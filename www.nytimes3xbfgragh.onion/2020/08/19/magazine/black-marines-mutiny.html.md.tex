The Untold Story of the Black Marines Charged With Mutiny at Sea

\url{https://nyti.ms/2Qjqe1f}

\begin{itemize}
\item
\item
\item
\item
\item
\item
\end{itemize}

\includegraphics{https://static01.graylady3jvrrxbe.onion/images/2020/08/19/magazine/19ATWAR-24/19ATWAR-24-articleLarge.jpg?quality=75\&auto=webp\&disable=upscale}

Sections

\protect\hyperlink{site-content}{Skip to
content}\protect\hyperlink{site-index}{Skip to site index}

The Great ReadAt War

\hypertarget{the-untold-story-of-the-black-marines-charged-with-mutiny-at-sea}{%
\section{The Untold Story of the Black Marines Charged With Mutiny at
Sea}\label{the-untold-story-of-the-black-marines-charged-with-mutiny-at-sea}}

Racial strife aboard a Navy ship left three men facing the threat of the
death penalty. They became little more than statistics in the military's
dismal record of race relations in the Vietnam era.

Pfc. Alexander Jenkins Jr. (back left, in glasses) and Pfc. Roy L.
Barnwell (far right) with other Black Marines on the U.S.S.
Sumter.Credit...From Bart Lubow

Supported by

\protect\hyperlink{after-sponsor}{Continue reading the main story}

\href{https://www.nytimes3xbfgragh.onion/by/john-ismay}{\includegraphics{https://static01.graylady3jvrrxbe.onion/images/2018/07/12/multimedia/author-john-ismay/author-john-ismay-thumbLarge.png}}

By \href{https://www.nytimes3xbfgragh.onion/by/john-ismay}{John Ismay}

\begin{itemize}
\item
  Aug. 19, 2020
\item
  \begin{itemize}
  \item
  \item
  \item
  \item
  \item
  \item
  \end{itemize}
\end{itemize}

One evening in late August 1972, as the American tank-landing ship
U.S.S. Sumter was steaming off the coast of Vietnam, a Marine onboard
dropped the needle on the turntable in front of him, sending music to
the loudspeakers bolted to the bulkheads in the cavernous spaces where
hundreds of sailors and Marines slept and hung out. Some members of the
crew were not ready for what they heard. ``Sun, up down. On the corner,
uptown. I turn around and hear the sound. A voice is talking about who's
gonna die next. Cause the white man's got a God complex.''

Though nobody knew it at the moment, that song was about to set off a
series of events that would leave three Black Marines facing charges of
mutiny and the possibility of execution or lengthy imprisonment. Others
were at risk of being thrown out of the Marine Corps with discharges
that would maim their job prospects in civilian America for the rest of
their lives. They were caught up in events that were not only about race
but also about structural racism; not just a matter of individuals and
personalities but of a U.S. military establishment that treated people
of color differently from white service members --- starting with
recruitment and induction, through combat deployments, right on through
the charges and punishments that arose when conflicts boiled over.

\includegraphics{https://static01.graylady3jvrrxbe.onion/images/2020/08/19/magazine/19ATWAR-11/19ATWAR-11-articleLarge.jpg?quality=75\&auto=webp\&disable=upscale}

The Marine spinning records that day was Pfc. Alexander Jenkins Jr., a
19-year-old from Newport News, Va., whose outgoing personality had
earned him a turn as the ship's D.J. During tedious weeks at sea, music
was one way to pass the time, but while Black Marines listened to songs
by white artists with no complaints, some white service members were not
so open in their tastes. Jenkins quickly found himself under verbal
attack from white sergeants and officers --- part of a campaign of
harassment and poor treatment that included mess cooks intentionally
handing him and his friends cold and inedible food, surprise uniform
inspections and capricious punishments from noncommissioned officers.
Eventually, it escalated to Black and white Marines physically fighting
each other on a ship at sea.

Jenkins kept playing the newest records and tapes he could find by Black
artists, many of which reflected the antiwar and Black-liberation
movements happening at home, alongside country and western albums and
hits by the Beatles. ``I was playing `What's Going On' by Marvin Gaye,
and I was playing `Bring the Boys Home' by Freda Payne,'' Jenkins
recalls. ``But playing `White Man's Got a God Complex' by the Last Poets
really set the white guys off.''

Jenkins remembers being pulled into a small room on the ship and
questioned by a group of higher-ranking white Marines about the
Harlem-based hip-hop pioneers' spoken-word song, which touched on
poverty, prostitution, drugs, the military-industrial complex, white
supremacy and the killings of Native Americans and Blacks. They accused
Jenkins of playing music that would incite a riot. ``If you don't have a
God complex, then this doesn't apply to you, now does it?'' Jenkins told
them. ``But if you do have a God complex, then you've got to listen,''
he added. A white Marine captain jumped out of his chair so forcefully
that it flipped over. ``You think you're so smart, don't you?'' the
Marine screamed in Jenkins's face. ``I'm sorry, sir. I really don't
understand,'' Jenkins countered. ``It's a damn record, OK? It's got a
nice beat.'' Jenkins was incensed, but he decided against pushing things
much further. ``I didn't want to get shot without a trial,'' he
recalled. Despite Jenkins's attempt to keep tensions from escalating,
relations between white and Black Marines aboard the Sumter were about
to get much worse.

\textbf{Put into service just} two years earlier, the Sumter steamed off
the coast of Vietnam with more than 150 Marines from a hodgepodge of
different units from the American bases on Okinawa, Japan. Among them
were Black servicemen who had been pushed to become truck drivers or
infantry troops because of racial bias in assessment tests. They were
part of a quick-reaction force that could be put ashore anywhere along
the coast to fight the Viet Cong and North Vietnamese Army should the
need arise. Until that time, though, they waited.

Even as the Marine Corps publicly announced efforts to reduce racist
attacks within the ranks, harassment, mistreatment and violence against
Blacks was commonplace and accepted, both in the United States (on bases
like Camp Lejeune in North Carolina, where the Ku Klux Klan posted a
billboard reading ``This Is Klan Country'' on a nearby highway) and on
its outposts in Okinawa and elsewhere. The more besieged the Black
Marines on the Sumter felt, the tighter they drew together for mutual
support and protection. But such security was ephemeral. Jenkins and two
of his close friends were about to have their young lives upended by an
incident that was hardly reported and remained almost invisible to the
public. The three Marines became little more than statistics in the
Corps's dismal record of race relations in the Vietnam era.

{[}\href{https://www.nytimes3xbfgragh.onion/newsletters/at-war}{\emph{Sign
up for the weekly At War newsletter}} \emph{to receive stories about
duty, conflict and consequence.}{]}

Trouble had already flared up in July outside the gates of the U.S.
Navy's base in Subic Bay, Philippines, during a port call. There, in the
town of Olongapo, sailors and Marines availed themselves of every kind
of vice in the de facto racially segregated entertainment district.
Black Marines and sailors tended to hang out in a neighborhood called
the Jungle, while their white counterparts had the run of the bars and
brothels elsewhere. An investigation by the director of naval
intelligence mentioned ``racial incidents'' between whites and Blacks
during Sumter's port visit there, where fistfights in the streets and
bars were not unusual. Sailors and Marines used the port visit to bring
a fresh supply of marijuana and heroin onto the ship for some diversion
during long days at sea.

Back on the ship, white officers harassed Black Marines for minor
infractions involving their hair and uniforms. Tight quarters left
little room for the men to blow off steam, and small routine squabbles
soon escalated. The ship's radio station --- the loudspeaker system
Jenkins played music on in the evenings --- was one of the few sources
of entertainment, and now even that became a point of contention.

Days after Jenkins was reprimanded, larger and more intense fights among
the Marines broke out. There are varying accounts of what happened and
why. Black and white Marines alike recall that a series of fistfights
throughout the deployment increased in frequency in the early days of
September on Sumter. In interviews with The Times, a half-dozen sailors
and Marines who were on the Sumter recalled these fights --- some
started by whites, others by Blacks. The Marines' leadership, however,
zeroed in on Jenkins, along with Pfc. Roy L. Barnwell and Lance Cpl.
James S. Blackwell, as the ``ringleaders'' who were instigating general
unrest and resistance to their orders.

After Jenkins was told he couldn't play the Last Poets, 64 of the 65
Black Marines on the ship submitted an informal complaint to the
highest-ranking Marine officer on board, Capt. John B. Krueger,
according to an account written a few months afterward by the defense
team that Jenkins, Barnwell and Blackwell soon needed. In their note,
the Black Marines told Krueger that they were being denied the right to
play their own music. ``Being that races are different in certain
aspects, and music being one,'' it read, ``then the proper officials
must make way as to the satisfaction of each and every race regardless
of minority.'' The Marines then submitted a request for a formal meeting
with their battalion commander, who was located on another ship nearby.
It was denied, further inflaming interactions between the men on board.

Tense conditions and simmering violence are detailed in the 1973 account
written by the legal team. White noncommissioned officers prowled the
berthing areas, harassing Black Marines. And when they talked back, they
were formally punished. One white lieutenant is said to have had a Black
Marine thrown into the ship's brig --- a jail with barred cells --- and
fed only bread and water for three days for nothing more than not having
his uniform completely in order. The same officer returned to the brig
to further harass and physically beat the man, according to the legal
team's account. In three separate incidents, one Black Marine had a
wrench thrown at him, another was cut with a sharp object and a third
was attacked with a knife, though those incidents were never
investigated by Marine leadership.

Image

The U.S.S. Sumter steamed off the coast of Vietnam with more than 150
Marines from a hodgepodge of different units.Credit...Alexander Jenkins
Jr.

Image

Incidents like what happened on the Sumter were not uncommon on military
bases around the world in the late 1960s and early
1970s.Credit...Alexander Jenkins Jr.

Joe Mueller, a white Marine officer who was then a second lieutenant on
his first deployment, remembers differently. In an interview, he
recalled Black Marines testing the limits of discipline in a number of
ways, including humming the tune of ``White Man's Got a God Complex'' as
a form of protest. On duty as the officer of the day on Sept. 7, he
heard a verbal disagreement outside the mess decks that quickly
escalated into the smacking sounds of fists. Somebody hit the switch
that flipped the overhead lights from nighttime red to bright white, and
everyone froze. Among the dozen or more men involved in the fight,
Mueller says, he saw three Black Marines --- Jenkins, Barnwell and
Blackwell --- standing over a white Marine. Forty-eight years later,
Jenkins has no recollection of this particular incident.

Another fight between Black and white Marines broke out the next day on
the ship's tank deck at lunchtime. First Lt. Al Vargas, the commander of
the embarked infantry company, remembers being struck in his side as he
dove in to help break up the melee. He then ordered all of the men under
his command back to their bunks. That's when Krueger, two first
lieutenants, a gunnery sergeant and a staff sergeant came to arrest
Jenkins. Jenkins doesn't deny that he was involved in this fight, but
his memory isn't clear on the details. ``I don't think I hit him, but
I'm the one they arrested for it,'' Jenkins says.

A twin-rotor CH-46 helicopter landed on the Sumter, loaded at least six
Marines --- Jenkins, Barnwell and Blackwell among them --- and flew off.
A Marine officer assured the ship's leaders that the ``troublemakers,''
the oldest of whom was 22 years old, would face discipline elsewhere.
For Jenkins, Barnwell and Blackwell, the days and weeks that followed
would have lasting repercussions on the rest of their lives.

The helicopter put the men ashore in Vietnam. In Danang, Jenkins
recalled, a colonel sat him down in a room and accused him of either
being a communist or a part of the Black power movement. Jenkins was
mystified, pointing out that he had volunteered for the Marine Corps,
and being on a ship in the middle of the Pacific, he had no telephone
and no possible communication with either group. ``I said, `Sir, this is
what's going on: We're being treated unfairly. Black men are getting
written up for the length of our hair, and harassed about our
uniforms.'''

Jenkins says that all the Marines on the ship wanted to go ashore and
fight the Viet Cong, but now, without any other outlets, they were
fighting each other. ``I got to love and trust that guy next to me,''
Jenkins told the colonel. ``And I'm not going to fight the enemy with
him if he doesn't like Black people.''

\textbf{The incidents on the} Sumter led the Marine Corps to charge
Jenkins, Barnwell and Blackwell with mutiny, for which they could have
faced the death penalty if found guilty. It was the first time since the
Civil War that American sailors or Marines had been charged with mutiny
at sea, according to two people who worked on the case in 1973. They
were also charged with various counts of assault, riot and resisting
arrest. Although two white Marines initially were charged with assault
and one with inciting to riot, all three were acquitted. Only one white
Marine, Sgt. Gary L. Wright, was convicted of any crime: dereliction of
duty for having ``refereed'' a fight between Barnwell and a white Marine
rather than breaking it up, but he received no punishment. The case did
not attract wide public attention, though it was one of many that
revealed the institutional racial biases that held strong across the
American military decades after the armed forces were desegregated.

Incidents like what happened on the Sumter were common on military bases
and warships around the world in the late 1960s and early 1970s --- a
reflection of what was happening more broadly as the civil rights
movement gained traction across the United States. Pervasive
mistreatment of Black inmates in base stockades --- essentially military
jails --- sparked riots in 1968 and 1969 at Fort Bragg in North
Carolina, Fort Carson in Colorado, Fort Dix in New Jersey, Fort Riley in
Kansas, Camp Pendleton in California and at Long Binh and Danang in
Vietnam. In May 1971, a fight between hundreds of Black and white airmen
at Travis Air Force Base in California resulted in the officers' club
being burned to the ground.

Camp Lejeune in North Carolina saw some of the most vicious and
persistent fighting between Black and white Marines in 1969. On
\href{https://www.nytimes3xbfgragh.onion/1969/07/24/archives/3-marines-hurt-in-lejeune-fight-hospitalized-after-a-racial.html}{Jul.
20}, three white Marines were hospitalized --- one with stab wounds to
the back --- after 44 Marines fought it out on base; one white Marine
\href{https://www.nytimes3xbfgragh.onion/1969/07/28/archives/corporal-20-dies-of-injuries-week-after-marine-base-fight.html}{later
died} from his injuries. The commanding officer of the Second Marine
Division there called it an isolated incident, but his Army counterpart
at the 82nd Airborne at nearby Fort Bragg recognized the seriousness of
the problem,
\href{https://www.nytimes3xbfgragh.onion/1969/08/23/archives/airborne-general-derides-marines-on-racial-fights.html}{saying}
``my men will not sink to the level of the Marines at Camp Lejeune.'' A
1971 report by the Congressional Black Caucus laid out the issues in
stark relief, saying ``subtle racism'' had ``crippled and impaired the
effectiveness of American troops'' and observed that ``the explosiveness
which prevails is made more serious by the amazing fact that many of
those in command positions on all levels refuse to realize that even in
a relatively controlled society as the military racism can and does
exist.''

Image

Jenkins in March 1972 in the barracks on base at Camp Foster, where he
was stationed for one year.Credit...From Alexander Jenkins Jr.

Image

Barnwell (right) and a fellow Marine on the Sumter's flight deck in
September 1972.Credit...Alexander Jenkins Jr.

Image

Lance Cpl. James S. Blackwell (right) with a sailor on the flight
deck.Credit...Alexander Jenkins Jr.

Just a month after the Sumter fights, a riot aboard the aircraft carrier
\href{https://www.nytimes3xbfgragh.onion/1972/11/29/archives/kitty-hawk-back-at-home-port-sailors-describe-racial-conflict.html}{U.S.S.
Kitty Hawk}, a tense sit-down strike on the carrier
\href{https://www.nytimes3xbfgragh.onion/1973/02/18/archives/the-constellation-incident-a-sort-of-mutiny.html}{U.S.S.
Constellation}, and a beating on the supply ship
\href{https://www.nytimes3xbfgragh.onion/1972/11/11/archives/zumwalt-rebukes-top-navy-leaders-on-racial-unrest-chief-berates.html}{U.S.N.S.
Hassayampa} made national headlines and moved the military to
investigate the broader source of the unrest. Adm. Elmo Zumwalt, the
Navy's top admiral, ordered an investigation into racial strife. The
resulting report found that from July 10 to Nov. 5, 1972, a total of 318
race-related incidents were documented at major Marine Corps
installations and that nearly half of those took place on two of the
service's bases in Okinawa, where Jenkins, Blackwell, Barnwell and the
rest of the Marines aboard the Sumter had come from. Despite these
findings, there would be little accountability among leaders for the
racial injustices that were festering within the ranks.

The House Armed Services Committee, led by the staunch segregationist
\href{https://www.nytimes3xbfgragh.onion/1979/12/30/archives/f-edward-hebert-exlawmaker-dies-conservative-louisiana-democrat.html}{F.
Edward Hébert} of Louisiana, immediately ordered an investigation of the
events aboard the two carriers. The Sumter incident was not included. On
Jan. 2, 1973, the subcommittee issued its report, placing all of the
blame on Black sailors it called ``thugs'' and deemed to be mostly of
``below-average mental capacity.'' It further blamed the programs
Zumwalt had instituted to eradicate systemic racism within the Navy for
creating a culture of ``permissiveness'' instead of taking a strict
law-and-order approach with Black sailors and Marines.

``The idea of this committee was to show that these equal-opportunity
programs were fomenting racial unrest,'' said the Navy historian John
Sherwood. ``The congressmen felt the reforms were the problem, and
hopefully Zumwalt would be fired, his programs abolished and the Navy
would go back to the way it was in the 1950s.''

Sherwood notes that Hébert was part of a broad coalition of Southern
segregationists in Congress --- two of whom, Representative
\href{https://www.nytimes3xbfgragh.onion/1981/06/02/obituaries/carl-vinson-97-ex-congressman-who-was-with-house-50-years-dies.html}{Carl
Vinson} of Georgia and Senator John C. Stennis of Mississippi, the Navy
later named aircraft carriers for --- that had a great deal of influence
on the Navy, and by extension, the Marine Corps, in the pre-Zumwalt era.
For members of Congress like Hébert, Vinson and Stennis, the civil
rights movement was an existential threat to the established order.

Zumwalt held onto his job, retiring in 1974. In the years that followed,
his successor continued his efforts on racial equity, but over time the
attention to reform petered out. The services have made progress in
adding Black and female officers, but have largely failed to place
people of color into leadership roles at the very top, which in 2020 are
still
\href{https://www.nytimes3xbfgragh.onion/2020/05/25/us/politics/military-minorities-leadership.html}{almost
entirely filled by white men}. Recently the service chiefs announced a
new round of task forces devoted to stamping out structural racism. ``We
must work to identify and eliminate individual and systemic racism
within our force,'' the Navy's top uniformed officer, Adm. Mike Gilday,
said in June, adding that the new program would ``work to identify and
remove racial barriers and improve inclusion within our Navy.'' But even
as these top-down initiatives are being put into place, experts are
repeatedly warning of
\href{https://www.nytimes3xbfgragh.onion/2019/02/27/us/military-white-nationalists-extremists.html}{white
supremacy} in the ranks.

\textbf{Back on the ship,} 20-year-old Lance Cpl. Alexander Holmes of
Brooklyn realized that Jenkins, Barnwell and Blackwell were in real
trouble. He felt that if things on the Sumter quieted down completely,
the Marine leadership would think that those three were the only
problem. ``I wanted to keep the tension up,'' Holmes recalls.

Holmes was joined by Pfc. Harry R. Wilson and Pfc. Charles S. Ross in
trying to keep the heat off their friends who had just been flown off
the ship. Holmes passed out butter knives to other Black Marines while
on the mess deck at mealtime, just so the white Marines would know that
things had not smoothed over. ``I knew from listening to Malcolm X and
Martin Luther King Jr. that the oppressor always feels like when they
cut the head off the snake that things will go back to normal,'' Holmes
says. ``But we wanted them to know that, no, the tension is still
here.''

It was only when Holmes disembarked the ship in Okinawa in October that
he learned that he too was in trouble. He was shown 20 to 25 witness
statements from white Marines recounting the incident with the butter
knives. Holmes readily admitted what happened and expressed regret.
``This white Marine lawyer sits me down and says if I just blame
everything on Jenkins, Barnwell and Blackwell, I'd be home for
Christmas,'' Holmes said. ``He knew I was supposed to be out of the
Marine Corps in November anyway, so he was just trying to get me to flip
on my friends.'' Holmes refused. The Marines eventually dropped their
charges of incitement against Holmes, and he flew to Naval Station
Treasure Island in San Francisco in February 1973, collected his
honorable-discharge paperwork and returned to Brooklyn to begin college.

Back in their jail cells on Okinawa, Jenkins, Barnwell and Blackwell
awaited the arrival of a lawyer from the States. One of Blackwell's
cousins in Chicago got the attention of the National Conference of Black
Lawyers, who promised to send a defense attorney. They tapped Ed Bell, a
young Oakland-based lawyer who planned to catch a military cargo flight
to meet his clients in Okinawa. After informing a Marine officer in
nearby Alameda that he intended to spread word of the Black liberation
movement among the troops in Okinawa upon his arrival, Bell was told by
Marine officials that all charges against Jenkins, Barnwell and
Blackwell had been dropped. Bell took them at their word, turned around
and went home. But it was a lie.

The three Marines in Okinawa were never told why the lawyer promised to
them never arrived, and they came to rely on a free legal clinic in
Koza, outside of Kadena Air Base, where Bart Lubow, a 25-year-old
civilian from Long Island, N.Y., worked as a legal assistant. Along with
the lawyers Bill Schaap and Doug Sorensen, the legal assistants Ellen
Ray and Lubow helped Jenkins, Barnwell and Blackwell mount a defense
during the military's equivalent of a grand jury hearing. It was Lubow
who wrote the near-contemporaneous account of the clashes on the ship.
That record, which he shared with The Times, details a military justice
system on Okinawa rife with racial animus that disproportionately
punished Black Marines, even for noncrimes like dapping, or for showing
a closed-fist gesture among other Black service members.

Jenkins denies that he, Barnwell and Blackwell were ringleaders, saying
instead that they were perhaps three of the most visible Black Marines
who challenged senior leaders for mistreating them on the Sumter. ``I
think I was singled out not just for the music, but because I was the
most boisterous,'' Jenkins recalls. ``We held classes on Black history
on the ship, and I would talk to the other Black Marines about
nonviolent resistance.'' That didn't matter. The response the Black
Marines received to their organizing, Jenkins said, was violence.

Image

From left: Jenkins, Barnwell and Blackwell at the judge advocate
general's office for a meeting with their lawyers in early
1973.Credit...From Bart Lubow

\textbf{Jenkins, Barnwell and} Blackwell, who spent months in the brig
in Okinawa, became known as the ``Sumter Three'' in the Black and
underground G.I. newspapers that covered their case. The former Marine
lawyer David Nelson recalls that the matter consumed the entire legal
office on Okinawa for months. With Schaap and Sorensen pushing for
exoneration and the Marine Corps not eager for more bad publicity, the
prosecutor eventually felt pressured to resolve the case. The mutiny
charges were dropped and eventually the other charges were too, in
exchange for the three Marines accepting unfavorable administrative
separations in lieu of courts-martial. The outcome could have been much
worse. The prosecutor had been pushing for 65 years of prison for each
man, with Blackwell facing an additional charge of slander for calling
his commanding officer a racist. Jenkins received a general discharge
under honorable conditions --- a discharge status that is not considered
fully honorable and denies veterans certain government benefits --- and
Lubow recalls that Barnwell and Blackwell each received an ``undesirable
discharge,'' which is another step worse than the one Jenkins received.

Between 1950 and 1980, 1.5 million service members received less than
fully honorable discharges, often referred to as ``bad paper''
discharges, through administrative separations --- with racial bias
often playing a role in those decisions. In 1972, a Department of
Defense task force found that Black service members ``received a higher
proportion of general and undesirable discharges than whites of similar
aptitude and education.'' That same year, the rate of service members
being discharged with general or other-than-honorable discharges from
the Marine Corps was 13 percent --- the highest percentage of all of the
services. (While the military has taken some steps to rectify racial
disparities within its ranks, people of color continue to suffer
disproportionately under the military justice system. As recently as
2015, Black service members were ``substantially more likely than white
service members to face military justice or disciplinary action,''
according to the legal justice group Protect Our Defenders.)

The consequences of less than fully honorable discharges are lifelong.
Numerous studies have found higher rates of unemployment, homelessness,
substance abuse and suicide among veterans with bad paper. The 1972 task
force, which even then called for greater protections of service
members' ``fundamental rights,'' argued that the issuance of bad paper
to a veteran ``will haunt him forever: affecting the respect of his
family, his standing in the community, impeding his effort to regain a
productive and meaningful role in society. The bad discharge is a
constant reinforcement of a negative self-image, a reminder that the
individual is `unsuitable, unfit or undesirable' in the eyes of his
country.'' With that stigma, the Sumter Three were all but guaranteed a
life of hardship without reprieve.

\textbf{Upon being released} from Okinawa, Jenkins briefly returned to
live with his mother and father in Virginia, but feeling that he had
outgrown his hometown, he moved to Detroit, where he stayed with his
sister and enrolled in college. Using the G.I. Bill to fund his
education, he started in the pre-med program at Wayne State University
but soon found himself interested in the new up-and-coming technology of
computer programming. He married, and when he had a family to support,
he left school in favor of getting a full-time job as a truck driver.
But Jenkins had trouble sleeping and suffered from depression, paranoia
and frequent anxiety attacks that developed after he returned home from
Japan. For self-defense, he bought an AR-15 for \$500, similar to the
M16 he carried in the Marines. One night he fired it at a thief who
tried to steal a barbecue from his yard. The experience so shook Jenkins
that he sold the rifle for almost half of what he paid, just to get it
out of his house. ``I felt besieged by the system,'' Jenkins says,
``because the system was always trying to get me, on something.''

In Detroit's withering economy, jobs came and went --- but sometimes the
layoffs were unexplained, in ways that suggested that employers were
acting out of racial bias or had found out about his discharge from the
Marines. In one case, after excelling as a computer programmer for a
bank and earning promotions, Jenkins was called in one day and
terminated, with no explanation other than an ominous hint that they had
found out something about his past. The stress and frustration grew over
decades, leading to an emotional collapse at age 38 that left him
briefly hospitalized.

James Blackwell also struggled when he got home. His sister Linda Page
puts it bluntly: ``When he got out he was a total mess.'' In one of
Page's spare bedrooms, he kicked the heroin habit he brought back with
him, but he continued to drink heavily. In 1994, at 43 years old, he
died suddenly of an aneurysm right outside the Cook County Circuit
Courthouse in Chicago. Page says Blackwell worked for the Yellow Pages
delivering telephone books and made money as an alley mechanic on the
side. She recalls him talking about his time on Okinawa awaiting his
court-martial. ``They kept him in a shed, and he could only see from
peeking out through the cracks,'' she says. ``He had real bad PTSD.''

Image

Jenkins at home in Detroit~in July.Credit...Cydni Elledge for The New
York Times

Barnwell seems to have fared even worse. His sister Patricia Gorman says
Barnwell lived in San Diego after leaving the Marine Corps, frequently
moving from one apartment to another. But she only learned that from him
much later: When he returned from Okinawa, he didn't contact his family
for more than 25 years. He got in touch in 1998, and she bought him a
round-trip train ticket to visit her in Choctaw County, Ala., where they
grew up. It was the first time she saw him since he went away to boot
camp in 1970. It was soon apparent that he wasn't about to make himself
at home there. Encountering slow service at a restaurant run by white
people, he suspected racism and wasn't quiet about it. On a different
day, he was pulled over by the police while driving. After that visit,
he never went back to Alabama. In 2001, Barnwell called Gorman to say
the cancer he had once beaten was back and he might have H.I.V. Public
records indicate Barnwell died April 9, 2001, in Los Angeles of
complications from AIDS. His family was never notified of his death, and
after 90 days, his remains were cremated and his ashes interred in a
mass grave for unclaimed bodies in Los Angeles County.

Jenkins still lives in Detroit, where he has quietly spent the last four
decades distancing himself from what happened on the Sumter, while still
maintaining a fierce pride in having been a Marine. Jenkins had wanted
to join the Corps since he was very young, and studied its history
before joining at age 17. He initially hoped to make the military a
career, but quickly chafed against systemic racism in the service. ``I
was full of piss and vinegar back then,'' Jenkins says. ``I look back to
my 19-year-old self and think, What the hell was I thinking?''

He says the only thing that saved him was some advice he got from his
uncle, John A. Jenkins, a Korean War combat vet, when he first got home
from Okinawa. ``I was mad as hell, angry at the world then,'' Jenkins
says. ``He drove it into me that if the cops stop you, that's their
chance to mess you up. It's almost like coming to America as a
foreigner: You have to learn the rules as a Black man to survive. You
have to know what to do and what not to do.'' Jenkins set out on the
straight and narrow, opting out of joints passed around at parties and
being meticulous about observing traffic laws. He says he has been
pulled over by the police only once or twice since 1973.

After his brief hospitalization in 1991, Jenkins stopped working outside
his home and devoted himself to helping his wife, Jerry, advance in her
career, and shepherding his daughter, Tanzania, through school to a
successful life as a systems engineer. Being charged with mutiny at sea
in a time of war shattered Jenkins emotionally --- and readily brought
tears 48 years later as he discussed it. ``I've been a recluse all these
years, because I didn't want these questions asked, and didn't want to
talk about it,'' Jenkins says. About 15 years ago, he joined a local
V.F.W. post to try to meet people. ``Most of the guys were Korea and
World War II guys who carried these same issues,'' Jenkins says. It
became difficult for him to keep going back, because so many appeared to
be drinking themselves to death.

As Jenkins slowly rebuilt his life, he lost track of the only two people
who truly understood what happened to him: Barnwell and Blackwell.
Jenkins only just learned of their deaths. ``I was hoping that at least
one of the two of them would be in a stable situation and be able to be
here now,'' Jenkins says. ``That's why I feel so alone, you know. I feel
very --- almost guilty about this situation that neither of those two
are here.''

While most days are better, Jenkins struggled with thoughts of suicide
as recently as 10 years ago. On days when his mind goes back to the
Sumter, his wife can tell, because he falls quiet for hours at a time.
``That situation on the Sumter screwed up my whole life,'' Jenkins says.
``I had to put on a different face to the world just to survive.''

\begin{center}\rule{0.5\linewidth}{\linethickness}\end{center}

\textbf{Sign up for}
\textbf{\href{https://www.nytimes3xbfgragh.onion/newsletters/at-war}{our
newsletter}} \textbf{to get the best of At War delivered to your inbox
every week. For more coverage of conflict, visit}
\textbf{\href{https://www.nytimes3xbfgragh.onion/spotlight/atwar}{NYTimes.com/atwar}.}

Advertisement

\protect\hyperlink{after-bottom}{Continue reading the main story}

\hypertarget{site-index}{%
\subsection{Site Index}\label{site-index}}

\hypertarget{site-information-navigation}{%
\subsection{Site Information
Navigation}\label{site-information-navigation}}

\begin{itemize}
\tightlist
\item
  \href{https://help.nytimes3xbfgragh.onion/hc/en-us/articles/115014792127-Copyright-notice}{©~2020~The
  New York Times Company}
\end{itemize}

\begin{itemize}
\tightlist
\item
  \href{https://www.nytco.com/}{NYTCo}
\item
  \href{https://help.nytimes3xbfgragh.onion/hc/en-us/articles/115015385887-Contact-Us}{Contact
  Us}
\item
  \href{https://www.nytco.com/careers/}{Work with us}
\item
  \href{https://nytmediakit.com/}{Advertise}
\item
  \href{http://www.tbrandstudio.com/}{T Brand Studio}
\item
  \href{https://www.nytimes3xbfgragh.onion/privacy/cookie-policy\#how-do-i-manage-trackers}{Your
  Ad Choices}
\item
  \href{https://www.nytimes3xbfgragh.onion/privacy}{Privacy}
\item
  \href{https://help.nytimes3xbfgragh.onion/hc/en-us/articles/115014893428-Terms-of-service}{Terms
  of Service}
\item
  \href{https://help.nytimes3xbfgragh.onion/hc/en-us/articles/115014893968-Terms-of-sale}{Terms
  of Sale}
\item
  \href{https://spiderbites.nytimes3xbfgragh.onion}{Site Map}
\item
  \href{https://help.nytimes3xbfgragh.onion/hc/en-us}{Help}
\item
  \href{https://www.nytimes3xbfgragh.onion/subscription?campaignId=37WXW}{Subscriptions}
\end{itemize}
