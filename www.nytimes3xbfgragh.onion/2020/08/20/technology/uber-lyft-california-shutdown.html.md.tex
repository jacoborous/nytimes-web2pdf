Sections

SEARCH

\protect\hyperlink{site-content}{Skip to
content}\protect\hyperlink{site-index}{Skip to site index}

\href{https://www.nytimes3xbfgragh.onion/section/technology}{Technology}

\href{https://myaccount.nytimes3xbfgragh.onion/auth/login?response_type=cookie\&client_id=vi}{}

\href{https://www.nytimes3xbfgragh.onion/section/todayspaper}{Today's
Paper}

\href{/section/technology}{Technology}\textbar{}Uber and Lyft Get
Reprieve After Threatening to Shut Down

\url{https://nyti.ms/2QarROv}

\begin{itemize}
\item
\item
\item
\item
\item
\end{itemize}

Advertisement

\protect\hyperlink{after-top}{Continue reading the main story}

Supported by

\protect\hyperlink{after-sponsor}{Continue reading the main story}

\hypertarget{uber-and-lyft-get-reprieve-after-threatening-to-shut-down}{%
\section{Uber and Lyft Get Reprieve After Threatening to Shut
Down}\label{uber-and-lyft-get-reprieve-after-threatening-to-shut-down}}

The companies, under legal pressure to reclassify their California
drivers as employees, said they would halt rides before an appeals court
gave them permission to continue.

\includegraphics{https://static01.graylady3jvrrxbe.onion/images/2020/08/21/business/20rideshutdown-print/merlin_160662759_0951fbab-4d08-4a7d-b4bf-c4d238af7dc4-articleLarge.jpg?quality=75\&auto=webp\&disable=upscale}

By \href{https://www.nytimes3xbfgragh.onion/by/kate-conger}{Kate Conger}

\begin{itemize}
\item
  Aug. 20, 2020
\item
  \begin{itemize}
  \item
  \item
  \item
  \item
  \item
  \end{itemize}
\end{itemize}

OAKLAND, Calif. --- Uber and Lyft threatened to suspend ride-hailing
services throughout California on Thursday night, a defiant reaction to
a judge who ordered the companies to reclassify their drivers as
employees.

But hours before the ride-hailing blackout was set to begin, an appeals
court granted Uber and Lyft a temporary reprieve, allowing them to
continue operating while the court weighs their appeal. Oral arguments
in the case are set for mid-October.

``We are glad that the court of appeal recognized the important
questions raised in this case, and that access to these critical
services won't be cut off while we continue to advocate for drivers'
ability to work with the freedom they want,'' said Matt Kallman, a
spokesman for Uber.

The fight could drag on for months, as Uber and Lyft battle a state
labor law intended to give employment benefits to gig workers. An
appeals court is weighing the companies' requests to overturn a judge's
order to employ drivers, but it is not clear when the court will issue a
ruling. The court has ordered Uber and Lyft to submit plans for hiring
employees by early September, in case the court does not decide in their
favor.

``These companies may have bought themselves a little more time, but the
price is that they have to demonstrate --- under oath --- that they have
an implementation plan that complies with the law,'' said John Cote, a
spokesman for the San Francisco city attorney, one of the officials
suing Uber and Lyft. ``The court of appeal is calling Uber and Lyft's
bluff.''

If Uber and Lyft are forced to reclassify drivers, they are considering
\href{https://www.nytimes3xbfgragh.onion/2020/08/18/technology/uber-lyft-franchise-california.html}{plans
to establish franchise-like operations in California}, inviting third
parties to hire their drivers rather than becoming employers themselves.

State officials said the companies must comply with the a new law, known
as
\href{https://www.nytimes3xbfgragh.onion/2019/09/11/technology/california-gig-economy-bill.html}{Assembly
Bill 5}, so that workers have access to sick leave, overtime and other
benefits --- a need that has become more dire during the coronavirus
pandemic.

But Uber and Lyft have argued that employing drivers would have a
catastrophic impact on their businesses, forcing them to raise fares and
hire only a small fraction of the drivers who currently work for them.
They would temporarily shutter the businesses rather than comply, they
said.

``While we won't have to suspend operations tonight, we do need to
continue fighting for independence plus benefits for drivers,'' said
Julie Wood, a Lyft spokeswoman.

Uber and Lyft have long categorized drivers as independent contractors,
an arrangement that the companies say allows drivers to have more
control over where and when they drive. But this model imposes a
financial burden on drivers, who are responsible for their own vehicle
maintenance, health insurance and other expenses that employers
traditionally cover.

Last year, the California Legislature passed A.B. 5 in an attempt to set
clearer employment standards for the state and rein in gig-economy
giants like Uber. Legislators argued that Uber shortchanged its drivers
and exploited an unfair advantage over law-abiding businesses in the
state.

Although the law went into effect in January, Uber and Lyft did not
change their practices. They argued that A.B. 5 did not apply to them
and spent tens of millions of dollars on a ballot initiative that, if
passed in November, would exempt them from the law.

In May,
\href{https://www.nytimes3xbfgragh.onion/2020/05/05/technology/california-uber-lyft-lawsuit.html}{California's
attorney general sued Uber and Lyft} to force them to comply with A.B.
5. The standoff came to a head last week when a San Francisco Superior
Court judge, Ethan Schulman, sided with the state, ordering Uber and
Lyft to reclassify their drivers by Thursday.

Uber and Lyft have argued that they are technology companies and that
drivers are not a core part of their business. But that ``flies in the
face of economic reality and common sense,'' Judge Schulman wrote in his
ruling. ``Were this reasoning to be accepted, the rapidly expanding
majority of industries that rely heavily on technology could with
impunity deprive legions of workers of the basic protections afforded to
employees by state labor and employment laws.''

``Our state and workers shouldn't have to foot the bill when big
businesses try to skip out on their responsibilities,'' said the
California attorney general, Xavier Becerra.

Rather than hire drivers, Uber and Lyft threatened to shut down. The
decision could have caused the businesses, which have already struggled
financially because of travel restrictions during the pandemic, to lose
even more money.

San Francisco and Los Angeles are among Uber's largest markets, and Lyft
has said it draws about 16 percent of its business from California. Uber
planned to continue operating Uber Eats, its food delivery service,
which has bolstered its revenue during the pandemic, a spokesman said.

Although the potential shutdown felt drastic to drivers and riders who
depend on Uber and Lyft, the move was not without precedent. The
companies have terminated their services in other regions rather than
complying with local laws they oppose. The shutdowns have often
pressured local governments to pass laws that are more friendly to Uber
and Lyft.

In 2016,
\href{https://www.nytimes3xbfgragh.onion/2016/05/10/technology/uber-and-lyft-stop-rides-in-austin-to-protest-fingerprint-background-checks.html}{Uber
and Lyft shut down in Austin}, Texas, to protest an ordinance that
required background checks that used fingerprints for drivers. They
returned the next year after Texas passed a
\href{https://gov.texas.gov/news/post/governor-abbott-signs-bill-ending-local-regulations-of-transportation-netwo}{statewide
law} that excludes fingerprinting from the background check
requirements.

That strategy could work again for Uber and Lyft if California voters
approve the ballot measure in November. If the companies lose that vote,
they would be required to employ drivers and may put in place
\href{https://www.nytimes3xbfgragh.onion/2020/08/18/technology/uber-lyft-franchise-california.html}{plans
to establish franchise-like operations in California}.

Advertisement

\protect\hyperlink{after-bottom}{Continue reading the main story}

\hypertarget{site-index}{%
\subsection{Site Index}\label{site-index}}

\hypertarget{site-information-navigation}{%
\subsection{Site Information
Navigation}\label{site-information-navigation}}

\begin{itemize}
\tightlist
\item
  \href{https://help.nytimes3xbfgragh.onion/hc/en-us/articles/115014792127-Copyright-notice}{©~2020~The
  New York Times Company}
\end{itemize}

\begin{itemize}
\tightlist
\item
  \href{https://www.nytco.com/}{NYTCo}
\item
  \href{https://help.nytimes3xbfgragh.onion/hc/en-us/articles/115015385887-Contact-Us}{Contact
  Us}
\item
  \href{https://www.nytco.com/careers/}{Work with us}
\item
  \href{https://nytmediakit.com/}{Advertise}
\item
  \href{http://www.tbrandstudio.com/}{T Brand Studio}
\item
  \href{https://www.nytimes3xbfgragh.onion/privacy/cookie-policy\#how-do-i-manage-trackers}{Your
  Ad Choices}
\item
  \href{https://www.nytimes3xbfgragh.onion/privacy}{Privacy}
\item
  \href{https://help.nytimes3xbfgragh.onion/hc/en-us/articles/115014893428-Terms-of-service}{Terms
  of Service}
\item
  \href{https://help.nytimes3xbfgragh.onion/hc/en-us/articles/115014893968-Terms-of-sale}{Terms
  of Sale}
\item
  \href{https://spiderbites.nytimes3xbfgragh.onion}{Site Map}
\item
  \href{https://help.nytimes3xbfgragh.onion/hc/en-us}{Help}
\item
  \href{https://www.nytimes3xbfgragh.onion/subscription?campaignId=37WXW}{Subscriptions}
\end{itemize}
