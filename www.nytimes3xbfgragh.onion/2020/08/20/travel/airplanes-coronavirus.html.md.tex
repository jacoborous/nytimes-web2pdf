Sections

SEARCH

\protect\hyperlink{site-content}{Skip to
content}\protect\hyperlink{site-index}{Skip to site index}

\href{https://www.nytimes3xbfgragh.onion/section/travel}{Travel}

\href{https://myaccount.nytimes3xbfgragh.onion/auth/login?response_type=cookie\&client_id=vi}{}

\href{https://www.nytimes3xbfgragh.onion/section/todayspaper}{Today's
Paper}

\href{/section/travel}{Travel}\textbar{}5 Things We Know About Flying
Right Now

\url{https://nyti.ms/3hgfV9J}

\begin{itemize}
\item
\item
\item
\item
\item
\item
\end{itemize}

\hypertarget{the-coronavirus-outbreak}{%
\subsubsection{\texorpdfstring{\href{https://www.nytimes3xbfgragh.onion/news-event/coronavirus?name=styln-coronavirus-national\&region=TOP_BANNER\&block=storyline_menu_recirc\&action=click\&pgtype=Article\&impression_id=0fae8cc0-f1df-11ea-98d7-2bcdd79e137a\&variant=undefined}{The
Coronavirus
Outbreak}}{The Coronavirus Outbreak}}\label{the-coronavirus-outbreak}}

\begin{itemize}
\tightlist
\item
  live\href{https://www.nytimes3xbfgragh.onion/2020/09/08/world/covid-19-coronavirus.html?name=styln-coronavirus-national\&region=TOP_BANNER\&block=storyline_menu_recirc\&action=click\&pgtype=Article\&impression_id=0faeb3d0-f1df-11ea-98d7-2bcdd79e137a\&variant=undefined}{Latest
  Updates}
\item
  \href{https://www.nytimes3xbfgragh.onion/interactive/2020/us/coronavirus-us-cases.html?name=styln-coronavirus-national\&region=TOP_BANNER\&block=storyline_menu_recirc\&action=click\&pgtype=Article\&impression_id=0faeb3d1-f1df-11ea-98d7-2bcdd79e137a\&variant=undefined}{Maps
  and Cases}
\item
  \href{https://www.nytimes3xbfgragh.onion/interactive/2020/science/coronavirus-vaccine-tracker.html?name=styln-coronavirus-national\&region=TOP_BANNER\&block=storyline_menu_recirc\&action=click\&pgtype=Article\&impression_id=0faeb3d2-f1df-11ea-98d7-2bcdd79e137a\&variant=undefined}{Vaccine
  Tracker}
\item
  \href{https://www.nytimes3xbfgragh.onion/2020/09/02/your-money/eviction-moratorium-covid.html?name=styln-coronavirus-national\&region=TOP_BANNER\&block=storyline_menu_recirc\&action=click\&pgtype=Article\&impression_id=0faeb3d3-f1df-11ea-98d7-2bcdd79e137a\&variant=undefined}{Eviction
  Moratorium}
\item
  \href{https://www.nytimes3xbfgragh.onion/interactive/2020/09/02/magazine/food-insecurity-hunger-us.html?name=styln-coronavirus-national\&region=TOP_BANNER\&block=storyline_menu_recirc\&action=click\&pgtype=Article\&impression_id=0faeb3d4-f1df-11ea-98d7-2bcdd79e137a\&variant=undefined}{American
  Hunger}
\end{itemize}

Advertisement

\protect\hyperlink{after-top}{Continue reading the main story}

Supported by

\protect\hyperlink{after-sponsor}{Continue reading the main story}

\hypertarget{5-things-we-know-about-flying-right-now}{%
\section{5 Things We Know About Flying Right
Now}\label{5-things-we-know-about-flying-right-now}}

As passengers cautiously return to air travel, there are a few issues
worth considering --- from middle-seat policies to questions about virus
transmission on airlines.

\includegraphics{https://static01.graylady3jvrrxbe.onion/images/2020/08/20/travel/20flying-coronavirus1/20flying-coronavirus1-articleLarge.jpg?quality=75\&auto=webp\&disable=upscale}

By Elaine Glusac

\begin{itemize}
\item
  Aug. 20, 2020
\item
  \begin{itemize}
  \item
  \item
  \item
  \item
  \item
  \item
  \end{itemize}
\end{itemize}

After passenger numbers plummeted earlier this year, air travel has
taken a significant step forward. On Aug. 16, nearly 863,000
\href{https://www.tsa.gov/coronavirus/passenger-throughput}{fliers}
passed through Transportation Security Administration checkpoints around
the country, the highest figure since March 17. Though just one third of
last year's 2.5 million passengers, the traffic is sharply higher than
the 87,534 who traveled on April 14 in the depths of the pandemic.

Commercial flights are down 43 percent in the United States, according
to
\href{https://flightaware.com/news/article/Aviation-Recovery-at-50/1486}{FlightAware.com},
a service that tracks flights, but that is the best figure since the
pandemic began, and up from a roughly 77 percent drop in April.

While the future of aviation remains uncertain --- the industry is
lobbying for more government funding to ward off future layoffs and
route cuts expected when the Coronavirus Aid, Relief and Economic
Security (CARES) Act funding expires Sept. 30 --- here are five things
we know about flying now.

\hypertarget{the-middle-seat-saga-continues}{%
\subsection{The middle seat saga
continues}\label{the-middle-seat-saga-continues}}

After the pandemic hit, three of the four biggest airlines in the
country --- American, Delta and Southwest --- vowed to block the sale of
middle seats to provide more social distancing in the air. United
Airlines was the sole holdout.

Now, American has joined United in selling all available seats on its
planes as demand allows, while Southwest has extended its commitment to
less density through Oct. 31.
\href{https://www.alaskaair.com/content/advisories/coronavirus}{Alaska
Airlines} is also blocking middle seats through Oct. 31, though it says
it may make exceptions for unforeseen circumstances, such as
accommodating fliers from a previously canceled flight.
\href{https://www.jetblue.com/safety}{JetBlue} has extended its
open-middle-seat policy through Oct. 15, and Delta is leaving adjacent
seats
\href{https://news.delta.com/more-space-through-summer-delta-will-block-middle-seat-selection-cap-cabin-seating-through-sept-30}{open}
through Jan. 6.

If you're looking for an uncrowded flight, the odds might be in your
favor. Airlines for America, the trade group that represents the major
airlines in the United States, says that as of Aug. 9, flights are
running about 47 percent full, versus 88 percent a year ago. Still,
\href{https://www.nytimes3xbfgragh.onion/2020/07/21/travel/crowded-flights-coronavirus.html}{complaints
about full flights} have continued on social media.

Lee Abbamonte, a travel blogger who lives in New York City, has flown
for pleasure and for work during the pandemic, upgrading to business
class with points, or sticking to carriers that block middle seats in
economy.

``I feel that a little research and legwork can make a big difference in
terms of load size,'' he wrote in an email. He uses online seat maps to
gauge how busy a flight is and recommends flying at off-peak times and
midweek for the lightest loads.

\hypertarget{low-fares-may-not-be-enough-to-lure-many-travelers}{%
\subsection{Low fares may not be enough to lure many
travelers}\label{low-fares-may-not-be-enough-to-lure-many-travelers}}

Traditionally, fall is a good time to look for cheap airfares and this
year is no different.

``After Aug. 15, fares go down because college kids and younger children
are going back to school,'' said George Hobica, the founder of
\href{https://www.airfarewatchdog.com/}{Airfarewatchdog.com}, a website
that finds flight deals. ``They would stay down to Thanksgiving, and
then after Thanksgiving to Dec. 14. You could set your almanac by it.''

This year, fares have been low all summer. According to the airfare
prediction app \href{https://www.hopper.com/}{Hopper}, the current
average price for a round-trip domestic ticket is \$176, down 38 percent
compared to the same time last year when it was \$282.

\hypertarget{latest-updates-the-coronavirus-outbreak}{%
\section{\texorpdfstring{\href{https://www.nytimes3xbfgragh.onion/2020/09/08/world/covid-19-coronavirus.html?action=click\&pgtype=Article\&state=default\&region=MAIN_CONTENT_1\&context=storylines_live_updates}{Latest
Updates: The Coronavirus
Outbreak}}{Latest Updates: The Coronavirus Outbreak}}\label{latest-updates-the-coronavirus-outbreak}}

Updated 2020-09-08T14:21:26.129Z

\begin{itemize}
\tightlist
\item
  \href{https://www.nytimes3xbfgragh.onion/2020/09/08/world/covid-19-coronavirus.html?action=click\&pgtype=Article\&state=default\&region=MAIN_CONTENT_1\&context=storylines_live_updates\#link-547feae1}{Senate
  Republicans plan to move forward with a scaled-back stimulus package.}
\item
  \href{https://www.nytimes3xbfgragh.onion/2020/09/08/world/covid-19-coronavirus.html?action=click\&pgtype=Article\&state=default\&region=MAIN_CONTENT_1\&context=storylines_live_updates\#link-679303d7}{Nine
  drugmakers pledge to thoroughly vet any coronavirus vaccine.}
\item
  \href{https://www.nytimes3xbfgragh.onion/2020/09/08/world/covid-19-coronavirus.html?action=click\&pgtype=Article\&state=default\&region=MAIN_CONTENT_1\&context=storylines_live_updates\#link-1c973131}{`The
  lockdown killed my father': Farmer suicides add to India's virus
  misery.}
\end{itemize}

\href{https://www.nytimes3xbfgragh.onion/2020/09/08/world/covid-19-coronavirus.html?action=click\&pgtype=Article\&state=default\&region=MAIN_CONTENT_1\&context=storylines_live_updates}{See
more updates}

More live coverage:
\href{https://www.nytimes3xbfgragh.onion/live/2020/09/08/business/stock-market-today-coronavirus?action=click\&pgtype=Article\&state=default\&region=MAIN_CONTENT_1\&context=storylines_live_updates}{Markets}

What consumers have gained in savings, they may give up in convenience,
as airlines have cut back the number of flights to consolidate traffic.
Mr. Hobica cited Delta's schedule from New York to Miami, which showed
just two nonstop flights on a Friday in August. Delta, which is running
half of its normal domestic schedule compared to last year, had nine
flights on this route last August.

``People will not have the choices that they had as far as schedules, at
least nonstop,'' he said.

In the current sales season, passengers can fly from as little as \$67
round trip from Newark to Tampa, Fla., on United. Many destinations on
American and Southwest are selling for about \$100 round trip.

Most sales seem aligned with where travelers say they want to go, which
is, generally speaking, away from other people.

Destination Analysts, a travel marketing research firm, has been doing a
weekly survey of traveler sentiment for the past 22 weeks and found most
recently that perennial destinations like Las Vegas and Orlando, Fla.,
remain high on the list of where people want to go, both before and
during the pandemic. New to the top 10 list since the virus are places
like Colorado and Alaska.

``People still want to go to Seattle and New Orleans, but because of the
pandemic we're seeing Colorado and wilderness destinations edge out
those urban experiences,'' said Erin Francis-Cummings, the chief
executive of Destination Analysts.

Still, most of the 1,200 adult American travelers surveyed told the firm
they won't travel no matter how low the fares go. In the last survey,
completed Aug. 9, 70 percent said no price cut would be large enough to
get them to travel.

``They are staying firm,'' Ms. Francis-Cummings said, comparing the
sentiment to the airline industry's recovery from the 2008 recession.
``Then, discounts were motivational. In the pandemic, a sizable percent
are not budging.''

\hypertarget{holiday-travel-may-be-cheaper-than-normal}{%
\subsection{Holiday travel may be cheaper than
normal}\label{holiday-travel-may-be-cheaper-than-normal}}

Typically, airlines hike fares beginning the weekend before Thanksgiving
in anticipation of the rush to the skies for the holiday. Currently, an
American flight from Chicago to Miami that sells from \$75 in October,
goes to \$356 the week of the holiday.

This year, of course, is like no other, and the number of college
students studying from home or families fearful of gatherings may
depress holiday travel. Hopper found prices are 30 percent lower
presently than they were in 2019 for Thanksgiving travel, with an
average round-trip domestic ticket at \$216.

At this point, only Delta has committed to blocking middle seats past
October. Additionally, flexible policies that waive fees for flight
cancellations or changes will expire well before the holiday (except at
Southwest, which is the only carrier that does not charge a fee for
ticket changes).
\href{https://www.united.com/ual/en/us/fly/travel/notices.html}{United}
is waiving change fees on new bookings through Aug. 31, though travel
may take place later.
\href{https://www.alaskaair.com/content/advisories/coronavirus?int=_AS_HomePage_AdvisoryBR_L1\%7C\%7C2020_CV_AW\%7C\%7C-prodID:Awareness\#weve-got-you-covered}{Alaska's}
waiver runs through Sept. 8.
\href{https://www.aa.com/i18n/travel-info/coronavirus-updates.jsp}{American}
and Delta have extended their waivers to Sept. 30, and
\href{https://www.jetblue.com/travel-alerts}{JetBlue} to Oct. 15.

``Right now, the priority for airlines is to make prices accessible and
terms flexible,'' said Hayley Berg, the economist at Hopper. ``Customers
are increasingly prioritizing flexibility in fares and their travel
experience over anything else.''

As a result, she advised booking holiday travel now while restrictions
are relaxed, if the price is right.

``On the whole, you will pay less than last year, but how much less and
when still remains to be seen,'' Ms. Berg said.

\hypertarget{first-class-doesnt-necessarily-guarantee-space}{%
\subsection{First class doesn't necessarily guarantee
space}\label{first-class-doesnt-necessarily-guarantee-space}}

Flying back recently from his second home in Tucson, Ariz., to Chicago,
George Fink, who works in finance, upgraded to first class on American,
using 55,000 miles for the one-way ticket in hopes of having more space.
Instead, he found himself with a seatmate wearing a mask that did not
cover his nose. He implored his fellow flier, who ignored him, to cover
up. He next tried the flight attendant, who would not help. The back of
the plane was full, too, making it impossible to move. Then the
attendants served a meal.

``That meant everyone in first class took off their masks and ate for
half an hour so all the masking and spacing was for naught,'' he said.

\href{https://www.nytimes3xbfgragh.onion/news-event/coronavirus?action=click\&pgtype=Article\&state=default\&region=MAIN_CONTENT_3\&context=storylines_faq}{}

\hypertarget{the-coronavirus-outbreak-}{%
\subsubsection{The Coronavirus Outbreak
›}\label{the-coronavirus-outbreak-}}

\hypertarget{frequently-asked-questions}{%
\paragraph{Frequently Asked
Questions}\label{frequently-asked-questions}}

Updated September 4, 2020

\begin{itemize}
\item ~
  \hypertarget{what-are-the-symptoms-of-coronavirus}{%
  \paragraph{What are the symptoms of
  coronavirus?}\label{what-are-the-symptoms-of-coronavirus}}

  \begin{itemize}
  \tightlist
  \item
    In the beginning, the coronavirus
    \href{https://www.nytimes3xbfgragh.onion/article/coronavirus-facts-history.html?action=click\&pgtype=Article\&state=default\&region=MAIN_CONTENT_3\&context=storylines_faq\#link-6817bab5}{seemed
    like it was primarily a respiratory illness}~--- many patients had
    fever and chills, were weak and tired, and coughed a lot, though
    some people don't show many symptoms at all. Those who seemed
    sickest had pneumonia or acute respiratory distress syndrome and
    received supplemental oxygen. By now, doctors have identified many
    more symptoms and syndromes. In April,
    \href{https://www.nytimes3xbfgragh.onion/2020/04/27/health/coronavirus-symptoms-cdc.html?action=click\&pgtype=Article\&state=default\&region=MAIN_CONTENT_3\&context=storylines_faq}{the
    C.D.C. added to the list of early signs}~sore throat, fever, chills
    and muscle aches. Gastrointestinal upset, such as diarrhea and
    nausea, has also been observed. Another telltale sign of infection
    may be a sudden, profound diminution of one's
    \href{https://www.nytimes3xbfgragh.onion/2020/03/22/health/coronavirus-symptoms-smell-taste.html?action=click\&pgtype=Article\&state=default\&region=MAIN_CONTENT_3\&context=storylines_faq}{sense
    of smell and taste.}~Teenagers and young adults in some cases have
    developed painful red and purple lesions on their fingers and toes
    --- nicknamed ``Covid toe'' --- but few other serious symptoms.
  \end{itemize}
\item ~
  \hypertarget{why-is-it-safer-to-spend-time-together-outside}{%
  \paragraph{Why is it safer to spend time together
  outside?}\label{why-is-it-safer-to-spend-time-together-outside}}

  \begin{itemize}
  \tightlist
  \item
    \href{https://www.nytimes3xbfgragh.onion/2020/05/15/us/coronavirus-what-to-do-outside.html?action=click\&pgtype=Article\&state=default\&region=MAIN_CONTENT_3\&context=storylines_faq}{Outdoor
    gatherings}~lower risk because wind disperses viral droplets, and
    sunlight can kill some of the virus. Open spaces prevent the virus
    from building up in concentrated amounts and being inhaled, which
    can happen when infected people exhale in a confined space for long
    stretches of time, said Dr. Julian W. Tang, a virologist at the
    University of Leicester.
  \end{itemize}
\item ~
  \hypertarget{why-does-standing-six-feet-away-from-others-help}{%
  \paragraph{Why does standing six feet away from others
  help?}\label{why-does-standing-six-feet-away-from-others-help}}

  \begin{itemize}
  \tightlist
  \item
    The coronavirus spreads primarily through droplets from your mouth
    and nose, especially when you cough or sneeze. The C.D.C., one of
    the organizations using that measure,
    \href{https://www.nytimes3xbfgragh.onion/2020/04/14/health/coronavirus-six-feet.html?action=click\&pgtype=Article\&state=default\&region=MAIN_CONTENT_3\&context=storylines_faq}{bases
    its recommendation of six feet}~on the idea that most large droplets
    that people expel when they cough or sneeze will fall to the ground
    within six feet. But six feet has never been a magic number that
    guarantees complete protection. Sneezes, for instance, can launch
    droplets a lot farther than six feet,
    \href{https://jamanetwork.com/journals/jama/fullarticle/2763852}{according
    to a recent study}. It's a rule of thumb: You should be safest
    standing six feet apart outside, especially when it's windy. But
    keep a mask on at all times, even when you think you're far enough
    apart.
  \end{itemize}
\item ~
  \hypertarget{i-have-antibodies-am-i-now-immune}{%
  \paragraph{I have antibodies. Am I now
  immune?}\label{i-have-antibodies-am-i-now-immune}}

  \begin{itemize}
  \tightlist
  \item
    As of right
    now,\href{https://www.nytimes3xbfgragh.onion/2020/07/22/health/covid-antibodies-herd-immunity.html?action=click\&pgtype=Article\&state=default\&region=MAIN_CONTENT_3\&context=storylines_faq}{~that
    seems likely, for at least several months.}~There have been
    frightening accounts of people suffering what seems to be a second
    bout of Covid-19. But experts say these patients may have a
    drawn-out course of infection, with the virus taking a slow toll
    weeks to months after initial exposure.~People infected with the
    coronavirus typically
    \href{https://www.nature.com/articles/s41586-020-2456-9}{produce}~immune
    molecules called antibodies, which are
    \href{https://www.nytimes3xbfgragh.onion/2020/05/07/health/coronavirus-antibody-prevalence.html?action=click\&pgtype=Article\&state=default\&region=MAIN_CONTENT_3\&context=storylines_faq}{protective
    proteins made in response to an
    infection}\href{https://www.nytimes3xbfgragh.onion/2020/05/07/health/coronavirus-antibody-prevalence.html?action=click\&pgtype=Article\&state=default\&region=MAIN_CONTENT_3\&context=storylines_faq}{.
    These antibodies may}~last in the body
    \href{https://www.nature.com/articles/s41591-020-0965-6}{only two to
    three months}, which may seem worrisome, but that's~perfectly normal
    after an acute infection subsides, said Dr. Michael Mina, an
    immunologist at Harvard University. It may be possible to get the
    coronavirus again, but it's highly unlikely that it would be
    possible in a short window of time from initial infection or make
    people sicker the second time.
  \end{itemize}
\item ~
  \hypertarget{what-are-my-rights-if-i-am-worried-about-going-back-to-work}{%
  \paragraph{What are my rights if I am worried about going back to
  work?}\label{what-are-my-rights-if-i-am-worried-about-going-back-to-work}}

  \begin{itemize}
  \tightlist
  \item
    Employers have to provide
    \href{https://www.osha.gov/SLTC/covid-19/standards.html}{a safe
    workplace}~with policies that protect everyone equally.
    \href{https://www.nytimes3xbfgragh.onion/article/coronavirus-money-unemployment.html?action=click\&pgtype=Article\&state=default\&region=MAIN_CONTENT_3\&context=storylines_faq}{And
    if one of your co-workers tests positive for the coronavirus, the
    C.D.C.}~has said that
    \href{https://www.cdc.gov/coronavirus/2019-ncov/community/guidance-business-response.html}{employers
    should tell their employees}~-\/- without giving you the sick
    employee's name -\/- that they may have been exposed to the virus.
  \end{itemize}
\end{itemize}

Only Delta and Alaska have committed to reducing density in the forward
cabin to 50 percent. (Southwest doesn't have a first or business class;
JetBlue is blocking six of 16 seats in its forward-class Mint cabins.)
Fliers opting for the upgrade on other carriers --- sometimes at very
attractive prices --- may very well find their wider, more spacious
seats just inches from the passengers next to them.

Experts advise looking for airplane configurations that include single
seat configurations. For example, on the Dreamliner that Mr. Hobica, the
Airfarewatchdog.com founder, booked from Los Angeles to Newark, the seat
configuration in business class was 1-2-1. (Bear in mind that carriers
have the right to change aircraft per their contracts of carriage.)

``It's a good way to fly if you don't want someone next to you,'' Mr.
Hobica said.

Megan Solis, a teacher in Chicago, bought three \$450 round-trip tickets
in business class on United in early September so that she and her
husband could take her oldest child to college in Boston, using credits
from a previously canceled trip. Currently, the last seat in the
family's four-seat row is empty and they're hoping it stays that way.

The relatively low fare was secondary in her decision. ``I was more
comfortable with the space up there, even before I saw the price,'' she
said.

\hypertarget{calculating-in-flight-transmission-risks}{%
\subsection{Calculating in-flight transmission
risks}\label{calculating-in-flight-transmission-risks}}

While the airlines tout their HEPA filters, which scrub more than 99
percent of germs in the air, there has been very little data on the
risks of catching coronavirus in-flight, even as evidence
\href{https://www.nytimes3xbfgragh.onion/2020/08/11/health/coronavirus-aerosols-indoors.html}{emerges}
that respiratory droplets containing live virus may linger in the air in
indoor spaces. To date, no super-spreading events have been traced to a
flight.

German researchers recently published a
\href{https://jamanetwork.com/journals/jamanetworkopen/fullarticle/2769383}{study}
in the JAMA Network on a group of 24 tourists in March who were
unwittingly exposed to Covid-19 in Israel a week before flying to
Frankfurt on a four-hour-and-40-minute flight carrying 102 passengers.
They found two likely cases of virus transmission on the flight, both
seated within two rows of an infected passenger. Notably, no one was
wearing masks on this flight, which took place before that public health
mandate was adopted by airlines beginning in May in the United States.

On June 8, when The New York Times
\href{https://www.nytimes3xbfgragh.onion/interactive/2020/06/08/upshot/when-epidemiologists-will-do-everyday-things-coronavirus.html}{surveyed}
511 epidemiologists about when they would travel again by airplane, the
largest contingent, 44 percent, said in three to 12 months. They deemed
other activities, including attending a sporting event, concert, funeral
or wedding, as riskier.

The question of whether a middle seat left open would improve a
passenger's odds of not getting sick compelled Arnold Barnett, a
statistician and professor of management science at Massachusetts
Institute of Technology, to look into it.

``United said it was a P.R. strategy and not about safety, whereas Delta
went to the other extreme,'' he said.

His mathematical model multiplied the number of Covid-19 cases by 10,
based on the Centers for Disease Control and Prevention's estimate that
the number of undetected cases could be 10 times the known infection
rate. He factored in the barrier created by seat backs, the in-flight
air purification systems and the effectiveness of masks in stopping
contagion, which he said was about a four-in-five chance.

He found that on a fully loaded flight, the chance of contracting
Covid-19 was one in 4,300. If the middle seat is empty, the risk falls
to one in 7,700. Taking into account the possibility of spreading the
virus to others not on the plane, he estimated the death risk to be one
per 400,000 passengers on full flights and one in 600,000 with open
middle seats.

These risks are considerably higher, he said, than the risk of a plane
crash, but comparable to risks associated with two hours of everyday
activities --- for instance, grocery shopping --- during the pandemic.

``United said there's no such thing as social distancing on airplanes,''
Mr. Barnett said, granting that an open seat doesn't give you the
recommended six feet of social distancing.

``I do think there's a difference, and I would rather fly the airlines
that are being more cautious in this regard,'' he added.

\begin{center}\rule{0.5\linewidth}{\linethickness}\end{center}

\emph{\textbf{Follow New York Times Travel}} \emph{on}
\href{https://www.instagram.com/nytimestravel/}{\emph{Instagram}}\emph{,}
\href{https://twitter.com/nytimestravel}{\emph{Twitter}} \emph{and}
\href{https://www.facebookcorewwwi.onion/nytimestravel/}{\emph{Facebook}}\emph{.
And}
\href{https://www.nytimes3xbfgragh.onion/newsletters/traveldispatch}{\emph{sign
up for our weekly Travel Dispatch newsletter}} \emph{to receive expert
tips on traveling smarter and inspiration for your next vacation.}

Advertisement

\protect\hyperlink{after-bottom}{Continue reading the main story}

\hypertarget{site-index}{%
\subsection{Site Index}\label{site-index}}

\hypertarget{site-information-navigation}{%
\subsection{Site Information
Navigation}\label{site-information-navigation}}

\begin{itemize}
\tightlist
\item
  \href{https://help.nytimes3xbfgragh.onion/hc/en-us/articles/115014792127-Copyright-notice}{©~2020~The
  New York Times Company}
\end{itemize}

\begin{itemize}
\tightlist
\item
  \href{https://www.nytco.com/}{NYTCo}
\item
  \href{https://help.nytimes3xbfgragh.onion/hc/en-us/articles/115015385887-Contact-Us}{Contact
  Us}
\item
  \href{https://www.nytco.com/careers/}{Work with us}
\item
  \href{https://nytmediakit.com/}{Advertise}
\item
  \href{http://www.tbrandstudio.com/}{T Brand Studio}
\item
  \href{https://www.nytimes3xbfgragh.onion/privacy/cookie-policy\#how-do-i-manage-trackers}{Your
  Ad Choices}
\item
  \href{https://www.nytimes3xbfgragh.onion/privacy}{Privacy}
\item
  \href{https://help.nytimes3xbfgragh.onion/hc/en-us/articles/115014893428-Terms-of-service}{Terms
  of Service}
\item
  \href{https://help.nytimes3xbfgragh.onion/hc/en-us/articles/115014893968-Terms-of-sale}{Terms
  of Sale}
\item
  \href{https://spiderbites.nytimes3xbfgragh.onion}{Site Map}
\item
  \href{https://help.nytimes3xbfgragh.onion/hc/en-us}{Help}
\item
  \href{https://www.nytimes3xbfgragh.onion/subscription?campaignId=37WXW}{Subscriptions}
\end{itemize}
