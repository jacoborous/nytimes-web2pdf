Sections

SEARCH

\protect\hyperlink{site-content}{Skip to
content}\protect\hyperlink{site-index}{Skip to site index}

\href{https://www.nytimes3xbfgragh.onion/section/nyregion}{New York}

\href{https://myaccount.nytimes3xbfgragh.onion/auth/login?response_type=cookie\&client_id=vi}{}

\href{https://www.nytimes3xbfgragh.onion/section/todayspaper}{Today's
Paper}

\href{/section/nyregion}{New York}\textbar{}M.T.A. Warns of Doomsday
Subway Cuts Without \$12 Billion in Federal Aid

\url{https://nyti.ms/3gvfhUI}

\begin{itemize}
\item
\item
\item
\item
\item
\end{itemize}

\hypertarget{the-coronavirus-outbreak}{%
\subsubsection{\texorpdfstring{\href{https://www.nytimes3xbfgragh.onion/news-event/coronavirus?name=styln-coronavirus-national\&region=TOP_BANNER\&block=storyline_menu_recirc\&action=click\&pgtype=Article\&impression_id=ab260d30-f4bf-11ea-946f-8d50c943ff29\&variant=undefined}{The
Coronavirus
Outbreak}}{The Coronavirus Outbreak}}\label{the-coronavirus-outbreak}}

\begin{itemize}
\tightlist
\item
  live\href{https://www.nytimes3xbfgragh.onion/2020/09/11/world/covid-19-coronavirus.html?name=styln-coronavirus-national\&region=TOP_BANNER\&block=storyline_menu_recirc\&action=click\&pgtype=Article\&impression_id=ab260d31-f4bf-11ea-946f-8d50c943ff29\&variant=undefined}{Latest
  Updates}
\item
  \href{https://www.nytimes3xbfgragh.onion/interactive/2020/us/coronavirus-us-cases.html?name=styln-coronavirus-national\&region=TOP_BANNER\&block=storyline_menu_recirc\&action=click\&pgtype=Article\&impression_id=ab263440-f4bf-11ea-946f-8d50c943ff29\&variant=undefined}{Maps
  and Cases}
\item
  \href{https://www.nytimes3xbfgragh.onion/interactive/2020/science/coronavirus-vaccine-tracker.html?name=styln-coronavirus-national\&region=TOP_BANNER\&block=storyline_menu_recirc\&action=click\&pgtype=Article\&impression_id=ab263441-f4bf-11ea-946f-8d50c943ff29\&variant=undefined}{Vaccine
  Tracker}
\item
  \href{https://www.nytimes3xbfgragh.onion/2020/09/10/us/politics/fda-coronavirus-vaccine.html?name=styln-coronavirus-national\&region=TOP_BANNER\&block=storyline_menu_recirc\&action=click\&pgtype=Article\&impression_id=ab263442-f4bf-11ea-946f-8d50c943ff29\&variant=undefined}{F.D.A.
  Regulators' Self-Defense}
\item
  \href{https://www.nytimes3xbfgragh.onion/2020/09/09/upshot/coronavirus-surprise-test-fees.html?name=styln-coronavirus-national\&region=TOP_BANNER\&block=storyline_menu_recirc\&action=click\&pgtype=Article\&impression_id=ab263443-f4bf-11ea-946f-8d50c943ff29\&variant=undefined}{Surprise
  Test Fees}
\end{itemize}

Advertisement

\protect\hyperlink{after-top}{Continue reading the main story}

Supported by

\protect\hyperlink{after-sponsor}{Continue reading the main story}

\hypertarget{mta-warns-of-doomsday-subway-cuts-without-12-billion-in-federal-aid}{%
\section{M.T.A. Warns of Doomsday Subway Cuts Without \$12 Billion in
Federal
Aid}\label{mta-warns-of-doomsday-subway-cuts-without-12-billion-in-federal-aid}}

The agency, facing staggering financial losses because of the pandemic,
said it would have to reduce subway and bus service by 40 percent.

\includegraphics{https://static01.graylady3jvrrxbe.onion/images/2020/08/26/nyregion/26subway1/merlin_176122542_850a2b22-123b-4cad-b9e8-6aa1ccb71da9-articleLarge.jpg?quality=75\&auto=webp\&disable=upscale}

\href{https://www.nytimes3xbfgragh.onion/by/christina-goldbaum}{\includegraphics{https://static01.graylady3jvrrxbe.onion/images/2019/11/22/reader-center/author-christina-goldbaum/author-christina-goldbaum-thumbLarge.png}}

By
\href{https://www.nytimes3xbfgragh.onion/by/christina-goldbaum}{Christina
Goldbaum}

\begin{itemize}
\item
  Aug. 26, 2020
\item
  \begin{itemize}
  \item
  \item
  \item
  \item
  \item
  \end{itemize}
\end{itemize}

Facing a staggering financial crisis and a stalemate in Washington, the
Metropolitan Transportation Authority threatened on Wednesday to adopt a
doomsday plan if it did not receive as much as \$12 billion in federal
aid, including slashing subway and bus service in New York City by 40
percent.

The plan paints a bleak picture for riders: Wait times would increase by
eight minutes on the subway and 15 minutes on buses; Long Island Rail
Road and Metro-North trains would run at 60- or 120-minute intervals.
Upgrades to the subway's signal systems, which have been the source of
many delays, would be scrapped.

The M.T.A. --- which runs the city's subway, buses and two commuter
rails --- laid out the plan as part of a broader political strategy
\href{https://www.nytimes3xbfgragh.onion/2020/07/21/nyregion/mta-subway-financial-cuts.html}{to
pressure Washington} to
\href{https://www.nytimes3xbfgragh.onion/2020/04/20/nyregion/nyc-mta-subway-coronavirus.html}{provide
assistance}.

The agency is facing a staggering \$16.2 billion deficit through 2024,
after the coronavirus pandemic wiped out its operating revenue --- which
comes from fares, tolls and subsidies --- virtually overnight.
\href{https://www.nytimes3xbfgragh.onion/2020/03/24/nyregion/coronavirus-nyc-mta-cuts-.html}{Ridership
on the subway}, which plummeted by 90 percent in April, has only reached
a quarter of usual levels, even as more and more New Yorkers return to
work.

The state-run transit agency has requested \$12 billion in aid to cover
its operating losses through 2021. But after negotiations over the next
stimulus package stalled this month, immediate federal support did not
appear to be forthcoming.

``The future of the M.T.A. and the future of the New York region lies
squarely in the hands of the federal government,'' the authority's
chairman, Patrick J. Foye, said on Wednesday. ``Without this additional
federal funding, we will be forced to take draconian measures, the
impact of which will be felt across the system and the region for
decades to come.''

The plan, which transit officials have said the agency would not enact
before next year, is the first detailed depiction of what the sprawling
public transportation network and backbone of the New York region's
economy could look like in the wake of the pandemic and the current
financial crisis.

Hallmark infrastructure projects, like extending the Second Avenue
Subway into Harlem and connecting commuter trains to Manhattan's west
side at Pennsylvania Station, would be paused indefinitely. Purchasing a
new fleet of electric buses and new subway cars, as well as adding
elevators to stations to make them more accessible, would also be
indefinitely delayed.

In addition, the agency would eliminate a widely hailed program that
provides accessible vehicles on demand to paratransit customers. Fares
and tolls would be raised by one percent and one dollar, respectively,
above already scheduled increases in 2021 and 2023.

Talks between Democrats and administration officials over the details of
another coronavirus relief package stalled earlier this month and have
remained in a political stalemate, with negotiators unable to reach an
agreement on the overall scope and price tag of the package.

\hypertarget{latest-updates-the-coronavirus-outbreak}{%
\section{\texorpdfstring{\href{https://www.nytimes3xbfgragh.onion/2020/09/11/world/covid-19-coronavirus.html?action=click\&pgtype=Article\&state=default\&region=MAIN_CONTENT_1\&context=storylines_live_updates}{Latest
Updates: The Coronavirus
Outbreak}}{Latest Updates: The Coronavirus Outbreak}}\label{latest-updates-the-coronavirus-outbreak}}

Updated 2020-09-12T06:16:33.399Z

\begin{itemize}
\tightlist
\item
  \href{https://www.nytimes3xbfgragh.onion/2020/09/11/world/covid-19-coronavirus.html?action=click\&pgtype=Article\&state=default\&region=MAIN_CONTENT_1\&context=storylines_live_updates\#link-dfb8a16}{Fauci
  cautions the virus could disrupt life in the U.S. until `maybe even
  towards the end of 2021.'}
\item
  \href{https://www.nytimes3xbfgragh.onion/2020/09/11/world/covid-19-coronavirus.html?action=click\&pgtype=Article\&state=default\&region=MAIN_CONTENT_1\&context=storylines_live_updates\#link-7104d154}{From
  Asia to Africa, China promotes its vaccine candidates to win friends.}
\item
  \href{https://www.nytimes3xbfgragh.onion/2020/09/11/world/covid-19-coronavirus.html?action=click\&pgtype=Article\&state=default\&region=MAIN_CONTENT_1\&context=storylines_live_updates\#link-393ad215}{The
  other way the virus will kill: hunger.}
\end{itemize}

\href{https://www.nytimes3xbfgragh.onion/2020/09/11/world/covid-19-coronavirus.html?action=click\&pgtype=Article\&state=default\&region=MAIN_CONTENT_1\&context=storylines_live_updates}{See
more updates}

More live coverage:
\href{https://www.nytimes3xbfgragh.onion/live/2020/09/11/business/stock-market-today-coronavirus?action=click\&pgtype=Article\&state=default\&region=MAIN_CONTENT_1\&context=storylines_live_updates}{Markets}

The \$3.4 trillion legislation approved by House Democrats in May
included \$15.75 billion in grants for transit agencies, while the \$1
trillion offer introduced by Senate Republicans last month did not
include a similar allocation in aid. It was unclear how much of that
would go to the M.T.A.

A spokesman for Senator Chuck Schumer, the minority leader and a New
York Democrat, said that the Senator is fighting for robust funding for
the transit authority in the current round of negotiations. In March,
Democratic leaders successfully secured \$3.9 billion for the M.T.A. in
the first federal stimulus package.

Without funding, though, transit advocates warned that riders would feel
the effects of the proposed cuts for decades.

``Something like this, it would fundamentally change New York,'' said
Nick Sifuentes, the executive director of Tri-State Transportation
Campaign, an advocacy group. ``It would kick off the death spiral of
people using anything other than public transit and then transit funding
would never recover.''

The Transport Workers Union Local 100, which represents many M.T.A.
workers, also decried the possibility of slashing the work force.

``Transit workers put this city and state on their backs and carried
them through the deadly pandemic, risking their own health and lives,''
the union's president, Tony Utano, said in a statement. ``Layoffs would
be an unimaginable shameful betrayal.''

Making any of the drastic cuts would come at a high political cost to
Gov. Andrew M. Cuomo, who controls the M.T.A. and has in recent years
positioned himself as the only one capable of saving its subway system.

But framing possible cuts to service and fare increases as dependent on
the federal authorities may lay the groundwork for blaming Washington if
New York officials are forced to make unpopular decisions.

``Only the federal government can come to the rescue of the M.T.A. I
think it's important we recognize that,'' said Mr. Foye, the authority
chairman.

At a news conference on Wednesday, Mr. Cuomo said that the state's own
financial crisis hampers its ability to assist the M.T.A. if federal aid
is not granted to the transit system and the state at large.

``It would be a financial catastrophe for the State of New York. There
wouldn't be one hole in the dike, called the M.T.A., there would be 50
holes in the dike,'' he said.

Earlier this month, the agency borrowed \$451 million from the Federal
Reserve, becoming only the second state government borrower to use the
program. The M.T.A. has also made some initial cuts to nonessential
services, including reducing overtime and eliminating consulting
contracts, that will save the agency \$540 million next year.

Still, in recent weeks the M.T.A. has come under fire for not doing more
to shore up its finances.

Financial experts and some state lawmakers have said that delaying
cost-saving measures deepens the agency's budget hole, which worsens by
\$200 million every week, and deferring conversations with legislators
about possible new revenue streams delays earnings from them.

This, they say, risks plunging the system into an even more dire crisis
in the years to come.

``The governor and the M.T.A. seemingly think they can keep the M.T.A.
on ice permanently and that it will then just wake up one day,'' said
Assemblyman Robert Carroll, a Democrat whose district covers part of
Brooklyn.

Adding to the urgency, many observers say it is unlikely that the
authority will receive as much as \$12 billion from federal authorities
in the next stimulus package. The M.T.A. had originally asked the
authorities
\href{https://www.nytimes3xbfgragh.onion/2020/03/17/nyregion/coronavirus-nyc-subway-federal-aid-.html}{for
\$3.9 billion in April} to cover its operating losses through the end of
this year, but now says it needs \$12 billion to cover its deficit
through the end of next year.

For comparison, the American Public Transportation Association, a
lobbying group, has called for a total of \$32 billion in the next
stimulus package for the country's transit agencies.

Already,
\href{https://www.nytimes3xbfgragh.onion/2020/07/19/us/coronavirus-public-transit.html}{other
transit agencies} around the country have begun slashing service and
furloughing their work forces. But in New York, officials have hesitated
taking any cost-saving steps that would immediately affect riders.

Historically, the agency has dug its way out of crises by cutting
service, slashing fares, acquiring new state subsidies and taking on
more debt.

But in this crisis, slashing service would lead to more packed trains,
which would risk contributing to the virus's spread and also discourage
riders from returning to the system. Raising fares when ridership hovers
at around 25 percent of usual would yield little new revenue and burden
the lower-income New Yorkers and essential workers who make up many of
the current commuters.

Meanwhile, the state and city are facing their own fiscal emergencies,
and debt repayment already consumes nearly a quarter of the agency's
operating budget.

``The M.T.A. needs this huge helping of federal emergency aid to get
through the now,'' said John Kaehny, executive director of Reinvent
Albany, a watchdog group. ``But the future looks grim even with the
federal aid. That's the problem.''

``It's a transit agency that has lurched from crisis to crisis,'' he
added. ``Now it's paying for decades of not restructuring where the
M.T.A. gets its operating revenue.''

Emily Cochrane contributed reporting.

Advertisement

\protect\hyperlink{after-bottom}{Continue reading the main story}

\hypertarget{site-index}{%
\subsection{Site Index}\label{site-index}}

\hypertarget{site-information-navigation}{%
\subsection{Site Information
Navigation}\label{site-information-navigation}}

\begin{itemize}
\tightlist
\item
  \href{https://help.nytimes3xbfgragh.onion/hc/en-us/articles/115014792127-Copyright-notice}{©~2020~The
  New York Times Company}
\end{itemize}

\begin{itemize}
\tightlist
\item
  \href{https://www.nytco.com/}{NYTCo}
\item
  \href{https://help.nytimes3xbfgragh.onion/hc/en-us/articles/115015385887-Contact-Us}{Contact
  Us}
\item
  \href{https://www.nytco.com/careers/}{Work with us}
\item
  \href{https://nytmediakit.com/}{Advertise}
\item
  \href{http://www.tbrandstudio.com/}{T Brand Studio}
\item
  \href{https://www.nytimes3xbfgragh.onion/privacy/cookie-policy\#how-do-i-manage-trackers}{Your
  Ad Choices}
\item
  \href{https://www.nytimes3xbfgragh.onion/privacy}{Privacy}
\item
  \href{https://help.nytimes3xbfgragh.onion/hc/en-us/articles/115014893428-Terms-of-service}{Terms
  of Service}
\item
  \href{https://help.nytimes3xbfgragh.onion/hc/en-us/articles/115014893968-Terms-of-sale}{Terms
  of Sale}
\item
  \href{https://spiderbites.nytimes3xbfgragh.onion}{Site Map}
\item
  \href{https://help.nytimes3xbfgragh.onion/hc/en-us}{Help}
\item
  \href{https://www.nytimes3xbfgragh.onion/subscription?campaignId=37WXW}{Subscriptions}
\end{itemize}
