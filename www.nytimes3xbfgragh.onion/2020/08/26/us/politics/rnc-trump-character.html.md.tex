Sections

SEARCH

\protect\hyperlink{site-content}{Skip to
content}\protect\hyperlink{site-index}{Skip to site index}

\href{https://www.nytimes3xbfgragh.onion/section/politics}{Politics}

\href{https://myaccount.nytimes3xbfgragh.onion/auth/login?response_type=cookie\&client_id=vi}{}

\href{https://www.nytimes3xbfgragh.onion/section/todayspaper}{Today's
Paper}

\href{/section/politics}{Politics}\textbar{}How Trump's Convention Has
Become a Crucial Play for the Suburbs

\url{https://nyti.ms/31uwRnu}

\begin{itemize}
\item
\item
\item
\item
\item
\end{itemize}

\begin{itemize}
\item
  \href{https://www.nytimes3xbfgragh.onion/live/2020/09/11/us/trump-vs-biden?action=click\&pgtype=Article\&state=default\&region=TOP_BANNER\&context=storylines_menu}{Election
  Updates}
\item
  \href{https://www.nytimes3xbfgragh.onion/interactive/2020/us/elections/election-states-biden-trump.html?action=click\&pgtype=Article\&state=default\&region=TOP_BANNER\&context=storylines_menu}{Paths
  to 270}
\item
  \href{https://www.nytimes3xbfgragh.onion/interactive/2019/us/elections/2020-presidential-election-calendar.html?action=click\&pgtype=Article\&state=default\&region=TOP_BANNER\&context=storylines_menu}{Key
  Dates}
\item
  \href{https://www.nytimes3xbfgragh.onion/interactive/2020/08/31/us/politics/vote-by-mail-deadlines.html?action=click\&pgtype=Article\&state=default\&region=TOP_BANNER\&context=storylines_menu}{Voting
  by Mail}
\item
  \href{https://www.nytimes3xbfgragh.onion/newsletters/politics?action=click\&pgtype=Article\&state=default\&region=TOP_BANNER\&context=storylines_menu}{Politics
  Newsletter}
\end{itemize}

Advertisement

\protect\hyperlink{after-top}{Continue reading the main story}

Supported by

\protect\hyperlink{after-sponsor}{Continue reading the main story}

News Analysis

\hypertarget{how-trumps-convention-has-become-a-crucial-play-for-the-suburbs}{%
\section{How Trump's Convention Has Become a Crucial Play for the
Suburbs}\label{how-trumps-convention-has-become-a-crucial-play-for-the-suburbs}}

Conciliatory messages at the Republican convention were an
acknowledgment by the president's campaign that appealing to his
right-wing base will not be enough to win re-election.

\includegraphics{https://static01.graylady3jvrrxbe.onion/images/2020/08/26/us/politics/26trump-image1/merlin_176179707_52448aba-5df8-46c8-ab44-82ed8fdfd86f-articleLarge.jpg?quality=75\&auto=webp\&disable=upscale}

\href{https://www.nytimes3xbfgragh.onion/by/jeremy-w-peters}{\includegraphics{https://static01.graylady3jvrrxbe.onion/images/2018/11/06/multimedia/author-jeremy-w-peters/author-jeremy-w-peters-thumbLarge.png}}\href{https://www.nytimes3xbfgragh.onion/by/annie-karni}{\includegraphics{https://static01.graylady3jvrrxbe.onion/images/2019/02/05/multimedia/author-annie-karni/author-annie-karni-thumbLarge.png}}\href{https://www.nytimes3xbfgragh.onion/by/nick-corasaniti}{\includegraphics{https://static01.graylady3jvrrxbe.onion/images/2018/06/13/multimedia/author-nick-corasaniti/author-nick-corasaniti-thumbLarge-v2.png}}

By \href{https://www.nytimes3xbfgragh.onion/by/jeremy-w-peters}{Jeremy
W. Peters},
\href{https://www.nytimes3xbfgragh.onion/by/annie-karni}{Annie Karni}
and \href{https://www.nytimes3xbfgragh.onion/by/nick-corasaniti}{Nick
Corasaniti}

\begin{itemize}
\item
  Aug. 26, 2020
\item
  \begin{itemize}
  \item
  \item
  \item
  \item
  \item
  \end{itemize}
\end{itemize}

\href{https://www.nytimes3xbfgragh.onion/es/2020/08/27/espanol/estados-unidos/trump-convencion-republicana.html}{Leer
en español}

It is one of President Trump's clearest paths to re-election: winning
back the suburbs in a handful of swing states that drifted from the
Republican Party in the 2018 midterms. And that imperative has been
vividly apparent each night of the party's national convention, with
speakers and videos that are trying to recast Mr. Trump's divisive
record, which had hurt the G.O.P. two years ago.

There have been glowing personal tributes from women, scenes of friendly
banter between Mr. Trump and immigrants and a Black family, and stories
from people he reached out to in times of despair. If all political
conventions cast their candidates in the best possible light, the
Republican National Convention has been going all-out for bright and
sunny.

But it was also an acknowledgment by the president's campaign that
appealing to his right-wing base will not be enough to win re-election,
and that voters who have soured on him after three and a half years are
not responding to a strategy that leans heavily into attacking his
opponent, Joseph R. Biden Jr., and other Democrats as radicals and
extremists.

Instead of sustained attacks on Mr. Biden, Wednesday night featured
personal speeches from a trio of female Trump aides, who pointed to
their own experiences of Mr. Trump, describing a version of him that is
rarely seen in public and effectively asking voters to take their word
for it.

``I have seen firsthand, many times the President comforting and
encouraging a child who has lost a parent, a parent who has lost a
child,'' said Kellyanne Conway, the outgoing White House adviser.
Kayleigh McEnany, the White House press secretary, talked about getting
a personal call from Mr. Trump after undergoing a preventive mastectomy.
Lara Trump, the president's daughter-in-law and a campaign adviser,
described the Trump family as ``warm and caring'' and ``down to earth.''

Trump advisers said on Wednesday that they did not intend to change
people's minds about the president. Voter opinions about him have been
remarkably impervious to the good and bad news about him, fluctuating
little since he took office. Rather, the aides said, they were seeking
to remind suburban voters of policies Mr. Trump has supported --- like
granting citizenship for legal immigrants and reducing harsh criminal
statutes --- that will give them something to hang onto in the voting
booth in November.

In 2016,
\href{https://www.cnn.com/election/2016/results/exit-polls}{exit polls}
showed Mr. Trump winning suburban areas, 49 percent to 45 percent,
helping to offset his deep deficit among city voters. By 2018's midterm
elections, Democrats had caught up: Each party captured 49 percent of
votes cast in the suburbs in House races that year,
\href{https://www.cnn.com/election/2018/exit-polls}{according to exit
polls}.

Now, Mr. Trump's job approval is worse among suburbanites than even
among city dwellers. Sixty-one percent of suburban voters disapproved of
his job performance while just 38 percent approved, according to a
\href{https://static.foxnews.com/foxnews.com/content/uploads/2020/08/Fox_August-9-12-2020_Complete_National_Topline_August-13-Release.pdf}{Fox
News poll} this month. Among suburban women in particular, Mr. Trump's
net approval rating was only 34 percent.

Rarely, if ever, have political image makers succeeded in scaffolding
over the most blemished parts of a presidential candidate's record
during a few nights of prime time programming. And no campaign has
attempted that feat with a candidate like Mr. Trump, whose entire
political persona has been built on playing into white fears of
immigrants and minorities --- beginning with his campaign announcement
five years ago warning of Mexican immigrants: ``They`re bringing drugs.
They're bringing crime. They're rapists.''

``A couple nice video clips and speeches from people of color in a
convention isn't going to do it because these voters know who Donald
Trump is,'' said Patrick Murray, the director of the Monmouth University
Polling Institute. ``It's going to be what happens in the next two
months with his actions and his off-the-cuff rhetoric.''

To prevail in November, Mr. Trump will need to improve his performance
in swing states like Wisconsin, Pennsylvania and North Carolina where
his appeal with white women helped him win in 2016. Over all,
\href{https://www.nytimes3xbfgragh.onion/2016/12/01/us/politics/white-women-helped-elect-donald-trump.html}{he
received} the support of 53 percent of white women, including 51 percent
of those with college degrees.

Sarah Longwell, a longtime Republican strategist who opposes Mr. Trump,
said the decision to show a version of him that most Americans don't see
was right, insofar as voters were looking for something to point to in
order to justify their support for a president who has encouraged racist
conspiracy theories, lashed out at women and repeatedly insulted the
intelligence of his Black critics.

``If there are people who are looking for permission to vote for him,''
Ms. Longwell said, ``it does give them something to point to.''

On television advertising, the Biden campaign has vastly outspent the
Trump campaign, with \$57.7 million on television in the month of August
compared to \$24.5 million by the Trump campaign. On Tuesday, the Trump
campaign pulled down all their broadcast ads, and have none scheduled to
air until Sept. 8; the campaign pledged to return to broadcast ``well
before'' that date, and it still has a national cable presence.

On Facebook, the Trump campaign and allied committees have spent \$22.8
million, and the Biden campaign has spent \$17.7 million in August.

But a monthslong ad campaign seeking to sow fear in the suburbs, using
selectively edited scenes to exaggerate violence from the summer protest
movement, has done little to win back the suburban voters that
Republicans lost in 2018, which cost them control of the House of
Representatives.

``We haven't seen anything that has shifted from what we saw in the blue
wave in 2018 where white college-educated women in the suburbs in
particular had enough of his caustic approach,'' Mr. Murray said.

\includegraphics{https://static01.graylady3jvrrxbe.onion/images/2020/08/26/us/politics/26trump-image2/merlin_175832241_24dc0d98-c997-44dc-b62f-ab7f16007c7c-articleLarge.jpg?quality=75\&auto=webp\&disable=upscale}

Before the convention, the Trump campaign was running a trio of ads that
depicted American cities under siege, warping momentary scenes of
violence from largely peaceful protests earlier this year into scenes of
chaos. One included a staged scene of a break-in at a senior citizens
home.

If the softer focus on Mr. Trump at the convention does not square with
the ads that his campaign has been producing, it was also incongruous
with the messaging at most other points during the convention. Mr. Trump
has given top billing to some of the most provocative defenders of his
style of politics. The result has been a program that can seem
discordant --- with one segment featuring the activist Charlie Kirk
declaring Mr. Trump ``the bodyguard of Western civilization'' who is
protecting Americans from ``bitter, deceitful, vengeful activists,'' and
in the next, a video with two millennial Latina women praising Mr. Trump
for the federal loan that kept their small business afloat during the
pandemic lockdown.

Interviews with several voters in swing states on Wednesday found
skepticism toward the convention's portrayal of Mr. Trump.

``I'm certainly aware he's trying to win back people he's lost,'' said
Maureen Thomas, 61, a resident of suburban Detroit who voted for the
Republican nominee in 2012, Mitt Romney, and now supports Mr. Biden. Ms.
Thomas, a retired lawyer, found the president's presiding over a
naturalization ceremony on Tuesday night, after years of hostility to
immigrants, ``fake, false, a show.''

Jeffrey Timlin, 26, a registered Republican in Montgomery County outside
Philadelphia, said that the convention's portrait of Mr. Trump ``just
doesn't feel authentic.'' Mr. Timlin, an engineer, plans to vote for Mr.
Biden, but said he is jaded about both candidates and their parties.

``I think that the idea of changing and putting Biden in would take at
least a little away from this big public smoke screen that has been the
Trump presidency,'' he said.

The image of Mr. Trump at the convention is a far cry from the president
who has spent his first term focused on strengthening his relationship
with his conservative base of support. For Americans who hoped Mr. Trump
would become the leader he vowed to be in his victory speech in 2016 ---
``I will be a president for all Americans,'' he declared --- his record
and priorities in office rarely reflect that pledge.

``That is the first and most important test of leadership that he failed
immediately and has failed every day since,'' said Carly Fiorina, the
former Hewlett-Packard chief executive and Republican who ran against
Mr. Trump in 2016. She is now supporting Mr. Biden, and encouraging
other Republicans to do so.

But Ms. Fiorina cautioned that Democrats should not assume that swing
voters will tune out Mr. Trump's appeals just because they have a
problem with the messenger.

``Most Americans don't believe that Trump is `the guardian of Western
civilization' for heaven's sake,'' Ms. Fiorina said. ``But I believe
that Biden and the Democratic Party, in order to win, need to keep their
eyes on where the majority of Americans are.''

And Republicans do see openings where they believe that Democrats are
vulnerable. For instance, if eruptions of violence in cities like
Kenosha, Wis., continue or worsen, Republicans believe there are
suburban voters who will blame Democrats, even those who say they are
with the majority of Americans who support the recent demonstrations
against racial injustice.

What many Republicans say they find frustrating is the way Mr. Trump's
provocations and outbursts tend to obscure the disagreement they have
been trying to raise about the Democratic Party's leadership and
policies, like the push from some on the party's left to ``defund''
police departments. Despite the soft-sell approach that some speakers
have taken during the first two nights of the convention, it is the
loudest voices --- the ones that mimic Mr. Trump's attacks --- that
often break through most memorably.

The Trump campaign is trying to give the milder moments a longer shelf
life, quickly cutting them up into ads and sending them out across the
Trump campaign's vast digital messaging operation. By Wednesday morning,
the campaign had clipped scenes
\href{https://www.facebookcorewwwi.onion/ads/library/?id=611652599723644}{from
the pardon of Joe Ponder}, a convicted bank robber, and
\href{https://www.facebookcorewwwi.onion/ads/library/?id=975219909621168}{Melania
Trump's Tuesday night speech}, turning them into Facebook ads.

The Biden campaign, for its part, has been spending heavily to try to
tamp down any potential convention bump for Mr. Trump. While the Trump
campaign is dark on broadcast, the Biden campaign will spend about \$20
million on television ads in battleground states.

But the post-convention glow around any candidate has traditionally
lasted only as long as the candidate has stayed on message. And that is
never a guarantee with Mr. Trump.

``In 1988, they thought that Dukakis had a great convention. In 2004,
everyone agreed that John Kerry had a good convention,'' said Russ
Schriefer, a Republican consultant who has worked on conventions for
George W. Bush and Mitt Romney.

``Having a good convention,'' he added, ``isn't in itself a great
predictor of winning.''

\emph{Trip Gabriel and Giovanni Russonello contributed reporting.}

\hypertarget{our-2020-election-guide}{%
\section{Our 2020 Election Guide}\label{our-2020-election-guide}}

Updated ~Sept. 11, 2020

\begin{center}\rule{0.5\linewidth}{\linethickness}\end{center}

\begin{itemize}
\item ~
  \hypertarget{the-latest}{%
  \subsection{The Latest}\label{the-latest}}

  \begin{itemize}
  \item
    Joe Biden and President Trump put
    \href{https://www.nytimes3xbfgragh.onion/2020/09/11/us/politics/shanksville-trump-biden.html?action=click\&pgtype=Article\&state=default\&region=BELOW_MAIN_CONTENT\&context=storylines_guide}{hostilities
    on hold today to travel to ground zero and then to Shanksville, Pa.,
    where they separately honored 9/11 victims}.
  \end{itemize}
\item ~
  \hypertarget{how-to-win-270}{%
  \subsection{How to Win 270}\label{how-to-win-270}}

  \begin{itemize}
  \item
    Joe Biden and Donald Trump need 270 electoral votes to reach the
    White House. Try building
    \href{https://www.nytimes3xbfgragh.onion/interactive/2020/us/elections/election-states-biden-trump.html?action=click\&pgtype=Article\&state=default\&region=BELOW_MAIN_CONTENT\&context=storylines_guide}{your
    own coalition of battleground states}~to see potential outcomes.
  \end{itemize}
\item ~
  \hypertarget{voting-by-mail}{%
  \subsection{Voting by Mail}\label{voting-by-mail}}

  \begin{itemize}
  \item
    Will you have enough time to vote by mail in your state? Yes, but
    it's risky to procrastinate.
    \href{https://www.nytimes3xbfgragh.onion/interactive/2020/08/31/us/politics/vote-by-mail-deadlines.html?action=click\&pgtype=Article\&state=default\&region=BELOW_MAIN_CONTENT\&context=storylines_guide}{Check
    your state's deadline.}
  \item
    \href{https://www.nytimes3xbfgragh.onion/interactive/2020/us/elections/joe-biden.html?action=click\&pgtype=Article\&state=default\&region=BELOW_MAIN_CONTENT\&context=storylines_guide}{}

    \hypertarget{joe-biden}{%
    \section{Joe Biden}\label{joe-biden}}

    \hypertarget{democrat}{%
    \subsection{Democrat}\label{democrat}}

    \href{https://www.nytimes3xbfgragh.onion/interactive/2020/us/elections/donald-trump.html?action=click\&pgtype=Article\&state=default\&region=BELOW_MAIN_CONTENT\&context=storylines_guide}{}

    \hypertarget{donald-trump}{%
    \section{Donald Trump}\label{donald-trump}}

    \hypertarget{republican}{%
    \subsection{Republican}\label{republican}}
  \end{itemize}
\item
  \hypertarget{keep-up-with-our-coverage}{%
  \subsection{Keep Up With Our
  Coverage}\label{keep-up-with-our-coverage}}

  \begin{itemize}
  \item
    Get an
    \href{https://www.nytimes3xbfgragh.onion/newsletters/politics?action=click\&pgtype=Article\&state=default\&region=BELOW_MAIN_CONTENT\&context=storylines_guide}{email}~recapping
    the day's news
  \item
    Download our mobile app on
    \href{https://apps.apple.com/us/app/nytimes/id284862083?ls=1\&mat_click_id=5c79ae7455014fd1bd66b5610c05b8f2-20191112-16948\&referrer=mat_click_id\%3D5c79ae7455014fd1bd66b5610c05b8f2-20191112-16948\%26link_click_id\%3D722930677036718082}{iOS}~and
    \href{http://a.localytics.com/android?id=com.nytimes.android\&referrer=utm_source\%3Dother_nyt_mobile_web\%26utm_medium\%3DWeb\%2520page\%26utm_term\%3DGeneral\%2520Mobile\%2520Page\%26utm_campaign\%3DNYT\%2520Mobile\%2520General\%2520Page}{Android}~and
    turn on Breaking News and Politics alerts
  \end{itemize}
\end{itemize}

Advertisement

\protect\hyperlink{after-bottom}{Continue reading the main story}

\hypertarget{site-index}{%
\subsection{Site Index}\label{site-index}}

\hypertarget{site-information-navigation}{%
\subsection{Site Information
Navigation}\label{site-information-navigation}}

\begin{itemize}
\tightlist
\item
  \href{https://help.nytimes3xbfgragh.onion/hc/en-us/articles/115014792127-Copyright-notice}{©~2020~The
  New York Times Company}
\end{itemize}

\begin{itemize}
\tightlist
\item
  \href{https://www.nytco.com/}{NYTCo}
\item
  \href{https://help.nytimes3xbfgragh.onion/hc/en-us/articles/115015385887-Contact-Us}{Contact
  Us}
\item
  \href{https://www.nytco.com/careers/}{Work with us}
\item
  \href{https://nytmediakit.com/}{Advertise}
\item
  \href{http://www.tbrandstudio.com/}{T Brand Studio}
\item
  \href{https://www.nytimes3xbfgragh.onion/privacy/cookie-policy\#how-do-i-manage-trackers}{Your
  Ad Choices}
\item
  \href{https://www.nytimes3xbfgragh.onion/privacy}{Privacy}
\item
  \href{https://help.nytimes3xbfgragh.onion/hc/en-us/articles/115014893428-Terms-of-service}{Terms
  of Service}
\item
  \href{https://help.nytimes3xbfgragh.onion/hc/en-us/articles/115014893968-Terms-of-sale}{Terms
  of Sale}
\item
  \href{https://spiderbites.nytimes3xbfgragh.onion}{Site Map}
\item
  \href{https://help.nytimes3xbfgragh.onion/hc/en-us}{Help}
\item
  \href{https://www.nytimes3xbfgragh.onion/subscription?campaignId=37WXW}{Subscriptions}
\end{itemize}
