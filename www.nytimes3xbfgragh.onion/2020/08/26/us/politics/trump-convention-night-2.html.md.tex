Sections

SEARCH

\protect\hyperlink{site-content}{Skip to
content}\protect\hyperlink{site-index}{Skip to site index}

\href{https://www.nytimes3xbfgragh.onion/section/politics}{Politics}

\href{https://myaccount.nytimes3xbfgragh.onion/auth/login?response_type=cookie\&client_id=vi}{}

\href{https://www.nytimes3xbfgragh.onion/section/todayspaper}{Today's
Paper}

\href{/section/politics}{Politics}\textbar{}R.N.C. Presents Donald
Trump, the American Protector

\url{https://nyti.ms/3jgvpeI}

\begin{itemize}
\item
\item
\item
\item
\item
\item
\end{itemize}

\begin{itemize}
\item
  \href{https://www.nytimes3xbfgragh.onion/interactive/2020/09/08/us/elections/results-new-hampshire-primary-elections.html?action=click\&pgtype=Article\&state=default\&region=TOP_BANNER\&context=storylines_menu}{New
  Hampshire Results}
\item
  \href{https://www.nytimes3xbfgragh.onion/live/2020/09/08/us/trump-vs-biden?action=click\&pgtype=Article\&state=default\&region=TOP_BANNER\&context=storylines_menu}{Election
  Updates}
\item
  \href{https://www.nytimes3xbfgragh.onion/interactive/2020/us/elections/election-states-biden-trump.html?action=click\&pgtype=Article\&state=default\&region=TOP_BANNER\&context=storylines_menu}{Paths
  to 270}
\item
  \href{https://www.nytimes3xbfgragh.onion/interactive/2020/08/31/us/politics/vote-by-mail-deadlines.html?action=click\&pgtype=Article\&state=default\&region=TOP_BANNER\&context=storylines_menu}{Voting
  by Mail}
\item
  \href{https://www.nytimes3xbfgragh.onion/interactive/2019/us/elections/2020-presidential-election-calendar.html?action=click\&pgtype=Article\&state=default\&region=TOP_BANNER\&context=storylines_menu}{Key
  Dates}
\item
  \href{https://www.nytimes3xbfgragh.onion/newsletters/politics?action=click\&pgtype=Article\&state=default\&region=TOP_BANNER\&context=storylines_menu}{Politics
  Newsletter}
\end{itemize}

Advertisement

\protect\hyperlink{after-top}{Continue reading the main story}

Supported by

\protect\hyperlink{after-sponsor}{Continue reading the main story}

News Analysis

\hypertarget{rnc-presents-donald-trump-the-american-protector}{%
\section{R.N.C. Presents Donald Trump, the American
Protector}\label{rnc-presents-donald-trump-the-american-protector}}

The Republicans want white suburban voters to believe that the president
is on their side --- and that he's on the side of immigrants and women,
too.

\includegraphics{https://static01.graylady3jvrrxbe.onion/images/2020/08/25/us/politics/25repubs-assess1/merlin_176179632_28dc1a65-e293-4146-888d-84ed14d6e39a-articleLarge.jpg?quality=75\&auto=webp\&disable=upscale}

\href{https://www.nytimes3xbfgragh.onion/by/lisa-lerer}{\includegraphics{https://static01.graylady3jvrrxbe.onion/images/2018/09/11/us/politics/author-lisa-lerer/lisa-lerer-headshot-thumbLarge.png}}\href{https://www.nytimes3xbfgragh.onion/by/sydney-ember}{\includegraphics{https://static01.graylady3jvrrxbe.onion/images/2018/06/12/multimedia/author-sydney-ember/author-sydney-ember-thumbLarge.png}}

By \href{https://www.nytimes3xbfgragh.onion/by/lisa-lerer}{Lisa Lerer}
and \href{https://www.nytimes3xbfgragh.onion/by/sydney-ember}{Sydney
Ember}

\begin{itemize}
\item
  Aug. 26, 2020
\item
  \begin{itemize}
  \item
  \item
  \item
  \item
  \item
  \item
  \end{itemize}
\end{itemize}

On the first night of their
\href{https://www.nytimes3xbfgragh.onion/2020/08/26/us/politics/republican-convention-recap.html}{convention},
Republicans appealed to ``quiet neighborhoods'' by warning of vengeful
mobs of ``anarchists.'' Twenty-four hours later,
\href{https://www.nytimes3xbfgragh.onion/2020/08/26/us/politics/rnc-trump-character.html}{Mr.
Trump} and his party cast the president as the country's greatest
protector --- not only from mobs of protesters but also of free speech,
economic opportunity and even faith itself.

``This is a fight for freedom versus oppression,'' said Tiffany Trump,
the president's younger daughter. ``A fight to keep America true to
America.''

Speaker after speaker extolled the president as steady steward of the
country's promise, casting Democrats as radical leftists intent on
destroying the American way of life.

The message seemed tailored to suburban voters --- the people who were
instrumental in Mr. Trump's win in 2016. It built on the previous
evening's effort to convince those voters that the president wasn't
racist, despite years trafficking in racist tropes and nicknames.

On Tuesday night, his campaign wanted viewers to know that he was not
anti-immigrant or sexist, either.

``To mothers and parents everywhere, you are warriors,'' said Melania
Trump, the first lady. ``In my husband, you have a president who will
not stop fighting for you and your families.''

Anti-abortion activists praised the president for his steps to limit
access to abortion. A dairy farmer from Wisconsin, a lobster fisherman
from Maine and a mayor from Minnesota's Iron Range cast Mr. Trump as
saving their economic livelihoods. Nick Sandmann, a Covington Catholic
High School graduate and the focus of a viral video from last year,
applauded the president's war on so-called ``cancel culture'' and the
mainstream media.

Mr. Trump's own son, Eric, put it the most bluntly, promising ``every
proud American who bleeds red, white and blue, my father will continue
to fight for you.''

\includegraphics{https://static01.graylady3jvrrxbe.onion/images/2020/08/25/us/politics/25repubs-assess-2/merlin_176184801_ee5c7e99-9665-4bbc-8171-d622d0437a92-articleLarge.jpg?quality=75\&auto=webp\&disable=upscale}

The tone was far more uplifting than the dark dystopia painted by
speakers the previous night, featuring heartwarming friendships between
FBI investigators and reformed criminals, police officers and drug
addicts.

Amid sunny promises for the future, the illness and economic devastation
of the coronavirus was barely acknowledged until the first lady's
address at the end of the program, nor were the nearly 180,000 Americans
who have died from Covid-19.

In his remarks, Larry Kudlow, the president's top economic adviser,
pointedly spoke of the pandemic using the past tense.

``It was awful,'' Mr. Kudlow said in a video address. ``Hardship and
heartbreak were everywhere. But presidential leadership came swiftly and
effectively with an extraordinary rescue for health and safety to
successfully fight the Covid virus.''

Mrs. Trump promised that her husband would not stop fighting until a
treatment for the virus became widely available, comments that were a
notable break from the standard rhetoric of the president and his
administration that seek to downplay the virus.

``Donald will not rest until he has done all he can to take care of
everyone impacted by this terrible pandemic,'' Mrs. Trump said in her
address, which was delivered live from the Rose Garden.

Much of the evening was devoted to recasting the president's image as an
inclusive leader, welcoming to immigrants, eager to promote women and
embracing of racial diversity. The president was happy to play his part,
issuing a presidential pardon for convention viewers and hosting a
naturalization ceremony at the White House.

Mr. Trump welcomed five new citizens he described as ``absolutely
incredible people into our great American family.''

``You followed the rules, you obeyed the laws,'' he said. ``You learned
your history, embraced our values and proved yourselves to be men and
women of the highest integrity.''

Suburban white voters, and particularly suburban white women, remained
the heart of the convention's appeal. Mrs. Trump spoke about the impact
of women's voices ``in our nation's story.'' A video featuring the women
working in the White House highlighted Mr. Trump's qualities as a boss.

``The level of genius is unbelievable, frankly,'' Mr. Trump --- who has
been accused of sexual assault and misconduct by more than a dozen women
and was caught on tape bragging about forcing himself on women --- said
in a recorded segment.

``This president has been a champion for women, mostly because he speaks
to them as if they can handle and tackle all issues,'' said Kellyanne
Conway, President Trump's counselor and one of his longest-serving
aides. (Ms. Conway announced on Sunday that she was leaving the White
House to focus on her children.)

And Kimberly Guilfoyle, a former Fox News host, praised Mr. Trump for
treating women equally in the workplace. ``I don't want a job because of
my gender, I want the job because I'm the best person for that position.
That's it,'' said Ms. Guilfoyle, who is now dating Mr. Trump's son,
Donald Trump Jr. ``And he respects that.''

Republicans broadcast their convention from an imposing, columned
auditorium in Washington, but they want the message to resonate in the
suburbs. Places where, in their telling, Americans are hunkered down,
terrified of ``radical left Democrats'' intent on ``taking away the
American dream.''

Republicans have lost the suburbs
\href{https://www.nytimes3xbfgragh.onion/elections/2008/results/president/national-exit-polls.html}{only
three times since 1980}: In 1992, 1996 and 2008 --- all three,
Democratic presidential wins. Even in those races, the G.O.P. candidate
lost by no more than five points. In 2016, suburbs powered Mr. Trump's
victory, with
\href{https://www.cnn.com/election/2016/results/exit-polls}{exit polls
showing he won them by four points.}

Image

``Every proud American who bleeds red, white and blue, my father will
continue to fight for you,'' Eric Trump said in his
remarks.Credit...Pete Marovich for The New York Times

An
\href{https://www.langerresearch.com/wp-content/uploads/1214a22020Election.pdf}{ABC
News/Washington Post} poll released in July shows Joseph R. Biden Jr.
leading Mr. Trump in the suburbs by a margin of nine percentage points.
Another survey
\href{https://www.foxnews.com/politics/fox-news-poll-biden-holds-lead-over-trump-as-coronavirus-concerns-grip-nation}{by
Fox News} found an 11-point advantage for the former vice president.

Mr. Trump won the suburbs of North Carolina by 24 points, according to
2016 exit polls. He's now losing them
\href{https://static.foxnews.com/foxnews.com/content/uploads/2020/06/Fox_June-20-23-2020_Complete_North-Carolina_Topline_June-25-Release.pdf}{by
21 points}, according to a Fox News poll in June. In Florida, a state
that was expected to favor Mr. Trump,
\href{https://www.cnn.com/election/2016/results/exit-polls/florida/president}{a
10-point advantage in 2016 exit polls} is now a
\href{https://static.foxnews.com/foxnews.com/content/uploads/2020/06/Fox_June-20-23-2020_Complete_Florida_Topline_June-25-Release.pdf}{six-point}
deficit.

The president's attempts to shore up support among white suburban voters
by playing on racist fears began in earnest this summer, when he moved
to eliminate an Obama-era program intended to combat racial segregation
in suburban housing, claiming the program had a ``devastating impact on
these once thriving Suburban areas.''

During the first opening of the R.N.C., much of the appeal to these
voters was also cloaked in retrograde racial tropes, a return to a brand
of Republican identity politics with deep roots in the conservative
movement. In the midst of the social unrest of 1968, President Richard
Nixon's so-called Southern strategy focused on a law and order message
--- a phrase lifted by Mr. Trump and tweeted dozens of times as racial
justice protests spread across the country this spring and summer.

\includegraphics{https://static01.graylady3jvrrxbe.onion/images/2017/01/29/podcasts/the-daily-album-art/the-daily-album-art-articleInline-v2.jpg?quality=75\&auto=webp\&disable=upscale}

\hypertarget{listen-to-the-daily-trumps-suburban-strategy}{%
\subsubsection{Listen to `The Daily': Trump's Suburban
Strategy}\label{listen-to-the-daily-trumps-suburban-strategy}}

Set against the backdrop of unrest, Republicans are using a law and
order message to appeal to a powerful voting bloc.

transcript

Back to The Daily

bars

0:00/33:18

-33:18

transcript

\hypertarget{listen-to-the-daily-trumps-suburban-strategy-1}{%
\subsection{Listen to `The Daily': Trump's Suburban
Strategy}\label{listen-to-the-daily-trumps-suburban-strategy-1}}

\hypertarget{hosted-by-michael-barbaro-produced-by-rachel-quester-robert-jimison-and-jessica-cheung-with-help-from-michael-simon-johnson-and-andy-mills-and-edited-by-lisa-tobin-and-mj-davis-lin}{%
\subsubsection{Hosted by Michael Barbaro; produced by Rachel Quester,
Robert Jimison and Jessica Cheung; with help from Michael Simon Johnson
and Andy Mills; and edited by Lisa Tobin and M.J. Davis
Lin}\label{hosted-by-michael-barbaro-produced-by-rachel-quester-robert-jimison-and-jessica-cheung-with-help-from-michael-simon-johnson-and-andy-mills-and-edited-by-lisa-tobin-and-mj-davis-lin}}

\hypertarget{set-against-the-backdrop-of-unrest-republicans-are-using-a-law-and-order-message-to-appeal-to-a-powerful-voting-bloc}{%
\paragraph{Set against the backdrop of unrest, Republicans are using a
law and order message to appeal to a powerful voting
bloc.}\label{set-against-the-backdrop-of-unrest-republicans-are-using-a-law-and-order-message-to-appeal-to-a-powerful-voting-bloc}}

\begin{itemize}
\item
  michael barbaro\\
  From The New York Times, I'm Michael Barbaro. This is ``The Daily.''
\item
  archived recording\\
  {[}SOUNDS OF PROTESTS{]}
\end{itemize}

michael barbaro

As protests and unrest over racial justice and policing continue to
erupt across the U.S. ---

\begin{itemize}
\tightlist
\item
  archived recording (mark mccloskey)\\
  These radicals are not content with marching in the streets. These are
  the people who will be in charge of your future and the future of your
  children.
\end{itemize}

michael barbaro

--- speaker after speaker at the Republican National Convention this
week ---

\begin{itemize}
\tightlist
\item
  archived recording (patricia mccloskey)\\
  When we don't have basic safety and security in our communities, we'll
  never be free to build a brighter future for ourselves, for our
  children or for our country.
\end{itemize}

michael barbaro

--- have put them at the center of their appeal to a key group of
voters.

\begin{itemize}
\tightlist
\item
  archived recording (patricia mccloskey)\\
  They're not satisfied with spreading the chaos and violence into our
  communities. They want to abolish the suburbs altogether.
\end{itemize}

michael barbaro

Today: My colleague Emily Badger on the power of the suburban vote.

\begin{itemize}
\tightlist
\item
  archived recording (sean purnell)\\
  If you're a traditional democrat who's become disillusioned with how
  radical your party has become, then stand with us. You are most
  welcome.
\end{itemize}

michael barbaro

And the Republican Party's pitch to win it back. It's Wednesday, August
26. Emily, you have been thinking a lot about the upcoming election and
how suburban America fits into it. Why are we hearing so much about that
demographic right now at the Republican National Convention? Because it
feels like such a specific and explicit form of outreach.

emily badger

So suburban voters have really been the focal point of presidential
elections going all the way back to the 1960s. We have seen this pattern
over time where it's increasingly clear that voters in cities are going
to vote Democratic, rural voters are going to vote Republican. All of
the suspense, the whole ballgame, is in the suburbs. And so, you know,
these are the voters that we have been fighting over for a very long
time, and so it's not surprising that coming down to the last couple
months of this election that these are the voters that we're talking
about.

michael barbaro

Right. So when we think about purple America, swing America, we're
really talking about the suburbs.

emily badger

That's right.

michael barbaro

Well, where does that story of politics and the suburbs start?

emily badger

Well, I have been thinking a lot this year about 1968 in particular.

{[}music{]}

1968 was this really pivotal year for a lot of reasons in American
politics. We think about what was happening in the country at that
moment.

\begin{itemize}
\tightlist
\item
  archived recording\\
  Good evening. The Reverend Dr. Martin Luther King, 39 years old and a
  Nobel Peace Prize winner, and the leader of the nonviolent Civil
  Rights Movement in the United States, was assassinated in Memphis
  tonight.
\end{itemize}

emily badger

Martin Luther King gets assassinated in the first week of April.

\begin{itemize}
\item
  archived recording 1\\
  {[}POLICE RADIO CHATTER{]}
\item
  archived recording 2\\
  Police report that the murder has touched off sporadic acts of
  violence in a negro section of the city.
\end{itemize}

emily badger

There is a wave of civil unrest that happens not just in one or two
cities, but in more than 100 cities across the country.

\begin{itemize}
\item
  archived recording 1\\
  Police report having made more than 600 arrests, with over half these
  still in custody. Three deaths have been reported so far.
\item
  archived recording 2\\
  Some of the worst trouble of the day occurred in Washington D.C., the
  very heart of the nation.
\item
  archived recording 3\\
  In some negro ghettos, there was looting, arson and bloodshed during
  the night.
\item
  archived recording 4\\
  4,000 National Guard and federal troops are in this uneasy town
  tonight.
\item
  archived recording (richard j. daley)\\
  We shoot to kill any arsonist, or anyone with a molotov cocktail in
  their hand in Chicago.
\item
  archived recording\\
  {[}POLICE RADIO CHATTER{]}
\item
  archived recording (richard j. daley)\\
  And to issue a police order to shoot to maim or cripple anyone
  looting.
\item
  speaker\\
  Well, I'm saddened and angered by what has happened. We've marked the
  death of a man of peace, a man of goodwill with colossal violence,
  destruction and death. I would insist that law and order must prevail,
  and I am of course angered by the needless, silly, stupid destruction
  that I've seen in both Washington and Baltimore. I never believed this
  could happen in our nation's capital, or in my city.
\end{itemize}

emily badger

Out of this moment, there really emerges, you know, this very strong
backlash, particularly among a white middle class suburban voters,
against all of this unrest and against the sense that there's crime and
there's violence, and we're fed up with it. We just want order.

michael barbaro

And as a reminder, what do the suburbs in America look like at this
point in the 1960s? Because my sense is that the concept of a suburb,
right, these kind of planned communities on the edges of cities --- tidy
yards, white picket fences --- that that's kind of new in this moment.

emily badger

Yeah. So we see this huge explosion of suburbia after World War II.

\begin{itemize}
\tightlist
\item
  archived recording\\
  At last, the Bryants have all the space they need.
\end{itemize}

emily badger

And the people who were able to move to suburbia in that moment are not
sort of representative of the entire American population. It's very
specific groups of people who get to go.

\begin{itemize}
\tightlist
\item
  archived recording\\
  The home they've always dreamed of, the happiest investment they have
  ever made.
\end{itemize}

emily badger

So it's primarily white residents who get to go.

\begin{itemize}
\tightlist
\item
  archived recording\\
  The separate dining room is another feature that delights Margaret
  Bryant in her new home, for it permits her to enjoy her guests while
  entertaining graciously.
\end{itemize}

emily badger

It's primarily middle class and upper income white residents who get to
go.

\begin{itemize}
\tightlist
\item
  archived recording\\
  The patio, easily reached through a sliding glass doors, provides an
  outdoor living room ideal for separate activities.
\end{itemize}

emily badger

So in this moment in the 1960s, when we talk about the suburbs, they are
racially exclusionary by design. It is intentional that
African-Americans cannot move out there at that point.

\begin{itemize}
\tightlist
\item
  archived recording\\
  This is how American families are living in their new homes.
\end{itemize}

emily badger

And so the Civil Rights Movement begins to threaten that sense of
exclusion, because now we're talking about busing. Now we're talking
about fair housing. We're talking about whether or not it's fair for
homeowners to be able to say, I don't want to have Black neighbors, for
realtors to say, I don't want to work with Black homebuyers.

And at the same moment, we also see the rise of a number of politicians,
who are themselves sort of suburban politicians, who figure out how to
give voice to that anxiety. How to take this growing group of the
electorate who live in the suburbs and turn them into a voting block
where you are speaking directly to their concerns about their own
suburban security.

michael barbaro

And of course, 1968 is a presidential election year. So how do we see
all of that play out?

\begin{itemize}
\tightlist
\item
  archived recording\\
  {[}MENACING MUSIC{]}
\end{itemize}

emily badger

So ---

\begin{itemize}
\tightlist
\item
  archived recording (richard nixon)\\
  In recent years, crime in this country has grown nine times as fast as
  population.
\end{itemize}

emily badger

We see Richard Nixon increasingly make law and order a centerpiece of
his stump speeches.

\begin{itemize}
\tightlist
\item
  archived recording (richard nixon)\\
  We owe it to the decent and law abiding citizens of America to take
  the offensive against the criminal forces that threaten their peace
  and their security.
\end{itemize}

emily badger

And he is talking more and more about crime ---

\begin{itemize}
\tightlist
\item
  archived recording (richard nixon)\\
  I pledge to you, the wave of crime is not going to be the wave of the
  future in America.
\end{itemize}

emily badger

--- spending more money on the police.

\begin{itemize}
\tightlist
\item
  archived recording (richard nixon)\\
  Dissent is a necessary ingredient of change, but in new system of
  government that provides for peaceful change, there is no cause that
  justifies resort to violence.
\end{itemize}

emily badger

Sort of speaking to these issues about, you know, how you should be able
to protect what you have earned as a kind of hardworking American who's
bought your way into the suburbs.

\begin{itemize}
\tightlist
\item
  archived recording (richard nixon)\\
  Let us recognize that the first civil right of every American is to be
  freed from domestic violence.
\end{itemize}

emily badger

You should be able to protect that without fear that all of this chaos
that's happening in cities is going to come to your doorstep.

\begin{itemize}
\item
  archived recording (richard nixon)\\
  So I pledge to you, we shall have order in the United States.
\item
  archived recording\\
  {[}CROWDS CHANTING{]}
\end{itemize}

emily badger

And then we get to the Republican National Convention in 1968 in Miami.

\begin{itemize}
\tightlist
\item
  archived recording (richard nixon)\\
  All right. Thank you very much.
\end{itemize}

emily badger

Which is actually taking place at a moment when there is unrest
happening in the Liberty City neighborhood of Miami.

\begin{itemize}
\tightlist
\item
  archived recording (richard nixon)\\
  We make history tonight, not for ourselves, but for the ages.
\end{itemize}

emily badger

And Richard Nixon gives this speech where he talks about ---

\begin{itemize}
\tightlist
\item
  archived recording (richard nixon)\\
  As we look at America, we see cities enveloped in smoke and flame. We
  hear sirens in the night.
\end{itemize}

emily badger

--- cities enveloped in smoke and flame.

\begin{itemize}
\tightlist
\item
  archived recording (richard nixon)\\
  We see Americans hating each other, fighting each other, killing each
  other at home.
\end{itemize}

emily badger

And he devotes a long passage to talking about law and order.

\begin{itemize}
\tightlist
\item
  archived recording (richard nixon)\\
  The American Revolution was and is dedicated to progress, but our
  founders recognized that the first requisite of progress is order.
\end{itemize}

emily badger

And one of the things that's most striking to me about that speech is he
even says ---

\begin{itemize}
\tightlist
\item
  archived recording (richard nixon)\\
  And to those who say that law and order is the code word for racism,
  there and here is a reply.
\end{itemize}

emily badger

--- this emphasis on law and order is not racist.

\begin{itemize}
\tightlist
\item
  archived recording (richard nixon)\\
  Our goal is justice --- justice for every American. If we are to have
  respect for law in America, we must have laws that deserve respect.
  Just as we cannot have progress without order, we cannot have order
  without progress. And so as we commit to order tonight, let us commit
  to progress.
\end{itemize}

michael barbaro

Right. He's giving white suburbanites permission to be upset --- to be
fearful.

emily badger

He's giving them not only permission, but he's giving them a language to
talk about their grievances that doesn't sound like the language of
racism. It sounds instead like the language of property values and
quality schools and security and prosperity.

michael barbaro

Mm-hmm.

\begin{itemize}
\tightlist
\item
  archived recording (richard nixon)\\
  For the past five years, we have been deluged by government programs
  for the unemployed, programs for the cities, programs for the poor,
  and we have reaped from these programs an ugly harvest of frustration,
  violence and failure across the land.
\end{itemize}

emily badger

And he also sort of says that he is speaking to ---

\begin{itemize}
\tightlist
\item
  archived recording (richard nixon)\\
  It is a quiet voice in the tumult of the shouting. It is the voice of
  the great majority of Americans, the forgotten Americans, the
  non-shouters, the non-demonstrators.
\end{itemize}

emily badger

--- the forgotten and the silent Americans who are not demonstrating.
Those people who are sort of silently watching everything that's
happening in America from their quiet neighborhoods in the suburbs.
Those are the people who he wants to speak to.

\begin{itemize}
\tightlist
\item
  archived recording (richard nixon)\\
  They're not racist or sick. They're not guilty of the crime that
  plagues the land. They are Black and they are white.
\end{itemize}

michael barbaro

Right. And this is where we get that phrase that Nixon uses in 1968, and
you've begun to hint at it: the silent majority.

emily badger

Right.

\begin{itemize}
\tightlist
\item
  archived recording\\
  {[}APPLAUSE{]}
\end{itemize}

michael barbaro

And Emily, what is our understanding of the role that this strategy
ultimately played in that election?

emily badger

1968 is this year when suburban voters deliver the presidency to Richard
Nixon, and suburban voters and their preferences become central to
American politics. And they have largely been central to presidential
elections ever since then.

michael barbaro

So how do we see that play out in the years that follow?

emily badger

So after 1968, as it becomes clear that suburban voters are the swing
voters, the pivotal voters in American elections, their concerns come to
dominate not just what the Republican Party is doing, but also what the
Democratic party is doing. And so these are themes that we hear from
Ronald Reagan.

\begin{itemize}
\tightlist
\item
  archived recording (ronald reagan)\\
  Crime is an American epidemic. It takes the lives of 25,000 Americans.
  It touches nearly one-third of American households.
\end{itemize}

emily badger

They're also themes that we hear from Bill Clinton.

\begin{itemize}
\tightlist
\item
  archived recording (bill clinton)\\
  Let us roll up our sleeves to roll back this awful tide of violence
  and reduce crime in our country. We have the tools now. Let us get
  about the business of using them.
\end{itemize}

emily badger

And this carries us all the way through to 2016, when Donald Trump comes
on the scene.

\begin{itemize}
\tightlist
\item
  archived recording (donald trump)\\
  We have a situation where we have our inner cities, African-Americans,
  Hispanics are living in hell because it's so dangerous. You walk down
  the street, you get shot.
\end{itemize}

michael barbaro

Right. And Emily, I feel like when most people think about the 2016
campaign, they probably think about Trump's language and his message
around immigration. But it wasn't just limited to that. I covered the
2016 Republican National Convention, and I remember that Trump ---

\begin{itemize}
\tightlist
\item
  archived recording (donald trump)\\
  I have a message to every last person threatening the peace on our
  streets.
\end{itemize}

michael barbaro

--- explicitly modeled his message that year on Richard Nixon's message
from 1968, and that he was not bashful about it.

\begin{itemize}
\tightlist
\item
  archived recording (donald trump)\\
  When I take the oath of office next year, I will restore law and order
  to our country.
\end{itemize}

michael barbaro

I actually wrote a story about this. And around that time, Trump said
--- and I'm going to quote from him --- ``I think what Nixon understood
is that when the world is falling apart, people want a strong leader
whose highest priority is protecting America first. The `60s were bad
--- really bad --- and it's really bad now. Americans feel like it's
chaos again.''

emily badger

Yeah. I mean, he picked up these themes in 2016 in such a forceful way
that almost felt kind of discordant with what was going on around us in
America at the time.

\begin{itemize}
\tightlist
\item
  archived recording (donald trump)\\
  You look at Baltimore. You look at the violence that's taking place in
  the inner cities --- Chicago. You take a look at Washington, D.C. We
  have a increase in murder.
\end{itemize}

emily badger

So he was speaking a lot about these tremendous crime spikes. There were
some cities where crime was increasing at the time, but we were still in
one of the lowest crime eras that we've had in decades in America.

michael barbaro

Right.

\begin{itemize}
\tightlist
\item
  archived recording (donald trump)\\
  But for too many of our citizens, a different reality exists. This
  American carnage stops right here and stops right now. {[}APPLAUSE{]}
\end{itemize}

emily badger

But it remains true at the same time that, even as crime has fallen
precipitously in America, fears about crime and law and order have
always remained really strong for many people. And we consistently see
in polling across time that Americans believe that crime is increasing
even when it's declining --- that they believe that it is worse than it
really is. So it is possible for Trump to tap into those fears, I think,
even in a moment where it looks like crime is at a historic low.

michael barbaro

Mm-hmm. And Emily, what do we know about how this message in 2016 landed
in the suburbs?

emily badger

So the 2016 election is, again, most closely fought in the suburbs.
Trump gets wiped out in big cities and in the densest places in America.
Hillary Clinton fares even worse in rural America than Barack Obama did.
And then, in these in-between places in the suburbs, it is incredibly
closely contested to the point where whether or not Trump won the
suburbs is heavily dependent on exactly how you define them. And so this
launches us into the Trump administration itself, when white,
college-educated suburban women in these highly educated suburban
districts wind up being pivotal to the backlash against Trump. They wind
up giving Democrats control of the House of Representatives in the 2018
midterms.

michael barbaro

So that would seem to set up the suburban white woman voter as an
essential --- maybe the essential --- demographic for the 2020
presidential race.

emily badger

It's clear in 2018 that, as Trump has lost a lot of support,
particularly among white women, among white suburban women, that if he
is going to gain ground in the 2020 election, he is going to need to win
some of those women back. So we see that coming. We know that that's
going to be an issue in 2020. But I think what we don't see coming ---

\begin{itemize}
\item
  archived recording\\
  {[}CROWD CHANTING{]}
\item
  archived recording (police officer)\\
  You need to disperse. Gas will be deployed if you do not disperse.
\end{itemize}

emily badger

--- is that we're going to be back in this moment a couple months before
the 2020 election, where we are again talking about racial unrest in the
United States.

michael barbaro

Just like in 1968.

emily badger

I think if you're Donald Trump, you look at this moment and that's what
you think.

{[}music{]}

michael barbaro

We'll be right back.

\begin{itemize}
\tightlist
\item
  archived recording (ronna mcdaniel)\\
  Good evening. I'm Ronna McDaniel, chairwoman of the Republican
  National Committee, and on behalf of everyone in our party and
  President Trump, thank you for tuning in as we kick off this historic
  convention.
\end{itemize}

michael barbaro

So you know, Emily, I was watching the Republican National Convention on
the opening night, and having heard you now explain the messaging from
the R.N.C. in 1968, it's sort of astonishing just how much the messaging
from the R.N.C. in 2020 hits the same notes.

emily badger

There are these exact same themes and even identical language about ---

\begin{itemize}
\tightlist
\item
  archived recording (donald trump jr.)\\
  It's almost like this election is shaping up to be church, work and
  school versus rioting, looting and vandalism.
\end{itemize}

emily badger

--- cities on fire, looting, vandalism. We don't have law and order. We
need to restore law and order. And it's coming from ---

\begin{itemize}
\tightlist
\item
  archived recording (donald trump jr.)\\
  Law and order is on the ballot.
\end{itemize}

emily badger

--- speaker ---

\begin{itemize}
\tightlist
\item
  archived recording (vernon jones)\\
  They call it defunding, and it's a danger to our cities, our
  neighborhoods and our children.
\end{itemize}

emily badger

--- after speaker ---

\begin{itemize}
\tightlist
\item
  archived recording (jim jordan)\\
  Look at what's happening in American cities, cities all run by
  Democrats --- crime, violence and mob rule.
\end{itemize}

emily badger

--- after speaker. It's really a theme that they return to throughout
the night, and it's embedded in this idea that this is what will happen
in a Democratic administration. We're seeing all of this chaos in cities
that are run by Democratic mayors that have long been strongholds of
Democratic politicians.

\begin{itemize}
\tightlist
\item
  archived recording (kimberly guilfoyles)\\
  Just take a look at California. It is a place of immense wealth,
  immeasurable innovation, an immaculate environment. And the Democrats
  turned it into a land of discarded heroin needles in parks, riots in
  streets, and blackouts in homes.
\end{itemize}

emily badger

And if we give Democrats control of the entire country, this is what you
can expect in your community where you live too.

michael barbaro

Mm-hmm. And it felt like the sort of ultimate example of this was this
couple from St. Louis who were given a prime speaking spot on the first
night of the Republican Convention. This is the couple who, back in
June, drew a lot of attention when images surfaced of them standing in
the front of their mansion pointing guns at protesters as those
protesters walked in front of the couple's house on their way to a
protest in front of the local mayor's house.

emily badger

Yeah. So this was a segment and a pair of speakers who I don't think we
could have expected to see in any prior Republican National Convention.
We have this couple ---

\begin{itemize}
\tightlist
\item
  mark mccloskey\\
  Good evening, America.
\end{itemize}

emily badger

--- Mark and Patricia McCloskey.

\begin{itemize}
\tightlist
\item
  archived recording (mark mccloskey)\\
  We are Mark and Patty McCloskey. We're speaking with you tonight from
  St. Louis, Missouri, where just weeks ago you may have seen us
  defending our home as a mob of protesters descended on our
  neighborhood.
\end{itemize}

emily badger

And they live on a gated, very upscale street in St. Louis that's
technically inside the city, but has very much sort of the trappings of
suburbia. And they're speaking to us from what looks like a couch in
their living room or their sitting room, and they're both wearing
blazers.

\begin{itemize}
\tightlist
\item
  archived recording (mark mccloskey)\\
  Not a single person in the out-of-control mob you saw at our house was
  charged with a crime. But you know who was? We were. They have
  actually charged us with felonies for daring to defend our home.
\end{itemize}

emily badger

And one of the things that was most striking to me about their segment
was that, in contrast to a lot of the other politicians who spoke with
really sort of forceful rhetoric, they had this calming presence.

\begin{itemize}
\item
  archived recording (patricia mccloskey)\\
  What you saw happen to us could just as easily happen to any of you
  who are watching from quiet neighborhoods around our country. And
  that's what we want to speak to you about tonight.
\item
  archived recording (mark mccloskey)\\
  That's exactly right.
\end{itemize}

emily badger

Even as they were saying, you need to worry about mobs coming for you in
your quiet neighborhood around the country. And it is Patricia McCloskey
who specifically tells us ---

\begin{itemize}
\tightlist
\item
  archived recording (patricia mccloskey)\\
  They're not satisfied with spreading the chaos and violence into our
  communities.
\end{itemize}

emily badger

Not only do Democrats want to spread chaos into the suburbs ---

\begin{itemize}
\tightlist
\item
  archived recording (patricia mccloskey)\\
  They want to abolish the suburbs altogether by ending single-family
  home zoning. This forced rezoning would bring crime, lawlessness and
  low quality apartments into now thriving suburban neighborhoods.
\end{itemize}

emily badger

--- the want to make it such that you can't have sort of your nice,
quiet suburban neighborhood full of single-family houses.

\begin{itemize}
\item
  archived recording (mark mccloskey)\\
  The Democrats have brought us nothing but destruction.
\item
  archived recording (patricia mccloskey)\\
  When we don't have basic safety and security in our communities, we'll
  never be free to build a brighter future for ourselves, for our
  children or for our country. That's what's at stake in this election,
  and that's why we must re-elect Donald Trump.
\end{itemize}

michael barbaro

So here again, as in 1968, we have a focus on housing regulations as a
way of talking about this.

emily badger

Yeah. When she says that the Democrats want to abolish the suburbs, she
is alluding to a piece of the 1968 Fair Housing Act --- again, we're
coming back to 1968 --- that the Obama administration had adopted a rule
trying to encourage communities all over the country, not just the
suburbs, to embrace integration. And earlier this summer, the Trump
administration rolled back that rule, and Trump announced that the
suburban housewives of America should be thrilled that I have done this,
and your very quality of life you will not have control over if the
federal government will come in and remake your neighborhood.

michael barbaro

Hmm. The other parallel that felt most overt to me, Emily, was this
recurring message that all this talk, as in 1968, is not racist. But in
this case it wasn't the Republican nominee. It wasn't Donald Trump
saying this. It was Republicans of color. It was, for example, Kim
Klacik ---

\begin{itemize}
\tightlist
\item
  archived recording (kim klacik)\\
  My name is Kim Klacik, and I'm running for Congress in Maryland's 7th
  district.
\end{itemize}

michael barbaro

--- a Black woman running for Congress in Baltimore. And she's running
on a message that Democrats have let Baltimore down.

\begin{itemize}
\tightlist
\item
  archived recording (kim klacik)\\
  Sadly the same cycle of decay exists in many of America's Democrat-run
  cities. And yet the Democrats still assume that Black people will vote
  for them no matter how much they let us down and take us for granted.
  We're sick of it. We're not going to take it anymore.
\end{itemize}

michael barbaro

And we also heard a similar message from Tim Scott ---

\begin{itemize}
\tightlist
\item
  archived recording (tim scott)\\
  We live in a world that only wants you to believe in the bad news ---
  racially, economically and culturally polarizing news. The truth is,
  our nation's arc always bends back towards fairness.
\end{itemize}

michael barbaro

--- a Black senator from South Carolina, and from Nikki Haley ---

\begin{itemize}
\tightlist
\item
  archived recording (nikki haley)\\
  In much of the Democratic party, it's now fashionable to say that
  America is racist. That is a lie. America is not a racist country.
  This is personal for me.
\end{itemize}

michael barbaro

--- who is of Indian descent, and who was Trump's Ambassador to the
United Nations.

emily badger

Yeah. We also heard something similar from Herschel Walker, who is the
former football player who has had this very long-running, almost
four-decade long relationship with Donald Trump, who is also
African-American, and effectively said ---

\begin{itemize}
\tightlist
\item
  archived recording (herschel walker)\\
  It hurt my soul to hear the terrible names that people called Donald.
  The worst one is racist. I take that as a personal insult that people
  would think I've had a 37-year friendship with a racist. People who
  think that don't know what they're talking about.
\end{itemize}

emily badger

You know, I am offended by the idea that anyone would think that I have
been friends with a racist for the last 37.

\begin{itemize}
\tightlist
\item
  archived recording (herschel walker)\\
  Just because someone loves and respect the flag, our national anthem
  and our country, doesn't mean they don't care about social justice. I
  care about all of those things, so does Donald Trump.
\end{itemize}

michael barbaro

So Emily, from Trump's point of view, this strategy would seem to have a
very solid track record. So is there any reason to think that it would
not work now in 2020?

emily badger

One big reason is that the suburbs themselves have changed dramatically
since the 1960s. The women who live in the suburbs today are much more
racially diverse. They're more economically diverse. When we talk about
suburban voters in suburbia today, it is much less clear exactly who
we're talking about, because it's no longer just middle class and upper
income white voters who are living in these communities. There's poverty
in these communities. There are immigrant communities who live in the
suburbs. And so this is not the voting bloc that Richard Nixon was
speaking to in 1968 or the voting block that it seems like Donald Trump
has in mind.

But the other reason why I think we should be really skeptical is that
we see in polling data that voters in the suburbs today, majorities of
them are supportive of the Black Lives Matter movement. They're
supportive of these protests. They're even participating in these
protests. And so they're really sort of not necessarily receptive to the
issues that Trump is trying to elevate, but he's also trying to get them
to focus on a set of issues which are not their primary concern right
now. I mean, between the pandemic and the collapse of the economy and
millions of Americans losing their health care as a result of that, you
know, those are the three issues that really sit at the top of suburban
voters' and female voters' concerns when we ask them what they're
concerned about right now.

michael barbaro

Mm-hmm. But it does seem like there might be an X factor here that
Donald Trump has been priming suburban voters for. And an example of
that would be what's going on right now in Wisconsin ---

\begin{itemize}
\tightlist
\item
  archived recording\\
  {[}CROWD CHANTING{]}
\end{itemize}

michael barbaro

--- where there are protests against the shooting of a Black man, Jacob
Blake, and those protests have turned into fires and looting.

\begin{itemize}
\tightlist
\item
  archived recording (police officer)\\
  You need to disperse. Gas will be deployed if you do not disperse.
\end{itemize}

michael barbaro

On top of the situation that we have had in cities like Portland. And
could it be that the polling that you're referring to is not quite up to
date, and that there may be voters who hear the president talking at
this convention and think to themselves, I do support Black Lives
Matter, but I don't support this. I don't support what I'm seeing on my
television screen in places like Kenosha, Wisconsin.

emily badger

I think the biggest unknown over the next two months, which could play
to the president's advantage, is that there will be more Kenoshas. And I
think we don't know at this point how more scenes like that might change
or erode public opinion about these issues in the months to come. but I
think that in order for this strategy to work for Trump, suburban women
need to not only become concerned about these scenes, but they have to
believe that their own neighborhoods are threatened in this moment.

And the question is, will suburban voters really see it that way in
2020, or has simply too much changed since 1968?

michael barbaro

Well, Emily, thank you very much. We appreciate it.

emily badger

Yeah. Thanks for the conversation.

michael barbaro

We'll be right back.

{[}music{]}

Here's what else you need to know today.

\begin{itemize}
\tightlist
\item
  archived recording (melania trump)\\
  I want to acknowledge the fact that, since March, our lives have
  changed drastically. The invisible enemy, Covid-19, swept across our
  beautiful country and impacted all of us.
\end{itemize}

michael barbaro

On the second night of the Republican National Convention, First Lady
Melania Trump confronted a topic that has been largely missing from the
proceedings so far --- the painful impact of the coronavirus pandemic.

\begin{itemize}
\tightlist
\item
  archived recording (melania trump)\\
  My deepest sympathy goes out to everyone who has lost a loved one, and
  my prayers are with those who are ill or suffering. I know many people
  are anxious and some feel helpless. I want you to know, you're not
  alone.
\end{itemize}

michael barbaro

The First Lady focused much of her speech on appealing to women and
mothers by seeking to portray her husband as their protector.

\begin{itemize}
\tightlist
\item
  archived recording (melania trump)\\
  To mothers and parents everywhere, you are warriors. In my husband,
  you have a president who will not stop fighting for you and your
  families. I see how hard he works each day and night. And despite the
  unprecedented attacks from the media and opposition, he will not give
  up. In fact, if you tell him it cannot be done, he just works harder.
\end{itemize}

michael barbaro

And on Tuesday, in a sign of the pandemic's ongoing economic toll,
American Airlines said it would furlough 19,000 workers when its federal
financial aid, which totaled nearly \$6 billion, comes to an end this
fall. By October, the airline will have reduced its workforce by 30
percent. American's rivals, Delta and United, say that they too may need
to cut jobs this fall. The announcements are likely to increase pressure
on Congress to pass new economic relief, something that lawmakers have
been unable to do for weeks.

{[}music{]}

That's it for ``The Daily.'' I'm Michael Barbaro. See you tomorrow.

On Monday night, Charlie Kirk, the founder of a conservative student
group, praised Mr. Trump as ``the bodyguard of western civilization,'' a
phrase generally used among white supremacists as code for white
identity. Vernon Jordan, a Democratic state representative from Georgia,
accused Democrats of keeping Black voters on a ``mental Plantation.''

Race played little role on Tuesday evening, even as demonstrations and
destruction rocked the city of Kenosha, Wis., after police there shot
Jacob Blake, a Black man, in front of his three young children.

Shortly before the start of his convention, Mr. Trump tweeted that the
National Guard should be called into the city to control the
demonstrations and destruction that have rocked the city ---
\href{https://www.nytimes3xbfgragh.onion/2020/08/25/us/jacob-blake-kenosha-fires.html}{a
step already taken by the Democratic governor.} An opening prayer was
the only mention of Mr. Blake.

Still, echoes of Mr. Nixon pervaded the proceedings, the former
president's ``silent majority'' becoming the ``forgotten man.'' In the
Trump campaign's telling, those are the working men and women who
elected him to the White House four years ago.

``I'm proud to watch you give them hell,'' said the younger Mr. Trump,
directly addressing his remarks to his father. ``You are making America
proud again. And yes, together, with the forgotten man and woman who are
finally forgotten no more, you are making America great again.''

\hypertarget{our-2020-election-guide}{%
\section{Our 2020 Election Guide}\label{our-2020-election-guide}}

Updated ~Sept. 8, 2020

\begin{center}\rule{0.5\linewidth}{\linethickness}\end{center}

\begin{itemize}
\item ~
  \hypertarget{the-latest}{%
  \subsection{The Latest}\label{the-latest}}

  \begin{itemize}
  \item
    President Trump and his party are using a playbook that aims to
    alarm people about crime in their backyards. It didn't work in 2018,
    but
    \href{https://www.nytimes3xbfgragh.onion/2020/09/08/us/politics/trump-republicans-fear-strategy.html?action=click\&pgtype=Article\&state=default\&region=BELOW_MAIN_CONTENT\&context=storylines_guide}{both
    parties think it could resonate more this year}.
  \end{itemize}
\item ~
  \hypertarget{how-to-win-270}{%
  \subsection{How to Win 270}\label{how-to-win-270}}

  \begin{itemize}
  \item
    Joe Biden and Donald Trump need 270 electoral votes to reach the
    White House. Try building
    \href{https://www.nytimes3xbfgragh.onion/interactive/2020/us/elections/election-states-biden-trump.html?action=click\&pgtype=Article\&state=default\&region=BELOW_MAIN_CONTENT\&context=storylines_guide}{your
    own coalition of battleground states}~to see potential outcomes.
  \end{itemize}
\item ~
  \hypertarget{voting-by-mail}{%
  \subsection{Voting by Mail}\label{voting-by-mail}}

  \begin{itemize}
  \item
    Will you have enough time to vote by mail in your state? Yes, but
    it's risky to procrastinate.
    \href{https://www.nytimes3xbfgragh.onion/interactive/2020/08/31/us/politics/vote-by-mail-deadlines.html?action=click\&pgtype=Article\&state=default\&region=BELOW_MAIN_CONTENT\&context=storylines_guide}{Check
    your state's deadline.}
  \item
    \href{https://www.nytimes3xbfgragh.onion/interactive/2020/us/elections/joe-biden.html?action=click\&pgtype=Article\&state=default\&region=BELOW_MAIN_CONTENT\&context=storylines_guide}{}

    \hypertarget{joe-biden}{%
    \section{Joe Biden}\label{joe-biden}}

    \hypertarget{democrat}{%
    \subsection{Democrat}\label{democrat}}

    \href{https://www.nytimes3xbfgragh.onion/interactive/2020/us/elections/donald-trump.html?action=click\&pgtype=Article\&state=default\&region=BELOW_MAIN_CONTENT\&context=storylines_guide}{}

    \hypertarget{donald-trump}{%
    \section{Donald Trump}\label{donald-trump}}

    \hypertarget{republican}{%
    \subsection{Republican}\label{republican}}
  \end{itemize}
\item
  \hypertarget{keep-up-with-our-coverage}{%
  \subsection{Keep Up With Our
  Coverage}\label{keep-up-with-our-coverage}}

  \begin{itemize}
  \item
    Get an
    \href{https://www.nytimes3xbfgragh.onion/newsletters/politics?action=click\&pgtype=Article\&state=default\&region=BELOW_MAIN_CONTENT\&context=storylines_guide}{email}~recapping
    the day's news
  \item
    Download our mobile app on
    \href{https://apps.apple.com/us/app/nytimes/id284862083?ls=1\&mat_click_id=5c79ae7455014fd1bd66b5610c05b8f2-20191112-16948\&referrer=mat_click_id\%3D5c79ae7455014fd1bd66b5610c05b8f2-20191112-16948\%26link_click_id\%3D722930677036718082}{iOS}~and
    \href{http://a.localytics.com/android?id=com.nytimes.android\&referrer=utm_source\%3Dother_nyt_mobile_web\%26utm_medium\%3DWeb\%2520page\%26utm_term\%3DGeneral\%2520Mobile\%2520Page\%26utm_campaign\%3DNYT\%2520Mobile\%2520General\%2520Page}{Android}~and
    turn on Breaking News and Politics alerts
  \end{itemize}
\end{itemize}

Advertisement

\protect\hyperlink{after-bottom}{Continue reading the main story}

\hypertarget{site-index}{%
\subsection{Site Index}\label{site-index}}

\hypertarget{site-information-navigation}{%
\subsection{Site Information
Navigation}\label{site-information-navigation}}

\begin{itemize}
\tightlist
\item
  \href{https://help.nytimes3xbfgragh.onion/hc/en-us/articles/115014792127-Copyright-notice}{©~2020~The
  New York Times Company}
\end{itemize}

\begin{itemize}
\tightlist
\item
  \href{https://www.nytco.com/}{NYTCo}
\item
  \href{https://help.nytimes3xbfgragh.onion/hc/en-us/articles/115015385887-Contact-Us}{Contact
  Us}
\item
  \href{https://www.nytco.com/careers/}{Work with us}
\item
  \href{https://nytmediakit.com/}{Advertise}
\item
  \href{http://www.tbrandstudio.com/}{T Brand Studio}
\item
  \href{https://www.nytimes3xbfgragh.onion/privacy/cookie-policy\#how-do-i-manage-trackers}{Your
  Ad Choices}
\item
  \href{https://www.nytimes3xbfgragh.onion/privacy}{Privacy}
\item
  \href{https://help.nytimes3xbfgragh.onion/hc/en-us/articles/115014893428-Terms-of-service}{Terms
  of Service}
\item
  \href{https://help.nytimes3xbfgragh.onion/hc/en-us/articles/115014893968-Terms-of-sale}{Terms
  of Sale}
\item
  \href{https://spiderbites.nytimes3xbfgragh.onion}{Site Map}
\item
  \href{https://help.nytimes3xbfgragh.onion/hc/en-us}{Help}
\item
  \href{https://www.nytimes3xbfgragh.onion/subscription?campaignId=37WXW}{Subscriptions}
\end{itemize}
