Sections

SEARCH

\protect\hyperlink{site-content}{Skip to
content}\protect\hyperlink{site-index}{Skip to site index}

\href{/section/us}{U.S.}\textbar{}`2020 Can Go to Hell': The Story
Behind the Viral Fire Photo That Said It All

\url{https://nyti.ms/3guJPpO}

\begin{itemize}
\item
\item
\item
\item
\item
\end{itemize}

\hypertarget{wildfires-in-the-west}{%
\subsubsection{\texorpdfstring{\href{https://www.nytimes3xbfgragh.onion/spotlight/california-wildfires?name=styln-california-wildfires\&region=TOP_BANNER\&block=storyline_menu_recirc\&action=click\&pgtype=Article\&impression_id=9ed63a20-f2ba-11ea-a96b-f1b6e8bed79f\&variant=undefined}{Wildfires
in the West}}{Wildfires in the West}}\label{wildfires-in-the-west}}

\begin{itemize}
\tightlist
\item
  live\href{https://www.nytimes3xbfgragh.onion/2020/09/09/us/wildfires-live-updates.html?name=styln-california-wildfires\&region=TOP_BANNER\&block=storyline_menu_recirc\&action=click\&pgtype=Article\&impression_id=9ed63a21-f2ba-11ea-a96b-f1b6e8bed79f\&variant=undefined}{Fires
  Updates}
\item
  \href{https://www.nytimes3xbfgragh.onion/2020/09/07/us/ca-wildfires-heatwave.html?name=styln-california-wildfires\&region=TOP_BANNER\&block=storyline_menu_recirc\&action=click\&pgtype=Article\&impression_id=9ed66130-f2ba-11ea-a96b-f1b6e8bed79f\&variant=undefined}{Fires
  Span California}
\item
  \href{https://www.nytimes3xbfgragh.onion/2020/08/26/us/california-wildfires-lake-berryessa.html?name=styln-california-wildfires\&region=TOP_BANNER\&block=storyline_menu_recirc\&action=click\&pgtype=Article\&impression_id=9ed66131-f2ba-11ea-a96b-f1b6e8bed79f\&variant=undefined}{The
  Trauma of 2020}
\item
  \href{https://www.nytimes3xbfgragh.onion/article/why-does-california-have-wildfires.html?name=styln-california-wildfires\&region=TOP_BANNER\&block=storyline_menu_recirc\&action=click\&pgtype=Article\&impression_id=9ed66132-f2ba-11ea-a96b-f1b6e8bed79f\&variant=undefined}{California's
  Disastrous Season}
\item
  \href{https://www.nytimes3xbfgragh.onion/2020/09/08/us/california-wildfire-heat-wave.html?name=styln-california-wildfires\&region=TOP_BANNER\&block=storyline_menu_recirc\&action=click\&pgtype=Article\&impression_id=9ed66133-f2ba-11ea-a96b-f1b6e8bed79f\&variant=undefined}{Newsletter}
\end{itemize}

\includegraphics{https://static01.graylady3jvrrxbe.onion/images/2020/08/26/us/26calfires-01/merlin_176187000_e3256d53-600c-4369-a943-250c3c5cf47e-articleLarge.jpg?quality=75\&auto=webp\&disable=upscale}

\hypertarget{2020-can-go-to-hell-the-story-behind-the-viral-fire-photo-that-said-it-all}{%
\section{`2020 Can Go to Hell': The Story Behind the Viral Fire Photo
That Said It
All}\label{2020-can-go-to-hell-the-story-behind-the-viral-fire-photo-that-said-it-all}}

Pandemic, unemployment, wildfire --- Lake Berryessa, Calif., has seen it
all this year. A photo from the town that evoked the trauma of 2020 sped
across the internet. Here, though, it was real life.

Chris Lacombe looked over the remains of his home, which was destroyed
by the L.N.U. Lightning Complex, in Spanish Flat, Calif.Credit...Max
Whittaker for The New York Times

Supported by

\protect\hyperlink{after-sponsor}{Continue reading the main story}

By \href{https://www.nytimes3xbfgragh.onion/by/jack-healy}{Jack Healy}

\begin{itemize}
\item
  Aug. 26, 2020
\item
  \begin{itemize}
  \item
  \item
  \item
  \item
  \item
  \end{itemize}
\end{itemize}

\href{https://cn.nytimes3xbfgragh.onion/usa/20200827/california-wildfires-lake-berryessa/}{阅读简体中文版}\href{https://cn.nytimes3xbfgragh.onion/usa/20200827/california-wildfires-lake-berryessa/zh-hant/}{閱讀繁體中文版}

LAKE BERRYESSA, Calif. --- In the sprawling destruction of California's
wildfires,
\href{https://ktla.com/news/california/thats-2020-photographers-california-wildfire-image-a-sign-of-the-times/}{one
photo became an instant icon} for 2020's miseries: On a hillside roaring
with flames stood a sign that asked visitors to a senior center to wear
masks, wash their hands and be safe. ``Come Join Us,'' it beckoned
creepily.

The virus. Lost jobs. A world aflame.

Yep, said Judi Vollmer, whose trailer home down the road from the sign
burned down last week, just days after she learned that her 92-year-old
father had tested positive for the coronavirus --- that pretty much sums
up life right now.

Ms. Vollmer, 65, was succinct: ``2020 can go to hell. This has been the
worst year of my life.''

Somehow, that welcome sign outside the Lake Berryessa senior center was
still standing on Tuesday as residents trickled back through police
barricades and road closures to check out what little had survived.

\href{https://napavalleyregister.com/news/local/family-identifies-remembers-3-victims-claimed-by-hennessey-fire/article_9d95b2f2-02a2-5b5d-988e-8de48ae42a66.html}{Three
people were killed} --- one of them a 71-year-old man in a wheelchair
--- when
\href{https://www.latimes.com/california/story/2020-08-26/fire-lake-lnu-berryessa-family-couldnt-escape}{flames
swarmed their mountainside property}. Family members said they had tried
to escape, but as a last resort took refuge in a homemade ``burn
shelter.'' Relatives identified the victims as Mary Hintemeyer, 70, her
boyfriend, Leo McDermott, 71, and Mr. McDermott's 41-year-old son, Tom.

Much of the lakefront community of retirees and young families who
commute to landscaping, winery and service jobs in wealthier corners of
Napa County had been reduced to a thicket of tangled steel and ash.

\includegraphics{https://static01.graylady3jvrrxbe.onion/images/2020/08/27/us/27calfires-2/merlin_175884909_4c503f4c-6f58-408a-b47f-e6384501b614-articleLarge.jpg?quality=75\&auto=webp\&disable=upscale}

Now, as people in this community of 1,700 salvaged chipped tea saucers
and wooden lanterns from the char of about 100 destroyed homes, their
worries were a microcosm of the question haunting so many people during
this season of pandemic and strife: Would they ever get their old lives
back?

``We've lost so many people who won't be back,'' said Jerry Rehmke, 80,
who runs the country store with his wife, Marcia Ritz, 77. Her trailer
home, with all of the drawings and paintings she had made, burned in the
Spanish Flat Villa mobile home park, along with Ms. Vollmer's trailer
and about 50 others.

``Everything,'' Ms. Ritz said. ``It's down to the ground.''

The constellation of wildfires staining California's skies and stinging
people's lungs across the West have now killed seven and destroyed at
least 1,690 homes and other buildings, officials said. It is still early
in a wildfire season expected to rage through the fall. So as 15,000
firefighters pushed to gain control of the blazes around the state,
thousands of families who evacuated are now streaming back and wondering
whether they will have to flee again.

On Wednesday, Gov. Gavin Newsom said the accounting of death and damage
could rise as people return home. ``We've never seen fire of this scale
in this part of the state,'' he said. ``It demonstrates the reality ---
not just the point of view --- of climate change and its impact in this
state.''

Ms. Ritz moved to Lake Berryessa 13 years ago and took over running the
country store (which survived, as did some marinas and campgrounds).
Their store actually boomed during the pandemic as stir-crazy boaters
and anglers flooded the area and snapped up orders of chicken sandwiches
and meatloaf. That is over now, and faced with years of rebuilding and a
bleak economic future, Ms. Ritz said she was ready to quit altogether.

Image

Marlene Eining's home in the Spanish Flat Villa mobile home park was
destroyed.Credit...Max Whittaker for The New York Times

Image

The community of Lake Barryessa is sorting through damaged belongings
this week, trying to find items that can be salvaged.Credit...Max
Whittaker for The New York Times

Image

Around 50 homes were destroyed in the mobile home park in Spanish
Flat.Credit...Max Whittaker for The New York Times

Image

Andrea Shumate comforted her husband Josh as he sifted through the
remains of his grandmother's home.Credit...Max Whittaker for The New
York Times

``Our customers have gone,'' Ms. Ritz said on Tuesday morning, a few
minutes after she woke up from another night sleeping outside on an air
mattress beside the country store. ``By the end of the year I'll be out.
This is it.''

Her husband piped up: ``We should take down the sign that says `Only
Five People in the Store.' There may not be five people up here.''

It was never simple living along Lake Berryessa, a reservoir stocked
with trout and catfish that is also famous for a drain that creates a
vortex-like hole during wet years. Work is scarce, and cities and
groceries are a 40-minute drive along vertiginous mountain roads. The
roads can glaze with ice in the winter, and on 90-degree summer days,
pints of ice cream melt into soup before you can get them home.

People said they moved from bigger cities because they liked the rural
quiet and seeing mountain lions out their windows. On Tuesday morning, a
singed fox limped through the mobile home park, paying no heed to the
residents and power crews in the street.

Some people had been drawn to the lake by California's
affordable-housing crisis, pushed out of the rest of Napa. They said
this was one of the last corners of affordable housing for people
earning minimum wage or living off Social Security in a county where the
average home costs more than \$700,000.

Fire had always been a threat, but evacuations and smoke have gotten
even more common as climate change compounds the risk of fires in what
is known as the wildland-urban interface. Hillsides overgrown with dry
fuel are broiling, and the greenery that people say they cherish about
life here has gone as brown as scorched crust.

For the past four years, people around the lake said they watched fires
march toward their homes, only to be beaten back. The local Lions Club
would donate money to fire victims. Local officials installed a cache of
emergency beds and supplies and a big new generator at the senior center
to be used as a fallback spot, residents said.

``We know what devastation it does,'' Pam Stadnyk, whose trailer home
burned, including the wood deck she had just put in, said as she walked
through the area on Tuesday for the first time since the fires. ``We've
been living with it. You just get to a point where you ---'' and she
trailed off.

Months of the pandemic already had worn on the mobile home park's
working-class residents. Some lost work at Napa's wineries and
restaurants.

Edward Morrison, 57, had lost overtime work doing delivery runs to
businesses that closed as the pandemic dragged on. One of his sons had
been living near Paradise last year when a wildfire gutted the town and
killed more than 50 people. Now, his trailer was rubble and his cat was
missing. He called a dispatcher.

``Your address?'' she asked Mr. Morrison.

``Well my address burned down,'' he said.

Ms. Vollmer, who had lived at the lake for 18 years, kept working
throughout the pandemic. Her \$13-an-hour job at the country store was
considered essential work, and though she had asthma and customers
sometimes refused to wear masks, she kept going and did not get sick.

Image

From left, Jerry Rehmke, Marcia Ritz and Pam Stadnyk surveyed the damage
in Spanish Flat. ``We've lost so many people who won't be back,'' Mr.
Rehmke said.Credit...Max Whittaker for The New York Times

Image

``This has been the worst year of my life,'' Judi Vollmer said. She held
one of the three cats she saved when evacuating her home at the mobile
home park.Credit...Max Whittaker for The New York Times

Image

The constellation of wildfires across the state have now killed seven
and destroyed at least 1,690 homes and other buildings, officials
said.Credit...Max Whittaker for The New York Times

Image

It is still early in a wildfire season expected to rage through the
fall.Credit...Max Whittaker for The New York Times

She had stayed away from her 92-year-old father's nursing home since
February until a couple of weeks ago, when Ms. Vollmer said she got a
call telling her that he had tested positive for the coronavirus. Ms.
Vollmer said that he had Alzheimer's disease and sometimes did not know
if she was his daughter or wife, but that he seemed fine when she
visited him through his window recently.

``I don't know if it could get any more stressful than this,'' she said.

The fire, like the pandemic, has hit California's poorest residents
hardest. Homeowners able to keep up with the complications and rising
costs of insuring property in a fire zone had a safety net. But Ms.
Vollmer said her carrier dropped her after a wildfire a few years ago.
The trailer was her life's investment and her retirement plan, and it
burned alongside the \$3,000 in cash she had tucked away inside.

The Red Cross is putting her up in a hotel near the airport in Napa
along with three of her five cats --- the ones she was able to rescue.
She received a paper bag stuffed with donated clothes, but said she did
not know where to go at the end of the week when her hotel stay was up.

She said she loved the community. When her husband died eight years ago,
people took up a collection to pay for his cremation. She said she did
not know how to start over at 65.

``We're survivors from up there,'' she said. ``We dodged the bullet so
many times. We always were OK.''

Jill Cowan contributed reporting from Los Angeles.

Advertisement

\protect\hyperlink{after-bottom}{Continue reading the main story}

\hypertarget{site-index}{%
\subsection{Site Index}\label{site-index}}

\hypertarget{site-information-navigation}{%
\subsection{Site Information
Navigation}\label{site-information-navigation}}

\begin{itemize}
\tightlist
\item
  \href{https://help.nytimes3xbfgragh.onion/hc/en-us/articles/115014792127-Copyright-notice}{©~2020~The
  New York Times Company}
\end{itemize}

\begin{itemize}
\tightlist
\item
  \href{https://www.nytco.com/}{NYTCo}
\item
  \href{https://help.nytimes3xbfgragh.onion/hc/en-us/articles/115015385887-Contact-Us}{Contact
  Us}
\item
  \href{https://www.nytco.com/careers/}{Work with us}
\item
  \href{https://nytmediakit.com/}{Advertise}
\item
  \href{http://www.tbrandstudio.com/}{T Brand Studio}
\item
  \href{https://www.nytimes3xbfgragh.onion/privacy/cookie-policy\#how-do-i-manage-trackers}{Your
  Ad Choices}
\item
  \href{https://www.nytimes3xbfgragh.onion/privacy}{Privacy}
\item
  \href{https://help.nytimes3xbfgragh.onion/hc/en-us/articles/115014893428-Terms-of-service}{Terms
  of Service}
\item
  \href{https://help.nytimes3xbfgragh.onion/hc/en-us/articles/115014893968-Terms-of-sale}{Terms
  of Sale}
\item
  \href{https://spiderbites.nytimes3xbfgragh.onion}{Site Map}
\item
  \href{https://help.nytimes3xbfgragh.onion/hc/en-us}{Help}
\item
  \href{https://www.nytimes3xbfgragh.onion/subscription?campaignId=37WXW}{Subscriptions}
\end{itemize}
