Sections

SEARCH

\protect\hyperlink{site-content}{Skip to
content}\protect\hyperlink{site-index}{Skip to site index}

\href{https://www.nytimes3xbfgragh.onion/section/science}{Science}

\href{https://myaccount.nytimes3xbfgragh.onion/auth/login?response_type=cookie\&client_id=vi}{}

\href{https://www.nytimes3xbfgragh.onion/section/todayspaper}{Today's
Paper}

\href{/section/science}{Science}\textbar{}You Can't Escape Lice, Even
6,500 Feet Below the Ocean

\url{https://nyti.ms/2EvO4UV}

\begin{itemize}
\item
\item
\item
\item
\item
\end{itemize}

Advertisement

\protect\hyperlink{after-top}{Continue reading the main story}

Supported by

\protect\hyperlink{after-sponsor}{Continue reading the main story}

Trilobites

\hypertarget{you-cant-escape-lice-even-6500-feet-below-the-ocean}{%
\section{You Can't Escape Lice, Even 6,500 Feet Below the
Ocean}\label{you-cant-escape-lice-even-6500-feet-below-the-ocean}}

A species of insect tags along with elephant seals as they spend months
at sea, enduring the crushing pressure changes of the mammals' dives.

\includegraphics{https://static01.graylady3jvrrxbe.onion/images/2020/09/01/science/26TB-LICE1/merlin_176170911_4d2970b6-edc4-45ea-a810-14b32c4a56d3-articleLarge.jpg?quality=75\&auto=webp\&disable=upscale}

By Priyanka Runwal

\begin{itemize}
\item
  Aug. 26, 2020
\item
  \begin{itemize}
  \item
  \item
  \item
  \item
  \item
  \end{itemize}
\end{itemize}

\href{https://www.nytimes3xbfgragh.onion/es/2020/08/29/espanol/piojos-sobreviven.html}{Leer
en español}

Darling, it's better under the sea, unless you're an insect. You might
find some bugs skimming the surface of a pond or even
\href{https://www.nytimes3xbfgragh.onion/2017/11/21/science/diving-flies-mono-lake.html}{creating
their own scuba bubble} to dive beneath the surface of inland lakes. But
insects are virtually absent from the open ocean.

If you look at the hind flippers of southern elephant seals, however,
you will find some insects that have made their way to a partially
aquatic life. Lice of the species Lepidophthirus macrorhini ** dwell on
the rear limbs of the large aquatic mammals, which spend nearly 10
months of the year in Antarctic waters and dive up to 6,500 feet below
the surface in search of food and may stay under for nearly two hours at
a time.

These lice could be the deepest surviving insects in marine ecosystems,
according to
a\href{https://jeb.biologists.org/content/early/2020/07/16/jeb.226811}{study}
published in July in the Journal of Experimental Biology. By enduring
such extreme environments, elephant seal lice can help scientists
unravel the mystery of why so few insects have made a home in the
ocean's vastness.

L. macrorhini are parasitic, bloodsucking lice that burrow into the
seal's top skin layer to feed. In 2015, María Soledad Leonardi, a marine
biologist at the Instituto de Biología de Organismos Marinos in
Argentina, found live lice on male elephant seals that surfaced to breed
on King George Island off the coast of Antarctica.

``You can see them with your naked eye,'' she said. ``They look like
miniature crabs.''

To her, the presence of lice on adult seals emerging from lengthy
offshore excursions suggested that the insects could survive the deep
dives and steep climbs of the seals' aquatic journeys. And that meant
the lice might be able to endure the crushing pressure of the ocean's
depths.

\includegraphics{https://static01.graylady3jvrrxbe.onion/images/2020/08/26/science/26TB-LICE2/26TB-LICE2-articleLarge.jpg?quality=75\&auto=webp\&disable=upscale}

Catching 8,000-pound seals at sea to check if lice braved these extreme
conditions would be very tricky, Dr. Soledad Leonardi said. So, her team
decided to bring the lice to the lab.

Using tweezers, they pulled the insects from the hind flippers of 15
elephant seal pups born on the beaches of Península Valdés in Argentina.
The pups\href{https://www.publish.csiro.au/zo/ZO9650437}{harbor adult
lice} that are transferred from their mothers' bodies within a few days
of birth. The lice quickly reproduce, taking advantage of the initial
weeks that the pups are confined to land, as
their\href{https://www.publish.csiro.au/zo/ZO9650437}{eggs don't hatch
underwater}.

In the lab, the team immersed the lice in individual flash-drive-size
chambers filled with seawater that connected to a scuba tank. Then, they
exposed each louse to a range of water pressures, as much as 200 times
greater than that at the sea surface and equivalent to depths ranging
between 980 and 6,500 feet. After experiencing 10 minutes of this
deep-sea environment, 69 of 75 lice emerged alive.

``It was fascinating for me to see that they survived the high
pressure,'' said Claudio Lazzari, an insect physiologist at the
University of Tours in France and a co-author of the study. ``It shows
that these lice can cope. We can exclude that they just die.''

The researchers then exposed surviving lice to a water pressure higher
or lower than what they were subject to earlier.

``The idea was to reproduce the situation that lice would experience
when their host dives through different pressure levels,'' Dr. Lazzari
said. All of the lice were able to tolerate the quick pressure change,
with adults recovering faster and exhibiting mobility after the
experiment, as compared to the nymphs.

Stuart Humphries, an evolutionary biophysicist at the University of
Lincoln in England, called the study ``neat,'' but also said that ``it'd
be interesting to know how the lice do it.''

So far, the researchers don't know if seal lice have special
adaptations. ``My guess is that these guys just shut down and lock their
tracheal system,'' Dr. Humphries said, meaning that the lice could hold
their breath in deep water.

The researchers are now looking to conduct experiments to see if these
insects arrest their activity and energy expenditure in the deep sea or
if they continue breathing.

``Understanding how this group of insects manages to survive underwater
will be the key to understanding why other groups couldn't,'' Dr.
Lazzari said.

But some scientists think the lice could be a unique case.

``Seal lice are a specialized case; they only live attached to their
host in marine environments and reproduce when the seals are on land,''
said Lanna Cheng, an emeritus marine biologist from Scripps Institution
of Oceanography in San Diego. ``Whether or not they have the ability to
survive as free-living insects at those depths, we have no idea.''

Advertisement

\protect\hyperlink{after-bottom}{Continue reading the main story}

\hypertarget{site-index}{%
\subsection{Site Index}\label{site-index}}

\hypertarget{site-information-navigation}{%
\subsection{Site Information
Navigation}\label{site-information-navigation}}

\begin{itemize}
\tightlist
\item
  \href{https://help.nytimes3xbfgragh.onion/hc/en-us/articles/115014792127-Copyright-notice}{©~2020~The
  New York Times Company}
\end{itemize}

\begin{itemize}
\tightlist
\item
  \href{https://www.nytco.com/}{NYTCo}
\item
  \href{https://help.nytimes3xbfgragh.onion/hc/en-us/articles/115015385887-Contact-Us}{Contact
  Us}
\item
  \href{https://www.nytco.com/careers/}{Work with us}
\item
  \href{https://nytmediakit.com/}{Advertise}
\item
  \href{http://www.tbrandstudio.com/}{T Brand Studio}
\item
  \href{https://www.nytimes3xbfgragh.onion/privacy/cookie-policy\#how-do-i-manage-trackers}{Your
  Ad Choices}
\item
  \href{https://www.nytimes3xbfgragh.onion/privacy}{Privacy}
\item
  \href{https://help.nytimes3xbfgragh.onion/hc/en-us/articles/115014893428-Terms-of-service}{Terms
  of Service}
\item
  \href{https://help.nytimes3xbfgragh.onion/hc/en-us/articles/115014893968-Terms-of-sale}{Terms
  of Sale}
\item
  \href{https://spiderbites.nytimes3xbfgragh.onion}{Site Map}
\item
  \href{https://help.nytimes3xbfgragh.onion/hc/en-us}{Help}
\item
  \href{https://www.nytimes3xbfgragh.onion/subscription?campaignId=37WXW}{Subscriptions}
\end{itemize}
