Sections

SEARCH

\protect\hyperlink{site-content}{Skip to
content}\protect\hyperlink{site-index}{Skip to site index}

\href{/section/climate}{Climate}\textbar{}Heat, Smoke and Covid Are
Battering the Workers Who Feed America

\url{https://nyti.ms/31sgb07}

\begin{itemize}
\item
\item
\item
\item
\item
\end{itemize}

\hypertarget{wildfires-in-the-west}{%
\subsubsection{\texorpdfstring{\href{https://www.nytimes3xbfgragh.onion/spotlight/california-wildfires?name=styln-california-wildfires\&region=TOP_BANNER\&block=storyline_menu_recirc\&action=click\&pgtype=Article\&impression_id=47e74880-f52b-11ea-acb7-bd1162cef823\&variant=undefined}{Wildfires
in the West}}{Wildfires in the West}}\label{wildfires-in-the-west}}

\begin{itemize}
\tightlist
\item
  live\href{https://www.nytimes3xbfgragh.onion/2020/09/12/us/wildfires-live-updates.html?name=styln-california-wildfires\&region=TOP_BANNER\&block=storyline_menu_recirc\&action=click\&pgtype=Article\&impression_id=47e74881-f52b-11ea-acb7-bd1162cef823\&variant=undefined}{Fires
  Updates}
\item
  \href{https://www.nytimes3xbfgragh.onion/interactive/2020/us/fires-map-tracker.html?name=styln-california-wildfires\&region=TOP_BANNER\&block=storyline_menu_recirc\&action=click\&pgtype=Article\&impression_id=47e74882-f52b-11ea-acb7-bd1162cef823\&variant=undefined}{Maps
  of the Fires}
\item
  \href{https://www.nytimes3xbfgragh.onion/article/wildfires-photos-california-oregon-washington-state.html?name=styln-california-wildfires\&region=TOP_BANNER\&block=storyline_menu_recirc\&action=click\&pgtype=Article\&impression_id=47e74883-f52b-11ea-acb7-bd1162cef823\&variant=undefined}{Photos}
\item
  \href{https://www.nytimes3xbfgragh.onion/2020/09/10/us/climate-change-california-wildfires.html?name=styln-california-wildfires\&region=TOP_BANNER\&block=storyline_menu_recirc\&action=click\&pgtype=Article\&impression_id=47e74884-f52b-11ea-acb7-bd1162cef823\&variant=undefined}{A
  Climate Reckoning}
\item
  \href{https://www.nytimes3xbfgragh.onion/article/wildfires-california-oregon-washington.html?name=styln-california-wildfires\&region=TOP_BANNER\&block=storyline_menu_recirc\&action=click\&pgtype=Article\&impression_id=47e76f90-f52b-11ea-acb7-bd1162cef823\&variant=undefined}{Answers
  to Your Questions}
\item
  \href{https://www.nytimes3xbfgragh.onion/2020/09/09/us/california-wildfires.html?name=styln-california-wildfires\&region=TOP_BANNER\&block=storyline_menu_recirc\&action=click\&pgtype=Article\&impression_id=47e76f91-f52b-11ea-acb7-bd1162cef823\&variant=undefined}{Newsletter}
\end{itemize}

\includegraphics{https://static01.graylady3jvrrxbe.onion/images/2020/08/24/climate/00CLI-FARMWORKERS1/merlin_176031285_574f29c7-62d6-442b-b4a5-68a24de3095d-articleLarge.jpg?quality=75\&auto=webp\&disable=upscale}

Inequity at the boiling point

\hypertarget{heat-smoke-and-covid-are-battering-the-workers-who-feed-america}{%
\section{Heat, Smoke and Covid Are Battering the Workers Who Feed
America}\label{heat-smoke-and-covid-are-battering-the-workers-who-feed-america}}

Farmworkers in California harvested corn in the predawn hours during a
heat wave this month.Credit...

Supported by

\protect\hyperlink{after-sponsor}{Continue reading the main story}

By \href{https://www.nytimes3xbfgragh.onion/by/somini-sengupta}{Somini
Sengupta}

Photographs by Brian L. Frank

\begin{itemize}
\item
  Published Aug. 25, 2020Updated Sept. 4, 2020
\item
  \begin{itemize}
  \item
  \item
  \item
  \item
  \item
  \end{itemize}
\end{itemize}

STOCKTON, Calif --- Work began in the dark. At 4 a.m., Briseida Flores
could make out a fire burning in the distance. Floodlights illuminated
the fields. And shoulder to shoulder with dozens of others, Ms. Flores
pushed into the rows of corn. Swiftly, they plucked. One after the
other. First under the lights, then by the first rays of daylight.

By 10:30 a.m., it was unbearably hot. Hundreds of wildfires were burning
to the north, and so much smoke was settling into the San Joaquin Valley
that the local air pollution agency issued a health alert. Ms. Flores,
19, who had joined her mother in the fields after her father lost his
job in the early days of the coronavirus pandemic, found it hard to
breathe in between the tightly planted rows. Her jeans were soaked with
sweat.

``It felt like a hundred degrees in there,'' Ms. Flores said. ``We said
we don't want to go in anymore.''

She went home, exhausted, and slept for an hour.

All this to harvest dried, ocher-colored ears of corn meant to decorate
the autumn table.

Like the gossamer layer of ash and dust that is settling on the trees in
Central
\href{https://www.nytimes3xbfgragh.onion/2020/09/04/us/california-heat-wave.html}{California},
climate change is adding on to the hazards already faced by some of the
country's poorest, most neglected laborers. So far this year, more than
7,000 fires have scorched 1.4 million acres, and there is no reprieve in
sight, officials warned.

\includegraphics{https://static01.graylady3jvrrxbe.onion/images/2020/08/24/climate/00CLI-FARMWORKERS2/merlin_176031540_ea6efa19-6d90-4f85-8403-75b00244f145-articleLarge.jpg?quality=75\&auto=webp\&disable=upscale}

Image

Workers in the San Joaquin Valley recently harvested corn before dawn to
avoid the worst heat of the day.

Image

Briseida Flores, 19, went to work in the fields after her father lost
his job.

Summer days are
\href{https://www.ncdc.noaa.gov/cag/statewide/time-series/4/tavg/ann/7/1895-2020?base_prd=true\&begbaseyear=1901\&endbaseyear=2000}{hotter
than they were a century ago} in the already scorching San Joaquin
Valley; the nights, when the body would normally cool down, are warming
faster.
\href{https://oehha.ca.gov/media/downloads/climate-change/report/2018caindicatorsreportmay2018.pdf}{Heat
waves are more frequent}. And across the state, fires have burned over a
million acres in less than two weeks. One recent scientific paper
concluded that
\href{https://iopscience.iop.org/article/10.1088/1748-9326/ab83a7}{climate
change had doubled the frequency of extreme fire weather days} since the
1980s.

In the valley is where the smoke gets stuck when the wind blows it in
from the north and south.

Still, hundreds of thousands of men and women like Ms. Flores continue
to pluck, weed, and pack produce for the nation here, as temperatures
soar into the triple digits for days at a time and the air turns to a
soup of dust and smoke, stirred with pollution from truck tailpipes and
chemicals sprayed on the fields, not to mention
\href{https://theconversation.com/living-near-active-oil-and-gas-wells-in-california-tied-to-low-birth-weight-and-smaller-babies-140034}{pollution
from the old oil wells} that dot parts of the valley.

I drove through the valley last week, from Lodi, just below Sacramento,
to Arvin, nearly 300 miles to the south, during a calamitous wave of
heat, fire and surging coronavirus infections. I wanted to see it
through the eyes of those worst affected: agricultural workers. Most of
them are immigrants from Mexico. Mostly, they earn minimum wage (\$13 an
hour in California). Mostly, they lack health insurance and they live
amid chronic pollution, making them susceptible to a host of respiratory
ailments.

Climate change exacerbates these horrors.

By noon one day last week, temperatures had soared to 100 degrees
Fahrenheit in Lodi, in the valley's northern stretch. Still, Leonor
Hernández, 38, mother of three, was at work. Dressed as usual in an
oversized full-sleeved shirt and hat, bandanna covering all but her
eyes, water bottle stuffed into her pocket, she walked up and down the
cherry orchard, scooping up stray branches hacked off after the harvest,
hoisting them into a bin. The ground had to be cleared for the next
spraying of pesticides, smoke or no smoke.

As the week progressed and more acres burned, the air grew increasingly
toxic. Her head and chest hurt. She was coughing. The
\href{https://www.valleyair.org/recent_news/Media_releases/2020/HC-Smoke-Impacts-from-various-wildfires-08-21-20.pdf}{San
Joaquin Valley Air Pollution Control District} urged residents to stay
indoors.

Good advice, in theory, Ms. Hernández said. ``But we need to work, and
if we stay indoors we don't get paid,'' she said. ``We have bills for
food and rent to pay.''

California is one of two states, along with Washington, with heat
standards for outdoor workers. Employers must provide shade, usually a
bench with a canopy, and drinking water. Many labor contractors stop
work when it gets too hot, but the law doesn't require a halt at any
given temperature threshold.

Image

Leonor Hernández at her home in Lodi, Calif. ``If we stay indoors we
don't get paid,'' she said.

Image

Haze settled over the San Joaquin Valley. At one point this month,
wildfires burned more than a million acres across California in less
than two weeks.

Image

Farmworkers tended carrots near Arvin, Calif., during an August heat
wave.

The problem of intensifying heat underscores a more basic problem. If
you work fewer hours, you make less. And for those who get paid at piece
rates --- wine grape pickers generally get paid by the bin --- there can
be a perverse incentive to work as fast as possible, even if it means
skipping a water break.

``It's the price of cheap food,'' said Armando Elenes,
secretary-treasurer of the United Farm Workers of America, which
advocated for heat standards in California 15 years ago after a spate of
farmworker deaths. The union is pushing for similar national
legislation.

In the cherry orchard, Ms. Hernández yelled out to one of her
co-workers, an older woman whose face and arms were exposed to the
elements and wet with sweat. She told her to take a break, drink water.
``We are taking a lot of care of each other,'' Ms. Hernández said.

Like many of her co-workers, she doesn't have health insurance, so
seeing a doctor is an unaffordable luxury. Twice last year in a heat
wave, Ms. Hernández was sick: nausea, headache, stomach ache. ``I
learned,'' she recalled. ``I said, `No more.'''

Work stopped shortly after noon. It was 102 degrees Fahrenheit, or about
39 Celsius. Ms. Hernández drove home, showered, prepared to meet with
her 12-year-old son's teacher about remote learning. School, she hoped,
would save her children from the fields. ``School is very important to
me,'' she said.

Not far from the cherry orchard, the residents of the Shady Rest mobile
home park came home in the afternoon to find neither shade nor rest. The
power had gone off because, the residents said, the electricity supply
in the complex is insufficient for the number of trailers. That meant no
water. No air-conditioning. And, with no internet, no school.

\href{https://www.nytimes3xbfgragh.onion/spotlight/california-wildfires}{Wildfires
in the West ›}

\hypertarget{live-updates}{%
\subsection{\texorpdfstring{\href{https://www.nytimes3xbfgragh.onion/2020/09/12/us/wildfires-live-updates.html}{Live
Updates}}{Live Updates}}\label{live-updates}}

Updated~

Sept. 12, 2020, 2:53 p.m. ET

\begin{itemize}
\tightlist
\item
  \href{https://www.nytimes3xbfgragh.onion/2020/09/12/us/wildfires-live-updates.html\#link-f3961ff}{President
  Trump will visit California on Monday after destructive fires.}
\item
  \href{https://www.nytimes3xbfgragh.onion/2020/09/12/us/wildfires-live-updates.html\#link-7e503ae9}{Shifting
  weather may improve firefighting conditions on the West Coast.}
\item
  \href{https://www.nytimes3xbfgragh.onion/2020/09/12/us/wildfires-live-updates.html\#link-5e4c548d}{Oregon's
  fire marshal is temporarily replaced as firefighters battle blazes.}
\end{itemize}

``All you want to do is shower, cook and stay cool, but you can't,''
said Laura Villagran, who came home from her shift at a tree nursery,
covered in grime and sweat.

The owner, Lal Singh Toor, said he did not know why the power was out.
The complex, he said, has a 400 amp electrical service, a level
\href{https://homeguides.sfgate.com/check-amp-size-houses-electrical-service-72409.html}{usually
adequate} for two to three large single-family homes. Shady Rest has 49
units.

Image

Residents of the Shady Rest mobile home park in Stockton, Calif.
Inadequate power in the complex cut children off from online classes
this month .

Image

Geography and industry have cursed parts of California with some of the
country's worst air, but many residents of the Shady Rest park lack
health insurance.

Image

Laundry day at Shady Rest.

The San Joaquin Valley is a vast bowl of industrial farmland nestled
between the Pacific Coast ranges and the Sierra Nevadas. Table grapes,
wine grapes, watermelons, carrots, and blueberries are all grown and
packed here. So are acres and acres of almonds and walnuts.

Geography and industry curse the valley with some of the
\href{https://www.eurekalert.org/pub_releases/2020-04/ala-nho041720.php}{country's
worst air}. Rates of
\href{https://www.cdph.ca.gov/Programs/CCDPHP/DEODC/EHIB/CPE/Pages/CaliforniaBreathingCountyAsthmaProfiles.aspx}{asthma}
and chronic obstructive pulmonary disease run high, according to doctors
at Clinica Sierra Vista, a network of medical centers in the valley.
Kidney functions decline with prolonged dehydration among many
agricultural workers, doctors in the region say. Diabetes --- associated
with eating inexpensive, starchy food --- is common. There's even a
respiratory ailment named for the area: Valley Fever, caused by
coccidioides fungus in the soil.

Dr. Olga Meave, chief medical officer at the Clinica Sierra Vista, spoke
of the battery of ailments that agricultural workers face. ``They're
going to be more prone to chronic respiratory ailments,'' she said.

Little wonder, then, that coronavirus infection rates in the valley
\href{https://www.cdph.ca.gov/Programs/CID/DCDC/Pages/COVID-19/COVID19CountyDataTable.aspx}{are
among the highest in California}. Latinos are disproportionately
infected.

``Work is seasonal,'' said Jose Rodriguez, head of a Stockton-based
group called El Concilio, which provides services for agricultural
workers. ``If they don't work, they're not going to make it through the
year.'' Hunger runs high. Twice as many people showed up for his group's
food distribution session last week as he had food for.

In the fields outside Stockton last week, the air became thicker and
smokier each day. By the week's end, Ms. Flores could feel it. ``It's
really bad,'' she said. ``You can smell the smoke and it hurts your
head.''

Image

Rates of asthma and chronic obstructive pulmonary disease run high in
the San Joaquin Valley.

Image

``If they don't work, they're not going to make it through the year,'' a
community activist said of field laborers.

Image

Mamy grape pickers get paid by the bin, so there can be a perverse
incentive to work as fast as possible, even if it means skipping a water
break.

The valley is
\href{https://droughtmonitor.unl.edu/CurrentMap/StateDroughtMonitor.aspx?CA}{abnormally
dry in parts,} and in drought in others. Dust swirls up from the fields
like a genie. Many creek beds are parched. The rivers have been twisted
and bent every which way to bring water from the north for the fields.
Since mid-August, for over two weeks, daily high temperatures have
ranged from 97 degrees Fahrenheit to 108.

By Thursday, ash fell over Kern County, the valley's southernmost
stretch. The sun struggled to break through. By midafternoon, it looked
like a glowing, ghostly orb.

In the fields near the town of Arvin, Alejandro Díaz, knife in hand,
bucket strapped to his chest, clipped the last grapes hanging on the
vines. Snip. Toss. Unload buckets into bins to make inexpensive table
wine. Two bins would fetch \$65, and if he and his work partner, Rafael
Pacheco, could put in a few hours before the heat roasted them, they
might pocket \$100 each.

It was muggy among the vines. ``Suffocating,'' Mr. Pacheco said. ``You
can't breathe.''

Mr. Diaz's face was wet with sweat. Dust from the vines filled in the
grooves. He said they would stop at 11 a.m., before it got to 102
degrees Fahrenheit. ``My life,'' Mr. Diaz said, ``is worth more than
another round of grapes.''

Image

Advertisement

\protect\hyperlink{after-bottom}{Continue reading the main story}

\hypertarget{site-index}{%
\subsection{Site Index}\label{site-index}}

\hypertarget{site-information-navigation}{%
\subsection{Site Information
Navigation}\label{site-information-navigation}}

\begin{itemize}
\tightlist
\item
  \href{https://help.nytimes3xbfgragh.onion/hc/en-us/articles/115014792127-Copyright-notice}{©~2020~The
  New York Times Company}
\end{itemize}

\begin{itemize}
\tightlist
\item
  \href{https://www.nytco.com/}{NYTCo}
\item
  \href{https://help.nytimes3xbfgragh.onion/hc/en-us/articles/115015385887-Contact-Us}{Contact
  Us}
\item
  \href{https://www.nytco.com/careers/}{Work with us}
\item
  \href{https://nytmediakit.com/}{Advertise}
\item
  \href{http://www.tbrandstudio.com/}{T Brand Studio}
\item
  \href{https://www.nytimes3xbfgragh.onion/privacy/cookie-policy\#how-do-i-manage-trackers}{Your
  Ad Choices}
\item
  \href{https://www.nytimes3xbfgragh.onion/privacy}{Privacy}
\item
  \href{https://help.nytimes3xbfgragh.onion/hc/en-us/articles/115014893428-Terms-of-service}{Terms
  of Service}
\item
  \href{https://help.nytimes3xbfgragh.onion/hc/en-us/articles/115014893968-Terms-of-sale}{Terms
  of Sale}
\item
  \href{https://spiderbites.nytimes3xbfgragh.onion}{Site Map}
\item
  \href{https://help.nytimes3xbfgragh.onion/hc/en-us}{Help}
\item
  \href{https://www.nytimes3xbfgragh.onion/subscription?campaignId=37WXW}{Subscriptions}
\end{itemize}
