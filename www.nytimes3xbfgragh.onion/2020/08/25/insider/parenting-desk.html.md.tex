Sections

SEARCH

\protect\hyperlink{site-content}{Skip to
content}\protect\hyperlink{site-index}{Skip to site index}

\href{https://www.nytimes3xbfgragh.onion/section/reader-center}{Times
Insider}

\href{https://myaccount.nytimes3xbfgragh.onion/auth/login?response_type=cookie\&client_id=vi}{}

\href{https://www.nytimes3xbfgragh.onion/section/todayspaper}{Today's
Paper}

\href{/section/reader-center}{Times Insider}\textbar{}Helping Parents
Hold the Line

\url{https://nyti.ms/3gvEBtI}

\begin{itemize}
\item
\item
\item
\item
\item
\item
\end{itemize}

Advertisement

\protect\hyperlink{after-top}{Continue reading the main story}

Supported by

\protect\hyperlink{after-sponsor}{Continue reading the main story}

Times Insider

\hypertarget{helping-parents-hold-the-line}{%
\section{Helping Parents Hold the
Line}\label{helping-parents-hold-the-line}}

Whether it's covering issues about safety, schooling or burnout, the
members of the Parenting desk are working during this ``new reality'' to
support caregivers on all fronts.

\includegraphics{https://static01.graylady3jvrrxbe.onion/images/2020/08/25/insider/25insider-parenting-03/25insider-parenting-03-articleLarge.jpg?quality=75\&auto=webp\&disable=upscale}

By Emma Grillo

\begin{itemize}
\item
  Aug. 25, 2020
\item
  \begin{itemize}
  \item
  \item
  \item
  \item
  \item
  \item
  \end{itemize}
\end{itemize}

\href{https://www.nytimes3xbfgragh.onion/series/times-insider}{\emph{Times
Insider}} \emph{explains who we are and what we do, and delivers
behind-the-scenes insights into how our journalism comes together.}

If Farah Miller, the director of content strategy for the
\href{https://www.nytimes3xbfgragh.onion/section/parenting}{Parenting
desk}, ever needs a reminder about how important the coverage of her
field is during the pandemic, she needs only to take a look through the
door of her home office, where she
\href{https://www.instagram.com/stories/highlights/17847143816283838/}{recently
snapped a picture} of her young daughter Bergen with her face pressed
against the glass, trying to get in.

``We are trying to parent and work through this madness,'' Ms. Miller
said, ``and it is really, really hard.''

Giving parents guidance on that challenge, which has encompassed a vast
number of issues that the pandemic has raised, has been the focus of the
Parenting team since March. If that work has only intensified some five
months later because of the uncertainty lingering about how schools
nationwide will teach children this fall, so has the team's resolve.

``Many of us, for months, had no child care outside our homes, and we
were also trying to manage distance learning like so many other folks,''
Jessica Grose, the editor of Parenting, said. ``I think everyone on the
desk is so grateful that we still have jobs, and knows that many of our
readers are in much tougher straits, and we've felt a real sense of
responsibility to get them the information they've needed about how to
best live their lives in this new reality.''

\includegraphics{https://static01.graylady3jvrrxbe.onion/images/2020/08/25/insider/25insider-parenting/25insider-parenting-articleLarge.jpg?quality=75\&auto=webp\&disable=upscale}

Ms. Grose leads a team of eight full-time staff members and a group of
freelance contributors. Together, they're producing 40 to 50 articles
per month, she said, spanning a spectrum of parental concerns:
educational issues, like what to know about
\href{https://www.nytimes3xbfgragh.onion/2020/07/22/parenting/school-pods-coronavirus.html}{learning
pods}; physical health issues, like why children
\href{https://www.nytimes3xbfgragh.onion/2020/06/24/parenting/virus-kids-sick-quarantine-infection.html}{still
get sick even during lockdown}; development issues, like furthering
\href{https://www.nytimes3xbfgragh.onion/2020/06/18/parenting/kids-social-needs-quarantine.html}{children's
social skills} despite isolation; societal issues, like how to
\href{https://www.nytimes3xbfgragh.onion/2020/06/03/parenting/kids-books-racism.html}{talk
to kids about racism}; and emotional issues, like ways to handle
\href{https://www.nytimes3xbfgragh.onion/2020/06/23/parenting/parental-burnout-coronavirus.html?searchResultPosition=3}{burnout}.

``In March, when everything started, the entire national conversation
became a conversation about parenting, in a way that was beneficial and
overwhelming to us as a desk,'' Ms. Miller said. ``We focus a lot on
service journalism, so there were just so many questions that parents
had that we wanted to answer. This was a story that just had a million
different threads to follow.''

No matter the topic, though, all Parenting articles reflect the team's
guiding principles that all material must be nonjudgmental and be
grounded in evidence and science.

The team also produces a newsletter twice a week, uses its
\href{https://www.instagram.com/nytparenting/?hl=en}{social media}
\href{https://twitter.com/NYTParenting?ref_src=twsrc\%5Egoogle\%7Ctwcamp\%5Eserp\%7Ctwgr\%5Eauthor}{accounts}
to spotlight articles related to parenting from across the newsroom and
continues to create special projects: In May, the team published a
collection of essays focused on
\href{https://www.nytimes3xbfgragh.onion/2020/05/07/parenting/motherhood-changes-us-all.html}{motherhood
and transformation}, and a recent package of profiles featured
\href{https://www.nytimes3xbfgragh.onion/2020/08/18/parenting/homeschool-coronavirus.html}{families
that have opted for home-schooling}.

Ms. Grose said that in the beginning of the pandemic, the desk talked to
infectious disease pediatricians every day to bring readers the most
up-to-date information. As the pandemic wears on, the desk continues to
consult pediatricians, as well as psychologists, psychiatrists and
social workers. And the team stays in close contact with the Science,
Metro and National desks to ensure that Parenting's coverage is
``additive, and not duplicative,'' Ms. Grose said.

When brainstorming ideas, the team pays close attention to reader
feedback. It also helps that many of the Parenting staff members are
raising young children themselves, and often come to pitch meetings with
ideas that stem from their own lives.

Ms. Miller handles audience development as part of her role and
frequently reads comments from, and interacts with, readers on the
desk's \href{https://www.instagram.com/nytparenting/?hl=en}{Instagram
page}. She often brings this feedback to the team so members can think
about the best way to address caregivers' most pressing concerns.

The team has also made it a priority to talk about the mental health of
parents. ``We do a lot of coverage about how parents can cope right now;
it's not just all about kids,'' Ms. Grose said. ``The amount of anxiety
and rapid change that has happened is really hard.''

Early in the pandemic, as editors and reporters explored recurring
themes of parents and children living similar lives in front of screens,
Deanna Donegan, the senior visual editor for Parenting, created a visual
approach to the articles that counterbalanced the heaviness of the
moment.

``Content-wise, Parenting has done a really good job of giving parents
resources of how to handle this very anxious time when there isn't a
one-size-fits-all plan for anyone,'' Ms. Donegan said. ``It's nice to
try to infuse some of that calm and reassurance, and also provide some
levity maybe, in the art.''

Because the team members know firsthand just how stressful it is to
raise children right now, they've built flexibility into their
schedules.

One editor with a newborn starts work at 5 a.m. so she can finish by the
afternoon, and Ms. Grose plans her day around taking a few hours after
lunch to spend time with her children.

Ms. Miller, who has been covering parenting for over a decade, said she
felt lucky to be working on this subject during such an extraordinary
time. She has seen many of the same hot-button issues come up repeatedly
during her career, but because of the pandemic, ``there are new problems
to solve,'' she said. ``Being able to help parents do that has been
really gratifying, as scary as everything is.''

Image

Someone has to take a break for Cosmic Kids Yoga. The wife of the
parenting reporter Christina Caron tagged in for a session with their
3-year-old.Credit...Christina Caron

Advertisement

\protect\hyperlink{after-bottom}{Continue reading the main story}

\hypertarget{site-index}{%
\subsection{Site Index}\label{site-index}}

\hypertarget{site-information-navigation}{%
\subsection{Site Information
Navigation}\label{site-information-navigation}}

\begin{itemize}
\tightlist
\item
  \href{https://help.nytimes3xbfgragh.onion/hc/en-us/articles/115014792127-Copyright-notice}{©~2020~The
  New York Times Company}
\end{itemize}

\begin{itemize}
\tightlist
\item
  \href{https://www.nytco.com/}{NYTCo}
\item
  \href{https://help.nytimes3xbfgragh.onion/hc/en-us/articles/115015385887-Contact-Us}{Contact
  Us}
\item
  \href{https://www.nytco.com/careers/}{Work with us}
\item
  \href{https://nytmediakit.com/}{Advertise}
\item
  \href{http://www.tbrandstudio.com/}{T Brand Studio}
\item
  \href{https://www.nytimes3xbfgragh.onion/privacy/cookie-policy\#how-do-i-manage-trackers}{Your
  Ad Choices}
\item
  \href{https://www.nytimes3xbfgragh.onion/privacy}{Privacy}
\item
  \href{https://help.nytimes3xbfgragh.onion/hc/en-us/articles/115014893428-Terms-of-service}{Terms
  of Service}
\item
  \href{https://help.nytimes3xbfgragh.onion/hc/en-us/articles/115014893968-Terms-of-sale}{Terms
  of Sale}
\item
  \href{https://spiderbites.nytimes3xbfgragh.onion}{Site Map}
\item
  \href{https://help.nytimes3xbfgragh.onion/hc/en-us}{Help}
\item
  \href{https://www.nytimes3xbfgragh.onion/subscription?campaignId=37WXW}{Subscriptions}
\end{itemize}
