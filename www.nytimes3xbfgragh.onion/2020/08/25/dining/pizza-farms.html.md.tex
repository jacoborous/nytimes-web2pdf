Sections

SEARCH

\protect\hyperlink{site-content}{Skip to
content}\protect\hyperlink{site-index}{Skip to site index}

\href{https://www.nytimes3xbfgragh.onion/section/food}{Food}

\href{https://myaccount.nytimes3xbfgragh.onion/auth/login?response_type=cookie\&client_id=vi}{}

\href{https://www.nytimes3xbfgragh.onion/section/todayspaper}{Today's
Paper}

\href{/section/food}{Food}\textbar{}Where the Produce Includes
Pepperoni: The Pizza Farm

\url{https://nyti.ms/2Qo8eCE}

\begin{itemize}
\item
\item
\item
\item
\item
\item
\end{itemize}

\href{https://www.nytimes3xbfgragh.onion/spotlight/at-home?action=click\&pgtype=Article\&state=default\&region=TOP_BANNER\&context=at_home_menu}{At
Home}

\begin{itemize}
\tightlist
\item
  \href{https://www.nytimes3xbfgragh.onion/2020/09/07/travel/route-66.html?action=click\&pgtype=Article\&state=default\&region=TOP_BANNER\&context=at_home_menu}{Cruise
  Along: Route 66}
\item
  \href{https://www.nytimes3xbfgragh.onion/2020/09/04/dining/sheet-pan-chicken.html?action=click\&pgtype=Article\&state=default\&region=TOP_BANNER\&context=at_home_menu}{Roast:
  Chicken With Plums}
\item
  \href{https://www.nytimes3xbfgragh.onion/2020/09/04/arts/television/dark-shadows-stream.html?action=click\&pgtype=Article\&state=default\&region=TOP_BANNER\&context=at_home_menu}{Watch:
  Dark Shadows}
\item
  \href{https://www.nytimes3xbfgragh.onion/interactive/2020/at-home/even-more-reporters-editors-diaries-lists-recommendations.html?action=click\&pgtype=Article\&state=default\&region=TOP_BANNER\&context=at_home_menu}{Explore:
  Reporters' Google Docs}
\end{itemize}

Advertisement

\protect\hyperlink{after-top}{Continue reading the main story}

Supported by

\protect\hyperlink{after-sponsor}{Continue reading the main story}

\hypertarget{where-the-produce-includes-pepperoni-the-pizza-farm}{%
\section{Where the Produce Includes Pepperoni: The Pizza
Farm}\label{where-the-produce-includes-pepperoni-the-pizza-farm}}

This Midwestern staple has grown ever more popular in the pandemic,
bringing farmers and diners together in a socially distanced summer
destination.

\includegraphics{https://static01.graylady3jvrrxbe.onion/images/2020/08/25/dining/25pizzafarms11/25pizzafarms11-articleLarge-v2.jpg?quality=75\&auto=webp\&disable=upscale}

\href{https://www.nytimes3xbfgragh.onion/by/julia-moskin}{\includegraphics{https://static01.graylady3jvrrxbe.onion/images/2018/09/25/multimedia/author-julia-moskin/author-julia-moskin-thumbLarge.png}}

By \href{https://www.nytimes3xbfgragh.onion/by/julia-moskin}{Julia
Moskin}

\begin{itemize}
\item
  Aug. 25, 2020
\item
  \begin{itemize}
  \item
  \item
  \item
  \item
  \item
  \item
  \end{itemize}
\end{itemize}

In parts of the Upper Midwest, the phrase ``pizza farm'' is as evocative
of summer food as ``juneberry pie,''
``\href{https://www.nytimes3xbfgragh.onion/2019/08/12/us/politics/on-politics-iowa-fair-booker-ferris-wheel.html}{butter
cow}'' and ``rutabaga festival.'' (The
\href{http://www.cumberland-wisconsin.com/rutabaga-events-2020}{88th
annual celebration} begins Saturday in Cumberland, Wis.)

Trace the route of the Mississippi River down from
\href{https://twincities.eater.com/maps/pizza-farm-minnesota-wisconsin}{Minneapolis}
and along the Wisconsin-Iowa border. In the last two decades, dozens of
farms in this region have built wood-fired ovens, studied the basics of
crust, sauce and cheese, and begun serving pizza on summer nights.

Families haul in stacks of camp chairs; couples on dates sip wine on
picnic blankets; children poke at the animals and run themselves out by
dark. Pizza is usually the only item on the menu, but the homegrown
toppings change with the season, from baby onions and roasted carrots in
the spring to zucchini, fried eggplant and red peppers as the season
closes out in September. In tune with the farm-to-table, local and
sustainable food movements, pizza farming has spread across the country.

\includegraphics{https://static01.graylady3jvrrxbe.onion/images/2020/08/25/dining/25pizzafarms9/merlin_176115768_4653a4ee-616f-4e11-92b2-34b403485478-articleLarge.jpg?quality=75\&auto=webp\&disable=upscale}

Ted Fisher and Robbi Bannen of
\href{https://atozproduceandbakery.com/}{the A to Z farm} in Stockholm,
Wis., started serving pizza in 1998, and many other farmers credit them
as local pioneers. They kept it up every summer because, they say, it
has supported them in so many ways: using up gluts of summer produce,
giving their three children skills and spending money, and drawing
locals onto the farm. Those customers became regulars --- and, often,
C.S.A. members, the lifeblood of the farm.

``That's why this is like a death in the family,'' Ms. Bannen said of
their decision to cancel pizza nights this summer because of the
coronavirus pandemic.

But some pizza farms are having a blockbuster season: Like drive-in
movie theaters, they happen to be perfectly suited to outdoor operations
and social distancing. Across the country, new pizza farms have sprung
up this summer, as farmers
\href{https://www.nytimes3xbfgragh.onion/2020/04/09/dining/farm-to-table-coronavirus.html}{scramble
to make up} for lost sales to restaurants and other big customers.

``The landscape has completely changed,'' said Liz Neumark, who just
opened up her \href{https://www.katchkiefarm.com/}{Katchkie Farm} in
Kinderhook, N.Y., to a weekend collaboration with
\href{https://www.instagram.com/hilltownhotpies/}{Hilltown Hot Pies}, a
local specialist. ``I never knew there were pizza groupies.''

Farm-to-table pizza has been around since the food revolution of the
1970s, but usually closer to the table than to the farm. By the time the
Los Angeles chef
\href{https://www.nytimes3xbfgragh.onion/2012/10/31/dining/wolfgang-puck-the-original-celebrity-chef-is-still-keeping-busy.html}{Wolfgang
Puck} made his reputation by turning pizza into high-end restaurant food
in the 1980s, Chez Panisse, in Berkeley, Calif., was
\href{https://www.nytimes3xbfgragh.onion/1984/07/08/magazine/getting-to-know-calzone.html}{already
famous} for its
\href{https://food52.com/blog/25282-my-family-recipe-coming-home-pizza-dough}{wood-fired
pies}, topped with foraged greens like nettles and wild herbs.

Image

Maren and Tom Beard of Luna Valley Farm. Like many pizza farmers, they
emphasize sustainability and community.Credit...Narayan Mahon for The
New York Times

Particularly determined pizza farmers even raise cattle for cheese and
pigs for meat. The popular Iowa Margherita pie at
\href{https://www.lunavalleyfarm.com/}{Luna Valley Farm}, in Decorah,
Iowa, has Italian sausage crumbled over the basic formula of fresh
tomatoes, mozzarella and basil leaves. The only thing Tom and Maren
Beard don't make themselves is drinks.

``We have three breweries in a 10-mile radius, local soda and natural
wine,'' Ms. Beard said, and partnerships with local businesses have also
become part of the pizza farm ethos. (Liquor laws in these states often
allow farms to serve beer and wine.)

Many pizza farmers also see themselves as activists and teachers.
Educational sites like Katchkie Farm, the
\href{https://edibleschoolyard.org/berkeley}{Edible Schoolyard} in
Berkeley, Calif., and the
\href{https://www.sustainablefood.yale.edu/the-yale-farm}{Yale Farm} in
New Haven, Conn., have long used pizza ovens to introduce students to
the principles (and pleasures) of diverse, local and sustainable
agriculture. For these farmers, pizza is an ideal teaching tool: Even a
small farm can raise wheat for dough, tomatoes and herbs for sauce, and
an array of vegetables for toppings.

``We didn't start making pizza as a concept; we were just trying to find
something that worked for us.'' Ms. Bannen said. ``I'm waiting for
someone to open a taco farm.''

Here are five popular pizza farms:

Image

Emily Knudsen and her partner, Bill Bartz, started Pleasant Grove Pizza
Farm in Waseca, Minn., in 2014.Credit...Jenn Ackerman for The New York
Times

\hypertarget{pleasant-grove-pizza-farm}{%
\subsection{Pleasant Grove Pizza Farm}\label{pleasant-grove-pizza-farm}}

\hypertarget{waseca-minn}{%
\subsubsection{Waseca, Minn.}\label{waseca-minn}}

When Emily Knudsen took her new boyfriend, Bill Bartz, to his first
pizza night at a nearby farm, he had a ``Sound of Music'' moment. ``He
was like Julie Andrews twirling around on that mountain,'' she said,
ready to jump feet-first into her longtime dream.

Unlike many pizza farmers, neither of them had worked on a farm --- Ms.
Knudsen, 37, has a background in hospitality and wedding planning, and
Mr. Bartz, also 37, in construction --- but they threw themselves into
it, starting from scratch in 2014. Although they grow some ingredients
(basil, tomatoes, peppers and honey), their specialty is lavish
toppings, loaded onto pies like the Buster (two kinds of mozzarella,
sausage, caramelized onions, mushrooms, basil), the Sweet Georgia
(mozzarella, prosciutto, arugula, goat cheese, honey) and the Pig and
Pork (tomato sauce, mozzarella, sausage, pepperoni and green olives).

Image

Like many pizza farms, Pleasant Grove is having a bumper season this
year as people try to gather safely and eat outdoors.Credit...Jenn
Ackerman for The New York Times

Like many modern farmers, they have branched out from agriculture in
multiple directions, hosting corporate picnics and wedding rehearsal
dinners, and adding attractions like baby animals, live music and local
crafts to pull in a wider audience. (Waseca is a little more than an
hour's drive south of the Twin Cities). By staying open this summer, the
farm has had its busiest pizza season ever, while enforcing a new
reservations-only policy and strict social distancing rules. ``I'm a
hammer,'' Ms. Knudsen said with pride. ``You should see me with those
brides and their mothers.''

41142 160th Street, Waseca, Minn.; 715-523-0857;
\href{https://www.pleasantgrovepizzafarm.com/}{pleasantgrovepizzafarm.com}

Image

This summer, pizza night at Hawkins Family Farm is even more
labor-intensive than usual.Credit...A J Mast for The New York Times

\hypertarget{hawkins-family-farm}{%
\subsection{Hawkins Family Farm}\label{hawkins-family-farm}}

\hypertarget{north-manchester-ind}{%
\subsubsection{North Manchester, Ind.}\label{north-manchester-ind}}

This farm in Wabash County has been in the Hawkins family for four
generations. But Jeff Hawkins, 65, said he became a farmer only when he
resigned as pastor of a Lutheran Evangelical parish to devote himself to
agriculture. ``Just like farming had become industrial,'' he said,
churches had become concerned with cost-effectiveness and productivity.
``Pastors were becoming more like C.E.O.s and less like shepherds.''

Image

In the Upper Midwest, where pepperoni and sausage are must-have
toppings, some pizza farmers raise pigs so that all the ingredients come
from the farm.~ ~Credit...A J Mast for The New York Times

With his son, Zach, 38, Mr. Hawkins began diversifying the farm, where
they now produce vegetables, honey, grains, pigs, chickens, goats,
cattle and --- on Friday nights --- pizza. They committed to making the
farm a sustainable business that also funds a nonprofit dedicated to
reconnecting members of the clergy with agriculture, which Mr. Hawkins
believes promotes healthy and holistic leadership. ``I was called to
this farm,'' he said.

Image

To some pizza farmers, like Zach Hawkins, the crust is as important as
the farm produce on top of it.Credit...A J Mast for The New York Times

Zach had also left the farm to attend college in Iowa, where he dove
into sustainable farming and dabbled in sourdough; on his return, he was
handed responsibility for the family's pizza recipe and style. ``I'm
always pushing the char,'' he said: blistered, blackened, puffy edges
are prized signs of a well-cooked wood-fired pizza.

They bring the same holistic approach to partnerships with the
community: A local bakery makes the dough, and a young farmer trying to
keep his family's nearby dairy farm going makes the mozzarella. Chefs
from restaurants in Fort Wayne, 40 miles to the east, are invited to
visit and experiment, using anything that grows on the farm as a pizza
topping. Kimchi and carrot-top pesto were recent hits.

``There aren't many `Old MacDonald' farms left,'' Mr. Hawkins said,
explaining why opening diverse, sustainable farms to the public is so
important. ``Pizza helps people see what agriculture really means.''

10373 N 300 E, North Manchester, Ind.; no phone;
\href{http://www.hawkinsfamilyfarm.com/}{hawkinsfamilyfarm.com}

Image

One of the original pizza farms in the Upper Midwest, A to Z Produce and
Bakery is closed this summer for the first time since 1998.Credit...AtoZ
Produce and Bakery

\hypertarget{a-to-z-produce-and-bakery}{%
\subsection{A to Z Produce and Bakery}\label{a-to-z-produce-and-bakery}}

\hypertarget{stockholm-wis}{%
\subsubsection{Stockholm, Wis.}\label{stockholm-wis}}

Ted Fisher and Robbi Bannen are not only credited as pioneers of the
pizza farm; they also believe they coined the term soon after they began
in 1998, quoting a 5-year-old visitor, Ms. Bannen said.

``We were food people way back when,'' in the 1970s, Mr. Fisher said.
(He is now 62, Ms. Bannen is 61.) A skilled bread baker, he already knew
how to work with naturally leavened doughs, and adapted those skills to
pizza with help from Alice Waters's classic 1980 cookbook, ``Chez
Panisse Pizza, Pasta and Calzones.'' When they put the word out that
they would be selling pizza for takeout one night a week, ``to most
people it looked like an insane thing to do,'' she said. At the time,
most farms did not serve hot food, any more than restaurants grew
produce --- though now both practices are common.

They grow all the vegetables for tomato sauce and toppings, and wheat
for the flour. ``There's nothing better than being able to use up the
food that we grow,'' Ms. Bannen said. Rather than guess how many peppers
they'll need in a given year, or whether anyone will buy a new kind of
kale they want to try growing, they can take chances, knowing that
everything can end up on a pizza.

One night, a takeout customer asked to set up a folding table in the
yard. After that, everyone was invited to sprawl out under the stars. (A
to Z has been closed this summer, however, because of the virus.) Now
the farm has fed a generation: Visitors have had first dates, proposed
marriage and fed their babies at A to Z pizza nights.

``A lot of people around here no longer have family farms that they
remember,'' she said. ``After 22 years, that's what we are ---
everyone's family farm.''

N2956 Anker Lane, Stockholm, Wis.; 715-448-4802;
\href{https://atozproduceandbakery.com/}{atozproduceandbakery.com}

Image

When Amber Waves Farm in Amagansett, N.Y., resumed pizza nights in July,
locals showed up in droves.~ Many pizza farms are having blockbuster
summers.~Credit...Joe Carrotta for The New York Times

\hypertarget{amber-waves-farm}{%
\subsection{Amber Waves Farm}\label{amber-waves-farm}}

\hypertarget{amagansett-ny}{%
\subsubsection{Amagansett, N.Y.}\label{amagansett-ny}}

To most pizza farmers (and pizza lovers), the basic components of pizza
are the crust, the sauce, and the cheese. For Amanda Merrow and Katie
Baldwin, there is a crucial fourth: the flour.

Their farm, Amber Waves, is part of a growing movement to return the
cultivation of grain --- like wheat, oats, sorghum, barley --- to small
regional farms, instead of relying on industrial giants in the Great
Plains. Local grain, they say, has the
\href{https://www.nytimes3xbfgragh.onion/2018/02/27/dining/row-7-seed-company-dan-barber.html}{same
benefits} as local fruits and vegetables, contributing to flavor,
sustainability, and community.

Image

On ambitious pizza farms like Amber Waves, as many ingredients as
possible are grown on the farm, like wheat, tomatoes, garlic and
basil.Credit...Joe Carrotta for The New York Times

The sourdough starter for their pizza crust was donated by a previous
owner of the farm who had reportedly kept it going
\href{https://www.nytimes3xbfgragh.onion/2015/02/01/nyregion/on-the-east-end-of-long-island-embracing-local-wheat.html}{since
1968}, and they have collaborated with
\href{https://www.carissasthebakery.com/}{Carissa's}, a nearby bakery
famous for creative breads like pickled rye and slow-risen baguettes.
``The taste of flour made from freshly harvested grain is a totally
different thing,'' said Ms. Merrow, 35. ``And bakers around here are
able to appreciate that.''

When they opened Amber Waves in 2008, one of their first investments was
a mobile pizza oven. They introduced themselves to the community with
D.I.Y. family pizza nights: Children grab a dough ball, learn to roll it
out, then wander into the garden to pick the toppings.

``Even kids who don't eat vegetables will eat pizza,'' said Ms. Baldwin,
39. ``Once a cherry tomato is on a pizza, it's not a vegetable anymore
--- it's a topping.''

Image

At some new pizza farms, the farmers put an expert in charge on pizza
nights. (Here, Liam Stegman of Amber Waves.)Credit...Joe Carrotta for
The New York Times

Amber Waves is moving in the opposite direction of development in its
ultrawealthy part of the Hamptons, using land for cultivation instead of
installing swimming pools. But much of the area was farmland until
relatively recently; potatoes, corn, wheat and more have thrived in this
soil for hundreds of years.

``We're not bringing this in,'' Ms. Merrow said. ``We're bringing it
back.''

367 Main St, Amagansett, N.Y.; 631-267-5664;
\href{https://www.amberwavesfarm.org/}{amberwavesfarm.org}

Image

Curtis Millsap of Millsap Farms started an all-you-can-eat weekly pizza
night to draw locals to visit the farm --- and to appreciate the
diversity of its agriculture.~Credit...Christopher Smith for The New
York Times

\hypertarget{millsap-farms}{%
\subsection{Millsap Farms}\label{millsap-farms}}

\hypertarget{springfield-mo}{%
\subsubsection{Springfield, Mo.}\label{springfield-mo}}

Curtis Millsap first heard about pizza farms soon after he and his wife
started farming in 2007, from Wwoofers arriving from Wisconsin. Those
volunteer workers from
\href{https://wwoofusa.org/?gclid=EAIaIQobChMImPqDu5Cq6wIVCACGCh30LQTkEAAYASAAEgJnvPD_BwE}{Worldwide
Opportunities on Organic Farms} hop from farm to farm as needed, often
carrying new ideas and innovations with them.

Image

Sarah and Curtis Millsap have raised 10 children on their farm. This
month, they celebrated their 20th anniversary on pizza
night.Credit...Christopher Smith for The New York Times

``It sounded like this magical thing,'' with fairy lights in the trees,
live music and wood-fired pizza under the stars, said Mr. Millsap, 44.
He and his wife, Sarah, 42, have raised 10 children on the farm, four
miles north of central Springfield. As the family has grown, their goals
have shifted from raising more vegetables to building more
relationships. They brought on a local mason to make them a mud oven ---
the original high-heat baking technology --- and enlisted their extended
family to help.

``If we were going to have the kind of community we wanted,'' Mr.
Millsap said, ``we would have to build it right here on the farm.''

Today, they host volunteers from all over the world, ask C.S.A. members
to work alongside them during planting and harvest, and entice locals to
the farm with all-you-can-eat pizza on Thursday nights from May to
October. The conversation about the week's offerings begins around the
kitchen table on Monday mornings, with the question: What do we have a
lot of this week?

Image

To maintain social distancing, pizza farms have spaced out their tables
and eliminated buffet-style service this summer.~~Credit...Christopher
Smith for The New York Times

Their most popular pie in the spring is topped with roasted beets, feta
cheese and walnuts. High summer brings bacon, arugula and tomato, with a
schmear of garlic mayonnaise.

``If you can grow it,'' Mr. Millsap said, ``we have put it on a pizza.''

6593 North Emu Lane, Springfield, Mo.; 417-839-0847;
\href{https://www.millsapfarms.com/}{millsapfarms.com}

\emph{Follow} \href{https://twitter.com/nytfood}{\emph{NYT Food on
Twitter}} \emph{and}
\href{https://www.instagram.com/nytcooking/}{\emph{NYT Cooking on
Instagram}}\emph{,}
\href{https://www.facebookcorewwwi.onion/nytcooking/}{\emph{Facebook}}\emph{,}
\href{https://www.youtube.com/nytcooking}{\emph{YouTube}} \emph{and}
\href{https://www.pinterest.com/nytcooking/}{\emph{Pinterest}}\emph{.}
\href{https://www.nytimes3xbfgragh.onion/newsletters/cooking}{\emph{Get
regular updates from NYT Cooking, with recipe suggestions, cooking tips
and shopping advice}}\emph{.}

Advertisement

\protect\hyperlink{after-bottom}{Continue reading the main story}

\hypertarget{site-index}{%
\subsection{Site Index}\label{site-index}}

\hypertarget{site-information-navigation}{%
\subsection{Site Information
Navigation}\label{site-information-navigation}}

\begin{itemize}
\tightlist
\item
  \href{https://help.nytimes3xbfgragh.onion/hc/en-us/articles/115014792127-Copyright-notice}{©~2020~The
  New York Times Company}
\end{itemize}

\begin{itemize}
\tightlist
\item
  \href{https://www.nytco.com/}{NYTCo}
\item
  \href{https://help.nytimes3xbfgragh.onion/hc/en-us/articles/115015385887-Contact-Us}{Contact
  Us}
\item
  \href{https://www.nytco.com/careers/}{Work with us}
\item
  \href{https://nytmediakit.com/}{Advertise}
\item
  \href{http://www.tbrandstudio.com/}{T Brand Studio}
\item
  \href{https://www.nytimes3xbfgragh.onion/privacy/cookie-policy\#how-do-i-manage-trackers}{Your
  Ad Choices}
\item
  \href{https://www.nytimes3xbfgragh.onion/privacy}{Privacy}
\item
  \href{https://help.nytimes3xbfgragh.onion/hc/en-us/articles/115014893428-Terms-of-service}{Terms
  of Service}
\item
  \href{https://help.nytimes3xbfgragh.onion/hc/en-us/articles/115014893968-Terms-of-sale}{Terms
  of Sale}
\item
  \href{https://spiderbites.nytimes3xbfgragh.onion}{Site Map}
\item
  \href{https://help.nytimes3xbfgragh.onion/hc/en-us}{Help}
\item
  \href{https://www.nytimes3xbfgragh.onion/subscription?campaignId=37WXW}{Subscriptions}
\end{itemize}
