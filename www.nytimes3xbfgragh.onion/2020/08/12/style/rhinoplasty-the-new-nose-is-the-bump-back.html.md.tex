Sections

SEARCH

\protect\hyperlink{site-content}{Skip to
content}\protect\hyperlink{site-index}{Skip to site index}

\href{https://www.nytimes3xbfgragh.onion/section/style}{Style}

\href{https://myaccount.nytimes3xbfgragh.onion/auth/login?response_type=cookie\&client_id=vi}{}

\href{https://www.nytimes3xbfgragh.onion/section/todayspaper}{Today's
Paper}

\href{/section/style}{Style}\textbar{}The New Nose: Is the Bump Back?

\url{https://nyti.ms/31Nkh1T}

\begin{itemize}
\item
\item
\item
\item
\item
\end{itemize}

Advertisement

\protect\hyperlink{after-top}{Continue reading the main story}

Supported by

\protect\hyperlink{after-sponsor}{Continue reading the main story}

Skin Deep

\hypertarget{the-new-nose-is-the-bump-back}{%
\section{The New Nose: Is the Bump
Back?}\label{the-new-nose-is-the-bump-back}}

The cookie-cutter ski jump nose is a relic now, as cosmetic doctors
approach rhinoplasties with a more artistic, individualized approach.

\includegraphics{https://static01.graylady3jvrrxbe.onion/images/2020/08/13/fashion/12SKIN-NOSES-Art/12SKIN-NOSES-Art-articleLarge.gif?quality=75\&auto=webp\&disable=upscale}

By Tatiana Boncompagni

\begin{itemize}
\item
  Published Aug. 12, 2020Updated Aug. 14, 2020
\item
  \begin{itemize}
  \item
  \item
  \item
  \item
  \item
  \end{itemize}
\end{itemize}

Twelve years ago, a young woman named Elina met with a number of plastic
surgeons about her nose, which she described as having a pronounced bump
and bulbous tip, traits she had inherited from her father. Each surgeon
showed Elina, then 18 years old, an ``after'' rendition of herself that
she described as jarring.

``I remember telling my mom: `This doesn't look like me. This doesn't
feel right,''' she said. ``I thought, `I'm going to have to live with my
nose,' and I put it on the back burner.''

But then, while scrolling through Instagram, Elina, now 30 and a
creative director in Manhattan who asked to be identified only by her
first name, stumbled upon Dr. Dara Liotta, a facial plastic surgeon who
specializes in natural-looking nose jobs. After a consultation, she
plunked down \$15,000 for a rhinoplasty with Dr. Liotta and is thrilled
with her new, slightly altered profile.

``She left me this bump,'' Elina said. ``It looks so natural. I saw a
bunch of friends, and they didn't notice.''

Gone is the era of the cookie-cutter, ski jump nose. Today's cosmetic
doctors are approaching rhinoplasties with a more artistic,
individualized approach, using new surgical and nonsurgical techniques
to refine rather than revamp one of the most prominent features of the
face. ``You are taught to do rhinoplasty a certain way, but now it's
much less mathematical,'' said Dr. Matthew White, a facial plastic
surgeon in New York. ``It's about really spending time talking to the
patient and figuring out what they are looking for.''

Wendy Lewis, an aesthetics industry consultant, remembers when a
particular New York surgeon --- Dr. Howard Diamond --- was so well known
for creating a specific nose, one with a scooped-out bridge and pinched
tip, that it was called ``a Dr. Diamond'' nose.

``That was the standard, and by today's standards that would be
unacceptable,'' Ms. Lewis said.

According to the American Academy of Facial Plastic and Reconstructive
Surgery, rhinoplasty, which typically costs between \$5,000 and
\$15,000, remained the most popular facial plastic surgery in 2019. The
group surveyed 774 of its members and from the information they
provided, the academy estimated that 55,000 surgeries were performed in
the United States last year.

``The data shows that rhinoplasties are as popular as ever,'' said Dr.
Patrick Byrne, a plastic surgeon at the Cleveland Clinic, pointing to
``the ubiquitous nature of social media'' and use of smartphone cameras
--- he says they distort the proportions of the face, making the nose
look larger --- as key drivers of the demand.

For Nikki Kanter, 36, a digital marketing consultant in Chicago, wanting
to look better in pictures was a factor in her decision to get a nose
job. ``In pictures, if I didn't angle my face right, I looked like a
boxer,'' said Ms. Kanter, who had broken her nose in college when she
was hit in the face by a bag of beads thrown from a float at a Mardi
Gras parade. Her rhinoplasty fixed the damage that had been done
internally, straightened the bridge of her nose and fixed the
``droopiness'' of the tip.

``Honestly it's one of the best things I've done for myself,'' said Ms.
Kanter, who added that she doesn't look drastically different but is
more confident.

In the last decade, doctors have abandoned so-called reduction
rhinoplasty techniques in favor of a new ``structural'' approach.
Instead of breaking the nose and then cutting away and removing
cartilage and bone to reshape the nose and make it smaller, surgeons now
rework (but don't remove) the cartilage in an effort to build a more
structurally sound nose that isn't going to collapse with the passing of
years.

``That's why the tip of noses pinch with time, because there's not
enough cartilage to support it,'' Dr. Liotta said.

Aside from being more structurally sound, the new, subtle nose job also
reflects evolving social norms and standards of beauty. Laurie Essig, a
professor and director of gender, sexuality and feminist studies at
Middlebury College, pointed out that the first nose jobs were performed
in the 1800s, mostly on Irish men who wanted to look more ``white.''

``It is a racialized and gendered project,'' said Ms. Essig, the author
of ``American Plastic: Boob Jobs, Credit Cards, and Our Quest for
Perfection.''

Dr. Melissa Doft, a plastic surgeon in New York, noted that many
patients seeking nose jobs don't want to lose their ethnicity. ``Our
grandparents' generation wanted to assimilate as quickly as possible,
but the next generation is proud of their heritage,'' Dr. Doft said.

Dr. Doft has also noted a recent uptick in patients who had reduction
rhinoplasties years ago and now want to correct the ``sharp angles which
we used to think were beautiful but have aged the patient.'' She cited
\href{https://academic.oup.com/asj/article-abstract/40/5/493/5648144?redirectedFrom=fulltext}{a
study recently published in
the}\href{https://academic.oup.com/asj/article-abstract/40/5/493/5648144?redirectedFrom=fulltext}{**}\href{https://academic.oup.com/asj/article-abstract/40/5/493/5648144?redirectedFrom=fulltext}{Aesthetic
Medical Journal} that used computer software to demonstrate that a
well-done rhinoplasty could make a patient look three years younger.

``Very little has been talked about the nose and aging, but as we get
older, the nose loses its softness and roundness,'' Dr. Doft said.

To make noses look more youthful, Dr. Babak Azizzadeh, a plastic surgeon
in Beverly Hills, Calif., grafts fascia --- that is, connective tissue
--- taken from above a patient's ear on to the bridge of the nose. The
fascia provides a layer of tissue that mimics the look of thicker, more
supple skin. ``It's a game changer,'' Dr. Azizzadeh said of the
technique.

The same results can come from injectable fillers, say some
dermatologists, who use hyaluronic acid fillers not only to soften
angles on the nose, but also to more dramatically contour it. These are
often referred to as liquid rhinoplasties, in which aesthetic doctors
use filler and neurotoxins to camouflage a bump or lift the tip of the
nose by injecting the muscle beneath the nose with Botox.

Dr. Dendy Engelman, a dermatologist in New York, said she has seen an
increase in patients asking about liquid rhinoplasties and attributes
the rise to social media, where the procedure is often touted by doctors
because of the dramatic before-and-after photos that appear to be tricks
of the eye.

``With noses, you think about tiny tweaks, almost airbrushing of the
nose,'' Dr. Engelman said. ``It's what we do with makeup tricks, using
how light reflects to make the nose look less prominent.''

But some doctors caution that injecting filler into the nose is riskier
than in other parts of the face because of the number of arteries at the
midline of the face. If filler gets into an artery, it can cause skin
necrosis (dying) and possibly blindness, as Dr. Engelman acknowledged.
She emphasized that patients should make sure their doctor has plenty of
experience and training in liquid rhinoplasty.

Dr. Paul Frank, a dermatologist in New York and the author of ``The
Pro-Aging Playbook,'' said he is concerned about the longer-term effects
of injecting filler into cartilage.

``We just don't know what kind of damage this could be doing to the
nose,'' Dr. Frank said. ``In the right hands, a rhinoplasty is safe and
long-lasting.''

Advertisement

\protect\hyperlink{after-bottom}{Continue reading the main story}

\hypertarget{site-index}{%
\subsection{Site Index}\label{site-index}}

\hypertarget{site-information-navigation}{%
\subsection{Site Information
Navigation}\label{site-information-navigation}}

\begin{itemize}
\tightlist
\item
  \href{https://help.nytimes3xbfgragh.onion/hc/en-us/articles/115014792127-Copyright-notice}{©~2020~The
  New York Times Company}
\end{itemize}

\begin{itemize}
\tightlist
\item
  \href{https://www.nytco.com/}{NYTCo}
\item
  \href{https://help.nytimes3xbfgragh.onion/hc/en-us/articles/115015385887-Contact-Us}{Contact
  Us}
\item
  \href{https://www.nytco.com/careers/}{Work with us}
\item
  \href{https://nytmediakit.com/}{Advertise}
\item
  \href{http://www.tbrandstudio.com/}{T Brand Studio}
\item
  \href{https://www.nytimes3xbfgragh.onion/privacy/cookie-policy\#how-do-i-manage-trackers}{Your
  Ad Choices}
\item
  \href{https://www.nytimes3xbfgragh.onion/privacy}{Privacy}
\item
  \href{https://help.nytimes3xbfgragh.onion/hc/en-us/articles/115014893428-Terms-of-service}{Terms
  of Service}
\item
  \href{https://help.nytimes3xbfgragh.onion/hc/en-us/articles/115014893968-Terms-of-sale}{Terms
  of Sale}
\item
  \href{https://spiderbites.nytimes3xbfgragh.onion}{Site Map}
\item
  \href{https://help.nytimes3xbfgragh.onion/hc/en-us}{Help}
\item
  \href{https://www.nytimes3xbfgragh.onion/subscription?campaignId=37WXW}{Subscriptions}
\end{itemize}
