Sections

SEARCH

\protect\hyperlink{site-content}{Skip to
content}\protect\hyperlink{site-index}{Skip to site index}

\href{https://www.nytimes3xbfgragh.onion/section/style}{Style}

\href{https://myaccount.nytimes3xbfgragh.onion/auth/login?response_type=cookie\&client_id=vi}{}

\href{https://www.nytimes3xbfgragh.onion/section/todayspaper}{Today's
Paper}

\href{/section/style}{Style}\textbar{}The Great Gotham Vroom Boom of
2020

\url{https://nyti.ms/2DJOhnc}

\begin{itemize}
\item
\item
\item
\item
\item
\item
\end{itemize}

\href{https://www.nytimes3xbfgragh.onion/spotlight/at-home?action=click\&pgtype=Article\&state=default\&region=TOP_BANNER\&context=at_home_menu}{At
Home}

\begin{itemize}
\tightlist
\item
  \href{https://www.nytimes3xbfgragh.onion/2020/09/07/travel/route-66.html?action=click\&pgtype=Article\&state=default\&region=TOP_BANNER\&context=at_home_menu}{Cruise
  Along: Route 66}
\item
  \href{https://www.nytimes3xbfgragh.onion/2020/09/04/dining/sheet-pan-chicken.html?action=click\&pgtype=Article\&state=default\&region=TOP_BANNER\&context=at_home_menu}{Roast:
  Chicken With Plums}
\item
  \href{https://www.nytimes3xbfgragh.onion/2020/09/04/arts/television/dark-shadows-stream.html?action=click\&pgtype=Article\&state=default\&region=TOP_BANNER\&context=at_home_menu}{Watch:
  Dark Shadows}
\item
  \href{https://www.nytimes3xbfgragh.onion/interactive/2020/at-home/even-more-reporters-editors-diaries-lists-recommendations.html?action=click\&pgtype=Article\&state=default\&region=TOP_BANNER\&context=at_home_menu}{Explore:
  Reporters' Google Docs}
\end{itemize}

Advertisement

\protect\hyperlink{after-top}{Continue reading the main story}

Supported by

\protect\hyperlink{after-sponsor}{Continue reading the main story}

\hypertarget{the-great-gotham-vroom-boom-of-2020}{%
\section{The Great Gotham Vroom Boom of
2020}\label{the-great-gotham-vroom-boom-of-2020}}

New Yorkers have historically been lukewarm on car ownership. Then came
the you-know-what.

\includegraphics{https://static01.graylady3jvrrxbe.onion/images/2020/08/13/fashion/12VIRUS-CARS-eli/12VIRUS-CARS-eli-articleLarge-v2.jpg?quality=75\&auto=webp\&disable=upscale}

By Foster Kamer

\begin{itemize}
\item
  Aug. 12, 2020
\item
  \begin{itemize}
  \item
  \item
  \item
  \item
  \item
  \item
  \end{itemize}
\end{itemize}

There's a special place in hell for the authors of those trend stories
that concern New Yorkers discovering things for the first time as though
they are new.

This is absolutely one of those trends, and its author, one of its
subjects.

Why hell? By the logic of these trend stories, nothing is truly
discovered until a certain subset of New Yorkers --- the most hardheaded
middle-class cosmopolitan denizens at the absolute center of modern
civilization --- try it on for size, for the first time, however
fashionably late we may be. And what have we discovered now?

Cars! Buying and leasing, four wheels, the smell of gas and the joys of
the open road. Emissions-spewing, fossil-fuel-guzzling automobiles ---
have you heard? They help you leave places quickly, and in a
hermetically sealed bubble where the only expelled aerosols we have to
inhale are the best kind: our own.

For a large segment of city-dwelling, taxi-taking, Citi Biking New
Yorkers who swore off private transport long ago, the prospect of owning
a ride here seemed at best unnecessary; at worst, a colossal and
cumbersome time suck.

Cars can also be money pits, and most clash with the green ideals of
many a city dweller.

In the early spring throes of bending the curve, like many large cities,
New York saw emptier roadways and lower levels of emissions than there
had been in decades --- and with this, cleaner air. But drivers are
hitting the road again, and those emissions are
\href{https://www.nytimes3xbfgragh.onion/interactive/2020/06/17/climate/virus-emissions-reopening.html}{already
surging back with them}. Some are referring to the perfect storm of a
mistrust in public transit paired with an increase in car ownership as a
``mobility crisis'' of gridlock:
\href{https://nyc.streetsblog.org/2020/08/04/yet-another-report-yes-mr-mayor-carmageddon-is-coming/}{Carmageddon}.

\includegraphics{https://static01.graylady3jvrrxbe.onion/images/2020/08/13/fashion/12VIRUS-CARS-troy/12VIRUS-CARS-troy-articleLarge-v2.jpg?quality=75\&auto=webp\&disable=upscale}

\hypertarget{how-did-we-get-here}{%
\subsection{How Did We Get Here?}\label{how-did-we-get-here}}

Car ownership in New York was once generally reserved for those who
worked in
\href{https://www.nytimes3xbfgragh.onion/2019/10/11/realestate/whos-afraid-of-a-transit-desert.html}{transit
deserts}, those who use ``summer'' as a verb, those with second homes
(these last two often being the same person), who could afford garages,
or those who lived in the more suburban parts of the outer boroughs,
with actual driveways. Not owning a car, not knowing how to drive, or
doing so poorly was a point of pride. At least in some circles, to own a
car in New York was to be nothing if not an outlier.

And yet, since mid-March, quarantine-weary New Yorkers started buying
way more cars. And now,
\href{https://www.nytimes3xbfgragh.onion/2020/05/18/nyregion/bike-shortage-coronavirus.html}{like
bicycles before them}, there's a shortage, particularly in the city.

As the pandemic arrived,
\href{https://www.nytimes3xbfgragh.onion/2020/03/18/business/economy/gm-ford-fiatchrysler-factories-virus.html}{auto
factories shut down operations}. New car shoppers weren't leaving their
houses. Dealerships are currently faced with a shortage of factory-new
cars and a shortage of the used cars (whose sales are typically applied
to new car purchases). Pair that with a sudden onslaught of car buyers,
and the result are proliferate supply-side problems
\href{https://twitter.com/thestalwart/status/1291110174879514628?s=21}{keeping
up with demand}.

According to data it provided The New York Times, the New York State
D.M.V. processed 73,933 original car registrations in the five boroughs
over June and July, a 18 percent increase over the 62,507 registrations
from the same time last year.

So many New Yorkers are hitting the road that last Thursday, Mayor Bill
de Blasio even went so far as to
\href{https://twitter.com/danarubinstein/status/1291387068736851969}{advise
them not to buy cars}, saying that they represent ``the past.''

Tom McParland is a columnist for car blogs like Jalopnik and The Drive.
He also owns a business called Automatch Consulting, which helps guide
hapless, often first-time car-buyers (like me, and two others I spoke
with for this article) through the car buying process for a \$500 to
\$1000 fee --- advice on picking the right model, getting a fair price
from dealers, a pre-purchase inspection, and so on.

May was a record month for Mr. McParland. In April, he estimates about a
third of his 30 sales were to New Yorkers. By May he had sold nearly 60
cars with almost half going to New Yorkers.

Image

Jane Lerner bought a used Mazda CX-5 from Carvana.

``The common theme,'' he said, ``is: `I don't trust the safety of
transit, and I don't trust the safety of ride share. I want my own
transportation.'''

Clayton Mantell, the sales manager of Ramsey Subaru, in Ramsey, N.J.,
has seen the same surge himself. ``Easily 50 New Yorkers in July,'' he
said, where normally, ``it'd be like 10.'' Were many of them first-time
buyers?

``Definitely,'' Mr. Mantell said. ``A lot of them just need an escape
pod.''

Among buyers, another common theme emerged. For those lucky enough to
still be collecting a paycheck in these thunderstruck economic times,
disposable income appeared where it once wasn't. (Mr. McParland said
most of his customers were looking to spend between \$15,000 and
\$25,000 for their cars; the people I spoke to paid between \$3,900 and
\$28,000 for theirs.) Money previously taken up for bars, restaurants,
travel and even daily commutes sat untouched for the better part of
three months of a lockdown.

And for some, like Troy Kelley, buying was the financially prudent thing
to do: He had to get to work in Manhattan from New Jersey, the shuttle
service he normally used shut down and he was ripping through money on
private cars. ``I just had to really buckle down and say, `I need to get
a vehicle,''' he said. He ended up with a used white Mercedes c300.
``I've never really enjoyed driving, per se,'' he said, ``but it is good
to have the freedom.''

So, as the logic goes, why not maybe save money and your health with a
roving, sterilizable social distancing machine with air-conditioning,
cupholders and a Bluetooth stereo instead of
\href{https://www.nytimes3xbfgragh.onion/2020/06/30/dining/restaurant-risks-coronavirus.html}{other
options} that might increase close contact with potential disease
carriers, or at the very least require wearing a mask and keeping one's
distance?

And so tree-lined borough blocks that felt bucolic in March are
fiercer-than-ever battlegrounds for alternate-side parking spots, with
temp-plated gladiators duking it out for a place to spend the night.
Cars from Carvana, the online used-car retailer, are being dropped off
in the city, purchased sans test drive (you have seven days to return).

Many of these New Yorkers are finding one of the most valuable
quantities money can buy right now: peace amid a pandemic, relief in
actual escapism.

\hypertarget{a-range-of-rides}{%
\subsection{A Range of Rides}\label{a-range-of-rides}}

Rachel Weiss, 49, a head of innovation at L'Oreal who lives in the
Chelsea neighborhood of Manhattan, leased her new garnet-colored
Cadillac CT5 at the end of July. ``This is my first car, so I went all
in,'' she said. ``I dipped into my savings. I felt trapped.''

On the Upper East Side, Lauren Caretsky, a 35-year-old advertising
manager, finally took up the mantle of her patronymic, springing to
lease a new, gray Honda CR-V in May. ``Haven't had a car since college
about 13 years ago,'' she said.

In Clinton Hill, the political donor organizer Jane Lerner, 49, bought a
used Mazda CX-5 off Carvana, delivered straight to her door. The last
time she had a car? It was 2002.

``I sold that car (a green Honda Civic) to a friend (who drove it off a
cliff in Big Sur) and have driven rarely since,'' Ms. Lerner explained
over a chat app (the friend is OK). ``Over my years in Brooklyn, I've
considered getting a car, but never seriously.'' But just one week into
lockdown? ``It became a very clear decision. Couldn't be happier with
it.''

Erica Lyon of Prospect Heights and Biz Lindsay of Greenpoint, both 33,
sprung for Volkswagens: a used white Golf and a new black Golf Alltrack,
respectively. Alex Faille in Fort Greene, also 33, bought a used Jeep
Wrangler. In Bedford-Stuyvesant, Subarus seemed to be the rage: Al Risi,
a 53-year-old music supervisor, scored a used Forester. Oliver Klein,
31, a restaurant manager, and his fiancé, Molly Stein, 30, who works in
art publishing: A used Crosstrek.

And in Williamsburg, Eli Razavi, 31, and his fiancée Tamara Fine, 27,
went in on a petrol blue-green 1987 Mercedes-Benz 560 SL convertible.
Mr. Razavi had moved to New York from Los Angeles 10 years ago and sold
his car for cash on arrival. ```I'm never getting a car again''' is how
he felt at the time. And then, of course, enter Covid-19:

``I just felt like I was trapped. And I didn't want to take public
transportation or Uber. I didn't want to deal with renting cars because
everything was booked.''

On the other side of the neighborhood, I went much the same route: a red
1990 325i BMW convertible, the kind of Ferris Bueller fever-dream a
single 35-year-old man would make fun of his friends for buying if he
didn't beat them to it.

All of us used to see a private car as an unnecessary luxury, or extra
closet space you had to repark several times a week (also,
\href{https://www.nytimes3xbfgragh.onion/2020/04/30/style/rats-car-engines.html}{rat
housing}). In a pandemic, it's an aerosol-free ticket to essential
places or to freedom: the beach, the countryside,
\href{https://www.nytimes3xbfgragh.onion/2019/10/24/nyregion/wegmans-brooklyn.html}{spacious
grocery stores with parking lots.}

Will that remain once New York sees some semblance of normalcy? Or are
the autos here to stay?

Ms. Lerner, the political donor organizer, admitted that she wasn't
crazy about the implications of buying a car in the city. ``I'm aware
that I'm contributing to the degradation of our environment,'' she said.
``I don't want to drive to Manhattan to go to dinner, you know? I'll
take the subway or I'll walk.'' For her, getting a car ``wasn't so much
to get around New York City --- it was to get \emph{out} of New York
City.'' She's now trying to map out how to visit her sister in Colorado.

Mr. Risi, the music supervisor from Bedford-Stuyvesant, took his new car
into Manhattan this weekend to run some errands, and see some friends.
He's taken it to the beach; he's taken it to Wegmans. He hadn't owned a
car for 16 years until this summer. After being raised in New York, he
moved to California for a stint, before heading back east in 2004. He
had a Volkswagen Jetta, and on moving back to the city, ``gave it to a
friend of mine for really cheap, and it was really liberating. I felt
very liberated.''

``And now, I'm getting a car again, for the same reason,'' he said.
``For liberty.''

Advertisement

\protect\hyperlink{after-bottom}{Continue reading the main story}

\hypertarget{site-index}{%
\subsection{Site Index}\label{site-index}}

\hypertarget{site-information-navigation}{%
\subsection{Site Information
Navigation}\label{site-information-navigation}}

\begin{itemize}
\tightlist
\item
  \href{https://help.nytimes3xbfgragh.onion/hc/en-us/articles/115014792127-Copyright-notice}{©~2020~The
  New York Times Company}
\end{itemize}

\begin{itemize}
\tightlist
\item
  \href{https://www.nytco.com/}{NYTCo}
\item
  \href{https://help.nytimes3xbfgragh.onion/hc/en-us/articles/115015385887-Contact-Us}{Contact
  Us}
\item
  \href{https://www.nytco.com/careers/}{Work with us}
\item
  \href{https://nytmediakit.com/}{Advertise}
\item
  \href{http://www.tbrandstudio.com/}{T Brand Studio}
\item
  \href{https://www.nytimes3xbfgragh.onion/privacy/cookie-policy\#how-do-i-manage-trackers}{Your
  Ad Choices}
\item
  \href{https://www.nytimes3xbfgragh.onion/privacy}{Privacy}
\item
  \href{https://help.nytimes3xbfgragh.onion/hc/en-us/articles/115014893428-Terms-of-service}{Terms
  of Service}
\item
  \href{https://help.nytimes3xbfgragh.onion/hc/en-us/articles/115014893968-Terms-of-sale}{Terms
  of Sale}
\item
  \href{https://spiderbites.nytimes3xbfgragh.onion}{Site Map}
\item
  \href{https://help.nytimes3xbfgragh.onion/hc/en-us}{Help}
\item
  \href{https://www.nytimes3xbfgragh.onion/subscription?campaignId=37WXW}{Subscriptions}
\end{itemize}
