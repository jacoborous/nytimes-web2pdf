Sections

SEARCH

\protect\hyperlink{site-content}{Skip to
content}\protect\hyperlink{site-index}{Skip to site index}

\href{https://myaccount.nytimes3xbfgragh.onion/auth/login?response_type=cookie\&client_id=vi}{}

\href{https://www.nytimes3xbfgragh.onion/section/todayspaper}{Today's
Paper}

Is It OK That I Haven't Told My Parents I Was Fired?

\url{https://nyti.ms/2Xqn9jK}

\begin{itemize}
\item
\item
\item
\item
\item
\item
\end{itemize}

\hypertarget{the-coronavirus-outbreak}{%
\subsubsection{\texorpdfstring{\href{https://www.nytimes3xbfgragh.onion/news-event/coronavirus?name=styln-coronavirus-national\&region=TOP_BANNER\&block=storyline_menu_recirc\&action=click\&pgtype=Article\&impression_id=33bb7a60-f1e9-11ea-8403-e93824161a97\&variant=undefined}{The
Coronavirus
Outbreak}}{The Coronavirus Outbreak}}\label{the-coronavirus-outbreak}}

\begin{itemize}
\tightlist
\item
  live\href{https://www.nytimes3xbfgragh.onion/2020/09/08/world/covid-19-coronavirus.html?name=styln-coronavirus-national\&region=TOP_BANNER\&block=storyline_menu_recirc\&action=click\&pgtype=Article\&impression_id=33bb7a61-f1e9-11ea-8403-e93824161a97\&variant=undefined}{Latest
  Updates}
\item
  \href{https://www.nytimes3xbfgragh.onion/interactive/2020/us/coronavirus-us-cases.html?name=styln-coronavirus-national\&region=TOP_BANNER\&block=storyline_menu_recirc\&action=click\&pgtype=Article\&impression_id=33bb7a62-f1e9-11ea-8403-e93824161a97\&variant=undefined}{Maps
  and Cases}
\item
  \href{https://www.nytimes3xbfgragh.onion/interactive/2020/science/coronavirus-vaccine-tracker.html?name=styln-coronavirus-national\&region=TOP_BANNER\&block=storyline_menu_recirc\&action=click\&pgtype=Article\&impression_id=33bb7a63-f1e9-11ea-8403-e93824161a97\&variant=undefined}{Vaccine
  Tracker}
\item
  \href{https://www.nytimes3xbfgragh.onion/2020/09/02/your-money/eviction-moratorium-covid.html?name=styln-coronavirus-national\&region=TOP_BANNER\&block=storyline_menu_recirc\&action=click\&pgtype=Article\&impression_id=33bb7a64-f1e9-11ea-8403-e93824161a97\&variant=undefined}{Eviction
  Moratorium}
\item
  \href{https://www.nytimes3xbfgragh.onion/interactive/2020/09/02/magazine/food-insecurity-hunger-us.html?name=styln-coronavirus-national\&region=TOP_BANNER\&block=storyline_menu_recirc\&action=click\&pgtype=Article\&impression_id=33bb7a65-f1e9-11ea-8403-e93824161a97\&variant=undefined}{American
  Hunger}
\end{itemize}

Advertisement

\protect\hyperlink{after-top}{Continue reading the main story}

Supported by

\protect\hyperlink{after-sponsor}{Continue reading the main story}

\href{/column/the-ethicist}{The Ethicist}

\hypertarget{is-it-ok-that-i-havent-told-my-parents-i-was-fired}{%
\section{Is It OK That I Haven't Told My Parents I Was
Fired?}\label{is-it-ok-that-i-havent-told-my-parents-i-was-fired}}

\includegraphics{https://static01.graylady3jvrrxbe.onion/images/2020/08/09/magazine/09Ethicist/09Ethicist-articleLarge.jpg?quality=75\&auto=webp\&disable=upscale}

By Kwame Anthony Appiah

\begin{itemize}
\item
  Aug. 4, 2020
\item
  \begin{itemize}
  \item
  \item
  \item
  \item
  \item
  \item
  \end{itemize}
\end{itemize}

\emph{Last year I was abruptly let go from my job. I ramped up my side
business, which was fine except that now the pandemic has caused an
immediate decline --- and potential future decrease --- in business
because my clients cannot quickly adapt to technology to allow for work
to continue. My clients requested that I reschedule work until the fall
of this year; they think that is when business will return to pre-crisis
levels. (This is an industrywide issue as well.)}

\emph{The trouble is that I never told my elderly parents that I was let
go. I no longer talk about the job, a topic we used to discuss
frequently. Now I discourage discussion about it, saying it is a ``toxic
work environment.'' Instead, I steer the discussion toward small
triumphs with my side job, which is now my primary source of income.
During this economic turbulence caused by the pandemic, both parents
often verbalize how blessed our family is that all of their children are
still employed. I continue to play along because I don't want my
octogenarian parents worrying about it.}

\emph{I have not shared the news with my family because I am hurt and
embarrassed by the circumstances of my departure from the job. I am
considering taking legal action to address what I believe to be
disparate treatment. I feel as if discussing this matter will be a
``downer'' for everyone at a time when we all crave good news. I believe
that I have good reasons for not sharing my job loss with my parents,
who tend to share details I would rather keep private.} \emph{Still, I
think I behaved unethically by not fully sharing my situation with them.
What do you think?} Name Withheld

\textbf{It's a sign} of something morally odd about our attitude to
employment that you feel ashamed of having been fired, even though you
think the firing was unjust. (You say ``embarrassed,'' but that's
usually a gentler way of saying the same thing.) If you're right, the
shame should attach not to you but to the people who fired you. And yet
your response is entirely representative. People tend to think that
having a job gives them a kind of moral standing, marks them out as a
contributor, a giver and not a taker.

I'm tempted to say that this is a very American idea. Benjamin Franklin,
in his autobiography, wrote of his ``bold and arduous project of
arriving at moral perfection,'' which involved such maxims as
``Industry: Lose no time; be always employ'd in something useful; cut
off all unnecessary actions.'' (Max Weber later cited Franklin's
cherishing of industry as evidence for what he called the Protestant
ethic.) But then I think of the French film ``Time Out'' and the
Japanese film ``Tokyo Sonata,'' each of which memorably depicts a fired
white-collar worker whose family thinks he is still going to the office.
The workplace as a source of worth is a widespread tenet. So I can see
why you wanted to keep this from your parents. Losing your job --- a
condition that the pandemic has now visited upon tens of millions of
Americans --- imposes harms beyond the financial ones.

\hypertarget{latest-updates-the-coronavirus-outbreak}{%
\section{\texorpdfstring{\href{https://www.nytimes3xbfgragh.onion/2020/09/08/world/covid-19-coronavirus.html?action=click\&pgtype=Article\&state=default\&region=MAIN_CONTENT_1\&context=storylines_live_updates}{Latest
Updates: The Coronavirus
Outbreak}}{Latest Updates: The Coronavirus Outbreak}}\label{latest-updates-the-coronavirus-outbreak}}

Updated 2020-09-08T15:29:57.612Z

\begin{itemize}
\tightlist
\item
  \href{https://www.nytimes3xbfgragh.onion/2020/09/08/world/covid-19-coronavirus.html?action=click\&pgtype=Article\&state=default\&region=MAIN_CONTENT_1\&context=storylines_live_updates\#link-547feae1}{Senate
  Republicans plan to move forward with a scaled-back stimulus package.}
\item
  \href{https://www.nytimes3xbfgragh.onion/2020/09/08/world/covid-19-coronavirus.html?action=click\&pgtype=Article\&state=default\&region=MAIN_CONTENT_1\&context=storylines_live_updates\#link-679303d7}{Nine
  drugmakers pledge to thoroughly vet any coronavirus vaccine.}
\item
  \href{https://www.nytimes3xbfgragh.onion/2020/09/08/world/covid-19-coronavirus.html?action=click\&pgtype=Article\&state=default\&region=MAIN_CONTENT_1\&context=storylines_live_updates\#link-1c973131}{`The
  lockdown killed my father': Farmer suicides add to India's virus
  misery.}
\end{itemize}

\href{https://www.nytimes3xbfgragh.onion/2020/09/08/world/covid-19-coronavirus.html?action=click\&pgtype=Article\&state=default\&region=MAIN_CONTENT_1\&context=storylines_live_updates}{See
more updates}

More live coverage:
\href{https://www.nytimes3xbfgragh.onion/live/2020/09/08/business/stock-market-today-coronavirus?action=click\&pgtype=Article\&state=default\&region=MAIN_CONTENT_1\&context=storylines_live_updates}{Markets}

You understandably don't want your folks to worry, and you might well
feel that it is up to you whether you disclose what happened. Family
relationships don't require frankness about everything. They may, on the
contrary, require discretion about certain things. But deceiving your
parents about your employment status is wrong, a pattern of deception
that isn't in keeping with a loving relationship. You should end the
charade before your parents learn the truth from someone else and are
left feeling betrayed. It won't be easy, I realize. Letting them know
that you haven't been honest with them is bound to be a source of shame
--- this time justified.

\emph{My husband and I are very fortunate financially. We are the
quintessential ``DINKs'' (double income, no kids): We have white-collar
jobs and our savings are good. We could live on his salary even if I
were to lose my job. That seems like a real possibility: While layoffs
may not be imminent, the organization I work for, a nonprofit, was
already on a shrinking budget before the current economic shock. Now the
odds are even higher that they will have to let people go before the
year is out. (I am actively looking for another position.)}

\emph{Should I lose my job, is it ethical for me to claim unemployment
benefits, at a time when an unprecedented number of people are doing the
same? Sure, I would be legally entitled to do so. But I fear that I
would be taking money that I don't really need --- and that someone else
desperately does --- out of a system that seems likely to be spread even
thinner the longer the downturn goes on.}

\emph{If someone can afford not to take unemployment benefits, are they
ethically obligated not to? During these scary economic times, do we
have an obligation to ``flatten the curve'' at the unemployment office
as well as the hospital?} Name Withheld, California

\href{https://www.nytimes3xbfgragh.onion/news-event/coronavirus?action=click\&pgtype=Article\&state=default\&region=MAIN_CONTENT_3\&context=storylines_faq}{}

\hypertarget{the-coronavirus-outbreak-}{%
\subsubsection{The Coronavirus Outbreak
›}\label{the-coronavirus-outbreak-}}

\hypertarget{frequently-asked-questions}{%
\paragraph{Frequently Asked
Questions}\label{frequently-asked-questions}}

Updated September 4, 2020

\begin{itemize}
\item ~
  \hypertarget{what-are-the-symptoms-of-coronavirus}{%
  \paragraph{What are the symptoms of
  coronavirus?}\label{what-are-the-symptoms-of-coronavirus}}

  \begin{itemize}
  \tightlist
  \item
    In the beginning, the coronavirus
    \href{https://www.nytimes3xbfgragh.onion/article/coronavirus-facts-history.html?action=click\&pgtype=Article\&state=default\&region=MAIN_CONTENT_3\&context=storylines_faq\#link-6817bab5}{seemed
    like it was primarily a respiratory illness}~--- many patients had
    fever and chills, were weak and tired, and coughed a lot, though
    some people don't show many symptoms at all. Those who seemed
    sickest had pneumonia or acute respiratory distress syndrome and
    received supplemental oxygen. By now, doctors have identified many
    more symptoms and syndromes. In April,
    \href{https://www.nytimes3xbfgragh.onion/2020/04/27/health/coronavirus-symptoms-cdc.html?action=click\&pgtype=Article\&state=default\&region=MAIN_CONTENT_3\&context=storylines_faq}{the
    C.D.C. added to the list of early signs}~sore throat, fever, chills
    and muscle aches. Gastrointestinal upset, such as diarrhea and
    nausea, has also been observed. Another telltale sign of infection
    may be a sudden, profound diminution of one's
    \href{https://www.nytimes3xbfgragh.onion/2020/03/22/health/coronavirus-symptoms-smell-taste.html?action=click\&pgtype=Article\&state=default\&region=MAIN_CONTENT_3\&context=storylines_faq}{sense
    of smell and taste.}~Teenagers and young adults in some cases have
    developed painful red and purple lesions on their fingers and toes
    --- nicknamed ``Covid toe'' --- but few other serious symptoms.
  \end{itemize}
\item ~
  \hypertarget{why-is-it-safer-to-spend-time-together-outside}{%
  \paragraph{Why is it safer to spend time together
  outside?}\label{why-is-it-safer-to-spend-time-together-outside}}

  \begin{itemize}
  \tightlist
  \item
    \href{https://www.nytimes3xbfgragh.onion/2020/05/15/us/coronavirus-what-to-do-outside.html?action=click\&pgtype=Article\&state=default\&region=MAIN_CONTENT_3\&context=storylines_faq}{Outdoor
    gatherings}~lower risk because wind disperses viral droplets, and
    sunlight can kill some of the virus. Open spaces prevent the virus
    from building up in concentrated amounts and being inhaled, which
    can happen when infected people exhale in a confined space for long
    stretches of time, said Dr. Julian W. Tang, a virologist at the
    University of Leicester.
  \end{itemize}
\item ~
  \hypertarget{why-does-standing-six-feet-away-from-others-help}{%
  \paragraph{Why does standing six feet away from others
  help?}\label{why-does-standing-six-feet-away-from-others-help}}

  \begin{itemize}
  \tightlist
  \item
    The coronavirus spreads primarily through droplets from your mouth
    and nose, especially when you cough or sneeze. The C.D.C., one of
    the organizations using that measure,
    \href{https://www.nytimes3xbfgragh.onion/2020/04/14/health/coronavirus-six-feet.html?action=click\&pgtype=Article\&state=default\&region=MAIN_CONTENT_3\&context=storylines_faq}{bases
    its recommendation of six feet}~on the idea that most large droplets
    that people expel when they cough or sneeze will fall to the ground
    within six feet. But six feet has never been a magic number that
    guarantees complete protection. Sneezes, for instance, can launch
    droplets a lot farther than six feet,
    \href{https://jamanetwork.com/journals/jama/fullarticle/2763852}{according
    to a recent study}. It's a rule of thumb: You should be safest
    standing six feet apart outside, especially when it's windy. But
    keep a mask on at all times, even when you think you're far enough
    apart.
  \end{itemize}
\item ~
  \hypertarget{i-have-antibodies-am-i-now-immune}{%
  \paragraph{I have antibodies. Am I now
  immune?}\label{i-have-antibodies-am-i-now-immune}}

  \begin{itemize}
  \tightlist
  \item
    As of right
    now,\href{https://www.nytimes3xbfgragh.onion/2020/07/22/health/covid-antibodies-herd-immunity.html?action=click\&pgtype=Article\&state=default\&region=MAIN_CONTENT_3\&context=storylines_faq}{~that
    seems likely, for at least several months.}~There have been
    frightening accounts of people suffering what seems to be a second
    bout of Covid-19. But experts say these patients may have a
    drawn-out course of infection, with the virus taking a slow toll
    weeks to months after initial exposure.~People infected with the
    coronavirus typically
    \href{https://www.nature.com/articles/s41586-020-2456-9}{produce}~immune
    molecules called antibodies, which are
    \href{https://www.nytimes3xbfgragh.onion/2020/05/07/health/coronavirus-antibody-prevalence.html?action=click\&pgtype=Article\&state=default\&region=MAIN_CONTENT_3\&context=storylines_faq}{protective
    proteins made in response to an
    infection}\href{https://www.nytimes3xbfgragh.onion/2020/05/07/health/coronavirus-antibody-prevalence.html?action=click\&pgtype=Article\&state=default\&region=MAIN_CONTENT_3\&context=storylines_faq}{.
    These antibodies may}~last in the body
    \href{https://www.nature.com/articles/s41591-020-0965-6}{only two to
    three months}, which may seem worrisome, but that's~perfectly normal
    after an acute infection subsides, said Dr. Michael Mina, an
    immunologist at Harvard University. It may be possible to get the
    coronavirus again, but it's highly unlikely that it would be
    possible in a short window of time from initial infection or make
    people sicker the second time.
  \end{itemize}
\item ~
  \hypertarget{what-are-my-rights-if-i-am-worried-about-going-back-to-work}{%
  \paragraph{What are my rights if I am worried about going back to
  work?}\label{what-are-my-rights-if-i-am-worried-about-going-back-to-work}}

  \begin{itemize}
  \tightlist
  \item
    Employers have to provide
    \href{https://www.osha.gov/SLTC/covid-19/standards.html}{a safe
    workplace}~with policies that protect everyone equally.
    \href{https://www.nytimes3xbfgragh.onion/article/coronavirus-money-unemployment.html?action=click\&pgtype=Article\&state=default\&region=MAIN_CONTENT_3\&context=storylines_faq}{And
    if one of your co-workers tests positive for the coronavirus, the
    C.D.C.}~has said that
    \href{https://www.cdc.gov/coronavirus/2019-ncov/community/guidance-business-response.html}{employers
    should tell their employees}~-\/- without giving you the sick
    employee's name -\/- that they may have been exposed to the virus.
  \end{itemize}
\end{itemize}

\textbf{I just mentioned} the concern many people have to be a giver,
not a taker, and that's clearly one that you share. Unemployment
payments are a legal right for those who qualify, which doesn't mean you
have to exercise the right. But it isn't incidental that, like Social
Security, the program of unemployment insurance to which you would be
applying is not means-tested. There may be political advantages to that
design: In a society like ours, benefits that aren't means-tested tend
to garner more support and carry less stigma. In our present economic
circumstances, too, we don't want people to cut back on spending, and
your unemployment pay will reduce the temptation to do so.

Of course, your particular decision will not make much difference to the
world: Not taking these benefits will save the government an amount
that's well within the rounding error of the budget, and spending it
won't increase demand detectably. (At least not in the aggregate: I
suppose it might make a detectable difference for a small neighborhood
store you patronize regularly.) But there's nothing wrong about taking
part in a basically just system in ways permitted by its rules.

\emph{I am Jewish by birth, upbringing and culture but consider myself
an agnostic. Questions about the existence of God have no interest for
me, and my religious observance is more family- than God-oriented. In
this period of intense suffering and loss, though, I've struggled with
how to honestly express feelings of hope and sorrow. ``You're in my
heart'' or ``I hope your loved one gets better'' just doesn't seem to
carry the same weight as ``I'm praying for you'' --- and recently, I've
found myself saying that to people I know who are deeply religious, even
though I don't literally pray. I don't want to be facile or deceptive,
but it is what people say in our country. If the other person has
already spoken about God's will, is responding in their lingua franca a
sign of empathy or a kind of appropriation?} Jezra Kaye, New York

\textbf{Not every use} of religious language signals religious
commitment. An atheist can say ``Bless you'' when someone sneezes; it's
just a conventional formula. The same is true of ``goodbye,'' said at
parting, even though it's an abbreviation of ``God be with you.'' When
people leave my presence in Asante, the region of Ghana where I grew up,
I often say ``Wo ne Nyame nko,'' which literally means ``Go with God.''
(It's more traditional than the alternative, which is an Asantified
version of ``bye-bye,'' namely ``Baabae-o.'') I'm pretty sure I'm not
misleading anyone. ``I'm praying for you'' could, against a certain
cultural background, be formulaic in just this way.

But what if ``I'm praying for you'' brings someone consolation not just
by expressing compassion and concern but by suggesting, falsely, that
you are actually praying? Perhaps religious people would be especially
prone to mistake your parrotings for promises. What's at stake, in these
circumstances, isn't appropriation but deception. Nor is this a white
lie, a trivial fib meant to spare someone's feelings. The act depends
upon deceiving listeners about something that is --- from their point of
view, if not yours --- genuinely important. Like many attempts at
kindness, it would be, at the very least, condescending.

Advertisement

\protect\hyperlink{after-bottom}{Continue reading the main story}

\hypertarget{site-index}{%
\subsection{Site Index}\label{site-index}}

\hypertarget{site-information-navigation}{%
\subsection{Site Information
Navigation}\label{site-information-navigation}}

\begin{itemize}
\tightlist
\item
  \href{https://help.nytimes3xbfgragh.onion/hc/en-us/articles/115014792127-Copyright-notice}{©~2020~The
  New York Times Company}
\end{itemize}

\begin{itemize}
\tightlist
\item
  \href{https://www.nytco.com/}{NYTCo}
\item
  \href{https://help.nytimes3xbfgragh.onion/hc/en-us/articles/115015385887-Contact-Us}{Contact
  Us}
\item
  \href{https://www.nytco.com/careers/}{Work with us}
\item
  \href{https://nytmediakit.com/}{Advertise}
\item
  \href{http://www.tbrandstudio.com/}{T Brand Studio}
\item
  \href{https://www.nytimes3xbfgragh.onion/privacy/cookie-policy\#how-do-i-manage-trackers}{Your
  Ad Choices}
\item
  \href{https://www.nytimes3xbfgragh.onion/privacy}{Privacy}
\item
  \href{https://help.nytimes3xbfgragh.onion/hc/en-us/articles/115014893428-Terms-of-service}{Terms
  of Service}
\item
  \href{https://help.nytimes3xbfgragh.onion/hc/en-us/articles/115014893968-Terms-of-sale}{Terms
  of Sale}
\item
  \href{https://spiderbites.nytimes3xbfgragh.onion}{Site Map}
\item
  \href{https://help.nytimes3xbfgragh.onion/hc/en-us}{Help}
\item
  \href{https://www.nytimes3xbfgragh.onion/subscription?campaignId=37WXW}{Subscriptions}
\end{itemize}
