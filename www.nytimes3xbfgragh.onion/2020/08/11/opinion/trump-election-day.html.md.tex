Sections

SEARCH

\protect\hyperlink{site-content}{Skip to
content}\protect\hyperlink{site-index}{Skip to site index}

\href{https://myaccount.nytimes3xbfgragh.onion/auth/login?response_type=cookie\&client_id=vi}{}

\href{https://www.nytimes3xbfgragh.onion/section/todayspaper}{Today's
Paper}

\href{/section/opinion}{Opinion}\textbar{}How to Foil Trump's Election
Night Strategy

\url{https://nyti.ms/33ON3So}

\begin{itemize}
\item
\item
\item
\item
\item
\item
\end{itemize}

Advertisement

\protect\hyperlink{after-top}{Continue reading the main story}

\href{/section/opinion}{Opinion}

Supported by

\protect\hyperlink{after-sponsor}{Continue reading the main story}

\hypertarget{how-to-foil-trumps-election-night-strategy}{%
\section{How to Foil Trump's Election Night
Strategy}\label{how-to-foil-trumps-election-night-strategy}}

To keep the president from claiming victory on Nov. 3, Biden supporters
who can vote in person may well have to.

\href{https://www.nytimes3xbfgragh.onion/column/jamelle-bouie}{\includegraphics{https://static01.graylady3jvrrxbe.onion/images/2019/01/24/opinion/jamelle-bouie/jamelle-bouie-thumbLarge-v3.png}}

By
\href{https://www.nytimes3xbfgragh.onion/column/jamelle-bouie}{Jamelle
Bouie}

Opinion Columnist

\begin{itemize}
\item
  Aug. 11, 2020
\item
  \begin{itemize}
  \item
  \item
  \item
  \item
  \item
  \item
  \end{itemize}
\end{itemize}

\includegraphics{https://static01.graylady3jvrrxbe.onion/images/2020/08/11/opinion/11bouie1/merlin_173838498_b791a911-0c46-40ca-a205-8feaf068447d-articleLarge.jpg?quality=75\&auto=webp\&disable=upscale}

\hypertarget{listen-to-this-op-ed}{%
\subsubsection{Listen to This Op-Ed}\label{listen-to-this-op-ed}}

Audio Recording by Audm

\emph{To hear more audio stories from publishers like The New York
Times,
download}\href{https://www.audm.com/?utm_source=nytmag\&utm_medium=embed\&utm_campaign=left_behind_draper}{**}\href{https://www.audm.com/?utm_source=nytopinion\&utm_medium=embed\&utm_campaign=foil_trump_strategy}{\emph{Audm
for iPhone or Android}}\emph{.}

There's no mystery about what
\href{https://www.nytimes3xbfgragh.onion/2020/09/03/us/politics/trump-2020-election.html}{President
Trump} intends to do if he holds a lead on election night in November.
He's practically broadcasting it.

First, he'll claim victory. Then, having spent most of the year
denouncing vote-by-mail as corrupt, fraudulent and prone to abuse, he'll
demand that authorities stop counting mail-in and absentee ballots.
He'll have teams of lawyers challenging counts and ballots across the
country.

He also seems to be counting on having the advantage of
\href{https://www.nytimes3xbfgragh.onion/2020/08/07/us/politics/postal-service-reorganization-mail-ballots.html}{mail
slowdowns}, engineered by the recently installed Postmaster General
Louis DeJoy. Fewer pickups and deliveries could mean more late-arriving
ballots and a better shot at dismissing votes before they're even
opened, especially if the campaign has successfully sued to block states
from extending deadlines. We might even see
\href{https://www.washingtonpost.com/wp-dyn/articles/A31074-2005Jan23.html}{a
Brooks Brothers riot} or two, where well-heeled Republican operatives
stage angry and voluble protests against ballot counts and recounts.

If Trump is leading on election night, in other words, there's a good
chance he'll try to disrupt and delegitimize the counting process. That
way, if Joe Biden pulls ahead in the days (or weeks) after voting ends
--- if we experience a ``blue shift'' like the one in 2018, in which the
Democratic majority in the House grew as votes came in --- the president
will have given himself grounds to reject the outcome as ``fake news.''

The only way to prevent this scenario, or at least, rob it of the oxygen
it needs to burn, is to deliver an election night lead to Biden. This
means voting in person. No, not everyone will be able to do that. But if
you plan to vote against Trump and can take appropriate precautions,
then some kind of hand delivery --- going to the polls or bringing your
mail-in ballot to a ``drop box'' --- will be the best way to protect
your vote from the president's concerted attempt to undermine the
election for his benefit.

Trump is the underdog in this year's race for president. He trails by
8.2 percentage points in
\href{https://projects.fivethirtyeight.com/polls/president-general/national/}{the
FiveThirtyEight average}; by 6.9 percentage points in
\href{https://www.realclearpolitics.com/epolls/2020/president/us/general_election_trump_vs_biden-6247.html}{the
RealClearPolitics average} and by 9 percentage points in
\href{https://www.270towin.com/2020-polls-biden-trump/}{the 270toWin
average}. He's given up on expanding his coalition or winning a majority
of voters (if he ever cared in the first place). And he's botched the
coronavirus pandemic, leaving the United States with an ever-climbing
six-figure death toll and a severe economic downturn. Trump is desperate
to hold on to power, but he probably can't win a fair fight. His
solution, then, is to do everything in his power to hinder the
opposition and either win an Electoral College majority or claim victory
before all the votes have been counted.

A key element of Trump's strategy is to undermine the Postal Service's
ability to deliver and collect mail. The president's postmaster general
has removed experienced officials, implemented cuts and
\href{https://twitter.com/marceelias/status/1292877962958823427?s=20}{raised
postage rates} for ballots mailed to voters, increasing the cost if
states want the post office to prioritize election mail. And
\href{https://www.politico.com/news/2020/08/08/trump-wants-to-cut-mail-in-voting-the-republican-machine-is-helping-him-392428}{Politico
reports} that Trump's aides and advisers in the White House have been
searching for ways to curb mail-in voting through executive action,
``from directing the Postal Service to not deliver certain ballots to
stopping local officials from counting them after Election Day.''

If vote-by-mail is the safest option in a pandemic, then the point of
the White House's effort is to create a dilemma for voters who place a
premium on safety. Do they mail a ballot and run the risk of a discarded
vote, or do they go to the polls and run the risk of infection and
illness? Consider the partisan split as well. Fifty-four percent of
Biden supporters prefer mail-in voting, according to
\href{https://www.langerresearch.com/wp-content/uploads/1214a22020Election.pdf}{a
July poll} from ABC News and The Washington Post, while only 17 percent
of Trump supporters say the same.

If in-person voters are disproportionately pro-Trump, and mail-in voters
are disproportionately pro-Biden, then you have the ingredients for an
election night standoff, where the president claims victory before all
the votes have been counted and tries to secure his ``win'' by keeping
mail-in ballots off the table.

There are reforms that could keep the president from taking this tack.
To account for postal delays, states can pledge to count ballots
postmarked on or before Nov. 3, so that they're included in the total
even if they arrive late. To speed up the process, states could permit
election officials to verify and count mail-in ballots even before
Election Day. They could also decline to release results until all polls
close and all votes are in. News organizations, similarly, could set
expectations for viewers and bring as much transparency as possible to
vote counts and other forms of election analysis.

Nonetheless, there is a chance that the president takes this path
regardless of state officials and the media. And there's every reason to
think that some portion of the Republican Party will back him. The Trump
campaign and the Republican National Committee are
\href{https://www.cnn.com/2020/08/01/politics/donald-trump-election-voting/index.html}{already
challenging} mail-in voting laws and suing to keep states like Nevada
and Pennsylvania from enlarging their scope. It is easy to imagine a
replay of Florida 2000, except on
\href{https://www.nytimes3xbfgragh.onion/2020/08/08/us/politics/voting-nov-3-election.html}{a
national scale}.

The best defense for the president's political opponents is, if
possible, to vote in person. For some, this will mean going to the polls
in November, in the middle of flu season, when the spread of Covid-19
may worsen. In most states, however, there are multiple ways to cast or
hand in a ballot. Every state offers some form of early or absentee
voting, and
\href{https://bipartisanpolicy.org/blog/voting-in-the-time-of-corona-the-difference-between-absentee-voting-and-voting-by-mail/}{33
states} --- including swing states like Arizona and Wisconsin --- allow
absentee voting without an excuse. Trump supports absentee voting ---
it's how his
\href{https://apnews.com/19ade6dafb5b6f82f324e4be9b12a7a0}{older
supporters in Florida} vote --- and his opponents should take advantage
of the fact that those systems won't be under the same kind of attack.
Many vote-by-mail states also offer drop boxes so that voters can
deliver ballots directly to the registrar. And if you must mail in your
ballot, the best practice would be to post it as early as possible, to
account for potential delays.

Earlier this year, a group of more than 100 people --- Republicans,
Democrats, senior political operatives and members of the media ---
gathered to role play the November election, using predetermined rules
and procedures. ``In each scenario other than a Biden landslide,''
\href{https://www.the-american-interest.com/2020/08/06/getting-from-november-to-january/}{writes}
Nils Gilman of the Berggruen Institute, who helped organize the
exercise, ``we ended up with a constitutional crisis that lasted until
the inauguration, featuring violence in the streets and a severely
disrupted administrative transition.''

There you have it. To head off the worst outcomes, Trump must go down in
a decisive defeat. He's on that path already. The task for his opponents
is to sustain that momentum and work to make his defeat as obvious as
possible, as early as possible. The pandemic makes that a risk, but it's
a risk many of us may have to take.

\emph{The Times is committed to publishing}
\href{https://www.nytimes3xbfgragh.onion/2019/01/31/opinion/letters/letters-to-editor-new-york-times-women.html}{\emph{a
diversity of letters}} \emph{to the editor. We'd like to hear what you
think about this or any of our articles. Here are some}
\href{https://help.nytimes3xbfgragh.onion/hc/en-us/articles/115014925288-How-to-submit-a-letter-to-the-editor}{\emph{tips}}\emph{.
And here's our email:}
\href{mailto:letters@NYTimes.com}{\emph{letters@NYTimes.com}}\emph{.}

\emph{Follow The New York Times Opinion section on}
\href{https://www.facebookcorewwwi.onion/nytopinion}{\emph{Facebook}}\emph{,}
\href{http://twitter.com/NYTOpinion}{\emph{Twitter (@NYTopinion)}}
\emph{and}
\href{https://www.instagram.com/nytopinion/}{\emph{Instagram}}\emph{.}

Advertisement

\protect\hyperlink{after-bottom}{Continue reading the main story}

\hypertarget{site-index}{%
\subsection{Site Index}\label{site-index}}

\hypertarget{site-information-navigation}{%
\subsection{Site Information
Navigation}\label{site-information-navigation}}

\begin{itemize}
\tightlist
\item
  \href{https://help.nytimes3xbfgragh.onion/hc/en-us/articles/115014792127-Copyright-notice}{©~2020~The
  New York Times Company}
\end{itemize}

\begin{itemize}
\tightlist
\item
  \href{https://www.nytco.com/}{NYTCo}
\item
  \href{https://help.nytimes3xbfgragh.onion/hc/en-us/articles/115015385887-Contact-Us}{Contact
  Us}
\item
  \href{https://www.nytco.com/careers/}{Work with us}
\item
  \href{https://nytmediakit.com/}{Advertise}
\item
  \href{http://www.tbrandstudio.com/}{T Brand Studio}
\item
  \href{https://www.nytimes3xbfgragh.onion/privacy/cookie-policy\#how-do-i-manage-trackers}{Your
  Ad Choices}
\item
  \href{https://www.nytimes3xbfgragh.onion/privacy}{Privacy}
\item
  \href{https://help.nytimes3xbfgragh.onion/hc/en-us/articles/115014893428-Terms-of-service}{Terms
  of Service}
\item
  \href{https://help.nytimes3xbfgragh.onion/hc/en-us/articles/115014893968-Terms-of-sale}{Terms
  of Sale}
\item
  \href{https://spiderbites.nytimes3xbfgragh.onion}{Site Map}
\item
  \href{https://help.nytimes3xbfgragh.onion/hc/en-us}{Help}
\item
  \href{https://www.nytimes3xbfgragh.onion/subscription?campaignId=37WXW}{Subscriptions}
\end{itemize}
