Sections

SEARCH

\protect\hyperlink{site-content}{Skip to
content}\protect\hyperlink{site-index}{Skip to site index}

\href{https://www.nytimes3xbfgragh.onion/section/books/review}{Book
Review}

\href{https://myaccount.nytimes3xbfgragh.onion/auth/login?response_type=cookie\&client_id=vi}{}

\href{https://www.nytimes3xbfgragh.onion/section/todayspaper}{Today's
Paper}

\href{/section/books/review}{Book Review}\textbar{}Kurt Andersen Asks:
What Is the Future of America?

\url{https://nyti.ms/3kD9obc}

\begin{itemize}
\item
\item
\item
\item
\item
\item
\end{itemize}

Advertisement

\protect\hyperlink{after-top}{Continue reading the main story}

Supported by

\protect\hyperlink{after-sponsor}{Continue reading the main story}

Nonfiction

\hypertarget{kurt-andersen-asks-what-is-the-future-of-america}{%
\section{Kurt Andersen Asks: What Is the Future of
America?}\label{kurt-andersen-asks-what-is-the-future-of-america}}

\includegraphics{https://static01.graylady3jvrrxbe.onion/images/2020/08/23/books/review/23Giridharads-COVER/23Giridharads-COVER-articleLarge.jpg?quality=75\&auto=webp\&disable=upscale}

Buy Book ▾

\begin{itemize}
\tightlist
\item
  \href{https://www.amazon.com/gp/search?index=books\&tag=NYTBSREV-20\&field-keywords=Evil+Geniuses+Kurt+Andersen}{Amazon}
\item
  \href{https://du-gae-books-dot-nyt-du-prd.appspot.com/buy?title=Evil+Geniuses\&author=Kurt+Andersen}{Apple
  Books}
\item
  \href{https://www.anrdoezrs.net/click-7990613-11819508?url=https\%3A\%2F\%2Fwww.barnesandnoble.com\%2Fw\%2F\%3Fean\%3D9781984801340}{Barnes
  and Noble}
\item
  \href{https://www.anrdoezrs.net/click-7990613-35140?url=https\%3A\%2F\%2Fwww.booksamillion.com\%2Fp\%2FEvil\%2BGeniuses\%2FKurt\%2BAndersen\%2F9781984801340}{Books-A-Million}
\item
  \href{https://bookshop.org/a/3546/9781984801340}{Bookshop}
\item
  \href{https://www.indiebound.org/book/9781984801340?aff=NYT}{Indiebound}
\end{itemize}

When you purchase an independently reviewed book through our site, we
earn an affiliate commission.

By \href{https://www.nytimes3xbfgragh.onion/by/anand-giridharadas}{Anand
Giridharadas}

\begin{itemize}
\item
  Aug. 11, 2020
\item
  \begin{itemize}
  \item
  \item
  \item
  \item
  \item
  \item
  \end{itemize}
\end{itemize}

\textbf{EVIL GENIUSES}\\
\textbf{The Unmaking of America: A Recent History}\\
By Kurt Andersen

It used to be called the New World. Now it's run by a man who wants to
make it great ``again.''

Sometime between then and now, \href{https://www.kurtandersen.com}{the
writer Kurt Andersen} argues in his essential, absorbing, infuriating,
full-of-facts-you-didn't-know, saxophonely written new book, America
lost one of its ``defining'' traits: ``openness to the new,'' its
gee-whiz tendency to ``try the untried and explore the uncharted,'' its
``innovative, novelty-seeking, risk-taking attitudes,'' its ``new
conceptions of freedom and fairness and self-government and national
identity'' --- built, it must be said, atop tyrannies new and old.

Andersen traces this ``cultural U-turn'' to the 1970s. (Reading this
book will disabuse you of any notion that hair was the biggest problem
of that era.) In those years, Andersen writes, America swerved away from
the new on two distinct but intersecting levels. In culture, it fell
into a ``mass nostalgia'' that became a ``cultural listlessness,'' a
slowing of the rate at which life looked, felt and sounded new ---
Americans from the 1950s and 1970s appearing as if from different
planets, but Americans from the 2000s and today looking not all that
dissimilar. Meanwhile, in political economy, America was hijacked by
capital supremacists, who preached and enacted, as Andersen details with
wallets-full of receipts, a return to a pre-New Deal order: ``everybody
for themselves, everything's for sale, greed is good, the rich get
richer, buyer beware, unfairness can't be helped, nothing but thoughts
and prayers for the losers.''

To begin with the conclusion of ``Evil Geniuses,'' Andersen, the author
of several books and an accomplished magazine editor and radio host,
argues that this double reversion threatens the endurance of the country
he has long chronicled. America, he says, risks being ``the first large
modern society to go from fully developed to failing.'' In an
illustration of his gift for connection-making and framing, he suggests
that what could save the country ``is a transformative pivot almost as
radical for us as the one China made'' when it abandoned Communism for
capitalism, while, Andersen notes for our benefit, more or less
preserving the chassis of its political system.

\emph{{[} Read an excerpt from}
\href{https://www.nytimes3xbfgragh.onion/2020/08/11/books/review/evil-genuises-by-kurt-andersen-an-excerpt.html}{\emph{``Evil
Genuises.''}} \emph{{]}}

Starting here at the end is one way of revealing the immensity of
Andersen's project, which is to explain everything that went wrong.
Andersen sets out to narrate a complex, many-layered history of how a
band of rich people, corporate executives and political right-wingers,
aided and abetted by gullible ``useful idiots'' in the media and the
political left, transformed the nation into a casino where only they
ever win.

Now, this is one of those situations where the book is better than the
review, so you should read it, but let me give you a sense of the many
dimensions of the hijacking Andersen details. He makes a definitive,
exhaustive and only very occasionally exhausting case that life changed
profoundly in America starting in the 1970s and well into the 1980s, in
ways that trap us still.

Thanks to a series of secret and not-secret memos, corporate America got
organized to pursue political power in a way it hadn't before. Through
policy changes like corporate and high-earner tax cuts, society was
reorganized. Just as important --- Andersen is very useful here --- the
rising capital supremacists ruled through ``countless nuts-and-bolts
changes so dweeby and tedious, and so often bipartisan, that they
appeared inconsequential and were uncontroversial,'' as well as by even
stealthier ``screwing-by-inaction'' or ``malign neglect,'' changing
things by letting things expire, failing to index things. Year by year,
continuing into the present, through these policy changes explicit and
subtle, American life was turned upside down --- in a way that many
people seemed not to realize. Pensions were gutted. The minimum wage was
effectively slashed. Companies started spending much more money buying
back their own stock than on research and development. Wall Street took
over the management of companies. Antitrust enforcement largely
disappeared. An app-guided, app-stiffed servant class was born.

The rich and the right correctly understood what they were seeking as a
cultural project with economic benefits. They acted accordingly. In
territory that has already been reported by
\href{https://www.jane-mayer.com}{Jane Mayer}, in her
must-read-if-you-care-about-your-country-even-just-a-little book ``Dark
Money,'' they reserved a fraction of the spoils of widening economic
inequality to invest in the yanking open of political inequality, so as
to widen the economic inequality yet further. Obviously, this meant
political giving. But it also meant funding universities, think tanks
and nonprofits. It meant ensuring that cause-boosting thinkers like
\href{https://www.aei.org/profile/charles-murray/}{Charles Murray} were
well tended. It meant developing new academic fields like ``law and
economics'' and new campus organizations like
\href{https://fedsoc.org}{the Federalist Society}. It meant buying up
media so that the capital-supremacist viewpoint could reach ordinary
people through Fox News and elites through The Wall Street Journal.

Rigging pays. Members of the Koch family are, Andersen reports, 20 times
richer than they were 40 years ago (maybe you're the same way). And the
wealthiest 1 percent of American households have gained \$12 million on
average since the 1980s (maybe same with you).

Now, Mayer and others --- \href{https://robertreich.org}{Robert Reich},
\href{https://www.theatlantic.com/author/george-packer/}{George Packer},
\href{https://www.jacobhacker.com}{Jacob S. Hacker} --- have told
versions of this story before. And at first, while reading ``Evil
Geniuses,'' that annoyed me. Until I understood what this book really
is: Andersen's retrospective on the bigger themes and trend lines and
power grabs that he, and so many others, missed, even as he was writing
magazine stories about the people and institutions in question. The book
is an intellectual double take, a rereporting of the great neoliberal
conquest, by a writer who kicks himself for missing it at the time.

At one point, he says, revealingly, that he admires Elizabeth Warren
``because I identify with her middle-aged illumination concerning the
political economy.'' Warren was famously a Republican until she became a
vocal progressive. Andersen also seems to see himself in
\href{https://www.britannica.com/biography/Walter-Lippmann}{Walter
Lippmann}, the early-20th-century writer who was ``pragmatic, in many
ways conservative, in no way a utopian,'' but came, through reporting,
to plead with gusto for social reform.

Andersen's sense of culpability and his permeability to new facts give
his book its particular power. It is a radicalized moderate's moderate
case for radical change. Andersen is unambiguous about where America
needs to go; he is honest about what it took to get him to his current
views; and he writes not as a haranguer who presumes you're with him but
as a journalist who presumes you're not, that you might even think as he
once did. So, carefully, meticulously, overwhelmingly, he argues through
facts.

And with his own complicity in mind, one place Andersen does break some
new ground is in the portrayal of the shameful liberal complicity that
was essential to the long plutocratic hijacking. For someone of my age,
gray but elder-millennial, for whom polarization has been the oxygen in
the air, it is head-spinning to be reminded how much of the nation's
turn to the right and to the rich the Democrats enabled. The
Democrat-controlled House voted to cut taxes on the richest Americans,
to a rate lower than at any time in the previous half century. Ronald
Reagan couldn't do it without them and, in the Senate, one
\href{https://www.govtrack.us/congress/votes/97-1981/s251}{Joe Biden
voted for the cuts, too}. As the tide turned against antitrust
regulation, writers in The New York Times (oh yeah, this newspaper) and
Newsweek cheered. When Gary Hart sought the Democratic nomination for
president for the second time in 1988, he actually enlisted as a tax
policy adviser Arthur Laffer, the clownish economist who invented the
hokum of supply-side economics (and who once asked me, before a
television debate, if ``the Indians'' still brush teeth with twigs).

Voting for cable deregulation? Hiring Goldman Sachs bankers as advisers?
Praising Charles Murray's advocacy of punishing mothers on welfare? Each
time, Democrats were \#OnIt. With Democrats like these, do we even need
a second party representing the plutes?

As he makes this wide-ranging case, Andersen never loses the texture of
actual human beings. He flies his plane over vast territory, but he
flies at low altitude so you're always able to see real people sowing
this future, going down these roads. He is a graceful, authoritative
guide, and he has a Writer-with-a-capital-W's ability to defamiliarize
the known. He also isn't afraid to play around. He even manages to work
in the word ``wankerish,'' on Page 69 --- nice.

``Evil Geniuses'' is not a perfect book. At certain moments, more than a
few times, there is a broad-brush characterization of the American
spirit or temperament or enjoyment of liberty that clearly did not apply
to Black Americans, a caveat Andersen omits. (``Almost anybody was
unusually free to give any business (or religion) a go''; ``Ours was a
nation built from scratch meant to embody the best Enlightenment
principles and habits of mind.'') It is also a long book that
occasionally loops back on itself. And, to my taste at least, it didn't
need the brief history of Covid at the end. I like my books like I like
my exes: at a remove from my current situation.

As he works his way to the end, Andersen actually does the thing I once
told my editor I would do and then just didn't: propose solutions on the
heels of his criticisms. Some are basic, unoriginal to him: stronger
unions, a universal basic income. But there is a more foundational piece
of advice he offers the political left at this moment, and I would like
it if everyone who so identifies would hear this:

At this moment of five intersecting crises --- health, economic, racial,
democratic and climatic --- things can feel hopeless. The rich and the
right did it. We all live in their world now, and while in their world
it keeps getting warmer the consolation is you should avoid the outdoors
anyway, because the defunding and defanging and delegitimizing of
government has left the virus rampaging.

Into this depressing thought Andersen jumps. Do not despair, for the
hijacking that got us here is evidence of what is possible today: ``If
you need proof that ideas have power and that radical change is
possible, it's there in the rearview mirror. Evil genius is genius
nonetheless. In the early 1970s, at the zenith of liberal-left
influence, an improbable, quixotic, out-of-power economic right ---
intellectuals, capitalists, politicians --- launched their crusade and
then kept at it tenaciously. The unthinkable became the inevitable in a
single decade. They envisioned a new American trajectory, then
popularized and arranged it with remarkable success.''

Andersen argues that this moment of crisis on multiple vectors is
precisely the kind of hour known through history to occasion great
change. But only if those who seek change know how to play the game. He
offers the left two pieces of counsel of particular note: to focus
relentlessly on the amassing of power, the building of institutions,
networks, societies, associations --- the long game. And to embrace the
power of evil genius. He says the mainstream left has been too nice, too
nonideological, too pragmatic, too nuanced. Winning back America as a
country that works for Americans will be the battle of a generation, and
this book raises the question of whether the left should seek to achieve
that country by embracing the cabalistic power-building, linguistic
cunning, intellectual patronage and media stewardship the right has
employed.

``For Americans now, will surviving a year (or more) of radical
uncertainty help persuade a majority to make radical changes in our
political economy to reduce their chronic economic uncertainty and
insecurity?'' Andersen writes. Or, he wonders, ``will Americans remain
hunkered forever, as confused and anxious and paralyzed as we were
before 2020, descend into digital feudalism, forgo a renaissance and
retreat into cocoons of comfortable cultural stasis providing the
illusion that nothing much is changing or ever can change?''

Advertisement

\protect\hyperlink{after-bottom}{Continue reading the main story}

\hypertarget{site-index}{%
\subsection{Site Index}\label{site-index}}

\hypertarget{site-information-navigation}{%
\subsection{Site Information
Navigation}\label{site-information-navigation}}

\begin{itemize}
\tightlist
\item
  \href{https://help.nytimes3xbfgragh.onion/hc/en-us/articles/115014792127-Copyright-notice}{©~2020~The
  New York Times Company}
\end{itemize}

\begin{itemize}
\tightlist
\item
  \href{https://www.nytco.com/}{NYTCo}
\item
  \href{https://help.nytimes3xbfgragh.onion/hc/en-us/articles/115015385887-Contact-Us}{Contact
  Us}
\item
  \href{https://www.nytco.com/careers/}{Work with us}
\item
  \href{https://nytmediakit.com/}{Advertise}
\item
  \href{http://www.tbrandstudio.com/}{T Brand Studio}
\item
  \href{https://www.nytimes3xbfgragh.onion/privacy/cookie-policy\#how-do-i-manage-trackers}{Your
  Ad Choices}
\item
  \href{https://www.nytimes3xbfgragh.onion/privacy}{Privacy}
\item
  \href{https://help.nytimes3xbfgragh.onion/hc/en-us/articles/115014893428-Terms-of-service}{Terms
  of Service}
\item
  \href{https://help.nytimes3xbfgragh.onion/hc/en-us/articles/115014893968-Terms-of-sale}{Terms
  of Sale}
\item
  \href{https://spiderbites.nytimes3xbfgragh.onion}{Site Map}
\item
  \href{https://help.nytimes3xbfgragh.onion/hc/en-us}{Help}
\item
  \href{https://www.nytimes3xbfgragh.onion/subscription?campaignId=37WXW}{Subscriptions}
\end{itemize}
