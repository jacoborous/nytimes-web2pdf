Sections

SEARCH

\protect\hyperlink{site-content}{Skip to
content}\protect\hyperlink{site-index}{Skip to site index}

\href{https://www.nytimes3xbfgragh.onion/section/style}{Style}

\href{https://myaccount.nytimes3xbfgragh.onion/auth/login?response_type=cookie\&client_id=vi}{}

\href{https://www.nytimes3xbfgragh.onion/section/todayspaper}{Today's
Paper}

\href{/section/style}{Style}\textbar{}All Dressed Up and Nowhere to Go

\url{https://nyti.ms/3isiZzW}

\begin{itemize}
\item
\item
\item
\item
\item
\item
\end{itemize}

\href{https://www.nytimes3xbfgragh.onion/spotlight/at-home?action=click\&pgtype=Article\&state=default\&region=TOP_BANNER\&context=at_home_menu}{At
Home}

\begin{itemize}
\tightlist
\item
  \href{https://www.nytimes3xbfgragh.onion/2020/09/07/travel/route-66.html?action=click\&pgtype=Article\&state=default\&region=TOP_BANNER\&context=at_home_menu}{Cruise
  Along: Route 66}
\item
  \href{https://www.nytimes3xbfgragh.onion/2020/09/04/dining/sheet-pan-chicken.html?action=click\&pgtype=Article\&state=default\&region=TOP_BANNER\&context=at_home_menu}{Roast:
  Chicken With Plums}
\item
  \href{https://www.nytimes3xbfgragh.onion/2020/09/04/arts/television/dark-shadows-stream.html?action=click\&pgtype=Article\&state=default\&region=TOP_BANNER\&context=at_home_menu}{Watch:
  Dark Shadows}
\item
  \href{https://www.nytimes3xbfgragh.onion/interactive/2020/at-home/even-more-reporters-editors-diaries-lists-recommendations.html?action=click\&pgtype=Article\&state=default\&region=TOP_BANNER\&context=at_home_menu}{Explore:
  Reporters' Google Docs}
\end{itemize}

Advertisement

\protect\hyperlink{after-top}{Continue reading the main story}

Supported by

\protect\hyperlink{after-sponsor}{Continue reading the main story}

\hypertarget{all-dressed-up-and-nowhere-to-go}{%
\section{All Dressed Up and Nowhere to
Go}\label{all-dressed-up-and-nowhere-to-go}}

Robbed of a real-life stage, the frustrated fashion faithful continue to
pull out all the stops on Instagram.

\includegraphics{https://static01.graylady3jvrrxbe.onion/images/2020/08/13/fashion/11FASHIONJUNKIES-1/merlin_175268004_83e60a07-db4b-41af-b249-55bde413c2b8-articleLarge.jpg?quality=75\&auto=webp\&disable=upscale}

By \href{https://www.nytimes3xbfgragh.onion/by/ruth-la-ferla}{Ruth La
Ferla}

\begin{itemize}
\item
  Published Aug. 11, 2020Updated Aug. 13, 2020
\item
  \begin{itemize}
  \item
  \item
  \item
  \item
  \item
  \item
  \end{itemize}
\end{itemize}

Last summer Nicole Gordon posted an Instagram snap of herself framed in
a doorway at home. In a slinky sleeveless dress, vivid makeup and
towering heels, Ms. Gordon, a writer and art adviser, was the picture of
cocktail-hour glamour.

Just weeks ago she posted a nearly identical image: her lips tinted
scarlet, hair swept back from her face. That dress, as she noted, still
fit, though she'd filled out in the interim. Her caption, a cross
between boast and lament, read: ``What a difference a year makes.''

Ms. Gordon, 51, was alluding of course to the pain and sense of
powerlessness that the pandemic has sown. ``It has stripped me of
everything that I knew of myself,'' she said last week --- not least the
semimonthly lash extensions and Botox treatments that were among her
cherished maintenance rituals.

She had rigorously prepped for her most recent post, tugging on two pair
of Spanx, rimming her eyes in dark liner, and coating her feet in
Lidocaine to help her squeeze into the stilettos she had not worn since
March.

``I told myself,'' Ms. Gordon said, ``that I was doing all this just so
I could feel like my old self again.''

That sentiment has swelled among like-minded artists, fashion
influencers and style-minded civilians, for whom pre-coronavirus life
was a runway and personal style a performance. Robbed of a stage, some
are at sea.

\includegraphics{https://static01.graylady3jvrrxbe.onion/images/2020/08/13/fashion/11FASHIONJUNKIES-3/11FASHIONJUNKIES-3-articleLarge.jpg?quality=75\&auto=webp\&disable=upscale}

``How do we continue to express ourselves through the joy of dressing
with no place to go?'' Ari Seth Cohen, asked plaintively. During
lockdown, Mr. Cohen, 38, the creator of
\href{https://www.advanced.style/}{Advanced Style,} a popular street
blog, three books, and a film celebrating the sartorial quirks of the
senior set, was hard-pressed to find subjects. Instead he posted
pictures of himself turned out in gaudy turbans and leopard-print
caftans.

At a time of widespread suffering and social unrest, that gesture may
seem brazen. ``Even among high-level fashion people, posting outfits is
apt to be viewed as kind of tone-deaf,'' said Lyn Slater, a professor at
the graduate School of Social Service at Fordham University.

Ms. Slater, 66, who moonlights as a model and blogger, persisted
nonetheless, coolly vamping on @accidentalicon, her Instagram account,
in a wardrobe of slogan T-shirts and rainbow-hued kimonos, her trademark
silver bob grown out during quarantine to shoulder length.

Social feeds have lately teemed with similarly colorful, often wickedly
over-the-top fashion portraits and selfies. They proliferate these days
on strikingly varied individual accounts and with hashtags like
\#quarantinelookoftheday and \#quarantinefashionchallenge, reinforcing a
sense of joy and connection, serving as a platform for self-promotion
(and more rarely, social activism), and restoring, for many, a sense of
self as fragile and faded as an old postcard.

Image

Lyn Slater, professor and devoted online vamper.Credit...Calvin Lom

``We're all cobbling behaviors together to get through the days,'' said
Leandra Medine, 31, the founder of
\href{https://www.manrepeller.com/}{Man Repeller,} a popular blog. Ms.
Medine announced in June that she would ``step back'' from the company
after being called out for a lack of diversity on the site. But on
@leandramcohen, her personal Instagram feed, she shows off a playful
cacophony of wildflower, stripe and kaleidoscopic tie-dye motifs.

Her posts are a reflexive response to the dreariness of lockdown, she
said, ``when there is no one to evaluate who you're telling the world
you believe yourself to be.''

To some social media die-hards, posting in that kind of vacuum is life
affirming. ``It's a joy to be your own muse,'' Mr. Cohen said,
illustrating that notion in posts that show him garbed in a manner that
is partly inspired by his grandmother.

``I'm wearing all her old jewelry,'' he said. ``During quarantine that
makes me feel connected to her again.'' He also draws for inspiration
from a well that includes Marc Jacobs, who has created a minor internet
sensation posting quasi-comic makeup tutorials and high-glam images that
show him wreathed in pearls, and balancing on king platform boots.

No question, Mr. Cohen said, such flamboyant get-ups can bring comfort
now and then, and express the hint of optimism that is a tonic during
somber times.

The impulse to fan out one's feathers can be deeply ingrained. As
Eleanor Lambert, the venerable American fashion publicist, once
observed: ``You cannot separate people, their yearning, their dreams and
their inborn vanity from an interest in clothes.''

Image

``We're dressing for the audience in our head,'' Merle Ginsberg said.

Sharing that itch on social media ``is no different from any other
effort to be seen,'' Ms. Medine said. ``What any one of us is doing is
trying to prove that we are worthy, lovable and socially acceptable.''

If only to ourselves. Online, as in life, ``We're dressing for the
audience in our head,'' said Merle Ginsberg, a fashion writer and former
judge on ``RuPaul's Drag Race'' who prefers not to disclose her age,
citing bias in the industry. Ms. Ginsberg recently posted a photograph
of her favorite John Fluevog floral-patterned Mary Jane pumps. ``I used
to get excited for fall around mid-July,'' she said in a caption. ``Now
what? Nowhere to wear these puppies.''

``Still, the aesthetic impulse never goes,'' she said in a phone
conversation from rural Michigan, where she has been sheltering. She
recently unearthed a pair of powder blue Dr. Scholl's sandals at a local
thrift store. ``They cost \$2,'' said Ms. Ginsberg, who was thinking
about showing them off in a post. ``But I feel like I'm wearing
Louboutins.''

Shawna Ferguson, 37, a stylist and art director, has sprinkled her
account @Ferguson\_darling, with a series of sassy self-portraits. She
was turned out in a recent post in a peach-colored off-the-shoulder
dress, a bit fancy, she acknowledged, for a day spent at home.

``This may or may not be a bridesmaid's gown,'' her caption read.
Inappropriate, for sure. But so what. ``I do what I want,'' she wrote,
adding with a dash of gallows humor, ``Dressing for the end of the
world.''

Image

D' Smith Alexander, 33, an architect and musician.

Bella McFadden, 24, a.k.a. @Internetgirl, publishes selfies primarily as
a way of keeping her brand afloat. (Her thrift store finds are in high
demand on \href{http://www.depop.com}{Depop}, a popular e-commerce app.)
But posting during quarantine boosts her confidence as well, she said,
and lends form to her vision, a fusion of late-1990s mall rat and Y2K
Goth. Posing feline-style in a black sweatshirt and tiny kilt, she asks
in a caption, ``Anyone else playing dress up for a slice of
excitement?''

Ms. McFadden occasionally splices her feed with enjoinders. ``Pause,''
she urged fans last week. ``All lives won't matter until Black lives
matter.'' Her account, like those of some contemporaries, doubles as a
platform for activism.

On his Instagram account, @youngblackarchitect, D' Smith Alexander, an
architect and musician, captioned a portrait of himself dapperly turned
out in a bright blue suit and patent leather loafers. ``I am a Black
man,'' Mr. Alexander, 33, wrote, his post a call for unity. ``I build. I
don't tear down other Black Men!''

Jason Rice, 44, has taken his activism in another, equally pointed
direction. ``For me posting is an act of rebellion,'' said Mr. Rice, a
partner in Changez Hair Salon in Royal Oak, Mich.: one way, he explained
of ridding himself of the stigma of wearing women's clothes.

``I grew up a queer kid,'' said Mr. Rice, who appears online variously
garbed in ultrawide paisley neckwear, layered jeweled chokers and, in
one instance, a filmy blush-tone off-the-shoulder dress. ``For me
posting is a way of stating, `I refuse to let this moment take me
down,''' he said.

Image

Paula Sutton, a lifestyle blogger in Norfolk, England.

Ms. Slater, of @accidentalicon, posts, she said, in part as a retort to
ageism. ``Many older women have felt vulnerable this crisis,'' she said,
adding that implicit in the ubiquitous messaging about at-risk
populations is the notion that older people should remove themselves
from society.

She is not having it. ``For me posting is much more about expressing who
I am,'' she said, ``regardless of my age or what others think or have to
say about it.''

Paula Sutton, a lifestyle blogger in Norfolk, England, has taken up the
gauntlet. On @hillhousevintage, her Instagram account, she fans out her
skirts or cavorts on her lawn in a series of colorful garden-worthy
frocks, her poses expressions of unfettered joy.

``I am fifty years of age and I see no shame in enjoying pretty dresses
and attempting to live life as beautifully and positively as I can,''
Ms. Sutton declares in one of her extended captions.

In the text accompanying an image that shows her in a gingham dress with
extravagantly puffy sleeves, she urges fans to follow her lead. ``Show
your face, show your homes, show your gardens,'' she writes, ``and
celebrate your version of beauty.

``Pose like Dovima,'' she adds. ``After all, life is hard enough without
feeling pressured into being self-censored by the frivolity police!''

Advertisement

\protect\hyperlink{after-bottom}{Continue reading the main story}

\hypertarget{site-index}{%
\subsection{Site Index}\label{site-index}}

\hypertarget{site-information-navigation}{%
\subsection{Site Information
Navigation}\label{site-information-navigation}}

\begin{itemize}
\tightlist
\item
  \href{https://help.nytimes3xbfgragh.onion/hc/en-us/articles/115014792127-Copyright-notice}{©~2020~The
  New York Times Company}
\end{itemize}

\begin{itemize}
\tightlist
\item
  \href{https://www.nytco.com/}{NYTCo}
\item
  \href{https://help.nytimes3xbfgragh.onion/hc/en-us/articles/115015385887-Contact-Us}{Contact
  Us}
\item
  \href{https://www.nytco.com/careers/}{Work with us}
\item
  \href{https://nytmediakit.com/}{Advertise}
\item
  \href{http://www.tbrandstudio.com/}{T Brand Studio}
\item
  \href{https://www.nytimes3xbfgragh.onion/privacy/cookie-policy\#how-do-i-manage-trackers}{Your
  Ad Choices}
\item
  \href{https://www.nytimes3xbfgragh.onion/privacy}{Privacy}
\item
  \href{https://help.nytimes3xbfgragh.onion/hc/en-us/articles/115014893428-Terms-of-service}{Terms
  of Service}
\item
  \href{https://help.nytimes3xbfgragh.onion/hc/en-us/articles/115014893968-Terms-of-sale}{Terms
  of Sale}
\item
  \href{https://spiderbites.nytimes3xbfgragh.onion}{Site Map}
\item
  \href{https://help.nytimes3xbfgragh.onion/hc/en-us}{Help}
\item
  \href{https://www.nytimes3xbfgragh.onion/subscription?campaignId=37WXW}{Subscriptions}
\end{itemize}
