Sections

SEARCH

\protect\hyperlink{site-content}{Skip to
content}\protect\hyperlink{site-index}{Skip to site index}

\href{https://www.nytimes3xbfgragh.onion/section/us}{U.S.}

\href{https://myaccount.nytimes3xbfgragh.onion/auth/login?response_type=cookie\&client_id=vi}{}

\href{https://www.nytimes3xbfgragh.onion/section/todayspaper}{Today's
Paper}

\href{/section/us}{U.S.}\textbar{}In the Wake of Covid-19 Lockdowns, a
Troubling Surge in Homicides

\url{https://nyti.ms/31Ikx1N}

\begin{itemize}
\item
\item
\item
\item
\item
\item
\end{itemize}

\hypertarget{race-and-america}{%
\subsubsection{\texorpdfstring{\href{https://www.nytimes3xbfgragh.onion/news-event/george-floyd-protests-minneapolis-new-york-los-angeles?name=styln-george-floyd\&region=TOP_BANNER\&block=storyline_menu_recirc\&action=click\&pgtype=Article\&impression_id=6c8fa1e0-f52c-11ea-9407-63c7aed3b7c8\&variant=undefined}{Race
and America}}{Race and America}}\label{race-and-america}}

\begin{itemize}
\tightlist
\item
  \href{https://www.nytimes3xbfgragh.onion/2020/09/11/us/black-police-chiefs-reform.html?name=styln-george-floyd\&region=TOP_BANNER\&block=storyline_menu_recirc\&action=click\&pgtype=Article\&impression_id=6c8fa1e1-f52c-11ea-9407-63c7aed3b7c8\&variant=undefined}{Black
  Police Chiefs}
\item
  \href{https://www.nytimes3xbfgragh.onion/2020/09/04/nyregion/rochester-police-daniel-prude.html?name=styln-george-floyd\&region=TOP_BANNER\&block=storyline_menu_recirc\&action=click\&pgtype=Article\&impression_id=6c8fc8f0-f52c-11ea-9407-63c7aed3b7c8\&variant=undefined}{What
  Happened in Rochester, N.Y.}
\item
  \href{https://www.nytimes3xbfgragh.onion/2020/08/30/us/portland-shooting-explained.html?name=styln-george-floyd\&region=TOP_BANNER\&block=storyline_menu_recirc\&action=click\&pgtype=Article\&impression_id=6c8fc8f1-f52c-11ea-9407-63c7aed3b7c8\&variant=undefined}{Portland
  Shooting}
\item
  \href{https://www.nytimes3xbfgragh.onion/2020/08/30/us/breonna-taylor-police-killing.html?name=styln-george-floyd\&region=TOP_BANNER\&block=storyline_menu_recirc\&action=click\&pgtype=Article\&impression_id=6c8fc8f2-f52c-11ea-9407-63c7aed3b7c8\&variant=undefined}{Breonna
  Taylor's Life and Death}
\end{itemize}

Advertisement

\protect\hyperlink{after-top}{Continue reading the main story}

Supported by

\protect\hyperlink{after-sponsor}{Continue reading the main story}

\hypertarget{in-the-wake-of-covid-19-lockdowns-a-troubling-surge-in-homicides}{%
\section{In the Wake of Covid-19 Lockdowns, a Troubling Surge in
Homicides}\label{in-the-wake-of-covid-19-lockdowns-a-troubling-surge-in-homicides}}

``People have gotten to the point where they just don't give a damn,''
said a minister in Kansas City, which is on pace for a record number of
killings.

\includegraphics{https://static01.graylady3jvrrxbe.onion/images/2020/08/07/us/00VIRUS-CRIME-kc/merlin_172298298_9ce47f34-97e3-4935-ac57-da7595b9f270-articleLarge.jpg?quality=75\&auto=webp\&disable=upscale}

By \href{https://www.nytimes3xbfgragh.onion/by/john-eligon}{John
Eligon},
\href{https://www.nytimes3xbfgragh.onion/by/shaila-dewan}{Shaila Dewan}
and
\href{https://www.nytimes3xbfgragh.onion/by/nicholas-bogel-burroughs}{Nicholas
Bogel-Burroughs}

\begin{itemize}
\item
  Published Aug. 11, 2020Updated Aug. 24, 2020
\item
  \begin{itemize}
  \item
  \item
  \item
  \item
  \item
  \item
  \end{itemize}
\end{itemize}

KANSAS CITY, Mo. --- It started with an afternoon stop at a gas station.
Two customers began exchanging angry stares near the pumps outside ---
and no one can explain exactly why.

That led to an argument, and it escalated quickly as one of them pulled
a gun and they struggled over it, according to the police.

``There's too many shootings. Please don't do this,'' the wife of one of
the men pleaded, stepping between them.

But by the time the fight was over at the station on Kansas City's East
Side late last month, the all-too-familiar crackle of gunfire pierced
the humid air, leaving another person dead in what has been an
exceedingly bloody summer.

The onset of warm weather nearly always brings with it a spike in
violent crime, but with much of the country emerging from weeks of
lockdown from the coronavirus, the increase this year has been much
steeper than usual.

Across 20 major cities, the murder rate at the end of June was on
average 37 percent higher than it was at the end of May, according to
Richard Rosenfeld, a criminologist at the University of Missouri-St.
Louis. The increase over the same period a year ago was just 6 percent.

In few places has the bloodshed been more devastating than in Kansas
City, where the city is on pace to shatter its record for homicides in a
year. Much of it has involved incidents of random, angry violence like
the conflict at the gas station --- disputes between strangers that left
someone dead, or killings that simply cannot be explained. They have
claimed the lives of a pregnant woman pushing a stroller, a 4-year-old
boy asleep in his grandmother's home and a teenage girl sitting in a
car.

They have also prompted a much-debated
\href{https://www.justice.gov/opa/pr/attorney-general-william-p-barr-announces-launch-operation-legend}{intervention
from the federal government}, an operation named after the 4-year-old
Kansas City boy, LeGend Taliferro, that has sent federal law enforcement
agents to
\href{https://www.justice.gov/opa/pr/operation-legend-expanded-cleveland-detroit-and-milwaukee}{at
least six cities} in an attempt to intervene.

``We're surrounded by murder, and it's almost like your number is up,''
said Erica Mosby, whose niece, Diamon Eichelburger, 20, was the pregnant
victim pushing the stroller in Kansas City. ``It's terrible.''

Nationally, crime remains at or near a generational low, and experts
caution against drawing conclusions from just a few months.

But President Trump has used the rising homicide numbers to paint
Democratic-led cities as out of control and to blame protests against
police brutality that broke out after the killing of George Floyd in
Minneapolis in late May.

Image

Diamon Eichelburger with her daughter, Belle.Credit...family

``Extreme politicians have joined this anti-police crusade and
relentlessly vilified our law enforcement heroes,'' Mr. Trump said
\href{https://www.whitehouse.gov/briefings-statements/remarks-president-trump-operation-legend-combatting-violent-crime-american-cities/}{during
a White House news conference} last month to announce Operation LeGend.
He added that ``the effort to shut down policing in their own
communities has led to a shocking explosion of shootings, killings,
murders.''

Criminologists dispute the president's suggestion that the increase is
tied to any pullback by the police in response to criticism or defunding
efforts, and fluctuations in the crime rate are notoriously hard to
explain. In many cities, the murder rate was
\href{https://www.nytimes3xbfgragh.onion/2020/07/06/upshot/murders-rising-crime-coronavirus.html}{on
the rise before the pandemic}, and a steep decline in arrests coincided
with the start of social distancing, as measured by mobile phone
records, according to \href{https://citycrimestats.com/covid/}{a
database} compiled by David Abrams, an economist at the University of
Pennsylvania law school.

Some experts have pointed to the pandemic's destabilization of community
institutions, or theorized that people with a propensity for violence
may have been less likely to heed stay-at-home orders. But in city after
city, crime overall is down, including all types of major crime except
murder, aggravated assault and in some places, car theft.

In New York, where murders are up 30 percent over last year, city and
police officials have tried to lay blame on a new law that lets many
defendants go free without posting bond, as well as on the
coronavirus-related mass release of people from jail. But the evidence
shows that a
\href{https://www.nytimes3xbfgragh.onion/2020/08/04/nyregion/nyc-shootings-coronavirus.html?searchResultPosition=2}{steep
decline in gun arrests} beginning in mid-May was a more likely cause.
Police officials in several cities have said the protests diverted
officers from crime-fighting duty or emboldened criminals.

\includegraphics{https://static01.graylady3jvrrxbe.onion/images/2017/01/29/podcasts/the-daily-album-art/the-daily-album-art-articleInline-v2.jpg?quality=75\&auto=webp\&disable=upscale}

\hypertarget{listen-to-the-daily-a-surge-in-shootings}{%
\subsubsection{Listen to `The Daily': A Surge in
Shootings}\label{listen-to-the-daily-a-surge-in-shootings}}

Gun violence in the U.S. typically rises in summer, but the problem has
been especially fierce this year. Why?

transcript

Back to The Daily

bars

0:00/33:12

-33:12

transcript

\hypertarget{listen-to-the-daily-a-surge-in-shootings-1}{%
\subsection{Listen to `The Daily': A Surge in
Shootings}\label{listen-to-the-daily-a-surge-in-shootings-1}}

\hypertarget{hosted-by-michael-barbaro-produced-by-michael-simon-johnson-and-neena-pathak-and-edited-by-lisa-tobin}{%
\subsubsection{Hosted by Michael Barbaro, produced by Michael Simon
Johnson and Neena Pathak, and edited by Lisa
Tobin}\label{hosted-by-michael-barbaro-produced-by-michael-simon-johnson-and-neena-pathak-and-edited-by-lisa-tobin}}

\hypertarget{gun-violence-in-the-us-typically-rises-in-summer-but-the-problem-has-been-especially-fierce-this-year-why}{%
\paragraph{Gun violence in the U.S. typically rises in summer, but the
problem has been especially fierce this year.
Why?}\label{gun-violence-in-the-us-typically-rises-in-summer-but-the-problem-has-been-especially-fierce-this-year-why}}

\begin{itemize}
\item
  {[}music{]}
\item
  michael barbaro\\
  From The New York Times, I'm Michael Barbaro. This is ``The Daily.''

  Today: In major cities across the U.S., gun violence is surging just
  as activists are calling to defund the police. My colleague, Ashley
  Southall, on how that is playing out in New York City. It's Monday,
  August 24.

  Ashley, what has been the story of crime in New York City up until
  this moment?
\item
  ashley southall\\
  So around the `70s and `80s, we saw crime start to really rise in New
  York. And we went through the drug war, the crack epidemic. And in the
  `90s, we saw crime sort of peak all across the United States, but
  especially in New York. And there were thousands of murders each year
  and even more shootings. But as the `90s wore on, and into the 2000s
  ---
\item
  archived recording\\
  The crime rate in this country is continuing to fall. The F.B.I.
  reported today that violent crime fell 5.5 percent last year. That's
  three years in a row now.
\end{itemize}

ashley southall

--- we saw crime decline. And there was a strong decline for years, to
the point where we were having just hundreds of murders a year.

\begin{itemize}
\item
  archived recording 1\\
  The N.Y.P.D. says there have been 289 murders this year. That's down
  from a peak of more than 2,200 in 1990.
\item
  archived recording 2\\
  Crime in New York City dropping yet again, this time to levels not
  seen since Harry Truman was president.
\item
  archived recording 3\\
  Police admit it's probably the worst kept secret in the city, that New
  York is now the safest big city in America.
\end{itemize}

ashley southall

And the question was when was New York going to reach bottom?

\begin{itemize}
\item
  archived recording 1\\
  Breaking news on this Sunday night.
\item
  archived recording 2\\
  Crime is up across the city. That's according to the latest numbers
  released by the N.Y.P.D. this ---
\item
  archived recording 3\\
  The N.Y.P.D. now having to adapt to a noticeable rise in crime since
  the beginning of the year.
\end{itemize}

ashley southall

At the beginning of the year, we started to see crime rise over several
categories. Then in June, we see it really take off.

\begin{itemize}
\item
  archived recording 1\\
  It has been a violent weekend in New York City.
\item
  archived recording 2\\
  The N.Y.P.D. investigating a spree of shootings on this first official
  weekend of summer.
\item
  archived recording 3\\
  More than a dozen people were shot between Friday night and Saturday
  morning.
\end{itemize}

ashley southall

And in July, the trend seems to just accelerate.

\begin{itemize}
\item
  archived recording 1\\
  Nine victims fatally shot tonight across four boroughs.
\item
  archived recording 2\\
  Deadly shootings started just before 1:00 this morning in Brooklyn and
  continued throughout the day in Manhattan, the Bronx and Staten
  Island.
\end{itemize}

ashley southall

So that by the end of the month, we'd had more shootings this year than
all of last year.

{[}sirens blaring{]}

And we still have five months to go.

michael barbaro

Mhm. I guess we should acknowledge, Ashley, that there is a siren behind
you, which has nothing to do with this conversation. It's just what it
means to live in New York.

ashley southall

Yeah. I'm five blocks from a hospital. So it's part of the soundtrack of
my everyday life.

michael barbaro

So at this point, there are meaningfully more shootings and meaningfully
more murders. So Ashley, how should we think about the causes of this?
What exactly is going on here? I think for a lot of people, the instinct
might be to think that this is somehow related to this moment in
American policing --- this forceful rejection of how policing has
occurred, the protests against it and the response of police to those
protests.

ashley southall

I mean, sure, it's easy to think that because it's right in front of us.
It almost seems obvious. But it's really much more complex. And there
are really three big theories about what's happening. The one that has
the most consensus around it is that the pandemic is exacerbating
longstanding problems of gun violence and particularly unemployment.
There are a lot of people with no jobs and nothing to do. And people are
sinking deeper into poverty, which creates a greater desperation. And
there's also a lot of illegal guns in New York City. And that's a really
combustible situation.

michael barbaro

Well, what do you mean? How does that work?

ashley southall

What we've long known in New York City is that a small number of people
are responsible for the vast majority of the violence. And those are
people who are largely in gangs and crews, particularly in public
housing. And those groups compete in normal times. Some gangs sell
drugs. And they have turf that they cover. But when you have a pandemic
hit that causes lots of unemployment, there's less people buying drugs.
And so there's less of a market. And when you have less of a market,
that increases the competition. So then you have these violent gangs
fighting more fiercely for less of a pie.

michael barbaro

OK. So under this theory that the pandemic is exacerbating existing
problems in the city, that explains the pandemic's effect on the people
doing the shooting. What about the effect the pandemic had on the
police?

ashley southall

Immediately what we saw was officers being removed from duty because of
the virus. We had thousands of officers calling out sick every day. And
at the height of the pandemic, almost 20 percent of the force was out
sick because of the virus.

michael barbaro

Wow.

ashley southall

And that had two effects. On-the-street officers, their presence is a
deterrent to violence. You're probably not going to shoot somebody with
a officer standing right there watching you.

michael barbaro

Right.

ashley southall

But if there's no officer, there's no real deterrent. And then, after a
shooting occurs, there's an investigation. In the police department,
detectives work in these squad rooms where they're very close together.
They're practically breathing on top of each other. So if one of them
gets sick, then there's a chance that everybody's sick. So they all go
out sick, and investigations come to a halt. Well, that's a problem.
Because one shooting often leads to another shooting, whether that
person continues shooting rivals, or whether someone is now looking for
him. And it's almost always a him. So when you have cases that can't be
investigated, that means that the ones that can be solved aren't being
solved and that that pattern of violence continues uninterrupted.

michael barbaro

All right. That's the first theory, that the pandemic is both fueling
more shootings and sidelining the officers who could potentially prevent
those shootings. Ashley, what is the second theory of why crime is going
up in New York?

ashley southall

Well, you touched on this earlier. We had massive protests break out in
May after the death of George Floyd in police custody in Minneapolis,
just as police were coming back from the virus. That required resources.
We saw in those early days, the police deployed about 4,000 officers to
the protest. But there was rampant looting in some parts of Manhattan
and Brooklyn. And they had to double the force to about 8,000 police
officers. So that meant that officers have to come from somewhere. And
they typically come from the streets. And some of them come from
detective squads. So that means that fewer police resources are involved
in investigating shootings. And there are less officers in the
neighborhoods where shootings are more likely to occur.

michael barbaro

And how do the protests themselves play into this theory?

ashley southall

Amid these protests, you start to see a political response to the
demands to reform the police.

\begin{itemize}
\tightlist
\item
  archived recording (dermot f. shea)\\
  Within the last hour, I just chaired a meeting of senior executives of
  the police department.
\end{itemize}

ashley southall

And one of those changes was to disband the anti-crime units.

\begin{itemize}
\tightlist
\item
  archived recording (dermot f. shea)\\
  It is regarding the deployment of precinct level and P.S.A. level
  anti-crime units. These are the plainclothes units that operate our
  traditional anti-crime.
\end{itemize}

michael barbaro

And explain what an anti-crime unit is.

ashley southall

Anti-crime officers are the main group responsible for getting guns off
the street in New York City. And they confiscated thousands of guns each
year.

\begin{itemize}
\tightlist
\item
  archived recording (dermot f. shea)\\
  The men and women of the police department were doing what I and
  others before me asked. They have done an exceptional job.
\end{itemize}

ashley southall

But they were controversial. And they became known as the ``jump out
boys.'' Because what they would do sometimes is stop in their cars, get
out and throw somebody up against the wall, search them, and leave
without any explanation or finding any weapons. And it had led to a lot
of tension between police and communities of color.

\begin{itemize}
\tightlist
\item
  archived recording (dermot f. shea)\\
  And it is lost on no one, certainly not the people that live in the
  neighborhoods that we serve, that endure being stopped, or their
  children being stopped.
\end{itemize}

ashley southall

But what was also problematic about anti-crime is that they were
involved in a lot of police-involved shootings.

michael barbaro

And why exactly is that?

ashley southall

Because anti-crime officers are trained to look for guns. They're
looking for violent people. And so in expecting violence, they are
prepared to use it. And the calculation was that New York City could not
afford to have a police-involved shooting at that time. So the
commissioner had been thinking about this, about disbanding anti-crime
for a long time.

\begin{itemize}
\tightlist
\item
  archived recording (dermot f. shea)\\
  Effective immediately, we will be transitioning those units, roughly
  600 people citywide, into a variety of assignments, including
  detective bureau, neighborhood policing and other assignments.
\end{itemize}

ashley southall

And finally, he made the decision in June to pull the plug.

\begin{itemize}
\tightlist
\item
  archived recording (dermot f. shea)\\
  It's a big move when you look at, culturally, how we police this city.
  And what we always struggle with, I believe, as police executives, is
  not keeping crime down. It's keeping crime down and keeping the
  community working with us. And I think those two things at times have
  been at odds. We can do it better. We can do it smarter. And we will.
\end{itemize}

michael barbaro

Is there a feeling, Ashley, that disbanding this unit, at a time when
gun violence is going up, has contributed to the increase in shootings
and in murders?

ashley southall

There's certainly a sense that removing anti-crime officers from the
street and publicly announcing that has let people who carry guns and
who are also involved in violence know that there's no one there to stop
them. And one of the things that we've seen as shootings and murders
have risen, is that gun arrests have declined. And since the anti-crime
unit was disbanded, shootings have only continued to go up.

michael barbaro

And under this second theory, that's centered around the rule of these
protests in the rise of crime, what role does the actual frustration at
the heart of these protests play here --- the mistrust on the side of
the protesters toward the police?

ashley southall

Well, there's always been a level of mistrust of police in communities
that experience higher levels of crime. Because they have experienced
aggressive policing in the past. But what happens when you have a death
like George Floyd that's captured on video, where he's saying the last
words of Eric Garner, who died here six years ago, is that those old
wounds are scratched again and they reopen. And that mistrust for a
while is laid bare again. You know, that can manifest as people not
calling the police or not seeking them out to help with the problem. And
one of the things that the police have said is that they had people who
were cooperating with shooting and murder investigations before the
protests who have now backed out. Because they don't want to have
anything to do with the police. And that makes it harder to solve crimes
that might lead to future ones, like shootings.

michael barbaro

And what about the police side of this equation, this message of the
protests? How are officers in New York City, from your reporting,
internalizing this message and reacting to it, and perhaps changing
their behavior and their approach in response in ways that might be
fueling all of this?

ashley southall

So when you talk to the cops, or even if you listen to the things that
top brass say, what cops take from the protests and the political
response is that their work isn't valuable. And nobody supports them.
That the city doesn't have their backs. And so what we're seeing is
that, in moments where police might have been proactive in making an
arrest or in intervening in a situation that could spiral into violence,
they're actually standing down.

One of the examples that people often give to me is that there are these
barbecues and street parties that go late into the night. Usually the
police would break them up. But instead, they're going on well past
midnight to 2:00, 4:00, sometimes even 6:00 a.m. And the police aren't
doing anything about it. And what we've seen in some instances is that
those parties often lead to a shooting. And one or more people get hurt
or are killed.

michael barbaro

And the cops you're talking to are telling you, yeah, we're hesitant to
engage with these kinds of situations right now.

ashley southall

Yeah. I mean, think about it from the perspective of the cop on the
street who, he's thinking, you know, well, the public doesn't want me
here. And if I step into a situation where I think someone might have a
gun or this thing could get out of control, then I could be arrested.
And then that's the end of my career, and I can't support my family.
That's a very scary thing for cops.

michael barbaro

But I'm curious, Ashley, if in your reporting you get the sense that, on
some level, this may be police saying to the protesters, OK, so you're
so disgusted by us, and you are making all these demands for budget cuts
to fund the police. Do you want to know what the world would look like
if you get your vision implemented? Well, here, we'll give you a taste
of it. We will not go in and break up the barbecue. We will not make the
arrest. In other words, is there some form of kind of protest going on
by the police of the work that they typically do in order to say to the
public, this is what it would be like if you get what protesters out on
the streets are demanding?

ashley southall

It certainly seems that in some way the cops are responding in a way
that almost seems flippant. You want to abolish the police? You want to
defund the police? Well, this is what it looks like when you don't have
the police. And there is a prevailing sense that cops are stepping back
to prove that they are, in fact, necessary.

michael barbaro

Ashley, it doesn't seem like there's a ton of debate around these two
theories that you have laid out here --- the role of the pandemic and
the role of the protests against police. Do I have that right?

ashley southall

Yeah. There's a lot of agreement on the role of the pandemic and also
the effect of the protests. But the third theory is where there's a lot
of debate. And it's heated.

michael barbaro

We'll be right back.

OK. So Ashley, tell me about this third theory about why crime in New
York City is rising, and why is that theory such a source of
disagreement.

ashley southall

So the third theory is one advanced almost exclusively by the New York
Police Department itself. And that theory is that a couple of measures
taken by both the city and the state over the last few years --- first
with criminal justice reform and then through some of the steps they've
taken to slow the spread of the coronavirus --- that those measures are
contributing to putting people back out onto the street who have not
just engaged in criminal activity, but who are also violent.

michael barbaro

I wonder if you can break those two things down. Maybe start with the
criminal justice reforms. What's the theory there?

ashley southall

So last year, the New York state legislature passed a bail law ---

\begin{itemize}
\tightlist
\item
  archived recording (andrew cuomo)\\
  The old system was too reliant on cash.
\end{itemize}

ashley southall

--- that allowed more people to get out of jail while their cases were
winding through the courts.

\begin{itemize}
\tightlist
\item
  archived recording (andrew cuomo)\\
  The old system was if you were rich and you could pay bail, you'd get
  out. If you were poor and you couldn't pay bail, you went to jail.
\end{itemize}

ashley southall

And in most cases, people who are assigned bail do make bail. But it
takes some time. And being in jail for a day, two days, a week, a month,
can have serious consequences for people's lives. They lose their job.
They lose their housing. And they might even lose custody of their kids.

michael barbaro

Right.

ashley southall

And that's all before the case is resolved. So the bail law was intended
to fix this problem that was largely affecting poor people in
communities of color.

\begin{itemize}
\tightlist
\item
  archived recording (andrew cuomo)\\
  It was never supposed to be about money. The whole justice system was
  not supposed to be about who's rich, who's poor.
\end{itemize}

ashley southall

But what the theory is now, at least from the police department, is that
that law put more people onto the streets who were not only criminal,
but are also violent. The police commissioner, Dermot Shea, he's gone
after this bail law ever since he became commissioner and predicted that
it would cause a rise in crime.

\begin{itemize}
\tightlist
\item
  archived recording (dermot f. shea)\\
  When you have individuals that are standing before a judge and
  immediately being released, and essentially everyone in the room knows
  that this person is a danger to the community, I think we need to look
  at the system and make sure that judges can make common sense
  decisions.
\end{itemize}

michael barbaro

So because that law was passed last year and is being implemented now,
the theory here is that a criminal justice reform with a good intention,
right, of not making people buy their freedom before their charges even
get processed by the court, may result in more people who might be
violent, who might do something like get involved in a shooting, being
out of prison and on the street.

ashley southall

That's exactly right. And the second part of this theory is that the
measures undertaken to slow the spread of the coronavirus in the city's
jails and in the state prison system have also put more violent people
onto the streets.

\begin{itemize}
\tightlist
\item
  archived recording (bill de blasio)\\
  Well everyone, I want to talk about what we all will be doing and will
  need to do to make these adjustments and to deal with our new reality.
  We're seeing milestones in the growth of this disease that are just
  absolutely staggering. Things we could not have imagined.
\end{itemize}

ashley southall

So as the coronavirus was spreading, the city started thinking about who
it could release from its jails.

\begin{itemize}
\tightlist
\item
  archived recording (bill de blasio)\\
  In the course of this evening, I will be given results of an effort by
  Department of Corrections and N.Y.P.D., and our mayor's office for
  criminal justice, to review a list of approximately 200 inmates for a
  potential release. Those individuals will be released tomorrow. We
  hope to make decisions on them very quickly. But this process ---
\end{itemize}

ashley southall

And so the argument is that some of these people who have been coming
back onto the streets are involved in the shootings and some of the
murders that we are seeing on the streets today.

michael barbaro

Is there evidence to support that? Because my sense is that the bail
reform deliberately avoided letting violent people --- or allegedly
violent people --- out. And the Covid-19 mitigation plans also attempted
to avoid letting out violent people. So is there any evidence that
people charged with violent crimes were let out through either of those
measures?

ashley southall

Not really. The problem with this theory is that the N.Y.P.D.`s own data
don't really support what they say.

\begin{itemize}
\tightlist
\item
  archived recording (alexandria ocasio-cortez)\\
  At first, the N.Y.P.D., they went out and they said, OK, this uptick
  in crime is happening because of bail reform.
\end{itemize}

ashley southall

And sometimes the data contradict them. This is something that's been
noted by critics including Alexandria Ocasio-Cortez, who is, of course,
a congresswoman representing part of the Bronx and Queens.

\begin{itemize}
\tightlist
\item
  archived recording (alexandria ocasio-cortez)\\
  They just released data a couple weeks ago that showed that out of
  almost all the people who have committed crimes, et cetera, almost
  none have been re-released due to the bail reform. So why is this
  uptick in crime happening? Well, do we think this has to do with the
  fact that there's record unemployment in the United States right now?
  The fact that people are at a level of economic desperation that we
  have not seen since the Great Recession?
\end{itemize}

ashley southall

The police department's data doesn't establish a connection between the
people who are out there shooting other people and the people who are
getting out on bail. There are some examples of people who have been
released who have been rearrested on a gun charge. But the police
haven't provided any evidence linking them to shootings in any kind of
substantial way. It's mostly been, like, an anecdote here or there, but
no consistent pattern.

michael barbaro

And Ashley, what about the release of people from places like Rikers
Island, that famous New York City prison, because of Covid-19? Has there
been any evidence that has led to the surge in shootings and murders?

ashley southall

So this theory plays out similar to the one about bail. And once again,
we have a situation where the N.Y.P.D.`s own data does not show a strong
connection between those people and the violence the police are trying
to pin on them.

michael barbaro

So Ashley, you have laid out these three theories. And the first two
seem like there's a sort of interplay between them. That people are
increasingly desperate and out of work because of the pandemic, and with
cops out sick, the shootings were escalating even further. And then the
protests started, and people were reluctant to engage with the police at
the same time that the police were reluctant to engage with the
community. And I'm mindful that one of the things that seems to support
these first two theories is that all these forces were also at play in
the other cities around America where we have seen rising unrest in
recent months. Other cities with historic issues between police and
communities of color --- so Chicago, Atlanta, Seattle. Whereas bail
reform and prisoner release, those are factors that are specific to New
York.

But even so, I have to think that regardless of the explanation for the
rise in crime, that it gives critics of the defund movement an argument
for opposing that move. Right? They can make the case that a moment
where shootings and murders are up is a pretty bad time to cut funding
for traditional policing. So do you feel like the fact that these
numbers are going up is going to make the case for defunding the police
in New York City harder? Is that what you're seeing?

ashley southall

On the one hand, yes, because no politician wants to be the person who
takes money away from the police department while more people are
getting shot, and more people are dying, and dying from violence that is
preventable. The police presence is supposed to deter violence. And
you're taking money away from them. They don't show up. And so the
violence continues. And so you're going to take the blame for that. It
makes the moment inopportune.

But at the same time, people point to the police failure to get a handle
on shootings and murders that have been rising now for three months
straight. And people see that as evidence of the failure of police to do
their most basic job, which is to keep people safe. So they say, let's
cut their funding and try something else.

michael barbaro

So this moment could actually strengthen the case for defunding the
police?

ashley southall

Yeah. That's it. Because it's not keeping people safe in communities of
color at this moment in New York City.

michael barbaro

Mhm. But Ashley, if you believe that the forces behind the rising crime
rate are connected to the pandemic, and that situation will resolve
itself in the next couple months or a year, and police could start doing
their jobs again as they did before, wouldn't that be enough to keep
shootings historically low like they were over the past few years? Or
has something now changed that will make it difficult to go back to
before?

ashley southall

I wish I could go back in the past and kind of do some tweaks to tell
you what it might look like if things had played out differently, if
there had been no pandemic and no protests. But I can't. What I can tell
you is that even before these events, the police recognized that there
was a really big impediment to fighting crime. And that was a broken
trust between the police, and particularly communities of color, that
was decades in the making.

Commissioner Shea, in disbanding anti-crime, acknowledged that although
they were effective in getting guns off the streets, their tactics were
sometimes harmful. And that was counterproductive to building the kind
of public confidence the police need to have people not only reporting
crimes --- because remember, not all crimes are reported to the police
and many people don't report crime at all --- it also keeps people from
helping the police to solve crimes. Repairing that trust was paramount
before the pandemic, before any protests, and it's only gotten worse.
And so it's not going to be an easy fix.

michael barbaro

So what you're saying is that the crime rate, the number of shootings
and the number of murders reported, has not necessarily been a great
reflection of whether policing in New York is actually working. In fact,
it might have been a pretty misleading number to begin with because of
these deeper root issues that have been there during that period of
quote, unquote ``low crime.'' So going back to before is a kind of
flawed way of thinking about this.

ashley southall

That's right. In this moment, we see these wild cards --- the pandemic
and the protests --- exacerbating the factors that already fuel gun
violence. But even before those came into play and up till now, we're
talking about a very small group of people who carry guns. And the
research shows that they do so because they feel caught between two
worlds that make them feel unsafe. One is the violence in their
communities, whether it be conflicts between gangs with rivalries or
crews in neighboring housing projects that have beefs. They also feel
unsafe around the police. They feel at any moment they could be
targeted, that they could be a George Floyd or an Eric Garner. So there
is no one in their lives who they feel can protect them. And so they
pick up a gun to do it for themselves.

So pandemic or no pandemic, protests or no protests, ultimately it's the
problems in their lives that need to be addressed. And what the experts
will tell you is that policing, even with reforms, is just one part of a
very large puzzle.

michael barbaro

Ashley, thank you very much. I really appreciate it.

ashley southall

Thanks for having me.

michael barbaro

We'll be right back.

Here's what else you need to know today.

\begin{itemize}
\tightlist
\item
  archived recording (louis dejoy)\\
  As we head into the election season, I want to assure this committee
  and the American public that the postal service is fully capable and
  committed to delivering the nation's election mail securely and on
  time.
\end{itemize}

michael barbaro

In testimony before Congress, the postmaster general, Louis DeJoy, said
he was confident that the U.S. Postal Service could handle a major surge
in voting by mail this fall, despite the cost-cutting plans that he has
put in place, which have slowed delivery across the country.

\begin{itemize}
\tightlist
\item
  archived recording (louis dejoy)\\
  We deliver 433 million pieces of mail a day, so 150 million ballots,
  160 million ballots over the course of a week is a very, very small
  amount. Adequate capacity.
\end{itemize}

michael barbaro

Nevertheless, over the weekend, the House of Representatives passed
emergency legislation that would block DeJoy's changes and inject \$25
billion dollars into the Postal Service before the election.

\begin{itemize}
\tightlist
\item
  archived recording (carolyn b. maloney)\\
  This is not a partisan issue. It makes absolutely no sense to
  implement these dramatic changes in the middle of a pandemic less than
  three months before the November elections. The American people do not
  want anyone messing with the Post Office.
\end{itemize}

michael barbaro

And the Republican National Convention will begin tonight with a heavy
emphasis on the president, his family and his White House staff. The
Times reports that, in an unusual decision, Trump is scheduled to speak
on all four nights. And that no previous Republican presidents or
presidential nominees will appear at the convention.

That's it for ``The Daily.'' I'm Michael Barbaro. See you tomorrow.

In Detroit, Chief James Craig said violence spiked but has started to go
down over the past two weekends. ``We haven't relaxed our enforcement
posture like some cities,'' he said.

In Kansas City, homicides have been on a swift upward trajectory from
the time a 41-year-old man named Earl Finch III was gunned down in a
driveway in broad daylight on Jan. 5, the first murder of the year. Even
the coronavirus lockdown did not slow the violence, though as in other
cities it has escalated even further in the wake of reopenings.

After six new deaths over the weekend, 122 people have been killed this
year, compared with 90 through the same time last year. The city is well
on its way to surpassing its grim record of 153 murders in 1993. And by
the end of July the city had matched the number of nonfatal shootings
--- about 490 --- that it had all of last year.

Much of the violence in Kansas City has had little rhyme or reason,
often stemming from petty arguments that boil over.

The short fuses may indicate restlessness and anger, criminologists and
law enforcement officials said. The police have attributed about 30 of
the homicides this year to arguments, some involving people with no
serious criminal history. Economic hardship also appeared to be a factor
in some of the killings. Only 15 were deemed drug-related. In almost 50
cases, the police have not yet determined a motive.

While disparities in things like education and employment have long
plagued Kansas City's East Side, a predominantly Black part of the city
that has the city's highest murder rate, community leaders said there
seemed to be an added sense of despair this year.

The Rev. Darren Faulkner, who runs a program that provides social
support to those deemed most at risk of violence, said the latest wave
of police killings of Black people has left many of his clients feeling
hopelessly trapped in a system in which they will never thrive.

``People have gotten to the point where they just don't give a damn,''
he said. ``I don't care about me. I certainly don't care about you. And
so I can go shoot your house or shoot you right on the spot because you
talked to me crazy, you looked at me crazy.''

Spontaneous, one-on-one beefs have replaced gang feuds as a driver of
shootings, said Maj. Greg Volker of the Kansas City Police Department.

``If people could settle an argument without having to resort to
shooting, violence would reduce,'' he said.

Another atypical trend this year is that in several cases, the gunmen
and victims were not otherwise involved in criminal activity, Major
Volker said, pointing to the gas station shooting in July.

The man now charged with murder in the case is a meatpacking worker,
Isaac Knighten, 40, who devotes much of his time to mentoring Black men
and boys, including teaching conflict resolution through Alpha Male
Nation, a mentoring organization his brother started. His wife said he
had turned his life around after serving time on drug charges from more
than a decade ago.

After Mr. Knighten had a brief, hostile exchange with the other man in
the parking lot, the man, Jayvon McCray, 28, pulled a gun and the men
began to fight, according to the police.

Image

Isaac Knighten, who is charged with murdering a man in a gas station
dispute.

Mr. Knighten's wife, Shaynan, said in an interview that she had their
five children get out of the car and run to a relative's house nearby.
She and Mr. McCray's girlfriend both got between the men and urged them
to calm down, according to the police.

Mr. Knighten eventually retrieved a gun from his car and fatally shot
Mr. McCray, who the police said appeared to no longer be holding a gun.

Mr. Knighten's lawyer, Dan Ross, said his client, who has been charged
with second-degree murder, was defending himself. Surveillance footage
shows that Mr. Knighten attempted to walk away from the dispute at least
six times, but Mr. McCray kept coming after him, the lawyer said.

Another contributing factor to this year's violence, Major Volker said,
was the impact of the coronavirus stay-at-home order on the drug trade.
Some dealers lost their regular buyers, so they sold to people they did
not know --- people who may have been intent on robbing them. The result
has been an uptick in drug robberies and shootings, especially in late
March and early April.

The real explosion of killings in Kansas City came in May and June, with
44 murders combined, more than twice as many during those same months
last year.

``I'm sure there will be academic studies for years to come as to what
caused the spike of 2020,'' said Tim Garrison, the United States
Attorney for the Western District of Missouri. ``I'm sure the lockdown
didn't help. When you already have a stressed economic situation and you
put a lot of folks out of work, and a lot of teenagers out of school,
it's a volatile situation.''

Mr. Garrison oversees Operation LeGend, a surge of some 200 federal
agents into Kansas City in an effort to help stem the violence. It has
been met with suspicion and street protests, in part because the
operation coincided with a militaristic
\href{https://www.nytimes3xbfgragh.onion/2020/07/26/us/protests-portland-seattle-trump.html}{federal
intervention on the streets of Portland} that was widely criticized for
inflaming tensions there.

Mr. Garrison said trained federal investigators have beefed up existing
task forces, seized dozens of guns, brought in suspects on existing
warrants and helped arrest a dozen homicide suspects.

Jean Peters Baker, the prosecutor in Jackson County, said that in the
murder cases she has received from Operation LeGend so far, the federal
agents did not appear to have contributed the forensic investigative
expertise that the federal authorities had promised.

Mr. Garrison pointed to a more recent arrest by the U.S. Marshals
Service, saying that federal investigators had linked a firearm found in
the suspect's possession to four other shootings.

The operation has been expanded to seven other cities, all but one of
which have seen an increase in homicides over last year. Some officials
have welcomed the help, while others have promised to monitor federal
agents for civil rights violations.

The federal operation represents the latest in a string of efforts that
Kansas City has undertaken to try to get its violence under control over
the years. Homicides dropped to a near record low of 80 in 2014, after
the launch of a joint federal-local operation known as the Kansas City
No Violence Alliance. But murders began ticking back up in subsequent
years, and the police pulled back from the program. An effort the
department launched last year to get the worst violent offenders off the
streets is expected to be expanded in September to help improve
coordination with the county prosecutor.

Charron Powell, LeGend's mother, said she gave permission for her son's
name to be used in the federal operation because she wanted the fight
against violence to be his legacy. She called the killings ``senseless''
and said those responsible had met with too few consequences.

``It may not work,'' she said, noting the opposition Operation LeGend
has encountered from many in the city. Still, she said, ``it's a good
thing they're trying --- they're trying something.''

John Eligon reported from Kansas City, Mo., and Shaila Dewan and
Nicholas Bogel-Burroughs reported from New York. Ashley Southall
contributed reporting from New York. Alain Delaquérière contributed
research.

Advertisement

\protect\hyperlink{after-bottom}{Continue reading the main story}

\hypertarget{site-index}{%
\subsection{Site Index}\label{site-index}}

\hypertarget{site-information-navigation}{%
\subsection{Site Information
Navigation}\label{site-information-navigation}}

\begin{itemize}
\tightlist
\item
  \href{https://help.nytimes3xbfgragh.onion/hc/en-us/articles/115014792127-Copyright-notice}{©~2020~The
  New York Times Company}
\end{itemize}

\begin{itemize}
\tightlist
\item
  \href{https://www.nytco.com/}{NYTCo}
\item
  \href{https://help.nytimes3xbfgragh.onion/hc/en-us/articles/115015385887-Contact-Us}{Contact
  Us}
\item
  \href{https://www.nytco.com/careers/}{Work with us}
\item
  \href{https://nytmediakit.com/}{Advertise}
\item
  \href{http://www.tbrandstudio.com/}{T Brand Studio}
\item
  \href{https://www.nytimes3xbfgragh.onion/privacy/cookie-policy\#how-do-i-manage-trackers}{Your
  Ad Choices}
\item
  \href{https://www.nytimes3xbfgragh.onion/privacy}{Privacy}
\item
  \href{https://help.nytimes3xbfgragh.onion/hc/en-us/articles/115014893428-Terms-of-service}{Terms
  of Service}
\item
  \href{https://help.nytimes3xbfgragh.onion/hc/en-us/articles/115014893968-Terms-of-sale}{Terms
  of Sale}
\item
  \href{https://spiderbites.nytimes3xbfgragh.onion}{Site Map}
\item
  \href{https://help.nytimes3xbfgragh.onion/hc/en-us}{Help}
\item
  \href{https://www.nytimes3xbfgragh.onion/subscription?campaignId=37WXW}{Subscriptions}
\end{itemize}
