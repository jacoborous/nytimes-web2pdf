Sections

SEARCH

\protect\hyperlink{site-content}{Skip to
content}\protect\hyperlink{site-index}{Skip to site index}

\href{https://www.nytimes3xbfgragh.onion/section/food/drinks}{Wine, Beer
\& Cocktails}

\href{https://myaccount.nytimes3xbfgragh.onion/auth/login?response_type=cookie\&client_id=vi}{}

\href{https://www.nytimes3xbfgragh.onion/section/todayspaper}{Today's
Paper}

\href{/section/food/drinks}{Wine, Beer \& Cocktails}\textbar{}Wine Joins
the 2020 Debate Over Privilege and Justice

\url{https://nyti.ms/3ifTiTc}

\begin{itemize}
\item
\item
\item
\item
\item
\item
\end{itemize}

Advertisement

\protect\hyperlink{after-top}{Continue reading the main story}

Supported by

\protect\hyperlink{after-sponsor}{Continue reading the main story}

The Pour

\hypertarget{wine-joins-the-2020-debate-over-privilege-and-justice}{%
\section{Wine Joins the 2020 Debate Over Privilege and
Justice}\label{wine-joins-the-2020-debate-over-privilege-and-justice}}

An accusation involving migrant labor in Puglia leads to
self-examination and, perhaps, new awareness of the treatment of
agricultural workers.

\includegraphics{https://static01.graylady3jvrrxbe.onion/images/2020/08/12/dining/12pour-final/12pour-final-articleLarge.jpg?quality=75\&auto=webp\&disable=upscale}

\href{https://www.nytimes3xbfgragh.onion/by/eric-asimov}{\includegraphics{https://static01.graylady3jvrrxbe.onion/images/2018/06/13/multimedia/author-eric-asimov/author-eric-asimov-thumbLarge.jpg}}

By \href{https://www.nytimes3xbfgragh.onion/by/eric-asimov}{Eric Asimov}

\begin{itemize}
\item
  Published Aug. 6, 2020Updated Aug. 10, 2020
\item
  \begin{itemize}
  \item
  \item
  \item
  \item
  \item
  \item
  \end{itemize}
\end{itemize}

In this topsy-turvy year of the Covid-19 pandemic and a national uproar
over politics and racial injustice, few things are immune from the
widespread cultural re-evaluation.

The wine world, too, is re-examining its business practices and
responsibilities. In recent weeks, the focus has turned to the case of
\href{https://wonderwomenofwine.com/femmefriday-interviews/2020/2/28/valentina-passalacqua-22820}{Valentina
Passalacqua} --- a natural-wine producer in Puglia, the region at the
heel of Italy's boot --- whom few Americans had ever heard of until
recently.

Over the last year, though, she drew a meteoric rise in attention as her
products were picked up by two of New York's most important importers of
natural wines, \href{https://zrswines.com/}{Zev Rovine Selections} and
\href{http://www.jennyandfrancois.com/}{Jenny \& François Selections}.
\href{http://valentinapassalacqua.it/?lang=en}{Her wines} were also
featured by \href{https://www.dryfarmwines.com/}{Dry Farm Wines}, a
natural-wine club that ships to 44 states, promising bottles that
``whisper in nature's perfect logic and design.''

But her upward trajectory as a natural-wine exemplar took a swift
nosedive in early July when her father, Settimio Passalacqua, a marble
and agriculture magnate in Puglia, was placed under house arrest by the
carabinieri, the national police. Prosecutors accused him of the
systematic and illegal exploitation of migrant workers in his produce
operation.

The Italian authorities have not suggested that Ms. Passalacqua was
complicit in the crimes they say her father committed. But over the last
month, many people in natural-wine circles, using the social justice
language of 2020, turned on her, questioning both whether she was
operating separately from her father and whether she had benefited from
the economic privilege of his actions, regardless of her personal
culpability.

\includegraphics{https://static01.graylady3jvrrxbe.onion/images/2020/08/12/dining/06pour/06pour-articleLarge.jpg?quality=75\&auto=webp\&disable=upscale}

By the end of July, Ms. Passalacqua's wines had been dropped by both her
New York-based importers, as well as by Dry Farm.

Ms. Passalacqua has maintained that her winery and vineyard are
independent of her father, and has strenuously denied any involvement
with his business.

``I am outraged by the working conditions my father is accused of
creating at this farm, and he should be punished if he did what he is
accused of,'' she said in a statement from
\href{https://www.goldin.com/}{Goldin Solutions}, a crisis public
relations firm in New York.

``Every person deserves the respect and dignity of a living wage and
good working conditions, which I am proud to provide at my vineyard. I
am optimistic that the importers will resume work with me quickly as
they become assured of the fact that blaming me for what my father
allegedly did at a totally different business is contrary to the spirit
of supporting women entrepreneurs who run ethical operations.''

Mr. Passalacqua is accused of engaging in
\href{https://ec.europa.eu/migrant-integration/news/italian-parliamentary-investigation-on-exploitation-of-migrant-workers-in-agriculture}{caporalato},
in which intermediaries act as labor contractors, arranging for
migrants, in this case from Northern Africa and Eastern Europe, to do
agricultural work while confining them in slum conditions and paying
them substandard wages.

It's a problem that has particularly plagued southern Italy, often in
conjunction with organized crime. Back in 2010, immigrant agricultural
workers near Rosarno, in Calabria, the toe of the boot,
\href{https://www.nytimes3xbfgragh.onion/2010/01/11/world/europe/11italy.html}{rebelled
violently} against exploitation and shameful conditions. The violence
shocked the country, and prompted many, including Pope Benedict XVI, to
criticize the exploitation of immigrants.

In 2015, the
\href{https://www.nytimes3xbfgragh.onion/2017/04/11/world/europe/a-womans-death-sorting-grapes-exposes-italys-slavery.html}{death
of a vineyard worker} in Puglia inspired new laws aimed at protecting
agricultural workers. But experts contend that many agricultural workers
in southern Italy continue to face slavelike conditions.

The accusations, though centered on Mr. Passalacqua's agricultural
operation and not his daughter's vineyards, are a reminder of the
precarious position of agricultural workers all over the wine world,
whose work is often unrecognized and who frequently depend on the
conscience of their employers to assure them of proper working
conditions and benefits.

It's an issue of human dignity that the entire wine world must confront,
particularly in the United States, where stringent immigration policies
and the Covid-19 pandemic have compounded risks for agricultural
workers.

But the suggestion of human exploitation has particular resonance in the
natural-wine realm, which --- whatever the motivations of individual
producers, importers and retailers --- often portrays its environmental,
ecological and production methods as moral and ethical choices.

Nonetheless, questions regarding migrant workers rarely come up. Most
estates are small enough, 10 to 30 acres, to be farmed with a local
labor force. For harvests, vineyard owners typically find the necessary
hands among friends and family.

But Ms. Passalacqua farmed 80 hectares, almost 200 acres, making her an
outlier in natural wine as well as a sort of unicorn for importers.

They saw a rare opportunity to scale up their businesses, to buy in
quantity and sell bottles that would retail in the moderate \$20-to-\$30
range, especially important at a time when most wines from France, their
prime source for natural wine, have been subject to a
\href{https://www.nytimes3xbfgragh.onion/2019/10/08/dining/trump-tariffs-wine.html}{25
percent tariff.}

``When you throw an 80-hectare winery onto the market all of a sudden,
it fills these critical holes in natural wine,'' said Zev Rovine of Zev
Rovine Selections, which imported her Valentina Passalacqua wines, one
of several Passalacqua brands,
\href{https://zrswines.com/wine-producer/valentina-passlacqua/}{until
mid-July}. ``Very few natural wines are cheap, and she filled that hole
with as much wine as you might want.''

The question of whether to continue doing business with Ms. Passalacqua
fell squarely into the larger discussion of social and economic
privilege. While some people scoffed at Ms. Passalacqua's efforts to
distance herself from her father, others pointed to benefits that she
enjoyed as a result of the wealth he created over many years in
businesses that may not have always been above the law.

In a sense, her case could be likened to that of white American families
in the 20th century who were able to build wealth by buying real estate
in areas that racially discriminated against Black people, creating
economic advantages that extended for generations. Though perhaps
descendants of those families have done nothing wrong personally, they
have still benefited from past injustices.

``I do believe Valentina in her heart is a really good person, that she
sees injustice and wants to change things,'' Mr. Rovine said. ``She says
she's fought her father all her life, and that she doesn't believe in
her father's way of business.

``But it was too hard to separate her from her family's history. Not
knowing what the truth is, it's too close for us to say this producer
doesn't do any of this stuff. I can't tell my clients that, I can't tell
my employees that, I can't tell myself that.''

For Jenny Lefcourt of Jenny \& François Selections, which imported Ms.
Passalacqua's Calcarius brand, the question was not so clear-cut. When
the initial reports came out, she stood by Ms. Passalacqua, not wanting
to blame the daughter for the sins of the father.

Ms. Lefcourt's hesitancy opened her up to accusations of hypocrisy, of
refusing to sacrifice economically, even though Jenny \& François has
portrayed itself as a company that stands up for social justice.

``This isn't about cancel culture,'' wrote Jennifer Green --- who
publishes \href{https://www.glougloumagazine.com/}{Glou Glou}, a wine
zine, and runs \href{https://www.superglou.com/new-glou}{Super Glou}, a
small natural-wine importing business --- on Instagram. ``This is about
our impulse to preach at the altar of wokeness, only to abandon that
platform when it suits our whims and especially our wallets.''

The response stung Ms. Lefcourt, who has been a pioneer in American
natural-wine culture and recently marked Jenny \& François's 20th
anniversary as an importer.

``I'm a political person, and I hope to represent people whose beliefs
align with my own, who respect human dignity and never discriminate or
exploit,'' she said. ``I wanted to give her a chance to defend
herself.''

By the end of July, though, she, too, had decided to
\href{https://www.instagram.com/p/CDUham7J6JP/}{drop the brand}.

``There's land that her father owns that her vines are planted on, and
even if the labor she used was paid fairly, if she's using that land
she's profiting from the exploitation of labor,'' Ms. Lefcourt said.
``Even that's not clear, but it's still too close for comfort, and I
don't feel she separated her interests enough from his.''

Regardless of whether Ms. Passalacqua's wines are sold in the United
States --- and plenty of the wines are still on retail shelves --- it
should not be forgotten that this is ultimately a story about the
vulnerability of agricultural workers and wine's role in assuring them
safe, humane and dignified working conditions.

Romanticizing wine as a natural, pastoral product often results in
omitting the human labor that goes into its creation. This omission can
often create the conditions for exploitation.

``We have been willing to fetishize agricultural products that are
appealing to us, without scrutinizing the entire supply chain,'' Ms.
Green said. ``When we're discussing farming, we leave out the
farmworkers.''

\emph{Follow} \href{https://twitter.com/nytfood}{\emph{NYT Food on
Twitter}} \emph{and}
\href{https://www.instagram.com/nytcooking/}{\emph{NYT Cooking on
Instagram}}\emph{,}
\href{https://www.facebookcorewwwi.onion/nytcooking/}{\emph{Facebook}}\emph{,}
\href{https://www.youtube.com/nytcooking}{\emph{YouTube}} \emph{and}
\href{https://www.pinterest.com/nytcooking/}{\emph{Pinterest}}\emph{.}
\href{https://www.nytimes3xbfgragh.onion/newsletters/cooking}{\emph{Get
regular updates from NYT Cooking, with recipe suggestions, cooking tips
and shopping advice}}\emph{.}

Advertisement

\protect\hyperlink{after-bottom}{Continue reading the main story}

\hypertarget{site-index}{%
\subsection{Site Index}\label{site-index}}

\hypertarget{site-information-navigation}{%
\subsection{Site Information
Navigation}\label{site-information-navigation}}

\begin{itemize}
\tightlist
\item
  \href{https://help.nytimes3xbfgragh.onion/hc/en-us/articles/115014792127-Copyright-notice}{©~2020~The
  New York Times Company}
\end{itemize}

\begin{itemize}
\tightlist
\item
  \href{https://www.nytco.com/}{NYTCo}
\item
  \href{https://help.nytimes3xbfgragh.onion/hc/en-us/articles/115015385887-Contact-Us}{Contact
  Us}
\item
  \href{https://www.nytco.com/careers/}{Work with us}
\item
  \href{https://nytmediakit.com/}{Advertise}
\item
  \href{http://www.tbrandstudio.com/}{T Brand Studio}
\item
  \href{https://www.nytimes3xbfgragh.onion/privacy/cookie-policy\#how-do-i-manage-trackers}{Your
  Ad Choices}
\item
  \href{https://www.nytimes3xbfgragh.onion/privacy}{Privacy}
\item
  \href{https://help.nytimes3xbfgragh.onion/hc/en-us/articles/115014893428-Terms-of-service}{Terms
  of Service}
\item
  \href{https://help.nytimes3xbfgragh.onion/hc/en-us/articles/115014893968-Terms-of-sale}{Terms
  of Sale}
\item
  \href{https://spiderbites.nytimes3xbfgragh.onion}{Site Map}
\item
  \href{https://help.nytimes3xbfgragh.onion/hc/en-us}{Help}
\item
  \href{https://www.nytimes3xbfgragh.onion/subscription?campaignId=37WXW}{Subscriptions}
\end{itemize}
