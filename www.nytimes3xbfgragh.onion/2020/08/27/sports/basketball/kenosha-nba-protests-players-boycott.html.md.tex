\href{/section/sports/basketball}{Pro Basketball}\textbar{}With
Walkouts, a New High Bar for Protests in Sports Is Set

\url{https://nyti.ms/31AtATH}

\begin{itemize}
\item
\item
\item
\item
\item
\item
\end{itemize}

\hypertarget{race-and-america}{%
\subsubsection{\texorpdfstring{\href{https://www.nytimes3xbfgragh.onion/news-event/george-floyd-protests-minneapolis-new-york-los-angeles?name=styln-george-floyd\&region=TOP_BANNER\&block=storyline_menu_recirc\&action=click\&pgtype=Article\&impression_id=73314a90-f1bb-11ea-a1b0-a5a6897b98d6\&variant=undefined}{Race
and America}}{Race and America}}\label{race-and-america}}

\begin{itemize}
\tightlist
\item
  \href{https://www.nytimes3xbfgragh.onion/2020/09/04/nyregion/rochester-police-daniel-prude.html?name=styln-george-floyd\&region=TOP_BANNER\&block=storyline_menu_recirc\&action=click\&pgtype=Article\&impression_id=73314a91-f1bb-11ea-a1b0-a5a6897b98d6\&variant=undefined}{How
  Police Handled Death of Daniel Prude}
\item
  \href{https://www.nytimes3xbfgragh.onion/2020/09/01/us/politics/trump-fact-check-protests.html?name=styln-george-floyd\&region=TOP_BANNER\&block=storyline_menu_recirc\&action=click\&pgtype=Article\&impression_id=73314a92-f1bb-11ea-a1b0-a5a6897b98d6\&variant=undefined}{Trump
  Fact Check}
\item
  \href{https://www.nytimes3xbfgragh.onion/2020/08/30/us/portland-shooting-explained.html?name=styln-george-floyd\&region=TOP_BANNER\&block=storyline_menu_recirc\&action=click\&pgtype=Article\&impression_id=733171a0-f1bb-11ea-a1b0-a5a6897b98d6\&variant=undefined}{Portland
  Shooting}
\item
  \href{https://www.nytimes3xbfgragh.onion/2020/08/30/us/breonna-taylor-police-killing.html?name=styln-george-floyd\&region=TOP_BANNER\&block=storyline_menu_recirc\&action=click\&pgtype=Article\&impression_id=733171a1-f1bb-11ea-a1b0-a5a6897b98d6\&variant=undefined}{Breonna
  Taylor's Life and Death}
\end{itemize}

\includegraphics{https://static01.graylady3jvrrxbe.onion/images/2020/08/28/sports/28Streeter-Illo/28Streeter-Illo-articleLarge.jpg?quality=75\&auto=webp\&disable=upscale}

Sections

\protect\hyperlink{site-content}{Skip to
content}\protect\hyperlink{site-index}{Skip to site index}

sports of the times

\hypertarget{with-walkouts-a-new-high-bar-for-protests-in-sports-is-set}{%
\section{With Walkouts, a New High Bar for Protests in Sports Is
Set}\label{with-walkouts-a-new-high-bar-for-protests-in-sports-is-set}}

The deafening silence of not having sports provided a powerful message
from athletes demanding racial justice, after kneeling and slogans on
jerseys seemed to have lost potency.

Credit...Dave Murray

Supported by

\protect\hyperlink{after-sponsor}{Continue reading the main story}

\href{https://www.nytimes3xbfgragh.onion/by/kurt-streeter}{\includegraphics{https://static01.graylady3jvrrxbe.onion/images/2018/11/26/multimedia/author-kurt-streeter/author-kurt-streeter-thumbLarge.png}}

By \href{https://www.nytimes3xbfgragh.onion/by/kurt-streeter}{Kurt
Streeter}

\begin{itemize}
\item
  Aug. 27, 2020
\item
  \begin{itemize}
  \item
  \item
  \item
  \item
  \item
  \item
  \end{itemize}
\end{itemize}

It was the silence that spoke loudest.

No basketballs pounding on hardwood. All games canceled. No baseballs
cracking off bats. Three games canceled. No soccer balls ricocheting
down the field. Five games canceled. No booming aces. The Western \&
Southern Open tennis tournament halted for a day.

This is what the silence said: No more Jacob Blakes. No more George
Floyds. No more Breonna Taylors. No more Natasha McKennas. No more
Philando Castiles. No more Michael Browns. No more Tamir Rices. No more
Eric Garners. No more Alton Sterlings.

No more pain.

Never before has the world of sports spoken so emphatically. The timing
was unmistakably significant. The athlete walkouts were set starkly
against a frightened Trumpian vision presented at the Republican
National Convention.

We watched this week as two Americas clashed in front of us, separated
by generations and by oceans-apart views of race, justice and what it
means to be a patriot.

No longer was sports offering gentrified protest, with league-endorsed
slogans on basketball jerseys. Calm collapsed in the face of the
inevitably growing power of players to make more than a statement. They
took action. It shattered the bubble of normalcy that had settled upon
the
\href{https://www.nytimes3xbfgragh.onion/2020/08/28/sports/basketball/nba-playoffs-resume.html}{N.B.A.}
and its fans, who watched happily from home as a pandemic and protests
raged.

``We are scared as Black people in America,'' LeBron James said,
downcast as he spoke at a news conference inside the N.B.A.'s so-called
bubble at Walt Disney World near Orlando, Fla.

``Because you don't know, you have no idea, how that cop that day left
the house,'' he added. ``You don't know if he woke up on a good side of
the bed, if he woke up on the wrong side of the bed. \ldots{} Or maybe
he just left the house saying, `Today is going to be the end for one of
these Black people.' That is what it feels like.''

Jaylen Brown of the Boston Celtics spoke with equally raw emotion: ``Are
we not human beings? Is Jacob Blake not a human being? He deserved to be
treated like a human being and did not deserve to be shot.''

Sterling Brown, the Milwaukee forward who in 2018 was tackled by police
officers and shocked with a Taser gun after a parking violation, read a
statement for his team that concluded, ``Despite the overwhelming plea
for change there has been no action, so our focus today cannot be on
basketball.''

Jaylen Brown is 23. Sterling Brown is 25. LeBron James, one of the older
players in the league, is only 35. All three, like so many of their
N.B.A. compatriots, are part of an emboldened generation of Black
athletes, a vanguard challenging America's norms in numbers never seen
before.

At the very same time,
\href{https://www.nytimes3xbfgragh.onion/2020/08/27/us/politics/republican-national-convention-recap.html}{the
Republican National Convention} represented and embraced an entirely
different vision --- one nostalgic for the past, wary of change and
angry for an entirely different reason. Sports personalities from an era
when player protests were rarer figured prominently. Lou Holtz, the
renowned 83-year-old college football coach who last led a team 16 years
ago, proclaimed steadfast devotion to President Trump and spoke
triumphantly of a mythical America where anyone can succeed by just
working hard enough.

Herschel Walker and Jack Brewer, both Black former N.F.L. players out of
the league for well over a decade, struck the same tone, hailing Trump
as a heaven-sent crusader against racism and a proponent of social
justice, ignoring a reality that says the opposite.

Two visions. Two Americas.

2020 vs. years gone by.

The N.B.A. is hardly alone. Walkouts
\href{https://www.nytimes3xbfgragh.onion/2020/08/27/sports/basketball/nba-resume.html}{rippled
this week} through the W.N.B.A., and through predominantly white sports
like professional tennis and soccer. Games were
\href{https://www.nytimes3xbfgragh.onion/2020/08/27/us/difference-boycott-strike-nba.html}{postponed
because of protesting players} in conservative, tradition-bound Major
League Baseball. At first the National Hockey League continued with its
playoff schedule, before bending to pressure and taking a pause.

This was the logical next step in the fervent activism inspired this
year by the killing of George Floyd. As the nation grappled with
\href{https://www.nytimes3xbfgragh.onion/interactive/2019/08/14/magazine/1619-america-slavery.html}{401
years of racial trauma}, it searched for ways to break apart systemic
injustice and violence against Black Americans.

Players as prominent as Kyrie Irving of the Brooklyn Nets declared that
holding a season now, resuming amid the pandemic, was a mistake and a
distraction --- and called for athletes to stay home and work within
their communities for change.

But the N.B.A. and the W.N.B.A got back to work. The players chose to
use nationally televised games as a platform for their grievances. They
draped their courts and jerseys with slogans and calls for change. They
knelt during the national anthem.

Yet those protests had lost their power. Slogans and refusing to stand
for the anthem seemed less edgy when everyone --- even corporate
sponsors and team owners --- glommed onto the movement like a fad.

Indeed, violence against Black people escalated. And that brought on
this week's refusal by the N.B.A. and others to play sports. It was a
swift jolt to leagues, owners and networks that live off televised
broadcast games.

Sports have long been a platform capable of providing shocks to the
status quo. More than 50 years ago,
\href{https://www.nytimes3xbfgragh.onion/2020/06/13/sports/tommie-smith-protest-colin-kaepernick.html}{Tommie
Smith} and John Carlos raised their black-gloved fists at the 1968
Summer Olympics in Mexico City.

Muhammad Ali refused to fight in the Vietnam War. Billie Jean King, the
W.N.B.A. star Maya Moore and a long line of female athletes fought for
justice and equal pay. And, of course, four years ago this week, Colin
Kaepernick was spotted for the first time in his protest of police
brutality, refusing to stand during the national anthem.

The current refusal to play, however, is not simply a shock. This is an
earthquake. Walkouts like this have never happened before in pro sports.
Though this appears to be a temporary work stoppage --- N.B.A. players
have voted to return, probably this weekend, and other sports seemed to
be following suit --- a new high bar of protest has been established.

Black athletes and their allies will not hesitate to effectively strike
again. Next time, the stoppage may well last longer than a few days.
Maybe players will sit out an entire season. Maybe they will be from the
N.F.L. Maybe Black college football players and their teammates at
schools like Alabama, Florida and Oklahoma will walk.

Perhaps the other America will stiffen in tried-and-true backlash and
persist in harkening to days of old.

But the silence will speak to us all.

Kurt Streeter is the new Sports of The Times columnist. He has been a
sports feature writer at The Times since 2017 and previously worked at
ESPN and The Los Angeles Times. See his work
\href{https://www.nytimes3xbfgragh.onion/by/kurt-streeter}{here}.

Advertisement

\protect\hyperlink{after-bottom}{Continue reading the main story}

\hypertarget{site-index}{%
\subsection{Site Index}\label{site-index}}

\hypertarget{site-information-navigation}{%
\subsection{Site Information
Navigation}\label{site-information-navigation}}

\begin{itemize}
\tightlist
\item
  \href{https://help.nytimes3xbfgragh.onion/hc/en-us/articles/115014792127-Copyright-notice}{©~2020~The
  New York Times Company}
\end{itemize}

\begin{itemize}
\tightlist
\item
  \href{https://www.nytco.com/}{NYTCo}
\item
  \href{https://help.nytimes3xbfgragh.onion/hc/en-us/articles/115015385887-Contact-Us}{Contact
  Us}
\item
  \href{https://www.nytco.com/careers/}{Work with us}
\item
  \href{https://nytmediakit.com/}{Advertise}
\item
  \href{http://www.tbrandstudio.com/}{T Brand Studio}
\item
  \href{https://www.nytimes3xbfgragh.onion/privacy/cookie-policy\#how-do-i-manage-trackers}{Your
  Ad Choices}
\item
  \href{https://www.nytimes3xbfgragh.onion/privacy}{Privacy}
\item
  \href{https://help.nytimes3xbfgragh.onion/hc/en-us/articles/115014893428-Terms-of-service}{Terms
  of Service}
\item
  \href{https://help.nytimes3xbfgragh.onion/hc/en-us/articles/115014893968-Terms-of-sale}{Terms
  of Sale}
\item
  \href{https://spiderbites.nytimes3xbfgragh.onion}{Site Map}
\item
  \href{https://help.nytimes3xbfgragh.onion/hc/en-us}{Help}
\item
  \href{https://www.nytimes3xbfgragh.onion/subscription?campaignId=37WXW}{Subscriptions}
\end{itemize}
