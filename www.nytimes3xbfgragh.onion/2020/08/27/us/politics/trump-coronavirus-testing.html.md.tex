Sections

SEARCH

\protect\hyperlink{site-content}{Skip to
content}\protect\hyperlink{site-index}{Skip to site index}

\href{https://www.nytimes3xbfgragh.onion/section/politics}{Politics}

\href{https://myaccount.nytimes3xbfgragh.onion/auth/login?response_type=cookie\&client_id=vi}{}

\href{https://www.nytimes3xbfgragh.onion/section/todayspaper}{Today's
Paper}

\href{/section/politics}{Politics}\textbar{}C.D.C.'s `Clarification' on
Coronavirus Testing Offers More Confusion

\url{https://nyti.ms/2YElWpw}

\begin{itemize}
\item
\item
\item
\item
\item
\end{itemize}

\hypertarget{the-coronavirus-outbreak}{%
\subsubsection{\texorpdfstring{\href{https://www.nytimes3xbfgragh.onion/news-event/coronavirus?name=styln-coronavirus-national\&region=TOP_BANNER\&block=storyline_menu_recirc\&action=click\&pgtype=Article\&impression_id=b6433390-f4ca-11ea-aafb-7b2c4efa9239\&variant=undefined}{The
Coronavirus
Outbreak}}{The Coronavirus Outbreak}}\label{the-coronavirus-outbreak}}

\begin{itemize}
\tightlist
\item
  live\href{https://www.nytimes3xbfgragh.onion/2020/09/11/world/covid-19-coronavirus.html?name=styln-coronavirus-national\&region=TOP_BANNER\&block=storyline_menu_recirc\&action=click\&pgtype=Article\&impression_id=b6433391-f4ca-11ea-aafb-7b2c4efa9239\&variant=undefined}{Latest
  Updates}
\item
  \href{https://www.nytimes3xbfgragh.onion/interactive/2020/us/coronavirus-us-cases.html?name=styln-coronavirus-national\&region=TOP_BANNER\&block=storyline_menu_recirc\&action=click\&pgtype=Article\&impression_id=b6435aa0-f4ca-11ea-aafb-7b2c4efa9239\&variant=undefined}{Maps
  and Cases}
\item
  \href{https://www.nytimes3xbfgragh.onion/interactive/2020/science/coronavirus-vaccine-tracker.html?name=styln-coronavirus-national\&region=TOP_BANNER\&block=storyline_menu_recirc\&action=click\&pgtype=Article\&impression_id=b6435aa1-f4ca-11ea-aafb-7b2c4efa9239\&variant=undefined}{Vaccine
  Tracker}
\item
  \href{https://www.nytimes3xbfgragh.onion/2020/09/10/us/politics/fda-coronavirus-vaccine.html?name=styln-coronavirus-national\&region=TOP_BANNER\&block=storyline_menu_recirc\&action=click\&pgtype=Article\&impression_id=b6435aa2-f4ca-11ea-aafb-7b2c4efa9239\&variant=undefined}{F.D.A.
  Regulators' Self-Defense}
\item
  \href{https://www.nytimes3xbfgragh.onion/2020/09/09/upshot/coronavirus-surprise-test-fees.html?name=styln-coronavirus-national\&region=TOP_BANNER\&block=storyline_menu_recirc\&action=click\&pgtype=Article\&impression_id=b6435aa3-f4ca-11ea-aafb-7b2c4efa9239\&variant=undefined}{Surprise
  Test Fees}
\end{itemize}

Advertisement

\protect\hyperlink{after-top}{Continue reading the main story}

Supported by

\protect\hyperlink{after-sponsor}{Continue reading the main story}

\hypertarget{cdcs-clarification-on-coronavirus-testing-offers-more-confusion}{%
\section{C.D.C.'s `Clarification' on Coronavirus Testing Offers More
Confusion}\label{cdcs-clarification-on-coronavirus-testing-offers-more-confusion}}

After saying that those exposed to the virus need not get tested, the
agency's director clarified that ``testing may be considered'' for those
people.

\includegraphics{https://static01.graylady3jvrrxbe.onion/images/2020/08/27/us/27dc-cdc/27dc-cdc-articleLarge.jpg?quality=75\&auto=webp\&disable=upscale}

\href{https://www.nytimes3xbfgragh.onion/by/sheryl-gay-stolberg}{\includegraphics{https://static01.graylady3jvrrxbe.onion/images/2018/11/26/multimedia/author-sheryl-gay-stolberg/author-sheryl-gay-stolberg-thumbLarge.png}}

By
\href{https://www.nytimes3xbfgragh.onion/by/sheryl-gay-stolberg}{Sheryl
Gay Stolberg}

\begin{itemize}
\item
  Aug. 27, 2020
\item
  \begin{itemize}
  \item
  \item
  \item
  \item
  \item
  \end{itemize}
\end{itemize}

WASHINGTON --- The director of the Centers for Disease Control and
Prevention, seeking to clarify recommendations on coronavirus testing
that incited an uproar, said that ``testing may be considered for all
close contacts of confirmed or probable Covid-19 patients.''

But his clarification may have further confused the issue.

The statement by the director, Dr. Robert R. Redfield, was issued to
some news outlets late Wednesday, and more broadly Thursday morning,
after a
\href{https://www.nytimes3xbfgragh.onion/2020/08/26/us/politics/coronavirus-testing-trump-cdc.html}{storm
of criticism} over
\href{https://www.nytimes3xbfgragh.onion/2020/08/25/health/covid-19-testing-cdc.html}{new
C.D.C. guidelines}. Those guidelines asserted that people who had been
in close contact with an infected individual --- typically defined as
being within six feet of a person with the coronavirus for at least 15
minutes --- ``do not necessarily need a test'' if they do not have
symptoms.

Administration officials said that ``not necessarily'' needing a test
was consistent with ``may be considered'' for one. But experts said the
shift in language was leaving patients, doctors and state and local
public health officials --- who rely on the C.D.C. for guidance ---
perplexed.

```May be'?'' asked Dr. Carlos del Rio, an infectious disease expert at
Emory University. ``I want a little more than that in a recommendation.
`May be' doesn't help.''

Democrats including the governors of California and New York as well as
Speaker Nancy Pelosi have accused the C.D.C. of bowing to political
pressure from President Trump, who wants to minimize the number of cases
of infection. Administration officials say the guidelines were the
product of a vigorous debate in the White House coronavirus task force.

In his statement, Dr. Redfield sought to explain: ``Testing is meant to
drive actions and achieve specific public health objectives. Everyone
who needs a Covid-19 test can get a test. Everyone who wants a test does
not necessarily need a test; the key is to engage the needed public
health community in the decision with the appropriate follow-up
action.''

The clarification does not change the new guidelines, which remain on
the C.D.C.'s website. But it is unusual. Public health experts say
clear, consistent communications are essential to fighting an infectious
disease outbreak, and in interviews several said that statements from
the C.D.C. and Dr. Redfield had fallen far short of that goal.

\hypertarget{latest-updates-the-coronavirus-outbreak}{%
\section{\texorpdfstring{\href{https://www.nytimes3xbfgragh.onion/2020/09/11/world/covid-19-coronavirus.html?action=click\&pgtype=Article\&state=default\&region=MAIN_CONTENT_1\&context=storylines_live_updates}{Latest
Updates: The Coronavirus
Outbreak}}{Latest Updates: The Coronavirus Outbreak}}\label{latest-updates-the-coronavirus-outbreak}}

Updated 2020-09-12T07:09:04.082Z

\begin{itemize}
\tightlist
\item
  \href{https://www.nytimes3xbfgragh.onion/2020/09/11/world/covid-19-coronavirus.html?action=click\&pgtype=Article\&state=default\&region=MAIN_CONTENT_1\&context=storylines_live_updates\#link-dfb8a16}{Fauci
  cautions the virus could disrupt life in the U.S. until `maybe even
  towards the end of 2021.'}
\item
  \href{https://www.nytimes3xbfgragh.onion/2020/09/11/world/covid-19-coronavirus.html?action=click\&pgtype=Article\&state=default\&region=MAIN_CONTENT_1\&context=storylines_live_updates\#link-7104d154}{From
  Asia to Africa, China promotes its vaccine candidates to win friends.}
\item
  \href{https://www.nytimes3xbfgragh.onion/2020/09/11/world/covid-19-coronavirus.html?action=click\&pgtype=Article\&state=default\&region=MAIN_CONTENT_1\&context=storylines_live_updates\#link-393ad215}{The
  other way the virus will kill: hunger.}
\end{itemize}

\href{https://www.nytimes3xbfgragh.onion/2020/09/11/world/covid-19-coronavirus.html?action=click\&pgtype=Article\&state=default\&region=MAIN_CONTENT_1\&context=storylines_live_updates}{See
more updates}

More live coverage:
\href{https://www.nytimes3xbfgragh.onion/live/2020/09/11/business/stock-market-today-coronavirus?action=click\&pgtype=Article\&state=default\&region=MAIN_CONTENT_1\&context=storylines_live_updates}{Markets}

``What we need from the C.D.C. is clear, specific, directive guidance,''
said Dr. Leana Wen, a former health commissioner of Baltimore. ``It
shouldn't be a Rorschach blot that we're looking at, and everybody's
getting a different response by looking at the same guidance.''

Dr. Wen said she was concerned about the effect of the rule on insurance
coverage for testing. Insurers have been chafing against the mandate to
pay for all tests without requiring a co-payment from patients. Gregg
Gonsalves, an assistant professor of epidemiology at the Yale School of
Medicine, said the new guidance suggested the administration was ``not
going to support asymptomatic testing with new money or allow Medicaid
to pay for it.''

``I don't think the C.D.C.'s decision forbids states from covering tests
beyond the C.D.C.'s authorization, but it might give states cover to
save money by making cutbacks,'' said Stan Dorn, a senior fellow at
Families USA, a nonpartisan health consumer advocacy group.

One person close to the C.D.C. and the White House said the new
guidelines were put in place in part to make them comport with testing
for other infectious diseases, like Zika, and in part because of a sense
among administration scientists --- as well as doctors and insurers ---
that ``too many people were getting tested out of fear and emotion.''

The flap came as the Trump administration announced the purchase and
production of 150 million rapid tests to be distributed across the
country. White House officials said the administration had teamed with
Abbott Laboratories to produce inexpensive and easy-to-use BinaxNOW
tests.

In the new testing guidelines, posted on Monday, the C.D.C. said close
\href{https://www.nytimes3xbfgragh.onion/2020/08/25/health/covid-19-testing-cdc.html}{contacts
of Covid-19 patients ``do not necessarily need a test''} unless they are
vulnerable or their doctor or a state or local public heath official
recommended it.

On a conference call with reporters Wednesday, Admiral Brett P. Giroir,
the administration's coronavirus testing czar, said that the policy
mirrored the existing recommendation for health care and other frontline
workers, and that the task force had simply decided to extend it to the
general population.

But the guidance was met with protest from public health experts, who
said that the nation needed more testing, not less, and that it made no
sense to advise anyone who had been exposed not to get a test,
particularly because the virus is transmitted by asymptomatic people.

``I'm very confused by it,'' Dr. del Rio said, adding, ``I really do not
understand what C.D.C. is thinking, and it doesn't make sense from an
infectious disease standpoint.''

The chief scientific officer of the Association of American Medical
Colleges, Dr. Ross McKinney Jr., slammed the move as ``irresponsible,''
saying the guidelines released on Monday ``go against the best interests
of the American people and are a step backward in fighting the
pandemic.''

Mr. Trump has suggested that the nation should do less testing, arguing
that administering more tests was driving up case numbers and making the
United States look bad. But experts say the true measure of the pandemic
is not case numbers but
\href{https://www.nytimes3xbfgragh.onion/interactive/2020/us/coronavirus-testing.html}{test
positivity rates --- the percentage of tests coming back positive}.

In an interview on Wednesday, Dr. Anthony S. Fauci, a member of the task
force and the government's top infectious disease expert, said he was
concerned that the guidelines could be misinterpreted. Dr. Fauci had
signed off on an early version of the rule but was undergoing surgery
for
\href{https://www.nytimes3xbfgragh.onion/interactive/2020/08/27/upshot/fauci-media-appearances.html}{removal
of a polyp on his vocal cord} when it was completed last Thursday.

In the statement, Dr. Redfield said the agency was ``placing an emphasis
on testing individuals with symptomatic illness, individuals with a
significant exposure, vulnerable populations including nursing homes or
long-term care facilities, critical infrastructure workers, health care
workers and first responders, or those individuals who may be
asymptomatic when prioritized by medical and public health officials.''

Dr. Redfield also said that anyone --- even people who tested negative
--- exposed to someone who is or may be infected should ``strictly
adhere'' to public health guidelines, like social distancing, wearing a
mask, avoiding crowded indoor spaces and frequently washing their hands.

Advertisement

\protect\hyperlink{after-bottom}{Continue reading the main story}

\hypertarget{site-index}{%
\subsection{Site Index}\label{site-index}}

\hypertarget{site-information-navigation}{%
\subsection{Site Information
Navigation}\label{site-information-navigation}}

\begin{itemize}
\tightlist
\item
  \href{https://help.nytimes3xbfgragh.onion/hc/en-us/articles/115014792127-Copyright-notice}{©~2020~The
  New York Times Company}
\end{itemize}

\begin{itemize}
\tightlist
\item
  \href{https://www.nytco.com/}{NYTCo}
\item
  \href{https://help.nytimes3xbfgragh.onion/hc/en-us/articles/115015385887-Contact-Us}{Contact
  Us}
\item
  \href{https://www.nytco.com/careers/}{Work with us}
\item
  \href{https://nytmediakit.com/}{Advertise}
\item
  \href{http://www.tbrandstudio.com/}{T Brand Studio}
\item
  \href{https://www.nytimes3xbfgragh.onion/privacy/cookie-policy\#how-do-i-manage-trackers}{Your
  Ad Choices}
\item
  \href{https://www.nytimes3xbfgragh.onion/privacy}{Privacy}
\item
  \href{https://help.nytimes3xbfgragh.onion/hc/en-us/articles/115014893428-Terms-of-service}{Terms
  of Service}
\item
  \href{https://help.nytimes3xbfgragh.onion/hc/en-us/articles/115014893968-Terms-of-sale}{Terms
  of Sale}
\item
  \href{https://spiderbites.nytimes3xbfgragh.onion}{Site Map}
\item
  \href{https://help.nytimes3xbfgragh.onion/hc/en-us}{Help}
\item
  \href{https://www.nytimes3xbfgragh.onion/subscription?campaignId=37WXW}{Subscriptions}
\end{itemize}
