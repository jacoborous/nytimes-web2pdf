Sections

SEARCH

\protect\hyperlink{site-content}{Skip to
content}\protect\hyperlink{site-index}{Skip to site index}

\href{https://www.nytimes3xbfgragh.onion/section/business/economy}{Economy}

\href{https://myaccount.nytimes3xbfgragh.onion/auth/login?response_type=cookie\&client_id=vi}{}

\href{https://www.nytimes3xbfgragh.onion/section/todayspaper}{Today's
Paper}

\href{/section/business/economy}{Economy}\textbar{}Fed Chair Sets Stage
for Longer Periods of Lower Rates

\url{https://nyti.ms/31xk7MS}

\begin{itemize}
\item
\item
\item
\item
\item
\end{itemize}

Advertisement

\protect\hyperlink{after-top}{Continue reading the main story}

Supported by

\protect\hyperlink{after-sponsor}{Continue reading the main story}

\hypertarget{fed-chair-sets-stage-for-longer-periods-of-lower-rates}{%
\section{Fed Chair Sets Stage for Longer Periods of Lower
Rates}\label{fed-chair-sets-stage-for-longer-periods-of-lower-rates}}

Jerome H. Powell said the central bank would focus its efforts on
fostering a strong labor market while tolerating higher inflation.

\includegraphics{https://static01.graylady3jvrrxbe.onion/images/2020/08/27/business/27vid-dc-fed/27vid-dc-fed-videoSixteenByNine3000.jpg}

\href{https://www.nytimes3xbfgragh.onion/by/jeanna-smialek}{\includegraphics{https://static01.graylady3jvrrxbe.onion/images/2020/07/03/reader-center/author-jeanna-smialek/author-jeanna-smialek-thumbLarge.png}}

By \href{https://www.nytimes3xbfgragh.onion/by/jeanna-smialek}{Jeanna
Smialek}

\begin{itemize}
\item
  Aug. 27, 2020
\item
  \begin{itemize}
  \item
  \item
  \item
  \item
  \item
  \end{itemize}
\end{itemize}

Jerome H. Powell, the chair of the Federal Reserve, announced a major
shift in how the central bank guides the economy, signaling it will make
job growth pre-eminent and will not raise interest rates to guard
against coming inflation just because the
\href{https://www.nytimes3xbfgragh.onion/2020/08/27/business/economy/unemployment-claims.html}{unemployment}
rate is low.

In emphasizing the importance of a strong labor market and saying the
Fed will tolerate slightly faster price gains, Mr. Powell and his
colleagues laid the groundwork for years of low interest rates. That
could translate into long periods of cheap mortgages and business loans
that foster strong demand and a solid job market.

The changes, which Mr. Powell detailed at the Kansas City Fed's annual
Jackson Hole policy symposium, follow a year-and-a-half long review of
the central bank's monetary policy strategy. In conjunction with his
remarks, the Fed released an outline of its long-run policy plan.

``Our revised statement emphasizes that maximum employment is a
broad-based and inclusive goal,'' Mr. Powell said in the remarks. ``This
change reflects our appreciation for the benefits of a strong labor
market, particularly for many in low- and moderate-income communities.''

Market reaction to Mr. Powell's announcement was mixed. Investors had
already penciled in years of rock-bottom interest rates and analysts
will be watching for more concrete rate guidance at the Fed's upcoming
meetings.

Still, Mr. Powell's announcement could mark a defining moment in his
tenure as chair, which began in early 2018 in the midst of the longest
economic expansion on record and has run straight into the sharpest
downturn since the Great Depression. The Fed raised rates nine times
between 2015 and late 2018, with four of those increases under Mr.
Powell's watch, as it tried to guard against inflation. Price increases
instead stagnated, making the Fed's moves seem like overkill and helping
to inspire and inform the policy review.

The central bank is facing major long run challenges as price gains
prove tepid and as interest rates have
\href{https://www.nytimes3xbfgragh.onion/2020/02/29/business/economy/coronavirus-central-banks-economy.html}{slipped
lower} across advanced economies including the United States, leaving
Fed officials with less room to cut borrowing costs and coax higher
growth following recessions. Those slow-burn problems are what prompted
Mr. Powell and his colleagues to revamp their policy framework. At the
same time, the coronavirus pandemic has created a significant short-run
threat, shuttering businesses and costing millions of people their jobs.

Mr. Powell's announcement codifies a critical change in how the central
bank tries to achieve its twin goals of maximum employment and stable
inflation --- one that could inform how the Fed sets monetary policy in
the wake of the pandemic-induced recession.

The Fed had long raised rates as joblessness fell to avoid an economic
overheating that might result in breakaway inflation --- the boogeyman
that has haunted monetary policy ever since price gains hit double-digit
levels in the 1970s. But the Fed's updated framework recognizes that too
low inflation is now the problem, rather than too high.

``It seems like a pretty subtle shift to most normal human beings,''
said Janet L. Yellen, the former Fed Chair. But ``most of the Fed's
history has revolved around keeping inflation under control. This really
does reflect a decisive recognition that we're in a very different
environment.''

\href{https://www.nytimes3xbfgragh.onion/2019/11/01/business/economy/federal-reserve-inflation.html}{\emph{{[}Read
more about how the Fed's view on inflation has been shifting.{]}}}

The Fed's revised statement says that its policies will be informed by
``shortfalls'' of employment from its maximum level, rather than by
``deviations'' --- suggesting that the central bank is no longer
planning to raise rates to cool off the economy simply because
unemployment has dipped to low levels.

The central bank is also formally shifting its inflation approach,
aiming to average 2 percent inflation over time, rather than as an
absolute goal. In doing so, the Fed is trying to convince the public and
investors that it will allow prices to rise a little bit faster. If
\href{https://www.nytimes3xbfgragh.onion/2019/11/01/business/economy/federal-reserve-inflation.html}{public
inflation expectations slip}, it can lock in slow increases. Those feed
directly into the level of interest rates, and leave the central bank
with even less room to cut them during times of crisis.

``If inflation expectations fall below our 2 percent objective, interest
rates would decline in tandem,'' Mr. Powell said. ``In turn, we would
have less scope to cut interest rates to boost employment during an
economic downturn.''

Higher inflation may seem like an odd goal to anyone who buys milk or
pays rent, but excessively weak price gains can actually have damaging
effects on the economy. A circle of stagnation has played out in
countries including Japan, in which lower price gains leave less room to
cut rates, limiting policymakers' ability to stimulate the economy and
resurrect inflation.

``We are certainly mindful that higher prices for essential items, such
as food, gasoline, and shelter, add to the burdens faced by many
families, especially those struggling with lost jobs and incomes,'' Mr.
Powell said. ``However, inflation that is persistently too low can pose
serious risks to the economy.''

In a question-and-answer session after the speech, Mr. Powell said the
Fed was ``talking about inflation moving moderately.''

If the Fed can achieve slightly higher price gains, it will translate
into more room for future rate cuts --- and buying that extra headroom
is a crucial goal in 2020. Long-running economic changes, such as an
aging population with different saving habits and weaker productivity
gains, have weighed on the interest rate setting that neither stokes nor
slows the economy. That has left the central bank with less
recession-fighting wiggle room.

Still, Mr. Powell pointed out that he and his colleagues ``are not tying
ourselves to a particular mathematical formula that defines the
average.''

Some economists questioned whether the Fed will actually manage to
achieve its new inflation target.

``The Fed is announcing this policy framework in part to push up
inflation expectations,'' said Seth Carpenter, a former Fed research
official now at UBS. ``In practice, however, getting above 2 percent is
a long way off.''

Many of the changes the Fed announced Thursday formalize an approach it
has edged toward over the past decade. The Fed was patient in beginning
to lift interest rates following the recession from 2007 to 2009, even
as unemployment fell.

When it did start to raise borrowing costs in late 2015, under Ms.
Yellen, it did so slowly.

Under Mr. Powell's leadership, the Fed has increasingly emphasized the
benefits of that strong labor market, which
\href{https://www.nytimes3xbfgragh.onion/2020/02/07/business/black-unemployment-wages.html}{pulled
long-sidelined workers} into jobs and helped to foster strong wage
growth for those who earn the least.

Ms. Yellen, who has long argued that a strong labor market could boost
marginalized groups, said the Fed's shift is ``great'' and ``a
recognition that tight labor markets are beneficial.''

The long-run document promises that the central bank will continue to
hold reviews, roughly every five years, and will
\href{https://www.nytimes3xbfgragh.onion/2020/08/26/business/economy/fed-meeting-powell.html}{continue
to consult the public} as it has done over the past year through its
``Fed Listens'' events.

``Public faith in large institutions around the world is under
pressure,'' Mr. Powell said in a question-and-answer session following
his speech. ``Institutions like the Fed have to aggressively seek
transparency and accountability to preserve our democratic legitimacy.''

The Fed also explicitly noted in its statement that financial stability
ranks among its key goals. In recent decades, expansions have ended when
asset price bubbles --- like the mid-2000s housing boom --- got out of
control, rather than at the hands of too-high inflation.

``Sustainably achieving maximum employment and price stability depends
on a stable financial system,'' the Fed said in its statement.
``Therefore, the committee's policy decisions reflect its longer-run
goals, its medium-term outlook, and its assessments of the balance of
risks, including risks to the financial system that could impede the
attainment of the committee's goals.''

Mr. Powell's remarks, and the Fed's shift, are set against an unhappy
backdrop that has highlighted the central bank's limits.

Fed officials have taken action to support the economy as the
pandemic-induced downturn drags on --- cutting interest rates to
near-zero, buying government-backed bonds in vast sums, and rolling out
emergency lending programs. Still, more than one
\href{https://slack-redir.net/link?url=https\%3A\%2F\%2Fwww.nytimes3xbfgragh.onion\%2Flive\%2F2020\%2F08\%2F27\%2Fbusiness\%2Fstock-market-today-coronavirus}{million
people filed} initial state jobless claims last week, data released
Thursday morning showed.

The Fed has repeatedly emphasized that a strong job market and economy
is an imperative goal, but that Congress will need to help achieve it.

``It is hard to overstate the benefits of sustaining a strong labor
market, a key national goal that will require a range of policies in
addition to supportive monetary policy,'' Mr. Powell said.

He added that there was a strong economy under the surface of the
ongoing weakness.

``We will get through this period, maybe with some starts and stops,''
he said. Still, ``we're looking at a long tail'' as people who work in
industries heavily impacted, like travel and service, struggle to find
new work in a process that could take years.

``We need to support them,'' Mr. Powell said.

Advertisement

\protect\hyperlink{after-bottom}{Continue reading the main story}

\hypertarget{site-index}{%
\subsection{Site Index}\label{site-index}}

\hypertarget{site-information-navigation}{%
\subsection{Site Information
Navigation}\label{site-information-navigation}}

\begin{itemize}
\tightlist
\item
  \href{https://help.nytimes3xbfgragh.onion/hc/en-us/articles/115014792127-Copyright-notice}{©~2020~The
  New York Times Company}
\end{itemize}

\begin{itemize}
\tightlist
\item
  \href{https://www.nytco.com/}{NYTCo}
\item
  \href{https://help.nytimes3xbfgragh.onion/hc/en-us/articles/115015385887-Contact-Us}{Contact
  Us}
\item
  \href{https://www.nytco.com/careers/}{Work with us}
\item
  \href{https://nytmediakit.com/}{Advertise}
\item
  \href{http://www.tbrandstudio.com/}{T Brand Studio}
\item
  \href{https://www.nytimes3xbfgragh.onion/privacy/cookie-policy\#how-do-i-manage-trackers}{Your
  Ad Choices}
\item
  \href{https://www.nytimes3xbfgragh.onion/privacy}{Privacy}
\item
  \href{https://help.nytimes3xbfgragh.onion/hc/en-us/articles/115014893428-Terms-of-service}{Terms
  of Service}
\item
  \href{https://help.nytimes3xbfgragh.onion/hc/en-us/articles/115014893968-Terms-of-sale}{Terms
  of Sale}
\item
  \href{https://spiderbites.nytimes3xbfgragh.onion}{Site Map}
\item
  \href{https://help.nytimes3xbfgragh.onion/hc/en-us}{Help}
\item
  \href{https://www.nytimes3xbfgragh.onion/subscription?campaignId=37WXW}{Subscriptions}
\end{itemize}
