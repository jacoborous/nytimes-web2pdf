Sections

SEARCH

\protect\hyperlink{site-content}{Skip to
content}\protect\hyperlink{site-index}{Skip to site index}

\href{https://myaccount.nytimes3xbfgragh.onion/auth/login?response_type=cookie\&client_id=vi}{}

\href{https://www.nytimes3xbfgragh.onion/section/todayspaper}{Today's
Paper}

A Portrait of a Man Named Jonathan

\url{https://nyti.ms/2QdWvqb}

\begin{itemize}
\item
\item
\item
\item
\item
\end{itemize}

Advertisement

\protect\hyperlink{after-top}{Continue reading the main story}

Supported by

\protect\hyperlink{after-sponsor}{Continue reading the main story}

The Artists

\hypertarget{a-portrait-of-a-man-named-jonathan}{%
\section{A Portrait of a Man Named
Jonathan}\label{a-portrait-of-a-man-named-jonathan}}

Erica Deeman's thoughtful pictures of members of the African diaspora
highlight her subjects' humanity.

\includegraphics{https://static01.graylady3jvrrxbe.onion/images/2020/07/13/t-magazine/13tmag-deeman/13tmag-deeman-articleLarge.jpg?quality=75\&auto=webp\&disable=upscale}

Aug. 21, 2020

\begin{itemize}
\item
\item
\item
\item
\item
\end{itemize}

\emph{In each installment of The Artists, T highlights a recent or
little-shown work by a Black artist, along with a few words from that
artist putting the work into context. This week, we're looking at a
piece by Erica Deeman, who focuses on portrait photography, exploring
questions of race and identity. Her new show, ``Familiar Stranger,''
opens next week at}
\href{http://www.anthonymeierfinearts.com/exhibitions}{\emph{Anthony
Meier Fine Arts.}}

\textbf{Name:} Erica Deeman

\textbf{Age:} 43

\textbf{Based in:} San Francisco, Calif.

\textbf{Originally from:} Nottingham, England

\textbf{When and where did you make this work?} I made this photograph
with Jonathan in my living room in San Francisco in 2016.

\textbf{Can you describe what is going on in the work?} Jonathan ---~a
friend of my roommate at the time --- was visiting the United States
from London. I'd been working on my
``\href{http://www.ericadeeman.com/new-gallery}{Brown}'' portraiture
series for a while, which focuses on men from the African diaspora and
is centered on what it means to see and be seen as a Black **** man in
the United States and beyond. I asked if he would consider sitting for
me, and thankfully he agreed. I set up the background and lights in my
living room, and we began the collaboration. For me, coming to the U.S.
has had a profound impact on my sense of self and belonging. The topics
of race, gender **** and representation drove our conversation ---
sharing the familiarity of our homeland experiences, moments of joy,
frustration and reflection. I wondered how he felt in the limited time
he had spent here. This image was made in a moment of pause and deep
thought.

\textbf{What inspired you to make this work?} So many thoughts and ideas
were pulsing through my mind at the time. First, a tenderness in seeing
Black **** men, set against the historical archive of portraiture and
scientific photography. Seeing the beauty in Blackness, I am constantly
reminded of
\href{https://www.nytimes3xbfgragh.onion/2020/07/21/t-magazine/carrie-mae-weems-moma-garden.html}{Carrie
Mae Weems}'s piece, ****
\href{https://artmuseum.mtholyoke.edu/object/i-looked-and-looked-and-failed-see-what-so-terrified-you-louisiana-project-series}{``I
Looked and Looked and Failed to See What so Terrified You''} (2003). **
There was also a significant community-building aspect in making the
photographs. I was still finding my feet, so to speak, so making this
work allowed me to fold into the Black community in the Bay Area,
feeling a deeper connection to my new home.

\textbf{What's the work of art in any medium that changed your life?} I
admire artists who can use any medium to share their ideas. For me,
\href{https://www.nytimes3xbfgragh.onion/2020/07/24/t-magazine/howardena-pindell.html}{Howardena
Pindell} is one of those artists who has the mastery of transforming her
audience. Of her pieces, I choose the video work
\href{https://www.youtube.com/watch?v=b5tJNXiB9Ko}{``Free, White, and
21''} (1980). I first saw it at the Pérez **** Art Museum Miami a few
days after Art Basel one year. The museum was very quiet, and I had the
opportunity to spend uninterrupted time with it.

Advertisement

\protect\hyperlink{after-bottom}{Continue reading the main story}

\hypertarget{site-index}{%
\subsection{Site Index}\label{site-index}}

\hypertarget{site-information-navigation}{%
\subsection{Site Information
Navigation}\label{site-information-navigation}}

\begin{itemize}
\tightlist
\item
  \href{https://help.nytimes3xbfgragh.onion/hc/en-us/articles/115014792127-Copyright-notice}{©~2020~The
  New York Times Company}
\end{itemize}

\begin{itemize}
\tightlist
\item
  \href{https://www.nytco.com/}{NYTCo}
\item
  \href{https://help.nytimes3xbfgragh.onion/hc/en-us/articles/115015385887-Contact-Us}{Contact
  Us}
\item
  \href{https://www.nytco.com/careers/}{Work with us}
\item
  \href{https://nytmediakit.com/}{Advertise}
\item
  \href{http://www.tbrandstudio.com/}{T Brand Studio}
\item
  \href{https://www.nytimes3xbfgragh.onion/privacy/cookie-policy\#how-do-i-manage-trackers}{Your
  Ad Choices}
\item
  \href{https://www.nytimes3xbfgragh.onion/privacy}{Privacy}
\item
  \href{https://help.nytimes3xbfgragh.onion/hc/en-us/articles/115014893428-Terms-of-service}{Terms
  of Service}
\item
  \href{https://help.nytimes3xbfgragh.onion/hc/en-us/articles/115014893968-Terms-of-sale}{Terms
  of Sale}
\item
  \href{https://spiderbites.nytimes3xbfgragh.onion}{Site Map}
\item
  \href{https://help.nytimes3xbfgragh.onion/hc/en-us}{Help}
\item
  \href{https://www.nytimes3xbfgragh.onion/subscription?campaignId=37WXW}{Subscriptions}
\end{itemize}
