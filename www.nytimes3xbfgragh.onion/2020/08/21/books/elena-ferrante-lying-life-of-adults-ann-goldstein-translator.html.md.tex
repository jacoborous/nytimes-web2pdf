\href{/section/books}{Books}\textbar{}Reading Elena Ferrante in English?
You're Also Reading Ann Goldstein

\begin{itemize}
\item
\item
\item
\item
\item
\item
\end{itemize}

\includegraphics{https://static01.graylady3jvrrxbe.onion/images/2020/08/24/books/21Goldstein1/21Goldstein1-articleLarge-v2.jpg?quality=75\&auto=webp\&disable=upscale}

Sections

\protect\hyperlink{site-content}{Skip to
content}\protect\hyperlink{site-index}{Skip to site index}

\hypertarget{reading-elena-ferrante-in-english-youre-also-reading-ann-goldstein}{%
\section{Reading Elena Ferrante in English? You're Also Reading Ann
Goldstein}\label{reading-elena-ferrante-in-english-youre-also-reading-ann-goldstein}}

The self-effacing translator worked with the ``My Brilliant Friend''
author again for her latest book, ``The Lying Life of Adults.''

Ann Goldstein, the translator who worked with Elena Ferrante again for
her latest book, ``The Lying Life of Adults.''Credit...September Dawn
Bottoms/The New York Times

Supported by

\protect\hyperlink{after-sponsor}{Continue reading the main story}

\href{https://nytimes3xbfgragh.onion/by/joumana-khatib}{\includegraphics{https://static01.graylady3jvrrxbe.onion/images/2018/09/13/multimedia/author-joumana-khatib/author-joumana-khatib-thumbLarge.png}}

By \href{https://nytimes3xbfgragh.onion/by/joumana-khatib}{Joumana
Khatib}

\begin{itemize}
\item
  Aug. 21, 2020
\item
  \begin{itemize}
  \item
  \item
  \item
  \item
  \item
  \item
  \end{itemize}
\end{itemize}

Millions of readers in thrall to Elena Ferrante, the secretive and
wildly popular Italian novelist, must accept certain conditions.

They won't be meeting her, virtually or in-person, at any sort of book
signing or literary festival. Her stories will be rooted in Italy, and
often focus on women trying to tame the chaos of their lives
\href{https://www.nytimes3xbfgragh.onion/2019/05/17/opinion/elena-ferrante-on-women-power.html}{through
writing}.

And if they are reading Ferrante's books in English, they are absorbing,
whether they realize it or not, the nimble translation work of Ann
Goldstein.

Goldstein has never met Ferrante and communicates with her through her
publisher, but she has become one of the best known and most celebrated
literary translators in the world as a result of her work on
``\href{https://www.nytimes3xbfgragh.onion/2012/12/23/books/review/my-brilliant-friend-by-elena-ferrante-and-more.html}{My
Brilliant Friend}'' and the rest of the author's Neapolitan quartet. In
many ways, their relationship is reciprocal: While Italian readers have
known Ferrante for years, it was the translation of her books into
English and other languages that catapulted her to international fame.

Their collaboration will come into view again next month when Ferrante's
latest novel, ``The Lying Life of Adults,'' is released across the world
on Sept. 1. It was previously
\href{https://www.nytimes3xbfgragh.onion/2019/10/28/books/elena-ferrante-lying-life-adults-brilliant-friend.html}{slated
for June 9}, but the publishers delayed it because of the coronavirus
pandemic. (Netflix is planning to
\href{https://media.netflix.com/en/press-releases/the-lying-life-of-adults-elena-ferrantes-latest-novel-will-be-developed-into-a-netflix-original-series-together-with-fandango}{adapt
the novel} into an original series.)

\includegraphics{https://static01.graylady3jvrrxbe.onion/images/2020/08/24/books/19Goldstein3/merlin_175893573_92f6c67a-931c-4ea7-9f8c-5429d799506c-articleLarge.jpg?quality=75\&auto=webp\&disable=upscale}

Like several of Ferrante's other books, ``The Lying Life of Adults'' is
set in Naples. It follows the unraveling of an adolescent, Giovanna,
after she overhears her father say that she is becoming ugly like her
fearsome aunt, Vittoria. Giovanna's quest to meet her aunt leads her
through a grittier part of the city, revealing unsavory family truths
along the way.

``It was a surprising book,'' Goldstein said in a Zoom interview from
her downtown Manhattan home. ``It was such a different view of Naples,
from such a different point of view both in terms of class and social
life, and of having a teenage narrator.''

She added: ``I just hope that I got it right.''

That humility was a hallmark of her approach as the head of The New
Yorker's copy desk. Goldstein worked at the magazine for over 40 years,
steadfastly defending its diereses, ``which'' and ``that'' rules and
other grammatical diktats **** that ``writers get cranky about,'' she
said.

But the most essential part of the job was to make a writer sound as
much like him or herself as possible, she said. ``The writers I edited
were the great writers. I was really lucky.''

After Janet Malcolm's husband and editor,
\href{https://www.nytimes3xbfgragh.onion/2004/09/29/nyregion/obituaries/gardner-botsford-87-dies-editor-at-the-new-yorker.html}{Gardner
Botsford}, died in 2004, Goldstein took over as her editor. ``I could
not have wished for a better successor,'' Malcolm wrote in an email.
``Ann's most outstanding trait --- apart from her beautiful work --- is
her modesty. She is known for her reticence and self-effacement.''

In the mid-1980s, Goldstein and a few New Yorker colleagues formed an
evening class to learn Italian. (``Enlightened employers used to pay for
classes,'' she said.) Goldstein had been enchanted by Dante in college
and wanted to read him in his original language. The group spent a year
each on ``Inferno,'' ``Purgatory'' and ``Paradise.''

``Normally people read `Inferno' and that's all, but it's worth seeing
it through to `Paradise,''' Goldstein said. ``You deserve it.''

She began translating a few years later, starting with Aldo Buzzi's
short story
``\href{https://archives.newyorker.com/newyorker/1992-09-14/flipbook/028/}{Chekhov
in Sondrio},'' and moving on to Pier Paolo Pasolini's ``Petrolio,'' ``a
totally crazy book'' with complicated Italian that, according to her,
hardly anyone has read in either language. Before retiring from The New
Yorker in 2017, Goldstein did all her translations at night or over
weekends and vacations.

``I'm willing to try anything,'' she said of the work she's drawn to.
``I don't think it's necessary to have an affinity for the writer, but
with Ferrante, I do.''

Image

``It was a surprising book,'' Ann Goldstein said of ``The Lying Life of
Adults.'' ``It was such a different view of Naples, from such a
different point of view both in terms of class and social life, and of
having a teenage narrator.''Credit...September Dawn Bottoms/The New York
Times

Europa Editions, Ferrante's U.S. publisher, declined to make the author
available for an interview. ``Elena Ferrante'' is a pseudonym, and while
there has been
\href{https://www.nytimes3xbfgragh.onion/2016/10/03/books/elena-ferrante-anita-raja-domenico-starnone.html}{speculation
about her identity}, she has never revealed herself publicly. Ferrante's
Italian publisher, Edizioni E/O, mediates her correspondence with
Goldstein.

Their working relationship goes back to 2004, before ``My Brilliant
Friend,'' when Goldstein translated
``\href{https://www.nytimes3xbfgragh.onion/2005/08/25/books/a-scorned-wifes-bumpy-road-of-raging-selfawareness.html}{The
Days of Abandonment},'' Ferrante's first book with Europa. Goldstein,
one of a handful of people invited to submit a sample translation, got
the job over --- among others --- Europa's editor in chief, Michael
Reynolds.

Goldstein describes herself as a highly literal translator, an approach
that serves Ferrante's idiosyncratic prose well, Reynolds said. ``It
takes a great deal of humility and a great deal of courage to represent
so closely what an author wrote in the original language.''

One of the reasons for Ferrante's success in English ``is the degree to
which the reader feels involved and engaged,'' he added. ``Ann's style
of translation helps that.''

Ferrante is known for her long, emotive sentences, and in Goldstein's
translation of ``The Lying Life of Adults,'' that comes through even in
the first paragraph: ``Everything --- the spaces of Naples, the blue
light of a frigid February, those words --- remained fixed. But I
slipped away, and am still slipping away, within these lines that are
intended to give me a story, while in fact I am nothing, nothing of my
own, nothing that has really begun or really been brought to
completion.''

Mary Norris, a former longtime copy editor at The New Yorker, worked
with Goldstein for decades. ``The virtues of a copy editor served her
well as a translator,'' Norris said. ``She disappears, in a sense. In
the way that a copy editor is a sieve for the writer and the language,
the same is true of a translator.''

But Norris came to see later that ``translating is not just like copy
editing,'' she said. ``It also involves being a writer. Ann gives that
part of herself to it.''

While she is most closely associated with Ferrante, Goldstein has
translated books by Elsa Morante and Giacomo Leopardi, as well as Jhumpa
Lahiri's 2017 collection of essays, ``In Other Words,'' which the author
wrote in Italian. Goldstein also edited and contributed to the 2015
translation of
``\href{https://www.nytimes3xbfgragh.onion/2015/11/29/books/review/the-complete-works-of-primo-levi.html}{The
Complete Works of Primo Levi},'' an enormous project involving
translations by several writers, including Jenny McPhee.

``She'd always say, `I'm not a writer, I'm not creative,' but there's a
certain creativity you really need, and she has it even if she doesn't
own it,'' McPhee said.

Of the Ferrante novels, McPhee added: ``Ann is all over those books
\ldots{} If somebody else had done it, it may have never taken off.''

The relationship between Goldstein and Ferrante resembles the one
between Lenù and Lila, the main characters of the Neapolitan quartet.
``Those are books about who's doing the narrating and the dichotomous
relationship between two women --- who's out front and who's behind,
who's left and who's stayed, who's the brilliant friend and who isn't
--- and I think that has repeated itself in the relationship between
author and translator,'' Reynolds said.

For Goldstein, who has remained in New York City through the pandemic,
it has been a strange time to be promoting a book. She is keeping busy
with more translation work and still meeting with her fellow Italian
students, after all these years, over Zoom.

``The idea was to read Dante,'' she said, ``and here we are, reading
Dante again.''

\emph{\textbf{Correction:}} ** \emph{\textbf{Aug. 21, 2020}}\\
\emph{Because of an editing error, an earlier version of this article
misstated the previous release date for ``The Lying Life of Adults'' in
English. It was June 9, not June 1.}

\emph{Follow New York Times Books on}
\href{https://www.facebookcorewwwi.onion/nytbooks/}{\emph{Facebook}}\emph{,}
\href{https://twitter.com/nytimesbooks}{\emph{Twitter}} \emph{and}
\href{https://www.instagram.com/nytbooks/}{\emph{Instagram}}\emph{, sign
up for}
\href{https://www.nytimes3xbfgragh.onion/newsletters/books-review}{\emph{our
newsletter}} \emph{or}
\href{https://www.nytimes3xbfgragh.onion/interactive/2017/books/books-calendar.html}{\emph{our
literary calendar}}\emph{. And listen to us on the}
\href{https://www.nytimes3xbfgragh.onion/column/book-review-podcast}{\emph{Book
Review podcast}}\emph{.}

Advertisement

\protect\hyperlink{after-bottom}{Continue reading the main story}

\hypertarget{site-index}{%
\subsection{Site Index}\label{site-index}}

\hypertarget{site-information-navigation}{%
\subsection{Site Information
Navigation}\label{site-information-navigation}}

\begin{itemize}
\tightlist
\item
  \href{https://help.nytimes3xbfgragh.onion/hc/en-us/articles/115014792127-Copyright-notice}{©~2020~The
  New York Times Company}
\end{itemize}

\begin{itemize}
\tightlist
\item
  \href{https://www.nytco.com/}{NYTCo}
\item
  \href{https://help.nytimes3xbfgragh.onion/hc/en-us/articles/115015385887-Contact-Us}{Contact
  Us}
\item
  \href{https://www.nytco.com/careers/}{Work with us}
\item
  \href{https://nytmediakit.com/}{Advertise}
\item
  \href{http://www.tbrandstudio.com/}{T Brand Studio}
\item
  \href{https://www.nytimes3xbfgragh.onion/privacy/cookie-policy\#how-do-i-manage-trackers}{Your
  Ad Choices}
\item
  \href{https://www.nytimes3xbfgragh.onion/privacy}{Privacy}
\item
  \href{https://help.nytimes3xbfgragh.onion/hc/en-us/articles/115014893428-Terms-of-service}{Terms
  of Service}
\item
  \href{https://help.nytimes3xbfgragh.onion/hc/en-us/articles/115014893968-Terms-of-sale}{Terms
  of Sale}
\item
  \href{https://spiderbites.nytimes3xbfgragh.onion}{Site Map}
\item
  \href{https://help.nytimes3xbfgragh.onion/hc/en-us}{Help}
\item
  \href{https://www.nytimes3xbfgragh.onion/subscription?campaignId=37WXW}{Subscriptions}
\end{itemize}
