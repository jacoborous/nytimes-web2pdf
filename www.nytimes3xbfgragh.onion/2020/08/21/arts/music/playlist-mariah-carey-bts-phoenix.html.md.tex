Sections

SEARCH

\protect\hyperlink{site-content}{Skip to
content}\protect\hyperlink{site-index}{Skip to site index}

\href{https://www.nytimes3xbfgragh.onion/section/arts/music}{Music}

\href{https://myaccount.nytimes3xbfgragh.onion/auth/login?response_type=cookie\&client_id=vi}{}

\href{https://www.nytimes3xbfgragh.onion/section/todayspaper}{Today's
Paper}

\href{/section/arts/music}{Music}\textbar{}Mariah Carey Calls for
Action, and 12 More New Songs

\url{https://nyti.ms/34njVSy}

\begin{itemize}
\item
\item
\item
\item
\item
\end{itemize}

\href{https://www.nytimes3xbfgragh.onion/spotlight/at-home?action=click\&pgtype=Article\&state=default\&region=TOP_BANNER\&context=at_home_menu}{At
Home}

\begin{itemize}
\tightlist
\item
  \href{https://www.nytimes3xbfgragh.onion/2020/09/07/travel/route-66.html?action=click\&pgtype=Article\&state=default\&region=TOP_BANNER\&context=at_home_menu}{Cruise
  Along: Route 66}
\item
  \href{https://www.nytimes3xbfgragh.onion/2020/09/04/dining/sheet-pan-chicken.html?action=click\&pgtype=Article\&state=default\&region=TOP_BANNER\&context=at_home_menu}{Roast:
  Chicken With Plums}
\item
  \href{https://www.nytimes3xbfgragh.onion/2020/09/04/arts/television/dark-shadows-stream.html?action=click\&pgtype=Article\&state=default\&region=TOP_BANNER\&context=at_home_menu}{Watch:
  Dark Shadows}
\item
  \href{https://www.nytimes3xbfgragh.onion/interactive/2020/at-home/even-more-reporters-editors-diaries-lists-recommendations.html?action=click\&pgtype=Article\&state=default\&region=TOP_BANNER\&context=at_home_menu}{Explore:
  Reporters' Google Docs}
\end{itemize}

Advertisement

\protect\hyperlink{after-top}{Continue reading the main story}

Supported by

\protect\hyperlink{after-sponsor}{Continue reading the main story}

The Playlist

\hypertarget{mariah-carey-calls-for-action-and-12-more-new-songs}{%
\section{Mariah Carey Calls for Action, and 12 More New
Songs}\label{mariah-carey-calls-for-action-and-12-more-new-songs}}

Hear tracks by Phoenix, Angel Olsen, Bebel Gilberto and others.

\includegraphics{https://static01.graylady3jvrrxbe.onion/images/2020/08/24/arts/21playlist/merlin_167224821_397e02a5-a803-49fb-9333-6c35f308b1bf-articleLarge.jpg?quality=75\&auto=webp\&disable=upscale}

By \href{https://www.nytimes3xbfgragh.onion/by/jon-pareles}{Jon
Pareles},
\href{https://www.nytimes3xbfgragh.onion/by/jon-caramanica}{Jon
Caramanica},
\href{https://www.nytimes3xbfgragh.onion/by/giovanni-russonello}{Giovanni
Russonello} and Lindsay Zoladz

\begin{itemize}
\item
  Aug. 21, 2020
\item
  \begin{itemize}
  \item
  \item
  \item
  \item
  \item
  \end{itemize}
\end{itemize}

\emph{Every Friday, pop critics for The New York Times weigh in on the
week's most notable new songs and videos. Just want the music?}
\href{https://open.spotify.com/playlist/01oygvLrsZCaxLSGUPWls3?si=rMOznFH3Qa2V90_zlu1I_Q}{\emph{Listen
to the Playlist on Spotify here}} \emph{(or find our profile: nytimes).
Like what you hear? Let us know at}
\href{mailto:theplaylist@NYTimes.com}{\emph{theplaylist@NYTimes.com}}
\emph{and}
\href{https://www.nytimes3xbfgragh.onion/newsletters/louder?module=inline}{\emph{sign
up for our Louder newsletter}}\emph{, a once-a-week blast of our pop
music coverage.}

\hypertarget{mariah-carey-with-ms-lauryn-hill-save-the-day}{%
\subsection{Mariah Carey with Ms. Lauryn Hill, `Save the
Day'}\label{mariah-carey-with-ms-lauryn-hill-save-the-day}}

This fall, Mariah Carey will throw the vault wide open: Not only is she
releasing her long-awaited memoir ``The Meaning of Mariah Carey'' on
Sept. 29 (mark your calendar, Eminem!) but she'll also be putting out a
career-spanning rarities collection on Oct. 2, featuring a trove of
previously unreleased material. The first taste is ``Save the Day,'' a
Jermaine Dupri-produced track that effectively samples Ms. Lauryn Hill's
iconic vocal from the Fugees' 1996 cover of ``Killing Me Softly.'' The
song's message of sweeping uplift certainly fits the current moment
(``If he won't and she won't, and they won't, then we won't, we won't
ever learn to save the day'') but the thumping beat and breathy vocals
are a throwback to mid-90s Mariah. LINDSAY ZOLADZ

\hypertarget{phoenix-identical}{%
\subsection{Phoenix, `Identical'}\label{phoenix-identical}}

Sofia Coppola's movies are known for their well-curated needle drops:
\href{https://www.youtube.com/watch?v=OZByH4Wai_I}{Bow Wow Wow in
Versailles!}\href{https://www.youtube.com/watch?v=dPly3e12ca8}{The Jesus
and Mary Chain in Tokyo!} One boon of being married to the Phoenix
frontman Thomas Mars, though, is that it's probably pretty easy to
commission an '80s-inspired pop song custom-made for the mood of your
latest film. ``Identical,'' from the soundtrack of Coppola's forthcoming
father-daughter movie ``On the Rocks'' (starring Bill Murray and Rashida
Jones) has a sleek surface and an appealing undercurrent of nostalgia,
driven by sunset-hued synths and Mars' sweet falsetto. You can almost
picture the montage. ZOLADZ

\hypertarget{chloe-moriondo-i-want-to-be-with-you}{%
\subsection{Chloe Moriondo, `I Want to Be With
You'}\label{chloe-moriondo-i-want-to-be-with-you}}

Acutely observed bedroom pop about losing yourself on the path to
someone else, served with a side of arena-emo triumph. JON CARAMANICA

\hypertarget{bts-dynamite}{%
\subsection{BTS, `Dynamite'}\label{bts-dynamite}}

BTS's first single wholly in English is a sprightly bit of lite
disco-funk somewhere in between Jamiroquai and Charlie Puth. Less
musically adventurous than the songs that made the group a worldwide
phenomenon, it relies on brightness, exuberance and relentless good
cheer. Sadly, though, ``Dynamite'' includes no real rapping --- always
one of the group's strong points, and a weapon that makes it among the
most versatile of pop outfits. CARAMANICA

\hypertarget{nubya-garcia-stand-with-each-other}{%
\subsection{Nubya Garcia, `Stand With Each
Other'}\label{nubya-garcia-stand-with-each-other}}

The decorated young
\href{https://www.nytimes3xbfgragh.onion/2020/08/17/arts/music/nubya-garcia-source.html}{British
tenor saxophonist Nubya Garcia} recorded her new album, ``Source,'' with
thoughts swirling about her own identity and family history. Of course,
that's all inseparable from the work of communal engagement. Speaking to
DownBeat, she wondered: ``What's the source of humanity's power when the
world's falling apart?'' Of the album's nine tracks --- recorded at
sessions in both Colombia and the U.K. --- perhaps none addresses that
question more directly than ``Stand With Each Other,'' a lapping,
mesmeric tune with inflections of reggae and cumbia. Musing and
even-toned, doused in reverb, Ms. Garcia's saxophone does a patient
dance with the harmonizing voices of three women and a loose clatter of
percussion. There's no big climax to speak of; what you hear is the
sound of musicians in deep communication, listening and feeling as they
go. GIOVANNI RUSSONELLO

\hypertarget{bebel-gilberto-featuring-martnuxe1lia-na-cara}{%
\subsection{Bebel Gilberto featuring Mart'nália, `Na
Cara'}\label{bebel-gilberto-featuring-martnuxe1lia-na-cara}}

Bebel Gilberto's new album, ``Agora'' (``Now''), produced by Thomas
Bartlett, captures the untranslatable Brazilian mood of saudade --- a
knowing, nostalgia-tinged melancholy --- by placing her whispery voice
within a subtle electronic mix of instruments, loops and what sound like
blurred old samples. Gilberto shares ``Na Cara'' (``To My Face''), a
demand for truthfulness, with the raspy samba singer Mart'nália; around
its slinky bass line are fleeting glimmers of vibraphone, piano and
string orchestra, appearing and vanishing like fading memories. JON
PARELES

\hypertarget{tomberlin-wasted}{%
\subsection{Tomberlin, `Wasted'}\label{tomberlin-wasted}}

``At Weddings,'' the 2018 debut album from the indie-folk artist Sarah
Beth Tomberlin, evoked reflective solitude. On ``Wasted,'' the first
single from her upcoming EP ``Projections,'' she's let a few well-known
collaborators into the mix and captured an even more complex mood.
Coproduction from the D.I.Y. maestro Alex G layers Tomberlin's guitar
atop a skittering, off-kilter beat that sounds like a children's
clapping game. And the music video, directed by the actress Busy
Phillips and starring one of her daughters, is poignant and compelling:
Two children idly explore their neighborhood while Tomberlin, in a
striking neon green dress, haunts the background of the frame like a
ghost of her own carefree past. ZOLADZ

\hypertarget{hardy-boyfriend}{%
\subsection{Hardy, `Boyfriend'}\label{hardy-boyfriend}}

The rising country star Hardy takes a detour toward nice-guy balladry
with ``Boyfriend,'' a sharply written song about taking things to the
next level. ``I'm so sick of driving clear across town every night from
my place to yours/I don't wanna be your boyfriend anymore'': This is how
sensitive country bros put a ring on it. CARAMANICA

\hypertarget{angel-olsen-waving-smiling}{%
\subsection{Angel Olsen, `Waving,
Smiling'}\label{angel-olsen-waving-smiling}}

``Waving, Smiling'' couldn't be more sparse. It's just guitar and voice,
Angel Olsen picking slow, tentative arpeggios like the skeleton of a
soul ballad as she sing about the aftermath of a heartbreak. She moves
gradually from accusation to sorrow to acceptance. ``I've laid out all
those tears/I've made my bed, made up of all my fears,'' she sings,
letting her voice tremble and then rise toward a tentative peace.
PARELES

\hypertarget{father-john-misty-to-s}{%
\subsection{Father John Misty, `To S.'}\label{father-john-misty-to-s}}

Father John Misty's music wriggles in and out of scare quotes, but ``To
S.,'' one of two new songs he released this week, spends most of its run
time unfurling gently outside of them. Yes, there are a few wry
Misty-isms here (``Guess what? I love you/Someone's gotta clean up the
mess'') but mostly the song finds Josh Tillman reveling in piano-driven
Laurel Canyon melancholy. ``What about life on the ground makes you feel
so strange?'' he croons, as backing strings render the song
appropriately weightless. ZOLADZ

\hypertarget{alex-the-astronaut-i-like-to-dance}{%
\subsection{Alex the Astronaut, `I Like to
Dance'}\label{alex-the-astronaut-i-like-to-dance}}

Alex the Astronaut --- the Australian songwriter Alex Lynn --- sums up a
trapped, longtime abusive relationship in her character study ``I Like
to Dance,'' from her debut album, ``The Theory of Absolutely Nothing.''
A guitar strumming four chords, a piano and a violin accompany the
first-person narration: ``He didn't want to be like this/I just wish
he'd stop,'' she sings, as the harrowing details, the children and
financial dependency, build up: ``I just wish he'd stop hitting me.''
PARELES

\hypertarget{jazz-is-dead-featuring-marcos-valle-queira-bem}{%
\subsection{Jazz Is Dead featuring Marcos Valle, `Queira
Bem'}\label{jazz-is-dead-featuring-marcos-valle-queira-bem}}

The producers and musical polymaths Adrian Younge and Ali Shaheed
Muhammad have been treating their new collaborative project, Jazz Is
Dead, as a chance to go past crate-digging, connecting directly with
some of their biggest influences from elder generations. In June they
released an album in collaboration with one of the most copiously
sampled figures in music, the vibraphonist and vocalist Roy Ayers. Now
they're already back with another full-length, this time joined by the
76-year-old Brazilian musician Marcos Valle, whose
\href{https://www.waxpoetics.com/blog/features/articles/rockin-eternally-leon-ware-marcos-valle-story/}{stamp}
has also landed on countless hip-hop records through the art of
appropriation. On ``Queira Bem,'' Valle's wispy, vulnerable vocals drift
over a placid backing of electric piano, analog synth, flutes, crinkly
guitar and a stubbornly coolheaded beat carried by the bass and drums.
RUSSONELLO

\hypertarget{moor-mother-and-billy-woods-furies}{%
\subsection{Moor Mother and Billy Woods,
`Furies'}\label{moor-mother-and-billy-woods-furies}}

The beat, a loop of hand drums, a distant choir and a flute line, is
resolute and unhurried. The lyrics, from a pitch-shifting Moor Mother
and a calm, deliberate Billy Woods, are splintered but combative: ``They
don't want me to shine 'cause I remind them of the fight,'' Moor Mother
intones. PARELES

Advertisement

\protect\hyperlink{after-bottom}{Continue reading the main story}

\hypertarget{site-index}{%
\subsection{Site Index}\label{site-index}}

\hypertarget{site-information-navigation}{%
\subsection{Site Information
Navigation}\label{site-information-navigation}}

\begin{itemize}
\tightlist
\item
  \href{https://help.nytimes3xbfgragh.onion/hc/en-us/articles/115014792127-Copyright-notice}{©~2020~The
  New York Times Company}
\end{itemize}

\begin{itemize}
\tightlist
\item
  \href{https://www.nytco.com/}{NYTCo}
\item
  \href{https://help.nytimes3xbfgragh.onion/hc/en-us/articles/115015385887-Contact-Us}{Contact
  Us}
\item
  \href{https://www.nytco.com/careers/}{Work with us}
\item
  \href{https://nytmediakit.com/}{Advertise}
\item
  \href{http://www.tbrandstudio.com/}{T Brand Studio}
\item
  \href{https://www.nytimes3xbfgragh.onion/privacy/cookie-policy\#how-do-i-manage-trackers}{Your
  Ad Choices}
\item
  \href{https://www.nytimes3xbfgragh.onion/privacy}{Privacy}
\item
  \href{https://help.nytimes3xbfgragh.onion/hc/en-us/articles/115014893428-Terms-of-service}{Terms
  of Service}
\item
  \href{https://help.nytimes3xbfgragh.onion/hc/en-us/articles/115014893968-Terms-of-sale}{Terms
  of Sale}
\item
  \href{https://spiderbites.nytimes3xbfgragh.onion}{Site Map}
\item
  \href{https://help.nytimes3xbfgragh.onion/hc/en-us}{Help}
\item
  \href{https://www.nytimes3xbfgragh.onion/subscription?campaignId=37WXW}{Subscriptions}
\end{itemize}
