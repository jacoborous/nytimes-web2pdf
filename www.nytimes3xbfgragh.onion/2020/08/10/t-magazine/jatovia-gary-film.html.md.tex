Sections

SEARCH

\protect\hyperlink{site-content}{Skip to
content}\protect\hyperlink{site-index}{Skip to site index}

\href{https://myaccount.nytimes3xbfgragh.onion/auth/login?response_type=cookie\&client_id=vi}{}

\href{https://www.nytimes3xbfgragh.onion/section/todayspaper}{Today's
Paper}

The Artist and Filmmaker Envisioning a Safer World for Black Women

\begin{itemize}
\item
\item
\item
\item
\item
\item
\end{itemize}

Advertisement

\protect\hyperlink{after-top}{Continue reading the main story}

Supported by

\protect\hyperlink{after-sponsor}{Continue reading the main story}

T Presents

\hypertarget{the-artist-and-filmmaker-envisioning-a-safer-world-for-black-women}{%
\section{The Artist and Filmmaker Envisioning a Safer World for Black
Women}\label{the-artist-and-filmmaker-envisioning-a-safer-world-for-black-women}}

Interspersing found footage with original interviews, Ja'Tovia Gary
makes poetic video works that elucidate the realities of racial
injustice --- and how the country might change.

\includegraphics{https://static01.graylady3jvrrxbe.onion/images/2020/08/10/t-magazine/art/Tadobe-slide-781B/Tadobe-slide-781B-articleLarge.jpg?quality=75\&auto=webp\&disable=upscale}

By \href{https://www.nytimes3xbfgragh.onion/by/lovia-gyarkye}{Lovia
Gyarkye}

\begin{itemize}
\item
  Aug. 10, 2020
\item
  \begin{itemize}
  \item
  \item
  \item
  \item
  \item
  \item
  \end{itemize}
\end{itemize}

Ti'jhae Beecher is 7 years old when she happens upon the artist and
filmmaker Ja'Tovia Gary on the corner of West **** 116th Street and
Malcolm X Boulevard in Harlem in 2019. Gary, flanked by a camera crew
and with a microphone in hand, had stopped her to ask a simple question:
``Do you feel safe?'' Beecher, who is standing with her grandfather,
answers yes and proceeds to tell Gary about how she feels prepared to
leave the house every day, but that there are times when she is upset
about having to wake up early for school. Gary agrees that waking up
early is, in fact, the worst, and then asks Beecher if, for the most
part, she feels good. Beecher replies plainly: ``I don't feel like I'm
in danger.''

``That's good,'' Gary says in response. ``I hope you never ever feel
like you are in danger, that you always feel safe and strong.'' Gary's
interview with Beecher is one of many that she conducts with Black women
and girls in Harlem that day, asking those passers-by --- who differ in
age, ethnicity and spiritual identification --- if they feel safe in
their bodies, and in the world. She includes their wide-ranging
responses to her question in
``\href{https://www.jatovia.com/the-giverny-document-2019\#1}{The
Giverny Document}'' (2019), an experimental film that explores what it
means to exist in the world for Black women, and Black women only. The
roughly 40-minute feature was part of a three-channel video installation
presented at both Paula Cooper Gallery in New York and the Hammer Museum
in Los Angeles this year (both shows closed early because of Covid-19),
but has also been shown as a stand-alone project at film festivals,
garnering Gary awards and critical acclaim.

Image

Gary's studio is filled with film reels, art books and photographs,
including a picture of her great-grandmother, taped to her chalkboard.
``I wanted to bring an ancestral energy to the space,'' she
says.Credit...JerSean Golatt

Image

An image of the actress Ruby Dee identifies a reel containing parts of
Gary's film ``An Ecstatic Experience'' (2015), which includes clips of
Dee.Credit...JerSean Golatt

When Gary, 35, talked about that moment with Beecher over the phone the
Monday after this past Juneteenth, she described her own response as
naïve, because, she said, ``there is no way that she is going to feel
safe forever.'' But, Gary added, ``our job is to create a world where
that is possible.'' As she sees it, part of her mission as a Black
Southern queer artist is to help shape this imagined world, one in which
Black women do not live in a state of precarity, forever teetering
between violence and safety. \textbf{Indeed,} Gary has been engaging
with themes of power and representations of Black womanhood in her work
since 2010. Her films are painterly and essayistic, combining seemingly
disparate archival footage to elicit new and different emotional
responses from the viewer. ``Healing is at the root of the work,'' she
said. ``Making art is a transformative process that transmutes pain or
trauma into something beautiful, useful, functional, instructive for
those who can engage with the work, and for me.''

Before Gary made films, she wanted to be in them. She was born in Dallas
and raised in Cedar Hill, a lush suburb about 20 minutes outside of the
city, which she characterizes as alienating. She came from a family of
religious storytellers --- there are preachers and ministers on both of
her parents' sides --- and always loved an audience: ``I was the girl
who was going to read out loud in language arts class,'' she said. In
fifth grade, her stepfather died, and the experience left her with
anxiety that complicated her relationship to attention. Still, Gary went
on to become ``a theater geek,'' and, when she was in the 11th grade,
she transferred to Dallas's Booker T. Washington High School for the
Performing and Visual Arts, a highly competitive arts school whose
alumni include Erykah Badu and Norah Jones. To say the experience
changed her life is an understatement. There, Gary studied ancient Greek
theater, Shakespeare and Molière alongside foundational Black texts: ``I
learned about Black women playwrights, the Black feminist literary
tradition as well, everyone from Ntozake Shange to Lorraine Hansberry.''

\includegraphics{https://static01.graylady3jvrrxbe.onion/images/2020/08/10/t-magazine/tmag-adobe-thumbnail-slide-D6RA/tmag-adobe-thumbnail-slide-D6RA-mediumSquareAt3X.jpg}

Armed with this knowledge and ambition, she left Texas and moved to New
York in 2002 to study at Marymount Manhattan College. After a year, she
dropped out (for a combination of financial and mental health reasons),
started waiting tables and got herself a manager. She landed some roles
in television commercials and doing voiceovers, and was relatively
successful, but the opportunities were limiting and ``demoralizing,''
she said, recalling one gig for which she was instructed to be ``a
little more urban.''

While assessing her feelings about acting and the roles available to
her, Gary re-enrolled in school --- this time at Brooklyn College ---
and started taking classes in Africana studies. ``Figuring out how I was
going to move forward as an artist became a serious conundrum,'' she
said. ``For me, moving to the position of director was about gaining
agency and power and autonomy.''

From behind the camera, Gary has been able to create work that explores
and reclaims Black women's subjectivity. In her 2015 short
``\href{https://www.jatovia.com/gallery-3\#1}{An Ecstatic Experience},''
she combines, among other clips, one from a 1965 television show in
which the actress Ruby Dee dramatizes the narrative of Fannie Moore, a
woman born into slavery in 1849, with one of the activist Assata Shakur
talking to the reporter Gil Noble in 1987 about her escape from the
United States to Cuba. Gary employs a technique called direct animation,
etching cubes around Dee's face, as well as stripes, waves and halos ---
marks that emphasize a feeling of frenzy and ecstasy, temporarily
destabilizing the viewer and connecting one Black woman's liberation to
another. She uses a similar approach in
``\href{https://www.jatovia.com/giverny-i-negresse-imperiale\#1}{Giverny
I (Négresse Impériale)},'' a six-minute short she made in 2016 during a
summer residency with the Terra Foundation for American Art in Giverny,
France. ``Giverny I,'' all of which is included in ``The Giverny
Document,'' addresses the insecurity of the Black woman's body by
juxtaposing video of Gary in Claude Monet's garden with Diamond
Reynold's video of the cop who fatally shot her boyfriend, Philando
Castile, that same summer. The result is a chaotic yet clarifying
compendium of images. ``I don't want the work to lull people into a
sense of complacency,'' Gary said. ``I don't want them to be merely
satisfied or entertained.''

\includegraphics{https://static01.graylady3jvrrxbe.onion/images/2020/08/10/t-magazine/art/Tadobe-slide-YJRW/Tadobe-slide-YJRW-articleLarge.jpg?quality=75\&auto=webp\&disable=upscale}

The power of her films comes in part from her intimate and involved
process. Gary spends hours scouring the internet for archival footage
--- from live performances to interviews --- and treats the material as
a canvas, scratching or painting onto the film surface or even adhering
flower petals to clear film strips before digitizing and editing them.
In the case of the Facebook Live footage she uses in ``The Giverny
Document,'' Gary obscures Castile's bloodied body with leaves and petals
she plucked from the plants in Monet's garden to interrogate the ease
with which society consumes images of Black death and violence.

Over the last five years, Gary has been working on an autobiographical
documentary titled
``\href{https://www.jatovia.com/things-not-seen\#1}{The Evidence of
Things Not Seen}.'' It's a project about her relationship to her family
that recently brought her back to Dallas (after 16 years in New York and
a brief stint in Boston), where she has been self-isolating during the
pandemic. In some ways it has allowed her to embark on her own healing
journey, she explained. ``It requires honesty with not simply my family
but myself,'' she said of the work. ``It's very much about leaping with
no net, you know, and asking, do you have the courage to do that?''

\hypertarget{t-presents-15-creative-women-for-our-time}{%
\subsubsection{\texorpdfstring{\href{https://www.nytimes3xbfgragh.onion/interactive/2020/08/10/t-magazine/creative-women-designers-artists-chefs.html}{T
Presents: 15 Creative Women for Our
Time}}{T Presents: 15 Creative Women for Our Time}}\label{t-presents-15-creative-women-for-our-time}}

\href{https://www.nytimes3xbfgragh.onion/section/t-magazine}{}

\href{https://www.nytimes3xbfgragh.onion/2020/08/10/t-magazine/priya-ahluwalia-fashion-menswear.html}{\includegraphics{https://static01.graylady3jvrrxbe.onion/newsgraphics/2020/06/17/tmag-adobe/assets/images/ahluwalia-460.jpg}}

Priya Ahluwalia

Fashion Designer

\href{https://www.nytimes3xbfgragh.onion/2020/08/10/t-magazine/alice-cicolini-jewelry-art.html}{\includegraphics{https://static01.graylady3jvrrxbe.onion/newsgraphics/2020/06/17/tmag-adobe/assets/images/cicolini-460.jpg}}

Alice Cicolini

Jewelry Designer

\href{https://nytimes3xbfgragh.onion/2020/08/10/t-magazine/sonya-clark-flags-art.html}{\includegraphics{https://static01.graylady3jvrrxbe.onion/newsgraphics/2020/06/17/tmag-adobe/assets/images/clark-460.jpg}}

Sonya Clark

Artist

\href{https://www.nytimes3xbfgragh.onion/2020/08/10/t-magazine/pierre-davis-no-sesso.html}{\includegraphics{https://static01.graylady3jvrrxbe.onion/newsgraphics/2020/06/17/tmag-adobe/assets/images/davis-460.jpg}}

Pierre Davis

Fashion Designer

\href{https://www.nytimes3xbfgragh.onion/2020/08/10/t-magazine/paria-farzaneh-fashion-menswear.html}{\includegraphics{https://static01.graylady3jvrrxbe.onion/newsgraphics/2020/06/17/tmag-adobe/assets/images/farzaneh-460.jpg}}

Paria Farzaneh

Fashion Designer

\href{https://www.nytimes3xbfgragh.onion/2020/08/10/t-magazine/elizabeth-garouste-interior-design.html}{\includegraphics{https://static01.graylady3jvrrxbe.onion/newsgraphics/2020/06/17/tmag-adobe/assets/images/garouste-460.jpg}}

Elizabeth Garouste

Furniture Designer and Artist

\href{https://www.nytimes3xbfgragh.onion/2020/08/10/t-magazine/jatovia-gary-film.html}{\includegraphics{https://static01.graylady3jvrrxbe.onion/newsgraphics/2020/06/17/tmag-adobe/assets/images/gary-460.jpg}}

Ja'Tovia Gary

Artist and Filmmaker

\href{https://www.nytimes3xbfgragh.onion/2020/08/10/t-magazine/aiko-hachisuka-art-sculpture.html}{\includegraphics{https://static01.graylady3jvrrxbe.onion/newsgraphics/2020/06/17/tmag-adobe/assets/images/hachisuka-460.jpg}}

Aiko Hachisuka

Artist

\href{https://www.nytimes3xbfgragh.onion/2020/08/10/t-magazine/juliana-huxtable.html}{\includegraphics{https://static01.graylady3jvrrxbe.onion/newsgraphics/2020/06/17/tmag-adobe/assets/images/huxtable-460.jpg}}

Juliana Huxtable

Artist

\href{https://www.nytimes3xbfgragh.onion/2020/08/10/t-magazine/mariam-kamara-architect-design.html}{\includegraphics{https://static01.graylady3jvrrxbe.onion/newsgraphics/2020/06/17/tmag-adobe/assets/images/kamara-460.jpg}}

Mariam Kamara

Architect

\href{https://www.nytimes3xbfgragh.onion/2020/08/10/t-magazine/sophia-moreno-bunge-floral-design.html}{\includegraphics{https://static01.graylady3jvrrxbe.onion/newsgraphics/2020/06/17/tmag-adobe/assets/images/bunge-460.jpg}}

Sophia Moreno-Bunge

Floral Designer

\href{https://www.nytimes3xbfgragh.onion/2020/08/10/t-magazine/marina-moscone-fashion-design.html}{\includegraphics{https://static01.graylady3jvrrxbe.onion/newsgraphics/2020/06/17/tmag-adobe/assets/images/moscone-460.jpg}}

Marina Moscone

Fashion Designer

\href{https://www.nytimes3xbfgragh.onion/2020/08/10/t-magazine/amber-pinkerton-photography.html}{\includegraphics{https://static01.graylady3jvrrxbe.onion/newsgraphics/2020/06/17/tmag-adobe/assets/images/pinkerton-460.jpg}}

Amber Pinkerton

Photographer

\href{https://www.nytimes3xbfgragh.onion/2020/08/10/t-magazine/sonoko-sakai-chef-cooking-soba.html}{\includegraphics{https://static01.graylady3jvrrxbe.onion/newsgraphics/2020/06/17/tmag-adobe/assets/images/sakai-460.jpg}}

Sonoko Sakai

Cookbook Author and Food Activist

\href{https://www.nytimes3xbfgragh.onion/2020/08/10/t-magazine/daniela-soto-innes-cooking-chef.html}{\includegraphics{https://static01.graylady3jvrrxbe.onion/newsgraphics/2020/06/17/tmag-adobe/assets/images/ines-460.jpg}}

Daniela Soto-Innes

Chef

Advertisement

\protect\hyperlink{after-bottom}{Continue reading the main story}

\hypertarget{site-index}{%
\subsection{Site Index}\label{site-index}}

\hypertarget{site-information-navigation}{%
\subsection{Site Information
Navigation}\label{site-information-navigation}}

\begin{itemize}
\tightlist
\item
  \href{https://help.nytimes3xbfgragh.onion/hc/en-us/articles/115014792127-Copyright-notice}{©~2020~The
  New York Times Company}
\end{itemize}

\begin{itemize}
\tightlist
\item
  \href{https://www.nytco.com/}{NYTCo}
\item
  \href{https://help.nytimes3xbfgragh.onion/hc/en-us/articles/115015385887-Contact-Us}{Contact
  Us}
\item
  \href{https://www.nytco.com/careers/}{Work with us}
\item
  \href{https://nytmediakit.com/}{Advertise}
\item
  \href{http://www.tbrandstudio.com/}{T Brand Studio}
\item
  \href{https://www.nytimes3xbfgragh.onion/privacy/cookie-policy\#how-do-i-manage-trackers}{Your
  Ad Choices}
\item
  \href{https://www.nytimes3xbfgragh.onion/privacy}{Privacy}
\item
  \href{https://help.nytimes3xbfgragh.onion/hc/en-us/articles/115014893428-Terms-of-service}{Terms
  of Service}
\item
  \href{https://help.nytimes3xbfgragh.onion/hc/en-us/articles/115014893968-Terms-of-sale}{Terms
  of Sale}
\item
  \href{https://spiderbites.nytimes3xbfgragh.onion}{Site Map}
\item
  \href{https://help.nytimes3xbfgragh.onion/hc/en-us}{Help}
\item
  \href{https://www.nytimes3xbfgragh.onion/subscription?campaignId=37WXW}{Subscriptions}
\end{itemize}
