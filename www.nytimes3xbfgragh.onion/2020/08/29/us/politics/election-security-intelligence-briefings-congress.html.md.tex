Sections

SEARCH

\protect\hyperlink{site-content}{Skip to
content}\protect\hyperlink{site-index}{Skip to site index}

\href{https://www.nytimes3xbfgragh.onion/section/politics}{Politics}

\href{https://myaccount.nytimes3xbfgragh.onion/auth/login?response_type=cookie\&client_id=vi}{}

\href{https://www.nytimes3xbfgragh.onion/section/todayspaper}{Today's
Paper}

\href{/section/politics}{Politics}\textbar{}No More In-Person Election
Briefings for Congress, Intelligence Chief Says

\url{https://nyti.ms/3jsyYP6}

\begin{itemize}
\item
\item
\item
\item
\item
\end{itemize}

\begin{itemize}
\item
  \href{https://www.nytimes3xbfgragh.onion/interactive/2020/09/08/us/elections/results-new-hampshire-primary-elections.html?action=click\&pgtype=Article\&state=default\&region=TOP_BANNER\&context=storylines_menu}{New
  Hampshire Results}
\item
  \href{https://www.nytimes3xbfgragh.onion/live/2020/09/08/us/trump-vs-biden?action=click\&pgtype=Article\&state=default\&region=TOP_BANNER\&context=storylines_menu}{Election
  Updates}
\item
  \href{https://www.nytimes3xbfgragh.onion/interactive/2020/us/elections/election-states-biden-trump.html?action=click\&pgtype=Article\&state=default\&region=TOP_BANNER\&context=storylines_menu}{Paths
  to 270}
\item
  \href{https://www.nytimes3xbfgragh.onion/interactive/2020/08/31/us/politics/vote-by-mail-deadlines.html?action=click\&pgtype=Article\&state=default\&region=TOP_BANNER\&context=storylines_menu}{Voting
  by Mail}
\item
  \href{https://www.nytimes3xbfgragh.onion/interactive/2019/us/elections/2020-presidential-election-calendar.html?action=click\&pgtype=Article\&state=default\&region=TOP_BANNER\&context=storylines_menu}{Key
  Dates}
\item
  \href{https://www.nytimes3xbfgragh.onion/newsletters/politics?action=click\&pgtype=Article\&state=default\&region=TOP_BANNER\&context=storylines_menu}{Politics
  Newsletter}
\end{itemize}

Advertisement

\protect\hyperlink{after-top}{Continue reading the main story}

Supported by

\protect\hyperlink{after-sponsor}{Continue reading the main story}

\hypertarget{no-more-in-person-election-briefings-for-congress-intelligence-chief-says}{%
\section{No More In-Person Election Briefings for Congress, Intelligence
Chief
Says}\label{no-more-in-person-election-briefings-for-congress-intelligence-chief-says}}

Lawmakers in both parties worry the move will block their ability to
question and test intelligence assessments that can be crucial to
ensuring that foreign powers do not undermine election results.

\includegraphics{https://static01.graylady3jvrrxbe.onion/images/2020/08/30/lens/29dc-intel-print/29dc-intel-pif-articleLarge.jpg?quality=75\&auto=webp\&disable=upscale}

\href{https://www.nytimes3xbfgragh.onion/by/nicholas-fandos}{\includegraphics{https://static01.graylady3jvrrxbe.onion/images/2018/11/06/multimedia/author-nicholas-fandos/author-nicholas-fandos-thumbLarge-v2.png}}\href{https://www.nytimes3xbfgragh.onion/by/julian-e-barnes}{\includegraphics{https://static01.graylady3jvrrxbe.onion/images/2019/12/13/reader-center/author-julian-barnes/author-julian-barnes-thumbLarge.png}}

By \href{https://www.nytimes3xbfgragh.onion/by/nicholas-fandos}{Nicholas
Fandos} and
\href{https://www.nytimes3xbfgragh.onion/by/julian-e-barnes}{Julian E.
Barnes}

\begin{itemize}
\item
  Aug. 29, 2020
\item
  \begin{itemize}
  \item
  \item
  \item
  \item
  \item
  \end{itemize}
\end{itemize}

The nation's top intelligence officials moved on Saturday to tighten
control over the flow of sensitive intelligence about foreign threats to
November's election, telling Congress that they would no longer provide
in-person briefings about
\href{https://www.nytimes3xbfgragh.onion/live/2020/08/31/us/trump-vs-biden}{election}
security and would rely solely on written updates instead.

Representatives from the Office of the Director of National Intelligence
informed the House and Senate Intelligence Committees of the policy
change by telephone on Friday and followed up with a batch of letters to
congressional leaders on Saturday.

In the letters, the chief of the intelligence office, John L. Ratcliffe,
framed the move as an attempt to ``ensure clarity and consistency'' in
intelligence agencies' interactions with Congress and to crack down on
leaks that have infuriated some intelligence officials.

``I believe this approach helps ensure, to the maximum extent possible,
that the information O.D.N.I. provides the Congress in support of your
oversight responsibilities on elections security, foreign malign
influence and election interference is not misunderstood nor
politicized,'' he wrote, according to
\href{https://int.graylady3jvrrxbe.onion/data/documenttools/letter-from-the-office-of-the-director-of-national-intelligence/728f2b21d6a4b371/full.pdf}{a
copy obtained by The New York Times}. ``It will also better protect our
sources and methods and most sensitive intelligence from additional
unauthorized disclosures or misuse.''

But coming just 10 weeks before Election Day, the change drew complaints
from lawmakers in both parties who worried the move would block their
ability to question and test intelligence assessments from the executive
branch at a time when they are crucial to ensuring that foreign powers
do not undermine the results. Intelligence agencies have revealed that
\href{https://www.nytimes3xbfgragh.onion/2020/08/07/us/politics/russia-china-trump-biden-election-interference.html}{Russia
is again trying to bolster the campaign of President Trump}, who has
insisted he is actually ``the last person Russia wants to see in
office'' and consistently attacked the intelligence agencies during his
tenure.

Democrats, who fear Mr. Trump's appointees have moved to color
intelligence assessments for his political benefit, were particularly
furious.

Speaker Nancy Pelosi and Representative Adam B. Schiff of California,
the chairman of the House Intelligence Committee, called the new policy
``shameful'' and said intelligence officials had also canceled briefings
with committees and the full House on election security threats already
scheduled for September at the request of Mr. Ratcliffe's office. They
vowed to try to force their reinstatement

``This is a shocking abdication of its lawful responsibility to keep the
Congress currently informed, and a betrayal of the public's right to
know how foreign powers are trying to subvert our democracy,'' the two
senior Democrats wrote.

On Saturday night, Senator Marco Rubio of Florida, the acting chairman
of the Senate Intelligence Committee, issued a statement that praised
Mr. Ratcliffe and said material from the congressional briefings had
been improperly leaked in an effort to damage Mr. Trump.

``This situation we now face is due, in no small part, to the
willingness of some to commit federal crimes for the purpose of
advancing their electoral aims,'' he said.

But the leaks should not ``release the intelligence community'' from the
legal obligation to brief Congress, he said, adding that he had
discussed the issue with Mr. Ratcliffe.

``In particular, he made explicitly clear that the Senate Select
Committee on Intelligence will continue receiving briefings on all
oversight topics, including election matters,'' Mr. Rubio said.

His statement did not clarify whether those briefings would be in person
or in writing.

A spokeswoman with the Office of the Director of National Intelligence
declined to comment. CNN
\href{https://www.cnn.com/2020/08/29/politics/office-of-director-of-national-intelligence-congress-election-security/index.html}{first
reported the change}.

Lawmakers far prefer hearing in person from executive branch officials
responsible for national security and intelligence. While written
briefings can be carefully edited to present the message the agencies
want heard, in-person sessions allow lawmakers to ask questions; press
for additional information about sources, methods and assumptions
involved in drawing up intelligence assessments; and analyze the tone of
those briefing them.

Eliminating in-person briefings about intelligence threats to the
election could also undermine a key lesson for lawmakers after Russian
interference in the 2016 election. This month, the Senate Intelligence
Committee
\href{https://www.nytimes3xbfgragh.onion/2020/08/18/us/politics/senate-intelligence-russian-interference-report.html}{concluded
its bipartisan, yearslong investigation} of the 2016 election and
recommended that intelligence agencies release as much information about
foreign interference efforts to lawmakers and the public as quickly as
possible, to undercut any propaganda and disinformation.

Senator Angus King of Maine, an independent member of the Senate
Intelligence Committee who votes with the Democrats, said dry written
briefings never had the breadth of information that a full
question-and-answer session had.

``It is an outrage,'' Mr. King said in an interview. ``It smacks of a
cover-up of information about foreign interference in our elections. If
there is foreign interference in our election, we should know about it
before the election. The intelligence committee is not a history
organization, it is a current events organization.''

Mr. King said that the classified information that had been presented to
his committee in recent sessions had been protected.

Composed of lawmakers from both parties, the House and Senate
intelligence committees are highly secretive bodies that are responsible
for overseeing intelligence policies and the operations of the nation's
intelligence agencies. Both panels were expecting additional in-person
briefings before Nov. 3.

While details of the new restrictions are murky, in-person briefings to
the political campaigns and the Democratic and Republican National
Committees are likely to continue, according to a person briefed on the
administration's plans.

The directive appears to apply to all intelligence agencies that report
to Mr. Ratcliffe, though not necessarily other entities in the Justice
Department, Defense Department and Homeland Security Department that are
responsible for election security and that also regularly apprise
Congress of their work. An official from Homeland Security's
Cybersecurity and Infrastructure Security Agency, which monitors the
security of voting machines, said it would continue to brief Congress.

Still, the distinction may matter little. In May, the Trump
administration
\href{https://www.nytimes3xbfgragh.onion/2020/05/15/us/politics/counterterrorism-intelligence.html}{consolidated
election-related congressional briefings} under William R. Evanina, the
director of the National Counterintelligence and Security Center, who
reports to Mr. Ratcliffe.

That change ensured that a single voice speaks for the sprawling
intelligence community, but it also effectively sidelined other agencies
and officials like the election czar at the Office of the Director of
National Intelligence, Shelby Pierson, from doing so. Ms. Pierson was
appointed to the post last year, but when she
\href{https://www.nytimes3xbfgragh.onion/2020/02/20/us/politics/russian-interference-trump-democrats.html}{briefed
House lawmakers that the Russian government preferred that Mr. Trump be
re-elected}, it prompted widespread anger among Republicans.

During a round table on Saturday in hurricane-stricken Texas, Mr. Trump
said it had been necessary to restrict the briefings because of what he
said were unspecified leaks by Mr. Schiff, whom the president called
``shifty Schiff,'' and other Democrats on his committee after earlier
in-person sessions. But he did not specify what he was referring to or
why he would cut off all members of Congress, including Republicans.

``He got tired of it so he wants to do it in a different form, because
you have leakers on the committee,'' Mr. Trump said of Mr. Ratcliffe.
Mr. Schiff
\href{https://twitter.com/repadamschiff/status/1299837041858031623?s=11}{accused
the president of lying} in a post on Twitter on Saturday.

Richard Grenell, Mr. Ratcliffe's predecessor, praised the move on
Saturday, saying he had heard from career intelligence officials that
they no longer wanted to brief lawmakers, ``because the partial
information leaks and manipulation of their words were detrimental to
their careers.''

Lawmakers saw it differently. The decision by Mr. Ratcliffe, who once
received similar briefings as
\href{https://www.nytimes3xbfgragh.onion/2020/02/28/us/politics/john-ratcliffe-director-national-intelligence.html}{a
Republican congressman} on the House Intelligence Committee, fanned
accusations by Democrats that he had politicized the intelligence
community since
\href{https://www.nytimes3xbfgragh.onion/2020/05/21/us/politics/john-ratcliffe-intelligence-director.html}{becoming
its director} this year.

``Ratcliffe has made clear he's in the job only to protect Trump from
democracy, not democracy from Trump,'' said Senator Chuck Schumer of New
York, the top Senate Democrat.

They have charged that under Mr. Trump, intelligence briefings to
Congress have been tainted by partisan politics, especially this year as
the Office of the Director of National Intelligence has pushed to
declassify information that conservatives believe is embarrassing to the
Obama administration or undercut the F.B.I.'s investigation into
Russia's multipronged election interference efforts in 2016, and
possible ties to the Trump campaign.

House Democrats, in particular, have delivered withering criticisms of
recent election security briefings from the intelligence agencies,
frustrating the top officials responsible for them.

After House members were briefed in late July on threats to the coming
election at the request of Democratic leaders, Democrats publicly
castigated the administration officials present for withholding what
they viewed as crucial information from the American people.

Behind closed doors, Ms. Pelosi had more pointedly accused Mr. Evanina
of keeping Americans blind to the true nature of the threats,
complaining that
\href{https://www.dni.gov/index.php/newsroom/press-releases/item/2135-statement-by-ncsc-director-william-evanina-100-days-until-election-2020}{a
public assessment he issued in late July} was too vague to be useful.
People familiar with the exchange, which was earlier
\href{https://www.politico.com/news/2020/07/31/nancy-pelosi-william-evanina-russia-meddling-389847}{reported
by Politico}, confirmed it on Saturday.

An intelligence official at the time defended Mr. Evanina's briefing and
accused Democrats of leaking classified information that they were
legally obligated to keep private for partisan gain.

But about a week later, Mr. Evanina
\href{https://www.dni.gov/index.php/newsroom/press-releases/item/2139-statement-by-ncsc-director-william-evanina-election-threat-update-for-the-american-public}{issued
a more detailed statement} saying Russia was
\href{https://www.nytimes3xbfgragh.onion/2020/08/07/us/politics/russia-china-trump-biden-election-interference.html}{using
a range of tactics to smear Joseph R. Biden Jr.}, the Democratic nominee
for president, in an effort to help Mr. Trump's campaign. The statement
said China preferred Mr. Trump's defeat and was deciding whether to more
actively try to influence the election against a president it considers
``unpredictable.'' And he noted that Iran was seeking to undermine
American democracy, as well, and sow divisions ahead of the election.

On Saturday, Ms. Pelosi and Mr. Schiff accused the Trump administration
of promoting ``a false narrative'' that the three nations pose the same
threat.

``Only one country --- Russia --- is actively undertaking a range of
measures to undermine the presidential election and to secure the
outcome that the Kremlin sees as best serving its interests,'' they
wrote.

\hypertarget{our-2020-election-guide}{%
\section{Our 2020 Election Guide}\label{our-2020-election-guide}}

Updated ~Sept. 8, 2020

\begin{center}\rule{0.5\linewidth}{\linethickness}\end{center}

\begin{itemize}
\item ~
  \hypertarget{the-latest}{%
  \subsection{The Latest}\label{the-latest}}

  \begin{itemize}
  \item
    President Trump and his party are using a playbook that aims to
    alarm people about crime in their backyards. It didn't work in 2018,
    but
    \href{https://www.nytimes3xbfgragh.onion/2020/09/08/us/politics/trump-republicans-fear-strategy.html?action=click\&pgtype=Article\&state=default\&region=BELOW_MAIN_CONTENT\&context=storylines_guide}{both
    parties think it could resonate more this year}.
  \end{itemize}
\item ~
  \hypertarget{how-to-win-270}{%
  \subsection{How to Win 270}\label{how-to-win-270}}

  \begin{itemize}
  \item
    Joe Biden and Donald Trump need 270 electoral votes to reach the
    White House. Try building
    \href{https://www.nytimes3xbfgragh.onion/interactive/2020/us/elections/election-states-biden-trump.html?action=click\&pgtype=Article\&state=default\&region=BELOW_MAIN_CONTENT\&context=storylines_guide}{your
    own coalition of battleground states}~to see potential outcomes.
  \end{itemize}
\item ~
  \hypertarget{voting-by-mail}{%
  \subsection{Voting by Mail}\label{voting-by-mail}}

  \begin{itemize}
  \item
    Will you have enough time to vote by mail in your state? Yes, but
    it's risky to procrastinate.
    \href{https://www.nytimes3xbfgragh.onion/interactive/2020/08/31/us/politics/vote-by-mail-deadlines.html?action=click\&pgtype=Article\&state=default\&region=BELOW_MAIN_CONTENT\&context=storylines_guide}{Check
    your state's deadline.}
  \item
    \href{https://www.nytimes3xbfgragh.onion/interactive/2020/us/elections/joe-biden.html?action=click\&pgtype=Article\&state=default\&region=BELOW_MAIN_CONTENT\&context=storylines_guide}{}

    \hypertarget{joe-biden}{%
    \section{Joe Biden}\label{joe-biden}}

    \hypertarget{democrat}{%
    \subsection{Democrat}\label{democrat}}

    \href{https://www.nytimes3xbfgragh.onion/interactive/2020/us/elections/donald-trump.html?action=click\&pgtype=Article\&state=default\&region=BELOW_MAIN_CONTENT\&context=storylines_guide}{}

    \hypertarget{donald-trump}{%
    \section{Donald Trump}\label{donald-trump}}

    \hypertarget{republican}{%
    \subsection{Republican}\label{republican}}
  \end{itemize}
\item
  \hypertarget{keep-up-with-our-coverage}{%
  \subsection{Keep Up With Our
  Coverage}\label{keep-up-with-our-coverage}}

  \begin{itemize}
  \item
    Get an
    \href{https://www.nytimes3xbfgragh.onion/newsletters/politics?action=click\&pgtype=Article\&state=default\&region=BELOW_MAIN_CONTENT\&context=storylines_guide}{email}~recapping
    the day's news
  \item
    Download our mobile app on
    \href{https://apps.apple.com/us/app/nytimes/id284862083?ls=1\&mat_click_id=5c79ae7455014fd1bd66b5610c05b8f2-20191112-16948\&referrer=mat_click_id\%3D5c79ae7455014fd1bd66b5610c05b8f2-20191112-16948\%26link_click_id\%3D722930677036718082}{iOS}~and
    \href{http://a.localytics.com/android?id=com.nytimes.android\&referrer=utm_source\%3Dother_nyt_mobile_web\%26utm_medium\%3DWeb\%2520page\%26utm_term\%3DGeneral\%2520Mobile\%2520Page\%26utm_campaign\%3DNYT\%2520Mobile\%2520General\%2520Page}{Android}~and
    turn on Breaking News and Politics alerts
  \end{itemize}
\end{itemize}

Advertisement

\protect\hyperlink{after-bottom}{Continue reading the main story}

\hypertarget{site-index}{%
\subsection{Site Index}\label{site-index}}

\hypertarget{site-information-navigation}{%
\subsection{Site Information
Navigation}\label{site-information-navigation}}

\begin{itemize}
\tightlist
\item
  \href{https://help.nytimes3xbfgragh.onion/hc/en-us/articles/115014792127-Copyright-notice}{©~2020~The
  New York Times Company}
\end{itemize}

\begin{itemize}
\tightlist
\item
  \href{https://www.nytco.com/}{NYTCo}
\item
  \href{https://help.nytimes3xbfgragh.onion/hc/en-us/articles/115015385887-Contact-Us}{Contact
  Us}
\item
  \href{https://www.nytco.com/careers/}{Work with us}
\item
  \href{https://nytmediakit.com/}{Advertise}
\item
  \href{http://www.tbrandstudio.com/}{T Brand Studio}
\item
  \href{https://www.nytimes3xbfgragh.onion/privacy/cookie-policy\#how-do-i-manage-trackers}{Your
  Ad Choices}
\item
  \href{https://www.nytimes3xbfgragh.onion/privacy}{Privacy}
\item
  \href{https://help.nytimes3xbfgragh.onion/hc/en-us/articles/115014893428-Terms-of-service}{Terms
  of Service}
\item
  \href{https://help.nytimes3xbfgragh.onion/hc/en-us/articles/115014893968-Terms-of-sale}{Terms
  of Sale}
\item
  \href{https://spiderbites.nytimes3xbfgragh.onion}{Site Map}
\item
  \href{https://help.nytimes3xbfgragh.onion/hc/en-us}{Help}
\item
  \href{https://www.nytimes3xbfgragh.onion/subscription?campaignId=37WXW}{Subscriptions}
\end{itemize}
