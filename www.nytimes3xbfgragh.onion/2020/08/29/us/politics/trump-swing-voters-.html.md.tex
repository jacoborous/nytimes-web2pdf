Sections

SEARCH

\protect\hyperlink{site-content}{Skip to
content}\protect\hyperlink{site-index}{Skip to site index}

\href{https://www.nytimes3xbfgragh.onion/section/politics}{Politics}

\href{https://myaccount.nytimes3xbfgragh.onion/auth/login?response_type=cookie\&client_id=vi}{}

\href{https://www.nytimes3xbfgragh.onion/section/todayspaper}{Today's
Paper}

\href{/section/politics}{Politics}\textbar{}Trump's Strategy: Make
`Wobbly Republicans' Think Their Party Is Great Again

\url{https://nyti.ms/31BskzQ}

\begin{itemize}
\item
\item
\item
\item
\item
\end{itemize}

\begin{itemize}
\item
  \href{https://www.nytimes3xbfgragh.onion/interactive/2020/09/08/us/elections/results-new-hampshire-primary-elections.html?action=click\&pgtype=Article\&state=default\&region=TOP_BANNER\&context=storylines_menu}{New
  Hampshire Results}
\item
  \href{https://www.nytimes3xbfgragh.onion/live/2020/09/08/us/trump-vs-biden?action=click\&pgtype=Article\&state=default\&region=TOP_BANNER\&context=storylines_menu}{Election
  Updates}
\item
  \href{https://www.nytimes3xbfgragh.onion/interactive/2020/us/elections/election-states-biden-trump.html?action=click\&pgtype=Article\&state=default\&region=TOP_BANNER\&context=storylines_menu}{Paths
  to 270}
\item
  \href{https://www.nytimes3xbfgragh.onion/interactive/2020/08/31/us/politics/vote-by-mail-deadlines.html?action=click\&pgtype=Article\&state=default\&region=TOP_BANNER\&context=storylines_menu}{Voting
  by Mail}
\item
  \href{https://www.nytimes3xbfgragh.onion/interactive/2019/us/elections/2020-presidential-election-calendar.html?action=click\&pgtype=Article\&state=default\&region=TOP_BANNER\&context=storylines_menu}{Key
  Dates}
\item
  \href{https://www.nytimes3xbfgragh.onion/newsletters/politics?action=click\&pgtype=Article\&state=default\&region=TOP_BANNER\&context=storylines_menu}{Politics
  Newsletter}
\end{itemize}

Advertisement

\protect\hyperlink{after-top}{Continue reading the main story}

Supported by

\protect\hyperlink{after-sponsor}{Continue reading the main story}

News Analysis

\hypertarget{trumps-strategy-make-wobbly-republicans-think-their-party-is-great-again}{%
\section{Trump's Strategy: Make `Wobbly Republicans' Think Their Party
Is Great
Again}\label{trumps-strategy-make-wobbly-republicans-think-their-party-is-great-again}}

The Trump campaign wants to lure back moderates who believe that the
president has lurched too far right and openly courted extremists.

\includegraphics{https://static01.graylady3jvrrxbe.onion/images/2020/08/30/us/politics/30gop-reality-print1/merlin_176289138_1c6fae18-c859-42aa-bf6f-506f65c5d3cc-articleLarge.jpg?quality=75\&auto=webp\&disable=upscale}

\href{https://www.nytimes3xbfgragh.onion/by/jeremy-w-peters}{\includegraphics{https://static01.graylady3jvrrxbe.onion/images/2018/11/06/multimedia/author-jeremy-w-peters/author-jeremy-w-peters-thumbLarge.png}}

By \href{https://www.nytimes3xbfgragh.onion/by/jeremy-w-peters}{Jeremy
W. Peters}

\begin{itemize}
\item
  Published Aug. 29, 2020Updated Sept. 8, 2020
\item
  \begin{itemize}
  \item
  \item
  \item
  \item
  \item
  \end{itemize}
\end{itemize}

\href{https://www.nytimes3xbfgragh.onion/2020/09/01/us/politics/trump-republicans.html}{President
Trump} and his party are entering the nine-week stretch until
\href{https://www.nytimes3xbfgragh.onion/live/2020/08/31/us/trump-vs-biden}{Election}
Day with their success riding on a hope that enough voters come to the
following conclusion: You're not as bad as we thought.

As part of this strategy,
\href{https://www.nytimes3xbfgragh.onion/2020/09/08/us/politics/trump-republicans-fear-strategy.html}{Republicans}
and the Trump campaign are attempting to focus voters' minds away from
the pandemic and economic crisis and on a narrower set of cultural
issues. For example, the nation is in dire straits, they say, not
because of Covid-19 deaths or double-digit unemployment or racial
discord, but because of liberals who want to ``cancel'' conservatives,
criminals who are rampaging from the cities into white suburbs and a
Democratic presidential ticket that is a ``Trojan horse'' for Fidel
Castro-style authoritarianism.

These topics often consume conservative media, are already appearing in
ads from the Trump campaign and his allies --- and were mentioned
repeatedly over the four nights of the Republican National Convention.

To complement the drumbeat of panic about the radical left, convention
speakers stressed their personal experiences with the president and
shared affirmations about his character --- a man of ``deep compassion''
and ``endless kindness'' with an ``exceptional work ethic,'' they said.
And they attempted to refute the idea that he dislikes Black people,
women and immigrants, calling those divisive characterizations by the
left and distortions of a media ``fog machine.'' Ben Carson, the only
Black member of Mr. Trump's cabinet, said people who called the
president racist ``could not be more wrong.''

Strategists in both parties say this attempt to reframe the country's
understanding of who the president is could backfire, coming off as
dismissive of the acute racial awareness that has been leading people of
all political beliefs and races to re-examine their attitudes about
discrimination.

But they also said it just might work.

``It is a fascinating tightrope they're walking on because on the one
hand Trump is saying some of the most racist, bombastic stuff ever to
come out of the mouth of a nominee of a major political party,'' said
Cornell Belcher, a Democratic pollster who worked for Barack Obama's two
presidential campaigns, when racial attitudes were never far from the
surface. ``At the same time, they are professing and having others
validate that he is not a racist. The contradiction is mind boggling.''

Mr. Belcher added, ``It is diabolical, but it's also brilliant.''

Inside Republican campaigns across the country, operatives are intent on
reaching a relatively small slice of the electorate: Republicans and
Republican-leaning independents who have either not voted or voted for
Democrats since Mr. Trump took office because they dislike his style and
leadership. The goal between now and the election is to make these more
moderate voters feel comfortable again being associated with a party
they think has lurched far to the right, unapologetically condoning and
courting racists, bigots and other extremists.

Kristen Soltis Anderson, a Republican pollster, called these ``wobbly
Republicans'' and said they are often deeply conflicted about voting in
November. The goal of Republicans this fall, she said, is to ``remind
them why they're Republicans.''

\includegraphics{https://static01.graylady3jvrrxbe.onion/images/2020/08/29/us/politics/29gop-reality2/merlin_176262933_156d5f0d-e618-4263-929e-a4c0376d55f3-articleLarge.jpg?quality=75\&auto=webp\&disable=upscale}

``The idea is to aggressively push back against the caricature of
Republicans, which is something that a wobbly Republican doesn't want to
be,'' she said. ``If you're a Trump Republican, you reject that
characterization. But if you're a wobbly Republican, you probably
internalize that.''

Convincing voters to accept this less blemished version of the
president, which strategists say is probably Mr. Trump's best hope of
winning enough of the roughly 5 to 7 percent of the country that is
still undecided, is problematic in that it essentially requires people
to imagine that he has not been in charge all along. To forget that the
immigrants he joined in a naturalization ceremony in front of the
cameras weren't people from the kinds of countries he profanely
denigrated. To excuse him from responsibility for inflaming the tensions
in cities that now resemble scenes of the very ``American carnage'' he
vowed to end on Inauguration Day.

Polls show that the percentage of Americans who think the country is on
the wrong track --- which experts look to as a reliable predictor of how
the incumbent president will perform --- is near or exceeding
\href{https://cookpolitical.com/index.php/analysis/national/national-politics/right-direction-and-wrong-track-numbers-tell-story-election}{70
percent}.

To bring ``wobbly Republicans'' back on board --- not just with the
party but with the man currently leading it --- they need to change
minds about the best known person in the country. That is difficult
though not unprecedented. Richard Nixon prevailed in the 1968 election
after two humiliating losses that even he believed had crushed his hopes
of becoming president --- first to John F. Kennedy in 1960 and two years
later in the California governor's race. (The California loss was what
prompted Nixon's
\href{https://www.nixonfoundation.org/2017/11/55-years-ago-last-press-conference/}{most
famous} utterance of self-pity, ``You won't have Nixon to kick around
anymore,'' at what he said was his final news conference.)

But even Nixon had nothing like the level of mass media exposure that
Mr. Trump has had, or anything like his insatiable desire to be in the
spotlight. And over the five years since he first started running for
president, his ubiquity has left few Americans without a firm opinion
about him. His job approval ratings have been extraordinarily steady and
more aligned with partisan affiliation than any president in the history
of modern polling,
\href{https://www.pewresearch.org/fact-tank/2020/08/24/trumps-approval-ratings-so-far-are-unusually-stable-and-deeply-partisan/}{according
to} the Pew Research Center.

``He is who he is,'' said Carly Fiorina, the former chief executive of
Hewlett-Packard who ran in the Republican presidential primary in 2016
against Mr. Trump. The idea that a glossy messaging operation could act
as a facade over Mr. Trump's flaws, she said, is far-fetched.

``I think what we know now is that Donald Trump cannot rise to the
occasion, he cannot grow into the job,'' Ms. Fiorina added. ``He is
someone who stokes controversy and conflict and outrage. It's who he was
in his reality TV days and who he is as president of the United
States.''

Most Americans, she said, are focused on issues that the president and
his campaign hope voters will overlook: ``When are we going to get this
virus under control? When are we going to get my kids back to school?
When is my favorite restaurant down the street going to reopen? And some
of those Americans voted for Trump.''

On the issues where Republicans are trying to shift the most negative
perceptions of the president, the displays and affirmations at the
convention do not match public opinion, polls show.

In July, Fox News asked registered voters about whether the believed Mr.
Trump and his opponent, former Vice President Joseph R. Biden, had
traits like compassion, judgment and mental soundness. On the question
of compassion, only 36 percent responded that Mr. Trump did. Fifty-six
percent said yes about Mr. Biden.

While pro-Trump speakers like Rudolph W. Giuliani disparaged the Black
Lives Matter movement and the demonstrators who marched in its name,
polling from June and July showed that majorities of Americans have been
supportive of the marchers and
\href{https://www.reuters.com/article/us-usa-election-poll/support-dips-for-protests-but-many-americans-reject-trumps-response-reuters-ipsos-poll-idUSKCN24U1EX}{disapprove}
of the way the president has handled them. The percentage of people who
believe racism and discrimination is a problem, including whites,
\href{https://www.nytimes3xbfgragh.onion/2020/06/05/us/politics/polling-george-floyd-protests-racism.html}{soared}
north of 70 percent as protests grew. How the recent unrest and eruption
of violence in Kenosha, Wis., will affect these attitudes is unclear.

And 75 percent of Black Americans ``strongly disagree'' with Mr. Trump's
\href{https://twitter.com/realDonaldTrump/status/1268167419132084230?s=20}{claims}
that he has ``done much more'' to improve their lives than any other
president since Abraham Lincoln,
\href{https://navigatorresearch.org/public-opinion-on-coronavirus-navigator-update-36/}{according
to}one recent survey.

The
\href{https://www.pewresearch.org/politics/2020/08/06/views-of-covid-19-response-by-trump-hospitals-cdc-and-other-officials/}{majority}of
the American public also continues to rate Mr. Trump's response to the
pandemic poorly. And polls show that even Republicans overwhelmingly
consider themselves in favor of wearing masks, despite the Trump
administration's inconsistent and often dismissive approach to
encouraging them.

Image

Delegates wearing masks at the Charlotte Convention Center for the roll
call vote to renominate President Trump on Monday.Credit...Travis Dove
for The New York Times

Polling does show a
\href{https://www.cato.org/publications/survey-reports/poll-62-americans-say-they-have-political-views-theyre-afraid-share\#liberals-are-divided-political-expression}{growing
percentage} of Americans of every political persuasion say they have
been afraid to express their political opinion. But priorities matter.
And the issues Americans continue to say they are concerned about more
than any other, aside from the economy, are the coronavirus, leadership
and race relations,
\href{https://news.gallup.com/poll/1675/most-important-problem.aspx}{according
to Gallup}.

David Winston, a Republican pollster, said that any politician who is
not making the coronavirus recovery their focus is misguided. ``Given
that that's what everyone in the country is dealing with, if you're not
talking about that, what exactly are you talking about?'' he asked.

Strategists in both parties said Mr. Trump and the Republicans do not
want to be in a situation where they are seen as preaching to the choir
to the exclusion of gettable swing voters like those wayward
Republicans. But the base-first strategy that Mr. Trump is most
comfortable with --- which his party has dutifully followed in races
down the ballot --- has not served him well.

In 2018, Republicans in Congressional races focused on a set of issues
they assumed would drive up turnout in a party so thoroughly consumed by
Mr. Trump's issues and persona. From Ohio to California, they ran ads
warning about criminal gangs and drugs invading the suburbs. Following
the president's lead, they pointed to a threatening caravan of
immigrants across the southern border.
\href{https://www.nytimes3xbfgragh.onion/2018/11/03/us/politics/republicans-house-races-caravan-voters.html}{Some}
invoked the image of \href{https://youtu.be/9-22S-kFong}{Colin
Kaepernick} kneeling for the national anthem in their ads.

``It's not clear that at any point since the 2016 election that strategy
has worked, other than in some deep red states in Senate races,'' said
Nick Gourevitch the head of research for the Global Strategy Group, a
Democratic firm. ``He's trying to recreate the exact circumstances of
his 2016 victory,'' Mr. Gourevitch said. ``But the migrant caravan, all
this stuff, they've tried it and it hasn't worked.''

\hypertarget{our-2020-election-guide}{%
\section{Our 2020 Election Guide}\label{our-2020-election-guide}}

Updated ~Sept. 8, 2020

\begin{center}\rule{0.5\linewidth}{\linethickness}\end{center}

\begin{itemize}
\item ~
  \hypertarget{the-latest}{%
  \subsection{The Latest}\label{the-latest}}

  \begin{itemize}
  \item
    President Trump and his party are using a playbook that aims to
    alarm people about crime in their backyards. It didn't work in 2018,
    but
    \href{https://www.nytimes3xbfgragh.onion/2020/09/08/us/politics/trump-republicans-fear-strategy.html?action=click\&pgtype=Article\&state=default\&region=BELOW_MAIN_CONTENT\&context=storylines_guide}{both
    parties think it could resonate more this year}.
  \end{itemize}
\item ~
  \hypertarget{how-to-win-270}{%
  \subsection{How to Win 270}\label{how-to-win-270}}

  \begin{itemize}
  \item
    Joe Biden and Donald Trump need 270 electoral votes to reach the
    White House. Try building
    \href{https://www.nytimes3xbfgragh.onion/interactive/2020/us/elections/election-states-biden-trump.html?action=click\&pgtype=Article\&state=default\&region=BELOW_MAIN_CONTENT\&context=storylines_guide}{your
    own coalition of battleground states}~to see potential outcomes.
  \end{itemize}
\item ~
  \hypertarget{voting-by-mail}{%
  \subsection{Voting by Mail}\label{voting-by-mail}}

  \begin{itemize}
  \item
    Will you have enough time to vote by mail in your state? Yes, but
    it's risky to procrastinate.
    \href{https://www.nytimes3xbfgragh.onion/interactive/2020/08/31/us/politics/vote-by-mail-deadlines.html?action=click\&pgtype=Article\&state=default\&region=BELOW_MAIN_CONTENT\&context=storylines_guide}{Check
    your state's deadline.}
  \item
    \href{https://www.nytimes3xbfgragh.onion/interactive/2020/us/elections/joe-biden.html?action=click\&pgtype=Article\&state=default\&region=BELOW_MAIN_CONTENT\&context=storylines_guide}{}

    \hypertarget{joe-biden}{%
    \section{Joe Biden}\label{joe-biden}}

    \hypertarget{democrat}{%
    \subsection{Democrat}\label{democrat}}

    \href{https://www.nytimes3xbfgragh.onion/interactive/2020/us/elections/donald-trump.html?action=click\&pgtype=Article\&state=default\&region=BELOW_MAIN_CONTENT\&context=storylines_guide}{}

    \hypertarget{donald-trump}{%
    \section{Donald Trump}\label{donald-trump}}

    \hypertarget{republican}{%
    \subsection{Republican}\label{republican}}
  \end{itemize}
\item
  \hypertarget{keep-up-with-our-coverage}{%
  \subsection{Keep Up With Our
  Coverage}\label{keep-up-with-our-coverage}}

  \begin{itemize}
  \item
    Get an
    \href{https://www.nytimes3xbfgragh.onion/newsletters/politics?action=click\&pgtype=Article\&state=default\&region=BELOW_MAIN_CONTENT\&context=storylines_guide}{email}~recapping
    the day's news
  \item
    Download our mobile app on
    \href{https://apps.apple.com/us/app/nytimes/id284862083?ls=1\&mat_click_id=5c79ae7455014fd1bd66b5610c05b8f2-20191112-16948\&referrer=mat_click_id\%3D5c79ae7455014fd1bd66b5610c05b8f2-20191112-16948\%26link_click_id\%3D722930677036718082}{iOS}~and
    \href{http://a.localytics.com/android?id=com.nytimes.android\&referrer=utm_source\%3Dother_nyt_mobile_web\%26utm_medium\%3DWeb\%2520page\%26utm_term\%3DGeneral\%2520Mobile\%2520Page\%26utm_campaign\%3DNYT\%2520Mobile\%2520General\%2520Page}{Android}~and
    turn on Breaking News and Politics alerts
  \end{itemize}
\end{itemize}

Advertisement

\protect\hyperlink{after-bottom}{Continue reading the main story}

\hypertarget{site-index}{%
\subsection{Site Index}\label{site-index}}

\hypertarget{site-information-navigation}{%
\subsection{Site Information
Navigation}\label{site-information-navigation}}

\begin{itemize}
\tightlist
\item
  \href{https://help.nytimes3xbfgragh.onion/hc/en-us/articles/115014792127-Copyright-notice}{©~2020~The
  New York Times Company}
\end{itemize}

\begin{itemize}
\tightlist
\item
  \href{https://www.nytco.com/}{NYTCo}
\item
  \href{https://help.nytimes3xbfgragh.onion/hc/en-us/articles/115015385887-Contact-Us}{Contact
  Us}
\item
  \href{https://www.nytco.com/careers/}{Work with us}
\item
  \href{https://nytmediakit.com/}{Advertise}
\item
  \href{http://www.tbrandstudio.com/}{T Brand Studio}
\item
  \href{https://www.nytimes3xbfgragh.onion/privacy/cookie-policy\#how-do-i-manage-trackers}{Your
  Ad Choices}
\item
  \href{https://www.nytimes3xbfgragh.onion/privacy}{Privacy}
\item
  \href{https://help.nytimes3xbfgragh.onion/hc/en-us/articles/115014893428-Terms-of-service}{Terms
  of Service}
\item
  \href{https://help.nytimes3xbfgragh.onion/hc/en-us/articles/115014893968-Terms-of-sale}{Terms
  of Sale}
\item
  \href{https://spiderbites.nytimes3xbfgragh.onion}{Site Map}
\item
  \href{https://help.nytimes3xbfgragh.onion/hc/en-us}{Help}
\item
  \href{https://www.nytimes3xbfgragh.onion/subscription?campaignId=37WXW}{Subscriptions}
\end{itemize}
