Sections

SEARCH

\protect\hyperlink{site-content}{Skip to
content}\protect\hyperlink{site-index}{Skip to site index}

\href{https://www.nytimes3xbfgragh.onion/section/politics}{Politics}

\href{https://myaccount.nytimes3xbfgragh.onion/auth/login?response_type=cookie\&client_id=vi}{}

\href{https://www.nytimes3xbfgragh.onion/section/todayspaper}{Today's
Paper}

\href{/section/politics}{Politics}\textbar{}C.I.A. Uncensors Memoir of
F.B.I. Agent Who Protested Torture of Terrorists

\url{https://nyti.ms/3hGqYtc}

\begin{itemize}
\item
\item
\item
\item
\item
\end{itemize}

Advertisement

\protect\hyperlink{after-top}{Continue reading the main story}

Supported by

\protect\hyperlink{after-sponsor}{Continue reading the main story}

\hypertarget{cia-uncensors-memoir-of-fbi-agent-who-protested-torture-of-terrorists}{%
\section{C.I.A. Uncensors Memoir of F.B.I. Agent Who Protested Torture
of
Terrorists}\label{cia-uncensors-memoir-of-fbi-agent-who-protested-torture-of-terrorists}}

Nine years after the C.I.A. blacked out parts of Ali Soufan's book, the
agency has finally allowed a more complete version of his story to be
published.

\includegraphics{https://static01.graylady3jvrrxbe.onion/images/2020/08/29/us/politics/29dc-torture/merlin_137962539_0de37bad-f907-4ecb-90ec-6da36dc7a888-articleLarge.jpg?quality=75\&auto=webp\&disable=upscale}

By \href{https://www.nytimes3xbfgragh.onion/by/charlie-savage}{Charlie
Savage} and
\href{https://www.nytimes3xbfgragh.onion/by/carol-rosenberg}{Carol
Rosenberg}

\begin{itemize}
\item
  Aug. 29, 2020
\item
  \begin{itemize}
  \item
  \item
  \item
  \item
  \item
  \end{itemize}
\end{itemize}

WASHINGTON --- After a group of Qaeda suspects was captured in September
2002, the C.I.A. flew Ali Soufan, an experienced F.B.I. counterterrorism
agent, to Afghanistan to help interrogate them. But when he arrived,
C.I.A. officials abruptly sent word to keep him from the two most
significant new detainees.

Mr. Soufan had made enemies for opposing the C.I.A.'s abusive
interrogation of its first prized prisoner, Abu Zubaydah. He eventually
won permission to question the two detainees after all, but when he
sought to tell the world about those sessions in his 2011 memoir, the
C.I.A. censored much of his account as classified.

Nine years later, and after a lawsuit, the C.I.A. has relented. W.W.
Norton will republish his book next month under the revised title ``The
Black Banners (Declassified): How Torture Derailed the War on Terror
After 9/11.'' Its restored sections add new details to the history of
the United States' early post-Sept. 11 fight against Al Qaeda.

The lesson of the release, Mr. Soufan said in an interview, is ``if you
fight for the truth hard enough, eventually you will win.''

The C.I.A. declined to comment.

\href{https://www.nytimes3xbfgragh.onion/interactive/2020/08/29/us/black-banners-400-405-side-by-side.html}{}

\includegraphics{https://static01.graylady3jvrrxbe.onion/images/2020/08/28/doc-77760-black-banners-400-405-side-by-side-promo/doc-77760-black-banners-400-405-side-by-side-promo-articleLarge.png}

\hypertarget{compare-pages-from-an-uncensored-book-on-investigating-terrorism}{%
\subsection{Compare Pages From an Uncensored Book on Investigating
Terrorism}\label{compare-pages-from-an-uncensored-book-on-investigating-terrorism}}

In the interim, some of what Mr. Soufan sought to discuss has become
public, including in the 2014 declassification of
\href{https://www.nytimes3xbfgragh.onion/interactive/2014/12/09/world/cia-torture-report-document.html}{a
lengthy summary of a landmark Senate study about C.I.A. torture}. But
his book adds richer details about how F.B.I. interrogators manipulated
detainees into divulging information --- using rapport-building tactics,
he says --- and offers a preview of how Mr. Soufan might testify in
\href{https://www.nytimes3xbfgragh.onion/2020/04/17/us/politics/911-trial-guantanamo.html}{the
delayed military commission case} at Guantánamo Bay, Cuba, against five
men accused of conspiring in the Sept. 11, 2001, attacks.

Ramzi bin al-Shibh, a defendant in
\href{https://www.nytimes3xbfgragh.onion/2020/02/03/us/politics/september-11-trial-guantanamo-bay.html}{the
Sept. 11 case}, was one of the two detainees Mr. Soufan was eventually
permitted to question in September 2002 --- for 45 minutes. After that,
the C.I.A. whisked the prisoner to a ``black site'' prison, where its
torture of him rendered his subsequent statements inadmissible as
evidence.

Defense lawyers may try to suppress what Mr. bin al-Shibh told Mr.
Soufan, too, as contaminated by Mr. bin al-Shibh's detention in C.I.A.
custody. As Mr. Soufan tells it, he persuaded the prisoner to talk
without inflicting abuse by leading him to believe that Mr. Soufan
already knew everything anyway and it was in his interest to cooperate.
He ``gave me lots of information about the planning for 9/11,'' like
describing interactions with the operation's planner, Khalid Shaikh
Mohammed, Mr. Soufan wrote.

Mr. Soufan also interrogated a younger brother of Walid bin Attash, who
is charged in the Sept. 11 case with Mr. Mohammed and Mr. bin al-Shibh.
The younger bin Attash, a teenager at the time who is given a pseudonym
in the book, had been captured with Mr. bin al-Shibh and was brought to
Mr. Soufan naked.

Mr. Soufan gave him a towel to cover himself and then revealed that he
had earlier interrogated yet another of his brothers in Yemen, for whom
he had arranged a phone call with their mother. The detainee went from
arrogant and hostile to sobbing on Mr. Soufan's shoulder.

``He completely opened up and gave me specific information about Al
Qaeda's operations in Afghanistan, including who was in charge of which
region and where they were based,'' he wrote.

The C.I.A. allowed Mr. Soufan to republish his book without the black
bars that had concealed those kinds of details after a journalist and
documentary filmmaker
\href{https://pacer-documents.s3.amazonaws.com/119/505795/127123716189.pdf}{sued
the agency}. It then conducted a new review and deemed nearly all of the
contents unclassified.

Shown a copy of the restored memoir, Daniel Jones, who led the research
of millions of pages of C.I.A. files for the Senate study, said that
version aligned with the agency's own contemporaneous records.

``Ali has been, by far, the most consistently accurate person in his
writings about this and his discussions about it,'' Mr. Jones said.
``His version of events most closely resembles the collective record of
the U.S. government.''

One of the most significant restored sections in the book recounts the
agent's interactions with Mr. Zubaydah, who was captured badly wounded.
Mr. Soufan helped keep him alive and then got him talking in a hospital
room, for what the agent described as a conversation that lasted about
10 days.

According to the book, the accidental act of showing Mr. Zubaydah a
wrong photograph while asking about someone else helped Mr. Soufan
uncover a crucial fact: Mr. Mohammed was the planner of the Sept. 11
attacks. Three years later, in 2005, the C.I.A. misled the Justice
Department into believing that big break followed its use of ``enhanced
interrogation techniques'' when seeking
\href{https://www.justice.gov/sites/default/files/olc/legacy/2013/10/21/memo-bradbury2005.pdf}{renewed
legal blessing for its torture program}.

Mr. Soufan also describes how the F.B.I. learned from Mr. Zubaydah about
a Latino terrorist recruit with an American passport who had vague
aspirations --- though no real-world ability, in Mr. Zubaydah's view ---
to detonate a radioactive ``dirty'' bomb in the United States. This led
to the identification and arrest of Jose Padilla. Accusations against
him of a dirty-bomb plot were later dropped; he was
\href{https://www.nytimes3xbfgragh.onion/2007/08/17/us/17padilla.html}{convicted
in a conspiracy} to help Islamic jihadist fighters abroad.

The C.I.A. in 2005 had also credited Mr. Zubaydah's dirty-bomb
disclosure to the use of its ``enhanced'' techniques. Mr. Soufan
disagreed, and the reissued book puts in sharper relief the interagency
struggle over how to interpret the complex circumstances of his
interrogation.

Mr. Soufan was still talking steadily with Mr. Zubaydah in the days
after his capture when the C.I.A. abruptly cut off the F.B.I.'s access,
saying the bureau's agents would never question him again. An agency
psychologist contractor began inflicting suffering and humiliation on
the prisoner --- like forced nudity and prolonged sleep deprivation ---
arguing he could get more and better information that way. These were
the first steps in what would become a more formal C.I.A. torture
program.

By Mr. Soufan's previously censored account, the psychologist obtained
no information at all from Mr. Zubaydah. Frustrated that the flow of
intelligence had stopped, C.I.A. headquarters grudgingly let the F.B.I.
take over again after all. The agents let Mr. Zubaydah sleep, gave him a
towel to cover himself and sought to repair the relationship so they
could pick up where they had left off. In that second round of
questioning, they elicited the information about Mr. Padilla and his
dirty-bomb aspirations.

The psychologist --- James E. Mitchell, who is identified by a pseudonym
in Mr. Soufan's book --- wrote his own C.I.A.-cleared memoir in 2016,
``Enhanced Interrogation: Inside the Minds and Motives of the Islamic
Terrorists Trying To Destroy America.'' He suggested that the F.B.I. got
the information by playing good cop to the C.I.A.'s bad cop.

``You can't discount the role of sleep deprivation in weakening Abu
Zubaydah's resolve and shifting his priorities from protecting
information to getting some rest,'' Dr. Mitchell wrote.

In an interview, however, Mr. Soufan argued that it was not a case of
``good cop, bad cop.'' He said the C.I.A.'s experimentation with torture
was not just ineffective, but interrupted and hindered a professional
interrogation that obtained useful information without abuse. The
uncensored details of the story as he told it in 2011, he said, proves
this point.

But officials at C.I.A. headquarters were determined to put the
psychologists in charge again, and Mr. Soufan decided to leave the black
site. In 2011, the C.I.A. redacted the reason. But the new edition lets
readers know Mr. Soufan's breaking point: A coffin had arrived, and the
C.I.A. planned to lock Mr. Zubaydah in it.

Adam Goldman contributed reporting.

Advertisement

\protect\hyperlink{after-bottom}{Continue reading the main story}

\hypertarget{site-index}{%
\subsection{Site Index}\label{site-index}}

\hypertarget{site-information-navigation}{%
\subsection{Site Information
Navigation}\label{site-information-navigation}}

\begin{itemize}
\tightlist
\item
  \href{https://help.nytimes3xbfgragh.onion/hc/en-us/articles/115014792127-Copyright-notice}{©~2020~The
  New York Times Company}
\end{itemize}

\begin{itemize}
\tightlist
\item
  \href{https://www.nytco.com/}{NYTCo}
\item
  \href{https://help.nytimes3xbfgragh.onion/hc/en-us/articles/115015385887-Contact-Us}{Contact
  Us}
\item
  \href{https://www.nytco.com/careers/}{Work with us}
\item
  \href{https://nytmediakit.com/}{Advertise}
\item
  \href{http://www.tbrandstudio.com/}{T Brand Studio}
\item
  \href{https://www.nytimes3xbfgragh.onion/privacy/cookie-policy\#how-do-i-manage-trackers}{Your
  Ad Choices}
\item
  \href{https://www.nytimes3xbfgragh.onion/privacy}{Privacy}
\item
  \href{https://help.nytimes3xbfgragh.onion/hc/en-us/articles/115014893428-Terms-of-service}{Terms
  of Service}
\item
  \href{https://help.nytimes3xbfgragh.onion/hc/en-us/articles/115014893968-Terms-of-sale}{Terms
  of Sale}
\item
  \href{https://spiderbites.nytimes3xbfgragh.onion}{Site Map}
\item
  \href{https://help.nytimes3xbfgragh.onion/hc/en-us}{Help}
\item
  \href{https://www.nytimes3xbfgragh.onion/subscription?campaignId=37WXW}{Subscriptions}
\end{itemize}
