Sections

SEARCH

\protect\hyperlink{site-content}{Skip to
content}\protect\hyperlink{site-index}{Skip to site index}

\href{https://www.nytimes3xbfgragh.onion/section/science}{Science}

\href{https://myaccount.nytimes3xbfgragh.onion/auth/login?response_type=cookie\&client_id=vi}{}

\href{https://www.nytimes3xbfgragh.onion/section/todayspaper}{Today's
Paper}

\href{/section/science}{Science}\textbar{}Did Something Burp? It Was an
Earthquake

\url{https://nyti.ms/31G81Br}

\begin{itemize}
\item
\item
\item
\item
\item
\end{itemize}

Advertisement

\protect\hyperlink{after-top}{Continue reading the main story}

Supported by

\protect\hyperlink{after-sponsor}{Continue reading the main story}

Trilobites

\hypertarget{did-something-burp-it-was-an-earthquake}{%
\section{Did Something Burp? It Was an
Earthquake}\label{did-something-burp-it-was-an-earthquake}}

Years of observations in central Italy show that more carbon dioxide
percolates through Earth's crust during periods of strong seismic
activity.

\includegraphics{https://static01.graylady3jvrrxbe.onion/images/2020/09/01/science/29TB-QUAKE1/29TB-QUAKE1-articleLarge.jpg?quality=75\&auto=webp\&disable=upscale}

By Katherine Kornei

\begin{itemize}
\item
  Aug. 29, 2020
\item
  \begin{itemize}
  \item
  \item
  \item
  \item
  \item
  \end{itemize}
\end{itemize}

During earthquakes, spider webs of faults open up below ground, allowing
gases deep within our planet to percolate upward. Researchers have now
compiled the first long-term record that shows a relationship between
earthquakes and the release of carbon dioxide gas.

While the amount of carbon dioxide released by tectonic activity is a
pittance compared with the
\href{https://www.epa.gov/ghgemissions/sources-greenhouse-gas-emissions}{billions
of tons} that human activity pumps into the atmosphere each year, the
research
\href{https://advances.sciencemag.org/lookup/doi/10.1126/sciadv.abc2938}{published
Wednesday}in Science Advances sheds light on the planet's
climate-controlling carbon cycle.

``It's modulating Earth's climate on geological time scales,'' said
James Muirhead, a geologist at the University of Auckland in New
Zealand, not involved in the research who praised the data set a team of
Italian scientists had collected.

The results could also potentially pave the way toward forecasting
seismic activity.

The region around Italy's central Apennine Mountains, roughly an hour
east of Rome, is
\href{https://www.bbc.com/news/science-environment-37176502}{riddled
with faults}. Devastating earthquakes have repeatedly struck the area,
including the L'Aquila earthquake in 2009. That temblor, which killed
hundreds of people, made headlines again in 2012 when a judge ruled that
seven Italian earthquake experts were
\href{https://www.nytimes3xbfgragh.onion/2012/10/23/world/europe/italy-convicts-7-for-failure-to-warn-of-quake.html?searchResultPosition=2}{guilty
of manslaughter} because they had failed to warn nearby residents of the
potential risk. The area's seismic activity has been
\href{https://www.nature.com/articles/ngeo2622}{linked to escaping
carbon dioxide}.

Giovanni Chiodini, a geochemist at the National Institute of Geophysics
and Volcanology in Bologna, and his colleagues analyzed the carbon
content of groundwater in the Apennines. From April 2009 through
December 2018, the researchers gathered hundreds of water samples from
36 different springs. They calculated the carbon dioxide concentration
in each sample after subtracting contributions from rainwater and soil.

The researchers estimated that about 1.7 million tons of carbon dioxide
were discharged by tectonic activity in the study area over a decade.
That's roughly equivalent to the carbon dioxide emitted by one volcanic
eruption.

\includegraphics{https://static01.graylady3jvrrxbe.onion/images/2020/08/29/science/29TB-QUAKE2/merlin_176298201_84158114-ec68-495d-a282-6c76e90c6ceb-articleLarge.jpg?quality=75\&auto=webp\&disable=upscale}

The real surprise came when Dr. Chiodini and his collaborators compared
their data with the records of roughly 17,000 earthquakes that had
occurred nearby. The researchers found that seismic activity and carbon
dioxide degassing clearly tracked one another in time --- periods of
high earthquake activity lined up with peaks in gas release. For
example, carbon dioxide concentrations measured in the months following
the L'Aquila earthquake were roughly twice as high as they were in 2013,
a period of low earthquake activity, the team showed. High
concentrations were again measured in September and November 2016, just
a few months after
\href{https://earthquake.usgs.gov/contactus/menlo/seminars/1169}{several
large earthquakes} rocked the region.

This link makes sense, the scientists propose, based on what's going on
miles beneath the surface. The central Apennines sit on top of a
subduction zone where slabs of carbon-rich rock are continuously diving
downward. As those rocks sink, they're exposed to hotter and hotter
conditions until they melt, which releases gases, Dr. Chiodini said.
``One hundred kilometers below the Apennines, you have a huge source of
carbon dioxide,'' he said. As pressure builds underground, the crust
eventually fractures, resulting in earthquakes.

Earthquakes themselves might also trigger more carbon dioxide degassing,
the scientists suggest. That's because ground movement might cause
bubbles of gas to form deep underground, akin to shaking a bottle of
champagne, Dr. Chiodini said. This feedback loop might help explain
aftershocks in the Apennines, the researchers propose.

The big question is what happens first, the earthquake or the carbon
dioxide degassing.

``If the carbon dioxide discharges are leading the large earthquakes in
time, then perhaps these methods could be used alongside other tools as
earthquake indicators,'' Dr. Muirhead said.

But a lot more observations --- with measurements spaced closer together
in time --- would be needed, he said, and even then it would still be a
fraught endeavor to forecast seismic activity. ``The uncertainties are
still too high and the consequences too huge,'' he said.

Advertisement

\protect\hyperlink{after-bottom}{Continue reading the main story}

\hypertarget{site-index}{%
\subsection{Site Index}\label{site-index}}

\hypertarget{site-information-navigation}{%
\subsection{Site Information
Navigation}\label{site-information-navigation}}

\begin{itemize}
\tightlist
\item
  \href{https://help.nytimes3xbfgragh.onion/hc/en-us/articles/115014792127-Copyright-notice}{©~2020~The
  New York Times Company}
\end{itemize}

\begin{itemize}
\tightlist
\item
  \href{https://www.nytco.com/}{NYTCo}
\item
  \href{https://help.nytimes3xbfgragh.onion/hc/en-us/articles/115015385887-Contact-Us}{Contact
  Us}
\item
  \href{https://www.nytco.com/careers/}{Work with us}
\item
  \href{https://nytmediakit.com/}{Advertise}
\item
  \href{http://www.tbrandstudio.com/}{T Brand Studio}
\item
  \href{https://www.nytimes3xbfgragh.onion/privacy/cookie-policy\#how-do-i-manage-trackers}{Your
  Ad Choices}
\item
  \href{https://www.nytimes3xbfgragh.onion/privacy}{Privacy}
\item
  \href{https://help.nytimes3xbfgragh.onion/hc/en-us/articles/115014893428-Terms-of-service}{Terms
  of Service}
\item
  \href{https://help.nytimes3xbfgragh.onion/hc/en-us/articles/115014893968-Terms-of-sale}{Terms
  of Sale}
\item
  \href{https://spiderbites.nytimes3xbfgragh.onion}{Site Map}
\item
  \href{https://help.nytimes3xbfgragh.onion/hc/en-us}{Help}
\item
  \href{https://www.nytimes3xbfgragh.onion/subscription?campaignId=37WXW}{Subscriptions}
\end{itemize}
