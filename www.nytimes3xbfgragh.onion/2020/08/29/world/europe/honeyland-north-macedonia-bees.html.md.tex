Sections

SEARCH

\protect\hyperlink{site-content}{Skip to
content}\protect\hyperlink{site-index}{Skip to site index}

\href{https://www.nytimes3xbfgragh.onion/section/world/europe}{Europe}

\href{https://myaccount.nytimes3xbfgragh.onion/auth/login?response_type=cookie\&client_id=vi}{}

\href{https://www.nytimes3xbfgragh.onion/section/todayspaper}{Today's
Paper}

\href{/section/world/europe}{Europe}\textbar{}Film Crew Spent 3 Years in
Remote Balkan Hamlet. Will They Ever Leave?

\url{https://nyti.ms/2DaCMVD}

\begin{itemize}
\item
\item
\item
\item
\item
\end{itemize}

Advertisement

\protect\hyperlink{after-top}{Continue reading the main story}

Supported by

\protect\hyperlink{after-sponsor}{Continue reading the main story}

North Macedonia Dispatch

\hypertarget{film-crew-spent-3-years-in-remote-balkan-hamlet-will-they-ever-leave}{%
\section{Film Crew Spent 3 Years in Remote Balkan Hamlet. Will They Ever
Leave?}\label{film-crew-spent-3-years-in-remote-balkan-hamlet-will-they-ever-leave}}

Nominated for two Oscars, ``Honeyland'' charted the tensions between a
hermitic beekeeper and her disorderly neighbors. Now the filmmakers are
struggling to disentangle themselves from their subjects.

\includegraphics{https://static01.graylady3jvrrxbe.onion/images/2020/08/07/world/00northmacedonia-dispatch1/merlin_174764457_12fce047-45da-40f6-87de-0f5baee8b9db-articleLarge.jpg?quality=75\&auto=webp\&disable=upscale}

\href{https://www.nytimes3xbfgragh.onion/by/patrick-kingsley}{\includegraphics{https://static01.graylady3jvrrxbe.onion/images/2018/10/15/multimedia/author-patrick-kingsley/author-patrick-kingsley-thumbLarge.png}}

By \href{https://www.nytimes3xbfgragh.onion/by/patrick-kingsley}{Patrick
Kingsley}

\begin{itemize}
\item
  Aug. 29, 2020
\item
  \begin{itemize}
  \item
  \item
  \item
  \item
  \item
  \end{itemize}
\end{itemize}

BEKIRLIJA, North Macedonia --- When the producer and two directors of
``Honeyland'' returned to the setting of their documentary in North
Macedonia for the first time since it
\href{https://www.nytimes3xbfgragh.onion/2020/01/24/movies/honeyland-oscars.html}{earned
two Oscar nominations in February}, something fundamental had changed.

The film chronicles the tensions between~Hatidze Muratova, a local
beekeeper, and a farmer in the remote hamlet of Bekirlija.~Squeezed
between two rocky hills and circled by imperial eagles, the village was
still reachable only in an off-road vehicle, via a steep, rutted track.
Most of the houses were still in ruins, slowly sliding into the
undergrowth.

And Hatidze Muratova, one of the hamlet's last inhabitants and the star
of
``\href{https://www.nytimes3xbfgragh.onion/2019/07/25/movies/honeyland-review.html}{Honeyland},''
was still waiting for the filmmakers with a smile and a strong coffee.

But Ms. Muratova's cramped, dark living room, site of the movie's most
moving scene, no longer felt lived in.

``Now this place and these people are different,'' said a wistful
Ljubomir Stefanov, one of the film's two co-directors, sitting in Ms.
Muratova's garden. ``And I can feel that she feels that this is not her
only home.''

That was largely thanks to Mr. Stefanov and his fellow filmmakers. Using
prize money won by the film, he and his colleagues had found her a new
house in Dorfulija, a larger and wealthier village about half an hour's
drive away. She now divides her time between the two villages.

Serbia

50 miles

KOSOVO

Bulgaria

Skopje

Dorfulija

Bekirlija

North Macedonia

GREECE

ALBANIA

By The New York Times

And that change speaks to a wider ethical conundrum that Mr. Stefanov
and his colleagues have grappled with since finishing filming --- one
that
has\href{https://www.nytimes3xbfgragh.onion/2010/11/07/movies/07lives.html}{long
troubled documentary filmmakers}. As observers, should they ever help
their subjects? And as humans, how could they ever not?

\includegraphics{https://static01.graylady3jvrrxbe.onion/images/2020/08/07/world/00northmacedonia-dispatch2/merlin_174766437_a5bc87af-74cd-482a-97ae-225c9fcfb23c-articleLarge.jpg?quality=75\&auto=webp\&disable=upscale}

Image

Calm and gentle, Ms. Muratova has a deep and respectful relationship
with nature, treating her bees almost as collaborators.Credit...Laura
Boushnak for The New York Times

Some documentary crews maintain a professional distance even after
filming stops.

``But we decided to break that rule,'' said Atanas Georgiev, the film's
producer.

The film depicts how Ms. Muratova and Hussein Sam, a seminomadic farmer
tried to coexist in one of the poorest pockets of North Macedonia.

At the time of filming, Ms. Muratova lived year-round in Bekirlija,
while Mr. Sam's chaotic family usually only visited during the summer,
disrupting Ms. Muratova's quiet existence.

The film was shot on a shoestring budget, but grossed a little over \$1
million, and turned its makers into darlings of the documentary circuit.
It also made Ms. Muratova perhaps the world's most famous beekeeper.

It won three prizes at the prestigious Sundance Film Festival and was
nominated for best documentary and best international feature at this
year's Academy Awards. A.O. Scott, the co-chief film critic of The New
York Times,
\href{https://www.nytimes3xbfgragh.onion/2019/12/04/movies/best-films.html}{named
it the No. 1 movie of 2019}.

In a simpler world, Mr. Georgiev, Mr. Stefanov and his co-director,
Tamara Kotevska, would be basking in their newfound success, and
focusing on new projects.

But the ``Honeyland'' directors, together with two cameramen, spent
three years, on and off, visiting these families.

That intense process ultimately shoved Ms. Muratova and Mr. Sam,
vulnerable people who had never previously even been to a cinema, into
the media glare. And the complexities of this transition created a clash
between the filmmakers' professional duties as cleareyed observers and
their subjects' emotional expectations of them as humans and friends.

Now the filmmakers find themselves unable to leave entirely --- serving
as mediators to, and occasionally protagonists in, the local tensions to
which they once only bore witness.

Image

Mr. Georgiev and Ms. Muratova visiting her new house in Dorfulija, which
is still under construction.Credit...Laura Boushnak for The New York
Times

Image

Ms.~Kotevska with~Ms. Muratova in her new house.Credit...Laura Boushnak
for The New York Times

``For the film crew,'' said Ms. Muratova, ``it was more demanding to
deal with us after the film than during the filming itself.''

Both members of North Macedonia's Turkish minority, Ms. Muratova, 56,
and Mr. Sam, 70, have similar roots in rural poverty, but markedly
different approaches to life.

Calm and gentle, Ms. Muratova has a deep and respectful relationship
with nature, treating her bees almost as collaborators. Mr. Sam has a
more haphazard way with his cows, viewing them almost as antagonists.

Ms. Muratova never married, while Mr. Sam and his wife, Ljutvie, have
eight rambunctious children.

In the film, relations between Ms. Muratova and Mr. Sam were bad; Mr.
Sam ignored Ms. Muratova's advice about how to start his own bee colony,
leading his bees to attack hers and ruining Ms. Muratova's entire
livelihood.

But in relative terms, the period depicted in the film proved to be a
rare period of calm in a conflict that had begun long before filming
started, and which has escalated since it finished. During
postproduction, the two became locked in a dispute about a communal well
in Bekirlija. Mr. Sam wanted its water for his cows, while Ms. Muratova
said it was only for human use.

Then there was a legal battle over an incident that predated the film,
in which Ms. Muratova was attacked by Mr. Sam's dogs.

Sucked into the dispute, the crew tried to stay neutral by providing
legal assistance to both parties, and they mediated an agreement by
which Ms. Muratova would withdraw her complaint in exchange for Mr.
Sam's promise to abide by a set of principles about his future behavior.

To lessen the pressure on himself in refereeing the relationship, Mr.
Georgiev created a foundation that works with the families independently
of the crew. A volunteer social worker now helps both families overcome
a never-ending list of logistical and social challenges, including
setting up bank accounts and enrolling them in social security.

``It's an avalanche'' of issues, said Julijana Daskalov, the
foundation's program manager.

Image

The Sam family home in~Dorfulija.Credit...Laura Boushnak for The New
York Times

Image

Ms. Muratova, pouring some honey for her guests at her home in
Bekirlija.Credit...Laura Boushnak for The New York Times

Even with this help, Mr. Georgiev is still often drawn into the
disputes. When he and the two directors visited in July, they were
immediately overwhelmed by a new barrage of issues.

Ms. Muratova had lost her new house key, so the producer had to find the
workman with the spare. Inside the house, the taps were dry, so Mr.
Georgiev called the mayor to reconnect the water. Then Mr. Sam wanted
help with a grant application, and griped about the problem with the
well. Meanwhile, someone had pilfered Ms. Muratova's honey, and she
blamed Mr. Sam.

``It's impossible!'' sighed Mr. Georgiev. ``We are filmmakers, not
social workers.''

Intervening is often thankless anyway.

Moments before leaving, Mr. Sam pulled him aside to ask why he hadn't
been in contact so much during the coronavirus lockdown.

``You haven't been calling,'' Mr. Sam said. ``I thought you'd abandoned
us.''

Yet Mr. Georgiev has been anything but absent. In fact, the social
worker felt he had made himself too available. In addition to finding
Ms. Muratova a new house, he and his team bought Mr. Sam a new truck and
fixed his family's chimney.

This kind of involvement is partly a self-interested act, Mr. Georgiev
said --- a means of both salving the crew's conscience for deriving
professional benefit from the lives of both Mr. Sam and Ms. Muratova,
and warding off public criticism.

But it is also ``kind of a payback,'' Mr. Georgiev said. ``Usually you
don't interfere with your protagonists --- but as soon as we realized
`Honeyland' would be very successful, we thought we had to do
something.''

Still, the transformation of Ms. Muratova's life is not necessarily
something to mourn, Mr. Stefanov said.

``Life is not an infinite process --- it has phases,'' he said. ``And
this is her wish.''

And even with her newfound fame, Ms. Muratova said she still remained
true to her vocation. At her old home in Bekirlija, she proudly unveiled
her latest bowl of liquid honey. But she refused a request to open up
her hives, hidden in the crags of a nearby mountain, for fear the heat
of the noon sun would harm the honeycombs.

``Even if I'm in a film,'' she said, ``I'm still going to take care of
my bees.''

Image

The road leading to Bekirlija.Credit...Laura Boushnak for The New York
Times

Advertisement

\protect\hyperlink{after-bottom}{Continue reading the main story}

\hypertarget{site-index}{%
\subsection{Site Index}\label{site-index}}

\hypertarget{site-information-navigation}{%
\subsection{Site Information
Navigation}\label{site-information-navigation}}

\begin{itemize}
\tightlist
\item
  \href{https://help.nytimes3xbfgragh.onion/hc/en-us/articles/115014792127-Copyright-notice}{©~2020~The
  New York Times Company}
\end{itemize}

\begin{itemize}
\tightlist
\item
  \href{https://www.nytco.com/}{NYTCo}
\item
  \href{https://help.nytimes3xbfgragh.onion/hc/en-us/articles/115015385887-Contact-Us}{Contact
  Us}
\item
  \href{https://www.nytco.com/careers/}{Work with us}
\item
  \href{https://nytmediakit.com/}{Advertise}
\item
  \href{http://www.tbrandstudio.com/}{T Brand Studio}
\item
  \href{https://www.nytimes3xbfgragh.onion/privacy/cookie-policy\#how-do-i-manage-trackers}{Your
  Ad Choices}
\item
  \href{https://www.nytimes3xbfgragh.onion/privacy}{Privacy}
\item
  \href{https://help.nytimes3xbfgragh.onion/hc/en-us/articles/115014893428-Terms-of-service}{Terms
  of Service}
\item
  \href{https://help.nytimes3xbfgragh.onion/hc/en-us/articles/115014893968-Terms-of-sale}{Terms
  of Sale}
\item
  \href{https://spiderbites.nytimes3xbfgragh.onion}{Site Map}
\item
  \href{https://help.nytimes3xbfgragh.onion/hc/en-us}{Help}
\item
  \href{https://www.nytimes3xbfgragh.onion/subscription?campaignId=37WXW}{Subscriptions}
\end{itemize}
