Sections

SEARCH

\protect\hyperlink{site-content}{Skip to
content}\protect\hyperlink{site-index}{Skip to site index}

\href{https://www.nytimes3xbfgragh.onion/section/politics}{Politics}

\href{https://myaccount.nytimes3xbfgragh.onion/auth/login?response_type=cookie\&client_id=vi}{}

\href{https://www.nytimes3xbfgragh.onion/section/todayspaper}{Today's
Paper}

\href{/section/politics}{Politics}\textbar{}Shift on Election Briefings
Could Create an Information Gap for Voters

\url{https://nyti.ms/34O40fV}

\begin{itemize}
\item
\item
\item
\item
\item
\end{itemize}

\begin{itemize}
\item
  \href{https://www.nytimes3xbfgragh.onion/live/2020/09/09/us/trump-vs-biden?action=click\&pgtype=Article\&state=default\&region=TOP_BANNER\&context=storylines_menu}{Election
  Updates}
\item
  \href{https://www.nytimes3xbfgragh.onion/interactive/2020/09/08/us/elections/results-new-hampshire-primary-elections.html?action=click\&pgtype=Article\&state=default\&region=TOP_BANNER\&context=storylines_menu}{New
  Hampshire Results}
\item
  \href{https://www.nytimes3xbfgragh.onion/interactive/2020/us/elections/election-states-biden-trump.html?action=click\&pgtype=Article\&state=default\&region=TOP_BANNER\&context=storylines_menu}{Paths
  to 270}
\item
  \href{https://www.nytimes3xbfgragh.onion/interactive/2020/08/31/us/politics/vote-by-mail-deadlines.html?action=click\&pgtype=Article\&state=default\&region=TOP_BANNER\&context=storylines_menu}{Voting
  by Mail}
\item
  \href{https://www.nytimes3xbfgragh.onion/interactive/2019/us/elections/2020-presidential-election-calendar.html?action=click\&pgtype=Article\&state=default\&region=TOP_BANNER\&context=storylines_menu}{Key
  Dates}
\item
  \href{https://www.nytimes3xbfgragh.onion/newsletters/politics?action=click\&pgtype=Article\&state=default\&region=TOP_BANNER\&context=storylines_menu}{Politics
  Newsletter}
\end{itemize}

Advertisement

\protect\hyperlink{after-top}{Continue reading the main story}

Supported by

\protect\hyperlink{after-sponsor}{Continue reading the main story}

News Analysis

\hypertarget{shift-on-election-briefings-could-create-an-information-gap-for-voters}{%
\section{Shift on Election Briefings Could Create an Information Gap for
Voters}\label{shift-on-election-briefings-could-create-an-information-gap-for-voters}}

The elimination of in-person election security briefings to Congress
could leave the public with a diminished understanding of the threats
facing the election as it enters a critical phase.

\includegraphics{https://static01.graylady3jvrrxbe.onion/images/2020/08/30/us/politics/30dc-intel/merlin_172211034_1dd03939-263b-44df-9d3d-fe8a5e757616-articleLarge.jpg?quality=75\&auto=webp\&disable=upscale}

\href{https://www.nytimes3xbfgragh.onion/by/david-e-sanger}{\includegraphics{https://static01.graylady3jvrrxbe.onion/images/2018/10/03/multimedia/author-david-e-sanger/author-david-e-sanger-thumbLarge.png}}\href{https://www.nytimes3xbfgragh.onion/by/julian-e-barnes}{\includegraphics{https://static01.graylady3jvrrxbe.onion/images/2019/12/13/reader-center/author-julian-barnes/author-julian-barnes-thumbLarge.png}}

By \href{https://www.nytimes3xbfgragh.onion/by/david-e-sanger}{David E.
Sanger} and
\href{https://www.nytimes3xbfgragh.onion/by/julian-e-barnes}{Julian E.
Barnes}

\begin{itemize}
\item
  Aug. 30, 2020
\item
  \begin{itemize}
  \item
  \item
  \item
  \item
  \item
  \end{itemize}
\end{itemize}

The decision by the nation's top intelligence official
\href{https://www.nytimes3xbfgragh.onion/2020/08/29/us/politics/election-security-intelligence-briefings-congress.html}{to
halt classified, in-person briefings to Congress} about foreign
interference in a
\href{https://www.nytimes3xbfgragh.onion/live/2020/09/01/us/trump-vs-biden}{presidential
election} that is just nine weeks away exposes the fundamental tension
about who needs to know this information: just the president, or the
voters whose election infrastructure, and minds, are the target of the
hacking?

The intelligence agencies are built to funnel a stream of secret
findings to the president, his staff and the military to inform their
actions.

President Trump has made it abundantly clear that he does not believe
the overwhelming evidence, detailed in thousands of pages of
investigative reports
\href{https://www.nytimes3xbfgragh.onion/2020/08/18/us/politics/senate-intelligence-russian-interference-report.html}{by
the Republican-led Senate Intelligence Committee} and indictments of
Russian intelligence officers by his own Justice Department, that Moscow
interfered in the 2016 election, and is at it again.

One of the bitter lessons of the last election is that intelligence
about hacking into voter registration systems and the spreading of
disinformation must be handled in a very different way. Those defending
against misinformation include state and city election officials;
Facebook, Twitter and Google; and voters themselves, who need to know
who is generating or amplifying the messages they see running across
their screens.

And if they do not understand the threat assessments, they will enter
the most critical phase of the election --- those vulnerable weeks when
everything counts and adversaries have a brief window to take their best
shot --- without understanding the battle space.

So it is no surprise that as soon as word leaked about the decision by
the director of national intelligence, John Ratcliffe, to give Congress
only written updates about the latest intelligence, former Vice
President Joseph R. Biden Jr. led the parade of accusations that Mr.
Trump is paving the way for a second round of election interference.

``Nothing is more important than the security and integrity of our
elections,'' Mr. Biden, the Democratic nominee, said in a statement on
Saturday. ``And we know that President Trump is unwilling to take action
to protect them. That leaves Congress as the best defender of our
democracy.''

``There can be only one conclusion: President Trump is hoping Vladimir
Putin will once more boost his candidacy and cover his horrific failures
to lead our country through the multiple crises we are facing,'' Mr.
Biden added. ``And he does not want the American people to know the
steps Vladimir Putin is taking to help Trump get re-elected or why Putin
is eager to intervene, because Donald Trump's foreign policy has been a
gift to the Kremlin.''

Whether or not Mr. Biden's accusation of malicious intent is correct,
the White House is once again seeking to marginalize Congress and the
committees that are charged with overseeing, and funding, the \$80
billion intelligence enterprise.

Intelligence officials sorting through the complexities of the 2020
intelligence note that the real danger arises from the swirl of
conflicting signals about how the
\href{https://www.nytimes3xbfgragh.onion/2020/07/24/us/politics/election-interference-russia-china-iran.html?searchResultPosition=8}{Russians,
the Chinese and the Iranians} are writing new playbooks for 2020.

Interpreting their intentions --- and their feints --- would be hard
enough in normal times.

Mr. Ratcliffe, a Trump partisan who is new to his job, is discovering
that he does not have a monopoly on the intelligence. Every week dozens
of cybersecurity firms issue reports that sift through evidence of
malware and disinformation.

So Mr. Trump and Mr. Ratcliffe will not be stopping the flow of data
about what foreign actors are up to, or whether they are succeeding.
They will just be pulling the U.S. intelligence services back from
publicly assessing what is important and what is background noise --- at
the most critical moment in a highly contested, highly divisive race
that the president himself declared a month ago will be ``the most
rigged election'' in history.

(Mr. Trump was referring to the surge in mail-in ballots, which he
claimed, without evidence, that Russia and China could gain access to.
Intelligence agencies last week
\href{https://www.nytimes3xbfgragh.onion/2020/08/26/us/politics/mail-in-voting-foreign-intervention.html}{contradicted
those claims}.)

\href{https://www.nytimes3xbfgragh.onion/news-event/2020-election}{Election
2020 ›}

\hypertarget{live-updates}{%
\subsection{\texorpdfstring{\href{https://www.nytimes3xbfgragh.onion/live/2020/09/09/us/trump-vs-biden}{Live
Updates}}{Live Updates}}\label{live-updates}}

\href{https://www.nytimes3xbfgragh.onion/live/2020/09/09/us/trump-vs-biden\#biden-looks-to-keep-focus-on-trumps-reported-remarks-disparaging-american-soldiers}{}

Sept. 9, 2020, 12:49 p.m. ET

\href{https://www.nytimes3xbfgragh.onion/live/2020/09/09/us/trump-vs-biden\#biden-looks-to-keep-focus-on-trumps-reported-remarks-disparaging-american-soldiers}{Biden
looks to keep focus on Trump's reported remarks disparaging American
soldiers.}\href{https://www.nytimes3xbfgragh.onion/live/2020/09/09/us/trump-vs-biden\#trump-admitted-to-bob-woodward-that-he-intentionally-downplayed-the-threat-of-the-coronavirus}{}

Sept. 9, 2020, 12:27 p.m. ET

\href{https://www.nytimes3xbfgragh.onion/live/2020/09/09/us/trump-vs-biden\#trump-admitted-to-bob-woodward-that-he-intentionally-downplayed-the-threat-of-the-coronavirus}{Trump
admitted to Bob Woodward that he intentionally downplayed the threat of
the
coronavirus.}\href{https://www.nytimes3xbfgragh.onion/live/2020/09/09/us/trump-vs-biden\#trumps-decision-to-go-maskless-at-a-north-carolina-rally-concerns-dr-fauci}{}

Sept. 9, 2020, 11:40 a.m. ET

\href{https://www.nytimes3xbfgragh.onion/live/2020/09/09/us/trump-vs-biden\#trumps-decision-to-go-maskless-at-a-north-carolina-rally-concerns-dr-fauci}{Trump's
decision to go maskless at a North Carolina rally concerns Dr. Fauci.}

Until a few days ago, there seemed to be a movement inside the
intelligence agencies to say a bit more about election threats --- but
not much more. Under pressure from congressional Democrats, who demanded
more public disclosures about Russian activity, intelligence officials
this month
\href{https://www.nytimes3xbfgragh.onion/2020/08/07/us/politics/russia-china-trump-biden-election-interference.html}{issued
a new public warning} about Moscow's interference. But they also
cautioned that China and Iran were coming in on Mr. Biden's behalf, even
though their activities so far have been marginal at best.

Meanwhile, the director of the National Security Agency, who also serves
as commander of Cyber Command, the vast military operation designed to
push back in the daily cyberconflict among nations, published
\href{https://www.foreignaffairs.com/articles/united-states/2020-08-25/cybersecurity}{a
vaguely worded essay} in Foreign Affairs magazine reminding American
rivals that the United States was pursuing a new strategy of
``persistent engagement'' deep inside adversary computer networks ---
but he was not specific about the threats.

``This is a new concept for the intelligence community,'' Senator Angus
King, a Maine independent who led
\href{https://www.nytimes3xbfgragh.onion/2020/03/11/us/politics/congress-cyber-solarium.html?searchResultPosition=1}{a
lengthy congressional study into enhancing the nation's cyberdefenses},
said in an interview on Saturday. ``Their fallback position is always
secret. And their second fallback position is that we only give this to
the national security apparatus. Maybe we will give it to Congress. We
will never give it to the American people unless someone demands it.''

``But I argue the American people are the decision makers and they are
entitled to the information and it has to be given to them in a form
that is useful and thoroughly examined,'' he added, noting that ``a cold
written statement'' does not meet that standard because those statements
can be watered down to fit Mr. Trump's agenda.

In fact, the challenge is that American intelligence analysts do not
write for the public. They employ code words understandable to those who
read their reports, but which need translation for a public that is
struggling to comprehend spear-phishing and ransomware and cannot agree
on what constitutes disinformation. The result is that even the
best-intentioned warnings often fail at their purpose.

That is one reason Democrats are pressing to interrogate the analysts
and force them to state their conclusions in plain terms. Mr. Ratcliffe
insists that is too risky.

In an appearance on Fox News on Sunday, Mr. Ratcliffe said he had
decided to end in-person briefings on election security because, a few
weeks ago, ``within minutes of one of those briefings ending, a number
of members of Congress went to a number of different publications and
leaked classified information, again, for political purposes to create a
narrative that simply isn't true: that somehow Russia is a greater
national security threat than China.''

Mr. Ratcliffe insisted there was ``a pandemic of information being
leaked out of the intelligence community, and I'm going to take the
measures to make sure that that stops.''

Mr. King disputes that any sources and methods were compromised, and
several federal officials agreed.

What Mr. Ratcliffe ignored was the risk ahead. If the complaint about
the intelligence agencies under President Barack Obama in 2016 was that
they had their radar off and never saw the Russians coming until it was
too late, the concern in 2020 may be a deliberate failure to
communicate.

\hypertarget{our-2020-election-guide}{%
\section{Our 2020 Election Guide}\label{our-2020-election-guide}}

Updated ~Sept. 9, 2020

\begin{center}\rule{0.5\linewidth}{\linethickness}\end{center}

\begin{itemize}
\item ~
  \hypertarget{the-latest}{%
  \subsection{The Latest}\label{the-latest}}

  \begin{itemize}
  \item
    Joe Biden heads today to Michigan, a battleground state where
    President Trump has resumed advertising ahead of a visit there on
    Thursday.
    \href{https://www.nytimes3xbfgragh.onion/live/2020/09/09/us/trump-vs-biden?action=click\&pgtype=Article\&state=default\&region=BELOW_MAIN_CONTENT\&context=storylines_guide}{Read
    live updates}.
  \end{itemize}
\item ~
  \hypertarget{how-to-win-270}{%
  \subsection{How to Win 270}\label{how-to-win-270}}

  \begin{itemize}
  \item
    Joe Biden and Donald Trump need 270 electoral votes to reach the
    White House. Try building
    \href{https://www.nytimes3xbfgragh.onion/interactive/2020/us/elections/election-states-biden-trump.html?action=click\&pgtype=Article\&state=default\&region=BELOW_MAIN_CONTENT\&context=storylines_guide}{your
    own coalition of battleground states}~to see potential outcomes.
  \end{itemize}
\item ~
  \hypertarget{voting-by-mail}{%
  \subsection{Voting by Mail}\label{voting-by-mail}}

  \begin{itemize}
  \item
    Will you have enough time to vote by mail in your state? Yes, but
    it's risky to procrastinate.
    \href{https://www.nytimes3xbfgragh.onion/interactive/2020/08/31/us/politics/vote-by-mail-deadlines.html?action=click\&pgtype=Article\&state=default\&region=BELOW_MAIN_CONTENT\&context=storylines_guide}{Check
    your state's deadline.}
  \item
    \href{https://www.nytimes3xbfgragh.onion/interactive/2020/us/elections/joe-biden.html?action=click\&pgtype=Article\&state=default\&region=BELOW_MAIN_CONTENT\&context=storylines_guide}{}

    \hypertarget{joe-biden}{%
    \section{Joe Biden}\label{joe-biden}}

    \hypertarget{democrat}{%
    \subsection{Democrat}\label{democrat}}

    \href{https://www.nytimes3xbfgragh.onion/interactive/2020/us/elections/donald-trump.html?action=click\&pgtype=Article\&state=default\&region=BELOW_MAIN_CONTENT\&context=storylines_guide}{}

    \hypertarget{donald-trump}{%
    \section{Donald Trump}\label{donald-trump}}

    \hypertarget{republican}{%
    \subsection{Republican}\label{republican}}
  \end{itemize}
\item
  \hypertarget{keep-up-with-our-coverage}{%
  \subsection{Keep Up With Our
  Coverage}\label{keep-up-with-our-coverage}}

  \begin{itemize}
  \item
    Get an
    \href{https://www.nytimes3xbfgragh.onion/newsletters/politics?action=click\&pgtype=Article\&state=default\&region=BELOW_MAIN_CONTENT\&context=storylines_guide}{email}~recapping
    the day's news
  \item
    Download our mobile app on
    \href{https://apps.apple.com/us/app/nytimes/id284862083?ls=1\&mat_click_id=5c79ae7455014fd1bd66b5610c05b8f2-20191112-16948\&referrer=mat_click_id\%3D5c79ae7455014fd1bd66b5610c05b8f2-20191112-16948\%26link_click_id\%3D722930677036718082}{iOS}~and
    \href{http://a.localytics.com/android?id=com.nytimes.android\&referrer=utm_source\%3Dother_nyt_mobile_web\%26utm_medium\%3DWeb\%2520page\%26utm_term\%3DGeneral\%2520Mobile\%2520Page\%26utm_campaign\%3DNYT\%2520Mobile\%2520General\%2520Page}{Android}~and
    turn on Breaking News and Politics alerts
  \end{itemize}
\end{itemize}

Advertisement

\protect\hyperlink{after-bottom}{Continue reading the main story}

\hypertarget{site-index}{%
\subsection{Site Index}\label{site-index}}

\hypertarget{site-information-navigation}{%
\subsection{Site Information
Navigation}\label{site-information-navigation}}

\begin{itemize}
\tightlist
\item
  \href{https://help.nytimes3xbfgragh.onion/hc/en-us/articles/115014792127-Copyright-notice}{©~2020~The
  New York Times Company}
\end{itemize}

\begin{itemize}
\tightlist
\item
  \href{https://www.nytco.com/}{NYTCo}
\item
  \href{https://help.nytimes3xbfgragh.onion/hc/en-us/articles/115015385887-Contact-Us}{Contact
  Us}
\item
  \href{https://www.nytco.com/careers/}{Work with us}
\item
  \href{https://nytmediakit.com/}{Advertise}
\item
  \href{http://www.tbrandstudio.com/}{T Brand Studio}
\item
  \href{https://www.nytimes3xbfgragh.onion/privacy/cookie-policy\#how-do-i-manage-trackers}{Your
  Ad Choices}
\item
  \href{https://www.nytimes3xbfgragh.onion/privacy}{Privacy}
\item
  \href{https://help.nytimes3xbfgragh.onion/hc/en-us/articles/115014893428-Terms-of-service}{Terms
  of Service}
\item
  \href{https://help.nytimes3xbfgragh.onion/hc/en-us/articles/115014893968-Terms-of-sale}{Terms
  of Sale}
\item
  \href{https://spiderbites.nytimes3xbfgragh.onion}{Site Map}
\item
  \href{https://help.nytimes3xbfgragh.onion/hc/en-us}{Help}
\item
  \href{https://www.nytimes3xbfgragh.onion/subscription?campaignId=37WXW}{Subscriptions}
\end{itemize}
