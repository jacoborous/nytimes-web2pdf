Sections

SEARCH

\protect\hyperlink{site-content}{Skip to
content}\protect\hyperlink{site-index}{Skip to site index}

\href{https://www.nytimes3xbfgragh.onion/section/politics}{Politics}

\href{https://myaccount.nytimes3xbfgragh.onion/auth/login?response_type=cookie\&client_id=vi}{}

\href{https://www.nytimes3xbfgragh.onion/section/todayspaper}{Today's
Paper}

\href{/section/politics}{Politics}\textbar{}Postal Service Warns States
It May Not Meet Mail-In Ballot Deadlines

\url{https://nyti.ms/2Y4WskV}

\begin{itemize}
\item
\item
\item
\item
\item
\item
\end{itemize}

\begin{itemize}
\item
  \href{https://www.nytimes3xbfgragh.onion/2020/09/12/us/politics/biden-trump-poll-wisconsin-minnesota.html?action=click\&pgtype=Article\&state=default\&region=TOP_BANNER\&context=storylines_menu}{New
  York Times Poll}
\item
  \href{https://www.nytimes3xbfgragh.onion/interactive/2020/us/elections/election-states-biden-trump.html?action=click\&pgtype=Article\&state=default\&region=TOP_BANNER\&context=storylines_menu}{Paths
  to 270}
\item
  \href{https://www.nytimes3xbfgragh.onion/interactive/2019/us/elections/2020-presidential-election-calendar.html?action=click\&pgtype=Article\&state=default\&region=TOP_BANNER\&context=storylines_menu}{Voting
  Deadlines}
\item
  \href{https://www.nytimes3xbfgragh.onion/interactive/2020/08/31/us/politics/vote-by-mail-deadlines.html?action=click\&pgtype=Article\&state=default\&region=TOP_BANNER\&context=storylines_menu}{Voting
  by Mail}
\item
  \href{https://www.nytimes3xbfgragh.onion/newsletters/politics?action=click\&pgtype=Article\&state=default\&region=TOP_BANNER\&context=storylines_menu}{Politics
  Newsletter}
\end{itemize}

Advertisement

\protect\hyperlink{after-top}{Continue reading the main story}

Supported by

\protect\hyperlink{after-sponsor}{Continue reading the main story}

\hypertarget{postal-service-warns-states-it-may-not-meet-mail-in-ballot-deadlines}{%
\section{Postal Service Warns States It May Not Meet Mail-In Ballot
Deadlines}\label{postal-service-warns-states-it-may-not-meet-mail-in-ballot-deadlines}}

The agency suggested allowing at least 15 days to request a ballot,
fueling further criticism that it was making it harder to vote during
the pandemic.

\includegraphics{https://static01.graylady3jvrrxbe.onion/images/2020/08/14/us/politics/14dc-postal/merlin_175321878_e144de87-0930-452b-96cc-8c4e28cc0ef4-articleLarge.jpg?quality=75\&auto=webp\&disable=upscale}

By \href{https://www.nytimes3xbfgragh.onion/by/luke-broadwater}{Luke
Broadwater},
\href{https://www.nytimes3xbfgragh.onion/by/hailey-fuchs}{Hailey Fuchs}
and \href{https://www.nytimes3xbfgragh.onion/by/nick-corasaniti}{Nick
Corasaniti}

\begin{itemize}
\item
  Published Aug. 14, 2020Updated Aug. 31, 2020
\item
  \begin{itemize}
  \item
  \item
  \item
  \item
  \item
  \item
  \end{itemize}
\end{itemize}

WASHINGTON --- The
\href{https://www.nytimes3xbfgragh.onion/2020/08/18/us/politics/postal-service-suspends-changes.html}{Postal
Service} has warned states that it may not be able to meet their
deadlines for delivering last-minute
\href{https://www.nytimes3xbfgragh.onion/interactive/2020/08/31/us/politics/vote-by-mail-deadlines.html}{mail-in
ballots}, further fueling the clash over the new postmaster general's
handling of vote-by-mail operations as
\href{https://www.nytimes3xbfgragh.onion/2020/08/18/us/politics/postal-service-suspends-changes.html}{President
Trump} continued to
\href{https://www.nytimes3xbfgragh.onion/2020/08/13/us/politics/trump-postal-service-mail-voting.html}{rail
against the practice}.

In letters sent in July to all 50 states and the District of Columbia,
Thomas J. Marshall, the general counsel for the Postal Service, told
most of them that ``certain deadlines for requesting and casting mail-in
ballots are incongruous with the Postal Service's delivery standards.''

As many states
\href{https://www.nytimes3xbfgragh.onion/interactive/2020/08/11/us/politics/vote-by-mail-us-states.html}{turn
to vote-by-mail operations} to carry out elections safely amid the
\href{https://www.nytimes3xbfgragh.onion/interactive/2020/us/coronavirus-us-cases.html}{coronavirus
pandemic}, Mr. Marshall urged those with tight schedules to require that
residents request ballots at least 15 days before an election --- rather
than the shorter periods currently allowed under the laws of many
states.

``This mismatch creates a risk that ballots requested near the deadline
under state law will not be returned by mail in time to be counted,''
Mr. Marshall wrote.

Many states have long allowed voters to request a mail ballot close to
the election, but the Postal Service suggested that the large volume of
voting by mail at a time of widespread delivery delays meant that states
would be better off building more time into their systems.

Mr. Marshall said Nevada, New Mexico, Oregon and Rhode Island should not
have any trouble, based on their laws, while he requested more
information from Vermont and Washington, D.C. The other 45 states, he
told them in the letters, face the risk that the timetables set by their
laws could leave some voters unable to get their ballots postmarked by
Election Day or received by election boards in time to be counted.

The letters prompted some states to consider changes that would give
voters more time to vote by mail or ensure their ballot would be
counted. And their release intensified
\href{https://www.nytimes3xbfgragh.onion/2020/07/31/us/politics/trump-usps-mail-delays.html}{the
criticism directed at the Postal Service and Mr. Trump} by Democrats and
voting rights advocates, who say the president is deliberately stoking
unfounded concerns that voting by mail will lead to fraud and miscounts
as a way to cast doubt about the outcome of the election.

Word that the letters had been sent across the country,
\href{https://www.washingtonpost.com/local/md-politics/usps-states-delayed-mail-in-ballots/2020/08/14/64bf3c3c-dcc7-11ea-8051-d5f887d73381_story.html?hpid=hp_hp-top-table-high_uspsstates-230pm\%3Ahomepage\%2Fstory-ans}{first
reported} by The Washington Post, came as the Postal Service's inspector
general said she had begun an investigation into the postmaster general,
Louis DeJoy, a Republican megadonor and Trump ally
\href{https://www.nytimes3xbfgragh.onion/2020/05/07/us/politics/postmaster-general-louis-dejoy.html}{who
was appointed in May}.

Democrats in Congress who had urged the inspector general inquiry said
they expected it to encompass operational changes to the mail service
imposed by Mr. DeJoy and questions about his personal finances,
including his ownership of stock in a Postal Service contractor and
options in a competitor.

In response to the warning letters, some states, including Pennsylvania
and Michigan, have called for extensions on counting late-arriving
ballots in the November election.

``We have asked the Legislature to change Michigan law to allow ballots
postmarked by Election Day that arrive within a certain window to be
counted,'' said Tracy Wimmer, a spokeswoman for Michigan's secretary of
state, Jocelyn Benson.

In Pennsylvania, the secretary of state, Kathy Boockvar, asked a court
to allow eligible ballots to be counted if they are postmarked by
Election Day and received by the following Friday.

``The letter made clear that the Postal Service is experiencing
significant delays with mail delivery and expects this problem to
continue through Election Day,'' Ms. Boockvar said in a statement. ``The
department's action is simple in its goal --- to prevent the
disenfranchisement of eligible Pennsylvania voters.''

She added that her office continued to have ``great confidence'' in the
vote-by-mail system in the state, which had more than 1.5 million
mail-in ballots in the primary, and that the office would continue to
advocate for Pennsylvania residents to use voting by mail.

Richard Fiesta, the executive director of the Alliance for Retired
Americans, which had filed a related lawsuit seeking to ensure voting
rights, called the letter ``nothing short of outrageous.''

``A government agency is taking proactive steps to make voting by mail
harder,'' he said, adding that the Postal Service was trying to
``deprive people the ability to make sure their ballot is counted on
time.''

In another battleground state, Wisconsin, which suffered through some
postal delays during its primary in April, changing the deadlines would
require an act of the Legislature. Given that the Republican-controlled
body was unwilling to move the April elections amid the peak of the
pandemic, there is little hope it would convene a special session now.

Instead, the Wisconsin Election Commission is planning to mail
voting-information packets along with the absentee ballot request forms
to 2.6 million registered voters in September.

``It strongly encourages those who choose to vote absentee by mail to
request absentee ballots and return them as soon as possible,'' said
Meagan Wolfe, the administrator of the Wisconsin Elections Commission.

After receiving the letter this week, Kim Wyman, the Republican
secretary of state in Washington, said she was ``highly concerned'' and
immediately convened a meeting with local and national Postal Service
representatives and Washington election lawyers.

Ms. Wyman said that after the meeting, it became clearer that most of
the letter was a warning about pre-existing deadlines, and that the
state would only have to adjust its ballots that go out later than the
initial batch. Washington, a vote-by-mail state, mails ballots to its
4.6 million voters about 20 days before the election, but address
changes are accepted up to eight days beforehand.

``My frustration with both Congress and the White House is this
politicization of administrative functions,'' Ms. Wyman said.
``Attacking the postal process, it undermines people's confidence in our
elections.''

In Colorado, a state that votes almost entirely by mail, the secretary
of state, Jena Griswold, said she was ``very concerned'' about the Trump
administration's actions at the Postal Service.

``The president has shown a degree of disrespect to this country that is
just un-American,'' Ms. Griswold said. ``There is something that we
should all be able to agree on and that is well functioning elections.''

Roughly 75 percent of Colorado ballots were returned to a drop box
rather than mailed in, and Ms. Griswold said her office would continue
to educate voters on their options to return ballots.

Ms. Griswold said the local Colorado Postal Service has been a great
partner, but she is concerned about the headquarters in Washington and
will be investigating.

``It is very concerning that there is not clear communication of what is
exactly happening from the U.S. Postal Service,'' Ms. Griswold said. ``I
have requested a meeting with the postmaster general. I think he owes
Colorado, Coloradans, and every American, an answer of why he is trying
to shake it up, fire people, increase costs, in the middle of the
pandemic, with the intention to suppress voters.''

In California, the state is planning on expanding its already ample and
aggressive voter information program.

``Am I concerned? Yes. But am I panicking? No,'' said Alex Padilla, the
California secretary of state. He said that new policies this year that
allowed for a 17-day cushion for ballots to be received if they were
postmarked by Election Day should help the state weather any major
delays at the post office. But voters should still know not to take the
potential delays lightly.

``Every time anybody attacks the integrity of vote by mail or makes
false claims about the integrity of our elections, voter education
becomes that much more important and we're going to be committed to do
that,'' Mr. Padilla said.

Mr. DeJoy has argued that he is modernizing the money-losing agency to
make it more efficient. Among his moves have been cuts to overtime for
postal workers, restrictions on transportation and the reduction of the
quantity and use of mail-processing equipment.

In June, union officials received a notice that Postal Service
management was removing 671 machines used to sort mail quickly because
of a ``reduction to letter and flat mail volume.''

Mail operations in several battleground states were hit hard by the
cuts. On the list for removal were 24 delivery bar code sorters in Ohio,
11 in Detroit, 11 in Florida, nine in Wisconsin, eight in Philadelphia
and five in Arizona.

In July, Postal Service management sent to employees a ``mandatory''
order called, ``Pivoting for Our Future.'' In the memo, the Postal
Service said it was banning a practice of postal workers making daily
additional trips beyond their initial runs in an effort to save some
\$200 million.

``One aspect of these changes that may be difficult for employees is
that --- temporarily --- we may see mail left behind or mail on the
workroom floor or docks,'' the memo stated.

Last week, the Postal Service put in place a hiring freeze and canceled
promotions for nonunionized workers.

Mr. Trump has repeatedly made unfounded claims that mail-in voting will
lead to widespread fraud, and has pointed to delays in counting in some
primary elections this year as evidence that the general election could
be chaotic.

Even as he assailed the practice, Mr. Trump and his wife, Melania Trump,
requested mail-in ballots in Florida, according to Palm Beach County
records.

Speaker Nancy Pelosi and Senator Chuck Schumer of New York issued a
joint statement on Friday condemning Mr. Trump's ``attacks on vote by
mail.''

``The president's comments today affirm that no patriotic tradition is
immune from his abuse of power,'' they said. ``The president made plain
that he will manipulate the operations of the post office to deny
eligible voters the ballot in pursuit of his own re-election.''

Luke Broadwater and Hailey Fuchs reported from Washington, and Nick
Corasaniti from New York.

\hypertarget{our-2020-election-guide}{%
\section{Our 2020 Election Guide}\label{our-2020-election-guide}}

Updated ~Sept. 12, 2020

\begin{center}\rule{0.5\linewidth}{\linethickness}\end{center}

\begin{itemize}
\item ~
  \hypertarget{the-latest}{%
  \subsection{The Latest}\label{the-latest}}

  \begin{itemize}
  \item
    President Trump has failed to erase Joseph R. Biden Jr.'s lead
    across a set of key swing states,
    \href{https://www.nytimes3xbfgragh.onion/2020/09/12/us/politics/biden-trump-poll-wisconsin-minnesota.html?action=click\&pgtype=Article\&state=default\&region=BELOW_MAIN_CONTENT\&context=storylines_guide}{according
    to a poll}~conducted by The Times and Siena College.
  \end{itemize}
\item ~
  \hypertarget{paths-to-270}{%
  \subsection{Paths to 270}\label{paths-to-270}}

  \begin{itemize}
  \item
    Joe Biden and Donald Trump need 270 electoral votes to reach the
    White House. Try building
    \href{https://www.nytimes3xbfgragh.onion/interactive/2020/us/elections/election-states-biden-trump.html?action=click\&pgtype=Article\&state=default\&region=BELOW_MAIN_CONTENT\&context=storylines_guide}{your
    own coalition of battleground states}~to see potential outcomes.
  \end{itemize}
\item ~
  \hypertarget{voting-deadlines}{%
  \subsection{Voting Deadlines}\label{voting-deadlines}}

  \begin{itemize}
  \item
    Early voting for the presidential election starts in September~in
    some states. Take a look at
    \href{https://www.nytimes3xbfgragh.onion/interactive/2019/us/elections/2020-presidential-election-calendar.html?action=click\&pgtype=Article\&state=default\&region=BELOW_MAIN_CONTENT\&context=storylines_guide}{key
    dates}\href{https://www.nytimes3xbfgragh.onion/interactive/2019/us/elections/2020-presidential-election-calendar.html?action=click\&pgtype=Article\&state=default\&region=BELOW_MAIN_CONTENT\&context=storylines_guide}{where
    you
    liv}\href{https://www.nytimes3xbfgragh.onion/interactive/2019/us/elections/2020-presidential-election-calendar.html?action=click\&pgtype=Article\&state=default\&region=BELOW_MAIN_CONTENT\&context=storylines_guide}{e}.
    If you're voting by
    mail,~\href{https://www.nytimes3xbfgragh.onion/interactive/2020/08/31/us/politics/vote-by-mail-deadlines.html?action=click\&pgtype=Article\&state=default\&region=BELOW_MAIN_CONTENT\&context=storylines_guide}{it's
    risky to procrastinate}.
  \item
    \href{https://www.nytimes3xbfgragh.onion/interactive/2020/us/elections/joe-biden.html?action=click\&pgtype=Article\&state=default\&region=BELOW_MAIN_CONTENT\&context=storylines_guide}{}

    \hypertarget{joe-biden}{%
    \section{Joe Biden}\label{joe-biden}}

    \hypertarget{democrat}{%
    \subsection{Democrat}\label{democrat}}

    \href{https://www.nytimes3xbfgragh.onion/interactive/2020/us/elections/donald-trump.html?action=click\&pgtype=Article\&state=default\&region=BELOW_MAIN_CONTENT\&context=storylines_guide}{}

    \hypertarget{donald-trump}{%
    \section{Donald Trump}\label{donald-trump}}

    \hypertarget{republican}{%
    \subsection{Republican}\label{republican}}
  \end{itemize}
\item
  \hypertarget{keep-up-with-our-coverage}{%
  \subsection{Keep Up With Our
  Coverage}\label{keep-up-with-our-coverage}}

  \begin{itemize}
  \item
    Get an
    \href{https://www.nytimes3xbfgragh.onion/newsletters/politics?action=click\&pgtype=Article\&state=default\&region=BELOW_MAIN_CONTENT\&context=storylines_guide}{email}~recapping
    the day's news
  \item
    Download our mobile app on
    \href{https://apps.apple.com/us/app/nytimes/id284862083?ls=1\&mat_click_id=5c79ae7455014fd1bd66b5610c05b8f2-20191112-16948\&referrer=mat_click_id\%3D5c79ae7455014fd1bd66b5610c05b8f2-20191112-16948\%26link_click_id\%3D722930677036718082}{iOS}~and
    \href{http://a.localytics.com/android?id=com.nytimes.android\&referrer=utm_source\%3Dother_nyt_mobile_web\%26utm_medium\%3DWeb\%2520page\%26utm_term\%3DGeneral\%2520Mobile\%2520Page\%26utm_campaign\%3DNYT\%2520Mobile\%2520General\%2520Page}{Android}~and
    turn on Breaking News and Politics alerts
  \end{itemize}
\end{itemize}

Advertisement

\protect\hyperlink{after-bottom}{Continue reading the main story}

\hypertarget{site-index}{%
\subsection{Site Index}\label{site-index}}

\hypertarget{site-information-navigation}{%
\subsection{Site Information
Navigation}\label{site-information-navigation}}

\begin{itemize}
\tightlist
\item
  \href{https://help.nytimes3xbfgragh.onion/hc/en-us/articles/115014792127-Copyright-notice}{©~2020~The
  New York Times Company}
\end{itemize}

\begin{itemize}
\tightlist
\item
  \href{https://www.nytco.com/}{NYTCo}
\item
  \href{https://help.nytimes3xbfgragh.onion/hc/en-us/articles/115015385887-Contact-Us}{Contact
  Us}
\item
  \href{https://www.nytco.com/careers/}{Work with us}
\item
  \href{https://nytmediakit.com/}{Advertise}
\item
  \href{http://www.tbrandstudio.com/}{T Brand Studio}
\item
  \href{https://www.nytimes3xbfgragh.onion/privacy/cookie-policy\#how-do-i-manage-trackers}{Your
  Ad Choices}
\item
  \href{https://www.nytimes3xbfgragh.onion/privacy}{Privacy}
\item
  \href{https://help.nytimes3xbfgragh.onion/hc/en-us/articles/115014893428-Terms-of-service}{Terms
  of Service}
\item
  \href{https://help.nytimes3xbfgragh.onion/hc/en-us/articles/115014893968-Terms-of-sale}{Terms
  of Sale}
\item
  \href{https://spiderbites.nytimes3xbfgragh.onion}{Site Map}
\item
  \href{https://help.nytimes3xbfgragh.onion/hc/en-us}{Help}
\item
  \href{https://www.nytimes3xbfgragh.onion/subscription?campaignId=37WXW}{Subscriptions}
\end{itemize}
