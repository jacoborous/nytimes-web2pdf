Sections

SEARCH

\protect\hyperlink{site-content}{Skip to
content}\protect\hyperlink{site-index}{Skip to site index}

\href{https://www.nytimes3xbfgragh.onion/section/politics}{Politics}

\href{https://myaccount.nytimes3xbfgragh.onion/auth/login?response_type=cookie\&client_id=vi}{}

\href{https://www.nytimes3xbfgragh.onion/section/todayspaper}{Today's
Paper}

\href{/section/politics}{Politics}\textbar{}DeJoy Earned Millions From
Company With Financial Ties to Postal Service

\url{https://nyti.ms/3avZrbd}

\begin{itemize}
\item
\item
\item
\item
\item
\item
\end{itemize}

\begin{itemize}
\item
  \href{https://www.nytimes3xbfgragh.onion/interactive/2020/09/08/us/elections/results-new-hampshire-primary-elections.html?action=click\&pgtype=Article\&state=default\&region=TOP_BANNER\&context=storylines_menu}{New
  Hampshire Results}
\item
  \href{https://www.nytimes3xbfgragh.onion/live/2020/09/08/us/trump-vs-biden?action=click\&pgtype=Article\&state=default\&region=TOP_BANNER\&context=storylines_menu}{Election
  Updates}
\item
  \href{https://www.nytimes3xbfgragh.onion/interactive/2020/us/elections/election-states-biden-trump.html?action=click\&pgtype=Article\&state=default\&region=TOP_BANNER\&context=storylines_menu}{Paths
  to 270}
\item
  \href{https://www.nytimes3xbfgragh.onion/interactive/2020/08/31/us/politics/vote-by-mail-deadlines.html?action=click\&pgtype=Article\&state=default\&region=TOP_BANNER\&context=storylines_menu}{Voting
  by Mail}
\item
  \href{https://www.nytimes3xbfgragh.onion/interactive/2019/us/elections/2020-presidential-election-calendar.html?action=click\&pgtype=Article\&state=default\&region=TOP_BANNER\&context=storylines_menu}{Key
  Dates}
\item
  \href{https://www.nytimes3xbfgragh.onion/newsletters/politics?action=click\&pgtype=Article\&state=default\&region=TOP_BANNER\&context=storylines_menu}{Politics
  Newsletter}
\end{itemize}

Advertisement

\protect\hyperlink{after-top}{Continue reading the main story}

Supported by

\protect\hyperlink{after-sponsor}{Continue reading the main story}

\hypertarget{dejoy-earned-millions-from-company-with-financial-ties-to-postal-service}{%
\section{DeJoy Earned Millions From Company With Financial Ties to
Postal
Service}\label{dejoy-earned-millions-from-company-with-financial-ties-to-postal-service}}

The postmaster general, under fire for his business ties and his
cost-cutting measures, will testify before the House next week.

\includegraphics{https://static01.graylady3jvrrxbe.onion/images/2020/08/17/us/politics/17dc-postal/merlin_175823529_c0f99542-4a90-4cde-85da-afba5b8be8a5-articleLarge.jpg?quality=75\&auto=webp\&disable=upscale}

\href{https://www.nytimes3xbfgragh.onion/by/catie-edmondson}{\includegraphics{https://static01.graylady3jvrrxbe.onion/images/2019/11/20/us/politics/catie-edmonson-twitter-chatblog/catie-edmonson-twitter-chatblog-thumbLarge.png}}\href{https://www.nytimes3xbfgragh.onion/by/emily-cochrane}{\includegraphics{https://static01.graylady3jvrrxbe.onion/images/2018/11/28/multimedia/author-emily-cochrane/author-emily-cochrane-thumbLarge-v3.png}}

By \href{https://www.nytimes3xbfgragh.onion/by/catie-edmondson}{Catie
Edmondson} and
\href{https://www.nytimes3xbfgragh.onion/by/emily-cochrane}{Emily
Cochrane}

\begin{itemize}
\item
  Published Aug. 17, 2020Updated Aug. 24, 2020
\item
  \begin{itemize}
  \item
  \item
  \item
  \item
  \item
  \item
  \end{itemize}
\end{itemize}

WASHINGTON --- Postmaster General
\href{https://www.nytimes3xbfgragh.onion/2020/09/02/us/politics/louis-dejoy-usps-paid.html}{Louis
DeJoy}, who has come under fire for his continuing financial ties to a
company that does business with the Postal Service, received \$1.2
million to \$7 million in income last year from that firm, according to
financial disclosure forms reviewed by The New York Times.

Mr. DeJoy continues to hold \$25 million to \$50 million in that
company, XPO Logistics, where he served as the chief executive of the
company's supply chain business until 2015 and was a board member until
2018. Documents filed with the Office of Government Ethics show that Mr.
DeJoy also received millions of dollars in rental payments from XPO
through leasing agreements at buildings that he owns.

The revelations are likely to further fuel scrutiny of Mr. DeJoy, a
major donor to President Trump who has made a series of cost-cutting
moves and other changes at the Postal Service that Democrats warn are
aimed at undermining the 2020 election. Mr. DeJoy agreed on Monday to
testify before the House Oversight Committee next week, and Democrats
are expected to press him on the justification behind his new policies
and question his potential conflicts of interest.

XPO, a \$16 billion logistics and transportation company, assists the
Postal Service during busy shipping periods, such as around the
holidays, moving bulk shipments of packages from fulfillment centers and
taking them to local Postal Service centers so mail carriers can deliver
them to residences.

Reports filed by XPO to the Securities and Exchange Commission show the
company paid Mr. DeJoy
\href{https://www.sec.gov/Archives/edgar/data/0001166003/000104746919002443/a2238430zdef14a.htm}{\$1.86
million in rent} in 2018. Mr. DeJoy reported to the Office of Government
Ethics --- which requires government officials to provide a range of
income, rather than a specific amount --- that he stood to earn \$1.2
million to \$7 million from the arrangements.

Mr. DeJoy has insisted that he has fully complied with federal ethics
rules and maintained that the new measures he has put in place are
necessary to modernize the Postal Service.

``I take my ethical obligations seriously, and I have done what is
necessary to ensure that I am and will remain in compliance with those
obligations,'' Mr. DeJoy said in a statement.

Image

Voters from six states on Monday filed a lawsuit against Postmaster
General Louis DeJoy and President Trump.Credit...Erin Schaff/The New
York Times

As part of their broader push to accelerate their oversight of Mr. DeJoy
and his agency, House lawmakers
\href{https://www.nytimes3xbfgragh.onion/2020/08/16/us/politics/coronavirus-postal-service-stimulus-bill.html}{are
expected to return on Saturday} to vote on legislation to provide the
Postal Service with \$25 billion in emergency funding and reverse the
changes he has made since taking charge in May, according to two people
familiar with the legislation, who asked for anonymity to disclose
details of unfinished legislation. The bill is also expected to prevent
any policy changes until Jan. 1, 2021, or until the end of the pandemic.

The vote poses a challenge for politically vulnerable House Republicans.
They will have to decide whether to break with Mr. Trump, who has
continued a near-daily assault on voting by mail, and support emergency
aid for the beleaguered post office after negotiators failed to reach an
agreement on a broader coronavirus relief package.

Mr. Trump, who has long criticized the Postal Service, complained about
the hearing on Monday because it coincided with the opening day of the
Republican National Convention, declaring that Democrats ``are always
playing games.''

``GET TOUGH REPUBLICANS!!!'' he said on Twitter.

Speaker Nancy Pelosi of California and Senator Chuck Schumer of New
York, the minority leader, had originally pressed for \$25 billion for
the agency during coronavirus relief negotiations, though they had
tentatively discussed lowering the amount to \$10 billion, which
administration officials signaled they were open to. But talks have
since stalled, and top Republicans on Monday criticized Ms. Pelosi's
decision to interrupt the annual summer recess to vote on the
stand-alone legislation.

``It's not only unrealistic, it's unnecessary,'' Mark Meadows, the White
House chief of staff, told reporters aboard Air Force One. ``This is not
a funding issue as much as it is a long-term reform issue.''

Mr. Meadows said the president was ``willing to provide money for the
post office as long as it is included in some other skinny measure if we
cannot agree to a larger deal.''

Senator Mitch McConnell of Kentucky, the majority leader, has rebuffed
Democrats' claims that the Postal Service needs more money to handle
\href{https://slack-redir.net/link?url=https\%3A\%2F\%2Fwww.nytimes3xbfgragh.onion\%2Finteractive\%2F2020\%2F08\%2F11\%2Fus\%2Fpolitics\%2Fvote-by-mail-us-states.html}{as
many as 80 million ballots} that could be cast by mail in the November
election, telling reporters that ``the Postal Service is going to be
just fine.''

``We're going to make sure that the ability to function going into the
election is not adversely affected,'' Mr. McConnell said at a news
conference in Horse Cave, Ky.

Senate Republicans on Monday were discussing putting forward a narrow
coronavirus relief proposal of their own, which could include allocating
new funds for a \$300 weekly federal unemployment benefit through late
December, providing liability protections and reviving a lapsed
small-business loan program, according to two people familiar with the
plans.

Those people, who cautioned that the legislation was not yet finalized,
also said the proposal would convert a
\href{https://home.treasury.gov/news/press-releases/sm1071}{\$10 billion
loan} the Treasury Department offered to the Postal Service into a
grant.

The service's financial woes pose a challenge for Republican lawmakers
--- particularly senators from mail-dependent states who are up for
re-election --- who risk angering millions of voters who rely on the
post office for medicine, Social Security checks and other essential
resources if they do not take action. Ms. Pelosi told MSNBC on Monday
that she expected Republicans to support the Democrats' bill.

Some top Republicans criticized Ms. Pelosi's move to bring up
stand-alone legislation, including Representative Kevin McCarthy of
California, the minority leader, who accused Democrats of ``pushing
conspiracy theories about the USPS to undermine faith in the election.''
Other, more moderate Republicans have been more vocal in their criticism
of the policy changes and have been more open to providing the agency
with billions of dollars, particularly if included in a broader stimulus
package.

Senator Susan Collins, Republican of Maine,
\href{https://twitter.com/SenatorCollins/status/1295161050481004544?s=20}{has
called on the Senate}, currently out on recess, to return to Washington
to consider a coronavirus relief package that would include legislation
she wrote with Senator Dianne Feinstein, Democrat of California. Their
measure would allocate up to \$25 billion for the agency to cover losses
resulting from the pandemic.

``We cannot shortchange service to the public to fix USPS' financial
issues,'' Ms. Collins
\href{https://twitter.com/SenatorCollins/status/1295365846211137537?s=20}{said
on Monday}. ``The Postmaster General must take immediate action to
remedy these delays.''

\includegraphics{https://static01.graylady3jvrrxbe.onion/images/2020/08/17/us/politics/17dc-postal-sub/merlin_175807149_bfa43852-f3e3-48fe-848b-6d3e5656c59e-articleLarge.jpg?quality=75\&auto=webp\&disable=upscale}

Senator Steve Daines, Republican of Montana, wrote to Mr. DeJoy on
Friday requesting information about the removal of collection boxes in
his state, calling it unacceptable if ``the removals of collection boxes
could result in delayed mail delivery and reduced mail options for
Montanans.''

But Democrats running for the Senate in states that rely heavily on the
mail made clear they would continue to press the issue. John
Hickenlooper, the former Democratic governor of Colorado, took to
Twitter
\href{https://twitter.com/Hickenlooper/status/1295102236574408704?s=20}{in
a campaign video} to upbraid the impact of the delays and laid the blame
squarely on Mr. Trump and his Republican opponent, the incumbent Cory
Gardner.

``It just makes me want to pull my hair out, and Cory Gardner hasn't
said a word,'' Mr. Hickenlooper said.

Senator Thom Tillis, Republican of North Carolina, ``supports providing
relief to the Postal Service as part of a larger Covid-19 package that
also provides more relief to North Carolina families, job creators and
the doctors, nurses and health care providers on the front lines,''
Daniel Keylin, a spokesman for the senator, said in a statement on
Monday. ``Senator Tillis has confidence in North Carolina's absentee
voting system and the ability of the state to provide North Carolinians
with safe and secure ways to cast a ballot.''

Voters from six states on Monday filed a lawsuit against both Mr. DeJoy
and Mr. Trump, asking a federal court to block what plaintiffs said was
a calculated dismantling of the Postal Service ahead of the November
election. The suit was filed in the Southern District of New York on
behalf of 17 plaintiffs from California, Pennsylvania, Illinois, New
Jersey, Wisconsin and New York, including
\href{https://www.nytimes3xbfgragh.onion/2020/07/14/nyregion/mondaire-jones-house-primary.html}{Mondaire
Jones}, a progressive candidate all but guaranteed to join Congress next
year as a representative for the suburbs north of New York City.

The lawsuit asks the court to declare that Mr. Trump and Mr. DeJoy
violated voters' rights through an effort to slash the Postal Service in
opposition to mail-in voting. It also wants the court to order the
defendants to take ``all steps necessary and sufficient to ensure that
the U.S.P.S. is adequately funded'' and ensure that ``absentee and other
mail ballots are treated equal to in-person ballots'' and fund
``sufficient staffing and overtime to handle a record level of mail
voting.''

The suit also asks the court to ``mitigate any harms that may flow from
already accomplished harms,'' including the destruction or disposal of
postal machinery.

Two Democrats on the House Judiciary Committee on Monday urged the
F.B.I. director, Christopher A. Wray, to open a criminal investigation
into the role Mr. DeJoy has played in curtailing services. Top Senate
Democrats also urged the Postal Service's board of governors to reverse
the changes ``that degrade or delay postal operations and the delivery
of the mail'' and consider removing Mr. DeJoy if he refuses to comply.

``Mr. DeJoy appears to be engaged in a partisan effort, with the support
of President Trump, to delay and degrade mail service and undermine the
mission of the United States Postal Service,'' the senators wrote. ``You
have the responsibility to reverse those changes and the authority to do
so.''

Luke Broadwater and Nicholas Fandos contributed reporting.

\hypertarget{our-2020-election-guide}{%
\section{Our 2020 Election Guide}\label{our-2020-election-guide}}

Updated ~Sept. 8, 2020

\begin{center}\rule{0.5\linewidth}{\linethickness}\end{center}

\begin{itemize}
\item ~
  \hypertarget{the-latest}{%
  \subsection{The Latest}\label{the-latest}}

  \begin{itemize}
  \item
    President Trump and his party are using a playbook that aims to
    alarm people about crime in their backyards. It didn't work in 2018,
    but
    \href{https://www.nytimes3xbfgragh.onion/2020/09/08/us/politics/trump-republicans-fear-strategy.html?action=click\&pgtype=Article\&state=default\&region=BELOW_MAIN_CONTENT\&context=storylines_guide}{both
    parties think it could resonate more this year}.
  \end{itemize}
\item ~
  \hypertarget{how-to-win-270}{%
  \subsection{How to Win 270}\label{how-to-win-270}}

  \begin{itemize}
  \item
    Joe Biden and Donald Trump need 270 electoral votes to reach the
    White House. Try building
    \href{https://www.nytimes3xbfgragh.onion/interactive/2020/us/elections/election-states-biden-trump.html?action=click\&pgtype=Article\&state=default\&region=BELOW_MAIN_CONTENT\&context=storylines_guide}{your
    own coalition of battleground states}~to see potential outcomes.
  \end{itemize}
\item ~
  \hypertarget{voting-by-mail}{%
  \subsection{Voting by Mail}\label{voting-by-mail}}

  \begin{itemize}
  \item
    Will you have enough time to vote by mail in your state? Yes, but
    it's risky to procrastinate.
    \href{https://www.nytimes3xbfgragh.onion/interactive/2020/08/31/us/politics/vote-by-mail-deadlines.html?action=click\&pgtype=Article\&state=default\&region=BELOW_MAIN_CONTENT\&context=storylines_guide}{Check
    your state's deadline.}
  \item
    \href{https://www.nytimes3xbfgragh.onion/interactive/2020/us/elections/joe-biden.html?action=click\&pgtype=Article\&state=default\&region=BELOW_MAIN_CONTENT\&context=storylines_guide}{}

    \hypertarget{joe-biden}{%
    \section{Joe Biden}\label{joe-biden}}

    \hypertarget{democrat}{%
    \subsection{Democrat}\label{democrat}}

    \href{https://www.nytimes3xbfgragh.onion/interactive/2020/us/elections/donald-trump.html?action=click\&pgtype=Article\&state=default\&region=BELOW_MAIN_CONTENT\&context=storylines_guide}{}

    \hypertarget{donald-trump}{%
    \section{Donald Trump}\label{donald-trump}}

    \hypertarget{republican}{%
    \subsection{Republican}\label{republican}}
  \end{itemize}
\item
  \hypertarget{keep-up-with-our-coverage}{%
  \subsection{Keep Up With Our
  Coverage}\label{keep-up-with-our-coverage}}

  \begin{itemize}
  \item
    Get an
    \href{https://www.nytimes3xbfgragh.onion/newsletters/politics?action=click\&pgtype=Article\&state=default\&region=BELOW_MAIN_CONTENT\&context=storylines_guide}{email}~recapping
    the day's news
  \item
    Download our mobile app on
    \href{https://apps.apple.com/us/app/nytimes/id284862083?ls=1\&mat_click_id=5c79ae7455014fd1bd66b5610c05b8f2-20191112-16948\&referrer=mat_click_id\%3D5c79ae7455014fd1bd66b5610c05b8f2-20191112-16948\%26link_click_id\%3D722930677036718082}{iOS}~and
    \href{http://a.localytics.com/android?id=com.nytimes.android\&referrer=utm_source\%3Dother_nyt_mobile_web\%26utm_medium\%3DWeb\%2520page\%26utm_term\%3DGeneral\%2520Mobile\%2520Page\%26utm_campaign\%3DNYT\%2520Mobile\%2520General\%2520Page}{Android}~and
    turn on Breaking News and Politics alerts
  \end{itemize}
\end{itemize}

Advertisement

\protect\hyperlink{after-bottom}{Continue reading the main story}

\hypertarget{site-index}{%
\subsection{Site Index}\label{site-index}}

\hypertarget{site-information-navigation}{%
\subsection{Site Information
Navigation}\label{site-information-navigation}}

\begin{itemize}
\tightlist
\item
  \href{https://help.nytimes3xbfgragh.onion/hc/en-us/articles/115014792127-Copyright-notice}{©~2020~The
  New York Times Company}
\end{itemize}

\begin{itemize}
\tightlist
\item
  \href{https://www.nytco.com/}{NYTCo}
\item
  \href{https://help.nytimes3xbfgragh.onion/hc/en-us/articles/115015385887-Contact-Us}{Contact
  Us}
\item
  \href{https://www.nytco.com/careers/}{Work with us}
\item
  \href{https://nytmediakit.com/}{Advertise}
\item
  \href{http://www.tbrandstudio.com/}{T Brand Studio}
\item
  \href{https://www.nytimes3xbfgragh.onion/privacy/cookie-policy\#how-do-i-manage-trackers}{Your
  Ad Choices}
\item
  \href{https://www.nytimes3xbfgragh.onion/privacy}{Privacy}
\item
  \href{https://help.nytimes3xbfgragh.onion/hc/en-us/articles/115014893428-Terms-of-service}{Terms
  of Service}
\item
  \href{https://help.nytimes3xbfgragh.onion/hc/en-us/articles/115014893968-Terms-of-sale}{Terms
  of Sale}
\item
  \href{https://spiderbites.nytimes3xbfgragh.onion}{Site Map}
\item
  \href{https://help.nytimes3xbfgragh.onion/hc/en-us}{Help}
\item
  \href{https://www.nytimes3xbfgragh.onion/subscription?campaignId=37WXW}{Subscriptions}
\end{itemize}
