\href{/section/travel}{Travel}\textbar{}Roaming Through Lanzarote's
Otherworldly Vineyards

\url{https://nyti.ms/2QjjPTz}

\begin{itemize}
\item
\item
\item
\item
\item
\item
\end{itemize}

\includegraphics{https://static01.graylady3jvrrxbe.onion/images/2020/08/24/travel/24travel-lanzarote-01/24travel-lanzarote-01-articleLarge-v3.jpg?quality=75\&auto=webp\&disable=upscale}

Sections

\protect\hyperlink{site-content}{Skip to
content}\protect\hyperlink{site-index}{Skip to site index}

The World Through a Lens

\hypertarget{roaming-through-lanzarotes-otherworldly-vineyards}{%
\section{Roaming Through Lanzarote's Otherworldly
Vineyards}\label{roaming-through-lanzarotes-otherworldly-vineyards}}

The desolate~beauty of the winemaking tradition on Lanzarote, the
easternmost of the Canary Islands, is evidence of human resilience in
the face of adversity.

Sunset over a vineyard near the village of Uga.Credit...

Supported by

\protect\hyperlink{after-sponsor}{Continue reading the main story}

Photographs and Text by Mónica R. Goya

\begin{itemize}
\item
  Published Aug. 24, 2020Updated Aug. 29, 2020
\item
  \begin{itemize}
  \item
  \item
  \item
  \item
  \item
  \item
  \end{itemize}
\end{itemize}

\emph{At the onset of the coronavirus pandemic, with travel restrictions
in place worldwide, we launched a series ---}
\href{https://www.nytimes3xbfgragh.onion/column/the-world-through-a-lens}{\emph{The
World Through a Lens}} \emph{--- in which photojournalists help
transport you, virtually, to some of our planet's most beautiful and
intriguing places. This week, Mónica R. Goya shares a collection of
images from the Spanish island of Lanzarote.}

\begin{center}\rule{0.5\linewidth}{\linethickness}\end{center}

Situated some 80 miles off the southwest coast of Morocco, Lanzarote ---
with its stunning coastline, desert-like climate and plethora of
volcanoes --- is the easternmost of Spain's Canary Islands. Major
volcanic activity between 1730 and 1736, and again in 1824,
\href{http://volcano.oregonstate.edu/lanzarote\#:~:text=Lanzarote\%2C\%20a\%20shield\%20volcano\%20made,are\%20near\%20Montanas\%20del\%20Fuego.}{indelibly
altered the island's landscape} and helped pave the way for an
improbable sight: a vast expanse of otherworldly vineyards.

In recent years, Spain has devoted more land to vines than
\href{https://www.foodswinesfromspain.com/spanishfoodwine/wcm/idc/groups/public/documents/documento_anexo/mde3/nzcw/~edisp/dax2017770176.pdf}{any
other country in the world}. And while the Canary Islands, more broadly,
have a longstanding wine tradition --- the archipelago's wines, for
example, were mentioned in several of Shakespeare's plays --- nothing
could prepare me for the uniqueness of Lanzarote's vines.

Image

At the coastal rock formations known as Los Hervideros, underwater caves
create dramatic sound (and visual) effects.

Image

The dark sand beach near Charco de los Clicos, on Lanzarote's western
coast.

\includegraphics{https://static01.graylady3jvrrxbe.onion/images/2020/08/24/travel/24travel-lanzarote-06/24travel-lanzarote-06-articleLarge.jpg?quality=75\&auto=webp\&disable=upscale}

Image

Famara Beach, in the north, is three miles long and popular among
surfers.

The most remarkable wine area on the island is La Geria, a 13,000-acre
protected landscape which lies at the foot of
\href{https://turismolanzarote.com/en/visiting-the-timanfaya-national-park/}{Timanfaya
National Park}, one of Lanzarote's main tourist attractions. It was here
in Timanfaya that volcanic eruptions buried around a quarter of the
island (including La Geria) under a thick layer of lava and ash,
creating a breathtakingly barren scene --- and eventually leading to a
new way of growing vines.

Image

The 10-mile-long Route of the Volcanoes at Timanfaya National Park
follows a single-track road.

Image

Characteristic scenery --- with almost extraterrestrial qualities --- at
Timanfaya National Park.

Many of the vines on Lanzarote are planted in inverted conical holes
known as hoyos, which are dug by hand to various depths, each one made
in search of the fertile soil underneath the ash and lapilli. In a
counterintuitive twist, the ash plays an essential role in the
vineyards' success: It protects the ground from erosion, helps retain
moisture and regulates soil temperature.

Low semicircular rock walls protect the vines from the merciless winds.
Together with the hoyos, they contribute to an inventive growing method
that might easily be mistaken for a network of sculptural art.

Image

Harvesting grapes from deep hoyos near the village of Masdache, in La
Geria.

Image

A bunch of white grapes of the
\href{https://en.wikipedia.org/wiki/Vigiriega}{Diego variety}, which are
famous for their acidity.

Image

Vines grow in a hillside plot near the village of Uga.

La Geria is a superb example of humans working hand-in-hand with nature.
In a way, the immense --- if desolate --- beauty of this area is
evidence of human resilience in the face of adversity: For hundreds of
years, inhabitants here have managed to extract life from volcanic ash
on an island often plagued by drought.

Image

Houses across the island are typically rectangular in shape, and are
painted white.

But changing weather patterns (including scarcer-than-usual rainfall)
and harsh economic realities are persistent threats. The traditional
hoyos system can yield about 1,200 pounds of grapes per acre. Other less
traditional (and less time intensive) cultivation systems on the island
can yield up to 6,000 pounds per acre ---~by utilizing higher-density
growing techniques and some forms of mechanization.

Image

Timanfaya National Park covers about 20 square miles.

An economist by trade and environmentalist at heart, the winegrower
Ascensión Robayna has a strong connection to Lanzarote and a serious
commitment to conservation. For years she has tended high-maintenance
and low-yielding organic vineyards, adamantly asserting that this unique
landscape, and the traditions embedded within it, must be kept alive.

``Growing vines in hoyos means that farmers adapted to the special
circumstances of soil and climate, creating the most singular of the
agrarian ecosystems,'' she said.

Image

Ascensión Robayna stands beside a lava fissure, called a chaboco, where
ancient muscat vines are grown.

Image

Ms. Robayna outside Puro Rofe, a winery founded on the island in 2018.

There's an obvious sparkle in Ms. Robayna's eyes whenever she descends
into the lava fissures, called chabocos, where trees and grapevines ---
especially muscat grapes, among the oldest of varieties --- are grown.
(\href{https://www.europeancellars.com/producer/puro-rofe/}{Puro Rofe},
a winery founded on the island in 2018, recently released a wine made
exclusively from her chaboco-grown grapes.)

Image

Ms. Robayna among her vines, with the El Cuervo volcano in the distance.

In the late 19th century, a pestilent aphid, phylloxera, decimated
grapevines throughout mainland Europe. (The wine industry there was
salvaged by grafting European vines onto American rootstocks, which were
immune to phylloxera.) By contrast, phylloxera never reached Canarian
shores. As a result, vines here can be planted on their own roots --- a
relative rarity in the wine world.

Hundred-year-old vines and unique grape varieties are a common sight
across the islands. Malvasia Volcánica is arguably the island's most
well-known grape variety; others include Listán Negro, Diego and Listán
Blanco.

Image

In La Geria, grapes from the hoyos are harvested by hand.

Once, while visiting a set of vineyards near Uga, a small village in
southern Lanzarote, I followed the winegrower Vicente Torres as he
climbed barefoot --- the traditional way of working here --- up the
hillside to inspect his vines. With the lapilli tickling my feet, and
while sinking slightly with each step, I found the ascent more arduous
than I'd anticipated. Growing anything in this soil, I learned, is hard
work.

Image

Vicente Torres, of the Puro Rofe winery, walks around his vineyards
barefoot, the traditional way of working in the hoyos.

Image

Mr. Torres in his vineyard, near the village of Uga.

According to regulatory data, this year's harvest is expected to be less
than half of last year's, with a forecast of about 2.6 million pounds of
grapes.

``The oldest men around here say they don't recall a year as bad for
vineyards as this,'' said Pablo Matallana, an oenologist who grew up on
neighboring Tenerife but has family roots on Lanzarote. ``We have been
enduring two years of extreme drought. Some plots have debilitated
considerably, and the vigor of the vines has decreased,'' he said.

Image

\href{https://en.wikipedia.org/wiki/Listán_negro}{Listán Negro} grapes
in a palm leaf basket, which were traditionally used for harvest until
lighter options became available.

Rayco Fernández, a founding member of the Puro Rofe winery and a
distributor praised for having been one of the first to showcase quality
Canarian wines, agreed. ``The drought is ruining vineyards,'' he said,
adding that the ash, where there is a thick enough layer of it, has been
a lifeline.

Image

Pablo Matallana's Vinícola Taro white wine is made from Malvasia grapes.

Image

Mr. Torres tastes wine straight from the barrel at Puro Rofe.

But Lanzarote faces other threats, too. Tourism accounts for a
significant portion of the island's gross domestic product. And, despite
a relatively low number of confirmed coronavirus infections, this
economic sector has largely evaporated.

According to a Covid-19 economic impact study conducted at La Laguna
University, Lanzarote's G.D.P. is projected to drop by 21 percent.

Image

Harvested grapes are carried off a plot in La Geria.

Image

A hoyo vineyard in Masdache.

With the number of winegrowers falling, and climate change wreaking
havoc, the future of winemaking on Lanzarote appears more challenging
than ever.

There's no doubt, though, that the island holds a kind of mythical sway
over its visitors. It's been almost a year since my last trip to
Lanzarote, yet I continue to revisit certain images in my mind: of vines
emerging from the majestic hoyos at the foot of Timanfaya --- a splendor
still to be treasured there, at least for now.

Image

Morning light over freshly harvested Diego grapes at a vineyard in La
Geria.

\href{http://www.monicargoya.com/}{\emph{Mónica R. Goya}} \emph{is a
London-based journalist and photographer. Her last World Through a Lens
essay was about a}
\href{https://www.nytimes3xbfgragh.onion/2020/06/24/travel/dolomites-italy-hut-hiking.html}{\emph{hut-to-hut
hike in the Dolomites}}\emph{. You can follow her work on}
\href{https://www.instagram.com/monicargoya/}{\emph{Instagram}}\emph{.}

\emph{\textbf{Follow New York Times Travel}} \emph{on}
\href{https://www.instagram.com/nytimestravel/}{\emph{Instagram}}\emph{,}
\href{https://twitter.com/nytimestravel}{\emph{Twitter}} \emph{and}
\href{https://www.facebookcorewwwi.onion/nytimestravel/}{\emph{Facebook}}\emph{.
And}
\href{https://www.nytimes3xbfgragh.onion/newsletters/traveldispatch}{\emph{sign
up for our weekly Travel Dispatch newsletter}} \emph{to receive expert
tips on traveling smarter and inspiration for your next vacation.}

Advertisement

\protect\hyperlink{after-bottom}{Continue reading the main story}

\hypertarget{site-index}{%
\subsection{Site Index}\label{site-index}}

\hypertarget{site-information-navigation}{%
\subsection{Site Information
Navigation}\label{site-information-navigation}}

\begin{itemize}
\tightlist
\item
  \href{https://help.nytimes3xbfgragh.onion/hc/en-us/articles/115014792127-Copyright-notice}{©~2020~The
  New York Times Company}
\end{itemize}

\begin{itemize}
\tightlist
\item
  \href{https://www.nytco.com/}{NYTCo}
\item
  \href{https://help.nytimes3xbfgragh.onion/hc/en-us/articles/115015385887-Contact-Us}{Contact
  Us}
\item
  \href{https://www.nytco.com/careers/}{Work with us}
\item
  \href{https://nytmediakit.com/}{Advertise}
\item
  \href{http://www.tbrandstudio.com/}{T Brand Studio}
\item
  \href{https://www.nytimes3xbfgragh.onion/privacy/cookie-policy\#how-do-i-manage-trackers}{Your
  Ad Choices}
\item
  \href{https://www.nytimes3xbfgragh.onion/privacy}{Privacy}
\item
  \href{https://help.nytimes3xbfgragh.onion/hc/en-us/articles/115014893428-Terms-of-service}{Terms
  of Service}
\item
  \href{https://help.nytimes3xbfgragh.onion/hc/en-us/articles/115014893968-Terms-of-sale}{Terms
  of Sale}
\item
  \href{https://spiderbites.nytimes3xbfgragh.onion}{Site Map}
\item
  \href{https://help.nytimes3xbfgragh.onion/hc/en-us}{Help}
\item
  \href{https://www.nytimes3xbfgragh.onion/subscription?campaignId=37WXW}{Subscriptions}
\end{itemize}
