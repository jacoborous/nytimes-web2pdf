Donald Trump Jr. Is Ready. But for What, Exactly?

\url{https://nyti.ms/34vdqNk}

\begin{itemize}
\item
\item
\item
\item
\item
\item
\end{itemize}

\begin{itemize}
\item
  \href{https://www.nytimes3xbfgragh.onion/interactive/2020/09/08/us/elections/results-new-hampshire-primary-elections.html?action=click\&pgtype=Article\&state=default\&region=TOP_BANNER\&context=storylines_menu}{New
  Hampshire Results}
\item
  \href{https://www.nytimes3xbfgragh.onion/live/2020/09/08/us/trump-vs-biden?action=click\&pgtype=Article\&state=default\&region=TOP_BANNER\&context=storylines_menu}{Election
  Updates}
\item
  \href{https://www.nytimes3xbfgragh.onion/interactive/2020/us/elections/election-states-biden-trump.html?action=click\&pgtype=Article\&state=default\&region=TOP_BANNER\&context=storylines_menu}{Paths
  to 270}
\item
  \href{https://www.nytimes3xbfgragh.onion/interactive/2020/08/31/us/politics/vote-by-mail-deadlines.html?action=click\&pgtype=Article\&state=default\&region=TOP_BANNER\&context=storylines_menu}{Voting
  by Mail}
\item
  \href{https://www.nytimes3xbfgragh.onion/interactive/2019/us/elections/2020-presidential-election-calendar.html?action=click\&pgtype=Article\&state=default\&region=TOP_BANNER\&context=storylines_menu}{Key
  Dates}
\item
  \href{https://www.nytimes3xbfgragh.onion/newsletters/politics?action=click\&pgtype=Article\&state=default\&region=TOP_BANNER\&context=storylines_menu}{Politics
  Newsletter}
\end{itemize}

\includegraphics{https://static01.graylady3jvrrxbe.onion/images/2020/08/30/magazine/30mag-donjr/30mag-donjr-articleLarge.jpg?quality=75\&auto=webp\&disable=upscale}

Sections

\protect\hyperlink{site-content}{Skip to
content}\protect\hyperlink{site-index}{Skip to site index}

Feature

\hypertarget{donald-trump-jr-is-ready-but-for-what-exactly}{%
\section{Donald Trump Jr. Is Ready. But for What,
Exactly?}\label{donald-trump-jr-is-ready-but-for-what-exactly}}

Of all the president's children, he has the strongest connection to the
politics, voters and online disinformation ecosystem that put his father
in the White House. What will he do with it?

Credit...Illustration by Eric Yahnker

Supported by

\protect\hyperlink{after-sponsor}{Continue reading the main story}

By Jason Zengerle

\begin{itemize}
\item
  Aug. 24, 2020
\item
  \begin{itemize}
  \item
  \item
  \item
  \item
  \item
  \item
  \end{itemize}
\end{itemize}

\hypertarget{listen-to-this-article}{%
\subsubsection{Listen to This Article}\label{listen-to-this-article}}

Audio Recording by Audm

\emph{To hear more audio stories from publishers like The New York
Times, download}
\emph{\href{https://www.audm.com/?utm_source=nytmag\&utm_medium=embed\&utm_campaign=the_next_trump}{Audm
for iPhone or Android}.}

A case can be made that the apex of Donald Trump's presidency occurred
early this year, around the time of the State of the Union address. The
Feb. 4
\href{https://www.nytimes3xbfgragh.onion/2020/02/05/us/politics/state-of-union-speech-address.html}{speech}
to a joint session of Congress began with
\href{https://www.nytimes3xbfgragh.onion/2020/08/24/nyregion/letitia-james-trump-projects-investigation.html}{Trump's}
ignoring the outstretched hand of Speaker Nancy Pelosi --- a pointed
snub of the Democrat who, two months earlier, led his impeachment. For
the next 78 minutes, Trump boasted about his accomplishments, like
building an economy that ``is the best it has ever been''; dished out
red meat to his base, such as pledging a national ban on late-term
abortions; and theatrically dispensed favors, including a Presidential
Medal of Freedom for the conservative talk-radio host Rush Limbaugh.

On that February evening, the first reported death from the coronavirus
in the United States was more than three weeks away, and it appeared as
if Trump had bent the office of the president, its trappings, the
institutions of government and, indeed, all of American politics to his
will. After he finished, Pelosi, standing behind Trump on the House
rostrum, dramatically ripped up her copy of his speech. It was a
made-for-meme moment --- and at first, the meme's natural constituency
seemed to be the left, for whom Pelosi's paper-shredding provided a rare
flash of emotional gratification in an otherwise dark time. (The next
day, the Republican-led Senate would vote to acquit Trump of all the
impeachment articles, mooting the Democrats' monthslong crusade.) But
then Trump's eldest child, Donald Trump Jr., had an idea.

It came to him while he was eating lunch at the Trump International
Hotel in Washington the day after the speech. Trump Jr. envisioned a
video featuring the most benign and unobjectionable parts of his
father's address --- hailing a Tuskegee Airman, reuniting a soldier who
had just returned from Afghanistan with his wife and children, giving a
private-school scholarship to a Black fourth grader from Philadelphia
--- spliced with footage of Pelosi ripping up the speech, as if she were
objecting to these beneficent gestures and not to the president himself.
Trump Jr. called Benny Johnson, a veteran right-wing meme maker who
works for the conservative student group Turning Point USA, and asked
him to get cracking.

A few hours later, Trump Jr. posted the results of Johnson's handiwork
to his social media accounts. ``Pelosi ripped up @realDonaldTrump's
speech last night,'' he wrote on Twitter with a link to the video. ``In
that speech were stories of American Heroes \& American Dreams. Their
stories are more powerful than her hate.'' The video immediately went
viral, with the president himself tweeting it the next day.

Pelosi demanded that Twitter and Facebook take down the video, arguing
that it was deceptively edited. The social media companies refused.
Trump's allies used the spat to drive even more traffic. ``It would be
soooooo terrible if this video hits 10,000,000 views,'' tweeted Dan
Scavino, the White House social media director. In the end, the video
racked up 50 million views, according to Johnson. (Thirty-seven million
people watched the State of the Union address on TV.) ``Don Jr. is a
meme general in the meme wars,'' Johnson says, ``and he is commanding
the D-Day invasion.''

The episode was emblematic of Trump Jr.'s role in his father's political
carnival. In one respect, the brazen disingenuousness and virality of
the meme --- and the way in which one led to the other --- was
unmistakably Trumpian. But there was a discipline and polish to Trump
Jr.'s move that his father's shambolic, logorrheic self-expressions so
often lack. (Trump's own initial reaction to the speech-ripping was to
go on a late-night-into-early-morning Twitter tear, retweeting Pelosi
criticism by everyone from his former U.N. ambassador Nikki Haley to the
300-follower Twitter account @JonMart93888215.) And yet Trump Jr.'s
gloss did nothing to soften his father's message. It wasn't Trump
watered down. It was Trump distilled.

When Trump ran for president in 2016, Trump Jr., who is now 42, was
involved but hardly central to the effort. His sister Ivanka and her
husband, Jared Kushner, exercised sweeping influence over the campaign.
Trump Jr., by contrast, was assigned small, discrete tasks, like putting
his outdoorsmanship on display in a pheasant-hunting photo op with his
brother, Eric, before the Iowa caucuses to counter attacks that his
father was a liberal city slicker. (``Don, you can \emph{finally} do
something for me --- you can go hunting,'' his father told him,
according to
\href{https://www.gq.com/story/real-story-of-donald-trump-jr}{GQ
magazine}.) If he tried to go outside his narrow lane, disaster tended
to follow. In the summer of 2016, he arranged for a Trump Tower meeting
between Trump campaign officials and a Russian lawyer promising dirt on
Hillary Clinton, an encounter that later became a focus of Robert
Mueller's investigation into possible collusion between the Trump
campaign and the Russian government. The Trump team's defense of Trump
Jr. boiled down to the argument that he wasn't a traitor, just an idiot
--- ``by no means a sophisticated political actor,'' Chris Christie
said. Michael Cohen, at the time Trump's personal attorney, later told a
Senate panel that ``Mr. Trump was very quick to tell everybody that he
thinks Don Jr. has the worst judgment of anyone he's ever met in the
world.'' Or, as the president himself put it, according to
\href{https://www.theatlantic.com/magazine/archive/2019/10/trump-dynasty/596674/}{The
Atlantic}, ``He's not the sharpest knife in the drawer.''

So it is one of the many surprises of Trump's presidency that Trump Jr.
has grown into arguably his father's most valuable political weapon.
``Don Jr. represents the emotional center of the MAGA universe,'' says
Jason Miller, a senior adviser on Trump's re-election campaign. Before
the pandemic, he was crisscrossing the country as his father's
most-requested campaign surrogate. Since the coronavirus curtailed his
travel plans, he has become one of the Republican Party's top virtual
fund-raisers. His Twitter, Instagram and Facebook accounts have a
combined 11 million followers and are vital cogs in the Republican
messaging machine.

After spending much of his adult life searching for a purpose, Trump Jr.
seems to have found one in politics. His siblings can often seem to be
patiently waiting out their father's presidency. Eric, who has been
running the Trump Organization in his father's absence, continues
building hotels and luxury condominiums. Ivanka and Kushner went to work
in the White House, but she has told friends that she's looking forward
to returning to New York and to her lifestyle brand.

But Trump Jr. does not want to go back to the way things were before. He
has been electrified, and transformed, by his father's presidency. He
has largely given up the duties that go along with his title as an
executive vice president of the Trump Organization in exchange for
full-time politics. He has divorced --- after 12 years of marriage and
five children --- Vanessa Haydon, who generally shied away from
politics. His girlfriend of the last couple of years, with whom he
recently bought a house in the Hamptons, is Kimberly Guilfoyle, a former
Fox News host and conservative commentator who serves as the finance
chair for his father's re-election campaign.

Now, as he works to secure a second term for Trump this November, Trump
Jr. is also thinking about his own political future. He is wagering that
by going all in on his father's presidency and the tribal passions it
has unleashed, he can claim his own durable place in American politics
--- that whether his father leaves the White House in 2021 or 2025, the
answer to what comes after Trump will be more Trump.

\textbf{On Saturday, March 7,} about 100 people gathered in a gilded
ballroom at Trump's Mar-a-Lago club in Palm Beach, Fla. The resort town
was playing host to a retreat for major donors to Trump's re-election
campaign that weekend, and the highlight was a lavish party on Saturday
night to celebrate Guilfoyle's 51st birthday. The Trump family was
there, save for Trump's wife, Melania. So were a who's who of the MAGA
universe, including members of Congress, like Representative Matt Gaetz
of Florida and Senator Lindsey Graham of South Carolina; Fox News stars
like Jesse Watters and Tucker Carlson; administration officials
including National Security Adviser Robert O'Brien and Trump loyalists
like Rudy Giuliani; and even, for a time, President Jair M. Bolsonaro of
Brazil, who was meeting with Trump at the club that weekend. Sergio Gor,
Guilfoyle's chief of staff on the Trump campaign and Trump Jr.'s
collaborator on a forthcoming book, played M.C. and D.J., standing on a
stage between two spinning disco balls. ``It was like a Gatsby-esque
extravaganza,'' one guest recalls.

Trump Jr., an avid fisherman, had been up since before dawn,
unsuccessfully pursuing a hammerhead shark at a nearby inlet. Poolside
at Mar-a-Lago later that morning, among the club's guests outfitted in
white linen, he showed off pictures of the six-foot nurse shark he did
catch to Gaetz; with his camouflage and fishing rod, he looked as if
``he just came off the set of `Duck Dynasty,''' Gaetz recalls. By the
evening, he had traded camo for a suit and tie and was seated at the
head table alongside Guilfoyle in her gold-sequined minidress.
``Princess, you are the best,'' he
said,\href{https://www.washingtonexaminer.com/news/trump-world-celebrates-kimberly-guilfoyles-birthday-at-mar-a-lago}{according
to The Washington Examiner,} when it was his turn for a toast. ``Thank
you for everything that you do. I love you very much, and get back to
work, OK?''

He turned to the guests. ``You are in this room for a reason,'' he said.
``You guys have been the warriors, the fighters, the people who have
been there every time we have made a call, every time we made a
request.'' He added, ``I'm sure Kimberly will hit you up.'' As the
president stood beside Guilfoyle and led the group in a rousing
rendition of ``Happy Birthday,'' Trump Jr. looked on, beaming. When the
song was finished, Guilfoyle shouted, ``Four more years!'' The president
kissed her on the head and smiled at his son.

The two men had for years had a difficult relationship. Trump's ex-wife
Ivana recounts in her 2017 book, ``Raising Trump,'' that when she
suggested naming their newly born first child Donald Jr., Trump
protested: ``You can't do that! What if he's a loser?'' After his
parents divorced, a 12-year-old Trump Jr. refused to speak to his father
for a year. Later, he seemed intent on escaping the celebrity
businessman's shadow and reputation. At the Fiji fraternity at the
University of Pennsylvania, Trump Jr.'s nickname was Ron Rump, and his
fraternity brothers called him Ron. ``He loved it, perhaps because it
gave him an extra level of anonymity,'' one of them recalls. Rather than
working for the Trump Organization immediately after college, Trump Jr.
spent a year and half in Aspen, Colo., skiing, hunting, fishing and
tending bar at night.

In 2001, he moved back to New York City and took his place at the
company. But his greatest contribution to the family business came on
the set of ``The Apprentice,'' which he joined as an occasional
boardroom judge in the show's 2006 season. He was valued by the
producers as a stabilizing presence, running interference between the
cast and crew and the volatile star, his father. When Trump would berate
crew members for a mistake, one ``Apprentice'' producer recalls, Trump
Jr., speaking from a well of personal experience, would console them:
``It's not your fault; it's your turn.''

People who worked on the show remember him often trying to lighten the
mood. ``He provided the comic relief, because his dad doesn't have a
sense of humor and Ivanka wasn't someone who made jokes,'' says Clay
Aiken, the ``American Idol'' winner who appeared on ``The Celebrity
Apprentice'' in 2012. ``He was perfectly fine to take the piss out of
himself, but sometimes he'd make a joke about his dad --- and then you
could tell he was really nervous his dad wouldn't like it. His
self-esteem was in the gutter.''

Much of the popular image of Trump Jr., especially among liberals, seems
to stem from those years: ``uselessly trying to impress a man who can
only be impressed by himself''
\href{https://www.gq.com/story/real-story-of-donald-trump-jr}{(GQ)}; ``a
recurring liability and a chronic headache''
\href{https://www.thedailybeast.com/trump-aides-russia-flap-proves-don-jr-is-the-fredo-of-the-first-family}{(The
Daily Beast)}; the ``Fredo'' of the Trump family (Twitter). In the first
days of Trump's presidency, he seemed poised for more of the same. After
the election, while Ivanka and Kushner headed to Washington, Trump Jr.
stayed behind in New York, ostensibly to run the Trump Organization with
Eric. But he had little to do. He was in charge of the company's
international portfolio, and while he could continue working on overseas
projects that predated his father's election, he couldn't embark on new
ones.

For a time, he tried to play a role in shaping the administration's
public-lands policy and other issues related to his outdoor activities,
which had earned him the Secret Service code name Mountaineer. Senator
Steve Daines, a Republican from Montana, used an elk-hunting trip with
Trump Jr. in November 2016 to lobby the incoming administration to pick
an interior secretary from the Mountain West. ``I wanted a Westerner,''
Daines says, ``and Westerner doesn't mean West Virginia. It doesn't mean
Oklahoma.'' Trump Jr. recommended Ryan Zinke, then a Montana congressman
and a friend of Daines's, for the Department of Interior job. Zinke got
the nod but resigned in December 2018 after a scandal-plagued tenure.

\textbf{Trump Jr.'s relatively} low public profile ended on July 8,
2017,
\href{https://www.nytimes3xbfgragh.onion/2017/07/08/us/politics/trump-russia-kushner-manafort.html}{when
The New York Times revealed} his role in arranging the Trump Tower
meeting the previous summer between Trump campaign officials and the
Russian lawyer and her associates. Though little seems to have come out
of the meeting, a bipartisan Senate Intelligence Committee report
released this month found that the Russians had ``significant
connections to the Russian government, including the Russian
intelligence services.''

A few days after the Times article ran, Trump Jr. went on Sean Hannity's
Fox News show to defend himself in a softball interview. ``There was
nothing to tell,'' he said of the meeting. ``I wouldn't have even
remembered it until you started scouring through this stuff.'' His stock
among conservatives rose as he proceeded to wage a sustained campaign
against the news media, Mueller and congressional investigators pursuing
their own Russia inquiry. (It was
\href{https://www.nbcnews.com/politics/justice-department/senate-made-criminal-referral-trump-jr-bannon-kushner-two-others-n1237155}{reported
this month} that in 2019, the Senate Intelligence Committee's Republican
and Democratic leaders made a criminal referral of Trump Jr. and several
other Trump associates to the Justice Department for lying or providing
contradictory testimony to the panel.) He became a frequent guest on Fox
News and an enthusiastic participant in the political fights of the
moment. ``Don's favorite part of politics is getting punched in the face
with a jab and responding with a haymaker,'' one person close to him
says.

To those who know Trump Jr., his attraction to politics was not
surprising. ``He was the only family member who talked politics before
his dad ran for president,'' the person close to him says. ``He's the
only one of the kids who would have found a way into politics if the dad
hadn't run for office.'' And those politics have always tilted hard to
the right. Speaking to the Senate Intelligence Committee in 2018,
Stephen K. Bannon, the Trump adviser who had run the right-wing website
Breitbart, said, ``I'd describe Don Jr., who I think very highly of, as
a guy who believes everything on Breitbart is true.'' Or as Sam Nunberg,
an adviser to Trump's 2016 campaign, says, ``Don's a real winger, and I
mean that as a compliment.''

\includegraphics{https://static01.graylady3jvrrxbe.onion/images/2020/08/30/magazine/30mag-donjr-02/30mag-donjr-02-articleLarge.jpg?quality=75\&auto=webp\&disable=upscale}

In early 2018, Trump Jr. approached Andy Surabian, a young Republican
operative who worked on the 2016 campaign and then in the White House
for Bannon. Trump Jr. was by then a formidable presence on social media
and Fox News, but, he explained to Surabian, he wanted to move into real
politics by stumping for Republican congressional candidates in the
midterm elections. Surabian put together a campaign schedule for him
that, from May to November, featured 70 events in 17 states. Among the
candidates he campaigned for was Matt Gaetz, a young congressman from
Florida who spent much of his first term loudly demonstrating his
loyalty to Trump. ``We need fighters!'' Trump Jr. said from behind a
lectern decorated with a ``Make America Gaetz Again!'' sign. Now, Gaetz
says, ``constantly candidates are begging me to get his phone number, or
a photo with him, or a chance for a retweet or an endorsement.''

The president can still be brutally dismissive of his son, grousing
about his enthusiasm for firearms or questioning his political
intelligence, according to multiple people present for such
conversations. When Trump appeared on a special Father's Day edition of
``Triggered,'' Trump Jr.'s biweekly online talk show for the Trump
campaign, the awkwardness between the men was painful. Trump Jr. asked
the president if he liked his beard. ``Get rid of it,'' Trump growled,
to peals of nervous laughter from Trump Jr.

But the president is, at heart, a transactional person. As Trump Jr.'s
political star has risen, Trump advisers say, so has Trump's
appreciation for him. Cliff Sims, a former White House communications
aide, recalls watching television with the president in the private
dining room off the Oval Office one afternoon when Trump Jr. appeared on
the screen. ``The president stopped what he was doing and turned up the
volume,'' Sims says. ``He literally said to me, `He's really good at
this, isn't he?' He had this kind of feeling like, He's a chip off the
old block.''

Trump Jr. is now a key player in the Republican Party's 2020 operation.
He and Guilfoyle have become fund-raising powerhouses, coaxing large
donations from high-dollar donors. (Guilfoyle is reportedly paid
\$15,000 a month by the Trump campaign.) Email solicitations sent out by
the National Republican Congressional Committee, the House Republicans'
election arm, under Trump Jr.'s name have so far raised more than \$3
million in small-dollar donations. ``Triggered'' is the most watched of
the Trump campaign's slate of digital shows. In September, Trump Jr.
plans to return to the campaign trail four days a week; in October,
that's expected to increase to six days a week.

The greatest measure of his newfound political clout is the heated
competition among Republicans to offer the most sycophantic quote about
him. Gaetz hails Trump Jr. as ``the most dynamic voice that you hear in
American politics other than when it's preceded by `Hail to the
Chief.''' Sean Spicer, the former White House press secretary, calls him
``a downright rock star.'' Jeff Roe, Senator Ted Cruz's political
strategist, deems him ``a next-level, generational talent.'' Republicans
speak of Trump Jr.'s hunting-and-fishing prowess the way Red Guards once
talked about Mao swimming the Yangtze. ``I've shot with Green Beret
snipers,'' Daines says, ``and Donald Trump Jr. is as good a shot as
anybody I've ever shot with. He's a remarkable marksman. And by the way,
on fly fishing, too --- I'm not trying to exaggerate his skills, but
I've been around a lot of guys that fly fish, and he's a guide-quality
fly fisherman.''

\textbf{In addition to} Surabian, Trump Jr.'s innermost inner circle
consists of Arthur Schwartz, a New York Republican operative with a
reputation for the political dark arts; Tommy Hicks, a Texas
private-equity scion and hunting buddy of Trump Jr.'s who's now
co-chairman of the Republican National Committee; and Charlie Kirk, the
founder of Turning Point USA who became friendly with Trump Jr. when he
served as his body man during the final months of the 2016 campaign. A
little further outside are people like Richard Grenell, who served as
Trump's ambassador to Germany and acting director of national
intelligence; Cliff Sims; Sergio Gor; and young Republican congressmen
like Gaetz and Lee Zeldin of New York.

Through phone calls and text chains, the group --- which Gaetz calls
``the wolf pack''--- formulates Trump Jr.'s political moves. ``It could
be fairly argued that Don Jr. and his political team,'' a Trump adviser
says, ``have a better rapid-response operation than the White House
communications office has ever had.'' And Trump Jr.'s favorite form of
rapid response, like his father's, is the social media post. ``He stares
at his iPhone all the time,'' says a Republican operative who has
traveled with Trump Jr. ``He's locked and loaded.''

The wolf pack is made up of some of the most cynical and situational
people in G.O.P. politics, whose priorities oscillate between ``owning
the libs'' and loyalty enforcement among Republicans. Last October,
Trump Jr. began tweeting against Lindsey Graham for not doing enough, as
the Senate Judiciary Committee chairman, to protect his father from
impeachment, raising an online army under the hashtag \#WheresLindsey to
demand that Graham issue subpoenas on Trump's behalf. That month, Graham
attended a World Series game with the president. ``For at least three
innings, Lindsey was squawking at the president to get Don Jr. off his
ass,'' says Gaetz, who was with them at the game. (Graham's office
declined to comment.)

In February, Trump Jr. posted to his Instagram account a picture of Mitt
Romney, who had just voted to convict his father in the Senate
impeachment trial, in some tragically high-waisted jeans with the
caption, ``MOM JEANS: Because you're a pussy.'' It was a juvenile move,
and it was the subject of some debate within the wolf pack. Over lunch
that day, Trump Jr. asked Surabian what he thought of the meme. (If the
message is deemed too inflammatory, it will often appear on Schwartz's
Twitter account instead of Trump Jr.'s.) The two ultimately concluded
that while the language and image would undoubtedly generate negative
headlines, it would also grab eyeballs, and that Trump Jr.'s own
attached comment, calling on Romney to ``be expelled from the @GOP,''
justified the post. ``The meme got attention and guaranteed it went
viral,'' the person close to Trump Jr. says, ``but it was the message
that Mitt should be kicked out of the caucus that we cared about and
wanted to make sure got out there.''

During the 2016 campaign, Trump Jr. posted to Instagram a picture,
titled ``The Deplorables,'' of the faces of various high-profile Trump
supporters superimposed on the bodies of characters from the action
movie ``The Expendables''; one face was that of Pepe the Frog, a cartoon
character that had by then been embraced as a mascot by white
supremacists online. He also posted on Twitter a picture of a candy bowl
with the text: ``If I had a bowl of Skittles and I told you just three
would kill you. Would you take a handful? That's our Syrian refugee
problem.'' The trope of undesirable people hiding among good ones dates
to the Holocaust, and the ``poisoned candy'' metaphor had become popular
with xenophobes online.

In each instance, Trump Jr. professed ignorance. ``I've never even heard
of Pepe the Frog,'' he told George Stephanopoulos of ABC News. ``I
thought it was a frog in a wig. I thought it was funny.'' Elsewhere, he
said the Skittles picture was ``a statistical thing.'' And yet
throughout his father's presidency, Trump Jr. has preserved his winking
proximity to the far-right and conspiracist fringe, while avoiding his
father's clumsier cycles of embraces and disavowals.

The president in the past several months has routinely retweeted the
Twitter accounts of followers of the QAnon conspiracy theory, which
posits that Trump is doing battle with a cabal of Democratic and ``deep
state'' elites who run a child-sex-trafficking ring. When Trump was
asked about QAnon, which the F.B.I. has labeled a domestic terrorism
threat, at a White House news conference this month, he replied, ``I've
heard these are people that love our country.'' Trump Jr. has himself
avoided such signal-boosting and overt praise of QAnon, but in May he
posted to Instagram a picture of Joe Biden saying, ``See you later,
alligator!'' alongside an image of an alligator responding, ``In a
while, pedophile!'' When Jake Tapper of CNN subsequently called out the
Trumps for the smear, Trump Jr. responded on Twitter: ``Jake, I'm sorry
that you're more upset (Triggered!) about a joke meme than you are
@JoeBiden's gross habit of touching \& sniffing young girls.'' Early
this year, he posted a picture of himself to Instagram holding a custom
AR-15-style rifle emblazoned with the ``Jerusalem cross,'' a symbol used
by Christian soldiers during the Crusades that has been adopted by
far-right extremist groups; the rifle's magazine clip was decorated with
an image of Hillary Clinton behind bars.

Surabian, speaking for Trump Jr., told CNN that ``symbols on firearms
depicting various historical warriors are extremely common within the
Second Amendment community,'' and that the Clinton-behind-bars image was
a meme intended to ``mock Hillary Clinton'' and ``trigger humorless
liberals.'' As for the Biden-as-pedophile posts, Trump Jr. maintained
that he was just ``joking around.'' When I asked whether Trump Jr.
believes in the QAnon conspiracy theory, Surabian replied, ``Of course
not.'' But in July, Trump Jr. finally ran afoul of Twitter by tweeting a
viral video making false claims about hydroxychloroquine's efficacy in
treating Covid-19. Twitter hid the post from view and suspended his
tweeting privileges for 12 hours. ``Big tech is activist liberal,'' he
complained on Fox News.

By then, he was already adept at strategically picking fights outside
the conservative media bubble. Last fall, when he published his book
``Triggered'' --- a farrago of tossed-off personal history and
predictable political attacks that sold 287,000 hardcover copies,
thanks, in part, to bulk purchases by the Republican National Committee
--- Hachette Book Group pressed him to do some mainstream media
appearances. Trump Jr.'s team, seeking a spectacle, reached out to ``The
View.'' He came prepared. When Joy Behar asked him about his father
boasting on the ``Access Hollywood'' tape of sexually assaulting women,
Trump Jr. fired back that Behar had worn blackface to a Halloween party
in the 1970s and that Whoopi Goldberg had once defended Roman Polanski.
``We've all done things that we regret,'' he said, ``if we're talking
about bringing the discourse down.'' The only ``View'' host he didn't go
out of his way to antagonize was Meghan McCain, even offering her a
semi-apology for his father's attacks against hers. ``We realized that
the biggest headline to come out of his appearance could not be Meghan
McCain confronting him about his dad,'' says the person close to Trump
Jr.

\textbf{In February, Trump Jr.} traveled to Iowa on the eve of the
state's caucuses in a show of force. Although his father faced no
serious opposition for the Republican nomination, he was leading a group
of some 80 congressmen, cabinet members and other Republicans to stump
for Trump there and, more important, rough up the Democrats. He was just
about to speak at a ``Keep Iowa Great'' event outside Des Moines when a
Jewish protester began yelling that Trump Jr.'s father was responsible
for a rise in anti-Semitism. As the protester was hauled out of the
rally by security guards, Trump Jr. shouted, ``I don't think anyone's
done more for Israel and American Jews than Donald Trump!''

With the crowd cheering him on, he launched into a tirade against the
reporters in the room, then pledged to do everything in his power to
help his father win re-election. ``We don't just have to lose,'' he
said. ``We don't just have to roll over and die because the other side
wants us to and their buddies in the mainstream media want us to. That's
not how it works anymore.'' He added, ``We will fight harder than any
people you've ever seen for the next 10 months to make sure that this
continues.''

At the time, it seemed likely that Trump would win a second term. But
then the coronavirus happened, infecting over 5.5 million Americans
(including Guilfoyle) and killing more than 170,000 of them, paralyzing
the economy and imperiling Trump's re-election prospects. At the end of
the Democratic National Convention this month, he trailed Biden in the
RealClearPolitics national polling average by 7.6 points --- not an
insurmountable deficit, but a daunting one.

At the White House and inside the Trump campaign, there remains a
stubborn, almost defiant sense of optimism --- born, they believe, out
of experience --- that the president will win in November. ``I can say
that, having been there four years ago, things looked a lot worse back
in 2016 than whatever crisis politically the president might be going
through right now,'' Charlie Kirk says. When I asked Jason Miller about
the degree of worry inside his office, he replied, ``I haven't picked up
on any of the W-word that you just threw around in such a cavalier
fashion.''

But Trump Jr. is apparently worried. ``Don's the only person who thinks
they're going to lose,'' says a prominent conservative activist who is
in regular contact with him and other key members of Trump's political
operation. ``He's like, `We're losing, dude, and we're going to get
really hurt when we lose.''' An electoral defeat in November, Trump Jr.
fears, could result in federal prosecutions of Trump, his family and his
political allies. He has told the conservative activist that he expects
that a Biden administration will not participate in a ``peaceful
transition'' and instead will ``shoot the prisoners.'' (``This is 100
percent false,'' Surabian says. ``Don does not have these concerns.'')

Even assuming his worst fears aren't realized, a Trump defeat in
November would pose an existential question for Trump Jr. He has become
a figure of genuine political value, but that value remains mostly a
function of his status as the premier surrogate for his father. This is
the most treasured currency there is in a Republican Party in which
political fortunes now rise and fall based on proximity and devotion to
Donald Trump --- but what happens to that currency if Trump leaves the
stage? At the same time, it is difficult to see Trump Jr. coming fully
into his own as a political figure until he does what he struggled
unsuccessfully to do in his younger years: escape his father's shadow.
Although he would obviously prefer that his father win in November,
people close to him say that, in some ways, having Trump out of the
White House would be advantageous for Trump Jr. They use words like
``unshackled'' and ``free'' and speculate excitedly about his running
for office in Montana or Florida --- or, a few dare to dream, even the
presidency --- in 2024.

Those who are familiar with Trump Jr.'s thinking, though, say that's not
going to happen --- at least not in the next four years. ``Don can do
everything he wants to do in politics,'' says the person close to him,
``without running for office.'' The wolf pack talks about how Trump Jr.
would be a natural podcaster or talk-radio host; Fox News --- or perhaps
a new conservative TV channel --- could give him his own show. He has
expressed interest in playing a prominent role in a revivified National
Rifle Association and is open to the idea of serving as chairman of the
Republican National Committee. This month, he will release his second
book, ``Liberal Privilege,'' a rehash of Biden's various supposed sins
that he wrote during the pandemic lockdown.

Trump has marveled to aides at the response Trump Jr. received when,
before the pandemic, he appeared at Trump rallies, where he was
typically greeted with cheers of ``46! 46!'' (Donald Trump is the 45th
president of the United States.) ``It's sort of cool if you're at a
stadium of 15,000 people and they start chanting `46' when you're
speaking,'' Trump Jr. told the comedian Jim Norton in February when he
appeared on the satellite-radio show Norton hosts with Sam Roberts.
Still, he said, ``I don't know that I'd like the day job, and that's a
big part of it.''

Later in the interview, he complained that ``someone in the mainstream
media will write an article'' about his cursing on the show. (No one in
the mainstream media ever did, but then, ``owning the libs'' has never
required actual owning of the libs.) ``Did you ever think, though, that
you'd get to be an adult, and then there'd be somebody who wanted to
write a newspaper article that you used the F-word?'' Roberts asked.

``No, I did not,'' Trump Jr. said, ``because we're not adults, guys. The
reality is, like, there are no adults in the room anymore.''

\hypertarget{our-2020-election-guide}{%
\section{Our 2020 Election Guide}\label{our-2020-election-guide}}

Updated ~Sept. 8, 2020

\begin{center}\rule{0.5\linewidth}{\linethickness}\end{center}

\begin{itemize}
\item ~
  \hypertarget{the-latest}{%
  \subsection{The Latest}\label{the-latest}}

  \begin{itemize}
  \item
    President Trump and his party are using a playbook that aims to
    alarm people about crime in their backyards. It didn't work in 2018,
    but
    \href{https://www.nytimes3xbfgragh.onion/2020/09/08/us/politics/trump-republicans-fear-strategy.html?action=click\&pgtype=Article\&state=default\&region=BELOW_MAIN_CONTENT\&context=storylines_guide}{both
    parties think it could resonate more this year}.
  \end{itemize}
\item ~
  \hypertarget{how-to-win-270}{%
  \subsection{How to Win 270}\label{how-to-win-270}}

  \begin{itemize}
  \item
    Joe Biden and Donald Trump need 270 electoral votes to reach the
    White House. Try building
    \href{https://www.nytimes3xbfgragh.onion/interactive/2020/us/elections/election-states-biden-trump.html?action=click\&pgtype=Article\&state=default\&region=BELOW_MAIN_CONTENT\&context=storylines_guide}{your
    own coalition of battleground states}~to see potential outcomes.
  \end{itemize}
\item ~
  \hypertarget{voting-by-mail}{%
  \subsection{Voting by Mail}\label{voting-by-mail}}

  \begin{itemize}
  \item
    Will you have enough time to vote by mail in your state? Yes, but
    it's risky to procrastinate.
    \href{https://www.nytimes3xbfgragh.onion/interactive/2020/08/31/us/politics/vote-by-mail-deadlines.html?action=click\&pgtype=Article\&state=default\&region=BELOW_MAIN_CONTENT\&context=storylines_guide}{Check
    your state's deadline.}
  \item
    \href{https://www.nytimes3xbfgragh.onion/interactive/2020/us/elections/joe-biden.html?action=click\&pgtype=Article\&state=default\&region=BELOW_MAIN_CONTENT\&context=storylines_guide}{}

    \hypertarget{joe-biden}{%
    \section{Joe Biden}\label{joe-biden}}

    \hypertarget{democrat}{%
    \subsection{Democrat}\label{democrat}}

    \href{https://www.nytimes3xbfgragh.onion/interactive/2020/us/elections/donald-trump.html?action=click\&pgtype=Article\&state=default\&region=BELOW_MAIN_CONTENT\&context=storylines_guide}{}

    \hypertarget{donald-trump}{%
    \section{Donald Trump}\label{donald-trump}}

    \hypertarget{republican}{%
    \subsection{Republican}\label{republican}}
  \end{itemize}
\item
  \hypertarget{keep-up-with-our-coverage}{%
  \subsection{Keep Up With Our
  Coverage}\label{keep-up-with-our-coverage}}

  \begin{itemize}
  \item
    Get an
    \href{https://www.nytimes3xbfgragh.onion/newsletters/politics?action=click\&pgtype=Article\&state=default\&region=BELOW_MAIN_CONTENT\&context=storylines_guide}{email}~recapping
    the day's news
  \item
    Download our mobile app on
    \href{https://apps.apple.com/us/app/nytimes/id284862083?ls=1\&mat_click_id=5c79ae7455014fd1bd66b5610c05b8f2-20191112-16948\&referrer=mat_click_id\%3D5c79ae7455014fd1bd66b5610c05b8f2-20191112-16948\%26link_click_id\%3D722930677036718082}{iOS}~and
    \href{http://a.localytics.com/android?id=com.nytimes.android\&referrer=utm_source\%3Dother_nyt_mobile_web\%26utm_medium\%3DWeb\%2520page\%26utm_term\%3DGeneral\%2520Mobile\%2520Page\%26utm_campaign\%3DNYT\%2520Mobile\%2520General\%2520Page}{Android}~and
    turn on Breaking News and Politics alerts
  \end{itemize}
\end{itemize}

Advertisement

\protect\hyperlink{after-bottom}{Continue reading the main story}

\hypertarget{site-index}{%
\subsection{Site Index}\label{site-index}}

\hypertarget{site-information-navigation}{%
\subsection{Site Information
Navigation}\label{site-information-navigation}}

\begin{itemize}
\tightlist
\item
  \href{https://help.nytimes3xbfgragh.onion/hc/en-us/articles/115014792127-Copyright-notice}{©~2020~The
  New York Times Company}
\end{itemize}

\begin{itemize}
\tightlist
\item
  \href{https://www.nytco.com/}{NYTCo}
\item
  \href{https://help.nytimes3xbfgragh.onion/hc/en-us/articles/115015385887-Contact-Us}{Contact
  Us}
\item
  \href{https://www.nytco.com/careers/}{Work with us}
\item
  \href{https://nytmediakit.com/}{Advertise}
\item
  \href{http://www.tbrandstudio.com/}{T Brand Studio}
\item
  \href{https://www.nytimes3xbfgragh.onion/privacy/cookie-policy\#how-do-i-manage-trackers}{Your
  Ad Choices}
\item
  \href{https://www.nytimes3xbfgragh.onion/privacy}{Privacy}
\item
  \href{https://help.nytimes3xbfgragh.onion/hc/en-us/articles/115014893428-Terms-of-service}{Terms
  of Service}
\item
  \href{https://help.nytimes3xbfgragh.onion/hc/en-us/articles/115014893968-Terms-of-sale}{Terms
  of Sale}
\item
  \href{https://spiderbites.nytimes3xbfgragh.onion}{Site Map}
\item
  \href{https://help.nytimes3xbfgragh.onion/hc/en-us}{Help}
\item
  \href{https://www.nytimes3xbfgragh.onion/subscription?campaignId=37WXW}{Subscriptions}
\end{itemize}
