Sections

SEARCH

\protect\hyperlink{site-content}{Skip to
content}\protect\hyperlink{site-index}{Skip to site index}

\href{https://www.nytimes3xbfgragh.onion/section/nyregion}{New York}

\href{https://myaccount.nytimes3xbfgragh.onion/auth/login?response_type=cookie\&client_id=vi}{}

\href{https://www.nytimes3xbfgragh.onion/section/todayspaper}{Today's
Paper}

\href{/section/nyregion}{New York}\textbar{}Can N.Y.C. Schools Open on
Time? De Blasio Is Under Pressure to Delay

\url{https://nyti.ms/3axTomm}

\begin{itemize}
\item
\item
\item
\item
\item
\end{itemize}

\hypertarget{school-reopenings}{%
\subsubsection{\texorpdfstring{\href{https://www.nytimes3xbfgragh.onion/spotlight/schools-reopening?name=styln-coronavirus-schools-reopening\&region=TOP_BANNER\&block=storyline_menu_recirc\&action=click\&pgtype=Article\&impression_id=f9f414c0-f280-11ea-8e1e-df7c7f9e9d81\&variant=undefined}{School
Reopenings}}{School Reopenings}}\label{school-reopenings}}

\begin{itemize}
\tightlist
\item
  \href{https://www.nytimes3xbfgragh.onion/2020/09/04/us/bar-exam-coronavirus.html?name=styln-coronavirus-schools-reopening\&region=TOP_BANNER\&block=storyline_menu_recirc\&action=click\&pgtype=Article\&impression_id=f9f414c1-f280-11ea-8e1e-df7c7f9e9d81\&variant=undefined}{Delayed
  Licensing Exams}
\item
  \href{https://www.nytimes3xbfgragh.onion/2020/09/08/upshot/children-testing-shortfalls-virus.html?name=styln-coronavirus-schools-reopening\&region=TOP_BANNER\&block=storyline_menu_recirc\&action=click\&pgtype=Article\&impression_id=f9f414c2-f280-11ea-8e1e-df7c7f9e9d81\&variant=undefined}{Limited
  Testing for Children}
\item
  \href{https://www.nytimes3xbfgragh.onion/2020/09/01/world/schools-reopen-globe-students.html?name=styln-coronavirus-schools-reopening\&region=TOP_BANNER\&block=storyline_menu_recirc\&action=click\&pgtype=Article\&impression_id=f9f414c3-f280-11ea-8e1e-df7c7f9e9d81\&variant=undefined}{School
  Around the World}
\item
  \href{https://www.nytimes3xbfgragh.onion/interactive/2020/us/covid-college-cases-tracker.html?name=styln-coronavirus-schools-reopening\&region=TOP_BANNER\&block=storyline_menu_recirc\&action=click\&pgtype=Article\&impression_id=f9f43bd0-f280-11ea-8e1e-df7c7f9e9d81\&variant=undefined}{Tracking
  College Cases}
\end{itemize}

Advertisement

\protect\hyperlink{after-top}{Continue reading the main story}

Supported by

\protect\hyperlink{after-sponsor}{Continue reading the main story}

\hypertarget{can-nyc-schools-open-on-time-de-blasio-is-under-pressure-to-delay}{%
\section{Can N.Y.C. Schools Open on Time? De Blasio Is Under Pressure to
Delay}\label{can-nyc-schools-open-on-time-de-blasio-is-under-pressure-to-delay}}

New York City is learning that getting the virus under control is only a
first step on the path to reopening schools.

\includegraphics{https://static01.graylady3jvrrxbe.onion/images/2020/08/17/nyregion/00nyvirus-schools1/merlin_175313658_1e2ab1fd-5893-41ea-81e7-3d2edf36f976-articleLarge.jpg?quality=75\&auto=webp\&disable=upscale}

\href{https://www.nytimes3xbfgragh.onion/by/eliza-shapiro}{\includegraphics{https://static01.graylady3jvrrxbe.onion/images/2018/12/28/multimedia/author-eliza-shapiro/author-eliza-shapiro-thumbLarge.png}}

By \href{https://www.nytimes3xbfgragh.onion/by/eliza-shapiro}{Eliza
Shapiro}

\begin{itemize}
\item
  Published Aug. 18, 2020Updated Sept. 1, 2020
\item
  \begin{itemize}
  \item
  \item
  \item
  \item
  \item
  \end{itemize}
\end{itemize}

With the
\href{https://www.nytimes3xbfgragh.onion/2020/08/05/nyregion/nyc-schools-reopening.html}{planned
first day of school in New York City rapidly approaching}, Mayor Bill de
Blasio is facing mounting pressure from the city's teachers, principals
and even members of his own administration to delay the start of
in-person instruction by several weeks to give educators more time to
prepare.

Mr. de Blasio has been hoping to reopen the nation's largest school
system on a part-time basis for the city's 1.1 million schoolchildren
this fall --- a feat no other big-city mayor is currently even
attempting.

If
\href{https://www.nytimes3xbfgragh.onion/2020/09/01/nyregion/schools-open-coronavirus-nyc.html}{New
York is able to reopen schools} safely, it would be an extraordinary
turnaround for a city that was a global epicenter of the pandemic just a
few months ago. Schools are the key to the city's long path back to
normalcy: opening classrooms would help jump start the struggling
economy by allowing more parents to return to work, and would provide
desperately needed services for tens of thousands of vulnerable
students.

But Mr. de Blasio's push to reopen on time is now facing its most
serious obstacle yet:
\href{https://www.nytimes3xbfgragh.onion/2020/08/14/nyregion/nyc-schools-reopening-plan.html}{the
city's principals, tasked with actually implementing the reopening plan,
are questioning the city's readiness}.

``We are now less than one month away from the first day of school and
still without sufficient answers to many of the important safety and
instructional questions we've raised,'' Mark Cannizzaro, president of
the city's principals' union, wrote in a letter last week, calling on
the mayor to heed his members' ``dire warnings.''

New York City has a virus transmission rate so low that it is closer to
that of South Korea's than of many other American cities, and there is
\href{https://www.nytimes3xbfgragh.onion/2020/08/07/health/coronavirus-ny-schools-reopen.html}{agreement
among many public health experts that the city's infection rate is low
enough to reopen at least some schools, with strict safety measures in
place}.

But New York is learning that having the virus under control ---
something few other places in the country have managed to do --- is only
the first step to reopening schools.

Pulling off a hybrid learning plan, with children reporting to school a
few days a week to allow for social distancing, has presented profound
logistical challenges.
\href{https://www.nytimes3xbfgragh.onion/2020/08/07/us/remote-learning-fall-2020.html}{Like
many other districts}, the city is discovering that months spent
figuring out how to safely reopen buildings may not have left enough
time to focus on instruction.

The city's scramble to make a hybrid model work has daunting
implications for other school systems, which are waiting for their virus
case loads to go down before they even consider partially reopening.

Some New York City charter schools and private schools have already
delayed their own plans for a part-time reopening, and have opted to
start the year with only remote learning.

The city's public school principals say they do not know how many of
their students will actually report to buildings on the planned first
day, Sept. 10, because there is no deadline for families to switch from
hybrid learning to remote only. So far, about 30 percent of city
families have said they will start the year remotely, but that number
could change significantly before the start of school.

That has made it all but impossible for principals to plan their class
schedules, and to determine how many teachers they will need to staff
remote instruction, in-person learning, or both. Principals say they
need more than the two work days in September that the city has allotted
for them to meet with teachers in order to make decisions about
staffing.

And though the city has begun to ship personal protective equipment and
cleaning supplies to schools, and has made strides in preparing many of
its aging buildings for reopening, there are lingering questions about
how many classrooms will have proper ventilation, and about how
frequently staff and students will be tested after buildings open.

The city's biggest obstacle appears to be time.

Mr. de Blasio's administration began preparing for school reopening in
earnest late in the spring, according to several people with direct
knowledge of the planning process, both to focus on preparing for summer
school and to assess the viability of reopening at a moment when the
virus was only beginning to loosen its grip on the city.

But reopening would be a daunting logistical endeavor for even a small
district, and three months was not nearly enough time to pull together
plans for 1,800 schools, said Mr. Cannizzaro.

Though the principals' union is smaller and much less powerful than the
city's teachers' union, concerns from school leaders carry particular
weight since they rarely wade into political fights --- and because
principals have been tasked with actually making reopening work.

Over the last few days, a growing number of principals have come forward
to say that there just isn't enough time.

About fifty school leaders
\href{https://docs.google.com/document/d/1s6RLidO59mPKy_Qxqrr2Vp771y0f41LzNuMSgLK-CI8/edit}{signed
a letter that called for a delay to in-person instruction} until the end
of September and included a detailed plan for how to phase children into
schools over the course of the fall, starting with young children.

The specificity of their request may make it harder for Mr. de Blasio to
ignore. Michael Mulgrew, president of the United Federation of Teachers,
has said he does not believe city schools will be ready on Sept. 10, but
has not yet offered up an alternative plan.

In recent days, Jumaane Williams, the city's public advocate, and Mark
Treyger, chair of the City Council's education committee and a close
ally of the city's teachers' union, have joined the chorus asking Mr. de
Blasio to allow more time before the physical reopening of classrooms.

Last week, the union representing school aides, lunch cooks and other
school staff, many of whom have been working in school buildings
throughout the pandemic,
\href{https://gothamist.com/news/lunch-workers-union-school-aides-urge-de-blasio-delay-nyc-school-reopening}{requested
a delay of 30 days before schools physically reopen}.

The mayor has so far resisted a delay. The decision about whether to
reopen and when is his alone; Gov. Andrew M. Cuomo, who has contradicted
the mayor on a number of education decisions over the last several
months,
\href{https://www.nytimes3xbfgragh.onion/2020/08/07/nyregion/cuomo-schools-reopening.html}{said
schools across the state are cleared to reopen as long as they are in a
region with a test positivity rate under 5 percent}.

\href{https://www.nytimes3xbfgragh.onion/spotlight/schools-reopening?action=click\&pgtype=Article\&state=default\&region=MAIN_CONTENT_3\&context=storylines_keepup}{}

\hypertarget{school-reopenings-}{%
\subsubsection{School Reopenings ›}\label{school-reopenings-}}

\hypertarget{back-to-school}{%
\paragraph{Back to School}\label{back-to-school}}

Updated Sept. 8, 2020

The latest on how schools are reopening amid the pandemic.

\begin{itemize}
\item
  \begin{itemize}
  \tightlist
  \item
    The first day of school is an annual rite of passage. But this year,
    it looks very different for tens of millions of students.
    \href{https://www.nytimes3xbfgragh.onion/2020/09/05/us/virtual-return-to-school-covid.html?action=click\&pgtype=Article\&state=default\&region=MAIN_CONTENT_3\&context=storylines_keepup}{We
    talked to some about their hopes and fears}.
  \item
    Coronavirus cases
    \href{https://www.nytimes3xbfgragh.onion/2020/09/06/us/colleges-coronavirus-students.html?action=click\&pgtype=Article\&state=default\&region=MAIN_CONTENT_3\&context=storylines_keepup}{are
    spiking in America's college towns}, leading to concern that young
    people who are infected will contribute to a spread of the virus.
  \item
    A growing number of Catholic schools across the country are
    \href{https://www.nytimes3xbfgragh.onion/2020/09/05/us/catholic-school-closings.html?action=click\&pgtype=Article\&state=default\&region=MAIN_CONTENT_3\&context=storylines_keepup}{shutting
    down forever during the coronavirus pandemic}, citing insurmountable
    financial pressure.
  \item
    The magazine's Ethicist columnist answers a question from a
    spokesperson at a major university:
    \href{https://www.nytimes3xbfgragh.onion/2020/09/08/magazine/university-reopening-safety-ethics.html?action=click\&pgtype=Article\&state=default\&region=MAIN_CONTENT_3\&context=storylines_keepup}{Can
    I promote a reopening plan I have doubts about}?
  \end{itemize}
\end{itemize}

Last week, Mr. de Blasio partially chalked up concerns as part of
typical negotiations between City Hall and the city's labor force,
saying, ``unions will always sound various alarms, and unions will say
things sometimes in a very dramatic fashion.''

The mayor went further the following day, saying of educators' concerns,
``Sometimes people think that if you raise enough questions and doubts,
folks will run away and hide. That's not what I do. That's not what New
Yorkers do. We just don't surrender.''

Rachael Bedard, a doctor who serves inmates on Rikers Island and who
supports school reopening,
\href{https://twitter.com/rachaelbedard/status/1293926522605142017}{wrote
in a post on Twitter} that the mayor's remarks were emblematic of what
she called a misguided approach to reopening.

``No one is asking for surrender,'' Dr. Bedard said. ``All stakeholders
want to understand why the mayor thinks this is safe when so many others
seem to think it isn't. He misses every opportunity to reassure, to
educate and to empathize.''

Mr. de Blasio has said he is committed to having as many students back
in classrooms as safely possible, and has argued that in-person
instruction is crucial for the city's public school students, who are
mostly low-income and Black and Latino. That assertion is widely
supported by both public health and education experts.

The mayor's concerns about remote learning were the major reason he
resisted closing the schools in March as the virus was spreading largely
undetected throughout New York, despite pleas from parents and
educators. Mr. de Blasio's hesitation widened the rift between his
administration and the teachers' union, which began calling for schools
to close several days before the mayor ultimately relented.

That relationship has reached a nadir over the summer. Mr. Mulgrew has
said his members no longer trust the mayor after the delayed school
closure decision this spring.

Though it is illegal for teachers to strike in New York, Mr. Mulgrew has
threatened to sue the city if schools reopen before his union deems it
safe, and indicated to his members this week that he might support an
unauthorized strike.

That would be a nightmarish scenario for Mr. de Blasio, who was once a
public school parent himself and speaks frequently about his feeling of
deep connection to the city's schools and educators.

Some people in Mr. de Blasio's administration have been privately urging
him to avoid such an outcome by announcing a revised timeline on
in-person instruction, but the mayor is so far determined to forge ahead
with the original date, according to several people with knowledge of
his thinking.

Even if the mayor ultimately decides to delay, there is no guarantee
that classrooms will physically reopen later in the fall --- or even for
months to come.

Though the city's test positivity rate is currently around 1 percent,
Mr. de Blasio has said he will not open schools or will close them if
that rate reaches 3 percent. Experts predict that the city's rate could
tick up as cold weather arrives and New Yorkers start congregating
indoors.

Mr. de Blasio has said he expects school to return to normal after there
is a vaccine.

Though New York received a tragic lesson in the unpredictability of the
virus this spring, Mr. Cannizzaro said he believes a delayed start to
in-person instruction would be the best way to ensure that children can
actually return to school buildings in a sustained way.

``This ask for time is to make sure when we get kids in buildings,
families like it enough that they want to stay,'' he said.

Advertisement

\protect\hyperlink{after-bottom}{Continue reading the main story}

\hypertarget{site-index}{%
\subsection{Site Index}\label{site-index}}

\hypertarget{site-information-navigation}{%
\subsection{Site Information
Navigation}\label{site-information-navigation}}

\begin{itemize}
\tightlist
\item
  \href{https://help.nytimes3xbfgragh.onion/hc/en-us/articles/115014792127-Copyright-notice}{©~2020~The
  New York Times Company}
\end{itemize}

\begin{itemize}
\tightlist
\item
  \href{https://www.nytco.com/}{NYTCo}
\item
  \href{https://help.nytimes3xbfgragh.onion/hc/en-us/articles/115015385887-Contact-Us}{Contact
  Us}
\item
  \href{https://www.nytco.com/careers/}{Work with us}
\item
  \href{https://nytmediakit.com/}{Advertise}
\item
  \href{http://www.tbrandstudio.com/}{T Brand Studio}
\item
  \href{https://www.nytimes3xbfgragh.onion/privacy/cookie-policy\#how-do-i-manage-trackers}{Your
  Ad Choices}
\item
  \href{https://www.nytimes3xbfgragh.onion/privacy}{Privacy}
\item
  \href{https://help.nytimes3xbfgragh.onion/hc/en-us/articles/115014893428-Terms-of-service}{Terms
  of Service}
\item
  \href{https://help.nytimes3xbfgragh.onion/hc/en-us/articles/115014893968-Terms-of-sale}{Terms
  of Sale}
\item
  \href{https://spiderbites.nytimes3xbfgragh.onion}{Site Map}
\item
  \href{https://help.nytimes3xbfgragh.onion/hc/en-us}{Help}
\item
  \href{https://www.nytimes3xbfgragh.onion/subscription?campaignId=37WXW}{Subscriptions}
\end{itemize}
