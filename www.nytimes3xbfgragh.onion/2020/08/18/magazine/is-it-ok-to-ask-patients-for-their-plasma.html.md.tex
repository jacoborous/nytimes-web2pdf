Sections

SEARCH

\protect\hyperlink{site-content}{Skip to
content}\protect\hyperlink{site-index}{Skip to site index}

\href{https://myaccount.nytimes3xbfgragh.onion/auth/login?response_type=cookie\&client_id=vi}{}

\href{https://www.nytimes3xbfgragh.onion/section/todayspaper}{Today's
Paper}

Is It OK to Ask Patients for Their Plasma?

\url{https://nyti.ms/3kSrXbH}

\begin{itemize}
\item
\item
\item
\item
\item
\item
\end{itemize}

Advertisement

\protect\hyperlink{after-top}{Continue reading the main story}

Supported by

\protect\hyperlink{after-sponsor}{Continue reading the main story}

\href{/column/the-ethicist}{The Ethicist}

\hypertarget{is-it-ok-to-ask-patients-for-their-plasma}{%
\section{Is It OK to Ask Patients for Their
Plasma?}\label{is-it-ok-to-ask-patients-for-their-plasma}}

\includegraphics{https://static01.graylady3jvrrxbe.onion/images/2020/08/23/magazine/23Ethicist/23Ethicist-articleLarge.jpg?quality=75\&auto=webp\&disable=upscale}

By Kwame Anthony Appiah

\begin{itemize}
\item
  Aug. 18, 2020
\item
  \begin{itemize}
  \item
  \item
  \item
  \item
  \item
  \item
  \end{itemize}
\end{itemize}

\emph{A friend, who works for a hospital system as an administrative
nurse, has been tasked with cold-calling patients who tested positive
for the coronavirus to ask if they would donate their plasma to help
critically ill patients. Many of these patients are people of color,
live in households where several people are infected, are part of a
local immigrant community and may have limited insurance coverage and be
dependent on public assistance. Is it unethical to use private
information to solicit help from individuals who already suffer
disproportionately?} Name Withheld

\textbf{Basic privacy rules} prevent a health care provider from sharing
patients' medical information without their consent. They don't prevent
a health care provider from contacting patients and letting them know
that they could do some good by donating plasma. Privacy limits what can
be done with information. It doesn't mean that it can't be used at all.

The invitation to help others is one they are free to reject. But merely
extending that invitation isn't a burdensome intrusion. You note that
many of these patients are people of color and that many may be
dependent on public assistance. This vulnerable population, we know, has
suffered a disproportionate incidence of coronavirus infection. Yet you
don't express any doubt that the hospital would use the convalescent
plasma fairly, providing it solely on the basis of need. Under those
circumstances, any help that members of this community can offer may
benefit this community disproportionately, too.

Perhaps you think that, because these people are in financial need, it
would be better to offer them money for their plasma. There is a long
history of discussion about whether it is a good idea to pay people for
plasma (as we sometimes do in this country) or for organ donations
(which we don't). My own view is that our health care system would be
better served, ethically, by avoiding the creation of markets in things
like patient-derived antibodies. If many of these potential donors don't
have enough money, we should address this by more adequate public
assistance, of one sort or another, or a basic income. Exploitation
involves taking advantage of people's vulnerabilities to get them to do
things they otherwise wouldn't. Paradoxically, the offer of money in
these circumstances would be more exploitative, not less.

\emph{My wife and I are in our mid-60s, living in Manhattan; my
mother-in-law is in her early 90s and lives alone in a nearby suburb. At
the beginning of the pandemic, we were doing grocery shopping for her,
dropping bags on her front steps. We tried to get groceries delivered
(for us and for her) for weeks, without luck. Finally, we got a slot for
pickup at a local Fairway; an Instacart worker would shop for groceries,
and you could pick them up at a window. Just hours before our scheduled
pickup, our order was rescheduled for the following week. We found out
that Instacart workers were walking out two days before our pickup, in
protest of poor working conditions. My wife and I debated canceling our
order, because our family is pro-union and pro-worker, but we decided
not to. A relative found out about this and took us to task on Facebook.
Were we wrong not to cancel?} Name Withheld

\textbf{The Instacart walkouts} in the spring, as reported in this
paper, aimed to increase backlogs in service, thus putting pressure on
management to meet the workers' demands for better pay and conditions.
(Instacart, for the record, has denied that the protests affected its
operations.) But the workers involved didn't picket drop-off sites; once
customers paid for their goods and traveled to collect them, it would
have been asking a lot to have them turn around and return home
empty-handed. So you weren't failing to do something that these workers
were asking you to do. And because your aim was to reduce medical risk
to you and your mother-in-law, what you should have done would depend in
part on whether there was another reasonable way to achieve that.

You could still have tried to express solidarity with the workers by
canceling your order, as your relative suggested. But expressions of
solidarity are, precisely, expressions: They gain their point not from
making you feel good but from communicating support. And unless lots of
people did so in a coordinated way, your decision wouldn't have
communicated anything.

You might, on your own, conclude that it was a good idea to organize a
boycott of this company until certain conditions were met. But it's the
workers who are in the best position to balance, say, the threat of job
loss caused by reduced business against the prospects of a successful
campaign. And you were aware of no such campaign, no such request. If
you want to support the workers, shouldn't you take direction from the
workers themselves?

\emph{I am a women's health care provider who believes that health care
should be accessible to all. Usually, I receive one or two texts or
calls a week from family, friends and acquaintances seeking a second
opinion, a prescription or a diagnosis. With the pandemic, I am getting
at least one or two requests daily from people who are afraid to go see
their provider or can't get in touch with them. Most of these patients
are experiencing what they consider to be an emergency, and many are
pregnant. They will start a text or voice mail with some version of ``If
I don't hear from you in the next hour, I'll head to the hospital.'' A
vast majority of these people really should not be going to the
hospital, especially in the setting of this pandemic --- we are in a
large city considered a Covid-19 hot spot.}

\emph{Now, after a long shift at the hospital or my clinic, I often face
a series of frantic messages; many meals with my family are interrupted
by calls from people seeking reassurance. I feel obliged to help keep
them out of the emergency room, but I find losing even a few hours of my
limited free time to be overwhelming. I believe I have a duty to
disseminate the knowledge I have, but at what cost?} Name Withheld

\textbf{We should all} be grateful for what you've been doing. If one or
two of your calls a week are relieving the burden on emergency rooms,
this alone would represent a significant public benefit. Yet the
situation you're describing is unsustainable --- and a consequence of
systemic difficulties. In an affluent country with a well-organized
health care system, everyone (not least pregnant women) would have a
primary point of contact they could consult in circumstances like these.
Your career choices have been shaped by your awareness that we don't
have such a system.

Still, giving medical advice is what you are paid to do. When people ask
you to provide unremunerated medical consultation, they're effectively
taking advantage of you, relying on your unbounded sense of
responsibility. It's understandable --- they're fueled by fear and even
desperation --- but that doesn't make it right.

A major reason you should ration these free services is that you are a
precious resource yourself, and --- especially given that you're already
overworked --- you must safeguard your physical and emotional health. If
you're overwhelmed by these entreaties, you're at risk of being less
effective when it comes to your own patients. You mention meals with
your family, and you may justly feel that family members themselves may
have some claim on your time.

Think about a short message --- conveyed on voice mail and on email and
SMS auto-responders --- that says something like: ``As you can imagine,
I'm very busy at the moment. If you're not my patient, please consider
contacting your own health care provider rather than leaving questions
for me.'' If you're on social media, you could post about these
challenges so that your friends might pause before asking for your free
expertise. We should all think twice before calling on medical people
whose phone number or email address we happen to have for advice we
should be getting from our own health care providers or from online
sources like the Centers for Disease Control and Prevention.

Advertisement

\protect\hyperlink{after-bottom}{Continue reading the main story}

\hypertarget{site-index}{%
\subsection{Site Index}\label{site-index}}

\hypertarget{site-information-navigation}{%
\subsection{Site Information
Navigation}\label{site-information-navigation}}

\begin{itemize}
\tightlist
\item
  \href{https://help.nytimes3xbfgragh.onion/hc/en-us/articles/115014792127-Copyright-notice}{©~2020~The
  New York Times Company}
\end{itemize}

\begin{itemize}
\tightlist
\item
  \href{https://www.nytco.com/}{NYTCo}
\item
  \href{https://help.nytimes3xbfgragh.onion/hc/en-us/articles/115015385887-Contact-Us}{Contact
  Us}
\item
  \href{https://www.nytco.com/careers/}{Work with us}
\item
  \href{https://nytmediakit.com/}{Advertise}
\item
  \href{http://www.tbrandstudio.com/}{T Brand Studio}
\item
  \href{https://www.nytimes3xbfgragh.onion/privacy/cookie-policy\#how-do-i-manage-trackers}{Your
  Ad Choices}
\item
  \href{https://www.nytimes3xbfgragh.onion/privacy}{Privacy}
\item
  \href{https://help.nytimes3xbfgragh.onion/hc/en-us/articles/115014893428-Terms-of-service}{Terms
  of Service}
\item
  \href{https://help.nytimes3xbfgragh.onion/hc/en-us/articles/115014893968-Terms-of-sale}{Terms
  of Sale}
\item
  \href{https://spiderbites.nytimes3xbfgragh.onion}{Site Map}
\item
  \href{https://help.nytimes3xbfgragh.onion/hc/en-us}{Help}
\item
  \href{https://www.nytimes3xbfgragh.onion/subscription?campaignId=37WXW}{Subscriptions}
\end{itemize}
