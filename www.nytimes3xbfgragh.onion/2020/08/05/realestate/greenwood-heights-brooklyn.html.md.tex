Sections

SEARCH

\protect\hyperlink{site-content}{Skip to
content}\protect\hyperlink{site-index}{Skip to site index}

\href{https://www.nytimes3xbfgragh.onion/section/realestate}{Real
Estate}

\href{https://myaccount.nytimes3xbfgragh.onion/auth/login?response_type=cookie\&client_id=vi}{}

\href{https://www.nytimes3xbfgragh.onion/section/todayspaper}{Today's
Paper}

\href{/section/realestate}{Real Estate}\textbar{}Greenwood Heights,
Brooklyn: Open Space and Room to Breathe

\url{https://nyti.ms/3gvPwVf}

\begin{itemize}
\item
\item
\item
\item
\item
\item
\end{itemize}

Advertisement

\protect\hyperlink{after-top}{Continue reading the main story}

Supported by

\protect\hyperlink{after-sponsor}{Continue reading the main story}

Living in

\hypertarget{greenwood-heights-brooklyn-open-space-and-room-to-breathe}{%
\section{Greenwood Heights, Brooklyn: Open Space and Room to
Breathe}\label{greenwood-heights-brooklyn-open-space-and-room-to-breathe}}

The neighborhood around the Green-Wood Cemetery is known for being
`spacious and airy' --- an appealing quality in the age of coronavirus.

\href{https://www.nytimes3xbfgragh.onion/slideshow/2020/08/05/realestate/living-in-greenwood-heights-brooklyn.html}{}

\hypertarget{living-in--greenwood-heights-brooklyn}{%
\subsection{Living In ... Greenwood Heights,
Brooklyn}\label{living-in--greenwood-heights-brooklyn}}

11 Photos

View Slide Show ›

\includegraphics{https://static01.graylady3jvrrxbe.onion/images/2020/08/05/realestate/05LIVING-GREENWOODHEIGHTS-slide-GQMA/05LIVING-GREENWOODHEIGHTS-slide-GQMA-articleLarge.jpg?quality=75\&auto=webp\&disable=upscale}

Karsten Moran for The New York Times

By Andrew Cotto

\begin{itemize}
\item
  Published Aug. 5, 2020Updated Aug. 6, 2020
\item
  \begin{itemize}
  \item
  \item
  \item
  \item
  \item
  \item
  \end{itemize}
\end{itemize}

In this season of quarantine, New Yorkers have been flocking to
Green-Wood Cemetery, the 478-acre green space a few blocks south of
Prospect Park in Brooklyn, for socially distant excursions. Many are
also discovering the qualities of its lesser-known namesake, Greenwood
Heights, the neighborhood that cradles it on three sides.

``I've been bringing people out to Greenwood for years,'' said James
Kerby, a Brooklyn-based agent with Douglas Elliman Real Estate. ``Now
I'm here all the time. It's so spacious and airy, so quiet, safe.''

And because buying a place there is less expensive than in surrounding
neighborhoods like South Slope and Windsor Terrace, he added, ``you can
take people who are living in apartments and put them in homes.''

BROOKLYN

South Slope

Open Source Gallery

GOWANUS EXPWY.

PROSPECT EXPWY.

FOURTH AVE.

FIFTH AVE.

Greenwood

Heights

Green-Wood

CEMETERY

Sunset

Park

38TH ST.

FORT HAMILTON

PKWY.

Greenwood

Heights

D

Borough Park

1/4 mile

By The New York Times

Those homes include a mix of single- and multifamily dwellings, prewar
apartment buildings and new developments. Few of the buildings extend
above the tree line, which keeps density down and the vibe low-key and
friendly.

Fred Cray, 63, a photographer, has lived in the area since 1996.
``Initially, I was reluctant to move to this neighborhood, as it seemed
remote. I needed more space, and it was affordable compared to
elsewhere,'' Mr. Cray said. ``What I continue to appreciate, to this
day, is the interaction and developed friendships with neighbors, as
well as the low-rise neighborhood feel. As the city has changed, there
remains plenty of immediate outdoor space, relative quiet and good
neighborhood restaurants.''

A common refrain among real estate agents and residents is how often
people move within the neighborhood. Kirstie Pendergrass, 46, a massage
therapist, her husband, Keith Malvetti, 48, a technical program manager
at Google, and their 7-year old son are in the process of doing just
that.

``When we moved here in 2009, I remember telling people I liked the fact
that there were lots of people in the neighborhood who had been here for
many, many years,'' Ms. Pendergrass said. ``That's part of why I want to
stay.''

Another factor is the green space. ``Because I'm a country girl at heart
and would rather be living in open spaces, being very close to
Green-Wood Cemetery is so important to me,'' she said. ``That space has
also been a huge source of comfort to me during the pandemic.''

\includegraphics{https://static01.graylady3jvrrxbe.onion/images/2020/08/09/realestate/05GAZZ-Greenwood3/merlin_175263669_b00ddda5-100a-4ce5-ab96-7859651d7fdb-articleLarge.jpg?quality=75\&auto=webp\&disable=upscale}

\hypertarget{what-youll-find}{%
\subsection{What You'll Find}\label{what-youll-find}}

Greenwood Heights is bordered by South Slope to the north, Sunset Park
to the west and south, Borough Park to the south and Windsor Terrace to
the east. It is generally recognized as extending to Fourth Avenue (but
not all the way to the bay) on the west, the Prospect Expressway on the
north and 38th Street on the south.

Besides a few fast-food places on Fourth Avenue, there are no chain
stores or restaurants; there are no Citibanks or Citi Bikes. On Fifth
Avenue, the main commercial thoroughfare, many of the businesses and
restaurants cater to Spanish-speaking residents. Luigi's Pizza has been
serving slices and squares since 1973, and the once-prominent Polish
population is represented by
\href{https://www.nytimes3xbfgragh.onion/2013/01/13/nyregion/jubilat-provisions-a-polish-market-with-a-roll-call-of-kielbasa.html}{Jubilat
Provisions}, a Polish market.

Signs of modest gentrification abound in tattoo parlors, hipster bars
and restaurants like Tambour Bistro \& Wine Bar, on Fifth Avenue, where
live jazz and blues can once again be heard by those enjoying outdoor
table service. Greenwood Grape \& Still, a boutique wine shop, opened in
2016 on a busy four-lane stretch of Fourth Avenue dominated by tire
shops, storefront churches and bodegas.

A more tranquil stretch of urban living can be found along the leafy
blocks of Sixth Avenue, with its quartet of cozy restaurants:
Giuseppina's Brick Oven Pizza, Battle Hill Tavern, Brooklyn Pub and Lot
2, a neighborhood favorite that has been serving comfort food since
2008. All four remained open for delivery and takeout during the
lockdown; they are now offering alfresco dining in one of the few areas
of the city where the sidewalk cafes and parking-spot patios don't feel
entirely ad hoc. Striking sunsets and unobstructed views of New York
Harbor and the Statue of Liberty add to the appeal.

Near the northern border of the cemetery, on Seventh Avenue, is
Greenwood Park, a 13,000-square-foot, indoor-outdoor beer garden with
bocce courts and a seasonal menu. That part of the neighborhood,
bordering the Prospect Expressway, is also home to an esplanade, a dog
run and Open Source Gallery, a nonprofit arts organization.

But the main attraction is the cemetery, a verdant display of statues,
monuments and mausoleums on acres of rolling hills, with ponds and some
7,000 trees. The resting place of such New York luminaries as
Jean-Michel Basquiat and Leonard Bernstein, it offers wide lanes and
cobbled paths for strolling, as well as Brooklyn's highest natural
elevation. The main entrance, marked by an elaborate Gothic archway, is
at 25th Street and Fifth Avenue, directly across from one of the area's
best-kept secrets: Baked in Brooklyn, a commercial bakery with a small
retail space offering baked goods and savory snacks.

Image

152 18TH STREET \textbar{} A three-family house built in 1910, listed
for \$1.45 million. 917-921-7180Credit...Karsten Moran for The New York
Times

\hypertarget{what-youll-pay}{%
\subsection{What You'll Pay}\label{what-youll-pay}}

As of early August, there were 38 properties on the market in the
neighborhood, at a median listing price of \$945,034 --- from a
one-bedroom, one-bathroom condominium listed for \$499,000 to a
two-family house with four bedrooms and four bathrooms listed for \$2.4
million.

According to a recent report by the Corcoran Group, at the same time
last year there were 68 properties listed, at a median price of
\$993,195 --- a decrease of 4 percent in price and 44 percent in the
number of listings. So far in 2020, 69 listings had sold at a median
price of \$867,730, down from 76 sales at median price of \$1,004,198
during the same period in 2019.

But even with the decreases in prices and volume, Mr. Kerby at Douglas
Elliman is still bullish on the area, particularly for its housing
stock. ``I think there will be a shift toward townhouses,'' he said.
``Condo/co-op owners are looking for more realistic work-at-home
situations, private yards, more space, less restrictions on lobby and
amenity space and no elevators. Townhouses are also more sellable in the
Covid environment, where there have been restrictions on access for
buyers, appraisers, inspectors, etc.''

As for rental units, according to StreetEasy the median monthly rent in
Greenwood Heights is currently \$2,500, up slightly from \$2,450 during
the same period last year.

Image

289 23RD STREET \textbar{} A two-bedroom, two-bathroom house, built in
1901 on 932 square feet, listed for \$1.25 million.
917-679-8189Credit...Karsten Moran for The New York Times

\hypertarget{the-vibe}{%
\subsection{The Vibe}\label{the-vibe}}

Like the cemetery, which was founded in 1838 (about 30 years before
Prospect Park) as a rural escape from the crowded city, the neighborhood
offers plenty of room to breathe.

``Greenwood is my city-girl version of the suburbs,'' said Sephrah
Towbin, 46, an agent with the Corcoran Group, who moved to Greenwood
Heights in 2008 from Manhattan and lives with her husband, Bill Riley,
50, the beverage manager at the restaurant Claro, and their 9-year old
daughter. ``I can walk everywhere, access the subway, but most
important, I have space, trees, light, community and parking.''

She added: ``All of these things, of course, are even more important
now, as people are looking to work from home and have space to live, and
there's that for almost everyone in the market right now.''

Image

St. John's Condominium on 21st Street is housed in a building that was
once home to St. John the Evangelist School.Credit...Karsten Moran for
The New York Times

\hypertarget{the-schools}{%
\subsection{The Schools}\label{the-schools}}

There are four public elementary schools and one charter school in the
area.

\href{https://tools.nycenet.edu/snapshot/2019/15K295/EMS/}{P.S. 295},
the Studio School of Arts and Culture, enrolls almost 400 students in
prekindergarten through fifth grade, with a student body that's 40
percent Hispanic, 37 percent white, 10 percent Asian, and 8 percent
Black). On 2018-19 state tests, 59 percent of students met standards in
English, versus 47.7 percent citywide; 68 percent met standards in math,
versus 45.6 percent citywide.

\href{https://tools.nycenet.edu/snapshot/2019/15K010/EMS/}{P.S. 10}, the
Magnet School of Math, Science and Design Technology, enrolls more than
950 students in kindergarten through fifth grade (50 percent white, 27
percent Hispanic, 9 percent Asian and 9 percent Black). On 2018-19 state
tests, 72 percent of students met standards in English and 71 percent
met standards in math.

\href{https://tools.nycenet.edu/snapshot/2019/15K107/EMS/}{P.S. 107},
John W. Kimball, enrolls more than 550 students in kindergarten through
fifth grade (74 percent white, 11 percent Hispanic, 9 percent Asian and
4 percent Black. On 2018-19 state tests, 82 percent of students met
standards in English and 89 percent met standards in math.

\href{https://tools.nycenet.edu/snapshot/2019/15K172/EMS/}{P.S. 172},
the Beacon School of Excellence, enrolls 538 students in kindergarten
through fifth grade (73 percent Hispanic, 14 percent white, 8 percent
Asian and 4 percent Black). On 2018-19 state tests, 88 percent of
students met standards in English and 97 percent met standards in math.

The \href{https://tools.nycenet.edu/snapshot/2019/84K362/EMS/}{Hellenic
Classical Charter School} enrolls 480 students in kindergarten through
eighth grade (46 percent Hispanic, 31 percent white, 4 percent Asian and
15 percent Black). On 2018-19 state tests, 67 percent of students met
standards in English and 87 percent met standards in math.

Image

Battle Hill, in Green-Wood Cemetery, the highest natural elevation in
the borough, was the site of the Battle of Brooklyn in the Revolutionary
War.Credit...Karsten Moran for The New York Times

\hypertarget{the-commute}{%
\subsection{The Commute}\label{the-commute}}

The N, D and R trains stop at 36th Street and Fourth Avenue. The R also
stops at 25th Street and Prospect Avenue, with transfers to many major
lines a few stops away at Barclays Center. The F line stop at 15th
Street, near Prospect Park, is within walking distance of the
neighborhood's northern reaches.

The B63 bus, running from Bay Ridge to Brooklyn Bridge Park, stops along
Fifth Avenue.

\hypertarget{the-history}{%
\subsection{The History}\label{the-history}}

The Battle of Brooklyn was fought on the highest elevation in what is
now Green-Wood Cemetery. On Aug. 27, 1776, some 50,000 British and
American troops clashed in open battle. Had it not been for a timely
nor'easter that kept British ships from entering the East River, the
Revolution might have ended when General Washington's troops were
outmatched. The site, with its sweeping views, is marked by monuments
and plaques.

For weekly email updates on residential real estate news,
\href{http://www.nytimes3xbfgragh.onion/newsletters/realestate/}{sign up
here}. Follow us on Twitter:
\href{https://twitter.com/nytrealestate}{@nytrealestate}.

Advertisement

\protect\hyperlink{after-bottom}{Continue reading the main story}

\hypertarget{site-index}{%
\subsection{Site Index}\label{site-index}}

\hypertarget{site-information-navigation}{%
\subsection{Site Information
Navigation}\label{site-information-navigation}}

\begin{itemize}
\tightlist
\item
  \href{https://help.nytimes3xbfgragh.onion/hc/en-us/articles/115014792127-Copyright-notice}{©~2020~The
  New York Times Company}
\end{itemize}

\begin{itemize}
\tightlist
\item
  \href{https://www.nytco.com/}{NYTCo}
\item
  \href{https://help.nytimes3xbfgragh.onion/hc/en-us/articles/115015385887-Contact-Us}{Contact
  Us}
\item
  \href{https://www.nytco.com/careers/}{Work with us}
\item
  \href{https://nytmediakit.com/}{Advertise}
\item
  \href{http://www.tbrandstudio.com/}{T Brand Studio}
\item
  \href{https://www.nytimes3xbfgragh.onion/privacy/cookie-policy\#how-do-i-manage-trackers}{Your
  Ad Choices}
\item
  \href{https://www.nytimes3xbfgragh.onion/privacy}{Privacy}
\item
  \href{https://help.nytimes3xbfgragh.onion/hc/en-us/articles/115014893428-Terms-of-service}{Terms
  of Service}
\item
  \href{https://help.nytimes3xbfgragh.onion/hc/en-us/articles/115014893968-Terms-of-sale}{Terms
  of Sale}
\item
  \href{https://spiderbites.nytimes3xbfgragh.onion}{Site Map}
\item
  \href{https://help.nytimes3xbfgragh.onion/hc/en-us}{Help}
\item
  \href{https://www.nytimes3xbfgragh.onion/subscription?campaignId=37WXW}{Subscriptions}
\end{itemize}
