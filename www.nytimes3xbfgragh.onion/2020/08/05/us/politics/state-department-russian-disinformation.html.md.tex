Sections

SEARCH

\protect\hyperlink{site-content}{Skip to
content}\protect\hyperlink{site-index}{Skip to site index}

\href{https://www.nytimes3xbfgragh.onion/section/politics}{Politics}

\href{https://myaccount.nytimes3xbfgragh.onion/auth/login?response_type=cookie\&client_id=vi}{}

\href{https://www.nytimes3xbfgragh.onion/section/todayspaper}{Today's
Paper}

\href{/section/politics}{Politics}\textbar{}State Dept. Traces Russian
Disinformation Links

\url{https://nyti.ms/3gwmDIB}

\begin{itemize}
\item
\item
\item
\item
\item
\end{itemize}

Advertisement

\protect\hyperlink{after-top}{Continue reading the main story}

Supported by

\protect\hyperlink{after-sponsor}{Continue reading the main story}

\hypertarget{state-dept-traces-russian-disinformation-links}{%
\section{State Dept. Traces Russian Disinformation
Links}\label{state-dept-traces-russian-disinformation-links}}

A new government report avoids direct discussion of American election
interference by Moscow, despite lawmakers' call for more information.

\includegraphics{https://static01.graylady3jvrrxbe.onion/images/2020/08/05/us/politics/05dc-disinfo/merlin_173805921_cde41302-7b51-4119-aa14-c775cfd2a0ac-articleLarge.jpg?quality=75\&auto=webp\&disable=upscale}

\href{https://www.nytimes3xbfgragh.onion/by/julian-e-barnes}{\includegraphics{https://static01.graylady3jvrrxbe.onion/images/2019/12/13/reader-center/author-julian-barnes/author-julian-barnes-thumbLarge.png}}

By \href{https://www.nytimes3xbfgragh.onion/by/julian-e-barnes}{Julian
E. Barnes}

\begin{itemize}
\item
  Published Aug. 5, 2020Updated Aug. 18, 2020
\item
  \begin{itemize}
  \item
  \item
  \item
  \item
  \item
  \end{itemize}
\end{itemize}

WASHINGTON ---
\href{https://www.nytimes3xbfgragh.onion/2020/08/18/us/politics/trump-russia-senate.html}{Russia}
continues to use a network of proxy websites to spread pro-Kremlin
disinformation and propaganda in the
\href{https://www.nytimes3xbfgragh.onion/2020/08/18/us/politics/trump-russia-senate.html}{United
States} and other parts of the West, according to a
\href{https://content.govdelivery.com/attachments/USSTATEBPA/2020/08/05/file_attachments/1512230/Pillars\%20of\%20Russias\%20Disinformation\%20and\%20Propaganda\%20Ecosystem_08-04-20\%20\%281\%29.pdf}{State
Department report} released on Wednesday.

The report is one of the most detailed explanations yet from the
\href{https://www.nytimes3xbfgragh.onion/2020/08/18/us/politics/trump-russia-senate.html}{Trump
administration} on how Russia disseminates disinformation, but it
largely avoids discussing how Moscow is trying to influence the current
campaign. Even as Democrats on Capitol Hill have urged the American
government to declassify more information on
\href{https://www.nytimes3xbfgragh.onion/2020/08/18/us/politics/senate-intelligence-russian-interference-report.html}{Russia's
efforts to interfere with the election}, President Trump has repeatedly
told officials such disclosures are unwelcome.

Most of the report focuses on an ecosystem of websites, many of them
fringe or conspiracy minded, that Russia has used or directed to spread
propaganda on a variety of topics. Those include an online journal
called the Strategic Culture Foundation and other sites, like the
Canada-based Global Research. The document
\href{https://www.nytimes3xbfgragh.onion/2020/07/28/us/politics/russia-disinformation-coronavirus.html}{builds
on information disclosed} last week by American officials about Russian
intelligence's control of various propaganda sites.

Secretary of State Mike Pompeo, who announced the release of the report
on Wednesday, said the State Department would offer rewards of up to
\$10 million for information to help identify any person who, acting at
the direction of a foreign government, tries to hack into election or
campaign infrastructure.

The report was prepared by the department's Global Engagement Center,
whose mandate is only to examine propaganda efforts outside the United
States.

The report states that the Strategic Culture Foundation is directed by
Russia's foreign intelligence service, the S.V.R., and stands as ``a
prime example of longstanding Russian tactics to conceal direct state
involvement in disinformation and propaganda outlets.'' The organization
publishes a wide variety of fringe voices and conspiracy theories in
English, while trying to obscure its Russian government sponsorship.

``The Kremlin bears direct responsibility for cultivating these tactics
and platforms as part of its approach of using information and
disinformation as a weapon,'' said Lea Gabrielle, the coordinator of the
State Department's Global Engagement Center.

Absent from the report is any mention of how one of the writers for the
Strategic Culture Foundation weighed in this spring on a Democratic
primary race in New York. The writer, Michael Averko, published articles
on the foundation's website and in a local publication in Westchester
County, N.Y., attacking Evelyn N. Farkas, a former Obama administration
official who was running for Congress.

In recent weeks, the F.B.I. questioned Mr. Averko about the Strategic
Culture Foundation and its ties to Russia.

While those attacks did not have a decisive effect on the election, they
showed Moscow's continuing efforts to influence votes in the United
States, Dr. Farkas said Wednesday in an interview.

She criticized the State Department for failing to explain how the
Strategic Culture Foundation had tried to intervene in the current
election, arguing the report missed an opportunity to ``wake people
up.''

``The State Department should not be releasing information that is so
sanitized that it fails to convey the enormity of the situation,'' Dr.
Farkas said. ``The whole point of writing a report like this is to put
the American people on alert.''

Intelligence officials in recent days have briefed members of Congress
about election threats from Russia and other countries. Senator Richard
Blumenthal, Democrat of Connecticut, and other lawmakers have called on
the administration to declassify and release to the public information
about those threats.

``The fact that they are issuing this report about what the Russians are
doing around the globe but not in the United States shows all the more
how information relative to our own security should be declassified,''
Mr. Blumenthal said in an interview on Wednesday. ``The classified
briefings have been absolutely chilling and frankly terrifying in the
magnitude of foreign threat to our election security that we face. It
really is a break-the-glass moment.''

The State Department report tries to gauge the reach of the pro-Russia
propaganda sites. Global Research is by far the most popular. According
to the report, it has accumulated 12.4 million page views, drawing an
average of 351,247 people per article. Other sites, like News Front and
SouthFront, have nine million and 4.3 million readers each. The
Strategic Culture Foundation has far less substantial internet traffic,
having drawn only about 990,000 visitors.

The Russian Embassy in Washington said the State Department report was
an effort to stop proposals to resume security cooperation with Russia.

``The U.S. State Department is not very fond of the existence of
alternative sources of information,''
\href{https://twitter.com/rusembusa/status/1291184519316475904?s=21}{said
Nikolay Lakhonin, the embassy press secretary}. ``Serious resources are
employed to discredit them. Any voice that contradicts Washington is
dubbed `disinformation' in the service of the `Kremlin' and Russian
intelligence.''

Global Research, according to the report, is a ``home-grown Canadian
website'' that nonetheless has become enmeshed in Moscow's propaganda
ecosystem. The report talks about how the founder of the site, Michel
Chossudovsky, was a former contributor to RT, Moscow's state-sponsored
international broadcaster, and sits on the board of other pro-Russian
conspiracy sites.

Global Research has previously denied that it is part of a pro-Russian
network of websites, but did not respond to a request for comment on
Wednesday.

The State Department report also highlights how the websites have spread
disinformation and conspiracy theories surrounding the pandemic, most
notably the false story that the novel coronavirus was created in an
American military lab.

One false story by Global Research claiming that the coronavirus
pandemic was not real was then spread by 70 other sites and
publications, Ms. Gabrielle said.

``Senior Russian officials and pro-Russian media sought to capitalize on
the fear and confusion surrounding the Covid-19 pandemic by actively
promulgating conspiracy theories,'' the report said.

Advertisement

\protect\hyperlink{after-bottom}{Continue reading the main story}

\hypertarget{site-index}{%
\subsection{Site Index}\label{site-index}}

\hypertarget{site-information-navigation}{%
\subsection{Site Information
Navigation}\label{site-information-navigation}}

\begin{itemize}
\tightlist
\item
  \href{https://help.nytimes3xbfgragh.onion/hc/en-us/articles/115014792127-Copyright-notice}{©~2020~The
  New York Times Company}
\end{itemize}

\begin{itemize}
\tightlist
\item
  \href{https://www.nytco.com/}{NYTCo}
\item
  \href{https://help.nytimes3xbfgragh.onion/hc/en-us/articles/115015385887-Contact-Us}{Contact
  Us}
\item
  \href{https://www.nytco.com/careers/}{Work with us}
\item
  \href{https://nytmediakit.com/}{Advertise}
\item
  \href{http://www.tbrandstudio.com/}{T Brand Studio}
\item
  \href{https://www.nytimes3xbfgragh.onion/privacy/cookie-policy\#how-do-i-manage-trackers}{Your
  Ad Choices}
\item
  \href{https://www.nytimes3xbfgragh.onion/privacy}{Privacy}
\item
  \href{https://help.nytimes3xbfgragh.onion/hc/en-us/articles/115014893428-Terms-of-service}{Terms
  of Service}
\item
  \href{https://help.nytimes3xbfgragh.onion/hc/en-us/articles/115014893968-Terms-of-sale}{Terms
  of Sale}
\item
  \href{https://spiderbites.nytimes3xbfgragh.onion}{Site Map}
\item
  \href{https://help.nytimes3xbfgragh.onion/hc/en-us}{Help}
\item
  \href{https://www.nytimes3xbfgragh.onion/subscription?campaignId=37WXW}{Subscriptions}
\end{itemize}
