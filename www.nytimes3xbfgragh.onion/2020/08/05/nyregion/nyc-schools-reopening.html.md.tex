Sections

SEARCH

\protect\hyperlink{site-content}{Skip to
content}\protect\hyperlink{site-index}{Skip to site index}

\href{https://www.nytimes3xbfgragh.onion/section/nyregion}{New York}

\href{https://myaccount.nytimes3xbfgragh.onion/auth/login?response_type=cookie\&client_id=vi}{}

\href{https://www.nytimes3xbfgragh.onion/section/todayspaper}{Today's
Paper}

\href{/section/nyregion}{New York}\textbar{}Can N.Y.C. Reopen Schools?
The Whole Country Is Watching

\url{https://nyti.ms/3gtu3Mw}

\begin{itemize}
\item
\item
\item
\item
\item
\item
\end{itemize}

\hypertarget{school-reopenings}{%
\subsubsection{\texorpdfstring{\href{https://www.nytimes3xbfgragh.onion/spotlight/schools-reopening?name=styln-coronavirus-schools-reopening\&region=TOP_BANNER\&block=storyline_menu_recirc\&action=click\&pgtype=Article\&impression_id=6568bd20-f2ca-11ea-bc7c-8fd709e972c8\&variant=undefined}{School
Reopenings}}{School Reopenings}}\label{school-reopenings}}

\begin{itemize}
\tightlist
\item
  \href{https://www.nytimes3xbfgragh.onion/2020/09/08/us/school-districts-cyberattacks-glitches.html?name=styln-coronavirus-schools-reopening\&region=TOP_BANNER\&block=storyline_menu_recirc\&action=click\&pgtype=Article\&impression_id=6568e430-f2ca-11ea-bc7c-8fd709e972c8\&variant=undefined}{Remote
  Learning Glitches}
\item
  \href{https://www.nytimes3xbfgragh.onion/2020/09/08/upshot/children-testing-shortfalls-virus.html?name=styln-coronavirus-schools-reopening\&region=TOP_BANNER\&block=storyline_menu_recirc\&action=click\&pgtype=Article\&impression_id=6568e431-f2ca-11ea-bc7c-8fd709e972c8\&variant=undefined}{Limited
  Testing for Children}
\item
  \href{https://www.nytimes3xbfgragh.onion/2020/09/01/world/schools-reopen-globe-students.html?name=styln-coronavirus-schools-reopening\&region=TOP_BANNER\&block=storyline_menu_recirc\&action=click\&pgtype=Article\&impression_id=6568e432-f2ca-11ea-bc7c-8fd709e972c8\&variant=undefined}{School
  Around the World}
\item
  \href{https://www.nytimes3xbfgragh.onion/interactive/2020/us/covid-college-cases-tracker.html?name=styln-coronavirus-schools-reopening\&region=TOP_BANNER\&block=storyline_menu_recirc\&action=click\&pgtype=Article\&impression_id=65690b40-f2ca-11ea-bc7c-8fd709e972c8\&variant=undefined}{Tracking
  College Cases}
\end{itemize}

Advertisement

\protect\hyperlink{after-top}{Continue reading the main story}

Supported by

\protect\hyperlink{after-sponsor}{Continue reading the main story}

\hypertarget{can-nyc-reopen-schools-the-whole-country-is-watching}{%
\section{Can N.Y.C. Reopen Schools? The Whole Country Is
Watching}\label{can-nyc-reopen-schools-the-whole-country-is-watching}}

The city's low infection rate has raised hopes that students can return
to classrooms next month. But many obstacles remain.

\includegraphics{https://static01.graylady3jvrrxbe.onion/images/2020/08/05/nyregion/05nyvirus-schools-1/merlin_171509217_a0b4626e-3f80-4c59-9f1e-5f79f0d92410-articleLarge.jpg?quality=75\&auto=webp\&disable=upscale}

\href{https://www.nytimes3xbfgragh.onion/by/eliza-shapiro}{\includegraphics{https://static01.graylady3jvrrxbe.onion/images/2018/12/28/multimedia/author-eliza-shapiro/author-eliza-shapiro-thumbLarge.png}}

By \href{https://www.nytimes3xbfgragh.onion/by/eliza-shapiro}{Eliza
Shapiro}

\begin{itemize}
\item
  Published Aug. 5, 2020Updated Sept. 1, 2020
\item
  \begin{itemize}
  \item
  \item
  \item
  \item
  \item
  \item
  \end{itemize}
\end{itemize}

With about a month to go before
\href{https://www.nytimes3xbfgragh.onion/2020/09/01/nyregion/schools-open-coronavirus-nyc.html}{New
York City schools are scheduled to reopen}, the city is confronting a
torrent of logistical issues and political problems that could upend
Mayor Bill de Blasio's ambition to make New York one of the few major
districts in the country to bring students back into classrooms this
fall.

There are not yet enough nurses to staff all city school buildings, and
ventilation systems in aging buildings are in urgent need of upgrades.
There may not even be enough teachers available to offer in-person
instruction.

Some teachers are threatening to stage a sickout, and their union has
indicated it might sue over reopening. Gov. Andrew M. Cuomo,
\href{https://www.nytimes3xbfgragh.onion/2020/04/12/nyregion/schools-cuomo-de-blasio-nyc-coronavirus.html}{who
has contradicted the mayor on every major issue related to schools
during the pandemic,} has spent the last several days loudly noting that
Mr. de Blasio's plan is not yet complete.

And the parents of the city's 1.1 million public school students,
exhausted after nearly four excruciating months of remote learning, are
desperate for answers and still unsure if they will send their children
back into classrooms.

Despite all that, the city believes it can safely reopen schools on
Sept. 10 because New York has maintained a low infection rate. If it
succeeds, it will accomplish something almost no other big city district
is even attempting. In recent days, Los Angeles, Miami, Houston and
Washington, D.C., not to mention scores of smaller suburban and rural
districts, have opted to start the school year remote-only.

On Wednesday, Chicago, facing a teachers strike over health fears and an
uptick in infections in the city,
\href{https://www.nytimes3xbfgragh.onion/2020/08/05/world/coronavirus-covid-19.html}{joined
the list}. The district, the nation's third-largest, had originally
planned to open using a hybrid model, with students divided into pods of
15 children each and attending in-person classes two days a week. In
explaining the shift,
\href{https://www.nytimes3xbfgragh.onion/2020/08/05/world/coronavirus-covid-19.html}{Mayor
Lori Lightfoot} said: ``We have to be guided by the science, period.''

And on Thursday, Boston announced that its public schools could decide
for themselves whether to physically reopen on a hybrid model involving
staggered schedules, or opt for all-remote instruction.

Districts across the country are also struggling with the costs of
reopening during the pandemic, with the added uncertainty of not knowing
how much federal aid they can count on or whether it will come with
restrictions. Republicans in Congress want to tie some aid to the
reopening of in-person classes --- a priority of President Trump's ---
but Democrats adamantly oppose the idea.

New York could provide a template for reopening in other districts where
the virus is contained and strict safety measures are in place. But if
it fails, it could send a deeply discouraging message to school
officials elsewhere.

``It's now or never,'' said Emily Oster, an economist at Brown
University who has written extensively about reopening. While the city's
virus incidence rate is among the lowest in the country, it is widely
predicted that those numbers will tick up later this fall, she noted.
``Either you do it for September, or no one is opening until there's a
vaccine,'' she added.

The question of reopening has presented the mayor and governor with one
of the weightiest conundrums of their careers.

The city's former position as a global epicenter of the virus has made
many parents and teachers extremely wary of school reopening. That is
particularly true of Black and Latino New Yorkers who saw their
communities ravaged by the virus.

Mr. de Blasio has laid out a series of safety measures over the last few
days in an attempt to assuage fears and boost the chances that reopening
really happens --- and to try to quiet mounting criticism from the
teachers' union and Mr. Cuomo. The mayor's plan calls for children to
report to school one to three days a week --- with masks and social
distancing required --- and learn online the rest of the time.

The city is also home to vast numbers of vulnerable children. Remote
learning has been a failure for many of the city's children, but has
been
\href{https://www.nytimes3xbfgragh.onion/2020/04/16/nyregion/special-education-coronavirus-nyc.html}{particularly
disastrous for the 200,000 students with disabilities} and
\href{https://www.nytimes3xbfgragh.onion/interactive/2019/11/19/nyregion/student-homelessness-nyc.html}{114,000
who are homeless}.

Even if the city succeeds in opening schools, there is little certainty
that it will be able to keep them open all semester.
One\href{https://www.nytimes3xbfgragh.onion/2020/08/01/us/schools-reopening-indiana-coronavirus.html}{Indiana
school that opened last week reported a positive case on the very first
day of classes}. Health experts predict the same is almost certain to
happen at some point in some of New York's 1,800 schools next month.
Just two cases in different classrooms of the same school could force
its closing for two weeks.

Balancing the risks and rewards of reopening is hugely challenging on
its own. But the mayor and governor's mutual dislike --- and Mr. Cuomo's
determination to undermine the mayor --- have compounded the problem.

The two men have each trumpeted the city's plummeting case numbers as a
point of pride. But the searing criticism they have both faced for
\href{https://www.nytimes3xbfgragh.onion/2020/03/15/nyregion/nyc-schools-closed.html}{waiting
too long to close schools in mid-March}, when the virus was already
spreading rapidly, has put them on high alert over reopening.

Mr. de Blasio said last week he believed New York was up for the
challenge, calling reopening a ``big, tough job, but one this city is
ready for.''

He acknowledged that many parents and teachers are fearful about
returning to classrooms, and said he would not reopen schools --- or
would close them --- if the city's test positivity rate ticks above 3
percent.

The city's average test positivity rate is currently around 1 percent,
though lags in test results have compromised some recent data. Average
test positivity rates in some parts of Florida reached as high as 20
percent last month.

Mr. Cuomo is expected to announce later this week that school districts
across the state can tentatively plan to reopen because of the low virus
caseload. But that does not necessarily mean that New York City schools
will open --- the State Education Department will still need to sign
off, and the mayor himself has said he will not make a final call until
later this summer.

Though it is unlikely that Mr. Cuomo will veto the city's reopening if
the numbers stay low, the rancorous history between the two men on
schools has prompted confusion among parents.

``I'm not looking forward to a fight between Cuomo and de Blasio,'' said
Peter Kruty, the father of two children in city public schools. ``That's
not going to be constructive.''

Even if state education officials sign off on New York City's final
plan, which has not yet been submitted, Mr. de Blasio's administration
still faces obstacles.

\href{https://www.nytimes3xbfgragh.onion/2020/07/29/us/teacher-union-school-reopening-coronavirus.html}{Perhaps
chief among them is growing dissent from the teachers' union}, which
helped craft the city's plan and is an active participant in high-level
discussions about reopening, but has recently backed away as teachers'
fears have mounted.

President Trump's push to reopen alarmed many educators. But it also
handed unions in Democratic cities, including New York, a powerful tool
to whip up support among their Democratic-leaning membership to oppose
opening school doors.

\href{https://www.nytimes3xbfgragh.onion/spotlight/schools-reopening?action=click\&pgtype=Article\&state=default\&region=MAIN_CONTENT_3\&context=storylines_keepup}{}

\hypertarget{school-reopenings-}{%
\subsubsection{School Reopenings ›}\label{school-reopenings-}}

\hypertarget{back-to-school}{%
\paragraph{Back to School}\label{back-to-school}}

Updated Sept. 8, 2020

The latest on how schools are reopening amid the pandemic.

\begin{itemize}
\item
  \begin{itemize}
  \tightlist
  \item
    The first day of school was a rocky one in many places, as districts
    that started classes online dealt with
    \href{https://www.nytimes3xbfgragh.onion/2020/09/08/us/school-districts-cyberattacks-glitches.html?action=click\&pgtype=Article\&state=default\&region=MAIN_CONTENT_3\&context=storylines_keepup}{technical
    glitches, crashing websites and cyberattacks}.
  \item
    It's not easy to get a coronavirus test for a child. As schools
    reopen,
    \href{https://www.nytimes3xbfgragh.onion/2020/09/08/upshot/children-testing-shortfalls-virus.html?action=click\&pgtype=Article\&state=default\&region=MAIN_CONTENT_3\&context=storylines_keepup}{many
    parents still can't find one nearby}, impeding the fight against the
    pandemic.
  \item
    Life in a quarantine dorm: Colleges are trying to
    \href{https://www.nytimes3xbfgragh.onion/2020/09/09/business/colleges-coronavirus-dormitories-quarantine.html?action=click\&pgtype=Article\&state=default\&region=MAIN_CONTENT_3\&context=storylines_keepup}{isolate
    students who have been exposed to the virus}, but they are running
    into a host of problems.
  \item
    Penn State football defines fall in State College, Pa.
    \href{https://www.nytimes3xbfgragh.onion/2020/09/09/sports/penn-state-college-football-canceled.html?action=click\&pgtype=Article\&state=default\&region=MAIN_CONTENT_3\&context=storylines_keepup}{What
    is the town without it}?
  \end{itemize}
\end{itemize}

Though one national teachers' union has authorized health and safety
strikes, it is illegal for teachers to strike in New York City. But in a
call with members last month, Michael Mulgrew, president of the local
United Federation of Teachers, said, ``I am preparing to do whatever we
need to do if we think the schools are not safe and the city disagrees
with us.''

\includegraphics{https://static01.graylady3jvrrxbe.onion/images/2020/08/05/nyregion/05nyvirus-schools-3/merlin_175291017_86ee1fd7-fc59-48d8-a177-54f969aad13b-articleLarge.jpg?quality=75\&auto=webp\&disable=upscale}

On Monday,
\href{https://twitter.com/JoshuaPotash/status/1290476937899466752}{city
teachers marched in Lower Manhattan to protest reopening plans}, using
the hashtag \#WeWontDieforDOE, in reference to the Department of
Education.

Even some educators who say they are willing to go back to classrooms
said they were concerned that the highly charged climate in New York
over reopening has damaged bonds between teachers and parents.

``We went from being honored as the most amazing people in the world to
now we are lazy people who don't want to work,'' said Melissa Dorcemus,
a high school teacher in Manhattan. ``I'm like, which are we?''

City Hall officials said they were planning to meet the union's safety
demands, though some crucial details are still scarce.

Still, Mr. Mulgrew, a political ally of Mr. Cuomo, recently said that
even if all the safety boxes are checked, he may continue to oppose
reopening, because of a lack of trust between the union and the mayor.

Partly to appease the union, Mr. de Blasio said quick-turnaround tests
will be made available for all staff before school starts, though he has
not announced details about whether students and staff will be tested
after the school year begins.

One or two cases in a single classroom will prompt the members of that
class to learn remotely for two weeks. But if two or more people in
different classrooms test positive, the entire building will close while
disease detectives investigate links between the cases.

Many staffing questions remain. The teachers' union has said it would
not be comfortable returning to schools without a nurse in every
building, a goal that has still not been reached.

The district also does not know if it will have enough teachers for
students in classrooms, with an estimated 20 percent of teachers
eligible to work from home for medical reasons.

Uncertainties about federal stimulus money is especially worrisome for
the many city school buildings that are over a century old and have
windows that barely open, raising questions about whether there will be
enough air circulation to mitigate the risk of an airborne virus.

Joseph Allen, a professor at Harvard's T.H. Chan School of Public
Health, has advised districts with low levels of transmission to update
ventilation systems and purchase portable air filters that can circulate
air several times an hour.

``For anyone who says we can't get this done in the next 30 days, that's
just wrong,'' he said.

As the city rushes to retrofit buildings, parents across the city said
they felt deeply conflicted about whether to return.

Acola McKnight, a single mother who lives in Harlem, is worried that her
son won't receive crucial services for his attention deficit disorder if
he is not in school.

But she also can't picture dropping him off at the door and not being
racked with fear about whether he will keep his mask on, or whether
someone might test positive that day.

``There's just so much uncertainty,'' she said. ``I have so many
doubts.''

Advertisement

\protect\hyperlink{after-bottom}{Continue reading the main story}

\hypertarget{site-index}{%
\subsection{Site Index}\label{site-index}}

\hypertarget{site-information-navigation}{%
\subsection{Site Information
Navigation}\label{site-information-navigation}}

\begin{itemize}
\tightlist
\item
  \href{https://help.nytimes3xbfgragh.onion/hc/en-us/articles/115014792127-Copyright-notice}{©~2020~The
  New York Times Company}
\end{itemize}

\begin{itemize}
\tightlist
\item
  \href{https://www.nytco.com/}{NYTCo}
\item
  \href{https://help.nytimes3xbfgragh.onion/hc/en-us/articles/115015385887-Contact-Us}{Contact
  Us}
\item
  \href{https://www.nytco.com/careers/}{Work with us}
\item
  \href{https://nytmediakit.com/}{Advertise}
\item
  \href{http://www.tbrandstudio.com/}{T Brand Studio}
\item
  \href{https://www.nytimes3xbfgragh.onion/privacy/cookie-policy\#how-do-i-manage-trackers}{Your
  Ad Choices}
\item
  \href{https://www.nytimes3xbfgragh.onion/privacy}{Privacy}
\item
  \href{https://help.nytimes3xbfgragh.onion/hc/en-us/articles/115014893428-Terms-of-service}{Terms
  of Service}
\item
  \href{https://help.nytimes3xbfgragh.onion/hc/en-us/articles/115014893968-Terms-of-sale}{Terms
  of Sale}
\item
  \href{https://spiderbites.nytimes3xbfgragh.onion}{Site Map}
\item
  \href{https://help.nytimes3xbfgragh.onion/hc/en-us}{Help}
\item
  \href{https://www.nytimes3xbfgragh.onion/subscription?campaignId=37WXW}{Subscriptions}
\end{itemize}
