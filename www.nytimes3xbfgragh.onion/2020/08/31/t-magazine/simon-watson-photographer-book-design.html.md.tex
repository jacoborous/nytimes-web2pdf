A New Book from a Renowned Photographer of Interiors

\url{https://nyti.ms/31HOalb}

\begin{itemize}
\item
\item
\item
\item
\item
\end{itemize}

\includegraphics{https://static01.graylady3jvrrxbe.onion/images/2020/08/27/t-magazine/design/simon-watson-slide-PLB4/simon-watson-slide-PLB4-articleLarge.jpg?quality=75\&auto=webp\&disable=upscale}

Sections

\protect\hyperlink{site-content}{Skip to
content}\protect\hyperlink{site-index}{Skip to site index}

\hypertarget{a-new-book-from-a-renowned-photographer-of-interiors}{%
\section{A New Book from a Renowned Photographer of
Interiors}\label{a-new-book-from-a-renowned-photographer-of-interiors}}

Throughout his career, Simon Watson has brought his unique eye to
crumbling castles, monastic apartments and everything in between.

Roberto Peregalli's riad in Tangier, Morocco --- one of the few homes
Watson claims to ``absolutely covet,'' 2006.Credit...© Simon Watson

Supported by

\protect\hyperlink{after-sponsor}{Continue reading the main story}

By Jennifer Conrad

\begin{itemize}
\item
  Published Aug. 31, 2020Updated Sept. 3, 2020
\item
  \begin{itemize}
  \item
  \item
  \item
  \item
  \item
  \end{itemize}
\end{itemize}

In the 1530s, Cardinal Marcello Cervini, the future Pope Marcellus II,
bought Castello Cervini, a medieval monastery surrounded by chestnut
trees in Tuscany's Vivo d'Orcia estate. And while Marcellus II passed
away just 22 days after ascending to the papacy in 1555, his family has
retained the property for four centuries and counting. By the time the
photographer
\href{https://www.nytimes3xbfgragh.onion/2019/12/13/t-magazine/decade-in-photos-simon-watson.html}{Simon
Watson} visited, in 2002, Marcellus's descendants Daria Cervini and her
siblings had turned the castle into a summer retreat, but many tokens of
the family's past remained, and a portrait of the ill-fated pope in a
crimson mozzetta kept watch over all.

Watson set about capturing what he calls the residence's ``poetic
beauty,'' and the results --- an image of a wallpapered bedroom with a
pair of twin beds awash in a blueish light, another of the arm of a pink
sofa set next to a small vase of matching freshly picked flowers ****
--- are, along with his photos of 19 other homes, included in his
forthcoming book,
``\href{https://www.rizzoliusa.com/book/9780847869008/}{The Lives of
Others}.'' Paging through, one is immediately aware of why the
photographer is sought out by many magazines, including T. ``Aesthetic
beauty is what stirs me and what always will,'' says Watson, adding, ``I
suppose
\href{https://www.nytimes3xbfgragh.onion/2019/08/27/t-magazine/simon-watson-home-dublin.html}{because
I grew up in Dublin}, I had a natural affinity for the Georgian
sensibility. I spent time living in old houses with crumbling
plasterwork and broken floorboards.''

\includegraphics{https://static01.graylady3jvrrxbe.onion/images/2020/08/27/t-magazine/design/simon-watson-slide-8S85/simon-watson-slide-8S85-articleLarge.jpg?quality=75\&auto=webp\&disable=upscale}

Indeed, his pictures are often of homes that wear their history on their
sleeves. But the images' ease and elegance stem from much more than the
physical elements within the frame, however fine. Watson is a master of
light and color, and his work has earned comparisons to that of the
Dutch masters. He favors natural light --- as
\href{https://www.nytimes3xbfgragh.onion/by/tom-delavan}{Tom Delavan},
who is T's design/interiors director, writes in the book's foreword,
``shutters are his tool of choice, canted at seemingly random angles to
transform spaces into something dramatically beautiful'' --- and
painterly hues, from saturated aquas to misty greens to deep reds.

Watson also has a keen sense of people, or rather how people might exist
in a space, retreating to one favorite corner to work and another to
relax. His Castello Cervini pictures don't emphasize the property's
grandness so much as they convey a sense of intimacy and authenticity
--- there's some peeling floral wallpaper that reveals the older painted
trim beneath; there's a skylit table piled with books; there's a small,
beloved teddy bear. It's almost as though, as with Horst P. Horst's
20th-century pictures of grand homes, Watson is merely peeking from one
room into another, never wishing to intrude or disrupt the natural state
of things. If you look closely, even in a picture of the home's imposing
facade, taken with the camera tilted upward, a figure peers down from an
upstairs window, a seeming reminder, as the book title implies, of the
lives lived inside.

Image

The lounge in Christian Louboutin's Paris apartment, with a sofa and
screens from Egypt, 2016.Credit...© Simon Watson

Image

A bedroom in Clandeboye, the home of Lindy Guinness, the Marchioness of
Dufferin and Ava, in County Down, Northern Ireland, 2008.Credit...©
Simon Watson

Image

In the dining room in the Gaeta, Italy, home of
\href{https://www.nytimes3xbfgragh.onion/2015/03/26/t-magazine/nicola-del-roscio-cy-twombly-gaeta-cultivating-genius.html}{Nicola
Del Roscio}, president of the Cy Twombly Foundation, a copy of a Picasso
by Twombly hangs over a Gothic chair, 2015.Credit...© Simon Watson

Watson's own life has been just as rich. His parents immersed him and
his siblings in fine art and classical architecture from a young age.
The photographer, now 51, went on to study film at a small college in
his native Dublin, but soon dropped out and spent time working in cities
around Europe. ``Slowly, different events in my life gradually brought
me to photography. My father bought me a camera when I was 15 or 16, and
I'd wander around taking black-and-white pictures of anything and
everything,'' he says. ``I also just liked the idea of being a
photographer, probably for all the wrong reasons --- working as an
assistant and doing the odd magazine shoot here and there, maybe too
many viewings of Antonioni's
`\href{https://www.nytimes3xbfgragh.onion/2017/07/19/movies/blow-up-michaelangelo-antonioni-film-forum.html}{Blow-Up}'
{[}the 1966 film in which David Hemmings plays a fashion
photographer{]}. When one is young, one is impressionable.'' In 1989,
Watson moved to New York and launched what is now a three-decade career
that has spanned commercial and editorial work as well as personal
projects, such as his 2006 exhibition at Poland's Auschwitz-Birkenau
State Museum,
``\href{http://www.simonwatson.com/projects/a-lingering-presence/}{A
Lingering Presence},'' which documented spaces at the concentration camp
that had never been open to the public.

Image

The Georgian house at
\href{https://www.nytimes3xbfgragh.onion/2018/09/10/t-magazine/henrietta-street-dublin-building-history-design.html}{Twelve
Henrietta Street}, a 19th-century Dublin tenement that was purchased and
restored by Ian Lumley in the 1980s, 2018.Credit...© Simon Watson

With ``The Lives of Others,'' out from Rizzoli next week, Watson
presents an edit of the interiors he has been commissioned to photograph
over the years, and includes many previously unpublished images. Each
section opens with a vignette in which he riffs on the featured
characters and spaces. Castel Gardena, a 17th-century hunting lodge in
the Dolomites owned by the winemaker Andrea Franchetti, was, Watson
writes, in a ``glorious state of romantic disrepair,'' with various
animals chewing noisily in the walls and scurrying through his room at
night.

Image

The Milan apartment of the fashion designer Stephan Janson and the
writer and horticulturalist Umberto Pasti teems with treasures from the
couple's travels, 2014.Credit...© Simon Watson

Image

A private home outside Naples, Italy, abuts Solfatara, an ancient
volcanic crater, mid-2000s.Credit...© Simon Watson

Image

A sitting area in Le Petit Palais, one of four houses that
\href{https://www.nytimes3xbfgragh.onion/2013/05/09/t-magazine/taroudant-lost-in-time.html}{Christopher
Decarpentrie and Abel Naessens} turned into a compound in Taroudant,
Morocco, 2013.Credit...© Simon Watson

But mostly the photos speak for themselves. Those of Casa Horta, the
architect and furniture designer
\href{https://www.nytimes3xbfgragh.onion/2018/08/27/t-magazine/architect-guillermo-santoma-casa-horta-barcelona.html}{Guillermo
Santomà}'s surrealist Barcelona home, show a bubblegum pink dining room
with plastic lawn chairs that Santomà **** partially melted with a
blowtorch, as well as a still-life of fruit and architectural elements
painted directly on the wall by the artist
\href{https://www.instagram.com/marriapratts/}{Marria Pratts}. ``When I
stepped in, I immediately thought, this guy is great; this guy has got
such a vivid imagination,'' Watson says. Another architect with a
distinctive vision,
\href{https://www.nytimes3xbfgragh.onion/2016/03/10/t-magazine/arno-brandlhuber-brutalist-architecture-home.html}{Arno
Brandlhuber}, fashioned Antivilla, his lake house near Potsdam, out of
an old East German underwear factory, using sledgehammers to smash
asymmetrical windows into the concrete walls. ``It's abrupt and
Brutalist, not dissimilar to Brandlhuber himself,'' says Watson.
Clearly, it's the personal details that linger with him: ``The place
where someone lives is a portrait of who they are,'' Watson says. ``You
can tell their humor and their character.''

Advertisement

\protect\hyperlink{after-bottom}{Continue reading the main story}

\hypertarget{site-index}{%
\subsection{Site Index}\label{site-index}}

\hypertarget{site-information-navigation}{%
\subsection{Site Information
Navigation}\label{site-information-navigation}}

\begin{itemize}
\tightlist
\item
  \href{https://help.nytimes3xbfgragh.onion/hc/en-us/articles/115014792127-Copyright-notice}{©~2020~The
  New York Times Company}
\end{itemize}

\begin{itemize}
\tightlist
\item
  \href{https://www.nytco.com/}{NYTCo}
\item
  \href{https://help.nytimes3xbfgragh.onion/hc/en-us/articles/115015385887-Contact-Us}{Contact
  Us}
\item
  \href{https://www.nytco.com/careers/}{Work with us}
\item
  \href{https://nytmediakit.com/}{Advertise}
\item
  \href{http://www.tbrandstudio.com/}{T Brand Studio}
\item
  \href{https://www.nytimes3xbfgragh.onion/privacy/cookie-policy\#how-do-i-manage-trackers}{Your
  Ad Choices}
\item
  \href{https://www.nytimes3xbfgragh.onion/privacy}{Privacy}
\item
  \href{https://help.nytimes3xbfgragh.onion/hc/en-us/articles/115014893428-Terms-of-service}{Terms
  of Service}
\item
  \href{https://help.nytimes3xbfgragh.onion/hc/en-us/articles/115014893968-Terms-of-sale}{Terms
  of Sale}
\item
  \href{https://spiderbites.nytimes3xbfgragh.onion}{Site Map}
\item
  \href{https://help.nytimes3xbfgragh.onion/hc/en-us}{Help}
\item
  \href{https://www.nytimes3xbfgragh.onion/subscription?campaignId=37WXW}{Subscriptions}
\end{itemize}
