Sections

SEARCH

\protect\hyperlink{site-content}{Skip to
content}\protect\hyperlink{site-index}{Skip to site index}

\href{https://www.nytimes3xbfgragh.onion/section/food}{Food}

\href{https://myaccount.nytimes3xbfgragh.onion/auth/login?response_type=cookie\&client_id=vi}{}

\href{https://www.nytimes3xbfgragh.onion/section/todayspaper}{Today's
Paper}

\href{/section/food}{Food}\textbar{}5 Standout Recipes From Julia Reed,
an Irreverent Voice of the South

\url{https://nyti.ms/31JkzHX}

\begin{itemize}
\item
\item
\item
\item
\item
\item
\end{itemize}

\href{https://www.nytimes3xbfgragh.onion/spotlight/at-home?action=click\&pgtype=Article\&state=default\&region=TOP_BANNER\&context=at_home_menu}{At
Home}

\begin{itemize}
\tightlist
\item
  \href{https://www.nytimes3xbfgragh.onion/2020/09/07/travel/route-66.html?action=click\&pgtype=Article\&state=default\&region=TOP_BANNER\&context=at_home_menu}{Cruise
  Along: Route 66}
\item
  \href{https://www.nytimes3xbfgragh.onion/2020/09/04/dining/sheet-pan-chicken.html?action=click\&pgtype=Article\&state=default\&region=TOP_BANNER\&context=at_home_menu}{Roast:
  Chicken With Plums}
\item
  \href{https://www.nytimes3xbfgragh.onion/2020/09/04/arts/television/dark-shadows-stream.html?action=click\&pgtype=Article\&state=default\&region=TOP_BANNER\&context=at_home_menu}{Watch:
  Dark Shadows}
\item
  \href{https://www.nytimes3xbfgragh.onion/interactive/2020/at-home/even-more-reporters-editors-diaries-lists-recommendations.html?action=click\&pgtype=Article\&state=default\&region=TOP_BANNER\&context=at_home_menu}{Explore:
  Reporters' Google Docs}
\end{itemize}

Advertisement

\protect\hyperlink{after-top}{Continue reading the main story}

Supported by

\protect\hyperlink{after-sponsor}{Continue reading the main story}

\hypertarget{5-standout-recipes-from-julia-reed-an-irreverent-voice-of-the-south}{%
\section{5 Standout Recipes From Julia Reed, an Irreverent Voice of the
South}\label{5-standout-recipes-from-julia-reed-an-irreverent-voice-of-the-south}}

The journalist, who died last week at 59, mixed sophistication and
down-home pleasures in her cooking.

\includegraphics{https://static01.graylady3jvrrxbe.onion/images/2020/09/02/dining/31reed-cook/merlin_176408811_982c238e-2224-4771-8a69-23f74f8a64b3-articleLarge.jpg?quality=75\&auto=webp\&disable=upscale}

\href{https://www.nytimes3xbfgragh.onion/by/kim-severson}{\includegraphics{https://static01.graylady3jvrrxbe.onion/images/2018/06/13/multimedia/author-kim-severson/author-kim-severson-thumbLarge.jpg}}

By \href{https://www.nytimes3xbfgragh.onion/by/kim-severson}{Kim
Severson}

\begin{itemize}
\item
  Aug. 31, 2020
\item
  \begin{itemize}
  \item
  \item
  \item
  \item
  \item
  \item
  \end{itemize}
\end{itemize}

Julia Reed, the Southern journalist who
\href{https://www.nytimes3xbfgragh.onion/aponline/2020/08/29/us/ap-us-obit-julia-reed.html}{died
of cancer} on Friday at 59, cooked very much the way she lived. Which is
to say, she was \emph{really} into it. Both her appetite and her writing
reflected a bawdy sophistication wrapped in a very pretty Southern
shawl.

Ms. Reed was a daughter of the South and a woman of the world who had
made her name as a writer in Washington D.C., New York City and New
Orleans, eating both high and low in equal measure. She could recommend
a reliable spot for both albóndigas in Spain and hot tamales in
Greenville, Miss., the Delta town where she was born and where she built
a house on some family land a couple of years ago.

Ms. Reed loved it all --- a French 75, a thick Roman steak, chilled crab
meat Maison or a pile of Gulf shrimp ``boiled for a nano-second and
eaten, still steaming, from the colander in the sink.''

That bit of writing came from her column in the August edition of
\href{https://gardenandgun.com/}{Garden \& Gun}, the lifestyle magazine
out of Charleston, S.C., that for more than a decade relied on Ms.
Reed's talents as a contributing editor and writer. In the best
tradition of Southern storytelling, her columns walked the reader along
a long and winding path that turned out to be the perfect way to get to
her destination.

That August
\href{https://gardenandgun.com/articles/cooking-through-covid/}{column},
her next-to-last, is hard to read. She chronicled the aggressive online
ordering and ambitious recipes that she, like so many of us, embraced
during the early days of the pandemic. She took a side trip into her
experiences reporting on white supremacists, likening them to the biting
buffalo gnats that invaded Greenville last spring. (``These guys were
like the damn gnats: You don't always see them coming and you don't know
the harm they've done until you are practically bleeding to death.'')

She ended by reflecting on how the small act of cooking can help with
the great reckonings facing America, and some suggestions for what
should be on the table at a funeral lunch. We are in the midst of a
national wake, she wrote, grieving for lives lost and dreams deferred.

But Ms. Reed put cooking at the center of that column. When she wrote
it, she knew that the end of her life was probably not far away. Perhaps
she left it for us as a road map.

Over the course of her life, Ms. Reed contributed more than 100 recipes
to The New York Times. Here are some of our favorites.

\textbf{\href{https://cooking.nytimes3xbfgragh.onion/recipes/8665-hot-cheese-olives}{Hot
Cheese Olives}}

Ms. Reed loved to
\href{https://gardenandgun.com/articles/secrets-of-a-southern-hostess/}{entertain}
(she wrote
\href{https://www.goodreads.com/book/show/26067552-julia-reed-s-south}{books}
about it!) and hot cheese olives --- the classic, easy bites from the
1950s --- were regular guests at her parties.

\textbf{\href{https://cooking.nytimes3xbfgragh.onion/recipes/11404-summer-squash-casserole}{Summer
Squash Casserole}}

Every good Southern cook has a summer squash casserole recipe. Ms.
Reed's is a perfect mix of homey ingredients like crushed Ritz crackers
for lightness and a mix of chopped peppers for character.

\textbf{\href{https://cooking.nytimes3xbfgragh.onion/recipes/6970-roman-steaks}{Roman
Steaks}}

Ms. Reed was a traveler, and often told tales about great dishes she
enjoyed abroad. One was the rosemary-scented Roman steak she had at Nino
in Rome. She adapted
\href{https://www.nytimes3xbfgragh.onion/2017/03/21/dining/paula-wolfert-alzheimers.html}{Paula
Wolfert's}version for us in 2004.

\textbf{\href{https://cooking.nytimes3xbfgragh.onion/recipes/1801-milk-punch}{Milk
Punch}}

Ms. Reed famously hated eggnog, but she loved a good milk punch, which
is eggnog's lighter, frothier cousin. Serve this at lunch, she wrote,
and ``by evening, everyone will want a Santa hat.''

\includegraphics{https://static01.graylady3jvrrxbe.onion/images/2020/09/02/dining/31reed-cook2-Pralines/31reed-cook2-Pralines-articleLarge.jpg?quality=75\&auto=webp\&disable=upscale}

\textbf{\href{https://cooking.nytimes3xbfgragh.onion/recipes/1746-pralines}{Pralines}}

Ms. Reed's second hometown was New Orleans, where African-American cooks
\href{https://www.eater.com/2016/10/27/13422426/praline-new-orleans-pecan-candy}{adapted}the
French praline to American ingredients. In a rare fit of economy, Ms.
Reed made them for holiday gifts one year, using a recipe from her
friend Mary Cooper, who makes the best pralines she ever tasted.

\emph{Follow} \href{https://twitter.com/nytfood}{\emph{NYT Food on
Twitter}} \emph{and}
\href{https://www.instagram.com/nytcooking/}{\emph{NYT Cooking on
Instagram}}\emph{,}
\href{https://www.facebookcorewwwi.onion/nytcooking/}{\emph{Facebook}}\emph{,}
\href{https://www.youtube.com/nytcooking}{\emph{YouTube}} \emph{and}
\href{https://www.pinterest.com/nytcooking/}{\emph{Pinterest}}\emph{.}
\href{https://www.nytimes3xbfgragh.onion/newsletters/cooking}{\emph{Get
regular updates from NYT Cooking, with recipe suggestions, cooking tips
and shopping advice}}\emph{.}

Advertisement

\protect\hyperlink{after-bottom}{Continue reading the main story}

\hypertarget{site-index}{%
\subsection{Site Index}\label{site-index}}

\hypertarget{site-information-navigation}{%
\subsection{Site Information
Navigation}\label{site-information-navigation}}

\begin{itemize}
\tightlist
\item
  \href{https://help.nytimes3xbfgragh.onion/hc/en-us/articles/115014792127-Copyright-notice}{©~2020~The
  New York Times Company}
\end{itemize}

\begin{itemize}
\tightlist
\item
  \href{https://www.nytco.com/}{NYTCo}
\item
  \href{https://help.nytimes3xbfgragh.onion/hc/en-us/articles/115015385887-Contact-Us}{Contact
  Us}
\item
  \href{https://www.nytco.com/careers/}{Work with us}
\item
  \href{https://nytmediakit.com/}{Advertise}
\item
  \href{http://www.tbrandstudio.com/}{T Brand Studio}
\item
  \href{https://www.nytimes3xbfgragh.onion/privacy/cookie-policy\#how-do-i-manage-trackers}{Your
  Ad Choices}
\item
  \href{https://www.nytimes3xbfgragh.onion/privacy}{Privacy}
\item
  \href{https://help.nytimes3xbfgragh.onion/hc/en-us/articles/115014893428-Terms-of-service}{Terms
  of Service}
\item
  \href{https://help.nytimes3xbfgragh.onion/hc/en-us/articles/115014893968-Terms-of-sale}{Terms
  of Sale}
\item
  \href{https://spiderbites.nytimes3xbfgragh.onion}{Site Map}
\item
  \href{https://help.nytimes3xbfgragh.onion/hc/en-us}{Help}
\item
  \href{https://www.nytimes3xbfgragh.onion/subscription?campaignId=37WXW}{Subscriptions}
\end{itemize}
