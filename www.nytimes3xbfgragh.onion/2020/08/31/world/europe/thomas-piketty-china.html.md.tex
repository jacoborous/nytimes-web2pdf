Sections

SEARCH

\protect\hyperlink{site-content}{Skip to
content}\protect\hyperlink{site-index}{Skip to site index}

\href{https://www.nytimes3xbfgragh.onion/section/world/europe}{Europe}

\href{https://myaccount.nytimes3xbfgragh.onion/auth/login?response_type=cookie\&client_id=vi}{}

\href{https://www.nytimes3xbfgragh.onion/section/todayspaper}{Today's
Paper}

\href{/section/world/europe}{Europe}\textbar{}Rejecting Censorship of
His Book, a French Star Economist Stands Up to China

\url{https://nyti.ms/3bdmCHp}

\begin{itemize}
\item
\item
\item
\item
\item
\end{itemize}

Advertisement

\protect\hyperlink{after-top}{Continue reading the main story}

Supported by

\protect\hyperlink{after-sponsor}{Continue reading the main story}

\hypertarget{rejecting-censorship-of-his-book-a-french-star-economist-stands-up-to-china}{%
\section{Rejecting Censorship of His Book, a French Star Economist
Stands Up to
China}\label{rejecting-censorship-of-his-book-a-french-star-economist-stands-up-to-china}}

Thomas Piketty, the renowned economist, refused Chinese demands that he
cut parts of his book if he wanted it published on the mainland.

\includegraphics{https://static01.graylady3jvrrxbe.onion/images/2020/08/31/world/31piketty-china/merlin_176411571_e9f7257a-1537-4b4c-a06b-1d1ed0c61c02-articleLarge.jpg?quality=75\&auto=webp\&disable=upscale}

By \href{https://www.nytimes3xbfgragh.onion/by/constant-meheut}{Constant
Méheut}

\begin{itemize}
\item
  Aug. 31, 2020
\item
  \begin{itemize}
  \item
  \item
  \item
  \item
  \item
  \end{itemize}
\end{itemize}

\href{https://cn.nytimes3xbfgragh.onion/world/20200902/thomas-piketty-china/}{阅读简体中文版}\href{https://cn.nytimes3xbfgragh.onion/world/20200902/thomas-piketty-china/zh-hant/}{閱讀繁體中文版}

PARIS --- With his in-depth critique of Western capitalism, detailed in
a 700-page book that enjoyed record sales in 2014, France's rock-star
economist Thomas Piketty was well regarded by Chinese leaders.

That was until he turned his attention to China.

Mr. Piketty said Monday that his follow-up book, ``Capital and
Ideology,'' which broadens his study of the rise of economic inequality
to non-Western countries such as China and India, is unlikely to be
published in mainland China because he refused requests from Chinese
publishers to cut parts of it.

``For the time being, there will be no book in China,'' said Mr.
Piketty, one of the most high-profile academics to stand up to China,
calling the requests ``ridiculous'' and equating them with censorship.

``They shouldn't be afraid of a book like that, it's a sign of
weakness,'' Mr. Piketty said in a phone interview.

Publishing foreign books in China
\href{https://www.nytimes3xbfgragh.onion/2019/12/27/business/us-china-books-trade-war.html}{has
long been a contentious process}, with Chinese publishers often cutting
or changing sexual or political content to gain government approval. In
recent years, the environment has grown even more challenging, with the
Chinese Communist Party's publicity department unveiling new rules
favoring domestic authors and titles that promote the country's
political and economic model.

Fearful of being barred from China's vast market, some
\href{https://www.nytimes3xbfgragh.onion/2013/10/20/world/asia/authors-accept-censors-rules-to-sell-in-china.html}{Western
authors} and
\href{https://www.nytimes3xbfgragh.onion/2017/11/01/world/asia/china-springer-nature-censorship.html}{academic
publishers} have bowed to Chinese censorship.

Mr. Piketty, who
\href{https://www.nytimes3xbfgragh.onion/2014/04/27/fashion/Thomas-Piketty-the-Economist-Behind-Capital-in-the-Twenty-First-Century-sensation.html}{attained
worldwide celebrity} in 2014 with his book
``\href{https://www.nytimes3xbfgragh.onion/2015/08/03/books/review-the-economics-of-inequality-by-thomas-piketty.html}{Capital
in the Twenty-First Century},'' appears unfazed. ``Asking me to cut all
this and publishing the rest would make no sense.''

Image

Mr. Piketty's newest book expands his study of economic inequality to
include non-Western countries like China and India.Credit...Michelle
Gaps/Michelle Gaps, via Associated Press

He added: ``To agree to this would amount to be compromised with the
regime and to accept to be instrumentalized in their propaganda
enterprise.''

Mr. Piketty's new book,
``\href{https://www.nytimes3xbfgragh.onion/2020/03/08/books/review/capital-and-ideology-thomas-piketty.html}{Capital
and Ideology},'' which was published in France in 2019 and in the United
States last March, is an attempt to describe what he calls ``inequality
regimes'' across the ages and around the world.

Unlike ``Capital in the Twenty-First Century,'' which was published in
2013 and focused on Europe and the United States, the new book widens
the scope and gives an important place to China and its
\href{https://www.nytimes3xbfgragh.onion/2018/02/26/world/asia/xi-jinping-thought-explained-a-new-ideology-for-a-new-era.html}{capitalism-infused
version of socialism}.

``There is a constructive criticism in this book, and, frankly, it does
not blame the Chinese model more than other models in the United States,
Europe, India, Brazil,'' Mr. Piketty said.

But starting in June, Mr. Piketty said, Citic Press --- one of the
largest Chinese publishing houses, which handled the Chinese version of
``Capital in the Twenty-First Century'' --- sent his French publisher,
Les Editions du Seuil, two 10-page lists of requested cuts from the
French and English editions of his book. Other Chinese publishers
interested in the book sent similar requests, Mr. Piketty said.

Citic Press and Les Editions du Seuil did not immediately respond to
requests for comment.

The requested cuts include parts that point out the ``extremely rapid
rise of inequality'' in China, to levels comparable to those seen in the
United States. Others highlight issues like China's lack of an
inheritance tax, which Mr. Piketty says results in a significant
concentration of wealth.

``It is truly paradoxical that a country led by a Communist Party, which
proclaims its adherence to `socialism with Chinese characteristics,'
could make such a choice,'' Mr. Piketty wrote in a paragraph that he
said Citic Press asked to be cut.

The Chinese government has long sought to defend its economic model as
best suited to a country of 1.4 billion inhabitants. Writing
\href{https://www.nytimes3xbfgragh.onion/interactive/2018/11/18/world/asia/china-rules.html}{its
own playbook}, China gradually asserted itself as an economic superpower
capable of challenging the United States.

Chinese leaders cited Mr. Piketty's 2013 book on rising inequality in
the United States and Europe as proof of the superiority of their
economic model.

Several million copies of Mr. Piketty's book ``Capital in the
Twenty-First Century'' have been sold worldwide, including tens of
thousands in China.

Among the requested cuts were sections critical of the Chinese
government, which Mr. Piketty wrote, ``has yet to demonstrate its
superiority over Western electoral democracy.''

The appearance of Mr. Piketty's book comes as China has been confronted
with an
\href{https://www.nytimes3xbfgragh.onion/2019/10/17/business/china-economic-growth.html}{unprecedented
economic slowdown}. A
\href{https://www.nytimes3xbfgragh.onion/2019/08/06/world/asia/china-xi-jingping-trade.html}{trade
war} with the United States and the effects of the
\href{https://www.nytimes3xbfgragh.onion/2020/04/16/business/china-coronavirus-economy.html}{coronavirus
crisis} have brought China's nearly half-century-long run of growth to
an end.

Mr. Piketty said that censoring his book ``seems to illustrate the
growing nervousness of the Chinese regime and their refusal of an open
debate on the different economic and political systems.''

The book, he said, will be published in
\href{https://www.nytimes3xbfgragh.onion/2020/08/30/world/asia/taiwan-china-military.html}{Taiwan}
and, he hopes,
\href{https://www.nytimes3xbfgragh.onion/2020/07/31/world/asia/hong-kong-election-national-security-law.html}{Hong
Kong}, which has come under increasing pressure from the Chinese
government in recent months with the introduction of a wide-ranging
national security law following big government protests.

``If they're afraid of a book like this, what are they going to do with
the demonstrators in Hong Kong or one day in Beijing or Shanghai, as it
will eventually happen?'' Mr. Piketty asked.

Gillian Wong contributed research from Hong Kong

Advertisement

\protect\hyperlink{after-bottom}{Continue reading the main story}

\hypertarget{site-index}{%
\subsection{Site Index}\label{site-index}}

\hypertarget{site-information-navigation}{%
\subsection{Site Information
Navigation}\label{site-information-navigation}}

\begin{itemize}
\tightlist
\item
  \href{https://help.nytimes3xbfgragh.onion/hc/en-us/articles/115014792127-Copyright-notice}{©~2020~The
  New York Times Company}
\end{itemize}

\begin{itemize}
\tightlist
\item
  \href{https://www.nytco.com/}{NYTCo}
\item
  \href{https://help.nytimes3xbfgragh.onion/hc/en-us/articles/115015385887-Contact-Us}{Contact
  Us}
\item
  \href{https://www.nytco.com/careers/}{Work with us}
\item
  \href{https://nytmediakit.com/}{Advertise}
\item
  \href{http://www.tbrandstudio.com/}{T Brand Studio}
\item
  \href{https://www.nytimes3xbfgragh.onion/privacy/cookie-policy\#how-do-i-manage-trackers}{Your
  Ad Choices}
\item
  \href{https://www.nytimes3xbfgragh.onion/privacy}{Privacy}
\item
  \href{https://help.nytimes3xbfgragh.onion/hc/en-us/articles/115014893428-Terms-of-service}{Terms
  of Service}
\item
  \href{https://help.nytimes3xbfgragh.onion/hc/en-us/articles/115014893968-Terms-of-sale}{Terms
  of Sale}
\item
  \href{https://spiderbites.nytimes3xbfgragh.onion}{Site Map}
\item
  \href{https://help.nytimes3xbfgragh.onion/hc/en-us}{Help}
\item
  \href{https://www.nytimes3xbfgragh.onion/subscription?campaignId=37WXW}{Subscriptions}
\end{itemize}
