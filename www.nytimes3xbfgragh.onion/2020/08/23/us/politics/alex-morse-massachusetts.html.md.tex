Sections

SEARCH

\protect\hyperlink{site-content}{Skip to
content}\protect\hyperlink{site-index}{Skip to site index}

\href{https://www.nytimes3xbfgragh.onion/section/politics}{Politics}

\href{https://myaccount.nytimes3xbfgragh.onion/auth/login?response_type=cookie\&client_id=vi}{}

\href{https://www.nytimes3xbfgragh.onion/section/todayspaper}{Today's
Paper}

\href{/section/politics}{Politics}\textbar{}Alex Morse Was Accused,
Condemned and Then Vindicated. Will His Experience Change Anything?

\url{https://nyti.ms/34oJYZm}

\begin{itemize}
\item
\item
\item
\item
\item
\end{itemize}

Advertisement

\protect\hyperlink{after-top}{Continue reading the main story}

Supported by

\protect\hyperlink{after-sponsor}{Continue reading the main story}

\hypertarget{alex-morse-was-accused-condemned-and-then-vindicated-will-his-experience-change-anything}{%
\section{Alex Morse Was Accused, Condemned and Then Vindicated. Will His
Experience Change
Anything?}\label{alex-morse-was-accused-condemned-and-then-vindicated-will-his-experience-change-anything}}

Vague allegations against the Massachusetts congressional candidate
offer a case study in how progressives navigate issues of sex and power
in politics when judgment is often swift and unforgiving.

\includegraphics{https://static01.graylady3jvrrxbe.onion/images/2020/08/21/us/politics/00morse1/merlin_175959009_a3ce0ae8-412e-45e3-841f-f38e9c0f2d51-articleLarge.jpg?quality=75\&auto=webp\&disable=upscale}

\href{https://www.nytimes3xbfgragh.onion/by/jeremy-w-peters}{\includegraphics{https://static01.graylady3jvrrxbe.onion/images/2018/11/06/multimedia/author-jeremy-w-peters/author-jeremy-w-peters-thumbLarge.png}}

By \href{https://www.nytimes3xbfgragh.onion/by/jeremy-w-peters}{Jeremy
W. Peters}

\begin{itemize}
\item
  Aug. 23, 2020
\item
  \begin{itemize}
  \item
  \item
  \item
  \item
  \item
  \end{itemize}
\end{itemize}

Political careers usually don't survive allegations like these. And for
about a week, it seemed as if
\href{https://www.nytimes3xbfgragh.onion/2020/08/25/us/politics/alex-morse-richard-neal-aoc.html}{Alex
Morse's} might not either.

On Aug. 7, the student newspaper at the University of Massachusetts,
Amherst reported that the state chapter of the College Democrats had
disinvited Mr. Morse, a congressional candidate and former guest
lecturer at the university, from its future events, claiming ``numerous
incidents'' of unwanted and inappropriate advances toward students.

Mr. Morse is a 31-year-old, gay, small-city mayor and a rising star in
national progressive politics. It wasn't just his job on the line, but
also the hopes of an entire movement: His primary challenge against one
of the most powerful Democrats in Congress is being closely watched as
an indicator of the strength of the Democratic Party's insurgent left
wing.

He quickly apologized to anyone he made uncomfortable with his behavior,
while also
\href{https://twitter.com/AlexBMorse/status/1292634553669091329/photo/1}{acknowledging}
some consensual sexual relationships with college students over the
years. He said none were with anyone he taught or supervised.

Nevertheless, within hours after the story broke, Mr. Morse went from
role model to pariah. Progressive groups said they would stop
campaigning for him. The university called the news ``deeply
concerning'' and opened an investigation.

Mr. Morse said he even considered dropping out, despite his suspicions
about the motives of his accusers and the vagueness of the charges.
``This was no accident that it was happening three weeks before the
primary,'' he said in an interview.

But then the story flipped, with a cascade of head-spinning revelations.
\href{https://theintercept.com/2020/08/12/alex-morse-college-democrats-chats/}{Messages}
between some of the students that were published by The Intercept showed
they had discussed how they might damage Mr. Morse's campaign, with one
suggesting it might help his career prospects with Mr. Morse's opponent,
Representative Richard E. Neal, the chairman of the tax-writing Ways and
Means Committee.

There was more. The Massachusetts Democratic Party acknowledged that it
had provided legal advice to the College Democrats about the letter,
leading Mr. Morse to accuse Mr. Neal and his allies in the state's
Democratic leadership of having a hand in a homophobic plot to smear
him. Mr. Neal has denied any involvement.

Today, Mr. Morse is still in the race and says the allegations have only
helped his campaign. Since the initial story appeared, he has raised
more than \$410,000, and 800 volunteers have reached out offering to
help.

Though the university is still investigating, the activists who
distanced themselves from him at first are now back on board. The
students involved in writing the letter have mostly gone underground,
and the state Democratic Party has opened an independent investigation
to determine if anyone acted improperly.

\includegraphics{https://static01.graylady3jvrrxbe.onion/images/2020/08/21/us/politics/00morse2/00morse2-articleLarge.jpg?quality=75\&auto=webp\&disable=upscale}

Behind the drama and plot twists --- imagine scenes from ``Veep''
spliced into an episode of ``House of Cards'' --- is a case study of how
first impressions can be misleading when someone is accused of having
improper sexual relationships, and what happens when those charges are
leveled against a popular progressive politician in the social
media-turbocharged culture of swift retribution.

News of the complaints against Mr. Morse first surfaced in an
\href{https://dailycollegian.com/2020/08/college-democrats-allege-inappropriate-misconduct-between-holyoke-mayor-alex-morse-and-college-students/}{article}
in The Daily Collegian, the student newspaper at the University of
Massachusetts. The paper reported that the state chapter of the College
Democrats had sent a letter to Mr. Morse claiming that he had
``routinely'' made ``sexual or romantic advances'' toward students.

But there were no specific episodes of misconduct cited, no named
victims or sources and no indication of how many students had
complained, beyond the assertion that the group had heard ``countless''
reports of Mr. Morse adding students as his friends on Instagram and
sending them messages ``in a way that makes these students feel
pressured to respond due to his status.''

The reaction was swift. As the news rocketed across social media,
several groups suspended their support for Mr. Morse, including the
Sunrise Movement, an environmental organization made up mostly of young
people.

``We are a youth movement and most of our members are students,'' said
Evan Weber, the group's political director. ``We felt a really deep
accountability to listen to and take allegations from young people and
students seriously. And we really wanted this guy in office.''

The group reinstated its support once The Intercept's reporting came to
light, and Mr. Weber said he now believes the allegations were a cynical
attempt to manipulate the powerful emotions surrounding sexual
misconduct and assault.

``They exploited young people, our generation's very good inclination to
listen to people who are speaking out about harm,'' he said.

But other groups questioned the veracity of the accusations from the
beginning, leading to a schism inside the coalition of progressive
activists and politicians backing his candidacy.

Rumors about Mr. Morse's dating life had circulated anonymously for
months, and his campaign said it received several queries from reporters
looking into them. His supporters were warned about them. The Victory
Fund, a group that has endorsed Mr. Morse and works to elect lesbian,
bisexual, gay and transgender candidates, heard through an intermediary
in June about a vague and unsourced complaint that the candidate had a
checkered dating past.

But the timing of that warning was suspicious, coming just after the
organization voted to formally endorse Mr. Morse but before word had
been put out publicly, said Elliot Imse, the Victory Fund's
communications director.

The group looked into the allegations, spoke with Mr. Morse and was
satisfied there appeared to be nothing there. And shortly after The
Daily Collegian published its article, the Victory Fund was one of the
few to publicly declare it still supported him.

``It's really important for us that attacks on a candidate's sex life
and sexuality backfire,'' Mr. Imse said. ``We do not want this to go
down as a successful tactic to use with L.G.B.T.Q. candidates.''

Mr. Morse's defenders said that whatever the motives of the students
were --- in a follow-up
\href{https://twitter.com/CollegeDemsofMA/status/1292521048861220864/photo/1}{statement}
the president of the College Democrats insisted that the group's intent
was always ``to hold the mayor accountable for his actions, and to
protect our members'' --- they clearly understood how little benefit of
the doubt the accused often get in these situations, and how damaging
the charges would be.

``This is the concern around a trigger-happy cancel culture, as it gives
undue credence to the initial allegation without due diligence,'' said
Julian Cyr, a state senator in Massachusetts who
\href{https://twitter.com/JulianCyr/status/1293604310450950144/photo/1}{argued}
shortly after the accusations surfaced that withdrawing support from Mr.
Morse would set a bad precedent for progressive candidates, especially
gay ones.

``There is a very sad, well-documented history of the dating lives of
L.G.B.T.Q. people being used against them,'' said Mr. Cyr, who is gay.
``And there are potential L.G.B.T.Q. candidates who look at what Alex
Morse has gone through and decided there is a price, a risk that they
don't want to put themselves through.''

Image

The allegations came weeks before Mr. Morse and Mr. Neal face off in the
Democratic primary on Sept. 1.Credit...Don Treeger/The Republican, via
Associated Press

The students involved in writing the letter and discussing how they
could harm Mr. Morse's career --- including a suggestion to find him on
dating apps and bait him into saying something incriminating, according
to the messages reviewed by The Intercept --- have mostly gone quiet
since the episode started receiving widespread attention online. Several
of them did not respond to requests asking for elaboration on the claims
in the letter or declined to comment.

For left-leaning groups that work with the Democratic Party like the
Victory Fund, supporting Mr. Morse was no small matter because they were
going up against one of the most powerful Democrats in the House. Mr.
Neal, in his 16th term, runs the committee that oversees the tax code,
Social Security and other government functions dealing with funding.

Mr. Neal's clout, combined with the legal advice the state party
provided the College Democrats, has led Mr. Morse and his supporters to
conclude that bigger players and egos were at work. The state party,
which has a policy of not involving itself in primaries, has
commissioned an independent investigation into whether any rules were
broken.

But the lawyer who reviewed the letter, James Roosevelt Jr., said the
party's involvement had been overstated. As the party's counsel, he
offered the same help he would to any affiliated organization, which in
this case involved advising the students to change ``two or three
words'' that he said were ``too inflammatory or accusatory'' in the
draft.

He said he also advised the group not to make the letter public, as it
told him it planned to do.

``In a case of libel and slander, truth is a defense,'' Mr. Roosevelt
said he advised them, adding: ``I don't know what the truth is and you
don't either. So make it a private letter.'' The students sent the
letter to Mr. Morse privately, but soon The Daily Collegian had a copy
and published its article.

Experts who study questions of sex and power in politics said that Mr.
Morse would most likely not be the last L.G.B.T.Q. politician thrown on
the defensive about his sex life, and that we have most likely only seen
the beginning of those attacks as more people who are open about their
sexual orientation and gender identity run for office.

Even though the presidential candidacy of Pete Buttigieg broke barriers,
his sexual history was never much of an issue because he came out
relatively late in adulthood and has been with his husband, Chasten
Buttigieg, for much of that time.

But for younger, single men like Mr. Morse, their dating history is
often subject to a troubling level of scrutiny, said Joseph Fischel, who
teaches a class at Yale University called ``Theory and Politics of
Sexual Consent'' and has written extensively on the subject of sex and
power dynamics.

``Americans are OK with gay politicians as long as they're sexless,''
Mr. Fischel said. That thinking, combined with the quick judgment people
often make about political sex scandals, he added, could be especially
dangerous for L.G.B.T.Q. candidates.

``There are other things we can fabricate or make up that would sink
someone's career. But suggestions of sexual impropriety take on a life
of their own and so often lead to quick and sloppy thought,'' Mr.
Fischel added.

Mr. Morse said the situation left him deeply conflicted --- as someone
who was wrongly accused but who believes victims should not feel
intimidated to speak out, and as a gay man worried about enduring
homophobia in American society.

``The expectation shouldn't be that we have to be in monogamous,
heteronormative relationships before we enter public life,'' he said.

As he watched the condemnation of him from complete strangers spread
across social media, Mr. Morse said he was devastated.

``I have often been an observer of this but never at the center of it,''
he said. ``But what I don't want this to lead to is a diminishment of
people's very real experiences and trauma.''

``It's a difficult line to walk,'' he added.

Advertisement

\protect\hyperlink{after-bottom}{Continue reading the main story}

\hypertarget{site-index}{%
\subsection{Site Index}\label{site-index}}

\hypertarget{site-information-navigation}{%
\subsection{Site Information
Navigation}\label{site-information-navigation}}

\begin{itemize}
\tightlist
\item
  \href{https://help.nytimes3xbfgragh.onion/hc/en-us/articles/115014792127-Copyright-notice}{©~2020~The
  New York Times Company}
\end{itemize}

\begin{itemize}
\tightlist
\item
  \href{https://www.nytco.com/}{NYTCo}
\item
  \href{https://help.nytimes3xbfgragh.onion/hc/en-us/articles/115015385887-Contact-Us}{Contact
  Us}
\item
  \href{https://www.nytco.com/careers/}{Work with us}
\item
  \href{https://nytmediakit.com/}{Advertise}
\item
  \href{http://www.tbrandstudio.com/}{T Brand Studio}
\item
  \href{https://www.nytimes3xbfgragh.onion/privacy/cookie-policy\#how-do-i-manage-trackers}{Your
  Ad Choices}
\item
  \href{https://www.nytimes3xbfgragh.onion/privacy}{Privacy}
\item
  \href{https://help.nytimes3xbfgragh.onion/hc/en-us/articles/115014893428-Terms-of-service}{Terms
  of Service}
\item
  \href{https://help.nytimes3xbfgragh.onion/hc/en-us/articles/115014893968-Terms-of-sale}{Terms
  of Sale}
\item
  \href{https://spiderbites.nytimes3xbfgragh.onion}{Site Map}
\item
  \href{https://help.nytimes3xbfgragh.onion/hc/en-us}{Help}
\item
  \href{https://www.nytimes3xbfgragh.onion/subscription?campaignId=37WXW}{Subscriptions}
\end{itemize}
