Sections

SEARCH

\protect\hyperlink{site-content}{Skip to
content}\protect\hyperlink{site-index}{Skip to site index}

\href{https://www.nytimes3xbfgragh.onion/section/arts}{Arts}

\href{https://myaccount.nytimes3xbfgragh.onion/auth/login?response_type=cookie\&client_id=vi}{}

\href{https://www.nytimes3xbfgragh.onion/section/todayspaper}{Today's
Paper}

\href{/section/arts}{Arts}\textbar{}The Social-Distancing Shamers Are
Watching

\url{https://nyti.ms/3cmNeFZ}

\begin{itemize}
\item
\item
\item
\item
\item
\item
\end{itemize}

\href{https://www.nytimes3xbfgragh.onion/spotlight/at-home?action=click\&pgtype=Article\&state=default\&region=TOP_BANNER\&context=at_home_menu}{At
Home}

\begin{itemize}
\tightlist
\item
  \href{https://www.nytimes3xbfgragh.onion/2020/09/07/travel/route-66.html?action=click\&pgtype=Article\&state=default\&region=TOP_BANNER\&context=at_home_menu}{Cruise
  Along: Route 66}
\item
  \href{https://www.nytimes3xbfgragh.onion/2020/09/04/dining/sheet-pan-chicken.html?action=click\&pgtype=Article\&state=default\&region=TOP_BANNER\&context=at_home_menu}{Roast:
  Chicken With Plums}
\item
  \href{https://www.nytimes3xbfgragh.onion/2020/09/04/arts/television/dark-shadows-stream.html?action=click\&pgtype=Article\&state=default\&region=TOP_BANNER\&context=at_home_menu}{Watch:
  Dark Shadows}
\item
  \href{https://www.nytimes3xbfgragh.onion/interactive/2020/at-home/even-more-reporters-editors-diaries-lists-recommendations.html?action=click\&pgtype=Article\&state=default\&region=TOP_BANNER\&context=at_home_menu}{Explore:
  Reporters' Google Docs}
\end{itemize}

Advertisement

\protect\hyperlink{after-top}{Continue reading the main story}

Supported by

\protect\hyperlink{after-sponsor}{Continue reading the main story}

Critic's Notebook

\hypertarget{the-social-distancing-shamers-are-watching}{%
\section{The Social-Distancing Shamers Are
Watching}\label{the-social-distancing-shamers-are-watching}}

The internet has long been identified as a breeding ground for public
shaming, but the coronavirus has advanced the game.

\includegraphics{https://static01.graylady3jvrrxbe.onion/images/2020/05/12/arts/11virus-shaming/11virus-shaming-articleLarge.jpg?quality=75\&auto=webp\&disable=upscale}

\href{https://www.nytimes3xbfgragh.onion/by/amanda-hess}{\includegraphics{https://static01.graylady3jvrrxbe.onion/images/2018/02/16/multimedia/author-amanda-hess/author-amanda-hess-thumbLarge-v2.png}}

By \href{https://www.nytimes3xbfgragh.onion/by/amanda-hess}{Amanda Hess}

\begin{itemize}
\item
  May 11, 2020
\item
  \begin{itemize}
  \item
  \item
  \item
  \item
  \item
  \item
  \end{itemize}
\end{itemize}

It was 80 degrees and sunny in
\href{https://www.nytimes3xbfgragh.onion/2020/06/18/nyregion/coronavirus-ny-social-distancing.html}{New
York City}, so I was lying slack on the couch, scrolling aimlessly
through my phone, when I
\href{https://twitter.com/LachCartwright/status/1257049652559577096}{paused
on a photograph} shot a few miles away. A throng of people were
sunbathing on the green lawn of a Manhattan pier. In the center stood a
man in a pair of floral briefs and nothing else. With his arms akimbo
and his chin turned artfully to the side, he looked as if he were posing
in a bodybuilding lineup. In front of him was a woman with her face
thrown up to the sky, her hands rising as if to tousle her hair.

The photograph was taken at an angle that looked straight down the pier,
collapsing hundreds of feet of space so that the sunbathers appeared to
join together as one golden organism with toned muscles, a Celtic back
tattoo and a languid smile. The only indications that this was unlike
any other spring reverie were the baby blue surgical masks obscuring the
faces of two passers-by.

When
\href{https://twitter.com/LachCartwright/status/1257049652559577096}{that
photo} and
\href{https://twitter.com/Welcome2theBX/status/1256763447209066496}{others
of the pier} appeared on Twitter and Instagram, the scene was exposed to
another kind of crowd: people shut into their homes, skin drained of
vitamin D, their own spring plans frustrated by stay-at-home orders,
sickness or grief. Their replies to the posts swarmed with
recriminations: The sunbathers were the picture of privilege; the masks
in the photo could be counted on one hand; nobody would feel sorry if
they died.

\begin{quote}
Christopher Street Pier - 4:48pm. Not sure if the message is getting
through. \href{https://t.co/RXNQ5pUWkn}{pic.twitter.com/RXNQ5pUWkn}

--- Lachlan Cartwright (@LachCartwright)
\href{https://twitter.com/LachCartwright/status/1257049652559577096?ref_src=twsrc\%5Etfw}{May
3, 2020}
\end{quote}

In recent months, similar images --- of people strolling on a
\href{https://twitter.com/alukeonlife/status/1254568419116175367?s=21}{boardwalk
in Dorset}, England, relaxing on a
\href{https://petapixel.com/2020/05/02/controversial-photo-of-crowds-on-ca-beach-was-shot-with-a-telephoto-lens/}{beach
in Orange County} or shopping at
\href{https://www.thecanary.co/trending/2020/03/22/selfish-reckless-members-of-the-british-public-are-still-rejecting-social-distancing-during-the-coronavirus-pandemic/}{a
flower market in London} --- have been dragged around the internet for
rounds of judgment. The internet has long been identified as a breeding
ground for public shame, but the coronavirus has advanced the game. If
some benefit of the doubt between strangers still existed in online
discourse, this mysterious, highly communicable and deadly illness has
annihilated it.

Now, we are grieving, afraid and confused. We are desperate for an
outlet, and indoor finger-pointing is one of the few hobbies still
accessible to those sheltering in place. Joggers have been
\href{https://www.independent.co.uk/life-style/women/manspreading-running-social-distancing-jogging-men-coronavirus-a9491926.html}{accused
of ``manspreading'' their droplets} across public airways. An
\href{https://twitter.com/naanking/status/1258219591177629697?s=12}{infant
was scolded} for appearing maskless outdoors. Somebody
\href{https://twitter.com/benyt/status/1256758795038085121}{called the
cops} on a guy for playing the trumpet, describing it as an ``instrument
that uses saliva and wind.''

But the photograph of the crowded public space has become the defining
image of Covid shaming. We used to post photos of ourselves picnicking
in the park or sunbathing with our friends, and these shaming images
look eerily similar to those old tokens of springtime. Except now we're
taking photos of other people, and saying that those people are bad.

In the strange absence of
\href{https://www.nytimes3xbfgragh.onion/2020/05/01/opinion/coronavirus-photography.html}{photos
documenting the coronavirus death toll}, sights of gaiety stand in for
the morbid. The
\href{https://twitter.com/DWUhlfelderLaw/status/1256281243500625929}{Florida
attorney Daniel Uhlfelder} has tried to make that association literal by
dressing as the Grim Reaper and stalking open beaches. To some,
\href{https://twitter.com/MichaelHartney/status/1256731376751120388/photo/1}{the
scene on the pier} recalled ``A Sunday Afternoon on the Island of La
Grande Jatte,'' Georges Seurat's pointillist painting of bourgeois
Parisians lounging near the Seine, which is shaded with a rebuke of
their leisure-class pursuits.

The people who appear in these crowd shots are often suspected of
occupying a privileged tier of society. They may be affluent and white,
or at least young and healthy --- the kinds of people least likely to be
seriously harmed by the virus, or by police. The pier photos circulated
just as
\href{https://www.cbsnews.com/news/nypd-social-distancing-arrest-video-lower-east-side/}{video
emerged of a New York police officer} beating and arresting a black man
while enforcing social distancing. Of
\href{https://www.nytimes3xbfgragh.onion/2020/05/07/nyregion/nypd-social-distancing-race-coronavirus.html}{the
40 people the New York Police Department has arrested} for
social-distancing violations in Brooklyn, only one has been white;
during a pandemic, black New Yorkers have several reasons to feel unsafe
outside. Social-media shaming is often analogized to ``violence'' or
``policing,'' but it looks nothing like actual violent policing.

There are other cultural messages embedded in these photos, too. The
images of people gathering on Florida beaches conjure
\href{https://www.washingtonpost.com/news/magazine/wp/2019/07/15/feature/is-it-okay-to-laugh-at-florida-man-2/}{the
dynamics of the Florida Man meme}, which delights in the broad-brush
painting of Floridians as criminally stupid hicks. The Christopher
Street Pier, the site of New York's viral photos, is practically
\href{https://www.nyclgbtsites.org/site/greenwich-village-waterfront-and-the-christopher-street-pier/}{a
gay historic landmark} --- rainbow flags hang from its lampposts --- and
the outrage resurrects old tropes that gay men make irresponsible and
hedonistic use of the body. Those photos were striking not just because
the men in the photos were not wearing masks, but also because they were
not wearing shirts.

A backlash to the backlash is brewing. Some of the photographs fueling
the shaming
\href{https://www.buzzfeed.com/joeydurso/coronavirus-social-distancing-lockdown-photos}{have
appeared misleading}, collapsing depth so that cyclists appear to be
\href{https://road.cc/content/news/times-latest-paper-try-shame-cyclists-dodgy-pics-272525}{breathing
down each other's necks} and sunbathers seem to be
\href{https://petapixel.com/2020/05/02/controversial-photo-of-crowds-on-ca-beach-was-shot-with-a-telephoto-lens/?fbclid=IwAR2rFh0hA_HqeI_7BUV9lgLhyuEnbI8qQJaradf-HMDSHNoZgoWQNJExytc}{lying
all over one another}. Other photos taken of the
\href{https://www.instagram.com/p/B_vdTP3DeHC/?utm_source=ig_embed}{gathering
on the pier}
\href{https://www.instagram.com/p/B_vWvBnDdRN/?igshid=tjlen2z39jvu}{reveal
ample green space} between bodies. Defenders of the crowds have quickly
evolved into photographic forensic analysts, scouring metadata,
analyzing shadows and becoming conversant in the effects of Telephoto
lenses.

They have also noted that the existence of the photo means that the
photographer was also in the crowded place. Flocking outside on a
beautiful day, even under the cloud of the virus, is a pretty universal
impulse, and Gov. Andrew Cuomo has recognized that New Yorkers
\href{https://www.nytimes3xbfgragh.onion/article/what-is-shelter-in-place-coronavirus.html}{need
to do it} to maintain our sanity. (He has also ordered the wearing of
masks in public only
when\href{https://www.governor.ny.gov/news/no-20217-continuing-temporary-suspension-and-modification-laws-relating-disaster-emergency}{``unable
to maintain'' social distance}.) What distinguishes the self-appointed
enforcers of the crowd are their attitudes. They too may be contributing
to the city's congested public arteries, but at least they don't like
what they see.

Jon Ronson, in his 2015 account of social media ruinings,
``\href{https://www.nytimes3xbfgragh.onion/2015/04/19/books/review/jon-ronsons-so-youve-been-publicly-shamed.html}{So
You've Been Publicly Shamed},'' hypothesizes that online shamers believe
they are doing a kind of public service. Now that feeling has been
wildly elevated: It feels as if shaming could actually save lives. These
photos have already inspired policy shifts, elevating internet policing
to actual policing: Days after the pier photos were passed around, Mayor
Bill de Blasio of New York announced
\href{https://www.nytimes3xbfgragh.onion/2020/05/08/nyregion/coronavirus-new-york-update.html\#link-368d0d35}{a
plan}to limit the number of people allowed in some parks.

It's not clear that further restricting already scarce public space will
help prevent the virus's spread. What it will do is further entrench
\href{https://time.com/5832403/nypd-pandemic-police-social-distancing-arrests/}{the
power of the police}. (De Blasio
\href{https://twitter.com/NYCMayor/status/1258581714877722630}{expressed
regret} for the racial ``disparity'' in arrests while continuing to
insist that cops are ``saving lives.'') But the shaming feels righteous
in the moment. I know, from experience, that laying into a stranger
online is one of the easiest ways to convert anger into pleasure.
Shaming the happy crowd feels like stealing a little piece of their joy.

Most days I feel like both the shamer and the crowd. I have forgotten to
wear a mask, and I have eyed the maskless with suspicion. Both the
crowds and the shamers are reacting to the same stressor --- a
catastrophically incompetent government response, which boils down to
warning us all to avoid human contact indefinitely.

The day that I saw those sunbathing photos, I went to the park, too,
where I picked my way through the pack to get some exercise and see a
goose. I have even, in the past several weeks, stretched my body across
a public lawn. I'm trying to be careful, but I'm also trying to still
feel like a person. I thought I was leaving a safe distance between
myself and others, but what did I know? What does anyone? I'm lucky
nobody decided that, actually, they did know better, and took my picture
to shame me --- or worse.

Advertisement

\protect\hyperlink{after-bottom}{Continue reading the main story}

\hypertarget{site-index}{%
\subsection{Site Index}\label{site-index}}

\hypertarget{site-information-navigation}{%
\subsection{Site Information
Navigation}\label{site-information-navigation}}

\begin{itemize}
\tightlist
\item
  \href{https://help.nytimes3xbfgragh.onion/hc/en-us/articles/115014792127-Copyright-notice}{©~2020~The
  New York Times Company}
\end{itemize}

\begin{itemize}
\tightlist
\item
  \href{https://www.nytco.com/}{NYTCo}
\item
  \href{https://help.nytimes3xbfgragh.onion/hc/en-us/articles/115015385887-Contact-Us}{Contact
  Us}
\item
  \href{https://www.nytco.com/careers/}{Work with us}
\item
  \href{https://nytmediakit.com/}{Advertise}
\item
  \href{http://www.tbrandstudio.com/}{T Brand Studio}
\item
  \href{https://www.nytimes3xbfgragh.onion/privacy/cookie-policy\#how-do-i-manage-trackers}{Your
  Ad Choices}
\item
  \href{https://www.nytimes3xbfgragh.onion/privacy}{Privacy}
\item
  \href{https://help.nytimes3xbfgragh.onion/hc/en-us/articles/115014893428-Terms-of-service}{Terms
  of Service}
\item
  \href{https://help.nytimes3xbfgragh.onion/hc/en-us/articles/115014893968-Terms-of-sale}{Terms
  of Sale}
\item
  \href{https://spiderbites.nytimes3xbfgragh.onion}{Site Map}
\item
  \href{https://help.nytimes3xbfgragh.onion/hc/en-us}{Help}
\item
  \href{https://www.nytimes3xbfgragh.onion/subscription?campaignId=37WXW}{Subscriptions}
\end{itemize}
