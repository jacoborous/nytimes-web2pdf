Sections

SEARCH

\protect\hyperlink{site-content}{Skip to
content}\protect\hyperlink{site-index}{Skip to site index}

\href{https://www.nytimes3xbfgragh.onion/section/food/drinks}{Wine, Beer
\& Cocktails}

\href{https://myaccount.nytimes3xbfgragh.onion/auth/login?response_type=cookie\&client_id=vi}{}

\href{https://www.nytimes3xbfgragh.onion/section/todayspaper}{Today's
Paper}

\href{/section/food/drinks}{Wine, Beer \& Cocktails}\textbar{}The
Polarizing Power of Orange Wine

\url{https://nyti.ms/2yEEzjt}

\begin{itemize}
\item
\item
\item
\item
\item
\item
\end{itemize}

Advertisement

\protect\hyperlink{after-top}{Continue reading the main story}

Supported by

\protect\hyperlink{after-sponsor}{Continue reading the main story}

The Pour

\hypertarget{the-polarizing-power-of-orange-wine}{%
\section{The Polarizing Power of Orange
Wine}\label{the-polarizing-power-of-orange-wine}}

The best examples of these white wines, made with red techniques, are
striking and wonderful. Still some dismiss this ancient wine, now trendy
once more.

\includegraphics{https://static01.graylady3jvrrxbe.onion/images/2020/05/13/dining/07pour4/merlin_171976677_cea1e3a8-cca5-4df9-b324-02236780c778-articleLarge.jpg?quality=75\&auto=webp\&disable=upscale}

\href{https://www.nytimes3xbfgragh.onion/by/eric-asimov}{\includegraphics{https://static01.graylady3jvrrxbe.onion/images/2018/06/13/multimedia/author-eric-asimov/author-eric-asimov-thumbLarge.jpg}}

By \href{https://www.nytimes3xbfgragh.onion/by/eric-asimov}{Eric Asimov}

\begin{itemize}
\item
  Published May 7, 2020Updated May 12, 2020
\item
  \begin{itemize}
  \item
  \item
  \item
  \item
  \item
  \item
  \end{itemize}
\end{itemize}

From a distance, what divides white wines from reds seems pretty clear.
Yes, the color is obvious, but it's also the methods of production.

To make red wine, the producer begins by macerating the juice of the
grapes with the pigment-bearing skins. This adds not only color to the
juice but also tannins, which contribute texture and structure to the
darkening wine. When the fermentation is complete and the winemaker is
satisfied, the wine is drawn off the skins to begin the aging process.

Conventional white wines are made differently. A winemaker might allow
the juice to soak for a few hours, or a day, with the skins, which are
pale but not entirely without pigment. The juice is whisked away far
more quickly than it would be with reds, to begin its fermentation with
minimal color and undetectable tannins.

What if the producer switched things up? Let's say you had red grapes,
but processed them using the techniques for making white wine. A lot of
winemakers do just that, leaving the juice in contact with the skins for
only a short while, just long enough to gain a pink tinge. Millions of
bottles of these wines are sold each summer --- ``Waiter, bring me
rosé!''

By contrast, imagine that you had white grapes and wanted to make wine
using the method for reds, keeping the juice in prolonged contact with
the pale skins. This yields something altogether different, a wine seen
far less often than rosé. People are still grappling with what to call
these sorts of wines.

Some use the phrase amber wine, which describes the color of at least a
few examples. They can otherwise range from pale pink to rosy copper,
russet or a dark, almost ruby-shaded brown.

Others prefer the phrase skin-contact whites or even skin-macerated
whites, which, while technically correct, lacks any sort of evocative
power.

Most, though, have settled on orange wine, a term that has piqued the
imagination of a generous percentage of the wine-buying public. It's not
literally accurate, of course, but for that matter, neither are white
wines white nor reds red.

Orange wines may
\href{https://www.manrepeller.com/2019/07/orange-wine.html}{seem
trendy}, like a hipster fashion that bewitches bargoers from London to
Tokyo. That view would not be unreasonable, given the 20-year trajectory
of modern orange wines.

But wines made like this are the oldest form of whites, stretching back
centuries if not millenniums. In some parts of the world, like the
Republic of Georgia, white wine never stopped being made this way,
despite formidable political and cultural pressure during the years of
Soviet domination to adopt more conventional methods of mass production.

The twin forces of history and fashion combine to create a remarkable
tension around orange wines. They have been celebrated in places where
they have a long history, like Georgia and Slovenia, as emblems of
cultural identity, while simultaneously being dismissed elsewhere for
their vapid trendiness.

They have been hailed as innovative expressions, and
\href{https://www.forbes.com/sites/richardbetts/2013/01/14/tecate-orange-wine/\#25f9b53d765d}{damned}
as flawed, oxidized, repetitive and dull.

Both the praise and the criticism have been well earned.

Some orange wines are in fact insipid. People tire of them quickly when
the novelty disappears. But why would it be surprising to find bad
examples of any type of wine?

Others are thought to be flawed only because their unfiltered haziness
and tannins are so strange among white wines.

The best orange wines succeed not because they are orange but because
they express nuances of beauty and culture in profound and distinctive
ways.

Wines like those from
\href{https://www.nytimes3xbfgragh.onion/2005/05/25/dining/new-wine-in-really-old-bottles.html}{Josko
Gravner} and
\href{https://www.nytimes3xbfgragh.onion/2016/10/05/dining/organic-wine-radikon-italy.html}{Radikon}
--- who both work in the
\href{http://www.movimentoturismovino.it/en/routes/friuli-venezia-giulia/2/the-collio/}{Collio}
region of
\href{https://www.wine-searcher.com/regions-friuli-venezia+giulia}{Friuli-Venezia
Giulia}, where northeast Italy touches Slovenia --- were among the
earliest and most influential in the renaissance of these wines. They
resonate not just because they are so good qualitatively, but also
because, in their reckoning with centuries of history and conflict, they
speak powerfully, both emotionally and intellectually.

Mr. Gravner was a successful conventional winemaker in the 1980s and
'90s before rejecting modern trappings. His quest for something more
satisfying took him to Georgia, where he was inspired by the local
technique, stretching back thousands of years, of fermenting wine in
\href{https://www.tastinggeorgia.com/wine}{qvevri}, or amphorae, buried
in the earth to stay cool.

He brought qvevri and the techniques of skin-maceration back to his
winery, where, since 2001, his production has set a standard for orange
wines.

Stanko Radikon never adopted amphorae. But he made his own experiments
and developed techniques that radically reimagined how his ancestors
might have made wine. His son, Sasa, has continued the tradition since
Mr. Radikon's death in 2016.

The Gravner and Radikon wines share some qualities. They are both
textured with lightly raspy tannins. They are energetic, with aromas and
flavors that, like many orange wines, tend toward pressed flowers and
dried fruits rather than fresh ones. But they are different as well.

The Gravner wines are vibrant yet intense. They are deeply
philosophical, yet not without joy. The Radikons are equally complex,
but seem a little more lighthearted and sunny.

Both are beautiful, and both are made to exacting standards by artisans
who work the edge, taking risks that they manage with scrupulous
attention to details.

\includegraphics{https://static01.graylady3jvrrxbe.onion/images/2020/05/13/dining/07pour/merlin_171974760_3cdf9323-7ba1-4e0e-b467-f663412cf9fe-articleLarge.jpg?quality=75\&auto=webp\&disable=upscale}

Few orange wines are as uncompromising as these. But I've had plenty of
excellent examples. Just to name a few that I've enjoyed recently,
\href{https://www.lastoppa.it/homepage-eng}{La Stoppa}, Elena
Pantaleoni's wonderful estate in Emilia-Romagna, makes Ageno, a dark,
spicy, herbal wine that is made mostly of malvasia. It's a lovely,
full-on example.

Farther south, in Umbria,
\href{http://www.paolobea.com/en/1/1/home.html}{Paolo Bea} produces
Arboreus, a waxy, bright and juicy wine made of trebbiano spoletino. In
the Tyrnavos region of Greece, south of Thessaloniki,
\href{http://www.eklektikon.com/portfolio/papras/}{Papras} makes
Pleiades, a hazy amber, floral wine, made of the roditis grape, with
flavors reminiscent of orange.

From Portugal, the \href{https://aphros-wine.com/en/}{Aphros} Phaunus,
made of the loureiro grape, was fermented in amphorae. It's bright,
lively and balanced.

And from Georgia, I had a wonderful wine that could have been mistaken
for a light red. It was the 2017 \href{https://www.iberieli.com/}{Zurab
Topuridze} from the Guria region. Made of the chkhaveri grape, it had
wild flavors of dried flowers, dried fruits and spices, and it went
beautifully with grilled flank steak.

From closer to home, I recently had two skin-fermented pinot gris. One
was from \href{https://www.twoshepherds.com/}{Two Shepherds} in Sonoma
Valley, the other from \href{https://www.donkeyandgoat.com/}{Donkey \&
Goat}, which buys the grapes from the Anderson Valley in Mendocino.
Pinot gris, like the Greek roditis, is a white grape with a distinct
pink cast, which gives the wines a coppery hue when made with
skin-contact techniques.

Image

Tracey Brandt, the proprietor of Donkey \& Goat with her husband, Jared,
at their winery in Berkeley, Calif. They make several orange
wines.Credit...Sari Blum

The Two Shepherds was savory, with a lightly musky quality, while the
Donkey \& Goat seemed fresher and more zesty, with bright floral aromas.

Far more wonderful examples exist, particularly in Georgia, Slovenia and
the Friuli-Venezia Giulia region. For more suggestions, Simon J. Woolf's
book ``\href{https://amber-revolution.com/}{Amber Revolution}'' is an
excellent source.

The best places to buy orange wines tend to be shops that also have
selections of natural wines.

Sometimes orange wines turn up where you least expect to find them. I
recently came upon a 2017 Akatcha, a skin-macerated white from
\href{https://www.domainebize.fr/en/the-domain.html}{Simon Bize et
Fils}, one of my favorite Burgundy producers. It was made of pinot
beurot, the Burgundian name for pinot gris, which is legally permitted
in Burgundy but rarely seen. It was spicy, energetic and herbal, with
long, lingering flavors.

Image

Chisa Bize of Domaine Simon Bize et Fils in Burgundy, which released its
first orange wine, a 2017 made of pinot gris.Credit...Ellen Silverman
for The New York Times

I was surprised to see it, and contacted Chisa Bize, the proprietor, to
learn more about it. She said by email that Bize's small plot of pinot
beurot ordinarily went into its Bourgogne Blanc, but that occasionally a
varietal pinot beurot had been released.

She had been curious about both making an orange wine, she said, and
about working without sulfur dioxide, an almost universally used
antioxidant and stabilizer. After experimentation, the 2017 was its
first release. She plans on doing it again.

Aside from questions of terroir, grape and intention, what distinguishes
bottles within the orange wine genre is often how long the juice has
been macerated with the skins. On the extreme end, Gravner and Radikon
macerate for many months, depending on the vintage. The Topuridze wine
from Georgia saw six months of maceration, and the Ageno around four
months.

On the tamer side, the two California examples both were macerated for
about five days, and the Bize for three to four days.

As with all types of wine, good and bad versions abound. Flaws that used
to be common in wine, like oxidation or too much volatile acidity, have
largely been vanquished in conventional wines. But they can still appear
in wines, like many in the orange genre, that avoid technological
solutions to these sorts of problems.

That may sound as if I'm placing orange wines in the natural-wine
category. Many would qualify, but not nearly all. Natural wines must
satisfy a much greater set of imperatives, including how the grapes are
farmed, while orange wines must simply be skin-macerated.

While faults may seem starkly obvious in orange and natural wines, it's
important to understand that the faults of dullness, sameness and bland
sterility are accepted without comment in conventional wines because
they are so common.

That does not excuse either set of faults. But it helps in understanding
them.

\emph{Follow} \href{https://twitter.com/nytfood}{\emph{NYT Food on
Twitter}} \emph{and}
\href{https://www.instagram.com/nytcooking/}{\emph{NYT Cooking on
Instagram}}\emph{,}
\href{https://www.facebookcorewwwi.onion/nytcooking/}{\emph{Facebook}}\emph{,}
\href{https://www.youtube.com/nytcooking}{\emph{YouTube}} \emph{and}
\href{https://www.pinterest.com/nytcooking/}{\emph{Pinterest}}\emph{.}
\href{https://www.nytimes3xbfgragh.onion/newsletters/cooking}{\emph{Get
regular updates from NYT Cooking, with recipe suggestions, cooking tips
and shopping advice}}\emph{.}

Advertisement

\protect\hyperlink{after-bottom}{Continue reading the main story}

\hypertarget{site-index}{%
\subsection{Site Index}\label{site-index}}

\hypertarget{site-information-navigation}{%
\subsection{Site Information
Navigation}\label{site-information-navigation}}

\begin{itemize}
\tightlist
\item
  \href{https://help.nytimes3xbfgragh.onion/hc/en-us/articles/115014792127-Copyright-notice}{©~2020~The
  New York Times Company}
\end{itemize}

\begin{itemize}
\tightlist
\item
  \href{https://www.nytco.com/}{NYTCo}
\item
  \href{https://help.nytimes3xbfgragh.onion/hc/en-us/articles/115015385887-Contact-Us}{Contact
  Us}
\item
  \href{https://www.nytco.com/careers/}{Work with us}
\item
  \href{https://nytmediakit.com/}{Advertise}
\item
  \href{http://www.tbrandstudio.com/}{T Brand Studio}
\item
  \href{https://www.nytimes3xbfgragh.onion/privacy/cookie-policy\#how-do-i-manage-trackers}{Your
  Ad Choices}
\item
  \href{https://www.nytimes3xbfgragh.onion/privacy}{Privacy}
\item
  \href{https://help.nytimes3xbfgragh.onion/hc/en-us/articles/115014893428-Terms-of-service}{Terms
  of Service}
\item
  \href{https://help.nytimes3xbfgragh.onion/hc/en-us/articles/115014893968-Terms-of-sale}{Terms
  of Sale}
\item
  \href{https://spiderbites.nytimes3xbfgragh.onion}{Site Map}
\item
  \href{https://help.nytimes3xbfgragh.onion/hc/en-us}{Help}
\item
  \href{https://www.nytimes3xbfgragh.onion/subscription?campaignId=37WXW}{Subscriptions}
\end{itemize}
