Sections

SEARCH

\protect\hyperlink{site-content}{Skip to
content}\protect\hyperlink{site-index}{Skip to site index}

\href{https://www.nytimes3xbfgragh.onion/section/health}{Health}

\href{https://myaccount.nytimes3xbfgragh.onion/auth/login?response_type=cookie\&client_id=vi}{}

\href{https://www.nytimes3xbfgragh.onion/section/todayspaper}{Today's
Paper}

\href{/section/health}{Health}\textbar{}After Recovery From the
Coronavirus, Most People Carry Antibodies

\url{https://nyti.ms/3fsjclX}

\begin{itemize}
\item
\item
\item
\item
\item
\end{itemize}

\hypertarget{the-coronavirus-outbreak}{%
\subsubsection{\texorpdfstring{\href{https://www.nytimes3xbfgragh.onion/news-event/coronavirus?name=styln-coronavirus-national\&region=TOP_BANNER\&block=storyline_menu_recirc\&action=click\&pgtype=Article\&impression_id=c9eecf10-f283-11ea-b022-6936941775de\&variant=undefined}{The
Coronavirus
Outbreak}}{The Coronavirus Outbreak}}\label{the-coronavirus-outbreak}}

\begin{itemize}
\tightlist
\item
  live\href{https://www.nytimes3xbfgragh.onion/2020/09/09/world/covid-19-coronavirus.html?name=styln-coronavirus-national\&region=TOP_BANNER\&block=storyline_menu_recirc\&action=click\&pgtype=Article\&impression_id=c9eecf11-f283-11ea-b022-6936941775de\&variant=undefined}{Latest
  Updates}
\item
  \href{https://www.nytimes3xbfgragh.onion/interactive/2020/us/coronavirus-us-cases.html?name=styln-coronavirus-national\&region=TOP_BANNER\&block=storyline_menu_recirc\&action=click\&pgtype=Article\&impression_id=c9eef620-f283-11ea-b022-6936941775de\&variant=undefined}{Maps
  and Cases}
\item
  \href{https://www.nytimes3xbfgragh.onion/interactive/2020/science/coronavirus-vaccine-tracker.html?name=styln-coronavirus-national\&region=TOP_BANNER\&block=storyline_menu_recirc\&action=click\&pgtype=Article\&impression_id=c9eef621-f283-11ea-b022-6936941775de\&variant=undefined}{Vaccine
  Tracker}
\item
  \href{https://www.nytimes3xbfgragh.onion/2020/09/02/your-money/eviction-moratorium-covid.html?name=styln-coronavirus-national\&region=TOP_BANNER\&block=storyline_menu_recirc\&action=click\&pgtype=Article\&impression_id=c9eef622-f283-11ea-b022-6936941775de\&variant=undefined}{Eviction
  Moratorium}
\item
  \href{https://www.nytimes3xbfgragh.onion/interactive/2020/09/02/magazine/food-insecurity-hunger-us.html?name=styln-coronavirus-national\&region=TOP_BANNER\&block=storyline_menu_recirc\&action=click\&pgtype=Article\&impression_id=c9eef623-f283-11ea-b022-6936941775de\&variant=undefined}{American
  Hunger}
\end{itemize}

Advertisement

\protect\hyperlink{after-top}{Continue reading the main story}

Supported by

\protect\hyperlink{after-sponsor}{Continue reading the main story}

\hypertarget{after-recovery-from-the-coronavirus-most-people-carry-antibodies}{%
\section{After Recovery From the Coronavirus, Most People Carry
Antibodies}\label{after-recovery-from-the-coronavirus-most-people-carry-antibodies}}

A new study adds to evidence of immunity among those who have already
been exposed to the pathogen.

\includegraphics{https://static01.graylady3jvrrxbe.onion/images/2020/05/10/autossell/antibody-testing-cover/antibody-testing-cover-videoSixteenByNineJumbo1600.jpg}

By
\href{https://www.nytimes3xbfgragh.onion/by/apoorva-mandavilli}{Apoorva
Mandavilli}

\begin{itemize}
\item
  Published May 7, 2020Updated Aug. 20, 2020
\item
  \begin{itemize}
  \item
  \item
  \item
  \item
  \item
  \end{itemize}
\end{itemize}

A new study offers a glimmer of hope in the grim fight against the
coronavirus: Nearly everyone who has had the disease --- regardless of
age, sex or severity of illness ---
\href{https://www.medrxiv.org/content/10.1101/2020.04.30.20085613v1}{makes
antibodies to the virus}.

The study, posted online on Tuesday but not yet reviewed by experts,
also hints that anyone who has
\href{https://www.nytimes3xbfgragh.onion/2020/06/17/nyregion/coronavirus-recovery-hospital.html}{recovered
from infection} may safely return to work --- although it is unclear how
long their protection might last.

``This is very good news,'' said Angela Rasmussen, a virologist at
Columbia University in New York who was not involved with the work.

\href{https://www.nytimes3xbfgragh.onion/2020/08/20/nyregion/nyc-coronavirus-antibody-testing.html}{Antibodies}
are immune molecules produced by the body to fight pathogens. The
presence of antibodies in the blood typically confers at least some
protection against the invader.

Health officials in several countries, including the United States, have
hung their hopes on tests that identify coronavirus antibodies to decide
who is immune and can go back to work. People who are immune could
replace vulnerable individuals, especially in high-transmission settings
like hospitals, building what researchers call
\href{https://www.nature.com/articles/s41591-020-0895-3}{``shield
immunity'' in the population}.

But most antibody tests are
\href{https://www.nytimes3xbfgragh.onion/2020/04/24/health/coronavirus-antibody-tests.html}{fraught
with false positives} --- picking up antibody signals where there are
none. The new study relied on a test developed by Florian Krammer, a
virologist at the Icahn School of Medicine at Mount Sinai, that has a
less than 1 percent chance of producing false-positive results.

Several small studies have given reason to hope that people who have had
Covid-19, the illness caused by the coronavirus, would gain some
immunity for some period of time. The new study is the largest by far,
with results from 1,343 people in and around New York City.

The study also eased a niggling worry that only some people --- only
those who were severely ill, for example --- might make antibodies. In
fact, the level of antibodies did not differ by age or sex, and even
people who had only mild symptoms produced a healthy amount.

Having antibodies is not the same as having immunity to the virus. But
in previous research, Dr. Krammer's team has
\href{https://www.medrxiv.org/content/10.1101/2020.03.17.20037713v2}{shown}
that antibody levels are closely linked with the ability to disarm the
virus, the key to immunity.

``It really shows that most people do develop antibodies, and that
there's very good correlation between those antibodies and their
capability to neutralize virus,'' Dr. Rasmussen said.

\hypertarget{latest-updates-the-coronavirus-outbreak}{%
\section{\texorpdfstring{\href{https://www.nytimes3xbfgragh.onion/2020/09/09/world/covid-19-coronavirus.html?action=click\&pgtype=Article\&state=default\&region=MAIN_CONTENT_1\&context=storylines_live_updates}{Latest
Updates: The Coronavirus
Outbreak}}{Latest Updates: The Coronavirus Outbreak}}\label{latest-updates-the-coronavirus-outbreak}}

Updated 2020-09-09T09:58:22.117Z

\begin{itemize}
\tightlist
\item
  \href{https://www.nytimes3xbfgragh.onion/2020/09/09/world/covid-19-coronavirus.html?action=click\&pgtype=Article\&state=default\&region=MAIN_CONTENT_1\&context=storylines_live_updates\#link-70cea8bb}{As
  drugmakers pledge to thoroughly vet a vaccine, one company pauses its
  trials for a safety review.}
\item
  \href{https://www.nytimes3xbfgragh.onion/2020/09/09/world/covid-19-coronavirus.html?action=click\&pgtype=Article\&state=default\&region=MAIN_CONTENT_1\&context=storylines_live_updates\#link-4438dd7}{Facing
  a surge in cases, Britain plans to limit most gatherings to six
  people.}
\item
  \href{https://www.nytimes3xbfgragh.onion/2020/09/09/world/covid-19-coronavirus.html?action=click\&pgtype=Article\&state=default\&region=MAIN_CONTENT_1\&context=storylines_live_updates\#link-11cec4c0}{Quarantine
  breakdowns at colleges in the U.S. are leaving some at risk.}
\end{itemize}

\href{https://www.nytimes3xbfgragh.onion/2020/09/09/world/covid-19-coronavirus.html?action=click\&pgtype=Article\&state=default\&region=MAIN_CONTENT_1\&context=storylines_live_updates}{See
more updates}

More live coverage:
\href{https://www.nytimes3xbfgragh.onion/live/2020/09/08/business/stock-market-today-coronavirus?action=click\&pgtype=Article\&state=default\&region=MAIN_CONTENT_1\&context=storylines_live_updates}{Markets}

Researchers at Mount Sinai tested people who signed up to be
\href{https://www.nytimes3xbfgragh.onion/2020/03/26/health/plasma-coronavirus-treatment.html}{donors
of convalescent plasma}, antibodies extracted from blood. The project
has enrolled more than 15,000 people so far, according to Dr. Ania
Wajnberg, who is leading the effort.

The new study is an analysis of results of the first set of donors. Over
all, only 3 percent of these participants had been seen in the emergency
department or had been hospitalized. The remaining subjects had only
mild or moderate symptoms.

``To my knowledge, this is the largest group of people described with
mild disease,'' Dr. Wajnberg said.

The criteria for inclusion became more stringent as the team learned
more about the coronavirus. For example, they initially required the
potential donors to be free of symptoms for only three days but later
extended that to 14 days.

The team tested 624 people who had tested positive for the virus and had
recovered. At first, just 511 of them had high antibody levels; 42 had
low levels; and 71 had none. When 64 of the subjects with weak or no
levels were retested more than a week later, however, all but three had
at least some antibodies.

That suggests the timing of testing for antibodies can greatly affect
the results, the researchers said. ``We weren't looking exactly at this,
but we had enough to say that 14 days is probably a little too early,''
Dr. Wajnberg said.

There was even a difference between levels at 20 days versus 24 days,
she said, suggesting that the optimal time for an antibody test is well
after symptoms begin. ``What we're telling people now is at least three
weeks after symptom onset,'' Dr. Wajnberg said.

\includegraphics{https://static01.graylady3jvrrxbe.onion/images/2020/05/07/science/07VIRUS-ANTIBODIES2/merlin_171931419_5665ae30-223f-4027-bb7e-c1136f68a8e9-articleLarge.jpg?quality=75\&auto=webp\&disable=upscale}

Because tests to diagnose coronavirus infection were unavailable to most
people in New York City in March, the researchers included another 719
people in their study who suspected they had Covid-19 based on symptoms
and exposure to the virus, but in whom the illness had not been
diagnosed.

In this group, the researchers found a different picture altogether. The
majority of these people --- 62 percent --- did not seem to have
antibodies.

Some of them may have been tested too soon after their illness for
antibodies to be detectable. But many probably mistook influenza,
another viral infection or even allergies for Covid-19, Dr. Wajnberg
said.

``I think literally everybody in New York thinks they've had it,'' she
said. ``People shouldn't assume the fever they had in January was Covid
and they're immune.''

Other experts were more struck by the percentage of people who turned
out to have antibodies, even though the coronavirus had never been
diagnosed in them.

The number suggests that ``in cities like New York, there are a
tremendous number of undiagnosed infections,'' said Taia Wang, a viral
immunologist at Stanford University.

An antibody survey conducted by New York State officials found that 20
percent of city residents had been infected.

\href{https://www.nytimes3xbfgragh.onion/news-event/coronavirus?action=click\&pgtype=Article\&state=default\&region=MAIN_CONTENT_3\&context=storylines_faq}{}

\hypertarget{the-coronavirus-outbreak-}{%
\subsubsection{The Coronavirus Outbreak
›}\label{the-coronavirus-outbreak-}}

\hypertarget{frequently-asked-questions}{%
\paragraph{Frequently Asked
Questions}\label{frequently-asked-questions}}

Updated September 4, 2020

\begin{itemize}
\item ~
  \hypertarget{what-are-the-symptoms-of-coronavirus}{%
  \paragraph{What are the symptoms of
  coronavirus?}\label{what-are-the-symptoms-of-coronavirus}}

  \begin{itemize}
  \tightlist
  \item
    In the beginning, the coronavirus
    \href{https://www.nytimes3xbfgragh.onion/article/coronavirus-facts-history.html?action=click\&pgtype=Article\&state=default\&region=MAIN_CONTENT_3\&context=storylines_faq\#link-6817bab5}{seemed
    like it was primarily a respiratory illness}~--- many patients had
    fever and chills, were weak and tired, and coughed a lot, though
    some people don't show many symptoms at all. Those who seemed
    sickest had pneumonia or acute respiratory distress syndrome and
    received supplemental oxygen. By now, doctors have identified many
    more symptoms and syndromes. In April,
    \href{https://www.nytimes3xbfgragh.onion/2020/04/27/health/coronavirus-symptoms-cdc.html?action=click\&pgtype=Article\&state=default\&region=MAIN_CONTENT_3\&context=storylines_faq}{the
    C.D.C. added to the list of early signs}~sore throat, fever, chills
    and muscle aches. Gastrointestinal upset, such as diarrhea and
    nausea, has also been observed. Another telltale sign of infection
    may be a sudden, profound diminution of one's
    \href{https://www.nytimes3xbfgragh.onion/2020/03/22/health/coronavirus-symptoms-smell-taste.html?action=click\&pgtype=Article\&state=default\&region=MAIN_CONTENT_3\&context=storylines_faq}{sense
    of smell and taste.}~Teenagers and young adults in some cases have
    developed painful red and purple lesions on their fingers and toes
    --- nicknamed ``Covid toe'' --- but few other serious symptoms.
  \end{itemize}
\item ~
  \hypertarget{why-is-it-safer-to-spend-time-together-outside}{%
  \paragraph{Why is it safer to spend time together
  outside?}\label{why-is-it-safer-to-spend-time-together-outside}}

  \begin{itemize}
  \tightlist
  \item
    \href{https://www.nytimes3xbfgragh.onion/2020/05/15/us/coronavirus-what-to-do-outside.html?action=click\&pgtype=Article\&state=default\&region=MAIN_CONTENT_3\&context=storylines_faq}{Outdoor
    gatherings}~lower risk because wind disperses viral droplets, and
    sunlight can kill some of the virus. Open spaces prevent the virus
    from building up in concentrated amounts and being inhaled, which
    can happen when infected people exhale in a confined space for long
    stretches of time, said Dr. Julian W. Tang, a virologist at the
    University of Leicester.
  \end{itemize}
\item ~
  \hypertarget{why-does-standing-six-feet-away-from-others-help}{%
  \paragraph{Why does standing six feet away from others
  help?}\label{why-does-standing-six-feet-away-from-others-help}}

  \begin{itemize}
  \tightlist
  \item
    The coronavirus spreads primarily through droplets from your mouth
    and nose, especially when you cough or sneeze. The C.D.C., one of
    the organizations using that measure,
    \href{https://www.nytimes3xbfgragh.onion/2020/04/14/health/coronavirus-six-feet.html?action=click\&pgtype=Article\&state=default\&region=MAIN_CONTENT_3\&context=storylines_faq}{bases
    its recommendation of six feet}~on the idea that most large droplets
    that people expel when they cough or sneeze will fall to the ground
    within six feet. But six feet has never been a magic number that
    guarantees complete protection. Sneezes, for instance, can launch
    droplets a lot farther than six feet,
    \href{https://jamanetwork.com/journals/jama/fullarticle/2763852}{according
    to a recent study}. It's a rule of thumb: You should be safest
    standing six feet apart outside, especially when it's windy. But
    keep a mask on at all times, even when you think you're far enough
    apart.
  \end{itemize}
\item ~
  \hypertarget{i-have-antibodies-am-i-now-immune}{%
  \paragraph{I have antibodies. Am I now
  immune?}\label{i-have-antibodies-am-i-now-immune}}

  \begin{itemize}
  \tightlist
  \item
    As of right
    now,\href{https://www.nytimes3xbfgragh.onion/2020/07/22/health/covid-antibodies-herd-immunity.html?action=click\&pgtype=Article\&state=default\&region=MAIN_CONTENT_3\&context=storylines_faq}{~that
    seems likely, for at least several months.}~There have been
    frightening accounts of people suffering what seems to be a second
    bout of Covid-19. But experts say these patients may have a
    drawn-out course of infection, with the virus taking a slow toll
    weeks to months after initial exposure.~People infected with the
    coronavirus typically
    \href{https://www.nature.com/articles/s41586-020-2456-9}{produce}~immune
    molecules called antibodies, which are
    \href{https://www.nytimes3xbfgragh.onion/2020/05/07/health/coronavirus-antibody-prevalence.html?action=click\&pgtype=Article\&state=default\&region=MAIN_CONTENT_3\&context=storylines_faq}{protective
    proteins made in response to an
    infection}\href{https://www.nytimes3xbfgragh.onion/2020/05/07/health/coronavirus-antibody-prevalence.html?action=click\&pgtype=Article\&state=default\&region=MAIN_CONTENT_3\&context=storylines_faq}{.
    These antibodies may}~last in the body
    \href{https://www.nature.com/articles/s41591-020-0965-6}{only two to
    three months}, which may seem worrisome, but that's~perfectly normal
    after an acute infection subsides, said Dr. Michael Mina, an
    immunologist at Harvard University. It may be possible to get the
    coronavirus again, but it's highly unlikely that it would be
    possible in a short window of time from initial infection or make
    people sicker the second time.
  \end{itemize}
\item ~
  \hypertarget{what-are-my-rights-if-i-am-worried-about-going-back-to-work}{%
  \paragraph{What are my rights if I am worried about going back to
  work?}\label{what-are-my-rights-if-i-am-worried-about-going-back-to-work}}

  \begin{itemize}
  \tightlist
  \item
    Employers have to provide
    \href{https://www.osha.gov/SLTC/covid-19/standards.html}{a safe
    workplace}~with policies that protect everyone equally.
    \href{https://www.nytimes3xbfgragh.onion/article/coronavirus-money-unemployment.html?action=click\&pgtype=Article\&state=default\&region=MAIN_CONTENT_3\&context=storylines_faq}{And
    if one of your co-workers tests positive for the coronavirus, the
    C.D.C.}~has said that
    \href{https://www.cdc.gov/coronavirus/2019-ncov/community/guidance-business-response.html}{employers
    should tell their employees}~-\/- without giving you the sick
    employee's name -\/- that they may have been exposed to the virus.
  \end{itemize}
\end{itemize}

Another finding from the study --- that diagnostic PCR tests can be
positive up to 28 days after the start of infection --- is also
important, Dr. Wang said. These tests look for genetic fragments, not
antibodies, and suggest an active or waning infection.

``As far as known unknowns about SARS-CoV-2, this one really stands
out,'' she said. ``We really need to know, how long does it take the
body to clear the virus? How long are people contagious? We don't know
the answer to that.''

She and other scientists said it was highly unlikely that a positive
test so long after symptoms appeared represents infectious virus.
Researchers in South Korea recently announced, for example, that several
suspected cases of ``reinfection'' were a result of PCR tests picking up
\href{https://www.newsweek.com/south-korea-experts-say-recovered-coronavirus-patients-retested-positive-because-dead-virus-parts-1500998}{remnants
of dead virus}.

Genetic material from the measles virus can show up in tests six months
after the illness, Dr. Krammer noted. And genetic fragments of Ebola and
Zika viruses are known to persist even longer in the body.

Still, Dr. Wang said, ``Until we do know, it's prudent for everyone to
proceed as if a positive PCR test means contagious virus.'' The Centers
for Disease Control and Prevention recommends that people isolate for 10
days after the onset of symptoms, but that period may need to be longer.

\textbf{\emph{{[}}\href{http://on.fb.me/1paTQ1h}{\emph{Like the Science
Times page on Facebook.}}} ****** \emph{\textbar{} Sign up for the}
\textbf{\href{http://nyti.ms/1MbHaRU}{\emph{Science Times
newsletter.}}\emph{{]}}}

Experts said the next step would be to confirm that the presence of
antibodies in the blood means protection from the coronavirus. The body
depends on a subset of antibodies, called neutralizing antibodies, to
shield it from the coronavirus.

``The question now becomes to what extent those are neutralizing
antibodies and whether that leads to protection from infection --- all
of which we should presume are yes,'' said Sean Whelan, a virologist at
Washington University in St. Louis.

In Dr. Krammer's previous work, to be published in the journal Nature
Medicine, his team tested whether the antibodies have neutralizing
power. The researchers found that in about a dozen people, including
some who had mild symptoms, the level of antibodies in the blood
corresponded to the level of neutralizing activity.

So everyone who makes antibodies is likely to have some immunity to the
virus, Dr. Krammer said: ``I'm fairly confident about this.'' Another
way to assess immunity would be to show that purified antibodies can
prevent coronavirus infection in an animal.

But perhaps the most urgent question, especially as
\href{https://www.nytimes3xbfgragh.onion/2020/05/02/us/politics/vaccines-coronavirus-research.html}{research
on vaccines ramps up}, is how long that immunity might last.

Even if the levels of antibodies fall over time to undetectable levels,
people may still retain some protection from the coronavirus.

Immune cells called T cells are valuable soldiers in fighting pathogens,
and at least
\href{https://www.sciencedirect.com/science/article/pii/S1074761320301813}{one
study has shown} that the coronavirus provokes a strong response from
these cells. So-called memory cells, or B cells, may also kick into gear
when they encounter the coronavirus, churning out more antibodies.

Ultimately, however, the answer to how long immunity lasts will come
only with patience.

``Unless someone has come up with some way to speed that process up,''
Dr. Rasmussen said, ``the only way to tell that is by following these
patients over time.''

Advertisement

\protect\hyperlink{after-bottom}{Continue reading the main story}

\hypertarget{site-index}{%
\subsection{Site Index}\label{site-index}}

\hypertarget{site-information-navigation}{%
\subsection{Site Information
Navigation}\label{site-information-navigation}}

\begin{itemize}
\tightlist
\item
  \href{https://help.nytimes3xbfgragh.onion/hc/en-us/articles/115014792127-Copyright-notice}{©~2020~The
  New York Times Company}
\end{itemize}

\begin{itemize}
\tightlist
\item
  \href{https://www.nytco.com/}{NYTCo}
\item
  \href{https://help.nytimes3xbfgragh.onion/hc/en-us/articles/115015385887-Contact-Us}{Contact
  Us}
\item
  \href{https://www.nytco.com/careers/}{Work with us}
\item
  \href{https://nytmediakit.com/}{Advertise}
\item
  \href{http://www.tbrandstudio.com/}{T Brand Studio}
\item
  \href{https://www.nytimes3xbfgragh.onion/privacy/cookie-policy\#how-do-i-manage-trackers}{Your
  Ad Choices}
\item
  \href{https://www.nytimes3xbfgragh.onion/privacy}{Privacy}
\item
  \href{https://help.nytimes3xbfgragh.onion/hc/en-us/articles/115014893428-Terms-of-service}{Terms
  of Service}
\item
  \href{https://help.nytimes3xbfgragh.onion/hc/en-us/articles/115014893968-Terms-of-sale}{Terms
  of Sale}
\item
  \href{https://spiderbites.nytimes3xbfgragh.onion}{Site Map}
\item
  \href{https://help.nytimes3xbfgragh.onion/hc/en-us}{Help}
\item
  \href{https://www.nytimes3xbfgragh.onion/subscription?campaignId=37WXW}{Subscriptions}
\end{itemize}
