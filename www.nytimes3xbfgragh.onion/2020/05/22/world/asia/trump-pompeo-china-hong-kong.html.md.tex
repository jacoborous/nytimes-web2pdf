Sections

SEARCH

\protect\hyperlink{site-content}{Skip to
content}\protect\hyperlink{site-index}{Skip to site index}

\href{https://www.nytimes3xbfgragh.onion/section/world/asia}{Asia
Pacific}

\href{https://myaccount.nytimes3xbfgragh.onion/auth/login?response_type=cookie\&client_id=vi}{}

\href{https://www.nytimes3xbfgragh.onion/section/todayspaper}{Today's
Paper}

\href{/section/world/asia}{Asia Pacific}\textbar{}Top U.S. Officials
Threaten Action on China Over Hong Kong Security Law

\url{https://nyti.ms/2AWe9KZ}

\begin{itemize}
\item
\item
\item
\item
\item
\item
\end{itemize}

Advertisement

\protect\hyperlink{after-top}{Continue reading the main story}

Supported by

\protect\hyperlink{after-sponsor}{Continue reading the main story}

\hypertarget{top-us-officials-threaten-action-on-china-over-hong-kong-security-law}{%
\section{Top U.S. Officials Threaten Action on China Over Hong Kong
Security
Law}\label{top-us-officials-threaten-action-on-china-over-hong-kong-security-law}}

Secretary of State Mike Pompeo and the White House economic adviser,
Kevin Hassett, signaled the Trump administration would punish China.

\includegraphics{https://static01.graylady3jvrrxbe.onion/images/2020/05/22/us/politics/22dc-trump-china/merlin_172730544_faa4f3a7-a743-449f-be2b-e22eb4c39fbd-articleLarge.jpg?quality=75\&auto=webp\&disable=upscale}

\href{https://www.nytimes3xbfgragh.onion/by/michael-crowley}{\includegraphics{https://static01.graylady3jvrrxbe.onion/images/2019/10/25/reader-center/author-michael-crowley/author-michael-crowley-thumbLarge-v2.png}}\href{https://www.nytimes3xbfgragh.onion/by/edward-wong}{\includegraphics{https://static01.graylady3jvrrxbe.onion/images/2018/09/24/multimedia/author-edward-wong/author-edward-wong-thumbLarge-v5.png}}\href{https://www.nytimes3xbfgragh.onion/by/ana-swanson}{\includegraphics{https://static01.graylady3jvrrxbe.onion/images/2018/12/10/multimedia/author-ana-swanson/author-ana-swanson-thumbLarge.png}}

By \href{https://www.nytimes3xbfgragh.onion/by/michael-crowley}{Michael
Crowley},
\href{https://www.nytimes3xbfgragh.onion/by/edward-wong}{Edward Wong}
and \href{https://www.nytimes3xbfgragh.onion/by/ana-swanson}{Ana
Swanson}

\begin{itemize}
\item
  Published May 22, 2020Updated May 24, 2020
\item
  \begin{itemize}
  \item
  \item
  \item
  \item
  \item
  \item
  \end{itemize}
\end{itemize}

WASHINGTON ---
\href{https://www.nytimes3xbfgragh.onion/2020/05/21/us/politics/mike-pompeo-inspector-general.html}{Secretary
of State Mike Pompeo} said on Friday that a broad new security measure
proposed by China would amount to a ``death knell'' for Hong Kong's
\href{https://www.nytimes3xbfgragh.onion/2019/11/14/business/hong-kong-protests-recession.html}{political
freedoms}, as Trump administration officials warned of punishments ---
possibly including revoking the territory's special economic and trading
status.

A move by the United States to end that special status would subject
goods from Hong Kong to the same American tariffs now applied to ones
from mainland China.

Other economic and visa restrictions that Washington has imposed on
China would also be applied to the semiautonomous territory. That could
deal a major blow to Hong Kong's historic role as a base for
multinational companies that command global lanes of trade and commerce
from a cluster of skyscrapers on the edge of the South China Sea.

It would also deepen the
\href{https://www.nytimes3xbfgragh.onion/2020/03/22/us/politics/coronavirus-us-china.html}{sense
of crisis} forming around United States-China relations --- now at their
\href{https://www.nytimes3xbfgragh.onion/2020/05/01/us/politics/coronavirus-china-trump.html}{worst
point in decades} --- in the wake of the coronavirus pandemic and a 2020
presidential campaign in which China already features prominently.

``If Hong Kong loses preferential trade treatment, U.S. tariffs and
export controls on China would apply to Hong Kong,'' said
\href{https://www.brookings.edu/experts/ryan-hass/}{Ryan Hass}, a China
director on President Barack Obama's National Security Council and
former diplomat who is now at the Brookings Institution. ``This
action-reaction sequence of China tightening its hold on Hong Kong and
America responding by withdrawing preferential treatment would weaken
Hong Kong's status as a global financial hub.''

On another front to pressure Beijing, the Trump administration
\href{https://www.commerce.gov/news/press-releases/2020/05/commerce-department-add-nine-chinese-entities-related-human-rights}{added
more than}
\href{https://www.commerce.gov/news/press-releases/2020/05/commerce-department-add-two-dozen-chinese-companies-ties-wmd-and}{30
Chinese companies} and other entities on Friday to a blacklist that
restricts their access to American technology, saying that they had
provided support to the Chinese military, helped Beijing construct a
high-tech surveillance system and aided in human rights violations
against Muslims.

Bipartisan support is also growing in Congress for sanctions on mainland
Chinese officials and entities associated with the proposed Hong Kong
security measure, which is likely to be enacted in coming days.

Mr. Pompeo issued his statement hours after Beijing
\href{https://www.nytimes3xbfgragh.onion/2020/05/22/business/china-hong-kong-national-security.html}{unveiled
details of the new security measure} that would allow greater crackdowns
on political dissent in Hong Kong, becoming the most senior Trump
administration official to aggressively react against China's drastic
proposal.

A senior White House economic adviser, Kevin Hassett, who formerly
oversaw President Trump's Council of Economic Advisers, echoed Mr.
Pompeo's tough talk on Friday.

``We're absolutely not going to give China a pass,''
\href{https://twitter.com/jimsciutto/status/1263829250785923073}{Mr.
Hassett told CNN}, adding, ``It's a very difficult, scary move.''

Mr. Trump did not comment, however, leaving his intentions unclear. He
has shown limited interest in Hong Kong's anti-Beijing protest movement
and, despite his public anger at China for its management of the
coronavirus outbreak in Wuhan, has stopped short of condemning its
president, Xi Jinping, with whom he has spent years trying to strike a
huge trade deal.

The pointed statement from Mr. Pompeo, among the government's most
aggressive China critics, said that the proposal ``would be a death
knell for the high degree of autonomy Beijing promised for Hong Kong
under the Sino-British joint declaration'' and that it ``would
inevitably impact our assessment of `one country, two systems' and the
status of the territory.''

``The United States strongly urges Beijing to reconsider its disastrous
proposal,'' he added. ``We stand with the people of Hong Kong.''

State Department officials have discussed how China's proposed national
security law might affect a potential decision to revoke the
preferential trade and economic status that the United States has given
Hong Kong since 1997, when Britain handed control of the territory to
China. A 1992 law has allowed administrations, for purposes of commerce,
to treat Hong Kong under Chinese rule in the same way the American
government did when the territory was a British colony.

Under the Hong Kong Human Rights and Democracy Act, which passed last
fall with bipartisan support in Congress, the State Department is
required to deliver a report each year certifying the semiautonomous
status of Hong Kong --- and thus endorsing the continuation of the
preferential treatment. Mr. Trump reluctantly
\href{https://www.nytimes3xbfgragh.onion/2019/11/27/us/politics/trump-hong-kong.html}{signed
the bill into law in November}, at a time when he was trying not to
anger Mr. Xi because he was aiming to secure a trade deal. China's
Foreign Ministry called that measure ``a naked interference in China's
internal affairs.''

Mr. Pompeo said this month he would delay issuing a report to Congress
in order to measure actions that Beijing has taken on Hong Kong. That
was intended to signal to Chinese leaders to ease their hard-line
policies on Hong Kong, but it has not worked so far.

The State Department spokeswoman, Morgan Ortagus, said in a statement
Thursday that the department was delaying the report to ``allow us to
account for any additional actions'' that ``would further undermine Hong
Kong's high degree of autonomy.''

The message was clear: American officials are tracking the proposed law,
and American policy toward Hong Kong is in the balance.

On Friday, a spokesman for the Chinese Foreign Ministry, Zhao Lijian,
accused the United States of adopting a ``Cold War mentality'' and said
it should not interfere in China's internal affairs. ``No one can stop
China from growing stronger,'' he added,
\href{https://www.globaltimes.cn/content/1189178.shtml}{according to the
Chinese newspaper Global Times}.

In the Trump administration, China policy debates have generally been
divided between economic officials wary of conflict that could rattle
markets or disrupt trade talks, and national security officials
\href{https://www.nytimes3xbfgragh.onion/2020/05/01/us/politics/coronavirus-china-trump.html}{determined
to confront} Beijing's growing strategic power. That made the harsh
words of Mr. Hassett, a senior economic aide, all the more notable.

``China's move in Hong Kong is going to be very, very bad for the
Chinese economy and the Hong Kong economy,'' Mr. Hassett added in
remarks to reporters at the White House.

Concerns among Trump administration national security officials over
Hong Kong escalated last year, when pro-democracy protesters took to the
streets beginning in the summer and were
\href{https://www.nytimes3xbfgragh.onion/2019/11/13/world/asia/hong-kong-protests-students.html}{confronted
at every turn by the police}. In Beijing, hard-line Chinese officials
determined that the police were not using enough force to suppress the
protests.

``This latest move is yet another case of how the Chinese leadership's
preoccupation with domestic national security considerations override
its considerations of the international repercussions of such a move,''
said Dali L. Yang, a political scientist at the University of Chicago.

In considering whether to revoke the territory's preferential status,
Trump administration officials must weigh the damage it would do to both
mainland China and Hong Kong itself. China relies on Hong Kong, whose
currency is pegged to the dollar, as the first financial stop in trade
with the rest of the world. Chinese financial institutions and
state-owned companies hold vast assets there. And many Chinese companies
list on the Hong Kong Stock Exchange.

Some American officials have asked whether a status change would
ultimately harm Hong Kong, and the many corporations based there, while
failing to get China to back off.

The preferential trade status of Hong Kong became even more important
for American companies under the trade war begun by Mr. Trump in 2018.
Some companies that wanted to avoid tariffs imposed by the Trump
administration arranged for goods from China to be sold first to Hong
Kong trading companies before going to the United States, allowing them
to
\href{https://www.scmp.com/economy/china-economy/article/2182215/trumps-trade-war-tariffs-send-firms-hong-kong-little-known}{avoid
some of the burden of tariffs}.

The withdrawal of Hong Kong's special status would make all exports from
the territory subject to the tariffs Mr. Trump has slapped on Chinese
products, including goods that transit through the territory.
Investments routed through Hong Kong are also likely to be subject to
greater scrutiny for national security risks, said Eswar Prasad, a
professor of trade policy at Cornell University.

Even if the Trump administration takes no action, Congress is likely to:
A bipartisan Senate measure introduced on Thursday would penalize
Chinese officials, or entities like police units, involved in the
security measure, as well as banks that conduct ``significant
transactions'' with them.

In an interview, Senator Chris Van Hollen, Democrat of Maryland, said
that he and his co-sponsor, Senator Pat Toomey, Republican of
Pennsylvania, would ``urge the Senate to move very quickly,'' and
predicted the measure would win a veto-proof majority in the Senate and
House.

On Friday, the Commerce Department added 24 organizations, including
Qihoo 360 Technology Company, FiberHome Technologies Group and Beijing
Cloudmind Technology, to its ``entity list'' for activities that run
counter to America's national security interests, including supporting
the Chinese military. An additional nine entities were added for aiding
human rights violations in the Xinjiang region, home to members of the
Uighur minority who have
\href{https://www.nytimes3xbfgragh.onion/2020/05/09/us/politics/china-uighurs-arrest.html}{been
targeted by Chinese officials}. The designation will bar American
companies from selling them certain high-tech items without first
obtaining a special license.

Advertisement

\protect\hyperlink{after-bottom}{Continue reading the main story}

\hypertarget{site-index}{%
\subsection{Site Index}\label{site-index}}

\hypertarget{site-information-navigation}{%
\subsection{Site Information
Navigation}\label{site-information-navigation}}

\begin{itemize}
\tightlist
\item
  \href{https://help.nytimes3xbfgragh.onion/hc/en-us/articles/115014792127-Copyright-notice}{©~2020~The
  New York Times Company}
\end{itemize}

\begin{itemize}
\tightlist
\item
  \href{https://www.nytco.com/}{NYTCo}
\item
  \href{https://help.nytimes3xbfgragh.onion/hc/en-us/articles/115015385887-Contact-Us}{Contact
  Us}
\item
  \href{https://www.nytco.com/careers/}{Work with us}
\item
  \href{https://nytmediakit.com/}{Advertise}
\item
  \href{http://www.tbrandstudio.com/}{T Brand Studio}
\item
  \href{https://www.nytimes3xbfgragh.onion/privacy/cookie-policy\#how-do-i-manage-trackers}{Your
  Ad Choices}
\item
  \href{https://www.nytimes3xbfgragh.onion/privacy}{Privacy}
\item
  \href{https://help.nytimes3xbfgragh.onion/hc/en-us/articles/115014893428-Terms-of-service}{Terms
  of Service}
\item
  \href{https://help.nytimes3xbfgragh.onion/hc/en-us/articles/115014893968-Terms-of-sale}{Terms
  of Sale}
\item
  \href{https://spiderbites.nytimes3xbfgragh.onion}{Site Map}
\item
  \href{https://help.nytimes3xbfgragh.onion/hc/en-us}{Help}
\item
  \href{https://www.nytimes3xbfgragh.onion/subscription?campaignId=37WXW}{Subscriptions}
\end{itemize}
