Sections

SEARCH

\protect\hyperlink{site-content}{Skip to
content}\protect\hyperlink{site-index}{Skip to site index}

\href{https://www.nytimes3xbfgragh.onion/section/us}{U.S.}

\href{https://myaccount.nytimes3xbfgragh.onion/auth/login?response_type=cookie\&client_id=vi}{}

\href{https://www.nytimes3xbfgragh.onion/section/todayspaper}{Today's
Paper}

\href{/section/us}{U.S.}\textbar{}Coronavirus Cases Slow in U.S., but
the Big Picture Remains Tenuous

\url{https://nyti.ms/3dRDvrn}

\begin{itemize}
\item
\item
\item
\item
\item
\end{itemize}

\hypertarget{the-coronavirus-outbreak}{%
\subsubsection{\texorpdfstring{\href{https://www.nytimes3xbfgragh.onion/news-event/coronavirus?name=styln-coronavirus-national\&region=TOP_BANNER\&block=storyline_menu_recirc\&action=click\&pgtype=Article\&impression_id=817df6d0-f4b7-11ea-bfcd-b5364c666c0c\&variant=undefined}{The
Coronavirus
Outbreak}}{The Coronavirus Outbreak}}\label{the-coronavirus-outbreak}}

\begin{itemize}
\tightlist
\item
  live\href{https://www.nytimes3xbfgragh.onion/2020/09/11/world/covid-19-coronavirus.html?name=styln-coronavirus-national\&region=TOP_BANNER\&block=storyline_menu_recirc\&action=click\&pgtype=Article\&impression_id=817df6d1-f4b7-11ea-bfcd-b5364c666c0c\&variant=undefined}{Latest
  Updates}
\item
  \href{https://www.nytimes3xbfgragh.onion/interactive/2020/us/coronavirus-us-cases.html?name=styln-coronavirus-national\&region=TOP_BANNER\&block=storyline_menu_recirc\&action=click\&pgtype=Article\&impression_id=817df6d2-f4b7-11ea-bfcd-b5364c666c0c\&variant=undefined}{Maps
  and Cases}
\item
  \href{https://www.nytimes3xbfgragh.onion/interactive/2020/science/coronavirus-vaccine-tracker.html?name=styln-coronavirus-national\&region=TOP_BANNER\&block=storyline_menu_recirc\&action=click\&pgtype=Article\&impression_id=817df6d3-f4b7-11ea-bfcd-b5364c666c0c\&variant=undefined}{Vaccine
  Tracker}
\item
  \href{https://www.nytimes3xbfgragh.onion/2020/09/10/us/politics/fda-coronavirus-vaccine.html?name=styln-coronavirus-national\&region=TOP_BANNER\&block=storyline_menu_recirc\&action=click\&pgtype=Article\&impression_id=817e1de0-f4b7-11ea-bfcd-b5364c666c0c\&variant=undefined}{F.D.A.
  Regulators' Self-Defense}
\item
  \href{https://www.nytimes3xbfgragh.onion/2020/09/09/upshot/coronavirus-surprise-test-fees.html?name=styln-coronavirus-national\&region=TOP_BANNER\&block=storyline_menu_recirc\&action=click\&pgtype=Article\&impression_id=817e1de1-f4b7-11ea-bfcd-b5364c666c0c\&variant=undefined}{Surprise
  Test Fees}
\end{itemize}

Advertisement

\protect\hyperlink{after-top}{Continue reading the main story}

Supported by

\protect\hyperlink{after-sponsor}{Continue reading the main story}

\hypertarget{coronavirus-cases-slow-in-us-but-the-big-picture-remains-tenuous}{%
\section{Coronavirus Cases Slow in U.S., but the Big Picture Remains
Tenuous}\label{coronavirus-cases-slow-in-us-but-the-big-picture-remains-tenuous}}

Reports of new cases have declined nationally, and deaths have slowed.
But reopening plans leave unanswered questions.

\href{https://www.nytimes3xbfgragh.onion/by/julie-bosman}{\includegraphics{https://static01.graylady3jvrrxbe.onion/images/2018/11/09/multimedia/author-julie-bosman/author-julie-bosman-thumbLarge.png}}\href{https://www.nytimes3xbfgragh.onion/by/amy-harmon}{\includegraphics{https://static01.graylady3jvrrxbe.onion/images/2020/04/29/reader-center/author-amy-harmon/author-amy-harmon-thumbLarge-v2.png}}\href{https://www.nytimes3xbfgragh.onion/by/mitch-smith}{\includegraphics{https://static01.graylady3jvrrxbe.onion/images/2018/09/10/multimedia/author-mitch-smith/author-mitch-smith-thumbLarge.png}}

By \href{https://www.nytimes3xbfgragh.onion/by/julie-bosman}{Julie
Bosman}, \href{https://www.nytimes3xbfgragh.onion/by/amy-harmon}{Amy
Harmon} and
\href{https://www.nytimes3xbfgragh.onion/by/mitch-smith}{Mitch Smith}

\begin{itemize}
\item
  May 16, 2020
\item
  \begin{itemize}
  \item
  \item
  \item
  \item
  \item
  \end{itemize}
\end{itemize}

CHICAGO --- The number of new coronavirus cases confirmed in the United
States has steadily declined in recent days. In New York, the figure has
dropped over the past month. The numbers have also plunged in hard-hit
Massachusetts and Rhode Island, and some states, including Vermont,
Hawaii and Alaska, are reporting few new cases at all.

But that progress is tenuous and uncertain.

\hypertarget{where-the-number-of-cases-is-rising-and-falling}{%
\subsubsection{Where the number of cases is rising and
falling}\label{where-the-number-of-cases-is-rising-and-falling}}

Change in the average number of new cases each day,

from April 30 to May 14. Per 100,000 residents.

Getting worse

Getting better

+20

+3

--3

--20

No cases reported or

low population density

Wash.

Mont.

N.D.

Minn.

Maine

Vt.

Ore.

N.H.

Idaho

Wis.

S.D.

Mass.

N.Y.

Mich.

Conn.

Wyo.

Iowa

Pa.

N.J.

Neb.

Md.

Nev.

Ill.

Ind.

Ohio

Utah

W.Va.

Colo.

Va.

Ky.

Mo.

Kan.

Calif.

Tenn.

N.C.

Okla.

Ariz.

N.M.

Ark.

S.C.

Ala.

Ga.

La.

Miss.

Texas

Alaska

Fla.

Hawaii

Change in the average number of new cases each day, from April 30 to May
14. Per 100,000 residents.

Getting worse

Getting better

+20

+3

--3

--20

No cases reported

or low population

density

Change in the average number of

new cases each day, from April 30 to

May 14. Per 100,000 residents.

+20

+3

--3

--20

No cases reported

or low population

density

By Weiyi Cai and Lauren Leatherby·Source:
\href{https://www.nytimes3xbfgragh.onion/interactive/2020/us/coronavirus-us-cases.html}{New
York Times database} of reports from state and local health agencies.
Parts of counties with population densities less than 10 people per
square mile are not shaded.

The nation has reached a perilous moment in the course of the epidemic,
embracing signs of hope and beginning to reopen businesses and ease the
very measures that slowed the virus, despite the risk of a resurgence.
With more than two-thirds of states significantly relaxing restrictions
on how Americans can move about over the last few weeks, an uptick in
cases is widely predicted.

Months after the virus began spreading, only about 3 percent of the
population has been tested for it, leaving its true scale and path
unknown even as it continues to sicken and kill people at alarming
rates. More than 20,000 new cases are identified on most days. And
almost **** every day this past week, more than 1,000 Americans died
from the virus.

``We're seeing a decline; undoubtedly, that is something good to see,''
Jeffrey Shaman, an epidemiologist at Columbia University, said. ``But
what we are also seeing is a lot of places right on the edge of
controlling the disease.''

The slowing of new cases is a stark change from two weeks ago, when
coronavirus cases were
\href{https://www.nytimes3xbfgragh.onion/2020/05/05/us/coronavirus-deaths-cases-united-states.html}{stuck
on a stubborn plateau nationally} and case numbers were rising in many
states. As of Friday, new cases were decreasing in 19 states and
increasing in three, while staying mostly the same in the rest,
according to a database maintained by The New York Times.

New reported cases by day

in the rest of the U.S.

New reported cases by day

in New York City

20,000 cases

20,000 cases

10,000

10,000

7-day

average

0

0

April

March

May

April

March

May

New reported cases by day in New York City

20,000 cases

10,000

7-day

average

0

April

March

May

New reported cases by day

in the rest of the U.S.

20,000 cases

10,000

0

April

March

May

By Weiyi Cai and Lauren Leatherby·Source:
\href{https://www.nytimes3xbfgragh.onion/interactive/2020/us/coronavirus-us-cases.html}{New
York Times database} of reports from state and local health agencies.

Encouraging signs have emerged in some of the hardest-hit places.

In New Orleans,
\href{https://www.nytimes3xbfgragh.onion/2020/03/26/us/coronavirus-louisiana-new-orleans.html}{where
hundreds of new cases were being identified} each day in early April,
fewer than 50 have been announced daily in the last three weeks. In the
Detroit area,
\href{https://www.nytimes3xbfgragh.onion/2020/03/30/us/coronavirus-detroit.html}{which
saw exponential case growth} beginning in late March, numbers have
fallen sharply. And in Cass County, Ind., where a meatpacking outbreak
sickened at least 900 people, only a handful of cases have been reported
most days this past week.

Even as many large cities saw their cases drop, increasing infections
continue to be
\href{https://www.nytimes3xbfgragh.onion/interactive/2020/04/08/us/coronavirus-rural-america-cases.html}{reported
in parts of rural America}. Some communities that have been fighting to
get outbreaks under control finally appear to have succeeded, but have
little idea how long it will last.

In Sioux Falls, S.D., where
\href{https://www.nytimes3xbfgragh.onion/2020/04/15/us/coronavirus-south-dakota-meat-plant-refugees.html}{the
virus sickened more than 1,000 people} at a Smithfield pork processing
plant, the outbreak appears to be slowing, Mayor Paul TenHaken said.
More than 4,000 Smithfield employees, their family members and close
contacts, were recently tested.

Yet the mayor fears that his city's progress could be temporary. On
Monday, the plant will begin slaughtering hogs again. Hundreds of
employees will be back together at work.

``I'll be honest, it makes me nervous,'' Mr. TenHaken said. ``We've seen
how a zero-case facility can become a 1,000-case facility.''

\hypertarget{latest-updates-the-coronavirus-outbreak}{%
\section{\texorpdfstring{\href{https://www.nytimes3xbfgragh.onion/2020/09/11/world/covid-19-coronavirus.html?action=click\&pgtype=Article\&state=default\&region=MAIN_CONTENT_1\&context=storylines_live_updates}{Latest
Updates: The Coronavirus
Outbreak}}{Latest Updates: The Coronavirus Outbreak}}\label{latest-updates-the-coronavirus-outbreak}}

Updated 2020-09-12T04:56:54.924Z

\begin{itemize}
\tightlist
\item
  \href{https://www.nytimes3xbfgragh.onion/2020/09/11/world/covid-19-coronavirus.html?action=click\&pgtype=Article\&state=default\&region=MAIN_CONTENT_1\&context=storylines_live_updates\#link-dfb8a16}{Fauci
  cautions the virus could disrupt life in the U.S. until `maybe even
  towards the end of 2021.'}
\item
  \href{https://www.nytimes3xbfgragh.onion/2020/09/11/world/covid-19-coronavirus.html?action=click\&pgtype=Article\&state=default\&region=MAIN_CONTENT_1\&context=storylines_live_updates\#link-7104d154}{From
  Asia to Africa, China promotes its vaccine candidates to win friends.}
\item
  \href{https://www.nytimes3xbfgragh.onion/2020/09/11/world/covid-19-coronavirus.html?action=click\&pgtype=Article\&state=default\&region=MAIN_CONTENT_1\&context=storylines_live_updates\#link-393ad215}{The
  other way the virus will kill: hunger.}
\end{itemize}

\href{https://www.nytimes3xbfgragh.onion/2020/09/11/world/covid-19-coronavirus.html?action=click\&pgtype=Article\&state=default\&region=MAIN_CONTENT_1\&context=storylines_live_updates}{See
more updates}

More live coverage:
\href{https://www.nytimes3xbfgragh.onion/live/2020/09/11/business/stock-market-today-coronavirus?action=click\&pgtype=Article\&state=default\&region=MAIN_CONTENT_1\&context=storylines_live_updates}{Markets}

Epidemiologists pointed to one overarching reason for the decline in new
cases: the success of widespread social distancing.

\includegraphics{https://static01.graylady3jvrrxbe.onion/images/2020/05/15/us/00VIRUS-STATEOFTHESTATE-texas2/merlin_172186413_951e3317-e415-4ba6-bd39-f11b5fc33adf-articleLarge.jpg?quality=75\&auto=webp\&disable=upscale}

Americans began to change their behavior in March, and it has
undoubtedly helped control the spread of the coronavirus. Between
mid-March, when public officials began to close schools and some
workplaces, and late April, when the
\href{https://www.nytimes3xbfgragh.onion/interactive/2020/us/states-reopen-map-coronavirus.html}{restrictions
were lifted or eased} in many states, 43.8 percent of the nation's
residents stayed home, according to
\href{https://www.nytimes3xbfgragh.onion/interactive/2020/05/12/us/coronavirus-reopening-shutdown.html}{cellphone
data analyzed by The Times}.

The major clusters of cases that have arisen have been almost
exclusively in three settings without effective social distancing:
nursing homes, correctional facilities and food-processing plants.

But in settings where distancing took place, the results have been
overwhelming, researchers say. More than 70 percent of the U.S.
population \href{https://jbayham.github.io/maps/distancing/}{lives in
counties} where coronavirus cases were reduced as a result of less time
spent outside the home,
\href{https://www.medrxiv.org/content/10.1101/2020.04.20.20073098v1}{according
to one estimate} by a research team led by economists at Yale
University. Without government orders to stay at home, 10 million more
people in the United States would have been infected with the virus by
the end of April, suggested a paper
\href{https://www.healthaffairs.org/doi/full/10.1377/hlthaff.2020.00608}{published
this past week in the journal Health Affairs}.

``There's this disconnect of why it got better,'' said Mayor Thomas P.
McNamara of Rockford, Ill., who has repeatedly stressed to his
constituents that it is not yet time to relax the measures that
contributed: ``Social distancing, stay at home, wear your face
covering.''

Image

Diners ate outside Franks Restaurant in the French Quarter of New
Orleans on Friday. Reports of new cases have slowed
there.~Credit...Emily Kask for The New York Times

The challenge has been convincing impatient Americans to continue taking
precautions that will continue to slow the spread of the virus while a
cure or vaccine remains far out of reach.

``I just received an email from someone yesterday who said, `I don't
think people in our community are taking it seriously,''' said Kelly
Chandler, the public health division manager for Itasca County, Minn., a
lightly populated community with 42 cases of the coronavirus and six
deaths.

Influxes of new cases were already turning up in some places that had
seemed to tamp down earlier outbreaks.

In Arizona, which began reopening its economy without seeing a sustained
drop in cases, infection numbers have continued to rise. More than
13,100 cases had been identified as of Friday. In Alabama, case numbers
have grown since the state began to reopen its economy. And in
Minnesota, cases around St. Cloud and Minneapolis have surged over the
past two weeks, even as there were signs that the situation could be
stabilizing.

In Kankakee County, Ill., confirmed cases have climbed in recent days
because testing has been ramping up significantly, said John J. Bevis,
the administrator for the Kankakee County Health Department. He
predicted that cases would decline soon --- but also that the recovery
could be short-lived.

``Down the road, as things begin to reopen, there is the possibility of
an increase in numbers again,'' Mr. Bevis said in an email.

Along with cases, the number of deaths has slowed nationally.

Case and death reports vary greatly by day of the week, with spikes
around midweek and steep drops on weekends. But on eight of the past
nine days, there have been fewer deaths announced than there were seven
days prior, an indication that the virus's toll seems to be easing. More
than half of the 24 counties that have recorded the most coronavirus
deaths, including Oakland County, Mich., and Hartford County, Conn., are
\href{https://twitter.com/familyunequal/status/1261426618544742400?s=21}{seeing
sustained declines}.

Deaths are a lagging sign of the virus's progression because people who
die of Covid-19 were typically infected three weeks earlier. But because
death counts are not distorted by uneven testing practices, they are ``a
very clearly observed indicator,'' said Nicholas Reich, a
biostatistician at the University of Massachusetts, Amherst, who has
begun to synthesize the projections of deaths produced by several
modeling teams on a weekly basis. The ``ensemble'' model released on
Tuesday sees the number drifting down from about 10,000 this week to
about 7,000 in the first week of June.

Still, even with the slowing growth in new cases and deaths, the
cumulative death toll in the United States is projected to reach about
113,000 by June 6, according to Dr. Reich's
\href{https://twitter.com/reichlab/status/1260279329025556480}{latest
ensemble model}.

Image

Washington State has seen case numbers rise some in recent days after
weeks of sustained reductions.Credit...Ruth Fremson/The New York Times

The effects of relaxing of restrictions on how Americans move about
remain ahead. As more states lifted limits on businesses and movement,
about 25 million more people ventured outside their homes on an average
day last week than during the preceding six weeks, the analysis of
cellphone data found.

But the lag after states reopen, combined with insufficient testing, may
mask a rebound until it is underway for several weeks. The states that
have reopened have offered a mixed picture --- one more mysterious
element of this virus, which doctors and scientists have grappled to
understand as it has spread, swiftly and invisibly, through rural
communities, on public transit, and in nursing homes, prisons and
factories.

Georgia, which drew national attention
\href{https://www.nytimes3xbfgragh.onion/2020/04/24/us/coronavirus-georgia-oklahoma-alaska-reopen.html}{when
it eased its restrictions} late last month, has not seen much change in
its case numbers. Its curve has trended slightly downward this week.

Yet in Texas, officials reported a spike in coronavirus cases two weeks
after
\href{https://www.nytimes3xbfgragh.onion/2020/05/01/us/coronavirus-texas-reopening.html}{the
state began to reopen}.

``At this point, there is uncertainty,'' said Alessandro Vespignani,
director of the Network Science Institute at Northeastern University,
who has been modeling the path of the virus. ``Probably the next week
will be one of the crucial ones because if we see more decrease of cases
we are still on a `good' trajectory --- if not, it really might be more
problematic for the future.''

Julie Bosman reported from Chicago, Amy Harmon from New York, and Mitch
Smith from Overland Park, Kan.

Advertisement

\protect\hyperlink{after-bottom}{Continue reading the main story}

\hypertarget{site-index}{%
\subsection{Site Index}\label{site-index}}

\hypertarget{site-information-navigation}{%
\subsection{Site Information
Navigation}\label{site-information-navigation}}

\begin{itemize}
\tightlist
\item
  \href{https://help.nytimes3xbfgragh.onion/hc/en-us/articles/115014792127-Copyright-notice}{©~2020~The
  New York Times Company}
\end{itemize}

\begin{itemize}
\tightlist
\item
  \href{https://www.nytco.com/}{NYTCo}
\item
  \href{https://help.nytimes3xbfgragh.onion/hc/en-us/articles/115015385887-Contact-Us}{Contact
  Us}
\item
  \href{https://www.nytco.com/careers/}{Work with us}
\item
  \href{https://nytmediakit.com/}{Advertise}
\item
  \href{http://www.tbrandstudio.com/}{T Brand Studio}
\item
  \href{https://www.nytimes3xbfgragh.onion/privacy/cookie-policy\#how-do-i-manage-trackers}{Your
  Ad Choices}
\item
  \href{https://www.nytimes3xbfgragh.onion/privacy}{Privacy}
\item
  \href{https://help.nytimes3xbfgragh.onion/hc/en-us/articles/115014893428-Terms-of-service}{Terms
  of Service}
\item
  \href{https://help.nytimes3xbfgragh.onion/hc/en-us/articles/115014893968-Terms-of-sale}{Terms
  of Sale}
\item
  \href{https://spiderbites.nytimes3xbfgragh.onion}{Site Map}
\item
  \href{https://help.nytimes3xbfgragh.onion/hc/en-us}{Help}
\item
  \href{https://www.nytimes3xbfgragh.onion/subscription?campaignId=37WXW}{Subscriptions}
\end{itemize}
