Sections

SEARCH

\protect\hyperlink{site-content}{Skip to
content}\protect\hyperlink{site-index}{Skip to site index}

\href{https://www.nytimes3xbfgragh.onion/section/world/europe}{Europe}

\href{https://myaccount.nytimes3xbfgragh.onion/auth/login?response_type=cookie\&client_id=vi}{}

\href{https://www.nytimes3xbfgragh.onion/section/todayspaper}{Today's
Paper}

\href{/section/world/europe}{Europe}\textbar{}Rwandan Genocide Suspect
Arrested After 23 Years on the Run

\url{https://nyti.ms/2WEOr65}

\begin{itemize}
\item
\item
\item
\item
\item
\end{itemize}

Advertisement

\protect\hyperlink{after-top}{Continue reading the main story}

Supported by

\protect\hyperlink{after-sponsor}{Continue reading the main story}

\hypertarget{rwandan-genocide-suspect-arrested-after-23-years-on-the-run}{%
\section{Rwandan Genocide Suspect Arrested After 23 Years on the
Run}\label{rwandan-genocide-suspect-arrested-after-23-years-on-the-run}}

Félicien Kabuga, 84, had been on the run for 23 years, since he was
indicted by the International Criminal Tribunal for Rwanda on multiple
charges of genocide.

\includegraphics{https://static01.graylady3jvrrxbe.onion/images/2020/05/17/world/17rwanda/merlin_172549857_fadb7232-8752-4195-94dc-ff3ed72c8279-articleLarge.jpg?quality=75\&auto=webp\&disable=upscale}

By \href{https://www.nytimes3xbfgragh.onion/by/marlise-simons}{Marlise
Simons} and
\href{https://www.nytimes3xbfgragh.onion/by/norimitsu-onishi}{Norimitsu
Onishi}

\begin{itemize}
\item
  May 16, 2020
\item
  \begin{itemize}
  \item
  \item
  \item
  \item
  \item
  \end{itemize}
\end{itemize}

PARIS --- He was behind the radio station whose hate-filled invectives
turned Rwandan against Rwandan, neighbor against neighbor, even spouse
against spouse. He was the man, it was said, who imported the hundreds
of thousands of machetes that allowed countless ordinary people to act
upon that hatred in one of the last genocides of the past century.

One of the most-wanted fugitives of the 1994 Rwandan genocide, Félicien
Kabuga, was arrested Saturday morning in a rented home just outside
Paris, protected by his children, the French authorities said. The
capture of Mr. Kabuga, 84, who was living under a false identity, was
the culmination of a decades-long international hunt across many
countries on at least two continents.

His arrest --- considered the most important apprehension by an
international tribunal in the past decade --- could help bring
long-awaited justice for his actions more than a generation after the
killing of at least 800,000 and
\href{https://www.nytimes3xbfgragh.onion/2019/04/06/world/africa/rwanda-genocide-25-years.html}{perhaps
as many as one million ethnic Tutsis and moderate Hutus} in the small
central African nation.

His trial could also help unravel some of the enduring mysteries of the
killings, particularly how much planning went into the genocide, which
also led to a catastrophic war in the neighboring Democratic Republic of
Congo and continues to destabilize much of central Africa today.

Mr. Kabuga, one of Rwanda's richest men before the genocide, is accused
by the International Criminal Tribunal for Rwanda of being
\href{https://2009-2017.state.gov/j/gcj/wcrp/206}{the main financier and
logistical backer} of the political and militia groups that committed
the genocide. He had been on the run for 23 years since he was indicted
on multiple charges of genocide.

``It is historical on many levels,'' Rwandan's justice minister,
Johnston Busingye, said in a phone interview from the country's capital,
Kigali. ``You can run, but you cannot hide. It can't be forever.''

A tribunal official said on Saturday that Mr. Kabuga had been tracked
down in France after investigators followed communications among members
of his family who, the official said, had acted as his support network.

It was not known how and when Mr. Kabuga entered France, and how he had
managed to evade detection while living in Asnières-sur-Seine, a
well-off suburb just northwest of Paris.

He was arrested at his home around 7 a.m. after a long investigation by
French national police specializing in crimes against humanity, with
help from the federal police in Belgium and the Metropolitan Police in
London, according to France's justice ministry.

Mr. Kabuga was expected to be handed over to United Nations prosecutors,
with his trial expected to take place in the tribunal's successor court
in Arusha, Tanzania.

``Kabuga has always been seen by the victims and survivors as one of the
leading figures,'' Serge Brammertz, the chief prosecutor at the
tribunal, said by phone on Saturday from The Hague. ``For them, after
waiting so many years, his arrest is an important step toward justice.''

Mr. Kabuga's capture could be the most important arrest of a figure
wanted by an international tribunal since
\href{https://www.nytimes3xbfgragh.onion/2011/05/27/world/europe/27ratko-mladic.html}{the
2011 apprehension of Gen. Ratko Mladic}, the Serbian military leader who
was later convicted of having committed genocide during the Bosnian war
of the early 1990s, Mr. Brammertz said.

The arrest ended a lengthy and often-frustrating search for Mr. Kabuga
by international investigators across multiple countries.

Stephen Rapp, a former chief prosecutor at the United Nations Rwanda
tribunal, said that immediately after the genocide Mr. Kabuga fled to
Switzerland, where he unsuccessfully applied for asylum, and was then
seen in other European countries before settling in Kenya for several
years. Mr. Rapp said the fugitive had used assumed names and several
different passports.

In 2002, the United States government began circulating wanted posters
in Nairobi, the Kenyan capital, one of his known hide-outs. In an
attempt to use its own resources and official connections to catch him,
the United States had offered a reward of up to \$5 million for his
capture.

But with his huge bank account and high-level connections, Mr. Kabuga
had managed until Saturday to escape an arrest warrant issued by the
tribunal in 1997.

In the late 1990s, Mr. Kabuga
\href{https://www.nytimes3xbfgragh.onion/2002/06/13/world/face-of-rwa}{was
traced to a house} owned by Hosea Kiplagat, a nephew of Kenya's
president at the time,
\href{https://www.nytimes3xbfgragh.onion/2020/02/03/obituaries/daniel-arap-moi-dead.html}{Daniel
arap Moi}of Kenya, according to a report published in 2001 by the
International Crisis Group, a research organization. The study also
detailed how investigators for the International Criminal Tribunal
uncovered evidence that a Kenyan police officer might have tipped off
Mr. Kabuga in 1997 that an arrest was imminent.

The Kenyan government at the time disputed the allegations that it had
not been diligent in its search for Mr. Kabuga.

In 2001, the United Nations court froze bank accounts that Mr. Kabuga
held or had access to in Switzerland, France, Belgium and Germany.

Believed to have been one the most powerful men in Rwanda before the
genocide, Mr. Kabuga, an ethnic Hutu, made his fortune in trade. Through
the marriage of a daughter, he was linked to a former president, Juvénal
Habyarimana, a Hutu, who was killed after his plane was shot down by a
missile over the Rwandan capital in 1994.

Extremist Hutus accused Tutsis of carrying out the assassination,
eventually triggering 100 days of killings in which tens of thousands of
Rwandans, including civilians, militia and the police, participated.

The Rwandan government has tried thousands of people, and the United
Nations Criminal Tribunal for Rwanda has tried close to 80, among them
senior government figures. After Mr. Kabuga's capture, at least six
senior figures suspected of participating in or orchestrating the
genocide remain on an international most wanted list.

Mr. Kabuga was charged with using his fortune to fund and organize the
notorious Interahamwe militia, which carried out the brunt of the
slaughter, often carried out by hacking people to death.

He is accused of issuing them weapons, including several hundred
thousand machetes imported from China, which were shipped to his
companies, as well as providing them transport in his company's
vehicles.

The indictment against him also alleges that his radio station,
Radio-Television Mille Collines, incited the killings through broadcasts
that directed roaming gangs of killers to roadblocks and sites where
Tutsi could be located.

``His trial may help us understand to what extent the genocide was
planned,'' said Filip Reyntjens, a Belgian expert on the genocide.
``Kabuga is often mentioned as someone who was involved through the
funding of the extremist radio station. He's also mentioned in the
context of the purchase of machetes. All of that will need to be proven,
but a trial could unearth of a lot of things 26 years after the
genocide.''

Abdi Latif Dahir contributed reporting from Nairobi, Kenya.

Advertisement

\protect\hyperlink{after-bottom}{Continue reading the main story}

\hypertarget{site-index}{%
\subsection{Site Index}\label{site-index}}

\hypertarget{site-information-navigation}{%
\subsection{Site Information
Navigation}\label{site-information-navigation}}

\begin{itemize}
\tightlist
\item
  \href{https://help.nytimes3xbfgragh.onion/hc/en-us/articles/115014792127-Copyright-notice}{©~2020~The
  New York Times Company}
\end{itemize}

\begin{itemize}
\tightlist
\item
  \href{https://www.nytco.com/}{NYTCo}
\item
  \href{https://help.nytimes3xbfgragh.onion/hc/en-us/articles/115015385887-Contact-Us}{Contact
  Us}
\item
  \href{https://www.nytco.com/careers/}{Work with us}
\item
  \href{https://nytmediakit.com/}{Advertise}
\item
  \href{http://www.tbrandstudio.com/}{T Brand Studio}
\item
  \href{https://www.nytimes3xbfgragh.onion/privacy/cookie-policy\#how-do-i-manage-trackers}{Your
  Ad Choices}
\item
  \href{https://www.nytimes3xbfgragh.onion/privacy}{Privacy}
\item
  \href{https://help.nytimes3xbfgragh.onion/hc/en-us/articles/115014893428-Terms-of-service}{Terms
  of Service}
\item
  \href{https://help.nytimes3xbfgragh.onion/hc/en-us/articles/115014893968-Terms-of-sale}{Terms
  of Sale}
\item
  \href{https://spiderbites.nytimes3xbfgragh.onion}{Site Map}
\item
  \href{https://help.nytimes3xbfgragh.onion/hc/en-us}{Help}
\item
  \href{https://www.nytimes3xbfgragh.onion/subscription?campaignId=37WXW}{Subscriptions}
\end{itemize}
