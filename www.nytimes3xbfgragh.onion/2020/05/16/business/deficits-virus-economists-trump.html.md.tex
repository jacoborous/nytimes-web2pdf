Sections

SEARCH

\protect\hyperlink{site-content}{Skip to
content}\protect\hyperlink{site-index}{Skip to site index}

\href{https://www.nytimes3xbfgragh.onion/section/business}{Business}

\href{https://myaccount.nytimes3xbfgragh.onion/auth/login?response_type=cookie\&client_id=vi}{}

\href{https://www.nytimes3xbfgragh.onion/section/todayspaper}{Today's
Paper}

\href{/section/business}{Business}\textbar{}How Washington Learned to
Embrace the Budget Deficit

\url{https://nyti.ms/3cGJgrU}

\begin{itemize}
\item
\item
\item
\item
\item
\end{itemize}

Advertisement

\protect\hyperlink{after-top}{Continue reading the main story}

Supported by

\protect\hyperlink{after-sponsor}{Continue reading the main story}

\hypertarget{how-washington-learned-to-embrace-the-budget-deficit}{%
\section{How Washington Learned to Embrace the Budget
Deficit}\label{how-washington-learned-to-embrace-the-budget-deficit}}

Many economists who once warned of mounting debt are now urging
lawmakers to spend more to combat economic pain from coronavirus.

\includegraphics{https://static01.graylady3jvrrxbe.onion/images/2020/05/15/business/00dc-virus-deficit/merlin_172205091_d211d5b6-2646-470b-aaef-84a810e6a0be-articleLarge.jpg?quality=75\&auto=webp\&disable=upscale}

\href{https://www.nytimes3xbfgragh.onion/by/jim-tankersley}{\includegraphics{https://static01.graylady3jvrrxbe.onion/images/2018/10/19/multimedia/author-jim-tankersley/author-jim-tankersley-thumbLarge.png}}

By \href{https://www.nytimes3xbfgragh.onion/by/jim-tankersley}{Jim
Tankersley}

\begin{itemize}
\item
  May 16, 2020
\item
  \begin{itemize}
  \item
  \item
  \item
  \item
  \item
  \end{itemize}
\end{itemize}

WASHINGTON --- This month, the federal government said it would borrow a
\href{https://www.reuters.com/article/us-usa-treasuries-borrowing/us-treasury-blasts-records-with-3-trillion-borrowing-need-this-quarter-idUSKBN22G28G}{record-breaking
\$3 trillion} from April to June to help businesses and workers get
through the coronavirus-induced recession. In April alone, the United
States recorded a larger budget deficit in a single month than it did
for all of the 2017 fiscal year, a total of \$738 billion.

Running such a large deficit would have been politically untenable just
a year ago; since the end of World War II, economists have often warned
that doing so would risk runaway inflation and possibly unsustainable
tax hikes on future generations. But now, even some of the country's
most ardent deficit hawks have watched the debt pile up and said: More,
please.

The coronavirus pandemic has brought a new sort of deficit scolding to
Capitol Hill, with economists and lawmakers warning the United States is
not borrowing enough to carry the nation through a debilitating
recession that could turn into a second Great Depression.

A legion of economists, Federal Reserve officials and even some of the
most outspoken proponents of deficit reduction in recent years are now
urging Congress and President Trump to continue spending trillions of
dollars to prevent a long-term collapse in business activity and
prolonged joblessness.

Behind those calls is a confluence of events that have enhanced the
economic case for rising deficits --- a combination of rock-bottom
interest rates, falling consumer prices and a deep plunge in consumer
and business activity.

Many economists said in the past that large public deficits and debt
would bog down the economy, by pushing up borrowing costs for businesses
and sending consumer prices soaring. Now, the Federal Reserve has
\href{https://www.nytimes3xbfgragh.onion/2020/04/29/business/economy/fed-coronavirus-interest-rates.html}{made
clear}that low interest rates, which have been slashed to near zero, are
here to stay, making it cheaper for the United States to borrow money.
Inflation, which struggled to get out of the gate during an 11-year
expansion, seems confined to the woodshed.

In order for America to survive the recession and minimize the damage,
many economists are now urging lawmakers to spend more. They want
additional aid to small businesses, continued enhanced unemployment
benefits for workers and more assistance for state and local governments
that have seen a steep falloff in tax revenue and have already laid off
1 million workers. Such spending, economists argue, would hasten a
rebound in economic growth and help save businesses that might otherwise
fail, generating a return to the economy that exceeds the relatively low
future interest costs incurred by prolific borrowing.

``I think we're still in the early innings of dealing with this crisis,
and we're probably in the early innings of throwing out trillions of
dollars to help us get by,'' said Kenneth Rogoff, a Harvard University
economist whose work on government debt and economic growth was
frequently cited by lawmakers pushing rapid deficit reduction under
President Barack Obama.

``Any sensible policy is going to have us racking up the deficit for a
long time, if you can,'' Mr. Rogoff said. ``If we go up another \$10
trillion, I wouldn't even blink at that now.''

Deficit critics still exist, at least in sound bites. Republican leaders
in the Senate have cited debt concerns as a reason to move slowly on a
new package of economic assistance amid the pandemic. Democratic leaders
in the House crafted and passed
\href{https://www.nytimes3xbfgragh.onion/2020/05/15/us/politics/house-simulus-vote.html}{a
\$3 trillion opening bid} for a new rescue package this week, but they
pared it back and dropped some members' top priorities from the bill out
of deficit concerns.

Mr. Trump's advisers and top Republican leaders, citing the enormous
sums of money already out the door, have said they would prefer to wait
and see whether the existing support provided by Congress will suffice
now that states are beginning to lift economic restrictions that were
imposed to slow the spread of the virus.

``I don't believe we can spend ourselves into prosperity,'' the head of
the National Economic Council, Larry Kudlow, told reporters on Friday.

Even as Republicans point to the deficit in resisting more support for
workers and businesses, they continue to push for tax cuts, which would
also grow the deficit but represents spending that they argue would be
more effective for the economic recovery.

``There's a huge fiscal problem growing,'' said Senator Rob Portman,
Republican of Ohio. ``We don't know what the impact of that is
economically, but we know that it's bad for our future economy and for
future generations. So we've got to take that into account. But we also
know that we have to provide a rescue package.''

There is little argument among either conservative or liberal economists
that the deficit needs to grow, as tax revenues fall and spending needs
rise amid a pandemic that has shuttered business activity and already
thrown at least 20 million people out of work.

Federal deficits typically grow during recessions and many economists
note that the deficit was going to rise this year whether lawmakers took
action or stood pat and allowed the economic damage to mount, forcing
more workers to utilize government benefits like food stamps and
unemployment.

``The deficit will move on its own,'' said Stephanie Kelton, an
economist at Stony Brook University who has advised Senator Bernie
Sanders of Vermont and is a leading champion of the theory that the
federal government's spending levels should be limited not by tax
collections or debt levels, but by how much the economy can actually
produce.

``We can move it proactively, and we can direct that deficit spending in
ways that are strategic and thoughtful,'' Ms. Kelton said, ``or we can
not do that, and it can move the ugly way.''

Lawmakers' decision to dive quickly into additional deficit spending has
been cheered by those who previously preached fiscal restraint,
including the Federal Reserve chair, Jerome H. Powell, who urged
Congress and Mr. Trump to ``go big'' on fiscal support for the economy.

\href{https://www.federalreserve.gov/newsevents/speech/powell20200513a.htm}{Mr.
Powell, who had been a longtime fiscal hawk, repeated his call} this
week, saying in a speech that ``additional fiscal support could be
costly, but worth it if it helps avoid long-term economic damage and
leaves us with a stronger recovery.'' He and the Fed have lent
considerable support to that effort, by promising to keep interest rates
near zero for as long as the economy remains in crisis and buying vast
sums of the Treasury bonds that support government borrowing.

\href{https://news.gallup.com/poll/147626/federal-budget-deficit.aspx}{Polls
show} Americans
\href{https://www.people-press.org/2019/12/17/views-of-the-major-problems-facing-the-country/}{worry
about the nation's deficit} and debt, but that those worries have
declined in recent years. Many economists' worries have declined, too,
in an era of persistently low interest rates and inflation that has
remained lower than the Fed's target rate of 2 percent, and receding
fears of the government ``crowding out'' private borrowers --- which is
to say, government borrowing pushing up interest rates to such a degree
that private companies find it harder to get access to capital.

In his
\href{https://www.piie.com/commentary/speeches-papers/public-debt-and-low-interest-rates}{presidential
address to the American Economic Association} in 2019, the economist
Olivier Blanchard made the case that policymakers in such an environment
should be much more willing to take on additional debt. Many other
leading economists agree.

``Interest rates are lower than they've ever been when we've done fiscal
stimulus,'' said Jason Furman, a former top economist under Mr. Obama
who is now a professor at Harvard University's Kennedy School of
Government. ``Inflation is lower than it's ever been when we've done
fiscal stimulus. There's not a business in the country that is
constrained from borrowing by the general level of interest rates.''

Yet some economists caution that, as deficits rise quickly, lawmakers
need to make sure they target the dollars effectively.

Michael J. Boskin, a Stanford University economist,
\href{https://www.nber.org/papers/w26727}{warned in a paper posted in
February} that rising debt as a share of nation's economy risks higher
taxes, lower future incomes and a reduced ability for children to climb
past their parents on the economic ladder, because large debt loads
reduce savings and crowd out private-sector investment. Mr. Boskin, in
an interview, said that he supported many of the government's
deficit-financed efforts in this crisis thus far, but that lawmakers
should target future spending on getting people back to work and helping
businesses reopen, in order to best help the economy recover.

``We owe that to ourselves,'' Mr. Boskin said, ``but especially to
future generations, who at some point are going to be paying for this.''

Other economists who have long championed deficit reduction have, in
this moment of crisis, called for higher and effectively targeted
spending. They include R. Glenn Hubbard, a Columbia University economist
who was a top adviser to President George W. Bush, and Maya MacGuineas,
the president of the Committee for a Responsible Federal Budget, who has
spent years advocating deficit reduction.

``There are times you should be borrowing and times you shouldn't be
borrowing,'' Ms. MacGuineas said. ``This is exactly the moment that we
should be borrowing.''

Republicans have grown more tolerant of deficits under Mr. Trump, who
famously said as a presidential candidate that he would eliminate the
national debt within eight years, but has instead swelled borrowing. Mr.
Trump's sweeping package of tax cuts in 2017
\href{https://www.nytimes3xbfgragh.onion/2019/01/11/business/trump-tax-cuts-revenue.html}{did
not pay for itself} as promised, and he has signed bipartisan agreements
to boost federal spending. That helped to push the
\href{https://www.nytimes3xbfgragh.onion/2020/01/13/business/budget-deficit-1-trillion-trump.html}{deficit
above \$1 trillion} in 2019, well before the current health emergency.

The crisis sharply accelerated the deficit. It will hit \$3.7 trillion
for the fiscal year, the
\href{https://www.cbo.gov/publication/56335}{Congressional Budget Office
projects} and, by the end of September, the budget office says, the
amount of debt held by the public will be larger than a full year of
economic output in the United States.

Fiscal hawks had warned that growing deficits under Mr. Trump, which
came despite an unemployment rate that fell to 50-year lows, could
hamstring the federal response to an economic crisis. Now that such a
crisis has arrived, and deficit fears have begun to surface in Congress,
many of those hawks say they feel vindicated.

Emily Cochrane, Neil Irwin and Jeanna Smialek contributed reporting.

Advertisement

\protect\hyperlink{after-bottom}{Continue reading the main story}

\hypertarget{site-index}{%
\subsection{Site Index}\label{site-index}}

\hypertarget{site-information-navigation}{%
\subsection{Site Information
Navigation}\label{site-information-navigation}}

\begin{itemize}
\tightlist
\item
  \href{https://help.nytimes3xbfgragh.onion/hc/en-us/articles/115014792127-Copyright-notice}{©~2020~The
  New York Times Company}
\end{itemize}

\begin{itemize}
\tightlist
\item
  \href{https://www.nytco.com/}{NYTCo}
\item
  \href{https://help.nytimes3xbfgragh.onion/hc/en-us/articles/115015385887-Contact-Us}{Contact
  Us}
\item
  \href{https://www.nytco.com/careers/}{Work with us}
\item
  \href{https://nytmediakit.com/}{Advertise}
\item
  \href{http://www.tbrandstudio.com/}{T Brand Studio}
\item
  \href{https://www.nytimes3xbfgragh.onion/privacy/cookie-policy\#how-do-i-manage-trackers}{Your
  Ad Choices}
\item
  \href{https://www.nytimes3xbfgragh.onion/privacy}{Privacy}
\item
  \href{https://help.nytimes3xbfgragh.onion/hc/en-us/articles/115014893428-Terms-of-service}{Terms
  of Service}
\item
  \href{https://help.nytimes3xbfgragh.onion/hc/en-us/articles/115014893968-Terms-of-sale}{Terms
  of Sale}
\item
  \href{https://spiderbites.nytimes3xbfgragh.onion}{Site Map}
\item
  \href{https://help.nytimes3xbfgragh.onion/hc/en-us}{Help}
\item
  \href{https://www.nytimes3xbfgragh.onion/subscription?campaignId=37WXW}{Subscriptions}
\end{itemize}
