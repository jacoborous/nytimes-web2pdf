Sections

SEARCH

\protect\hyperlink{site-content}{Skip to
content}\protect\hyperlink{site-index}{Skip to site index}

\href{https://www.nytimes3xbfgragh.onion/section/health}{Health}

\href{https://myaccount.nytimes3xbfgragh.onion/auth/login?response_type=cookie\&client_id=vi}{}

\href{https://www.nytimes3xbfgragh.onion/section/todayspaper}{Today's
Paper}

\href{/section/health}{Health}\textbar{}You're Getting Used to Masks.
Will You Wear a Face Shield?

\url{https://nyti.ms/3gcXXoJ}

\begin{itemize}
\item
\item
\item
\item
\item
\end{itemize}

\href{https://www.nytimes3xbfgragh.onion/spotlight/at-home?action=click\&pgtype=Article\&state=default\&region=TOP_BANNER\&context=at_home_menu}{At
Home}

\begin{itemize}
\tightlist
\item
  \href{https://www.nytimes3xbfgragh.onion/2020/09/07/travel/route-66.html?action=click\&pgtype=Article\&state=default\&region=TOP_BANNER\&context=at_home_menu}{Cruise
  Along: Route 66}
\item
  \href{https://www.nytimes3xbfgragh.onion/2020/09/04/dining/sheet-pan-chicken.html?action=click\&pgtype=Article\&state=default\&region=TOP_BANNER\&context=at_home_menu}{Roast:
  Chicken With Plums}
\item
  \href{https://www.nytimes3xbfgragh.onion/2020/09/04/arts/television/dark-shadows-stream.html?action=click\&pgtype=Article\&state=default\&region=TOP_BANNER\&context=at_home_menu}{Watch:
  Dark Shadows}
\item
  \href{https://www.nytimes3xbfgragh.onion/interactive/2020/at-home/even-more-reporters-editors-diaries-lists-recommendations.html?action=click\&pgtype=Article\&state=default\&region=TOP_BANNER\&context=at_home_menu}{Explore:
  Reporters' Google Docs}
\end{itemize}

Advertisement

\protect\hyperlink{after-top}{Continue reading the main story}

Supported by

\protect\hyperlink{after-sponsor}{Continue reading the main story}

\hypertarget{youre-getting-used-to-masks-will-you-wear-a-face-shield}{%
\section{You're Getting Used to Masks. Will You Wear a Face
Shield?}\label{youre-getting-used-to-masks-will-you-wear-a-face-shield}}

The clear plastic guards may be easier to wear, disinfect and reuse than
cloth or surgical face coverings, although they don't entirely replace
the need for masks.

\includegraphics{https://static01.graylady3jvrrxbe.onion/images/2020/05/26/science/25SCI-FACESHIELDS/merlin_171862293_9108b1a5-2713-4f9d-aca0-1f24a1aa95aa-articleLarge.jpg?quality=75\&auto=webp\&disable=upscale}

\href{https://www.nytimes3xbfgragh.onion/by/knvul-sheikh}{\includegraphics{https://static01.graylady3jvrrxbe.onion/images/2020/01/03/reader-center/author-knvul-sheikh/author-knvul-sheikh-thumbLarge.png}}

By \href{https://www.nytimes3xbfgragh.onion/by/knvul-sheikh}{Knvul
Sheikh}

\begin{itemize}
\item
  Published May 24, 2020Updated Sept. 1, 2020
\item
  \begin{itemize}
  \item
  \item
  \item
  \item
  \item
  \end{itemize}
\end{itemize}

The debate over whether
\href{https://www.nytimes3xbfgragh.onion/interactive/2020/health/coronavirus-best-face-masks.html}{Americans
should wear face masks to control coronavirus transmission} has been
settled. Governments and businesses now require or at least recommend
them in many public settings. But as parts of the country reopen, some
doctors want you to consider another layer of personal protective
equipment in your daily life: clear plastic
\href{https://www.nytimes3xbfgragh.onion/2020/09/01/well/live/face-shields-masks-valves-vents.html}{face
shields}.

``I wear a face shield every time I enter a store or other building,''
said Dr. Eli Perencevich. ``Sometimes I also wear a cloth
\href{https://www.nytimes3xbfgragh.onion/2020/06/02/health/coronavirus-face-masks-surveys.html}{mask}
if required by the store's policy.''

Dr. Perencevich is an infectious disease physician at the University of
Iowa and the Iowa City Veterans Affairs Health Care System. In an
opinion article published last month in
\href{https://jamanetwork.com/journals/jama/fullarticle/2765525}{JAMA},
he and two colleagues argued that simple, clear plastic face shields
could help reduce the transmission of infections when added to public
health measures like increased testing, contact tracing, social
distancing and hand hygiene.

\begin{quote}
Face shield worn during a one-hour walk in 13mph sustained winds with
35mph gusts. No problem - actually no watery eyes, which was nice. Ball
cap did blow off once
\href{https://twitter.com/hashtag/faceshields?src=hash\&ref_src=twsrc\%5Etfw}{\#faceshields}
\href{https://t.co/zScHp0Kvom}{pic.twitter.com/zScHp0Kvom}

--- 𝙀𝙡𝙞 𝙋𝙚𝙧𝙚𝙣𝙘𝙚𝙫𝙞𝙘𝙝 🤚 🧼😷 (@eliowa)
\href{https://twitter.com/eliowa/status/1251907985527517186?ref_src=twsrc\%5Etfw}{April
19, 2020}
\end{quote}

The idea is not just a thought experiment. In Singapore, preschool
students and their teachers will receive
\href{https://www.channelnewsasia.com/news/singapore/covid-19-face-shield-hand-sanitiser-students-school-reopen-12759972}{face
shields}when they return to school next month. Local health experts
recommended
\href{https://wskg.org/news/playbook-for-reopening-pa-schools-face-shields-staggered-schedules-temp-checks/}{teachers
in Philadelphia wear shields} when schools reopen, and a teachers union
in Palo Alto, Calif.
\href{https://paloaltoonline.com/news/2020/05/22/face-shields-and-testing-teachers-union-asks-for-protection-flexibility-in-working-conditions-this-fall}{requested
them as well}.

But it can be difficult to imagine Americans being willing to put on
another form of protective equipment.
\href{https://www.nytimes3xbfgragh.onion/video/us/politics/100000007124695/trump-coronavirus-mask-factory.html}{President
Trump} and
\href{https://www.nytimes3xbfgragh.onion/2020/04/28/us/politics/coronavirus-pence-mask.html}{Vice
President Mike Pence} have shirked wearing masks in settings that would
seem to call for them, and simple
\href{https://www.nytimes3xbfgragh.onion/2020/05/15/us/coronavirus-masks-violence.html}{face
covering requirements have provoked conflicts} at grocery stores and
restaurants.

Face shields have long been required equipment for many procedures in
hospitals. Doctors and nurses wear them when intubating Covid-19
patients and during surgeries that may cause blood and bone fragments to
fly out.

As debate arose over whether tiny coronavirus droplets could float on
air currents, protecting the eyes and the entire face became a bigger
issue in health care settings, said Dr. Sherry Yu, a dermatology
resident affiliated with Brigham and Women's Hospital in Boston. People
needed them to do nasal swabs for the coronavirus test or for triage in
the emergency room. As shortages loomed, Dr. Yu was among the
\href{https://www.nytimes3xbfgragh.onion/2020/03/30/health/coronavirus-innovators.html}{many
people and groups around the country who began fabricating face shields}
for front-line health care workers.

``The nice thing about face shields is that they can be resterilized and
cleaned by the user, so they're reusable indefinitely until some
component breaks or cracks,'' Dr. Yu said. A simple alcohol wipe or
rinse with soap and hot water is all it takes for the shields to be
contaminant-free again.

Surgical
\href{https://www.nytimes3xbfgragh.onion/2020/06/12/health/coronavirus-cdc-masks-gatherings.html}{masks}
and N95s, on the other hand, are meant to be disposed after each use,
although some studies have shown
\href{https://www.nytimes3xbfgragh.onion/article/face-shield-mask-california-coronavirus.html}{masks}
can be reused
\href{https://www.nytimes3xbfgragh.onion/2020/04/16/health/n95-masks-decontaminated-coronavirus.html}{two
or three times} after being sterilized before they lose integrity.

Dr. Perencevich believes that face shields should be the preferred
personal protective equipment of everyone for the same reason health
care workers use them. They protect the entire face, including the eyes,
and prevent people from touching their faces or inadvertently exposing
themselves to the coronavirus.

Face shields may be easier to wear than masks, he said, comparing them
with
\href{https://twitter.com/eliowa/status/1261299983560716288}{wearing
glasses or a hat}. They wrap around a small portion of a person's
forehead rather than covering more than half their face with material
that can create the urge to itch.

Many people also wear masks incorrectly, letting them dangle off the
tips of their noses, or concealing just their mouths. People also tend
to readjust face masks frequently, or remove them to communicate with
others, which increases their risk of being exposed or infecting others,
he said. And while cloth masks can prevent people from spreading germs
to others, they don't usually protect the wearer from infection.

Face shields can also aid people who depend on lip-reading, Dr.
Perencevich said. They may be slightly dorky-looking, but the shields
allow facial expressions and lip movements to remain visible, while
serving as an obvious reminder to maintain social distancing.

\textbf{\emph{{[}}\href{http://on.fb.me/1paTQ1h}{\emph{Like the Science
Times page on Facebook.}}} ****** \emph{\textbar{} Sign up for the}
\textbf{\href{http://nyti.ms/1MbHaRU}{\emph{Science Times
newsletter.}}\emph{{]}}}

Still, he and other experts acknowledge that face shields have their
limits.

Just like masks, they must be removed when eating in cafeterias or
restaurants. And studies on how effectively they can reduce a person's
viral exposure are scarce.

\includegraphics{https://static01.graylady3jvrrxbe.onion/images/2020/05/23/health/24virus-faceshields-video-image/24virus-faceshields-video-image-superJumbo.jpg}

One \href{https://www.ncbi.nlm.nih.gov/pubmed/24467190}{cough simulation
study in 2014} suggested that a shield could reduce a user's viral
exposure by 96 percent when worn within 18 inches of someone who was
coughing. But most people in the general public are much farther away
from others they are interacting with, said William Lindsley, a
bioengineer at the National Institute for Occupational Safety and Health
who led the study. Large droplets that may contain virus will fall to
the ground quickly, reducing the need for a face shield worn when
standing farther away.

Even in close range, there can be scenarios where face shields are not
as effective as masks like N95s that create a seal around one's face.
``If you're facing sideways, or I'm behind you, maybe you're sitting at
a desk and I'm standing, there's other scenarios you can imagine where
droplets can come around a face shield,'' Dr. Lindsley said.

There is also no research on how well one person's face shield protects
other people from viral transmission, the concept called source control
that is a primary benefit of surgical and cloth masks.

One of the main reasons the Centers for Disease Control and Prevention
\href{https://www.nytimes3xbfgragh.onion/2020/03/27/health/us-coronavirus-face-masks.html}{changed
their recommendations to suggest everyone wear a face covering} in
public was to protect others in case they were among the asymptomatic or
pre-symptomatic group of people infected with the virus.

``I'm a huge fan of face shields,'' said Saskia Popescu, a senior
infection-prevention specialist at George Mason University in Fairfax,
Va. ``But I don't think we can swap them out for face masks just yet.''

Dr. Perencevich and his colleagues expect that more research will show
shields to be superior to cloth masks, not only because shields provide
full face protection but as they are nearly impossible to wear
incorrectly.

``Remember, effectiveness depends not only on the inherent properties of
the facial covering but also how well the facial covering is worn,'' he
said.

And he and his co-authors like to imagine that people who are reluctant
to wear masks will find face shields more comfortable: Once a person
tries one on, they say, the wearer realizes its many benefits.

Advertisement

\protect\hyperlink{after-bottom}{Continue reading the main story}

\hypertarget{site-index}{%
\subsection{Site Index}\label{site-index}}

\hypertarget{site-information-navigation}{%
\subsection{Site Information
Navigation}\label{site-information-navigation}}

\begin{itemize}
\tightlist
\item
  \href{https://help.nytimes3xbfgragh.onion/hc/en-us/articles/115014792127-Copyright-notice}{©~2020~The
  New York Times Company}
\end{itemize}

\begin{itemize}
\tightlist
\item
  \href{https://www.nytco.com/}{NYTCo}
\item
  \href{https://help.nytimes3xbfgragh.onion/hc/en-us/articles/115015385887-Contact-Us}{Contact
  Us}
\item
  \href{https://www.nytco.com/careers/}{Work with us}
\item
  \href{https://nytmediakit.com/}{Advertise}
\item
  \href{http://www.tbrandstudio.com/}{T Brand Studio}
\item
  \href{https://www.nytimes3xbfgragh.onion/privacy/cookie-policy\#how-do-i-manage-trackers}{Your
  Ad Choices}
\item
  \href{https://www.nytimes3xbfgragh.onion/privacy}{Privacy}
\item
  \href{https://help.nytimes3xbfgragh.onion/hc/en-us/articles/115014893428-Terms-of-service}{Terms
  of Service}
\item
  \href{https://help.nytimes3xbfgragh.onion/hc/en-us/articles/115014893968-Terms-of-sale}{Terms
  of Sale}
\item
  \href{https://spiderbites.nytimes3xbfgragh.onion}{Site Map}
\item
  \href{https://help.nytimes3xbfgragh.onion/hc/en-us}{Help}
\item
  \href{https://www.nytimes3xbfgragh.onion/subscription?campaignId=37WXW}{Subscriptions}
\end{itemize}
