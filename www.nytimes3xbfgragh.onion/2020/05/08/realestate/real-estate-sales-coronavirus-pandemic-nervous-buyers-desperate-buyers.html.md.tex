Sections

SEARCH

\protect\hyperlink{site-content}{Skip to
content}\protect\hyperlink{site-index}{Skip to site index}

\href{https://www.nytimes3xbfgragh.onion/section/realestate}{Real
Estate}

\href{https://myaccount.nytimes3xbfgragh.onion/auth/login?response_type=cookie\&client_id=vi}{}

\href{https://www.nytimes3xbfgragh.onion/section/todayspaper}{Today's
Paper}

\href{/section/realestate}{Real Estate}\textbar{}Desperate Sellers,
Nervous Buyers: Real Estate Sales in a Pandemic

\url{https://nyti.ms/2zimjwn}

\begin{itemize}
\item
\item
\item
\item
\item
\item
\end{itemize}

\href{https://www.nytimes3xbfgragh.onion/spotlight/at-home?action=click\&pgtype=Article\&state=default\&region=TOP_BANNER\&context=at_home_menu}{At
Home}

\begin{itemize}
\tightlist
\item
  \href{https://www.nytimes3xbfgragh.onion/2020/09/07/travel/route-66.html?action=click\&pgtype=Article\&state=default\&region=TOP_BANNER\&context=at_home_menu}{Cruise
  Along: Route 66}
\item
  \href{https://www.nytimes3xbfgragh.onion/2020/09/04/dining/sheet-pan-chicken.html?action=click\&pgtype=Article\&state=default\&region=TOP_BANNER\&context=at_home_menu}{Roast:
  Chicken With Plums}
\item
  \href{https://www.nytimes3xbfgragh.onion/2020/09/04/arts/television/dark-shadows-stream.html?action=click\&pgtype=Article\&state=default\&region=TOP_BANNER\&context=at_home_menu}{Watch:
  Dark Shadows}
\item
  \href{https://www.nytimes3xbfgragh.onion/interactive/2020/at-home/even-more-reporters-editors-diaries-lists-recommendations.html?action=click\&pgtype=Article\&state=default\&region=TOP_BANNER\&context=at_home_menu}{Explore:
  Reporters' Google Docs}
\end{itemize}

Advertisement

\protect\hyperlink{after-top}{Continue reading the main story}

Supported by

\protect\hyperlink{after-sponsor}{Continue reading the main story}

\hypertarget{desperate-sellers-nervous-buyers-real-estate-sales-in-a-pandemic}{%
\section{Desperate Sellers, Nervous Buyers: Real Estate Sales in a
Pandemic}\label{desperate-sellers-nervous-buyers-real-estate-sales-in-a-pandemic}}

The sales market in New York City has slowed to a crawl, but bold buyers
are forging ahead, sometimes buying sight unseen.

\includegraphics{https://static01.graylady3jvrrxbe.onion/images/2020/05/10/realestate/08virus-selling/oakImage-1588788278106-articleLarge.png?quality=75\&auto=webp\&disable=upscale}

\href{https://www.nytimes3xbfgragh.onion/by/stefanos-chen}{\includegraphics{https://static01.graylady3jvrrxbe.onion/images/2018/06/13/multimedia/author-stefanos-chen/author-stefanos-chen-thumbLarge-v2.png}}

By \href{https://www.nytimes3xbfgragh.onion/by/stefanos-chen}{Stefanos
Chen}

\begin{itemize}
\item
  May 8, 2020
\item
  \begin{itemize}
  \item
  \item
  \item
  \item
  \item
  \item
  \end{itemize}
\end{itemize}

Late last month, about 60 agents from some of New York's top brokerages
gathered in a virtual conference room for what was billed as the first
open house of its kind for real estate agents. One by one, presenters
shared pictures and videos of eight listings in Chelsea, with varying
degrees of candor.

``It does look onto a brick wall,'' said one agent.

``We just reduced the price,'' said another.

The gallery of muted agents looked on. One chewed a sandwich and yelled
at someone offscreen; another sat stone-faced in front of a virtual
jungle background. One wore a suit and tie, while another splayed out on
a couch in sweats.

``I feel like I'm inside an acid trip,'' an agent wrote in a private
text.

Six weeks after New York State issued its stay-at-home order to combat
the coronavirus, agents, consumers and developers are finding their way
through an
\href{https://www.nytimes3xbfgragh.onion/2020/06/24/realestate/phase-2-showings.html}{unrecognizable
home-buying market}, devising new and unfamiliar methods to push deals
along against long odds. Some are proving more successful than others.

It was already going to be a challenging spring in Manhattan, where
prices are down about 20 percent from the peak in 2016
\href{https://www.nytimes3xbfgragh.onion/2020/01/10/realestate/new-york-decade-real-estate.html}{amid
a glut of luxury condos}. But as sellers pitch million-dollar apartments
over FaceTime and buyers grapple with purchasing a home they've never
set foot in, sales and listings are evaporating during what is supposed
to be the peak of spring buying season.

From March 22, when the stay-at-home order took effect, to April 29,
there were 643 contracts signed in Manhattan, fewer than half signed
during the same period last year, according to GS Data Services, a real
estate data firm. The median sale price of \$1.025 million marked a 6
percent drop from the same time last spring. In Brooklyn, where the
median sale price was \$900,000 from March 22 to April 29, signings were
down 65 percent from the same period last year.

Now agents are bracing for deeper cuts. There were just 59 new listings
posted in Manhattan in the week ending April 26, including resales and
new development, a stunning 88 percent decline from the 519 listings
added during the same week last year, according to UrbanDigs, a real
estate data site.

``The drop in deal volume is staggering and unprecedented for the
industry,'' said Garrett Derderian, the chief executive of GS Data
Services, adding that most of the recent buyers still had a chance to
visit in person, before the lockdown.

\includegraphics{https://static01.graylady3jvrrxbe.onion/images/2020/05/08/realestate/08virus-selling/08virus-selling-articleLarge.jpg?quality=75\&auto=webp\&disable=upscale}

``I don't see deals going fully virtual,'' he said, adding that sales
will decline further because so much of the buying process is normally
tactile and emotional.

The organizer of the virtual open house, Gerald Germany, an agent with
Douglas Elliman, said his remote event was the best way for agents to
gain exposure for their listings while in-person showings remain
prohibited. So far, though, he has had just one signed contract since
the lockdown began: a one-bedroom apartment listed for \$995,000, which
the buyer visited just before the restrictions began.

``We're going to have to wait until these people can get in and see the
units,'' he said. In early May, Gov. Andrew M. Cuomo announced that the
second phase of reopening the city would include real estate services,
though the timeline is unclear.

Still, as frantic sellers hunt for buyers, deals are still happening ---
from first-timers hoping to take advantage of near record-low mortgage
rates and soft prices, to all-cash investors buying units in bulk. To
lure sheepish buyers, agents and developers are trying everything from
millennial-friendly Instagram tours to deeper discounts and even
``satisfaction guarantees.''

It's unclear where prices will settle, but the first batch of new buyers
could set the tone for months to come. In a small survey of 43 offers
entered after the stay-at-home order in Manhattan, Queens and Brooklyn,
the average offer was 14.5 percent below asking price, according to
Fritz Frigan with Halstead Real Estate. Among accepted offers, the
discount was about 8 percent. (Discounts are likely to be smaller at the
lower end of the market, where supply remains tight, agents said.)

Image

To sell her Upper West Side co-op, Lara Sullivan drove to New York from
her current home in Boston to open all the doors, then waited in the car
as a prospective buyer toured the space.Credit...Compass

Kathy Murray, a Douglas Elliman agent, is still getting deals done, but
from the confines of her home. If there is an upside to having to show
apartments virtually, it's that habitual open-house tourists rarely
bother, leaving only determined buyers to contend with.

``Once they want a FaceTime tour, they tend to be more serious about
making a deal,'' said Ms. Murray, who has four deals in the works ---
three of which involve international buyers.

In late April she closed a deal on an Upper East Side studio listed a
year ago. Before the pandemic, the price was cut twice, from \$745,000
to the last asking price of \$695,000, and she said the buyer, a Harvard
student from Hong Kong, negotiated an additional 9 percent discount as
the market grew more uncertain. Crucially, the buyer and his parents
requested to include the seller's furniture, so they wouldn't face
move-in challenges with the condo board.

Deals today require good timing and adaptability. Lara Sullivan, who
rented out her Upper West Side co-op before moving to Boston for a job
in the health industry, drove back to New York recently to retrieve the
keys from her last tenant, not thinking she'd soon find another renter,
let alone a buyer.

Her agent, Alyssa Brody with Compass, wasn't expecting much interest
when she listed the apartment for sale in late March, but she received a
call from an interested broker within half an hour, she said.

With gloves and a mask in tow, Ms. Sullivan drove for three hours to
open all the doors in the apartment, then waited in the car as the
prospective buyer toured the space. Ms. Brody, who was two hours away in
Sag Harbor, gave live updates over FaceTime, conveyed by the buyer's
agent.

Image

Alyssa Brody, a Compass agent, credited a recent sale to this
selfie-style apartment tour posted to Instagram. ``It has to be
authentic,'' she said, or buyers become suspicious of the
content.Credit...Alyssa Brody

It paid off: The buyer, who caught wind of the upcoming listing three
weeks before it came to market, agreed to pay close to the asking price
of \$1,968,300 for the three-bedroom duplex at the top of a prewar
walk-up. Ms. Brody credits an amateur Instagram tour of the apartment
she recorded in 2018 for catching the buyer's eye.

``It has to be authentic,'' she said of the
\href{https://www.instagram.com/p/B_FmVj1jMkC/}{selfie-style video},
with captions like ``the most magical part!'' and the hashtag
\#mondaymotivation. Overproduced videos, she said, can make buyers
question the content.

As Ms. Sullivan discovered, motivated buyers are out there. At Manhattan
House, the well-regarded midcentury condo on the Upper East Side, Shelly
Bleier, an agent with Douglas Elliman, sold a one-bedroom apartment,
sight unseen, to another resident of the building in an off-market deal.
It is in contract for \$70,000 more than the \$2.01 million the seller
**** paid for it in 2016, at the peak of the market. Ms. Bleier said she
would have listed the unit for about \$1.65 million, based on recent
comparable sales.

``I think it's the pandemic deal of the century,'' said Ms. Bleier,
whose client, an investor in India, was preparing to rent rather than
sell because she feared she wouldn't be able to turn a profit.

The buyer had heard the apartment was going to list for rent, and jumped
at the chance to buy it for a family member. ``I told the woman, `I
can't show it to you,' and she said, `I don't care,''' apparently
because she had seen the apartment before.

Image

A rendering of the Waterline Square complex on the West Side of
Manhattan, where a group of South American buyers bought eight units in
April for close to \$27 million. The buyers visited prior to New York's
stay-at-home order.Credit...Noe \& Associates with The Boundary

In the luxury condo market, where prices have been lagging for years,
there were just 11 new development contracts signed in the week ending
April 26, marking the lowest weekly tally in several years, according to
the brokerage CORE. Now developers, some of whom were under pressure to
move units long before the pandemic, are offering substantial
concessions.

While few expect the wave of defaults seen after 2008, there will likely
be some urgent sales in the coming months, said Elliot Bogod, president
of Broadway Realty. He said he was trying to purchase about 20 units at
a 20 percent discount from an Upper West Side condo he would not name,
because of competition from other bidders. (In an unusual move, he said
he was negotiating with the lender, not the developer, suggesting the
property might be in financial distress.) The units start at around \$4
million, he said, and his clients plan to rent them.

At Waterline Square, a new luxury complex on the West Side, a group of
South American buyers bought eight units in April for close to \$27
million in cash --- an average discount of about 7 percent, according to
people familiar with the deal. James Linsley, the president of GID
Development Group, the developer, would not comment on the buyers, who
visited the property before the lockdown, but said the sales team has
signed six deals since the lockdown began, none of which were sight
unseen.

Despite its current role as the epicenter of a global pandemic,
investors ``still look at New York City as a safe haven,'' said Melissa
Ziweslin, a senior managing director at Corcoran Sunshine, a large
development marketing firm.

At One Manhattan Square, the 815-unit tower on the Lower East Side, the
developer, Extell, has announced its deepest discounts yet: up to 20
percent off select units in a building where prices ranged from \$1.2
million to over \$13 million. Before the pandemic, the developer was
already offering to cover up to 10 years of common charges on the most
expensive units, at a cost of tens of millions of dollars to the
project. About 33 percent of the units are now closed or in contract,
according to an analysis by the data company MarketProof. A spokeswoman
said the developer would not release new sales numbers.

Image

A rendering of the Rowan, an under-construction condo project in
Astoria, Queens. Buyers who signed after the stay-at-home order will be
able to tour the project when the lockdown is eased, after which they
will have five days to walk away from any deal.Credit...Redundant Pixel

Less expensive condo projects have taken other unusual steps. At the
Rowan in Astoria, Queens, where prices range from \$540,000 to about
\$2.5 million, the developer, RockFarmer Properties, is offering a
``satisfaction guaranteed'' clause. Buyers who signed after the
stay-at-home order will be able to tour the under-construction project
when the lockdown is eased, after which they will have five days to walk
away from any deal. Down payment requirements were also reduced to 5
percent from 10 percent.

Shan Chowdhury, of Halstead, signed a contract for a client, a
first-time buyer in Miami who works in the medical field, to purchase a
one-bedroom apartment, sight unseen, at the Vernon 123 complex in Long
Island City, Queens.

The buyer watched a virtual tour and looked at an aerial view from
Google Street View, Mr. Chowdhury said. That was enough to seal the
deal. The 650-square-foot apartment initially listed for \$895,000 in
2019, and was cut several times to the last asking price of \$799,000.

Mr. Chowdhury wouldn't reveal the final price, because the deal hadn't
closed yet, but could confidently say: ``We renegotiated, hard.''

For weekly email updates on residential real estate news,
\href{http://www.nytimes3xbfgragh.onion/newsletters/realestate/}{sign up
here}. Follow us on Twitter:
\href{https://twitter.com/nytrealestate}{@nytrealestate}.

Advertisement

\protect\hyperlink{after-bottom}{Continue reading the main story}

\hypertarget{site-index}{%
\subsection{Site Index}\label{site-index}}

\hypertarget{site-information-navigation}{%
\subsection{Site Information
Navigation}\label{site-information-navigation}}

\begin{itemize}
\tightlist
\item
  \href{https://help.nytimes3xbfgragh.onion/hc/en-us/articles/115014792127-Copyright-notice}{©~2020~The
  New York Times Company}
\end{itemize}

\begin{itemize}
\tightlist
\item
  \href{https://www.nytco.com/}{NYTCo}
\item
  \href{https://help.nytimes3xbfgragh.onion/hc/en-us/articles/115015385887-Contact-Us}{Contact
  Us}
\item
  \href{https://www.nytco.com/careers/}{Work with us}
\item
  \href{https://nytmediakit.com/}{Advertise}
\item
  \href{http://www.tbrandstudio.com/}{T Brand Studio}
\item
  \href{https://www.nytimes3xbfgragh.onion/privacy/cookie-policy\#how-do-i-manage-trackers}{Your
  Ad Choices}
\item
  \href{https://www.nytimes3xbfgragh.onion/privacy}{Privacy}
\item
  \href{https://help.nytimes3xbfgragh.onion/hc/en-us/articles/115014893428-Terms-of-service}{Terms
  of Service}
\item
  \href{https://help.nytimes3xbfgragh.onion/hc/en-us/articles/115014893968-Terms-of-sale}{Terms
  of Sale}
\item
  \href{https://spiderbites.nytimes3xbfgragh.onion}{Site Map}
\item
  \href{https://help.nytimes3xbfgragh.onion/hc/en-us}{Help}
\item
  \href{https://www.nytimes3xbfgragh.onion/subscription?campaignId=37WXW}{Subscriptions}
\end{itemize}
