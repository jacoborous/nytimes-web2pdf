Sections

SEARCH

\protect\hyperlink{site-content}{Skip to
content}\protect\hyperlink{site-index}{Skip to site index}

\href{https://www.nytimes3xbfgragh.onion/section/technology}{Technology}

\href{https://myaccount.nytimes3xbfgragh.onion/auth/login?response_type=cookie\&client_id=vi}{}

\href{https://www.nytimes3xbfgragh.onion/section/todayspaper}{Today's
Paper}

\href{/section/technology}{Technology}\textbar{}Twitter Had Been Drawing
a Line for Months When Trump Crossed It

\url{https://nyti.ms/2MbCcaZ}

\begin{itemize}
\item
\item
\item
\item
\item
\item
\end{itemize}

Advertisement

\protect\hyperlink{after-top}{Continue reading the main story}

Supported by

\protect\hyperlink{after-sponsor}{Continue reading the main story}

\hypertarget{twitter-had-been-drawing-a-line-for-months-when-trump-crossed-it}{%
\section{Twitter Had Been Drawing a Line for Months When Trump Crossed
It}\label{twitter-had-been-drawing-a-line-for-months-when-trump-crossed-it}}

Inside the company, one faction wanted Jack Dorsey, Twitter's chief, to
take a hard line against the president's tweets while another urged him
to remain hands-off.

\includegraphics{https://static01.graylady3jvrrxbe.onion/images/2020/05/31/business/31twitter-print/merlin_168451299_16e0499e-5f91-4ccf-93bf-64fdda3a5b50-articleLarge.jpg?quality=75\&auto=webp\&disable=upscale}

By \href{https://www.nytimes3xbfgragh.onion/by/kate-conger}{Kate Conger}

\begin{itemize}
\item
  Published May 30, 2020Updated June 23, 2020
\item
  \begin{itemize}
  \item
  \item
  \item
  \item
  \item
  \item
  \end{itemize}
\end{itemize}

OAKLAND, Calif. ---
\href{https://www.nytimes3xbfgragh.onion/2020/06/23/technology/trump-twitter-label-seattle.html}{Jack
Dorsey} was up late Thursday at his home in San Francisco talking online
with his executives when their conversation was interrupted:
\href{https://www.nytimes3xbfgragh.onion/2020/06/23/technology/trump-twitter-label-seattle.html}{President
Trump} had just posted another inflammatory
\href{https://twitter.com/realDonaldTrump/status/1266231100780744704}{message}
on
\href{https://www.nytimes3xbfgragh.onion/2020/06/09/us/politics/trump-twitter.html}{Twitter}.

Tensions between
\href{https://www.nytimes3xbfgragh.onion/2020/06/10/technology/trump-twitter-protests.html}{Twitter},
where Mr. Dorsey is chief executive, and Mr. Trump had been running high
for days over the president's aggressive tweets and the company's
decision to begin labeling some of them. In his latest message, Mr.
Trump weighed in on the clashes between the police and protesters in
Minneapolis, saying, ``when the looting starts, the shooting starts.''

A group of more than 10 Twitter officials, including lawyers and
policymakers, quickly gathered virtually to review
\href{https://www.nytimes3xbfgragh.onion/2020/06/03/us/politics/trump-twitter-fact-check.html}{Mr.
Trump's post} and debate over the messaging system Slack and Google Docs
whether it pushed people toward violence.

They soon came to a conclusion. And after midnight, Mr. Dorsey gave his
go-ahead: Twitter would hide
\href{https://www.nytimes3xbfgragh.onion/2020/06/03/us/politics/trump-twitter-fact-check.html}{Mr.
Trump's tweet} behind a warning label that said the message violated its
policy against glorifying violence. It was the first time Twitter
applied that specific warning to any public figure's tweets.

The action has prompted a broad fight over whether and how social media
companies should be held responsible for what appears on their sites,
and was the culmination of months of debate inside Twitter. For more
than a year, the company had been building an infrastructure to limit
the impact of objectionable messages from world leaders, creating rules
on what would and would not be allowed and designing a plan for when Mr.
Trump inevitably broke them.

But the path to that point was not smooth. Inside Twitter, dealing with
Mr. Trump's tweets --- which are the equivalent of a presidential
megaphone --- was a fitful and uneven process. Some executives
repeatedly urged Mr. Dorsey to take action on the inflammatory posts
while others insisted he hold back, staying hands-off as the company had
done for years.

Outside Twitter, the president's critics urged the company to shut him
down as he pushed the limits with insults and untruths, noting ordinary
users were sometimes suspended for lesser transgressions. But Twitter
argued that posts by Mr. Trump and other world leaders deserved special
leeway because of their news value.

The efforts were complicated by Mr. Dorsey, 43, who was sometimes
\href{https://www.nytimes3xbfgragh.onion/2020/02/29/business/dealbook/elliott-twitter-jack-dorsey.html}{absent
on travels} and meditative retreats before
\href{https://www.nytimes3xbfgragh.onion/news-event/coronavirus?action=click\&pgtype=Article\&state=default\&module=styln-coronavirus-national\&variant=show\&region=TOP_BANNER\&context=storylines_menu}{the
coronavirus pandemic}. He often delegated policy decisions, watching the
debate from the sidelines so he would not dominate with his own views.
And he frequently did not weigh in until the last minute.

Now Twitter is at war with Mr. Trump over its treatment of his posts,
which has implications for the future of speech on social media. In the
past week, the company for the first time
\href{https://www.nytimes3xbfgragh.onion/2020/05/26/technology/twitter-trump-mail-in-ballots.html}{added
fact-checking} and other warning labels to three of Mr. Trump's
messages, refuting their accuracy or marking them as inappropriate.

In response, an irate Mr. Trump
\href{https://www.nytimes3xbfgragh.onion/2020/05/28/us/politics/trump-jack-dorsey.html}{issued
an executive order} designed to limit legal protections that tech
companies enjoy and posted more angry messages.

Twitter's position is precarious. The company is grappling with charges
of bias from the right over its labeling of Mr. Trump's tweets; one of
its executives has faced a sustained campaign of online harassment. Yet
Twitter's critics on the left said that by leaving Mr. Trump's tweets up
and not banning him from the site, it was enabling the president.

``It really is about whether or not Twitter blinks,'' said James
Grimmelmann, a law professor at Cornell University. ``You really have to
stick to your guns and ensure you do it right.''

Twitter is girding for a protracted battle with Mr. Trump. Some
employees have locked down their social media accounts and deleted their
professional affiliation to avoid being harassed. Executives, holed up
at home, are meeting virtually to discuss next steps while also handling
\href{https://www.nytimes3xbfgragh.onion/2020/03/08/technology/coronavirus-misinformation-social-media.html}{a
surge of misinformation} related to the pandemic.

This account of how Twitter came to take action on Mr. Trump's tweets
was based on interviews with nine current and former company employees
and others who work with Mr. Dorsey outside of Twitter. They declined to
be identified because they were not authorized to speak publicly and
because they feared being targeted by Mr. Trump's supporters.

A Twitter spokesman declined to comment. Mr. Dorsey tweeted on Friday
that the fact-checking process should be open to the public so that the
facts are ``verifiable by everyone.''

Mr. Trump said on Twitter that his recent statements were ``very
simple'' and that ``nobody should have any problem with this other than
the haters, and those looking to cause trouble on social media.'' The
White House declined to comment.

The confrontation between Mr. Trump and Twitter has raised questions
about free speech. Under
\href{https://www.nytimes3xbfgragh.onion/2020/05/28/business/section-230-internet-speech.html}{Section
230 of the Communications Decency Act}, social media companies are
shielded from most liability for the content posted on their platforms.
Republican lawmakers have argued the companies are acting as publishers
and not mere distributors of content and should be stripped of those
protections.

But a hands-off approach by the companies has allowed harassment and
abuse to proliferate online, said Lee Bollinger, the president of
Columbia University and a First Amendment scholar. So now the companies,
he said, have to grapple with how to moderate content and take more
responsibility, without losing their legal protections.

``These platforms have achieved incredible power and influence,'' Mr.
Bollinger said, adding that moderation was a necessary response.
``There's a greater risk to American democracy in allowing unbridled
speech on these private platforms.''

For years, Twitter did not touch Mr. Trump's messages. But as he
continued using Twitter to deride rivals and spread falsehoods, the
company faced mounting criticism.

That set off internal debates. Mr. Dorsey observed the discussions,
sometimes raising questions about who could be harmed by posts on
Twitter or its moderation decisions, executives said.

In 2018, two of the president's tweets stood out to Twitter officials.
In one,
\href{https://twitter.com/realdonaldtrump/status/948355557022420992?lang=en}{Mr.
Trump discussed launching nuclear weapons at North Korea}, which some
employees believed violated company policy against violent threats. In
the other, he called a former aide, Omarosa Manigault Newman, ``a
crazed, crying lowlife'' and ``that dog.''

At the time, Twitter had rules against harassing messages like the tweet
about Ms. Manigault Newman, but left the tweet up.

The company began working on a specific solution to allow it to respond
to violent and inaccurate posts from Mr. Trump and other world leaders
without removing the messages. Mr. Dorsey had expressed interest in
finding a middle ground, executives said. It also rolled out labels to
denote that a tweet needed fact-checking or had videos and photos that
had been altered to be misleading.

The effort was overseen by Vijaya Gadde, who leads Twitter's legal,
policy, trust and safety teams. The labels for world leaders, unveiled
last June, explained how a politician's message had broken a Twitter
policy and took away tools that could amplify it, like retweets and
likes.

``We want to elevate healthy conversations on Twitter and that may
sometimes mean offering context,'' Del Harvey, Twitter's vice president
of trust and safety, said in an interview this year.

By the time the labels were introduced, Mr. Trump was not the only head
of state testing Twitter's boundaries. Shortly before Twitter released
them, the president of Brazil, Jair Bolsonaro, tweeted a sexually
explicit video and the Iranian leader Ali Khamenei posted threatening
remarks about Israel.

Last month, Twitter used the labels on
\href{https://twitter.com/OsmarTerra/status/1246474430676643842}{a
tweet} from the Brazilian politician Osmar Terra in which he falsely
claimed that quarantine increased cases of the coronavirus.

``This Tweet violated the Twitter Rules,'' the label read. ``However,
Twitter has determined that it may be in the public's interest for the
Tweet to remain accessible.''

On Tuesday, Twitter officials began discussing labeling Mr. Trump's
messages after he falsely asserted that mail-in ballots were illegally
printed and implied they would lead to fraud in the November election.
His tweets were flagged to Twitter through a portal it had opened
specifically for nonprofit groups and local officials involved in
election integrity to report content that could discourage or interfere
with voting.

Twitter quickly concluded that Mr. Trump had posted false information
about mail-in ballots. The company then labeled two of his tweets,
urging people to ``get the facts'' about voting by mail. An in-house
team of fact checkers also assembled a list of what people should know
about mail-in ballots.

Mr. Trump struck back, drafting an executive order designed to chip away
at Section 230. He and his allies also singled out a Twitter employee
who had publicly criticized him and other Republicans, falsely
suggesting that employee was responsible for the labels.

Mr. Dorsey and his executives kept on alert. On Wednesday, Twitter
\href{https://www.nytimes3xbfgragh.onion/2020/05/28/technology/trump-twitter-fact-check.html}{labeled
hundreds of other tweets}, including those that falsely claimed to
include images of Derek Chauvin, the white police officer who was
charged with third-degree murder and second-degree manslaughter in
\href{https://www.nytimes3xbfgragh.onion/2020/05/27/us/george-floyd-minneapolis-death.html}{the
death of George Floyd}, an African-American man in Minnesota.

Mr. Trump did not let up. Even after Twitter called out his shooting
tweet for glorifying violence, he
\href{https://twitter.com/realDonaldTrump/status/1266434153932894208}{posted}
the same sentiment again.

``Looting leads to shooting,'' Mr. Trump wrote, adding that he did not
want violence to occur. ``It was spoken as a fact.''

This time, Twitter did not label the tweet.

Advertisement

\protect\hyperlink{after-bottom}{Continue reading the main story}

\hypertarget{site-index}{%
\subsection{Site Index}\label{site-index}}

\hypertarget{site-information-navigation}{%
\subsection{Site Information
Navigation}\label{site-information-navigation}}

\begin{itemize}
\tightlist
\item
  \href{https://help.nytimes3xbfgragh.onion/hc/en-us/articles/115014792127-Copyright-notice}{©~2020~The
  New York Times Company}
\end{itemize}

\begin{itemize}
\tightlist
\item
  \href{https://www.nytco.com/}{NYTCo}
\item
  \href{https://help.nytimes3xbfgragh.onion/hc/en-us/articles/115015385887-Contact-Us}{Contact
  Us}
\item
  \href{https://www.nytco.com/careers/}{Work with us}
\item
  \href{https://nytmediakit.com/}{Advertise}
\item
  \href{http://www.tbrandstudio.com/}{T Brand Studio}
\item
  \href{https://www.nytimes3xbfgragh.onion/privacy/cookie-policy\#how-do-i-manage-trackers}{Your
  Ad Choices}
\item
  \href{https://www.nytimes3xbfgragh.onion/privacy}{Privacy}
\item
  \href{https://help.nytimes3xbfgragh.onion/hc/en-us/articles/115014893428-Terms-of-service}{Terms
  of Service}
\item
  \href{https://help.nytimes3xbfgragh.onion/hc/en-us/articles/115014893968-Terms-of-sale}{Terms
  of Sale}
\item
  \href{https://spiderbites.nytimes3xbfgragh.onion}{Site Map}
\item
  \href{https://help.nytimes3xbfgragh.onion/hc/en-us}{Help}
\item
  \href{https://www.nytimes3xbfgragh.onion/subscription?campaignId=37WXW}{Subscriptions}
\end{itemize}
