Sections

SEARCH

\protect\hyperlink{site-content}{Skip to
content}\protect\hyperlink{site-index}{Skip to site index}

\href{https://www.nytimes3xbfgragh.onion/section/style}{Style}

\href{https://myaccount.nytimes3xbfgragh.onion/auth/login?response_type=cookie\&client_id=vi}{}

\href{https://www.nytimes3xbfgragh.onion/section/todayspaper}{Today's
Paper}

\href{/section/style}{Style}\textbar{}The `Call Her Daddy' Feud: What
Happened?

\url{https://nyti.ms/3g70SPy}

\begin{itemize}
\item
\item
\item
\item
\item
\item
\end{itemize}

Advertisement

\protect\hyperlink{after-top}{Continue reading the main story}

Supported by

\protect\hyperlink{after-sponsor}{Continue reading the main story}

\hypertarget{the-call-her-daddy-feud-what-happened}{%
\section{The `Call Her Daddy' Feud: What
Happened?}\label{the-call-her-daddy-feud-what-happened}}

The podcast built a loyal audience. So did its hosts.

\includegraphics{https://static01.graylady3jvrrxbe.onion/images/2020/05/19/fashion/19CALL-HER-DADDY-cooper-franklyn/19CALL-HER-DADDY--articleLarge.jpg?quality=75\&auto=webp\&disable=upscale}

\href{https://www.nytimes3xbfgragh.onion/by/taylor-lorenz}{\includegraphics{https://static01.graylady3jvrrxbe.onion/images/2020/03/18/reader-center/author-taylor-lorenz/author-taylor-lorenz-thumbLarge.png}}

By \href{https://www.nytimes3xbfgragh.onion/by/taylor-lorenz}{Taylor
Lorenz}

\begin{itemize}
\item
  Published May 19, 2020Updated July 16, 2020
\item
  \begin{itemize}
  \item
  \item
  \item
  \item
  \item
  \item
  \end{itemize}
\end{itemize}

In just two years, ``Call Her Daddy,'' a raunchy podcast about sex,
dating, culture and life in New York, became a cult sensation. This
week, it descended into a fireball of chaos and internet drama.

Two friends, Sofia Franklyn, 27, and Alexandra Cooper, 26, started the
show in 2018 after coming up with the idea over a vacation. The podcast
was acquired by Barstool Sports about a month after the first episode
aired and quickly began raking in hundreds of thousands of subscribers.

It jumped from
\href{https://www.adexchanger.com/publishers/influencers-are-the-new-publishers-barstool-sports-ceo-erika-nardini/}{12,000
to two million downloads in just two months} and topped the podcast
charts. It is one of the top 20 most popular on
\href{https://www.podcastinsights.com/top-us-podcasts/}{Apple Podcasts}.

``They talk about things everyone can relate to, especially people in
college and their 20s,'' said Addison Rose, 19, a college student and
longtime fan. ``It's interesting to hear sex talked about in such an
honest way from a women's perspective. It was refreshing.''

\begin{center}\rule{0.5\linewidth}{\linethickness}\end{center}

What happens online? More by
\href{https://www.nytimes3xbfgragh.onion/by/taylor-lorenz}{Taylor
Lorenz}:

\begin{itemize}
\item
  \href{https://www.nytimes3xbfgragh.onion/2020/06/08/style/sway-house-neighbors-tiktok.html}{How
  the Sway House TikTok Collab Made Enemies}
\item
  \href{https://www.nytimes3xbfgragh.onion/2020/06/29/style/shane-dawson-jeffree-star-youtube.html}{Inside
  the Shane Dawson and Jeffree Star YouTube Drama}
\item
  \href{https://www.nytimes3xbfgragh.onion/2020/06/26/us/jenna-marbles-leaves-youtube.html}{YouTube
  Star Jenna Marbles Apologizes for Blackface Video}
\item
  \href{https://www.nytimes3xbfgragh.onion/2020/07/16/style/taylor-lorenz-internet-culture-reporting.html}{How
  We Report on Internet Culture and the Teens Who Rule It}
\end{itemize}

\begin{center}\rule{0.5\linewidth}{\linethickness}\end{center}

Ms. Cooper wrote in a post titled
``\href{https://www.barstoolsports.com/blog/1108118/how-did-two-blow-up-dolls-land-a-barstool-deal}{How
Did Two `Blow Up Dolls' Land a Barstool Deal}'': ``Barstool liked our
idea of uncensored, real, female locker room talk, which quite frankly
is just as nasty as guy locker room talk. And we had no problem
exploiting our experiences --- as well as ourselves --- for our
listeners' entertainment.''

But in the past month, the ``Call Her Daddy'' empire has tumbled,
exposing the inevitable issues media companies face when their star
employees morph into powerful influencers.

``This controversy is starting to show some of the macro issues in the
media businesses built around these personalities,'' said Blake Robbins,
a partner at Ludlow Ventures, a venture capital firm. ``It's the pull of
influencers outgrowing the media brands that invested in their
careers.''

\hypertarget{an-uncertain-future}{%
\subsection{An Uncertain Future}\label{an-uncertain-future}}

The last time anyone heard from the two ``Call Her Daddy'' hosts
directly was on April 8, when they uploaded an episode with the cryptic
subtitle ``Kesha\ldots{} The End.'' Then, the podcast's feed went dark.

Fans began tweeting the hashtag \#FreeTheFathers, speculating that the
hosts were being silenced by their parent company, Barstool Sports.
``Fathers'' is an affectionate nickname given to the ``Call Her Daddy''
hosts.

On April 21, Ms. Franklyn and Ms. Cooper posted a statement to Instagram
saying that they ``legally can't speak out yet'' about what has been
taking place behind the scenes. (Neither Ms. Cooper nor Ms. Franklyn
responded to requests to comment for this article.) Tabloids
\href{https://pagesix.com/2020/04/27/battle-brewing-between-barstool-sports-and-raunchy-hit-podcast-call-her-daddy/}{speculated
on a fallout between the two}. Fans
\href{https://twtext.com/article/1254463034522820610}{posted theories}
that they'd dropped hidden messages in the titles of recent episodes
indicating trouble.

On Sunday, the tension that had been quietly brewing for weeks boiled
over.

Dave Portnoy, the Barstool Sports founder and president, posted a
30-minute tell-all episode to the ``Call Her Daddy'' podcast feed,
calling Ms. Franklyn and Ms. Cooper ``unprofessional, disloyal and
greedy'' before revealing the full details of his side of the messy
contract dispute.

According to Erika Nardini, the C.E.O. of Barstool,
\href{https://variety.com/2019/digital/news/barstool-sports-podcast-revenue-1203305912/}{more
than 35 percent of Barstool's revenue} now comes from the company's
podcast business, and ``Call Her Daddy'' was a crown jewel. Mr. Portnoy
said the company was losing \$100,000 per missed episode and said that
he offered the hosts a guaranteed base salary of \$500,000 a year, plus
bonuses, among other incentives that he estimated would ultimately net
them millions, to return to the show.

Ms. Cooper, he said, had agreed to the new terms, but Ms. Franklyn
refused, according to Mr. Portnoy, on the advice of her boyfriend, the
\href{https://nypost.com/2020/05/19/what-to-know-about-peter-nelson-and-call-her-daddy-drama/}{HBO
Sports executive Peter Nelson}, referred to as ``suitman'' on the
podcast. Mr. Nelson helped Ms. Franklyn and Ms. Cooper shop the podcast
around to other networks, according to Mr. Portnoy.

On Tuesday, Mr. Portnoy announced new ``Call Her Daddy'' merch
emblazoned with the phrase
``\href{https://twitter.com/stoolpresidente/status/1262390184613740544}{cancel
suitman},'' alluding to Mr. Nelson. Fans lashed out at Mr. Nelson online
for
\href{https://twitter.com/knicks_tape99/status/1262412507060338689}{coming
between the two friends} and
``\href{https://twitter.com/realitykiwi/status/1262209743478808576}{ruining
the podcast}.''

Now, the future of the show is uncertain. According to Mr. Portnoy, Ms.
Cooper will likely return and take over ``Call Her Daddy,'' and an offer
is out to Ms. Franklyn. They might host competing podcasts, both on the
Barstool Network.

On Tuesday afternoon, Ms. Franklyn addressed the controversy on
Instagram Stories. ``Did Barstool help blow up `Call Her Daddy?' 100
percent,'' she said. But, that doesn't mean she'll be returning. ``I
found out that Alex had gone behind my back and done something and I
found out that it wasn't the first time. I'm willing to do `Call Her
Daddy,' I really am. I can't do it under the circumstances that she
wants,'' Ms. Franklyn said.

In the meantime, thousands of fans have continued weighing in online.
Many have begged for a reconciliation. Others feel deeply betrayed.

In his tell-all, Mr. Portnoy said that Ms. Franklyn and Ms. Cooper had
each taken home nearly \$500,000 last year, a figure far higher than
many fans imagined. ``They'd frequently talk in the podcast about being
broke girls in their 20s, which is relatable,'' Ms. Rose said. ``Then it
came out how much they were making.''

``\$500k a year to talk about sex once a week and that's not good enough
for you? Imagine having it THAT easy.. feel betrayed as a fan tbh,''
\href{https://twitter.com/ballerinaswifts/status/1262198053743853573}{another
fan tweeted}.

\hypertarget{people-expect-transparency}{%
\subsection{`People Expect
Transparency'}\label{people-expect-transparency}}

Media companies have long acted as talent incubators, providing content
producers name-brand recognition and access to a larger audience. But,
as that talent builds a following on social media, the balance of power
shifts. Often, talent no longer needs the media company to operate as a
middleman, and many realize they could monetize their own platforms more
effectively by themselves.

``Even though this feels messy and salacious, it actually does touch on
much larger questions about media institutions and talent, and how they
create value and contracts with each other,'' said Nicholas Quah, the
founder of Hot Pod, a newsletter about podcasts.

When ``Call Her Daddy'' started, Ms. Franklyn and Ms. Cooper were
relatively unknown. Now they have about one million followers on
Instagram each and wield a wide and loyal audience online. To Barstool
Sports, however, they were simply employees.

``We're entering a period where creators are business owners and media
brands of their own. They can't just be seen as employees,'' said Jordi
Hays, a digital strategist who works with online creators in Los
Angeles. ``The tools are available to them to become founders and
C.E.O.s of their brand, and develop businesses with multiple powerful
revenue streams like merch, ad sales and subscription revenue.''

This isn't the first time high-profile internet creators decided their
media company wasn't working for them anymore and decided to forge out
their own. In 2016, a slew of employees of BuzzFeed Video
\href{https://whatstrending.com/trending-videos/trending-now/23343-why-do-creators-leave-buzzfeed/}{left
to become full-time YouTubers}. Scott Rogowsky went through
\href{https://www.thedailybeast.com/ceo-of-hq-the-hottest-app-going-if-you-run-this-profile-well-fire-our-host}{extensive
complex contract negotiations} after HQ Trivia minted him as a star.

And Tfue, an elite Fortnite player, sued
\href{https://www.nytimes3xbfgragh.onion/2019/11/15/style/faze-clan-house.html}{FaZe
Clan}, a gaming content collective, in 2019, to escape
\href{https://www.theatlantic.com/technology/archive/2019/05/why-tfues-lawsuit-against-faze-clan-matters/589900/}{what
he considered an exploitive contract}. ``The time is now for content
creators, gamers and streamers to stop being taken advantage of through
oppressive, unfair and illegal agreements,'' he
\href{https://www.theatlantic.com/technology/archive/2019/05/why-tfues-lawsuit-against-faze-clan-matters/589900/}{wrote
in the suit}.

While most traditional publishers and media brands would balk at
litigating a contract dispute in public, Mr. Portnoy and Barstool Sports
have proven themselves adept at leveraging online attention in their
favor.

``A lot of companies would try to sweep this under the rug and put out a
press release, but in this new world of media where these people are
huge personalities, people expect transparency,'' Mr. Robbins said.
``This is a Barstool gold mine. They want these story lines. They can
now create these competing podcasts between the `Call Her Daddy' hosts,
and I have no doubt that both would do really well.''

That's assuming both women would return. Grace Atwood, a lifestyle
influencer and co-host of the podcast ``Bad on Paper,'' said she could
see why they may not want to. ``These girls have nearly one million
followers, they can make \$10,000 or \$20,000 for a single Instagram
post,'' she said. ``You can take that audience and do anything. You
could start a fashion label, you could start a spinoff podcast. If that
audience likes and trusts you, there's literally endless things you can
do with it. Imagine getting to that level and then taking a \$75,000
salary.''

Emma Gray, a host of ``Here to Make Friends,'' a podcast about ``The
Bachelor,'' owned by HuffPost, said that she has loved the production,
sales and support HuffPost has offered since she teamed with the company
five years ago to start her show.

Still, ``we all know that followers are literal capital at this point,''
she said. ``I think it's important for media companies to treat their
podcast hosts as talent and therefore use talent contracts in their
negotiations rather than a general employment agreement.''

For talent negotiations, there's no better person to have on your side
than an experienced talent agent or a lawyer. Quinn Heraty, the founder
of Heraty Law, has worked extensively with podcasters. Ms. Heraty said
that she encourages nearly all hosts to go independent. ``Being an
employee is just working to create value for someone else,'' she said.

But for those who do take a deal with a more established media brand,
the important thing is to negotiate equitable terms from the beginning.
``When you have a deal that's inherently more favorable to one side or
another, that's going to, over the course of the deal, cause stress on
the relationship,'' she said. ``Deals change over the lifetime of a
business relationship because these dynamics are dynamic.''

The systems for monetization are also increasingly complex. Most
aspiring podcasters don't think of negotiating things like ownership of
their back catalog, licensing agreements, platform exclusivity,
intellectual property and more, upfront. ``The deal making is only
becoming more complex,'' said Oren Rosenbaum, the head of emerging
platforms at United Talent Agency. And, ``we're still on the ground
floor. We're still in the infancy of this business.''

No matter how the situation with ``Call Her Daddy'' resolves, these
conflicts won't go away as long as the traditional media ecosystem
remains in flux. But fans and those in the online creator industry are
rapt.

``This is one of the first times in this new personality-driven media
world that we've gotten insight into how the contracts are actually
structured,'' Mr. Robbins said. ``Dave and Barstool clearly do bring
something to the table, because they're making this one of the most
talked-about things in pop culture.''

Advertisement

\protect\hyperlink{after-bottom}{Continue reading the main story}

\hypertarget{site-index}{%
\subsection{Site Index}\label{site-index}}

\hypertarget{site-information-navigation}{%
\subsection{Site Information
Navigation}\label{site-information-navigation}}

\begin{itemize}
\tightlist
\item
  \href{https://help.nytimes3xbfgragh.onion/hc/en-us/articles/115014792127-Copyright-notice}{©~2020~The
  New York Times Company}
\end{itemize}

\begin{itemize}
\tightlist
\item
  \href{https://www.nytco.com/}{NYTCo}
\item
  \href{https://help.nytimes3xbfgragh.onion/hc/en-us/articles/115015385887-Contact-Us}{Contact
  Us}
\item
  \href{https://www.nytco.com/careers/}{Work with us}
\item
  \href{https://nytmediakit.com/}{Advertise}
\item
  \href{http://www.tbrandstudio.com/}{T Brand Studio}
\item
  \href{https://www.nytimes3xbfgragh.onion/privacy/cookie-policy\#how-do-i-manage-trackers}{Your
  Ad Choices}
\item
  \href{https://www.nytimes3xbfgragh.onion/privacy}{Privacy}
\item
  \href{https://help.nytimes3xbfgragh.onion/hc/en-us/articles/115014893428-Terms-of-service}{Terms
  of Service}
\item
  \href{https://help.nytimes3xbfgragh.onion/hc/en-us/articles/115014893968-Terms-of-sale}{Terms
  of Sale}
\item
  \href{https://spiderbites.nytimes3xbfgragh.onion}{Site Map}
\item
  \href{https://help.nytimes3xbfgragh.onion/hc/en-us}{Help}
\item
  \href{https://www.nytimes3xbfgragh.onion/subscription?campaignId=37WXW}{Subscriptions}
\end{itemize}
