Sections

SEARCH

\protect\hyperlink{site-content}{Skip to
content}\protect\hyperlink{site-index}{Skip to site index}

\href{https://www.nytimes3xbfgragh.onion/section/movies}{Movies}

\href{https://myaccount.nytimes3xbfgragh.onion/auth/login?response_type=cookie\&client_id=vi}{}

\href{https://www.nytimes3xbfgragh.onion/section/todayspaper}{Today's
Paper}

\href{/section/movies}{Movies}\textbar{}`On the Record' Review: A Black
Woman Says `\#MeToo'

\url{https://nyti.ms/2M5gu8y}

\begin{itemize}
\item
\item
\item
\item
\item
\end{itemize}

Advertisement

\protect\hyperlink{after-top}{Continue reading the main story}

Supported by

\protect\hyperlink{after-sponsor}{Continue reading the main story}

\hypertarget{on-the-record-review-a-black-woman-says-metoo}{%
\section{`On the Record' Review: A Black Woman Says
`\#MeToo'}\label{on-the-record-review-a-black-woman-says-metoo}}

This documentary about allegations against the music mogul Russell
Simmons generated controversy at Sundance when Oprah Winfrey pulled out
as an executive producer.

\includegraphics{https://static01.graylady3jvrrxbe.onion/images/2020/05/29/arts/27ontherecord1/27ontherecord1-articleLarge-v2.jpg?quality=75\&auto=webp\&disable=upscale}

By Devika Girish

\begin{itemize}
\item
  May 27, 2020
\item
  \begin{itemize}
  \item
  \item
  \item
  \item
  \item
  \end{itemize}
\end{itemize}

\begin{itemize}
\tightlist
\item
  On the Record\\
  Directed by Kirby Dick, Amy Ziering Documentary 1h 35m
\end{itemize}

\href{https://www.imdb.com/showtimes/title/tt11394650?ref_=ref_ext_NYT}{Find
Tickets}

When you purchase a ticket for an independently reviewed film through
our site, we earn an affiliate commission.

In her 1999 book, ``When Chickenheads Come Home to Roost: A Hip-Hop
Feminist Breaks It Down,'' the cultural critic
\href{https://www.nytimes3xbfgragh.onion/2018/08/15/books/review-she-begat-this-lauryn-hill-joan-morgan.html}{Joan
Morgan} describes eloquently the ways in which racism often makes it
difficult for black women to call out sexism within their own
communities. ``I needed a feminism that would allow us to continue
loving ourselves and the brothers who hurt us without letting race
loyalty buy us early tombstones,'' she wrote.

She reiterates that sentiment in ``On the Record,'' a wrenching new
documentary by Kirby Dick and Amy Ziering,
\href{https://www.hbomax.com/feature/urn:hbo:feature:GXrw_sA07SZGXWQEAAADb}{streaming
on HBO Max}. The film details the allegations of sexual assault against
the music mogul Russell Simmons, but its scope is much wider: It
explores the particular (and often overlooked) struggles of black women
in the \#MeToo movement. At its center is Drew Dixon, 48, who says
Simmons raped her in 1995 while she was a rising A\&R executive at his
pioneering company, Def Jam Records. Dixon didn't speak publicly about
the incident for more than two decades, fearing that challenging the
``godfather of hip-hop'' would amount to a betrayal of her community.
``I didn't want to let the culture down,'' she says poignantly. ``I love
the culture. I loved Russell, too.''

``On the Record'' closely follows Dixon before, during and after her
decision to go public with her accusations in a
\href{https://www.nytimes3xbfgragh.onion/2017/12/13/arts/music/russell-simmons-rape.html}{December
2017 article} in The New York Times. It also weaves in the testimonies
of seven other women who say they were raped by Simmons --- including
the writer
\href{https://www.hollywoodreporter.com/news/writer-jenny-lumet-russell-simmons-sexually-violated-me-guest-column-1062934}{Jenny
Lume}t, the former assistant and model
\href{https://www.usatoday.com/story/entertainment/tv/2019/10/21/nbc-russell-simmons-aj-calloway-accuser-sil-lai-abrams/4051745002/}{Sil
Lai Abrams} and the hip-hop artist
\href{https://www.wnyc.org/story/woman-speaks-out-against-russell-simmons-alleged-rape/}{Sherri
Hines}. (Simmons has denied all accusations of nonconsensual sex and
described his life as ``devoid of violence'' in a written response to
the filmmakers.)

The stories of these women hit the familiar beats of the countless
\#MeToo narratives that have emerged since the reckoning of Harvey
Weinstein three years ago: abuses of power, derailed careers, fearful
silences, doubts and dismissals. But for black women who have been
assaulted by black men, the quest for justice is intersectional. It
involves negotiations between solidarity and salvation.

The film communicates these complex ideas with quiet, forceful emotional
clarity. It's the latest in Dick and Ziering's formidable oeuvre of
documentaries on the subject: Their previous collaborations, like
``\href{https://www.nytimes3xbfgragh.onion/2012/06/22/movies/the-invisible-war-directed-by-kirby-dick.html}{The
Invisible War},'' about sexual assault in the military, and
``\href{https://www.nytimes3xbfgragh.onion/2015/02/27/movies/review-the-hunting-ground-documentary-a-searing-look-at-campus-rape.html}{The
Hunting Ground},'' about rape on college campuses, provoked strong,
polarized reactions and substantive policy changes. ``On the Record'' is
a relatively modest film --- more character study than exposé --- but it
has already attracted considerable controversy. Shortly before it
premiered at the Sundance Film Festival in January,
\href{https://www.nytimes3xbfgragh.onion/2020/01/11/business/media/oprah-winfrey-documentary-russell-simmons-apple.html}{Oprah
Winfrey}, one of its executive producers, dropped out, citing creative
differences, and the filmmakers lost their distribution deal with Apple
TV and moved to HBO Max.

Amid speculation that Simmons might have pressured Winfrey to withdraw,
\href{https://www.nytimes3xbfgragh.onion/2020/01/17/movies/oprah-winfrey-russell-simmons-movie.html}{she
told The Times} that she continued to stand by the women in the film but
felt that their stories hadn't been sufficiently ``elevated,'' and that
the film lacked an appropriately broad context that took in the
``debauchery'' of the music business at the time. Winfrey's concerns
about the narrow focus and selective context don't seem misplaced, but
these strike me as strengths rather than shortcomings. ``On the Record''
is not a work of journalism. Most of the accusations have been reported
on extensively in the last two years in
\href{https://www.latimes.com/business/hollywood/la-fi-russell-simmons-yoga-20171213-htmlstory.html}{various
publications}. What the film does is bring these accounts to living,
breathing and \emph{moving} life, taking us beyond the media cycles of
allegation and denial to a survivor's intimate confrontations with
cultural pressures and trauma.

Dixon makes for a stirring and charismatic protagonist, her face
lighting up with joy as she describes walking the streets of Brooklyn
with Biggie Smalls and putting together a Grammy-winning record with
Method Man and Mary J. Blige. When the harassment began, she thought her
talent would protect her from danger. ``I am an executive with value,
and he's a businessperson,'' she says. When she's proved wrong, it's a
crushing lesson: Misogyny razes all ideas of value and worth, even in
seemingly meritocratic settings.

\begin{center}\rule{0.5\linewidth}{\linethickness}\end{center}

A few years later, while working a different job at Arista Records,
Dixon is confronted by this reality all over again. When she rebuffed
the advances of L.A. Reid, then the chief executive, she says he
punished her by passing on two up-and-coming artists she had scouted:
Kanye West and John Legend. (Reid has said
\href{https://www.buzzfeednews.com/article/michaelblackmon/drew-dixon-profile-russell-simmons-me-too-rape}{he
apologized if anything he had said or done had been
``misinterpreted.''})

Employing a plain, by-the-numbers style, Dick and Ziering make a
deliberate choice to let their interviewees take center stage,
contextualizing their stories with some archival images. These inserts
can feel glib. One hasty sequence, which seems to support some of
Winfrey's reservations, starts with excerpts from rap videos to
demonstrate the prevalence of sexism in hip-hop, then follows up with
clips from songs by the Beatles and the Rolling Stones to clarify,
rather awkwardly, that sexism isn't confined to hip-hop.

But these nominal bits of editorialization are supplemented by a superb
cast of cultural commentators, who represent the real, intellectual
force of ``On the Record.'' In addition to Morgan, they include Tarana
Burke, the founder of the \#MeToo movement, Kimberlé Crenshaw, the
scholar behind the theory of
\href{https://time.com/5786710/kimberle-crenshaw-intersectionality/}{intersectionality},
and Kierna Mayo, the former editor of Ebony magazine.

As Dixon and the other survivors describe their painful experiences of
harassment and shame, the commentators situate them eloquently within
the broader picture of African-American history and raise important
questions that have often remained at the periphery of the \#MeToo
movement. They explain, for instance, how black women's fear of speaking
out against their abusers is rooted in the legacy of false rape
accusations against black men --- and that the criminal justice system
offers little hope to communities that have long suffered its biases.

The filmmakers do an admirable job of switching between these micro and
macro perspectives in a tight and accessible 95 minutes. Their movie
makes a sincere case for Dixon's quiet plea to the camera: ``It's time
for somebody to acknowledge the burden and the plunder of black women.''

\begin{center}\rule{0.5\linewidth}{\linethickness}\end{center}

\textbf{On the Record}

Not rated. Running time: 1 hour 35 minutes. Watch on
\href{https://www.hbomax.com/on-the-record/}{HBO Max}.

Advertisement

\protect\hyperlink{after-bottom}{Continue reading the main story}

\hypertarget{site-index}{%
\subsection{Site Index}\label{site-index}}

\hypertarget{site-information-navigation}{%
\subsection{Site Information
Navigation}\label{site-information-navigation}}

\begin{itemize}
\tightlist
\item
  \href{https://help.nytimes3xbfgragh.onion/hc/en-us/articles/115014792127-Copyright-notice}{©~2020~The
  New York Times Company}
\end{itemize}

\begin{itemize}
\tightlist
\item
  \href{https://www.nytco.com/}{NYTCo}
\item
  \href{https://help.nytimes3xbfgragh.onion/hc/en-us/articles/115015385887-Contact-Us}{Contact
  Us}
\item
  \href{https://www.nytco.com/careers/}{Work with us}
\item
  \href{https://nytmediakit.com/}{Advertise}
\item
  \href{http://www.tbrandstudio.com/}{T Brand Studio}
\item
  \href{https://www.nytimes3xbfgragh.onion/privacy/cookie-policy\#how-do-i-manage-trackers}{Your
  Ad Choices}
\item
  \href{https://www.nytimes3xbfgragh.onion/privacy}{Privacy}
\item
  \href{https://help.nytimes3xbfgragh.onion/hc/en-us/articles/115014893428-Terms-of-service}{Terms
  of Service}
\item
  \href{https://help.nytimes3xbfgragh.onion/hc/en-us/articles/115014893968-Terms-of-sale}{Terms
  of Sale}
\item
  \href{https://spiderbites.nytimes3xbfgragh.onion}{Site Map}
\item
  \href{https://help.nytimes3xbfgragh.onion/hc/en-us}{Help}
\item
  \href{https://www.nytimes3xbfgragh.onion/subscription?campaignId=37WXW}{Subscriptions}
\end{itemize}
