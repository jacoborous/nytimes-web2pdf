Sections

SEARCH

\protect\hyperlink{site-content}{Skip to
content}\protect\hyperlink{site-index}{Skip to site index}

\href{https://www.nytimes3xbfgragh.onion/section/politics}{Politics}

\href{https://myaccount.nytimes3xbfgragh.onion/auth/login?response_type=cookie\&client_id=vi}{}

\href{https://www.nytimes3xbfgragh.onion/section/todayspaper}{Today's
Paper}

\href{/section/politics}{Politics}\textbar{}Hong Kong Has Lost Autonomy,
Pompeo Says, Opening Door to U.S. Action

\url{https://nyti.ms/2M6mGx7}

\begin{itemize}
\item
\item
\item
\item
\item
\item
\end{itemize}

Advertisement

\protect\hyperlink{after-top}{Continue reading the main story}

Supported by

\protect\hyperlink{after-sponsor}{Continue reading the main story}

\hypertarget{hong-kong-has-lost-autonomy-pompeo-says-opening-door-to-us-action}{%
\section{Hong Kong Has Lost Autonomy, Pompeo Says, Opening Door to U.S.
Action}\label{hong-kong-has-lost-autonomy-pompeo-says-opening-door-to-us-action}}

The State Department announcement comes as President Trump weighs hard
measures against China, which is expected to approve a national security
law on Hong Kong on Thursday.

\includegraphics{https://static01.graylady3jvrrxbe.onion/images/2020/05/27/us/politics/27dc-trumpchina1/merlin_172675209_b08d3a87-8faa-47b4-8a44-435a1dc6274c-articleLarge.jpg?quality=75\&auto=webp\&disable=upscale}

\href{https://www.nytimes3xbfgragh.onion/by/edward-wong}{\includegraphics{https://static01.graylady3jvrrxbe.onion/images/2018/09/24/multimedia/author-edward-wong/author-edward-wong-thumbLarge-v5.png}}

By \href{https://www.nytimes3xbfgragh.onion/by/edward-wong}{Edward Wong}

\begin{itemize}
\item
  Published May 27, 2020Updated June 29, 2020
\item
  \begin{itemize}
  \item
  \item
  \item
  \item
  \item
  \item
  \end{itemize}
\end{itemize}

\href{https://cn.nytimes3xbfgragh.onion/usa/20200528/china-hong-kong-pompeo-trade/}{阅读简体中文版}\href{https://cn.nytimes3xbfgragh.onion/usa/20200528/china-hong-kong-pompeo-trade/zh}{閱讀繁體中文版}

WASHINGTON --- Secretary of State
\href{https://www.nytimes3xbfgragh.onion/2020/06/23/us/politics/pompeo-state-human-rights.html}{Mike
Pompeo} announced on Wednesday that the State Department no longer
considered
\href{https://www.nytimes3xbfgragh.onion/2020/06/29/business/economy/us-halts-high-tech-exports-hong-kong.html}{Hong
Kong} to have significant autonomy under Chinese rule, a move that
indicated the Trump administration was likely to end some or all of the
United States government's
\href{https://www.nytimes3xbfgragh.onion/2020/05/28/business/hong-kong-special-status-explained.html}{special
trade and economic relations} with the territory in southern China.

Mr. Pompeo's action came just hours before China was expected to pass a
\href{https://www.nytimes3xbfgragh.onion/2020/06/19/world/asia/hong-kong-china-national-security.html}{national
security law} that would allow Chinese security agencies to take broad
actions limiting the liberties of Hong Kong residents,
\href{https://www.nytimes3xbfgragh.onion/2020/05/27/world/asia/hong-kong-protest-china-anthem.html?action=click\&module=Top\%20Stories\&pgtype=Homepage}{many
of whom} have
\href{https://www.nytimes3xbfgragh.onion/2020/05/24/world/asia/hong-kong-protest-coronavirus-china.html}{protested}
the proposed law and clashed with police officers.

The United States and China appear to be on a collision course over the
future of Hong Kong, a center of global capitalism and symbol of
resistance to the Chinese Communist Party. Relations between the two
nations are at their
\href{https://www.nytimes3xbfgragh.onion/2020/05/28/world/asia/china-united-states.html}{worst
in decades}, and disputes have flared over trade, national security and
the
\href{https://www.nytimes3xbfgragh.onion/2020/04/30/us/politics/trump-administration-intelligence-coronavirus-china.html}{origins
of the coronavirus}.

President Trump's foreign policy aides are discussing actions that would
be among the harshest punishments taken against China over the past
three years. The actions could have far-reaching consequences for global
commerce and transform how Chinese and foreign companies operate, as
well as upend the lives of many of Hong Kong's 7.5 million residents,
who have been under enormous pressure from years of political
crackdowns.

Hong Kong has been a financial and commercial hub since late last
century. China relies on the bustling city of ports and skyscrapers on
the edge of the South China Sea for transactions with other countries.
Many Chinese and foreign firms use Hong Kong as an international or
regional base, and members of elite Communist Party families or
executives with ties to them do business and own property there. Many
companies also raise capital by listing on the Hong Kong Stock Exchange.

Mr. Pompeo has said the security law would be a
\href{https://www.nytimes3xbfgragh.onion/2020/05/22/world/asia/trump-pompeo-china-hong-kong.html}{``death
knell''} for Hong Kong, which has had liberties under a semiautonomous
system of governance that do not exist in mainland China, including
freedoms of speech, the press and assembly, as well as an independent
judiciary.

In recent days, protesters in Hong Kong have taken to the streets to
\href{https://www.nytimes3xbfgragh.onion/2020/05/27/world/asia/hong-kong-protest-china-anthem.html}{voice
outrage} at the proposed law, only to be beaten back by police officers
clad in riot gear and firing tear gas.

American diplomats said they called on Wednesday for a virtual meeting
of the United Nations Security Council to discuss Hong Kong, but China
blocked the move.

If it proceeds with punishments, the Trump administration could impose
the same tariffs on exports from Hong Kong that it puts on goods from
mainland China, said officials with knowledge of the discussions. Other
trade restrictions that apply to China, including bans or limits on what
American companies can sell to Chinese companies because of national
security or
\href{https://www.nytimes3xbfgragh.onion/2020/06/23/us/politics/pompeo-state-human-rights.html}{human
right}s concerns, may be imposed on Hong Kong as well.

Some of Mr. Trump's advisers are discussing visa bans on Chinese
officials who enact the law.

``I certified to Congress today that Hong Kong does not continue to
warrant treatment under United States laws in the same manner as U.S.
laws were applied to Hong Kong before July 1997,'' Mr. Pompeo said
Wednesday. ``No reasonable person can assert today that Hong Kong
maintains a high degree of autonomy from China, given facts on the
ground.''

``Hong Kong and its dynamic, enterprising and free people have
flourished for decades as a bastion of liberty, and this decision gives
me no pleasure,'' he added. ``But sound policymaking requires a
recognition of reality. While the United States once hoped that free and
prosperous Hong Kong would provide a model for authoritarian China, it
is now clear that China is modeling Hong Kong after itself.''

Mr. Pompeo is the most vocal of a group of national security officials
who advocate tough policies on China. Some of Mr. Trump's top economic
advisers prefer a more conciliatory approach to dealing with China, the
world's second-largest economy, and will likely urge caution. American
corporate executives have said the administration should act with care.

Mr. Trump has rarely made any strong comments on the situation in Hong
Kong, and he has praised Xi Jinping, the president of China, throughout
his time in office, even insisting that they have a strong friendship.
Mr. Trump has also been eager to promote a trade agreement he signed
with China in January as an economic win for the United States. He wants
to avoid jeopardizing that deal, even though Beijing is not meeting
purchasing quotas mandated by it.

The president is keen to boost the U.S. economy, which has fallen into
recession during the pandemic, ahead of the November presidential
election.

But on Tuesday, when asked by reporters about China's proposed national
security law, Mr. Trump said he planned to act this week. ``I think
you'll find it very interesting,'' he said, adding that his response
would come ``very powerfully.''

\includegraphics{https://static01.graylady3jvrrxbe.onion/images/2020/05/27/us/politics/27dc-trumpchina2/merlin_172871856_1c5f4d61-6dad-417b-859d-87923dc3f8ab-articleLarge.jpg?quality=75\&auto=webp\&disable=upscale}

The certification by the State Department is a recommendation on policy
and does not itself catalyze actions immediately. American officials,
including Mr. Trump, will now weigh what steps to take.

The United States is likely to choose specific areas in which to break
off cooperation first with Hong Kong, including trade and law
enforcement.

The president would need to issue an executive order to end the special
relationship entirely, according to people familiar with the
discussions. One possibility is for the United States to take piecemeal
action over the next year before ending the special status if China does
not change course, they said.

``We're not hopeful that Beijing will reverse itself, but that is an
option,''
\href{https://www.state.gov/biographies/david-r-stilwell/}{David R.
Stilwell}, assistant secretary of state for East Asia and the Pacific,
said of the Chinese government's push on the national security law.

Britain handed Hong Kong to China in 1997, after the two nations reached
an agreement on the colony 13 years earlier. In 1992, the United States
passed a law that said the American government would treat a
Beijing-ruled Hong Kong under the same conditions it had applied to the
British colony.

In November, after months of pro-democracy protests and crackdowns by
the police in Hong Kong, Mr. Trump
\href{https://www.lawfareblog.com/hong-kong-human-rights-and-democracy-act-redundant-still-worthwhile}{signed
into law a bipartisan bill} requiring the State Department to provide an
annual certification to Congress to help determine whether to continue
the special relationship with Hong Kong.

That certification depends on a judgment by department officials of
whether China was ceding enough autonomy to Hong Kong.

\href{https://gps.ucsd.edu/faculty-directory/susan-shirk.html}{Susan
Shirk}, a former State Department official now at the University of
California, San Diego, said that given the mandate from Congress, Mr.
Pompeo had no choice on his assessment ``once Beijing blatantly
overruled the Hong Kong legislature with a new law that integrates Hong
Kong'' into the Chinese security state.

``Of course, the big losers will be the people of Hong Kong, not the
politicians in Beijing or Washington who produced this predicament,''
she added.

Mr. Pompeo's announcement is certain to draw condemnation from Beijing,
where the government is holding its annual legislative session this
week. Officials announced details of the proposed law Friday, at the
start of the session.

``If anyone insists on harming China's interests, China is determined to
take all necessary countermeasures,'' Zhao Lijian, a Chinese Foreign
Ministry spokesman, said at a news conference earlier Wednesday in
Beijing. ``The national security law for Hong Kong is purely China's
internal affair that allows no foreign interference.''

Some American business executives are advising the Trump administration
to tread carefully on changing the relationship with Hong Kong.

The U.S. Chamber of Commerce, which represents American companies in
Hong Kong, said in a statement Tuesday that it was ``deeply concerned''
about the proposed national security law. It asked the Chinese
government to ``peacefully de-escalate'' the situation and preserve the
semi-autonomy of the ``one country, two systems'' framework that, under
the 1984 treaty between Beijing and London, is supposed to exist until
2047.

``We likewise urge the Trump administration to continue to prioritize
the maintenance of a positive and constructive relationship between the
United States and Hong Kong,'' the group said.

It added that ``far-reaching changes'' to Hong Kong's status ``in
economic and trade matters would have serious implications for Hong Kong
and for U.S. business, particularly those with business operations
located there who exercise a positive influence in favor of Hong Kong's
core values.''

\href{https://law.hofstra.edu/directory/faculty/fulltime/ku/}{Julian
Ku}, a law professor at Hofstra University, said the Trump
administration had flexibility on which options to exercise.

``I would expect the president would act on some agreements, but not on
others,'' Mr. Ku said. For example, he noted, the administration might
terminate the extradition treaty with Hong Kong, since the national
security law makes fair adjudication less credible, or it could put Hong
Kong under the same controls that limit American technology exports to
China.

Mr. Trump and his aides might also leave visa rules for Hong Kong
residents alone for now, he added. ``So he has a little flexibility
which might allow him to negotiate with China as this process goes
forward,'' Mr. Ku said.

Mark Williams, the chief Asia economist at Capital Economics, said Mr.
Trump's tariffs on imports from mainland China --- which are paid by
American companies --- would not automatically extend to Hong Kong
despite the new State Department assessment. But the cumulative effect
of various actions would erode Hong Kong's status as an international
business center, Mr. Williams wrote in a note to clients.

``The irony is that in punishing Hong Kong, we wind up martyring it
rather than saving it,'' said
\href{https://asiasociety.org/policy-institute/daniel-russel}{Daniel
Russel}, an assistant secretary of state for East Asia and the Pacific
in the Obama administration. As for diplomacy between Washington and
Beijing, he said: ``The brake pads in the relationship have worn very,
very thin. And it's hard to see this confrontation going anywhere except
escalation.''

In Congress, Senator Marco Rubio, Republican of Florida and a sponsor of
the bill on Hong Kong that passed last fall, cheered Mr. Pompeo's
announcement.

``For years, the Chinese government and Communist Party have walked back
on its commitment to ensure autonomy and freedom for Hong Kong,'' Mr.
Rubio said. ``We cannot let Beijing profit from breaking the
Sino-British Joint Declaration and trying to crush the spirit of Hong
Kong's people.''

On another front, the State Department plans to expand the list of
Chinese state-run news organizations operating in the United States on
which it has
\href{https://www.nytimes3xbfgragh.onion/2020/03/26/us/politics/coronavirus-china-spies.html}{imposed
new restrictions}, including foreign employee quotas, American officials
said. And the agency is watching to see if China will retaliate against
American journalists in Hong Kong for the administration's
\href{https://www.nytimes3xbfgragh.onion/2020/05/09/us/politics/china-journalists-us-visa-crackdown.html}{most
recent round of visa restrictions} against Chinese journalists. In
March, China
\href{https://www.nytimes3xbfgragh.onion/2020/03/17/business/media/china-expels-american-journalists.html}{expelled}
American journalists from three news organizations, including The New
York Times.

Michael Crowley and Ana Swanson contributed reporting from Washington,
and Keith Bradsher from Beijing.

Advertisement

\protect\hyperlink{after-bottom}{Continue reading the main story}

\hypertarget{site-index}{%
\subsection{Site Index}\label{site-index}}

\hypertarget{site-information-navigation}{%
\subsection{Site Information
Navigation}\label{site-information-navigation}}

\begin{itemize}
\tightlist
\item
  \href{https://help.nytimes3xbfgragh.onion/hc/en-us/articles/115014792127-Copyright-notice}{©~2020~The
  New York Times Company}
\end{itemize}

\begin{itemize}
\tightlist
\item
  \href{https://www.nytco.com/}{NYTCo}
\item
  \href{https://help.nytimes3xbfgragh.onion/hc/en-us/articles/115015385887-Contact-Us}{Contact
  Us}
\item
  \href{https://www.nytco.com/careers/}{Work with us}
\item
  \href{https://nytmediakit.com/}{Advertise}
\item
  \href{http://www.tbrandstudio.com/}{T Brand Studio}
\item
  \href{https://www.nytimes3xbfgragh.onion/privacy/cookie-policy\#how-do-i-manage-trackers}{Your
  Ad Choices}
\item
  \href{https://www.nytimes3xbfgragh.onion/privacy}{Privacy}
\item
  \href{https://help.nytimes3xbfgragh.onion/hc/en-us/articles/115014893428-Terms-of-service}{Terms
  of Service}
\item
  \href{https://help.nytimes3xbfgragh.onion/hc/en-us/articles/115014893968-Terms-of-sale}{Terms
  of Sale}
\item
  \href{https://spiderbites.nytimes3xbfgragh.onion}{Site Map}
\item
  \href{https://help.nytimes3xbfgragh.onion/hc/en-us}{Help}
\item
  \href{https://www.nytimes3xbfgragh.onion/subscription?campaignId=37WXW}{Subscriptions}
\end{itemize}
