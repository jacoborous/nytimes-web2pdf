Sections

SEARCH

\protect\hyperlink{site-content}{Skip to
content}\protect\hyperlink{site-index}{Skip to site index}

\href{https://www.nytimes3xbfgragh.onion/section/world/canada}{Canada}

\href{https://myaccount.nytimes3xbfgragh.onion/auth/login?response_type=cookie\&client_id=vi}{}

\href{https://www.nytimes3xbfgragh.onion/section/todayspaper}{Today's
Paper}

\href{/section/world/canada}{Canada}\textbar{}Extradition of Huawei
Executive Clears a Major Legal Hurdle in Canada

\url{https://nyti.ms/2XzpVST}

\begin{itemize}
\item
\item
\item
\item
\item
\end{itemize}

Advertisement

\protect\hyperlink{after-top}{Continue reading the main story}

Supported by

\protect\hyperlink{after-sponsor}{Continue reading the main story}

\hypertarget{extradition-of-huawei-executive-clears-a-major-legal-hurdle-in-canada}{%
\section{Extradition of Huawei Executive Clears a Major Legal Hurdle in
Canada}\label{extradition-of-huawei-executive-clears-a-major-legal-hurdle-in-canada}}

A Vancouver court ruled that fraud charges against Meng Wanzhou in the
United States would constitute a crime in Canada, opening the way for
her extradition.

\includegraphics{https://static01.graylady3jvrrxbe.onion/images/2020/05/27/world/27meng-extradition-new/merlin_172889289_c60f01f4-d24b-4954-bab1-b184c00e9b11-articleLarge.jpg?quality=75\&auto=webp\&disable=upscale}

By Tracy Sherlock and
\href{https://www.nytimes3xbfgragh.onion/by/dan-bilefsky}{Dan Bilefsky}

\begin{itemize}
\item
  May 27, 2020
\item
  \begin{itemize}
  \item
  \item
  \item
  \item
  \item
  \end{itemize}
\end{itemize}

\href{https://cn.nytimes3xbfgragh.onion/world/20200528/huawei-extradition-meng-wanzhou/}{阅读简体中文版}\href{https://cn.nytimes3xbfgragh.onion/world/20200528/huawei-extradition-meng-wanzhou/zh-hant/}{閱讀繁體中文版}

VANCOUVER, British Columbia --- The chief financial officer of the
Chinese technology giant Huawei came one step closer to standing trial
in the United States on sweeping fraud charges after a Canadian court
ruled on Wednesday that prosecutors had satisfied a critical legal
requirement for her extradition from Canada.

The executive, Meng Wanzhou, was arrested in Vancouver in December 2018,
at the request of the United States, and
\href{https://www.nytimes3xbfgragh.onion/2019/01/28/us/politics/meng-wanzhou-huawei-iran.html}{indicted
in January 2019.} Her detention set off one of the biggest legal dramas
in recent Canadian memory, its twists and turns parsed on national
television.

Her arrest also thrust Canada into the middle of a diplomatic struggle
between the United States and China --- over trade, theft of technology
secrets and whether Huawei's efforts in helping countries build 5G
next-generation mobile networks present a threat to national security.

And it severely strained Canada's own relations with China. Shortly
after Ms. Meng's arrest, China detained --- in retaliation, some say ---
two Canadians and
\href{https://www.nytimes3xbfgragh.onion/2019/03/04/world/asia/china-canada-michael-kovrig-huawei.html?module=inline}{accused
them of espionage}. They are still in secret jails in China.

That relationship has become more fraught since, with Canadians
criticizing China's handling of the coronavirus pandemic and its human
rights policies. Wednesday's decision is expected to aggravate those
tensions.

Guy Saint-Jacques, a former Canadian ambassador to China, said the
ruling likely presaged ``both sides hardening their stances at a moment
when countries are already questioning China's role in the pandemic.''

On Wednesday, Huawei said that it was ``disappointed'' in the ruling,
and that it expected Ms. Meng would ultimately be proved innocent.

Chinese state media this week signaled there could be a backlash if the
ruling did not go in Ms. Meng's favor.
\href{https://www.globaltimes.cn/content/1189476.shtml}{Global Times}, a
state-owned tabloid with a nationalist bent, warned of ``resentment'' in
China should the judge make a decision that ``panders to the Trump
administration.''

After the decision, Canada's minister of foreign affairs,
\href{https://twitter.com/FP_Champagne/status/1265725645474213889}{François-Philippe
Champagne}, stressed that the Canadian judiciary was independent.

He said that Canada would continue to engage with China, and that its
top priority was the release of the two Canadians --- the former
diplomat Michael Kovrig and the businessman Michael Spavor --- ``who
have been arbitrarily detained for over 500 days.''

Ms. Meng will have another chance to fight for her release in the coming
months at hearings on whether her rights were violated during her
arrest.

While the Meng case has complicated Canadian diplomacy, Wednesday's
decision was a matter of law.

In her decision, Heather Holmes, associate chief justice of the British
Columbia Supreme Court, found that prosecutors had cleared a fundamental
hurdle for Ms. Meng's extradition under Canadian law --- demonstrating
that the conduct she is accused of in the United States, if proved, also
constitutes a crime in Canada.

The legal concept is known as ``double criminality.''

\includegraphics{https://static01.graylady3jvrrxbe.onion/images/2020/05/26/world/00meng-extradition3/00meng-extradition3-articleLarge.jpg?quality=75\&auto=webp\&disable=upscale}

The judge ruled that double criminality was met because the conduct Ms.
Meng is accused of --- ``the making of intentionally false statements''
--- meets the essence of fraud.

In their indictment against Ms. Meng, now 48, United States prosecutors
charged Ms. Meng with fraudulently deceiving four banks into making
transactions to help Huawei evade United States sanctions against Iran.

Her defense team argued that the extradition request did not satisfy the
requirement of double criminality because it was based on the accusation
that U.S. sanctions against Iran had been breached --- sanctions that
Canada no longer has in place.

Prosecutors said Ms. Meng lied to representatives of the bank HSBC in
2013 about Huawei's relationship with Skycom, a company that would clear
transactions between Huawei and HSBC in Iran, by saying Skycom was a
partner, rather than a subsidiary, of Huawei.

They said that misrepresentation amounted to fraud by exposing HSBC to
reputational and economic risk in light of the American sanctions.

``Lying to a bank in order to get banking services that creates a risk
of economic prejudice is fraud,'' said Robert Frater, Canada's chief
general counsel for the Department of Justice,
\href{https://www.nytimes3xbfgragh.onion/2020/01/20/world/canada/meng-wanzhou-huawei-detention-vancouver.html}{arguing
the United States' case} in court earlier this year.

Ms. Meng, the eldest daughter of Huawei's founder, Ren Zhengfei, one of
China's most prominent businessmen, has denied the allegations.

Image

United States prosecutors charged Ms. Meng, now 48, with fraudulently
deceiving four banks, including HSBC.Credit...Lam Yik Fei for The New
York Times

The U.S. indictment charging Ms. Meng also accused Huawei of stealing
trade secrets, and obstructing a criminal investigation.

The United States argues that Huawei's ties to the Chinese government
make it a threat to the national security of countries who adopt its
technology in their next generation mobile networks. Huawei vehemently
denies that.

The Trump administration escalated its campaign against Huawei this
month by
\href{https://www.nytimes3xbfgragh.onion/2020/05/15/business/economy/commerce-department-huawei.html}{restricting
its ability to work} with chip makers that produce many crucial
components in its smartphones and telecom equipment.

Canada itself
\href{https://www.nytimes3xbfgragh.onion/2019/02/27/world/canada/huawei-5g-meng-wanzhou-china.html?searchResultPosition=2}{has
been reviewing} whether it should allow Huawei technology in the
country's 5G network.

Image

The Trump administration has recently escalated its battle with
Huawei.Credit...Lam Yik Fei for The New York Times

Despite the assertion by Mr. Champagne, the foreign minister, that
Canada would continue to engage with China, experts said that China's
behavior was spurring Canadians to rethink how to deal with the economic
superpower.

In addition to detaining the two Canadians, China also restricted
imports of pork, canola oil and other Canadian products after Ms. Meng's
arrest.

David Mulroney, another former Canadian ambassador to China, said
China's handling of the coronavirus pandemic, its infringements of human
rights
\href{https://www.nytimes3xbfgragh.onion/2020/05/21/us/politics/trump-china-hong-kong.html}{in
Hong Kong} and
its\href{https://www.nytimes3xbfgragh.onion/interactive/2019/11/16/world/asia/china-xinjiang-documents.html}{mass
detention of minorities}in Xinjiang had also worsened relations with
Beijing.

``There has been a global awakening prompted by the pandemic that China
is an unreliable partner,'' he said.

Last week, Prime Minister Justin Trudeau said China had linked the
detention of the two Canadians with Ms. Meng's case from the beginning,
and had failed to understand that Canada had an independent judicial
system.

Image

In Vancouver, protesting the detention of two Canadians in
China.Credit...Jason Redmond/Agence France-Presse --- Getty Images

Ms. Meng's defense has filed a separate civil case against Canadian
authorities, arguing that her rights were breached when border officials
questioned her for three hours before making an arrest, seized her
phones, asked for her passcodes and searched her eight pieces of
luggage.

Government prosecutors counter that the border guards had every right to
search her.

In early 2019, Ms. Meng was released
\href{https://www.nytimes3xbfgragh.onion/2019/03/04/world/canada/huawei-canada-meng-wanzhou.html?module=inline}{on
bail} of 10 million Canadian dollars. She has spent the time since
living in two different mansions that her family owns in wealthy areas
of Vancouver.

Most recently, she has lived in a gated \$14 million, seven-bedroom
mansion in the city's exclusive Shaughnessy neighborhood. Until the
pandemic hit, she was able to travel relatively freely around the city
--- shopping and attending musical concerts --- though she had to wear a
GPS tracker on her ankle.

She takes daily painting lessons and her mother has been living with
her.

Last weekend, she appeared at the steps of the imposing modernist
British Columbia Supreme Court, making a
\href{https://www.cbc.ca/news/canada/british-columbia/meng-wanzhou-photographs-posing-court-1.5582689}{thumbs-up
gesture} for photographers.

Many in Canada have noted the contrast between her circumstances and
those of the Canadians imprisoned in China, whose access to lawyers and
their families is severely limited.

Image

A security tent outside the home where Ms. Meng has been living in
Vancouver while awaiting her extradition hearing.Credit...Jackie Dives
for The New York Times

Ms. Meng could appeal a final decision on extradition to the Supreme
Court of Canada, a process that experts say could drag on for years.

After a ruling from the Supreme Court, the case would move to the
political arena, with the justice minister of Canada making the final
decision on whether she must be sent to the United States to stand
trial.

Tracy Sherlock reported from Vancouver, British Columbia, and Dan
Bilefsky from Montreal. Raymond Zhong contributed reporting from Taipei,
Taiwan, and David McCabe from Washington.

Advertisement

\protect\hyperlink{after-bottom}{Continue reading the main story}

\hypertarget{site-index}{%
\subsection{Site Index}\label{site-index}}

\hypertarget{site-information-navigation}{%
\subsection{Site Information
Navigation}\label{site-information-navigation}}

\begin{itemize}
\tightlist
\item
  \href{https://help.nytimes3xbfgragh.onion/hc/en-us/articles/115014792127-Copyright-notice}{©~2020~The
  New York Times Company}
\end{itemize}

\begin{itemize}
\tightlist
\item
  \href{https://www.nytco.com/}{NYTCo}
\item
  \href{https://help.nytimes3xbfgragh.onion/hc/en-us/articles/115015385887-Contact-Us}{Contact
  Us}
\item
  \href{https://www.nytco.com/careers/}{Work with us}
\item
  \href{https://nytmediakit.com/}{Advertise}
\item
  \href{http://www.tbrandstudio.com/}{T Brand Studio}
\item
  \href{https://www.nytimes3xbfgragh.onion/privacy/cookie-policy\#how-do-i-manage-trackers}{Your
  Ad Choices}
\item
  \href{https://www.nytimes3xbfgragh.onion/privacy}{Privacy}
\item
  \href{https://help.nytimes3xbfgragh.onion/hc/en-us/articles/115014893428-Terms-of-service}{Terms
  of Service}
\item
  \href{https://help.nytimes3xbfgragh.onion/hc/en-us/articles/115014893968-Terms-of-sale}{Terms
  of Sale}
\item
  \href{https://spiderbites.nytimes3xbfgragh.onion}{Site Map}
\item
  \href{https://help.nytimes3xbfgragh.onion/hc/en-us}{Help}
\item
  \href{https://www.nytimes3xbfgragh.onion/subscription?campaignId=37WXW}{Subscriptions}
\end{itemize}
