Sections

SEARCH

\protect\hyperlink{site-content}{Skip to
content}\protect\hyperlink{site-index}{Skip to site index}

\href{https://www.nytimes3xbfgragh.onion/section/health}{Health}

\href{https://myaccount.nytimes3xbfgragh.onion/auth/login?response_type=cookie\&client_id=vi}{}

\href{https://www.nytimes3xbfgragh.onion/section/todayspaper}{Today's
Paper}

\href{/section/health}{Health}\textbar{}Moderna Coronavirus Vaccine
Trial Shows Promising Early Results

\url{https://nyti.ms/2X69NIz}

\begin{itemize}
\item
\item
\item
\item
\item
\item
\end{itemize}

\hypertarget{the-coronavirus-outbreak}{%
\subsubsection{\texorpdfstring{\href{https://www.nytimes3xbfgragh.onion/news-event/coronavirus?name=styln-coronavirus-national\&region=TOP_BANNER\&block=storyline_menu_recirc\&action=click\&pgtype=Article\&impression_id=0e25f610-f291-11ea-834c-e760e465928d\&variant=undefined}{The
Coronavirus
Outbreak}}{The Coronavirus Outbreak}}\label{the-coronavirus-outbreak}}

\begin{itemize}
\tightlist
\item
  live\href{https://www.nytimes3xbfgragh.onion/2020/09/09/world/covid-19-coronavirus.html?name=styln-coronavirus-national\&region=TOP_BANNER\&block=storyline_menu_recirc\&action=click\&pgtype=Article\&impression_id=0e25f611-f291-11ea-834c-e760e465928d\&variant=undefined}{Latest
  Updates}
\item
  \href{https://www.nytimes3xbfgragh.onion/interactive/2020/us/coronavirus-us-cases.html?name=styln-coronavirus-national\&region=TOP_BANNER\&block=storyline_menu_recirc\&action=click\&pgtype=Article\&impression_id=0e261d20-f291-11ea-834c-e760e465928d\&variant=undefined}{Maps
  and Cases}
\item
  \href{https://www.nytimes3xbfgragh.onion/interactive/2020/science/coronavirus-vaccine-tracker.html?name=styln-coronavirus-national\&region=TOP_BANNER\&block=storyline_menu_recirc\&action=click\&pgtype=Article\&impression_id=0e261d21-f291-11ea-834c-e760e465928d\&variant=undefined}{Vaccine
  Tracker}
\item
  \href{https://www.nytimes3xbfgragh.onion/2020/09/02/your-money/eviction-moratorium-covid.html?name=styln-coronavirus-national\&region=TOP_BANNER\&block=storyline_menu_recirc\&action=click\&pgtype=Article\&impression_id=0e261d22-f291-11ea-834c-e760e465928d\&variant=undefined}{Eviction
  Moratorium}
\item
  \href{https://www.nytimes3xbfgragh.onion/interactive/2020/09/02/magazine/food-insecurity-hunger-us.html?name=styln-coronavirus-national\&region=TOP_BANNER\&block=storyline_menu_recirc\&action=click\&pgtype=Article\&impression_id=0e261d23-f291-11ea-834c-e760e465928d\&variant=undefined}{American
  Hunger}
\end{itemize}

Advertisement

\protect\hyperlink{after-top}{Continue reading the main story}

Supported by

\protect\hyperlink{after-sponsor}{Continue reading the main story}

\hypertarget{moderna-coronavirus-vaccine-trial-shows-promising-early-results}{%
\section{Moderna Coronavirus Vaccine Trial Shows Promising Early
Results}\label{moderna-coronavirus-vaccine-trial-shows-promising-early-results}}

The company said a test in 8 healthy volunteers found its experimental
vaccine was safe and provoked a strong immune response. It is on an
accelerated timetable to begin larger human trials soon.

\includegraphics{https://static01.graylady3jvrrxbe.onion/images/2020/05/18/world/18virus-moderna/18virus-moderna-articleLarge-v3.jpg?quality=75\&auto=webp\&disable=upscale}

By \href{https://www.nytimes3xbfgragh.onion/by/denise-grady}{Denise
Grady}

\begin{itemize}
\item
  Published May 18, 2020Updated July 27, 2020
\item
  \begin{itemize}
  \item
  \item
  \item
  \item
  \item
  \item
  \end{itemize}
\end{itemize}

The first
\href{https://www.nytimes3xbfgragh.onion/2020/07/27/health/moderna-vaccine-covid.html}{coronavirus
vaccine} to be tested in people appears to be safe and able to stimulate
an immune response against the infection, the manufacturer,
\href{https://www.nytimes3xbfgragh.onion/2020/07/27/health/moderna-vaccine-covid.html}{Moderna},
announced on Monday, offering a glint of hope to a world desperate for
ways to stop the pandemic.

The preliminary findings, in the first eight people who each received
two doses of the experimental
\href{https://www.nytimes3xbfgragh.onion/2020/06/03/science/coronavirus-vaccine-horseshoe-crabs.html}{vaccine},
must now be repeated in far larger tests in hundreds and then thousands
of people, to find out if the vaccine can work in the real world.
Moderna's technology, involving genetic material from the virus called
mRNA, is relatively new and has yet to produce any approved vaccine.

The promising early news sent Moderna's stock soaring by more than 25
percent on Monday afternoon and helped drive Wall Street to its best day
in six weeks. Stocks were also lifted by statements from the Federal
Reserve chair, Jerome H. Powell, that the central bank would continue to
support the economy and markets.

Trading on Monday had all of the characteristics of a rally focused on
prospects for a return to normal: The S\&P 500 rose more than 3 percent;
stock benchmarks in Europe were 4 percent to 6 percent higher; and oil
prices also jumped. Among the best performers in the S\&P 500 were
travel-related companies, like United Airlines, Expedia Group and
Marriott International.

\emph{{[}}\href{https://www.nytimes3xbfgragh.onion/interactive/2020/science/coronavirus-vaccine-tracker.html}{\emph{Follow
our Live Coronavirus Vaccine Tracker}}\emph{.{]}}

With the weather warming and state after state starting to lift lockdown
restrictions, Americans are eager to regain their freedom to shop, go to
the beach and enjoy bars and restaurants. Still, more than 1,000 people
died most days last week in the United States.

Vaccines are now seen as the best and perhaps only hope of stopping or
even slowing a disease that has
\href{https://www.nytimes3xbfgragh.onion/interactive/2020/world/coronavirus-maps.html}{sickened
nearly five million people worldwide, killed 315,000} and locked down
entire countries, paralyzing their economies.

\hypertarget{latest-updates-the-coronavirus-outbreak}{%
\section{\texorpdfstring{\href{https://www.nytimes3xbfgragh.onion/2020/09/09/world/covid-19-coronavirus.html?action=click\&pgtype=Article\&state=default\&region=MAIN_CONTENT_1\&context=storylines_live_updates}{Latest
Updates: The Coronavirus
Outbreak}}{Latest Updates: The Coronavirus Outbreak}}\label{latest-updates-the-coronavirus-outbreak}}

Updated 2020-09-09T11:21:23.828Z

\begin{itemize}
\tightlist
\item
  \href{https://www.nytimes3xbfgragh.onion/2020/09/09/world/covid-19-coronavirus.html?action=click\&pgtype=Article\&state=default\&region=MAIN_CONTENT_1\&context=storylines_live_updates\#link-70cea8bb}{As
  drugmakers pledge to thoroughly vet a vaccine, one company pauses its
  trials for a safety review.}
\item
  \href{https://www.nytimes3xbfgragh.onion/2020/09/09/world/covid-19-coronavirus.html?action=click\&pgtype=Article\&state=default\&region=MAIN_CONTENT_1\&context=storylines_live_updates\#link-780eaa2f}{Britain
  is expected to ban gatherings of more than six people.}
\item
  \href{https://www.nytimes3xbfgragh.onion/2020/09/09/world/covid-19-coronavirus.html?action=click\&pgtype=Article\&state=default\&region=MAIN_CONTENT_1\&context=storylines_live_updates\#link-11cec4c0}{Quarantine
  breakdowns at colleges in the U.S. are leaving some at risk.}
\end{itemize}

\href{https://www.nytimes3xbfgragh.onion/2020/09/09/world/covid-19-coronavirus.html?action=click\&pgtype=Article\&state=default\&region=MAIN_CONTENT_1\&context=storylines_live_updates}{See
more updates}

More live coverage:
\href{https://www.nytimes3xbfgragh.onion/live/2020/09/09/business/stock-market-today-coronavirus?action=click\&pgtype=Article\&state=default\&region=MAIN_CONTENT_1\&context=storylines_live_updates}{Markets}

Dozens of companies and universities are rushing to create coronavirus
vaccines, and human trials have already started for several
manufacturers, including
\href{https://www.nytimes3xbfgragh.onion/2020/07/27/health/moderna-vaccine-covid.html}{Pfizer}
and its German partner BioNTech, the
\href{https://www.nytimes3xbfgragh.onion/2020/05/04/business/coronavirus-china-vaccine.html}{Chinese
company CanSino} and the
\href{https://www.nytimes3xbfgragh.onion/2020/04/27/world/europe/coronavirus-vaccine-update-oxford.html}{University
of Oxford, which is working with AstraZeneca}.

Experts agree that it is essential to develop multiple vaccines, because
the urgent global need for
\href{https://www.nytimes3xbfgragh.onion/2020/05/01/health/coronavirus-vaccine-supplies.html}{billions
of doses will far outstrip the production capacity} of any one
manufacturer. But there is widespread concern among scientists that
haste could compromise safety, resulting in a vaccine that does not work
or even harms patients.

The potential strength of Moderna's mRNA approach to vaccine making is
that it uses a genetic framework that can be quickly adapted for each
new viral threat. The company has said that it is proceeding on an
accelerated timetable, with a second phase of tests involving 600 people
to begin soon, and a third phase to begin in July involving thousands of
healthy people. The
\href{https://www.nytimes3xbfgragh.onion/2020/05/07/health/coronavirus-vaccine-moderna.html}{Food
and Drug Administration gave Moderna the go-ahead} this month for the
second phase.

If those trials go well, some doses of a vaccine could become available
for widespread use by the end of this year or early 2021, Dr. Tal Zaks,
Moderna's chief medical officer, said in an interview. ``We're doing our
best to make it as many millions as possible.''

President Trump said last week that a vaccine would be available before
the end of this year. His prediction was supported by Moncef Slaoui, the
newly appointed leader of Operation Warp Speed, the administration's
effort to speed vaccine development. At a briefing last week, Mr.
Slaoui, a former member of Moderna's board of directors who resigned
when he took up his new government post, said he had seen preliminary
research data that convinced him that a vaccine could be created by the
end of the year. He did not identify the data.

Also on Monday, Caitlin Oakley, a spokeswoman for the Department of
Health and Human Services, confirmed that Mr. Slaoui would divest his
Moderna stock options, valued at about \$10 million, on Tuesday morning.
Ms. Oakley added that Mr. Slaoui would donate to cancer research the
increased value his shares had accrued from last Thursday until
Tuesday's sale. The share price closed at \$80 on Monday, up from
\$64.56 last Thursday, adding \$2.4 million to the value of his options.

At a round table with restaurant executives at the White House on
Monday, Mr. Trump said, ``This was a very big day, cure wise and vaccine
wise,'' and noted that the markets were lifted by drug news.

Moderna produced the vaccine in collaboration with the National
Institute of Allergy and Infectious Diseases, the institute that is
headed by Dr. Anthony Fauci and has been leading the clinical trials.
Part of the National Institutes of Health, the agency is involved in
research on other experimental coronavirus vaccines. Moderna and Johnson
\& Johnson have each received roughly half a billion dollars from the
U.S. government, to speed development of a vaccine.

The people vaccinated in Moderna's Phase 1 study described on Monday
were healthy volunteers ages 18 to 55. Their immune systems made
antibodies that were then tested in infected cells in the lab, and were
able to stop the virus from replicating --- the key requirement for an
effective vaccine. The levels of those so-called neutralizing antibodies
matched or exceeded the levels found in patients who had recovered after
contracting the virus in the community.

Dr. Mark J. Mulligan, director of the N.Y.U. Langone Vaccine Center,
called the Moderna findings ``very encouraging.'' He added, ``It's a
small number of participants, but it appears to be a really good
start.'' Dr. Mulligan was not involved in the early testing but expected
to participate in a later phase of the Moderna vaccine research.

Moderna's early stage of testing, Phase 1, is continuing, Two more age
groups --- 55 to 70 and 71 and over --- are now being enrolled to test
the vaccine. The company did not mention plans to include children in
its studies and did not respond to an inquiry about it in time for
publication. But Dr. Mulligan said that tests in children were often
delayed until a vaccine was shown to be safe in young adults.

The actual data from the preliminary tests has not been published or
shared publicly, but has been submitted to the Food and Drug
Administration, which does not comment on trials still in progress. The
company said it hoped to make data publicly available this summer.

\href{https://www.nytimes3xbfgragh.onion/news-event/coronavirus?action=click\&pgtype=Article\&state=default\&region=MAIN_CONTENT_3\&context=storylines_faq}{}

\hypertarget{the-coronavirus-outbreak-}{%
\subsubsection{The Coronavirus Outbreak
›}\label{the-coronavirus-outbreak-}}

\hypertarget{frequently-asked-questions}{%
\paragraph{Frequently Asked
Questions}\label{frequently-asked-questions}}

Updated September 4, 2020

\begin{itemize}
\item ~
  \hypertarget{what-are-the-symptoms-of-coronavirus}{%
  \paragraph{What are the symptoms of
  coronavirus?}\label{what-are-the-symptoms-of-coronavirus}}

  \begin{itemize}
  \tightlist
  \item
    In the beginning, the coronavirus
    \href{https://www.nytimes3xbfgragh.onion/article/coronavirus-facts-history.html?action=click\&pgtype=Article\&state=default\&region=MAIN_CONTENT_3\&context=storylines_faq\#link-6817bab5}{seemed
    like it was primarily a respiratory illness}~--- many patients had
    fever and chills, were weak and tired, and coughed a lot, though
    some people don't show many symptoms at all. Those who seemed
    sickest had pneumonia or acute respiratory distress syndrome and
    received supplemental oxygen. By now, doctors have identified many
    more symptoms and syndromes. In April,
    \href{https://www.nytimes3xbfgragh.onion/2020/04/27/health/coronavirus-symptoms-cdc.html?action=click\&pgtype=Article\&state=default\&region=MAIN_CONTENT_3\&context=storylines_faq}{the
    C.D.C. added to the list of early signs}~sore throat, fever, chills
    and muscle aches. Gastrointestinal upset, such as diarrhea and
    nausea, has also been observed. Another telltale sign of infection
    may be a sudden, profound diminution of one's
    \href{https://www.nytimes3xbfgragh.onion/2020/03/22/health/coronavirus-symptoms-smell-taste.html?action=click\&pgtype=Article\&state=default\&region=MAIN_CONTENT_3\&context=storylines_faq}{sense
    of smell and taste.}~Teenagers and young adults in some cases have
    developed painful red and purple lesions on their fingers and toes
    --- nicknamed ``Covid toe'' --- but few other serious symptoms.
  \end{itemize}
\item ~
  \hypertarget{why-is-it-safer-to-spend-time-together-outside}{%
  \paragraph{Why is it safer to spend time together
  outside?}\label{why-is-it-safer-to-spend-time-together-outside}}

  \begin{itemize}
  \tightlist
  \item
    \href{https://www.nytimes3xbfgragh.onion/2020/05/15/us/coronavirus-what-to-do-outside.html?action=click\&pgtype=Article\&state=default\&region=MAIN_CONTENT_3\&context=storylines_faq}{Outdoor
    gatherings}~lower risk because wind disperses viral droplets, and
    sunlight can kill some of the virus. Open spaces prevent the virus
    from building up in concentrated amounts and being inhaled, which
    can happen when infected people exhale in a confined space for long
    stretches of time, said Dr. Julian W. Tang, a virologist at the
    University of Leicester.
  \end{itemize}
\item ~
  \hypertarget{why-does-standing-six-feet-away-from-others-help}{%
  \paragraph{Why does standing six feet away from others
  help?}\label{why-does-standing-six-feet-away-from-others-help}}

  \begin{itemize}
  \tightlist
  \item
    The coronavirus spreads primarily through droplets from your mouth
    and nose, especially when you cough or sneeze. The C.D.C., one of
    the organizations using that measure,
    \href{https://www.nytimes3xbfgragh.onion/2020/04/14/health/coronavirus-six-feet.html?action=click\&pgtype=Article\&state=default\&region=MAIN_CONTENT_3\&context=storylines_faq}{bases
    its recommendation of six feet}~on the idea that most large droplets
    that people expel when they cough or sneeze will fall to the ground
    within six feet. But six feet has never been a magic number that
    guarantees complete protection. Sneezes, for instance, can launch
    droplets a lot farther than six feet,
    \href{https://jamanetwork.com/journals/jama/fullarticle/2763852}{according
    to a recent study}. It's a rule of thumb: You should be safest
    standing six feet apart outside, especially when it's windy. But
    keep a mask on at all times, even when you think you're far enough
    apart.
  \end{itemize}
\item ~
  \hypertarget{i-have-antibodies-am-i-now-immune}{%
  \paragraph{I have antibodies. Am I now
  immune?}\label{i-have-antibodies-am-i-now-immune}}

  \begin{itemize}
  \tightlist
  \item
    As of right
    now,\href{https://www.nytimes3xbfgragh.onion/2020/07/22/health/covid-antibodies-herd-immunity.html?action=click\&pgtype=Article\&state=default\&region=MAIN_CONTENT_3\&context=storylines_faq}{~that
    seems likely, for at least several months.}~There have been
    frightening accounts of people suffering what seems to be a second
    bout of Covid-19. But experts say these patients may have a
    drawn-out course of infection, with the virus taking a slow toll
    weeks to months after initial exposure.~People infected with the
    coronavirus typically
    \href{https://www.nature.com/articles/s41586-020-2456-9}{produce}~immune
    molecules called antibodies, which are
    \href{https://www.nytimes3xbfgragh.onion/2020/05/07/health/coronavirus-antibody-prevalence.html?action=click\&pgtype=Article\&state=default\&region=MAIN_CONTENT_3\&context=storylines_faq}{protective
    proteins made in response to an
    infection}\href{https://www.nytimes3xbfgragh.onion/2020/05/07/health/coronavirus-antibody-prevalence.html?action=click\&pgtype=Article\&state=default\&region=MAIN_CONTENT_3\&context=storylines_faq}{.
    These antibodies may}~last in the body
    \href{https://www.nature.com/articles/s41591-020-0965-6}{only two to
    three months}, which may seem worrisome, but that's~perfectly normal
    after an acute infection subsides, said Dr. Michael Mina, an
    immunologist at Harvard University. It may be possible to get the
    coronavirus again, but it's highly unlikely that it would be
    possible in a short window of time from initial infection or make
    people sicker the second time.
  \end{itemize}
\item ~
  \hypertarget{what-are-my-rights-if-i-am-worried-about-going-back-to-work}{%
  \paragraph{What are my rights if I am worried about going back to
  work?}\label{what-are-my-rights-if-i-am-worried-about-going-back-to-work}}

  \begin{itemize}
  \tightlist
  \item
    Employers have to provide
    \href{https://www.osha.gov/SLTC/covid-19/standards.html}{a safe
    workplace}~with policies that protect everyone equally.
    \href{https://www.nytimes3xbfgragh.onion/article/coronavirus-money-unemployment.html?action=click\&pgtype=Article\&state=default\&region=MAIN_CONTENT_3\&context=storylines_faq}{And
    if one of your co-workers tests positive for the coronavirus, the
    C.D.C.}~has said that
    \href{https://www.cdc.gov/coronavirus/2019-ncov/community/guidance-business-response.html}{employers
    should tell their employees}~-\/- without giving you the sick
    employee's name -\/- that they may have been exposed to the virus.
  \end{itemize}
\end{itemize}

Two shots, four weeks apart, are likely to be needed, meaning that
however many doses are produced, only half that number of people can be
vaccinated.

Moderna said that additional tests in mice that were vaccinated and then
infected found that the vaccine could prevent the virus from replicating
in their lungs, and that the animals had levels of neutralizing
antibodies comparable to those in the people who had received the
vaccine.

Three doses of the vaccine were tested: low, medium and high. These
initial results are based on tests of the low and medium doses. The only
adverse effects at those doses were redness and soreness in one
patient's arm where the shot was given.

But at the highest dose, three patients had fever, muscle pains and
headaches, Dr. Zaks said, adding that the symptoms went away after a
day.

But the high dose is being eliminated from future studies, not so much
because of the side effects, but because the lower doses appear to work
so well that the high dose is not needed.

``The lower the dose, the more vaccine we'll be able to make,'' Dr. Zaks
said.

He added, ``Demand is going to far outstrip supply so I think there is
an ethical obligation to go with the lowest dose you can so you can make
as much vaccine as possible.''

The company, based in Cambridge, Mass., has a vaccine-making facility
nearby in Norwood, and recently announced a 10-year collaboration with
the Swiss contract drugmaker Lonza to manufacture
\href{https://www.cnbc.com/2020/05/01/moderna-and-switzerlands-lonza-strike-a-deal-on-a-potential-coronavirus-vaccine.html}{up
to one billion doses a year} for worldwide distribution, if the vaccine
proves successful. Dr. Zaks said Moderna was also working with other
vaccine makers in the United States to scale up production.

Moderna uses genetic material --- messenger RNA --- to make vaccines,
and the company has nine others in various stages of development,
including several for viruses that cause respiratory illnesses. But no
vaccine made with this technology has yet reached the market.

Work on the new coronavirus started in January, as soon as Chinese
scientists posted its genetic sequence on the internet. Researchers at
Moderna and the National Institute of Allergy and Infectious Diseases
identified part of the sequence that codes for
\href{https://www.nytimes3xbfgragh.onion/interactive/2020/03/11/science/how-coronavirus-hijacks-your-cells.html}{a
spikelike protein on the surface of the virus} that attaches to human
cells, helping the virus to invade them.

The idea behind Moderna's vaccine is to inject the mRNA for part of the
spike protein and have it slip into the cells of a healthy person, which
then follow its instruction and crank out the viral protein. That
protein should act as a red flag for the immune system, stimulating it
to produce antibodies that will prevent infection by blocking the action
of the spike if the person is exposed to the virus.

``The new technologies for genetic immunization are rapid and produce a
product that is highly potent at producing immune responses,'' Dr.
Mulligan said. ``Today's RNA results confirm that there is great
potential.''

Annie Karni and Sheila Kaplan contributed reporting from Washington.

Advertisement

\protect\hyperlink{after-bottom}{Continue reading the main story}

\hypertarget{site-index}{%
\subsection{Site Index}\label{site-index}}

\hypertarget{site-information-navigation}{%
\subsection{Site Information
Navigation}\label{site-information-navigation}}

\begin{itemize}
\tightlist
\item
  \href{https://help.nytimes3xbfgragh.onion/hc/en-us/articles/115014792127-Copyright-notice}{©~2020~The
  New York Times Company}
\end{itemize}

\begin{itemize}
\tightlist
\item
  \href{https://www.nytco.com/}{NYTCo}
\item
  \href{https://help.nytimes3xbfgragh.onion/hc/en-us/articles/115015385887-Contact-Us}{Contact
  Us}
\item
  \href{https://www.nytco.com/careers/}{Work with us}
\item
  \href{https://nytmediakit.com/}{Advertise}
\item
  \href{http://www.tbrandstudio.com/}{T Brand Studio}
\item
  \href{https://www.nytimes3xbfgragh.onion/privacy/cookie-policy\#how-do-i-manage-trackers}{Your
  Ad Choices}
\item
  \href{https://www.nytimes3xbfgragh.onion/privacy}{Privacy}
\item
  \href{https://help.nytimes3xbfgragh.onion/hc/en-us/articles/115014893428-Terms-of-service}{Terms
  of Service}
\item
  \href{https://help.nytimes3xbfgragh.onion/hc/en-us/articles/115014893968-Terms-of-sale}{Terms
  of Sale}
\item
  \href{https://spiderbites.nytimes3xbfgragh.onion}{Site Map}
\item
  \href{https://help.nytimes3xbfgragh.onion/hc/en-us}{Help}
\item
  \href{https://www.nytimes3xbfgragh.onion/subscription?campaignId=37WXW}{Subscriptions}
\end{itemize}
