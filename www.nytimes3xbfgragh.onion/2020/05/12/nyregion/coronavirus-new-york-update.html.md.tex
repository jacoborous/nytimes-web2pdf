Sections

SEARCH

\protect\hyperlink{site-content}{Skip to
content}\protect\hyperlink{site-index}{Skip to site index}

\href{https://www.nytimes3xbfgragh.onion/section/nyregion}{New York}

\href{https://myaccount.nytimes3xbfgragh.onion/auth/login?response_type=cookie\&client_id=vi}{}

\href{https://www.nytimes3xbfgragh.onion/section/todayspaper}{Today's
Paper}

\href{/section/nyregion}{New York}\textbar{}About 100 N.Y. Children
Treated for Illness Tied to Virus

\url{https://nyti.ms/2SZNfrz}

\begin{itemize}
\item
\item
\item
\item
\item
\item
\end{itemize}

\hypertarget{the-coronavirus-outbreak}{%
\subsubsection{\texorpdfstring{\href{https://www.nytimes3xbfgragh.onion/news-event/coronavirus?name=styln-coronavirus-national\&region=TOP_BANNER\&block=storyline_menu_recirc\&action=click\&pgtype=Article\&impression_id=66c6f1a0-f1da-11ea-8359-5d2adecad97a\&variant=undefined}{The
Coronavirus
Outbreak}}{The Coronavirus Outbreak}}\label{the-coronavirus-outbreak}}

\begin{itemize}
\tightlist
\item
  live\href{https://www.nytimes3xbfgragh.onion/2020/09/08/world/covid-19-coronavirus.html?name=styln-coronavirus-national\&region=TOP_BANNER\&block=storyline_menu_recirc\&action=click\&pgtype=Article\&impression_id=66c6f1a1-f1da-11ea-8359-5d2adecad97a\&variant=undefined}{Latest
  Updates}
\item
  \href{https://www.nytimes3xbfgragh.onion/interactive/2020/us/coronavirus-us-cases.html?name=styln-coronavirus-national\&region=TOP_BANNER\&block=storyline_menu_recirc\&action=click\&pgtype=Article\&impression_id=66c6f1a2-f1da-11ea-8359-5d2adecad97a\&variant=undefined}{Maps
  and Cases}
\item
  \href{https://www.nytimes3xbfgragh.onion/interactive/2020/science/coronavirus-vaccine-tracker.html?name=styln-coronavirus-national\&region=TOP_BANNER\&block=storyline_menu_recirc\&action=click\&pgtype=Article\&impression_id=66c718b0-f1da-11ea-8359-5d2adecad97a\&variant=undefined}{Vaccine
  Tracker}
\item
  \href{https://www.nytimes3xbfgragh.onion/2020/09/02/your-money/eviction-moratorium-covid.html?name=styln-coronavirus-national\&region=TOP_BANNER\&block=storyline_menu_recirc\&action=click\&pgtype=Article\&impression_id=66c718b1-f1da-11ea-8359-5d2adecad97a\&variant=undefined}{Eviction
  Moratorium}
\item
  \href{https://www.nytimes3xbfgragh.onion/interactive/2020/09/02/magazine/food-insecurity-hunger-us.html?name=styln-coronavirus-national\&region=TOP_BANNER\&block=storyline_menu_recirc\&action=click\&pgtype=Article\&impression_id=66c718b2-f1da-11ea-8359-5d2adecad97a\&variant=undefined}{American
  Hunger}
\end{itemize}

Advertisement

\protect\hyperlink{after-top}{Continue reading the main story}

Supported by

\protect\hyperlink{after-sponsor}{Continue reading the main story}

\hypertarget{about-100-ny-children-treated-for-illness-tied-to-virus}{%
\section{About 100 N.Y. Children Treated for Illness Tied to
Virus}\label{about-100-ny-children-treated-for-illness-tied-to-virus}}

``This is a truly disturbing situation,'' Governor Cuomo said of the
rising number of cases of the mysterious inflammatory syndrome.

\begin{itemize}
\item
  May 12, 2020
\item
  \begin{itemize}
  \item
  \item
  \item
  \item
  \item
  \item
  \end{itemize}
\end{itemize}

\hypertarget{heres-what-you-need-to-know}{%
\subsubsection{Here's what you need to
know:}\label{heres-what-you-need-to-know}}

\begin{itemize}
\tightlist
\item
  \protect\hyperlink{link-28c645f6}{About 100 children in N.Y. are
  suspected of having a rare illness tied to the virus.}
\item
  \protect\hyperlink{link-6283b16a}{With parts of N.Y. set to reopen,
  Cuomo again pushes for federal aid.}
\item
  \protect\hyperlink{link-59529e06}{N.J's governor outlined plans for
  contact tracing and more testing.}
\item
  \protect\hyperlink{link-f233ab6}{Lin-Manuel Miranda will help woo
  tourists back to N.Y.C.}
\item
  \protect\hyperlink{link-6f64d26f}{N.Y.C. is expanding testing and
  tracing, but a limited reopening is weeks away.}
\end{itemize}

Cases and deaths in New York State

0

5,000

10,000 cases

March

April

May

June

July

Aug.

Sept.

New cases

7-day average

Total cases

444,751

Deaths

32,600

Includes confirmed and probable cases where available

See maps of the coronavirus outbreak in New York »

\includegraphics{https://static01.graylady3jvrrxbe.onion/images/2020/05/12/nyregion/12nyvirus-briefing01/merlin_172401906_914f9404-e30e-410e-adbb-dd8c3104f40e-articleLarge.jpg?quality=75\&auto=webp\&disable=upscale}

\hypertarget{about-100-children-in-ny-are-suspected-of-having-a-rare-illness-tied-to-the-virus}{%
\subsection{About 100 children in N.Y. are suspected of having a rare
illness tied to the
virus.}\label{about-100-children-in-ny-are-suspected-of-having-a-rare-illness-tied-to-the-virus}}

New York State health officials are investigating about 100 cases of a
rare and dangerous inflammatory syndrome that afflicts children and
appears to be connected to the coronavirus, Gov. Andrew M. Cuomo said on
Tuesday.

So far, three deaths in the state have been linked to the illness, which
is known as
\href{https://www.nytimes3xbfgragh.onion/article/kawasaki-disease-coronavirus-children.html?module=inline}{pediatric
multisystem inflammatory syndrome} and causes life-threatening
inflammation in critical organs, Mr. Cuomo said.

More than half of the state's pediatric inflammatory syndrome cases ---
57 percent --- involved children ages 5 to 14.

Earlier in the day, Mayor Bill de Blasio said that 52 cases of the
syndrome, which has symptoms that overlap with those of toxic shock or
\href{https://www.heart.org/en/health-topics/kawasaki-disease}{Kawasaki
disease,} had been reported in New York City, and that 10 potential
cases were being evaluated.

The dead included a 5-year-old boy, who
\href{https://www.nytimes3xbfgragh.onion/2020/05/08/nyregion/child-dead-new-virus-kawasaki.html?module=inline}{died
last week} in New York City; a 7-year-old boy in Westchester County and
an 18-year-old girl on Long Island.

``This is a truly disturbing situation,'' Mr. Cuomo said at his daily
news briefing. ``And I know parents around the state and around the
country are very concerned about this, and they should be.''

Hospitals across the state should make it a priority to test any
children displaying the syndrome's symptoms for the coronavirus, the
governor said.

Mr. Cuomo's announcement came as he reported 195 more virus-related
deaths in the state, an increase from Monday's total but the second
consecutive day that the toll was under 200.

The pediatric illness
\href{https://www.nytimes3xbfgragh.onion/2020/05/05/nyregion/kawasaki-disease-coronavirus.html?module=inline}{began
to appear in the region} in recent weeks, and doctors and researchers
are still investigating how and why it affects children.

Connecticut reported its first cases of the syndrome on Monday. As of
Tuesday, six children in the state were being treated for the ailment,
officials said.

Gov. Ned Lamont announced three of the Connecticut cases at a briefing
on Monday.

``I think right now it's a very, very tiny risk of infection,'' he said.
``It was not really ever detected in Asia, which, I don't quite know
what that implies.''

Three other children were being treated for the syndrome at the
Connecticut Children's Medical Center in Hartford, a spokeswoman, Monica
Buchanan, said on Tuesday. Two of the three were confirmed to have the
illness, Ms. Buchanan said.

New Jersey health officials said on Tuesday that they were investigating
10 potential cases of the syndrome, none of which were fatal.

\hypertarget{with-parts-of-ny-set-to-reopen-cuomo-again-pushes-for-federal-aid}{%
\subsection{With parts of N.Y. set to reopen, Cuomo again pushes for
federal
aid.}\label{with-parts-of-ny-set-to-reopen-cuomo-again-pushes-for-federal-aid}}

Image

Looking south on Second Avenue in Manhattan on Monday
evening.~~Credit...Dave Sanders for The New York Times

With New York making steady progress in its battle against the virus and
three upstate regions poised to start a gradual reopening by this
weekend, Mr. Cuomo on Tuesday reiterated the importance of federal aid
as the state charts its recovery.

The number of people hospitalized in New York continued to decrease, Mr.
Cuomo said, one of the key metrics that officials are monitoring in
assessing whether the outbreak's severity is waning.

\hypertarget{latest-updates-the-coronavirus-outbreak}{%
\section{\texorpdfstring{\href{https://www.nytimes3xbfgragh.onion/2020/09/08/world/covid-19-coronavirus.html?action=click\&pgtype=Article\&state=default\&region=MAIN_CONTENT_1\&context=storylines_live_updates}{Latest
Updates: The Coronavirus
Outbreak}}{Latest Updates: The Coronavirus Outbreak}}\label{latest-updates-the-coronavirus-outbreak}}

Updated 2020-09-08T13:48:00.364Z

\begin{itemize}
\tightlist
\item
  \href{https://www.nytimes3xbfgragh.onion/2020/09/08/world/covid-19-coronavirus.html?action=click\&pgtype=Article\&state=default\&region=MAIN_CONTENT_1\&context=storylines_live_updates\#link-547feae1}{Senate
  Republicans plan to move forward with a scaled-back stimulus package.}
\item
  \href{https://www.nytimes3xbfgragh.onion/2020/09/08/world/covid-19-coronavirus.html?action=click\&pgtype=Article\&state=default\&region=MAIN_CONTENT_1\&context=storylines_live_updates\#link-679303d7}{Nine
  drugmakers pledge to thoroughly vet any coronavirus vaccine.}
\item
  \href{https://www.nytimes3xbfgragh.onion/2020/09/08/world/covid-19-coronavirus.html?action=click\&pgtype=Article\&state=default\&region=MAIN_CONTENT_1\&context=storylines_live_updates\#link-1c973131}{`The
  lockdown killed my father': Farmer suicides add to India's virus
  misery.}
\end{itemize}

\href{https://www.nytimes3xbfgragh.onion/2020/09/08/world/covid-19-coronavirus.html?action=click\&pgtype=Article\&state=default\&region=MAIN_CONTENT_1\&context=storylines_live_updates}{See
more updates}

More live coverage:
\href{https://www.nytimes3xbfgragh.onion/live/2020/09/08/business/stock-market-today-coronavirus?action=click\&pgtype=Article\&state=default\&region=MAIN_CONTENT_1\&context=storylines_live_updates}{Markets}

The number of new daily hospitalizations has fallen close to where it
was on March 19, just before Mr. Cuomo issued executive orders shutting
down much of the state.

``We're making real progress, there's no doubt,'' Mr. Cuomo said. ``But
there's also no doubt that it's no time to get cocky, no time to get
arrogant.''

While sounding that warning, Mr. Cuomo urged lawmakers in Washington to
give state and local governments whose budgets have been ravaged by the
pandemic the financial help they need to rebound.

``To get this economy up and running, we're going to need an intelligent
stimulus bill,'' Mr. Cuomo said.

\includegraphics{https://static01.graylady3jvrrxbe.onion/images/2020/05/12/nyregion/12vid-cuomo-still/12vid-cuomo-still-videoSixteenByNine3000.jpg}

New York state needs an estimated \$61 billion in federal support to
avoid enacting 20 percent cuts to schools, local governments and
hospitals, Mr. Cuomo said.

He also said it would be impossible for New York to resume business as
normal without the money it needs to develop a sophisticated testing and
contact tracing apparatus.

It is unclear whether Congress will give Mr. Cuomo the help he is
seeking. Like President Trump, Senator Mitch McConnell, the Republican
majority leader, said he last month that he did not support what he has
labeled a blue state bailout.

Mr. Cuomo called Mr. McConnell's characterization ``one of the really
dumb ideas of all time.''

\hypertarget{njs-governor-outlined-plans-for-contact-tracing-and-more-testing}{%
\subsection{N.J's governor outlined plans for contact tracing and more
testing.}\label{njs-governor-outlined-plans-for-contact-tracing-and-more-testing}}

Image

A drive-through coronavirus testing center at Bergen County Community
College in Paramus, N.J., last month.~Credit...Ryan Christopher Jones
for The New York Times

Gov. Philip D. Murphy of New Jersey on Tuesday outlined plans for the
testing and contact tracing of the virus that he said would be critical
to reopening the state's economy.

Still, Mr. Murphy made the case that New Jersey --- which, with New
York, has been at heart of the pandemic in the United States --- is now
more affected by outbreak than other states. New Jersey, he said, had
overtaken New York and Connecticut in the rate of new infections and
deaths.

``There are still thousands in our hospitals, and sadly an untold number
more will perish,'' Mr. Murphy said, while noting that the number of
hospitalizations, deaths and new cases had plunged since their mid-April
peak.

To continue to beat back the outbreak, New Jersey officials said they
planned to test up to 20,000 people a day by the end of May. The state
will also deploy hundreds of contact tracers to determine who has had
close interactions with a sick person, Mr. Murphy said.

The state's goal, the governor said, was to recruit a racially diverse
group of tracers who speak various languages and identify closely with
the communities where they will work. The job pays around \$25 an hour,
he said.

The drop in the number of new virus cases means that the state can
consider a limited reopening, Mr. Murphy said, but he warned impatient
residents about the risk of loosening restrictions too soon. After
closing parks and golf courses in early April,
\href{https://www.nytimes3xbfgragh.onion/2020/05/02/nyregion/weather-parks-nyc-nj-coronavirus.html}{the
state reopened them on May 2}. Mr. Murphy did not say which businesses
may be the first to reopen.

Also on Tuesday, Mr. Murphy announced 198 new deaths --- 139 more than
were reported the day before --- for a total of 9,508. About half of the
fatalities involved nursing home residents. The daily report of new
deaths in New Jersey may include deaths that occurred weeks ago and were
only recently confirmed.

``Those numbers don't lie,'' Mr. Murphy said. ``We are still the most
impacted state in America.''

\hypertarget{lin-manuel-miranda-will-help-woo-tourists-back-to-nyc}{%
\subsection{Lin-Manuel Miranda will help woo tourists back to
N.Y.C.}\label{lin-manuel-miranda-will-help-woo-tourists-back-to-nyc}}

Image

Duffy Square, in Midtown Manhattan, used to be thronged with Broadway
ticketbuyers. No more.Credit...Juan Arredondo for The New York Times

The puzzle of how to revive New York City's tourist trade is so vexing
that city officials are pulling together a group of industry experts ---
and one of the biggest names on Broadway --- to try to solve it.

On Tuesday, the city's tourism agency, NYC \& Company, said it was
establishing the Coalition for NYC Hospitality \& Tourism Recovery.
Among the group's leaders: Lin-Manuel Miranda, the composer, lyricist
and actor who created the musical ``Hamilton.''

The coalition's task is to come up with a plan for wooing people back to
the city once it starts to emerge from the coronavirus pandemic, a
chapter that appears to be months off at least after the Broadway League
said on Tuesday that
\href{https://www.nytimes3xbfgragh.onion/2020/05/12/theater/broadway-coronavirus.html}{its
members were canceling shows through Sept 6}.

``It is time to consider how we can begin to reopen our doors and safely
reconnect with our city and with each other, and with the visitors who
will one day again flock to New York,'' said Charles Flateman, NYC \&
Company's chairman and executive vice president of the Shubert
Organization.

\href{https://www.nytimes3xbfgragh.onion/news-event/coronavirus?action=click\&pgtype=Article\&state=default\&region=MAIN_CONTENT_3\&context=storylines_faq}{}

\hypertarget{the-coronavirus-outbreak-}{%
\subsubsection{The Coronavirus Outbreak
›}\label{the-coronavirus-outbreak-}}

\hypertarget{frequently-asked-questions}{%
\paragraph{Frequently Asked
Questions}\label{frequently-asked-questions}}

Updated September 4, 2020

\begin{itemize}
\item ~
  \hypertarget{what-are-the-symptoms-of-coronavirus}{%
  \paragraph{What are the symptoms of
  coronavirus?}\label{what-are-the-symptoms-of-coronavirus}}

  \begin{itemize}
  \tightlist
  \item
    In the beginning, the coronavirus
    \href{https://www.nytimes3xbfgragh.onion/article/coronavirus-facts-history.html?action=click\&pgtype=Article\&state=default\&region=MAIN_CONTENT_3\&context=storylines_faq\#link-6817bab5}{seemed
    like it was primarily a respiratory illness}~--- many patients had
    fever and chills, were weak and tired, and coughed a lot, though
    some people don't show many symptoms at all. Those who seemed
    sickest had pneumonia or acute respiratory distress syndrome and
    received supplemental oxygen. By now, doctors have identified many
    more symptoms and syndromes. In April,
    \href{https://www.nytimes3xbfgragh.onion/2020/04/27/health/coronavirus-symptoms-cdc.html?action=click\&pgtype=Article\&state=default\&region=MAIN_CONTENT_3\&context=storylines_faq}{the
    C.D.C. added to the list of early signs}~sore throat, fever, chills
    and muscle aches. Gastrointestinal upset, such as diarrhea and
    nausea, has also been observed. Another telltale sign of infection
    may be a sudden, profound diminution of one's
    \href{https://www.nytimes3xbfgragh.onion/2020/03/22/health/coronavirus-symptoms-smell-taste.html?action=click\&pgtype=Article\&state=default\&region=MAIN_CONTENT_3\&context=storylines_faq}{sense
    of smell and taste.}~Teenagers and young adults in some cases have
    developed painful red and purple lesions on their fingers and toes
    --- nicknamed ``Covid toe'' --- but few other serious symptoms.
  \end{itemize}
\item ~
  \hypertarget{why-is-it-safer-to-spend-time-together-outside}{%
  \paragraph{Why is it safer to spend time together
  outside?}\label{why-is-it-safer-to-spend-time-together-outside}}

  \begin{itemize}
  \tightlist
  \item
    \href{https://www.nytimes3xbfgragh.onion/2020/05/15/us/coronavirus-what-to-do-outside.html?action=click\&pgtype=Article\&state=default\&region=MAIN_CONTENT_3\&context=storylines_faq}{Outdoor
    gatherings}~lower risk because wind disperses viral droplets, and
    sunlight can kill some of the virus. Open spaces prevent the virus
    from building up in concentrated amounts and being inhaled, which
    can happen when infected people exhale in a confined space for long
    stretches of time, said Dr. Julian W. Tang, a virologist at the
    University of Leicester.
  \end{itemize}
\item ~
  \hypertarget{why-does-standing-six-feet-away-from-others-help}{%
  \paragraph{Why does standing six feet away from others
  help?}\label{why-does-standing-six-feet-away-from-others-help}}

  \begin{itemize}
  \tightlist
  \item
    The coronavirus spreads primarily through droplets from your mouth
    and nose, especially when you cough or sneeze. The C.D.C., one of
    the organizations using that measure,
    \href{https://www.nytimes3xbfgragh.onion/2020/04/14/health/coronavirus-six-feet.html?action=click\&pgtype=Article\&state=default\&region=MAIN_CONTENT_3\&context=storylines_faq}{bases
    its recommendation of six feet}~on the idea that most large droplets
    that people expel when they cough or sneeze will fall to the ground
    within six feet. But six feet has never been a magic number that
    guarantees complete protection. Sneezes, for instance, can launch
    droplets a lot farther than six feet,
    \href{https://jamanetwork.com/journals/jama/fullarticle/2763852}{according
    to a recent study}. It's a rule of thumb: You should be safest
    standing six feet apart outside, especially when it's windy. But
    keep a mask on at all times, even when you think you're far enough
    apart.
  \end{itemize}
\item ~
  \hypertarget{i-have-antibodies-am-i-now-immune}{%
  \paragraph{I have antibodies. Am I now
  immune?}\label{i-have-antibodies-am-i-now-immune}}

  \begin{itemize}
  \tightlist
  \item
    As of right
    now,\href{https://www.nytimes3xbfgragh.onion/2020/07/22/health/covid-antibodies-herd-immunity.html?action=click\&pgtype=Article\&state=default\&region=MAIN_CONTENT_3\&context=storylines_faq}{~that
    seems likely, for at least several months.}~There have been
    frightening accounts of people suffering what seems to be a second
    bout of Covid-19. But experts say these patients may have a
    drawn-out course of infection, with the virus taking a slow toll
    weeks to months after initial exposure.~People infected with the
    coronavirus typically
    \href{https://www.nature.com/articles/s41586-020-2456-9}{produce}~immune
    molecules called antibodies, which are
    \href{https://www.nytimes3xbfgragh.onion/2020/05/07/health/coronavirus-antibody-prevalence.html?action=click\&pgtype=Article\&state=default\&region=MAIN_CONTENT_3\&context=storylines_faq}{protective
    proteins made in response to an
    infection}\href{https://www.nytimes3xbfgragh.onion/2020/05/07/health/coronavirus-antibody-prevalence.html?action=click\&pgtype=Article\&state=default\&region=MAIN_CONTENT_3\&context=storylines_faq}{.
    These antibodies may}~last in the body
    \href{https://www.nature.com/articles/s41591-020-0965-6}{only two to
    three months}, which may seem worrisome, but that's~perfectly normal
    after an acute infection subsides, said Dr. Michael Mina, an
    immunologist at Harvard University. It may be possible to get the
    coronavirus again, but it's highly unlikely that it would be
    possible in a short window of time from initial infection or make
    people sicker the second time.
  \end{itemize}
\item ~
  \hypertarget{what-are-my-rights-if-i-am-worried-about-going-back-to-work}{%
  \paragraph{What are my rights if I am worried about going back to
  work?}\label{what-are-my-rights-if-i-am-worried-about-going-back-to-work}}

  \begin{itemize}
  \tightlist
  \item
    Employers have to provide
    \href{https://www.osha.gov/SLTC/covid-19/standards.html}{a safe
    workplace}~with policies that protect everyone equally.
    \href{https://www.nytimes3xbfgragh.onion/article/coronavirus-money-unemployment.html?action=click\&pgtype=Article\&state=default\&region=MAIN_CONTENT_3\&context=storylines_faq}{And
    if one of your co-workers tests positive for the coronavirus, the
    C.D.C.}~has said that
    \href{https://www.cdc.gov/coronavirus/2019-ncov/community/guidance-business-response.html}{employers
    should tell their employees}~-\/- without giving you the sick
    employee's name -\/- that they may have been exposed to the virus.
  \end{itemize}
\end{itemize}

Joining Mr. Flatemen and Mr. Miranda at the group's helm are Ellen
Futter, the president of the American Museum of Natural History,
\href{https://www.nytimes3xbfgragh.onion/2020/05/06/arts/design/natural-history-layoffs-virus.html}{which
recently announced a number of layoffs}; Thelma Golden, the Studio
Museum in Harlem's director and chief curator; the restaurateur Danny
Meyer; and Peter Ward, the president of the New York Hotel \& Motel
Trades Council.

Before the pandemic struck, NYC \& Company was forecasting an 11th
straight year of increased tourism. In 2019, the city had more than 66
million visitors who generated about \$70 billion of economic activity
that supported 400,000 jobs, according to the agency's estimates.

``Together, we will create a next act for our city,'' Mr. Miranda said
in a statement.

\hypertarget{nyc-is-expanding-testing-and-tracing-but-a-limited-reopening-is-weeks-away}{%
\subsection{N.Y.C. is expanding testing and tracing, but a limited
reopening is weeks
away.}\label{nyc-is-expanding-testing-and-tracing-but-a-limited-reopening-is-weeks-away}}

Image

A drive-through testing site in Brooklyn on Monday.Credit...Dave Sanders
for The New York Times

Mr. de Blasio on Tuesday announced an expansion of coronavirus testing
and tracing across New York City, but he warned again that a limited
reopening of the city was weeks away at best.

Twelve new testing sites will be set up in the next three weeks in a
push to double the public hospital system's testing capacity, the mayor
said at his daily news briefing. The city was also training 535
\href{https://www.nytimes3xbfgragh.onion/2020/05/07/nyregion/coronavirus-contact-tracing-nyc.html}{contact
tracers}, with a goal of having 2,500 in the field by early June.

Still, the city, the pandemic's U.S. epicenter, has met just four of the
seven
\href{https://www.nytimes3xbfgragh.onion/2020/05/04/nyregion/coronavirus-reopen-cuomo-ny.html}{criteria
required to start to reopen}, Mr. Cuomo said on Monday while announcing
that three upstate regions had achieved all of the necessary benchmarks.

Mr. de Blasio has said he is closely monitoring three measures in
weighing the city's progress toward reopening: the number of new virus
infections; the number of infected patients in intensive care units; and
the percentage of residents testing positive for the virus.

``Clearly, these indicators are not getting us the kind of answers we
need to change our restrictions in May,'' the mayor said. ``You've got
to have 10 days to two weeks of consistent, downward motion. We haven't
had that in a sustained way at all.''

More than 20,000 deaths in New York City have now been linked to the
virus, according to
\href{https://www1.nyc.gov/site/doh/covid/covid-19-data.page}{data from
the city's Department of Health}.

\hypertarget{connecticut-moves-toward-reopening-as-top-health-official-is-ousted}{%
\subsection{Connecticut moves toward reopening as top health official is
ousted.}\label{connecticut-moves-toward-reopening-as-top-health-official-is-ousted}}

With Connecticut nearing the May 20 date that officials have set for
starting a gradual reopening, Gov. Ned Lamont said on Tuesday that the
state continued to make steady progress in stemming the coronavirus
outbreak.

Mr. Lamont, speaking at a daily news briefing, reported 33 new virus
deaths, eight fewer than on Monday. The virus has now been linked to
3,041 deaths in Connecticut.

``Hopefully the fatalities continue to go down,'' Mr. Lamont said.

The governor also said the total number of patients hospitalized with
the virus had fallen to 1,189 from 1,212. He reported 568 new confirmed
virus cases, which he suggested was in line with an increase in testing.

Mr. Lamont's briefing came several hours after he said he was replacing
the state's public health commissioner, Renee Coleman-Mitchell, who had
been in the job about a year. She had not been at his daily briefings in
recent weeks.

The governor did not discuss his reasons for making the change, saying
only that he had appointed the state's social services commissioner,
Deidre Gifford, to fill the post for the time being.

Ms. Coleman-Mitchell's ``service over the last year has been a great
deal of help, particularly in the face of the global Covid-19 pandemic
that has brought disruption to many throughout the world,'' Mr. Lamont
said in a statement.

In a statement issued by a lawyer, Ms. Coleman-Mitchell said she had
been told by members of the governor's staff that the move ``was not
related to job performance.''

``I am proud of the work of the Department of Public Health during this
time of unprecedented turmoil and threat to the public health,'' she
added.

\hypertarget{tell-us-about-the-moments-that-have-brought-you-hope-strength-humor-and-relief}{%
\subsection{Tell us about the moments that have brought you hope,
strength, humor and
relief.}\label{tell-us-about-the-moments-that-have-brought-you-hope-strength-humor-and-relief}}

The coronavirus outbreak has brought much of life in New York to a halt
and there is no clear end in sight. But there are also moments that
offer a sliver of strength, hope, humor or some other type of relief: a
joke from a stranger on line at the supermarket; a favor from a friend
down the block; a great meal ordered from a restaurant we want to
survive; trivia night via Zoom with the bar down the street.

We'd like to hear about your moments, the ones that are helping you
through these dark times. A reporter or editor may contact you. Your
information will not be published without your consent.

Reporting was contributed by Andrea Salcedo, Maria Cramer, Michael Gold,
Patrick McGeehan, Jesse McKinley and Azi Paybarah.

Advertisement

\protect\hyperlink{after-bottom}{Continue reading the main story}

\hypertarget{site-index}{%
\subsection{Site Index}\label{site-index}}

\hypertarget{site-information-navigation}{%
\subsection{Site Information
Navigation}\label{site-information-navigation}}

\begin{itemize}
\tightlist
\item
  \href{https://help.nytimes3xbfgragh.onion/hc/en-us/articles/115014792127-Copyright-notice}{©~2020~The
  New York Times Company}
\end{itemize}

\begin{itemize}
\tightlist
\item
  \href{https://www.nytco.com/}{NYTCo}
\item
  \href{https://help.nytimes3xbfgragh.onion/hc/en-us/articles/115015385887-Contact-Us}{Contact
  Us}
\item
  \href{https://www.nytco.com/careers/}{Work with us}
\item
  \href{https://nytmediakit.com/}{Advertise}
\item
  \href{http://www.tbrandstudio.com/}{T Brand Studio}
\item
  \href{https://www.nytimes3xbfgragh.onion/privacy/cookie-policy\#how-do-i-manage-trackers}{Your
  Ad Choices}
\item
  \href{https://www.nytimes3xbfgragh.onion/privacy}{Privacy}
\item
  \href{https://help.nytimes3xbfgragh.onion/hc/en-us/articles/115014893428-Terms-of-service}{Terms
  of Service}
\item
  \href{https://help.nytimes3xbfgragh.onion/hc/en-us/articles/115014893968-Terms-of-sale}{Terms
  of Sale}
\item
  \href{https://spiderbites.nytimes3xbfgragh.onion}{Site Map}
\item
  \href{https://help.nytimes3xbfgragh.onion/hc/en-us}{Help}
\item
  \href{https://www.nytimes3xbfgragh.onion/subscription?campaignId=37WXW}{Subscriptions}
\end{itemize}
