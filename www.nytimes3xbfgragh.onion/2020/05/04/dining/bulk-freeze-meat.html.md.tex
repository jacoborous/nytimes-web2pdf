Sections

SEARCH

\protect\hyperlink{site-content}{Skip to
content}\protect\hyperlink{site-index}{Skip to site index}

\href{https://www.nytimes3xbfgragh.onion/section/food}{Food}

\href{https://myaccount.nytimes3xbfgragh.onion/auth/login?response_type=cookie\&client_id=vi}{}

\href{https://www.nytimes3xbfgragh.onion/section/todayspaper}{Today's
Paper}

\href{/section/food}{Food}\textbar{}How to Cook and Freeze a Large Piece
of Meat, and Eat for Weeks

\url{https://nyti.ms/2W1U25V}

\begin{itemize}
\item
\item
\item
\item
\item
\item
\end{itemize}

\href{https://www.nytimes3xbfgragh.onion/spotlight/at-home?action=click\&pgtype=Article\&state=default\&region=TOP_BANNER\&context=at_home_menu}{At
Home}

\begin{itemize}
\tightlist
\item
  \href{https://www.nytimes3xbfgragh.onion/2020/09/07/travel/route-66.html?action=click\&pgtype=Article\&state=default\&region=TOP_BANNER\&context=at_home_menu}{Cruise
  Along: Route 66}
\item
  \href{https://www.nytimes3xbfgragh.onion/2020/09/04/dining/sheet-pan-chicken.html?action=click\&pgtype=Article\&state=default\&region=TOP_BANNER\&context=at_home_menu}{Roast:
  Chicken With Plums}
\item
  \href{https://www.nytimes3xbfgragh.onion/2020/09/04/arts/television/dark-shadows-stream.html?action=click\&pgtype=Article\&state=default\&region=TOP_BANNER\&context=at_home_menu}{Watch:
  Dark Shadows}
\item
  \href{https://www.nytimes3xbfgragh.onion/interactive/2020/at-home/even-more-reporters-editors-diaries-lists-recommendations.html?action=click\&pgtype=Article\&state=default\&region=TOP_BANNER\&context=at_home_menu}{Explore:
  Reporters' Google Docs}
\end{itemize}

Advertisement

\protect\hyperlink{after-top}{Continue reading the main story}

Supported by

\protect\hyperlink{after-sponsor}{Continue reading the main story}

\hypertarget{how-to-cook-and-freeze-a-large-piece-of-meat-and-eat-for-weeks}{%
\section{How to Cook and Freeze a Large Piece of Meat, and Eat for
Weeks}\label{how-to-cook-and-freeze-a-large-piece-of-meat-and-eat-for-weeks}}

A large, inexpensive roast is a boon for busy home cooks: Prepare it
simply, then let it star in a number of fast weeknight meals. J. Kenji
López-Alt explains.

\includegraphics{https://static01.graylady3jvrrxbe.onion/images/2020/05/06/dining/04Kenji4/04Kenji4-articleLarge-v2.jpg?quality=75\&auto=webp\&disable=upscale}

By \href{https://www.nytimes3xbfgragh.onion/by/j-kenji-lopez-alt}{J.
Kenji López-Alt}

\begin{itemize}
\item
  May 4, 2020
\item
  \begin{itemize}
  \item
  \item
  \item
  \item
  \item
  \item
  \end{itemize}
\end{itemize}

On my last bimonthly trip to the supermarket to stock up on staples, I
noticed the beef offerings had transitioned from mostly steaks and chops
to large, inexpensive roasts like top round, eye rounds and
\href{https://www.nytimes3xbfgragh.onion/2014/10/29/dining/santa-maria-tri-tip-recipe.html}{tri-tip}.
This makes sense, given our slowed shopping cadence. But what's the best
way to deal with big, inexpensive cuts of beef?

Fattier, expensive cuts like prime rib or New York strip are celebratory
centerpieces that do best when simply roasted with salt and pepper and
served straight away. Leaner cuts can also be roasted successfully,
provided you go low and slow. I like to place my roast in a cold oven,
set it to 225 degrees Fahrenheit, and slow-roast the beef until it hits
125 degrees on a digital thermometer, which yields a rosy-pink doneness
that extends from edge-to-edge, as well as enhanced tenderness. (The
same enzymes that tenderize a dry-aged steak will work in overtime as
you slowly heat them.)

I then finish it with a sear, which is faster, minimizing the amount of
dry, overcooked meat around the exterior. (This technique, which I
published in Cook's Illustrated in a 2007 article about steaks, is now
commonly known as ``the reverse sear.'') Cooked this way and sliced
thin, even a relatively tough, lean cut will come out tender and
succulent.

But as good as warm roast beef can be, I'd suggest that leaner,
inexpensive cuts actually taste better served cold, the next day --- or,
in a variety of preparations throughout the week. A cold roast beef
sandwich on crusty bread slathered with horseradish sauce --- equal
parts mayo, sour cream or yogurt and drained prepared horseradish, with
plenty of black pepper and a dash of Worcestershire --- or adding slices
to a Parmesan-packed Caesar salad are easy places to start, but it can
get much better. (See below.)

Simply store whatever you don't finish on the first day in the fridge
overnight, then slice it thin for future use. (It's better to store
leftovers whole and slice them next day --- cold beef slices more easily
than warm.)

\includegraphics{https://static01.graylady3jvrrxbe.onion/images/2020/05/06/dining/04kenji2/merlin_171960177_b551f427-62ce-42a0-9b9f-2d1f194b37b4-articleLarge.jpg?quality=75\&auto=webp\&disable=upscale}

This type of roast is also fantastic to use for quick meals months down
the line, provided you freeze it the right way.

Air and bulkiness are the enemies of good freezing. Air exposure can
lead to freezer burn --- that's when ice evaporates directly from the
surface of frozen foods in a process called sublimation --- and, believe
it or not, thin plastic bags and plastic wrap are air-permeable, which
is why it's important to use freezer bags, Cryovac bags or reusable
silicone bags, making sure to squeeze all the air out of them before
sealing.

What about bulkiness? Unlike vegetables, which have rigid cells that
burst when their water-filled interiors turn into jagged ice crystals,
meat (and especially fattier meat) fares quite well in the freezer,
provided the freezing and thawing processes are relatively fast. The
slower a piece of meat freezes, the more large and jagged the ice
crystals that form in it will be, and the more moisture (and flavor) it
will lose as it thaws.

So, how do you speed the freezing and thawing process? There are two
tricks. The first is to pack your meat as thin and as flat as possible.
The higher the surface area-to-volume ratio of a given amount of meat,
the more efficiently it will freeze and the less damage it will suffer.
That means cutting the roast thinly with a sharp slicer, then fanning it
and packing it flat in a freezer bag or Cryovac bag (the way smoked
salmon or fancy sliced salami comes packaged at the supermarket) before
freezing.

The second is to harness the power of aluminum, a fantastic conductor of
heat. By placing your freezer bag on an aluminum baking sheet, heat is
conducted away faster than if it were simply placed in the freezer on
its own. Once frozen, the bag can be removed from the baking sheet and
stored in its conveniently flat, stackable form. When I had a
side-by-side-style freezer, I kept everything --- soup, ground meat,
steaks, cooked rice --- frozen in flat packs that I filed away
vertically like vinyl records.

When you're ready to defrost, placing those bags on an aluminum baking
sheet on the countertop can cut defrosting time in half. A gallon-size
freezer bag with a fanned layer of sliced roast beef will defrost in
about 25 minutes --- just enough time to throw together a vinaigrette
and prepare some vegetables.

Image

Credit...David Malosh for The New York Times. Food Stylist: Simon
Andrews.

\hypertarget{4-smart-uses-for-leftover-roast-beef}{%
\subsection{4 Smart Uses for Leftover Roast
Beef}\label{4-smart-uses-for-leftover-roast-beef}}

\begin{itemize}
\item
  \textbf{Make a quick Thai salad:} Pound 2 garlic cloves with a
  tablespoon of brown or palm sugar in a mortar-and-pestle. Add 1
  tablespoon fish sauce and the juice of a lime, and crushed Thai dried
  chiles or red-pepper flakes to taste. Add sliced beef and crunchy
  vegetables like shredded cabbage, cucumber and onions; herbs like
  cilantro and mint; and split cherry tomatoes to the bowl and toss.
  Garnish with crushed peanuts and fried shallots.
\item
  \textbf{Pair with salty, fermented sauces and peppery vegetables:}
  Whisk together 2 teaspoons miso paste, a teaspoon of soy sauce, a
  teaspoon of honey, 2 teaspoons of whole-grain mustard and 3
  tablespoons of extra-virgin olive oil. Drizzle it over a bowl of
  thinly sliced beef, cucumber, red onion and peppery greens like
  watercress, arugula or mizuna.
\item
  \textbf{Brighten with briny capers and olives:} Lay some slices
  carpaccio-style on a large, chilled platter, then drizzle with a
  vinaigrette made from a tablespoon lemon juice, a couple of
  tablespoons lightly chopped capers, a teaspoon whole-grain mustard, a
  tablespoon minced shallot or red onion, and 3 to 4 tablespoons
  extra-virgin olive oil. Sprinkle with coarse sea salt. For crunch,
  garnish with lightly chopped toasted hazelnuts or pine nuts, then toss
  a handful or arugula or watercress in the bowl you made the dressing
  in and mound it in the center of the plate. Finish with freshly grated
  Parmesan.
\item
  \textbf{Or just keep it simple:} Drizzle some slices with olive oil,
  crack some pepper on top, sprinkle with coarse sea salt and eat them
  with your fingers. (Add a funky blue cheese, like Roquefort or
  Gorgonzola, or an extra-sharp Cheddar, if you've got it.)
\end{itemize}

Recipe:
\textbf{\href{https://cooking.nytimes3xbfgragh.onion/recipes/1021032-slow-roasted-beef}{Slow-Roasted
Beef}}

\hypertarget{and-to-drink-}{%
\subsection{And to Drink \ldots{}}\label{and-to-drink-}}

A tender beef roast with a well-browned exterior is about as easy to
pair with wine as a dish can be. You have your pick of just about any
medium- to full-bodied red wine, from any place. Your selection is
entirely dependent on your own taste. If you are planning to flavor the
roast with a sauce, though, that might narrow the choices. Pan juices
wouldn't change things, but if you were to add an English-style
horseradish sauce, for example, you might prefer a richer, more forceful
wine, like a Châteauneuf-du-Pape. Same if you eat it with ketchup. I,
personally, would avoid pungent or sweet sauces, and pick a decent
Bordeaux, or maybe a Chianti Classico, or a good grenache-based Spanish
wine. Oh, the possibilities. \textbf{ERIC ASIMOV}

\emph{Follow} \href{https://twitter.com/nytfood}{\emph{NYT Food on
Twitter}} \emph{and}
\href{https://www.instagram.com/nytcooking/}{\emph{NYT Cooking on
Instagram}}\emph{,}
\href{https://www.facebookcorewwwi.onion/nytcooking/}{\emph{Facebook}}\emph{,}
\href{https://www.youtube.com/nytcooking}{\emph{YouTube}} \emph{and}
\href{https://www.pinterest.com/nytcooking/}{\emph{Pinterest}}\emph{.}
\href{https://www.nytimes3xbfgragh.onion/newsletters/cooking}{\emph{Get
regular updates from NYT Cooking, with recipe suggestions, cooking tips
and shopping advice}}\emph{.}

Advertisement

\protect\hyperlink{after-bottom}{Continue reading the main story}

\hypertarget{site-index}{%
\subsection{Site Index}\label{site-index}}

\hypertarget{site-information-navigation}{%
\subsection{Site Information
Navigation}\label{site-information-navigation}}

\begin{itemize}
\tightlist
\item
  \href{https://help.nytimes3xbfgragh.onion/hc/en-us/articles/115014792127-Copyright-notice}{©~2020~The
  New York Times Company}
\end{itemize}

\begin{itemize}
\tightlist
\item
  \href{https://www.nytco.com/}{NYTCo}
\item
  \href{https://help.nytimes3xbfgragh.onion/hc/en-us/articles/115015385887-Contact-Us}{Contact
  Us}
\item
  \href{https://www.nytco.com/careers/}{Work with us}
\item
  \href{https://nytmediakit.com/}{Advertise}
\item
  \href{http://www.tbrandstudio.com/}{T Brand Studio}
\item
  \href{https://www.nytimes3xbfgragh.onion/privacy/cookie-policy\#how-do-i-manage-trackers}{Your
  Ad Choices}
\item
  \href{https://www.nytimes3xbfgragh.onion/privacy}{Privacy}
\item
  \href{https://help.nytimes3xbfgragh.onion/hc/en-us/articles/115014893428-Terms-of-service}{Terms
  of Service}
\item
  \href{https://help.nytimes3xbfgragh.onion/hc/en-us/articles/115014893968-Terms-of-sale}{Terms
  of Sale}
\item
  \href{https://spiderbites.nytimes3xbfgragh.onion}{Site Map}
\item
  \href{https://help.nytimes3xbfgragh.onion/hc/en-us}{Help}
\item
  \href{https://www.nytimes3xbfgragh.onion/subscription?campaignId=37WXW}{Subscriptions}
\end{itemize}
