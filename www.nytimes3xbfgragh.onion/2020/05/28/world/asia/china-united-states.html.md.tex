Sections

SEARCH

\protect\hyperlink{site-content}{Skip to
content}\protect\hyperlink{site-index}{Skip to site index}

\href{https://www.nytimes3xbfgragh.onion/section/world/asia}{Asia
Pacific}

\href{https://myaccount.nytimes3xbfgragh.onion/auth/login?response_type=cookie\&client_id=vi}{}

\href{https://www.nytimes3xbfgragh.onion/section/todayspaper}{Today's
Paper}

\href{/section/world/asia}{Asia Pacific}\textbar{}Beijing Hardens
Resolve to Defy U.S., Even While Calling for Cooperation

\url{https://nyti.ms/3c89KkH}

\begin{itemize}
\item
\item
\item
\item
\item
\item
\end{itemize}

Advertisement

\protect\hyperlink{after-top}{Continue reading the main story}

Supported by

\protect\hyperlink{after-sponsor}{Continue reading the main story}

News analysis

\hypertarget{beijing-hardens-resolve-to-defy-us-even-while-calling-for-cooperation}{%
\section{Beijing Hardens Resolve to Defy U.S., Even While Calling for
Cooperation}\label{beijing-hardens-resolve-to-defy-us-even-while-calling-for-cooperation}}

From China's perspective, the punitive American measures on trade,
technology and Hong Kong have revealed a core American hostility. But
China does not want to incinerate the relationship.

\includegraphics{https://static01.graylady3jvrrxbe.onion/images/2020/05/28/world/28china-1/merlin_172908903_5b7d4269-84e6-4320-9baf-4cffd3c933e5-articleLarge.jpg?quality=75\&auto=webp\&disable=upscale}

\href{https://www.nytimes3xbfgragh.onion/by/keith-bradsher}{\includegraphics{https://static01.graylady3jvrrxbe.onion/images/2018/10/08/multimedia/author-keith-bradsher/author-keith-bradsher-thumbLarge.png}}\href{https://www.nytimes3xbfgragh.onion/by/steven-lee-myers}{\includegraphics{https://static01.graylady3jvrrxbe.onion/images/2018/10/15/multimedia/author-steven-lee-myers/author-steven-lee-myers-thumbLarge.png}}

By \href{https://www.nytimes3xbfgragh.onion/by/keith-bradsher}{Keith
Bradsher} and
\href{https://www.nytimes3xbfgragh.onion/by/steven-lee-myers}{Steven Lee
Myers}

\begin{itemize}
\item
  Published May 28, 2020Updated June 1, 2020
\item
  \begin{itemize}
  \item
  \item
  \item
  \item
  \item
  \item
  \end{itemize}
\end{itemize}

\href{https://cn.nytimes3xbfgragh.onion/china/20200529/china-united-states/}{阅读简体中文版}\href{https://cn.nytimes3xbfgragh.onion/china/20200529/china-united-states/zh-hant}{閱讀繁體中文版}

BEIJING --- Ignoring threats from Washington, China stripped another
layer of autonomy from
\href{https://www.nytimes3xbfgragh.onion/2020/06/01/world/asia/Hong-kong-Tiananmen-vigil-banned.html}{Hong
Kong} on Thursday, plowing ahead with a plan that would ban any form of
dissent deemed subversive in the territory reclaimed from Britain more
than two decades ago.

But even as the plan was approved by China's top legislative body, and
Chinese officials taunted the United States as an imperious meddler,
Premier Li Keqiang struck a conciliatory tone. While offering no
concessions to American demands, he called for close trade relations
between the two countries.

The clash over
\href{https://www.nytimes3xbfgragh.onion/2020/06/01/world/asia/Hong-kong-Tiananmen-vigil-banned.html}{Hong
Kong} and other issues points to the quandary facing China as it grows
in power and contends with an increasingly aggressive Trump
administration. The Chinese leadership does not want to incinerate the
relationship with the United States, given the enormous economic
benefits. Nor is it willing to back down, reflecting divisions in
Beijing between hawks and more moderating forces.

``Anything the U.S. says or does or will do, China will refuse,'' Shi
Yinhong, a professor of international relations at Renmin University in
Beijing, said in a telephone interview.

With both countries blaming each other for the coronavirus pandemic,
trade disputes and now the crisis roiling Hong Kong, the result has been
a downward spiral of actions and responses that may not let up before
Mr. Trump's re-election campaign ends in November.

The back-and-forth between Washington and Beijing intensified in the
past two days.

Secretary of State Mike Pompeo declared on Wednesday that Washington
would no longer consider
\href{https://www.nytimes3xbfgragh.onion/2019/11/14/business/hong-kong-protests-recession.html}{Hong
Kong to have significant autonomy}, clearing the way for Mr. Trump to
end the
\href{https://www.nytimes3xbfgragh.onion/2020/05/22/world/asia/trump-pompeo-china-hong-kong.html}{special
trade and economic relations} the territory now enjoys. Less than 24
hours later, the National People's Congress, China's legislature, did
precisely what Mr. Pompeo had railed against: authorizing new security
laws in Hong Kong.

After the United States won
\href{https://www.nytimes3xbfgragh.onion/2020/05/27/world/canada/huawei-extradition-meng-wanzhou.html}{an
initial victory} in a Canadian court on Wednesday in its long effort to
bring criminal charges against a senior executive of Huawei, the Chinese
telecommunications giant, China swiftly vowed to retaliate against both
Canada and the United States. China already had blocked some Canadian
exports and detained two Canadian citizens for more than 500 days.

Trump administration officials argue that they have brought China to the
table on trade by imposing tariffs. But they have failed so far to
achieve their goal of fundamentally shifting China's behavior --- on
trade or any other issue.

From Beijing's perspective, the punitive measures have simply revealed
the core of American hostility toward China.

``When China was rising as an economic power, the United States
tolerated it,'' Shen Dingli, an expert on relations with the United
States at Fudan University in Shanghai, said in a telephone interview.
``Now that China is strong, it cannot tolerate it anymore.''

When the Trump administration announced new restrictions to block
companies around the world from using American-made machinery and
software to help Huawei, Beijing promised to target American technology
companies operating in China.

When the administration capped the number of Chinese journalists in the
United States,
\href{https://www.nytimes3xbfgragh.onion/2020/03/17/business/media/china-expels-american-journalists.html}{China
kicked out} most of the American correspondents from three major news
organizations in the United States, including The New York Times.

\includegraphics{https://static01.graylady3jvrrxbe.onion/images/2020/05/28/world/28china-2/merlin_170660466_10678490-9cf3-4308-809c-0fb9cd49f8ae-articleLarge.jpg?quality=75\&auto=webp\&disable=upscale}

Both Mr. Trump and President Xi Jinping of China feel compelled to
appear strong. The American president views blaming China for the
coronavirus crisis in the United States as a path to re-election. The
Chinese leader faces enormous economic and diplomatic challenges that
could stir domestic opposition to his grip on power.

What the American moves have not done is chasten Mr. Xi's government,
which appears to feel simultaneously embattled and defiant.

Hu Xijin, the outspoken editor of Global Times, a nationalistic tabloid
controlled by the Communist Party, all but dared the Trump
administration to carry out its threat to end Hong Kong's favored trade
status. He noted that there were 85,000 Americans there and scores of
companies that would reap ``the bitter fruits'' of the American
decision.

``Washington is too narcissistic,'' he
\href{https://m.weibo.cn/status/4509393636362290?display=0\&retcode=6102}{wrote}
in Chinese on Weibo on Thursday. ``American politicians like Pompeo
arrogantly think that the fate of Hong Kong is in their hands.''

The National People's Congress on Thursday dutifully adopted the
government's proposals to impose new laws on Hong Kong to suppress
subversion, secession, terrorism and other acts that might threaten
China's national security --- as the authorities in Beijing define it.
The vote was nearly unanimous, with only one delegate voting against and
six abstaining.

Image

Protesters and bystanders crouching after the police fired pepper balls
in Central, a business district in Hong Kong, on Wednesday.Credit...Lam
Yik Fei for The New York Times

China's authoritarian system and state-run media give Mr. Xi enormous
power to control the message in the face of American hostility ---
exploiting it to rally popular outrage and tempering it to play the role
of magnanimous global partner.

At his ritual news conference wrapping up the National People's Congress
on Thursday, Mr. Li, the premier, singled out for praise an American
company, Honeywell, that on Tuesday announced an investment in Wuhan ---
the city from which the pandemic spread. A month before, the Pentagon
had awarded Honeywell a contract to supply protective masks.

Mr. Li twice called for ``peaceful'' relations with Taiwan, after
conspicuously dropping the word when he discussed Taiwan at the start of
the weeklong legislative session. And he underlined China's willingness
to look further for the origins of the coronavirus.

China, though, has shown little inclination for compromise.

Beijing reacted harshly to a Canadian court's ruling that cleared an
initial hurdle for the extradition of Meng Wanzhou, a senior executive
of Huawei charged by the United States with bank fraud related to
American sanctions against Iran.

The Chinese Embassy in Ottawa accused the United States and Canada of
abusing their bilateral extradition treaty and ``arbitrarily taking
forceful measures'' against Ms. Meng.

Image

Huawei Chief Financial Officer, Meng Wanzhou leaving British Columbia
Supreme Court in Vancouver, on Wednesday.Credit...Don Mackinnon/Agence
France-Presse --- Getty Images

``The purpose of the United States is to bring down Huawei and other
Chinese high-tech companies, and Canada has been acting in the process
as an accomplice of the United States,'' the embassy
\href{https://twitter.com/ChinaEmbOttawa/status/1265762374004494338}{said
on Twitter}, which is banned inside China. ``The whole case is entirely
a grave political incident.''

China has already retaliated against Canadian exports of pork, canola
oil and other products, and in recent days it has hinted that it will
target still more. It has also held two Canadians, Michael Kovrig and
Michael Spavor, in secret detention on state security charges widely
viewed as retaliatory.

Neither has appeared in a public court hearing or been afforded access
to lawyers during court proceedings. That has hardened anti-Chinese
sentiment in Canada, which had not historically been as suspicious of
Beijing as, say, the United States has.

The International Crisis Group, where Mr. Kovrig, a former diplomat,
worked, posted a message on Twitter noting that Thursday was his 535th
day in detention. ``Each passing day is a stain on China's reputation,''
the group said.

Mr. Xi, who has ruled with an increasingly authoritarian grip since
2012, seems
\href{https://www.nytimes3xbfgragh.onion/2020/05/24/world/asia/china-hong-kong-taiwan.html}{impervious}
to worries about China's reputation. He has
\href{https://www.nytimes3xbfgragh.onion/2020/05/20/world/asia/coronavirus-china-xi-jinping.html}{emerged
from the pandemic newly emboldened}, seizing on nationalist themes to
deflect from the government's early failures in stopping the
coronavirus's spread.

He has described the pandemic and its still-unfolding economic challenge
as
\href{https://www.nytimes3xbfgragh.onion/2020/05/20/world/asia/coronavirus-china-xi-jinping.html}{a
crucible} that will forge a stronger government and a stronger party.
China has also showed it will not be distracted from its defense of
territorial claims along its land and sea borders --- from the South
China Sea to the Himalayas.

The commander of the People's Liberation Army's garrison in Hong Kong
delivered\href{https://www.nytimes3xbfgragh.onion/2020/05/26/world/asia/china-military-hong-kong.html}{a
pointed reminder}of its duty to keep the peace there on the sidelines of
the congress in Beijing this week.

Image

Soldiers from the People's Liberation Army at barracks in Hong Kong in
October.Credit...Lam Yik Fei for The New York Times

The bravado has weakened what leverage the United States might once have
wielded: the threat of international condemnation, restrictions on
trade, even the prospect of decoupling the world's two largest
economies. Beijing now seems willing to bear any cost.

Lau Siu-kai, a former senior Hong Kong government official who advises
Beijing, said that American pressure had failed to prompt a
reconsideration in the Hong Kong issue, in part because China's
leadership has anticipated American opposition on many fronts.

``Beijing will stick with its new policy toward Hong Kong regardless of
U.S. reactions and is prepared to take countermeasures in a tit-for-tat
manner,'' he said.

Keith Bradsher reported from Beijing, and Steven Lee Myers from Seoul,
South Korea. Research was contributed by Claire Fu, Wang Yiwei, Amber
Wang and Liu Yi from Beijing.

Advertisement

\protect\hyperlink{after-bottom}{Continue reading the main story}

\hypertarget{site-index}{%
\subsection{Site Index}\label{site-index}}

\hypertarget{site-information-navigation}{%
\subsection{Site Information
Navigation}\label{site-information-navigation}}

\begin{itemize}
\tightlist
\item
  \href{https://help.nytimes3xbfgragh.onion/hc/en-us/articles/115014792127-Copyright-notice}{©~2020~The
  New York Times Company}
\end{itemize}

\begin{itemize}
\tightlist
\item
  \href{https://www.nytco.com/}{NYTCo}
\item
  \href{https://help.nytimes3xbfgragh.onion/hc/en-us/articles/115015385887-Contact-Us}{Contact
  Us}
\item
  \href{https://www.nytco.com/careers/}{Work with us}
\item
  \href{https://nytmediakit.com/}{Advertise}
\item
  \href{http://www.tbrandstudio.com/}{T Brand Studio}
\item
  \href{https://www.nytimes3xbfgragh.onion/privacy/cookie-policy\#how-do-i-manage-trackers}{Your
  Ad Choices}
\item
  \href{https://www.nytimes3xbfgragh.onion/privacy}{Privacy}
\item
  \href{https://help.nytimes3xbfgragh.onion/hc/en-us/articles/115014893428-Terms-of-service}{Terms
  of Service}
\item
  \href{https://help.nytimes3xbfgragh.onion/hc/en-us/articles/115014893968-Terms-of-sale}{Terms
  of Sale}
\item
  \href{https://spiderbites.nytimes3xbfgragh.onion}{Site Map}
\item
  \href{https://help.nytimes3xbfgragh.onion/hc/en-us}{Help}
\item
  \href{https://www.nytimes3xbfgragh.onion/subscription?campaignId=37WXW}{Subscriptions}
\end{itemize}
