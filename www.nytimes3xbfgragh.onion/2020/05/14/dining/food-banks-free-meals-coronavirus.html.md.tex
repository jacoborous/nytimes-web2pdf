Sections

SEARCH

\protect\hyperlink{site-content}{Skip to
content}\protect\hyperlink{site-index}{Skip to site index}

\href{https://www.nytimes3xbfgragh.onion/section/food}{Food}

\href{https://myaccount.nytimes3xbfgragh.onion/auth/login?response_type=cookie\&client_id=vi}{}

\href{https://www.nytimes3xbfgragh.onion/section/todayspaper}{Today's
Paper}

\href{/section/food}{Food}\textbar{}To Fight Waste and Hunger, Food
Banks Start Cooking

\url{https://nyti.ms/2Z1JjKI}

\begin{itemize}
\item
\item
\item
\item
\item
\item
\end{itemize}

\href{https://www.nytimes3xbfgragh.onion/spotlight/at-home?action=click\&pgtype=Article\&state=default\&region=TOP_BANNER\&context=at_home_menu}{At
Home}

\begin{itemize}
\tightlist
\item
  \href{https://www.nytimes3xbfgragh.onion/2020/09/07/travel/route-66.html?action=click\&pgtype=Article\&state=default\&region=TOP_BANNER\&context=at_home_menu}{Cruise
  Along: Route 66}
\item
  \href{https://www.nytimes3xbfgragh.onion/2020/09/04/dining/sheet-pan-chicken.html?action=click\&pgtype=Article\&state=default\&region=TOP_BANNER\&context=at_home_menu}{Roast:
  Chicken With Plums}
\item
  \href{https://www.nytimes3xbfgragh.onion/2020/09/04/arts/television/dark-shadows-stream.html?action=click\&pgtype=Article\&state=default\&region=TOP_BANNER\&context=at_home_menu}{Watch:
  Dark Shadows}
\item
  \href{https://www.nytimes3xbfgragh.onion/interactive/2020/at-home/even-more-reporters-editors-diaries-lists-recommendations.html?action=click\&pgtype=Article\&state=default\&region=TOP_BANNER\&context=at_home_menu}{Explore:
  Reporters' Google Docs}
\end{itemize}

Advertisement

\protect\hyperlink{after-top}{Continue reading the main story}

Supported by

\protect\hyperlink{after-sponsor}{Continue reading the main story}

\hypertarget{to-fight-waste-and-hunger-food-banks-start-cooking}{%
\section{To Fight Waste and Hunger, Food Banks Start
Cooking}\label{to-fight-waste-and-hunger-food-banks-start-cooking}}

As farmers throw away produce and other Americans line up for food,
relief groups are connecting the two by turning those ingredients into
meals.

\includegraphics{https://static01.graylady3jvrrxbe.onion/images/2020/05/20/dining/14virus-foodbank1/merlin_172297695_85bac692-3014-43ec-b162-a91551632824-articleLarge.jpg?quality=75\&auto=webp\&disable=upscale}

By \href{https://www.nytimes3xbfgragh.onion/by/brett-anderson}{Brett
Anderson}

\begin{itemize}
\item
  May 14, 2020
\item
  \begin{itemize}
  \item
  \item
  \item
  \item
  \item
  \item
  \end{itemize}
\end{itemize}

BOYNTON BEACH, Fla. --- The new kitchen was still under construction
here at the Palm Beach County branch of
\href{https://feedingsouthflorida.org/}{Feeding South Florida}, a food
bank, when Chrissy Benoit walked through the warehouse in March. Fruits
and vegetables had started arriving in unusually large amounts from
nearby farms, which got her thinking about the cooking she would do when
the kitchen was finished.

``Bell peppers, squashes, zucchini, cucumbers, apples --- the things
that you would think typically grow in abundance we see a lot of in a
fairly short window of time,'' said Ms. Benoit, the branch's general
manager. ``It's kind of like a `Top Chef' challenge, only instead of a
box, you've got a pallet.''

Food banks typically distribute boxes of ingredients for people in need
to prepare at home. But this one has sped up construction of its kitchen
during the pandemic to do something it hadn't before: cook.

The kitchen is a single response to two huge and seemingly paradoxical
problems vexing communities all over the nation: too much food, and too
little. It aims to use a glut of produce and meat from local farmers,
who have lost their pipeline to restaurants and other big customers, to
make meals for a soaring number of people suddenly facing hunger.

Since March, demand for food has leapt by 600 percent at Feeding South
Florida, the largest food bank in this agriculture-rich state. When the
new kitchen opens, as early as next week, Ms. Benoit and her staff will
begin preparing 10,000 hot meals a day.

At the same time, Feeding South Florida will join the ranks of food
banks across the country that are trying to better serve their
communities by stepping up to the stove, as unemployment spreads and the
start of the harvest moves north.

``Our typical client gets a box of produce to make food --- that's a
great model,'' said Brian Riendeau, executive director of
\href{https://daretocare.org/}{Dare to Care}, a food bank in Louisville,
Ky., that plans to open a large production kitchen later this month.
``But we also have a lot of single parents working two jobs. What they
don't have is time to cook a meal.''

Food banks that can turn tomatoes into marinara are fighting more than
hunger. Cooking allows them to process larger amounts of perishable food
as the coronavirus sows
\href{https://www.nytimes3xbfgragh.onion/2020/05/02/business/coronavirus-food-waste-destroyed.html}{chaos}
in the country's food system, forcing farmers to plow under crops and
dump milk, and meat processors to shut down plants and
\href{https://www.nytimes3xbfgragh.onion/2020/05/14/business/coronavirus-farmers-killing-pigs.html}{euthanize
animals}.

Farmers donate much of the food for relief efforts, but food banks are
trying to find money to at least pay them for the costs of harvesting
and delivery.

\includegraphics{https://static01.graylady3jvrrxbe.onion/images/2020/05/20/dining/14virus-foodbank2/merlin_172234335_07b520b8-aa37-4e81-b672-e8481314cdec-articleLarge.jpg?quality=75\&auto=webp\&disable=upscale}

The intersecting problems of growing hunger and food waste came early to
South Florida, where the harvest hit its stride this year just as local
farmers' biggest customers --- restaurants, hotels, school districts,
Walt Disney World --- shut down in response to the virus.

``When that demand stopped, the farmers had to readjust,'' said Nikki
Fried, the state agriculture commissioner. Farmers and distributors have
suffered more than \$500 million in losses so far, she said. (Even so,
with travel at a standstill, the state's \$137 billion agriculture
sector has overtaken tourism as Florida's largest industry.)

One Florida farmer plowed under 100 acres of green beans and 60 acres of
cabbage, according to the \href{https://ifas.ufl.edu/}{Institute of Food
and Agriculture Sciences} at the University of Florida. Another
predicted that 10 million pounds of his tomatoes would stay in the
fields.

Image

Surplus fruits and vegetables have been pouring into Feeding South
Florida's warehouse in Boynton Beach.~Credit...Saul Martinez for The New
York Times

Paco Vélez, the president and chief executive of Feeding South Florida,
said a local farmer donated eight semi-truck trailers of tomatoes to the
food bank and its partners in March. ``We'd never received that much
before,'' he said. ``Because of its perishability, we have to move it
real quick.''

The delivery was an early signal that food waste and hunger were rising
on parallel tracks. The United States Department of Agriculture recently
responded with \$300 million a month in grants that allow food
distributors to buy surplus meat, produce and dairy products and pass
them to food banks and other charities.

Image

Members of the National Guard are helping Feeding South Florida keep up
with demand that has grown 600 percent since March.Credit...Saul
Martinez for The New York Times

\href{https://feedingpa.org/}{Feeding Pennsylvania}, a sister food bank
network to Feeding South Florida, is benefiting from that money just as
it ramps up its ability to cook. This week, the network, which does not
have a kitchen, began a program that enlists Pennsylvania restaurants to
make ready-to-eat meals to distribute across the state.

Most of the meals, which come in pans large enough to feed a family of
four, will be frozen. This allows Feeding Pennsylvania to preserve
perishable food at a time when
\href{https://www.nytimes3xbfgragh.onion/2020/05/11/dining/bulk-food-buying-coronavirus.html}{bulk
buying} and
\href{https://www.nytimes3xbfgragh.onion/2020/03/31/smarter-living/wirecutter/dont-overdo-the-coronavirus-stockpiling.html}{panic-shopping}
are making shelf-stable items
\href{https://www.nytimes3xbfgragh.onion/2020/05/11/nyregion/Coronavirus-supermarkets-items-missing.html}{harder
to come by}, said Jane Clements-Smith, the network's director.

``Families can put the meals in the freezer and pull them out when they
need them,'' she said. ``It gives them an easy night, which everyone
wants right now.''

The program will produce 520,000 meals a week. It is funded into June by
money from the Federal Emergency Management Agency and the state, but a
similar operation could extend into the summer, when the Pennsylvania
harvest season hits its peak, Ms. Clements-Smith said. In anticipation,
she is encouraging charities that receive food from her food bank to
apply for
\href{https://www.dep.pa.gov/Business/Land/Waste/Recycling/Municipal-Resources/FinancialAssistance/Pages/default.aspx}{state
government grants} to increase their cold-storage capacity.

``We're watching what's happening in Florida,'' she said. ``Our growing
season is coming. We're putting our heads together, to be sure we're
prepared.''

The mass-cooking campaign in Pennsylvania is a partnership with
\href{https://operationbbqrelief.org/}{Operation BBQ Relief,} a
disaster-relief organization created in 2011 to feed tornado victims in
Joplin, Mo. The group provides know-how, money and ingredients from its
supplier network to restaurants joining the effort to feed people.

``The break-even is, we have to do 2,500 meals a day at each restaurant
to make it cost effective,'' said Stan Hays, a founder and the chief
executive. ``We get chefs who are like, `We've done catering for way
more than 2,500 people before.' I say, `Have you done it 14 days in a
row?' ''

Shannon Brown saw firsthand how difficult it can be for chefs to
transition from normal restaurant service to disaster work. Ms. Brown is
an owner of \href{http://www.sonderbakersfield.com/}{Sonder}, a
restaurant in Bakersfield, Calif., that helped Operation BBQ Relief
develop the pilot program it brought to Pennsylvania.

``I've never seen food in that volume,'' Ms. Brown said. ``When they
start unloading the truck, it looks like a clown car.''

The program is similar to those run in the pandemic by the
\href{https://www.nytimes3xbfgragh.onion/2020/03/24/dining/restaurants-closing-workers-coronavirus.html}{charities}
of well-known chefs like \href{https://wck.org/}{José Andrés} and
\href{https://leeinitiative.org/}{Edward Lee}. The difference for chefs
working full-time in hunger relief is that the high-volume work never
ends.

Image

Ms. Forman carrying a tray of macaroni and cheese in the kitchen at
FoodChain.Credit...Luke Sharrett for The New York Times

Leandra Forman, a restaurant-trained chef who runs the kitchen for
\href{http://foodchainlex.org/}{FoodChain}, a nonprofit group in
Lexington, Ky., said learning to turn donated ingredients into hundreds
of meals a day takes practice. ``You can't just expect restaurant chefs
to come in and know how to do this. I see 10,000 pounds of butternut
squash, and have to figure out how to turn it into lunch.''

FoodChain, which normally focuses on food sustainability and education,
has partnered with the area's largest food bank,
\href{https://godspantry.org/}{God's Pantry}, and other local charities
to cook meals for homebound seniors, out-of-school children and elderly
refugees.

Becca Self, FoodChain's founder, is looking ahead to the coming growing
season. ``We're trying to talk to farmers now about how to make it worth
it for them to make that harvest,'' she said. ``One of our biggest crops
coming up is asparagus. It's typically grown for a high-end restaurant
market --- and those restaurants are gone now.''

Image

A meal from the FoodChain kitchen: Beef and broccoli casserole with a
side of macaroni and cheese.Credit...Luke Sharrett for The New York
Times

The \href{https://safoodbank.org/}{San Antonio Food Bank} operates two
kitchens, with construction on a third underway, and has been feeding
120,000 people a week, twice as many as it did before the pandemic.

``One of the challenges of this coronavirus environment is the demand
being greater than the supply, and not being able to offer all the
varieties of food to nourish a family,'' said Eric Cooper, the food
bank's chief executive. Modern grocery stores ``are providing more
cooked, healthy meals. With our kitchens, we can offer our clients the
same thing.''

Inside Feeding South Florida's warehouse in Boynton Beach, Ms. Benoit
looked on in early May as a construction crew installed vents in the
kitchen.

``The goal is, how much more can we create to give people a more
interesting array of options,'' she said, speaking through a face mask
decorated with images of blue crabs and shellfish. ``There are only so
many families you can just hand a 25-pound bag of apples to.''

Honey-roasted chicken, fruit salad with torn mint, and farro risotto
with asparagus are some of the first dishes Ms. Benoit expects to cook.
They'll be distributed in sealed TV-dinner-style containers.

Image

Sari M. Vatske, executive vice president of Feeding South Florida, in
the food bank's new kitchen.~Credit...Saul Martinez for The New York
Times

``We want to do a lot of citrus marinades,'' she said. ``You get a ton
of flavor and it adds juiciness to the meats. We see a lot of oranges,
lemons and limes around here.''

That abundance normally dries up in the summer, when locally grown food
becomes scarce, said Sari M. Vatske, the executive vice president of
Feeding South Florida.

She worries that better times will not return in the fall. ``If the
economy doesn't rescale, the demand from the hospitality industry will
be down, but the charitable need will be up,'' Ms. Vatske said.

Whatever happens, Ms. Benoit said she'll remain focused on the kitchen.
``All the cooking everyone is doing right now, we're doing 365 days a
year,'' she said. ``We're always food banking.''

\emph{Follow} \href{https://twitter.com/nytfood}{\emph{NYT Food on
Twitter}} \emph{and}
\href{https://www.instagram.com/nytcooking/}{\emph{NYT Cooking on
Instagram}}\emph{,}
\href{https://www.facebookcorewwwi.onion/nytcooking/}{\emph{Facebook}}\emph{,}
\href{https://www.youtube.com/nytcooking}{\emph{YouTube}} \emph{and}
\href{https://www.pinterest.com/nytcooking/}{\emph{Pinterest}}\emph{.}
\href{https://www.nytimes3xbfgragh.onion/newsletters/cooking}{\emph{Get
regular updates from NYT Cooking, with recipe suggestions, cooking tips
and shopping advice}}\emph{.}

Advertisement

\protect\hyperlink{after-bottom}{Continue reading the main story}

\hypertarget{site-index}{%
\subsection{Site Index}\label{site-index}}

\hypertarget{site-information-navigation}{%
\subsection{Site Information
Navigation}\label{site-information-navigation}}

\begin{itemize}
\tightlist
\item
  \href{https://help.nytimes3xbfgragh.onion/hc/en-us/articles/115014792127-Copyright-notice}{©~2020~The
  New York Times Company}
\end{itemize}

\begin{itemize}
\tightlist
\item
  \href{https://www.nytco.com/}{NYTCo}
\item
  \href{https://help.nytimes3xbfgragh.onion/hc/en-us/articles/115015385887-Contact-Us}{Contact
  Us}
\item
  \href{https://www.nytco.com/careers/}{Work with us}
\item
  \href{https://nytmediakit.com/}{Advertise}
\item
  \href{http://www.tbrandstudio.com/}{T Brand Studio}
\item
  \href{https://www.nytimes3xbfgragh.onion/privacy/cookie-policy\#how-do-i-manage-trackers}{Your
  Ad Choices}
\item
  \href{https://www.nytimes3xbfgragh.onion/privacy}{Privacy}
\item
  \href{https://help.nytimes3xbfgragh.onion/hc/en-us/articles/115014893428-Terms-of-service}{Terms
  of Service}
\item
  \href{https://help.nytimes3xbfgragh.onion/hc/en-us/articles/115014893968-Terms-of-sale}{Terms
  of Sale}
\item
  \href{https://spiderbites.nytimes3xbfgragh.onion}{Site Map}
\item
  \href{https://help.nytimes3xbfgragh.onion/hc/en-us}{Help}
\item
  \href{https://www.nytimes3xbfgragh.onion/subscription?campaignId=37WXW}{Subscriptions}
\end{itemize}
