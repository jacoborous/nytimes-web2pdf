Sections

SEARCH

\protect\hyperlink{site-content}{Skip to
content}\protect\hyperlink{site-index}{Skip to site index}

\href{https://www.nytimes3xbfgragh.onion/section/business/economy}{Economy}

\href{https://myaccount.nytimes3xbfgragh.onion/auth/login?response_type=cookie\&client_id=vi}{}

\href{https://www.nytimes3xbfgragh.onion/section/todayspaper}{Today's
Paper}

\href{/section/business/economy}{Economy}\textbar{}`Rolling Shock' as
Job Losses Mount Even With Reopenings

\url{https://nyti.ms/3cy1r2T}

\begin{itemize}
\item
\item
\item
\item
\item
\item
\end{itemize}

\hypertarget{the-coronavirus-outbreak}{%
\subsubsection{\texorpdfstring{\href{https://www.nytimes3xbfgragh.onion/news-event/coronavirus?name=styln-coronavirus-markets\&region=TOP_BANNER\&block=storyline_menu_recirc\&action=click\&pgtype=Article\&impression_id=42c81110-f52e-11ea-bd9d-0b405bc82d81\&variant=undefined}{The
Coronavirus
Outbreak}}{The Coronavirus Outbreak}}\label{the-coronavirus-outbreak}}

\begin{itemize}
\tightlist
\item
  live\href{https://www.nytimes3xbfgragh.onion/2020/09/12/world/covid-19-coronavirus.html?name=styln-coronavirus-markets\&region=TOP_BANNER\&block=storyline_menu_recirc\&action=click\&pgtype=Article\&impression_id=42c81111-f52e-11ea-bd9d-0b405bc82d81\&variant=undefined}{Latest
  Updates}
\item
  \href{https://www.nytimes3xbfgragh.onion/interactive/2020/us/coronavirus-us-cases.html?name=styln-coronavirus-markets\&region=TOP_BANNER\&block=storyline_menu_recirc\&action=click\&pgtype=Article\&impression_id=42c81112-f52e-11ea-bd9d-0b405bc82d81\&variant=undefined}{Maps
  and Cases}
\item
  \href{https://www.nytimes3xbfgragh.onion/interactive/2020/science/coronavirus-vaccine-tracker.html?name=styln-coronavirus-markets\&region=TOP_BANNER\&block=storyline_menu_recirc\&action=click\&pgtype=Article\&impression_id=42c81113-f52e-11ea-bd9d-0b405bc82d81\&variant=undefined}{Vaccine
  Tracker}
\item
  \href{https://www.nytimes3xbfgragh.onion/2020/09/10/us/politics/fda-coronavirus-vaccine.html?name=styln-coronavirus-markets\&region=TOP_BANNER\&block=storyline_menu_recirc\&action=click\&pgtype=Article\&impression_id=42c83820-f52e-11ea-bd9d-0b405bc82d81\&variant=undefined}{F.D.A.
  Regulators' Self-Defense}
\item
  \href{https://www.nytimes3xbfgragh.onion/2020/09/09/upshot/coronavirus-surprise-test-fees.html?name=styln-coronavirus-markets\&region=TOP_BANNER\&block=storyline_menu_recirc\&action=click\&pgtype=Article\&impression_id=42c83821-f52e-11ea-bd9d-0b405bc82d81\&variant=undefined}{Surprise
  Test Fees}
\end{itemize}

Advertisement

\protect\hyperlink{after-top}{Continue reading the main story}

Supported by

\protect\hyperlink{after-sponsor}{Continue reading the main story}

\hypertarget{rolling-shock-as-job-losses-mount-even-with-reopenings}{%
\section{`Rolling Shock' as Job Losses Mount Even With
Reopenings}\label{rolling-shock-as-job-losses-mount-even-with-reopenings}}

Nearly three million new unemployment claims brought the two-month total
to more than 36 million, even with some still frustrated in seeking
benefits.

\href{https://www.nytimes3xbfgragh.onion/by/patricia-cohen}{\includegraphics{https://static01.graylady3jvrrxbe.onion/images/2018/02/16/multimedia/author-patricia-cohen/author-patricia-cohen-thumbLarge.jpg}}\href{https://www.nytimes3xbfgragh.onion/by/tiffany-hsu}{\includegraphics{https://static01.graylady3jvrrxbe.onion/images/2018/12/06/multimedia/author-tiffany-hsu/author-tiffany-hsu-thumbLarge.png}}

By \href{https://www.nytimes3xbfgragh.onion/by/patricia-cohen}{Patricia
Cohen} and
\href{https://www.nytimes3xbfgragh.onion/by/tiffany-hsu}{Tiffany Hsu}

\begin{itemize}
\item
  Published May 14, 2020Updated June 11, 2020
\item
  \begin{itemize}
  \item
  \item
  \item
  \item
  \item
  \item
  \end{itemize}
\end{itemize}

\href{https://www.nytimes3xbfgragh.onion/interactive/2020/us/states-reopen-map-coronavirus.html}{Scattershot
reopenings} of retail stores, nail salons and restaurants around the
country have not halted the flood of layoffs, with the
\href{https://www.dol.gov/ui/data.pdf}{government reporting} Thursday
that nearly three million people filed
\href{https://www.nytimes3xbfgragh.onion/2020/06/11/business/economy/unemployment-claims-coronavirus.html}{unemployment}
claims last week, bringing the two-month tally to more than 36 million.

The
\href{https://www.nytimes3xbfgragh.onion/2020/06/04/business/economy/coronavirus-unemployment-claims.html}{weekly
count of new claims has been declining} since late March, but that
hopeful flicker barely stands out in an otherwise grim and chaotic
economic landscape.

``This is a very protracted, painful situation for the labor market,''
said Rubeela Farooqi, chief U.S. economist at High Frequency Economics,
``and I just don't see anything positive.''

In places where the fitful reopening has started, workers called back to
their
\href{https://www.nytimes3xbfgragh.onion/2020/05/28/business/economy/coronavirus-stimulus-unemployment.html}{jobs}often
face reduced hours and paychecks as well as a heightened risk of
infection. Declining to return, however, is likely to put an end to any
jobless benefits.

``It's a very tough choice for those in the service industry and those
at the lower end of the pay scale,'' Ms. Farooqi said. ``Do you go back
and risk getting sick, or have no money coming in?''

Lags in data make it hard to calculate just how many workers may have
been rehired after the most recent shelter-in-place restrictions were
lifted. And Connecticut
\href{https://twitter.com/CTDOL/status/1260962610729619456/photo/1}{cited
an error} in the government's report that appeared to have inflated the
state's latest claims by more than 200,000.

But Michelle Meyer, head of U.S. economics at Bank of America, said she
doubted that callbacks to work outnumbered additional layoffs from other
sectors. The slowdown has been rippling beyond the early shutdowns in
retail and hospitality to professional business services, manufacturing
and health care.

``In a sense, it's a rolling shock,'' she said.

Georgia, one of the first states to reopen, is an example. ``The
reopening is bringing people back to work, reducing the total amount of
people receiving
\href{https://www.nytimes3xbfgragh.onion/2020/05/21/business/stock-market-today-coronavirus.html}{unemployment}
insurance,'' Ms. Meyer noted. ``But the number of initial jobless claims
is still rising, which suggests there is still residual weakness in the
economy.''

In an analysis of the latest
\href{https://www.nytimes3xbfgragh.onion/2020/05/28/business/economy/coronavirus-unemployment-claims.html}{unemployment}-claims
report, the U.S. Chamber of Commerce found that in
\href{https://www.uschamber.com/series/above-the-fold/analysis-these-10-states-have-seen-the-highest-share-of-their-workforce-file}{11
states}, more than a quarter of those in the work force in February were
now unemployed. And a
\href{https://www.nytimes3xbfgragh.onion/2020/05/14/business/economy/coronavirus-poor-economy.html}{survey
by the Federal Reserve} found that in households making less than
\$40,000 a year, nearly 40 percent of those who were working in February
lost their jobs in March or the beginning of April.

For millions of Americans, government benefits have provided a crucial
lifeline. Unemployment programs alone
delivered\href{https://fsapps.fiscal.treasury.gov/dts/files/20043000.pdf}{\$48
billion in payments} in April, according to the Treasury Department.

\hypertarget{latest-updates-the-coronavirus-outbreak-and-the-economy}{%
\section{\texorpdfstring{\href{https://www.nytimes3xbfgragh.onion/live/2020/09/11/business/stock-market-today-coronavirus?action=click\&pgtype=Article\&state=default\&region=MAIN_CONTENT_1\&context=storylines_live_updates}{Latest
Updates: The Coronavirus Outbreak and the
Economy}}{Latest Updates: The Coronavirus Outbreak and the Economy}}\label{latest-updates-the-coronavirus-outbreak-and-the-economy}}

\href{https://www.nytimes3xbfgragh.onion/live/2020/09/11/business/stock-market-today-coronavirus?action=click\&pgtype=Article\&state=default\&region=MAIN_CONTENT_1\&context=storylines_live_updates\#the-nyse-may-move-its-data-center-out-of-new-jersey-in-response-to-a-proposed-tax}{23h
ago}

\href{https://www.nytimes3xbfgragh.onion/live/2020/09/11/business/stock-market-today-coronavirus?action=click\&pgtype=Article\&state=default\&region=MAIN_CONTENT_1\&context=storylines_live_updates\#the-nyse-may-move-its-data-center-out-of-new-jersey-in-response-to-a-proposed-tax}{The
N.Y.S.E. may move its data center out of New Jersey in response to a
proposed tax.}

\href{https://www.nytimes3xbfgragh.onion/live/2020/09/11/business/stock-market-today-coronavirus?action=click\&pgtype=Article\&state=default\&region=MAIN_CONTENT_1\&context=storylines_live_updates\#the-federal-budget-deficit-hit-3-trillion-as-of-august}{25h
ago}

\href{https://www.nytimes3xbfgragh.onion/live/2020/09/11/business/stock-market-today-coronavirus?action=click\&pgtype=Article\&state=default\&region=MAIN_CONTENT_1\&context=storylines_live_updates\#the-federal-budget-deficit-hit-3-trillion-as-of-august}{The
federal budget deficit hit \$3 trillion as of August.}

\href{https://www.nytimes3xbfgragh.onion/live/2020/09/11/business/stock-market-today-coronavirus?action=click\&pgtype=Article\&state=default\&region=MAIN_CONTENT_1\&context=storylines_live_updates\#warner-bros-pushes-the-release-of-wonder-woman-1984-to-christmas}{26h
ago}

\href{https://www.nytimes3xbfgragh.onion/live/2020/09/11/business/stock-market-today-coronavirus?action=click\&pgtype=Article\&state=default\&region=MAIN_CONTENT_1\&context=storylines_live_updates\#warner-bros-pushes-the-release-of-wonder-woman-1984-to-christmas}{Warner
Bros. pushes the release of `Wonder Woman 1984' to Christmas.}

\href{https://www.nytimes3xbfgragh.onion/live/2020/09/11/business/stock-market-today-coronavirus?action=click\&pgtype=Article\&state=default\&region=MAIN_CONTENT_1\&context=storylines_live_updates}{See
more updates}

More live coverage:
\href{https://www.nytimes3xbfgragh.onion/2020/09/11/world/covid-19-coronavirus.html?action=click\&pgtype=Article\&state=default\&region=MAIN_CONTENT_1\&context=storylines_live_updates}{Global}

But even as states strive to keep up with the onslaught of claims, many
workers remain supremely frustrated, either by their inability to submit
applications or by payment delays. According to a poll for The New York
Times in early May by the online research firm SurveyMonkey, more than
half of those applying for unemployment benefits in recent weeks were
unsuccessful.

And as of Saturday, 20 states, the District of Columbia and Puerto Rico
had not paid out any money under the Pandemic Unemployment Assistance
program, which Congress passed in March to help freelancers, the
self-employed and other workers not normally eligible for state jobless
benefits.

Some of those being called back to work have never seen a penny of
government aid.

Jason Cooper, 43, went back to his serving job at the Savour restaurant
in Tallahassee, Fla., in early May without ever receiving the jobless
benefits he spent weeks trying to track down.

\includegraphics{https://static01.graylady3jvrrxbe.onion/images/2020/05/14/business/14virus-jobless8a/merlin_172469562_d0397da9-fde3-4137-91ab-95dea308b508-articleLarge.jpg?quality=75\&auto=webp\&disable=upscale}

He was furloughed on March 14 and tried for the next month to file for
government aid. In mid-April, a few days after he successfully submitted
a claim, his employer secured a loan and began paying workers again.

Mr. Cooper was lucky: he had some savings that helped cover four weeks
without income, and he temporarily moved in with his mother in St.
Augustine, Fla. He plans to wait out the pandemic before trying to go to
a state office to claim the backdated benefits he believes he is owed.

``It was so difficult to get through the first time that I have no real
faith that the system is going to work anytime in the near future,'' he
said.

At first, Mr. Cooper was concerned about returning to work, but now he
feels comfortable in the small restaurant, where employees are not
wearing masks but are using single-use gloves for serving. Tables have
also been spaced farther apart. ``It feels shockingly normal,'' he said.

Other workers who have been called back say they are scared of getting
sick but feel they have no choice. Some states are taking a hard line.
Nebraska \href{https://dol.nebraska.gov/PressRelease/Details/153}{posted
a notice online} declaring that failing to return to work ``could be
considered fraud'' and potentially disqualify people from receiving
benefits. In South Carolina, workers are
\href{https://dew.sc.gov/docs/default-source/default-document-library/covid-19-related-ui-information.pdf?sfvrsn=133a12f0_0}{not
eligible for benefits} if they don't work because they are isolating
themselves to avoid exposure to the virus or have to care for children
while schools are closed. Iowa has
\href{https://www.iowaworkforcedevelopment.gov/job-offer-decline-form-employers}{an
online form} for ``employers to report unemployed claimants who have
refused legitimate job offers.''

Sarah Parker, 26, a customer service manager at a store in Chillicothe,
Ohio, was asked by her employer last week to prepare to return to work
in a small shopping center.

Image

Sarah Parker, a customer service manager at an Ohio store, said her
paycheck after returning to work would be about half what she received
in government benefits.Credit...Maddie McGarvey for The New York Times

``On one side, I really love my job --- it's my favorite place to be,''
she said. ``On the other hand, I'm petrified. I'm afraid I'm going to
put myself at way more risk working harder for less pay.''

She estimates that her new paycheck will be half of the roughly \$800 a
week that she received in government aid, but said she did not want to
subsist on benefit payments.

After the pandemic grabbed hold in March, Washington began sending
\$1,200 stimulus checks to most households. Congress also enacted
emergency benefits that expanded unemployment insurance to gig workers
and others not covered under state programs, and provided a weekly
supplement of \$600 through July. To help laid-off workers who exhausted
their state benefits, the government extended unemployment insurance for
13 weeks.

For many low-wage workers who applied successfully, the additional \$600
meant their
\href{https://www.nytimes3xbfgragh.onion/interactive/2020/04/23/business/economy/unemployment-benefits-stimulus-coronavirus.html}{weekly
income matched or even exceeded their regular paycheck}. Many would
prefer to wait for the health risks to recede but fear losing their
benefits.

Kelty Stanton, an 18-year-old community college student, said that she
was happy to be asked to return to her job next week at a restaurant and
grocery near Titusville, N.J., but that she was stunned when she read an
email outlining the terms.

Image

``I'm relieved that I have a job again, but at the same time, I also
feel unsafe,'' said Kelty Stanton, who was called back to work at a
restaurant and grocery in New Jersey.Credit...Michelle Gustafson for The
New York Times

To stay on as an employee, she was required to return a signed and dated
copy of an attached letter or risk being fired. Her company said that if
she declined to go back to work, New Jersey's unemployment department
would be notified and her benefits removed.

The letter stressed that the company was committed to a safe and healthy
workplace, with ``frequent disinfection of surfaces, social distancing
rules, reduced customer capacity, staggered shifts.''

``I'm relieved that I have a job again,'' Ms. Stanton said, ``but at the
same time, I also feel unsafe going back. I'm nervous and kind of
anxious. But it's not like they're giving me much of a choice.''

She said she planned to buy some masks and extra hand sanitizer before
she starts on Monday.

Some employers are asking returning workers to sign
\href{https://twitter.com/Gilly0n/status/1258873334827061248?s=20}{waivers}
absolving companies of liability if their workers get sick.

A survey released last week by the National Federation of Independent
Business, a trade group that has often supported lower taxes and
deregulation, found that 68 percent of small-business owners were
worried about increased liability claims if they reopened.

This week, social media users
\href{https://twitter.com/Gilly0n/status/1258873334827061248?s=20}{shared
a letter} from the Las Vegas restaurant chain
\href{https://twitter.com/tripple_OD/status/1259698602185764865?s=20}{Nacho
Daddy} that asked returning employees to promise not to take legal
action if they contracted the virus at work.

The president of Nacho Daddy, Paul Hymas, said in an emailed statement
on Wednesday that the company had since removed that liability language
from its hiring process.

Ben Casselman contributed reporting.

Advertisement

\protect\hyperlink{after-bottom}{Continue reading the main story}

\hypertarget{site-index}{%
\subsection{Site Index}\label{site-index}}

\hypertarget{site-information-navigation}{%
\subsection{Site Information
Navigation}\label{site-information-navigation}}

\begin{itemize}
\tightlist
\item
  \href{https://help.nytimes3xbfgragh.onion/hc/en-us/articles/115014792127-Copyright-notice}{©~2020~The
  New York Times Company}
\end{itemize}

\begin{itemize}
\tightlist
\item
  \href{https://www.nytco.com/}{NYTCo}
\item
  \href{https://help.nytimes3xbfgragh.onion/hc/en-us/articles/115015385887-Contact-Us}{Contact
  Us}
\item
  \href{https://www.nytco.com/careers/}{Work with us}
\item
  \href{https://nytmediakit.com/}{Advertise}
\item
  \href{http://www.tbrandstudio.com/}{T Brand Studio}
\item
  \href{https://www.nytimes3xbfgragh.onion/privacy/cookie-policy\#how-do-i-manage-trackers}{Your
  Ad Choices}
\item
  \href{https://www.nytimes3xbfgragh.onion/privacy}{Privacy}
\item
  \href{https://help.nytimes3xbfgragh.onion/hc/en-us/articles/115014893428-Terms-of-service}{Terms
  of Service}
\item
  \href{https://help.nytimes3xbfgragh.onion/hc/en-us/articles/115014893968-Terms-of-sale}{Terms
  of Sale}
\item
  \href{https://spiderbites.nytimes3xbfgragh.onion}{Site Map}
\item
  \href{https://help.nytimes3xbfgragh.onion/hc/en-us}{Help}
\item
  \href{https://www.nytimes3xbfgragh.onion/subscription?campaignId=37WXW}{Subscriptions}
\end{itemize}
