Sections

SEARCH

\protect\hyperlink{site-content}{Skip to
content}\protect\hyperlink{site-index}{Skip to site index}

\href{https://www.nytimes3xbfgragh.onion/section/world/americas}{Americas}

\href{https://myaccount.nytimes3xbfgragh.onion/auth/login?response_type=cookie\&client_id=vi}{}

\href{https://www.nytimes3xbfgragh.onion/section/todayspaper}{Today's
Paper}

\href{/section/world/americas}{Americas}\textbar{}As Bolsonaro Keeps
Amazon Vows, Brazil's Indigenous Fear `Ethnocide'

\url{https://nyti.ms/34OWXl6}

\begin{itemize}
\item
\item
\item
\item
\item
\item
\end{itemize}

Advertisement

\protect\hyperlink{after-top}{Continue reading the main story}

Supported by

\protect\hyperlink{after-sponsor}{Continue reading the main story}

Promises Made

\hypertarget{as-bolsonaro-keeps-amazon-vows-brazils-indigenous-fear-ethnocide}{%
\section{As Bolsonaro Keeps Amazon Vows, Brazil's Indigenous Fear
`Ethnocide'}\label{as-bolsonaro-keeps-amazon-vows-brazils-indigenous-fear-ethnocide}}

President Jair Bolsonaro is moving aggressively to open up the Amazon
rainforest to commercial development, posing an existential threat to
the tribes living there.

\includegraphics{https://static01.graylady3jvrrxbe.onion/images/2020/03/05/world/Brazil-Promises-Top/Brazil-Promises-Top-articleLarge.jpg?quality=75\&auto=webp\&disable=upscale}

By \href{https://www.nytimes3xbfgragh.onion/by/ernesto-londono}{Ernesto
Londoño} and Letícia Casado

\begin{itemize}
\item
  Published April 19, 2020Updated April 24, 2020
\item
  \begin{itemize}
  \item
  \item
  \item
  \item
  \item
  \item
  \end{itemize}
\end{itemize}

\href{https://www.nytimes3xbfgragh.onion/es/2020/04/19/espanol/america-latina/bolsonaro-brasil-amazonia-indigena.html}{Leer
en
español}\href{https://www.nytimes3xbfgragh.onion/pt/2020/04/19/world/americas/bolsonaro-brasil-amazonia-indigenas-funai.html}{Ler
em português}

URU EU WAU WAU TERRITORY, Brazil --- The billboard at the entrance of a
tiny Indigenous village in the Amazon has become a relic in less than a
decade, boasting of something no longer true.

``Here, there is investment by the federal government,'' proclaims the
sign, erected in 2012, which is now shrouded by fallen palm tree fronds.

In fact, this tiny hamlet in Rondônia state, called Alto Jamari, home to
some 10 families of the Uru Eu Wau Wau tribe, is barely surviving, just
like scores of other struggling villages in the region that for decades
have served as havens for Indigenous culture and
\href{http://www.edf.org/sites/default/files/tropical-forest-carbon-in-indigenous-territories-a-global-analysis.pdf}{bulwarks
against deforestation} in
\href{https://www.nytimes3xbfgragh.onion/2020/04/24/world/americas/brazil-bolsonaro-moro.html}{Brazil.}

Federal aid is drying up at the same time that more outsiders are
trespassing on their lands, eager to illegally exploit the forest's
resources, and as the coronavirus poses a deadly threat, having already
reached a few remote villages.

Local leaders and Indigenous advocates direct their blame for this
deteriorating situation toward one person:
\href{https://www.nytimes3xbfgragh.onion/2020/04/24/world/americas/brazil-bolsonaro-moro.html}{President
Jair Bolsonaro}.

During his run for the presidency, Mr. Bolsonaro promised he would open
up the Amazon to more commercial development, including mining and
large-scale farming.

``Where there is Indigenous land,'' he has said, ``there is wealth
underneath it.''

\includegraphics{https://static01.graylady3jvrrxbe.onion/images/2020/04/19/world/19brazil-promises1/00brazil-promises9-articleLarge.jpg?quality=75\&auto=webp\&disable=upscale}

Since taking office a little more than a year ago, Mr. Bolsonaro has
moved aggressively to further those development goals, putting in place
policies that critics fear have set in motion a new era of ethnocide for
Indigenous communities.

He has started dismantling a system of protection for Indigenous
communities enshrined in Brazil's Constitution, with his government last
year slashing the funding of the National Indian Foundation, the federal
agency responsible for upholding those Indigenous rights.

As president, he has vowed not to designate ``one centimeter'' more as
protected Indigenous lands, arguing that living in isolation is an
anachronism in the 21st century and an impediment to economic growth.

``The Indigenous person can't remain in his land as if he were some
prehistoric creature,'' Mr. Bolsonaro
\href{https://valor.globo.com/politica/noticia/2020/02/18/indio-nao-pode-ficar-na-sua-terra-como-ser-pre-historico-diz-bolsonaro.ghtml}{said
in February}.

Also in February, Mr. Bolsonaro presented a bill to Congress that could
effectively legalize
\href{https://www.nytimes3xbfgragh.onion/2018/11/10/world/americas/brazil-indigenous-mining-bolsonaro.html}{the
illegal mining ventures that have polluted rivers} and torn down large
swaths of the Amazon.

The proposed legislation, which Congress has shown no appetite to
advance as Brazil battles the coronavirus, would also authorize oil and
gas exploration and hydropower plants on Indigenous territories. Under
the plan, native communities would be consulted about projects ---
\href{https://agenciabrasil.ebc.com.br/politica/noticia/2020-02/bolsonaro-envia-projeto-que-regulamenta-exploracao-de-terras-indigenas}{but
would not be given veto power.}

Last year, Mr. Bolsonaro bragged that he had
``\href{https://www.nytimes3xbfgragh.onion/2019/08/27/world/americas/bolsonaro-brazil-environment.html}{put
an end to}'' what he called ``astronomical fines'' against companies
that violate environmental law in the Amazon, removing one of the few
disincentives developers face.

Brazil's president is keeping his promises about expanding development
in the Amazon. And for many of the Indigenous people who live there, the
Bolsonaro era is posing an existential threat.

What We Found

\hypertarget{razing-down-our-forest}{%
\subsection{`Razing Down Our Forest'}\label{razing-down-our-forest}}

Image

Uru Eu Wau Wau children at play.Credit...Victor Moriyama for The New
York Times

Brazil's 1988 Constitution confers expansive rights to Brazil's
Indigenous people, a form of reparations for centuries of brutal
treatment.

While these rights have never been fully upheld, they are being
eviscerated in the Bolsonaro era, according to Indigenous leaders and
activists.

For communities with small populations, like the Uru Eu Wau Wau, the
government's stance could mean their total disappearance as distinct
tribes.

The schoolhouse at the largest of the Uru Eu Wau Wau's six villages ---
a modern facility surrounded by a cluster of modest huts --- sits empty.
Teachers stopped showing up last year because they weren't being paid.

VENEZUELA

FRENCH

GUIANA

Atlantic

Ocean

Amazon R.

amazon

rainforest

Porto Velho

RONDÔNIA

BRAZIL

Uru Eu Wau Wau

Territory

PERU

BOLIVIA

Rio de Janeiro

CHILE

ARGENTINA

500 miles

By The New York Times

Visits from doctors and nurses have become rare, in large part because
Cuban doctors who had been providing care in remote villages left
abruptly shortly before Mr. Bolsonaro took office in January 2019
\href{https://www.nytimes3xbfgragh.onion/2019/06/11/world/americas/brazil-cuba-doctors-jair-bolsonaro.html}{in
response to threats} from the incoming president.

Illegal incursions by loggers into the edges of the territory have
become increasingly frequent, putting its residents on a war footing.

``They're razing down our forest,'' Juvitai Uru Eu Wau Wau, 19, said
while swinging on a hammock as a toddler pushed a dusty tricycle around
a cluster of small huts. As is common, Juvitai uses the tribe's name as
her family name.

Children in the village have picked up on the collective angst, Juvitai
said, and constantly ask whether their days living in relative isolation
are coming to an end.

``I tell them to be calm,'' Juvitai said, sounding uncertain. ``This is
our land. We're staying here.''

What We Found

\hypertarget{a-government-that-is-in-favor-of-deforestation}{%
\subsection{`A Government That Is in Favor of
Deforestation'}\label{a-government-that-is-in-favor-of-deforestation}}

Image

People relaxing in an Uru Eu Wau Wau village.Credit...Victor Moriyama
for The New York Times

On a satellite image, the Uru Eu Wau Wau territory stands out as an
emerald green island surrounded by parcels of razed forest, most of
which are now
\href{https://www.nytimes3xbfgragh.onion/2019/10/10/world/americas/amazon-fires-brazil-cattle.html}{cattle
ranches}.

In 1991, the federal government officially designated the Uru Eu Wau Wau
territory. It encompasses a 6,950 square mile area --- a little smaller
than the state of New Jersey --- where the tribe has built a cluster of
small villages. This federal recognition is supposed to confer limited
political autonomy, prohibiting outsiders from entering without explicit
permission and barring large-scale commercial activity.

The territory, still technically owned by the federal government, is now
home to about 220 Uru Eu Wau Wau people, as well as a few smaller
\href{https://www.nytimes3xbfgragh.onion/2017/09/10/world/americas/brazil-amazon-tribe-killings.html}{uncontacted
tribes} whose exact populations are unknown.

The Uru Eu Wau Wau have endured illegal incursions from loggers for
years. But in February of last year, it became clear the tribe was
facing a far graver threat when some 200 men strode into their territory
with the apparent intent to establish a permanent settlement.

After the Uru Eu Wau Wau protested and the incursion drew the attention
of the Brazilian news media, the federal police did step in to expel the
men. But such enforcement actions are rare, and it's impossible for the
authorities to effectively patrol such a vast region, which both the
loggers and tribes know well.

Soon after the police left, someone opened fire on a government plaque
at one of the main entrances to the territory that signals that the area
is protected. It sent a chilling message to the Uru Eu Wau Wau.

``What we're seeing is the result of a government that is in favor of
deforestation in the Amazon,'' said Bitate Uru Eu Wau Wau, a leader in
the community. ``It has emboldened invaders to come into Indigenous
territories.''

What We Found

\hypertarget{lawbreakers-take-comfort-in-the-political-reality}{%
\subsection{Lawbreakers `Take Comfort in the Political
Reality'}\label{lawbreakers-take-comfort-in-the-political-reality}}

Image

Dairy cattle on the edge of Indigenous territory in Rondônia
state.Credit...Victor Moriyama for The New York Times

Federal prosecutors in the state said these incursions are part of a
wave of illegal squatters who raze protected land, harvest the wood and
then carve out land parcels for which they create fake titles.

Loggers, miners, cattle ranchers and others have used this approach in
the Amazon for many years, and it has often paid off because lawmakers
have time and again created pathways for squatters to rightfully own
land they took possession of unlawfully.

But while their tactics are not new, prosecutors say the squatters have
become increasingly brazen since Mr. Bolsonaro's election, abetted by
his disdain for environmental fines and the government's attitude toward
development.

``The objective is to create facts on the ground,'' said Daniel Azevedo,
a federal prosecutor in Porto Velho, the Rondônia state capital, who
focuses on environmental and Indigenous crimes.

Deforestation in Indigenous territories across Brazil has risen sharply
in recent months. From August 2018 to July 2019, 1,634 square miles of
forest cover was slashed, according to Brazil's National Institute for
Space Studies. That represents a 74 percent increase from the same
period a year before.

The Uru Eu Wau Wau territory was among the 10 hardest hit by
deforestation during that time.

Mr. Azevedo said law enforcement officials can build cases against
particularly egregious drivers of deforestation. But he added the
authorities are ill equipped to roll back the forces driving
deforestation at a time when squatters feel backed by elected officials.

``They take comfort in the political reality, sensing that local
politicians, senators, even the president supports their cause,'' Mr.
Azevedo said.

The Uru Eu Wau Wau is one of several Indigenous communities that have
seen a sharp rise in land incursions and threats in the Bolsonaro era.
Further north, the Yanomami and Munduruku tribes have
\href{https://news.mongabay.com/2019/07/yanomami-amazon-reserve-invaded-by-20000-miners-bolsonaro-fails-to-act/}{been
invaded by thousands of gold miners.}

In 2019,
\href{https://g1.globo.com/natureza/noticia/2019/12/10/mortes-de-liderancas-indigenas-batem-recorde-em-2019-diz-pastoral-da-terra.ghtml}{at
least seven Indigenous leaders}were killed in conflicts over land.

At a meeting last year with the governors of Brazil's nine Amazonian
states, Mr. Bolsonaro made clear he saw Indigenous lands and their
inhabitants as a drag on Brazil's potential.

``Indigenous people don't lobby, don't speak our language, and yet today
they manage to have 14 percent of our national territory,''
\href{https://www.nytimes3xbfgragh.onion/2019/08/27/world/americas/bolsonaro-brazil-environment.html}{he
said}, using a figure slightly larger than the government's own
statistics. ``One of their intentions is to hold us back.''

What We Found

\hypertarget{they-dont-bring-in-money-for-brazil-only-burdens}{%
\subsection{`They Don't Bring in Money for Brazil, Only
Burdens'}\label{they-dont-bring-in-money-for-brazil-only-burdens}}

Image

Daniel da Cunha owns a bar near the Uru Eu Wau Wau territory and is in
favor of clearing the surrounding forest for livestock
production.Credit...Victor Moriyama for The New York Times

Mr. Bolsonaro, who
\href{https://www.nytimes3xbfgragh.onion/2018/10/28/world/americas/jair-bolsonaro-brazil-election.html}{won
the presidency} with 55 percent of the vote, has many supporters who
agree with his contention that Indigenous communities should not be in
control of the 12.5 percent of the country's landmass demarcated as
Indigenous land.

Daniel da Cunha, 60, who lives just outside the Uru Eu Wau Wau
territory, said those territories should be carved up so jobless people
can put them to profitable use.

``They don't work,'' he said of Indigenous people. ``They don't bring in
money for Brazil, only burdens.''

Some lawmakers argue that Mr. Bolsonaro is right to want to upend
Brazil's Indigenous policy, but favor a more moderate approach.

Arthur Oliveira Maia, a center-right congressman from the state of
Bahia, said that under the current legal framework, no one, including
the Indigenous tribes themselves, can profit from the reserved
territories.

``Commercial endeavors in Indigenous territories could be done
gradually, setting aside 10 or 15 percent of the land,'' he said.

He added that he favored starting out with agriculture, which tends to
have a lower environmental impact, rather than mining.

``Today Indigenous people are struggling,'' he said. ``The emancipation
of these people is only possible through economic means.''

What We Found

\hypertarget{a-past-of-horrors-a-present-of-cuts}{%
\subsection{A Past of Horrors, a Present of
Cuts}\label{a-past-of-horrors-a-present-of-cuts}}

Image

Members of Uru Eu Wau Wau tribe, clockwise from top left; Tapé, Oatuto,
Tebu, Poajup, Mandeí and Felipe. All share the tribe's name as a family
name.Credit...Victor Moriyama for The New York Times

Mr. Bolsonaro has long spoken derisively about Indigenous people. In
1998, when he was a fringe far-right lawmaker, Mr. Bolsonaro said it was
a ``shame that the Brazilian cavalry hadn't been as efficient as the
American one, which exterminated the Indians.''

What Mr. Bolsonaro did not acknowledge is that Brazil's Indigenous
people were almost wiped out after Europeans arrived in the early 16th
century.

The Indigenous population in modern day Brazil plunged from estimates of
between three million and as many
as\href{https://www.survivalinternational.org/tribes/brazilian}{11
million} people in the
1500s\href{http://www.funai.gov.br/index.php/indios-no-brasil/quem-sao}{to
70,000} by the 1950s as entire tribes were killed off, while huge
numbers were enslaved.

After Brazil's generals seized power in the 1960s, the repressive
military government --- which Mr. Bolsonaro
\href{https://www.nytimes3xbfgragh.onion/2019/03/29/world/americas/brazil-bolsonaro-coup.html}{has
long lionized} --- treated Indigenous people living in the Amazon as
obstacles to economic growth.

The country's 1988 Constitution tried to redress some of these wrongs.

It ended the military-era policy that had encouraged the assimilation of
Indigenous people and recognized their ``customs, languages, beliefs and
traditions.''

The Constitution also established a process of land demarcation that
over the years created the vast patchwork of 567 protected Indigenous
territories. In 2010, when Brazil conducted its last census,
\href{https://censo2010.ibge.gov.br/noticias-censo?busca=1\&id=3\&idnoticia=2194\&t=censo-2010-poblacao-indigena-896-9-mil-tem-305-etnias-fala-274\&view=noticia}{about
517,000} of the country's 897,000 Indigenous people lived in those
lands.

On his first day in office, Mr. Bolsonaro transferred the land
demarcation process from the National Indian Foundation, known as FUNAI,
to the Ministry of Agriculture, which is heavily influenced by the
agribusiness lobby. The Supreme Court blocked the move, finding it
unconstitutional, but all pending demarcation cases remain frozen.

In addition to the challenge on transferring FUNAI, Mr. Bolsonaro has
encountered other setbacks or delays. Leaders in Congress have signaled
they
\href{https://g1.globo.com/politica/noticia/2020/02/18/momento-nao-e-adequado-para-discutir-projeto-sobre-mineracao-em-terras-indigenas-diz-maia.ghtml}{are
not in a hurry}to move forward on his bill to authorize energy projects
in Indigenous lands.

But the power of the presidency still gives him plenty of opportunity to
further his vision.

The government
recently\href{https://www.nytimes3xbfgragh.onion/2020/02/05/world/americas/Brazil-indigenous-missionary.html}{appointed
a former Christian missionary}, Ricardo Lopes Dias, to head the FUNAI
division in charge of protecting uncontacted tribes. While Mr. Dias has
pledged not to use his post to proselytize, his appointment incited
fears the government will allow missionaries to make contact with
isolated communities, which are vulnerable to dying en mass from common
diseases during such encounters.

A representative of FUNAI said the agency is investing in
entrepreneurship and sustainability programs like artisanal fishing and
small-scale honey-making ventures that are meant to encourage the
autonomy of Indigenous communities.

For years before Mr. Bolsonaro became president, FUNAI had already been
contending with personnel shortages and lean budgets, which forced the
agency to abandon several outposts in remote areas and cut the frequency
of visits to villages.

While the agency's authorized budget had remained relatively steady in
recent years, the Bolsonaro administration
\href{https://indigenistasassociados.org.br/2019/10/23/proposta-de-plano-plurianual-e-de-orcamento-de-2020-enfraquece-a-funai/}{made
a sharp cut to programmatic spending for 2020}, earmarking \$9 million
for programs to uphold Indigenous rights, about 40 percent less than the
year before.

The association that represents career employees at the agency said in a
statement the reduction means FUNAI has an increasingly thin presence on
the ground, leaving those communities besieged by land grabbers at
greater risk.

``This is the first time in which government planning,'' the employee
association said, ``does not contemplate the Indigenous rights
guaranteed by the Constitution.''

What We Found

\hypertarget{if-we-dont-kill-it-will-get-worse}{%
\subsection{`If We Don't Kill, It Will Get
Worse'}\label{if-we-dont-kill-it-will-get-worse}}

Image

Awapy Uru Eu Wau Wau heating arrows that carry a natural poison
extracted from a local tree.Credit...Victor Moriyama for The New York
Times

Whenever the Uru Eu Wau Wau learn of new incursions into their
territory, they set out on foot to survey the damage and burn settler
encampments. As a group prepared at dawn for one recent expedition,
warriors in the tribe slathered poison on the tips of their arrows.

Ivaneide Bandeira Cardozo, an activist who often accompanies the Uru Eu
Wau Wau, looked ashen, fearing what a confrontation with loggers could
lead to.

``You need to promise me that if you run into them you won't kill,'' she
pleaded.

``If we don't kill, it will get worse day by day,'' one of the men
responded.

During an arduous six-hour hike through dense forest, the Uru Eu Wau Wau
waded through water and clouds of buzzing insects to reach a large
stretch of land that had recently been reduced to ashes.

The Uru Eu Wau Wau could do little more than take photos of the damage
and then set fire to the small encampment.

When asked about what the Bolsonaro administration's policies may do to
communities like these, Ms. Cardozo, who has supported the tribe for
decades, looked dejected.

``Their objective is to force them from their lands and turn them into
ordinary citizens in the periphery of cities, into beggars,'' she said.
``To me that amounts to a policy of genocide and ethnocide.''

One of the oldest members of the tribe, Borea Uru Eu Wau Wau, has scars
on her back from bullet wounds she suffered during an ambush by rubber
trappers in the 1980s. A sister, aunt and grandmother were killed then,
she recalled.

Since the new wave of incursions began, Borea has experienced
flashbacks, which have left her with a fatalistic view about the future.

``It takes too long to wait for justice, for which we've waited and
waited,'' she said, speaking barely above a whisper. ``It's easier to
kill.''

\textbf{The Takeaway}: Mr. Bolsonaro is determined to expand the
economic exploitation of the Amazon, whatever the costs.

Ernesto Londoño reported from Uru Eu Wau Wau territory, and Letícia
Casado from Brasília.

Advertisement

\protect\hyperlink{after-bottom}{Continue reading the main story}

\hypertarget{site-index}{%
\subsection{Site Index}\label{site-index}}

\hypertarget{site-information-navigation}{%
\subsection{Site Information
Navigation}\label{site-information-navigation}}

\begin{itemize}
\tightlist
\item
  \href{https://help.nytimes3xbfgragh.onion/hc/en-us/articles/115014792127-Copyright-notice}{©~2020~The
  New York Times Company}
\end{itemize}

\begin{itemize}
\tightlist
\item
  \href{https://www.nytco.com/}{NYTCo}
\item
  \href{https://help.nytimes3xbfgragh.onion/hc/en-us/articles/115015385887-Contact-Us}{Contact
  Us}
\item
  \href{https://www.nytco.com/careers/}{Work with us}
\item
  \href{https://nytmediakit.com/}{Advertise}
\item
  \href{http://www.tbrandstudio.com/}{T Brand Studio}
\item
  \href{https://www.nytimes3xbfgragh.onion/privacy/cookie-policy\#how-do-i-manage-trackers}{Your
  Ad Choices}
\item
  \href{https://www.nytimes3xbfgragh.onion/privacy}{Privacy}
\item
  \href{https://help.nytimes3xbfgragh.onion/hc/en-us/articles/115014893428-Terms-of-service}{Terms
  of Service}
\item
  \href{https://help.nytimes3xbfgragh.onion/hc/en-us/articles/115014893968-Terms-of-sale}{Terms
  of Sale}
\item
  \href{https://spiderbites.nytimes3xbfgragh.onion}{Site Map}
\item
  \href{https://help.nytimes3xbfgragh.onion/hc/en-us}{Help}
\item
  \href{https://www.nytimes3xbfgragh.onion/subscription?campaignId=37WXW}{Subscriptions}
\end{itemize}
