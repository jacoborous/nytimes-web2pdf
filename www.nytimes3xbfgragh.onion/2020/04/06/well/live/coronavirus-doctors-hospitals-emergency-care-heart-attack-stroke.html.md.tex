Sections

SEARCH

\protect\hyperlink{site-content}{Skip to
content}\protect\hyperlink{site-index}{Skip to site index}

\href{https://www.nytimes3xbfgragh.onion/section/well/live}{Live}

\href{https://myaccount.nytimes3xbfgragh.onion/auth/login?response_type=cookie\&client_id=vi}{}

\href{https://www.nytimes3xbfgragh.onion/section/todayspaper}{Today's
Paper}

\href{/section/well/live}{Live}\textbar{}Where Have All the Heart
Attacks Gone?

\url{https://nyti.ms/2UN1tNN}

\begin{itemize}
\item
\item
\item
\item
\item
\item
\end{itemize}

\hypertarget{the-coronavirus-outbreak}{%
\subsubsection{\texorpdfstring{\href{https://www.nytimes3xbfgragh.onion/news-event/coronavirus?name=styln-coronavirus-national\&region=TOP_BANNER\&block=storyline_menu_recirc\&action=click\&pgtype=Article\&impression_id=f3bcfbf0-f4b8-11ea-b9f9-d5e9868d2682\&variant=undefined}{The
Coronavirus
Outbreak}}{The Coronavirus Outbreak}}\label{the-coronavirus-outbreak}}

\begin{itemize}
\tightlist
\item
  live\href{https://www.nytimes3xbfgragh.onion/2020/09/11/world/covid-19-coronavirus.html?name=styln-coronavirus-national\&region=TOP_BANNER\&block=storyline_menu_recirc\&action=click\&pgtype=Article\&impression_id=f3bd2300-f4b8-11ea-b9f9-d5e9868d2682\&variant=undefined}{Latest
  Updates}
\item
  \href{https://www.nytimes3xbfgragh.onion/interactive/2020/us/coronavirus-us-cases.html?name=styln-coronavirus-national\&region=TOP_BANNER\&block=storyline_menu_recirc\&action=click\&pgtype=Article\&impression_id=f3bd2301-f4b8-11ea-b9f9-d5e9868d2682\&variant=undefined}{Maps
  and Cases}
\item
  \href{https://www.nytimes3xbfgragh.onion/interactive/2020/science/coronavirus-vaccine-tracker.html?name=styln-coronavirus-national\&region=TOP_BANNER\&block=storyline_menu_recirc\&action=click\&pgtype=Article\&impression_id=f3bd2302-f4b8-11ea-b9f9-d5e9868d2682\&variant=undefined}{Vaccine
  Tracker}
\item
  \href{https://www.nytimes3xbfgragh.onion/2020/09/10/us/politics/fda-coronavirus-vaccine.html?name=styln-coronavirus-national\&region=TOP_BANNER\&block=storyline_menu_recirc\&action=click\&pgtype=Article\&impression_id=f3bd2303-f4b8-11ea-b9f9-d5e9868d2682\&variant=undefined}{F.D.A.
  Regulators' Self-Defense}
\item
  \href{https://www.nytimes3xbfgragh.onion/2020/09/09/upshot/coronavirus-surprise-test-fees.html?name=styln-coronavirus-national\&region=TOP_BANNER\&block=storyline_menu_recirc\&action=click\&pgtype=Article\&impression_id=f3bdbf40-f4b8-11ea-b9f9-d5e9868d2682\&variant=undefined}{Surprise
  Test Fees}
\end{itemize}

Advertisement

\protect\hyperlink{after-top}{Continue reading the main story}

Supported by

\protect\hyperlink{after-sponsor}{Continue reading the main story}

Doctors

\hypertarget{where-have-all-the-heart-attacks-gone}{%
\section{Where Have All the Heart Attacks
Gone?}\label{where-have-all-the-heart-attacks-gone}}

Except for treating Covid-19, many hospitals seem to be eerily quiet.

\includegraphics{https://static01.graylady3jvrrxbe.onion/images/2020/03/11/well/well_doctors_bradford/well_doctors_bradford-articleLarge.jpg?quality=75\&auto=webp\&disable=upscale}

By Harlan M. Krumholz, M.D.

\begin{itemize}
\item
  Published April 6, 2020Updated May 14, 2020
\item
  \begin{itemize}
  \item
  \item
  \item
  \item
  \item
  \item
  \end{itemize}
\end{itemize}

The hospitals are eerily quiet, except
for\href{https://www.nytimes3xbfgragh.onion/2020/05/14/health/coronavirus-strokes.html}{Covid-19}.

I have heard this sentiment from fellow doctors across the United States
and in many other countries. We are all asking: Where are all the
patients with
\href{https://www.nytimes3xbfgragh.onion/2020/05/14/health/coronavirus-strokes.html}{heart
attacks and strok}e? They are missing from our hospitals.

Yale New Haven Hospital, where I work, has almost 300 people stricken
with Covid-19, and the numbers keep rising --- and yet we are not yet at
capacity because of a marked decline in our usual types of patients. In
more normal times, we never have so many empty beds.

Our hospital is usually so full that patients wait in gurneys along the
walls of the emergency department for a bed to become available on the
general wards or even in the intensive care unit. We send people home
from the hospital as soon as possible so we can free up beds for those
who are waiting. But the pandemic has caused a previously unimaginable
shift in the demand for hospital services.

Some of the excess capacity is indeed by design. We canceled elective
procedures, though many of those patients never needed hospitalization.
We are now providing care at home through telemedicine, but those
services are for stable outpatients, not for those who are acutely ill.

What is striking is that many of the emergencies have disappeared. Heart
attack and stroke teams, always poised to rush in and save lives, are
mostly idle. This is not just at my hospital. My fellow cardiologists
have shared with me that their cardiology consultations have shrunk,
except those related to Covid-19. In an
\href{https://twitter.com/angioplastyorg/status/1245892249101074432?s=20}{informal
Twitter poll} by \href{http://angioplasty.org/}{@angioplastyorg, an
online community of cardiologists,} almost half of the respondents
reported that they are seeing a 40 percent to 60 percent reduction in
admissions for heart attacks; about 20 percent reported more than a 60
percent reduction.

\hypertarget{latest-updates-the-coronavirus-outbreak}{%
\section{\texorpdfstring{\href{https://www.nytimes3xbfgragh.onion/2020/09/11/world/covid-19-coronavirus.html?action=click\&pgtype=Article\&state=default\&region=MAIN_CONTENT_1\&context=storylines_live_updates}{Latest
Updates: The Coronavirus
Outbreak}}{Latest Updates: The Coronavirus Outbreak}}\label{latest-updates-the-coronavirus-outbreak}}

Updated 2020-09-12T04:56:54.924Z

\begin{itemize}
\tightlist
\item
  \href{https://www.nytimes3xbfgragh.onion/2020/09/11/world/covid-19-coronavirus.html?action=click\&pgtype=Article\&state=default\&region=MAIN_CONTENT_1\&context=storylines_live_updates\#link-dfb8a16}{Fauci
  cautions the virus could disrupt life in the U.S. until `maybe even
  towards the end of 2021.'}
\item
  \href{https://www.nytimes3xbfgragh.onion/2020/09/11/world/covid-19-coronavirus.html?action=click\&pgtype=Article\&state=default\&region=MAIN_CONTENT_1\&context=storylines_live_updates\#link-7104d154}{From
  Asia to Africa, China promotes its vaccine candidates to win friends.}
\item
  \href{https://www.nytimes3xbfgragh.onion/2020/09/11/world/covid-19-coronavirus.html?action=click\&pgtype=Article\&state=default\&region=MAIN_CONTENT_1\&context=storylines_live_updates\#link-393ad215}{The
  other way the virus will kill: hunger.}
\end{itemize}

\href{https://www.nytimes3xbfgragh.onion/2020/09/11/world/covid-19-coronavirus.html?action=click\&pgtype=Article\&state=default\&region=MAIN_CONTENT_1\&context=storylines_live_updates}{See
more updates}

More live coverage:
\href{https://www.nytimes3xbfgragh.onion/live/2020/09/11/business/stock-market-today-coronavirus?action=click\&pgtype=Article\&state=default\&region=MAIN_CONTENT_1\&context=storylines_live_updates}{Markets}

And this is not a phenomenon specific to the United States.
Investigators
\href{https://www.recintervcardiol.org/es/?option=com_content\&view=article\&id=344\&catid=14}{from
Spain reported} a 40 percent reduction in emergency procedures for heart
attacks during the last week of March compared with the period just
before the pandemic hit.

And it may not just be heart attacks and strokes.
\href{https://twitter.com/MandaChelednik/status/1246793095032852480?s=20}{Colleagues
on Twitter report} a decline in many other emergencies, including acute
appendicitis and acute gall bladder disease.

The most concerning possible explanation is that people stay home and
suffer rather than risk coming to the hospital and getting infected with
coronavirus. This theory suggests that Covid-19 has instilled fear of
face-to-face medical care. As a result, many people with urgent health
problems may be opting to remain at home rather than call for help. And
when they do finally seek medical attention, it is often only after
their condition has worsened. Doctors from Hong Kong
\href{https://www.ahajournals.org/doi/10.1161/CIRCOUTCOMES.120.006631}{reported}
an increase in patients coming to the hospital late in the course of
their heart attack, when treatment is less likely to be lifesaving.

There are other possible explanations for the missing patients. In this
time of social distancing, our meals, social interactions and physical
activity patterns tend to be very different. Maybe we have removed some
of the triggers for heart attacks and strokes, like excessive eating and
drinking or abrupt periods of physical exertion. This theory merits
research but seems unlikely to explain the dramatic changes we're
observing.

We actually expected to see more heart attacks during this time.
Respiratory infections typically increase the risk of heart attacks.
\href{https://www.ncbi.nlm.nih.gov/pmc/articles/PMC3139921/}{Studies}
suggest that recent respiratory infections can double the risk of a
heart attack or stroke. The risk seems to begin soon after the
respiratory infection develops, so any rise in heart attacks or strokes
should be evident by now. We
\href{https://www.cardiosmart.org/News-and-Events/2019/03/Flu-Shot-Helps-Prevent-Heart-Attack}{urge
people to get flu vaccines} every year, in part, to protect their
hearts.

Also, times of stress increase the risk of heart attacks and strokes.
Depression, anxiety and frustration, feelings that the pandemic might
exacerbate, are all associated with a doubling or more of heart attack
risks. Work and life stress, which also may be higher with the acute
disruptions we've all been going through, can markedly increase the risk
of a heart attack. Moreover, events like earthquakes or terrorist
attacks or war, in which an entire society is exposed to a stressor, are
risk factors for heart attacks. Finally, Covid-19 can actually affect
the heart, which should be increasing the number of patients with heart
problems.

Experts are bringing together data to confirm these patterns. We hope to
gain a greater understanding of their causes and consequences.

Meanwhile, the immediate message to patients is clear: Don't delay
needed treatment. If fear of the pandemic leads people to delay or avoid
care, then the death rate will extend far beyond those directly infected
by the virus. Time to treatment dictates the outcomes for people with
heart attacks and strokes. These deaths may not be labeled Covid-19
deaths, but surely, they are collateral damage.

The public needs to know that hospitals are equipped not only to care
for people with Covid-19 but also those who have other life-threatening
health problems. Yes, we in health care are working to keep people out
of the hospital if we can, but we can safely provide care for those
people who are not sick from Covid-19. Masks and protective gear for
health care workers and patients go a long way to ensure a safe
environment. Also, people with chronic conditions need to know that
avoidance of needed care could ultimately be as big a threat as the
virus itself.

As we fight coronavirus, we need to combat perceptions that everyone
else must stay away from the hospital. The pandemic toll will be much
worse if it leads people to avoid care for life-threatening, yet
treatable, conditions like heart attacks and strokes.

\begin{center}\rule{0.5\linewidth}{\linethickness}\end{center}

\emph{Harlan Krumholz, M.D., is professor of medicine at Yale and
director of the Yale New Haven Hospital Center for Outcomes Research and
Evaluation.}

Advertisement

\protect\hyperlink{after-bottom}{Continue reading the main story}

\hypertarget{site-index}{%
\subsection{Site Index}\label{site-index}}

\hypertarget{site-information-navigation}{%
\subsection{Site Information
Navigation}\label{site-information-navigation}}

\begin{itemize}
\tightlist
\item
  \href{https://help.nytimes3xbfgragh.onion/hc/en-us/articles/115014792127-Copyright-notice}{©~2020~The
  New York Times Company}
\end{itemize}

\begin{itemize}
\tightlist
\item
  \href{https://www.nytco.com/}{NYTCo}
\item
  \href{https://help.nytimes3xbfgragh.onion/hc/en-us/articles/115015385887-Contact-Us}{Contact
  Us}
\item
  \href{https://www.nytco.com/careers/}{Work with us}
\item
  \href{https://nytmediakit.com/}{Advertise}
\item
  \href{http://www.tbrandstudio.com/}{T Brand Studio}
\item
  \href{https://www.nytimes3xbfgragh.onion/privacy/cookie-policy\#how-do-i-manage-trackers}{Your
  Ad Choices}
\item
  \href{https://www.nytimes3xbfgragh.onion/privacy}{Privacy}
\item
  \href{https://help.nytimes3xbfgragh.onion/hc/en-us/articles/115014893428-Terms-of-service}{Terms
  of Service}
\item
  \href{https://help.nytimes3xbfgragh.onion/hc/en-us/articles/115014893968-Terms-of-sale}{Terms
  of Sale}
\item
  \href{https://spiderbites.nytimes3xbfgragh.onion}{Site Map}
\item
  \href{https://help.nytimes3xbfgragh.onion/hc/en-us}{Help}
\item
  \href{https://www.nytimes3xbfgragh.onion/subscription?campaignId=37WXW}{Subscriptions}
\end{itemize}
