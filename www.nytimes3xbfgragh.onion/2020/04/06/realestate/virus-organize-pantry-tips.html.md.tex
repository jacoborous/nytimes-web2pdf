Sections

SEARCH

\protect\hyperlink{site-content}{Skip to
content}\protect\hyperlink{site-index}{Skip to site index}

\href{https://www.nytimes3xbfgragh.onion/section/realestate}{Real
Estate}

\href{https://myaccount.nytimes3xbfgragh.onion/auth/login?response_type=cookie\&client_id=vi}{}

\href{https://www.nytimes3xbfgragh.onion/section/todayspaper}{Today's
Paper}

\href{/section/realestate}{Real Estate}\textbar{}How to Put Your Pantry
in Order (and Stop Wasting Food)

\url{https://nyti.ms/39JNQTF}

\begin{itemize}
\item
\item
\item
\item
\item
\item
\end{itemize}

\href{https://www.nytimes3xbfgragh.onion/spotlight/at-home?action=click\&pgtype=Article\&state=default\&region=TOP_BANNER\&context=at_home_menu}{At
Home}

\begin{itemize}
\tightlist
\item
  \href{https://www.nytimes3xbfgragh.onion/2020/09/07/travel/route-66.html?action=click\&pgtype=Article\&state=default\&region=TOP_BANNER\&context=at_home_menu}{Cruise
  Along: Route 66}
\item
  \href{https://www.nytimes3xbfgragh.onion/2020/09/04/dining/sheet-pan-chicken.html?action=click\&pgtype=Article\&state=default\&region=TOP_BANNER\&context=at_home_menu}{Roast:
  Chicken With Plums}
\item
  \href{https://www.nytimes3xbfgragh.onion/2020/09/04/arts/television/dark-shadows-stream.html?action=click\&pgtype=Article\&state=default\&region=TOP_BANNER\&context=at_home_menu}{Watch:
  Dark Shadows}
\item
  \href{https://www.nytimes3xbfgragh.onion/interactive/2020/at-home/even-more-reporters-editors-diaries-lists-recommendations.html?action=click\&pgtype=Article\&state=default\&region=TOP_BANNER\&context=at_home_menu}{Explore:
  Reporters' Google Docs}
\end{itemize}

Advertisement

\protect\hyperlink{after-top}{Continue reading the main story}

Supported by

\protect\hyperlink{after-sponsor}{Continue reading the main story}

Sheltering

\hypertarget{how-to-put-your-pantry-in-order-and-stop-wasting-food}{%
\section{How to Put Your Pantry in Order (and Stop Wasting
Food)}\label{how-to-put-your-pantry-in-order-and-stop-wasting-food}}

Step one: Start by making a mess. Here's what to do after that.

\includegraphics{https://static01.graylady3jvrrxbe.onion/images/2020/04/03/realestate/06sheltering1/06sheltering1-articleLarge.jpg?quality=75\&auto=webp\&disable=upscale}

\href{https://www.nytimes3xbfgragh.onion/by/tim-mckeough}{\includegraphics{https://static01.graylady3jvrrxbe.onion/images/2018/06/12/multimedia/author-tim-mckeough/author-tim-mckeough-thumbLarge.png}}

By \href{https://www.nytimes3xbfgragh.onion/by/tim-mckeough}{Tim
McKeough}

\begin{itemize}
\item
  April 6, 2020
\item
  \begin{itemize}
  \item
  \item
  \item
  \item
  \item
  \item
  \end{itemize}
\end{itemize}

I always thought my pantry was under control. I like to cook, my wife
enjoys baking, and we like to think of ourselves as reasonably organized
people.

But when we recently dug into the deep, dark recesses of our cabinet to
make way for the
\href{https://www.nytimes3xbfgragh.onion/2020/03/06/dining/how-to-stock-a-pantry.html}{additional
staples that would support an extended stay at home}, we were surprised
by what we found: canned goods and powdered mixes that had expired years
ago, old pasta boxes containing few noodles and multiple open bags of
pecans, which I had bought every year for Thanksgiving, believing we had
none.

When we began putting everything back, we were determined to create
\href{https://cooking.nytimes3xbfgragh.onion/guides/56-how-to-stock-a-modern-pantry}{a
more efficient pantry} --- and stop wasting food --- by devising a
system that would allow us to easily see and reach everything inside.

I asked several organization and kitchen pros for advice.

\includegraphics{https://static01.graylady3jvrrxbe.onion/images/2020/04/19/realestate/06sheltering2/oakImage-1585941489969-articleLarge.jpg?quality=75\&auto=webp\&disable=upscale}

\hypertarget{pull-everything-out}{%
\subsection{Pull Everything Out}\label{pull-everything-out}}

A pantry shouldn't be a dumping ground for things you can't figure out
what to do with.

``It should feel like your tool kit,'' said Amanda Hesser, a founder and
the chief executive of \href{https://food52.com/}{Food52} (and former
Times food editor and writer). ``With that mind-set, it makes it much
easier to clean it out, so you only have the stuff you really need.''

To get your pantry in order, start by making a mess. ``Take everything
out,'' said Sharon Lowenheim, the owner of
\href{http://organizinggoddess.com/}{Organizing Goddess}, in New York.
``Once all the shelves are clear, which probably hasn't happened since
you moved in, clean them.''

While you're doing that, she said, be on the lookout for items that have
expired, or that you no longer use, and throw them away. When most of us
buy groceries, we put the newest items in the front of the pantry, she
said, which pushes everything else to the back, where things can
disappear for years.

Image

Lazy Susans can help keep smaller jars accessible and make the most of
awkward cabinet corners.Credit...Laura Cattano

\hypertarget{group-similar-products}{%
\subsection{Group Similar Products}\label{group-similar-products}}

Once everything is out, look for products that should be grouped
together. ``As we pull things out, we're putting them into categories
like baking, dinner, snacks,'' said Fillip Hord, who founded
\href{https://horderly.com/}{Horderly Professional Organizing} with his
wife, Jamie Hord.

The goal, he explained, is to keep items that are normally used together
in proximity --- pasta and tomato sauce, chips and salsa, flour and
sugar --- so you rarely have to go searching for things once you restock
the pantry.

``The experience of using your pantry should be intuitive, graceful and
easy,'' said \href{https://lauracattano.com/}{Laura Cattano}, a
professional organizer in New York. ``You want to open one cabinet or
one drawer and have everything right there.''

\hypertarget{make-more-space}{%
\subsection{Make More Space}\label{make-more-space}}

If your pantry is overstuffed, and you know it will be difficult to fit
everything back inside, look for things that could be moved elsewhere,
Ms. Cattano suggested.

Snacks, coffee and tea are good candidates for being moved to another
cabinet. And if you're using the pantry to store pots and pans, move
them out, too.

``There's wall space,'' Ms. Cattano said. ``Getting a pot rack can clear
up a lot of space in a cabinet.''

Image

Once everything is in place, labeling is the last step. Ms. Hesser uses
simpler tools, like wet-erase markers to write directly on plastic
containers or Sharpies to write on masking tape.Credit...James
Ransom/Food52

\hypertarget{adjust-the-shelves}{%
\subsection{Adjust the Shelves}\label{adjust-the-shelves}}

Most kitchen cabinets have adjustable shelves, but few people bother to
rearrange them.

``When you move into a place, often the shelves are about a foot
apart,'' Ms. Hesser said, which can lead to wasted vertical space.
``When canned goods are all on the same shelf, it only needs, say, six
inches between it and the next shelf.''

By adjusting the space between shelves to fit your cans and containers,
she said, ``You can really maximize the space that you have.''

If there's enough room, you might even be able to squeeze in an extra
shelf or two. Or you could add a wire cabinet shelf with legs ---
sometimes called a helper shelf --- to create an additional, elevated
level on top of an existing shelf.

Ideally, you should avoid stacking products on top of each other. ``Once
you start stacking stuff, things are immediately hard to get to and
teetering on top of each other,'' Ms. Hesser said. ``Yes, you may feel
like you're packing more into your pantry, but you're also creating an
obstacle.''

\hypertarget{put-everyday-products-at-eye-level}{%
\subsection{Put Everyday Products at Eye
Level}\label{put-everyday-products-at-eye-level}}

When putting products back into the pantry, try to keep the items you
use for everyday cooking the most accessible, by positioning them at eye
level.

``That's prime real estate,'' Ms. Hord said, and an ideal place for
things like oils and vinegars. By comparison, she continued, ``Baking
can usually live up high,'' because most people bake infrequently.

If you have children, consider placing their snacks near the bottom of
the pantry, she said, so they can help themselves.

\hypertarget{add-bins-or-pullout-organizers}{%
\subsection{Add Bins or Pullout
Organizers}\label{add-bins-or-pullout-organizers}}

A few carefully chosen accessories can help ensure you'll never lose
anything in the back of the pantry again. Professional organizers often
use clear plastic bins or wire baskets that match the depth of pantry
shelves.

``Sometimes grains, pasta and rice come in bags that are impossible to
stand up,'' Ms. Lowenheim said. Bins can reduce frustration by keeping
such items upright while also preventing them from spilling out when you
open the pantry doors.

Bins function like drawers, she said, because they can easily be pulled
out to reach products that would otherwise be stuck in the back.

Ms. Lowenheim and the Hords frequently use
\href{https://www.containerstore.com/s?source=form\&refinements=\&q=linus+bins\&submit=}{iDesign
Linus bins} from the Container Store, while Ms. Cattano recommended
\href{https://www.containerstore.com/s/white-elfa-mesh-drawers/d?productId=10014211\&q=elfa\%20mesh}{Elfa
Mesh drawers}. There are also built-in alternatives with slides that get
fastened to a shelf, like
\href{https://www.simplehuman.com/kitchen-pull-out-cabinet-organizer}{Simplehuman's
pullout cabinet organizers}.

Lazy Susans can also help keep smaller jars easily accessible, Mr. Hord
said, and make the most of awkward cabinet corners.

Image

One debate in organizing circles is whether it makes sense to invest in
matching containers for dry goods. While the appearance is satisfying,
it's only worth it if you know you can keep up with refilling them over
time.Credit...Horderly

\hypertarget{consider-containers-for-individual-products}{%
\subsection{Consider Containers for Individual
Products}\label{consider-containers-for-individual-products}}

One of the big debates in organizing circles is whether it makes sense
to invest in matching storage containers for dry goods.

Ms. Hesser took the plunge late last year, overhauling her pantry with
matching
\href{https://food52.com/shop/products/3221-modula-stackable-storage-container-starter-set}{Mepal
Modula storage containers}. ``When I open my pantry now, it is just a
complete pleasure,'' she said.

While the resulting look can be satisfying, the Hords cautioned that
it's only worth buying containers if you know you can keep up with
refilling them over time. ``If you're not going to stay on top of it, it
just turns into excess bins,'' Mr. Hord said, which creates more
clutter.

Ms. Lowenheim recommended finding a middle ground: If you have large
packages of cereal or oats that are mostly empty, she said, transfer the
contents to a few containers to free up pantry space.

\hypertarget{label-containers-and-shelves}{%
\subsection{Label Containers and
Shelves}\label{label-containers-and-shelves}}

Once everything is in place, ``labeling is the last --- and most
important --- step,'' said Ms. Hord, who likes to use a label maker for
the job.

Ms. Hesser prefers simpler tools: wet-erase markers to write directly on
plastic containers and Sharpies to write on masking tape.

Whichever tools you choose, labeling containers and shelves ``makes it
so much easier to find everything,'' Ms. Hesser said.

But perhaps more crucial, labels help preserve all your hard work.

After you've finished cooking a meal, Mr. Hord said, ``it's how you know
where to put things back.''

For weekly email updates on residential real estate news,
\href{http://www.nytimes3xbfgragh.onion/newsletters/realestate/}{sign up
here}. Follow us on Twitter:
\href{https://twitter.com/nytrealestate}{@nytrealestate}.

Advertisement

\protect\hyperlink{after-bottom}{Continue reading the main story}

\hypertarget{site-index}{%
\subsection{Site Index}\label{site-index}}

\hypertarget{site-information-navigation}{%
\subsection{Site Information
Navigation}\label{site-information-navigation}}

\begin{itemize}
\tightlist
\item
  \href{https://help.nytimes3xbfgragh.onion/hc/en-us/articles/115014792127-Copyright-notice}{©~2020~The
  New York Times Company}
\end{itemize}

\begin{itemize}
\tightlist
\item
  \href{https://www.nytco.com/}{NYTCo}
\item
  \href{https://help.nytimes3xbfgragh.onion/hc/en-us/articles/115015385887-Contact-Us}{Contact
  Us}
\item
  \href{https://www.nytco.com/careers/}{Work with us}
\item
  \href{https://nytmediakit.com/}{Advertise}
\item
  \href{http://www.tbrandstudio.com/}{T Brand Studio}
\item
  \href{https://www.nytimes3xbfgragh.onion/privacy/cookie-policy\#how-do-i-manage-trackers}{Your
  Ad Choices}
\item
  \href{https://www.nytimes3xbfgragh.onion/privacy}{Privacy}
\item
  \href{https://help.nytimes3xbfgragh.onion/hc/en-us/articles/115014893428-Terms-of-service}{Terms
  of Service}
\item
  \href{https://help.nytimes3xbfgragh.onion/hc/en-us/articles/115014893968-Terms-of-sale}{Terms
  of Sale}
\item
  \href{https://spiderbites.nytimes3xbfgragh.onion}{Site Map}
\item
  \href{https://help.nytimes3xbfgragh.onion/hc/en-us}{Help}
\item
  \href{https://www.nytimes3xbfgragh.onion/subscription?campaignId=37WXW}{Subscriptions}
\end{itemize}
