Sections

SEARCH

\protect\hyperlink{site-content}{Skip to
content}\protect\hyperlink{site-index}{Skip to site index}

\href{https://www.nytimes3xbfgragh.onion/section/climate}{Climate}

\href{https://myaccount.nytimes3xbfgragh.onion/auth/login?response_type=cookie\&client_id=vi}{}

\href{https://www.nytimes3xbfgragh.onion/section/todayspaper}{Today's
Paper}

\href{/section/climate}{Climate}\textbar{}Emissions Declines Will Set
Records This Year. But It's Not Good News.

\url{https://nyti.ms/2WdQY5s}

\begin{itemize}
\item
\item
\item
\item
\item
\item
\end{itemize}

\hypertarget{climate-and-environment}{%
\subsubsection{\texorpdfstring{\href{https://www.nytimes3xbfgragh.onion/section/climate?name=styln-climate\&region=TOP_BANNER\&block=storyline_menu_recirc\&action=click\&pgtype=Article\&impression_id=600ce730-f1e3-11ea-8772-c75414e6f942\&variant=undefined}{Climate
and
Environment}}{Climate and Environment}}\label{climate-and-environment}}

\begin{itemize}
\tightlist
\item
  \href{https://www.nytimes3xbfgragh.onion/article/why-does-california-have-wildfires.html?name=styln-climate\&region=TOP_BANNER\&block=storyline_menu_recirc\&action=click\&pgtype=Article\&impression_id=600d0e40-f1e3-11ea-8772-c75414e6f942\&variant=undefined}{California
  Fires}
\item
  \href{https://www.nytimes3xbfgragh.onion/interactive/2020/climate/trump-environment-rollbacks.html?name=styln-climate\&region=TOP_BANNER\&block=storyline_menu_recirc\&action=click\&pgtype=Article\&impression_id=600d0e41-f1e3-11ea-8772-c75414e6f942\&variant=undefined}{Trump's
  Changes}
\item
  \href{https://www.nytimes3xbfgragh.onion/interactive/2020/04/19/climate/climate-crash-course-1.html?name=styln-climate\&region=TOP_BANNER\&block=storyline_menu_recirc\&action=click\&pgtype=Article\&impression_id=600d0e42-f1e3-11ea-8772-c75414e6f942\&variant=undefined}{Climate
  101}
\item
  \href{https://www.nytimes3xbfgragh.onion/interactive/2018/08/30/climate/how-much-hotter-is-your-hometown.html?name=styln-climate\&region=TOP_BANNER\&block=storyline_menu_recirc\&action=click\&pgtype=Article\&impression_id=600d0e43-f1e3-11ea-8772-c75414e6f942\&variant=undefined}{Is
  Your Hometown Hotter?}
\end{itemize}

Advertisement

\protect\hyperlink{after-top}{Continue reading the main story}

Supported by

\protect\hyperlink{after-sponsor}{Continue reading the main story}

\hypertarget{emissions-declines-will-set-records-this-year-but-its-not-good-news}{%
\section{Emissions Declines Will Set Records This Year. But It's Not
Good
News.}\label{emissions-declines-will-set-records-this-year-but-its-not-good-news}}

An ``unprecedented'' fall in fossil fuel use, driven by the Covid-19
crisis, is likely to lead to a nearly 8 percent drop, according to new
research.

\includegraphics{https://static01.graylady3jvrrxbe.onion/images/2020/05/02/climate/30CLI-VIRUS-EMISSIONS-print/30CLI-VIRUS-EMISSIONS1-articleLarge-v2.jpg?quality=75\&auto=webp\&disable=upscale}

\href{https://www.nytimes3xbfgragh.onion/by/brad-plumer}{\includegraphics{https://static01.graylady3jvrrxbe.onion/images/2018/02/20/multimedia/author-brad-plumer/author-brad-plumer-thumbLarge.jpg}}

By \href{https://www.nytimes3xbfgragh.onion/by/brad-plumer}{Brad Plumer}

\begin{itemize}
\item
  April 30, 2020
\item
  \begin{itemize}
  \item
  \item
  \item
  \item
  \item
  \item
  \end{itemize}
\end{itemize}

WASHINGTON ---~Global greenhouse gas emissions are on track to plunge
nearly 8 percent this year, the largest drop ever recorded, as worldwide
lockdowns to fight the coronavirus have triggered an ``unprecedented''
decline in the use of fossil fuels, the International Energy Agency
\href{https://www.iea.org/reports/global-energy-review-2020}{said in a
new report on Thursday}.

But experts cautioned that the drop should not be seen as good news for
efforts to tackle climate change. When the pandemic subsides and nations
take steps to restart their economies, emissions could easily soar
again~unless governments make concerted efforts to shift to cleaner
energy as part of their recovery efforts.

``This historic decline in emissions is happening for all the wrong
reasons,'' said Fatih Birol, the agency's executive director. ``People
are dying and countries are suffering enormous economic trauma right
now. The only way to sustainably reduce emissions is not through painful
lockdowns, but by putting the right energy and climate policies in
place.''

More than 4 billion people are living in countries that have imposed
partial or more extensive shutdowns on economic activity to slow the
spread of the virus. By mid-April, the report found, energy use in many
of those countries was 17 percent to 25 percent lower than it was in
2019, as factories idled, employees stopped driving to work and airlines
grounded their flights.

The agency currently expects many governments to start relaxing those
restrictions later in the year, as China
\href{https://www.nytimes3xbfgragh.onion/2020/03/24/world/asia/china-coronavirus-lockdown-hubei.html}{has
already done} and as
\href{https://www.nytimes3xbfgragh.onion/2020/04/27/us/coronavirus-governors-states-reopening.html}{some
states are starting to do} in the United States. Even so, the report
said, global carbon dioxide emissions were projected to fall by roughly
2.6 billion tons this year, an 8 percent drop from 2019.

That would put global emissions back at levels last seen in 2010, wiping
out an entire decade of growth in the use of fossil fuels worldwide. The
projected annual drop in emissions would be~six times the size of the
decline seen after the global financial crisis in 2009 and a far bigger
drop than at any point during the Great Depression or at the end of
World War II, when much of Europe lay in ruins.

\href{\%3Ca\%20href=\%22https://www.nytimes3xbfgragh.onion/section/climate?action=click\&pgtype=Article\&state=default\&region=MAIN_CONTENT_1\&context=storylines_keepup\%22\%3Ehttps://www.nytimes3xbfgragh.onion/section/climate?action=click\&pgtype=Article\&state=default\&region=MAIN_CONTENT_1\&context=storylines_keepup\%3C/a\%3E}{}

\hypertarget{climate-and-environment-}{%
\subsubsection{Climate and Environment
›}\label{climate-and-environment-}}

\hypertarget{keep-up-on-the-latest-climate-news}{%
\paragraph{Keep Up on the Latest Climate
News}\label{keep-up-on-the-latest-climate-news}}

Updated Sept. 6, 2020

Here's what you need to know this week:

\begin{itemize}
\item
  \begin{itemize}
  \tightlist
  \item
    Americans back
    \href{https://www.nytimes3xbfgragh.onion/2020/09/04/climate/flood-fire-building-restrictions.html?action=click\&pgtype=Article\&state=default\&region=MAIN_CONTENT_1\&context=storylines_keepup}{tough
    limits on building in fire and flood zones}, new research shows.
  \item
    California's wildfires are driving another crisis: More and more
    \href{https://www.nytimes3xbfgragh.onion/2020/09/02/climate/wildfires-insurance.html?action=click\&pgtype=Article\&state=default\&region=MAIN_CONTENT_1\&context=storylines_keepup}{homeowners
    can't get insurance}.
  \item
    The Trump administration has relaxed Obama-era rules limiting the
    release of
    \href{https://www.nytimes3xbfgragh.onion/2020/08/31/climate/trump-coal-plants.html?action=click\&pgtype=Article\&state=default\&region=MAIN_CONTENT_1\&context=storylines_keepup}{toxic
    waste from coal plants}.
  \end{itemize}
\end{itemize}

Still, there are many uncertainties around the early estimates.

If countries remain locked down for longer than expected, or if
businesses struggle to recover from the pandemic, the drop in emissions
could be larger. By contrast, getting the virus under control faster
would mean a smaller dip in emissions this year.

The report also noted that, after past crises, global emissions have
typically shot back up to previous levels once the initial shock passed.
And if countries like China try to help their ailing economies by
relaxing environmental rules or subsidizing polluting industries like
coal or steel, the resulting rebound in emissions could be even larger
than the decline.

That's what happened after the financial crisis a decade ago: By 2010,
global emissions had surged back higher than ever before as nations
invested heavily in fossil fuels to lift themselves out of the
recession.

``One of the big question marks now is whether countries decide to put
clean energy at the heart of their economic stimulus packages,'' Mr.
Birol said.

This week, leaders from Germany, Britain, Japan and elsewhere
\href{https://www.bmu.de/en/event/petersberg-climate-dialogue-xi/}{held
a video conference} urging nations to invest in technology to reduce
emissions, such as solar power or electric vehicles, as they chart their
economic recovery efforts. ``There will be a difficult debate about the
allocation of funds,'' said Chancellor Angela Merkel of Germany. ``But
it is important that recovery programs always keep an eye on the
climate.''

For now, the current crisis has dramatically reshaped the global energy
landscape.

The world's use of oil fell nearly 5 percent in the first quarter of
this year, the report said. By March, global road transport was down
nearly 50 percent, and air traffic was down 60 percent, compared to
2019. That slump in fuel demand
\href{https://www.nytimes3xbfgragh.onion/2020/03/08/business/saudi-arabia-oil-prices.html}{has
caused crude prices to crash worldwide}, straining the budgets of major
oil producers like Saudi Arabia and pushing drilling companies in places
like Texas to the brink of bankruptcy.

\includegraphics{https://static01.graylady3jvrrxbe.onion/images/2020/04/30/climate/30CLI-VIRUS-EMISSIONS2/merlin_170669919_f6dce9b0-c5a0-4196-b664-3440021c5a52-articleLarge.jpg?quality=75\&auto=webp\&disable=upscale}

The world's use of coal, the dirtiest of all fossil fuels, fell nearly 8
percent in the first quarter of the year. Much of that was triggered by
early coronavirus shutdowns in China, the world's biggest coal user. But
even though Chinese coal plants are now firing back up, the global coal
industry
\href{https://www.nytimes3xbfgragh.onion/2019/11/12/climate/energy-trends-climate-change.html}{faces
a continued threat from cheaper and cleaner energy sources} like natural
gas and renewables.

By contrast, wind and solar power have seen a slight uptick in demand
during the pandemic.

One big reason for that: Many countries are using significantly less
electricity as office buildings, restaurants and movie theaters close.
But because existing wind turbines and solar panels cost little to
operate, they tend to get priority on electric grids, which means they
are still operating closer to full capacity, while fossil-fuel plants
are allowed to run less frequently.

Despite the record drop in emissions, scientists cautioned that the
world faces an enormous task in getting global warming under control.

The United Nations has said that global emissions would have to fall
nearly 8 percent every single year between now and 2030 if countries
hoped to keep global warming well below 2 degrees Celsius (3.6 degrees
Fahrenheit), which world leaders have deemed necessary for avoiding
catastrophic social, economic and environmental damage from climate
change.

``A lockdown is just a one-off event, it can't get you all the way
there,'' said Glen Peters, research director at the Center for
International Climate Research in Norway.

He pointed out that even amid the sweeping lockdowns, the global economy
continues to rely heavily on fossil fuels for all the power plants,
trucks, planes, cars and heavy industries that continue to operate
during the crisis.

While experts noted that the current lockdowns won't last forever, some
expressed hope that they might reveal some of the benefits of switching
to cleaner energy. In recent weeks, for instance, cities like
\href{https://www.nytimes3xbfgragh.onion/interactive/2020/03/22/climate/coronavirus-usa-traffic.html}{Los
Angeles} and
\href{https://www.nytimes3xbfgragh.onion/interactive/2020/climate/coronavirus-pollution.html}{Milan}
have seen a dramatic reduction in air pollution and smog as fewer people
drive and cars stay off the road.

``I hope the striking improvements in air quality we've seen remind us
what things could be like if we shifted to green power and electric
vehicles,'' said Rob Jackson, an earth scientist at Stanford University.

Advertisement

\protect\hyperlink{after-bottom}{Continue reading the main story}

\hypertarget{site-index}{%
\subsection{Site Index}\label{site-index}}

\hypertarget{site-information-navigation}{%
\subsection{Site Information
Navigation}\label{site-information-navigation}}

\begin{itemize}
\tightlist
\item
  \href{https://help.nytimes3xbfgragh.onion/hc/en-us/articles/115014792127-Copyright-notice}{©~2020~The
  New York Times Company}
\end{itemize}

\begin{itemize}
\tightlist
\item
  \href{https://www.nytco.com/}{NYTCo}
\item
  \href{https://help.nytimes3xbfgragh.onion/hc/en-us/articles/115015385887-Contact-Us}{Contact
  Us}
\item
  \href{https://www.nytco.com/careers/}{Work with us}
\item
  \href{https://nytmediakit.com/}{Advertise}
\item
  \href{http://www.tbrandstudio.com/}{T Brand Studio}
\item
  \href{https://www.nytimes3xbfgragh.onion/privacy/cookie-policy\#how-do-i-manage-trackers}{Your
  Ad Choices}
\item
  \href{https://www.nytimes3xbfgragh.onion/privacy}{Privacy}
\item
  \href{https://help.nytimes3xbfgragh.onion/hc/en-us/articles/115014893428-Terms-of-service}{Terms
  of Service}
\item
  \href{https://help.nytimes3xbfgragh.onion/hc/en-us/articles/115014893968-Terms-of-sale}{Terms
  of Sale}
\item
  \href{https://spiderbites.nytimes3xbfgragh.onion}{Site Map}
\item
  \href{https://help.nytimes3xbfgragh.onion/hc/en-us}{Help}
\item
  \href{https://www.nytimes3xbfgragh.onion/subscription?campaignId=37WXW}{Subscriptions}
\end{itemize}
