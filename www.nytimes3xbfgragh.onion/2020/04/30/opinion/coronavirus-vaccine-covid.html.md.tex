Sections

SEARCH

\protect\hyperlink{site-content}{Skip to
content}\protect\hyperlink{site-index}{Skip to site index}

\href{https://myaccount.nytimes3xbfgragh.onion/auth/login?response_type=cookie\&client_id=vi}{}

\href{https://www.nytimes3xbfgragh.onion/section/todayspaper}{Today's
Paper}

\href{/section/opinion}{Opinion}\textbar{}There May Be a Dangerous
Shortcut to a Coronavirus Vaccine

\url{https://nyti.ms/2yeAUIT}

\begin{itemize}
\item
\item
\item
\item
\item
\item
\end{itemize}

Advertisement

\protect\hyperlink{after-top}{Continue reading the main story}

\href{/section/opinion}{Opinion}

Supported by

\protect\hyperlink{after-sponsor}{Continue reading the main story}

\hypertarget{there-may-be-a-dangerous-shortcut-to-a-coronavirus-vaccine}{%
\section{There May Be a Dangerous Shortcut to a Coronavirus
Vaccine}\label{there-may-be-a-dangerous-shortcut-to-a-coronavirus-vaccine}}

Should we take it?

\includegraphics{https://static01.graylady3jvrrxbe.onion/images/2019/08/23/opinion/Bokat-Lindell-headshot/Bokat-Lindell-headshot-thumbLarge.png}

By Spencer Bokat-Lindell

Mr. Bokat-Lindell is a writer in The New York Times Opinion section.

\begin{itemize}
\item
  April 30, 2020
\item
  \begin{itemize}
  \item
  \item
  \item
  \item
  \item
  \item
  \end{itemize}
\end{itemize}

\includegraphics{https://static01.graylady3jvrrxbe.onion/images/2020/04/30/opinion/30debatableillo/30debatableillo-articleLarge.jpg?quality=75\&auto=webp\&disable=upscale}

\emph{This article is part of the Debatable newsletter. You can}
\href{https://www.nytimes3xbfgragh.onion/newsletters/debatable}{\emph{sign
up here}} \emph{to receive it Tuesdays and Thursdays.}

Scientists at Oxford University's Jenner Institute raised hopes this
week when they announced plans to expand testing for a potential
\href{https://www.nytimes3xbfgragh.onion/2020/05/20/health/coronavirus-vaccines.html}{coronavirus
vaccine} that if proven effective could be ready for emergency use
\href{https://www.nytimes3xbfgragh.onion/2020/04/27/world/europe/coronavirus-vaccine-update-oxford.html}{as
soon as September}. But as my colleague Stuart Thompson
\href{https://www.nytimes3xbfgragh.onion/interactive/2020/04/30/opinion/coronavirus-covid-vaccine.html}{explains},
that's an enormous if: Most vaccines take a decade or longer to make,
and none has ever been developed in less than four years.

Cutting that record down to 12 or even 18 months would already require
moving at ``pandemic speed.'' But last week,
\href{https://foster.house.gov/sites/foster.house.gov/files/2020.04.20_Ltr\%20to\%20HHS\%20\%20FDA\%20on\%20Rapid\%20Vaccine\%20Deployment\%20for\%20COVID-19\%20-\%20Signed.pdf}{35
members of Congress} proposed an extraordinary practice that some
scientists think could compress the timeline even further: deliberately
infecting volunteers. Here's what people are saying about the idea,
known as human challenge.

\hypertarget{why-would-we-intentionally-infect-people}{%
\subsection{Why would we intentionally infect
people?}\label{why-would-we-intentionally-infect-people}}

As it happens, the Jenner Institute is named for
\href{https://www.jenner.ac.uk/about/edward-jenner}{Edward Jenner,} who
invented the world's first vaccine by doing just this. In 1796, more
than a century before anyone even knew what a virus was, Mr. Jenner
suspected that cowpox, a mild disease that milkmaids sometimes
contracted, might confer immunity against the far more deadly smallpox.
To test his theory, he inoculated an 8-year-old boy with the first virus
(courtesy of a Gloucester cow named Blossom) and then deliberately
infected --- or ``challenged'' --- him with the second.

Mr. Jenner's theory proved correct --- ``vaccine'' derives from
``vacca*,''* the Latin word for cow --- but his method has always been
\href{https://theconversation.com/judging-jenner-was-his-smallpox-experiment-really-unethical-54362}{ethically
fraught}. Today, human-challenge studies are subject to strict oversight
and authorized only for diseases that have treatments, like malaria or
the flu. Any potential vaccine now also has to undergo three phases of
clinical trials. Efficacy is tested in the third phase, where instead of
being deliberately infected, participants simply go about their daily
lives and researchers see whether those who received the vaccine prove
less likely to contract the disease than those who received a placebo.

\textbf{But the pandemic has raised two concerns with the process:}

\begin{itemize}
\item
  Right now, most people are trying \emph{not} to get sick: The Jenner
  Institute's director has said that if the infection rate continues to
  slow in Britain, researchers may not be able to determine whether the
  vaccine works.
\item
  Trials take time: The first phase typically takes months, and the
  second two take years.
\end{itemize}

\textbf{As a result, some experts have called for replacing conventional
Phase III testing with human-challenge trials.} In The Journal of
Infectious Diseases last month, the bioethicist Nir Eyal and the
epidemiologists Marc Lipsitch and Peter Smith
\href{https://academic.oup.com/jid/advance-article/doi/10.1093/infdis/jiaa152/5814216\#201727802}{argued}
that the idea, while risky, could shave months off the process. ``Every
week that vaccine rollout is delayed will be accompanied by many
thousands of deaths globally,'' they wrote. ``If the use of human
challenge helped to make the vaccine available before the epidemic has
completely passed, the savings in human lives could be in the thousands
or conceivably millions.''

\hypertarget{how-would-it-work}{%
\subsection{How would it work?}\label{how-would-it-work}}

In the authors' proposed design, the study would rely entirely on young,
healthy volunteers who fully understand the risks of participating. (To
the authors, ``young'' might mean 20 to 45; to others,
\href{https://www.sciencemag.org/news/2020/03/speed-coronavirus-vaccine-testing-deliberately-infecting-volunteers-not-so-fast-some\#}{18
to 30} or
\href{https://www.vox.com/future-perfect/2020/4/9/21209593/coronavirus-vaccine-human-trials-explained}{even
18 to 25}.) Conventional Phase III trials typically require thousands of
volunteers, but a human-challenge trial might need only 100. All
participants would remain isolated in comfortable state-of-the-art
facilities, with access to ``excellent'' health care.

\hypertarget{who-in-the-world-would-volunteer-to-get-infected}{%
\subsection{Who in the world would volunteer to get
infected?}\label{who-in-the-world-would-volunteer-to-get-infected}}

\textbf{Actually, a lot of people,} according to Josh Morrison and
Sophie Rose, the co-founders of an organization called
\href{https://1daysooner.org/}{1DaySooner}, which has gathered
signatures from over 8,000 potential volunteers. In The Washington Post,
Mr. Morrison and Ms. Rose
\href{https://www.washingtonpost.com/outlook/2020/04/27/vaccine-infection-volunteer-coronavirus/}{argue}
that the idea is not as radical as it sounds:
\href{https://www.thelancet.com/journals/laninf/article/PIIS1473-3099(20)30243-7/fulltext}{According}
to one study, the coronavirus's fatality rate for 20- to 29-year-olds in
China was 3 in 10,000 --- the same as that of
\href{https://www.ncbi.nlm.nih.gov/pubmed/22732041}{kidney donation}
surgery and roughly twice that of
\href{https://www.cdc.gov/reproductivehealth/maternal-mortality/pregnancy-mortality-surveillance-system.htm}{childbirth}in
the United States.

``We and many others are willing to take on what we see as an acceptable
individual risk to serve the public and the people we care about,'' they
write. ``As willing and well-informed volunteers, whose autonomy ought
to be respected, we feel challenge trials are justified if they mean a
vaccine arrives even one day sooner.''

\hypertarget{would-it-be-ethical}{%
\subsection{Would it be ethical?}\label{would-it-be-ethical}}

\textbf{We already allow people to risk their lives for the collective
good,} Dr. Lipsitch, Dr. Eyal and Dr. Smith say. Firefighters, for
example, are routinely called upon to rush into burning buildings. (And
today, of course, delivery drivers and grocery clerks are being asked to
accept a level of risk they did not sign up for.) The question, then, is
whether the study's potential cost would be low enough to warrant its
potential benefit. Besides recruiting only healthy, young volunteers and
guaranteeing them the best care, the authors delineate four ways in
which the study would minimize net risk:

\begin{itemize}
\item
  The vaccine may protect some of those who receive it.
\item
  Absent an effective vaccine, a high proportion of the general
  population is likely to get Covid-19, so some volunteers may simply be
  pushing their illnesses forward.
\item
  Only people who already have an especially high risk of exposure would
  be recruited (e.g., New Yorkers).
\item
  Volunteers would get priority for any treatments that may become
  available.
\end{itemize}

\textbf{But many researchers and bioethicists balk at the idea of
coronavirus human-challenge trials,} Jon Cohen
\href{https://www.sciencemag.org/news/2020/03/speed-coronavirus-vaccine-testing-deliberately-infecting-volunteers-not-so-fast-some\#}{writes}
in Science magazine. For one thing, the risks are hard to gauge, since
the virus is so new that we don't know how often people get seriously
ill or what its long-term complications are. ``Where you're going to
give somebody a virus on purpose, you really want to understand the
disease so that you know that what you're doing is a reasonable risk,''
Matthew Memoli, an immunologist who stages human-challenge studies of
influenza, told Mr. Cohen.

\textbf{There are also thorny ethical questions beyond risks and
benefits,} according to Seema Shah, a medical ethics professor at
Northwestern University. ``Justice considerations also matter, such as
whether the risks are fairly distributed,''
\href{https://www.vox.com/future-perfect/2020/4/9/21209593/coronavirus-vaccine-human-trials-explained}{she
told}Vox. ``There are also other criteria: community engagement, fair
selection of participants, robust informed consent, and payment that
compensates for time and inconveniences.'' And there lies another point
of contention: Dr. Eyal
\href{https://www.nature.com/articles/d41586-020-00927-3}{advises}
against using high payments, which he says could take advantage of the
poor.

\hypertarget{would-it-be-worth-it}{%
\subsection{Would it be worth it?}\label{would-it-be-worth-it}}

\textbf{It's possible that human-challenge trials wouldn't actually
speed up the process,} according to Myrone Levine, a vaccine researcher
at the University of Maryland who has conducted challenge trials since
the 1970s. Infections are still climbing rapidly in many places, so
conventional trials could reveal a vaccine's efficacy on the same
timeline. ``I cannot imagine that this would be ethical and would really
speed up what we have to do,'' Dr. Levine told Mr. Cohen.

\textbf{The benefit would also hinge on getting a lot of administrative
ducks in a row,} Dr. Shah said. For example, researchers would need to
coordinate globally to ensure consistency across trials and to ascertain
whether the Food and Drug Administration would even accept the results.
And as Dr. Lipsitch, Dr. Eyal and Dr. Smith acknowledged, even if all
goes well, more studies might be needed to prove the vaccine is safe and
effective for older populations.

``We're all looking for a Hail Mary, and it's easy to see challenge
studies as exciting and having a lot of promise,'' Dr. Shah said. ``But
a lot of things need to fall into place to achieve that promise.''

\emph{Do you have a point of view we missed? Email us at}
\href{mailto:debatable@NYTimes.com}{\emph{debatable@NYTimes.com}}\emph{.
Please note your name, age and location in your response, which may be
included in the next newsletter.}

\begin{center}\rule{0.5\linewidth}{\linethickness}\end{center}

\hypertarget{more-perspectives-on-human-challenge-trials}{%
\subsubsection{MORE PERSPECTIVES ON HUMAN-CHALLENGE
TRIALS}\label{more-perspectives-on-human-challenge-trials}}

\href{https://www.vice.com/en_us/article/5dm7na/why-intentionally-infecting-people-with-coronavirus-could-be-worth-it}{``People
Are Willing to Risk Their Lives for a COVID Vaccine. Should We Let
Them?''} \emph{{[}Vice{]}}

\href{https://nypost.com/2020/04/27/new-idea-to-speed-up-coronavirus-vaccine-research-is-bold-but-dangerous/}{``New
idea to speed up coronavirus vaccine research is bold but dangerous''}
\emph{{[}New York Post{]}}

\href{https://www.theatlantic.com/ideas/archive/2020/04/challenge-trial-ethical-imperative/610309/}{``A
coronavirus challenge trial is an ethical imperative''} \emph{{[}The
Atlantic{]}}

\begin{center}\rule{0.5\linewidth}{\linethickness}\end{center}

\hypertarget{what-youre-saying}{%
\subsubsection{WHAT YOU'RE SAYING}\label{what-youre-saying}}

\emph{Here's what readers had to say about the last edition:}
\href{https://www.nytimes3xbfgragh.onion/2020/04/28/opinion/biden-vice-president.html}{\emph{Biden's
vice-presidential pick}}

John C. Kornblum, a former U.S. ambassador to Germany, from Berlin:
``What about Val Demings, representative from Florida? She is
experienced, well spoken and did very well during impeachment hearings.
Demographically, who could be against a black, female former police
chief from FLORIDA.'' (Tom Coleman, a former Republican congressman,
also wrote in to recommend Ms. Demings.)

Alma from New Mexico: ``You did not mention Michelle Lujan Grisham,
governor of New Mexico and former chair of the congressional Hispanic
caucus. She has received national accolades for her handling of
Covid-19. She is popular with both progressives and moderates.''

Bruce from Hong Kong: ``I can't understand why no one is mentioning
Susan Rice, who should be the front-runner. Highly intelligent and
effective and demographically appropriate, Rice would be the perfect
foil for Biden.''

Deb from Minnesota: ``What prior experience did Trump have to qualify
him?''

Advertisement

\protect\hyperlink{after-bottom}{Continue reading the main story}

\hypertarget{site-index}{%
\subsection{Site Index}\label{site-index}}

\hypertarget{site-information-navigation}{%
\subsection{Site Information
Navigation}\label{site-information-navigation}}

\begin{itemize}
\tightlist
\item
  \href{https://help.nytimes3xbfgragh.onion/hc/en-us/articles/115014792127-Copyright-notice}{©~2020~The
  New York Times Company}
\end{itemize}

\begin{itemize}
\tightlist
\item
  \href{https://www.nytco.com/}{NYTCo}
\item
  \href{https://help.nytimes3xbfgragh.onion/hc/en-us/articles/115015385887-Contact-Us}{Contact
  Us}
\item
  \href{https://www.nytco.com/careers/}{Work with us}
\item
  \href{https://nytmediakit.com/}{Advertise}
\item
  \href{http://www.tbrandstudio.com/}{T Brand Studio}
\item
  \href{https://www.nytimes3xbfgragh.onion/privacy/cookie-policy\#how-do-i-manage-trackers}{Your
  Ad Choices}
\item
  \href{https://www.nytimes3xbfgragh.onion/privacy}{Privacy}
\item
  \href{https://help.nytimes3xbfgragh.onion/hc/en-us/articles/115014893428-Terms-of-service}{Terms
  of Service}
\item
  \href{https://help.nytimes3xbfgragh.onion/hc/en-us/articles/115014893968-Terms-of-sale}{Terms
  of Sale}
\item
  \href{https://spiderbites.nytimes3xbfgragh.onion}{Site Map}
\item
  \href{https://help.nytimes3xbfgragh.onion/hc/en-us}{Help}
\item
  \href{https://www.nytimes3xbfgragh.onion/subscription?campaignId=37WXW}{Subscriptions}
\end{itemize}
