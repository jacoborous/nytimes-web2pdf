I'm an E.R. Doctor in New York. None of Us Will Ever Be the Same.

\url{https://nyti.ms/3cg0Vpt}

\begin{itemize}
\item
\item
\item
\item
\item
\item
\end{itemize}

\hypertarget{the-coronavirus-outbreak}{%
\subsubsection{\texorpdfstring{\href{https://www.nytimes3xbfgragh.onion/news-event/coronavirus?name=styln-coronavirus-national\&region=TOP_BANNER\&block=storyline_menu_recirc\&action=click\&pgtype=Article\&impression_id=8fd33900-f1e9-11ea-bc23-b3853583cdbc\&variant=undefined}{The
Coronavirus
Outbreak}}{The Coronavirus Outbreak}}\label{the-coronavirus-outbreak}}

\begin{itemize}
\tightlist
\item
  live\href{https://www.nytimes3xbfgragh.onion/2020/09/08/world/covid-19-coronavirus.html?name=styln-coronavirus-national\&region=TOP_BANNER\&block=storyline_menu_recirc\&action=click\&pgtype=Article\&impression_id=8fd36010-f1e9-11ea-bc23-b3853583cdbc\&variant=undefined}{Latest
  Updates}
\item
  \href{https://www.nytimes3xbfgragh.onion/interactive/2020/us/coronavirus-us-cases.html?name=styln-coronavirus-national\&region=TOP_BANNER\&block=storyline_menu_recirc\&action=click\&pgtype=Article\&impression_id=8fd36011-f1e9-11ea-bc23-b3853583cdbc\&variant=undefined}{Maps
  and Cases}
\item
  \href{https://www.nytimes3xbfgragh.onion/interactive/2020/science/coronavirus-vaccine-tracker.html?name=styln-coronavirus-national\&region=TOP_BANNER\&block=storyline_menu_recirc\&action=click\&pgtype=Article\&impression_id=8fd36012-f1e9-11ea-bc23-b3853583cdbc\&variant=undefined}{Vaccine
  Tracker}
\item
  \href{https://www.nytimes3xbfgragh.onion/2020/09/02/your-money/eviction-moratorium-covid.html?name=styln-coronavirus-national\&region=TOP_BANNER\&block=storyline_menu_recirc\&action=click\&pgtype=Article\&impression_id=8fd36013-f1e9-11ea-bc23-b3853583cdbc\&variant=undefined}{Eviction
  Moratorium}
\item
  \href{https://www.nytimes3xbfgragh.onion/interactive/2020/09/02/magazine/food-insecurity-hunger-us.html?name=styln-coronavirus-national\&region=TOP_BANNER\&block=storyline_menu_recirc\&action=click\&pgtype=Article\&impression_id=8fd36014-f1e9-11ea-bc23-b3853583cdbc\&variant=undefined}{American
  Hunger}
\end{itemize}

\includegraphics{https://static01.graylady3jvrrxbe.onion/images/2020/04/19/magazine/19mag-ERDiary-images/19mag-ERDiary-images-articleLarge-v3.jpg?quality=75\&auto=webp\&disable=upscale}

Sections

\protect\hyperlink{site-content}{Skip to
content}\protect\hyperlink{site-index}{Skip to site index}

\hypertarget{im-an-er-doctor-in-new-york-none-of-us-will-ever-be-the-same}{%
\section{I'm an E.R. Doctor in New York. None of Us Will Ever Be the
Same.}\label{im-an-er-doctor-in-new-york-none-of-us-will-ever-be-the-same}}

A Covid diary: This is what I saw as the pandemic engulfed our
hospitals.

Credit...Philip Montgomery for The New York Times

Supported by

\protect\hyperlink{after-sponsor}{Continue reading the main story}

By Helen Ouyang

\begin{itemize}
\item
  Published April 14, 2020Updated May 27, 2020
\item
  \begin{itemize}
  \item
  \item
  \item
  \item
  \item
  \item
  \end{itemize}
\end{itemize}

\hypertarget{listen-to-this-article}{%
\subsubsection{Listen to This Article}\label{listen-to-this-article}}

Audio Recording by Audm

\emph{To hear more audio stories from publishers, like The New York
Times, download}
\href{https://www.audm.com/?utm_source=nyt\&utm_medium=embed\&utm_campaign=state_of_emergency}{\emph{Audm
for iPhone or Android}}\emph{.}

\hypertarget{first-week-of-march}{%
\subsection{First Week of March}\label{first-week-of-march}}

N.Y.C. Covid-19 cases, March 1: 1

I am in Karachi, Pakistan, on March 2, when I read the news: New York
City has its first patient hospitalized with the coronavirus. Though I
am more than 7,000 miles away --- reporting on a different disease
outbreak --- I am already worried about what I will face when I return
home in two days to my job as an emergency-room doctor in the city. Even
in the best of circumstances, the E.R. can be swamped, with patients
doubled up in rooms and too few monitors and beds to go around. Doctors
and nurses are always multitasking at the edge of their limits. ``Damage
control,'' we call it.

I know the situation with ­Covid-19 is already dire in different parts
of the world, Italy especially. Could our hospitals also be overtaken
that quickly? What would that look like? I need to know what might come,
what decisions I might be confronted with. I want to hear about them
directly from health care workers in Italy.

A few days from now, I will come across the name of Guido Bertolini, a
clinical epidemiologist who studies intensive care. Through a colleague
of his, I reach out to him over Whats­App, and we begin corresponding.
He had been high up in the Italian Alps through the last day of
February, when the distressing messages started to come in from
colleagues asking him to join a new Coronavirus Crisis Unit for
Lombardy, a region in northern Italy. Some of the pleas had an Excel
file attached. When Bertolini opened it, he tells me, he couldn't
believe the numbers. He had to see the situation for himself.

With an E.R. doctor from Milan, he drove to the Lombardy city of Lodi
the next day. He was horrified by what he witnessed. ``So many patients,
in every corner,'' he says. ``They were attached to oxygen in all
possible ways.'' Individual oxygen dispensers, meant for single
patients, were being split among four people at a time. ``When we came
out, we were silent for all the journey home,'' he says. ``We could not
speak.'' He knows the hospital has already passed its maximum capacity.

``From my position in the crisis unit, I see the whole picture,'' he
says. ``Which is dramatic.'' Lombardy is one of Italy's richest areas,
where there is ``almost no limit in resources,'' he explains. Yet the
region has only half the number of I.C.U. beds it needs to care for all
the critically ill patients infected with ­Covid-19. He knows doctors
are soon going to have to decide who lives and who doesn't. How could he
help them do that?

He begins rounding up --- virtually, over Skype --- a group of
bioethicists and I.C.U. physicians. They include Marco Vergano, a
45-year-old I.C.U. doctor in Turin, in the neighboring province of
Piedmont, who is also the chairman of the bioethics group of Italy's
society of intensivists (Siaarti). He's working back-to-back shifts in
the I.C.U., but he jumps online with the six other members of the task
force that Bertolini has set up. That night, he begins drafting a
document. The first version includes strict criteria. If you are over 80
or one of your organs isn't functioning well or your dementia has
advanced past a certain point, you are unlikely to get a breathing tube
or a spot in the I.C.U.

Soon after, the group decides to delete the specific cutoffs, so that
hospitals can adapt their responses to circumstances, which are changing
hourly. They want doctors to have flexibility but use these principles
to guide and justify their decision-making. The document's fundamental
thrust, though, is that those with the highest chances of survival ---
the young and the healthy --- get priority. They strongly advise against
allocating precious resources, like ventilators and beds, on the
traditional basis of first-come, first-served, which would reduce the
number of lives a hospital could save. ``Soft utilitarian'' is how
Vergano characterizes the approach.

Swift and fierce denunciation of the group and
\href{https://www.cssr.news/2020/03/overwhelmed-by-the-virus-the-issue-of-extreme-triage/}{its
recommendations} follows the document's release. ``You cannot imagine to
what extent we have to face harsh criticism,'' Vergano says. ``From
colleagues to journalists to bioethicists --- we are in firing lines
these days,'' Bertolini adds. ``They say we are God-playing.''

Vergano notes that most of the criticism has come from regions in Italy
that have yet to be hit as hard as Lombardy. They are ``completely
living in another world,'' Bertolini says, because ``unless you are
inside this situation, you cannot understand fully.'' People are, it
seems, woefully bad at grasping how future events will unfold, whether
in the city next door tomorrow or across the Atlantic a couple of weeks
later.

Back in New York, I work a couple of shifts in the E.R. Though we've
been given updated instructions for screening patients for coronavirus
--- which say that a person need not have a history of recent travel to
qualify for testing --- the hospital feels mostly like its normal,
hectic self, as it did before I left the country.

\begin{center}\rule{0.5\linewidth}{\linethickness}\end{center}

\hypertarget{second-week-of-march}{%
\subsection{Second Week of March}\label{second-week-of-march}}

N.Y.C. Covid-19 cases, March 8: 14

We have started to hold regular Covid-focused meetings over Zoom.
Participants ask questions about the availability of tests and how we
should protect ourselves, but no one seems very worried by what's
unfolding in Italy.

Bergamo, a city of 120,000, with about a million more in the surrounding
province, sits at the foothills of the Alps, 25 miles northeast of
Milan. Travel guides describe how the upper part of the city, perched
high on a hill and encircled by walls, is connected to the lower part by
walking paths and a funicular. The city is known for its spectacular
medieval architecture. The area, home to San Pellegrino sparkling water
and a manufacturer of brakes for Formula One cars, is also a busy
transit hub, with an airport that serves over 12 million passengers a
year. Doctors tell me the province of Bergamo has been hit the hardest
by this pandemic.

Papa Giovanni XXIII Hospital, which provides advanced, state-of-the-art
medical care, is one of the biggest hospitals in the region, housing
more than 900 beds. It probably has the highest number of Covid
infections in the country. Andrea Duca, an E.R. doctor there, has been
treating these patients for a couple of weeks now, since the first case
was detected. They had only sore throats and mild coughs to start, but
after a few days, patients were showing up with more severe symptoms.
They had significant lung infections and low oxygen levels, even when
they didn't look that ill. Some of them had diarrhea instead of
respiratory complaints, which made diagnosis confusing. The clinical
picture was different from what Duca and his colleagues expected. ``The
virus is as free as the wind,'' Pietro Brambillasca, an anesthesiologist
who works with Duca, tells me over the phone. ``It does whatever it
wants.''

The patients keep coming. Beds fill up. Ventilators get parceled out.
Quickly, there are many more patients than equipment and space. Doctors
can be recruited, or take on more patients than they are usually
comfortable with, but what to do about the lack of resources? Who gets
the precious few ventilators?

Those deemed too old or too sick don't get ventilators or have them
taken away so that they can be used for patients who are more likely to
survive. Duca recalls for me one of the first patients he subjected to
this calculation. The man, 68, had transplanted lungs. His oxygen level
had dropped; his breathing rate increased. ``I knew that he was not
doing well,'' Duca says. But there were no spots in the I.C.U., because
they were filled with younger and healthier patients whose prospects of
recovery were greater. Duca made the difficult decision not to give the
patient a breathing tube, to save the ventilator for someone more likely
to live.

Family members weren't allowed into the hospital because they, too,
could get infected or spread the virus to others if they themselves were
sick. But Duca asked for permission from his supervisor to let the man's
wife and daughter in, just for a few minutes. ``I saw his face when he
looked at his wife coming inside this room,'' Duca recalls. ``He smiled
at her. It was a fraction of a second. He had this wonderful smile.'' He
continues: ``Then I saw that he was looking at me. He realized that
there was something wrong if only \emph{his} relatives were coming
inside.'' The man knew in that instant that he was going to die, Duca
says. As the man's breathing worsened, morphine was started. He died 12
hours later.

``Which one is the lucky man of the day?'' Brambillasca asks. He
normally cares for very sick children who have had organ transplants,
but since the outbreak, he has been called to float between the E.R. and
the I.C.U. When we speak by phone one morning, on one of my days off, he
sounds defeated. His wife, an otolaryngologist, has also been recruited
to the effort: She is now working in a Covid unit in a neighboring
hospital. I can hear their 1-year-old daughter in the background. Every
day, Brambillasca feels inadequate. ``I ask myself if I'm more useful if
I go outside my home, take paper and alcohol and disinfect the doorknobs
of my neighbors instead of going to work as a doctor,'' he says.

Brambillasca tells me about how he had two patients side by side one
day. One man was around 65 and had been on a ventilator for 10 days. He
had heart problems, and he wasn't improving. To his left was another
man, about the same age but healthy. His breathing was becoming faster
and shallower. Over the course of two minutes, Brambillasca decided to
take the ventilator away from the first man and give it to the second
one. ``If you think of it as saving the most number of lives, that's it,
you have to do it,'' he says. ``But I'll become an ice-cream maker
instead of a doctor if I have to go on this way.''

Will I, too, feel that way soon? We are starting to see some cases in
our hospitals, but it's nothing like what doctors in Italy are
describing. They warn me that we are about two weeks behind them. Could
we really get to where they are in such a short time?

\begin{center}\rule{0.5\linewidth}{\linethickness}\end{center}

\hypertarget{third-week-of-march}{%
\subsection{Third Week of March}\label{third-week-of-march}}

N.Y.C. Covid-19 cases, March 15: 330

One of our E.R. doctors, who also works in the I.C.U., proposes an
extreme case during a Zoom meeting: We know from China's experience that
once a patient is in cardiac arrest from ­Covid-19, the chance of
survival is essentially zero. His words hang in the air, but the
question is clear: Should we try to resuscitate this patient, despite
our equipment shortages and the risks to ourselves? As this hypothetical
situation plays out in my head, I immediately want to know the age of
the patient. In practice, this decision comes sooner for me than I
expect. For now, it's only a shadow of what my Italian counterparts are
facing, but it forces the very real question of how to allocate
resources, whether ventilators or beds --- or those of us who work in
the E.R.

A man in his late 80s is sent in from a nursing home with a fever, cough
and diarrhea. He is my first patient who is most likely Covid-positive;
I can't know for sure, because tests are taking up to 24 hours to come
back in our internal lab. Although the man is designated D.N.R./D.N.I.
--- ``do not resuscitate'' and ``do not intubate,'' which instruct us
not to pursue aggressive interventions like electric shocks and
breathing tubes --- his family, with death now looming, reverses his
no-resuscitation order and decides, instead, that he should receive even
the most extraordinary lifesaving maneuvers. The man hasn't walked in
years; he has advanced dementia and was unable to talk even before this
most recent illness. He can't tell me what he wants, so under normal
circumstances we are to follow the family's orders. They are in the
waiting room, unable to come in because of our new, strictly enforced
no-visitor policy, to prevent virus transmission. With the man's
breathing rapidly worsening, I don't have time to call them. I am
supposed to obey their wishes, which the doctor from the nursing home
had, in his spare cursive handwriting, documented in a statement.

\hypertarget{latest-updates-the-coronavirus-outbreak}{%
\section{\texorpdfstring{\href{https://www.nytimes3xbfgragh.onion/2020/09/08/world/covid-19-coronavirus.html?action=click\&pgtype=Article\&state=default\&region=MAIN_CONTENT_1\&context=storylines_live_updates}{Latest
Updates: The Coronavirus
Outbreak}}{Latest Updates: The Coronavirus Outbreak}}\label{latest-updates-the-coronavirus-outbreak}}

Updated 2020-09-08T15:29:57.612Z

\begin{itemize}
\tightlist
\item
  \href{https://www.nytimes3xbfgragh.onion/2020/09/08/world/covid-19-coronavirus.html?action=click\&pgtype=Article\&state=default\&region=MAIN_CONTENT_1\&context=storylines_live_updates\#link-547feae1}{Senate
  Republicans plan to move forward with a scaled-back stimulus package.}
\item
  \href{https://www.nytimes3xbfgragh.onion/2020/09/08/world/covid-19-coronavirus.html?action=click\&pgtype=Article\&state=default\&region=MAIN_CONTENT_1\&context=storylines_live_updates\#link-679303d7}{Nine
  drugmakers pledge to thoroughly vet any coronavirus vaccine.}
\item
  \href{https://www.nytimes3xbfgragh.onion/2020/09/08/world/covid-19-coronavirus.html?action=click\&pgtype=Article\&state=default\&region=MAIN_CONTENT_1\&context=storylines_live_updates\#link-1c973131}{`The
  lockdown killed my father': Farmer suicides add to India's virus
  misery.}
\end{itemize}

\href{https://www.nytimes3xbfgragh.onion/2020/09/08/world/covid-19-coronavirus.html?action=click\&pgtype=Article\&state=default\&region=MAIN_CONTENT_1\&context=storylines_live_updates}{See
more updates}

More live coverage:
\href{https://www.nytimes3xbfgragh.onion/live/2020/09/08/business/stock-market-today-coronavirus?action=click\&pgtype=Article\&state=default\&region=MAIN_CONTENT_1\&context=storylines_live_updates}{Markets}

We are weeks away from the full impact of this outbreak, but we are
already trying to conserve masks, gowns and face shields. Because of how
infectious the virus is and the country's lack of preparation and
equipment, the decision to intervene is a question not only of how to
apportion tangible supplies, but also of how to best distribute risk
among health care workers. I want to do everything for my patients, as
much as they and their families want, just as we have always done. But
what do I owe future patients? What do I owe my colleagues? As I look at
my team of doctors and nurses and consider our next steps, I think of a
recent Facebook post from one of my supervising physicians, who trained
me during residency, William Binder, who is now in his 60s. ``As an
emergency physician, I understand anything can happen to anyone at any
time, but I have never felt exposed nor susceptible,'' he wrote. ``The
coronavirus has stripped away my veneer of invincibility.''

\includegraphics{https://static01.graylady3jvrrxbe.onion/images/2020/04/19/magazine/19mag-ERDiary-images-02/19mag-ERDiary-images-02-articleLarge-v3.jpg?quality=75\&auto=webp\&disable=upscale}

\begin{center}\rule{0.5\linewidth}{\linethickness}\end{center}

\hypertarget{fourth-week-of-march}{%
\subsection{Fourth Week of March}\label{fourth-week-of-march}}

N.Y.C. Covid-19 cases, March 22: 9,045

It has been only a week since my colleague first posed the hypothetical
case about resuscitating a Covid-infected patient whose heart has
stopped. I feel like that was a different world back then, one in which
we all held onto a thread of optimism that we would not have to face
Italy's choices. Early on, I joined several private Facebook groups for
doctors and browsed health care workers' feeds on Twitter. But these
posts soon feel unbearable; it's suddenly too much to see clinical
scenarios discussed hypothetically. In New York City, the hypothetical
is here. A bunch of us in the E.R. have started communicating through a
Whats­App group chat so that we can openly discuss how we're feeling
about the pandemic response. Three New York City hospitals are rumored
to be out of ventilators.
\href{https://www.nytimes3xbfgragh.onion/2020/03/25/nyregion/nyc-coronavirus-hospitals.html}{Thirteen
Covid patients died in one hospital in 24 hours}. A refrigerated truck
is sheltering dead bodies there because the morgue is already full.

We are starting trial runs of putting two patients on one ventilator at
my hospital. I can't believe we are coming up to this point already. So
many patients are overflowing into the hallways, relying on oxygen tanks
instead of the dispensers on the walls. Do we have enough tanks in the
E.R.? No one in the chat group knows. Should someone take control of
their supply? Yes, good idea. Someone suggests medical students, but the
school wants to protect them from exposure to the virus in the E.R.

It seems impossible to avoid getting infected. You would have to be
perfect, and in the mayhem of the E.R., it's nearly impossible to be
even good. I make mental calculations to keep all protective equipment
on for my eight-hour shifts; during my 12-hour shifts, I'll remove it
only twice, to eat or drink. Two Italian colleagues --- a doctor and a
nurse --- have already warned me about the physical toll of wearing this
equipment on their aching faces, their noses rubbed raw, the tracing of
their masks etched into their skin.

On days off, I try to learn what I can about this virus and its many
tricks. I watch videos on how to best manage patients on their
ventilators. Their respiratory needs are different from what I'm used
to. Go high on the oxygen and the post-exhalation pressure. Keep the
breaths small, though, because Covid lungs are thought to be stiff and
might overstretch. It's a delicate balance between trying to protect the
healthy parts of the lung while giving injured areas time to rest. We're
still learning about the disease, though, and it seems some patients'
lungs might have different needs than this.

OK, onto the heart. Someone sends me ultrasound images of profound heart
failure in a Covid patient he cared for. There's the gut, too ---
patients can experience a lot of diarrhea. But we can't give them too
much in the way of IV fluids or we could flood the respiratory system. I
recall a mantra from my days in residency: ``A dry lung is a happy
lung.'' An Italian doctor tells me that she's learning that the kidneys
could also take a hit, compromising their ability to filter waste from
the blood. Every part of the body comes under attack, it seems.

We're temporarily out of the proper disinfectant wipes at the E.R. in
one of our hospitals. Someone intubates two patients --- a procedure
that risks exposing the medical worker to discharge from a patient's
nose and mouth --- without a face shield because none were immediately
available. A co-worker is collaborating with others to 3-D-print face
shields. Emails come through from hospital leadership and the city's
health department telling us to be ``appropriate'' and conserve our N95
masks. A physician assistant is baking her masks in the oven to
sterilize them. She shares her recipe: 170 degrees for 30 minutes.
Others spray theirs down with Lysol after every shift. I was shocked
when they told us to use these single-use masks for the whole day; now
we are told they must last multiple shifts. We can discard them only if
they become visibly soiled. Otherwise, wear the same one --- ``for
multiple patients, for multiple shifts.'' How am I supposed to know when
a mask should be thrown out? What does a virus particle look like,
anyway? I start telling my residents that it's better to be lucky than
to be good.

The hospitals I work at are nearing maximum occupancy, even as new
quarters are constantly being opened to accommodate more patients
infected with Covid. In the meantime, updated clinical recommendations
are given to us to follow: If patients' oxygen levels are slightly below
normal, send them home anyway if they look OK. Let's hope they know when
to come back, I think. Brambillasca, the Italian anesthesiologist, tells
me that his patients often look well, but if their oxygen reading is
slightly low, they can ``crump'' --- medical slang for getting very ill
very quickly. When I share this with colleagues, a couple of them start
to counter: ``The current evidence says. \ldots{} '' What evidence? The
novel coronavirus has been around for only a few months.

Everyone in medicine knows that one of the most heart-dropping phrases
you can hear is: ``You know that patient you saw the other day? Well, he
came back and. \ldots{} '' I think of all the doctors who sent their
patients home because they looked well or are young or don't have
medical problems, and they came back to the E.R. needing a breathing
tube. I'm sure these patients all looked OK a few days ago.

Don't worry, we hear, Andrew Cuomo, New York's governor, is protecting
us from lawsuits. He had issued an executive order stating that
physicians ``shall be immune from civil liability for any injury or
death'' while caring for patients during the Covid outbreak, unless it's
a case of ``gross negligence.'' I ask my co-workers if anyone is still
concerned about getting sued. I think we're much more anxious about
having to live with people dying --- and possibly getting sick
ourselves. Not only do we have to think about patients not getting
ventilators, but now we have to worry about sending infected people
home, where they will likely worsen and may become critically sick,
unable to make it back to the hospital in time. Paramedics say they are
seeing 300 ``dead on arrival'' cases in one day, citywide, instead of
the usual 50 or so.

As soon as I hear this, I venture out that night to buy two pulse
oximeters, small devices that go on a person's finger to monitor his or
her respiratory status. I can keep track of friends and neighbors who
fall ill. Even when I'm at home, I can help triage. I bring one to work,
to test patients' oxygen levels, to see how much they drop when they
walk. I'm told we will give them to patients soon, so they can monitor
themselves --- and maybe to-go oxygen containers as well, if they're
needed. I hope this will be effective. But it seems a lot to ask of
someone who's very sick.

One colleague, who is over 60, already has a plan if she feels ill.
She'll check her oxygen measurement, and if it's less than normal,
she'll consult an
\href{https://www.nytimes3xbfgragh.onion/interactive/2020/us/coronavirus-us-cases.html}{outbreak
map online} and survey the surrounding states. She'll pick the closest
city with the smallest number of cases. Then she'll drive there and hope
that her age won't be considered when it comes to the care she gets. I
find out that more doctors are hospitalized with the virus. One E.R.
colleague across town is intubated. I get texts from colleagues across
the country about doctors who are infected and hospitalized, some in the
I.C.U., some intubated. I look at my reused mask. It doesn't seem soiled
yet. I put it back on my face. Better to be lucky than to be good, I
tell myself.

On Twitter, I see a
\href{https://twitter.com/sdslavin/status/1241327739954749440}{photo of
resident doctors} at Massachusetts General Hospital, where I trained,
holding up official documents explicitly designating who should make
decisions for their care if they become critically ill. The thought is
overwhelming, but I know, as a doctor, I want my patients to do the
same. I decide to do it unofficially, texting a close doctor friend I
work with and telling him what I want in writing. Please try, for as
long as possible, if there's a chance I can make a decent recovery. If
not, well, you know what to do. After all, someone else could probably
use that ventilator.

\begin{center}\rule{0.5\linewidth}{\linethickness}\end{center}

\hypertarget{fourth-week-of-march-1}{%
\subsection{Fourth Week of March}\label{fourth-week-of-march-1}}

N.Y.C. Covid-19 cases, March 24: 14,905

Though it has been only two weeks, I desperately ache for that time when
a patient testing positive for Covid was a surprise. I think back to the
man from the nursing home. I had made the decision that day to intubate
him, which would necessitate giving him a ventilator and an I.C.U. bed.
It was early on in New York's outbreak, and we were still in
patient-centered mode, as the doctors in Italy put it. They are deep
into community-centered care now. ``As physicians, we normally choose
the best option for the patient,'' Giovanna Colombo, an I.C.U. doctor at
Papa Giovanni XXIII Hospital, tells me. ``We don't have to think of the
community implications of what we're doing. But the epidemic setting is
completely different.''

I call up Mirco Nacoti, another I.C.U. doctor there. ``Without
guidelines,'' he tells me, ``it's impossible to work. And there is no
space for imagination during humanitarian crisis. If you use a lot for
the first patient, then you have no treatment for the next patient. You
have to reorganize everything. You have to reorganize your mind; you
have to reorganize your work; you have to reorganize your personnel and
health care people.''

Marco Vergano, a co-author of the controversial Siaarti guidelines, had
removed the criteria from the document because he wanted to give doctors
flexibility --- and because he knew the criticism would be overwhelming.
Yet clear criteria are what physicians want most. The doctors I speak to
in Italy all want a specific formula and decisions from a third-party
team, one whose members aren't directly treating the patients. ``You're
caring for patients who are complex enough,'' Colombo tells me. ``If you
keep thinking of this problem, you can never do this job.''

To help with this task in Bergamo, a few weeks into the outbreak, a
doctor at the hospital comes up with a scoring system. It's not meant to
be a strict make-or-break guideline, but it functions as a tool to help
in decision-making. It has specifications that the Siaarti document
lacks. In this tiered system, patients are scored for age, medical
problems and the severity of their current respiratory status. The
higher a patient's final tally, the higher the priority for intubation.

In New York City this week, the conversation shifts. The question of who
gets a ventilator and who doesn't comes up in every single Zoom meeting
among E.R. physicians that I participate in. A hospital committee is
discussing that, we're told. We want guidelines; nobody wants to
exclusively treat people first-come, first-served. I've thought and
written about what makes a meaningful life, and I generally agree that
means autonomy for patients and families; they should get to make
decisions about their treatment. But I do believe that when resources
are scarce, doctors can and should make judgments about who should get
more care. A colleague, feeling similarly, announces during a meeting:
Soon I'm just not going to intubate the 80-something-year-old patient
who doesn't talk or walk so that there will be a ventilator available
for the 30-year-old who comes in later. It sounds heartless, but we
agree with her. Future patients like the 30-year-old are not yet here,
but they are definitely on their way.

One of my residents asks me, ``Will there be ventilators for us if we
need them?'' As with many questions I've been receiving lately, I don't
know the answer to that one.

\begin{center}\rule{0.5\linewidth}{\linethickness}\end{center}

\hypertarget{fourth-week-of-march-2}{%
\subsection{Fourth Week of March}\label{fourth-week-of-march-2}}

N.Y.C. Covid-19 cases, March 26: 23,112

I am scheduled to be off from work for several days. The evening before
I'm due to return to the hospital, a colleague messages our group to say
that a 49-year-old Covid patient of hers, who was waiting in the E.R.
for an inpatient bed, was found blue and dead in a chair. Nobody even
knows if he gasped before he died. On my way to work, I hear on the
radio that a 48-year-old nurse from another New York City hospital has
died from coronavirus. Someone else tells me that an anesthesiologist at
our hospital is on a ventilator. A surgeon and an E.R. doctor across
town are in similar states.

When I walk through the hospital doors, the E.R. is a place I no longer
recognize. Intubated patients, of every age, are on ventilators
everywhere. It feels simultaneously electrifying and oppressive. But
it's also eerily quiet. Family members and friends haven't been allowed
into the E.R. for more than a week; most of the patients are too sick to
talk; the few without breathing tubes who are able to cough are muffled
by their masks. Oxygen hisses in the background. A couple of hours into
my shift, one of the nurses comes to me. She falls apart, tears
streaming down her inflamed, marked cheeks. She sobs out words of anger
and frustration and sadness. The morning, on top of the last several
days, has crushed her. I want to hug her, but I can't.

Soon after that, someone asks, ``Doctor, is it OK to take the patient to
the morgue?'' The other physician on duty and I look at each other. The
morgue? Who just died? Apparently, a patient who was waiting for an
inpatient bed, whose family had decided against extreme resuscitative
measures, had died, without us even knowing.

Several days ago, only a few patients had Covid, but suddenly it seems
we have become, like facilities in Italy, a Covid hospital. Every
patient seems to test positive for it. I am shocked by the one or two
negative results I receive during a shift. We have to function as if
everyone is infected.

A co-worker tells me he used three masks during the course of his shift.
Three masks?! I respond. That's crazy! Then I realize I am the absurd
one. The masks are meant for single use, one per patient encounter; my
colleague had used three masks over a 12-hour shift, most likely having
seen upward of 30 patients who potentially have Covid. It's idiotic that
I was shocked by his using three masks, especially when many of our
co-workers in the city have fallen ill.

Patients who test positive for the virus are unintentionally roomed with
those who test negative or whose tests are still pending, because the
E.R. is bursting. Even if we are exposed to a patient without proper
personal protective equipment, we are expected to return to work if we
don't have symptoms. In Italy, where 61 doctors have already died from
Covid (a number that will grow past 100 in the next couple of weeks),
health care workers believe that they themselves expedited the spread of
the virus. There, the doctors are routinely tested for any exposures,
even if they are asymptomatic.

I have to shut down thoughts about my own risks and mortality. I recall
the words of my old mentor, but I don't think I can do this job unless I
force myself to believe in my own invincibility. Otherwise, with every
violation of the protective barrier, every instance of less-than-ideal
protection, which is almost every time, I would be paralyzed by thoughts
of having infected myself. I see a patient around my age intubated, hear
about a hospital colleague getting critically ill. A co-worker texts
that her classmate from residency is now intubated. I read an article
about how health care workers seem to suffer more from serious Covid
infections, even if they're young, possibly as a result of being exposed
to higher initial doses of the virus. I'm not even sure this is true
anymore --- I've seen plenty of critically ill patients in their 30s and
40s. I push these thoughts away, immediately. Better to be lucky than to
be good, I remind myself. It's the only thing that provides some
reassurance. If I feel like it's not totally in my control, then I won't
completely lose my mind over every mistake I make donning and removing
my P.P.E. and recycling single-use equipment.

I look in the mirror for the first time when I get home one night. My
eyes are bloodshot. Deep horizontal creases run across my cheeks. A
faint abrasion has already settled into the bridge of my nose. I just
want to fall into my bed, but I force myself to shower. When I turn my
phone back on, a nurse in Bergamo, Stefania Cornelli, has texted me that
she crashed her car. The vehicle was totaled, but she wasn't seriously
hurt. It had been about one month into this crisis for her. ``We are so
tired, tired of a tiredness that no sleep can relieve,'' she writes. ``I
think I really need to ask help to a psychologist.''

\begin{center}\rule{0.5\linewidth}{\linethickness}\end{center}

\hypertarget{fourth-week-of-march-3}{%
\subsection{Fourth Week of March}\label{fourth-week-of-march-3}}

N.Y.C. Covid-19 cases, March 28: 30,766

When I get to work the next day, a patient who had a breathing tube
inserted overnight had woken up enough to pull it out. She was
delirious, lacking oxygen to her brain, and had also yanked out her IV
lines. Sputum and blood and sweat are flying everywhere in the room. My
instinct is to run in to help, but I force myself to pause, put on all
the equipment. I place the N95 respirator on my face --- and a surgical
mask over the N95 to keep it clean and reusable, as we're instructed ---
as well as a gown, goggles, gloves and a face shield, 3-D-printed by my
colleague. It's so hot. I start sweating immediately. We manage to
reinsert her breathing tube and replace her IV lines; she safely makes
it to the I.C.U. After an hour working like this, I feel lightheaded,
but it is too early to remove the mask and drink water. How do I make it
through the next 12 hours?

Later in the day, I start getting chills underneath all my equipment. I
briefly wonder if I'm getting sick, then I decide that it will become
obvious if I am, that I should just go on for the day. Even if I develop
symptoms, I'm not able to get a test from employee-health services at my
hospital anyway. Whenever I have patients come in telling me that they
tested positive at their doctor's office or at urgent care, I
immediately take note of where they got that done.

I get statistics from my hospital indicating that over 80 percent of the
admitted patients from the previous day have tested positive for
­Covid-19. I feel at odds with myself, conflicted between my emotional
response and my intellectual curiosity about this virus, which seems, as
Brambillasca said, to be mercurial --- reckless in what it inflicts on
its victim. I scroll through the lab tests of patients. I excitedly
exclaim out loud that one patient's lymphocytes, a type of white blood
cell, are very low, something I had read about. Then I pause, realizing
that this is a sign that the patient probably won't do well. These
observations happen repeatedly; I pendulum back and forth between my
fascination with the disease and my despair for my patients.

Six hours into my shift, I go to the bathroom for the first time. I
stand in the unvented bathroom for a minute and pull my mask away from
my face. The air is stale, but the rush of oxygen into my lungs feels
wonderful. I take big gulps of it through my nostrils before letting the
mask compress my face again. When I'm not in the hospital, I feel a
phantom mask on my face. I wake from sleep trying to adjust it, thinking
it's still on.

I try to preserve the equipment that I do have, but the steps seem
futile. In the E.R., I sanitize, glove, remove glove, sanitize again. I
have to touch a door handle to go into the workroom to type my clinical
notes. I'm unable to sanitize again because there are no more portable
hand sanitizers left. I get flustered when I accidentally touch my face,
wondering how I forgot and slipped. Sometimes, I can't remember if the
gloves on my hands are clean or dirty.

At so many points I ask myself, Does it even matter anymore? It feels
like the virus is everywhere, breathing on all the surfaces, exhaling
itself into the atmosphere. It feels exhausting wearing one mask all day
and covering it with another to keep it clean, having to think so much
about not getting it soiled and wondering if I accidentally contaminate
the inside of it when I hold it away from my face to breathe for a
minute or take it off to chug water. Sometimes I see the individual
virus particles --- round with red, protruding crown-shaped spikes, like
the C.D.C.'s rendering --- everywhere in the hospital, on beds and
monitors and phones and computers. I shudder, forcing myself to erase
the image from my mind.

I've taken part in humanitarian relief missions in more than 20
countries, in settings as resource-poor as mobile clinics in South Sudan
immediately after its secession, refugee camps in Kenya, an abandoned
war hospital in Liberia, medical facilities in Somalia. Never have I
personally felt unsafe, like I didn't have enough protection for myself.
People are now referring to ours as ``a third-world country,'' but in
terms of P.P.E. in this pandemic, it's actually worse than those
overseas hospitals. While most of the specialists have been
unflinchingly generous, offering extra hands in the E.R. and imperiling
their own lives, a few doctors who are consulted for their expertise on
certain medical conditions have balked at having to see patients here at
all. They feel unsafe, they say. Deep down, I know they're probably
right.

By the end of my shift, every patient begins to blend into a single
patient. ``Fever and cough,'' ``fever and cough and shortness of
breath,'' ``cough and trouble breathing,'' ``sent by doctor's office for
Covid rule-out,'' ``sent from urgent care for Covid test.'' I can't even
keep track of them anymore. Usually I remember patients by their faces,
but they all have masks on too, so all I see is their eyes, which more
often than not are closed.

I become obsessed with oxygen levels, which seem to be the only reliable
indication of how patients are doing. Is 92 percent much better than 90
percent? Should 93 be the cutoff to send someone home, or should I make
it 94? I used to be able to rely on my gut and clinical judgment when I
walked into a room and looked at the patient, but coronavirus is
lawless. It obeys no rules. What is unusual, in this illness, is that
many people come in talking to you, even as their breathing worsens.
They can speak, but their oxygen readings are frighteningly low. As the
hours tick by, they rapidly get sicker, to the point where they need a
breathing tube. In most other situations, people who require breathing
tubes in the E.R. arrive at the hospital too ill to interact with me,
needing mechanical ventilation right away. That makes it a little
easier.

Patients' oxygen tanks run out. (It's impossible to know unless you bend
over, look behind the stretcher and glimpse the thin black needle ticked
over to the red zone on the gauge.) Or whatever oxygen you did give them
becomes suddenly insufficient, as their lungs grasp for ever more. Maybe
an alarm bell sounds because their oxygen level has dropped. Or more
likely, they've become disconnected from the monitor, a far-too-frequent
occurrence, and you see them frantically trying to breathe. Or most
likely, the oxygen, even if it's blowing, is of no use, because they're
unable to take it in, barely inhaling at all, silently dying, alone.

What may have been unimaginable even a week ago seems completely
possible, even likely, now. A colleague informs me that she had to push
aside a dead body to plug in a ventilator for a new patient who was
recently intubated. Is this how the dead leave the world now?

Before, I would check in with the Italian doctors, concerned for their
and their patients' well-being, but our roles have now reversed. I am
now at the receiving end of their grief and sympathy. ``How are you?''
one texts me. ``We hear it's so bad there.'' Yes, it really is. ``Stay
strong,'' another says.

We're unable to reliably predict who does well and who doesn't. Old or
young, all seem wholly vulnerable. Politicians, epidemiologists, even
doctors have been saying that people in their 20s and 30s who get sick
already have medical problems or are obese, but then, right after I hear
that, I need to put a young and fit patient on a ventilator. The virus
is impulsive, attacking one person more ferociously than another. I feel
the compression from all sides --- the I.C.U. is full, the E.R. is full
--- I just don't see the end of this in sight. When I think about that,
I feel submerged, and my instinct is to rip off my mask and leave the
hospital. Then I try to convince myself that it's like running. When you
start off, your lungs burn and your legs ache, but as your stride hits a
rhythm, you start to feel good, and you know you can go on for miles. I
hope intensely for that moment to come soon.

I know many New York hospitals are working on their own
resource-allocation guidelines and designating a third-party team of
in-house doctors to decide which patients get to have their care
escalated. Now that I'm already involved in helping to make those
decisions, I'm less worried about getting the criteria in my hands. I'm
also hopeful that external relief will come. I used to travel to others
to provide humanitarian assistance, and now people and materials are
coming here to help. Makeshift hospitals are opening around the city and
will take some of the load off. When we hear that the Javits Center and
the Navy hospital ship Comfort will care only for non-Covid patients, my
colleagues and I find this laughable, because everyone has the virus.
They'll realize it soon enough, we say to one another. (A few days after
it opens, the Javits temporary hospital changes its admission policy to
take in Covid patients; the Comfort does so the following week.)

\href{https://www.nytimes3xbfgragh.onion/news-event/coronavirus?action=click\&pgtype=Article\&state=default\&region=MAIN_CONTENT_3\&context=storylines_faq}{}

\hypertarget{the-coronavirus-outbreak-}{%
\subsubsection{The Coronavirus Outbreak
›}\label{the-coronavirus-outbreak-}}

\hypertarget{frequently-asked-questions}{%
\paragraph{Frequently Asked
Questions}\label{frequently-asked-questions}}

Updated September 4, 2020

\begin{itemize}
\item ~
  \hypertarget{what-are-the-symptoms-of-coronavirus}{%
  \paragraph{What are the symptoms of
  coronavirus?}\label{what-are-the-symptoms-of-coronavirus}}

  \begin{itemize}
  \tightlist
  \item
    In the beginning, the coronavirus
    \href{https://www.nytimes3xbfgragh.onion/article/coronavirus-facts-history.html?action=click\&pgtype=Article\&state=default\&region=MAIN_CONTENT_3\&context=storylines_faq\#link-6817bab5}{seemed
    like it was primarily a respiratory illness}~--- many patients had
    fever and chills, were weak and tired, and coughed a lot, though
    some people don't show many symptoms at all. Those who seemed
    sickest had pneumonia or acute respiratory distress syndrome and
    received supplemental oxygen. By now, doctors have identified many
    more symptoms and syndromes. In April,
    \href{https://www.nytimes3xbfgragh.onion/2020/04/27/health/coronavirus-symptoms-cdc.html?action=click\&pgtype=Article\&state=default\&region=MAIN_CONTENT_3\&context=storylines_faq}{the
    C.D.C. added to the list of early signs}~sore throat, fever, chills
    and muscle aches. Gastrointestinal upset, such as diarrhea and
    nausea, has also been observed. Another telltale sign of infection
    may be a sudden, profound diminution of one's
    \href{https://www.nytimes3xbfgragh.onion/2020/03/22/health/coronavirus-symptoms-smell-taste.html?action=click\&pgtype=Article\&state=default\&region=MAIN_CONTENT_3\&context=storylines_faq}{sense
    of smell and taste.}~Teenagers and young adults in some cases have
    developed painful red and purple lesions on their fingers and toes
    --- nicknamed ``Covid toe'' --- but few other serious symptoms.
  \end{itemize}
\item ~
  \hypertarget{why-is-it-safer-to-spend-time-together-outside}{%
  \paragraph{Why is it safer to spend time together
  outside?}\label{why-is-it-safer-to-spend-time-together-outside}}

  \begin{itemize}
  \tightlist
  \item
    \href{https://www.nytimes3xbfgragh.onion/2020/05/15/us/coronavirus-what-to-do-outside.html?action=click\&pgtype=Article\&state=default\&region=MAIN_CONTENT_3\&context=storylines_faq}{Outdoor
    gatherings}~lower risk because wind disperses viral droplets, and
    sunlight can kill some of the virus. Open spaces prevent the virus
    from building up in concentrated amounts and being inhaled, which
    can happen when infected people exhale in a confined space for long
    stretches of time, said Dr. Julian W. Tang, a virologist at the
    University of Leicester.
  \end{itemize}
\item ~
  \hypertarget{why-does-standing-six-feet-away-from-others-help}{%
  \paragraph{Why does standing six feet away from others
  help?}\label{why-does-standing-six-feet-away-from-others-help}}

  \begin{itemize}
  \tightlist
  \item
    The coronavirus spreads primarily through droplets from your mouth
    and nose, especially when you cough or sneeze. The C.D.C., one of
    the organizations using that measure,
    \href{https://www.nytimes3xbfgragh.onion/2020/04/14/health/coronavirus-six-feet.html?action=click\&pgtype=Article\&state=default\&region=MAIN_CONTENT_3\&context=storylines_faq}{bases
    its recommendation of six feet}~on the idea that most large droplets
    that people expel when they cough or sneeze will fall to the ground
    within six feet. But six feet has never been a magic number that
    guarantees complete protection. Sneezes, for instance, can launch
    droplets a lot farther than six feet,
    \href{https://jamanetwork.com/journals/jama/fullarticle/2763852}{according
    to a recent study}. It's a rule of thumb: You should be safest
    standing six feet apart outside, especially when it's windy. But
    keep a mask on at all times, even when you think you're far enough
    apart.
  \end{itemize}
\item ~
  \hypertarget{i-have-antibodies-am-i-now-immune}{%
  \paragraph{I have antibodies. Am I now
  immune?}\label{i-have-antibodies-am-i-now-immune}}

  \begin{itemize}
  \tightlist
  \item
    As of right
    now,\href{https://www.nytimes3xbfgragh.onion/2020/07/22/health/covid-antibodies-herd-immunity.html?action=click\&pgtype=Article\&state=default\&region=MAIN_CONTENT_3\&context=storylines_faq}{~that
    seems likely, for at least several months.}~There have been
    frightening accounts of people suffering what seems to be a second
    bout of Covid-19. But experts say these patients may have a
    drawn-out course of infection, with the virus taking a slow toll
    weeks to months after initial exposure.~People infected with the
    coronavirus typically
    \href{https://www.nature.com/articles/s41586-020-2456-9}{produce}~immune
    molecules called antibodies, which are
    \href{https://www.nytimes3xbfgragh.onion/2020/05/07/health/coronavirus-antibody-prevalence.html?action=click\&pgtype=Article\&state=default\&region=MAIN_CONTENT_3\&context=storylines_faq}{protective
    proteins made in response to an
    infection}\href{https://www.nytimes3xbfgragh.onion/2020/05/07/health/coronavirus-antibody-prevalence.html?action=click\&pgtype=Article\&state=default\&region=MAIN_CONTENT_3\&context=storylines_faq}{.
    These antibodies may}~last in the body
    \href{https://www.nature.com/articles/s41591-020-0965-6}{only two to
    three months}, which may seem worrisome, but that's~perfectly normal
    after an acute infection subsides, said Dr. Michael Mina, an
    immunologist at Harvard University. It may be possible to get the
    coronavirus again, but it's highly unlikely that it would be
    possible in a short window of time from initial infection or make
    people sicker the second time.
  \end{itemize}
\item ~
  \hypertarget{what-are-my-rights-if-i-am-worried-about-going-back-to-work}{%
  \paragraph{What are my rights if I am worried about going back to
  work?}\label{what-are-my-rights-if-i-am-worried-about-going-back-to-work}}

  \begin{itemize}
  \tightlist
  \item
    Employers have to provide
    \href{https://www.osha.gov/SLTC/covid-19/standards.html}{a safe
    workplace}~with policies that protect everyone equally.
    \href{https://www.nytimes3xbfgragh.onion/article/coronavirus-money-unemployment.html?action=click\&pgtype=Article\&state=default\&region=MAIN_CONTENT_3\&context=storylines_faq}{And
    if one of your co-workers tests positive for the coronavirus, the
    C.D.C.}~has said that
    \href{https://www.cdc.gov/coronavirus/2019-ncov/community/guidance-business-response.html}{employers
    should tell their employees}~-\/- without giving you the sick
    employee's name -\/- that they may have been exposed to the virus.
  \end{itemize}
\end{itemize}

Health care workers and equipment are coming in from other states. I am
optimistic that for those who have a chance of surviving, we will be
able to do everything for them. Of course, hard choices will still have
to be made --- it's never easy withholding care from a patient --- but I
believe they will be rational decisions that most doctors would agree
on. We are not playing God, as those who made the Siaarti guidelines
were accused of, but we have been doing this long enough to know which
patients will have a possibility of recovery and which ones will
needlessly suffer. Even in Italy, Vergano tells me, his critics have all
backed off.

Still, mental-health professionals, especially those who treat combat
veterans, worry that doctors will sustain moral injury from having to
allocate medical equipment and care. The truth is, when treatment is
rationed or withheld, the decisions are almost always reasonable, and
hopefully the family will be involved. I've already had a few of those
conversations on the phone with family members, guiding them through
what would happen to their loved ones, explaining the extensive medical
procedures involved and the thin likelihood of survival, assuring them
that they should feel no guilt, that I would do the same for my mother.
I can't say with 100 percent certainty that they would not have
survived, but I can say that I didn't prolong their suffering. There is
a bit of solace in that.

I think back again to the elderly man I intubated, when we were still at
the foothills of this pandemic. If I were given a do-over, I would not
do it. I would save the ventilator for a future someone else. I would
override what the family wanted and hope that afterward, they would
understand. I believe it will be fairly obvious that in most of the
cases where we don't push forward with extreme medical interventions, we
would not have been able to save the patients anyway.

What I think will actually cause moral injury is seeing people die after
getting the most advanced care available. People who come in talking,
with stories to share. They get care --- the best that modern medicine
has to offer --- with life-prolonging machines and IV drips of all sorts
of critical-care drugs. We put our full minds and whole hearts into
trying to save them. Then I see their bodies shut down anyway. They are
alone. I'll see that over and over again, and it will reach a point when
it is numbing. What will affect me the most is not remembering them as
individual people, no particular detail that separates a person from the
one before and the one after, because they all come in sick with the
same symptoms, the same history, until they morph together, become
breathless bodies. That I am the last person they see before they die
--- not their families --- and that I won't remember them at all because
there will be hundreds more just like them. That it will become routine.

\begin{center}\rule{0.5\linewidth}{\linethickness}\end{center}

\hypertarget{last-days-of-march}{%
\subsection{Last Days of March}\label{last-days-of-march}}

N.Y.C. Covid-19 cases, March 30: 38,087

``By now, I think it's very hard to stay human,'' says Duca, the E.R.
doctor in Bergamo. ``You go on, you forget you have a person, a human in
front of you. You forget the patient has a life. I think that we do this
to protect ourselves. Otherwise, it would be impossible to work every
day.'' Colombo, his I.C.U. colleague, tells me she goes through an
``emotional shutdown,'' as she calls it, when she arrives at the
hospital. ``The first few days, I was crying when I was home,''
Brambillasca says. ``Then you transform, because you have to do it. You
become tough in a few days.''

The first patient hospitalized in New York is finally discharged, nearly
a month after his diagnosis. Many have died in the meantime, and many
more are uncounted in the ­Covid-19 death toll because they succumbed at
home or weren't tested. Some others wait days in the E.R. for an
inpatient bed, languishing in hallways. Their fates remain unknown. I
pass by them when I first arrive at the E.R. and when I leave at the end
of the day. Sometimes they are still there the next day. If they are
awake, I'm hesitant to make eye contact. I'm too ashamed that after
nearly 15 years as a doctor, I can't do much more for them except put an
oxygen mask over their nose and mouth.

One day I see a grandfatherly man, who speaks softly and smiles sweetly,
come in with oxygen numbers dipping as low as 75 percent. He feels good,
he says, and his breathing is fine. Just a little tired, don't worry, he
says. Despite everything I know so far, I think he will do OK because he
looks so well. The next day, when I return to the E.R., I see he is now
confused. Even wearing an oxygen mask, he could not sustain levels above
90 percent overnight. He had previously decided that he did not want
extraordinary measures taken to save his life; he did not want to be on
a breathing machine. His family, over the phone, is clear about his
wishes, so we make him comfortable with morphine.

I want to spend time with him, but more patients, much younger patients,
keep arriving, struggling to breathe. I have to tend to them instead.
The disease has won against him; the new patients have a chance. I don't
want to think that way, but it is the dismal truth of our new situation.
I hope the morphine is enough to blur the reality that he's all alone. I
move on, forcing myself not to think about him again. Too concerned
about the new patients, I never take the time to check on him again. Too
exhausted at the end of my shift, I don't say goodbye to him either. He
dies later that night.

Before the pandemic, I would typically see a fair number of nonwhite
patients. Yet Hispanic and black patients appear to be arriving at our
E.R. at higher rates now --- and they seem sicker than patients of other
ethnicities. A paramedic points out a similar pattern in what he's
seeing. (Data that comes out later confirms as much:
\href{https://www.nytimes3xbfgragh.onion/2020/04/08/nyregion/coronavirus-race-deaths.html}{Black
and Hispanic patients are dying at twice the rates} of their white and
Asian counterparts.)

I keep hearing about this ``apex,'' that we're still weeks away from it.
I can't bear this word anymore. Apex. When is it coming? How will we
know when we've reached it? What if cases start to slow down, then
increase again? But mostly, I think, how can I think that far ahead,
when I have to coach myself just to get through the next hour?

On the last day of March, I get several texts from Duca and his
colleagues. For the first time, they are seeing some light: The number
of new patients seems to be finally decreasing. Of all the messages I've
received from friends and strangers all over the world, these are the
ones that keep me going. To know that as bad as this is now, it will end
someday. But when?

\begin{center}\rule{0.5\linewidth}{\linethickness}\end{center}

\hypertarget{first-days-of-april}{%
\subsection{First Days of April}\label{first-days-of-april}}

N.Y.C. Covid-19 cases, April 1: 47,440

In the early evening, toward the end of one shift, a woman with
ash-blond hair in her 50s walks into the E.R. She converses with the
nurse about her week of fever and cough, but while an EKG is being done,
she suddenly becomes unresponsive. She loses her pulse. We shock her out
of the irregular, rapid rhythm her heart is in, put a breathing tube
down her throat and start drips of multiple IV medications to stimulate
her heart and constrict her blood vessels. Later that night, I get a
text from a colleague in her 60s, who had walked by during the
resuscitation. ``I have the sense that the world is ending,'' she
writes. ``The person you were coding was six years younger than me.''

The next morning, as I'm getting ready for work, I panic: I might not
have showered last night when I got home from the hospital. I try to
retrace my actions but fail. I simply cannot recall. Did I just fall
asleep? Am I infected? Should I change my sheets, scrub my apartment?
But I have to get to the hospital for my shift. There's nothing I can do
about this now. Better to be lucky.

When I arrive in the E.R., I look up the woman's electronic medical
record from yesterday's shift. I badly want to be able to text back to
my colleague that the patient is doing OK, that we'll all be OK. When I
open her chart, a warning flashes across the computer screen: ``You are
entering the medical record of a deceased patient. Are you sure you want
to proceed?'' She barely made it to daybreak.

A few days ago, palliative-care doctors started helping us with some of
the life-or-death conversations. They call families and talk to them
about procedures that patients might have to undergo if they want to
escalate the interventions; these doctors help figure out where the
limits should be drawn. They also explain how patients could otherwise
be made comfortable, if they don't want to continue with more aggressive
treatments. But the doctors are soon overloaded, unable to tend to all
the consultations. I try to do what I can. An 89-year-old patient is
brought in by ambulance, with an oxygen mask covering most of her small
face. I don't think she'll be able to talk, but she is actually able to
express herself and tell me: ``I don't want a breathing tube. I'm almost
90 years old. I've lived.'' She's originally from North Carolina, she
says. I call her niece, who is her health care proxy. She conferences in
other family members. ``Well, can't we overrule what she wants?'' one of
them asks me.

I'm not formally trained in this, as our palliative-care doctors are,
but I've had many of these discussions over the years. I tell them that
she has clearly expressed what she wants, and I promise to make her
comfortable. Think about what you know of her, I say. What does she
value? Would she want to be hooked up to a machine? How can we stay true
not only to her wishes but also to who she is as a person? They agree
that dying peacefully would be what she would want.

The patient is still awake, interacting with me. I call the patient's
family through Face­Time on my cellphone. Her niece comes on, her smooth
cheeks shiny with tears. She tries an upbeat hello. My patient isn't
fooled. ``Everyone's got to stop crying,'' she says. ``They're taking
good care of me here.'' We all laugh a little through our tears. I order
some morphine for the patient. Her breathing gets easier.

I run around, trying to care for more patients. I'm not sure if anything
I do makes a difference. I can't run away from Brambillasca's words
about the virus: ``It does whatever it wants.''

I wonder if I'm more useful Face­Timing patients' families rather than
applying my skills as a doctor. Three hours later, I pull out my phone
again and call my patient's niece. ``I love you,'' she says to her aunt.
My patient flutters her eyes half-open. ``I love you, too,'' she slowly
replies, her voice noticeably weaker now. I put my hand on her hand. She
grabs my fingers, tells me she feels cared for. She doesn't want to let
go. I don't want to either. I look down at my purple-gloved hand holding
hers, delicate and bony. I hate that she has to feel synthetic rubber,
that she doesn't get an actual human touch before she departs from the
living.

The next morning, a much-needed message comes through from Italy.
``Please, don't give up,'' writes Cornelli, the nurse in Lombardy. ``Our
jobs are difficult but are the most beautiful ones.''

We try putting a few patients prone on their stomachs. I first heard
about this weeks ago, from one of the private Facebook groups devoted to
caring for critically ill Covid patients. It's said to help intubated
patients --- why not give this a try with those who don't have breathing
tubes but aren't oxygenating well? I see a patient's oxygen level shoot
up. It works, I yell out, elatedly, prematurely. Something actually
works! We need massage tables with the cutout face holes for our
patients, I joke to my resident.

A couple days later, I see on Twitter that a Detroit-area oral-surgery
resident has died. His name and photo are in the tweet. ``Christopher
Firlit.'' I say his name out loud; I look at his photo. I want to honor
his death. I want people to know; I don't want doctors to die in
anonymity. Eventually, I put my phone away. Then I think back to my own
resident's question: What would happen if they need to be put on
ventilators?

\begin{center}\rule{0.5\linewidth}{\linethickness}\end{center}

\hypertarget{second-week-of-april}{%
\subsection{Second Week of April}\label{second-week-of-april}}

N.Y.C. Covid-19 cases, April 5: 67,552

``Messaging with you helps,'' I text Brambillasca. ``To hear it will
end.'' (I punctuate using a period, but in my mind it's a giant question
mark.)

``And it will,'' he immediately replies. ``We are seeing it here. So it
will come to NYC as well.''

I read his words three times. This will end, I tell myself. This will
end. I am hardly responding to family and friends anymore. It feels
impossible to explain to them what's going on. I can think of nothing
else, but the last thing I want to do is describe to each person what's
happening in the hospital. I rely on my co-workers --- they grasp
everything I'm feeling with just one glance or a three-word text. Even
doctor friends --- in Philadelphia, Boston, Los Angeles --- seem like
separate species now. No one from another region could understand what
was happening in Lombardy. Can someone from another city understand
what's happening in New York?

My phone vibrates again. Brambillasca just got his first non-Covid
patient in the ``clean'' I.C.U., intended for patients not infected with
coronavirus. The patient's heart had stopped twice during a rescue
abdominal surgery --- a terribly sick person with severe complications
whose outlook remained poor. But Brambillasca was still grateful, still
happy: ``What a soft lung to inflate.''

I happened to have been assigned to work at one hospital for a chain of
shifts, so I hadn't been inside one of our other hospitals in over a
week. As soon as I open the E.R. doors there, I shrink from the sights
and smells. Patients are now triple-bunked into single-person spaces,
curtains pushed aside. In one room, three men, who appear to be in their
80s or so, are side by side in their stretchers, each one pulling at his
oxygen mask, confused, their frail limbs swinging in the air. Some have
sat in their own feces for a day. Puddles of urine have pooled around
the wheels of some patients' stretchers. Nurses are out sick; the
remaining ones are coping the best they can. I have gotten texts from
colleagues about the chaos here, but I thought that those were just
about one bad day, that they had already gone through the worst.

Another doctor notices the bewilderment on my face and comes over.
``There are people literally dying of hypoxia in the hallways,'' he
says, ``and there's empty space with oxygen dispensers on the wall and
no one using them.'' What is he talking about? Isn't the hospital full?
He suggests that I take a walk down the hall and make a right, less than
100 yards away.

I swipe open the unit, which usually serves as a post­operative area,
with my ID. I see a room about half the size of the E.R. It's a Sunday,
a slow day usually, but still, there's only one patient, who's being
tended to by a nursing assistant. A nurse hovers nearby. I track the
green oxygen dispensers on the walls, these fountains of life that my
patients gravely need. I go upstairs to one of the regular floors. It's
calm and quiet. Unlike in the E.R., where I dodge patients, colleagues
and stretchers to get around --- forget six feet of separation; we're
not able to maintain six inches --- here the hallways are free and
unobstructed. It's just a regular hospital floor, but the space feels
glorious, luxurious.

It's the first day of our pulse-ox to-go program. Until this point, I
have been opposed to the idea of sending hypoxic patients home with
pulse oximeters, especially after learning from the Italian doctors that
their oxygen numbers often drop quickly to life-threatening levels ---
sometimes before the patients feel it. These guidelines seem too unsafe
to me. A colleague begs me to rethink this, telling me they will get
better care at home with their family members than here in the E.R., at
least in its current state. ``And I'm saying this as someone who doesn't
believe in these guidelines,'' he adds. After witnessing how many
patients are suffering in the E.R., I immediately discharge two to
self-monitor. I know I'll probably soon hear the dreaded words --- ``You
know that patient you sent home the other day? \ldots{} '' --- but I
have to do what's best for them right now, with what I have in front of
me. I'm hopeful that the field hospital being built at Columbia
University's soccer facility, to be staffed largely by former military
personnel, will open soon with a capacity for nearly 300 patients.

This week, our employee-health services is at last starting to routinely
test medical workers who develop symptoms that could be Covid-related.
Still, I wish we could regularly get swabbed and checked when we know we
have been exposed, even those of us without symptoms, so that we don't
inadvertently pass it on to our patients. Some of us are also eager for
antibody testing, seeking a sense of security if we end up having
antibodies, though it's probably too early to say whether or for how
long that could actually provide immunity.

In the E.R., I run into two co-workers who have recovered from the virus
and are back at work. Our E.R. colleague across town is out of the
I.C.U. I look at a photo of her eating and smiling on Facebook. The next
day, I see on Twitter that James Pruden, a 70-year-old doctor in New
Jersey, is
\href{https://twitter.com/emswami/status/1247966879110631425}{leaving
the hospital after spending nearly a month in the I.C.U.} He was one of
the first doctors hospitalized for coronavirus infection in the United
States. I didn't think he would make it, because of his age and how sick
he seemed. In a video clip, Pruden, in a blue dress shirt, is wheeled
out on a stretcher and points energetically at the surrounding crowd.
I've never met him, but I'm immediately tearful. I replay the recording
four more times. Then I send the tweet to a colleague who works with
him. ``Something going our way for a change,'' he responds. ``If he can
do this, we sure can.''

Later that same day, though, I get a text that several more of our E.R.
staff members are hospitalized, requiring oxygen. I learn that another
died a few days earlier. More co-workers are ill at home with symptoms.
At night, I open an email that a doctor in Brooklyn forwarded to me with
the names of more health care workers in New York City who have died. I
hadn't even heard of their deaths.

Over the next several days, I notice the tone changing during my shifts.
Conversations about dying and death are all around me now, the only kind
I hear. Either I am having one or the physician next to me is. We spend
our days talking to patients and families about the limits of medicine
and what doctors can do; we call people to tell them their loved ones
have passed away. Then we make another call. And another.

Those of us who work in the E.R. are accustomed to pushing our patients'
mortality to the edge. My promise to them has always been that when they
come through those E.R. doors, I will do everything I can to help them
live. This is how we approached every shift. In a way, that job was
easy. Do everything possible, unless the patient or family has
explicitly expressed otherwise.

This is no longer the sole operating principle of emergency medicine in
New York City. It has been less than six weeks, but I've never felt less
useful as a doctor. The one thing I can do --- what I think will matter
most, in the end --- is just to be a person first, for these patients
and their families.

``Staying human is painful, but it is what I need to keep working,''
Duca says. ``I realize now that keeping the emotions outside of me can
help to manage the shift and the stress, but I need to be human to keep
working.'' I know exactly what he means. It's no longer getting through
this day or this week; we are in the deep now, the interminable. For
doctors to survive this pandemic, we have to feel each moment --- even
if it makes each moment more difficult to endure.

\begin{center}\rule{0.5\linewidth}{\linethickness}\end{center}

\textbf{Helen Ouyang} is a physician, a writer and an assistant
professor at Columbia University. She last wrote for the magazine about
hospice homes for children. \textbf{Philip Montgomery} is a photographer
whose current work chronicles the fractured state of America. He won the
2018 National Magazine Award for feature photography on Ohio's opioid
epidemic.

Advertisement

\protect\hyperlink{after-bottom}{Continue reading the main story}

\hypertarget{site-index}{%
\subsection{Site Index}\label{site-index}}

\hypertarget{site-information-navigation}{%
\subsection{Site Information
Navigation}\label{site-information-navigation}}

\begin{itemize}
\tightlist
\item
  \href{https://help.nytimes3xbfgragh.onion/hc/en-us/articles/115014792127-Copyright-notice}{©~2020~The
  New York Times Company}
\end{itemize}

\begin{itemize}
\tightlist
\item
  \href{https://www.nytco.com/}{NYTCo}
\item
  \href{https://help.nytimes3xbfgragh.onion/hc/en-us/articles/115015385887-Contact-Us}{Contact
  Us}
\item
  \href{https://www.nytco.com/careers/}{Work with us}
\item
  \href{https://nytmediakit.com/}{Advertise}
\item
  \href{http://www.tbrandstudio.com/}{T Brand Studio}
\item
  \href{https://www.nytimes3xbfgragh.onion/privacy/cookie-policy\#how-do-i-manage-trackers}{Your
  Ad Choices}
\item
  \href{https://www.nytimes3xbfgragh.onion/privacy}{Privacy}
\item
  \href{https://help.nytimes3xbfgragh.onion/hc/en-us/articles/115014893428-Terms-of-service}{Terms
  of Service}
\item
  \href{https://help.nytimes3xbfgragh.onion/hc/en-us/articles/115014893968-Terms-of-sale}{Terms
  of Sale}
\item
  \href{https://spiderbites.nytimes3xbfgragh.onion}{Site Map}
\item
  \href{https://help.nytimes3xbfgragh.onion/hc/en-us}{Help}
\item
  \href{https://www.nytimes3xbfgragh.onion/subscription?campaignId=37WXW}{Subscriptions}
\end{itemize}
