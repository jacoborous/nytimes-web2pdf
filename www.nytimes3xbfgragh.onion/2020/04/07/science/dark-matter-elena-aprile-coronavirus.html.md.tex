\href{/section/science}{Science}\textbar{}Will Coronavirus Freeze the
Search for Dark Matter?

\url{https://nyti.ms/2xSwsis}

\begin{itemize}
\item
\item
\item
\item
\item
\item
\end{itemize}

\hypertarget{the-coronavirus-outbreak}{%
\subsubsection{\texorpdfstring{\href{https://www.nytimes3xbfgragh.onion/news-event/coronavirus?name=styln-coronavirus-national\&region=TOP_BANNER\&block=storyline_menu_recirc\&action=click\&pgtype=Article\&impression_id=a315c0f0-f2ac-11ea-987f-e1331429ff8f\&variant=undefined}{The
Coronavirus
Outbreak}}{The Coronavirus Outbreak}}\label{the-coronavirus-outbreak}}

\begin{itemize}
\tightlist
\item
  live\href{https://www.nytimes3xbfgragh.onion/2020/09/09/world/covid-19-coronavirus.html?name=styln-coronavirus-national\&region=TOP_BANNER\&block=storyline_menu_recirc\&action=click\&pgtype=Article\&impression_id=a315c0f1-f2ac-11ea-987f-e1331429ff8f\&variant=undefined}{Latest
  Updates}
\item
  \href{https://www.nytimes3xbfgragh.onion/interactive/2020/us/coronavirus-us-cases.html?name=styln-coronavirus-national\&region=TOP_BANNER\&block=storyline_menu_recirc\&action=click\&pgtype=Article\&impression_id=a315c0f2-f2ac-11ea-987f-e1331429ff8f\&variant=undefined}{Maps
  and Cases}
\item
  \href{https://www.nytimes3xbfgragh.onion/interactive/2020/science/coronavirus-vaccine-tracker.html?name=styln-coronavirus-national\&region=TOP_BANNER\&block=storyline_menu_recirc\&action=click\&pgtype=Article\&impression_id=a315c0f3-f2ac-11ea-987f-e1331429ff8f\&variant=undefined}{Vaccine
  Tracker}
\item
  \href{https://www.nytimes3xbfgragh.onion/2020/09/02/your-money/eviction-moratorium-covid.html?name=styln-coronavirus-national\&region=TOP_BANNER\&block=storyline_menu_recirc\&action=click\&pgtype=Article\&impression_id=a315e800-f2ac-11ea-987f-e1331429ff8f\&variant=undefined}{Eviction
  Moratorium}
\item
  \href{https://www.nytimes3xbfgragh.onion/2020/09/09/upshot/coronavirus-surprise-test-fees.html?name=styln-coronavirus-national\&region=TOP_BANNER\&block=storyline_menu_recirc\&action=click\&pgtype=Article\&impression_id=a315e801-f2ac-11ea-987f-e1331429ff8f\&variant=undefined}{Surprise
  Test Fees}
\end{itemize}

\includegraphics{https://static01.graylady3jvrrxbe.onion/images/2020/04/07/science/06SCI-DARKMATTERCOVERSUB1/06SCI-DARKMATTER1-articleLarge.jpg?quality=75\&auto=webp\&disable=upscale}

Sections

\protect\hyperlink{site-content}{Skip to
content}\protect\hyperlink{site-index}{Skip to site index}

\hypertarget{will-coronavirus-freeze-the-search-for-dark-matter}{%
\section{Will Coronavirus Freeze the Search for Dark
Matter?}\label{will-coronavirus-freeze-the-search-for-dark-matter}}

An experiment under 4,600 feet of Italian rock wasn't immune from the
pandemic's interruption.

Elena Aprile, a dark matter scientist and physics professor at Columbia
University, was supposed to return to Italy but has been stranded in
Brooklyn.Credit...Gabby Jones for The New York Times

Supported by

\protect\hyperlink{after-sponsor}{Continue reading the main story}

\href{https://www.nytimes3xbfgragh.onion/by/dennis-overbye}{\includegraphics{https://static01.graylady3jvrrxbe.onion/images/2018/07/30/multimedia/author-dennis-overbye/author-dennis-overbye-thumbLarge.png}}

By \href{https://www.nytimes3xbfgragh.onion/by/dennis-overbye}{Dennis
Overbye}

\begin{itemize}
\item
  April 7, 2020
\item
  \begin{itemize}
  \item
  \item
  \item
  \item
  \item
  \item
  \end{itemize}
\end{itemize}

Elena Aprile was in a race against time.

Her Xenon experiment, one of the world's largest and most expensive
investigations into the nature of dark matter, was coming together
beneath Gran Sasso, a mountain in Italy. But Dr. Aprile, a Columbia
University physics professor, was stuck in her apartment in Brooklyn as
New York entered an indeterminate period of lockdown to contain the
spread of the new coronavirus, and she was ``living on Cheerios and
milk,'' she said.

In Italy, about a month into its own lockdown, a skeleton crew was
trying to finish assembling her experiment's expensive and delicate
detector and safely seal it in place deep below the mountain's rocks,
before the virus brought down the hammer on even this much group
activity.

What followed was an illustration of how some science is managing to get
done during a plague. At stake was perhaps nothing less than the secret
of the universe.

\hypertarget{the-dark-side}{%
\subsection{The dark side}\label{the-dark-side}}

\includegraphics{https://static01.graylady3jvrrxbe.onion/images/2020/04/07/science/06SCI-DARKMATTER2/merlin_171253929_9a847e19-a19b-401d-bf1b-bfe31e4a583c-articleLarge.jpg?quality=75\&auto=webp\&disable=upscale}

Astronomers have reluctantly concluded over the last half-century that
most of the matter in the universe is invisible. They suspect that this
invisible stuff consists of giant cosmic clouds of subatomic particles
called ``wimps,'' for weakly interacting massive particles, left over
from the Big Bang.

Mostly impervious to normal forces like electromagnetism, these
particles drift through the world, and through us, like ghosts through a
wall.

In the quest to spot them, physicists have built a succession of bigger
and bigger detectors. But as they've gained greater and greater clarity,
they have seen no wimps, which has created a crisis in physics.

In the 1970s and 1980s, fashionable but speculative concepts in particle
physics were devised to explain some of the deeper mysteries of
fundamental physics. One, supersymmetry, suggested that the universe
might be littered with undiscovered particles that could act like dark
matter. But over the years, the most promising models of what these
particles might have been were slowly crossed out. This leaves many of
the mysteries of the universe --- like why stars are so big and atoms
are so small --- with no plausible explanation.

The wimp experiments keep improving. But eventually they could reach a
limit called the ``neutrino floor,'' becoming so sensitive that they are
overwhelmed by neutrinos, ghostly super-elusive particles that flood the
universe from the sun, the stars and the Big Bang. Any wimps passing
through will be impossible to discern in this sea, and there the wimp
search will end.

``So we have a few more years where this guy can hide, but it's not
there yet,'' she said.

Dr. Aprile and her team --- a globe-spanning confederation --- planned
to record the pit-pat of dark matter particles raining into a tank of
liquid xenon lined with 500 photomultipliers and other sensors, and
placed far underground to shield it from cosmic rays. The hope was that
her team's device would spot the rare collision of a wimp with a xenon
nucleus, an event she estimated might happen about once a year per ton
of xenon.

Image

The Gran Sasso National Laboratory lies off a tunnel through a mountain
beneath nearly 4,600 feet of rock. It is among the largest underground
research centers in the worldCredit...Stefano Montesi/Corbis, via Getty
Images

Image

A corridor in the Gran Sasso Laboratory in 2018.Credit...Stefano
Montesi/Corbis, via Getty Images

Dr. Aprile was reluctant to put a price on the project. An earlier
version of the experiment with 3.3 tons of xenon cost \$30 million. But
that didn't include the people, she said. A big part of the cost is
xenon itself, which costs around \$2 million per ton, she added. Her new
detector will have 8.5 tons.

\hypertarget{latest-updates-the-coronavirus-outbreak}{%
\section{\texorpdfstring{\href{https://www.nytimes3xbfgragh.onion/2020/09/09/world/covid-19-coronavirus.html?action=click\&pgtype=Article\&state=default\&region=MAIN_CONTENT_1\&context=storylines_live_updates}{Latest
Updates: The Coronavirus
Outbreak}}{Latest Updates: The Coronavirus Outbreak}}\label{latest-updates-the-coronavirus-outbreak}}

Updated 2020-09-09T14:50:13.661Z

\begin{itemize}
\tightlist
\item
  \href{https://www.nytimes3xbfgragh.onion/2020/09/09/world/covid-19-coronavirus.html?action=click\&pgtype=Article\&state=default\&region=MAIN_CONTENT_1\&context=storylines_live_updates\#link-5b0bf0d1}{As
  drugmakers pledge to thoroughly vet vaccines, one company pauses its
  trials for a safety review.}
\item
  \href{https://www.nytimes3xbfgragh.onion/2020/09/09/world/covid-19-coronavirus.html?action=click\&pgtype=Article\&state=default\&region=MAIN_CONTENT_1\&context=storylines_live_updates\#link-6e2052bd}{The
  director of the N.I.H. and the surgeon general answer senators'
  questions.}
\item
  \href{https://www.nytimes3xbfgragh.onion/2020/09/09/world/covid-19-coronavirus.html?action=click\&pgtype=Article\&state=default\&region=MAIN_CONTENT_1\&context=storylines_live_updates\#link-780eaa2f}{Britain
  is expected to ban gatherings of more than six people.}
\end{itemize}

\href{https://www.nytimes3xbfgragh.onion/2020/09/09/world/covid-19-coronavirus.html?action=click\&pgtype=Article\&state=default\&region=MAIN_CONTENT_1\&context=storylines_live_updates}{See
more updates}

More live coverage:
\href{https://www.nytimes3xbfgragh.onion/live/2020/09/09/business/stock-market-today-coronavirus?action=click\&pgtype=Article\&state=default\&region=MAIN_CONTENT_1\&context=storylines_live_updates}{Markets}

A rival experiment called the \href{https://lz.lbl.gov/}{LZ Dark Matter
Experiment}, also using eight tons of xenon, was being assembled in an
old gold mine that is now the Sanford Underground Research Facility, in
Lead, S.D. And there is a whole alphabet soup of other experiments
stashed in old mines and tunnels around the world, with names like
PandaX, DarkSide and SuperCDMS.

But now coronavirus was infecting even the cosmos. Richard Gaitskell of
Brown University, one of the principal scientists of the LZ experiment,
said in an email that their project had temporarily been mothballed
``out of an abundance of caution and to allow personnel to respect
shelter in place.''

Dr. Aprile said, ``All of us will have delays due to this damn thing. If
one of my people gets sick, I will feel so bad.''

\hypertarget{research-on-the-run}{%
\subsection{Research on the run}\label{research-on-the-run}}

Dr. Aprile was born in Milan. To say that she lives a peripatetic life
would be an understatement. She teaches at Columbia but commutes
regularly to L'Aquila, a town in central Italy near the Gran Sasso
National Laboratory, which lies off a tunnel through the mountain of the
same name, beneath nearly 4,600 feet of rock.

Until March she had been living the typical jet-setting life of particle
physicist. In November she attended a physics conference in South Korea.
In February, after a brief stop in New York, she was in Italy at Gran
Sasso for three days. From there she went to a conference in South
Africa, and on to the University of California, San Diego, where she was
a visiting professor.

Then the universities shut down. Worried about her two daughters, who
live in New York, Dr. Aprile returned home. She had planned to return to
Gran Sasso in early May after her professorship was done, when they
would start testing and running their detector. But the virus had other
plans.

Stefano Ragazzi, director of the Gran Sasso lab, said that the
experiments there are designed to be conducted remotely. As a result,
there were only about half a dozen scientists on site in March when the
coronavirus hit Italy.

It is safer and easier to keep experiments running, rather than shut
them off and later switch them back on, he explained, so the lab's
experiments have continued to operate as they would during the winter
holidays.

Dr. Ragazzi announced that, to ensure the safety of the people and the
equipment, work in Gran Sasso would be limited only to what was
necessary.

``Xenon was amid critical ongoing operations,'' Dr. Ragazzi said in an
email. ``We asked them to come to a safe stopping point and to pause
operations.''

Image

Left, the water tank of the xenon setup, adorned with a poster showing
what is inside, and at right, the three-story service building, in
2016.Credit...Andrea Sabbadini/Alamy

Image

The assembled time projection chamber.Credit...Xenon Dark Matter Project

Image

The cryostat hanging within the water tank.Credit...Xenon Dark Matter
Project

That stopping point would come once the detector had been sealed in its
cryostat --- a big thermos bottle that could keep the xenon inside at
minus 150 degrees Fahrenheit --- and all the air had been pumped out,
Dr. Aprile said: ``The point is to enclose it in a cryostat, seal it,
make it leak-tight.'' She spoke over the phone after a long day of
teleconferencing with Italy.

``We close this detector for the first time inside this big water
tank,'' she said. ``Then we spend a few months, if everything goes well,
commissioning it to understand how the hell it works. Hopefully it works
as you designed. You start to see if there's a signal. And that's when
you declare OK, and then you start to work.''

All did not go well.

An important step occurred on March 5, when a team led by Luca Grandi of
the University of Chicago installed the detector underground. It had
arrived in pieces at Gran Sasso from all over the world, ``like the
pieces of a puzzle,'' Dr. Aprile said, and had to be assembled in a
``clean room'' in a part of the Gran Sasso lab that was aboveground.

The finished detector, known as a time projection chamber, is about five
feet long and five feet wide, and weighs half a ton without the xenon in
it. The team had to rent a special truck and get a police escort to move
it to the underground part of the lab, which is accessible through a
highway tunnel under the mountain.

``We didn't realize it would be so hard to handle,'' Dr. Aprile said.

There the detector was installed under the dome of the cryostat. But the
cryostat was not ready to be closed. ``We were almost done, but now we
needed special permissions,'' Dr. Aprile said.

\hypertarget{whos-in-charge-there}{%
\subsection{Who's in charge there?}\label{whos-in-charge-there}}

Image

The top of the photomultiplier tubes array and reflector panels, seen
from below the field cage.Credit...Xenon Dark Matter Project

Image

``I fear, what happens if the team gets infected or gets hurt,'' Dr.
Aprile said. ``The lab gets the blame.''Credit...Gabby Jones for The New
York Times

Failure to finish installing the detector would leave the tank open to
the air, which would increase the chance of contamination by radon, a
radioactive gas found in underground spaces and the main source of
contamination in experiments like this one.

A minimum of three or four people were needed to handle these final
steps. Dr. Aprile had a half-dozen scientists and technicians at the
site, so the margin was getting thin. But Dr. Grandi had to leave to
teach in Chicago.

Dr. Aprile promoted Petr Chaguine, a scientist from Rice University who
had been living in Gran Sasso, to direct the team. He reported back to
his friends and family in Houston that his Italian colleagues were
``kindly translating news and new government regulations'' as they
appeared, which was often.

For a while, the team members approved by Dr. Ragazzi could car-pool
from their homes to the lab. Then the rules changed and they had to
drive separately.

Another rule required a Glimos --- Group Leader in Matter of Safety ---
to visit every day to make sure everything was in order. Roberto
Corrieri was doing the job, then announced that he would follow
governmental instructions and stay home in Assergi; then he changed his
mind and stayed. The only other person who could have done the safety
inspection had left to join his family in Naples.

``I did not want to push the boundary if he felt he wanted to stay
home,'' Dr. Aprile said of her conversations with Mr. Corrieri.
``Luckily he is a good guy and realized that doing it was important for
many people, so he agreed to do it.''

She added, ``I fear, what happens if the team gets infected or gets
hurt. The lab gets the blame."

Image

Masatoshi Kobayashi and Danilo Tatananni with the closed-up detector.
``We did it,'' they wrote Dr. Aprile.Credit...Masatoshi Kobayashi

That left enough people in the lab to continue working. ``I had to do a
lot of encouraging,'' Dr. Aprile said. It helped that they knew each
other, and that there were no strangers on the team: ``So they were
comfortable being close enough to work.''

On March 20, Dr. Aprile received a photo by email of a pair of her
scientists, Masatoshi Kobayashi and Danilo Tatananni. They were garbed
much like E.R. doctors, in ``bunny suits'' and masks, which are standard
apparel for the clean rooms where sensitive scientific gadgets are
assembled. The men were standing in front of her detector, which they
had just closed up.

``We did it,'' the email said.

The physicists will now spend two weeks pumping air from the vat, down
to a vacuum, at which point it can be monitored remotely. The task of
filling the vat with liquid xenon must wait.

``We cannot test drive our new car,'' Dr. Aprile said. She was happy and
relieved to no longer have to reluctantly urge her colleagues to enter a
field of danger.

``They feel like heroes,'' she said. ``Was it worth it? I'm wondering
myself.''

\textbf{\emph{{[}}\href{http://on.fb.me/1paTQ1h}{\emph{Like the Science
Times page on Facebook.}}} ****** \emph{\textbar{} Sign up for the}
\textbf{\href{http://nyti.ms/1MbHaRU}{\emph{Science Times
newsletter.}}\emph{{]}}}

Advertisement

\protect\hyperlink{after-bottom}{Continue reading the main story}

\hypertarget{site-index}{%
\subsection{Site Index}\label{site-index}}

\hypertarget{site-information-navigation}{%
\subsection{Site Information
Navigation}\label{site-information-navigation}}

\begin{itemize}
\tightlist
\item
  \href{https://help.nytimes3xbfgragh.onion/hc/en-us/articles/115014792127-Copyright-notice}{©~2020~The
  New York Times Company}
\end{itemize}

\begin{itemize}
\tightlist
\item
  \href{https://www.nytco.com/}{NYTCo}
\item
  \href{https://help.nytimes3xbfgragh.onion/hc/en-us/articles/115015385887-Contact-Us}{Contact
  Us}
\item
  \href{https://www.nytco.com/careers/}{Work with us}
\item
  \href{https://nytmediakit.com/}{Advertise}
\item
  \href{http://www.tbrandstudio.com/}{T Brand Studio}
\item
  \href{https://www.nytimes3xbfgragh.onion/privacy/cookie-policy\#how-do-i-manage-trackers}{Your
  Ad Choices}
\item
  \href{https://www.nytimes3xbfgragh.onion/privacy}{Privacy}
\item
  \href{https://help.nytimes3xbfgragh.onion/hc/en-us/articles/115014893428-Terms-of-service}{Terms
  of Service}
\item
  \href{https://help.nytimes3xbfgragh.onion/hc/en-us/articles/115014893968-Terms-of-sale}{Terms
  of Sale}
\item
  \href{https://spiderbites.nytimes3xbfgragh.onion}{Site Map}
\item
  \href{https://help.nytimes3xbfgragh.onion/hc/en-us}{Help}
\item
  \href{https://www.nytimes3xbfgragh.onion/subscription?campaignId=37WXW}{Subscriptions}
\end{itemize}
