\href{/section/business}{Business}\textbar{}`I Just Need the Comfort':
Processed Foods Make a Pandemic Comeback

\url{https://nyti.ms/34laelp}

\begin{itemize}
\item
\item
\item
\item
\item
\item
\end{itemize}

\hypertarget{the-coronavirus-outbreak}{%
\subsubsection{\texorpdfstring{\href{https://www.nytimes3xbfgragh.onion/news-event/coronavirus?name=styln-coronavirus-markets\&region=TOP_BANNER\&block=storyline_menu_recirc\&action=click\&pgtype=Article\&impression_id=014d9c20-f1cb-11ea-a821-81f048d561cf\&variant=undefined}{The
Coronavirus
Outbreak}}{The Coronavirus Outbreak}}\label{the-coronavirus-outbreak}}

\begin{itemize}
\tightlist
\item
  live\href{https://www.nytimes3xbfgragh.onion/2020/09/08/world/covid-19-coronavirus.html?name=styln-coronavirus-markets\&region=TOP_BANNER\&block=storyline_menu_recirc\&action=click\&pgtype=Article\&impression_id=014d9c21-f1cb-11ea-a821-81f048d561cf\&variant=undefined}{Latest
  Updates}
\item
  \href{https://www.nytimes3xbfgragh.onion/interactive/2020/us/coronavirus-us-cases.html?name=styln-coronavirus-markets\&region=TOP_BANNER\&block=storyline_menu_recirc\&action=click\&pgtype=Article\&impression_id=014d9c22-f1cb-11ea-a821-81f048d561cf\&variant=undefined}{Maps
  and Cases}
\item
  \href{https://www.nytimes3xbfgragh.onion/interactive/2020/science/coronavirus-vaccine-tracker.html?name=styln-coronavirus-markets\&region=TOP_BANNER\&block=storyline_menu_recirc\&action=click\&pgtype=Article\&impression_id=014d9c23-f1cb-11ea-a821-81f048d561cf\&variant=undefined}{Vaccine
  Tracker}
\item
  \href{https://www.nytimes3xbfgragh.onion/2020/09/02/your-money/eviction-moratorium-covid.html?name=styln-coronavirus-markets\&region=TOP_BANNER\&block=storyline_menu_recirc\&action=click\&pgtype=Article\&impression_id=014d9c24-f1cb-11ea-a821-81f048d561cf\&variant=undefined}{Eviction
  Moratorium}
\item
  \href{https://www.nytimes3xbfgragh.onion/interactive/2020/09/02/magazine/food-insecurity-hunger-us.html?name=styln-coronavirus-markets\&region=TOP_BANNER\&block=storyline_menu_recirc\&action=click\&pgtype=Article\&impression_id=014d9c25-f1cb-11ea-a821-81f048d561cf\&variant=undefined}{American
  Hunger}
\end{itemize}

\includegraphics{https://static01.graylady3jvrrxbe.onion/images/2020/04/08/business/08virus-badfood-1/merlin_171304212_fbd5c880-40ed-466f-a0d4-4c8028752684-articleLarge.jpg?quality=75\&auto=webp\&disable=upscale}

Sections

\protect\hyperlink{site-content}{Skip to
content}\protect\hyperlink{site-index}{Skip to site index}

\hypertarget{i-just-need-the-comfort-processed-foods-make-a-pandemic-comeback}{%
\section{`I Just Need the Comfort': Processed Foods Make a Pandemic
Comeback}\label{i-just-need-the-comfort-processed-foods-make-a-pandemic-comeback}}

Shoppers, moved by nostalgia and hunting for longer shelf lives, are
returning to old standbys like Chef Boyardee and Campbell's soup.

Credit...Daniel Dorsa for The New York Times

Supported by

\protect\hyperlink{after-sponsor}{Continue reading the main story}

By \href{https://www.nytimes3xbfgragh.onion/by/julie-creswell}{Julie
Creswell}

\begin{itemize}
\item
  April 7, 2020
\item
  \begin{itemize}
  \item
  \item
  \item
  \item
  \item
  \item
  \end{itemize}
\end{itemize}

Just a few months ago, Sue Smith considered herself a healthy eater. She
ate salads with kale and quinoa. She counted calories. She eliminated
processed sugar from her diet. She avoided dairy products.

But in the past month, as the coronavirus pandemic made her housebound,
Ms. Smith, a writer in Los Angeles, began shopping --- and eating ---
completely differently.

During a trip to the grocery store, she bought SpaghettiOs. She threw
two large boxes of Goldfish crackers into her shopping cart. And she
went all in on dairy.

``I'm eating ice cream. Ice cream bars,'' Ms. Smith said. ``And tonight,
I'm making a spinach-artichoke lasagna. There's so much dairy in it. But
I just need the comfort that I get from that food right now.''

As the coronavirus shutdowns continue across the United States, two
growing trends involving how people eat --- the rising amount of money
spent on meals outside the home and the increased purchase of fresh or
organic foods in grocery stores --- have been reversed. Many restaurants
have closed, and shoppers are reaching for frozen pizza and boxes of
cereal instead of organic greens and whole grains.

That's good news for big food companies like Kraft Heinz and J.M.
Smucker, which have struggled in recent years to adapt as Americans
shied away in great numbers from highly processed foods. Now, in a
moment of crisis, shoppers are turning to old standbys that they may not
have had in years or even decades.

\includegraphics{https://static01.graylady3jvrrxbe.onion/images/2020/04/09/business/09virus-badfood-2/08virus-badfood-2-articleLarge.jpg?quality=75\&auto=webp\&disable=upscale}

Many large food businesses like the Campbell Soup Company, which had
seen steady declines in soup sales the last two years, are now
\href{https://www.nasdaq.com/articles/coronavirus-campbell-soup-ramping-up-production-2020-03-05}{ramping
up} production and temporarily increasing wages for hourly employees to
meet the higher demand. In the last month, sales of Campbell's soup
soared 59 percent from a year earlier. Prego pasta sauce increased 52
percent, and sales of its Pepperidge Farm Goldfish crackers climbed
nearly 23 percent.

Similarly,
\href{https://www.nytimes3xbfgragh.onion/2019/02/21/business/kraft-heinz-earnings.html?searchResultPosition=3}{Kraft
Heinz,} whose products had fallen far out of favor with consumers,
resulting in massive write-downs in the values of its Kraft natural
cheese and Oscar Mayer cold cuts businesses a year ago, told investors
last week that some of its factories were working three shifts to meet
high demand for products like its macaroni and cheese. The company's
stock rose on Tuesday after it said first-quarter sales would be up 3
percent.

And Conagra Brands, which had reported a decline of more than 5 percent
in net sales for the quarter ending Feb. 23, said its shipments to
retailers and in-store sales in March had grown 50 percent as demand
increased for Slim Jim jerky snacks, Birds Eye frozen vegetables and
Chef Boyardee pastas.

Image

Conagra, which makes Chef Boyardee pastas, said it had seen a 50 percent
leap in demand.Credit...Daniel Dorsa for The New York Times

``We stocked up on the entire Chef Boyardee line. Chef Boyardee Ravioli.
Chef Boyardee Beefaroni,'' Ms. Smith said. ``I hadn't had that stuff in
20 years.''

Much of the early panic buying that cleared out stores of rice, cans of
tuna and soup, and
\href{https://www.nytimes3xbfgragh.onion/2020/03/22/business/coronavirus-beans-sales.html}{beans}
was rooted in a combination of fear and practicality. Shoppers,
uncertain of when they would be able to return to grocery stores and
whether they would find any food restocked, bought foods that could sit
on their shelves for months.

These simple and easy-to-make meals also fill the bill for people trying
to squeeze a fast lunch in between Zoom meetings for work or for parents
feeding their newly home-schooled children.

For the large food companies, a big question is whether the robust sales
will disappear once the shutdown ends. Some of that depends on how
quickly the economy rebounds, said Robert Moskow, an analyst at Credit
Suisse.

``We counted three economic recessions in the past 30 years, and in each
of them the data show that consumers shifted more toward at-home food
consumption to save money, away from the structural trend of eating away
from the home,'' Mr. Moskow said. ``I would expect food-at-home
consumption to increase, and not just for the next two months but for
the next 12 months.''

Image

``I'm eating ice cream. Ice cream bars,'' said Sue Smith, a writer in
Los Angeles.Credit...Rozette Rago for The New York Times

Others say the quarantine food shopping provides an opportunity for food
companies to convert first-time shoppers into longtime buyers with
packaged, refrigerated or frozen foods that they say are healthier and
tastier than they were a few years ago.

``We're seeing frozen dinners and entrees that are on trend with simple
ingredients and global cuisines,'' said David Portalatin, the national
analyst for food and beverage consumption at the NPD Group, a research
firm. ``The food companies have responded to the contemporary food
values over the last few years.''

Executives at General Mills said they had worked diligently to improve
the nutrition level and taste of many products. ``Right now, we have
people trying the products they haven't had for a while, and we hope
they're surprised and find that they're delicious and that we have them
come back,'' said Jon Nudi, who leads the company's North American
retail operations.

General Mills has seen across-the-board increases in its various product
lines in the last four weeks, from Yoplait yogurt to Cheerios cereal to
Progresso Soup and even baking products like Gold Medal flour and
Bisquick as consumers confined to their homes fill the endless hours by
trying new recipes or even baking
\href{https://www.nytimes3xbfgragh.onion/2020/03/30/style/bread-baking-coronavirus.html}{bread.}

``We've seen all of our categories go up, including dry packaged dinner
mixes like Hamburger Helper,'' Mr. Nudi said. ``There were a lot of
people who thought its best days were way behind it. But, mixed with
hamburger or tuna, it's a simple and delicious meal.''

For many people, some of the strict rules they had around food before
the quarantines are now being eased.

``We don't normally have chips at home. But now we have Doritos and
Cheetos. Chips made with orange stuff and all sorts of seasonings that
we normally don't eat,'' said Connie Huynh, an organizer with the
grass-roots activist network People's Action in Pasadena, Calif. ``We
are relaxing some of the rules during this stressful time just to get
through it.''

Image

In the midst of the pandemic, shoppers are reaching for processed,
shelf-stable foods instead of organic greens and whole
grains.Credit...Daniel Dorsa for The New York Times

For others, the food purchases are purely an emotional reaction.
Consumers are reaching for foods that trigger a comforting childhood
memory or are simply their go-to snack when they need to relieve stress.

``One of the first things I grabbed was Kraft Easy Cheese. The
disgusting orange stuff in a can. But it was one of the foods I ate
growing up, so it's a nostalgia thing,'' said Hana Thompson, who works
for a software start-up in Denver. ``I also have a bag of Flamin' Hot
Cheetos that I haven't opened. How long can I last and not eat those?
It's a low-entertainment game that I've been playing.''

Two days later, the Cheetos won.

Advertisement

\protect\hyperlink{after-bottom}{Continue reading the main story}

\hypertarget{site-index}{%
\subsection{Site Index}\label{site-index}}

\hypertarget{site-information-navigation}{%
\subsection{Site Information
Navigation}\label{site-information-navigation}}

\begin{itemize}
\tightlist
\item
  \href{https://help.nytimes3xbfgragh.onion/hc/en-us/articles/115014792127-Copyright-notice}{©~2020~The
  New York Times Company}
\end{itemize}

\begin{itemize}
\tightlist
\item
  \href{https://www.nytco.com/}{NYTCo}
\item
  \href{https://help.nytimes3xbfgragh.onion/hc/en-us/articles/115015385887-Contact-Us}{Contact
  Us}
\item
  \href{https://www.nytco.com/careers/}{Work with us}
\item
  \href{https://nytmediakit.com/}{Advertise}
\item
  \href{http://www.tbrandstudio.com/}{T Brand Studio}
\item
  \href{https://www.nytimes3xbfgragh.onion/privacy/cookie-policy\#how-do-i-manage-trackers}{Your
  Ad Choices}
\item
  \href{https://www.nytimes3xbfgragh.onion/privacy}{Privacy}
\item
  \href{https://help.nytimes3xbfgragh.onion/hc/en-us/articles/115014893428-Terms-of-service}{Terms
  of Service}
\item
  \href{https://help.nytimes3xbfgragh.onion/hc/en-us/articles/115014893968-Terms-of-sale}{Terms
  of Sale}
\item
  \href{https://spiderbites.nytimes3xbfgragh.onion}{Site Map}
\item
  \href{https://help.nytimes3xbfgragh.onion/hc/en-us}{Help}
\item
  \href{https://www.nytimes3xbfgragh.onion/subscription?campaignId=37WXW}{Subscriptions}
\end{itemize}
