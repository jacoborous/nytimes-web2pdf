Sections

SEARCH

\protect\hyperlink{site-content}{Skip to
content}\protect\hyperlink{site-index}{Skip to site index}

\href{https://www.nytimes3xbfgragh.onion/section/politics}{Politics}

\href{https://myaccount.nytimes3xbfgragh.onion/auth/login?response_type=cookie\&client_id=vi}{}

\href{https://www.nytimes3xbfgragh.onion/section/todayspaper}{Today's
Paper}

\href{/section/politics}{Politics}\textbar{}Wisconsin Primary Recap:
Voters Forced to Choose Between Their Health and Their Civic Duty

\url{https://nyti.ms/2xVDP8Z}

\begin{itemize}
\item
\item
\item
\item
\item
\item
\end{itemize}

\begin{itemize}
\item
  \href{https://www.nytimes3xbfgragh.onion/live/2020/09/08/us/trump-vs-biden?action=click\&pgtype=Article\&state=default\&region=TOP_BANNER\&context=storylines_menu}{Election
  Updates}
\item
  \href{https://www.nytimes3xbfgragh.onion/interactive/2020/us/elections/election-states-biden-trump.html?action=click\&pgtype=Article\&state=default\&region=TOP_BANNER\&context=storylines_menu}{Paths
  to 270}
\item
  \href{https://www.nytimes3xbfgragh.onion/interactive/2020/08/31/us/politics/vote-by-mail-deadlines.html?action=click\&pgtype=Article\&state=default\&region=TOP_BANNER\&context=storylines_menu}{Voting
  by Mail}
\item
  \href{https://www.nytimes3xbfgragh.onion/interactive/2019/us/elections/2020-presidential-election-calendar.html?action=click\&pgtype=Article\&state=default\&region=TOP_BANNER\&context=storylines_menu}{Key
  Dates}
\item
  \href{https://www.nytimes3xbfgragh.onion/newsletters/politics?action=click\&pgtype=Article\&state=default\&region=TOP_BANNER\&context=storylines_menu}{Politics
  Newsletter}
\end{itemize}

Advertisement

\protect\hyperlink{after-top}{Continue reading the main story}

Supported by

\protect\hyperlink{after-sponsor}{Continue reading the main story}

\hypertarget{wisconsin-primary-recap-voters-forced-to-choose-between-their-health-and-their-civic-duty}{%
\section{Wisconsin Primary Recap: Voters Forced to Choose Between Their
Health and Their Civic
Duty}\label{wisconsin-primary-recap-voters-forced-to-choose-between-their-health-and-their-civic-duty}}

The state was the first to hold a major election with in-person voting
since stay-at-home orders were widely instituted because of the
coronavirus.

By The New York Times

\begin{itemize}
\item
  April 7, 2020
\item
  \begin{itemize}
  \item
  \item
  \item
  \item
  \item
  \item
  \end{itemize}
\end{itemize}

\includegraphics{https://static01.graylady3jvrrxbe.onion/images/2020/04/07/us/politics/07wisconsin-briefing-videotop/07wisconsin-briefing-videotop-videoSixteenByNine3000.jpg}

\begin{itemize}
\item
  Wisconsin held its presidential primary between former Vice President
  \href{https://www.nytimes3xbfgragh.onion/interactive/2020/us/elections/joe-biden.html}{Joseph
  R. Biden Jr.} and Senator
  \href{https://www.nytimes3xbfgragh.onion/interactive/2020/us/elections/bernie-sanders.html}{Bernie
  Sanders}. Mr. Biden had a strong lead in a recent, widely respected
  poll.
\item
  Polls closed at 9 p.m. Eastern time. Long lines were seen in cities
  like Milwaukee, which had only five polling places open.
\item
  The state's elections commission had ordered municipal clerks not to
  release any results until April 13, in compliance with a federal court
  ruling.
\item
  Wisconsin Democrats wanted to extend absentee voting and even postpone
  the election altogether, but Republicans
  \href{https://www.nytimes3xbfgragh.onion/2020/04/06/us/politics/wisconsin-primary-voting-coronavirus.html}{successfully
  blocked both} in court. As a result, Democratic turnout was likely to
  be depressed because of the virus and the deadlines for absentee
  voting. A crucial seat on the State Supreme Court is on the ballot.
\end{itemize}

\hypertarget{an-election-almost-certain-to-be-tarred-as-illegitimate}{%
\subsection{An election almost certain to be tarred as
illegitimate.}\label{an-election-almost-certain-to-be-tarred-as-illegitimate}}

In Milwaukee, citizens were forced to choose between following public
health orders to stay home and stand in line for hours at one of just
five polling places the city kept open amid the coronavirus pandemic.

Across the state along the St. Croix River, a state senator who is her
county's chief medical examiner brought a homemade face mask to the
polls because she didn't want to take a surgical mask from her
co-workers who will have to inspect the bodies of people who die from
the coronavirus.

And everywhere in between, Wisconsinites reported an array of problems
with absentee ballots. Some didn't arrive, some couldn't be legally
witnessed and others were afraid to venture outside their homes to
return their ballots by Tuesday night's deadline.

It added up to an election almost certain to be tarred as illegitimate
and contested by whichever side loses --- especially if the conservative
State Supreme Court Justice Daniel Kelly wins a full 10-year term.

``People are going to be wondering about the authenticity of the vote no
matter what because of the politicalization,'' said Patty Schachtner, a
Democratic state senator from St. Croix County who made her own mask to
wear during a six-hour stint as a poll worker. Other poll workers, she
said, had no protection at all.

That Wisconsin's leaders, Gov. Tony Evers, a Democrat, and the
Republicans in charge of state legislative majorities, Robin Vos, the
Assembly speaker, and Scott Fitzgerald, the State Senate majority
leader, could not come to an agreement on how to alter the election is
an epic and predictable failure.

It follows a decade of bitter partisan wrangling that saw former Gov.
Scott Walker and his G.O.P. clinically attack and defang the state's
Democratic institutions, starting with organized labor and continuing
with voting laws making it far harder for poor and black residents of
urban areas to vote.

Like so much else in Wisconsin, Tuesday's vote brought divisions along
partisan and geographical lines. In Milwaukee, where just five of 180
polling sites remained open, voters who hadn't already cast absentee
ballots --- an overwhelmingly black and Hispanic population --- waited
in lines for hours.

Elsewhere in the state, especially in areas that have not yet been hit
hard by the outbreak, officials reported shorter lines and polling sites
that remained open for business as usual.

``We have people, out of the kindness of their heart, have volunteered
to drive around and witness ballots and deliver them to the clerk's
office,'' said Matt Lederer, the Democratic Party chairman in Outagamie
County, in the Fox Valley between Milwaukee and Green Bay. ``We're
making phone calls and we're doing our best but so far, I'm hearing that
the turnout seems low.''

Republicans, meanwhile, said they knew of few problems outside of
Milwaukee, which has long been portrayed by the state's conservatives as
the source of Wisconsin's problems. There was little sympathy.

``Everybody had a fair opportunity to vote,'' said Dennis Gasper, the
Republican Party chairman in Sheboygan County. ``Nobody's having a
problem voting. I went by a number of our polling places and there's no
lines out in the country.''

\hypertarget{black-voters-in-milwaukee-are-hit-hardest-by-coronavirus}{%
\subsection{Black voters in Milwaukee are hit hardest by
coronavirus.}\label{black-voters-in-milwaukee-are-hit-hardest-by-coronavirus}}

Milwaukee is the epicenter of Wisconsin's coronavirus pandemic, and the
black community in Milwaukee is among the most ravaged. As of Tuesday
afternoon, Milwaukee county's coronavirus dashboard showed black
Americans made up 626 of the county's 1,387 confirmed cases, and 36 of
its 51 deaths.

The numbers informed the fear among residents who decided to vote
Tuesday, braving crowds and even some hail to cast their ballots.
Milwaukee had a massive drop off in poll workers ahead of Tuesday's
election, forcing the city to close all but five of its polling station.
Some residents had to wait more than two hours, while covering their
faces with makeshift masks and trying to maintain proper social
distancing.

``Everything has changed,'' said Lt. Gov. Mandela Barnes, a native of
Milwaukee. ``And folks had more time to figure this out, and we don't.''

Mr. Barnes said it was already clear that the coronavirus pandemic had
upended the state's political landscape. He said the city should be
commended for releasing such race-driven data on the virus, and that
elected officials had to be creative in their solutions moving forward,
as typical forms of political organizing, like rallies or door-knocking,
won't work.

``Covid-19 has taken the day. It's front and center and everything that
we're dealing with,'' he said. ``I just don't know how you break through
the ice that is coronavirus.''

``Because even when you move to a mostly digital organizing model,
there's still a digital divide that exists,'' he added. ``And there's a
real fear that folks won't be able to get the right information or that
those people could be overloaded with misinformation.''

\href{https://www.nytimes3xbfgragh.onion/interactive/2020/us/wisconsin-coronavirus-cases.html}{}

\includegraphics{https://static01.graylady3jvrrxbe.onion/images/2020/03/29/us/wisconsin-coronavirus-cases-promo-1585539580772/wisconsin-coronavirus-cases-promo-1585539580772-articleLarge-v116.png}

\hypertarget{wisconsin-coronavirus-map-and-case-count}{%
\subsection{Wisconsin Coronavirus Map and Case
Count}\label{wisconsin-coronavirus-map-and-case-count}}

A detailed county map shows the extent of the coronavirus outbreak, with
tables of the number of cases by county.

\hypertarget{long-lines-also-plague-voting-in-green-bay}{%
\subsection{Long lines also plague voting in Green
Bay.}\label{long-lines-also-plague-voting-in-green-bay}}

Long lines weren't just confined to polling locations in Milwaukee.

In Green Bay, the third largest city in Wisconsin, polling locations
were reduced from 31 on a normal Election Day to just two on Tuesday, as
only 17 poll workers of the city's roster of 270 were able to work.

At West High School, the line ``went along the parking lot, snaked back
up and went around the building,'' said Seth Hoffmeister, 29, who lives
in Green Bay. ``I got out to film it but the line was so long I had to
get back in my car to film the rest.''

Mr. Hoffmeister, who voted early, does political organizing work for the
Wisconsin Conservation Voters, and had been hearing from field
organizers throughout the day of long lines in Green Bay.

``A gentleman that lives in Green Bay and has C.O.P.D. wore a mask to go
vote at East High School at 7 a.m. and he got back from the polls around
10 a.m.,'' Mr. Hoffmeister said, reading an email from one of his
organizers. ``And the lines were even longer when he left than he
arrived.''

Green Bay was one of the first cities to begin taking aggressive action
to try to postpone the Wisconsin primary, filing a lawsuit against the
state on March 25 seeking to delay the primary.

``The city is finding it functionally impossible to comply with both the
Wisconsin Election Commission's established procedures for administering
the election and the directives of health officials,'' the legal
complaint, written more than two weeks ago, said.

\includegraphics{https://static01.graylady3jvrrxbe.onion/images/2020/04/07/us/politics/07wisconsin-briefing-hintz/merlin_147720624_6131d2e5-0eb3-4607-bd8e-b9297d37e6a2-articleLarge.jpg?quality=75\&auto=webp\&disable=upscale}

\hypertarget{many-voters-say-their-absentee-ballots-never-arrived}{%
\subsection{Many voters say their absentee ballots never
arrived.}\label{many-voters-say-their-absentee-ballots-never-arrived}}

Across Wisconsin, would-be voters complained that the absentee ballots
they requested had never arrived in the mail, even though figures
released by the state seemed to indicate the problem was not widespread.

Representative Gordon Hintz, the Democratic minority leader in the State
Assembly, said there may have been a glitch in the system, perhaps
because of overwhelmed elections offices. ``It appears that people who
requested their ballots between March the 20th and 24th, or maybe the
25th, have not received their ballots,'' Mr. Hintz said.

Official state figures showed that of 1,282,762 ballots requested,
1,273,374 had been sent, a shortfall of about 9,000. Voters had returned
864,750 ballots by Tuesday morning. (Only 249,500 absentee ballots were
issued in the spring 2016 election.)

But Mr. Hintz estimated that hundreds, if not thousands, of voters in
his Oshkosh district alone had not received the ballots they asked for,
leaving them in a predicament over whether to vote in person and risk
contracting or spreading the coronavirus.

One of them was Mr. Hintz himself, who had decided not to vote Tuesday
because the ballot he requested on March 22 had not arrived. The
Wisconsin Elections Commission's website says it was mailed to him on
March 24.

Roger Luhn, a psychiatrist in Milwaukee, said Tuesday that he was also
among the voters who had not received an absentee ballot.

``According to the website, they mailed the ballot to me on March 23,''
said Dr. Luhn, who is medical director of a psychiatric hospital.
``Yesterday, I gave up. I called the election commission. They put you
on extended hold.''

Dr. Luhn said he would not go to the polls on Tuesday out of concern for
his family, his patients and his fellow staff members. ``There is no
good outcome for today's election,'' he said. ``No matter what happens,
not enough people will have had an opportunity to safely cast their
ballots.''

\hypertarget{voters-encounter-long-lines--and-social-distancing}{%
\subsection{Voters encounter long lines --- and social
distancing.}\label{voters-encounter-long-lines--and-social-distancing}}

The effects of shuttering so many polling sites in Milwaukee were
immediately apparent on Tuesday morning: Across the city, lines
stretched for blocks even before 7 a.m. local time.

On the South Side of the city, the parking lot of Alexander Hamilton
High School was already full as daylight broke. By 8 a.m., more than 300
voters waited in a line that snaked through the parking lot and down the
street.

At other locations nearby that would have normally been open for voting,
signs were posted directing voters to Hamilton High School. But many of
the locations were in heavily immigrant neighborhoods, predominantly
Spanish or Hmong, and the only signs posted were in English.

At Marshall High School, in the northern part of Milwaukee, the line
stretched for more than three blocks, with voters keeping six feet of
space between each other. Most wore masks or other facial coverings.

The northern part of the city, which is predominantly black, has been
hit the hardest by the coronavirus. Yet hundreds of voters had already
queued by early morning.

The lines weren't limited to Milwaukee. In Waukesha, a suburb just
outside of Milwaukee, only one polling location was open for a city of
70,000. A similarly long line wrapped around a parking lot, as cones
denoting a safe distance between voters helped break up the line.

Image

People lined up to vote at Riverside High School in Milwaukee on
Tuesday. The lines began before polls opened at 7 a.m. and continued
into the afternoon.Credit...Lauren Justice for The New York Times

\hypertarget{a-different-scene-outside-wisconsins-cities}{%
\subsection{A different scene outside Wisconsin's
cities.}\label{a-different-scene-outside-wisconsins-cities}}

Voting in Milwaukee was an unmitigated mess, but like so much else in
Wisconsin, the scene is markedly different outside the main urban areas.
Officials in rural counties --- especially Republican officials ---
reported few problems at the polls and said they weren't aware of people
having problems voting absentee.

``I don't know anyone personally who tried to get a ballot and didn't
get one,'' said Rohn Bishop, the Republican Party chairman in Fond du
Lac County. ``I think turnout might actually be slightly higher than
your typical spring election because there was such a push to get
absentee ballots up.''

\href{https://www.nytimes3xbfgragh.onion/news-event/2020-election}{Election
2020 ›}

\hypertarget{live-updates}{%
\subsection{\texorpdfstring{\href{https://www.nytimes3xbfgragh.onion/live/2020/09/08/us/trump-vs-biden}{Live
Updates}}{Live Updates}}\label{live-updates}}

\href{https://www.nytimes3xbfgragh.onion/live/2020/09/08/us/trump-vs-biden\#in-a-closely-watched-new-hampshire-primary-democrats-will-pick-a-challenger-to-governor-sununu}{}

Sept. 8, 2020, 9:37 a.m. ET

\href{https://www.nytimes3xbfgragh.onion/live/2020/09/08/us/trump-vs-biden\#in-a-closely-watched-new-hampshire-primary-democrats-will-pick-a-challenger-to-governor-sununu}{In
a closely watched New Hampshire primary, Democrats will pick a
challenger to Governor
Sununu.}\href{https://www.nytimes3xbfgragh.onion/live/2020/09/08/us/trump-vs-biden\#pence-and-harris-vied-for-wisconsin-a-pivotal-state-in-the-2020-race}{}

Sept. 8, 2020, 8:40 a.m. ET

\href{https://www.nytimes3xbfgragh.onion/live/2020/09/08/us/trump-vs-biden\#pence-and-harris-vied-for-wisconsin-a-pivotal-state-in-the-2020-race}{Pence
and Harris vied for Wisconsin, a pivotal state in the 2020
race.}\href{https://www.nytimes3xbfgragh.onion/live/2020/09/08/us/trump-vs-biden\#with-labor-day-behind-them-the-campaigns-take-flight-literally}{}

Sept. 8, 2020, 8:18 a.m. ET

\href{https://www.nytimes3xbfgragh.onion/live/2020/09/08/us/trump-vs-biden\#with-labor-day-behind-them-the-campaigns-take-flight-literally}{With
Labor Day behind them, the campaigns take flight --- literally.}

In Sheboygan County, about an hour north of Milwaukee up the Lake
Michigan shore, Dennis Gasper, the local G.O.P. chairman, said he drove
around local polling places Tuesday morning and found no issues.

``All the clerks have figured out how to deal with the coronavirus
thing, so nobody should be having a problem voting,'' he said.

Mr. Gasper said he was more focused on the prospect of ``fraudulent
voting'' from Milwaukee and Madison, a well-worn concern of Wisconsin
Republicans that has never been backed up by evidence. It was in fact
rural voters, he argued, who were being disenfranchised because their
local clerks did not offer as many days of early voting as did those in
Milwaukee and Madison.

``I can tell you right now that you're going to hear all about voter
suppression, this is what they're setting up to do, when in fact the
Milwaukee and Madison voters had much more opportunity to vote than the
rural communities did,'' Mr. Gasper said. ``That doesn't stop them from
making political points out of what would normally be a mundane election
process.''

Image

Workers sat behind a plexiglass barrier at a polling place in Dunn,
Wis.Credit...John Hart/Wisconsin State Journal, via Associated Press

\hypertarget{beyond-reduced-polling-sites-and-undelivered-absentee-ballots-another-hurdle-to-voting}{%
\subsection{Beyond reduced polling sites and undelivered absentee
ballots, another hurdle to
voting.}\label{beyond-reduced-polling-sites-and-undelivered-absentee-ballots-another-hurdle-to-voting}}

Countless Wisconsinites were unable to vote Tuesday because local public
health officials reduced polling sites or because municipal clerks were
swamped by a record number of absentee ballot requests.

Still others found themselves disenfranchised by the state's absentee
voting laws, which require a witness signature before a ballot can be
returned.

Jill Swenson, a 61-year-old literary agent, is self-quarantining in her
Appleton home. A widow, she lives alone, suffers from a chronic lung
disease and fears contracting the coronavirus.

``I received my ballot and was much surprised, since I had never voted
absentee before, to discover there was a witness requirement,'' Ms.
Swenson said in a phone interview Tuesday.

``I don't know anyone who has been self-isolating who could be a witness
for me.''

Her neighbors, Ms. Swenson said, include an Appleton police officer, a
factory worker and teenage boys who have been playing basketball in
their driveway --- no one with whom she felt comfortable sharing space
and air.

She contacted the Wisconsin Elections Commission, the bipartisan state
elections agency run by a former Republican state lawmaker, and was sent
advice on how to find a witness.

The \href{https://elections.wi.gov/node/6790}{commission proposed} Ms.
Swenson have a witness come to her window, or watch her mark her ballot
via FaceTime or Skype. Ms. Swenson would then have to sign her ballot,
leave it outside her house, and have the witness sign and return it.

Then on April 2, U.S. District Judge William Conley waived the absentee
witness requirement. Ms. Swenson put her ballot in the mail the next
day.

But the Supreme Court on Monday night overturned Judge Conley's ruling,
reinstating the witness requirement. Ms. Swenson's ballot, mailed three
days earlier, will not count.

When she logged on to the state elections website to check the status of
her ballot Tuesday, it said her completed ballot was not received.

Image

A drive-through polling place was set up outside City Hall in Beloit,
Wis.Credit...Daniel Acker/Reuters

\hypertarget{partisan-brawling-and-a-logistical-tangle-have-led-to-chaos}{%
\subsection{Partisan brawling and a logistical tangle have led to
chaos.}\label{partisan-brawling-and-a-logistical-tangle-have-led-to-chaos}}

Like so much else in Wisconsin over the last decade, the state's
coronavirus response and opinions about moving the election broke along
partisan lines.

Democrats, aiming to expand turnout especially in the state's largest
cities, Milwaukee and Madison, sought to expand mail voting and delay
the election until June. Republicans, wary of affording new powers to a
Democratic governor and content with suppressing turnout in urban
centers where the coronavirus has struck hardest, refused to entertain
proposals for relief.

``Thousands will wake up and have to choose between exercising their
right to vote and staying healthy and safe,'' Gov. Tony Evers said
Monday after the state's Supreme Court
\href{https://www.nytimes3xbfgragh.onion/2020/04/06/us/politics/wisconsin-primary-voting-coronavirus.html}{blocked
his effort to postpone the election}.

But Dean Knudson, a Republican former state legislator who is chairman
of the Wisconsin Elections Commission, said late Monday that voters who
wished to participate in Tuesday's contest would have no recourse but to
venture to the polls --- even if they had requested but had not yet
received an absentee ballot.

``If they haven't got their ballot in the mail,'' he said, ``they are
going to have to go to the polling place tomorrow.''

Other Republicans have played down the danger to public health of voting
during a pandemic. One Republican county chair, Jim Miller of Sawyer
County, said the process would be similar to people picking up food to
eat during the state's stay-at-home order.

``If you can go out and get fast food, you can go vote curbside,'' Mr.
Miller said. ``It's the same procedure.''

Image

Voters exiting Riverside High School.Credit...Lauren Justice for The New
York Times

\hypertarget{why-were-wisconsin-republicans-so-adamant-about-holding-tuesdays-elections}{%
\subsection{Why were Wisconsin Republicans so adamant about holding
Tuesday's
elections?}\label{why-were-wisconsin-republicans-so-adamant-about-holding-tuesdays-elections}}

It's not just a presidential primary on the ballot in Wisconsin. Also at
stake is the makeup of the Wisconsin Supreme Court --- the very court
that struck down Mr. Evers's effort to delay Tuesday's elections.

Statewide races in Wisconsin tend to be close, and Supreme Court
elections, which come with 10-year terms, are often even closer.

Last year Brian Hagedorn, a conservative judge, defeated a liberal
challenger by less than 6,000 votes out of 1.2 million cast. In 2011,
another conservative, David T. Prosser Jr., won by 7,000 votes after
officials in Waukesha County
\href{https://www.nytimes3xbfgragh.onion/2011/04/13/us/13wisconsin.html}{found
14,000 overlooked ballots} the day after the election.

For now, conservatives hold five of seven seats on the officially
nonpartisan court. The incumbent in Tuesday's contest, Justice Daniel
Kelly, was appointed to replace Justice Prosser by Gov. Scott Walker in
2016 and is seeking his first full term. He faces Jill Karofsky, a
circuit court judge.

President Trump has posted several messages on Twitter endorsing Justice
Kelly in recent days.

If Justice Kelly wins, it will cement the conservative majority's
ability to block future Democratic efforts to change the state's strict
voting laws and litigate an expected stalemate over congressional and
state legislative boundaries during post-2020 redistricting.

Liberals would need to flip just one of the conservatives' votes if
Judge Karofsky wins. Unless a justice retires or resigns, they would not
have an opportunity to win a court majority until the 2023 elections.

Reporting was contributed by Astead W. Herndon from Milwaukee, Nick
Corasaniti and Stephanie Saul from New York, and Reid J. Epstein from
Washington.

\hypertarget{our-2020-election-guide}{%
\section{Our 2020 Election Guide}\label{our-2020-election-guide}}

Updated ~Sept. 8, 2020

\begin{center}\rule{0.5\linewidth}{\linethickness}\end{center}

\begin{itemize}
\item ~
  \hypertarget{the-latest}{%
  \subsection{The Latest}\label{the-latest}}

  \begin{itemize}
  \item
    The campaign
    \href{https://www.nytimes3xbfgragh.onion/live/2020/09/08/us/trump-vs-biden?action=click\&pgtype=Article\&state=default\&region=BELOW_MAIN_CONTENT\&context=storylines_guide}{shifts
    to a higher gear this week}, with President Trump set to visit
    Florida and North Carolina today and Joseph R. Biden heading to
    Michigan tomorrow.
  \end{itemize}
\item ~
  \hypertarget{how-to-win-270}{%
  \subsection{How to Win 270}\label{how-to-win-270}}

  \begin{itemize}
  \item
    Joe Biden and Donald Trump need 270 electoral votes to reach the
    White House. Try building
    \href{https://www.nytimes3xbfgragh.onion/interactive/2020/us/elections/election-states-biden-trump.html?action=click\&pgtype=Article\&state=default\&region=BELOW_MAIN_CONTENT\&context=storylines_guide}{your
    own coalition of battleground states}~to see potential outcomes.
  \end{itemize}
\item ~
  \hypertarget{voting-by-mail}{%
  \subsection{Voting by Mail}\label{voting-by-mail}}

  \begin{itemize}
  \item
    Will you have enough time to vote by mail in your state? Yes, but
    it's risky to procrastinate.
    \href{https://www.nytimes3xbfgragh.onion/interactive/2020/08/31/us/politics/vote-by-mail-deadlines.html?action=click\&pgtype=Article\&state=default\&region=BELOW_MAIN_CONTENT\&context=storylines_guide}{Check
    your state's deadline.}
  \item
    \href{https://www.nytimes3xbfgragh.onion/interactive/2020/us/elections/joe-biden.html?action=click\&pgtype=Article\&state=default\&region=BELOW_MAIN_CONTENT\&context=storylines_guide}{}

    \hypertarget{joe-biden}{%
    \section{Joe Biden}\label{joe-biden}}

    \hypertarget{democrat}{%
    \subsection{Democrat}\label{democrat}}

    \href{https://www.nytimes3xbfgragh.onion/interactive/2020/us/elections/donald-trump.html?action=click\&pgtype=Article\&state=default\&region=BELOW_MAIN_CONTENT\&context=storylines_guide}{}

    \hypertarget{donald-trump}{%
    \section{Donald Trump}\label{donald-trump}}

    \hypertarget{republican}{%
    \subsection{Republican}\label{republican}}
  \end{itemize}
\item
  \hypertarget{keep-up-with-our-coverage}{%
  \subsection{Keep Up With Our
  Coverage}\label{keep-up-with-our-coverage}}

  \begin{itemize}
  \item
    Get an
    \href{https://www.nytimes3xbfgragh.onion/newsletters/politics?action=click\&pgtype=Article\&state=default\&region=BELOW_MAIN_CONTENT\&context=storylines_guide}{email}~recapping
    the day's news
  \item
    Download our mobile app on
    \href{https://apps.apple.com/us/app/nytimes/id284862083?ls=1\&mat_click_id=5c79ae7455014fd1bd66b5610c05b8f2-20191112-16948\&referrer=mat_click_id\%3D5c79ae7455014fd1bd66b5610c05b8f2-20191112-16948\%26link_click_id\%3D722930677036718082}{iOS}~and
    \href{http://a.localytics.com/android?id=com.nytimes.android\&referrer=utm_source\%3Dother_nyt_mobile_web\%26utm_medium\%3DWeb\%2520page\%26utm_term\%3DGeneral\%2520Mobile\%2520Page\%26utm_campaign\%3DNYT\%2520Mobile\%2520General\%2520Page}{Android}~and
    turn on Breaking News and Politics alerts
  \end{itemize}
\end{itemize}

Advertisement

\protect\hyperlink{after-bottom}{Continue reading the main story}

\hypertarget{site-index}{%
\subsection{Site Index}\label{site-index}}

\hypertarget{site-information-navigation}{%
\subsection{Site Information
Navigation}\label{site-information-navigation}}

\begin{itemize}
\tightlist
\item
  \href{https://help.nytimes3xbfgragh.onion/hc/en-us/articles/115014792127-Copyright-notice}{©~2020~The
  New York Times Company}
\end{itemize}

\begin{itemize}
\tightlist
\item
  \href{https://www.nytco.com/}{NYTCo}
\item
  \href{https://help.nytimes3xbfgragh.onion/hc/en-us/articles/115015385887-Contact-Us}{Contact
  Us}
\item
  \href{https://www.nytco.com/careers/}{Work with us}
\item
  \href{https://nytmediakit.com/}{Advertise}
\item
  \href{http://www.tbrandstudio.com/}{T Brand Studio}
\item
  \href{https://www.nytimes3xbfgragh.onion/privacy/cookie-policy\#how-do-i-manage-trackers}{Your
  Ad Choices}
\item
  \href{https://www.nytimes3xbfgragh.onion/privacy}{Privacy}
\item
  \href{https://help.nytimes3xbfgragh.onion/hc/en-us/articles/115014893428-Terms-of-service}{Terms
  of Service}
\item
  \href{https://help.nytimes3xbfgragh.onion/hc/en-us/articles/115014893968-Terms-of-sale}{Terms
  of Sale}
\item
  \href{https://spiderbites.nytimes3xbfgragh.onion}{Site Map}
\item
  \href{https://help.nytimes3xbfgragh.onion/hc/en-us}{Help}
\item
  \href{https://www.nytimes3xbfgragh.onion/subscription?campaignId=37WXW}{Subscriptions}
\end{itemize}
