Sections

SEARCH

\protect\hyperlink{site-content}{Skip to
content}\protect\hyperlink{site-index}{Skip to site index}

\href{https://www.nytimes3xbfgragh.onion/section/politics}{Politics}

\href{https://myaccount.nytimes3xbfgragh.onion/auth/login?response_type=cookie\&client_id=vi}{}

\href{https://www.nytimes3xbfgragh.onion/section/todayspaper}{Today's
Paper}

\href{/section/politics}{Politics}\textbar{}They Turned Out to Vote in
Wisconsin During a Health Crisis. Here's Why.

\url{https://nyti.ms/3e4SAXC}

\begin{itemize}
\item
\item
\item
\item
\item
\end{itemize}

\begin{itemize}
\item
  \href{https://www.nytimes3xbfgragh.onion/live/2020/09/07/us/trump-vs-biden?action=click\&pgtype=Article\&state=default\&region=TOP_BANNER\&context=storylines_menu}{Election
  Updates}
\item
  \href{https://www.nytimes3xbfgragh.onion/interactive/2020/us/elections/election-states-biden-trump.html?action=click\&pgtype=Article\&state=default\&region=TOP_BANNER\&context=storylines_menu}{Paths
  to 270}
\item
  \href{https://www.nytimes3xbfgragh.onion/interactive/2020/08/31/us/politics/vote-by-mail-deadlines.html?action=click\&pgtype=Article\&state=default\&region=TOP_BANNER\&context=storylines_menu}{Voting
  by Mail}
\item
  \href{https://www.nytimes3xbfgragh.onion/interactive/2019/us/elections/2020-presidential-election-calendar.html?action=click\&pgtype=Article\&state=default\&region=TOP_BANNER\&context=storylines_menu}{Key
  Dates}
\item
  \href{https://www.nytimes3xbfgragh.onion/newsletters/politics?action=click\&pgtype=Article\&state=default\&region=TOP_BANNER\&context=storylines_menu}{Politics
  Newsletter}
\end{itemize}

Advertisement

\protect\hyperlink{after-top}{Continue reading the main story}

Supported by

\protect\hyperlink{after-sponsor}{Continue reading the main story}

\hypertarget{they-turned-out-to-vote-in-wisconsin-during-a-health-crisis-heres-why}{%
\section{They Turned Out to Vote in Wisconsin During a Health Crisis.
Here's
Why.}\label{they-turned-out-to-vote-in-wisconsin-during-a-health-crisis-heres-why}}

Voters in Milwaukee and across the state braved long lines and a risk of
illness. Many said they wanted their voices to be heard. But for some,
the potential health risk was too great.

\includegraphics{https://static01.graylady3jvrrxbe.onion/images/2020/04/07/us/politics/07wisconsin-briefing-videotop/07wisconsin-briefing-videotop-videoSixteenByNine3000.jpg}

\href{https://www.nytimes3xbfgragh.onion/by/astead-w-herndon}{\includegraphics{https://static01.graylady3jvrrxbe.onion/images/2018/09/14/us/author-head-astead/author-head-astead-thumbLarge-v2.png}}

By \href{https://www.nytimes3xbfgragh.onion/by/astead-w-herndon}{Astead
W. Herndon}

\begin{itemize}
\item
  April 7, 2020
\item
  \begin{itemize}
  \item
  \item
  \item
  \item
  \item
  \end{itemize}
\end{itemize}

MILWAUKEE --- After days of legal wrangling, partisan mudslinging and
grave warnings from public health professionals, Wisconsin forged ahead
with its
\href{https://www.nytimes3xbfgragh.onion/2020/04/07/us/politics/wisconsin-primary-election.html}{elections
on Tuesday}, the first state to hold in-person voting during the height
of the coronavirus pandemic.

Thousands of Wisconsin residents, forced to weigh the risks to their
health against their willingness to exercise the right to vote, arrived
before polls opened at 7 a.m., casting ballots for the national
Democratic presidential primary and several contests between Republicans
and Democrats in major state and local races.

Some poll workers wore makeshift hazmat suits, more reminiscent of
health care professionals than electoral volunteers. Most voters came
prepared with masks, gloves, hand sanitizer and Clorox wipes. Many said
they were racked with fear.

``They say they don't want you to get sick, but then they send you out
here in the damn crowd,'' said Lawrence Johnson, a 70-year-old cleaning
worker in line to vote at Riverside High School. ``There are people like
me who are handicapped --- we have no business doing all this just to
vote.''

At Washington High School on Milwaukee's North Side, a woman carried a
homemade sign that read ``This Is Ridiculous.''

Despite their trepidation, voters who showed up at polling locations ---
there were only five in Milwaukee, compared with the typical 180 ---
said that this was their day to be heard. Some Democrats spoke of a
sense of defiance to their actions, a determination to challenge the
state Republicans who refused to move the election date even after
requests from public health experts.

Nyree Sanders, 45, contacted her work supervisor while she was in line,
after it became clear the voting process would take up much more of her
day than anticipated.

``I just don't get as to why they were so insistent that we have to vote
today,'' Ms. Sanders said. ``Why don't they take us into account? Why
don't they take our health into account?''

Chris Wheeler, an appliance repairman, said he thought about not voting,
but since he's been working a high-risk job anyway to sustain his
income, he decided he might as well ``exercise my constitutional
right.''

``It's just irresponsible,'' said Mr. Wheeler, who is 58. ``I've been in
places where people are infected, I've been in hospitals --- it's just
my reality right now. It is what it is.''

Officials in state after state have
\href{https://www.nytimes3xbfgragh.onion/article/2020-campaign-primary-calendar-coronavirus.html}{postponed
in-person voting} in the last month, grinding the Democratic primary to
a halt as the impact of the coronavirus has disrupted every aspect of
American life. However, in Wisconsin, pleas from state Democrats to
delay voting were ignored by the Republican leaders in the Legislature,
who said moving the election was too drastic a measure and an
infringement on personal liberty.

Republicans in particular expressed outrage when the state's Democratic
governor, Tony Evers, proposed an expanded absentee ballot voting system
that would mail a ballot to each of the state's millions of registered
voters. Republicans also successfully blocked an executive order by Mr.
Evers on Monday to use emergency powers to delay the election, after the
State Supreme Court, which is controlled by conservatives, reversed Mr.
Evers's order.

The consequence was an election on Tuesday that was criticized as both
unsafe and illegitimate. National figures like Senator
\href{https://www.nytimes3xbfgragh.onion/interactive/2020/us/elections/bernie-sanders.html}{Bernie
Sanders} of Vermont called for it to be delayed, and the leaders of the
Democratic Party of Wisconsin --- which was seeking to win a vital State
Supreme Court seat --- said it could not encourage people to vote in
good faith.

Erin Baldeon Fischer, a 32-year-old graduate student, said she saw the
line at her Milwaukee polling place on Tuesday and decided the health
risk might be too great.

``I have a 9-month-old; I'm not sure I feel comfortable being there with
an infant at home,'' she said. ``I'm not sure it's a responsible
decision as a mother.''

Mary Corder, 70, usually votes in her senior assisted living facility,
but decided she couldn't vote in this election after the location was
shut down. Ms. Corder said she found out her polling place was closed
only yesterday, after voting there for years.

``I was told by my doctor to stay in and stay away from crowds,'' Ms.
Corder said. ``That's why I didn't go. Plus, I have no ride. No
transportation. I just can't take the risk.''

But the distress was hardly noticeable in other parts of the state that
are less populous, more white, and more likely to vote Republican. In
communities like Beloit, more drive-through voting options were
available, and the elimination of some polling places was likely to have
less of an effect.

In Wauwatosa, which borders Milwaukee, polling locations were virtually
empty as supply outpaced the demand. Outside the region in Cedarburg, a
more Republican community, two polling locations had virtually no wait,
and some voters arrived and left within 10 minutes.

Charlotte Rasmussen, the Republican chair of Clark County in central
Wisconsin, said she had no concerns.

``It's been so far, so good,'' Ms. Rasmussen said. ``Things are going
well, everybody is in protective gear and everyone is trying to practice
social distancing.''

In Brookfield, a city in nearby Waukesha County, one resident said his
polling location was adequately staffed, had good social distancing and
was relatively easy to navigate.

``Our kids are voting in Milwaukee and they're definitely waiting longer
than we did,'' said Bruce Campbell, 65. ``You can feel the blue county,
red county dynamics. It's difficult to watch.''

On Tuesday in Milwaukee, the state's most populous city and the
Democratic power base of the state, voting logistics were a categorical
nightmare. The city cut more than 170 polling locations before the
elections, citing health concerns, and lines at Riverside High School
--- one of five places where city residents could cast a ballot ---
snaked for about four blocks.

\includegraphics{https://static01.graylady3jvrrxbe.onion/images/2020/04/07/us/politics/07wisconsin-scene2/merlin_171363690_725fca45-5759-44d4-bb28-257f0b2650d8-articleLarge.jpg?quality=75\&auto=webp\&disable=upscale}

Throughout the city, residents sought to help one another stay safe and
upbeat. One house opposite Riverside High School blared a playlist of
James Brown, Bill Withers and other soul artists to keep the waiting
voters energized. State Representative David Bowen, a Democrat in
Milwaukee who contracted the coronavirus but has recovered, picked up
absentee ballots from sick residents and delivered them to the post
office. Breana Stephens, a 29-year-old teacher, traveled to the school
on her own to pass out Clorox wipes, fresh pens and gloves for those in
need.

``Everybody is really on edge, and you can sense that,'' said Ms.
Stephens, who voted with an absentee ballot more than a month ago.
``People are worried and anxious and not in the best of spirits. As a
person with some free time, I wanted to do what I can.''

Image

Ballots were collected at a drive-up polling place in Beloit,
Wis.~Credit...Daniel Acker/Reuters

The differences added to the sense of grievance felt among some
Milwaukee voters. They said they believed that the city's racial makeup,
abundance of Democrats and areas of high poverty made Republicans less
willing to care about the health risks. Like other cities around the
country, Milwaukee has also been more acutely affected by the
coronavirus than the more rural parts of the state; it has a majority of
the state's confirmed cases and deaths --- particularly among the city's
black population.

Clarence Carter, 70, said he was voting in person on Tuesday because he
filed for an absentee ballot weeks ago but did not receive it. His wife
has health issues and couldn't stand in the line, he said.

``The polling place next to my house closed down, so I'm here,'' Mr.
Carter said. ``I'm just disappointed. This is really crazy.''

So why vote?

``It's the ballot or the bullet,'' he said, quoting the famous speech by
Malcolm X.

\hypertarget{our-2020-election-guide}{%
\section{Our 2020 Election Guide}\label{our-2020-election-guide}}

Updated ~Sept. 7, 2020

\begin{center}\rule{0.5\linewidth}{\linethickness}\end{center}

\begin{itemize}
\item ~
  \hypertarget{the-latest}{%
  \subsection{The Latest}\label{the-latest}}

  \begin{itemize}
  \item
    The unofficial Labor Day kickoff to the fall presidential campaign
    centered on Pennsylvania and Wisconsin,
    \href{https://www.nytimes3xbfgragh.onion/2020/09/07/us/politics/wisconsin-biden-harris-trump-pence.html?action=click\&pgtype=Article\&state=default\&region=BELOW_MAIN_CONTENT\&context=storylines_guide}{two
    pivotal states for both President Trump and Joseph R. Biden Jr}.
  \end{itemize}
\item ~
  \hypertarget{how-to-win-270}{%
  \subsection{How to Win 270}\label{how-to-win-270}}

  \begin{itemize}
  \item
    Joe Biden and Donald Trump need 270 electoral votes to reach the
    White House. Try building
    \href{https://www.nytimes3xbfgragh.onion/interactive/2020/us/elections/election-states-biden-trump.html?action=click\&pgtype=Article\&state=default\&region=BELOW_MAIN_CONTENT\&context=storylines_guide}{your
    own coalition of battleground states}~to see potential outcomes.
  \end{itemize}
\item ~
  \hypertarget{voting-by-mail}{%
  \subsection{Voting by Mail}\label{voting-by-mail}}

  \begin{itemize}
  \item
    Will you have enough time to vote by mail in your state? Yes, but
    it's risky to procrastinate.
    \href{https://www.nytimes3xbfgragh.onion/interactive/2020/08/31/us/politics/vote-by-mail-deadlines.html?action=click\&pgtype=Article\&state=default\&region=BELOW_MAIN_CONTENT\&context=storylines_guide}{Check
    your state's deadline.}
  \item
    \href{https://www.nytimes3xbfgragh.onion/interactive/2020/us/elections/joe-biden.html?action=click\&pgtype=Article\&state=default\&region=BELOW_MAIN_CONTENT\&context=storylines_guide}{}

    \hypertarget{joe-biden}{%
    \section{Joe Biden}\label{joe-biden}}

    \hypertarget{democrat}{%
    \subsection{Democrat}\label{democrat}}

    \href{https://www.nytimes3xbfgragh.onion/interactive/2020/us/elections/donald-trump.html?action=click\&pgtype=Article\&state=default\&region=BELOW_MAIN_CONTENT\&context=storylines_guide}{}

    \hypertarget{donald-trump}{%
    \section{Donald Trump}\label{donald-trump}}

    \hypertarget{republican}{%
    \subsection{Republican}\label{republican}}
  \end{itemize}
\item
  \hypertarget{keep-up-with-our-coverage}{%
  \subsection{Keep Up With Our
  Coverage}\label{keep-up-with-our-coverage}}

  \begin{itemize}
  \item
    Get an
    \href{https://www.nytimes3xbfgragh.onion/newsletters/politics?action=click\&pgtype=Article\&state=default\&region=BELOW_MAIN_CONTENT\&context=storylines_guide}{email}~recapping
    the day's news
  \item
    Download our mobile app on
    \href{https://apps.apple.com/us/app/nytimes/id284862083?ls=1\&mat_click_id=5c79ae7455014fd1bd66b5610c05b8f2-20191112-16948\&referrer=mat_click_id\%3D5c79ae7455014fd1bd66b5610c05b8f2-20191112-16948\%26link_click_id\%3D722930677036718082}{iOS}~and
    \href{http://a.localytics.com/android?id=com.nytimes.android\&referrer=utm_source\%3Dother_nyt_mobile_web\%26utm_medium\%3DWeb\%2520page\%26utm_term\%3DGeneral\%2520Mobile\%2520Page\%26utm_campaign\%3DNYT\%2520Mobile\%2520General\%2520Page}{Android}~and
    turn on Breaking News and Politics alerts
  \end{itemize}
\end{itemize}

Advertisement

\protect\hyperlink{after-bottom}{Continue reading the main story}

\hypertarget{site-index}{%
\subsection{Site Index}\label{site-index}}

\hypertarget{site-information-navigation}{%
\subsection{Site Information
Navigation}\label{site-information-navigation}}

\begin{itemize}
\tightlist
\item
  \href{https://help.nytimes3xbfgragh.onion/hc/en-us/articles/115014792127-Copyright-notice}{©~2020~The
  New York Times Company}
\end{itemize}

\begin{itemize}
\tightlist
\item
  \href{https://www.nytco.com/}{NYTCo}
\item
  \href{https://help.nytimes3xbfgragh.onion/hc/en-us/articles/115015385887-Contact-Us}{Contact
  Us}
\item
  \href{https://www.nytco.com/careers/}{Work with us}
\item
  \href{https://nytmediakit.com/}{Advertise}
\item
  \href{http://www.tbrandstudio.com/}{T Brand Studio}
\item
  \href{https://www.nytimes3xbfgragh.onion/privacy/cookie-policy\#how-do-i-manage-trackers}{Your
  Ad Choices}
\item
  \href{https://www.nytimes3xbfgragh.onion/privacy}{Privacy}
\item
  \href{https://help.nytimes3xbfgragh.onion/hc/en-us/articles/115014893428-Terms-of-service}{Terms
  of Service}
\item
  \href{https://help.nytimes3xbfgragh.onion/hc/en-us/articles/115014893968-Terms-of-sale}{Terms
  of Sale}
\item
  \href{https://spiderbites.nytimes3xbfgragh.onion}{Site Map}
\item
  \href{https://help.nytimes3xbfgragh.onion/hc/en-us}{Help}
\item
  \href{https://www.nytimes3xbfgragh.onion/subscription?campaignId=37WXW}{Subscriptions}
\end{itemize}
