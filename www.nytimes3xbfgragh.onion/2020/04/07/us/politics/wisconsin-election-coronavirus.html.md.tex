Sections

SEARCH

\protect\hyperlink{site-content}{Skip to
content}\protect\hyperlink{site-index}{Skip to site index}

\href{https://www.nytimes3xbfgragh.onion/section/politics}{Politics}

\href{https://myaccount.nytimes3xbfgragh.onion/auth/login?response_type=cookie\&client_id=vi}{}

\href{https://www.nytimes3xbfgragh.onion/section/todayspaper}{Today's
Paper}

\href{/section/politics}{Politics}\textbar{}Voting in Wisconsin During a
Pandemic: Lines, Masks and Plenty of Fear

\url{https://nyti.ms/2xbTXDc}

\begin{itemize}
\item
\item
\item
\item
\item
\end{itemize}

\begin{itemize}
\item
  \href{https://www.nytimes3xbfgragh.onion/live/2020/09/07/us/trump-vs-biden?action=click\&pgtype=Article\&state=default\&region=TOP_BANNER\&context=storylines_menu}{Election
  Updates}
\item
  \href{https://www.nytimes3xbfgragh.onion/interactive/2020/us/elections/election-states-biden-trump.html?action=click\&pgtype=Article\&state=default\&region=TOP_BANNER\&context=storylines_menu}{Paths
  to 270}
\item
  \href{https://www.nytimes3xbfgragh.onion/interactive/2020/08/31/us/politics/vote-by-mail-deadlines.html?action=click\&pgtype=Article\&state=default\&region=TOP_BANNER\&context=storylines_menu}{Voting
  by Mail}
\item
  \href{https://www.nytimes3xbfgragh.onion/interactive/2019/us/elections/2020-presidential-election-calendar.html?action=click\&pgtype=Article\&state=default\&region=TOP_BANNER\&context=storylines_menu}{Key
  Dates}
\item
  \href{https://www.nytimes3xbfgragh.onion/newsletters/politics?action=click\&pgtype=Article\&state=default\&region=TOP_BANNER\&context=storylines_menu}{Politics
  Newsletter}
\end{itemize}

Advertisement

\protect\hyperlink{after-top}{Continue reading the main story}

Supported by

\protect\hyperlink{after-sponsor}{Continue reading the main story}

\hypertarget{voting-in-wisconsin-during-a-pandemic-lines-masks-and-plenty-of-fear}{%
\section{Voting in Wisconsin During a Pandemic: Lines, Masks and Plenty
of
Fear}\label{voting-in-wisconsin-during-a-pandemic-lines-masks-and-plenty-of-fear}}

Wisconsin's primary showed an electoral system stretched to the breaking
point by the coronavirus crisis, as people weighed the health risks
against their desire to vote.

\includegraphics{https://static01.graylady3jvrrxbe.onion/images/2020/04/07/us/politics/07wisconsin-primary-top/merlin_171379461_5617863d-7cfe-471e-af28-3ae7a0307829-articleLarge.jpg?quality=75\&auto=webp\&disable=upscale}

\href{https://www.nytimes3xbfgragh.onion/by/astead-w-herndon}{\includegraphics{https://static01.graylady3jvrrxbe.onion/images/2018/09/14/us/author-head-astead/author-head-astead-thumbLarge-v2.png}}\href{https://www.nytimes3xbfgragh.onion/by/alexander-burns}{\includegraphics{https://static01.graylady3jvrrxbe.onion/images/2018/09/25/multimedia/author-alexander-burns/author-alexander-burns-thumbLarge-v2.png}}

By \href{https://www.nytimes3xbfgragh.onion/by/astead-w-herndon}{Astead
W. Herndon} and
\href{https://www.nytimes3xbfgragh.onion/by/alexander-burns}{Alexander
Burns}

\begin{itemize}
\item
  Published April 7, 2020Updated May 12, 2020
\item
  \begin{itemize}
  \item
  \item
  \item
  \item
  \item
  \end{itemize}
\end{itemize}

MILWAUKEE --- Even before voting began, there were lines outside polling
locations that stretched for several blocks. Some poll workers wore
hazmat suits. Nearly every voter wore a face mask, removing it only to
make small talk that reflected a combination of determination and grim
humor about the extraordinary experience of voting amid a deadly
pandemic.

For thousands of people across
\href{https://www.nytimes3xbfgragh.onion/interactive/2020/05/12/us/elections/live-updates-california-wisconsin-special-election.html}{Wisconsin}
on Tuesday, fears of the coronavirus outbreak
\href{https://www.nytimes3xbfgragh.onion/2020/04/07/us/politics/wisconsin-democratic-voters.html}{did
not stop them from participating in the state's elections}, where
critical races such as the Democratic presidential primary and a key
state Supreme Court seat were being decided.

``It feels bad to have to choose between your personal safety and your
right to vote,'' said Dan Bullock, 40, as he waited to vote at
Washington High School on Milwaukee's North Side. ``But you have to be
heard.''

Many others across the state, however, appeared inclined to stay home as
\href{https://www.nytimes3xbfgragh.onion/2020/04/07/us/politics/wisconsin-primary-election.html}{the
fear of contracting the disease outweighed their desire to participate}
in the most fundamental ritual of democracy. Late Monday,
\href{https://www.nytimes3xbfgragh.onion/2020/04/07/us/politics/wisconsin-pandemic-primary-republicans.html}{Republicans
in the state legislature had gone to court to block the Democratic
governor's order} to postpone the primary.

``No one should have to choose between risking their health and possibly
dying and going to vote,'' said Marcelia Nicholson, 31, a county
supervisor for Milwaukee. She said she was unsure she could vote safely
after having been exposed to the coronavirus herself.

In Milwaukee --- where the number of polling stations was reduced from
180 to only five --- voters tried to exercise proper social distancing
as they waited, in some cases, for more than two hours. But in other
areas of the state, including Madison, suburbs like Brookfield, and more
rural areas like Beloit, the voting process was altered but not totally
disrupted, with options that included curbside ballot access and poll
locations that were more fully staffed.

Milwaukee has the biggest minority population in the state, which means
that geographic and partisan differences in access to voting often
overlap with racial ones.

\includegraphics{https://static01.graylady3jvrrxbe.onion/images/2020/04/07/us/politics/07wisconsin-primary2/merlin_171375540_90231870-27d7-4bdb-b5f0-eff19c759f59-articleLarge.jpg?quality=75\&auto=webp\&disable=upscale}

The scenes that unfolded in Wisconsin showed an electoral system
stretched to the breaking point by the same public health catastrophe
that has killed thousands and brought the country's economic and social
patterns to a virtual standstill in recent weeks. And in Wisconsin, the
political institutions proved overmatched, with a Republican legislature
and a conservative state and federal judiciary resisting efforts to
reschedule the
\href{https://www.nytimes3xbfgragh.onion/interactive/2020/05/12/us/elections/live-updates-california-wisconsin-special-election.html}{election}
or revise the procedures for voting.

The result was a dangerous spectacle that forced voters to choose
between participating in an important election and protecting their
health. While election administrators said they were trying in myriad
ways to make the voting process safer, the long lines, last-minute
judicial rulings and backlogged absentee ballot requests added up to
something resembling system failure.

Ellie Bradish, for instance, said she was forced to vote in person in
Milwaukee after attempts at early voting and absentee voting failed.

``My friend and colleagues at work were worried about me coming out,''
Ms. Bradish, 40, said. ``But I was worried that if I didn't come, my
vote would be thrown out.''

\includegraphics{https://static01.graylady3jvrrxbe.onion/images/2020/04/07/us/politics/07wisconsin-briefing-videotop/07wisconsin-briefing-videotop-videoSixteenByNine3000.jpg}

The array of procedural problems led some state party officials to
predict that the results would be contested by whichever side loses.

``People are going to be wondering about the authenticity of the vote no
matter what because of the politicalization,'' said Patty Schachtner, a
Democratic state senator from St. Croix County, who made her own mask to
wear during a six-hour stint as a poll worker.

National voting rights experts said the turmoil and acrimony surrounding
the election could be an unsettling example of what might happen across
the country later this spring if states do not manage to implement new
methods of voting during the coronavirus outbreak --- or even in the
November general election if the pandemic has not abated by then.

In Wisconsin, Gov. Tony Evers and his fellow Democrats pushed for a
range of changes to the primary process, including rescheduling the
election and switching to mail-in voting. But Democrats faced a wall of
resistance from Republicans who saw political advantage in leaving
existing procedures intact.

Image

People waited to vote at Riverside High School in Milwaukee on Tuesday.
Only five polling places were open in the city of half a
million.Credit...Lauren Justice for The New York Times

Almost forgotten amid a life-or-death debate about voting procedures was
a Democratic presidential race that is still not formally finished:
former Vice President Joseph R. Biden Jr. and Senator Bernie Sanders
were both competing for delegates in Wisconsin, though neither man
campaigned actively in the state. Mr. Biden, with a nearly
insurmountable delegate lead overall, was expected to carry the primary,
but in a strange byproduct of the tangled judicial rulings there would
be no results released until next week.

It was not only the Democratic presidential primary on the ballot on
Tuesday in Wisconsin: there was intense competition over a seat on the
Wisconsin Supreme Court, with one of the justices in the conservative
majority battling for re-election. The winner will be in position to
cast a deciding vote on a case before the court that seeks to purge more
than 200,000 people from Wisconsin's voter rolls.

The panel has been a bulwark of Republican dominance in the state over
the last decade, along with a G.O.P. majority in the State Legislature
that has entrenched itself through aggressive gerrymandering.

Like so much else in Wisconsin, the scene was markedly different outside
the main urban areas. Republican county chairs boasted about their
smooth process throughout the day, with short lines and ample room for a
smattering of voters who often showed little signs of the current health
crisis --- no masks, no gloves. In Sheboygan County, about an hour north
of Milwaukee up the Lake Michigan shore, Dennis Gasper, a Republican
Party official, said he drove around local polling places and found no
issues.

``All the clerks have figured out how to deal with the coronavirus
thing, so nobody should be having a problem voting,'' Mr. Gasper said.

Mr. Gasper chalked up complaints from Milwaukee and elsewhere to
Democrats ``making political points out of what would normally be a
mundane election process.''

Image

A drive-through polling place was set up in New Richmond,
Wis.Credit...Nicholas Pfosi/Reuters

But in Milwaukee, where there are more than 1,000 confirmed cases of the
virus and at least 87 people have died of it, many voters cast their
ballots wearing full protective gear, some overtaken with fear.

Some Republican efforts to downplay the danger of the election ended up
highlighting the medical risks involved. For instance, Robin Vos, the
Republican speaker of the Wisconsin Assembly,
\href{https://www.facebookcorewwwi.onion/SpeakerVos/posts/2864610826921333?__tn__=-R}{posted
on Facebook}that he was volunteering as a poll worker, writing that ``an
impressive amount of planning and organization'' went into securing the
election. In photos, Mr. Vos looked ready to enter a contaminated zone:
He was wearing a face mask, a plastic body covering and gloves.

The partisan divide within Wisconsin over the safety and integrity of
the election was mirrored on the national level, with the Democratic
presidential candidates raising concerns about the safety of the vote
and President Trump urging Republican voters to the polls as though
little was out of the ordinary. On Twitter, Mr. Trump lauded the
incumbent Wisconsin Supreme Court justice seeking re-election, Daniel
Kelly, as a jurist who ``loves your Military, Vets, Farmers.'' And later
in the day he weighed in on the vote by mail effort, saying without
evidence, ``The mail ballots are corrupt, in my opinion.''

``Mail ballots, they cheat,'' the president said at his evening briefing
on the virus. ``Mail ballots are very dangerous for this country because
of cheaters. They go collect them. They are fraudulent in many cases.''

Mr. Trump himself voted by mail in the 2018 midterms and in last month's
Florida primary.

Mr. Sanders, who like Mr. Biden has held no campaign events in nearly a
month because of the virus, took a far more somber approach, rebuking
Wisconsin Republicans for risking ``the health and safety of many
thousands of Wisconsin voters'' to force an election under conditions of
extreme adversity.

Mr. Biden, who last week said he would defer ``to the scientists'' and
to state officials about in-person voting in Wisconsin, was more
skeptical Tuesday night. ``My gut is that we shouldn't have had the
election in the first place, the in-person election,'' he said on CNN.
``It should have been all-mail ballots in, it should have been moved in
the way that five other states have done it.''

It was not just Democratic candidates who called the election dangerous:
As a series of court decisions were announced on Monday evening ---
every one of them a setback for Democrats --- Speaker Nancy Pelosi and
other prominent party leaders accused the judiciary of siding with the
Republican Party over the interests of voters.

In perhaps the sternest comments of all, Ms. Pelosi rebuked the federal
Supreme Court for
\href{https://www.nytimes3xbfgragh.onion/2020/04/07/us/politics/wisconsin-elections-supreme-court.html}{rejecting
an effort late Monday to extend absentee balloting} in Wisconsin. The
court, Ms. Pelosi said on television, was ``undermining our democracy.''

Image

A poll worker checked in a voter from behind a plexiglass barrier in
Madison.Credit...Steve Apps/Wisconsin State Journal, via Associated
Press

Despite the collective sense of civic duty on display in Wisconsin, many
voters expressed dismay at the difficult circumstances. Kinnethia
Tolson-Johnson arrived at her polling place in Milwaukee before it
opened, hoping to avoid crowds and stay safe. But she had to wait
outside for more than an hour while fellow voters organized themselves
to keep proper distance. ``It was discouraging,'' Ms. Tolson-Johnson
said. ``And it was just hard to keep your energy.''

Hannah Gleeson, a health care worker in Milwaukee who is 17 weeks
pregnant, and who recently tested positive for coronavirus, said she was
despondent about being unable to vote. ``I've always said that every
vote matters, every vote counts, and it's your one chance to have your
voice heard,'' she said. ``And it's now something that I really feel has
been taken away from me, and my husband as well.''

Ms. Gleeson said she filed a request to vote by absentee ballot but
never received one.

``They're delivering refrigerated trucks out to Milwaukee because they
expect the death toll to be so high,'' she said. ``But they also expect
us to go out and vote.''

Representative Mark Pocan, a Democrat who represents Madison, said the
electoral troubles that have been on display in the state the last few
weeks should serve as a catalyst for national changes.

``We should be the poster child for a national vote by mail program in
November,'' Mr. Pocan said. ``We cannot risk having this confusion
headed into the national election.''

Astead W. Herndon reported from Milwaukee, and Alexander Burns from New
York. Reid J. Epstein contributed reporting from Washington, D.C., and
Nick Corasaniti from New York.

\hypertarget{our-2020-election-guide}{%
\section{Our 2020 Election Guide}\label{our-2020-election-guide}}

Updated ~Sept. 7, 2020

\begin{center}\rule{0.5\linewidth}{\linethickness}\end{center}

\begin{itemize}
\item ~
  \hypertarget{the-latest}{%
  \subsection{The Latest}\label{the-latest}}

  \begin{itemize}
  \item
    The unofficial Labor Day kickoff to the fall presidential campaign
    centered on Pennsylvania and Wisconsin,
    \href{https://www.nytimes3xbfgragh.onion/2020/09/07/us/politics/wisconsin-biden-harris-trump-pence.html?action=click\&pgtype=Article\&state=default\&region=BELOW_MAIN_CONTENT\&context=storylines_guide}{two
    pivotal states for both President Trump and Joseph R. Biden Jr}.
  \end{itemize}
\item ~
  \hypertarget{how-to-win-270}{%
  \subsection{How to Win 270}\label{how-to-win-270}}

  \begin{itemize}
  \item
    Joe Biden and Donald Trump need 270 electoral votes to reach the
    White House. Try building
    \href{https://www.nytimes3xbfgragh.onion/interactive/2020/us/elections/election-states-biden-trump.html?action=click\&pgtype=Article\&state=default\&region=BELOW_MAIN_CONTENT\&context=storylines_guide}{your
    own coalition of battleground states}~to see potential outcomes.
  \end{itemize}
\item ~
  \hypertarget{voting-by-mail}{%
  \subsection{Voting by Mail}\label{voting-by-mail}}

  \begin{itemize}
  \item
    Will you have enough time to vote by mail in your state? Yes, but
    it's risky to procrastinate.
    \href{https://www.nytimes3xbfgragh.onion/interactive/2020/08/31/us/politics/vote-by-mail-deadlines.html?action=click\&pgtype=Article\&state=default\&region=BELOW_MAIN_CONTENT\&context=storylines_guide}{Check
    your state's deadline.}
  \item
    \href{https://www.nytimes3xbfgragh.onion/interactive/2020/us/elections/joe-biden.html?action=click\&pgtype=Article\&state=default\&region=BELOW_MAIN_CONTENT\&context=storylines_guide}{}

    \hypertarget{joe-biden}{%
    \section{Joe Biden}\label{joe-biden}}

    \hypertarget{democrat}{%
    \subsection{Democrat}\label{democrat}}

    \href{https://www.nytimes3xbfgragh.onion/interactive/2020/us/elections/donald-trump.html?action=click\&pgtype=Article\&state=default\&region=BELOW_MAIN_CONTENT\&context=storylines_guide}{}

    \hypertarget{donald-trump}{%
    \section{Donald Trump}\label{donald-trump}}

    \hypertarget{republican}{%
    \subsection{Republican}\label{republican}}
  \end{itemize}
\item
  \hypertarget{keep-up-with-our-coverage}{%
  \subsection{Keep Up With Our
  Coverage}\label{keep-up-with-our-coverage}}

  \begin{itemize}
  \item
    Get an
    \href{https://www.nytimes3xbfgragh.onion/newsletters/politics?action=click\&pgtype=Article\&state=default\&region=BELOW_MAIN_CONTENT\&context=storylines_guide}{email}~recapping
    the day's news
  \item
    Download our mobile app on
    \href{https://apps.apple.com/us/app/nytimes/id284862083?ls=1\&mat_click_id=5c79ae7455014fd1bd66b5610c05b8f2-20191112-16948\&referrer=mat_click_id\%3D5c79ae7455014fd1bd66b5610c05b8f2-20191112-16948\%26link_click_id\%3D722930677036718082}{iOS}~and
    \href{http://a.localytics.com/android?id=com.nytimes.android\&referrer=utm_source\%3Dother_nyt_mobile_web\%26utm_medium\%3DWeb\%2520page\%26utm_term\%3DGeneral\%2520Mobile\%2520Page\%26utm_campaign\%3DNYT\%2520Mobile\%2520General\%2520Page}{Android}~and
    turn on Breaking News and Politics alerts
  \end{itemize}
\end{itemize}

Advertisement

\protect\hyperlink{after-bottom}{Continue reading the main story}

\hypertarget{site-index}{%
\subsection{Site Index}\label{site-index}}

\hypertarget{site-information-navigation}{%
\subsection{Site Information
Navigation}\label{site-information-navigation}}

\begin{itemize}
\tightlist
\item
  \href{https://help.nytimes3xbfgragh.onion/hc/en-us/articles/115014792127-Copyright-notice}{©~2020~The
  New York Times Company}
\end{itemize}

\begin{itemize}
\tightlist
\item
  \href{https://www.nytco.com/}{NYTCo}
\item
  \href{https://help.nytimes3xbfgragh.onion/hc/en-us/articles/115015385887-Contact-Us}{Contact
  Us}
\item
  \href{https://www.nytco.com/careers/}{Work with us}
\item
  \href{https://nytmediakit.com/}{Advertise}
\item
  \href{http://www.tbrandstudio.com/}{T Brand Studio}
\item
  \href{https://www.nytimes3xbfgragh.onion/privacy/cookie-policy\#how-do-i-manage-trackers}{Your
  Ad Choices}
\item
  \href{https://www.nytimes3xbfgragh.onion/privacy}{Privacy}
\item
  \href{https://help.nytimes3xbfgragh.onion/hc/en-us/articles/115014893428-Terms-of-service}{Terms
  of Service}
\item
  \href{https://help.nytimes3xbfgragh.onion/hc/en-us/articles/115014893968-Terms-of-sale}{Terms
  of Sale}
\item
  \href{https://spiderbites.nytimes3xbfgragh.onion}{Site Map}
\item
  \href{https://help.nytimes3xbfgragh.onion/hc/en-us}{Help}
\item
  \href{https://www.nytimes3xbfgragh.onion/subscription?campaignId=37WXW}{Subscriptions}
\end{itemize}
