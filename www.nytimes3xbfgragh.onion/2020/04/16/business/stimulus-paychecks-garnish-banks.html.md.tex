Sections

SEARCH

\protect\hyperlink{site-content}{Skip to
content}\protect\hyperlink{site-index}{Skip to site index}

\href{https://www.nytimes3xbfgragh.onion/section/business}{Business}

\href{https://myaccount.nytimes3xbfgragh.onion/auth/login?response_type=cookie\&client_id=vi}{}

\href{https://www.nytimes3xbfgragh.onion/section/todayspaper}{Today's
Paper}

\href{/section/business}{Business}\textbar{}Some Banks Keep Customers'
Stimulus Checks if Accounts Are Overdrawn

\url{https://nyti.ms/3a8nWJC}

\begin{itemize}
\item
\item
\item
\item
\item
\item
\end{itemize}

\hypertarget{the-coronavirus-outbreak}{%
\subsubsection{\texorpdfstring{\href{https://www.nytimes3xbfgragh.onion/news-event/coronavirus?name=styln-coronavirus-markets\&region=TOP_BANNER\&block=storyline_menu_recirc\&action=click\&pgtype=Article\&impression_id=dac9d7b0-f2a8-11ea-8f04-735a2495ff97\&variant=undefined}{The
Coronavirus
Outbreak}}{The Coronavirus Outbreak}}\label{the-coronavirus-outbreak}}

\begin{itemize}
\tightlist
\item
  live\href{https://www.nytimes3xbfgragh.onion/2020/09/09/world/covid-19-coronavirus.html?name=styln-coronavirus-markets\&region=TOP_BANNER\&block=storyline_menu_recirc\&action=click\&pgtype=Article\&impression_id=dac9fec0-f2a8-11ea-8f04-735a2495ff97\&variant=undefined}{Latest
  Updates}
\item
  \href{https://www.nytimes3xbfgragh.onion/interactive/2020/us/coronavirus-us-cases.html?name=styln-coronavirus-markets\&region=TOP_BANNER\&block=storyline_menu_recirc\&action=click\&pgtype=Article\&impression_id=dac9fec1-f2a8-11ea-8f04-735a2495ff97\&variant=undefined}{Maps
  and Cases}
\item
  \href{https://www.nytimes3xbfgragh.onion/interactive/2020/science/coronavirus-vaccine-tracker.html?name=styln-coronavirus-markets\&region=TOP_BANNER\&block=storyline_menu_recirc\&action=click\&pgtype=Article\&impression_id=dac9fec2-f2a8-11ea-8f04-735a2495ff97\&variant=undefined}{Vaccine
  Tracker}
\item
  \href{https://www.nytimes3xbfgragh.onion/2020/09/02/your-money/eviction-moratorium-covid.html?name=styln-coronavirus-markets\&region=TOP_BANNER\&block=storyline_menu_recirc\&action=click\&pgtype=Article\&impression_id=dac9fec3-f2a8-11ea-8f04-735a2495ff97\&variant=undefined}{Eviction
  Moratorium}
\item
  \href{https://www.nytimes3xbfgragh.onion/2020/09/09/upshot/coronavirus-surprise-test-fees.html?name=styln-coronavirus-markets\&region=TOP_BANNER\&block=storyline_menu_recirc\&action=click\&pgtype=Article\&impression_id=dac9fec4-f2a8-11ea-8f04-735a2495ff97\&variant=undefined}{Surprise
  Test Fees}
\end{itemize}

Advertisement

\protect\hyperlink{after-top}{Continue reading the main story}

Supported by

\protect\hyperlink{after-sponsor}{Continue reading the main story}

\hypertarget{some-banks-keep-customers-stimulus-checks-if-accounts-are-overdrawn}{%
\section{Some Banks Keep Customers' Stimulus Checks if Accounts Are
Overdrawn}\label{some-banks-keep-customers-stimulus-checks-if-accounts-are-overdrawn}}

Financial institutions can use the government deposits to make up for
recipients' negative balances.

\includegraphics{https://static01.graylady3jvrrxbe.onion/images/2020/04/17/business/15virus-garnish-print/15virus-garnish-articleLarge.jpg?quality=75\&auto=webp\&disable=upscale}

\href{https://www.nytimes3xbfgragh.onion/by/emily-flitter}{\includegraphics{https://static01.graylady3jvrrxbe.onion/images/2019/06/19/reader-center/author-emily-flitter/author-emily-flitter-thumbLarge.png}}\href{https://www.nytimes3xbfgragh.onion/by/alan-rappeport}{\includegraphics{https://static01.graylady3jvrrxbe.onion/images/2018/06/12/multimedia/author-alan-rappeport/author-alan-rappeport-thumbLarge-v2.png}}

By \href{https://www.nytimes3xbfgragh.onion/by/emily-flitter}{Emily
Flitter} and
\href{https://www.nytimes3xbfgragh.onion/by/alan-rappeport}{Alan
Rappeport}

\begin{itemize}
\item
  Published April 16, 2020Updated April 17, 2020
\item
  \begin{itemize}
  \item
  \item
  \item
  \item
  \item
  \item
  \end{itemize}
\end{itemize}

For some struggling Americans, the arrival of a deposit from the
Treasury Department to help with basic expenses like rent and groceries
during the coronavirus crisis was something to count on --- until their
financial institutions got in the way.

Frustrated customers say banks have been seizing some, or all, of their
relief payments because their accounts are overdrawn, in some cases as a
result of pandemic-caused hardship.

Joseph James Davis Jr. said his bank in Mena, Ark., took more than
\$2,000 after he fell victim to a check-cashing scam in a moment of
desperation.

``I've never been scammed before,'' said Mr. Davis, 41.

Mr. Davis said the work for his landscaping business had dried up
because of the coronavirus crisis, so he responded to an online ad that
promised payment to anyone who would agree to put advertising decals on
a vehicle.

Mr. Davis was sent a check, but was told he had to pay for the decals
with some of the money. After he sent off the payment, his bank, Union
Bank of Mena, told him the check had been bogus and he had to repay it
\$2,784.

He couldn't. And on Wednesday, Mr. Davis saw the \$3,400 relief payment
--- \$2,400 for himself and his wife and \$500 for each of his two
stepchildren --- land in his bank account. The bank kept all but \$611.

Kevin Williams, president of Union Bank of Mena, did not respond to
calls and emails seeking comment on Thursday.

\hypertarget{latest-updates-the-coronavirus-outbreak-and-the-economy}{%
\section{\texorpdfstring{\href{https://www.nytimes3xbfgragh.onion/live/2020/09/09/business/stock-market-today-coronavirus?action=click\&pgtype=Article\&state=default\&region=MAIN_CONTENT_1\&context=storylines_live_updates}{Latest
Updates: The Coronavirus Outbreak and the
Economy}}{Latest Updates: The Coronavirus Outbreak and the Economy}}\label{latest-updates-the-coronavirus-outbreak-and-the-economy}}

\href{https://www.nytimes3xbfgragh.onion/live/2020/09/09/business/stock-market-today-coronavirus?action=click\&pgtype=Article\&state=default\&region=MAIN_CONTENT_1\&context=storylines_live_updates\#amazon-announces-another-hiring-spree}{1h
ago}

\href{https://www.nytimes3xbfgragh.onion/live/2020/09/09/business/stock-market-today-coronavirus?action=click\&pgtype=Article\&state=default\&region=MAIN_CONTENT_1\&context=storylines_live_updates\#amazon-announces-another-hiring-spree}{Amazon
announces another hiring spree.}

\href{https://www.nytimes3xbfgragh.onion/live/2020/09/09/business/stock-market-today-coronavirus?action=click\&pgtype=Article\&state=default\&region=MAIN_CONTENT_1\&context=storylines_live_updates\#why-a-licensing-expert-and-a-mall-operator-bought-brooks-brothers-forever-21-and-others}{4h
ago}

\href{https://www.nytimes3xbfgragh.onion/live/2020/09/09/business/stock-market-today-coronavirus?action=click\&pgtype=Article\&state=default\&region=MAIN_CONTENT_1\&context=storylines_live_updates\#why-a-licensing-expert-and-a-mall-operator-bought-brooks-brothers-forever-21-and-others}{Why
a licensing expert and a mall operator bought Brooks Brothers, Forever
21 and others.}

\href{https://www.nytimes3xbfgragh.onion/live/2020/09/09/business/stock-market-today-coronavirus?action=click\&pgtype=Article\&state=default\&region=MAIN_CONTENT_1\&context=storylines_live_updates\#lvmh-says-it-is-pulling-out-of-its-16-billion-takeover-of-tiffany}{4h
ago}

\href{https://www.nytimes3xbfgragh.onion/live/2020/09/09/business/stock-market-today-coronavirus?action=click\&pgtype=Article\&state=default\&region=MAIN_CONTENT_1\&context=storylines_live_updates\#lvmh-says-it-is-pulling-out-of-its-16-billion-takeover-of-tiffany}{LVMH
says it is pulling out of its \$16 billion takeover of Tiffany.}

\href{https://www.nytimes3xbfgragh.onion/live/2020/09/09/business/stock-market-today-coronavirus?action=click\&pgtype=Article\&state=default\&region=MAIN_CONTENT_1\&context=storylines_live_updates}{See
more updates}

More live coverage:
\href{https://www.nytimes3xbfgragh.onion/2020/09/09/world/covid-19-coronavirus.html?action=click\&pgtype=Article\&state=default\&region=MAIN_CONTENT_1\&context=storylines_live_updates}{Global}

The phenomenon is swiftly becoming a political issue, with Treasury
Secretary Steven Mnuchin fielding calls from senators urging him to
ensure that relief money isn't garnished. Banks are legally allowed to
withhold funds that go into accounts that have negative balances, and no
specific provision in the CARES Act, the \$2 trillion relief package
that authorized the stimulus payments, prevents banks from taking
customers' stimulus money to cover debts.

USAA, a financial services company that serves members of the military
and their families, was also garnishing stimulus payments before
reversing itself on Thursday.

A Minneapolis couple with a USAA account --- a disabled veteran and his
wife --- were anxiously awaiting their relief payments, the wife said.
She and her young family had just moved into their own apartment after
living with their extended family while they struggled to get out from
under thousands of dollars of debt.

But the woman, who did not want to be identified by name out of concern
that her financial troubles could harm the careers of family members,
had to quit her job after being unable to find child care when some
Minnesota day care centers closed because of the virus. She had been
counting on a relief payment to help pay rent and buy formula for her
10-month-old daughter.

But USAA told the couple that it was keeping the money because their
account was overdrawn.

The woman showed The New York Times screenshots of a Twitter exchange
between her husband and a USAA representative. Using USAA's verified
Twitter account, the representative explained that if the family's bank
account had a negative balance, ``any deposits to the account will go
toward the negative amount owed to the bank.''

After this article was published on Thursday, USAA said it would pause
overdraft collections for the next 90 days.

``This will allow members access to their full stimulus payment to help
cover the costs of rent, food and other important necessities,'' Matthew
Hartwig, a bank spokesman, said in an email. ``Beginning as early as
today, we will apply this policy retroactively to any member accounts
with a negative balance at the time the first stimulus checks were
deposited, so that members will have access to their stimulus funds.''

The government checks are meant to cushion the pandemic's financial blow
to some of the hardest-hit Americans. Anyone who earns up to \$75,000 in
adjusted gross annual income and has a Social Security number will
receive \$1,200. Married couples who file joint tax returns will receive
\$2,400 if their adjusted gross income is under \$150,000. The amount
declines for those who make more.

In a March 2018 survey, the Pew Charitable Trusts, a nonpartisan
research institute,
\href{https://www.pewtrusts.org/en/research-and-analysis/articles/2018/03/21/millions-use-bank-overdrafts-as-credit}{found}
that more than 39 million Americans had incurred
\href{https://www.nytimes3xbfgragh.onion/2020/06/03/business/banks-overdraft-fees.html}{overdraft
fees} within the past year, with people essentially using overdrafting
as credit.

Several politicians are calling for banks to stop garnishing stimulus
payments. On Wednesday, Senators Elizabeth Warren of Massachusetts and
Sherrod Brown of Ohio, both Democrats, implored the head of a bank trade
group to tell its members to halt the practice.

``For weeks, we have pressed the Treasury Department to exercise its
authority and ensure that Americans receive the full amount of their
stimulus payments,'' the senators wrote in a letter to Rob Nichols, the
chief executive of the American Bankers Association. ``While Treasury
has refused to follow congressional intent, that does not give banks
license to steal the stimulus payments from their customers.''

By contrast, the CARES Act specifically prohibits garnishing stimulus
money for state or federal debts, except for court-mandated child
support.

Not every bank is keeping its overdrawn customers' money. Bank of
America, JPMorgan Chase, Citibank and Wells Fargo --- the nation's four
biggest banks --- are pausing their collections on negative account
balances to give customers access to the stimulus.

``We are temporarily crediting the overdrawn amount for customers,
giving them full access to their stimulus payment,'' said Anne Pace, a
spokeswoman for Chase, in an email to The Times on Wednesday. ``We hope
this gives them a chance to catch their breath.''

On Monday, a group of 25 state attorneys general also registered their
disapproval of garnishing relief checks. ``During this public health and
economic crisis, the states do not believe that the billions of dollars
appropriated by Congress to help keep hardworking Americans afloat
should be subject to garnishment,'' they wrote in a letter to Mr.
Mnuchin.

\href{https://prospect.org/coronavirus/banks-can-grab-stimulus-check-pay-debts/}{The
American Prospect this week unearthed}an audio recording of a Treasury
official discussing with banks how stimulus money should be handled when
there are outstanding loans or other debts. The official, Ronda Kent,
said that ``there's nothing in the law that precludes that action'' and
that it was up to the discretion of the banks.

Progressive watchdog groups have seized on the issue to criticize the
Trump administration's handling of the economic crisis, describing it as
a giveaway for banks.

``This money should be going toward food, rent and medicine --- it's not
the time to hand out favors to debt-collection industry donors or pad
some big bank's bottom line,'' said Jeremy Funk, a spokesman for Allied
Progress. ``Secretary Mnuchin needs to ensure that these \$1,200 checks
go straight into Americans' pockets, where they belong.''

The Treasury Department had no comment.

Another hopeful stimulus recipient described having to fight for hours
with her credit union on Wednesday before it would release the full
\$2,400 deposit. Initially, the institution, Digital Credit Union, which
is based in Marlborough, Mass., kept \$1,000 to make up for the
customer's overdrawn account balance.

The customer did not want to be identified because she was worried that
the lender would close her accounts or penalize her for speaking
publicly. She and her husband have four children. His hours at a group
home for children were recently cut to three days a week, she said. She
is out of work.

She said that after multiple calls, a representative had agreed to
return the \$1,000 to her. She said she was sharing her story because
she was worried that other people would not have the stamina to fight
for the money the way she had.

Edward Niser, a spokesman for Digital Credit Union, said in an email
that the institution could not comment on individuals, citing privacy
reasons*.*

*``*In these difficult times,'' he said, ``we are there to support our
members and we are making every possible effort to follow evolving
federal and state guidance.''

Advertisement

\protect\hyperlink{after-bottom}{Continue reading the main story}

\hypertarget{site-index}{%
\subsection{Site Index}\label{site-index}}

\hypertarget{site-information-navigation}{%
\subsection{Site Information
Navigation}\label{site-information-navigation}}

\begin{itemize}
\tightlist
\item
  \href{https://help.nytimes3xbfgragh.onion/hc/en-us/articles/115014792127-Copyright-notice}{©~2020~The
  New York Times Company}
\end{itemize}

\begin{itemize}
\tightlist
\item
  \href{https://www.nytco.com/}{NYTCo}
\item
  \href{https://help.nytimes3xbfgragh.onion/hc/en-us/articles/115015385887-Contact-Us}{Contact
  Us}
\item
  \href{https://www.nytco.com/careers/}{Work with us}
\item
  \href{https://nytmediakit.com/}{Advertise}
\item
  \href{http://www.tbrandstudio.com/}{T Brand Studio}
\item
  \href{https://www.nytimes3xbfgragh.onion/privacy/cookie-policy\#how-do-i-manage-trackers}{Your
  Ad Choices}
\item
  \href{https://www.nytimes3xbfgragh.onion/privacy}{Privacy}
\item
  \href{https://help.nytimes3xbfgragh.onion/hc/en-us/articles/115014893428-Terms-of-service}{Terms
  of Service}
\item
  \href{https://help.nytimes3xbfgragh.onion/hc/en-us/articles/115014893968-Terms-of-sale}{Terms
  of Sale}
\item
  \href{https://spiderbites.nytimes3xbfgragh.onion}{Site Map}
\item
  \href{https://help.nytimes3xbfgragh.onion/hc/en-us}{Help}
\item
  \href{https://www.nytimes3xbfgragh.onion/subscription?campaignId=37WXW}{Subscriptions}
\end{itemize}
