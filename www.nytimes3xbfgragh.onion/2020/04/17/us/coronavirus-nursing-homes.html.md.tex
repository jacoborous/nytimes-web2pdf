Sections

SEARCH

\protect\hyperlink{site-content}{Skip to
content}\protect\hyperlink{site-index}{Skip to site index}

\href{https://www.nytimes3xbfgragh.onion/section/us}{U.S.}

\href{https://myaccount.nytimes3xbfgragh.onion/auth/login?response_type=cookie\&client_id=vi}{}

\href{https://www.nytimes3xbfgragh.onion/section/todayspaper}{Today's
Paper}

\href{/section/us}{U.S.}\textbar{}`They're Death Pits': Virus Claims at
Least 7,000 Lives in U.S. Nursing Homes

\url{https://nyti.ms/3epAdwJ}

\begin{itemize}
\item
\item
\item
\item
\item
\item
\end{itemize}

\hypertarget{the-coronavirus-outbreak}{%
\subsubsection{\texorpdfstring{\href{https://www.nytimes3xbfgragh.onion/news-event/coronavirus?name=styln-coronavirus-national\&region=TOP_BANNER\&block=storyline_menu_recirc\&action=click\&pgtype=Article\&impression_id=23aaa1e0-f2bc-11ea-9197-0b8458576cb9\&variant=undefined}{The
Coronavirus
Outbreak}}{The Coronavirus Outbreak}}\label{the-coronavirus-outbreak}}

\begin{itemize}
\tightlist
\item
  live\href{https://www.nytimes3xbfgragh.onion/2020/09/09/world/covid-19-coronavirus.html?name=styln-coronavirus-national\&region=TOP_BANNER\&block=storyline_menu_recirc\&action=click\&pgtype=Article\&impression_id=23aac8f0-f2bc-11ea-9197-0b8458576cb9\&variant=undefined}{Latest
  Updates}
\item
  \href{https://www.nytimes3xbfgragh.onion/interactive/2020/us/coronavirus-us-cases.html?name=styln-coronavirus-national\&region=TOP_BANNER\&block=storyline_menu_recirc\&action=click\&pgtype=Article\&impression_id=23aaf000-f2bc-11ea-9197-0b8458576cb9\&variant=undefined}{Maps
  and Cases}
\item
  \href{https://www.nytimes3xbfgragh.onion/interactive/2020/science/coronavirus-vaccine-tracker.html?name=styln-coronavirus-national\&region=TOP_BANNER\&block=storyline_menu_recirc\&action=click\&pgtype=Article\&impression_id=23aaf001-f2bc-11ea-9197-0b8458576cb9\&variant=undefined}{Vaccine
  Tracker}
\item
  \href{https://www.nytimes3xbfgragh.onion/2020/09/02/your-money/eviction-moratorium-covid.html?name=styln-coronavirus-national\&region=TOP_BANNER\&block=storyline_menu_recirc\&action=click\&pgtype=Article\&impression_id=23aaf002-f2bc-11ea-9197-0b8458576cb9\&variant=undefined}{Eviction
  Moratorium}
\item
  \href{https://www.nytimes3xbfgragh.onion/2020/09/09/upshot/coronavirus-surprise-test-fees.html?name=styln-coronavirus-national\&region=TOP_BANNER\&block=storyline_menu_recirc\&action=click\&pgtype=Article\&impression_id=23aaf003-f2bc-11ea-9197-0b8458576cb9\&variant=undefined}{Surprise
  Test Fees}
\end{itemize}

Advertisement

\protect\hyperlink{after-top}{Continue reading the main story}

Supported by

\protect\hyperlink{after-sponsor}{Continue reading the main story}

\hypertarget{theyre-death-pits-virus-claims-at-least-7000-lives-in-us-nursing-homes}{%
\section{`They're Death Pits': Virus Claims at Least 7,000 Lives in U.S.
Nursing
Homes}\label{theyre-death-pits-virus-claims-at-least-7000-lives-in-us-nursing-homes}}

More than six weeks after the first coronavirus deaths in a nursing
home, outbreaks unfold across the country. About a fifth of U.S. virus
deaths are linked to nursing facilities.

\includegraphics{https://static01.graylady3jvrrxbe.onion/images/2020/04/16/us/00virus-nursing-magnolia2/merlin_171409395_a3f3797a-5a18-4d91-b427-e326157dfeea-articleLarge.jpg?quality=75\&auto=webp\&disable=upscale}

By \href{https://www.nytimes3xbfgragh.onion/by/farah-stockman}{Farah
Stockman},
\href{https://www.nytimes3xbfgragh.onion/by/matt-richtel}{Matt Richtel},
\href{https://www.nytimes3xbfgragh.onion/by/danielle-ivory}{Danielle
Ivory} and
\href{https://www.nytimes3xbfgragh.onion/by/mitch-smith}{Mitch Smith}

\begin{itemize}
\item
  Published April 17, 2020Updated May 13, 2020
\item
  \begin{itemize}
  \item
  \item
  \item
  \item
  \item
  \item
  \end{itemize}
\end{itemize}

The first warning of the devastation that the
\href{https://www.nytimes3xbfgragh.onion/2020/05/07/business/coronavirus-nursing-homes.html}{coronavirus}
could wreak inside American
\href{https://www.nytimes3xbfgragh.onion/2020/05/07/business/coronavirus-nursing-homes.html}{nursing
homes} came in late February, when residents of a
\href{https://www.nytimes3xbfgragh.onion/2020/03/21/us/coronavirus-nursing-home-kirkland-life-care.html}{facility
in suburban Seattle} perished, one by one, as families waited helplessly
outside.

In the ensuing six weeks, large and shockingly lethal outbreaks have
continued to ravage nursing homes across the nation, undeterred by
urgent new safety requirements. Now a nationwide tally by The New York
Times has found the number of people living in or connected to
\href{https://www.nytimes3xbfgragh.onion/2020/05/13/nyregion/nursing-homes-coronavirus-new-york.html}{nursing
homes} who have died of the coronavirus to be at least 7,000, far higher
than previously known.

In New Jersey, 17 bodies piled up in a nursing home morgue, and more
than a quarter of a Virginia home's residents have died. At least 24
people at a facility in Maryland have died; more than 100 residents and
workers have been infected at another in Kansas; and people have died in
centers for military veterans in Florida, Nevada, New York, Maine,
Massachusetts, Oregon and Washington.

On Friday, New York officials for the first time disclosed the names of
72 long-term care facilities that have had five or more deaths,
including the Cobble Hill Health Center in Brooklyn where 55 people have
died. At least 14 nursing homes in New York City and its suburbs
\href{https://www.nytimes3xbfgragh.onion/2020/04/17/nyregion/new-york-nursing-homes-coronavirus-deaths.html}{have
recorded more than 25 coronavirus-related deaths}. **** In New Jersey,
officials revealed that infections have broken out in 394 long-term
facilities --- almost two-thirds of the state's homes --- and that more
than 1,500 deaths were tied to nursing facilities.

Overall, about a fifth of deaths from the virus in the United States
have been tied to nursing homes or other long-term care facilities, the
Times review of cases shows. And more than 36,500 residents and
employees across the nation have contracted it.

In interviews with more than two dozen workers in long-term care
facilities as well as family members of residents and health care
experts, a portrait emerged of a system unequipped to handle the
onslaught and disintegrating further amid the growing crisis.

``They're death pits,'' said Betsy McCaughey, a former lieutenant
governor of New York who founded the Committee to Reduce Infection
Deaths, an education campaign aimed at stopping hospital-acquired
infections. ``These nursing homes are already overwhelmed. They're
crowded and they're understaffed. One Covid-positive patient in a
nursing home produces carnage.''

\includegraphics{https://static01.graylady3jvrrxbe.onion/images/2020/03/31/autossell/Video-Thumb/Video-Thumb-videoSixteenByNineJumbo1600.jpg}

It is a tragedy that is continuing to unfold, and one that even the dire
figures that are known only partially capture. The number of cases at
these facilities, which include nursing homes, assisted-living
facilities, memory care facilities, retirement and senior communities
and long-term rehabilitation facilities, is almost certainly still
higher since many facilities, counties and states have not provided
detailed information. The outbreaks have been spread across the
sprawling senior care industry, including at publicly run facilities,
those run by nonprofit groups and others managed by large corporations.
Some nursing homes with clusters have a history of safety violations,
persistent staffing problems and limited amenities. Other hard-hit
facilities have sterling health records, luxurious living arrangements
and pricey rents.

\hypertarget{latest-updates-the-coronavirus-outbreak}{%
\section{\texorpdfstring{\href{https://www.nytimes3xbfgragh.onion/2020/09/09/world/covid-19-coronavirus.html?action=click\&pgtype=Article\&state=default\&region=MAIN_CONTENT_1\&context=storylines_live_updates}{Latest
Updates: The Coronavirus
Outbreak}}{Latest Updates: The Coronavirus Outbreak}}\label{latest-updates-the-coronavirus-outbreak}}

Updated 2020-09-09T16:45:22.130Z

\begin{itemize}
\tightlist
\item
  \href{https://www.nytimes3xbfgragh.onion/2020/09/09/world/covid-19-coronavirus.html?action=click\&pgtype=Article\&state=default\&region=MAIN_CONTENT_1\&context=storylines_live_updates\#link-279e24e2}{Top
  U.S. health officials update Congress on vaccine development and
  distribution plans.}
\item
  \href{https://www.nytimes3xbfgragh.onion/2020/09/09/world/covid-19-coronavirus.html?action=click\&pgtype=Article\&state=default\&region=MAIN_CONTENT_1\&context=storylines_live_updates\#link-5b0bf0d1}{As
  drugmakers pledge to thoroughly vet vaccines, one company pauses its
  trials for a safety review.}
\item
  \href{https://www.nytimes3xbfgragh.onion/2020/09/09/world/covid-19-coronavirus.html?action=click\&pgtype=Article\&state=default\&region=MAIN_CONTENT_1\&context=storylines_live_updates\#link-58edc4cb}{Britain
  bans gatherings of more than six people.}
\end{itemize}

\href{https://www.nytimes3xbfgragh.onion/2020/09/09/world/covid-19-coronavirus.html?action=click\&pgtype=Article\&state=default\&region=MAIN_CONTENT_1\&context=storylines_live_updates}{See
more updates}

More live coverage:
\href{https://www.nytimes3xbfgragh.onion/live/2020/09/09/business/stock-market-today-coronavirus?action=click\&pgtype=Article\&state=default\&region=MAIN_CONTENT_1\&context=storylines_live_updates}{Markets}

The virus is known to be more deadly to aging, immune-compromised
people, and small, confined settings like nursing homes, where workers
frequently move from one room to the next, are particularly vulnerable
to spreading infection. But oversights and failures also have
contributed to the crisis.

Virus tests and protective gear have been scarce inside many of these
facilities, which are among the most overlooked players in the health
care system. These homes, with staff members who receive less extensive
training than those in hospitals, tend to struggle to slow infectious
diseases. Employees are often poorly paid workers who move between
multiple jobs and return home to communities at risk of contracting the
virus.

All of these factors have allowed the virus to thrive, making its way
into at least 4,100 American nursing homes and other long-term care
facilities, despite increasingly desperate efforts to stop the spread.

\includegraphics{https://static01.graylady3jvrrxbe.onion/images/2020/04/16/us/00virus-nursing-lifecare/merlin_170377002_d773c76a-d044-4599-8bfc-8a262624cd0e-articleLarge.jpg?quality=75\&auto=webp\&disable=upscale}

Facilities were late to require workers and residents to wear masks ---
and some were still not enforcing such policies, workers and family
members said. Facing shortages of tests and of masks, homes often
waited, they said, until residents were showing symptoms of Covid-19
before testing them for the virus and isolating them from others, even
if they had contact with people who had been infected.

``The residents and staff are being led to slaughter,'' said Judith
Regan, an editor and publishing executive whose 91-year-old father, Leo
Regan, lives at the Long Island State Veterans Home at Stony Brook
University. At least 57 residents and 37 staff members there have tested
positive for the virus, and 32 residents have died. ``He is on the
Titanic, but there are no lifeboats,'' Ms. Regan said. Officials at the
veterans home did not respond to requests for comment.

Employees at some facilities have stopped coming to work. In California,
83 patients with the virus had to be evacuated from a nursing facility
in Riverside County after only one of 13 scheduled certified nursing
assistants appeared at work, public health officials said. Sixteen
employees and dozens of patients had tested positive days earlier.

Even now, protective gear is in short supply at many homes. One nursing
assistant at a Detroit nursing facility said she had been issued an N95
mask but had to make it last three weeks. With no gowns available, she
said she and her co-workers were being told to suit up in the same gowns
that patients sleep in.

In Miami, Rosa Mercedes, a certified nursing assistant at a residential
facility, waited in line in her car for a coronavirus test on Thursday
at the Hard Rock Stadium. She said her facility, which she declined to
name, provides her with one mask each day as she feeds, bathes and helps
multiple patients use the bathroom.

Now she has a cough and sore throat. ``I don't know if I have it or
don't have it,'' she said. ``Everybody's living in a nightmare.''

Nursing home industry officials acknowledged this week that many of
their facilities were in crisis and said they lacked the protective
equipment and testing that hospitals have received.

``We don't have what we need to stop this,'' said Mark Parkinson, the
president and chief executive of the American Health Care Association
and the National Center for Assisted Living, a trade organization that
represents skilled nursing facilities and assisted living homes that
house more than a million people. ``We have got to have masks, and we
don't have masks.''

Mr. Parkinson said that federal health authorities have designated
nursing homes and long-term care facilities at a lower priority level
than hospitals, meaning longer turnaround times for test results --- a
significant problem for slowing spread.

He said that many employees do not have the option of isolating
themselves from sick patients.

``The cavalry hasn't arrived,'' Mr. Parkinson said. ``People will end up
blaming nursing homes and talking about how terrible we are, but it is
the complete lack of prioritization that has put us in the position that
we are in.''

Nursing home facilities have borne the brunt of a structural shift:
Hospitals, seeking to keep costs down, send more vulnerable patients
into a growing industry of nursing homes. Even before the pandemic,
380,000 people died each year from infection at long-term care
facilities, according to the Centers for Disease Control \& Prevention.

The Centers for Medicare and Medicaid Services, which regulates the
nation's more than 15,000 nursing homes, issued
\href{https://www.cms.gov/files/document/qso-20-14-nh-revised.pdf}{new
guidance} last month, telling administrators to restrict all visitors,
cancel group activities, shut down dining rooms and screen all residents
and staff members for fevers and respiratory illnesses.

Families of nursing home residents said some of the new restrictions
were enforced unevenly in some homes; unlocked front doors in some
places, for instance, have failed to stop visits. And even where
enforcement has been strict, staff members could still unwittingly carry
the virus inside.

Image

Adrina Rodriguez, left, talked with a nurse through a window as she
visited her father at the Gateway Care and Rehabilitation Center Tuesday
in Hayward, Calif.Credit...Justin Sullivan/Getty Images

At Canterbury Rehabilitation and Healthcare Center in Richmond, Va.,
where many residents rely on Medicaid to cover costs, triple rooms are
not uncommon. The facility has struggled to hire and retain nursing
staff. As recently as October, federal investigators found nearly two
dozen
\href{https://www.medicare.gov/nursinghomecompare/previousInspections.html?ID=495272\&Inspn=HEALTH\&profTab=1\&Distn=15.2\&loc=RICHMOND\%2C\%20VA\&lat=37.5407246\&lng=-77.4360481\&name=CANTERBURY\%20REHABILITATION\%20\%26\%20HEALTH\%20CARE\%20CENTER}{deficiencies}
at the facility, including a lack of appropriate respiratory care for a
resident and a lack of appropriate ulcer care.

When a few cases of the virus cropped up at Canterbury in mid-March, the
state had only a few hundred test kits available, limiting officials'
ability to figure out how far it had spread, according to Dr. Danny
Avula, the local county's health official. By the time officials were
able to test everyone a few weeks later, more than 60 residents tested
positive, including some who did not show symptoms. At least 46 of the
facility's estimated 160 residents have died from the virus, making it
one of the deadliest clusters in the country.

``Nursing homes were not designed to deal with this kind of crisis,''
Dr. Avula said.

Even in the best of times, the facilities struggle to retain staff, and
families of residents of some facilities heaped praise on staff members
for risking their own lives to keep working.The situation has led
anxious families to agonize about whether to try to bring their loved
ones home. But many cannot provide the extensive medical care that is
required and fear exposing others to the virus. As they wrestle with
what to do, many say that they are being given little information about
what is happening inside the homes.

``It's totally horrifying --- I can't even describe the feeling,'' said
Adam Zimmerman, whose 77-year-old mother lives in an acute-care facility
near Los Angeles where cases of the virus have been identified. He said
he speaks by phone regularly with his mother, who has a tracheotomy and
medical conditions, but has not been able to visit her in weeks.

Image

Residents from St. Joseph's Senior Home were helped onto buses in
Woodbridge, N.J., last month after some residents tested positive for
coronavirus.Credit...Seth Wenig/Associated Press

At a facility in New York City, there had been no indication of a
problem, a grandson of a resident said, until last week when he received
a call that his grandfather, who has Alzheimer's, was gravely ill. He
was coughing hard and had a fever.

By Sunday, he was dead. Only later was the family told that the man had
tested positive for the coronavirus, according to the grandson, who
wanted to be identified only by his first name, Andrew, because his
grandmother is still in the nursing home and has yet to be told of her
husband's death. She, too, has tested positive for the virus, he said.

He said it broke his heart that his grandfather died alone. ``I couldn't
even hold his hand,'' he said. ``He couldn't speak English, and he was
just surrounded by strangers. I can't imagine how scared he must have
been.''

Some facilities have found creative ways to combat the virus. At the
Park Springs Life Plan Community in Stone Mountain, Ga., four staff
members and one resident have tested positive for the virus, but they
have fully recovered. The facility decided to take a rare step: It asked
staff members to volunteer to live on the campus to avoid inadvertently
carrying the virus into the facility from home. Sixty workers
volunteered. Ginger Hansborough, the facility's accounting director, who
normally lives with a partner and his octogenarian mother, moved in, not
only to protect residents at the facility, but also to protect her
family.

``I didn't want to be the reason that anything happened to them,'' she
said.

Reporting was contributed by Simon Romero, Vanessa Swales, Jack Healy,
John Leland, Alison Saldanha, Karen Yourish, Sarah Almukhtar and Timothy
Williams.

Advertisement

\protect\hyperlink{after-bottom}{Continue reading the main story}

\hypertarget{site-index}{%
\subsection{Site Index}\label{site-index}}

\hypertarget{site-information-navigation}{%
\subsection{Site Information
Navigation}\label{site-information-navigation}}

\begin{itemize}
\tightlist
\item
  \href{https://help.nytimes3xbfgragh.onion/hc/en-us/articles/115014792127-Copyright-notice}{©~2020~The
  New York Times Company}
\end{itemize}

\begin{itemize}
\tightlist
\item
  \href{https://www.nytco.com/}{NYTCo}
\item
  \href{https://help.nytimes3xbfgragh.onion/hc/en-us/articles/115015385887-Contact-Us}{Contact
  Us}
\item
  \href{https://www.nytco.com/careers/}{Work with us}
\item
  \href{https://nytmediakit.com/}{Advertise}
\item
  \href{http://www.tbrandstudio.com/}{T Brand Studio}
\item
  \href{https://www.nytimes3xbfgragh.onion/privacy/cookie-policy\#how-do-i-manage-trackers}{Your
  Ad Choices}
\item
  \href{https://www.nytimes3xbfgragh.onion/privacy}{Privacy}
\item
  \href{https://help.nytimes3xbfgragh.onion/hc/en-us/articles/115014893428-Terms-of-service}{Terms
  of Service}
\item
  \href{https://help.nytimes3xbfgragh.onion/hc/en-us/articles/115014893968-Terms-of-sale}{Terms
  of Sale}
\item
  \href{https://spiderbites.nytimes3xbfgragh.onion}{Site Map}
\item
  \href{https://help.nytimes3xbfgragh.onion/hc/en-us}{Help}
\item
  \href{https://www.nytimes3xbfgragh.onion/subscription?campaignId=37WXW}{Subscriptions}
\end{itemize}
