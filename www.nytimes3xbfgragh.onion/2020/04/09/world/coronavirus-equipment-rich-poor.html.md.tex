Sections

SEARCH

\protect\hyperlink{site-content}{Skip to
content}\protect\hyperlink{site-index}{Skip to site index}

\href{https://www.nytimes3xbfgragh.onion/section/world}{World}

\href{https://myaccount.nytimes3xbfgragh.onion/auth/login?response_type=cookie\&client_id=vi}{}

\href{https://www.nytimes3xbfgragh.onion/section/todayspaper}{Today's
Paper}

\href{/section/world}{World}\textbar{}In Scramble for Coronavirus
Supplies, Rich Countries Push Poor Aside

\url{https://nyti.ms/3e7tyXP}

\begin{itemize}
\item
\item
\item
\item
\item
\item
\end{itemize}

\hypertarget{the-coronavirus-outbreak}{%
\subsubsection{\texorpdfstring{\href{https://www.nytimes3xbfgragh.onion/news-event/coronavirus?name=styln-coronavirus-national\&region=TOP_BANNER\&block=storyline_menu_recirc\&action=click\&pgtype=Article\&impression_id=97e055b0-f1f3-11ea-861d-9f9fbf97bb3c\&variant=undefined}{The
Coronavirus
Outbreak}}{The Coronavirus Outbreak}}\label{the-coronavirus-outbreak}}

\begin{itemize}
\tightlist
\item
  live\href{https://www.nytimes3xbfgragh.onion/2020/09/08/world/covid-19-coronavirus.html?name=styln-coronavirus-national\&region=TOP_BANNER\&block=storyline_menu_recirc\&action=click\&pgtype=Article\&impression_id=97e055b1-f1f3-11ea-861d-9f9fbf97bb3c\&variant=undefined}{Latest
  Updates}
\item
  \href{https://www.nytimes3xbfgragh.onion/interactive/2020/us/coronavirus-us-cases.html?name=styln-coronavirus-national\&region=TOP_BANNER\&block=storyline_menu_recirc\&action=click\&pgtype=Article\&impression_id=97e055b2-f1f3-11ea-861d-9f9fbf97bb3c\&variant=undefined}{Maps
  and Cases}
\item
  \href{https://www.nytimes3xbfgragh.onion/interactive/2020/science/coronavirus-vaccine-tracker.html?name=styln-coronavirus-national\&region=TOP_BANNER\&block=storyline_menu_recirc\&action=click\&pgtype=Article\&impression_id=97e07cc0-f1f3-11ea-861d-9f9fbf97bb3c\&variant=undefined}{Vaccine
  Tracker}
\item
  \href{https://www.nytimes3xbfgragh.onion/2020/09/02/your-money/eviction-moratorium-covid.html?name=styln-coronavirus-national\&region=TOP_BANNER\&block=storyline_menu_recirc\&action=click\&pgtype=Article\&impression_id=97e07cc1-f1f3-11ea-861d-9f9fbf97bb3c\&variant=undefined}{Eviction
  Moratorium}
\item
  \href{https://www.nytimes3xbfgragh.onion/interactive/2020/09/02/magazine/food-insecurity-hunger-us.html?name=styln-coronavirus-national\&region=TOP_BANNER\&block=storyline_menu_recirc\&action=click\&pgtype=Article\&impression_id=97e07cc2-f1f3-11ea-861d-9f9fbf97bb3c\&variant=undefined}{American
  Hunger}
\end{itemize}

Advertisement

\protect\hyperlink{after-top}{Continue reading the main story}

Supported by

\protect\hyperlink{after-sponsor}{Continue reading the main story}

\hypertarget{in-scramble-for-coronavirus-supplies-rich-countries-push-poor-aside}{%
\section{In Scramble for Coronavirus Supplies, Rich Countries Push Poor
Aside}\label{in-scramble-for-coronavirus-supplies-rich-countries-push-poor-aside}}

Developing nations in Latin America and Africa cannot find enough
materials and equipment to test for coronavirus, partly because the
United States and Europe are outspending them.

\includegraphics{https://static01.graylady3jvrrxbe.onion/images/2020/04/10/world/10virus-richpoor-p11/merlin_171377658_4d42f00b-5fcb-4145-a974-acab11b2a9ae-articleLarge.jpg?quality=75\&auto=webp\&disable=upscale}

\href{https://www.nytimes3xbfgragh.onion/by/jane-bradley}{\includegraphics{https://static01.graylady3jvrrxbe.onion/images/2020/03/04/reader-center/author-jane-bradley/author-jane-bradley-thumbLarge.png}}

By \href{https://www.nytimes3xbfgragh.onion/by/jane-bradley}{Jane
Bradley}

\begin{itemize}
\item
  April 9, 2020
\item
  \begin{itemize}
  \item
  \item
  \item
  \item
  \item
  \item
  \end{itemize}
\end{itemize}

\href{https://www.nytimes3xbfgragh.onion/es/2020/04/09/espanol/coronavirus-paises-desarrollo.html}{Leer
en español}

Crates of masks snatched from cargo planes on airport tarmacs. Countries
paying triple the market price to outbid others.
\href{https://www.nytimes3xbfgragh.onion/2020/04/06/business/economy/peter-navarro-coronavirus-defense-production-act.html}{Accusations
of ``modern piracy''} against governments trying to secure medical
supplies for their own people.

As the United States and European Union countries compete to acquire
scarce medical equipment to combat the coronavirus, another troubling
divide is also emerging, with poorer countries losing out to wealthier
ones in the global scrum for masks and testing materials.

Scientists in Africa and Latin America have been told by manufacturers
that orders for vital testing kits cannot be filled for months, because
the supply chain is in upheaval and almost everything they produce is
going to America or Europe. All countries report steep price increases,
from testing kits to masks.

The huge global demand for masks, alongside new distortions in the
private market, has forced some developing countries to turn to UNICEF
for help. Etleva Kadilli, who oversees supplies at the agency, said it
was trying to buy 240 million masks to help 100 countries but so far had
managed to source only around 28 million.

``There is a war going on behind the scenes, and we're most worried
about poorer countries losing out,'' said Dr. Catharina Boehme, the
chief executive of Foundation for Innovative New Diagnostics, which
\href{https://www.who.int/news-room/detail/10-02-2020-who-and-find-formalize-strategic-collaboration-to-drive-universal-access-to-essential-diagnostics}{collaborates}
with the World Health Organization in helping poorer countries gain
access to medical tests.

In Africa, Latin America and parts of Asia, many countries are already
at a disadvantage, with health systems that are underfunded, fragile and
often lacking in necessary equipment. A
\href{https://www.ncbi.nlm.nih.gov/pmc/articles/PMC4305307/}{recent
study} found that some poor countries have only one equipped intensive
care bed per million residents.

\includegraphics{https://static01.graylady3jvrrxbe.onion/images/2020/04/08/world/00virus-richpoor2/merlin_171271215_439023cd-915c-4496-aa6b-de5d41304075-articleLarge.jpg?quality=75\&auto=webp\&disable=upscale}

So far, the developing world has reported far fewer cases and deaths
from the coronavirus, but many experts fear that the pandemic could be
especially devastating for the poorest countries.

Testing is the first defense against the virus and an important tool to
stop so many patients from ending up hospitalized. Most manufacturers
want to help, but the niche industry that produces the testing equipment
and chemical reagents necessary to process lab tests is dealing with
huge global demand.

``There's never really been a shortage of chemical reagents before
now,'' said Doris-Ann Williams, chief executive of the British In Vitro
Diagnostics Association, which represents producers and distributors of
the lab tests used to detect coronavirus. ``If it was just one country
with an epidemic it would be fine, but all the major countries in the
world are wanting the same thing at the same time.''

\hypertarget{latest-updates-the-coronavirus-outbreak}{%
\section{\texorpdfstring{\href{https://www.nytimes3xbfgragh.onion/2020/09/08/world/covid-19-coronavirus.html?action=click\&pgtype=Article\&state=default\&region=MAIN_CONTENT_1\&context=storylines_live_updates}{Latest
Updates: The Coronavirus
Outbreak}}{Latest Updates: The Coronavirus Outbreak}}\label{latest-updates-the-coronavirus-outbreak}}

Updated 2020-09-08T16:13:48.390Z

\begin{itemize}
\tightlist
\item
  \href{https://www.nytimes3xbfgragh.onion/2020/09/08/world/covid-19-coronavirus.html?action=click\&pgtype=Article\&state=default\&region=MAIN_CONTENT_1\&context=storylines_live_updates\#link-679303d7}{Nine
  drugmakers pledge to thoroughly vet any coronavirus vaccine.}
\item
  \href{https://www.nytimes3xbfgragh.onion/2020/09/08/world/covid-19-coronavirus.html?action=click\&pgtype=Article\&state=default\&region=MAIN_CONTENT_1\&context=storylines_live_updates\#link-547feae1}{Senate
  Republicans plan to move forward with a scaled-back stimulus package.}
\item
  \href{https://www.nytimes3xbfgragh.onion/2020/09/08/world/covid-19-coronavirus.html?action=click\&pgtype=Article\&state=default\&region=MAIN_CONTENT_1\&context=storylines_live_updates\#link-1c973131}{`The
  lockdown killed my father': Farmer suicides add to India's virus
  misery.}
\end{itemize}

\href{https://www.nytimes3xbfgragh.onion/2020/09/08/world/covid-19-coronavirus.html?action=click\&pgtype=Article\&state=default\&region=MAIN_CONTENT_1\&context=storylines_live_updates}{See
more updates}

More live coverage:
\href{https://www.nytimes3xbfgragh.onion/live/2020/09/08/business/stock-market-today-coronavirus?action=click\&pgtype=Article\&state=default\&region=MAIN_CONTENT_1\&context=storylines_live_updates}{Markets}

For poorer countries, Dr. Boehme said the competition for resources is
potentially a ``global catastrophe,'' as a once-coherent supply chain
has rapidly devolved into an arm-twisting exercise. Leaders of ``every
country'' are personally calling manufacturing chief executives to
demand first-in-line access to vital supplies. Some governments have
even offered to send private jets.

In Brazil, Amilcar Tanuri cannot offer private jets. Dr. Tanuri runs
public laboratories at the Federal University of Rio de Janeiro, half of
which are ``stuck doing nothing,'' instead of testing health workers,
because he said the chemical reagents he needs are being routed to
wealthier countries.

``If you don't have reliable tests, you are blind,'' he said. ``This is
the beginning of the epidemic curve so I'm very concerned about the
public health system here being overwhelmed very fast.''

Brazil is Latin America's hardest hit country so far, with more than
\href{https://www.ecdc.europa.eu/en/publications-data/download-todays-data-geographic-distribution-covid-19-cases-worldwide}{10,000
confirmed cases} and a
\href{https://brazilian.report/coronavirus-brazil-live-blog/2020/04/02/over-23000-still-brazilians-await-covid-19-test-results/}{testing
backlog of at least 23,000}. It is also the region's most controversial
player in the pandemic, with a president, Jair Bolsonaro, who has been
an outspoken skeptic of the risks posed by the coronavirus.

Image

Wearing masks in the Kibera slum in Nairobi, Kenya.Credit...Tyler
Hicks/The New York Times

But below the political noise, the country's scientists began trying to
ramp up testing hours after the country's first case was announced.

Yet within weeks, Dr. Tanuri was left to frantically call private firms
on three continents, trying to source the chemical reagents needed for
the 200 testing samples his labs receive every day --- only to be told
that the United States and Europe had already bought up months of
production.

``If we purchase something to arrive in 60 days, it's too late,'' he
said. ``The virus goes faster than we can go.''

The situation is similar for some African countries.

After reporting its
\href{https://www.nytimes3xbfgragh.onion/2020/03/27/world/africa/south-africa-coronavirus.html}{first
death} on March 27, South Africa moved swiftly, introducing a strict
lockdown and announcing ambitious house-to-house canvassing that has
already seen 47,000 people tested. South Africa has
\href{https://nationalgovernment.co.za/units/view/251/national-health-laboratory-service-nhls}{more
than 200 public labs}, an impressive network that surpasses wealthier
countries like Britain and was developed in response to past outbreaks
of H.I.V. and tuberculosis.

But, like Brazil, it is reliant on international manufacturers for the
chemical reagents, and other equipment, needed to process the tests. Dr.
Francois Venter, an infectious diseases expert who is advising the South
African government, said the struggle to acquire the reagents was
endangering the country's overall response.

``We have the capacity to do large testing, but we've been bedeviled by
the fact the actual testing materials, reagents, haven't been coming,''
he said. ``We're not as wealthy. We don't have as many ventilators, we
don't have as many doctors, our health system was in a precarious
position before coronavirus.''

``The country is terrified,'' he added.

To address the problem, South Africa's National Health Laboratory
Services has set up a ``war room'' of around 20 people who are
continuously calling different suppliers --- yet running into problems
sourcing the test kits and protective equipment they need.

Image

The Institut Pasteur in Dakar, Senegal, is partnering with a British
company seeking to develop a home coronavirus
test.~Credit...Seyllou/Agence France-Presse --- Getty Images

``The suppliers are basically saying their production output does not
meet the needs,'' said Dr. Kamy Chetty, the director of the agency.
``They are working flat out.''

Experts say that the industry that produces test kits is quite small.
Ms. Williams, the industry representative in Britain, said there was no
shortage of chemical reagents but that delays were arising in the
production process, including the necessary checks and approvals,
because the huge demand was overwhelming the system.

\href{https://www.nytimes3xbfgragh.onion/news-event/coronavirus?action=click\&pgtype=Article\&state=default\&region=MAIN_CONTENT_3\&context=storylines_faq}{}

\hypertarget{the-coronavirus-outbreak-}{%
\subsubsection{The Coronavirus Outbreak
›}\label{the-coronavirus-outbreak-}}

\hypertarget{frequently-asked-questions}{%
\paragraph{Frequently Asked
Questions}\label{frequently-asked-questions}}

Updated September 4, 2020

\begin{itemize}
\item ~
  \hypertarget{what-are-the-symptoms-of-coronavirus}{%
  \paragraph{What are the symptoms of
  coronavirus?}\label{what-are-the-symptoms-of-coronavirus}}

  \begin{itemize}
  \tightlist
  \item
    In the beginning, the coronavirus
    \href{https://www.nytimes3xbfgragh.onion/article/coronavirus-facts-history.html?action=click\&pgtype=Article\&state=default\&region=MAIN_CONTENT_3\&context=storylines_faq\#link-6817bab5}{seemed
    like it was primarily a respiratory illness}~--- many patients had
    fever and chills, were weak and tired, and coughed a lot, though
    some people don't show many symptoms at all. Those who seemed
    sickest had pneumonia or acute respiratory distress syndrome and
    received supplemental oxygen. By now, doctors have identified many
    more symptoms and syndromes. In April,
    \href{https://www.nytimes3xbfgragh.onion/2020/04/27/health/coronavirus-symptoms-cdc.html?action=click\&pgtype=Article\&state=default\&region=MAIN_CONTENT_3\&context=storylines_faq}{the
    C.D.C. added to the list of early signs}~sore throat, fever, chills
    and muscle aches. Gastrointestinal upset, such as diarrhea and
    nausea, has also been observed. Another telltale sign of infection
    may be a sudden, profound diminution of one's
    \href{https://www.nytimes3xbfgragh.onion/2020/03/22/health/coronavirus-symptoms-smell-taste.html?action=click\&pgtype=Article\&state=default\&region=MAIN_CONTENT_3\&context=storylines_faq}{sense
    of smell and taste.}~Teenagers and young adults in some cases have
    developed painful red and purple lesions on their fingers and toes
    --- nicknamed ``Covid toe'' --- but few other serious symptoms.
  \end{itemize}
\item ~
  \hypertarget{why-is-it-safer-to-spend-time-together-outside}{%
  \paragraph{Why is it safer to spend time together
  outside?}\label{why-is-it-safer-to-spend-time-together-outside}}

  \begin{itemize}
  \tightlist
  \item
    \href{https://www.nytimes3xbfgragh.onion/2020/05/15/us/coronavirus-what-to-do-outside.html?action=click\&pgtype=Article\&state=default\&region=MAIN_CONTENT_3\&context=storylines_faq}{Outdoor
    gatherings}~lower risk because wind disperses viral droplets, and
    sunlight can kill some of the virus. Open spaces prevent the virus
    from building up in concentrated amounts and being inhaled, which
    can happen when infected people exhale in a confined space for long
    stretches of time, said Dr. Julian W. Tang, a virologist at the
    University of Leicester.
  \end{itemize}
\item ~
  \hypertarget{why-does-standing-six-feet-away-from-others-help}{%
  \paragraph{Why does standing six feet away from others
  help?}\label{why-does-standing-six-feet-away-from-others-help}}

  \begin{itemize}
  \tightlist
  \item
    The coronavirus spreads primarily through droplets from your mouth
    and nose, especially when you cough or sneeze. The C.D.C., one of
    the organizations using that measure,
    \href{https://www.nytimes3xbfgragh.onion/2020/04/14/health/coronavirus-six-feet.html?action=click\&pgtype=Article\&state=default\&region=MAIN_CONTENT_3\&context=storylines_faq}{bases
    its recommendation of six feet}~on the idea that most large droplets
    that people expel when they cough or sneeze will fall to the ground
    within six feet. But six feet has never been a magic number that
    guarantees complete protection. Sneezes, for instance, can launch
    droplets a lot farther than six feet,
    \href{https://jamanetwork.com/journals/jama/fullarticle/2763852}{according
    to a recent study}. It's a rule of thumb: You should be safest
    standing six feet apart outside, especially when it's windy. But
    keep a mask on at all times, even when you think you're far enough
    apart.
  \end{itemize}
\item ~
  \hypertarget{i-have-antibodies-am-i-now-immune}{%
  \paragraph{I have antibodies. Am I now
  immune?}\label{i-have-antibodies-am-i-now-immune}}

  \begin{itemize}
  \tightlist
  \item
    As of right
    now,\href{https://www.nytimes3xbfgragh.onion/2020/07/22/health/covid-antibodies-herd-immunity.html?action=click\&pgtype=Article\&state=default\&region=MAIN_CONTENT_3\&context=storylines_faq}{~that
    seems likely, for at least several months.}~There have been
    frightening accounts of people suffering what seems to be a second
    bout of Covid-19. But experts say these patients may have a
    drawn-out course of infection, with the virus taking a slow toll
    weeks to months after initial exposure.~People infected with the
    coronavirus typically
    \href{https://www.nature.com/articles/s41586-020-2456-9}{produce}~immune
    molecules called antibodies, which are
    \href{https://www.nytimes3xbfgragh.onion/2020/05/07/health/coronavirus-antibody-prevalence.html?action=click\&pgtype=Article\&state=default\&region=MAIN_CONTENT_3\&context=storylines_faq}{protective
    proteins made in response to an
    infection}\href{https://www.nytimes3xbfgragh.onion/2020/05/07/health/coronavirus-antibody-prevalence.html?action=click\&pgtype=Article\&state=default\&region=MAIN_CONTENT_3\&context=storylines_faq}{.
    These antibodies may}~last in the body
    \href{https://www.nature.com/articles/s41591-020-0965-6}{only two to
    three months}, which may seem worrisome, but that's~perfectly normal
    after an acute infection subsides, said Dr. Michael Mina, an
    immunologist at Harvard University. It may be possible to get the
    coronavirus again, but it's highly unlikely that it would be
    possible in a short window of time from initial infection or make
    people sicker the second time.
  \end{itemize}
\item ~
  \hypertarget{what-are-my-rights-if-i-am-worried-about-going-back-to-work}{%
  \paragraph{What are my rights if I am worried about going back to
  work?}\label{what-are-my-rights-if-i-am-worried-about-going-back-to-work}}

  \begin{itemize}
  \tightlist
  \item
    Employers have to provide
    \href{https://www.osha.gov/SLTC/covid-19/standards.html}{a safe
    workplace}~with policies that protect everyone equally.
    \href{https://www.nytimes3xbfgragh.onion/article/coronavirus-money-unemployment.html?action=click\&pgtype=Article\&state=default\&region=MAIN_CONTENT_3\&context=storylines_faq}{And
    if one of your co-workers tests positive for the coronavirus, the
    C.D.C.}~has said that
    \href{https://www.cdc.gov/coronavirus/2019-ncov/community/guidance-business-response.html}{employers
    should tell their employees}~-\/- without giving you the sick
    employee's name -\/- that they may have been exposed to the virus.
  \end{itemize}
\end{itemize}

``Manufacturers don't just want to sell to rich countries,'' said Paul
Molinaro, head of supply and logistics for the World Health
Organization. ``They want to diversify, but they've got all this
competing demand from different governments.''

He added: ``When it comes to the sharp end of a hypercompetitive
environment with price rises, these low- and middle-income countries are
going to end up at the back of the queue.''

Last week, President Trump invoked the Defense Production Act to
prohibit the export of face masks to other countries and demand that
American firms increase production of medical supplies.

One American company that makes masks, 3M, responded by warning of
``significant humanitarian implications'' if it stopped supplying masks
to Latin America and Canada. This week, the company and the Trump
administration reached a
\href{https://news.3m.com/blog/3m-stories/3m-and-trump-administration-announce-plan-import-1665-million-additional-respirators}{deal}
that allows 3M to continue exporting to developing countries, while also
providing the United States with 166 million masks over the next few
months.

Last month,
\href{https://ec.europa.eu/commission/presscorner/detail/en/IP_20_469}{Europe}
and
\href{https://uk.reuters.com/article/us-health-coronavirus-china-testkits/china-clamps-down-on-coronavirus-test-kit-exports-after-accuracy-questioned-idUKKBN21J51S}{China}
introduced their own export restrictions on tests and protective
equipment.

Some private firms, however, are putting profit aside to help developing
countries with more fragile health systems.

A British testing manufacturer, Mologic, has received government funding
to develop a 10-minute home coronavirus test in partnership with Senegal
that, if approved, would cost less than \$1 to produce. It would not be
reliant on labs, electricity or sourcing expensive supplies from global
manufacturers.

Mologic agreed to share its technology with Institut Pasteur de Dakar, a
flagship lab in Dakar, to help produce the kit ``at cost.'' While the
goal is to make it widely available, it is predominantly aimed at
slowing the spread of the virus in Africa.

Image

Enforcing the lockdown in Johannesburg.Credit...Marco Longari/Agence
France-Presse --- Getty Images

For poorer countries, the supply problem is bigger than just testing.

Zambia is at the very beginning of its epidemic curve with only
\href{https://af.reuters.com/article/zambiaNews/idAFL8N2BQ3WC}{one
death} so far, but it is already struggling to source masks, as well as
testing materials like swabs and reagents, says Charles Holmes, a board
member of the Centre for Infectious Disease Research in Zambia and the
former chief medical officer for the Obama administration's President's
Emergency Plan for AIDS Relief, known as PEPFAR.

When Zambia tried to place an order for N95 masks, Dr. Holmes said, the
broker tried to sell them for ``five to 10 times'' more than the usual
cost, despite checks revealing the masks expired in 2016.

``It's difficult for countries or governments having those conversations
with manufacturers, when much wealthier countries are having those same
conversations,'' he said. ``The private sector is likely to respond to
the highest bidder for many of these supplies, that's just business.''

He said manufacturers have told Zambian officials that they cannot
guarantee a delivery date for supplies because ``most of them are being
snapped up by the U.S. and Europe.''

While few would criticize governments for looking out for their own
people, health experts believe that it is in everyone's interest to help
poorer countries get the supplies they need.

``An infection with a highly transportable respiratory virus anywhere in
the world puts all countries at risk,'' Dr. Holmes said. ``Wealthy
nations not only have an obligation to look out for countries that are
going to struggle, but they should also have some self interest in
ensuring that the pandemic is contained in developing countries.''

Advertisement

\protect\hyperlink{after-bottom}{Continue reading the main story}

\hypertarget{site-index}{%
\subsection{Site Index}\label{site-index}}

\hypertarget{site-information-navigation}{%
\subsection{Site Information
Navigation}\label{site-information-navigation}}

\begin{itemize}
\tightlist
\item
  \href{https://help.nytimes3xbfgragh.onion/hc/en-us/articles/115014792127-Copyright-notice}{©~2020~The
  New York Times Company}
\end{itemize}

\begin{itemize}
\tightlist
\item
  \href{https://www.nytco.com/}{NYTCo}
\item
  \href{https://help.nytimes3xbfgragh.onion/hc/en-us/articles/115015385887-Contact-Us}{Contact
  Us}
\item
  \href{https://www.nytco.com/careers/}{Work with us}
\item
  \href{https://nytmediakit.com/}{Advertise}
\item
  \href{http://www.tbrandstudio.com/}{T Brand Studio}
\item
  \href{https://www.nytimes3xbfgragh.onion/privacy/cookie-policy\#how-do-i-manage-trackers}{Your
  Ad Choices}
\item
  \href{https://www.nytimes3xbfgragh.onion/privacy}{Privacy}
\item
  \href{https://help.nytimes3xbfgragh.onion/hc/en-us/articles/115014893428-Terms-of-service}{Terms
  of Service}
\item
  \href{https://help.nytimes3xbfgragh.onion/hc/en-us/articles/115014893968-Terms-of-sale}{Terms
  of Sale}
\item
  \href{https://spiderbites.nytimes3xbfgragh.onion}{Site Map}
\item
  \href{https://help.nytimes3xbfgragh.onion/hc/en-us}{Help}
\item
  \href{https://www.nytimes3xbfgragh.onion/subscription?campaignId=37WXW}{Subscriptions}
\end{itemize}
