Sections

SEARCH

\protect\hyperlink{site-content}{Skip to
content}\protect\hyperlink{site-index}{Skip to site index}

\href{https://www.nytimes3xbfgragh.onion/section/travel}{Travel}

\href{https://myaccount.nytimes3xbfgragh.onion/auth/login?response_type=cookie\&client_id=vi}{}

\href{https://www.nytimes3xbfgragh.onion/section/todayspaper}{Today's
Paper}

\href{/section/travel}{Travel}\textbar{}Flying via a Smaller Airport? It
May Take a While

\url{https://nyti.ms/2UVR1DR}

\begin{itemize}
\item
\item
\item
\item
\item
\end{itemize}

Advertisement

\protect\hyperlink{after-top}{Continue reading the main story}

Supported by

\protect\hyperlink{after-sponsor}{Continue reading the main story}

update

\hypertarget{flying-via-a-smaller-airport-it-may-take-a-while}{%
\section{Flying via a Smaller Airport? It May Take a
While}\label{flying-via-a-smaller-airport-it-may-take-a-while}}

Passengers looking to fly to and from regional airports are feeling the
brunt of airline cuts even more than those in bigger hubs, with many
cancellations and circuitous routes.

\includegraphics{https://static01.graylady3jvrrxbe.onion/images/2020/04/19/travel/09-update-virus-regional-airports/merlin_170924610_2ff7b050-dfc9-4f42-8a9a-dd6ff349d766-articleLarge.jpg?quality=75\&auto=webp\&disable=upscale}

By Julie Weed

\begin{itemize}
\item
  April 9, 2020
\item
  \begin{itemize}
  \item
  \item
  \item
  \item
  \item
  \end{itemize}
\end{itemize}

Cathy Cichowski was trying to get from Halifax, Nova Scotia, to San
Antonio, where her job in a hospital involves workplace safety. She came
to Canada to say goodbye to her terminally ill mother but was
desperately needed back at the hospital, where one of her tasks is
helping make sure nurses' masks are appropriately fitted.

Since the third week in March, her return flights were canceled twice
and then rescheduled for April 15, but after five calls and an estimated
seven hours on the phone with United Airlines, she was able to fly out
on April 8, on a route that went from Halifax to Toronto to Houston to
San Antonio, rather than her original one-stop through Newark. It took
her more than 14 hours to get home.

The steep drop in demand caused by the pandemic has the airlines
instituting last-minute cancellations while significantly reducing
future schedules, making it hard for passengers to know if a purchased
ticket will result in an actual flight.
\href{http://news.aa.com/news/news-details/2020/American-Airlines-Announces-Additional-Schedule-Suspensions-in-Response-to-Reduced-Customer-Demand-Related-to-Covid-19-OPS-DIS-03/}{American
Airlines},
\href{https://newsroom.alaskaair.com/2020-03-25-Alaska-Airlines-announces-schedule-reductions-and-other-changes-to-conserve-cash}{Alaska
Airlines},
\href{https://hub.united.com/united-flight-reductions-suspensions-2020-2645514815.html}{United}
and
\href{https://www.sec.gov/Archives/edgar/data/27904/000168316820000862/delta_8k-ex9901.htm}{Delta}
have all announced domestic flight reductions of about 70 percent.

Large airports have seen their number of flights drastically reduced.
United has cut its flights to
\href{https://www.newarkairport.com/}{Newark Liberty International
Airport} from 139 per day serving 62 destinations to 15 flights per day
serving nine destinations. Actual departures from Newark by all airlines
dropped to 99 on April 7 from 564 on March 7, according to
\href{https://nam12.safelinks.protection.outlook.com/?url=https\%3A\%2F\%2Fwww.flightradar24.com\%2F\&data=02\%7C01\%7C\%7Cf14d5395d0404c43711f08d7dbd96beb\%7C84df9e7fe9f640afb435aaaaaaaaaaaa\%7C1\%7C0\%7C637219600188882112\&sdata=YnA3uaivNvfdYOiOlpTmiyS7O\%2F6nldB\%2BFWs2sRQHmKE\%3D\&reserved=0}{Flightradar24},
a global flight tracking service and app.

But as bad as things are at major airports, for small-market cities,
which may have a limited number of carriers, the effect has been
amplified.

On April 5,
\href{https://www.flightradar24.com/data/airports/cho/statistics}{21 of
27 flights} scheduled to depart from the
\href{http://www.gocho.com/flight-info/arrivals-departures/}{Charlottesville,
Va., airport} were canceled, according to Flightradar24. Alaska Airlines
and United Airlines, which began commercial flights to California,
Denver and Las Vegas from \href{https://www.painefield.com/}{Paine Field
Airport} north of Seattle in March 2019, had 38 departures scheduled for
March 8 of this year. By April 5 that was down to 24 scheduled
departures, 13 of which were canceled.

In upstate New York, scheduled departures from the
\href{https://syrairport.org/}{Syracuse Hancock International Airport}
dropped to 18 per day from about 70 a day at the beginning of March. On
a recent day midweek, 12 of those were canceled. (Passengers are
supposed to
\href{https://www.nytimes3xbfgragh.onion/reuters/2020/04/07/business/07reuters-health-coronavirus-airlines-iata.html}{get
refunds when the airlines cancel,} though in some cases they have been
offered credits instead.)

When Sofia Dudas flew from Austin, Texas, to Seattle on April 5, only
one of the three scheduled nonstop flights took off, and even so, she
said there were only about 15 passengers on her flight, which could
accommodate more than 150 people.

As part of the government's Coronavirus Aid, Relief, and Economic
Security (CARES) Act that offers financial support to United States
airlines, the airlines must continue to serve the airports they did
before March 1, as long as it is ``reasonable and practicable,'' but
routing can change, international destinations are not included, and
exceptions are granted. Airlines serving different airports within one
metro area can consolidate service to one of those airports.

The frequency of the required flights is determined by the airlines'
schedule before March. If an airline was providing at least one flight a
day at least five days per week, it now is required to provide at least
one flight per day, five days per week, for that area. If the area was
served fewer than five days per week, the airline needs to service it
once a week. But if too few passengers book tickets on a flight or no
one shows up at the gate, the flight can be canceled.

The airlines can change routing, so cities that may have only recently
gained nonstops are losing them, and passengers hoping to travel from
those cities may have to fly circuitous routes that go through an
airline's hubs.

United changed about
\href{https://hub.united.com/united-flight-reductions-suspensions-2020-2645514815.html}{130
nonstops} to connecting flights through one or even two of its hubs.
Appleton, Wis., a small city in the northeast part of the state, got its
first nonstops to Denver in June 2018. Now, fliers are being offered a
route that goes from Denver south to Atlanta, back north to Chicago, and
then on to Appleton. The short United Airlines flight from Los Angeles
International Airport to Palm Springs will now go through San Francisco
or Denver. Delta's former nonstop between Raleigh-Durham, N.C., and
Seattle now flies through Detroit, Minneapolis or Atlanta.

Some of the suspended nonstop flights may still appear on the companies'
websites, but these will be updated soon.

Kathy Osborne decided on March 30 to leave New York for
\href{https://www.cityftmyers.com/}{Fort Myers, Fla.}, where her parents
own a home, and purchased a ticket for April 4. ``We wanted to get out
of there,'' said Ms. Osborne of her attempt with her fiancé to leave
town. ``It was truly scary because people in our neighborhood didn't
seem to be taking the quarantine seriously at all,'' she said.

Her first reservation with United Airlines was canceled the day after
she purchased it, and the airline said she wouldn't be able to fly for
two weeks. The airport at Fort Myers has dropped to about 85 arrivals
per day, down from about 230 per day, over the last month, according to
Flightradar24.

JetBlue was still flying between New York and Fort Myers, but only once
per day instead of the usual multiple times, and the agent on the line
couldn't guarantee that the flight would not be canceled. But Ms.
Osborne, who needed a day to pack and wrap up her work affairs, booked
it. The flight took off and the pair made it to Fort Myers, wearing
turtlenecks to pull up over their faces.

\begin{center}\rule{0.5\linewidth}{\linethickness}\end{center}

\emph{\textbf{Follow New York Times Travel}} \emph{on}
\href{https://www.instagram.com/nytimestravel/}{\emph{Instagram}}\emph{,}
\href{https://twitter.com/nytimestravel}{\emph{Twitter}} \emph{and}
\href{https://www.facebookcorewwwi.onion/nytimestravel/}{\emph{Facebook}}\emph{.
And}
\href{https://www.nytimes3xbfgragh.onion/newsletters/traveldispatch}{\emph{sign
up for our weekly Travel Dispatch newsletter}} \emph{to receive expert
tips on traveling smarter and inspiration for your next vacation.}

Advertisement

\protect\hyperlink{after-bottom}{Continue reading the main story}

\hypertarget{site-index}{%
\subsection{Site Index}\label{site-index}}

\hypertarget{site-information-navigation}{%
\subsection{Site Information
Navigation}\label{site-information-navigation}}

\begin{itemize}
\tightlist
\item
  \href{https://help.nytimes3xbfgragh.onion/hc/en-us/articles/115014792127-Copyright-notice}{©~2020~The
  New York Times Company}
\end{itemize}

\begin{itemize}
\tightlist
\item
  \href{https://www.nytco.com/}{NYTCo}
\item
  \href{https://help.nytimes3xbfgragh.onion/hc/en-us/articles/115015385887-Contact-Us}{Contact
  Us}
\item
  \href{https://www.nytco.com/careers/}{Work with us}
\item
  \href{https://nytmediakit.com/}{Advertise}
\item
  \href{http://www.tbrandstudio.com/}{T Brand Studio}
\item
  \href{https://www.nytimes3xbfgragh.onion/privacy/cookie-policy\#how-do-i-manage-trackers}{Your
  Ad Choices}
\item
  \href{https://www.nytimes3xbfgragh.onion/privacy}{Privacy}
\item
  \href{https://help.nytimes3xbfgragh.onion/hc/en-us/articles/115014893428-Terms-of-service}{Terms
  of Service}
\item
  \href{https://help.nytimes3xbfgragh.onion/hc/en-us/articles/115014893968-Terms-of-sale}{Terms
  of Sale}
\item
  \href{https://spiderbites.nytimes3xbfgragh.onion}{Site Map}
\item
  \href{https://help.nytimes3xbfgragh.onion/hc/en-us}{Help}
\item
  \href{https://www.nytimes3xbfgragh.onion/subscription?campaignId=37WXW}{Subscriptions}
\end{itemize}
