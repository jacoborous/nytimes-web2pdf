Sections

SEARCH

\protect\hyperlink{site-content}{Skip to
content}\protect\hyperlink{site-index}{Skip to site index}

\href{https://www.nytimes3xbfgragh.onion/section/smarter-living}{Smarter
Living}

\href{https://myaccount.nytimes3xbfgragh.onion/auth/login?response_type=cookie\&client_id=vi}{}

\href{https://www.nytimes3xbfgragh.onion/section/todayspaper}{Today's
Paper}

\href{/section/smarter-living}{Smarter Living}\textbar{}What Is
Convalescent Blood Plasma, and Why Do We Care About It?

\url{https://nyti.ms/3bzSJR9}

\begin{itemize}
\item
\item
\item
\item
\item
\end{itemize}

\hypertarget{the-coronavirus-outbreak}{%
\subsubsection{\texorpdfstring{\href{https://www.nytimes3xbfgragh.onion/news-event/coronavirus?name=styln-coronavirus-national\&region=TOP_BANNER\&block=storyline_menu_recirc\&action=click\&pgtype=Article\&impression_id=d24d9230-f4d2-11ea-bf97-05837a7d0676\&variant=undefined}{The
Coronavirus
Outbreak}}{The Coronavirus Outbreak}}\label{the-coronavirus-outbreak}}

\begin{itemize}
\tightlist
\item
  live\href{https://www.nytimes3xbfgragh.onion/2020/09/11/world/covid-19-coronavirus.html?name=styln-coronavirus-national\&region=TOP_BANNER\&block=storyline_menu_recirc\&action=click\&pgtype=Article\&impression_id=d24d9231-f4d2-11ea-bf97-05837a7d0676\&variant=undefined}{Latest
  Updates}
\item
  \href{https://www.nytimes3xbfgragh.onion/interactive/2020/us/coronavirus-us-cases.html?name=styln-coronavirus-national\&region=TOP_BANNER\&block=storyline_menu_recirc\&action=click\&pgtype=Article\&impression_id=d24db940-f4d2-11ea-bf97-05837a7d0676\&variant=undefined}{Maps
  and Cases}
\item
  \href{https://www.nytimes3xbfgragh.onion/interactive/2020/science/coronavirus-vaccine-tracker.html?name=styln-coronavirus-national\&region=TOP_BANNER\&block=storyline_menu_recirc\&action=click\&pgtype=Article\&impression_id=d24db941-f4d2-11ea-bf97-05837a7d0676\&variant=undefined}{Vaccine
  Tracker}
\item
  \href{https://www.nytimes3xbfgragh.onion/2020/09/10/us/politics/fda-coronavirus-vaccine.html?name=styln-coronavirus-national\&region=TOP_BANNER\&block=storyline_menu_recirc\&action=click\&pgtype=Article\&impression_id=d24db942-f4d2-11ea-bf97-05837a7d0676\&variant=undefined}{F.D.A.
  Regulators' Self-Defense}
\item
  \href{https://www.nytimes3xbfgragh.onion/2020/09/09/upshot/coronavirus-surprise-test-fees.html?name=styln-coronavirus-national\&region=TOP_BANNER\&block=storyline_menu_recirc\&action=click\&pgtype=Article\&impression_id=d24db943-f4d2-11ea-bf97-05837a7d0676\&variant=undefined}{Surprise
  Test Fees}
\end{itemize}

Advertisement

\protect\hyperlink{after-top}{Continue reading the main story}

Supported by

\protect\hyperlink{after-sponsor}{Continue reading the main story}

\hypertarget{what-is-convalescent-blood-plasma-and-why-do-we-care-about-it}{%
\section{What Is Convalescent Blood Plasma, and Why Do We Care About
It?}\label{what-is-convalescent-blood-plasma-and-why-do-we-care-about-it}}

Blood plasma from people who have recovered from Covid-19 may help
others fight the disease.

\includegraphics{https://static01.graylady3jvrrxbe.onion/images/2020/04/22/multimedia/22sl-plasma/22sl-plasma-articleLarge.jpg?quality=75\&auto=webp\&disable=upscale}

\href{https://www.nytimes3xbfgragh.onion/by/tim-herrera}{\includegraphics{https://static01.graylady3jvrrxbe.onion/images/2018/12/07/multimedia/author-tim-herrera/author-tim-herrera-thumbLarge.png}}

By \href{https://www.nytimes3xbfgragh.onion/by/tim-herrera}{Tim Herrera}

\begin{itemize}
\item
  Published April 24, 2020Updated Aug. 23, 2020
\item
  \begin{itemize}
  \item
  \item
  \item
  \item
  \item
  \end{itemize}
\end{itemize}

A medical procedure doctors have used to treat novel diseases for a
century has emerged as a focal point in the fight against
\href{https://www.nytimes3xbfgragh.onion/2020/08/23/us/politics/fda-plasma-coronavirus.html}{Covid-19:
convalescent plasma}.

Conva-what-now\ldots{}?

Convalescent plasma is the term used for plasma that is removed from the
blood of a person who has recovered from a disease, then transfused into
a patient still battling it.

Yes, it sounds a little confusing. So allow us to break down everything
you need to know about convalescent plasma --- and why it matters right
now.

\hypertarget{ok-first-what-exactly-is-plasma}{%
\subsection{OK, first: What, exactly, is
plasma?}\label{ok-first-what-exactly-is-plasma}}

Let's start with a little biology lesson. I promise it'll be fun.

Plasma is the liquid part of your blood. It's light yellow, and made up
of about 91 percent to 92 percent water. It accounts for around 55
percent of your blood, with the other 45 percent being red blood cells,
white blood cells and platelets. (Platelets help your blood clot.)

Here's why we're talking about it now: When your body is exposed to a
foreign pathogen, your body's response is to produce antibodies, ``which
are proteins that can bind to the virus and help to deactivate it, clear
it from circulation and prevent it from invading the body's cells,''
according to Dr. Jeffrey Jhang, medical director of clinical
laboratories and transfusion services for the Mount Sinai Health System
in New York.

Those antibodies --- your internal army working to vanquish foreign
invaders --- are contained in plasma. And once you've recovered, or
\emph{convalesced,} from a given virus, those antibodies stick around in
your plasma for a certain amount of time, ready to fight that virus if
it comes back. That length of time varies, and each virus requires its
own antibodies, meaning that SARS antibodies, for example, are powerless
to stop MERS.

But here's the thing: Your soldiers can sometimes fight for other
people. Doctors can extract convalescent plasma from a recovered
patient, then transfuse it into a patient who is fighting the disease
you recovered from. This means your antibodies may help that person's
own immune system in its war against the disease by accelerating the
time it takes to develop its own army of antibodies.

\hypertarget{latest-updates-the-coronavirus-outbreak}{%
\section{\texorpdfstring{\href{https://www.nytimes3xbfgragh.onion/2020/09/11/world/covid-19-coronavirus.html?action=click\&pgtype=Article\&state=default\&region=MAIN_CONTENT_1\&context=storylines_live_updates}{Latest
Updates: The Coronavirus
Outbreak}}{Latest Updates: The Coronavirus Outbreak}}\label{latest-updates-the-coronavirus-outbreak}}

Updated 2020-09-12T07:09:04.082Z

\begin{itemize}
\tightlist
\item
  \href{https://www.nytimes3xbfgragh.onion/2020/09/11/world/covid-19-coronavirus.html?action=click\&pgtype=Article\&state=default\&region=MAIN_CONTENT_1\&context=storylines_live_updates\#link-dfb8a16}{Fauci
  cautions the virus could disrupt life in the U.S. until `maybe even
  towards the end of 2021.'}
\item
  \href{https://www.nytimes3xbfgragh.onion/2020/09/11/world/covid-19-coronavirus.html?action=click\&pgtype=Article\&state=default\&region=MAIN_CONTENT_1\&context=storylines_live_updates\#link-7104d154}{From
  Asia to Africa, China promotes its vaccine candidates to win friends.}
\item
  \href{https://www.nytimes3xbfgragh.onion/2020/09/11/world/covid-19-coronavirus.html?action=click\&pgtype=Article\&state=default\&region=MAIN_CONTENT_1\&context=storylines_live_updates\#link-393ad215}{The
  other way the virus will kill: hunger.}
\end{itemize}

\href{https://www.nytimes3xbfgragh.onion/2020/09/11/world/covid-19-coronavirus.html?action=click\&pgtype=Article\&state=default\&region=MAIN_CONTENT_1\&context=storylines_live_updates}{See
more updates}

More live coverage:
\href{https://www.nytimes3xbfgragh.onion/live/2020/09/11/business/stock-market-today-coronavirus?action=click\&pgtype=Article\&state=default\&region=MAIN_CONTENT_1\&context=storylines_live_updates}{Markets}

However, before we get too excited, it's important to note that when it
comes to Covid-19, we don't actually know yet if our antibody soldiers
\emph{can} fight for other people. More on that below.

\hypertarget{wait-thats-crazy-has-this-worked-before}{%
\subsection{Wait, that's crazy. Has this worked
before?}\label{wait-thats-crazy-has-this-worked-before}}

Yes! Doctors have been using convalescent plasma transfusions to help
patients fight diseases as far back as the Spanish Flu of 1918. More
recently, the procedure has been used in patients with SARS, Ebola, H1N1
and more.

The name for this therapy is \emph{passive immunity}. When you develop
your own antibodies --- say, through a vaccine --- that's active
immunity. But when you ``borrow'' them from another person via
convalescent plasma, it's passive. This has been used when no other
treatment options are available, and
\href{https://www.jci.org/articles/view/138003}{studies have suggested}
that it can help improve the condition of patients still suffering from
various diseases, including H1N1 and SARS.

``Convalescent plasma has historically been used therapeutically and for
prophylaxis'' --- as prevention --- ``typically in times when a new
disease, virus, bacteria comes on the scene and we don't have any
viral-specific therapies for that new or novel disease,'' said Dr. Erin
Goodhue, executive medical director of the American Red Cross.

What is particularly noteworthy about using convalescent plasma to treat
Covid-19 is that the treatment has never been used as widely as is now
being proposed, Dr. Goodhue said. This means that, for the first time,
the medical and scientific communities will be able to conduct the type
of rigorous studies of the procedure itself to better determine its
effectiveness. That type of study hasn't been possible in previous uses
of the treatment.

It is also used to treat burn, trauma and cancer patients ---
\href{https://www.nybc.org/media/nybloodcenter/filer_public/74/2a/742a70e5-bca4-4390-b279-58063bc8a61a/nybc_about_donation_flyers_plasma.pdf}{a
plasma explainer} from the New York Blood Center calls it ``liquid
gold."

\hypertarget{so--does-it-work-for-covid-19}{%
\subsection{So \ldots{} does it work for
Covid-19?}\label{so--does-it-work-for-covid-19}}

Remember when I said not to get too excited yet?

As we learned from our biology lesson earlier, once a person recovers
from Covid-19, his or her blood --- more specifically, his or her
\emph{plasma} --- contains antibodies that can fight the virus. Since
this coronavirus is novel --- no one had been exposed to it before this
outbreak --- our bodies don't already contain the antibodies needed to
fight it.

The medical community is racing to determine whether convalescent plasma
will help those still battling Covid-19. But it's simply too soon to
know conclusively whether it works --- the Food and Drug Administration
only announced its initiative to collect convalescent plasma about a
month ago. Some hospitals got a small jump on that, but it will likely
be months before we have a definitive answer, experts said.

``I've heard anecdotes of some patients improving, but I don't have any
results of any kind of rigorous nature to be able to share,'' Dr. Jhang
said.

Dr. Goodhue echoed Dr. Jhang's caution in drawing conclusions, but said
there are encouraging signs.

``I'm still continuing to hear positive anecdotal evidence,'' Dr.
Goodhue said. ``But to balance that out, there have been some cases
where it just hasn't seemed to work for a patient at all,
unfortunately.''

Still, more convalescent plasma donations means more opportunities to
study its effectiveness. As of Wednesday, about 35,000 people have
reached out to Mount Sinai, in New York, to see if they are candidates;
about 6,000 have been screened for antibodies; and more than 1,000
high-antibody producers have been identified. Mount Sinai has given
plasma to more than 150 people and counting, according to a spokeswoman.

The Red Cross has ``distributed a couple hundred convalescent plasma
products, and is projected to collect and process hundreds more this
week,'' according to a spokeswoman. It will be collecting donations at
more than 170 locations nationwide. Thousands of potential donors have
reached out to the organization, but less than 10 percent have met the
F.D.A.'s eligibility requirements for donation.

\hypertarget{is-the-donation-process-painful-is-it-safe}{%
\subsection{Is the donation process painful? Is it
safe?}\label{is-the-donation-process-painful-is-it-safe}}

That's an easy one: no, and yes --- so long as you're OK with the prick
of a needle.

The process takes between an hour-and-a-half and two-and-a-half hours.
Though it's somewhat similar to a normal blood donation --- particularly
the screening process during which donors are asked questions about
their health history --- there are significant differences.

In a typical blood donation, a phlebotomist inserts a needle into a vein
in your arm. That needle is connected to a hose, and that hose is
connected to a bag. Your heart does the work here and pumps your blood
into that bag, with the usual donation amount being about a pint.

A plasma donation, however, is like a closed-loop system.

It starts the same as a regular blood donation, with a phlebotomist
inserting a needle into an arm vein. But instead of your blood being
pumped into a bag, it is drawn into a machine with a centrifuge, where
the plasma is separated from the blood. The plasma is collected in a
separate bag, and the blood that was drawn is returned to your body with
a saline solution through the needle already in your arm. This process
repeats a few times until the proper amount of plasma is collected,
which varies based on weight. Side-effects are mostly similar to a
normal blood donation, so you might feel a little faint or dizzy, but in
most cases that's about it. (Your body will replenish the lost plasma in
a day or two.)

A single donation can result in two to four units of plasma, each of
which can be transfused into an ill patient. Exactly what happens to
your plasma after donation depends on where you donate, but it will
undergo tests to identify transmittable diseases,
\href{https://www.ncbi.nlm.nih.gov/books/NBK531504/}{per F.D.A.
regulations}. Eventually it will end up being used to treat Covid-19
patients.

\hypertarget{ive-recovered-from-covid-19-how-can-i-donate}{%
\subsection{I've recovered from Covid-19. How can I
donate?}\label{ive-recovered-from-covid-19-how-can-i-donate}}

To qualify, donors must pass normal blood-donation requirements and be
symptom-free of Covid-19 for at least 14 days, and, in most cases, must
have positive results from a test. (Other restrictions may apply,
depending on the organization.)

Many health care institutions nationwide are involved in plasma
donations, including the Red Cross, so to find a location near you go to
the website for the \href{https://ccpp19.org/}{National Covid-19
Convalescent Plasma Project} or visit the
\href{https://www.redcrossblood.org/donate-blood/dlp/plasma-donations-from-recovered-covid-19-patients.html}{Red
Cross's website}.

\hypertarget{i-had-covid-19-symptoms-and-recovered-but-was-never-tested-can-i-donate}{%
\subsection{I had Covid-19 symptoms and recovered, but was never tested.
Can I
donate?}\label{i-had-covid-19-symptoms-and-recovered-but-was-never-tested-can-i-donate}}

Unfortunately, probably not. The F.D.A.'s eligibility criteria include a
positive diagnosis of Covid-19, along with being symptom-free for at
least 28 days before donation; or at least 14 days without symptoms
\emph{and} a negative Covid-19 test.

However, a regular blood donation is still an option. Check the
\href{http://www.aabb.org/tm/donation/Pages/Blood-Bank-Locator.aspx}{A.A.B.B.
(formerly the American Association of Blood Banks) locator}, visit the
\href{https://www.redcrossblood.org/}{Red Cross website} or call (800)
RED-CROSS. You can also find information through the
\href{https://americasblood.org/for-donors/}{America's Blood Centers}
website, or call (202) 393-5725.

Advertisement

\protect\hyperlink{after-bottom}{Continue reading the main story}

\hypertarget{site-index}{%
\subsection{Site Index}\label{site-index}}

\hypertarget{site-information-navigation}{%
\subsection{Site Information
Navigation}\label{site-information-navigation}}

\begin{itemize}
\tightlist
\item
  \href{https://help.nytimes3xbfgragh.onion/hc/en-us/articles/115014792127-Copyright-notice}{©~2020~The
  New York Times Company}
\end{itemize}

\begin{itemize}
\tightlist
\item
  \href{https://www.nytco.com/}{NYTCo}
\item
  \href{https://help.nytimes3xbfgragh.onion/hc/en-us/articles/115015385887-Contact-Us}{Contact
  Us}
\item
  \href{https://www.nytco.com/careers/}{Work with us}
\item
  \href{https://nytmediakit.com/}{Advertise}
\item
  \href{http://www.tbrandstudio.com/}{T Brand Studio}
\item
  \href{https://www.nytimes3xbfgragh.onion/privacy/cookie-policy\#how-do-i-manage-trackers}{Your
  Ad Choices}
\item
  \href{https://www.nytimes3xbfgragh.onion/privacy}{Privacy}
\item
  \href{https://help.nytimes3xbfgragh.onion/hc/en-us/articles/115014893428-Terms-of-service}{Terms
  of Service}
\item
  \href{https://help.nytimes3xbfgragh.onion/hc/en-us/articles/115014893968-Terms-of-sale}{Terms
  of Sale}
\item
  \href{https://spiderbites.nytimes3xbfgragh.onion}{Site Map}
\item
  \href{https://help.nytimes3xbfgragh.onion/hc/en-us}{Help}
\item
  \href{https://www.nytimes3xbfgragh.onion/subscription?campaignId=37WXW}{Subscriptions}
\end{itemize}
