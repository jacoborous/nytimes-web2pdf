Sections

SEARCH

\protect\hyperlink{site-content}{Skip to
content}\protect\hyperlink{site-index}{Skip to site index}

\href{https://www.nytimes3xbfgragh.onion/section/well/live}{Live}

\href{https://myaccount.nytimes3xbfgragh.onion/auth/login?response_type=cookie\&client_id=vi}{}

\href{https://www.nytimes3xbfgragh.onion/section/todayspaper}{Today's
Paper}

\href{/section/well/live}{Live}\textbar{}What's a Pulse Oximeter, and Do
I Really Need One at Home?

\url{https://nyti.ms/2Kxenti}

\begin{itemize}
\item
\item
\item
\item
\item
\item
\end{itemize}

\hypertarget{the-coronavirus-outbreak}{%
\subsubsection{\texorpdfstring{\href{https://www.nytimes3xbfgragh.onion/news-event/coronavirus?name=styln-coronavirus-national\&region=TOP_BANNER\&block=storyline_menu_recirc\&action=click\&pgtype=Article\&impression_id=85d5cd50-f4b9-11ea-a9ee-334f178d1f9e\&variant=undefined}{The
Coronavirus
Outbreak}}{The Coronavirus Outbreak}}\label{the-coronavirus-outbreak}}

\begin{itemize}
\tightlist
\item
  live\href{https://www.nytimes3xbfgragh.onion/2020/09/11/world/covid-19-coronavirus.html?name=styln-coronavirus-national\&region=TOP_BANNER\&block=storyline_menu_recirc\&action=click\&pgtype=Article\&impression_id=85d5cd51-f4b9-11ea-a9ee-334f178d1f9e\&variant=undefined}{Latest
  Updates}
\item
  \href{https://www.nytimes3xbfgragh.onion/interactive/2020/us/coronavirus-us-cases.html?name=styln-coronavirus-national\&region=TOP_BANNER\&block=storyline_menu_recirc\&action=click\&pgtype=Article\&impression_id=85d5cd52-f4b9-11ea-a9ee-334f178d1f9e\&variant=undefined}{Maps
  and Cases}
\item
  \href{https://www.nytimes3xbfgragh.onion/interactive/2020/science/coronavirus-vaccine-tracker.html?name=styln-coronavirus-national\&region=TOP_BANNER\&block=storyline_menu_recirc\&action=click\&pgtype=Article\&impression_id=85d5f460-f4b9-11ea-a9ee-334f178d1f9e\&variant=undefined}{Vaccine
  Tracker}
\item
  \href{https://www.nytimes3xbfgragh.onion/2020/09/10/us/politics/fda-coronavirus-vaccine.html?name=styln-coronavirus-national\&region=TOP_BANNER\&block=storyline_menu_recirc\&action=click\&pgtype=Article\&impression_id=85d5f461-f4b9-11ea-a9ee-334f178d1f9e\&variant=undefined}{F.D.A.
  Regulators' Self-Defense}
\item
  \href{https://www.nytimes3xbfgragh.onion/2020/09/09/upshot/coronavirus-surprise-test-fees.html?name=styln-coronavirus-national\&region=TOP_BANNER\&block=storyline_menu_recirc\&action=click\&pgtype=Article\&impression_id=85d5f462-f4b9-11ea-a9ee-334f178d1f9e\&variant=undefined}{Surprise
  Test Fees}
\end{itemize}

Advertisement

\protect\hyperlink{after-top}{Continue reading the main story}

Supported by

\protect\hyperlink{after-sponsor}{Continue reading the main story}

Ask Well

\hypertarget{whats-a-pulse-oximeter-and-do-i-really-need-one-at-home}{%
\section{What's a Pulse Oximeter, and Do I Really Need One at
Home?}\label{whats-a-pulse-oximeter-and-do-i-really-need-one-at-home}}

A tiny fingertip device can give you valuable information about your
health during a bout of Covid-19 or any respiratory illness.

\includegraphics{https://static01.graylady3jvrrxbe.onion/images/2020/05/03/well/well-pulse-oximeter2/well-pulse-oximeter2-articleLarge.jpg?quality=75\&auto=webp\&disable=upscale}

\href{https://www.nytimes3xbfgragh.onion/by/tara-parker-pope}{\includegraphics{https://static01.graylady3jvrrxbe.onion/images/2018/07/10/us/Parker-Pope-Tara/Parker-Pope-Tara-thumbLarge.png}}

By \href{https://www.nytimes3xbfgragh.onion/by/tara-parker-pope}{Tara
Parker-Pope}

\begin{itemize}
\item
  Published April 24, 2020Updated June 18, 2020
\item
  \begin{itemize}
  \item
  \item
  \item
  \item
  \item
  \item
  \end{itemize}
\end{itemize}

\href{https://www.nytimes3xbfgragh.onion/es/2020/04/29/espanol/estilos-de-vida/oximetro-para-que-sirve.html}{Leer
en español}

After working for 10 days at Bellevue Hospital in New York, Dr. Richard
Levitan decided to
\href{https://www.nytimes3xbfgragh.onion/2020/04/20/opinion/coronavirus-testing-pneumonia.html}{share
what he had learned} about Covid-19. Too many patients were showing up
at the hospital with perilously low oxygen levels, putting them at risk
for severe complications and death.

But a simple home gadget called a pulse oximeter could help alert
patients to seek help sooner, he said.

``In the hospital, when I'm trying to decide who I send home, a big part
of the criteria is `What is your oxygen? What is your pulse?''' said Dr.
Levitan from his home in New Hampshire, where he just finished
self-quarantine as a precaution. ``With a pulse oximeter and a
thermometer, Americans can be prepared and be diagnosed and treated
before they get really, really sick.''

Health officials are divided on whether home monitoring with a pulse
oximeter should be recommended on a widespread basis during Covid-19.
Studies of reliability show mixed results, and there's little guidance
on how to choose one. But many doctors are advising patients to get one,
making it the go-to gadget of the pandemic. We've answered common
questions about the device, how it works and what to do with the
information it gives you.

\hypertarget{what-is-a-pulse-oximeter}{%
\subsection{What is a pulse oximeter?}\label{what-is-a-pulse-oximeter}}

A pulse oximeter is a small device that looks sort of like a chip clip
or a big clothes pin. You place your finger snugly inside (most require
nail side up), and within seconds it lights up with numbers indicating
your blood oxygen level and heart rate. Most healthy people will get an
oxygen reading around 95 to 98 percent. Some people with existing health
conditions may have a lower normal reading. You should check in with
your doctor if the number falls to 92 or lower.

The device will also show your heart rate. A normal resting heart rate
for adults ranges from about 60 to 100 beats per minute, although
athletes with a higher cardiovascular fitness will have a lower pulse.

Pro tip: One of the things to remember about reading a pulse oximeter is
that many of them are designed to be read by someone facing you, not the
person wearing it. The first time I tried my home device, it looked like
my oxygen level was an alarming 86, but then I realized I was reading it
upside down. (It was 98.)

\hypertarget{latest-updates-the-coronavirus-outbreak}{%
\section{\texorpdfstring{\href{https://www.nytimes3xbfgragh.onion/2020/09/11/world/covid-19-coronavirus.html?action=click\&pgtype=Article\&state=default\&region=MAIN_CONTENT_1\&context=storylines_live_updates}{Latest
Updates: The Coronavirus
Outbreak}}{Latest Updates: The Coronavirus Outbreak}}\label{latest-updates-the-coronavirus-outbreak}}

Updated 2020-09-12T05:29:13.829Z

\begin{itemize}
\tightlist
\item
  \href{https://www.nytimes3xbfgragh.onion/2020/09/11/world/covid-19-coronavirus.html?action=click\&pgtype=Article\&state=default\&region=MAIN_CONTENT_1\&context=storylines_live_updates\#link-dfb8a16}{Fauci
  cautions the virus could disrupt life in the U.S. until `maybe even
  towards the end of 2021.'}
\item
  \href{https://www.nytimes3xbfgragh.onion/2020/09/11/world/covid-19-coronavirus.html?action=click\&pgtype=Article\&state=default\&region=MAIN_CONTENT_1\&context=storylines_live_updates\#link-7104d154}{From
  Asia to Africa, China promotes its vaccine candidates to win friends.}
\item
  \href{https://www.nytimes3xbfgragh.onion/2020/09/11/world/covid-19-coronavirus.html?action=click\&pgtype=Article\&state=default\&region=MAIN_CONTENT_1\&context=storylines_live_updates\#link-393ad215}{The
  other way the virus will kill: hunger.}
\end{itemize}

\href{https://www.nytimes3xbfgragh.onion/2020/09/11/world/covid-19-coronavirus.html?action=click\&pgtype=Article\&state=default\&region=MAIN_CONTENT_1\&context=storylines_live_updates}{See
more updates}

More live coverage:
\href{https://www.nytimes3xbfgragh.onion/live/2020/09/11/business/stock-market-today-coronavirus?action=click\&pgtype=Article\&state=default\&region=MAIN_CONTENT_1\&context=storylines_live_updates}{Markets}

\hypertarget{how-does-a-pulse-oximeter-work}{%
\subsection{How does a pulse oximeter
work?}\label{how-does-a-pulse-oximeter-work}}

When you insert your finger into a pulse oximeter, it beams different
wavelengths of light through your finger (you won't feel a thing). It's
targeting hemoglobin, a protein molecule in your blood that carries
oxygen. Hemoglobin absorbs different amounts and wavelengths of light
depending on the level of oxygen it's carrying. Your pulse oximeter will
give you a numerical reading --- a percentage that indicates the level
of oxygen saturation in your blood. If you've been to a doctor in the
past 20 years, you've experienced pulse oximetry.

The device works better with warmer hands than cold hands. And because
oxygen levels can fluctuate, consider taking measurements a few times a
day. Also try it in different positions, such as while lying flat on
your back or while walking. Keep notes to share with your doctor if
needed.

\hypertarget{does-it-matter-what-finger-i-use}{%
\subsection{Does it matter what finger I
use?}\label{does-it-matter-what-finger-i-use}}

Most health technicians will place the device on the index fingers, but
a study of 37 volunteers found that the highest reading came from
\href{https://www.ncbi.nlm.nih.gov/pmc/articles/PMC4627972/}{the third
finger on the dominant hand}. A close second was the dominant thumb. So
if you are right-handed, use the right middle finger. If you are
left-handed, use the left middle finger. The difference between fingers
is small, so if you prefer the index finger, that's fine.

\hypertarget{do-long-nails-or-nail-polish-make-a-difference}{%
\subsection{Do long nails or nail polish make a
difference?}\label{do-long-nails-or-nail-polish-make-a-difference}}

Yes. Dark nail polish can affect accuracy of the reading. Very long
nails would make it difficult to insert your finger properly in the
clip.

\hypertarget{what-happens-if-my-oxygen-level-falls-what-is-the-treatment}{%
\subsection{What happens if my oxygen level falls? What is the
treatment?}\label{what-happens-if-my-oxygen-level-falls-what-is-the-treatment}}

If your number dips to 92 or lower, you should check in with your
doctor. But don't panic.

The good news is that it's a lot easier to bolster an oxygen level that
is just starting to drop than one that is dangerously low. When Dr. Anna
Marie Chang, an emergency room physician in Philadelphia, tested
positive for coronavirus in mid-March, she felt lousy but was reassured
by daily checks that showed normal oxygen levels. Dr. Chang, an
associate professor of emergency medicine and director of clinical
research for Thomas Jefferson University, even started feeling better
but kept up her daily monitoring with her pulse oximeter. One morning
she felt severely fatigued and saw that her oxygen level had dropped to
88 percent.

``I texted my colleagues and said, `I think it's time to go in,''' she
said. Once admitted, she was placed on oxygen with a mask. She spent her
days resting in the prone position (on her stomach) because the position
opens up the lungs and is more comfortable. ``I was there for four days
and never needed to be intubated,'' Dr. Chang said. ``It was just
supplemental oxygen.''

Dr. Levitan noted that patients with Covid-19 can experience a
potentially dangerous drop in oxygen saturation without having obvious
breathing problems. Without a pulse oximeter, they might never know it
or get very used to how they feel, despite very low oxygen levels. By
the time they go to the hospital feeling shortness of breath, their
oxygen levels would have dropped significantly, and they could have very
advanced Covid pneumonia.

``They are still talking, thinking clearly, and not in obvious
distress,'' Dr. Levitan said. ``If the level of oxygen became this low
all of sudden, these patients would be unconscious, having seizures, or
otherwise affected. What that means to me is there is a period of days
where they were going silently down and they didn't know it.''

\hypertarget{is-there-a-risk-to-monitoring-oxygen-levels-at-home}{%
\subsection{Is there a risk to monitoring oxygen levels at
home?}\label{is-there-a-risk-to-monitoring-oxygen-levels-at-home}}

It's possible that a home monitor could give a faulty reading or be used
incorrectly, prompting a patient to seek care unnecessarily. If you or
someone in your home shows a very low reading, you may want to test your
device on a healthy person to confirm that it is working correctly and
discuss it with your doctor.

And home monitoring should not give you a false sense of security. Don't
ignore physical symptoms even if your oxygen level is fine. You should
still call a doctor if you have severe shortness of breath, a high
fever, confusion or any other concerning symptom. The benefit of
monitoring is that it potentially can flag a decline in your respiratory
health before you feel it. And if you feel really lousy --- as many
Covid patients do for a few weeks --- seeing a normal oxygen level can
relieve some of the stress of the illness.

\hypertarget{are-home-devices-accurate-which-one-should-i-buy}{%
\subsection{Are home devices accurate? Which one should I
buy?}\label{are-home-devices-accurate-which-one-should-i-buy}}

When Dr. Chang needed a home monitor, she called friends and told them
to pick one up from Target. ``I literally said to my friend, `Just find
me one,''' Dr. Chang said. ``It's fairly straightforward technology.''

The research data on home monitors has been mixed, but they tend to be
accurate within a few percentage points. In drugstores you can find
monitors in the \$20 to \$50 range, while some sell online for \$200 or
more. Paying a higher price doesn't guarantee a better monitor.

Wirecutter, a New York Times Company that reviews and recommends
products,
\href{https://thewirecutter.com/blog/coronavirus-pulse-oximeter/}{suggests}
starting with the
\href{https://www.accessdata.fda.gov/scripts/cdrh/cfdocs/cfpmn/pmn.cfm}{Food
and Drug Administration's 510(k) Premarket Notification Database} and
searching for ``oximeter.''

Given that pulse oximeters are in high demand, you may not find a model
listed in the F.D.A. database right now and will have to settle for what
you can find. One way to check how it's performing is to take your pulse
manually and compare the result to the rate shown on the device.
Remember, when you use a monitor, it's the trend that matters, not a
single reading.

``Consumer models are very reliable,'' Dr. Levitan said.

Experts advise sticking with the finger clip technology for now. Newer
wearable devices and camera-based apps use a different technology to
measure oxygen saturation, and so far most of these products appear to
be unreliable. A 2019 study in the
\href{https://www.ncbi.nlm.nih.gov/pubmed/31471076}{American Journal of
Emergency Medicine}tested three iPhone apps that offered pulse oximetry
function, but they all fell dangerously short of the mark. The apps were
``inaccurate'' and ``had limited to no ability to accurately detect
hypoxia,'' the authors concluded.

If you can't find a pulse oximeter right away, you can probably find one
that can be delivered in a few weeks or month. If you get sick and don't
have a home pulse oximeter, don't panic. Most people do fine without
them. You also can also borrow one from a friend (they are easily
sanitized) or talk to your doctor about getting your oxygen levels
checked at an urgent care center.

Advertisement

\protect\hyperlink{after-bottom}{Continue reading the main story}

\hypertarget{site-index}{%
\subsection{Site Index}\label{site-index}}

\hypertarget{site-information-navigation}{%
\subsection{Site Information
Navigation}\label{site-information-navigation}}

\begin{itemize}
\tightlist
\item
  \href{https://help.nytimes3xbfgragh.onion/hc/en-us/articles/115014792127-Copyright-notice}{©~2020~The
  New York Times Company}
\end{itemize}

\begin{itemize}
\tightlist
\item
  \href{https://www.nytco.com/}{NYTCo}
\item
  \href{https://help.nytimes3xbfgragh.onion/hc/en-us/articles/115015385887-Contact-Us}{Contact
  Us}
\item
  \href{https://www.nytco.com/careers/}{Work with us}
\item
  \href{https://nytmediakit.com/}{Advertise}
\item
  \href{http://www.tbrandstudio.com/}{T Brand Studio}
\item
  \href{https://www.nytimes3xbfgragh.onion/privacy/cookie-policy\#how-do-i-manage-trackers}{Your
  Ad Choices}
\item
  \href{https://www.nytimes3xbfgragh.onion/privacy}{Privacy}
\item
  \href{https://help.nytimes3xbfgragh.onion/hc/en-us/articles/115014893428-Terms-of-service}{Terms
  of Service}
\item
  \href{https://help.nytimes3xbfgragh.onion/hc/en-us/articles/115014893968-Terms-of-sale}{Terms
  of Sale}
\item
  \href{https://spiderbites.nytimes3xbfgragh.onion}{Site Map}
\item
  \href{https://help.nytimes3xbfgragh.onion/hc/en-us}{Help}
\item
  \href{https://www.nytimes3xbfgragh.onion/subscription?campaignId=37WXW}{Subscriptions}
\end{itemize}
