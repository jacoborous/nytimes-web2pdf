Sections

SEARCH

\protect\hyperlink{site-content}{Skip to
content}\protect\hyperlink{site-index}{Skip to site index}

\href{https://myaccount.nytimes3xbfgragh.onion/auth/login?response_type=cookie\&client_id=vi}{}

\href{https://www.nytimes3xbfgragh.onion/section/todayspaper}{Today's
Paper}

Enrique Olvera's Satisfying, Adaptable Vegetable Soup

\url{https://nyti.ms/34x3eSs}

\begin{itemize}
\item
\item
\item
\item
\item
\item
\end{itemize}

\href{https://www.nytimes3xbfgragh.onion/spotlight/at-home?action=click\&pgtype=Article\&state=default\&region=TOP_BANNER\&context=at_home_menu}{At
Home}

\begin{itemize}
\tightlist
\item
  \href{https://www.nytimes3xbfgragh.onion/2020/09/07/travel/route-66.html?action=click\&pgtype=Article\&state=default\&region=TOP_BANNER\&context=at_home_menu}{Cruise
  Along: Route 66}
\item
  \href{https://www.nytimes3xbfgragh.onion/2020/09/04/dining/sheet-pan-chicken.html?action=click\&pgtype=Article\&state=default\&region=TOP_BANNER\&context=at_home_menu}{Roast:
  Chicken With Plums}
\item
  \href{https://www.nytimes3xbfgragh.onion/2020/09/04/arts/television/dark-shadows-stream.html?action=click\&pgtype=Article\&state=default\&region=TOP_BANNER\&context=at_home_menu}{Watch:
  Dark Shadows}
\item
  \href{https://www.nytimes3xbfgragh.onion/interactive/2020/at-home/even-more-reporters-editors-diaries-lists-recommendations.html?action=click\&pgtype=Article\&state=default\&region=TOP_BANNER\&context=at_home_menu}{Explore:
  Reporters' Google Docs}
\end{itemize}

Advertisement

\protect\hyperlink{after-top}{Continue reading the main story}

Supported by

\protect\hyperlink{after-sponsor}{Continue reading the main story}

\hypertarget{enrique-olveras-satisfying-adaptable-vegetable-soup}{%
\section{Enrique Olvera's Satisfying, Adaptable Vegetable
Soup}\label{enrique-olveras-satisfying-adaptable-vegetable-soup}}

The chef shares his recipe for a hearty broth-based dish, inspired by
the version his grandmother used to make.

\includegraphics{https://static01.graylady3jvrrxbe.onion/images/2020/04/13/t-magazine/13tmag-olvera-recipe-02/13tmag-olvera-recipe-02-articleLarge.jpg?quality=75\&auto=webp\&disable=upscale}

By
\href{https://www.nytimes3xbfgragh.onion/by/merrell-hambleton}{Merrell
Hambleton}

\begin{itemize}
\item
  Published April 13, 2020Updated April 21, 2020
\item
  \begin{itemize}
  \item
  \item
  \item
  \item
  \item
  \item
  \end{itemize}
\end{itemize}

In high school, the Mexican chef
\href{https://www.instagram.com/enriqueolveraf/?hl=en}{Enrique Olvera}
was known to his friends as Pozole. The word, which means ``hominy'' in
Spanish, derived from the Nahuatl ``pozolli,'' is also the name of a
traditional Mexican soup or stew made from shelled kernels of corn, pork
and garlic. When Olvera opened his first restaurant in Mexico City in
2000, he called it \href{https://pujol.com.mx/en/}{Pujol}, a slurring of
the childhood nickname. In many **** ways, soup has been a throughline
in Olvera's culinary life. He has said that his grandmother's puchero,
made from vegetables and cheap cuts of meat, is the first dish that he
remembers eating. Other staples of her kitchen were a lentil soup with
plantains and bacon and a vegetable soup made with potatoes, carrots,
zucchini and whatever else was good and in season. ``My grandfather
really liked soups, so we always had them at home,'' Olvera says. ``He
found them heartwarming.''

Somewhere between the more formal pozole, traditionally served on
special occasions, and Olvera's standard weeknight meal of black beans
cooked down with onion, garlic and herbs, is the vegetable soup that the
chef shares below. The tomato-based broth, bolstered with potatoes,
broccoli and cauliflower and flavored with ancho chile, is based loosely
on the version his grandmother used to prepare. To make it heartier ---
into **** what Olvera calls ``a perfect one-course meal'' --- he added
ayocote beans. ``It's almost like a Mexican version of minestrone,'' he
says.

Another key ingredient is the aromatic epazote, a dark green, leafy herb
with notes of citrus, anise and mint. ``In Mexico, we obviously use a
lot of cilantro, but epazote is more distinctly Mexican,'' says the
chef. ``It's a digestive, so if you're feeling down or a little sick,
it's a good herb for that. It has a comforting quality.'' But if you
can't find epazote, it's no trouble. Olvera describes the dish as
endlessly adaptable. He recommends tarragon as a substitute for epazote,
but also encourages anyone making it to ``just grab any herb that's at
its peak and grown with care.'' Can't find fresh tomatoes? High-quality
canned San Marzanos will do. Or scrap the tomatoes altogether and opt
for a clear broth base. ``Some people add edible greens or change the
beans,'' he says. ``If you want to add some matzo balls into it, that
can work, too.'' Soothing and versatile, it's a soup especially perfect
for this moment. ``You can have it for dinner, sitting down and enjoying
conversation with your family. But it can also just be you.'' he says.
``Amidst all this craziness, everyone can use a soup.''

\includegraphics{https://static01.graylady3jvrrxbe.onion/images/2020/04/13/t-magazine/13tmag-olvera-recipe/13tmag-olvera-recipe-articleLarge.jpg?quality=75\&auto=webp\&disable=upscale}

\hypertarget{enrique-olveras-vegetable-soup}{%
\subsubsection{\texorpdfstring{\textbf{Enrique Olvera's Vegetable
Soup}}{Enrique Olvera's Vegetable Soup}}\label{enrique-olveras-vegetable-soup}}

\emph{Serves 10}

\textbf{Ingredients:}

\begin{itemize}
\item
  4 tablespoons extra virgin olive oil
\item
  1 large white onion, cut into quarters
\item
  6 large garlic cloves, finely chopped
\item
  3 medium-sized ancho chiles, roasted
\item
  1 cup dried ayocote beans (runner beans), white or scarlet
\item
  2 fingerling potatoes, cut into rounds
\item
  4 carrots, cut into thick rounds
\item
  Salt, to taste
\item
  10 very ripe plum tomatoes, quartered
\item
  1 bunch chard, roughly chopped
\item
  1 cup broccoli florets
\item
  1 cup cauliflower florets
\item
  2 handfuls fresh epazote (or tarragon) leaves, roughly chopped
\item
  1 bay leaf
\item
  Key lime wedges, for serving
\end{itemize}

1. Put the beans and bay leaf in a large pot. Add water to equal double
the volume of the beans (about 2 cups). Bring to a boil, then reduce to
a simmer. Keep a small pot with hot water on the stove in case the beans
need more liquid; they should be fully submerged. **** Cook until the
beans are tender enough to bite into, but before they are fully cooked
and soft, about 40 minutes. Add the potatoes and carrots, cook for 15
minutes. Remove the bay leaf and season with salt.

2. In another pot, heat the olive oil over medium heat. Cook the roasted
chiles, **** chopped onion and garlic until translucent, about 5
minutes. Add the tomatoes, cover the pot and let it cook for 20 more
minutes. Transfer into a blender and blend until smooth.

3. Strain the tomato mixture through a sieve, pressing with a spoon
until the liquid has passed through into the pot with the beans and
vegetables. Add the broccoli, cauliflower, chard and herbs and cook for
10 minutes. Season with salt.

Advertisement

\protect\hyperlink{after-bottom}{Continue reading the main story}

\hypertarget{site-index}{%
\subsection{Site Index}\label{site-index}}

\hypertarget{site-information-navigation}{%
\subsection{Site Information
Navigation}\label{site-information-navigation}}

\begin{itemize}
\tightlist
\item
  \href{https://help.nytimes3xbfgragh.onion/hc/en-us/articles/115014792127-Copyright-notice}{©~2020~The
  New York Times Company}
\end{itemize}

\begin{itemize}
\tightlist
\item
  \href{https://www.nytco.com/}{NYTCo}
\item
  \href{https://help.nytimes3xbfgragh.onion/hc/en-us/articles/115015385887-Contact-Us}{Contact
  Us}
\item
  \href{https://www.nytco.com/careers/}{Work with us}
\item
  \href{https://nytmediakit.com/}{Advertise}
\item
  \href{http://www.tbrandstudio.com/}{T Brand Studio}
\item
  \href{https://www.nytimes3xbfgragh.onion/privacy/cookie-policy\#how-do-i-manage-trackers}{Your
  Ad Choices}
\item
  \href{https://www.nytimes3xbfgragh.onion/privacy}{Privacy}
\item
  \href{https://help.nytimes3xbfgragh.onion/hc/en-us/articles/115014893428-Terms-of-service}{Terms
  of Service}
\item
  \href{https://help.nytimes3xbfgragh.onion/hc/en-us/articles/115014893968-Terms-of-sale}{Terms
  of Sale}
\item
  \href{https://spiderbites.nytimes3xbfgragh.onion}{Site Map}
\item
  \href{https://help.nytimes3xbfgragh.onion/hc/en-us}{Help}
\item
  \href{https://www.nytimes3xbfgragh.onion/subscription?campaignId=37WXW}{Subscriptions}
\end{itemize}
