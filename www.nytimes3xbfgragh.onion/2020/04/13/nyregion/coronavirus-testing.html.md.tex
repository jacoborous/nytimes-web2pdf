Sections

SEARCH

\protect\hyperlink{site-content}{Skip to
content}\protect\hyperlink{site-index}{Skip to site index}

\href{https://www.nytimes3xbfgragh.onion/section/nyregion}{New York}

\href{https://myaccount.nytimes3xbfgragh.onion/auth/login?response_type=cookie\&client_id=vi}{}

\href{https://www.nytimes3xbfgragh.onion/section/todayspaper}{Today's
Paper}

\href{/section/nyregion}{New York}\textbar{}3 Vans, 6 Coolers, a Plane,
a Storm and 2 Labs: A Nasal Swab's Journey

\url{https://nyti.ms/3esuDd0}

\begin{itemize}
\item
\item
\item
\item
\item
\item
\end{itemize}

\hypertarget{the-coronavirus-outbreak}{%
\subsubsection{\texorpdfstring{\href{https://www.nytimes3xbfgragh.onion/news-event/coronavirus?name=styln-coronavirus-national\&region=TOP_BANNER\&block=storyline_menu_recirc\&action=click\&pgtype=Article\&impression_id=d7db2fc0-f4b7-11ea-96c6-737460281ea1\&variant=undefined}{The
Coronavirus
Outbreak}}{The Coronavirus Outbreak}}\label{the-coronavirus-outbreak}}

\begin{itemize}
\tightlist
\item
  live\href{https://www.nytimes3xbfgragh.onion/2020/09/11/world/covid-19-coronavirus.html?name=styln-coronavirus-national\&region=TOP_BANNER\&block=storyline_menu_recirc\&action=click\&pgtype=Article\&impression_id=d7db2fc1-f4b7-11ea-96c6-737460281ea1\&variant=undefined}{Latest
  Updates}
\item
  \href{https://www.nytimes3xbfgragh.onion/interactive/2020/us/coronavirus-us-cases.html?name=styln-coronavirus-national\&region=TOP_BANNER\&block=storyline_menu_recirc\&action=click\&pgtype=Article\&impression_id=d7db2fc2-f4b7-11ea-96c6-737460281ea1\&variant=undefined}{Maps
  and Cases}
\item
  \href{https://www.nytimes3xbfgragh.onion/interactive/2020/science/coronavirus-vaccine-tracker.html?name=styln-coronavirus-national\&region=TOP_BANNER\&block=storyline_menu_recirc\&action=click\&pgtype=Article\&impression_id=d7db2fc3-f4b7-11ea-96c6-737460281ea1\&variant=undefined}{Vaccine
  Tracker}
\item
  \href{https://www.nytimes3xbfgragh.onion/2020/09/10/us/politics/fda-coronavirus-vaccine.html?name=styln-coronavirus-national\&region=TOP_BANNER\&block=storyline_menu_recirc\&action=click\&pgtype=Article\&impression_id=d7db56d0-f4b7-11ea-96c6-737460281ea1\&variant=undefined}{F.D.A.
  Regulators' Self-Defense}
\item
  \href{https://www.nytimes3xbfgragh.onion/2020/09/09/upshot/coronavirus-surprise-test-fees.html?name=styln-coronavirus-national\&region=TOP_BANNER\&block=storyline_menu_recirc\&action=click\&pgtype=Article\&impression_id=d7db56d1-f4b7-11ea-96c6-737460281ea1\&variant=undefined}{Surprise
  Test Fees}
\end{itemize}

Advertisement

\protect\hyperlink{after-top}{Continue reading the main story}

Supported by

\protect\hyperlink{after-sponsor}{Continue reading the main story}

\hypertarget{3-vans-6-coolers-a-plane-a-storm-and-2-labs-a-nasal-swabs-journey}{%
\section{3 Vans, 6 Coolers, a Plane, a Storm and 2 Labs: A Nasal Swab's
Journey}\label{3-vans-6-coolers-a-plane-a-storm-and-2-labs-a-nasal-swabs-journey}}

The backlog for Covid-19 testing in New Jersey and other parts of the
country is getting worse, not better. From the nose of a patient in a
mile-long line to a phone call days later, bottlenecks thwart its
progress.

\includegraphics{https://static01.graylady3jvrrxbe.onion/images/2020/04/11/multimedia/11virus-testing1/merlin_171490422_6f2eb368-294e-4ca6-9196-71c80ab281da-articleLarge.jpg?quality=75\&auto=webp\&disable=upscale}

\href{https://www.nytimes3xbfgragh.onion/by/rukmini-callimachi}{\includegraphics{https://static01.graylady3jvrrxbe.onion/images/2018/10/08/multimedia/author-rukmini-callimachi/author-rukmini-callimachi-thumbLarge-v2.png}}

By
\href{https://www.nytimes3xbfgragh.onion/by/rukmini-callimachi}{Rukmini
Callimachi}

\begin{itemize}
\item
  Published April 13, 2020Updated April 21, 2020
\item
  \begin{itemize}
  \item
  \item
  \item
  \item
  \item
  \item
  \end{itemize}
\end{itemize}

PARAMUS, N.J. --- The lines start forming the night before, as people
with glassy eyes and violent coughs try to get
\href{https://www.nytimes3xbfgragh.onion/2020/04/21/health/fda-in-home-test-coronavirus.html}{tested
for the virus}. In the darkness, they park their cars, cut their engines
and try to sleep.

The backlog for
\href{https://www.nytimes3xbfgragh.onion/2020/04/21/health/fda-in-home-test-coronavirus.html}{coronavirus
testing} in New Jersey, the state with the second-highest caseload in
the country, has been getting worse, not better, officials say.

So far, New Jersey has conducted over
\href{https://www.nj.gov/health/cd/topics/covid2019_dashboard.shtml}{115,000
tests}, about one for every 75 residents. Across the river in New York,
the epicenter of the crisis, there is about one for every 40. The tests
are~a critical tool in measuring the disease's spread and a requirement
for certain forms of treatment. Yet they remain hard to get, and many
are actively discouraged from trying.

``It's unequivocally worsening,'' Gov. Philip D. Murphy of New Jersey
\href{https://www.nj.gov/governor/news/news/562020/approved/20200402d.shtml}{said}
recently, adding, ``We've got constraints in the entire food chain.''

Initially, the strain came from a lack of test kits, but now there are
not enough nasal swabs, not enough nurses. There is a pileup at the labs
themselves and a limited supply of the chemicals needed to identify the
virus.

Two weeks ago at the Bergen Community College in Paramus, a
drive-through testing site in
\href{https://www.nj.gov/health/cd/topics/covid2019_dashboard.shtml}{the
hardest-hit area of New Jersey}, residents had to arrive by 3 a.m. to
get a spot. Within days, they were told to show up at 11 p.m. the night
before.

\includegraphics{https://static01.graylady3jvrrxbe.onion/images/2020/04/11/multimedia/11virus-testing8/merlin_171396021_cce7f662-7050-48b5-95bc-910a5445830f-articleLarge.jpg?quality=75\&auto=webp\&disable=upscale}

On Monday of last week, Anita Holmes-Perez felt so sick that she asked
her husband to drive her there even earlier, at 10:45, but a car was
already ahead of her. The entrance to the site, run by the Federal
Emergency Management Agency, was blocked off by an armored personnel
carrier. Members of the National Guard idled in camouflage nearby.

Ms. Holmes-Perez spent the night constantly adjusting the reclining seat
inside her Mercedes S-Class, lying down until the congestion in her
chest forced her to sit up again.

She was battling a fever, a cough, dizziness and a feeling of confusion.
``Like you don't know where you are,'' the 45-year-old said. ``You
forget what you're doing.''

When medical workers finally took a sample from her the next morning, it
would be shipped across the country because the local lab was too full.
Three vans would take it part of the way. A plane, sent on a detour by a
storm, would take it further. It would be days before she got a result.
Until then, Ms. Holmes-Perez waited.

\hypertarget{face-forward}{%
\subsection{`Face Forward'}\label{face-forward}}

Shortly before the drive-through opened at 8 a.m. last Tuesday, a police
car drove up and down the mile-long line of parked vehicles, sirens
blaring.

``I think they did it to wake us up,'' said 29-year-old Kayla Codina,
who had spent the hours before dawn swiping through TikTok, too anxious
to sleep inside her Ford Fusion.

When testing finally began, the cars surged forward, approaching a
triage site marked in orange cones.

Image

Face shields hanging in a tent at Bergen Community College, the
drive-through testing site in Paramus.Credit...Ryan Christopher Jones
for The New York Times

From a distance, the station appeared to be manned by astronauts. The
nurses were wearing face shields and bright white scrubs, their first
names written in black marker on the front along with the digits 0800:
the hour they got suited up. A grid of temperatures and times advised
how long before they would need to discard the uniforms.

``Please roll up your window,'' one of the nurses said. ``Higher,
please.''

The people coming to get tested are not allowed out of their cars, and
their windows can be open no more than an inch --- just enough for
workers to slip in a pink square of paper with a number on it. They give
out only 500 numbers a day.

\hypertarget{latest-updates-the-coronavirus-outbreak}{%
\section{\texorpdfstring{\href{https://www.nytimes3xbfgragh.onion/2020/09/11/world/covid-19-coronavirus.html?action=click\&pgtype=Article\&state=default\&region=MAIN_CONTENT_1\&context=storylines_live_updates}{Latest
Updates: The Coronavirus
Outbreak}}{Latest Updates: The Coronavirus Outbreak}}\label{latest-updates-the-coronavirus-outbreak}}

Updated 2020-09-12T04:56:54.924Z

\begin{itemize}
\tightlist
\item
  \href{https://www.nytimes3xbfgragh.onion/2020/09/11/world/covid-19-coronavirus.html?action=click\&pgtype=Article\&state=default\&region=MAIN_CONTENT_1\&context=storylines_live_updates\#link-dfb8a16}{Fauci
  cautions the virus could disrupt life in the U.S. until `maybe even
  towards the end of 2021.'}
\item
  \href{https://www.nytimes3xbfgragh.onion/2020/09/11/world/covid-19-coronavirus.html?action=click\&pgtype=Article\&state=default\&region=MAIN_CONTENT_1\&context=storylines_live_updates\#link-7104d154}{From
  Asia to Africa, China promotes its vaccine candidates to win friends.}
\item
  \href{https://www.nytimes3xbfgragh.onion/2020/09/11/world/covid-19-coronavirus.html?action=click\&pgtype=Article\&state=default\&region=MAIN_CONTENT_1\&context=storylines_live_updates\#link-393ad215}{The
  other way the virus will kill: hunger.}
\end{itemize}

\href{https://www.nytimes3xbfgragh.onion/2020/09/11/world/covid-19-coronavirus.html?action=click\&pgtype=Article\&state=default\&region=MAIN_CONTENT_1\&context=storylines_live_updates}{See
more updates}

More live coverage:
\href{https://www.nytimes3xbfgragh.onion/live/2020/09/11/business/stock-market-today-coronavirus?action=click\&pgtype=Article\&state=default\&region=MAIN_CONTENT_1\&context=storylines_live_updates}{Markets}

Before patients can get one, they are asked to hold their New Jersey
drivers' licenses against the glass to prove they are residents. On
previous days, desperate people drove in from out of state.

Image

Ms. Massarotti screening a resident for Covid-19 symptoms. Those who are
approved for a test get one of the day's 500 pink cards.Credit...Ryan
Christopher Jones for The New York Times

Those with no symptoms are also turned away. When the medical staff
asked Ms. Codina to describe how she was feeling, she said she could no
longer take a full breath. She was handed Square No. 14, allowing her to
drive a few dozen feet up to a white tent.

Farther down the line, Andres Chia, 54, who had tested positive days
earlier, was worried that he had infected his younger brother and their
84-year-old father. The nurse handed them Nos. 145 and 146 and waved
their Nissan forward.

``My father keeps asking: `Is it my time now? Is this how I am going to
go?''' said the younger brother, Israel Chia, 44.

Image

Andres Chia, who had tested positive for the virus, brought his
84-year-old father to be screened for it.Credit...Ryan Christopher Jones
for The New York Times

A nurse leaned in with a long Q-tip but was corrected by her manager.
``Tell him to face forward,'' he said to her.

When she inserted the swab into the older man's nose, pushing as far
back as it could go, he erupted into an explosive cough, the kind that
aerosolizes the virus, sending tiny, potentially dangerous droplets into
the air, most of them trapped inside the Nissan.

Image

Medical workers sealed specimens in bags before shipping them to the
lab.Credit...Ryan Christopher Jones for The New York Times

By midafternoon, they'd run out of tests. Workers placed the day's work
--- hundreds of test tubes in plastic bags --- into two large boxes
covered in ice packs.

Douglas Ortmann, a 20-year FedEx veteran, eased the two boxes into his
empty van. Since screening began weeks ago, he has waited for the phone
call each day, dashed to the Paramus testing site and then driven the
samples to Teterboro, 14 minutes away.

``I understand how important it is,'' he said.

At 2:55 p.m., the van pulled into the loading dock of Quest Diagnostics'
flagship lab.

\hypertarget{system-overload}{%
\subsection{System Overload}\label{system-overload}}

As tests were underway in the drive-through, their destination was being
decided 900 miles across the country.

In Brookfield, Wis., Quest's executive vice president James E. Davis was
working from home, like much of the country. The floor-to-ceiling
windows of his home office face a verdant lawn, but his eyes were
trained on his desktop monitor, where he had loaded a spreadsheet of how
many nasal swabs each of his labs had received.

Image

May Carrillo sorted through the samples at Quest's facility in
Teterboro, N.J. The lab's proximity to New York has left it
inundated.Credit...Ryan Christopher Jones for The New York Times

This much was obvious: The Teterboro lab, one of the company's largest,
was overwhelmed. Over the preceding 24 hours, FedEx drivers like Mr.
Ortmann had arrived with samples from elsewhere in New Jersey ---
including hospitals, whose patients take priority over those at
drive-throughs --- and from New York.

While other countries quickly ramped up screening, the United States
lost
\href{https://www.nytimes3xbfgragh.onion/2020/03/28/us/testing-coronavirus-pandemic.html}{valuable
time}. It wasn't until late February that private labs were given the
go-ahead to create tests of their own.

Quest began with a test in a single lab in California, and has since
expanded screening to 12 locations. It can process 35,000 specimens a
day, though not at the same facility. A fleet of 23 planes ferry coolers
of nasal swabs to one lab or another.

New Jersey's state-operated lab runs no more than 70 tests a day,
according to Christopher Neuwirth, an assistant commissioner at the
state's Health Department. Some hospitals can do on-site tests, but
they're a drop in the bucket. So the burden has fallen on private labs
like Quest, whose workload comes from all over the nation.

Mr. Neuwirth said that was hurting New Jersey right now, because it sits
within driving distance of New York. ``We happen to be in a region with
all these hot spots, and when you have commercial labs doing this across
the country, every state competes to get their tests done,'' he said.

At Quest, Mr. Davis jumps on a conference call twice a day to decide
which labs have wiggle room and which don't*.* For much of last week,
the Teterboro lab was at capacity, with almost half its caseload from
high-priority hospital patients.

Image

The Teterboro lab was overwhelmed, so the new specimens were packed with
dry ice to be shipped elsewhere.Credit...Ryan Christopher Jones for The
New York Times

Before the FedEx van pulled up to the 250,000-square-foot lab in New
Jersey, Mr. Davis's decision was already made: The test tubes would be
diverted to a lab in Chantilly, Va.

Workers brought them inside to prepare them for the next leg of the
journey. Two women in lab coats created a manifest, logging each
patient's information into a computer. The specimens were moved to six
coolers the color of key lime pie and packed with dry ice.

Image

Jamahl Carter loaded the samples onto a van. They would soon be
airborne.Credit...Ryan Christopher Jones for The New York Times

At 1:24 a.m. on Wednesday, another van drove them to the airport across
the street, where a small single-engine plane was waiting. The pilot,
George Fendley, would be flying solo, as he did several times a week. If
there ever were passenger seats inside the plane, they'd been ripped
out, the coolers filling the belly of the aircraft.

It was less than an hour to Virginia, about 270 miles away, and the
plane took off at 1:53 a.m. Along the way, Mr. Fendley flew into a
thunderstorm and had to touch down in Pennsylvania. He took off again,
trying to skirt around the lashing winds and rain. When he landed, it
was 4:28 a.m.

Image

George Fendley, a pilot, prepared to fly the specimens 270 miles south
to Virginia.Credit...Ryan Christopher Jones for The New York Times

Another van drove the cargo 27 minutes to Chantilly. Soon the sun was
rising, and as the operators unpacked the samples, almost 24 hours had
elapsed.

At the drive-through, New Jersey residents had been told to expect a
three- to five-day wait. ``Do you know if they mean business days or
regular days?'' Ms. Codina later asked.

\hypertarget{lives-on-pause}{%
\subsection{Lives on Pause}\label{lives-on-pause}}

In an effort to alleviate the strain on the system, some doctors have
advised patients to avoid testing.

That was the case for Ms. Holmes-Perez, who received the following
message from her doctor's office: ``There is no benefit to being tested
as you probably would not qualify anyway. Just stay on your
medications.''

But with each passing day, her symptoms worsened. She struggled to
breathe. Then she started to slur her speech. Next she began to feel
confused. ``I would enter a room and forget where I was,'' she recalled.

Image

Ms. Codina on her way out of the Paramus testing site. Notes on her
dashboard thanked the medical workers for their efforts.Credit...Ryan
Christopher Jones for The New York Times

For many, the concern is not just their own health but that of loved
ones.

Ms. Codina said she was waiting to find out her status so she could
decide where to live until she got better: Her roommate had been
experiencing Covid-like symptoms. If Ms. Codina's test came back
negative, she planned to move in with either her boyfriend or her
parents.

``This impacts many, many people in my life,'' she said

And in their shared home in Hackensack, N.J., the Chia brothers were
self-quarantining, each in his own bedroom. They wanted to know which
members of their family were positive so they could protect their
father.

\hypertarget{results-trickle-in}{%
\subsection{Results Trickle In}\label{results-trickle-in}}

In Chantilly, the machine used to analyze the samples, known as the
Roche Cobas 8800, can run 376 tests at once. It takes between three and
four hours for the machine to do one cycle. In theory, it can perform
500 tests in under eight hours.

Image

A technician at Quest's lab in Chantilly, Va., the facility that ended
up testing the Paramus samples.Credit...Erin Schaff/The New York Times

But before anything can happen, a human being needs to do the
painstaking work of loading each sample into the machine. As the sun
rose on Wednesday morning, a scientist used a pipette to remove liquid
from each tube and place it in the testing apparatus.

The machine gave off a purple hue. In the hours that followed, its
mechanized arms and levers carried out a process known as the
\href{https://www.nytimes3xbfgragh.onion/2019/08/15/science/kary-b-mullis-dead.html}{polymerase
chain reaction} to replicate genetic material. The lab was trying to
match the samples against a sequence unique to the new coronavirus,
according to Dr. Lawrence Tsao, Quest's East Region medical director.

The process requires high heat and a number of chemical reagents. One of
those, an enzyme called Taq polymerase, was originally sourced from
bacteria deep in the ocean's hydrothermal vents. Because it is in short
supply, there is a cap on how many tests Quest can do per day.

Image

The machine that screens for the virus, the Roche Cobas 8800, can run
376 tests at a time.Credit...Erin Schaff/The New York Times

``Usually we have extra reagents on hand,'' Dr. Tsao said. ``Now they
calculate what we're doing, and they give us just that.''

The last step is delivering the results. Twelve times a day, Quest sends
batches of results to \href{https://maximus.com/}{Maximus}, a federal
contractor. Because of privacy concerns, the Reston, Va.-based company's
call center does not leave voice messages. Its operators call back only
twice before moving to the next patient.

By the time they called Ms. Holmes-Perez, three days had passed since
she spent all night waiting in line. By then, the congestion was so bad
she had a hard time hearing the operator.

``Negative,'' the caller told her.

``At first, I felt relief, but then I started to worry about what is
actually going on with me,'' Ms. Holmes-Perez said.

In North Bergen, Ms. Codina feared she might not hear the call, so she
bought herself a new ringtone: a Ricky Martin song. On Friday, her phone
started blasting ``Livin' La Vida Loca'' just as she had started to make
herself a sandwich. She left the door to the refrigerator open in her
haste to answer. Like a little over half the people who have been tested
in the state, she, too, was told she was negative.

Image

Ms. Codina near here home in North Bergen this weekend after she
received her test results.Credit...Ryan Christopher Jones for The New
York Times

But when the operator called Israel Chia in Hackensack on Friday, it was
with bad news: Both he and his elderly father were positive.

Over the weekend, the situation got worse. The oxygen monitor he had
bought put his father in the range of 85 to 90 percent. Doctors are
concerned if a patient's reading falls below 95. On Saturday, he called
911.

Image

Andres Chia, left, with his brother Israel in Hackensack. Their father's
condition has worsened since he received the test.Credit...Ryan
Christopher Jones for The New York Times

Paramedics confirmed the low oxygen level but had a warning for the son:
Once your father goes to the hospital, you won't be able to see him. And
he will be surrounded by extremely sick people, who may have a worse
strain of the virus.

The test had brought one level of clarity, but now the brothers were
facing another dilemma. ``Here at home, I can give him his vitamins, his
Pedialyte, his protein shake. He's comfortable,'' Mr. Chia said.
``Should I keep him home? For now, that's what we've decided to do.''

Advertisement

\protect\hyperlink{after-bottom}{Continue reading the main story}

\hypertarget{site-index}{%
\subsection{Site Index}\label{site-index}}

\hypertarget{site-information-navigation}{%
\subsection{Site Information
Navigation}\label{site-information-navigation}}

\begin{itemize}
\tightlist
\item
  \href{https://help.nytimes3xbfgragh.onion/hc/en-us/articles/115014792127-Copyright-notice}{©~2020~The
  New York Times Company}
\end{itemize}

\begin{itemize}
\tightlist
\item
  \href{https://www.nytco.com/}{NYTCo}
\item
  \href{https://help.nytimes3xbfgragh.onion/hc/en-us/articles/115015385887-Contact-Us}{Contact
  Us}
\item
  \href{https://www.nytco.com/careers/}{Work with us}
\item
  \href{https://nytmediakit.com/}{Advertise}
\item
  \href{http://www.tbrandstudio.com/}{T Brand Studio}
\item
  \href{https://www.nytimes3xbfgragh.onion/privacy/cookie-policy\#how-do-i-manage-trackers}{Your
  Ad Choices}
\item
  \href{https://www.nytimes3xbfgragh.onion/privacy}{Privacy}
\item
  \href{https://help.nytimes3xbfgragh.onion/hc/en-us/articles/115014893428-Terms-of-service}{Terms
  of Service}
\item
  \href{https://help.nytimes3xbfgragh.onion/hc/en-us/articles/115014893968-Terms-of-sale}{Terms
  of Sale}
\item
  \href{https://spiderbites.nytimes3xbfgragh.onion}{Site Map}
\item
  \href{https://help.nytimes3xbfgragh.onion/hc/en-us}{Help}
\item
  \href{https://www.nytimes3xbfgragh.onion/subscription?campaignId=37WXW}{Subscriptions}
\end{itemize}
