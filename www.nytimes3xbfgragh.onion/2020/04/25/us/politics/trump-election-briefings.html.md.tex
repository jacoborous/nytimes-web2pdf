Sections

SEARCH

\protect\hyperlink{site-content}{Skip to
content}\protect\hyperlink{site-index}{Skip to site index}

\href{https://www.nytimes3xbfgragh.onion/section/politics}{Politics}

\href{https://myaccount.nytimes3xbfgragh.onion/auth/login?response_type=cookie\&client_id=vi}{}

\href{https://www.nytimes3xbfgragh.onion/section/todayspaper}{Today's
Paper}

\href{/section/politics}{Politics}\textbar{}Nervous Republicans See
Trump Sinking, and Taking Senate With Him

\url{https://nyti.ms/3cN7Ire}

\begin{itemize}
\item
\item
\item
\item
\item
\end{itemize}

\begin{itemize}
\item
  \href{https://www.nytimes3xbfgragh.onion/interactive/2020/09/08/us/elections/results-new-hampshire-primary-elections.html?action=click\&pgtype=Article\&state=default\&region=TOP_BANNER\&context=storylines_menu}{New
  Hampshire Results}
\item
  \href{https://www.nytimes3xbfgragh.onion/live/2020/09/09/us/trump-vs-biden?action=click\&pgtype=Article\&state=default\&region=TOP_BANNER\&context=storylines_menu}{Election
  Updates}
\item
  \href{https://www.nytimes3xbfgragh.onion/interactive/2020/us/elections/election-states-biden-trump.html?action=click\&pgtype=Article\&state=default\&region=TOP_BANNER\&context=storylines_menu}{Paths
  to 270}
\item
  \href{https://www.nytimes3xbfgragh.onion/interactive/2020/08/31/us/politics/vote-by-mail-deadlines.html?action=click\&pgtype=Article\&state=default\&region=TOP_BANNER\&context=storylines_menu}{Voting
  by Mail}
\item
  \href{https://www.nytimes3xbfgragh.onion/interactive/2019/us/elections/2020-presidential-election-calendar.html?action=click\&pgtype=Article\&state=default\&region=TOP_BANNER\&context=storylines_menu}{Key
  Dates}
\item
  \href{https://www.nytimes3xbfgragh.onion/newsletters/politics?action=click\&pgtype=Article\&state=default\&region=TOP_BANNER\&context=storylines_menu}{Politics
  Newsletter}
\end{itemize}

Advertisement

\protect\hyperlink{after-top}{Continue reading the main story}

Supported by

\protect\hyperlink{after-sponsor}{Continue reading the main story}

\hypertarget{nervous-republicans-see-trump-sinking-and-taking-senate-with-him}{%
\section{Nervous Republicans See Trump Sinking, and Taking Senate With
Him}\label{nervous-republicans-see-trump-sinking-and-taking-senate-with-him}}

The election is still six months away, but a rash of ominous new polls
and the president's erratic briefings have the G.O.P. worried about a
Democratic takeover.

\includegraphics{https://static01.graylady3jvrrxbe.onion/images/2020/04/24/us/politics/00trump-reelect/merlin_171873021_3bbb19f1-6572-4cde-ad35-e508f5f24ca4-articleLarge.jpg?quality=75\&auto=webp\&disable=upscale}

\href{https://www.nytimes3xbfgragh.onion/by/jonathan-martin}{\includegraphics{https://static01.graylady3jvrrxbe.onion/images/2018/11/06/multimedia/author-jonathan-martin/author-jonathan-martin-thumbLarge.png}}\href{https://www.nytimes3xbfgragh.onion/by/maggie-haberman}{\includegraphics{https://static01.graylady3jvrrxbe.onion/images/2018/07/12/multimedia/author-maggie-haberman/author-maggie-haberman-thumbLarge.png}}

By \href{https://www.nytimes3xbfgragh.onion/by/jonathan-martin}{Jonathan
Martin} and
\href{https://www.nytimes3xbfgragh.onion/by/maggie-haberman}{Maggie
Haberman}

\begin{itemize}
\item
  April 25, 2020
\item
  \begin{itemize}
  \item
  \item
  \item
  \item
  \item
  \end{itemize}
\end{itemize}

WASHINGTON --- President Trump's
\href{https://www.nytimes3xbfgragh.onion/2020/04/11/us/politics/coronavirus-trump-response.html}{erratic
handling} of the coronavirus outbreak, the worsening economy and a
cascade of ominous public and private polling have Republicans
increasingly nervous that they are at risk of losing the presidency and
the Senate if Mr. Trump does not put the nation on a radically improved
course.

The scale of the G.O.P.'s challenge has crystallized in the last week.
With 26 million Americans now having
\href{https://www.nytimes3xbfgragh.onion/interactive/2020/04/23/business/coronavirus-unemployment.html}{filed
for unemployment benefits}, Mr. Trump's standing in states that he
carried in 2016 looks increasingly wobbly: New surveys show him trailing
significantly in battleground states like Michigan and Pennsylvania, and
he is even narrowly behind in must-win Florida.

Democrats raised substantially more money than Republicans did in the
first quarter in the most pivotal congressional races, according to
recent campaign finance reports. And while Mr. Trump is
\href{https://www.nytimes3xbfgragh.onion/2020/04/21/us/politics/biden-2020-fundraising.html}{well
ahead in money} compared with the presumptive Democratic nominee, Joseph
R. Biden Jr., Democratic donors are only beginning to focus on the
general election, and several super PACs plan to spend heavily on behalf
of him and the party.

Perhaps most significantly, Mr. Trump's single best advantage as an
incumbent --- his access to the bully pulpit --- has effectively become
a platform for self-sabotage.

His daily news briefings on the coronavirus outbreak are inflicting
grave damage on his political standing, Republicans believe, and his
recent remarks about combating the virus with
\href{https://www.nytimes3xbfgragh.onion/2020/04/24/health/sunlight-coronavirus-trump.html}{sunlight
and disinfectant} were a breaking point for a number of senior party
officials.

On Friday evening, Mr. Trump conducted only a short briefing and took no
questions, a format that a senior administration official said was being
discussed as the best option for the president going forward.

Glen Bolger, a longtime Republican pollster, said the landscape for his
party had become far grimmer compared with the pre-virus plan to run
almost singularly around the country's prosperity.

``With the economy in free-fall, Republicans face a very challenging
environment and it's a total shift from where we were a few months
ago,'' Mr. Bolger said. ``Democrats are angry, and now we have the
foundation of the campaign yanked out from underneath us.''

Mr. Trump's advisers and allies have often blamed external events for
his most self-destructive acts, such as his repeated outbursts during
the two-year investigation into his campaign's dealings with Russia.
Now, there is no such explanation --- and, so far, there have been
exceedingly few successful interventions regarding Mr. Trump's behavior
at the podium.

Representative Tom Cole, Republican of Oklahoma, said the president had
to change his tone and offer more than a campaign of grievance.

``You got to have some hope to sell people,'' Mr. Cole said. ``But Trump
usually sells anger, division and `we're the victim.'''

There are still more than six months until the election, and many
Republicans are hoping that the dynamics of the race will shift once Mr.
Biden is thrust back into the campaign spotlight. At that point, they
believe, the race will not simply be the up-or-down referendum on the
president it is now, and Mr. Trump will be able to more effectively sell
himself as the person to rebuild the economy.

\emph{{[}Read about Joseph R. Biden Jr.'s}
\href{https://www.nytimes3xbfgragh.onion/2020/04/25/us/politics/joe-biden-coronavirus-quarantine.html}{\emph{cloistered
mode of campaigning}} \emph{during the coronavirus lockdown.{]}}

\includegraphics{https://static01.graylady3jvrrxbe.onion/images/2020/04/24/us/politics/00trump-reelect2/merlin_171873675_58d9d7fb-54d2-4cf7-a640-1fd1970915f7-articleLarge.jpg?quality=75\&auto=webp\&disable=upscale}

``We built the greatest economy in the world; I'll do it a second
time,'' Mr. Trump said earlier this month, road-testing a theme he will
deploy in the coming weeks.

Still, a recent wave of polling has fueled Republican anxieties, as Mr.
Biden leads in virtually every competitive state.

The surveys also showed Republican senators in Arizona, Colorado, North
Carolina and Maine trailing or locked in a dead heat with potential
Democratic rivals --- in part because their fate is linked to Mr.
Trump's job performance. If incumbents in those states lose, and
Republicans pick up only the Senate seat in Alabama, Democrats would
take control of the chamber should Mr. Biden win the presidency.

``He's got to run very close for us to keep the Senate,'' Charles R.
Black Jr., a veteran Republican consultant, said of Mr. Trump. ``I've
always thought we were favored to, but I can't say that now with all
these cards up in the air.''

Republicans were taken aback this past week by the results of a 17-state
survey commissioned by the Republican National Committee. It found the
president struggling in the Electoral College battlegrounds and likely
to lose without signs of an economic rebound this fall, according to a
party strategist outside the R.N.C. who is familiar with the poll's
results.

The Trump campaign's own surveys have also shown an erosion of support,
according to four people familiar with the data, as the coronavirus
remains the No. 1 issue worrying voters.

Polling this early is, of course, not determinative: In 2016 Hillary
Clinton also enjoyed a wide advantage in many states well before
November.

Yet Mr. Trump's best hope to win a state he lost in 2016, Minnesota,
also seems increasingly challenging. A Democratic survey taken by
Senator Tina Smith showed the president trailing by 10 percentage points
there, according to a Democratic strategist who viewed the poll.

The private data of the two parties is largely mirrored by public
surveys. Just last week, three Pennsylvania polls and two Michigan
surveys were released showing Mr. Trump losing outside the margin of
error. And a pair of Florida polls were released that showed Mr. Biden
enjoying a slim advantage in a state that is all but essential for
Republicans to retain the presidency.

To some in the party, this feels all too similar to the last time they
held the White House.

In 2006, anger at President George W. Bush and unease with the Iraq war
propelled Democrats to reclaim Congress; two years later they captured
the presidency thanks to the same anti-incumbent themes and an
unexpected crisis that accelerated their advantage, the economic
collapse of 2008. The two elections were effectively a single continuous
rejection of Republican rule, as some in the G.O.P. fear 2018 and 2020
could become in a worst-case scenario.

``It already feels very similar to the 2008 cycle,'' said Billy Piper, a
Republican lobbyist and former chief of staff to Senator Mitch
McConnell*.*

Significant questions remain that could tilt the outcome of this
election: whether Americans experience a second wave of the virus in the
fall, the condition of the economy and how well Mr. Biden performs after
he emerges from his Wilmington, Del., basement, which many in his party
are privately happy to keep him in so long as Mr. Trump is fumbling as
he governs amid a crisis.

Image

Defeating Senator Susan Collins, Republican of Maine,~has become a cause
for national liberal activists.Credit...Anna Moneymaker/The New York
Times

But if Republicans are comforted by the uncertainties that remain, they
are alarmed by one element of this election that is already abundantly
clear: The small-dollar fund-raising energy Democrats enjoyed in the
midterms has not abated.

\href{https://www.nytimes3xbfgragh.onion/news-event/2020-election}{Election
2020 ›}

\hypertarget{live-updates}{%
\subsection{\texorpdfstring{\href{https://www.nytimes3xbfgragh.onion/live/2020/09/09/us/trump-vs-biden}{Live
Updates}}{Live Updates}}\label{live-updates}}

\href{https://www.nytimes3xbfgragh.onion/live/2020/09/09/us/trump-vs-biden\#democrats-worry-about-a-partisan-slant-at-the-postal-service-where-trump-allies-dominate-the-board}{}

Sept. 9, 2020, 8:51 a.m. ET

\href{https://www.nytimes3xbfgragh.onion/live/2020/09/09/us/trump-vs-biden\#democrats-worry-about-a-partisan-slant-at-the-postal-service-where-trump-allies-dominate-the-board}{Democrats
worry about a partisan slant at the Postal Service, where Trump allies
dominate the
board.}\href{https://www.nytimes3xbfgragh.onion/live/2020/09/09/us/trump-vs-biden\#the-nras-new-candidate-grades-show-a-continuing-decline-in-support}{}

Sept. 9, 2020, 8:19 a.m. ET

\href{https://www.nytimes3xbfgragh.onion/live/2020/09/09/us/trump-vs-biden\#the-nras-new-candidate-grades-show-a-continuing-decline-in-support}{The
N.R.A.'s new candidate grades show a continuing decline in
support.}\href{https://www.nytimes3xbfgragh.onion/live/2020/09/09/us/trump-vs-biden\#biden-announces-a-plan-to-push-companies-to-keep-jobs-in-the-us}{}

Sept. 9, 2020, 8:15 a.m. ET

\href{https://www.nytimes3xbfgragh.onion/live/2020/09/09/us/trump-vs-biden\#biden-announces-a-plan-to-push-companies-to-keep-jobs-in-the-us}{Biden
announces a plan to push companies to keep jobs in the U.S.}

Most of the incumbent House Democrats facing competitive races enjoy a
vast financial advantage over Republican challengers, who are struggling
to garner attention as the virus overwhelms news coverage.

Still, few officials in either party believed the House was in play this
year. There was also similar skepticism about the Senate. Then the virus
struck and fund-raising reports covering the first three months of this
year were released in mid-April.

Republican senators facing difficult races were not only
\href{https://www.nytimes3xbfgragh.onion/2020/04/16/us/politics/senate-races-2020-fundraising.html}{all
outraised} by Democrats, they were also overwhelmed.

In Maine, for example, Senator Susan Collins brought in \$2.4 million
while her little-known rival, the House speaker Sara Gideon, raised more
than \$7 million. Even more concerning to Republicans is the
lesser-known Thom Tillis of North Carolina. Republican officials are
especially irritated at Mr. Tillis because he has little small-dollar
support and raised only \$2.1 million, which was more than doubled by
his Democratic opponent.

``These Senate first-quarter fund-raising numbers are a serious wake-up
call for the G.O.P.,'' said Scott Reed, the top political strategist at
the U.S. Chamber of Commerce.

The Republican Senate woes come as anger toward Mr. Trump is rising from
some of the party's most influential figures on Capitol Hill.

After working closely with Senate Republicans at the start of the year,
some of the party's top congressional strategists say the handful of
political advisers Mr. Trump retains have communicated little with them
since the health crisis began.

In a campaign steered by Mr. Trump, whose rallies drove fund-raising and
data harvesting, the center of gravity has of late shifted to the White
House. His campaign headquarters will remain closed for another few
weeks, and West Wing officials say the president's campaign manager,
Brad Parscale, hasn't been to the White House since last month, though
he is in touch by phone.

Then there is the president's conduct.

In just the last week, he has
\href{https://twitter.com/maggieNYT/status/1252325824512045058}{undercut}the
efforts of his campaign and his allies to attack Mr. Biden on China;
suddenly proposed a halt on immigration; and said governors should not
move too soon to reopen their economies --- a week after calling on
protesters to ``liberate'' their states. And that was all before his
digression into the potential healing powers of disinfectants.

Image

Senator Thom Tillis, Republican of North Carolina, has little
small-dollar support and raised only \$2.1 million in the first
quarter.~Credit...Pete Marovich for The New York Times

Republican lawmakers have gone from watching his lengthy daily briefings
with a tight-lipped grimace to looking upon them with horror.

``Any of us can be onstage too much,'' said the longtime Representative
Greg Walden of Oregon, noting that ``there's a burnout factor no matter
who you are, you've got to think about that.''

Privately, other party leaders are less restrained about the political
damage they believe Mr. Trump is doing to himself and Republican
candidates. One prominent G.O.P. senator said the nightly sessions were
so painful he could not bear watching any longer.

``I would urge the president to focus on the positive, all that has been
done and how we are preparing for a possible renewal of the pandemic in
the fall,'' said Representative Peter King, Republican of New York.

Asked about concerns over Mr. Trump's briefings, the White House press
secretary, Kayleigh McEnany, said, ``Millions and millions of Americans
tune in each day to hear directly from President Trump and appreciate
his leadership, unprecedented coronavirus response, and confident
outlook for America's future.''

Mr. Trump's thrashing about partly reflects his frustration with the
virus and his inability to slow Mr. Biden's rise in the polls. It's also
an illustration of his broader inability to shift the public
conversation to another topic, something he has almost always been able
to do when confronted with negative story lines ranging from impeachment
proceedings to payouts to adult film stars.

Mr. Trump is also restless. Administration officials said they were
looking to resume his travel in as soon as a week, although campaign
rallies remain distant for now.

As they look for ways to regain the advantage, some Republicans believe
the party must mount an immediate ad campaign blitzing Mr. Biden,
identifying him to their advantage and framing the election as a clear
choice.

``If Trump is the issue, he probably loses,'' said Mr. Black, the
consultant. ``If he makes it about Biden and the economy is getting
better, he has a chance.''

\hypertarget{our-2020-election-guide}{%
\section{Our 2020 Election Guide}\label{our-2020-election-guide}}

Updated ~Sept. 9, 2020

\begin{center}\rule{0.5\linewidth}{\linethickness}\end{center}

\begin{itemize}
\item ~
  \hypertarget{the-latest}{%
  \subsection{The Latest}\label{the-latest}}

  \begin{itemize}
  \item
    Joe Biden heads today to Michigan, a battleground state where
    President Trump has resumed advertising ahead of a visit there on
    Thursday.
    \href{https://www.nytimes3xbfgragh.onion/live/2020/09/09/us/trump-vs-biden?action=click\&pgtype=Article\&state=default\&region=BELOW_MAIN_CONTENT\&context=storylines_guide}{Read
    live updates}.
  \end{itemize}
\item ~
  \hypertarget{how-to-win-270}{%
  \subsection{How to Win 270}\label{how-to-win-270}}

  \begin{itemize}
  \item
    Joe Biden and Donald Trump need 270 electoral votes to reach the
    White House. Try building
    \href{https://www.nytimes3xbfgragh.onion/interactive/2020/us/elections/election-states-biden-trump.html?action=click\&pgtype=Article\&state=default\&region=BELOW_MAIN_CONTENT\&context=storylines_guide}{your
    own coalition of battleground states}~to see potential outcomes.
  \end{itemize}
\item ~
  \hypertarget{voting-by-mail}{%
  \subsection{Voting by Mail}\label{voting-by-mail}}

  \begin{itemize}
  \item
    Will you have enough time to vote by mail in your state? Yes, but
    it's risky to procrastinate.
    \href{https://www.nytimes3xbfgragh.onion/interactive/2020/08/31/us/politics/vote-by-mail-deadlines.html?action=click\&pgtype=Article\&state=default\&region=BELOW_MAIN_CONTENT\&context=storylines_guide}{Check
    your state's deadline.}
  \item
    \href{https://www.nytimes3xbfgragh.onion/interactive/2020/us/elections/joe-biden.html?action=click\&pgtype=Article\&state=default\&region=BELOW_MAIN_CONTENT\&context=storylines_guide}{}

    \hypertarget{joe-biden}{%
    \section{Joe Biden}\label{joe-biden}}

    \hypertarget{democrat}{%
    \subsection{Democrat}\label{democrat}}

    \href{https://www.nytimes3xbfgragh.onion/interactive/2020/us/elections/donald-trump.html?action=click\&pgtype=Article\&state=default\&region=BELOW_MAIN_CONTENT\&context=storylines_guide}{}

    \hypertarget{donald-trump}{%
    \section{Donald Trump}\label{donald-trump}}

    \hypertarget{republican}{%
    \subsection{Republican}\label{republican}}
  \end{itemize}
\item
  \hypertarget{keep-up-with-our-coverage}{%
  \subsection{Keep Up With Our
  Coverage}\label{keep-up-with-our-coverage}}

  \begin{itemize}
  \item
    Get an
    \href{https://www.nytimes3xbfgragh.onion/newsletters/politics?action=click\&pgtype=Article\&state=default\&region=BELOW_MAIN_CONTENT\&context=storylines_guide}{email}~recapping
    the day's news
  \item
    Download our mobile app on
    \href{https://apps.apple.com/us/app/nytimes/id284862083?ls=1\&mat_click_id=5c79ae7455014fd1bd66b5610c05b8f2-20191112-16948\&referrer=mat_click_id\%3D5c79ae7455014fd1bd66b5610c05b8f2-20191112-16948\%26link_click_id\%3D722930677036718082}{iOS}~and
    \href{http://a.localytics.com/android?id=com.nytimes.android\&referrer=utm_source\%3Dother_nyt_mobile_web\%26utm_medium\%3DWeb\%2520page\%26utm_term\%3DGeneral\%2520Mobile\%2520Page\%26utm_campaign\%3DNYT\%2520Mobile\%2520General\%2520Page}{Android}~and
    turn on Breaking News and Politics alerts
  \end{itemize}
\end{itemize}

Advertisement

\protect\hyperlink{after-bottom}{Continue reading the main story}

\hypertarget{site-index}{%
\subsection{Site Index}\label{site-index}}

\hypertarget{site-information-navigation}{%
\subsection{Site Information
Navigation}\label{site-information-navigation}}

\begin{itemize}
\tightlist
\item
  \href{https://help.nytimes3xbfgragh.onion/hc/en-us/articles/115014792127-Copyright-notice}{©~2020~The
  New York Times Company}
\end{itemize}

\begin{itemize}
\tightlist
\item
  \href{https://www.nytco.com/}{NYTCo}
\item
  \href{https://help.nytimes3xbfgragh.onion/hc/en-us/articles/115015385887-Contact-Us}{Contact
  Us}
\item
  \href{https://www.nytco.com/careers/}{Work with us}
\item
  \href{https://nytmediakit.com/}{Advertise}
\item
  \href{http://www.tbrandstudio.com/}{T Brand Studio}
\item
  \href{https://www.nytimes3xbfgragh.onion/privacy/cookie-policy\#how-do-i-manage-trackers}{Your
  Ad Choices}
\item
  \href{https://www.nytimes3xbfgragh.onion/privacy}{Privacy}
\item
  \href{https://help.nytimes3xbfgragh.onion/hc/en-us/articles/115014893428-Terms-of-service}{Terms
  of Service}
\item
  \href{https://help.nytimes3xbfgragh.onion/hc/en-us/articles/115014893968-Terms-of-sale}{Terms
  of Sale}
\item
  \href{https://spiderbites.nytimes3xbfgragh.onion}{Site Map}
\item
  \href{https://help.nytimes3xbfgragh.onion/hc/en-us}{Help}
\item
  \href{https://www.nytimes3xbfgragh.onion/subscription?campaignId=37WXW}{Subscriptions}
\end{itemize}
