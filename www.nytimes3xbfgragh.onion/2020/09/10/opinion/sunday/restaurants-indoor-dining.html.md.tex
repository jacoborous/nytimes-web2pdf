\href{/section/opinion/sunday}{Sunday Review}\textbar{}How to Save
Restaurants

\url{https://nyti.ms/35vLxW4}

\begin{itemize}
\item
\item
\item
\item
\item
\end{itemize}

\includegraphics{https://static01.graylady3jvrrxbe.onion/images/2020/09/13/opinion/09kris1/merlin_176543067_3253c2bd-9f9a-4b60-a4dd-9809da8f83d6-articleLarge.jpg?quality=75\&auto=webp\&disable=upscale}

Sections

\protect\hyperlink{site-content}{Skip to
content}\protect\hyperlink{site-index}{Skip to site index}

\href{/section/opinion}{Opinion}

\hypertarget{how-to-save-restaurants}{%
\section{How to Save Restaurants}\label{how-to-save-restaurants}}

Rebuilding the restaurant business requires a new model for its labor.

Glenn Vaulx III preparing a takeout order at the Four Way in
Memphis.Credit...Whitten Sabbatini for The New York Times

Supported by

\protect\hyperlink{after-sponsor}{Continue reading the main story}

By Priya Krishna

Ms. Krishna is a journalist and the author of the cookbook
``Indian-ish.''

\begin{itemize}
\item
  Sept. 10, 2020
\item
  \begin{itemize}
  \item
  \item
  \item
  \item
  \item
  \end{itemize}
\end{itemize}

When the pandemic hit America's restaurants, it was as if an anvil
dropped --- on a bubble.

To run a restaurant, any kind of restaurant, is a constant struggle to
keep that bubble aloft. Every day is a negotiation: of labor costs, food
costs, rent, insurance, health inspections, and the art and craft of
creating an experience special enough to keep people coming through the
doors. When the pandemic lockdown forced hundreds of thousands of
establishments to close, there was no backup plan. No one was prepared
for the extent of the fallout.

The restaurant and fast food industry, the second-largest private
employer in the United States, collapsed overnight. At least 5.5 million
jobs evaporated by the end of April, and the number of people employed
in food services is still 2.5 million fewer than in February. Technomic,
a consulting firm for the food-service industry, estimates that 20
percent to 25 percent of independently owned restaurants will never
reopen. And those restaurants uphold an ecosystem that extends to farms,
fishmongers, florists, ceramists, wineries and more. The damage has been
so severe that the James Beard Foundation announced in August that it
would cancel its restaurant awards this year because of the pandemic and
a need to re-examine structural bias.

The most deeply affected were restaurant workers, who were either laid
off so that they could file for unemployment or were asked to keep
working and risk their health. These are people who often do not have
access to health insurance, earn less than a living wage and
disproportionately include undocumented workers, immigrants, and Black
and brown people --- the most marginalized people in this country.

\includegraphics{https://static01.graylady3jvrrxbe.onion/images/2020/09/09/opinion/09kris5a/merlin_176542953_dde24ec1-23fa-4461-985a-2ff02caa0c79-articleLarge.jpg?quality=75\&auto=webp\&disable=upscale}

As the country begins to open up and restaurants slowly invite customers
back in --- New York City announced on Wednesday that indoor dining
could resume at 25 percent capacity --- many of those same people are
being asked to come back to work, with no change to their compensation
or promises of assistance in case they get sick. (A number of
restaurants have even had to close after reopening because workers
tested positive for the virus.)

It's unfathomable to imagine a country without restaurants, but even
more unfathomable to imagine a successful economic recovery that doesn't
include restaurant employees. As such a large slice of the American work
force, they are not only essential to growth. How we support them will
be a litmus test for whether the United States can ever build a fair,
equitable economy.

Image

Randi Owens at Rye Plaza in Kansas City, Mo.Credit...Christopher Smith
for The New York Times

Rebuilding the restaurant business can't be just about diversifying
revenue streams. It requires rethinking how employers and patrons value
labor, which means shifting the restaurant model to one that's centered
on workers.

The onus for change should not fall solely on restaurants. The success
of a worker-centered approach, especially in the middle of a recession,
requires cooperation from customers and help from government. With many
restaurants now welcoming customers who are tiring of home cooking after
an extended lockdown, and getting national attention by policymakers,
this is the time to make a structural shift.

The restaurants best equipped to kick-start this change won't be those
that are part of larger empires in densely populated,
high-cost-of-living cities like New York and San Francisco. In fact,
outside the fast-food industry, many restaurants in America have only
one location. These are the places that don't have a large corporate
infrastructure. They are run by a small group of people, often family
members. They buy ingredients from local producers. They are the
lifeblood of their communities, and for them, survival is about more
than just keeping a business running.

Image

Patrice Bates Thompson, owner of the Four Way.Credit...Whitten Sabbatini
for The New York Times

They are restaurants like the Four Way, a 74-year-old soul-food
institution in Memphis. At the beginning of March, it was racking up
record sales of its fried catfish and peach cobbler. Once the pandemic
hit, Patrice Bates Thompson, the owner, had to make changes fast. Like
many other restaurateurs, she shifted to mostly takeout and delivery,
has been buying her employees groceries when purchasing food for the
restaurant and is helping a few of them cover their utility bills.

Nearly every restaurant that survived the pandemic so far has had to
adjust its operations to survive. The shift to takeout is perhaps the
most visible and lasting change for restaurants. Takeout and delivery
services have allowed thousands of restaurants all over the country to
survive, and the experience is evolving as restaurants apply their
creativity to this now ubiquitous form. Seven Reasons, a fine-dining
restaurant in Washington, bundles orders with cocktails in Mason jars
and sends customers links to Spotify playlists to listen to while
they're eating --- a lockdown-appropriate approximation of dining out.
Junzi Kitchen, a mini-chain of Chinese restaurants in New York and
Connecticut, was inspired by restaurants in China to design an
interactive, rotating takeout menu, with an accompanying Instagram Live
by the chef, Lucas Sin, explaining the story behind each dish and how to
plate it. Other restaurants have turned into grocery stores, offering
their premium ingredients to home cooks. The New York restaurateur
Gabriel Stulman started selling meal kits out of his West Village spot,
Jeffrey's Grocery, so that customers can replicate popular dishes at
home.

Image

Lucas Sin, the chef at Junzi Kitchen in Manhattan.Credit...Laylah
Amatullah Barrayn for The New York Times

These innovations will certainly help restaurants to hang on in the near
term, keep paying their employees and even increase revenue once dine-in
service is more prevalent.

But they don't confront the larger issue: The business model of
restaurants is built on the assumption of cheap labor. One out of six
restaurant workers live below the poverty line, according to the
Economic Policy Institute, and the industry has an exceptionally high
turnover rate --- 75 percent in 2018, according to the Bureau of Labor
Statistics, compared with 49 percent for the rest of the private sector.
In other words, jobs in the restaurant industry look increasingly like
gig work --- unstable, poorly paid and with few protections for workers.

Image

Take-out orders at Junzi Kitchen in Manhattan.Credit...Laylah Amatullah
Barrayn for The New York Times

Image

The chef of Junzi Kitchen explains the story behind each dish and how to
plate it on Instagram.Credit...Laylah Amatullah Barrayn for The New York
Times

To make it worse, the practice of tipping front-of-house workers
(servers, bartenders and hosts), which is deeply ingrained in the
culture and business model of restaurants, creates a disparity in income
between front- and back-of-the-house workers, privileges white workers,
who are more likely to work in the front of the house, and
\href{https://www.nytimes3xbfgragh.onion/interactive/2018/03/11/business/tipping-sexual-harassment.html}{fuels
sexual harassment}. The Bureau of Labor Statistics reports that the
median hourly wage
\href{https://www.bls.gov/ooh/food-preparation-and-serving/cooks.htm}{for
cooks is \$12.67} and
\href{https://www.bls.gov/oes/current/oes353031.htm}{\$11 for servers},
but the I.R.S. estimates that about 40 percent of tips go unreported,
which would inflate that server's hourly wage to \$15.40.

Image

Anakaren Ibarra-DumovichCredit...Christopher Smith for The New York
Times

Anakaren Ibarra-Dumovich, a former sous chef at Rye Plaza in Kansas
City, Mo., said that too many restaurants are reopening without
considering what relying on tips for income means for workers. If people
aren't dining out as much, she explained, tip income could fall
drastically and workers could end up returning to the same job for much
less money. But if you're offered work and don't take it, you no longer
qualify for unemployment. ``Either you work and risk potential exposure
and make no money or you lose all of it,'' she said. ``It is scary.''

Image

Stephone Trodman behind the bar at Rye Plaza.Credit...Christopher Smith
for The New York Times

Image

A Rye Plaza server readying plates.Credit...Christopher Smith for The
New York Times

Devita Davison, the executive director of FoodLab Detroit, an incubator
for socially minded food businesses, believes the industry needs to use
this period of upheaval to think more radically. ``The question is not
`What does the restaurant look like?' but `What does it mean to have a
profitable restaurant?''' she said. ``Because guess what, for the sake
of profitability, who suffers?''

Image

Devita Davison, executive director of FoodLab Detroit.~Credit...Erin
Kirkland for The New York Times

This is the harder question, and one that a few restaurants are starting
to answer. One of FoodLab's partner restaurants, PizzaPlex, in southwest
Detroit, is already working to generate enough revenue to become a
worker-owned cooperative, where every employee has a financial stake in
the business and has a say in major decisions. For the owners,
Alessandra Carreon and Drew McUsic, the goal is not to maximize profits
for the two of them but to redistribute the wealth generated by their
restaurant back into the community.

Image

PizzaPlex's mobile pizza oven in the restaurant's outdoor patio
area.Credit...Erin Kirkland for The New York Times

Image

PizzaPlex, in southwest Detroit, is working to generate enough revenue
to become a worker-owned cooperative.Credit...Erin Kirkland for The New
York Times

There are other ways for a restaurant to change its relationship with
workers. Melissa Miranda owns Musang in Seattle, a restaurant she calls
``Filipinx'' to honor the genderqueer people she knows among immigrants
from the Philippines in white American neighborhoods. She instituted a
single hourly wage across the board, with a tip pool divided by hours of
work. She closed Musang for dine-in service before the stay-at-home
order was issued in Seattle, because many of her employees live in
multigenerational homes with older family members.

Image

Melissa Miranda, chef and owner of Musang in Seattle.Credit...Grant
Hindsley for The New York Times

She turned the restaurant into a community kitchen open two days a week,
where people can pick up a free meal on a first come first served basis.
Her team delivers meals three days a week, and that work is supported by
donations from suppliers and customers. (About 75 percent of online
orders include donations, she said.) Musang's takeout and outdoor dining
sales are buoying the kitchen's work as well, now that revenue has
returned almost to what it was before the pandemic and the restaurant is
profitable again.

The pandemic forced Ms. Miranda to re-examine what it means to run a
restaurant: How do you provide health insurance during a pandemic when
margins are so slim? What do you do with front-of-house workers when
you're a long way from dine-in service? How can you mentor employees and
encourage them to become business owners themselves?

Image

Jesse Tiamson preparing a cocktail as part of a to-go order at
Musang.Credit...Grant Hindsley for The New York Times

Image

A bountiful table at Musang.Credit...Grant Hindsley for The New York
Times

She has made some concrete changes. She eliminated tipping and plans to
offer everyone on her staff health care and retirement benefits. She has
reduced the size of her staff so that she can pay them more --- between
\$25 and \$30 an hour --- and spend more time on training them and
teaching them about the business. ``I have worked in this industry a
long time,'' Ms. Miranda told me. ``I never had a 401(k) or benefits or
anyone looking out for my financial future.'' Musang, she hopes, will be
different. The old-school model of restaurants is exclusively about
revenue. ``We didn't build this restaurant for that,'' she said. ``We
built this restaurant with the intent to make change.'' She wants it to
be the last place her employees work before they open up their own
restaurant.

Francesca Hong, the chef and a co-owner of Morris Ramen in Madison,
Wis., also started a community kitchen, which she and another restaurant
group, Rule No. One, have expanded into an initiative called Cook It
Forward Madison. The project works with nonprofits to provide meals to
people in need. In turn, the nonprofits provide the restaurant with
financial and technical assistance, like accounting and legal aid.

Image

Francesca Hong, the chef and a co-owner of Morris Ramen in Madison,
Wis.Credit...Lauren Justice for The New York Times

That aid, plus donations through Cook It Forward Madison, allowed her to
offer 100 percent of her staff their jobs back. (About three-quarters of
them accepted.) It also means that she gets to keep buying from local
farms and other producers, ensuring that those benefiting from the
restaurant include not just her employees, but also the broader
community of workers that support it. (Ms. Hong has also become more
active in the community politically;
\href{https://madison.com/wsj/news/local/govt-and-politics/francesca-hong-wins-packed-race-for-assembly-district-76-on-madisons-liberal-isthmus/article_17c27bb6-46da-526f-9c09-5e4a7d3bda10.html}{she
is a candidate for a Wisconsin State Assembly seat}.)

Image

Matt Morris preparing ramen carry-out orders at Morris
Ramen.Credit...Lauren Justice for The New York Times

Image

Specials and events at Morris Ramen.Credit...Lauren Justice for The New
York Times

All of this effort to come up with new business models will accomplish
little unless restaurant patrons understand the true costs of labor.
Communication is key. Ms. Miranda, for example, has raised menu prices
while trying to keep the restaurant accessible to locals. ``If we were
to introduce a 20 percent service charge, that would be a bit of a
shock,'' she said. So she has shifted the restaurant's repertoire to
more vegetable-heavy dishes, so that she can spend less on meat and more
on employee compensation. The menu includes a note to customers on how
the cost of taking care of employees is factored into food prices. Not
being transparent with guests ``is where the pushback happens,'' she
told me.

Image

A family dinner at Musang.Credit...Grant Hindsley for The New York Times

The rustic Berkshires restaurant, the Prairie Whale in Great Barrington,
Mass., reopened in June for outdoor dining, adding a 3 percent kitchen
service charge to offset the pay discrepancy between its front-of-house
and back-of-house workers. Claire Sprouse, who runs Hunky Dory, an
all-day cafe in Crown Heights, Brooklyn, announced in July that she was
reopening the restaurant's outdoor patio, but with no tipping and
slightly increased menu prices.

Image

Claire Sprouse, owner of Hunky Dory in Brooklyn.Credit...Laylah
Amatullah Barrayn for The New York Times

The biggest argument against worker-centric systems is an economic one:
Who is going to want, much less be able, to pay more for meals in the
middle of a recession? And reduced capacity in restaurants will also
mean reduced labor. Not every restaurant will be able to make big
changes; many have always been in survival mode and lack the resources
to alter how they do business. When Jacklyn Pham's father opened Saigon
Pagolac, a Vietnamese restaurant in Houston's Chinatown, in 1989, he
didn't have a mission, she said. Cooking was simply what he knew how to
do. He still does inventory with pen and paper. At the beginning of
2020, sales plummeted because anti-Chinese sentiment from the
coronavirus slowed traffic to Chinatown. The restaurant did takeout
through March and April, and reopened for dine-in service in May, as
soon as Texas allowed it. There was no other way, Ms. Pham said. There
were bills to pay.

At Monkey 68 in Roswell, Ga., which reopened to the public at 50 percent
capacity in mid-May, Tay Wunn, the general manager, is focused on how to
follow safety guidelines while still making money with fewer customers.
Mr. Wunn doesn't feel ready to ponder structural changes when the
restaurant is making half what it did before the pandemic. He reopened
because ``we needed the revenue,'' he said, and because so many of the
employees did not receive unemployment benefits and wanted to get back
to work.

Image

Tay Wunn, general manager of Monkey 68 in Roswell, Ga.Credit...Audra
Melton for The New York Times

Building a labor-centered model for restaurants may feel quixotic, an
idea that won't work at scale. But because most restaurants are
small-scale operations, the solutions don't need to be all-encompassing.
There are many ways, big and small, that a restaurant can value labor.
They can do it by eliminating tipping or switching to a cooperative
model, but a new model can also mean cutting down on food waste and
adding the savings to employees' paychecks or lobbying for government
policies that support workers' and immigrants' rights. For Ms. Thompson,
of the Four Way in Memphis, it means helping her employees pay their
bills while she researches health benefits packages for them.

Every restaurant should be able to find a model narrowly tailored to its
workers and community, so long as there is also a broader public safety
net. **** For an industrywide shift to take place, some government
assistance will be essential. In the United States, federal assistance
came via the Paycheck Protection Program, part of the emergency
Coronavirus Aid, Relief, and Economic Security Act passed in late March.
But the first round of the P.P.P. allocated only 9 percent of its loans
to the hospitality sector, and most of it went to chains with far
greater resources than independent restaurants.

Image

Workers at Monkey 68 preparing sushi orders for take-out.Credit...Audra
Melton for The New York Times

Image

Monkey 68 is focused on still making money with fewer
customers.Credit...Audra Melton for The New York Times

In the absence of bipartisan support for more wide-reaching federal
measures, especially universal health care, states and municipalities
will have to step in to fill the vacuum of national leadership, as they
have throughout the pandemic, to create a safety net for restaurants and
workers. ``The great thing about our federal system is that each state
can be a laboratory to experiment with policies,'' said David Henkes, a
senior principal at Technomic. A tax credit for providing health
insurance may work well in one state, and a stronger policy on rent
relief in another. Both are policies that ease the financial and
operational burden on restaurants, allowing them to invest in their
workers. Because the restaurant industry touches so many parts of the
economy, government assistance will help not just restaurants, but also
the broader ecosystem of farms and other suppliers they work with.

Image

Outside~Hunky Dory in Brooklyn.Credit...Laylah Amatullah Barrayn for The
New York Times

With no federal reopening regulations and no official customer guidance,
restaurants, workers and diners are left to make ethical calculations of
their own. Close forever or reopen under unsafe conditions? Take a job
in harm's way or forfeit a paycheck? Support a local restaurant or risk
a server's health for a plate of enchiladas?

These questions all come down to labor, and the willingness of
government, restaurant owners and customers to value it. More important
than any specific policy is an acknowledgment that the restaurant system
that we have all bought into for so long is broken. A system built to
serve the privileged by hurting the most vulnerable is not a system
worth having. Not in a pandemic, and not ever.

\hypertarget{restaurant-workers-how-has-the-pandemic-affected-you}{%
\subsection{Restaurant workers, how has the pandemic affected
you?}\label{restaurant-workers-how-has-the-pandemic-affected-you}}

Priya Krishna (\href{https://twitter.com/priyakrishna}{@priyakrishna})
is a journalist and the author of the cookbook ``Indian-ish.''

\emph{The Times is committed to publishing}
\href{https://www.nytimes3xbfgragh.onion/2019/01/31/opinion/letters/letters-to-editor-new-york-times-women.html}{\emph{a
diversity of letters}} \emph{to the editor. We'd like to hear what you
think about this or any of our articles. Here are some}
\href{https://help.nytimes3xbfgragh.onion/hc/en-us/articles/115014925288-How-to-submit-a-letter-to-the-editor}{\emph{tips}}\emph{.
And here's our email:}
\href{mailto:letters@NYTimes.com}{\emph{letters@NYTimes.com}}\emph{.}

\emph{Follow The New York Times Opinion section on}
\href{https://www.facebookcorewwwi.onion/nytopinion}{\emph{Facebook}}\emph{,}
\href{http://twitter.com/NYTOpinion}{\emph{Twitter (@NYTopinion)}}
\emph{and}
\href{https://www.instagram.com/nytopinion/}{\emph{Instagram}}\emph{.}

Advertisement

\protect\hyperlink{after-bottom}{Continue reading the main story}

\hypertarget{site-index}{%
\subsection{Site Index}\label{site-index}}

\hypertarget{site-information-navigation}{%
\subsection{Site Information
Navigation}\label{site-information-navigation}}

\begin{itemize}
\tightlist
\item
  \href{https://help.nytimes3xbfgragh.onion/hc/en-us/articles/115014792127-Copyright-notice}{©~2020~The
  New York Times Company}
\end{itemize}

\begin{itemize}
\tightlist
\item
  \href{https://www.nytco.com/}{NYTCo}
\item
  \href{https://help.nytimes3xbfgragh.onion/hc/en-us/articles/115015385887-Contact-Us}{Contact
  Us}
\item
  \href{https://www.nytco.com/careers/}{Work with us}
\item
  \href{https://nytmediakit.com/}{Advertise}
\item
  \href{http://www.tbrandstudio.com/}{T Brand Studio}
\item
  \href{https://www.nytimes3xbfgragh.onion/privacy/cookie-policy\#how-do-i-manage-trackers}{Your
  Ad Choices}
\item
  \href{https://www.nytimes3xbfgragh.onion/privacy}{Privacy}
\item
  \href{https://help.nytimes3xbfgragh.onion/hc/en-us/articles/115014893428-Terms-of-service}{Terms
  of Service}
\item
  \href{https://help.nytimes3xbfgragh.onion/hc/en-us/articles/115014893968-Terms-of-sale}{Terms
  of Sale}
\item
  \href{https://spiderbites.nytimes3xbfgragh.onion}{Site Map}
\item
  \href{https://help.nytimes3xbfgragh.onion/hc/en-us}{Help}
\item
  \href{https://www.nytimes3xbfgragh.onion/subscription?campaignId=37WXW}{Subscriptions}
\end{itemize}
