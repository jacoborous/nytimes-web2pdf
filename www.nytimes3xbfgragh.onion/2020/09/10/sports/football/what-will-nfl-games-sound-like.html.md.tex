Sections

SEARCH

\protect\hyperlink{site-content}{Skip to
content}\protect\hyperlink{site-index}{Skip to site index}

\href{https://www.nytimes3xbfgragh.onion/section/sports/football}{Pro
Football}

\href{https://myaccount.nytimes3xbfgragh.onion/auth/login?response_type=cookie\&client_id=vi}{}

\href{https://www.nytimes3xbfgragh.onion/section/todayspaper}{Today's
Paper}

\href{/section/sports/football}{Pro Football}\textbar{}What Will N.F.L.
Games Sound Like?

\url{https://nyti.ms/2ZpiCPx}

\begin{itemize}
\item
\item
\item
\item
\item
\end{itemize}

Advertisement

\protect\hyperlink{after-top}{Continue reading the main story}

Supported by

\protect\hyperlink{after-sponsor}{Continue reading the main story}

\hypertarget{what-will-nfl-games-sound-like}{%
\section{What Will N.F.L. Games Sound
Like?}\label{what-will-nfl-games-sound-like}}

N.F.L. teams have spent years trying to create over-the-top
entertainment for fans inside stadiums. This year, they'll just be
trying to cover up echoes from empty seats.

\includegraphics{https://static01.graylady3jvrrxbe.onion/images/2020/09/10/sports/10nfl-stadiumsWEB2/merlin_147690540_02204230-29f8-4fe6-949f-8bda1c42b6c3-articleLarge.jpg?quality=75\&auto=webp\&disable=upscale}

\href{https://www.nytimes3xbfgragh.onion/by/ken-belson}{\includegraphics{https://static01.graylady3jvrrxbe.onion/images/2018/02/16/multimedia/author-ken-belson/author-ken-belson-thumbLarge.jpg}}

By \href{https://www.nytimes3xbfgragh.onion/by/ken-belson}{Ken Belson}

\begin{itemize}
\item
  Sept. 10, 2020
\item
  \begin{itemize}
  \item
  \item
  \item
  \item
  \item
  \end{itemize}
\end{itemize}

\href{https://www.nytimes3xbfgragh.onion/2020/09/10/sports/football/nfl-picks-week-1.html}{N.F.L}.
games have become high-tech circuses. Players run through flames as they
enter the field, pyrotechnics are blasted after touchdowns and
heart-pounding music shakes the rafters, all in an effort to get the
home crowd to be as loud as possible.

But as the
\href{https://www.nytimes3xbfgragh.onion/2020/09/10/sports/football/nfl-season-open-covid.html}{2020
season} begins, nearly every N.F.L. team will enter uncharted territory
as they host games without fans. The ``12th Man,'' as raucous home
football crowds are called, is for now a victim of the pandemic.

The strict rules governing who can be in each stadium will dramatically
alter the look and feel of games. There will be no cheerleaders on the
sidelines, fewer cameras will pan the stands, and the national anthem
will be sung live elsewhere or taped in advance. Because all
\href{https://www.nytimes3xbfgragh.onion/2020/07/14/sports/football/nfl-players-training-camp.html}{preseason
games were canceled}, this weekend will be the first chance for many
stadium operators to produce their new game-day programs in real time.

Yet facing
\href{https://www.nytimes3xbfgragh.onion/2020/07/02/sports/football/nfl-salary-cap-no-fans.html}{billions
of dollars in lost revenue}, the league decided to let teams host some
fans if state and local guidelines permit, creating a lack of uniformity
among franchises. Some teams like the Kansas City Chiefs will welcome
tens of thousands of fans for each game right away while others,
\href{https://www.nytimes3xbfgragh.onion/2020/07/20/sports/football/jets-giants-rutgers-fans-metlife-stadium.html}{like
the Giants and Jets} and
\href{https://www.nfl.com/news/raiders-allegiant-stadium-will-be-closed-to-fans-for-2020-season}{Las
Vegas Raiders}, do not expect to host fans this season. Others will
pivot during the season, adjusting to add fans as local rules on large
gatherings change.

Coaches whose teams will play without fans to start the season have
complained that the imbalance runs counter to the N.F.L.'s goal of
maintaining what it calls ``competitive equity.''

``I think it's honestly ridiculous that there will be, on the surface,
what appears to be a playing field that's like that --- inconsistently
across the league with the different away stadiums,'' Buffalo Bills
coach Sean McDermott
\href{https://twitter.com/Marcel_LJ/status/1297894552364363777}{told
ESPN}. Spectators will not be allowed at Bills home games, but the team
visits Miami in Week 2 to play the Dolphins, who will host some fans.

Though the league's Competition Committee decided that there wasn't
enough of an advantage to rule out having fans anywhere --- Roger
Goodell, the N.F.L. commissioner, said that stadiums have ``varying
capacities'' even in normal years --- the N.F.L. has tacitly
acknowledged the importance that fans play in supporting home teams and
confusing opponents.

\hypertarget{when-will-your-team-allow-fans}{%
\subsection{When Will Your Team Allow
Fans?}\label{when-will-your-team-allow-fans}}

Only a handful of N.F.L. teams are allowing fans into their stadiums for
their first home games. Here are the plans for each team.

No fans for first two games

No fans through Sept.

No fans to start season

No fans through Sept.

No fans for season opener

No fans until further notice

No fans for season opener

About 10 percent of capacity

No decision yet (Texas allows up to 50\%)

No fans for season opener

No fans through October

No fans first two games

No fans through Sept.

Up to 2,500 fans for opener

25 percent of capacity

22 percent of capacity

No fans until further notice

No fans until further notice

No fans this season

Up to 13,000 fans to start

No fans through Sept.

No fans through Sept.

No fans for season opener

No fans until further notice

No fans until further notice

No fans until further notice

No fans first two games

No fans for opener

No fans for first three games

No fans for first two games

No fans through Sept.

No fans until further notice

Arizona Cardinals

Atlanta Falcons

Baltimore Ravens

Buffalo Bills

Carolina Panthers

Chicago Bears

Cincinnati Bengals

Cleveland Browns

Dallas Cowboys

Denver Broncos

Detroit Lions

Green Bay Packers

Houston Texans

Indianapolis Colts

Jacksonville Jaguars

Kansas City Chiefs

Los Angeles Chargers

Los Angeles Rams

Las Vegas Raiders

Miami Dolphins

Minnesota Vikings

New England Patriots

New Orleans Saints

New York Giants

New York Jets

Philadelphia Eagles

Pittsburgh Steelers

San Francisco 49ers

Seattle Seahawks

Tampa Bay Buccaneers

Tennessee Titans

Washington Football Team

No fans for first two games

No fans through Sept.

No fans to start season

No fans through Sept.

No fans for season opener

No fans until further notice

No fans for season opener

About 10 percent of capacity

No decision yet (Texas allows up to 50\%)

No fans for season opener

No fans through October

No fans first two games

No fans through Sept.

Up to 2,500 fans for opener

25 percent of capacity

22 percent of capacity

No fans until further notice

No fans until further notice

No fans this season

Up to 13,000 fans to start

No fans through Sept.

No fans through Sept.

No fans for season opener

No fans until further notice

No fans until further notice

No fans until further notice

No fans first two games

No fans for opener

No fans for first three games

No fans for first two games

No fans through Sept.

No fans until further notice

Arizona

Atlanta

Baltimore

Buffalo

Carolina

Chicago

Cincinnati

Cleveland

Dallas

Denver

Detroit

Green Bay

Houston

Indianapolis

Jacksonville

Kansas City

L.A. Chargers

L.A. Rams

Las Vegas

Miami

Minnesota

New England

New Orleans

N.Y Giants

N.Y. Jets

Philadelphia

Pittsburgh

San Francisco

Seattle

Tampa Bay

Tennessee

Washington

Arizona

Atlanta

Baltimore

Buffalo

Carolina

Chicago

Cincinnati

Cleveland

Dallas

Denver

Detroit

Green Bay

Houston

Indianapolis

Jacksonville

Kansas City

No fans for first two games

No fans through Sept.

No fans to start season

No fans through Sept.

No fans for season opener

No fans until further notice

No fans for season opener

About 10 percent of capacity

No decision yet (Texas allows up to 50\%)

No fans for season opener

No fans through October

No fans first two games

No fans through Sept.

Up to 2,500 fans for opener

25 percent capacity

22 percent capacity

L.A. Chargers

L.A. Rams

Las Vegas

Miami

Minnesota

New England

New Orleans

N.Y Giants

N.Y. Jets

Philadelphia

Pittsburgh

San Francisco

Seattle

Tampa Bay

Tennessee

Washington

No fans until further notice

No fans until further notice

No fans this season

Up to 13,000 fans to start

No fans through Sept.

No fans through Sept.

No fans for season opener

No fans until further notice

No fans until further notice

No fans until further notice

No fans first two games

No fans for opener

No fans for first three games

No fans for first two games

No fans through Sept.

No fans until further notice

By The New York Times

Last week, the N.F.L. sent teams instructions for how stadium operators
can make up for the lack of fans. They include the use of
``league-curated audio'' like a loop of prerecorded crowd noise
collected from N.F.L. games played no louder than 70 decibels. It will
be similar to the piped-in noise that has filled in for fans in the
N.B.A., W.N.B.A., and
\href{https://www.nytimes3xbfgragh.onion/2020/06/16/sports/coronavirus-stadium-fans-crowd-noise.html}{England's
Premier League} to help cover up the on-field chatter that may run afoul
of broadcast standards.

Viewers watching on television will hear not only what's going on in the
stadium, but crowd noise that was recorded over the past four years by
N.F.L. Films. ``Sound palettes'' were created with noise from each
team's home stadium that will be mixed into broadcasts.

N.F.L. Films grouped hundreds of sounds into positive and negative
categories and low, medium and high volumes that audio engineers at Fox,
NBC and other networks can match with the plays on the field. In between
plays, ambient sound will be played from a pool of sounds that includes
team chants and boos.

``Each stadium sounds unique and has its own local flavor,'' said Vince
Caputo, the supervising sound mixer at N.F.L. Films. ``So we decided,
why not lean in and make it as authentic as possible so we could say to
Eagles fans, `you're listening to you.'''

Caputo's crew also created sound sequences to be played in stadiums,
like call-and-answer chants that include music, which must be played no
louder than 75 decibels so teams can't overcompensate for the lack of
fans by turning up the sound. For instance, in Green Bay, the audio
mixer can play the music that precedes the ``Go Pack Go'' cheer and then
play the recording of the chant. In stadiums where there are fans,
broadcasters will also have live crowd noise to mix in.

\includegraphics{https://static01.graylady3jvrrxbe.onion/images/2020/09/10/sports/10nfl-stadiumsWEB1/merlin_176579382_29f101e5-f2a6-45b2-a2f1-7f918deb06c2-articleLarge.jpg?quality=75\&auto=webp\&disable=upscale}

Unlike most teams, which will start the season playing in front of empty
seats, the Dolphins plan to welcome up to 13,000 fans, or about 20
percent of capacity. To maintain safe distancing, cheerleaders, T.D. the
mascot and honorees like veterans and former players will not be allowed
on the field. Kim Rometo, the chief information officer for the
Dolphins, and Laura Sandall, the vice president of marketing, trimmed
their game day staff by about 20 percent, to two dozen or so workers.
The hype team and the workers who build the tunnel for players to run
through will be reduced.

The remaining staff will rework their operations. Some staff, like the
closed caption operator, will work from home on game days. Seats around
the camera wells have been blocked off so television cameramen are a
safe distance from fans. Video editors this season will work in rooms
adjacent to the control room to preserve physical distancing.

The Dolphins and other teams are offsetting the loss of exposure by
continuing to promote their sponsors on their social media accounts and
other digital platforms, some of which are available only to season
ticketholders.

``A lot of teams have said they will downsize, so they'll have a video
production show, but it will be less elaborate,'' said Will Ellerbruch,
who works in the live events group at Daktronics, the scoreboard
manufacturer that has helped teams automate functions, like the playing
of specific audio clips after certain plays on the field, as well as
streaming video.

Even with all the planning, the size of crowds could change as local
governments re-examine the rate of infection. In addition to the
league's guidance, Rometo said the team created a 65-page guide for
Dolphins home games that included a bevy of scenarios for how to safely
position the workers who run the stadium's cameras, scoreboards and
sound system, as well as the cheerleaders and mascot, whether there were
fans in the seats or not.

``We have plans for all these scenarios and can execute on all of them,
but we haven't had a pandemic in 100 years,'' Rometo said. ``You try
your best, but you just don't know what will happen.''

Advertisement

\protect\hyperlink{after-bottom}{Continue reading the main story}

\hypertarget{site-index}{%
\subsection{Site Index}\label{site-index}}

\hypertarget{site-information-navigation}{%
\subsection{Site Information
Navigation}\label{site-information-navigation}}

\begin{itemize}
\tightlist
\item
  \href{https://help.nytimes3xbfgragh.onion/hc/en-us/articles/115014792127-Copyright-notice}{©~2020~The
  New York Times Company}
\end{itemize}

\begin{itemize}
\tightlist
\item
  \href{https://www.nytco.com/}{NYTCo}
\item
  \href{https://help.nytimes3xbfgragh.onion/hc/en-us/articles/115015385887-Contact-Us}{Contact
  Us}
\item
  \href{https://www.nytco.com/careers/}{Work with us}
\item
  \href{https://nytmediakit.com/}{Advertise}
\item
  \href{http://www.tbrandstudio.com/}{T Brand Studio}
\item
  \href{https://www.nytimes3xbfgragh.onion/privacy/cookie-policy\#how-do-i-manage-trackers}{Your
  Ad Choices}
\item
  \href{https://www.nytimes3xbfgragh.onion/privacy}{Privacy}
\item
  \href{https://help.nytimes3xbfgragh.onion/hc/en-us/articles/115014893428-Terms-of-service}{Terms
  of Service}
\item
  \href{https://help.nytimes3xbfgragh.onion/hc/en-us/articles/115014893968-Terms-of-sale}{Terms
  of Sale}
\item
  \href{https://spiderbites.nytimes3xbfgragh.onion}{Site Map}
\item
  \href{https://help.nytimes3xbfgragh.onion/hc/en-us}{Help}
\item
  \href{https://www.nytimes3xbfgragh.onion/subscription?campaignId=37WXW}{Subscriptions}
\end{itemize}
