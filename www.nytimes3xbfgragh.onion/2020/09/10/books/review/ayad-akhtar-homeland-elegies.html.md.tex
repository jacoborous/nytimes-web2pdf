Sections

SEARCH

\protect\hyperlink{site-content}{Skip to
content}\protect\hyperlink{site-index}{Skip to site index}

\href{https://www.nytimes3xbfgragh.onion/section/books/review}{Book
Review}

\href{https://myaccount.nytimes3xbfgragh.onion/auth/login?response_type=cookie\&client_id=vi}{}

\href{https://www.nytimes3xbfgragh.onion/section/todayspaper}{Today's
Paper}

\href{/section/books/review}{Book Review}\textbar{}`Homeland Elegies'
Sings for a Fading Dream of National Belonging

\url{https://nyti.ms/35mM1gV}

\begin{itemize}
\item
\item
\item
\item
\item
\end{itemize}

Advertisement

\protect\hyperlink{after-top}{Continue reading the main story}

Supported by

\protect\hyperlink{after-sponsor}{Continue reading the main story}

Fiction

\hypertarget{homeland-elegies-sings-for-a-fading-dream-of-national-belonging}{%
\section{`Homeland Elegies' Sings for a Fading Dream of National
Belonging}\label{homeland-elegies-sings-for-a-fading-dream-of-national-belonging}}

\includegraphics{https://static01.graylady3jvrrxbe.onion/images/2020/09/13/books/review/13Kunzru/13Kunzru-articleLarge.jpg?quality=75\&auto=webp\&disable=upscale}

Buy Book ▾

\begin{itemize}
\tightlist
\item
  \href{https://www.amazon.com/gp/search?index=books\&tag=NYTBSREV-20\&field-keywords=Homeland+Elegies+Ayad+Akhtar}{Amazon}
\item
  \href{https://du-gae-books-dot-nyt-du-prd.appspot.com/buy?title=Homeland+Elegies\&author=Ayad+Akhtar}{Apple
  Books}
\item
  \href{https://www.anrdoezrs.net/click-7990613-11819508?url=https\%3A\%2F\%2Fwww.barnesandnoble.com\%2Fw\%2F\%3Fean\%3D9780316496421}{Barnes
  and Noble}
\item
  \href{https://www.anrdoezrs.net/click-7990613-35140?url=https\%3A\%2F\%2Fwww.booksamillion.com\%2Fp\%2FHomeland\%2BElegies\%2FAyad\%2BAkhtar\%2F9780316496421}{Books-A-Million}
\item
  \href{https://bookshop.org/a/3546/9780316496421}{Bookshop}
\item
  \href{https://www.indiebound.org/book/9780316496421?aff=NYT}{Indiebound}
\end{itemize}

When you purchase an independently reviewed book through our site, we
earn an affiliate commission.

By Hari Kunzru

\begin{itemize}
\item
  Sept. 10, 2020
\item
  \begin{itemize}
  \item
  \item
  \item
  \item
  \item
  \end{itemize}
\end{itemize}

\textbf{HOMELAND ELEGIES}\\
By Ayad Akhtar

The city of Abbottabad, in the former North-West Frontier Province of
Pakistan, was named after James Abbott, a 19th-century British Army
officer and player in the ``Great Game,'' the power struggle in Central
Asia between the British and Russian Empires. Today it's perhaps best
known as the garrison town that sheltered Osama bin Laden before he was
\href{https://www.nytimes3xbfgragh.onion/2011/05/02/world/asia/osama-bin-laden-is-killed.html}{discovered
and summarily executed} by American Special Forces in 2011. When the
narrator of Ayad Akhtar's moving and confrontational novel ``Homeland
Elegies'' goes there with his father in 2008 to visit relatives, he gets
a lecture from his uncle about the tactical genius of 9/11, and his
vision of a Muslim community based on principles espoused by the Prophet
Muhammad and his companions, one that ``does not bifurcate its military
and political aspirations.''

The narrator, like Akhtar, is an American-born dramatist, whose own
politics have been formed by a childhood in suburban Milwaukee and a
liberal arts education. While he disagrees with his uncle, sitting in
the man's Raj-era bungalow with William Morris wallpaper, the narrator
finds it easiest to listen without giving an opinion. His father, a
staunch American patriot and future Trump voter, is enraged. ``Trust
me,'' he snaps on the taxi ride home, ``you don't have a clue how
terrible your life would have been if I'd stayed here.''

\includegraphics{https://static01.graylady3jvrrxbe.onion/images/2020/08/19/books/review/Kunzru1/Kunzru1-articleLarge.jpg?quality=75\&auto=webp\&disable=upscale}

The political complexities of Abbottabad are inseparable from the
tensions within the narrator's family, and this fraught visit is just
one of a cascade of scenes and stories that vibrate with the stressful
contradictions of an American Muslim life. Like Akhtar's dramas
(``\href{https://www.nytimes3xbfgragh.onion/2012/10/23/theater/reviews/disgraced-by-ayad-akhtar-with-aasif-mandvi.html}{Disgraced},''
``The Invisible Hand''), ``Homeland Elegies'' deals in ambiguities that
were beyond the pale of public discourse in the years after 9/11. The
many unacknowledged failures of American policy and the coarsening of
popular attitudes form the matrix in which Akhtar's stories grow. He has
an unerring sense for the sore spots, the bitter truths that have
emerged from this history.

At one point, the narrator identifies as part of the ``Muslim world,''
noting that ``despite our ill usage at the hands of the American empire,
the defiling of America-as-symbol enacted on that fateful Tuesday in
September would only bring home anew to all the profundity of that
symbol's power.'' Then, in the same paragraph, he switches, to ``speak
as an American'' of how ``the world looked to us \ldots{} to uphold a
holy image, or as holy as it gets in this age of enlightenment.'' The
paradox is that only people who see the United States as ``the earthly
garden, the abundant idyll'' would have such a jealous compulsion to
destroy it. On either side of the ideological one-way mirror, the
spectacle of American exceptionalism mesmerizes.

``Homeland Elegies'' is presented as a novel, Akhtar's second, but often
reads like a series of personal essays, each one illustrating yet
another intriguing facet of the narrator's prismatic identity. Like all
autofiction, it induces the slightly prurient \emph{frisson} of
``truthiness,'' the genre's signature affect. The narrator, like Akhtar,
has won a Pulitzer Prize for drama. What other parts are ``true''? The
syphilis? The sudden windfall from shady investments? We are given a
portrait of a writer in the round, a sophisticated observer who is also
a newly minted member of the cultural elite, a little dazzled by the
bright lights but eager to heap his plate at the sexual and financial
buffet. For a while, he hobnobs with celebrities and billionaires,
imagining that he is ``penning a coruscating catalog of the new
aristocracy.'' Eventually he realizes that he is nothing more than a
``neoliberal courtier.''

The narrator finds himself thinking of Walt Whitman, and in particular
the poet's claims to be able to express through his
``\href{https://www.poetryfoundation.org/poems/48857/ones-self-i-sing}{simple
separate person}'' some kind of collective American experience. ``My
tongue, too, is homegrown,'' Akhtar writes, ``every atom of this blood
formed of this soil, this air. But these multitudes will not be my
own.'' ``Homeland Elegies'' is about being denied membership to the
Whitmanian crowd, a wound inflicted by 9/11 that has been painful for
many American Muslims, particularly those who feel ``at home,'' or
assumed they were, or aspired to be. The elegies of Akhtar's title are
sung for a dream of national belonging that has only receded since 2001.

Image

The reader's experience of the book is one of fragmentation. Akhtar
tells stories that fracture and ramify and negate. Sometimes they're
comic, like the visit to an absurd Sufi ceremony led by an Austrian
heiress. Sometimes they're wrenchingly tragic. The narrator's 9/11 tale
is one of abjection: He wets himself in terror after being harassed by
an Islamophobic man as he waits to give blood at St. Vincent's Hospital
in the West Village. To protect himself from further attacks, he steals
a crucifix pendant from a Salvation Army store and wears it for several
months, a camouflage that carries more than a tint of cultural shame.
His Pakistani-American girlfriend is shocked when he confesses, years
later. She could never wear a cross. ``We bought flags,'' she says.

The book's most memorable creation (or re-creation) is the narrator's
father, a larger-than-life figure whose most cherished memory is of the
time he spent as Donald Trump's doctor. He is a ``great fan of America''
who keeps a copy of ``The Art of the Deal'' in the living room, ``an
imam's son whose only sacred names \ldots{} were those of the big
California cabernets he adored.'' The family fortunes rise and fall as
he wastes the money he makes as a cardiologist on Trumpian real estate
scheming. Finally, after a series of personal and professional
disasters, his bluster fades, and his son concludes that ``he thinks
he's American, but what that really means is that he still \emph{wants}
to be American. He still doesn't really feel like one.''

Akhtar arranges people and situations with a dramatist's care to expose
the fault lines where community or communication cracks. Sometimes, the
pieces seem almost too carefully arranged. A Pennsylvania state trooper
stops the narrator and engages him in a probing conversation about
Lawrence Wright's
``\href{https://www.nytimes3xbfgragh.onion/2006/08/01/books/01kaku.html}{The
Looming Tower}.'' A cosmopolitan aunt, a university teacher of critical
theory who makes her young nephew read Fanon and Edward Said, draws the
line at ``The Satanic Verses.''

The unease reaches a high pitch with the narrator's trip to Los Angeles
to take meetings after he wins the Pulitzer. A Black Republican film
agent explains what he considers to be the fundamentally Jewish
character of Hollywood, ``founded by families from New York's garment
district,'' who value ``novelty, ephemerality, single use, mass
production.'' The agent tells the narrator that if he wants to get
hired, he needs, as a Muslim, to ``find ways to let them know up front
that you're not coming for them. \ldots{} Israel, the rest of it.'' The
narrator splutters that ``my favorite writers are all Jewish.'' The
absurdity of this, essentially a version of ``Some of my best friends
are Black,'' is like that of a punchline in a brilliant but queasy
racial farce, one written to make the audience look away, and wonder
when they'll be able to leave the theater.

Advertisement

\protect\hyperlink{after-bottom}{Continue reading the main story}

\hypertarget{site-index}{%
\subsection{Site Index}\label{site-index}}

\hypertarget{site-information-navigation}{%
\subsection{Site Information
Navigation}\label{site-information-navigation}}

\begin{itemize}
\tightlist
\item
  \href{https://help.nytimes3xbfgragh.onion/hc/en-us/articles/115014792127-Copyright-notice}{©~2020~The
  New York Times Company}
\end{itemize}

\begin{itemize}
\tightlist
\item
  \href{https://www.nytco.com/}{NYTCo}
\item
  \href{https://help.nytimes3xbfgragh.onion/hc/en-us/articles/115015385887-Contact-Us}{Contact
  Us}
\item
  \href{https://www.nytco.com/careers/}{Work with us}
\item
  \href{https://nytmediakit.com/}{Advertise}
\item
  \href{http://www.tbrandstudio.com/}{T Brand Studio}
\item
  \href{https://www.nytimes3xbfgragh.onion/privacy/cookie-policy\#how-do-i-manage-trackers}{Your
  Ad Choices}
\item
  \href{https://www.nytimes3xbfgragh.onion/privacy}{Privacy}
\item
  \href{https://help.nytimes3xbfgragh.onion/hc/en-us/articles/115014893428-Terms-of-service}{Terms
  of Service}
\item
  \href{https://help.nytimes3xbfgragh.onion/hc/en-us/articles/115014893968-Terms-of-sale}{Terms
  of Sale}
\item
  \href{https://spiderbites.nytimes3xbfgragh.onion}{Site Map}
\item
  \href{https://help.nytimes3xbfgragh.onion/hc/en-us}{Help}
\item
  \href{https://www.nytimes3xbfgragh.onion/subscription?campaignId=37WXW}{Subscriptions}
\end{itemize}
