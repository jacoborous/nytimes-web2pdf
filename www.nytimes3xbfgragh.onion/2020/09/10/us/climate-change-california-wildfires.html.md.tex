Sections

SEARCH

\protect\hyperlink{site-content}{Skip to
content}\protect\hyperlink{site-index}{Skip to site index}

\href{/section/us}{U.S.}\textbar{}A Climate Reckoning in Fire-Stricken
California

\url{https://nyti.ms/33nPCZQ}

\begin{itemize}
\item
\item
\item
\item
\item
\item
\end{itemize}

\hypertarget{wildfires-in-the-west}{%
\subsubsection{\texorpdfstring{\href{https://www.nytimes3xbfgragh.onion/spotlight/california-wildfires?name=styln-california-wildfires\&region=TOP_BANNER\&block=storyline_menu_recirc\&action=click\&pgtype=Article\&impression_id=4909cac0-f4c3-11ea-989a-419b1be2eba7\&variant=undefined}{Wildfires
in the West}}{Wildfires in the West}}\label{wildfires-in-the-west}}

\begin{itemize}
\tightlist
\item
  live\href{https://www.nytimes3xbfgragh.onion/2020/09/11/us/wildfires-live-updates.html?name=styln-california-wildfires\&region=TOP_BANNER\&block=storyline_menu_recirc\&action=click\&pgtype=Article\&impression_id=4909f1d0-f4c3-11ea-989a-419b1be2eba7\&variant=undefined}{Fires
  Updates}
\item
  \href{https://www.nytimes3xbfgragh.onion/interactive/2020/us/fires-map-tracker.html?name=styln-california-wildfires\&region=TOP_BANNER\&block=storyline_menu_recirc\&action=click\&pgtype=Article\&impression_id=4909f1d1-f4c3-11ea-989a-419b1be2eba7\&variant=undefined}{Maps
  of the Fires}
\item
  \href{https://www.nytimes3xbfgragh.onion/2020/09/10/us/climate-change-california-wildfires.html?name=styln-california-wildfires\&region=TOP_BANNER\&block=storyline_menu_recirc\&action=click\&pgtype=Article\&impression_id=4909f1d2-f4c3-11ea-989a-419b1be2eba7\&variant=undefined}{A
  Climate Reckoning}
\item
  \href{https://www.nytimes3xbfgragh.onion/article/wildfires-california-oregon-washington.html?name=styln-california-wildfires\&region=TOP_BANNER\&block=storyline_menu_recirc\&action=click\&pgtype=Article\&impression_id=4909f1d3-f4c3-11ea-989a-419b1be2eba7\&variant=undefined}{Answers
  to Your Questions}
\item
  \href{https://www.nytimes3xbfgragh.onion/article/wildfires-photos-california-oregon-washington-state.html?name=styln-california-wildfires\&region=TOP_BANNER\&block=storyline_menu_recirc\&action=click\&pgtype=Article\&impression_id=4909f1d4-f4c3-11ea-989a-419b1be2eba7\&variant=undefined}{Photos}
\item
  \href{https://www.nytimes3xbfgragh.onion/2020/09/09/us/california-wildfires.html?name=styln-california-wildfires\&region=TOP_BANNER\&block=storyline_menu_recirc\&action=click\&pgtype=Article\&impression_id=4909f1d5-f4c3-11ea-989a-419b1be2eba7\&variant=undefined}{Newsletter}
\end{itemize}

\includegraphics{https://static01.graylady3jvrrxbe.onion/images/2020/09/10/us/10FIRES-CLIMATE-oroville/merlin_176789718_5f263a18-3f3d-421d-ae11-781274a9d1bf-articleLarge.jpg?quality=75\&auto=webp\&disable=upscale}

\hypertarget{a-climate-reckoning-in-fire-stricken-california}{%
\section{A Climate Reckoning in Fire-Stricken
California}\label{a-climate-reckoning-in-fire-stricken-california}}

If climate change was a somewhat abstract notion a decade ago, today it
is all too real for Californians fleeing wildfires and smothered in a
blanket of smoke, the worst year of fires on record.

Looking over Lake Oroville after the Bear Fire, part of the North
Complex Fire in Oroville, Calif., burned through on
Wednesday.Credit...Max Whittaker for The New York Times

Supported by

\protect\hyperlink{after-sponsor}{Continue reading the main story}

\href{https://www.nytimes3xbfgragh.onion/by/thomas-fuller}{\includegraphics{https://static01.graylady3jvrrxbe.onion/images/2018/06/12/multimedia/author-thomas-fuller/author-thomas-fuller-thumbLarge.png}}\href{https://www.nytimes3xbfgragh.onion/by/christopher-flavelle}{\includegraphics{https://static01.graylady3jvrrxbe.onion/images/2019/06/28/climate/author-chris-flavelle/author-chris-flavelle-thumbLarge-v3.png}}

By \href{https://www.nytimes3xbfgragh.onion/by/thomas-fuller}{Thomas
Fuller} and
\href{https://www.nytimes3xbfgragh.onion/by/christopher-flavelle}{Christopher
Flavelle}

\begin{itemize}
\item
  Sept. 10, 2020
\item
  \begin{itemize}
  \item
  \item
  \item
  \item
  \item
  \item
  \end{itemize}
\end{itemize}

SAN FRANCISCO --- Multiple mega fires burning more than three million
acres. Millions of residents smothered in
\href{https://www.nytimes3xbfgragh.onion/2020/09/11/climate/california-smoke-wildfires.html}{toxic
air}. Rolling blackouts and triple-digit heat waves. Climate change, in
the words of one scientist, is smacking California in the face.

The crisis in the nation's most populous state is more than just an
accumulation of individual catastrophes. It is also an example of
something climate experts have long worried about, but which few
expected to see so soon: a cascade effect, in which a series of
disasters overlap, triggering or amplifying each other.

``You're toppling dominoes in ways that Americans haven't imagined,''
said Roy Wright, who directed resilience programs for the Federal
Emergency Management Agency until 2018 and grew up in Vacaville, Calif.,
near one of this year's largest fires. ``It's apocalyptic.''

The same could be said for the entire West Coast this week, to
Washington and Oregon, where towns were decimated by infernos as
firefighters were stretched to their limits.

California's simultaneous crises illustrate how the ripple effect works.
A scorching summer led to dry conditions never before experienced. That
aridity helped make the season's wildfires the biggest ever recorded.
Six of the 20 largest wildfires in modern California history have
occurred this year.

If climate change was a somewhat abstract notion a decade ago, today it
is all too real for Californians. The intensely hot wildfires are not
only chasing thousands of people from their homes but causing dangerous
chemicals to leach into drinking water. Excessive heat warnings and
suffocating smoky air have threatened the health of people already
struggling during the pandemic. And the threat of more wildfires has led
insurance companies to cancel homeowner policies and the state's main
utility to shut off power to tens of thousands of people pre-emptively.

``If you are in denial about climate change, come to California,'' Gov.
Gavin Newsom said last month.

Officials have worried about cascading disasters. They just did not
think they would start so soon.

``We used to worry about one natural hazard at a time,'' said Alice
Hill, a senior fellow at the Council on Foreign Relations who oversaw
resilience planning on the National Security Council during the Obama
administration. ``The acceleration of climate impacts has happened
faster than even we anticipated.''

\includegraphics{https://static01.graylady3jvrrxbe.onion/images/2020/09/10/us/10FIRES-CLIMATE-cooling2/merlin_176674653_7d9b30fb-cf08-460b-b0e9-49cea75e3cfe-articleLarge.jpg?quality=75\&auto=webp\&disable=upscale}

Climate scientists say the mechanism driving the wildfire crisis is
straightforward: Human behavior, chiefly the burning of fossil fuels
like coal and oil, has released greenhouse gases that increase
temperatures, desiccating forests and priming them to burn.

Mark Harvey, who was senior director for resilience at the National
Security Council until January, said the government had struggled to
prepare for situations like what was happening in California.

``The government does a very, very bad job looking at cascading
scenarios,'' Mr. Harvey said. ``Most of our systems are built to handle
one problem at a time.''

In some ways, this year's wildfires in California have been decades in
the making. A prolonged drought that ended in 2017 was a major reason
for the death of 163 million trees in California forests over the past
decade, according to the U.S. Forest Service. One of the fastest-moving
fires this year ravaged the forests that had the highest concentration
of dead trees, south of Yosemite National Park.

Further north, the Bear Fire became the 10th largest in modern
California history --- burning through an astonishing 230,000 acres in
one 24-hour period.

``It's really shocking to see the number of fast-moving, extremely large
and destructive fires simultaneously burning,'' said Daniel Swain, a
climate scientist in the Institute of the Environment and Sustainability
at the University of California, Los Angeles. ``I've spoken to maybe two
dozen fire and climate experts over the last 48 hours and pretty much
everyone is at a loss of words. There's certainly been nothing in living
memory on this scale.''

While the state mobilizes to deal with the immediate threats, the fires
will also leave California with difficult and costly longer-term
problems, everything from the effects of smoke inhalation to damaged
drinking water systems.

Wildfire smoke can in the worst cases be deadly, especially among older
people. Studies have shown that when waves of smoke hit,
\href{https://slack-redir.net/link?url=https\%3A\%2F\%2Finsights.ovid.com\%2Fepidemiology\%2Fepide\%2F2017\%2F01\%2F000\%2Fwildfire-specific-fine-particulate-matter-risk\%2F13\%2F00001648}{the
rate of hospitalizations rises}, and patients experience respiratory
problems, heart attacks and strokes.

The coronavirus pandemic adds a new layer of risk to an already perilous
situation. The Centers for Disease Control and Prevention have issued
statements warning that people with Covid-19 are at
\href{https://www.cdc.gov/disasters/covid-19/wildfire_smoke_covid-19.html}{increased
risk from wildfire smoke} during the pandemic.

``The longer we have bad air in California, the more we'll be concerned
about adverse health effects,'' said John Balmes, a spokesman for the
American Lung Association and a professor of medicine at the University
of California, San Francisco.

Image

Smoke hanging in the air turned the light orange over the Bay Area
Wednesday Credit...Jim Wilson/The New York Times

As for drinking water, scientists have known for years that runoff from
burned homes can put harmful chemicals into ground water and reservoirs.
But research in the aftermath of the 2017 wildfires in wine country
north of San Francisco and the 2018 fire that destroyed the town of
Paradise in the foothills of the Sierra discovered a different threat:
Benzene and other dangerous contaminants were found inside water
systems, possibly from heat-damaged plastics in the water
infrastructure.

``Communities need to recognize this vulnerability,'' said Andrew J.
Whelton, a professor in environmental engineering at Purdue University,
and an author of
\href{https://awwa.onlinelibrary.wiley.com/doi/full/10.1002/aws2.1183}{a
study on water contamination in Paradise}.

``Dangerous chemicals can leach from inside water systems for months
after a fire.''

The Environmental Protection Agency classifies water with benzene levels
above 500 parts per billion as hazardous. Some samples in Paradise after
the fire were found to have 2,000 parts per billion. In Sonoma County
after the wine country fires some samples had 40,000 parts per billion,
Dr. Whelton said.

Before now, many Californians assumed it would be an earthquake that
might knock out their power, damage their homes and render their
neighborhoods uninhabitable.

Susan Luten, a retired lawyer in Oakland, lives near the Hayward fault,
an area that seismologists warn is due for a major earthquake. But it is
the threat of fire that prompted her and her husband to put their go
bags by the door --- shoes, a change of clothes, flashlights, whistles,
medications, small bills and duct tape.

``We have a rope inside the house in case we have to escape down the
steep hillside on foot rather than by driving a car,'' Ms. Luten said.
Her husband studied Google Maps for escape routes.

Image

The small town of Berry Creek, Calif., was in ruins Wednesday after it
was destroyed by the Bear Fire.~Scientists have known for years that
run-off from burned homes can put harmful chemicals into ground water
and reservoirs. Credit...Max Whittaker for The New York Times

The whiplash of the multiple crises in California has played out in
their living room.

``Two days ago we were roasting inside with the windows closed in a heat
wave to avoid heavy smoke,'' Ms. Luten said.

``Today we are cool, but unable to see across the street,'' she said on
Wednesday, when the entire San Francisco Bay Area was shrouded in a
faint orange glow, the sun obscured by massive columns of smoke in the
atmosphere. ``Combine all of this with a pandemic and political menace
and it's hard not to think we are unwitting bit players in some sort of
end-of-days movie.''

Emily Szasz, a graduate art history student from Santa Cruz, said she
felt like she was in a strange, unfamiliar land.

``I feel as though I'm somewhere I've never been before,'' Ms. Szasz
said. ``There were wildfires occasionally throughout my life here, which
would be quickly fought and contained. Never do I remember 23 straight
days of orange, oppressive, smoky skies, leaving my house in fear that
I'd never return to it, or knowing someone whose home burned down in the
mountains near my house.''

Several years ago, as a student at the University of California,
Berkeley, a professor explained that California and the West were likely
to experience the effects of climate change sooner than the rest of the
country, Ms. Szasz said. The words now resonate with her.

``There is no greater proof, nor should we require it, that climate
change is here and is changing our lives,'' Ms. Szasz said of the
wildfires. ``I am only 25 years old and I do not know what future there
is for me, let alone my potential children and grandchildren.''

Even after this year's fires are put out, their ripple effects will keep
spreading, creating economic shocks --- in the insurance industry and
with the state's power grid, to name two examples --- well beyond the
physical and health damage of the disasters themselves.

This summer millions of Californians' homes went dark for an hour or
more as the smothering summer heat threatened to overload the grid.

Those blackouts are separate from the pre-emptive shut-offs carried out
by California utilities in an effort to prevent their equipment from
sparking wildfires. This week, Pacific Gas and Electric turned off power
to about 170,000 customers --- a continuation of
\href{https://www.nytimes3xbfgragh.onion/2019/10/09/us/pge-shut-off-power-outage.html}{a
program of extensive power shut-offs} that began last year.

In the insurance industry, years of heavy losses have pushed companies
to
\href{https://www.nytimes3xbfgragh.onion/2020/09/02/climate/wildfires-insurance.html}{pull
back from fire-prone areas}, in what state officials call a crisis of
its own. A lack of affordable insurance threatens to devastate housing
markets, by making homes less valuable and harder to sell.

Rex Frazier, president of the Personal Insurance Federation of
California, which represents insurers, said the industry was waiting to
see how big this year's losses were, and what the state does next.

``We have to use it as a clarion call,'' said Mr. Wright, the former
FEMA official who is now president of the Insurance Institute for
Business \& Home Safety, an industry-funded group that looks at how to
reduce damage from disasters. ``What we can't do is simply cover our
ears, hunker down and go, `I just want this to go away.'''

Philip B. Duffy, a climate scientist who is president of the Woodwell
Climate Research Center, said many people did not understand the
dynamics of a warming world.

``People are always asking, `Is this the new normal?''' he said. ``I
always say no. It's going to get worse.''

Thomas Fuller reported from San Francisco, and Christopher Flavelle from
Washington. Ivan Penn contributed reporting from Burbank, Calif., and
John Schwartz from West Orange, N.J.

Advertisement

\protect\hyperlink{after-bottom}{Continue reading the main story}

\hypertarget{site-index}{%
\subsection{Site Index}\label{site-index}}

\hypertarget{site-information-navigation}{%
\subsection{Site Information
Navigation}\label{site-information-navigation}}

\begin{itemize}
\tightlist
\item
  \href{https://help.nytimes3xbfgragh.onion/hc/en-us/articles/115014792127-Copyright-notice}{©~2020~The
  New York Times Company}
\end{itemize}

\begin{itemize}
\tightlist
\item
  \href{https://www.nytco.com/}{NYTCo}
\item
  \href{https://help.nytimes3xbfgragh.onion/hc/en-us/articles/115015385887-Contact-Us}{Contact
  Us}
\item
  \href{https://www.nytco.com/careers/}{Work with us}
\item
  \href{https://nytmediakit.com/}{Advertise}
\item
  \href{http://www.tbrandstudio.com/}{T Brand Studio}
\item
  \href{https://www.nytimes3xbfgragh.onion/privacy/cookie-policy\#how-do-i-manage-trackers}{Your
  Ad Choices}
\item
  \href{https://www.nytimes3xbfgragh.onion/privacy}{Privacy}
\item
  \href{https://help.nytimes3xbfgragh.onion/hc/en-us/articles/115014893428-Terms-of-service}{Terms
  of Service}
\item
  \href{https://help.nytimes3xbfgragh.onion/hc/en-us/articles/115014893968-Terms-of-sale}{Terms
  of Sale}
\item
  \href{https://spiderbites.nytimes3xbfgragh.onion}{Site Map}
\item
  \href{https://help.nytimes3xbfgragh.onion/hc/en-us}{Help}
\item
  \href{https://www.nytimes3xbfgragh.onion/subscription?campaignId=37WXW}{Subscriptions}
\end{itemize}
