Sections

SEARCH

\protect\hyperlink{site-content}{Skip to
content}\protect\hyperlink{site-index}{Skip to site index}

\href{https://www.nytimes3xbfgragh.onion/section/business}{Business}

\href{https://myaccount.nytimes3xbfgragh.onion/auth/login?response_type=cookie\&client_id=vi}{}

\href{https://www.nytimes3xbfgragh.onion/section/todayspaper}{Today's
Paper}

\href{/section/business}{Business}\textbar{}Chevy's Little Engine That
Could

\url{https://nyti.ms/2FrHv5P}

\begin{itemize}
\item
\item
\item
\item
\item
\end{itemize}

Advertisement

\protect\hyperlink{after-top}{Continue reading the main story}

Supported by

\protect\hyperlink{after-sponsor}{Continue reading the main story}

Wheels

\hypertarget{chevys-little-engine-that-could}{%
\section{Chevy's Little Engine That
Could}\label{chevys-little-engine-that-could}}

For six decades, a small-block Chevy has powered the dreams of
hot-rodders and Corvettes, and plenty of other cars, even a Ford
Mustang.

\includegraphics{https://static01.graylady3jvrrxbe.onion/images/2020/09/11/business/10wheels2-print/merlin_176804751_d987f0f7-554d-4a50-817d-f2b73b85e409-articleLarge.jpg?quality=75\&auto=webp\&disable=upscale}

By Roy Furchgott

\begin{itemize}
\item
  Sept. 10, 2020
\item
  \begin{itemize}
  \item
  \item
  \item
  \item
  \item
  \end{itemize}
\end{itemize}

In the great rivalry between Chevrolet and Ford, Jason Carlisle, a
former mechanic and current hobbyist racer, took sides decades ago.
``Cut me: I bleed Ford blue,'' he said. But when it came time to replace
the engine in his Mustang racecar, he did the unthinkable. He installed
a small-block Chevy engine.

Called the SBC by aficionados, it proved lighter and more reliable than
the Ford engine, and just as powerful, at a quarter of the price. ``It
hurt,'' Mr. Carlisle said. ``It's heartbreaking. I've ate a lot of crow
over this one.''

For six-plus decades, speed freaks like Mr. Carlisle have been seduced
by the strategy of an adventuresome World War II émigré and engineer,
Zora Arkus-Duntov. At midcentury, Ford performance left Chevy in the
dust. Working at Chevrolet in 1953, Arkus-Duntov wrote a memo,
\href{https://www.corvetteactioncenter.com/history/duntov_letter.html}{``Thoughts
Pertaining to Youth, Hot Rodders and Chevrolet,''} that mapped out the
path for Chevy to overtake Ford.

Today the Chevy V-8 that powers Mr. Carlisle's Mustang is the direct
descendant of the SBC from Arkus-Duntov and the iconoclastic chief
engineer Ed Cole, whose motto was ``Kick the hell out of the status
quo.'' The engine saved the Corvette from near oblivion and made
Chevrolet the source of automotive history's dominant domestic engine.

Subsequent SBC versions would sustain the legacy. The mid-1990s version
would inspire not one but two annual festivals, which continue today. It
can be found transplanted in everything from modern behemoth trucks to
vintage English sports cars. And now, 65 years after its debut, it is
still being produced, it is still the dominant domestic engine --- with
more than 109 million sold --- and its current factory version
(producing up to 495 horses) still powers the Corvette.

\includegraphics{https://static01.graylady3jvrrxbe.onion/images/2020/09/11/business/10wheels1-print/merlin_176804715_b0367e8d-97d8-4de2-b2fc-66544271b7ef-articleLarge.jpg?quality=75\&auto=webp\&disable=upscale}

Until the early 1950s, Chevy engine bays were largely the domain of
inline engines such as the Blue Flame Six. Chevrolet's parent company,
General Motors, used V-8s in high-end Cadillacs, Buicks and Oldsmobiles,
but little else.

Ford's Flathead V-8, in production since 1932, was the choice of lead
foots, hot-rodders and even criminals. The bank robber Clyde Barrow
praised Ford as a getaway car in an unsolicited (and not entirely
grammatical)\href{https://www.thehenryford.org/collections-and-research/digital-collections/artifact/281082/}{letter
to Ford}: ``For sustained speed and freedom from trouble the Ford has
got ever other car skinned.''

When Cole became Chevy's chief engineer in 1952, a V-8 was already in
development. ``It was too big, too conventional,'' said Christo Datini,
an archivist with the General Motors Heritage Center. ``He wanted
something that was more lightweight and compact.''

Image

The Blue Flame Six was a Chevrolet standby until the early 1950s, with
the V-8 mostly limited to high-end General Motors cars.Credit...General
Motors

Cole fast-tracked a new design that would become the small-block. He
sequestered his engineering team across the street from G.M.'s Hamtramck
assembly plant, presumably to work without corporate interference. The
team was obsessive. One member, Jack Golding, told MotorTrend, ``We
worked 60 hours a week at times --- all day Saturday plus long hours the
rest of the week.''

Image

Mr. Carlisle put a Ford sticker on the engine.Credit...Eve Edelheit for
The New York Times

The engineers were halfway through development of the small-block when,
on a dreary January weekend in New York, G.M. unveiled the new models at
the 1953 Motorama at the Waldorf Astoria. Among them was a two-seater
with futuristic fiberglass body panels and aluminum components. It was
called the Corvette.

That 'Vette was a concept car, but consumers were wowed. It was hurried
into production.

Among those consumers was Arkus-Duntov. He wrangled a job at G.M.,
where, drawing on his experience in aftermarket performance parts, he
wrote the ``Duntov Letter,'' a memo that defined the future for
Chevrolet.

``The hot rod movement and interest in things connected with hop-up and
speed is still growing,'' it began. The movement had spawned
large-circulation magazines that were ``full of Fords.'' Unless G.M.
took steps, Ford would dominate the market for years to come.

It's unclear whether Arkus-Duntov knew of Cole's project, but he called
for an engine suitable for supercharging and with an aftermarket line of
performance-enhancing parts.

``The interesting thing that can be gleaned from the letter Zora
wrote,'' said Derek E. Moore, director of collections at the National
Corvette Museum, ``is that although he is an incredible engineering
mind, he is looking at marketing to the youth that is going to be the
next generation of buyers.''

Arkus-Duntov conceded that image-conscious drivers would not accept
dowdy Chevrolets as a framework for hot-rodding. With an exception:
``Possibly the existence of the Corvette provides the loop hole,'' he
wrote.

The problem was that the Corvette was already failing, and failing
badly.

Rushing to market in 1953, Chevrolet produced only 300 Corvettes. To
build prestige, dealers were told to restrict sales to V.I.P.s, such as
mayors, business leaders and favorite customers.

For 1954, G.M. geared up to produce 10,000 Corvettes. But faced with low
demand, it built only 3,640. The year ended with about 1,100 unsold.
Part of the problem was the Corvette's humdrum 150-horsepower Blue Flame
Six. Then there was the expense; the Corvette listed for \$2,774 (nearly
\$27,000 in 2020 dollars), reduced from the \$3,498 (roughly \$34,000)
list price for the 1953 model. But thrill seekers could buy a Jaguar, a
Triumph, an MG or an Austin-Healey for about that price or less.

G.M. discussed ending production.

Then Chevy got wind of a Ford project --- a two-seater. The Thunderbird.
To cancel the Corvette when faced with the Thunderbird would look like
surrender. The Corvette got a fortuitous reprieve.

Fortuitous because Cole, who had joined forces with Arkus-Duntov, had
their small-block ready for production.

Image

A Corvette, Chevrolet's small-block rival to the Ford Thunderbird, at
Indianapolis Motor Speedway in 1955.Credit...General Motors

The hallmark of the small-block is, of course, the block. From 1955 on,
the spacing between the cylinder bores (the holes where the pistons go)
is 4.4 inches from center to center, compared with 4.8 for the Chevy
big-block. It was elegantly simple, light, reliable and powerful.

The small-block was available in full-size cars, such as the Bel Air,
the Task Force light trucks and the Corvette. Horsepower ranged from 140
in the trucks to 180 in the full-size cars. But the Corvette version
made 195 horsepower. The 0-to-60 time dropped from 11 seconds to 8.7.

Chevrolet officially named the small-block the Turbo-Fire. Hot-rodders
called it ``Mighty Mouse.''

In 1955, Chevrolet fretfully built only 700 Corvettes. ``They pretty
much sold out,'' Mr. Moore said. More than that, it put Chevy head to
head in the two-seater market against Ford.

Ford marketed the Thunderbird as a ``personal luxury car,'' stressing
comfort over performance. The T-Bird sold well, but Ford thought ---
correctly --- that it would sell more if it carried more passengers. By
1958, the redesigned Thunderbird seated four. Ford had ceded the
American sports car market to the Corvette. With it, Chevrolet cemented
its reputation as a performance brand.

Image

A 1955 Thunderbird, which Ford would redesign to seat four people before
the decade was out.Credit...Ford Motor, via Associated Press

The competition had just begun. True to Arkus-Duntov's vision, Chevy
introduced performance parts. In 1956, a \$188.30 high-performance cam
option boosted Corvette power to 240 horses. That year, at Daytona
Speedweek, Corvettes set two records, one with Arkus-Duntov behind the
wheel. Small-block-powered Chevys began to score race wins, notably
taking seven of the top 10 finishes at the Southern 500 super speedway
stock car race on Labor Day in 1955.

The small-block became a favorite of hot-rodders eking out horsepower.
And it was factory-standard in Chevy cars and trucks, powering classic
muscle cars such as the Camaro, Chevelle and Nova.

The engine's reputation attracted people who didn't have G.M. cars but
wanted small-block power. It became the standard for engine swaps. The
LS version, made from 1995 to 2014, is especially popular owing to its
lighter aluminum head and high-performance components. It inspired two
annual LS festivals, where people show off their builds.

``You can pull a 5.3-liter LS from a junkyard truck, add a cheap turbo
and make over 1,000 horsepower,'' said Jamie Meyer, who headed
performance parts marketing for much of his 15-year career at G.M. ``And
the engine probably costs \$400 or \$500.''

In Tampa, Fla., Patrick Laughlin sells kits to fit small-blocks into
other cars. ``I've seen them in MGs, Hondas, anything,'' said Mr.
Laughlin, whose auto shop, Conquer Custom, is just one of many such
operations. He sells 60 to 80 kits a year.

Anyone with basic mechanical, plumbing and wiring skills can handle a
conversion, Mr. Laughlin said, provided that person has an engine hoist
``or a big tree with a strap and a couple of friends.'' He added, ``It's
more of a confidence issue.''

Mr. Carlisle, the racer, said the Ford engine that had been in his
Mustang required new bearings and piston rings after every race. With
the small-block, he has run three 14-hour endurance races without a
repair.

He said he would tolerate the shame of running a small-block Chevy as
long as it was in a Ford body: ``I can close the hood.''

Advertisement

\protect\hyperlink{after-bottom}{Continue reading the main story}

\hypertarget{site-index}{%
\subsection{Site Index}\label{site-index}}

\hypertarget{site-information-navigation}{%
\subsection{Site Information
Navigation}\label{site-information-navigation}}

\begin{itemize}
\tightlist
\item
  \href{https://help.nytimes3xbfgragh.onion/hc/en-us/articles/115014792127-Copyright-notice}{©~2020~The
  New York Times Company}
\end{itemize}

\begin{itemize}
\tightlist
\item
  \href{https://www.nytco.com/}{NYTCo}
\item
  \href{https://help.nytimes3xbfgragh.onion/hc/en-us/articles/115015385887-Contact-Us}{Contact
  Us}
\item
  \href{https://www.nytco.com/careers/}{Work with us}
\item
  \href{https://nytmediakit.com/}{Advertise}
\item
  \href{http://www.tbrandstudio.com/}{T Brand Studio}
\item
  \href{https://www.nytimes3xbfgragh.onion/privacy/cookie-policy\#how-do-i-manage-trackers}{Your
  Ad Choices}
\item
  \href{https://www.nytimes3xbfgragh.onion/privacy}{Privacy}
\item
  \href{https://help.nytimes3xbfgragh.onion/hc/en-us/articles/115014893428-Terms-of-service}{Terms
  of Service}
\item
  \href{https://help.nytimes3xbfgragh.onion/hc/en-us/articles/115014893968-Terms-of-sale}{Terms
  of Sale}
\item
  \href{https://spiderbites.nytimes3xbfgragh.onion}{Site Map}
\item
  \href{https://help.nytimes3xbfgragh.onion/hc/en-us}{Help}
\item
  \href{https://www.nytimes3xbfgragh.onion/subscription?campaignId=37WXW}{Subscriptions}
\end{itemize}
