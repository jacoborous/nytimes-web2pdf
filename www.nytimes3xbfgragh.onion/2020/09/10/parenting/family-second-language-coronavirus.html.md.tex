Sections

SEARCH

\protect\hyperlink{site-content}{Skip to
content}\protect\hyperlink{site-index}{Skip to site index}

\href{https://www.nytimes3xbfgragh.onion/section/parenting}{Parenting}

\href{https://myaccount.nytimes3xbfgragh.onion/auth/login?response_type=cookie\&client_id=vi}{}

\href{https://www.nytimes3xbfgragh.onion/section/todayspaper}{Today's
Paper}

\href{/section/parenting}{Parenting}\textbar{}In Quarantine, Kids Pick
Up Parents' Mother Tongues

\url{https://nyti.ms/3iinpJT}

\begin{itemize}
\item
\item
\item
\item
\item
\item
\end{itemize}

\hypertarget{school-reopenings}{%
\subsubsection{\texorpdfstring{\href{https://www.nytimes3xbfgragh.onion/spotlight/schools-reopening?name=styln-coronavirus-schools-reopening\&region=TOP_BANNER\&block=storyline_menu_recirc\&action=click\&pgtype=Article\&impression_id=03e480f0-f4bb-11ea-be14-1f129ac58728\&variant=undefined}{School
Reopenings}}{School Reopenings}}\label{school-reopenings}}

\begin{itemize}
\tightlist
\item
  \href{https://www.nytimes3xbfgragh.onion/2020/09/08/us/school-districts-cyberattacks-glitches.html?name=styln-coronavirus-schools-reopening\&region=TOP_BANNER\&block=storyline_menu_recirc\&action=click\&pgtype=Article\&impression_id=03e480f1-f4bb-11ea-be14-1f129ac58728\&variant=undefined}{Remote
  Learning Glitches}
\item
  \href{https://www.nytimes3xbfgragh.onion/2020/09/08/upshot/children-testing-shortfalls-virus.html?name=styln-coronavirus-schools-reopening\&region=TOP_BANNER\&block=storyline_menu_recirc\&action=click\&pgtype=Article\&impression_id=03e480f2-f4bb-11ea-be14-1f129ac58728\&variant=undefined}{Limited
  Testing for Children}
\item
  \href{https://www.nytimes3xbfgragh.onion/2020/09/10/us/des-moines-school-opening-coronavirus.html?name=styln-coronavirus-schools-reopening\&region=TOP_BANNER\&block=storyline_menu_recirc\&action=click\&pgtype=Article\&impression_id=03e480f3-f4bb-11ea-be14-1f129ac58728\&variant=undefined}{District
  Defies Reopening Order}
\item
  \href{https://www.nytimes3xbfgragh.onion/interactive/2020/us/covid-college-cases-tracker.html?name=styln-coronavirus-schools-reopening\&region=TOP_BANNER\&block=storyline_menu_recirc\&action=click\&pgtype=Article\&impression_id=03e480f4-f4bb-11ea-be14-1f129ac58728\&variant=undefined}{Tracking
  College Cases}
\end{itemize}

Advertisement

\protect\hyperlink{after-top}{Continue reading the main story}

Supported by

\protect\hyperlink{after-sponsor}{Continue reading the main story}

\hypertarget{in-quarantine-kids-pick-up-parents-mother-tongues}{%
\section{In Quarantine, Kids Pick Up Parents' Mother
Tongues}\label{in-quarantine-kids-pick-up-parents-mother-tongues}}

For some families, the pandemic has meant a return to their native
languages.

\includegraphics{https://static01.graylady3jvrrxbe.onion/images/2020/09/04/multimedia/8parenting-mother-tongues/8parenting-mother-tongues-articleLarge.jpg?quality=75\&auto=webp\&disable=upscale}

By Sophie Hardach

\begin{itemize}
\item
  Sept. 10, 2020
\item
  \begin{itemize}
  \item
  \item
  \item
  \item
  \item
  \item
  \end{itemize}
\end{itemize}

A few days into the lockdown here in London, I noticed a surprising
side-effect of the pandemic: My 3-year-old son was speaking more German.

German is my mother tongue, and I have used it with him since he was
born, but because everyone around us speaks English, including my
British husband, we settled into a pattern typical of mixed families. I
spoke to my son in German, and he replied in English. Then Covid-19
reshuffled our linguistic deck. As all of us quarantined at home, my son
embraced German with unprecedented enthusiasm. Now, almost six months
on, it has become his preferred language. In a complete reversal, he
even replies to my husband in German.

My experience is far from unique. All over the world, Covid-19 has
forced children to stay inside. In some homes where different languages
coexist, this is changing how they speak. With schools and day cares
closed, previously dominant languages --- such as English in Britain and
the United States --- are no longer as overpowering. Instead, children
are hearing more of their parents' mother tongues.

``They're put into this little hothouse of less English, more other
languages,'' said Ludovica Serratrice, Ph.D., a professor specializing
in multilingualism at the University of Reading in Britain.

Together with researchers at the University of Oxford, the University of
Cambridge, University College London and other institutions, Dr.
Serratrice surveyed the language habits of over 700 multilingual
families in Britain and Ireland from April to the beginning of July,
when the countries were mostly shut down. And researchers in Norway
carried out an adapted version of the same survey, collecting responses
from almost 200 families.

The parents in these families spoke more than 40 different mother
tongues, including French, Polish, Spanish, Hindi, Punjabi, Urdu,
Kirundi and Zulu. Before the lockdown, the children tended to use the
dominant languages: English in Britain and Ireland, and Norwegian in
Norway (plus English, thanks to television, computer games and other
media).

Now, preliminary data suggests children were using the parents'
languages more during the lockdown, especially among younger kids.

``When you think about living in a different country and raising your
child in your native language, some people think, `Oh, it's the most
natural thing and it's easy,' because it's your native language. And
that couldn't be further from the truth,'' said Elisabet García
González, a doctoral research fellow at the University of Oslo who led
the Norwegian survey.

Instead, ``the language of the home becomes less and less important,''
she said, as children start school and make friends in the country's
dominant language. Unless parents take extra measures, the
\href{https://www.cambridge.org/core/journals/applied-psycholinguistics/article/parental-language-input-patterns-and-childrens-bilingual-use/821A5852222197491F4E7ABC8AA4B099}{ancestral
sound may fade}. School plays a crucial role with this;
i\href{https://www.jstor.org/stable/3588366?Search=yes\&resultItemClick=true\&searchText=birth\&searchText=order\&searchText=bilingual\&searchUri=\%2Faction\%2FdoBasicSearch\%3FQuery\%3Dbirth\%2Border\%2Bbilingual\&ab_segments=0\%2Fbasic_SYC-5187\%2Ftest\&refreqid=search\%3A5aa7f5c52530dc08951a6d8565c612a1\&seq=1}{n
a study} of 200 Korean-American families, the portion of firstborn
children who spoke Korean to their parents went from almost 80 percent
to 34 percent after starting school. Younger siblings spoke even less.

For parents, that sudden rejection of the mother tongue can be
bewildering and even painful. My son's first words were in German. He
preferred it as long as I was on maternity leave and we were both at
home. When he started day care, he switched to English, even in our own
conversations, and it was as if someone had snatched away our common
language.

The pandemic appears to have stopped that slide, at least for some.

Dr. Elizabeth Lanza, a professor of linguistics at the University of
Oslo, who supervised the Norwegian survey, observed the shift in her own
family. Lanza is American, but has lived in Norway for decades. Her
daughter also lives in Norway and speaks English to her young toddler,
while her partner speaks Norwegian. Before Covid-19, one of the boy's
favorite words was the Norwegian \emph{``mer!'',} echoing the language
he heard at his day care*.* About a week into the lockdown, he switched
to the English equivalent: \emph{``more!''}

Dr. Lanza cautioned that not all respondents saw their native tongue
strengthen. Some even said it was suffering because they were
home-schooling the children in Norwegian.

But where the languages did blossom, it made the parents happy. In the
midst of an incredibly
\href{https://www.ox.ac.uk/news/2020-05-06-major-stressors-parents-during-covid-19-revealed-new-report-0}{stressful
time}, the fact that the children were speaking these second languages
brought parents joy. This echoes
\href{https://journals.sagepub.com/doi/full/10.1177/1367006920920939}{research}
suggesting that passing on one's language can create better
communication between generations and a shared identity and heritage.

In the United States, researchers interested in language have launched
\href{https://kidtalk.app/home}{an app called KidTalk} to gather
recordings made before and during Covid-19. Yi Ting Huang, Ph.D., an
associate professor in the Department of Hearing and Speech Sciences at
the University of Maryland, and Joshua Hartshorne, Ph.D., an assistant
professor of psychology at Boston College, have recruited more than 300
families, about a third of which are multilingual, for the project.

Dr. Hartshorne said it's a good opportunity to study how children learn
languages, including multiple languages. ``I don't know that we've had
recent historical precedent for a child's world to be shrunk down to
just the immediate family for months at a time,'' he wrote in an email.

Dr. Huang plans to use speech-recognition software to analyze the
recordings; for example, identifying the number of speakers and
languages in each conversation, and tracking any changes. This could
help us understand how multilingual children deploy their different
languages as learning tools, such as using their knowledge of one to
acquire the next.

In the past, Dr. Huang said, many researchers and policymakers viewed a
child as a water glass that could only hold so much liquid. ``So we're
just trying to cram as much of one language as opposed to another
language,'' she said. This led to some parents and teachers using only
English with children, and suppressing second languages. Now most
experts say children use languages flexibly, changing them for the
situation.

Dr. Huang experienced this herself. When she was 5 years old, she moved
to the United States from Taiwan, where she'd spoken Mandarin to her
mother. But in the U.S., she began replying in English. It was only as
an adult that she realized how much Mandarin meant to her. Now she uses
it again with her mother.

\href{https://www.nytimes3xbfgragh.onion/spotlight/schools-reopening?action=click\&pgtype=Article\&state=default\&region=MAIN_CONTENT_3\&context=storylines_keepup}{}

\hypertarget{school-reopenings-}{%
\subsubsection{School Reopenings ›}\label{school-reopenings-}}

\hypertarget{back-to-school}{%
\paragraph{Back to School}\label{back-to-school}}

Updated Sept. 11, 2020

The latest on how schools are reopening amid the pandemic.

\begin{itemize}
\item
  \begin{itemize}
  \tightlist
  \item
    School officials in Des Moines are refusing to hold in-person
    classes,
    \href{https://www.nytimes3xbfgragh.onion/2020/09/10/us/des-moines-school-opening-coronavirus.html?action=click\&pgtype=Article\&state=default\&region=MAIN_CONTENT_3\&context=storylines_keepup}{despite
    an order from Iowa's governor and a judge's ruling}, risking school
    funding and their jobs because they think it's unsafe.
  \item
    The University of Illinois at Urbana-Champaign had one of the most
    comprehensive plans by a major college to keep the virus under
    control. But it
    \href{https://www.nytimes3xbfgragh.onion/2020/09/10/health/university-illinois-covid.html?action=click\&pgtype=Article\&state=default\&region=MAIN_CONTENT_3\&context=storylines_keepup}{failed
    to account for students partying}.
  \item
    College students are
    \href{https://www.nytimes3xbfgragh.onion/2020/09/10/technology/coronavirus-quarantines-college.html?action=click\&pgtype=Article\&state=default\&region=MAIN_CONTENT_3\&context=storylines_keepup}{using
    apps to shame their schools}~into better coronavirus plans.
  \item
    For some families, the pandemic
    \href{https://www.nytimes3xbfgragh.onion/2020/09/10/parenting/family-second-language-coronavirus.html?action=click\&pgtype=Article\&state=default\&region=MAIN_CONTENT_3\&context=storylines_keepup}{has
    meant a return to their native languages}.
  \end{itemize}
\end{itemize}

``I find a lot of comfort in Mandarin,'' she said. ``It reminds me of my
mom and home.'' In an email, she described language as a ``living marker
of history and cultural identity,'' linking immigrant families to their
place of origin.

Dr. Huang said that my son's radical switch to German might be similar.
Amid the upheaval of Covid-19, he might be turning to a reassuring
language that he associates with me: ``We're all reaching for some
things that feel familiar.''

Earlier this year, Dr. Huang's mother joined her to help with child
care, and within only a week or so, Huang noticed her 6-year-old
daughter engaging more with Mandarin.

Dr. Hartshorne and his partner are also raising their daughter in
English and Mandarin, but they have seen the opposite effect.
``Initially, we were full of energy and actually speaking more Mandarin
to our daughter, and her Mandarin actually started to improve relative
to her English,'' he wrote. ``As the weeks have worn on, we've worn
down.'' English has become the family's default language, though the
daughter still understands Mandarin.

For some parents, the school closures are an opportunity to challenge
bigger linguistic hierarchies. Medadi Ssentanda, Ph.D., is a lecturer in
African languages at Makerere University in Uganda, and a specialist in
mother tongue education. More than 42 Indigenous languages are spoken in
Uganda, but formal education is delivered in English, a legacy of
colonialism. Dr. Ssentanda has observed that the local language ---
Luganda --- in his own neighborhood has gained strength during lockdown.

Children are spending more time with their families, and are also seeing
the language used in ways they hadn't considered before. When his
12-year-old daughter asked him to judge a debate with other children in
English, Dr. Ssentanda agreed --- but only if they held it in Luganda.

``She was surprised!'' he wrote in an email. ``There is a belief that
academic issues must necessarily be discussed in the English language.''

Will all these languages continue to blossom after Covid-19? It's hard
to say. Some researchers expected children to revert to the dominant
language once life returns to normal. Others saw the possibility of a
virtuous cycle, with children growing more confident in their second
language and using it more in the long run.

In the meantime, I've decided to simply enjoy my son's new love of
German. Perhaps it will continue as our shared language this time. But
even if it doesn't, I'll always remember that when the whole world was
in turmoil, my mother tongue was there to provide warmth, laughter and a
feeling of safety.

\begin{center}\rule{0.5\linewidth}{\linethickness}\end{center}

Sophie Hardach is a journalist and author living in London. She is
working on a book about the joy of languages and linguistic diversity.

Advertisement

\protect\hyperlink{after-bottom}{Continue reading the main story}

\hypertarget{site-index}{%
\subsection{Site Index}\label{site-index}}

\hypertarget{site-information-navigation}{%
\subsection{Site Information
Navigation}\label{site-information-navigation}}

\begin{itemize}
\tightlist
\item
  \href{https://help.nytimes3xbfgragh.onion/hc/en-us/articles/115014792127-Copyright-notice}{©~2020~The
  New York Times Company}
\end{itemize}

\begin{itemize}
\tightlist
\item
  \href{https://www.nytco.com/}{NYTCo}
\item
  \href{https://help.nytimes3xbfgragh.onion/hc/en-us/articles/115015385887-Contact-Us}{Contact
  Us}
\item
  \href{https://www.nytco.com/careers/}{Work with us}
\item
  \href{https://nytmediakit.com/}{Advertise}
\item
  \href{http://www.tbrandstudio.com/}{T Brand Studio}
\item
  \href{https://www.nytimes3xbfgragh.onion/privacy/cookie-policy\#how-do-i-manage-trackers}{Your
  Ad Choices}
\item
  \href{https://www.nytimes3xbfgragh.onion/privacy}{Privacy}
\item
  \href{https://help.nytimes3xbfgragh.onion/hc/en-us/articles/115014893428-Terms-of-service}{Terms
  of Service}
\item
  \href{https://help.nytimes3xbfgragh.onion/hc/en-us/articles/115014893968-Terms-of-sale}{Terms
  of Sale}
\item
  \href{https://spiderbites.nytimes3xbfgragh.onion}{Site Map}
\item
  \href{https://help.nytimes3xbfgragh.onion/hc/en-us}{Help}
\item
  \href{https://www.nytimes3xbfgragh.onion/subscription?campaignId=37WXW}{Subscriptions}
\end{itemize}
