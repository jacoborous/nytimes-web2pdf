Sections

SEARCH

\protect\hyperlink{site-content}{Skip to
content}\protect\hyperlink{site-index}{Skip to site index}

\href{https://www.nytimes3xbfgragh.onion/section/health}{Health}

\href{https://myaccount.nytimes3xbfgragh.onion/auth/login?response_type=cookie\&client_id=vi}{}

\href{https://www.nytimes3xbfgragh.onion/section/todayspaper}{Today's
Paper}

\href{/section/health}{Health}\textbar{}A University Had a Great
Coronavirus Plan, but Students Partied On

\url{https://nyti.ms/2FqqLfg}

\begin{itemize}
\item
\item
\item
\item
\item
\end{itemize}

\hypertarget{school-reopenings}{%
\subsubsection{\texorpdfstring{\href{https://www.nytimes3xbfgragh.onion/spotlight/schools-reopening?name=styln-coronavirus-schools-reopening\&region=TOP_BANNER\&block=storyline_menu_recirc\&action=click\&pgtype=Article\&impression_id=82c9b0b0-f4b7-11ea-b564-bd1dfc5ec362\&variant=undefined}{School
Reopenings}}{School Reopenings}}\label{school-reopenings}}

\begin{itemize}
\tightlist
\item
  \href{https://www.nytimes3xbfgragh.onion/2020/09/08/us/school-districts-cyberattacks-glitches.html?name=styln-coronavirus-schools-reopening\&region=TOP_BANNER\&block=storyline_menu_recirc\&action=click\&pgtype=Article\&impression_id=82c9b0b1-f4b7-11ea-b564-bd1dfc5ec362\&variant=undefined}{Remote
  Learning Glitches}
\item
  \href{https://www.nytimes3xbfgragh.onion/2020/09/08/upshot/children-testing-shortfalls-virus.html?name=styln-coronavirus-schools-reopening\&region=TOP_BANNER\&block=storyline_menu_recirc\&action=click\&pgtype=Article\&impression_id=82c9b0b2-f4b7-11ea-b564-bd1dfc5ec362\&variant=undefined}{Limited
  Testing for Children}
\item
  \href{https://www.nytimes3xbfgragh.onion/2020/09/10/us/des-moines-school-opening-coronavirus.html?name=styln-coronavirus-schools-reopening\&region=TOP_BANNER\&block=storyline_menu_recirc\&action=click\&pgtype=Article\&impression_id=82c9b0b3-f4b7-11ea-b564-bd1dfc5ec362\&variant=undefined}{District
  Defies Reopening Order}
\item
  \href{https://www.nytimes3xbfgragh.onion/interactive/2020/us/covid-college-cases-tracker.html?name=styln-coronavirus-schools-reopening\&region=TOP_BANNER\&block=storyline_menu_recirc\&action=click\&pgtype=Article\&impression_id=82c9b0b4-f4b7-11ea-b564-bd1dfc5ec362\&variant=undefined}{Tracking
  College Cases}
\end{itemize}

Advertisement

\protect\hyperlink{after-top}{Continue reading the main story}

Supported by

\protect\hyperlink{after-sponsor}{Continue reading the main story}

\hypertarget{a-university-had-a-great-coronavirus-plan-but-students-partied-on}{%
\section{A University Had a Great Coronavirus Plan, but Students Partied
On}\label{a-university-had-a-great-coronavirus-plan-but-students-partied-on}}

An unexpected upswing in positive tests at the University of Illinois at
Urbana-Champaign showed how even the most comprehensive approaches to
limiting the virus's spread can break down.

\includegraphics{https://static01.graylady3jvrrxbe.onion/images/2020/09/10/science/10VIRUS-ILLINOIS1/10VIRUS-ILLINOIS1-articleLarge.jpg?quality=75\&auto=webp\&disable=upscale}

\href{https://www.nytimes3xbfgragh.onion/by/kenneth-chang}{\includegraphics{https://static01.graylady3jvrrxbe.onion/images/2018/02/16/multimedia/author-kenneth-chang/author-kenneth-chang-thumbLarge.jpg}}

By \href{https://www.nytimes3xbfgragh.onion/by/kenneth-chang}{Kenneth
Chang}

\begin{itemize}
\item
  Sept. 10, 2020
\item
  \begin{itemize}
  \item
  \item
  \item
  \item
  \item
  \end{itemize}
\end{itemize}

At the University of Illinois at Urbana-Champaign, more than 40,000
students take tests twice a week for the coronavirus. They cannot enter
campus buildings unless an app vouches that their test has come back
negative. Everyone has to wear masks.

This is one of the most comprehensive plans by a major college to keep
the virus under control. University scientists developed a quick,
inexpensive saliva test. Other researchers put together a detailed
computer model that suggested these measures would work, and that
in-person instruction could go forward this fall.

But the predictive model included an oversight: It assumed that all of
the students would do all of the things that they were told to.

Enough students continued to go to parties even after testing positive,
showing how even the best thought-out plans to keep college education
moving can fail when humans do not heed common sense or the commands
from public health officials.

Last week, the university reported an unexpected upswing of coronavirus
cases and imposed a lockdown. Students had to stay in their dorms or
off-campus housing except for essential activities, which included going
to class.

\includegraphics{https://static01.graylady3jvrrxbe.onion/images/2020/09/10/science/10virus-illinois-xkcd/10virus-illinois-xkcd-articleLarge.png?quality=75\&auto=webp\&disable=upscale}

Randall Munroe, the creator of \href{https://xkcd.com/2355/}{the popular
internet comic strip XKCD}, and
\href{https://www.nytimes3xbfgragh.onion/column/good-question-randall-munroe-xkcd}{a
contributor to The Times}, poked fun at the fact that two of the people
who were key players in developing the models were physicists.

``Can't understand why someone with a physics degree would be bad at
judging how often college students get invited to parties,'' comments
one of the XKCD characters.

Nigel Goldenfeld, one of the physicists who was the butt of the comic
strip, replied in good humor. ``We enjoyed the joke,'' he said. But, he
noted, it was not a completely accurate portrayal of what happened.

For one, while he and a fellow physicist, Sergei Maslov, had devoted
effort to an epidemiological model for all of Illinois, the more
detailed university simulation, modeling the movements of some 46,000
students, professors and others like servers in coffee shops and bars
who interact with students, was the effort of a larger group and led by
Dr. Goldenfeld and Ahmed E. Elbanna, a professor of civil and
environmental engineering. (It's harder to make jokes about civil
engineers.)

Second, they had indeed taken into account college partying and quite a
bit of it --- more than 7,000 students partying three times a week in
their model.

What the scientists had not taken into account was that some students
would continue partying after they received a positive test result. ``It
was willful noncompliance by a small group of people,'' Dr. Goldenfeld
said.

Those were the key ingredients for a few people infecting many others.
``If you know you are positive,'' Dr. Elbanna said, ``and you go to a
party, that's not just a bad act. That's very, very dangerous.''

\href{https://www.nytimes3xbfgragh.onion/interactive/2020/us/covid-college-cases-tracker.html}{}

\includegraphics{https://static01.graylady3jvrrxbe.onion/images/2020/09/11/us/covid-college-cases-tracker-promo-1599840058820/covid-college-cases-tracker-promo-1599840058820-articleLarge.png}

\hypertarget{tracking-covid-at-us-colleges-and-universities}{%
\subsection{Tracking Covid at U.S. Colleges and
Universities}\label{tracking-covid-at-us-colleges-and-universities}}

Large outbreaks expanded on campuses as new semesters were underway.

Some of the students who tested positive
\href{https://dailyillini.com/covid-10/2020/09/02/live-university-holds-news-conference-after-on-campus-covid-19-spike/}{even
tried to circumvent the app} so that they could enter buildings instead
of staying isolated in their rooms,
\href{https://massmail.illinois.edu/massmail/61640956.html}{university
administrators said in a letter to students}.

\href{https://www.nytimes3xbfgragh.onion/spotlight/schools-reopening?action=click\&pgtype=Article\&state=default\&region=MAIN_CONTENT_3\&context=storylines_keepup}{}

\hypertarget{school-reopenings-}{%
\subsubsection{School Reopenings ›}\label{school-reopenings-}}

\hypertarget{back-to-school}{%
\paragraph{Back to School}\label{back-to-school}}

Updated Sept. 11, 2020

The latest on how schools are reopening amid the pandemic.

\begin{itemize}
\item
  \begin{itemize}
  \tightlist
  \item
    School officials in Des Moines are refusing to hold in-person
    classes,
    \href{https://www.nytimes3xbfgragh.onion/2020/09/10/us/des-moines-school-opening-coronavirus.html?action=click\&pgtype=Article\&state=default\&region=MAIN_CONTENT_3\&context=storylines_keepup}{despite
    an order from Iowa's governor and a judge's ruling}, risking school
    funding and their jobs because they think it's unsafe.
  \item
    The University of Illinois at Urbana-Champaign had one of the most
    comprehensive plans by a major college to keep the virus under
    control. But it
    \href{https://www.nytimes3xbfgragh.onion/2020/09/10/health/university-illinois-covid.html?action=click\&pgtype=Article\&state=default\&region=MAIN_CONTENT_3\&context=storylines_keepup}{failed
    to account for students partying}.
  \item
    College students are
    \href{https://www.nytimes3xbfgragh.onion/2020/09/10/technology/coronavirus-quarantines-college.html?action=click\&pgtype=Article\&state=default\&region=MAIN_CONTENT_3\&context=storylines_keepup}{using
    apps to shame their schools}~into better coronavirus plans.
  \item
    For some families, the pandemic
    \href{https://www.nytimes3xbfgragh.onion/2020/09/10/parenting/family-second-language-coronavirus.html?action=click\&pgtype=Article\&state=default\&region=MAIN_CONTENT_3\&context=storylines_keepup}{has
    meant a return to their native languages}.
  \end{itemize}
\end{itemize}

Comprehensive testing of everyone on campus and prompt contact tracing
showed the trouble spots ---
\href{https://dailyillini.com/news/2020/09/09/greek-private-housing-becomes-center-of-campus-spread/}{some
fraternities and sororities, as well as some off-campus housing, that
were throwing parties} --- as well as where the containment plans were
working. There were few signs of the virus spreading in classrooms or
from students to the people in the surrounding towns of Champaign and
Urbana.

Dr. Goldenfeld said the main purpose of the model was not to make
precise predictions, but to help administrators make informed choices on
what precautions made sense.

For example, the model showed that once-a-week screening, as university
administrators originally planned, was too little, too slow. Students
who were infected soon after a test cleared them would be infectious for
days before the next test. The university increased the mandate to two
tests a week, although
\href{https://dailyillini.com/news/2020/09/10/university-changes-covid-19-testing-schedule/}{now
that schedule is only for undergraduate students}.

Since the university clamped down last week, the number of new cases has
dropped again, and the hope is that all students will now take the
protocols more seriously.

Carl T. Bergstrom, a professor of biology at the University of
Washington in Seattle and an infectious disease expert, said most of the
other large state universities were ``opening and hoping for the best
without doing any kind of serious testing or they've switched to largely
online education.''

What the University of Illinois has tried to do is ``pretty unusual,''
Dr. Bergstrom said. ``Being able to have an in-person semester at a
school of that size with that kind of social atmosphere is really a
remarkable accomplishment and I think they have a reasonable chance of
pulling it off,'' he said. ``If it if it doesn't work, they gave it a
damn good try.''

The positivity rate for the tests is currently about 1 percent. ``It's
higher than we would like, but it's coming down,'' said Rebecca Lee
Smith, a professor of epidemiology at the University of Illinois. ``I am
hopeful. Everything is going in the right direction.''

Dr. Goldenfeld said the notoriety of recent days even had some upside:
The XKCD zinging gave him ``street cred'' with his daughters, he said.

Advertisement

\protect\hyperlink{after-bottom}{Continue reading the main story}

\hypertarget{site-index}{%
\subsection{Site Index}\label{site-index}}

\hypertarget{site-information-navigation}{%
\subsection{Site Information
Navigation}\label{site-information-navigation}}

\begin{itemize}
\tightlist
\item
  \href{https://help.nytimes3xbfgragh.onion/hc/en-us/articles/115014792127-Copyright-notice}{©~2020~The
  New York Times Company}
\end{itemize}

\begin{itemize}
\tightlist
\item
  \href{https://www.nytco.com/}{NYTCo}
\item
  \href{https://help.nytimes3xbfgragh.onion/hc/en-us/articles/115015385887-Contact-Us}{Contact
  Us}
\item
  \href{https://www.nytco.com/careers/}{Work with us}
\item
  \href{https://nytmediakit.com/}{Advertise}
\item
  \href{http://www.tbrandstudio.com/}{T Brand Studio}
\item
  \href{https://www.nytimes3xbfgragh.onion/privacy/cookie-policy\#how-do-i-manage-trackers}{Your
  Ad Choices}
\item
  \href{https://www.nytimes3xbfgragh.onion/privacy}{Privacy}
\item
  \href{https://help.nytimes3xbfgragh.onion/hc/en-us/articles/115014893428-Terms-of-service}{Terms
  of Service}
\item
  \href{https://help.nytimes3xbfgragh.onion/hc/en-us/articles/115014893968-Terms-of-sale}{Terms
  of Sale}
\item
  \href{https://spiderbites.nytimes3xbfgragh.onion}{Site Map}
\item
  \href{https://help.nytimes3xbfgragh.onion/hc/en-us}{Help}
\item
  \href{https://www.nytimes3xbfgragh.onion/subscription?campaignId=37WXW}{Subscriptions}
\end{itemize}
