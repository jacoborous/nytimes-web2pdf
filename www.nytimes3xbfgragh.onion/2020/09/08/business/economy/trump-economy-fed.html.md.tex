Sections

SEARCH

\protect\hyperlink{site-content}{Skip to
content}\protect\hyperlink{site-index}{Skip to site index}

\href{https://www.nytimes3xbfgragh.onion/section/business/economy}{Economy}

\href{https://myaccount.nytimes3xbfgragh.onion/auth/login?response_type=cookie\&client_id=vi}{}

\href{https://www.nytimes3xbfgragh.onion/section/todayspaper}{Today's
Paper}

\href{/section/business/economy}{Economy}\textbar{}The Fed Enabled a
Record Expansion. Trump Is Taking Credit.

\url{https://nyti.ms/2ZfZqDB}

\begin{itemize}
\item
\item
\item
\item
\item
\end{itemize}

\begin{itemize}
\item
  \href{https://www.nytimes3xbfgragh.onion/live/2020/09/11/us/trump-vs-biden?action=click\&pgtype=Article\&state=default\&region=TOP_BANNER\&context=storylines_menu}{Election
  Updates}
\item
  \href{https://www.nytimes3xbfgragh.onion/interactive/2020/us/elections/election-states-biden-trump.html?action=click\&pgtype=Article\&state=default\&region=TOP_BANNER\&context=storylines_menu}{Paths
  to 270}
\item
  \href{https://www.nytimes3xbfgragh.onion/interactive/2019/us/elections/2020-presidential-election-calendar.html?action=click\&pgtype=Article\&state=default\&region=TOP_BANNER\&context=storylines_menu}{Key
  Dates}
\item
  \href{https://www.nytimes3xbfgragh.onion/interactive/2020/08/31/us/politics/vote-by-mail-deadlines.html?action=click\&pgtype=Article\&state=default\&region=TOP_BANNER\&context=storylines_menu}{Voting
  by Mail}
\item
  \href{https://www.nytimes3xbfgragh.onion/newsletters/politics?action=click\&pgtype=Article\&state=default\&region=TOP_BANNER\&context=storylines_menu}{Politics
  Newsletter}
\end{itemize}

Advertisement

\protect\hyperlink{after-top}{Continue reading the main story}

Supported by

\protect\hyperlink{after-sponsor}{Continue reading the main story}

\hypertarget{the-fed-enabled-a-record-expansion-trump-is-taking-credit}{%
\section{The Fed Enabled a Record Expansion. Trump Is Taking
Credit.}\label{the-fed-enabled-a-record-expansion-trump-is-taking-credit}}

President Trump has taken credit for the lowest unemployment rate in 50
years, but the Federal Reserve's patient policies laid the groundwork.

\includegraphics{https://static01.graylady3jvrrxbe.onion/images/2020/09/08/business/00DC-TrumpEcon-01/merlin_168830211_5c6565c2-779a-4b69-9430-1174db2d301c-articleLarge.jpg?quality=75\&auto=webp\&disable=upscale}

\href{https://www.nytimes3xbfgragh.onion/by/jeanna-smialek}{\includegraphics{https://static01.graylady3jvrrxbe.onion/images/2020/07/03/reader-center/author-jeanna-smialek/author-jeanna-smialek-thumbLarge.png}}\href{https://www.nytimes3xbfgragh.onion/by/jim-tankersley}{\includegraphics{https://static01.graylady3jvrrxbe.onion/images/2018/10/19/multimedia/author-jim-tankersley/author-jim-tankersley-thumbLarge.png}}

By \href{https://www.nytimes3xbfgragh.onion/by/jeanna-smialek}{Jeanna
Smialek} and
\href{https://www.nytimes3xbfgragh.onion/by/jim-tankersley}{Jim
Tankersley}

\begin{itemize}
\item
  Sept. 8, 2020
\item
  \begin{itemize}
  \item
  \item
  \item
  \item
  \item
  \end{itemize}
\end{itemize}

WASHINGTON --- President Trump is using the prepandemic economy to make
a case for his re-election, highlighting time and again that
unemployment rates fell to record low levels for Black and Hispanic
workers in 2019, and that wages were climbing steadily under his watch.

He is also seeking to convince voters that he is rapidly returning
America to that prosperous place following waves of pandemic-wrought job
loss --- fostering what he labeled a ``Super V'' rebound on Sunday ---
and that Joseph R. Biden Jr. would ``destroy'' the economy if he wins in
November.

But Mr. Trump's story line about his economic track record, particularly
what he showcased during his
\href{https://www.nytimes3xbfgragh.onion/2020/08/28/us/politics/trump-rnc-speech-transcript.html}{Republican
National Convention speech} last month, leaves out a crucial detail.
Lucky timing and a patient Federal Reserve were pivotal in driving the
strong labor market of the late 2010s, economists said. The Trump
administration's tax cuts and higher government spending temporarily
nudged the economy, but the trade wars cooled it off, so the
administration's track record was mixed.

That complicated reality is unlikely to stop Mr. Trump from laying claim
to the successes of the 2018 and 2019 job market. But voters who want to
understand what drove such strong hiring and growth might be better off
looking at the actions of the Fed and its chair, Jerome H. Powell, whom
Mr. Trump
\href{https://www.nytimes3xbfgragh.onion/2017/11/02/business/economy/jerome-powell-federal-reserve-trump.html}{nominated}
in late 2017 and then spent more than a year attacking on Twitter and in
speeches.

By retaining his predecessor's patient approach to rate increases ---
and then stopping them altogether as inflation, which the central bank
tries to keep under control, hovered at low levels --- Mr. Powell's Fed
helped to keep the longest economic expansion in United States history
chugging along. The stretch of unbroken growth pushed unemployment to
its lowest level in 50 years, prompting companies to cast a wider net
for employees, pulling long-sidelined workers back into jobs.

``Both monetary and fiscal policy were stimulative, and it did lead to a
strong labor market,'' said Stephanie Aaronson, a former Fed researcher
who is now at the Brookings Institution. Very low inflation ``has given
policymakers the latitude to try new things.''

That matters as more than a talking point: It could fundamentally shape
the post-pandemic economy. The Fed has signaled that it intends to leave
rates low to push unemployment down again, which could help return the
labor market to strong levels. But the challenges posed by business
closures and job reshuffling mean that elected officials, who have
taxing and spending powers that the Fed lacks, may prove crucial to the
speed and scope of the rebound.

``The single most important thing we can do here is to support a strong
labor market,'' Mr. Powell said in late August remarks. ``That is more
of an all-governmental society project,'' and ``to wait to the eighth
and ninth year of the cycle to get those results --- we can do better
than that with other policies.''

To be sure, it is easy to overstate how strong conditions were before
the pandemic struck.

About \href{https://fred.stlouisfed.org/series/LNS11300060}{83 percent
of adults} in their prime working years were in the labor force at the
start of 2020, which was a marked improvement but still down from an
84.6 percent high in the late 1990s.
\href{https://www.pewsocialtrends.org/2020/01/09/trends-in-income-and-wealth-inequality/}{Inequality
prevailed}. Wage growth had picked up from the expansion's early years,
but it remained
\href{https://www.frbatlanta.org/chcs/wage-growth-tracker?panel=2}{shy
of historical records}.

But there is no doubt that the prepandemic job market was robust.
Unemployment had declined to
\href{https://fred.stlouisfed.org/series/UNRATE}{3.5 percent}, its
lowest level in half a century. Prime-age workers who had dropped out of
the labor market were surprising economists by applying for jobs.
Unemployment for
\href{https://www.nytimes3xbfgragh.onion/2020/02/07/business/black-unemployment-wages.html}{Black}
and
\href{https://www.nytimes3xbfgragh.onion/2019/07/01/business/economy/minority-women-hispanics-jobs.html}{Hispanic
workers} hit record lows, and pay was picking up for those who earned
the least.

Now, the pandemic recession has thrown millions out of work, hitting
disadvantaged groups especially hard. Black unemployment stood at
\href{https://fred.stlouisfed.org/series/LNS14000006}{13 percent} in
August, for instance, compared to
\href{https://fred.stlouisfed.org/series/LNS14000003}{7.3 percent} for
white workers.

\includegraphics{https://static01.graylady3jvrrxbe.onion/images/2020/09/08/business/00DC-TrumpEcon-02/merlin_176699319_390c63db-00fa-42a8-a27a-8a49525c5b8b-articleLarge.jpg?quality=75\&auto=webp\&disable=upscale}

Mr. Trump is already taking a victory lap as the job market begins to
heal, calling the rebound ``the fastest labor market recovery from an
economic crisis in history'' during a Sunday news conference. But about
half of the people who have lost jobs since February remain unemployed.
Economists have warned that the return to full strength could become a
grinding process, and Mr. Powell
\href{https://www.npr.org/2020/09/04/909590044/transcript-nprs-full-interview-with-fed-chairman-jerome-powell}{has
said} that some workers may struggle to return to jobs.

Understanding the policy mix that helped make the labor market so strong
in 2019 will be critical to putting the United States back on track for
another robust period of growth.

Some of the policies pushed through by Mr. Trump and lawmakers did help
to bolster economic growth, which can drive hiring, economists said. The
government was
\href{https://fred.stlouisfed.org/series/W068RCQ027SBEA\#0}{spending
more freely}, and the administration's signature tax cuts, passed in
late 2017, seem to have delivered a fleeting jolt to the economy.

Economists at the University of Pennsylvania's Penn Wharton Budget Model
say that the Tax Cuts and Jobs Act helped growth to jump to about 3
percent for 2018, but the effect faded as growth returned to 2.2 percent
in 2019.

``We don't project any material impact on growth from T.C.J.A. in 2019
or going forward,'' said Alexander Arnon, a senior analyst at the Penn
Wharton Budget Model, a research center that analyzes and predicts the
effects of tax and other policy changes on the federal budget.

Data make it clear that the administration's policies were not the whole
story.

A chart of employment gains over the expansion show that they continued
with remarkable consistency, month over month and year after year,
\href{https://fred.stlouisfed.org/series/PAYEMS\#0}{starting from around
2010}. The jobless rate slowly and steadily dropped. And people
gradually trickled back from the labor market's sidelines.

Much of the improvement seems to have been driven by a long, steady
economic expansion, creating a self-sustaining cycle in which workers
got hired, spent more and fueled demand that created more jobs.

Fed policy helped to enable the progress. Starting under Mr. Powell's
predecessor Janet L. Yellen, the central bank chose to lift interest
rates at a historically slow pace, treading carefully to avoid crashing
the expansion while also trying to avoid runaway inflation.

Mr. Powell, who assumed the Fed chair in February 2018, raised rates
four times during his first year --- still a much slower pace than in
prior business cycles --- before pausing in early 2019 as markets
gyrated. Under his watch, the central bank allowed the unemployment rate
to fall to recent lows without trying to offset that change, and even
lowered interest rates in the second half of 2019 to help sustain the
expansion amid Mr. Trump's trade war, which included steep tariffs on
Chinese goods.

Mr. Trump himself was publicly pushing for rate cuts, viewing that as a
way to make the economy
\href{https://www.cnbc.com/2019/07/05/trump-if-we-had-a-fed-that-would-lower-interest-rates-wed-be-like-a-rocket-ship.html}{take
off like a ``rocket.''} He regularly criticized Mr. Powell for the 2018
rate increases and then the slow pace of 2019 rate cuts. The president
implied that Mr. Powell was an ``enemy'' and
\href{https://www.nytimes3xbfgragh.onion/2019/09/11/business/economy/bonehead-trump-jay-powell.html}{called
the Fed chair} and his colleagues ``boneheads'' for not pursuing easier
monetary policy sooner.

But the Fed sets rates independently of the White House and consistently
ignored his advice. When it did move, it neither did so at the speed the
president sought, nor by using the emergency monetary tools that he
wanted.

The good news for the post-crisis recovery and rebound is that the Fed
is likely to again let unemployment fall sharply.

In an update to its long-run framework in late August, the Fed
officially signaled that it will no longer raise interest rates because
of a low unemployment rate alone, effectively codifying the practice
adopted last year.

The bad news is that the central bank is low on ammunition to prod the
economy. It was able to cut interest rates by only 1.5 percentage points
when the pandemic started, compared to cuts that totaled about 5 percent
during the prior two recessions. Relying too much on low rates could
make for another very gradual recovery --- one like the last long
expansion, which took nearly a decade to really pull workers in from the
sidelines.

``We really need it to be broader than just the Fed,'' Mr. Powell said
of post-pandemic labor market policies, speaking at the Kansas City
Fed's conference in late August.

Mr. Trump and his allies are correct in arguing that leadership from the
White House could matter enormously. Government taxing and spending will
be paramount to the strength of the coming business cycle.

``President Trump and a President Biden would pursue different fiscal
policy paths,'' said Michael Strain, who studies economic policy at the
American Enterprise Institute. ``There might also be differences in how
they pursue the public health situation.''

Most economists argue that more government spending will be important to
keeping America on a path toward full recovery. There is less agreement
over shutdown versus reopening: Some stress the primacy of controlling
the virus, while others argue that the costs to business and jobs are
too steep.

``There are a lot of unknowns,'' Mr. Strain said.

\hypertarget{our-2020-election-guide}{%
\section{Our 2020 Election Guide}\label{our-2020-election-guide}}

Updated ~Sept. 11, 2020

\begin{center}\rule{0.5\linewidth}{\linethickness}\end{center}

\begin{itemize}
\item ~
  \hypertarget{the-latest}{%
  \subsection{The Latest}\label{the-latest}}

  \begin{itemize}
  \item
    Joe Biden and President Trump put
    \href{https://www.nytimes3xbfgragh.onion/2020/09/11/us/politics/shanksville-trump-biden.html?action=click\&pgtype=Article\&state=default\&region=BELOW_MAIN_CONTENT\&context=storylines_guide}{hostilities
    on hold today to travel to ground zero and then to Shanksville, Pa.,
    where they separately honored 9/11 victims}.
  \end{itemize}
\item ~
  \hypertarget{how-to-win-270}{%
  \subsection{How to Win 270}\label{how-to-win-270}}

  \begin{itemize}
  \item
    Joe Biden and Donald Trump need 270 electoral votes to reach the
    White House. Try building
    \href{https://www.nytimes3xbfgragh.onion/interactive/2020/us/elections/election-states-biden-trump.html?action=click\&pgtype=Article\&state=default\&region=BELOW_MAIN_CONTENT\&context=storylines_guide}{your
    own coalition of battleground states}~to see potential outcomes.
  \end{itemize}
\item ~
  \hypertarget{voting-by-mail}{%
  \subsection{Voting by Mail}\label{voting-by-mail}}

  \begin{itemize}
  \item
    Will you have enough time to vote by mail in your state? Yes, but
    it's risky to procrastinate.
    \href{https://www.nytimes3xbfgragh.onion/interactive/2020/08/31/us/politics/vote-by-mail-deadlines.html?action=click\&pgtype=Article\&state=default\&region=BELOW_MAIN_CONTENT\&context=storylines_guide}{Check
    your state's deadline.}
  \item
    \href{https://www.nytimes3xbfgragh.onion/interactive/2020/us/elections/joe-biden.html?action=click\&pgtype=Article\&state=default\&region=BELOW_MAIN_CONTENT\&context=storylines_guide}{}

    \hypertarget{joe-biden}{%
    \section{Joe Biden}\label{joe-biden}}

    \hypertarget{democrat}{%
    \subsection{Democrat}\label{democrat}}

    \href{https://www.nytimes3xbfgragh.onion/interactive/2020/us/elections/donald-trump.html?action=click\&pgtype=Article\&state=default\&region=BELOW_MAIN_CONTENT\&context=storylines_guide}{}

    \hypertarget{donald-trump}{%
    \section{Donald Trump}\label{donald-trump}}

    \hypertarget{republican}{%
    \subsection{Republican}\label{republican}}
  \end{itemize}
\item
  \hypertarget{keep-up-with-our-coverage}{%
  \subsection{Keep Up With Our
  Coverage}\label{keep-up-with-our-coverage}}

  \begin{itemize}
  \item
    Get an
    \href{https://www.nytimes3xbfgragh.onion/newsletters/politics?action=click\&pgtype=Article\&state=default\&region=BELOW_MAIN_CONTENT\&context=storylines_guide}{email}~recapping
    the day's news
  \item
    Download our mobile app on
    \href{https://apps.apple.com/us/app/nytimes/id284862083?ls=1\&mat_click_id=5c79ae7455014fd1bd66b5610c05b8f2-20191112-16948\&referrer=mat_click_id\%3D5c79ae7455014fd1bd66b5610c05b8f2-20191112-16948\%26link_click_id\%3D722930677036718082}{iOS}~and
    \href{http://a.localytics.com/android?id=com.nytimes.android\&referrer=utm_source\%3Dother_nyt_mobile_web\%26utm_medium\%3DWeb\%2520page\%26utm_term\%3DGeneral\%2520Mobile\%2520Page\%26utm_campaign\%3DNYT\%2520Mobile\%2520General\%2520Page}{Android}~and
    turn on Breaking News and Politics alerts
  \end{itemize}
\end{itemize}

Advertisement

\protect\hyperlink{after-bottom}{Continue reading the main story}

\hypertarget{site-index}{%
\subsection{Site Index}\label{site-index}}

\hypertarget{site-information-navigation}{%
\subsection{Site Information
Navigation}\label{site-information-navigation}}

\begin{itemize}
\tightlist
\item
  \href{https://help.nytimes3xbfgragh.onion/hc/en-us/articles/115014792127-Copyright-notice}{©~2020~The
  New York Times Company}
\end{itemize}

\begin{itemize}
\tightlist
\item
  \href{https://www.nytco.com/}{NYTCo}
\item
  \href{https://help.nytimes3xbfgragh.onion/hc/en-us/articles/115015385887-Contact-Us}{Contact
  Us}
\item
  \href{https://www.nytco.com/careers/}{Work with us}
\item
  \href{https://nytmediakit.com/}{Advertise}
\item
  \href{http://www.tbrandstudio.com/}{T Brand Studio}
\item
  \href{https://www.nytimes3xbfgragh.onion/privacy/cookie-policy\#how-do-i-manage-trackers}{Your
  Ad Choices}
\item
  \href{https://www.nytimes3xbfgragh.onion/privacy}{Privacy}
\item
  \href{https://help.nytimes3xbfgragh.onion/hc/en-us/articles/115014893428-Terms-of-service}{Terms
  of Service}
\item
  \href{https://help.nytimes3xbfgragh.onion/hc/en-us/articles/115014893968-Terms-of-sale}{Terms
  of Sale}
\item
  \href{https://spiderbites.nytimes3xbfgragh.onion}{Site Map}
\item
  \href{https://help.nytimes3xbfgragh.onion/hc/en-us}{Help}
\item
  \href{https://www.nytimes3xbfgragh.onion/subscription?campaignId=37WXW}{Subscriptions}
\end{itemize}
