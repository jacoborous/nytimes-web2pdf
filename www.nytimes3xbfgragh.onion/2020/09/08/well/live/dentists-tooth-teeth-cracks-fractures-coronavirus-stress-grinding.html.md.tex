Sections

SEARCH

\protect\hyperlink{site-content}{Skip to
content}\protect\hyperlink{site-index}{Skip to site index}

\href{https://www.nytimes3xbfgragh.onion/section/well/live}{Live}

\href{https://myaccount.nytimes3xbfgragh.onion/auth/login?response_type=cookie\&client_id=vi}{}

\href{https://www.nytimes3xbfgragh.onion/section/todayspaper}{Today's
Paper}

\href{/section/well/live}{Live}\textbar{}Dentists Are Seeing an Epidemic
of Cracked Teeth. What's Going On?

\url{https://nyti.ms/35eZ3Ny}

\begin{itemize}
\item
\item
\item
\item
\item
\item
\end{itemize}

\hypertarget{the-coronavirus-outbreak}{%
\subsubsection{\texorpdfstring{\href{https://www.nytimes3xbfgragh.onion/news-event/coronavirus?name=styln-coronavirus-national\&region=TOP_BANNER\&block=storyline_menu_recirc\&action=click\&pgtype=Article\&impression_id=72050c80-f1f0-11ea-a7d5-df0e27ecb6dd\&variant=undefined}{The
Coronavirus
Outbreak}}{The Coronavirus Outbreak}}\label{the-coronavirus-outbreak}}

\begin{itemize}
\tightlist
\item
  live\href{https://www.nytimes3xbfgragh.onion/2020/09/08/world/covid-19-coronavirus.html?name=styln-coronavirus-national\&region=TOP_BANNER\&block=storyline_menu_recirc\&action=click\&pgtype=Article\&impression_id=72050c81-f1f0-11ea-a7d5-df0e27ecb6dd\&variant=undefined}{Latest
  Updates}
\item
  \href{https://www.nytimes3xbfgragh.onion/interactive/2020/us/coronavirus-us-cases.html?name=styln-coronavirus-national\&region=TOP_BANNER\&block=storyline_menu_recirc\&action=click\&pgtype=Article\&impression_id=72050c82-f1f0-11ea-a7d5-df0e27ecb6dd\&variant=undefined}{Maps
  and Cases}
\item
  \href{https://www.nytimes3xbfgragh.onion/interactive/2020/science/coronavirus-vaccine-tracker.html?name=styln-coronavirus-national\&region=TOP_BANNER\&block=storyline_menu_recirc\&action=click\&pgtype=Article\&impression_id=72053390-f1f0-11ea-a7d5-df0e27ecb6dd\&variant=undefined}{Vaccine
  Tracker}
\item
  \href{https://www.nytimes3xbfgragh.onion/2020/09/02/your-money/eviction-moratorium-covid.html?name=styln-coronavirus-national\&region=TOP_BANNER\&block=storyline_menu_recirc\&action=click\&pgtype=Article\&impression_id=72053391-f1f0-11ea-a7d5-df0e27ecb6dd\&variant=undefined}{Eviction
  Moratorium}
\item
  \href{https://www.nytimes3xbfgragh.onion/interactive/2020/09/02/magazine/food-insecurity-hunger-us.html?name=styln-coronavirus-national\&region=TOP_BANNER\&block=storyline_menu_recirc\&action=click\&pgtype=Article\&impression_id=72053392-f1f0-11ea-a7d5-df0e27ecb6dd\&variant=undefined}{American
  Hunger}
\end{itemize}

Advertisement

\protect\hyperlink{after-top}{Continue reading the main story}

Supported by

\protect\hyperlink{after-sponsor}{Continue reading the main story}

\hypertarget{dentists-are-seeing-an-epidemic-of-cracked-teeth-whats-going-on}{%
\section{Dentists Are Seeing an Epidemic of Cracked Teeth. What's Going
On?}\label{dentists-are-seeing-an-epidemic-of-cracked-teeth-whats-going-on}}

When I reopened my dental practice in early June, the tooth fractures
started coming in: at least one a day, every single day that I've been
in the office.

\includegraphics{https://static01.graylady3jvrrxbe.onion/images/2020/09/02/well/well-teeth/well-teeth-articleLarge.jpg?quality=75\&auto=webp\&disable=upscale}

By Tammy Chen, D.D.S.

\begin{itemize}
\item
  Sept. 8, 2020, 5:00 a.m. ET
\item
  \begin{itemize}
  \item
  \item
  \item
  \item
  \item
  \item
  \end{itemize}
\end{itemize}

``How's your dental practice?'' a friend of mine asked, brow furrowed,
concern evident on her face.

I've seen this look a lot recently. Since the onset of the pandemic,
with a citywide shutdown and social distancing measures firmly
entrenched, more than a handful of friends and family members have
presumed I must be on the brink of closing. But I let her know that I am
busier than ever.

``Really?'' she asked. ``How's that possible?''

``I've seen more tooth fractures in the last six weeks than in the
previous six years,'' I explained.

Unfortunately, that's no exaggeration.

I closed my midtown Manhattan practice to all but dental emergencies in
mid-March, in line with American Dental Association guidelines and state
government mandate. Almost immediately, I noticed an uptick in phone
calls: jaw pain, tooth sensitivity, achiness in the cheeks, migraines.
Most of these patients I effectively treated via telemedicine.

But when I reopened my practice in early June, the fractures started
coming in: at least one a day, every single day that I've been in the
office. On average, I'm seeing three to four; the bad days are six-plus
fractures.

What's going on?

One obvious answer is stress. From
\href{https://www.nytimes3xbfgragh.onion/2020/04/13/style/why-weird-dreams-coronavirus.html}{Covid-induced
nightmares} to
``\href{https://www.nytimes3xbfgragh.onion/2020/03/20/technology/coronavirus-doomsurfing.html}{doomsurfing}''
to
``\href{https://www.wellandgood.com/coronavirus-anxiety-scale-coronaphobia/}{coronaphobia},''
it's no secret that pandemic-related anxiety is affecting our collective
mental health. That stress, in turn, leads to clenching and grinding,
which can damage the teeth.

\hypertarget{latest-updates-the-coronavirus-outbreak}{%
\section{\texorpdfstring{\href{https://www.nytimes3xbfgragh.onion/2020/09/08/world/covid-19-coronavirus.html?action=click\&pgtype=Article\&state=default\&region=MAIN_CONTENT_1\&context=storylines_live_updates}{Latest
Updates: The Coronavirus
Outbreak}}{Latest Updates: The Coronavirus Outbreak}}\label{latest-updates-the-coronavirus-outbreak}}

Updated 2020-09-08T16:13:48.390Z

\begin{itemize}
\tightlist
\item
  \href{https://www.nytimes3xbfgragh.onion/2020/09/08/world/covid-19-coronavirus.html?action=click\&pgtype=Article\&state=default\&region=MAIN_CONTENT_1\&context=storylines_live_updates\#link-679303d7}{Nine
  drugmakers pledge to thoroughly vet any coronavirus vaccine.}
\item
  \href{https://www.nytimes3xbfgragh.onion/2020/09/08/world/covid-19-coronavirus.html?action=click\&pgtype=Article\&state=default\&region=MAIN_CONTENT_1\&context=storylines_live_updates\#link-547feae1}{Senate
  Republicans plan to move forward with a scaled-back stimulus package.}
\item
  \href{https://www.nytimes3xbfgragh.onion/2020/09/08/world/covid-19-coronavirus.html?action=click\&pgtype=Article\&state=default\&region=MAIN_CONTENT_1\&context=storylines_live_updates\#link-1c973131}{`The
  lockdown killed my father': Farmer suicides add to India's virus
  misery.}
\end{itemize}

\href{https://www.nytimes3xbfgragh.onion/2020/09/08/world/covid-19-coronavirus.html?action=click\&pgtype=Article\&state=default\&region=MAIN_CONTENT_1\&context=storylines_live_updates}{See
more updates}

More live coverage:
\href{https://www.nytimes3xbfgragh.onion/live/2020/09/08/business/stock-market-today-coronavirus?action=click\&pgtype=Article\&state=default\&region=MAIN_CONTENT_1\&context=storylines_live_updates}{Markets}

But more specifically, the surge I'm seeing in tooth trauma may be a
result of two additional factors.

First, an unprecedented number of Americans are suddenly working from
home, often wherever they can cobble together a makeshift workstation:
on the sofa, perched on a barstool, tucked into a corner of the kitchen
counter. The awkward body positions that ensue can cause us to
\href{https://www.nytimes3xbfgragh.onion/2020/09/04/well/live/ergonomics-work-from-home-injuries.html}{hunch
our shoulders forward}, curving the spine into something resembling a
C-shape.

If you're wondering why a dentist cares about ergonomics, the simple
truth is that nerves in your neck and shoulder muscles lead into the
temporomandibular joint, or TMJ, which connects the jawbone to the
skull. Poor posture during the day can translate into a grinding problem
at night.

Second, most of us aren't getting the restorative sleep we need. Since
the onset of the pandemic, I've listened to patient after patient
describe sudden restlessness and insomnia. These are hallmarks of an
overactive or dominant sympathetic nervous system, which drives the
body's ``fight or flight'' response. Think of a gladiator preparing for
battle: balling his fists, clenching his jaw. Because of the stress of
coronavirus, the body stays in a battle-ready state of arousal, instead
of resting and recharging. All that tension goes straight to the teeth.

So what can we do?

You'd be surprised how many people are unaware that they're clenching
and grinding. Even patients who come into the office complaining of pain
and sensitivity are often incredulous when I point it out. ``Oh, no. I
don't grind my teeth,'' is a refrain I hear over and over again, despite
the fact that I'm often \emph{watching} them do it.

Awareness is key. Are your teeth currently touching? Even as you read
this article? If so, that's a sure sign that you're doing some damage
--- your teeth shouldn't actually touch throughout the day at all unless
you're actively eating and chewing your food. Instead, your jaw should
be relaxed, with a bit of space between the teeth when the lips are
closed. Be mindful, and try to stop yourself from grinding when you
catch yourself doing it.

If you have a night guard or retainer, devices that keep the teeth in
proper alignment and prevent grinding, try popping them in during the
day. These appliances provide a physical barrier, absorbing and
dispersing pressure. As I often tell my patients, I'd much rather you
crack a night guard than crack a tooth. Your dentist can custom make a
night guard to assure proper fit.

And since many of us will be continuing to work from home for months, it
is imperative to set up a proper work station. Ideally, when seated,
your shoulders should be over your hips, and your ears should be over
your shoulders. Computer screens should be at eye level; prop up your
monitor or laptop on a box or a stack of books if you don't have an
adjustable chair or desk.

\href{https://www.nytimes3xbfgragh.onion/news-event/coronavirus?action=click\&pgtype=Article\&state=default\&region=MAIN_CONTENT_3\&context=storylines_faq}{}

\hypertarget{the-coronavirus-outbreak-}{%
\subsubsection{The Coronavirus Outbreak
›}\label{the-coronavirus-outbreak-}}

\hypertarget{frequently-asked-questions}{%
\paragraph{Frequently Asked
Questions}\label{frequently-asked-questions}}

Updated September 4, 2020

\begin{itemize}
\item ~
  \hypertarget{what-are-the-symptoms-of-coronavirus}{%
  \paragraph{What are the symptoms of
  coronavirus?}\label{what-are-the-symptoms-of-coronavirus}}

  \begin{itemize}
  \tightlist
  \item
    In the beginning, the coronavirus
    \href{https://www.nytimes3xbfgragh.onion/article/coronavirus-facts-history.html?action=click\&pgtype=Article\&state=default\&region=MAIN_CONTENT_3\&context=storylines_faq\#link-6817bab5}{seemed
    like it was primarily a respiratory illness}~--- many patients had
    fever and chills, were weak and tired, and coughed a lot, though
    some people don't show many symptoms at all. Those who seemed
    sickest had pneumonia or acute respiratory distress syndrome and
    received supplemental oxygen. By now, doctors have identified many
    more symptoms and syndromes. In April,
    \href{https://www.nytimes3xbfgragh.onion/2020/04/27/health/coronavirus-symptoms-cdc.html?action=click\&pgtype=Article\&state=default\&region=MAIN_CONTENT_3\&context=storylines_faq}{the
    C.D.C. added to the list of early signs}~sore throat, fever, chills
    and muscle aches. Gastrointestinal upset, such as diarrhea and
    nausea, has also been observed. Another telltale sign of infection
    may be a sudden, profound diminution of one's
    \href{https://www.nytimes3xbfgragh.onion/2020/03/22/health/coronavirus-symptoms-smell-taste.html?action=click\&pgtype=Article\&state=default\&region=MAIN_CONTENT_3\&context=storylines_faq}{sense
    of smell and taste.}~Teenagers and young adults in some cases have
    developed painful red and purple lesions on their fingers and toes
    --- nicknamed ``Covid toe'' --- but few other serious symptoms.
  \end{itemize}
\item ~
  \hypertarget{why-is-it-safer-to-spend-time-together-outside}{%
  \paragraph{Why is it safer to spend time together
  outside?}\label{why-is-it-safer-to-spend-time-together-outside}}

  \begin{itemize}
  \tightlist
  \item
    \href{https://www.nytimes3xbfgragh.onion/2020/05/15/us/coronavirus-what-to-do-outside.html?action=click\&pgtype=Article\&state=default\&region=MAIN_CONTENT_3\&context=storylines_faq}{Outdoor
    gatherings}~lower risk because wind disperses viral droplets, and
    sunlight can kill some of the virus. Open spaces prevent the virus
    from building up in concentrated amounts and being inhaled, which
    can happen when infected people exhale in a confined space for long
    stretches of time, said Dr. Julian W. Tang, a virologist at the
    University of Leicester.
  \end{itemize}
\item ~
  \hypertarget{why-does-standing-six-feet-away-from-others-help}{%
  \paragraph{Why does standing six feet away from others
  help?}\label{why-does-standing-six-feet-away-from-others-help}}

  \begin{itemize}
  \tightlist
  \item
    The coronavirus spreads primarily through droplets from your mouth
    and nose, especially when you cough or sneeze. The C.D.C., one of
    the organizations using that measure,
    \href{https://www.nytimes3xbfgragh.onion/2020/04/14/health/coronavirus-six-feet.html?action=click\&pgtype=Article\&state=default\&region=MAIN_CONTENT_3\&context=storylines_faq}{bases
    its recommendation of six feet}~on the idea that most large droplets
    that people expel when they cough or sneeze will fall to the ground
    within six feet. But six feet has never been a magic number that
    guarantees complete protection. Sneezes, for instance, can launch
    droplets a lot farther than six feet,
    \href{https://jamanetwork.com/journals/jama/fullarticle/2763852}{according
    to a recent study}. It's a rule of thumb: You should be safest
    standing six feet apart outside, especially when it's windy. But
    keep a mask on at all times, even when you think you're far enough
    apart.
  \end{itemize}
\item ~
  \hypertarget{i-have-antibodies-am-i-now-immune}{%
  \paragraph{I have antibodies. Am I now
  immune?}\label{i-have-antibodies-am-i-now-immune}}

  \begin{itemize}
  \tightlist
  \item
    As of right
    now,\href{https://www.nytimes3xbfgragh.onion/2020/07/22/health/covid-antibodies-herd-immunity.html?action=click\&pgtype=Article\&state=default\&region=MAIN_CONTENT_3\&context=storylines_faq}{~that
    seems likely, for at least several months.}~There have been
    frightening accounts of people suffering what seems to be a second
    bout of Covid-19. But experts say these patients may have a
    drawn-out course of infection, with the virus taking a slow toll
    weeks to months after initial exposure.~People infected with the
    coronavirus typically
    \href{https://www.nature.com/articles/s41586-020-2456-9}{produce}~immune
    molecules called antibodies, which are
    \href{https://www.nytimes3xbfgragh.onion/2020/05/07/health/coronavirus-antibody-prevalence.html?action=click\&pgtype=Article\&state=default\&region=MAIN_CONTENT_3\&context=storylines_faq}{protective
    proteins made in response to an
    infection}\href{https://www.nytimes3xbfgragh.onion/2020/05/07/health/coronavirus-antibody-prevalence.html?action=click\&pgtype=Article\&state=default\&region=MAIN_CONTENT_3\&context=storylines_faq}{.
    These antibodies may}~last in the body
    \href{https://www.nature.com/articles/s41591-020-0965-6}{only two to
    three months}, which may seem worrisome, but that's~perfectly normal
    after an acute infection subsides, said Dr. Michael Mina, an
    immunologist at Harvard University. It may be possible to get the
    coronavirus again, but it's highly unlikely that it would be
    possible in a short window of time from initial infection or make
    people sicker the second time.
  \end{itemize}
\item ~
  \hypertarget{what-are-my-rights-if-i-am-worried-about-going-back-to-work}{%
  \paragraph{What are my rights if I am worried about going back to
  work?}\label{what-are-my-rights-if-i-am-worried-about-going-back-to-work}}

  \begin{itemize}
  \tightlist
  \item
    Employers have to provide
    \href{https://www.osha.gov/SLTC/covid-19/standards.html}{a safe
    workplace}~with policies that protect everyone equally.
    \href{https://www.nytimes3xbfgragh.onion/article/coronavirus-money-unemployment.html?action=click\&pgtype=Article\&state=default\&region=MAIN_CONTENT_3\&context=storylines_faq}{And
    if one of your co-workers tests positive for the coronavirus, the
    C.D.C.}~has said that
    \href{https://www.cdc.gov/coronavirus/2019-ncov/community/guidance-business-response.html}{employers
    should tell their employees}~-\/- without giving you the sick
    employee's name -\/- that they may have been exposed to the virus.
  \end{itemize}
\end{itemize}

Consider, too, that in our new home offices, it's not uncommon to roll
out of bed, find a couch, then sit for nine hours a day. Try to mix it
up with some standing, whenever possible, and incorporate more movement.
Use each and every bathroom break, or phone call, as an opportunity to
take more steps, no matter how small your home or apartment might be.

At the end of the workday, I advise my patients to --- excuse the very
technical, medical term here --- ``wiggle like a fish.'' Lie down on the
floor on your back, with your arms extended straight above your head,
and gently wiggle your arms, shoulders, hips and feet from side to side.
The goal is to decompress and elongate the spine, as well as release and
relieve some of that tension and pressure.

If you've got a bathtub, consider taking a 20-minute Epsom salt soak in
the evening. Focus on breathing through your nose and relaxing, rather
than thinking about work, scrolling through emails, or contemplating
your kids' back-to-school schedule (easier said than done, I know).

Then, right before bed, take five minutes to quiet your mind. Close your
eyes, suction your tongue to the roof of your mouth, and breathe in and
out through your nose, in and out through your nose. It's a decidedly
low-tech solution, but deep breathing is one of the most effective ways
to stimulate the vagus nerve, which controls the body's parasympathetic
nervous system. ** A counterpart to the fight or flight response, the
parasympathetic nervous system triggers the body's ``rest and digest''
mechanism, slowing the heart rate, lowering blood pressure, allowing for
more restful, restorative sleep. The more relaxed your body, the more
likely you are to wake up with less tension in the jaw. That means less
grinding at night.

Teeth are naturally brittle, and everyone has tiny fissures in their
teeth from chewing, grinding and everyday use. They can take only so
much trauma before they eventually break. Think of a wall that has a
tiny spider crack that, with weathering, can become bigger and bigger
until it becomes a gaping hole. We want to prevent any added stress from
grinding that could cause these microscopic cracks to propagate into
larger cracks and, ultimately, a catastrophic failure requiring root
canal, a crown or other major dental treatment.

If you haven't already done so, make an appointment with your dentist.
Stay up on your six-month screening and cleaning schedule.

And if you do nothing else, get a night guard.

\emph{Tammy Chen is a prosthodontist and the owner of}
\href{https://www.cpdanyc.com/}{\emph{Central Park Dental Aesthetics}}
\emph{in Midtown Manhattan.}

Advertisement

\protect\hyperlink{after-bottom}{Continue reading the main story}

\hypertarget{site-index}{%
\subsection{Site Index}\label{site-index}}

\hypertarget{site-information-navigation}{%
\subsection{Site Information
Navigation}\label{site-information-navigation}}

\begin{itemize}
\tightlist
\item
  \href{https://help.nytimes3xbfgragh.onion/hc/en-us/articles/115014792127-Copyright-notice}{©~2020~The
  New York Times Company}
\end{itemize}

\begin{itemize}
\tightlist
\item
  \href{https://www.nytco.com/}{NYTCo}
\item
  \href{https://help.nytimes3xbfgragh.onion/hc/en-us/articles/115015385887-Contact-Us}{Contact
  Us}
\item
  \href{https://www.nytco.com/careers/}{Work with us}
\item
  \href{https://nytmediakit.com/}{Advertise}
\item
  \href{http://www.tbrandstudio.com/}{T Brand Studio}
\item
  \href{https://www.nytimes3xbfgragh.onion/privacy/cookie-policy\#how-do-i-manage-trackers}{Your
  Ad Choices}
\item
  \href{https://www.nytimes3xbfgragh.onion/privacy}{Privacy}
\item
  \href{https://help.nytimes3xbfgragh.onion/hc/en-us/articles/115014893428-Terms-of-service}{Terms
  of Service}
\item
  \href{https://help.nytimes3xbfgragh.onion/hc/en-us/articles/115014893968-Terms-of-sale}{Terms
  of Sale}
\item
  \href{https://spiderbites.nytimes3xbfgragh.onion}{Site Map}
\item
  \href{https://help.nytimes3xbfgragh.onion/hc/en-us}{Help}
\item
  \href{https://www.nytimes3xbfgragh.onion/subscription?campaignId=37WXW}{Subscriptions}
\end{itemize}
