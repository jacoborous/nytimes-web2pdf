Sections

SEARCH

\protect\hyperlink{site-content}{Skip to
content}\protect\hyperlink{site-index}{Skip to site index}

\href{https://www.nytimes3xbfgragh.onion/section/sports/basketball}{Pro
Basketball}

\href{https://myaccount.nytimes3xbfgragh.onion/auth/login?response_type=cookie\&client_id=vi}{}

\href{https://www.nytimes3xbfgragh.onion/section/todayspaper}{Today's
Paper}

\href{/section/sports/basketball}{Pro Basketball}\textbar{}Tina Charles
Is a W.N.B.A. Superstar Hiding in Plain Sight

\url{https://nyti.ms/3m3nkfp}

\begin{itemize}
\item
\item
\item
\item
\item
\end{itemize}

Advertisement

\protect\hyperlink{after-top}{Continue reading the main story}

Supported by

\protect\hyperlink{after-sponsor}{Continue reading the main story}

\hypertarget{tina-charles-is-a-wnba-superstar-hiding-in-plain-sight}{%
\section{Tina Charles Is a W.N.B.A. Superstar Hiding in Plain
Sight}\label{tina-charles-is-a-wnba-superstar-hiding-in-plain-sight}}

Charles has stacks of awards and eye-popping stats from her run with the
Liberty. But a W.N.B.A. championship, and the spotlight, have eluded
her.

\includegraphics{https://static01.graylady3jvrrxbe.onion/images/2020/09/07/sports/07wnba-charles-1/07wnba-charles-1-articleLarge.jpg?quality=75\&auto=webp\&disable=upscale}

By Natalie Weiner

\begin{itemize}
\item
  Sept. 8, 2020
\item
  \begin{itemize}
  \item
  \item
  \item
  \item
  \item
  \end{itemize}
\end{itemize}

Even if Tina Charles never plays another basketball game, she's bound to
be a Hall of Famer. That's according to UConn's Geno Auriemma, her
already-inducted former coach --- but it doesn't take winning 11
national championships to see that her enshrinement is inevitable.

Some of her numerous accolades, though, already have a public place of
honor. At Charlie's Records on Fulton Street in Brooklyn, a banner
that's several feet tall bears a photo of Charles shooting in her
Connecticut Sun uniform. It hangs above racks of calypso, soca and
dancehall records, and next to several signs printed with Charles's No.
31 and ``MVP.'' A large adjacent banner reads ``Bud Light Salutes Tina
Charles, 2012 Olympic Gold Medalist.'' Smaller photos and news clippings
complete the tribute.

About a half-hour walk away is Barclays Center, where the New York
native might have continued chasing a W.N.B.A. championship, one of the
few awards missing from what must be straining shelves of trophies. The
accomplishment would certainly be recognized prominently at Charlie's
Records, her father Rawlston Charles's decades-old Bedford-Stuyvesant
record shop.

In spite of Tina Charles's many rings and medals, her lack of deep
postseason runs in the world's best women's basketball league has made
her an all-time talent who's inexplicably hiding in plain sight.

``I think that eats at her,'' Washington Mystics Coach Mike Thibault
said. ``I think she wants to re-establish that, `Look, I'm one of the
top players and I can help a team win a championship.' That was her big
goal when she went to New York, and it didn't work.''

Charles, who leveraged her franchise player status to get back to her
hometown from the Connecticut Sun in 2014, was traded by the Liberty to
the Washington Mystics earlier this year when the team hit a hard reset,
liquidating almost all of its veteran talent.

Instead of winning a title at Madison Square Garden for the team she
grew up watching, Charles had been relegated to
\href{https://www.nytimes3xbfgragh.onion/2018/06/17/sports/basketball/liberty-westchester-wnba.html}{playing
at the 5,000-seat Westchester County Center} after James L. Dolan put
the Liberty up for sale in late 2017. Joe Tsai, who owns the N.B.A.'s
Nets, purchased the Liberty, giving them a new home at Barclays. Along
with a new general manager, chief executive and head coach, the Liberty
now have a young, development-oriented roster
\href{https://www.nytimes3xbfgragh.onion/2020/07/24/sports/basketball/wnba-eastern-conference-preview.html}{framed
around their 2020 No. 1 draft pick Sabrina Ionescu.}

``I was very thankful to play for the hometown team. Not a lot of people
get that opportunity,'' Charles, 31, said in an interview. ``That's
where I'm leaving it.''

The Liberty declined to comment on the trade, instead sharing a
statement from their chief executive, Keia Clarke: ``While she is no
longer part of the team, her name will forever be synonymous with New
York basketball.''

Charles was traded to the reigning W.N.B.A. champion Mystics, which also
meant reuniting with Thibault --- who coached Charles after the Sun
picked her first over all in the 2010 draft. Thibault helped her win
awards like the rookie of the year (2010) and most valuable player
(2012).

``He was the first person to believe in me,'' Charles said of Thibault.
``When you're consistent as a coach, you know how to get the best out of
your players regardless of who's on your roster.''

What the Mystics also have is an established group of top-tier players,
something that has eluded Charles for much of her professional career.

The Mystics have struggled to a 5-13 record in the W.N.B.A. bubble in
Bradenton, Fla. without key starters like reigning M.V.P.
\href{https://www.nytimes3xbfgragh.onion/2020/07/14/sports/basketball/wnba-delle-donnes-opt-out.html}{Elena
Delle Donne} and
\href{https://www.nytimes3xbfgragh.onion/2020/09/02/sports/basketball/nba-wnba-activism-natasha-cloud.html}{Natasha
Cloud}. Charles is not with the team after receiving a medical exemption
this season because she has
\href{https://www.theplayerstribune.com/en-us/articles/tina-charles-wnba}{extrinsic
asthma} and is at high risk of having complications from Covid-19, the
disease caused by coronavirus. Thibault said that although Charles is on
a one-year deal, she has verbally committed to returning to the Mystics
next year.

The 6-foot-4 center has the fifth most rebounds and the 11th-most points
in W.N.B.A. history --- both because of her exceptional ability and
because she's more or less had to shoulder that much of the load to keep
her teams afloat.

``She's always been \emph{the} one, not one of,'' said Bill Laimbeer,
who coached the Liberty during four of the six seasons Charles played
for the team and now coaches the Las Vegas Aces. ``She's been put in
these situations where she has to be the standard-bearer without other
Olympians around her; she was carrying us so heavily, she just ran out
of gas.''

\includegraphics{https://static01.graylady3jvrrxbe.onion/images/2020/09/04/sports/basketball/07wnba-charles-web-2/merlin_34782540_5893ab58-d3b9-484d-9d73-ea73b5ec9509-articleLarge.jpg?quality=75\&auto=webp\&disable=upscale}

Being \emph{the} one is familiar for Charles. In New York's
always-competitive amateur basketball scene she was center stage,
playing at the Garden with her Amateur Athletic Union team during
Liberty halftimes and then again as the star player of the No. 1-ranked
Christ the King team in 2006
(\href{https://www.nytimes3xbfgragh.onion/2006/01/16/sports/christ-the-king-comes-through.html}{naturally,
she hit the game-winner}). As the best high school player in the
country, she went to UConn, the best women's basketball program in the
country. There, she became the best college player in the country,
leading that team to two more titles.

``The expectations are so high for a kid like that, that I don't know
that there's any way you could have said she exceeded expectations,''
Auriemma said. ``That would be impossible.''

Charles has at least met, if not surpassed, those sky-high expectations
at every level on the court, without necessarily getting much attention
for doing so. Whether that's because of her lack of W.N.B.A. titles or
her generally understated demeanor, Charles is unbothered by so often
being in the background.

``It doesn't make me feel any type of way,'' Charles said. ``If your
team isn't successful, you're not going to get individual success. I
know it has nothing to do with my skill or anything I've been able to
put out on the court.''

Since 2013, Charles has donated her W.N.B.A. salary to her Hopey's Heart
Foundation, which supplies automated external defibrillators to schools
and recreational centers. This year, she's shifting that donation to
organizations that support the Black Lives Matter movement, Black-owned
businesses and Covid-19 relief. Her contributions will be in \$846
increments, in recognition of the eight minutes and 46 seconds that have
come to symbolize how long the police in Minneapolis pressed a knee into
the neck of George Floyd, killing him. Charles attended a memorial
service for Floyd in Brooklyn's Cadman Square Plaza.

``It's gotten overlooked that us W.N.B.A. players were the ones who were
really on the front lines, who were always very vocal,'' Charles said.
She and her Liberty teammates
\href{https://www.nytimes3xbfgragh.onion/2016/07/31/sports/basketball/tina-charles-new-york-liberty-wnba-protest.html}{were
initially fined} for wearing T-shirts that bore the hashtag
\#BlackLivesMatter in 2016; even after being warned by the league about
violating uniform policies,
\href{http://keezonsports.com/photos/2016/7/21/wnba-indiana-fever-vs-new-york-liberty}{she
wore her warm-up shirt inside out in protest}.

``We didn't really have the support of the W.N.B.A. when Philando
Castile and Alton Sterling's lives were lost,'' she said. ``It was
totally different. So it's definitely very beautiful to see them support
this cause just as they support breast cancer awareness, Pride, any
other cause --- you know? It's really important.''

The league later rescinded the fines amid public outcry.

``Tina can be shy, can be quiet, but she's always understood her
responsibility,'' said Swin Cash, the four-time All-Star who played and
protested alongside Charles on the Liberty in 2016. ``It meant so much
to her to let people know, `As a franchise player, you guys are going to
see me. I want to lead in that regard.'''

With her championship aspirations on hold for now, Charles is working on
both her game and drawing attention to the achievements of those around
her via her production company, Thirty-One Enterprises, which has
partnered with Kevin Durant's ``The Boardroom''
\href{https://theboardroom.tv/episode/alana-beard-marissa-coleman-talk-entrepreneurship-with-tina-charles}{to
share the stories of her fellow W.N.B.A. players}. She also reciprocated
her father's in-store homage by directing a documentary about him and
his impact on the New York music scene. The film, called
``\href{https://www.youtube.com/watch?v=0LSyINKMQDg}{Charlie's
Records},'' was released last year.

``In basketball, for me it's winning a championship,'' Charles said.
``That's the ultimate goal for me. But at the end of the day,
championship or not, it's just about how I was able to leave a lasting
impact on someone that I came across. I'm thankful, regardless of how my
story ends basketball-wise.''

Advertisement

\protect\hyperlink{after-bottom}{Continue reading the main story}

\hypertarget{site-index}{%
\subsection{Site Index}\label{site-index}}

\hypertarget{site-information-navigation}{%
\subsection{Site Information
Navigation}\label{site-information-navigation}}

\begin{itemize}
\tightlist
\item
  \href{https://help.nytimes3xbfgragh.onion/hc/en-us/articles/115014792127-Copyright-notice}{©~2020~The
  New York Times Company}
\end{itemize}

\begin{itemize}
\tightlist
\item
  \href{https://www.nytco.com/}{NYTCo}
\item
  \href{https://help.nytimes3xbfgragh.onion/hc/en-us/articles/115015385887-Contact-Us}{Contact
  Us}
\item
  \href{https://www.nytco.com/careers/}{Work with us}
\item
  \href{https://nytmediakit.com/}{Advertise}
\item
  \href{http://www.tbrandstudio.com/}{T Brand Studio}
\item
  \href{https://www.nytimes3xbfgragh.onion/privacy/cookie-policy\#how-do-i-manage-trackers}{Your
  Ad Choices}
\item
  \href{https://www.nytimes3xbfgragh.onion/privacy}{Privacy}
\item
  \href{https://help.nytimes3xbfgragh.onion/hc/en-us/articles/115014893428-Terms-of-service}{Terms
  of Service}
\item
  \href{https://help.nytimes3xbfgragh.onion/hc/en-us/articles/115014893968-Terms-of-sale}{Terms
  of Sale}
\item
  \href{https://spiderbites.nytimes3xbfgragh.onion}{Site Map}
\item
  \href{https://help.nytimes3xbfgragh.onion/hc/en-us}{Help}
\item
  \href{https://www.nytimes3xbfgragh.onion/subscription?campaignId=37WXW}{Subscriptions}
\end{itemize}
