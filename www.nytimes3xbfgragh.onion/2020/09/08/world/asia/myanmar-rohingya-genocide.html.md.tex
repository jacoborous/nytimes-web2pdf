Sections

SEARCH

\protect\hyperlink{site-content}{Skip to
content}\protect\hyperlink{site-index}{Skip to site index}

\href{/section/world/asia}{Asia Pacific}\textbar{}`Kill All You See': In
a First, Myanmar Soldiers Tell of Rohingya Slaughter

\url{https://nyti.ms/2F9c1ld}

\begin{itemize}
\item
\item
\item
\item
\item
\item
\end{itemize}

\includegraphics{https://static01.graylady3jvrrxbe.onion/images/2020/09/08/world/08rohingya-graves-1/merlin_176684058_c9fb54cc-a16a-42b6-b331-4eaed1c08263-articleLarge.jpg?quality=75\&auto=webp\&disable=upscale}

\hypertarget{kill-all-you-see-in-a-first-myanmar-soldiers-tell-of-rohingya-slaughter}{%
\section{`Kill All You See': In a First, Myanmar Soldiers Tell of
Rohingya
Slaughter}\label{kill-all-you-see-in-a-first-myanmar-soldiers-tell-of-rohingya-slaughter}}

Video testimony from two soldiers supports widespread accusations that
Myanmar's military tried to eradicate the ethnic minority in a genocidal
campaign.

The remains of a Rohingya school in Rakhine State in western Myanmar
last year.Credit...Adam Dean for The New York Times

Supported by

\protect\hyperlink{after-sponsor}{Continue reading the main story}

By \href{https://www.nytimes3xbfgragh.onion/by/hannah-beech}{Hannah
Beech}, Saw Nang and
\href{https://www.nytimes3xbfgragh.onion/by/marlise-simons}{Marlise
Simons}

\begin{itemize}
\item
  Sept. 8, 2020
\item
  \begin{itemize}
  \item
  \item
  \item
  \item
  \item
  \item
  \end{itemize}
\end{itemize}

\href{https://www.nytimes3xbfgragh.onion/es/2020/09/08/espanol/mundo/rohinya-genocidio-birmania.html}{Leer
en
español}\href{https://cn.nytimes3xbfgragh.onion/world/20200909/myanmar-rohingya-genocide/}{阅读简体中文版}\href{https://cn.nytimes3xbfgragh.onion/world/20200909/myanmar-rohingya-genocide/zh-hant/}{閱讀繁體中文版}

The two soldiers confess their crimes in a monotone, a few blinks of the
eye their only betrayal of emotion: executions, mass burials, village
obliterations and rape.

The August 2017 order from his commanding officer was clear, Pvt. Myo
Win Tun said in video testimony. ``Shoot all you see and all you hear.''

He said he obeyed, taking part in the massacre of 30 Rohingya Muslims
and burying them in a mass grave near a cell tower and a military base.

Around the same time, in a neighboring township, Pvt. Zaw Naing Tun said
he and his comrades in another battalion followed a nearly identical
directive from his superior: ``Kill all you see, whether children or
adults.''

\includegraphics{https://static01.graylady3jvrrxbe.onion/images/2020/09/08/world/08rohingya-graves-soldiers/merlin_176717454_db6f0804-1f8c-44b4-a8d7-5f33bdcf53ce-articleLarge.jpg?quality=75\&auto=webp\&disable=upscale}

``We wiped out about 20 villages,'' Private Zaw Naing Tun said, adding
that he, too, dumped bodies in a mass grave.

The two soldiers' video testimony, recorded by a rebel militia, is the
first time that members of the Tatmadaw, as Myanmar's military is known,
have openly confessed to taking part in what United Nations officials
say was a genocidal campaign against the country's Rohingya Muslim
minority.

Image

Rohingya refugees from Myanmar in Bangladesh in August 2017. Many had
fled the area of Taung Bazar, where Pvt. Myo Win Tun has confessed to
taking part in atrocities.~Credit...Adam Dean for The New York Times

On Monday, the two men, who fled Myanmar last month, were transported to
The Hague, where the International Criminal Court has opened a case
examining whether Tatmadaw leaders committed large-scale crimes against
the Rohingya.

The atrocities described by the two men echo evidence of serious human
rights abuses gathered from among the more than one million Rohingya
refugees now sheltering in neighboring Bangladesh. What distinguishes
their testimony is that it comes from perpetrators, not victims.

``This is a monumental moment for Rohingya and the people of Myanmar in
their ongoing struggle for justice,'' said Matthew Smith, chief
executive officer at Fortify Rights, a human rights watchdog. ``These
men could be the first perpetrators from Myanmar tried at the I.C.C.,
and the first insider witnesses in the custody of the court.''

The New York Times cannot independently confirm that the two soldiers
committed the crimes to which they confessed. But details in their
narratives conform to descriptions provided by dozens of witnesses and
observers, including Rohingya refugees, Rakhine residents, Tatmadaw
soldiers and local politicians.

And multiple villagers independently confirmed the whereabouts of mass
graves that the soldiers provided in their testimony --- evidence that
will be seized on in investigations at the International Criminal Court
and other legal proceedings. The Myanmar government has repeatedly
denied that such sites exist across the region.

The crimes that the soldiers say were carried out by their infantry
battalions and other security forces --- some 150 civilians killed and
dozens of villages destroyed --- are just a part of Myanmar's long
campaign against the Rohingya. And they portray a concerted, calculated
operation to exterminate a single ethnic minority group, the issue at
the heart of ongoing genocide cases.

The massacres of Rohingya that culminated in 2017 catalyzed one of the
fastest flights of refugees anywhere in the world. Within weeks,
\href{https://www.nytimes3xbfgragh.onion/2019/08/22/world/asia/rohingya-myanmar-repatriation.html}{three-quarters
of a million stateless people} were uprooted from their homes in
Myanmar's western Rakhine State, as security forces attacked their
villages with rifles, machetes and flamethrowers.

Image

Rohingya refugees at a camp near Amtoli, Bangladesh, in August
2017.~Credit...Adam Dean for The New York Times

Old men were decapitated, and young girls were raped, their head scarves
torn off to use as blindfolds, witnesses and survivors said. Doctors
Without Borders estimated that at least
\href{https://www.nytimes3xbfgragh.onion/2017/12/14/world/asia/myanmar-rohingya-deaths.html}{6,700
Rohingya}, including 730 children, suffered violent deaths from late
August to late September 2017. Roughly 200 Rohingya settlements were
completely razed from 2017 to 2019, the United Nations said.

In a
\href{https://www.ohchr.org/Documents/HRBodies/HRCouncil/FFM-Myanmar/20190916/A_HRC_42_CRP.5.pdf}{report}
published last year, a fact-finding mission for the United Nations Human
Rights Council said ``there is a serious risk that genocidal actions may
occur or recur and that Myanmar is failing in its obligation to prevent
genocide, to investigate genocide and to enact effective legislation
criminalizing and punishing genocide.''

The Myanmar government has denied any orchestrated campaign against the
Rohingya. Last December,
\href{https://www.nytimes3xbfgragh.onion/2019/12/10/world/asia/aung-san-suu-kyi-myanmar-genocide-hague.html}{Daw
Aung San Suu Kyi}, the nation's civilian leader, defended Myanmar
against charges of genocide in another case, this one at the
International Court of Justice in The Hague. A Nobel Peace Prize
laureate, Ms. Aung San Suu Kyi has seen her legacy tarnished by her
support for the military and her refusal to vocally condemn the
persecution of the Rohingya.

Only a few Tatmadaw soldiers have been punished, with brief prison
terms, for what the military says were isolated missteps in a couple of
villages.

Although the Rohingya are from Rakhine State in Myanmar,
\href{https://www.nytimes3xbfgragh.onion/2017/10/24/world/asia/myanmar-rohingya-ethnic-cleansing.html}{the
country's government claims that they are foreign interlopers}. Myanmar
officials have suggested that the Rohingya burned down their own
villages to garner international sympathy.

The two soldiers' accounts shatter that official narrative.

It is not clear what will happen to the two men, who are not under
arrest but were effectively placed in the custody of the International
Criminal Court on Monday. They could provide testimony in court
proceedings and be put in witness protection. They could be tried. The
court's office of the prosecutor refused to publicly comment on an
ongoing case, but two people familiar with the investigations said that
the men had already been questioned extensively by court officials in
recent weeks.

Image

Rohingya refugees in September 2017 after crossing into Bangladesh.
Villages are burning in the background.~Credit...Adam Dean for The New
York Times

The International Criminal Court normally pursues prosecutions of
high-level figures accused of grave offenses such as genocide or crimes
against humanity, not rank-and-file soldiers.

Payam Akhavan, a Canadian lawyer who is representing Bangladesh in a
filing against Myanmar at the International Criminal Court, would not
comment on the identities of the two men. But he called for
accountability to prevent further atrocities against the 600,000
Rohingya who remain in Myanmar.

``Impunity is not an option,'' Mr. Akhavan said. ``Some justice is
better than no justice at all.''

The soldiers' accounts will also add weight to the separate case at the
International Court of Justice, where Myanmar is being accused of trying
to ``destroy the Rohingya as a group, in whole or in part, by the use of
\href{https://www.nytimes3xbfgragh.onion/2017/09/02/world/asia/rohingya-myanmar-bangladesh-refugees-massacre.html?action=click\&module=RelatedCoverage\&pgtype=Article\&region=Footer}{mass
murder, rape and other forms of sexual violence}, as well as the
systematic destruction by fire of their villages.''

That case was filed last year by Gambia on behalf of the 57-nation
Organization of Islamic Cooperation. Last week, the Netherlands and
Canada announced that they would provide legal support to the effort to
hold Myanmar accountable for genocide, calling it a matter ``of concern
to all of humanity.''

Image

A refugee camp near Cox's Bazar, Bangladesh, in November 2017. More than
a million Rohingya have taken refuge in Bangladesh.Credit...Adam Dean
for The New York Times

In August 2017, the 353 and 565 Light Infantry Battalions conducted
``clearance operations'' in the areas where the men said they did,
Buthidaung and Maungdaw Townships. Commanding officers whom Private Myo
Win Tun said ordered him to wipe out the Rohingya --- Col. Than Htike,
Capt. Tun Tun and Sgt. Aung San Oo --- were operational there at the
time, according to fellow soldiers.

There is a cell tower close to the 552 Light Infantry Battalion base, on
the outskirts of Taung Bazar town, near where Private Myo Win Tun said
he helped dig a mass grave. The base is well known in the area because
it, along with two dozen border guard posts, was attacked by
\href{https://www.nytimes3xbfgragh.onion/2017/09/17/world/asia/myanmar-rohingya-militants.html?module=inline}{Rohingya
insurgents} on Aug. 25, 2017, galvanizing the brutal military operations
against Rohingya civilians.

Rohingya refugees who lived in a village adjacent to the 552 encampment
said they recognized Private Myo Win Tun. They described in precise
detail the locations of two mass graves in that area. Residents still in
the region, who spoke with The Times, also said they knew of mass burial
sites near the military encampment.

NORTH

552 Light Infantry

Battalion base

Location of mass grave

confirmed by villagers

Cell tower

Thin Ga Net village

Location of another mass grave

confirmed by villagers

NORTH

552 Light Infantry

Battalion base

Location of mass grave

confirmed by villagers

Cell tower

Thin Ga Net village

Location of another mass grave

confirmed by villagers

NORTH

552 Light Infantry

Battalion base

Location of mass grave

confirmed by villagers

Cell tower

Thin Ga Net village

Location of another mass grave

confirmed by villagers

Location of mass grave

confirmed by villagers

Thin Ga Net village

552 Light Infantry

Battalion base

Cell tower

NORTH

Location of another mass grave

confirmed by villagers

By Jin Wu/The New York Times·Satellite image by Maxar Technologies,
taken on September 25, 2017.

Basha Miya, who is now a refugee in Bangladesh, said his grandmother was
buried in one of the mass graves by the base, along with at least 16
others from the neighboring village of Thin Ga Net, known in the
Rohingya language as Phirkhali.

``When I remember her, I just cry,'' he said. ``I feel bad that I
couldn't give her a proper funeral.''

After soldiers dumped the bodies in two graves by the banks of canals,
they brought in bulldozers to cover the corpses, eyewitnesses said.
Private Myo Win Tun said he and others buried eight women, seven
children and 15 men in one grave.

Thin Ga Net village was wiped from the map by fire. Today, only a couple
of water reservoirs hint that a bustling Rohingya village once stood
there.

\hypertarget{thin-ga-net-village}{%
\subsubsection{Thin Ga Net village}\label{thin-ga-net-village}}

\includegraphics{https://static01.graylady3jvrrxbe.onion/packages/flash/multimedia/ICONS/transparent.png}

\includegraphics{https://static01.graylady3jvrrxbe.onion/newsgraphics/2020/09/08/rohingya-2020/assets/images/thin-ga-net-23may2017-2000.jpg}

May 23, 2017

Burned Rohingya villages

Location of mass grave

confirmed by villagers

552 Light Infantry

Battalion base

Burned Rohingya villages

Location of mass grave

confirmed by villagers

552 Light Infantry

Battalion base

Burned

Rohingya

villages

Location of mass grave

confirmed by villagers

552 Light Infantry

Battalion base

Burned

Rohingya

villages

Location of mass grave

confirmed by villagers

552 Light Infantry

Battalion base

Sept. 25, 2017

By Jin Wu/The New York Times·Satellite images by Maxar Technologies

As they marauded through the villages around Taung Bazar, Private Myo
Win Tun, 33, seems to have lost track of how many Rohingya he and his
battalion killed. Was it 60 or 70? Maybe more?

``We indiscriminately shot at everybody,'' he said in video testimony.
``We shot the Muslim men in the foreheads and kicked the bodies into the
hole.''

He also raped a woman, he said.

Private Zaw Naing Tun, a former Buddhist monk, admits to a similar fog,
as his battalion's killing of some 80 Rohingya stretched from hours into
days. The soldier said he and other members of his battalion stormed
through 20 villages in Maungdaw Township, including Doe Tan, Ngan
Chaung, Kyet Yoe Pyin, Zin Paing Nyar and U Shey Kya.

Some of these villages were burned to the ground. Bashir Ahmed said that
Tatmadaw battalions entered his hometown, Zin Paing Nyar, early on Aug.
26, 2017.

\hypertarget{zin-paing-nyar-village}{%
\subsubsection{Zin Paing Nyar village}\label{zin-paing-nyar-village}}

\includegraphics{https://static01.graylady3jvrrxbe.onion/packages/flash/multimedia/ICONS/transparent.png}

\includegraphics{https://static01.graylady3jvrrxbe.onion/newsgraphics/2020/09/08/rohingya-2020/assets/images/cropped-zin-paing-nyar-15feb2017-2000.jpg}

Feb. 15, 2017

Burned

Rohingy

villages

Burned

Rohingya

villages

Burned

Rohingya

villages

Burned

Rohingya

villages

Nov. 26, 2017

Satellite image by Maxar Technologies

\hypertarget{doe-tan-village}{%
\subsubsection{Doe Tan village}\label{doe-tan-village}}

\includegraphics{https://static01.graylady3jvrrxbe.onion/packages/flash/multimedia/ICONS/transparent.png}

\includegraphics{https://static01.graylady3jvrrxbe.onion/newsgraphics/2020/09/08/rohingya-2020/assets/images/doe-tan-23may2017-2000.jpg}

May 23, 2017

Burned

Rohingy

villages

Burned

Rohingya

villages

Burned

Rohingya

villages

Burned

Rohingya

villages

Jan. 9, 2018

Satellite image by Maxar Technologies

``They opened fire whenever they found someone in front of them,'' he
said. ``They burned our houses. Nothing is left.''

More than 30 residents were killed in Zin Paing Nyar, according to
survivors' testimony.

Private Zaw Naing Tun, 30, said that he and four other members of his
battalion shot dead seven Rohingya in Zin Paing Nyar. They captured 10
unarmed men, tied them with ropes, killed them and buried them in a mass
grave north of the village, he said in the video testimony.

There are some discrepancies between the soldiers' accounts and those of
Rohingya villagers. Private Myo Win Tun described the cell tower as
being east of the 552 base when it is, in fact, southwest.

Image

A Rohingya Muslim man reading the Quran at one of the few undamaged
mosques in Ngan Chaung village in northern Rakhine State last
year.Credit...Adam Dean for The New York Times

But most of the other details are corroborated by statements from
witnesses and survivors. In Ngan Chaung village, part of which was
spared destruction, five or six soldiers from Light Infantry Battalion
353 arrived one afternoon in late August 2017 and singled out five women
for rape, said a resident who still lives in the hamlet. The women's
husbands were later killed, he and other residents said.

Private Zaw Naing Tun said he didn't commit sexual violence because he
was too low-ranking to participate. Instead, he stood sentry as others
raped Rohingya women, he said.

Both of the soldiers who admitted to killing Rohingya are themselves
members of ethnic minorities in a country where persecution of such
groups is institutionalized.

Earlier this year, the pair ended up in the custody of the Arakan Army,
an ethnic Rakhine militia currently fighting the Tatmadaw, which
recorded their video confessions. Both men said they deserted from the
Tatmadaw.

Desertion is a particular problem in ethnic minority conflict zones,
military insiders say. About 60 soldiers are believed to have gone
A.W.O.L. from Light Infantry Battalion 565.

``I was racially discriminated against,'' Private Myo Win Tun, a member
of the Shanni ethnic group, said in his video testimony, in a rare burst
of feeling.

Later, he would describe, in a flat voice, how his commanding officer,
Colonel Than Htike, had instructed the battalion to ``exterminate'' the
Rohingya.

``I was involved in the killing of 30 Muslim innocent men, women and
children buried in one grave,'' he said, as he stoically faced the
camera.

Image

The remains of a mosque in Sabal Khone, a razed Rohingya
village.~Credit...Adam Dean for The New York Times

Advertisement

\protect\hyperlink{after-bottom}{Continue reading the main story}

\hypertarget{site-index}{%
\subsection{Site Index}\label{site-index}}

\hypertarget{site-information-navigation}{%
\subsection{Site Information
Navigation}\label{site-information-navigation}}

\begin{itemize}
\tightlist
\item
  \href{https://help.nytimes3xbfgragh.onion/hc/en-us/articles/115014792127-Copyright-notice}{©~2020~The
  New York Times Company}
\end{itemize}

\begin{itemize}
\tightlist
\item
  \href{https://www.nytco.com/}{NYTCo}
\item
  \href{https://help.nytimes3xbfgragh.onion/hc/en-us/articles/115015385887-Contact-Us}{Contact
  Us}
\item
  \href{https://www.nytco.com/careers/}{Work with us}
\item
  \href{https://nytmediakit.com/}{Advertise}
\item
  \href{http://www.tbrandstudio.com/}{T Brand Studio}
\item
  \href{https://www.nytimes3xbfgragh.onion/privacy/cookie-policy\#how-do-i-manage-trackers}{Your
  Ad Choices}
\item
  \href{https://www.nytimes3xbfgragh.onion/privacy}{Privacy}
\item
  \href{https://help.nytimes3xbfgragh.onion/hc/en-us/articles/115014893428-Terms-of-service}{Terms
  of Service}
\item
  \href{https://help.nytimes3xbfgragh.onion/hc/en-us/articles/115014893968-Terms-of-sale}{Terms
  of Sale}
\item
  \href{https://spiderbites.nytimes3xbfgragh.onion}{Site Map}
\item
  \href{https://help.nytimes3xbfgragh.onion/hc/en-us}{Help}
\item
  \href{https://www.nytimes3xbfgragh.onion/subscription?campaignId=37WXW}{Subscriptions}
\end{itemize}
