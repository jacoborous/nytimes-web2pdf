Sections

SEARCH

\protect\hyperlink{site-content}{Skip to
content}\protect\hyperlink{site-index}{Skip to site index}

\href{https://www.nytimes3xbfgragh.onion/section/world/europe}{Europe}

\href{https://myaccount.nytimes3xbfgragh.onion/auth/login?response_type=cookie\&client_id=vi}{}

\href{https://www.nytimes3xbfgragh.onion/section/todayspaper}{Today's
Paper}

\href{/section/world/europe}{Europe}\textbar{}Boris Johnson Facing
Revolt Over Northern Ireland Pact

\url{https://nyti.ms/3bCc3hi}

\begin{itemize}
\item
\item
\item
\item
\item
\end{itemize}

Advertisement

\protect\hyperlink{after-top}{Continue reading the main story}

Supported by

\protect\hyperlink{after-sponsor}{Continue reading the main story}

\hypertarget{boris-johnson-facing-revolt-over-northern-ireland-pact}{%
\section{Boris Johnson Facing Revolt Over Northern Ireland
Pact}\label{boris-johnson-facing-revolt-over-northern-ireland-pact}}

The government's top lawyer has quit, and even a government minister
admits the prime minister's effort to rewrite the treaty violates
international law.

\includegraphics{https://static01.graylady3jvrrxbe.onion/images/2020/09/08/world/08brexit/merlin_176717913_9d809ff6-8477-4676-a16b-8bc27d780dcb-articleLarge.jpg?quality=75\&auto=webp\&disable=upscale}

\href{https://www.nytimes3xbfgragh.onion/by/mark-landler}{\includegraphics{https://static01.graylady3jvrrxbe.onion/images/2019/10/22/reader-center/author-mark-landler/author-mark-landler-thumbLarge-v3.png}}

By \href{https://www.nytimes3xbfgragh.onion/by/mark-landler}{Mark
Landler}

\begin{itemize}
\item
  Published Sept. 8, 2020Updated Sept. 10, 2020
\item
  \begin{itemize}
  \item
  \item
  \item
  \item
  \item
  \end{itemize}
\end{itemize}

LONDON --- Prime Minister
\href{https://www.nytimes3xbfgragh.onion/2020/09/10/world/europe/brexit-boris-johnson-ireland.html}{Boris
Johnson's Brexit} negotiations fell into disarray on Tuesday, as the
government's top lawyer resigned over Mr. Johnson's plan to override a
landmark agreement with the European Union, and one of his own ministers
admitted that the changes would break international law.

The head of the government's legal department, Jonathan Jones, resigned
abruptly, the day before the government planned to introduce legislation
that would rewrite provisions on the treatment of Northern Ireland,
should Britain fail to strike a permanent trade agreement with Brussels
by the end of this year.

Mr. Johnson's aggressive move to pull back from the agreement about
Northern Ireland underscored his determination for Britain to control
its own economic destiny --- even at the cost of triggering another
confrontation with the European Union, shredding his own diplomacy and
raising questions about his government's commitment to the rule of law.

It carried echoes of the moment he suspended Parliament last fall,
effectively squelching a debate over his drive to complete Brexit, a
maneuver that Britain's Supreme Court later declared unlawful.

Mr. Jones did not detail his reasons for resigning, but The Financial
Times, which first reported the news, said he clashed with officials in
Downing Street about plans to rewrite the so-called Northern Ireland
protocol, a central plank of the Withdrawal Agreement, under which
Britain left the European Union last January.

The prospect of Britain reneging on a treaty that Mr. Johnson himself
signed ignited a firestorm in Parliament, with even the prime minister's
Conservative predecessor, Theresa May, sharply criticizing the
government.

``How can the government reassure future international partners that the
U.K. can be trusted to abide by the legal obligations of the agreements
it signs?'' said Ms. May during a tense debate in the House of Commons.
She negotiated the bulk of the Withdrawal Agreement with Brussels.

Image

Jonathan Jones, the head of the government's legal department, resigned
over Prime Minister Boris Johnson's plan to override a landmark
agreement with the European Union.Credit...UK Government

Mr. Johnson's Northern Ireland secretary, Brandon Lewis, insisted that
the proposed changes were not intended to rip up the agreement but
merely to create a ``safety net'' for Northern Ireland businesses in the
event that London could not work out long-term trading arrangements with
Brussels.

But he acknowledged that the changes would ``break international law in
a very specific and limited way.'' Mr. Lewis argued there was a
precedent for Britain to reconsider its obligations when circumstances
change, drawing catcalls from opposition lawmakers and incredulous
looks, even from some Conservatives.

``One of the things the U.K. has always prided itself on is upholding
the rule of the law,'' said Professor Catherine Barnard, an expert on
European Union law at Cambridge University. ``To have a minister of
state stand up and say, `We're breaking international law, but only a
little' is extraordinary.''

Professor Barnard said there were legal contradictions in the agreement,
having to do with when customs duties must be imposed on goods shipped
from Britain to Northern Ireland. But she said the agreement provided
ways to resolve these issues and Britain had no right to change it
unilaterally.

The trade negotiations, which turn on broader issues like Britain's
ability to direct state subsidies to its companies, resumed on Tuesday
in London against an already acrimonious backdrop. Mr. Johnson declared
that a no-deal Brexit would be a ``good outcome'' for Britain and that
the Withdrawal Agreement, which he once hailed as a triumph, was a
flawed deal that needed to be fixed.

Mr. Johnson's reversal seemed driven in part by a desire to mollify the
hard-line Brexiteers in his party. They have long agitated for Britain
to negate the deal on Northern Ireland, complaining that it impinges on
British sovereignty.

Under the agreement, Northern Ireland would remain part of Britain's
customs territory but would abide by European Union rules on issues
ranging from safety standards to state subsidies to industry.

European Union leaders expressed alarm that Britain would go back on an
international treaty, ratified by all the members of the bloc. Ursula
von der Leyen, the president of the European Commission, warned Britain
that a future trade agreement depended on it carrying out this one.

\includegraphics{https://static01.graylady3jvrrxbe.onion/images/2020/09/08/world/08brexit2/merlin_164266566_64079c99-591c-4478-a0d0-4ea06394140e-articleLarge.jpg?quality=75\&auto=webp\&disable=upscale}

The resignation of Mr. Jones, the latest in a parade of civil servants
who have left the Johnson government, left little doubt that the
government's actions raised thorny legal problems. A 58-year-old
barrister who served in the Home Office and the attorney general's
office, Mr. Jones rose to become the government's senior lawyer,
rendering judgment on the legality of new legislation.

He is the sixth senior civil servant to quit since Mr. Johnson won a
landslide election victory last December and his chief political
adviser, Dominic Cummings, embarked on a campaign to shake up Britain's
bureaucracy.

``When your top lawyer resigns, it sends an extraordinary signal,'' said
Bobby McDonagh, who served as the Irish ambassador to Britain.

Britain's move to rewrite the agreement, he said, suggested the Johnson
government did not fully understand the implications of the language its
own diplomats agreed to. ``They're really amateurs when it comes to
trade negotiations,'' Mr. McDonagh said.

The damage to Britain's international reputation could be substantial,
at a time when it is trying to negotiate trade agreements, not just with
the European Union but also with the United States.

In Washington, Congressional Democrats, as well as the Democratic
presidential nominee, former Vice President Joseph R. Biden Jr., are
likely to view any changes in Britain's treatment of Northern Ireland as
jeopardizing the Good Friday Agreement, which ended decades of sectarian
violence there.

Some analysts said they viewed Mr. Johnson's moves as a way to increase
his leverage with the European Union as the trade talks enter a
make-or-break period. But others said the legislation underscored that
members of his government were determined to diverge as sharply as
possible from the bloc.

The biggest bone of contention is the European Union's demand that
Britain abide by its rules on state subsidies to businesses. Britain has
yet to offer a counterproposal, and Mr. Johnson's aides are determined
to keep control over industrial policy so they can pour money into
high-tech sectors like biotechnology and artificial intelligence.

Advertisement

\protect\hyperlink{after-bottom}{Continue reading the main story}

\hypertarget{site-index}{%
\subsection{Site Index}\label{site-index}}

\hypertarget{site-information-navigation}{%
\subsection{Site Information
Navigation}\label{site-information-navigation}}

\begin{itemize}
\tightlist
\item
  \href{https://help.nytimes3xbfgragh.onion/hc/en-us/articles/115014792127-Copyright-notice}{©~2020~The
  New York Times Company}
\end{itemize}

\begin{itemize}
\tightlist
\item
  \href{https://www.nytco.com/}{NYTCo}
\item
  \href{https://help.nytimes3xbfgragh.onion/hc/en-us/articles/115015385887-Contact-Us}{Contact
  Us}
\item
  \href{https://www.nytco.com/careers/}{Work with us}
\item
  \href{https://nytmediakit.com/}{Advertise}
\item
  \href{http://www.tbrandstudio.com/}{T Brand Studio}
\item
  \href{https://www.nytimes3xbfgragh.onion/privacy/cookie-policy\#how-do-i-manage-trackers}{Your
  Ad Choices}
\item
  \href{https://www.nytimes3xbfgragh.onion/privacy}{Privacy}
\item
  \href{https://help.nytimes3xbfgragh.onion/hc/en-us/articles/115014893428-Terms-of-service}{Terms
  of Service}
\item
  \href{https://help.nytimes3xbfgragh.onion/hc/en-us/articles/115014893968-Terms-of-sale}{Terms
  of Sale}
\item
  \href{https://spiderbites.nytimes3xbfgragh.onion}{Site Map}
\item
  \href{https://help.nytimes3xbfgragh.onion/hc/en-us}{Help}
\item
  \href{https://www.nytimes3xbfgragh.onion/subscription?campaignId=37WXW}{Subscriptions}
\end{itemize}
