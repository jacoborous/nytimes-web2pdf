Sections

SEARCH

\protect\hyperlink{site-content}{Skip to
content}\protect\hyperlink{site-index}{Skip to site index}

\href{https://www.nytimes3xbfgragh.onion/section/food}{Food}

\href{https://myaccount.nytimes3xbfgragh.onion/auth/login?response_type=cookie\&client_id=vi}{}

\href{https://www.nytimes3xbfgragh.onion/section/todayspaper}{Today's
Paper}

\href{/section/food}{Food}\textbar{}7 Ways the Pandemic Has Changed How
We Shop for Food

\url{https://nyti.ms/2R7fQK8}

\begin{itemize}
\item
\item
\item
\item
\item
\item
\end{itemize}

\href{https://www.nytimes3xbfgragh.onion/spotlight/at-home?action=click\&pgtype=Article\&state=default\&region=TOP_BANNER\&context=at_home_menu}{At
Home}

\begin{itemize}
\tightlist
\item
  \href{https://www.nytimes3xbfgragh.onion/2020/09/07/travel/route-66.html?action=click\&pgtype=Article\&state=default\&region=TOP_BANNER\&context=at_home_menu}{Cruise
  Along: Route 66}
\item
  \href{https://www.nytimes3xbfgragh.onion/2020/09/04/dining/sheet-pan-chicken.html?action=click\&pgtype=Article\&state=default\&region=TOP_BANNER\&context=at_home_menu}{Roast:
  Chicken With Plums}
\item
  \href{https://www.nytimes3xbfgragh.onion/2020/09/04/arts/television/dark-shadows-stream.html?action=click\&pgtype=Article\&state=default\&region=TOP_BANNER\&context=at_home_menu}{Watch:
  Dark Shadows}
\item
  \href{https://www.nytimes3xbfgragh.onion/interactive/2020/at-home/even-more-reporters-editors-diaries-lists-recommendations.html?action=click\&pgtype=Article\&state=default\&region=TOP_BANNER\&context=at_home_menu}{Explore:
  Reporters' Google Docs}
\end{itemize}

Advertisement

\protect\hyperlink{after-top}{Continue reading the main story}

Supported by

\protect\hyperlink{after-sponsor}{Continue reading the main story}

\hypertarget{7-ways-the-pandemic-has-changed-how-we-shop-for-food}{%
\section{7 Ways the Pandemic Has Changed How We Shop for
Food}\label{7-ways-the-pandemic-has-changed-how-we-shop-for-food}}

Oranges and frozen foods are being snapped up. Shelves have fewer
choices. And customers are steering their carts in surprising new
directions.

\includegraphics{https://static01.graylady3jvrrxbe.onion/images/2020/09/08/dining/08grocery14/merlin_176693238_c1299b27-af78-4d05-949d-2dc21c121e90-articleLarge.jpg?quality=75\&auto=webp\&disable=upscale}

\href{https://www.nytimes3xbfgragh.onion/by/kim-severson}{\includegraphics{https://static01.graylady3jvrrxbe.onion/images/2018/06/13/multimedia/author-kim-severson/author-kim-severson-thumbLarge.jpg}}

By \href{https://www.nytimes3xbfgragh.onion/by/kim-severson}{Kim
Severson}

\begin{itemize}
\item
  Sept. 8, 2020
\item
  \begin{itemize}
  \item
  \item
  \item
  \item
  \item
  \item
  \end{itemize}
\end{itemize}

When the coronavirus hit, even the most enthusiastic cooks had to adjust
to a new, more complicated relationship with their kitchens.

For the first time in a generation, Americans began spending more money
at the supermarket than at places where someone else made the food.
Grocers saw
\href{https://www.theshelbyreport.com/2020/06/12/fmi-report-american-shoppers-adjust/}{eight
years}of projected sales growth packed into one month. Shopping trends
that were in their infancy were turbocharged.

The six-month shift has been a behavioral scientist's dream. Shoppers
began by building bomb-shelter pantries. Then came
\href{https://www.nytimes3xbfgragh.onion/2020/04/07/business/coronavirus-processed-foods.html}{a
nostalgia phase}, with bowls of Lucky Charms and boxes of Little Debbies
offering throwback comfort. Soon, days were defined by elaborate
culinary stunts,
\href{https://www.nytimes3xbfgragh.onion/2020/03/30/style/bread-baking-coronavirus.html}{sourdough
starter} and kombucha clubs.

Although kitchen fatigue is setting in for many, a new set of kitchen
habits have been set.

``People are moving on to more complex cooking, and we don't see that
going away,'' said Rodney McMullen, the chairman and chief executive of
\href{https://www.kroger.com/}{Kroger}, where sales rose 30 percent at
the onset of the pandemic, including big jumps in the pasta aisles, the
beer and wine department and baking supplies, including a 600 percent
jump in sales of yeast.

He and others in the business say the Covid-driven return to the kitchen
could change grocery shopping forever.

``This is a pivotal time in our history,'' said
\href{https://people.umass.edu/nagurney/}{Anna Nagurney}, a professor in
the Isenberg School of Management at the University of Massachusetts who
studies supply chains. ``Not all of what we've seen will stick, but a
lot of it will.''

Here are seven ways the pandemic has already changed the way Americans
shop for food:

\hypertarget{1-trips-are-fewer-lists-are-better}{%
\subsection{1. Trips Are Fewer, Lists Are
Better}\label{1-trips-are-fewer-lists-are-better}}

The need to avoid infection has taught people how to get by on fewer
trips to the store, and to make good shopping lists.

``People now go to the store with purpose,'' said John Owen, the
associate director for food and retail with Mintel, the market analysis
group. ``The number of trips went way down, and the size of the basket
went way up in April. We have eased back on that, but not by much.''

Before the coronavirus, 19 percent of Americans shopped for food more
than three times a week, according to
\href{https://www.mckinsey.com/featured-insights/coronavirus-leading-through-the-crisis/charting-the-path-to-the-next-normal/some-changes-to-grocery-shopping-habits-likely-to-stick-after-the-crisis}{a
study by} the management firm McKinsey \& Company. That number had
dropped to 10 percent by June.

\includegraphics{https://static01.graylady3jvrrxbe.onion/images/2020/09/08/dining/08grocery7/merlin_176462025_47579b87-2833-4e56-ac31-0edbe276e616-articleLarge.jpg?quality=75\&auto=webp\&disable=upscale}

``My typical grocery shopping before the pandemic was very much `I am
going to decide today what I feel like making for dinner tonight, and
stop on the way home and get what I need,''' said Lizzie Bowman, 39, a
marketing director at American Public Media who lives in Minneapolis.

She has streamlined her shopping to once a week. ``It's more of a
stock-up, but not a crazy kind of hoarding stock-up.''

She won't go to stores that don't set good health protocols, and leans
into those that offer more local and regional food. Her regular rotation
includes a food co-op, \href{https://www.traderjoes.com/}{Trader Joe's}
and the regional chain \href{https://lundsandbyerlys.com/}{Lunds \&
Byerlys}.

``It has made me a better planner and more aware of what I like to buy
where,'' she said. ``I am so much more purposeful about where I choose
to shop.''

\hypertarget{2-online-aisles-are-bustling}{%
\subsection{2. Online Aisles Are
Bustling}\label{2-online-aisles-are-bustling}}

A year ago, 81 percent of shoppers
\href{https://news.gallup.com/poll/264857/online-grocery-shopping-rare.aspx}{surveyed
by Gallup}said they never turned to the internet for groceries. Online
shopping was lolling at around 3 percent of all grocery sales, or about
\$1.2 billion, according to
\href{https://www.supermarketnews.com/retail-financial/us-online-grocery-sales-growth-tails-june}{a
survey} by Brick Meets Click/Mercatus.

But in June, online grocery sales in the United States hit \$7.2
billion.

``Even my parents are getting increasingly used to using their iPad,''
said Mr. McMullen, 60, the Kroger chairman. ``There are millions of
people who have gotten used to cooking. They've found out they enjoyed
it, and they've gotten used to tech and are understanding the
benefits.''

The
\href{https://www.supermarketnews.com/online-retail/retailers-ramp-their-online-grocery-offerings-amid-continuing-pandemic/gallery?slide=13}{race
for their dollars is on}. In a challenge to Amazon Prime, Walmart
\href{https://www.detroitnews.com/story/business/2020/09/01/years-later-walmart-launches-answer-amazon-prime/113639782/}{last
week announced} a new \$98-a-year subscription service that offers
same-day delivery on 160,000 items.
\href{https://www.instacart.com/?click_id=EAIaIQobChMI3ayi5erP6wIVjMDICh0znwh3EAAYASAAEgIHuvD_BwE\&utm_medium=sem\&utm_source=instacart_google\&utm_campaign=ad_demand_search_brand_mkag_us-na-catchall_exact_us_LTV_Test\&utm_content=accountid-1732890876_campaignid-10842730770_adgroupid-104896959977_device-c\&utm_term=matchtype-e_keyword-instacart_targetid-kwd-297369219725_locationid-9060351\&kskwid=6106173\&ksadid=6107435\&gclid=EAIaIQobChMI3ayi5erP6wIVjMDICh0znwh3EAAYASAAEgIHuvD_BwE}{Instacart}
is more than doubling its work force, and new services like
\href{https://www.rosieapp.com/}{Rosie} are popping up.

Curbside pickup, delivery's sibling, has also exploded. Stores are
converting parking lots to better handle traffic from shoppers who drive
by to pick up orders. Companies including Kroger and
\href{https://www.wholefoodsmarket.com/stores?gclid=EAIaIQobChMIyoev9OrP6wIVYuW1Ch0v0Q-UEAAYASAAEgJILPD_BwE}{Whole
Foods Market} are opening what are becoming known as
``\href{https://www.supermarketnews.com/online-retail/whole-foods-opens-online-only-dark-store-brooklyn}{dark
stores},'' designed solely for picking up or delivering orders placed
online.

\href{https://www.localfarmfinder.com/}{Farmers have found} their way
onto the internet, too. Online orders are up more than 10 times over
last year for farms that use
\href{https://www.barn2door.com/}{Barn2Door,} an e-commerce site for
farmers, said James Maiocco,
\href{https://www.organicauthority.com/buzz-news/you-can-support-local-food-online-meet-barn2door-the-airbnb-of-food}{the
site's} chief operating officer.

Image

Kathy Moore, left, and Roxanne Wyss, center, are new converts to online
grocery shopping. Jessica Medina, an Instacart driver, dropped off an
order at Mrs. Wyss's home in Overland Park, Kan.Credit...Christopher
(KS) Smith for The New York Times

Roxanne Wyss and her work partner Kathy Moore, professional cooks in
their 60s who live about 25 miles apart in the Kansas City area, are two
unlikely converts to online food shopping. They met 38 years ago in the
test kitchen at the Rival Company, which invented the Crock-Pot, and
have been \href{https://www.pluggedintocooking.com/}{teaching and
writing cookbooks together} ever since.

With recipes for two cookbooks to test and no desire to risk infection,
they began to shop online in the spring. Neither dreamed that it would
be three and a half months until they stepped back into a supermarket.

They have found ways to work the angles online. Developing a texting
relationship with whoever picks out their groceries helps assure they
get the quality they expect. Some stores deliver more reliably than
others. Curbside pickup lets them avoid the extra costs that come with
delivery from services like Instacart.

Now they're back in the store, where they enjoy browsing for new
products and communing with other shoppers. And, of course, it's always
better to pick your own produce.

Still, they consider themselves permanent converts to online shopping.
``If there is a surge in the virus, we will return to ordering
everything online,'' Mrs. Moore said. ``And it will be wonderful to turn
to online when the weather is treacherous.''

Image

Oranges were one of the surprise breakouts this year, and sales remain
strong.Credit...Andrew Spear for The New York Times

\hypertarget{3-orange-is-the-new-snack}{%
\subsection{3. Orange Is the New
Snack}\label{3-orange-is-the-new-snack}}

Produce sales have been riding high since March, and are still up 11
percent from a year earlier, said Joe Watson, a vice president at the
\href{https://www.pma.com/}{Produce Marketing Association}. But one item
is a real outlier: oranges.

In May, grocers sold 73 percent more oranges than during the same month
in 2019. Even into July, sales remained 52 percent higher than a year
before.

``Oranges were a surprise, but they are popular from an immunity
standpoint,'' Mr. Watson said. They also last longer than some other
fruit, which matters when people are going to the store less often, he
said.

Sales in the category that grocers call ``natural products'' were
growing before the pandemic, but they blew up when it arrived. By
mid-March, they were up 78 percent over the year before, according to
\href{https://www.theshelbyreport.com/2020/06/09/natural-product-sales-up-significantly/}{the
market research firm IRI}.

``Consumers are very cognizant about doing what it takes to stay
healthy,'' said Shelley Balanko, a senior vice president at the Hartman
Group, a consumer research company. ``We think the trend is going to
stick around because people just really can't afford to get sick, on a
variety of levels.''

\hypertarget{4-redrawing-the-store}{%
\subsection{4. Redrawing the Store}\label{4-redrawing-the-store}}

Pandemic shopping has ushered in wider aisles, new methods of sanitation
and less-crowded stores. And shoppers want these changes to stay.

``It became clear to me pretty early on which stores were being
thoughtful and which were not,'' said Ms. Bowman, the Minneapolis
shopper, who spent almost 10 years working in the marketing department
of General Mills. ``I look at everything. I am a real nerd in the
grocery store, so store optics matter a lot to me.''

Image

Several grocery chains (including Kroger, here) have used the shift in
pandemic shopping habits to install more self-serve kiosks and explore
other touchless checkout methods.Credit...Andrew Spear for The New York
Times

Health concerns have also accelerated the growth in payment apps and
self-checkout. Walmart is testing
\href{https://chainstoreage.com/first-look-walmarts-new-self-checkout-store}{a
new system} that replaces traditional checkout lines with an open plaza
ringed by 34 terminals. Shoppers can scan their purchases, or wave down
an employee to do the scanning for them.

Kroger intends to double down on customer choice, offering an array of
options including self-checkout stations and an app that allows
consumers to scan and pay as they shop, as well as traditional cashiers.

``The infrastructure of the grocery store will continue to improve, and
service will continue to get better,'' said John Owen, the associate
director for food and retail at Mintel. ``And it's only a matter of time
before stores will be much bigger to accommodate the increase in
traffic.''

Still, some physical changes are fading.
\href{https://www.publix.com/}{Publix}, the 1,250-store chain based in
Florida,
\href{https://www.orlandosentinel.com/coronavirus/jobs-economy/os-cfb-coronavirus-publix-one-way-aisles-20200831-hlxyyrt4fnhbfoc4hv5hgkkiki-story.html}{recently
ended} its enforcement of one-way traffic in aisles, after customers
complained.

\hypertarget{5-choices-are-shrinking}{%
\subsection{5. Choices Are Shrinking}\label{5-choices-are-shrinking}}

After decades in which American supermarkets expanded to offer a
dizzying selection of products and brands, they are pulling back on
variety.

There are no more free samples (a health risk) and fewer specialty
promotions. Shoppers, intent on getting in and out quickly, are sticking
to items they already know. Online shoppers, guided by algorithms and
autofill, are less likely to make impulse purchases.

Grocers have found that they can still do a brisk business with fewer
choices. Displays at the end of aisles are more likely to hold bulk
packages of staples than new products looking to break into the market.
Instead of offering both conventional and organic leeks, for example, a
store may stock only the organic, Mr. Watson said. By reducing choices,
stores can more easily surf the ups and downs of the supply chain, which
are also limiting what's available.

Shoppers are being more economical. Retailers report more interest in
\href{https://www.foodnavigator-usa.com/Article/2020/09/04/Target-Good-Gather-private-label-line-to-nearly-double-number-of-SKUs?utm_source=newsletter_daily\&utm_medium=email\&utm_campaign=04-Sep-2020}{house
brands}. In a July study by the Food Industry Association, three in 10
shoppers said they were buying
\href{https://www.fmi.org/our-research/research-reports/u-s-grocery-shopper-trends}{more
store brands} than before the pandemic, a quirk that grocery analysts
say will likely become a habit, especially if the economy worsens.

Image

Beans, the darling of the early days of the pandemic, are still selling
well.Credit...Andrew Spear for The New York Times

Dried beans may be another economic indicator. They were
\href{https://www.nytimes3xbfgragh.onion/2020/03/22/business/coronavirus-beans-sales.html}{the
unexpected darling} in the early days of pandemic shopping, lifted by
the embrace of \href{https://www.ranchogordo.com/}{heirloom
varieties}and recipe-sharing on Instagram. Normally, sales drop in the
summer, but not this year.

``We are still seeing a surprisingly strong demand for dried beans,''
said Vince Hayward, a member of the third generation to lead the
\href{https://www.camelliabrand.com/about-camellia/}{Camellia brand},
whose red kidney beans are the staple of the New Orleans table. He likes
to think that demand is steady because people fell in love with beans,
but he realizes that economic insecurity could be driving sales.

``I feel like we've experienced the earthquake, and now the tsunami's on
the way,'' he said.

\hypertarget{6-the-freezer-is-hot}{%
\subsection{6. The Freezer Is Hot}\label{6-the-freezer-is-hot}}

Frozen food is another surprise breakout. Sales initially jumped by 94
percent in March from a year earlier, according to the
\href{https://affi.org/wp-content/uploads/2020/documents/Frozen\%20Food\%20Sales\%20Amid\%20COVID-19\%20Consumer\%20Research\%20-\%20FINAL.pdf}{American
Frozen Food Institute.} That initial rush abated, but even in August,
sales remained up almost 18 percent. Costco, whose
\href{https://www.supermarketnews.com/retail-financial/costco-reports-double-digit-sales-growth-august-q4?NL=SN-02\&Issue=SN-02_20200908_SN-02_551\&sfvc4enews=42\&cl=article_1\&utm_rid=CPG06000000192089\&utm_campaign=40944\&utm_medium=email\&elq2=f95a1058afc044959b8fedcc7da0433e}{sales
are up 15 percent} over August a year ago, attributes some of the growth
to strong frozen food sales.

Initially, shoppers were loading their freezers in what some in the
grocery business politely refer to as ``the initial pantry filling.''
For some consumers, frozen fruit and vegetables became a less expensive
and more reliable alternative to fresh. And then there was a simple
reality: Some days it is just easier to pull a meal from the freezer.

Once shoppers started exploring the freezer case, they found tastier new
options.

``Frozen had a lot of momentum coming into the pandemic,'' said Mr. Owen
from Mintel. ``A lot of the growth is coming from small brands that have
healthier, clean labels or vegetarian lines. People are discovering that
product quality and taste has improved.''

\hypertarget{7-local-is-a-bigger-lure}{%
\subsection{7. `Local' Is a Bigger
Lure}\label{7-local-is-a-bigger-lure}}

The fragility of the supply chain, concerns over health and safety and
an appreciation of community have buoyed the movement toward food that
is raised or produced locally.

Mrs. Moore and Mrs. Wyss both began ordering deliveries of eggs and milk
from a local dairy, and they split a quarter of beef. There are waiting
lists for community-supported agriculture subscriptions. Struggling
restaurants have turned into provisioners. Grocers are teaming up with
chefs to sell meal kits. Locally grown produce is selling out quickly.

It's all part of a greater awareness about healthy eating, food waste
and climate change, as well as a desire to keep money in the
neighborhood.

Image

The director and screenwriter Sean Gullette has developed a deeper
appreciation for the work of people like Ehab Jawad, whose family
co-owns the Foodtown in Prospect Heights, Brooklyn --- Mr. Gullette's
neighborhood store.Credit...Jose A. Alvarado Jr. for The New York Times

``I'll be damned if I'm buying a pear from Australia right now,'' said
Sean Gullette, 52, a
\href{https://www.imdb.com/name/nm0347797/}{filmmaker, writer and actor}
who feeds his family of four mostly from Foodtown, an
\href{https://www.foodtown.com/stores/foodtown-of-prospect-heights}{independently
run store} across the street from his home in Prospect Heights,
Brooklyn, that is part of a three-state grocery cooperative.

During the difficult, early days of the pandemic, Mr. Gullette watched
the store staff scramble to find creative ways to get staples like bread
on the shelves and deliver groceries to people who couldn't get to the
store.

He had already been friendly with the family that owns it, but now he
sees them in a new light.

``I love my Foodtown brothers,'' he said. ``You realize what a crucial
link of the chain these guys are. There are a bunch of people creating
this thing that we are deeply dependent on for something so intimate,
for what we put in our bodies. It has completely changed how I think
about grocery shopping.''

\emph{Follow} \href{https://twitter.com/nytfood}{\emph{NYT Food on
Twitter}} \emph{and}
\href{https://www.instagram.com/nytcooking/}{\emph{NYT Cooking on
Instagram}}\emph{,}
\href{https://www.facebookcorewwwi.onion/nytcooking/}{\emph{Facebook}}\emph{,}
\href{https://www.youtube.com/nytcooking}{\emph{YouTube}} \emph{and}
\href{https://www.pinterest.com/nytcooking/}{\emph{Pinterest}}\emph{.}
\href{https://www.nytimes3xbfgragh.onion/newsletters/cooking}{\emph{Get
regular updates from NYT Cooking, with recipe suggestions, cooking tips
and shopping advice}}\emph{.}

Advertisement

\protect\hyperlink{after-bottom}{Continue reading the main story}

\hypertarget{site-index}{%
\subsection{Site Index}\label{site-index}}

\hypertarget{site-information-navigation}{%
\subsection{Site Information
Navigation}\label{site-information-navigation}}

\begin{itemize}
\tightlist
\item
  \href{https://help.nytimes3xbfgragh.onion/hc/en-us/articles/115014792127-Copyright-notice}{©~2020~The
  New York Times Company}
\end{itemize}

\begin{itemize}
\tightlist
\item
  \href{https://www.nytco.com/}{NYTCo}
\item
  \href{https://help.nytimes3xbfgragh.onion/hc/en-us/articles/115015385887-Contact-Us}{Contact
  Us}
\item
  \href{https://www.nytco.com/careers/}{Work with us}
\item
  \href{https://nytmediakit.com/}{Advertise}
\item
  \href{http://www.tbrandstudio.com/}{T Brand Studio}
\item
  \href{https://www.nytimes3xbfgragh.onion/privacy/cookie-policy\#how-do-i-manage-trackers}{Your
  Ad Choices}
\item
  \href{https://www.nytimes3xbfgragh.onion/privacy}{Privacy}
\item
  \href{https://help.nytimes3xbfgragh.onion/hc/en-us/articles/115014893428-Terms-of-service}{Terms
  of Service}
\item
  \href{https://help.nytimes3xbfgragh.onion/hc/en-us/articles/115014893968-Terms-of-sale}{Terms
  of Sale}
\item
  \href{https://spiderbites.nytimes3xbfgragh.onion}{Site Map}
\item
  \href{https://help.nytimes3xbfgragh.onion/hc/en-us}{Help}
\item
  \href{https://www.nytimes3xbfgragh.onion/subscription?campaignId=37WXW}{Subscriptions}
\end{itemize}
