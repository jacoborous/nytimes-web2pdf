Sections

SEARCH

\protect\hyperlink{site-content}{Skip to
content}\protect\hyperlink{site-index}{Skip to site index}

\href{https://www.nytimes3xbfgragh.onion/section/movies}{Movies}

\href{https://myaccount.nytimes3xbfgragh.onion/auth/login?response_type=cookie\&client_id=vi}{}

\href{https://www.nytimes3xbfgragh.onion/section/todayspaper}{Today's
Paper}

\href{/section/movies}{Movies}\textbar{}Academy Explains Diversity Rules
for Best Picture Oscar

\url{https://nyti.ms/2ZeSSW3}

\begin{itemize}
\item
\item
\item
\item
\item
\end{itemize}

Advertisement

\protect\hyperlink{after-top}{Continue reading the main story}

Supported by

\protect\hyperlink{after-sponsor}{Continue reading the main story}

\hypertarget{academy-explains-diversity-rules-for-best-picture-oscar}{%
\section{Academy Explains Diversity Rules for Best Picture
Oscar}\label{academy-explains-diversity-rules-for-best-picture-oscar}}

Beginning in 2022, films aiming for the top award will have to turn in a
confidential ``inclusion standard form.'' In 2024, they will have to
meet two of four standards.

\includegraphics{https://static01.graylady3jvrrxbe.onion/images/2020/09/08/arts/08academy1/merlin_176740947_32814034-742a-4de4-a0e0-6a8cb4ca453f-articleLarge.jpg?quality=75\&auto=webp\&disable=upscale}

By \href{https://www.nytimes3xbfgragh.onion/by/nicole-sperling}{Nicole
Sperling}

\begin{itemize}
\item
  Sept. 8, 2020
\item
  \begin{itemize}
  \item
  \item
  \item
  \item
  \item
  \end{itemize}
\end{itemize}

In June, the
\href{https://www.nytimes3xbfgragh.onion/2020/06/12/movies/oscars-diversity-rule.html}{Academy
of Motion Picture Arts and Sciences}, which oversees the Oscars, said it
would add a diversity component to the Oscar race. On Tuesday, it
explained how it's going to work.

Beginning in 2024 with the 96th Oscars, films hoping to qualify for the
best picture category will have to meet inclusion standards both on
camera and behind the scenes.

To meet the onscreen representation standard, at least one of the lead
actors or a significant supporting actor must be from an
underrepresented racial or ethnic group, whether that means Asian,
Hispanic, Black, Indigenous, Native American, Middle Eastern, North
African, native Hawaiian or other Pacific Islander.

There are alternatives: Thirty percent of all actors in secondary or
more minor roles could come from two of the following categories: women,
L.G.B.T.Q., an underrepresented racial or ethnic group, or those with
cognitive or physical disabilities. Or the main story line must focus on
an underrepresented group.

The move is part of a continuing effort to improve inclusion both within
the organization and in the movies it honors. Over the years, the
academy has come under fire for presenting all-white acting slates at
nomination time, a fault many attribute to both the homogeneous makeup
of the organization and the industry at large. These standards are meant
to address the broader industry issues.

``The aperture must widen to reflect our diverse global population in
both the creation of motion pictures and in the audiences who connect
with them,'' the academy's president, David Rubin, and chief executive,
Dawn Hudson, said in a statement, adding that the pending standards will
``be a catalyst for long-lasting, essential change in our industry.''

Beginning in 2022, for the 94th Oscars and again in 2023 for the 95th
Oscars, each best picture candidate must first submit a confidential
academy inclusion standard form to be considered eligible --- a baby
step, if you will, to get the industry thinking more about inclusion.

Then in 2024, to qualify, a film must meet two of four standards in
areas of onscreen representation, offscreen creative leadership,
apprenticeship opportunities for members of underrepresented groups and
diversity in the ranks of the marketing and distribution departments.

Other standards involve filling the ranks behind the scenes with women
or people of color; offering both paid apprenticeships and training
opportunities to those in underrepresented groups; or hiring multiple
senior executives from those groups at either the studio or the film
company charged with marketing and distributing the films.

The standards will be enforced via spot checks of sets and through
dialogue between the academy and a movie's filmmakers and distributors.

Two academy governors --- the producer
\href{https://devonfranklin.com/}{DeVon Franklin} and Paramount
Pictures' chairman and chief executive, Jim Gianopulos --- headed up the
task force to develop the standards. They took their inspiration from
the
\href{https://www.bfi.org.uk/inclusion-film-industry/bfi-diversity-standards}{British
Film Institute}, which in 2019 became the first major awards body to
introduce diversity and inclusion criteria as part of its eligibility
requirements.

In recent years, the academy has made efforts to diversify its
membership in response to
\href{https://www.nytimes3xbfgragh.onion/2020/02/06/movies/oscarssowhite-history.html}{\#OscarsSoWhite},
a hashtag that emerged after the organization did not nominate any
actors of color for Academy Awards two years in a row. Academy leaders
vowed to double the number of people of color and women members by 2020,
a milestone they hit this summer
\href{https://www.nytimes3xbfgragh.onion/2020/06/30/movies/academy-oscars-new-members.html}{when
they invited} 819 new members including the actresses Awkwafina and
Zendaya to join. The organization is still predominantly white (81
percent) and male (67 percent).

In June, it announced that it would delay the 93rd Academy Awards to
April 25.

Advertisement

\protect\hyperlink{after-bottom}{Continue reading the main story}

\hypertarget{site-index}{%
\subsection{Site Index}\label{site-index}}

\hypertarget{site-information-navigation}{%
\subsection{Site Information
Navigation}\label{site-information-navigation}}

\begin{itemize}
\tightlist
\item
  \href{https://help.nytimes3xbfgragh.onion/hc/en-us/articles/115014792127-Copyright-notice}{©~2020~The
  New York Times Company}
\end{itemize}

\begin{itemize}
\tightlist
\item
  \href{https://www.nytco.com/}{NYTCo}
\item
  \href{https://help.nytimes3xbfgragh.onion/hc/en-us/articles/115015385887-Contact-Us}{Contact
  Us}
\item
  \href{https://www.nytco.com/careers/}{Work with us}
\item
  \href{https://nytmediakit.com/}{Advertise}
\item
  \href{http://www.tbrandstudio.com/}{T Brand Studio}
\item
  \href{https://www.nytimes3xbfgragh.onion/privacy/cookie-policy\#how-do-i-manage-trackers}{Your
  Ad Choices}
\item
  \href{https://www.nytimes3xbfgragh.onion/privacy}{Privacy}
\item
  \href{https://help.nytimes3xbfgragh.onion/hc/en-us/articles/115014893428-Terms-of-service}{Terms
  of Service}
\item
  \href{https://help.nytimes3xbfgragh.onion/hc/en-us/articles/115014893968-Terms-of-sale}{Terms
  of Sale}
\item
  \href{https://spiderbites.nytimes3xbfgragh.onion}{Site Map}
\item
  \href{https://help.nytimes3xbfgragh.onion/hc/en-us}{Help}
\item
  \href{https://www.nytimes3xbfgragh.onion/subscription?campaignId=37WXW}{Subscriptions}
\end{itemize}
