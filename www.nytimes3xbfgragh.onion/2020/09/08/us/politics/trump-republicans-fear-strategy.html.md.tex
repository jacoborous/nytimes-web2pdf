Sections

SEARCH

\protect\hyperlink{site-content}{Skip to
content}\protect\hyperlink{site-index}{Skip to site index}

\href{https://www.nytimes3xbfgragh.onion/section/politics}{Politics}

\href{https://myaccount.nytimes3xbfgragh.onion/auth/login?response_type=cookie\&client_id=vi}{}

\href{https://www.nytimes3xbfgragh.onion/section/todayspaper}{Today's
Paper}

\href{/section/politics}{Politics}\textbar{}Republicans Revive 2018
Strategy, Hoping for Better Result: Scare Voters

\url{https://nyti.ms/2R8Wa8R}

\begin{itemize}
\item
\item
\item
\item
\item
\item
\end{itemize}

\begin{itemize}
\item
  \href{https://www.nytimes3xbfgragh.onion/interactive/2020/09/08/us/elections/results-new-hampshire-primary-elections.html?action=click\&pgtype=Article\&state=default\&region=TOP_BANNER\&context=storylines_menu}{New
  Hampshire Results}
\item
  \href{https://www.nytimes3xbfgragh.onion/live/2020/09/08/us/trump-vs-biden?action=click\&pgtype=Article\&state=default\&region=TOP_BANNER\&context=storylines_menu}{Election
  Updates}
\item
  \href{https://www.nytimes3xbfgragh.onion/interactive/2020/us/elections/election-states-biden-trump.html?action=click\&pgtype=Article\&state=default\&region=TOP_BANNER\&context=storylines_menu}{Paths
  to 270}
\item
  \href{https://www.nytimes3xbfgragh.onion/interactive/2020/08/31/us/politics/vote-by-mail-deadlines.html?action=click\&pgtype=Article\&state=default\&region=TOP_BANNER\&context=storylines_menu}{Voting
  by Mail}
\item
  \href{https://www.nytimes3xbfgragh.onion/interactive/2019/us/elections/2020-presidential-election-calendar.html?action=click\&pgtype=Article\&state=default\&region=TOP_BANNER\&context=storylines_menu}{Key
  Dates}
\item
  \href{https://www.nytimes3xbfgragh.onion/newsletters/politics?action=click\&pgtype=Article\&state=default\&region=TOP_BANNER\&context=storylines_menu}{Politics
  Newsletter}
\end{itemize}

Advertisement

\protect\hyperlink{after-top}{Continue reading the main story}

Supported by

\protect\hyperlink{after-sponsor}{Continue reading the main story}

\hypertarget{republicans-revive-2018-strategy-hoping-for-better-result-scare-voters}{%
\section{Republicans Revive 2018 Strategy, Hoping for Better Result:
Scare
Voters}\label{republicans-revive-2018-strategy-hoping-for-better-result-scare-voters}}

President Trump and his party are using a playbook that aims to alarm
people about crime in their backyards. It didn't work in 2018, but both
parties think it could resonate more this year.

\includegraphics{https://static01.graylady3jvrrxbe.onion/images/2020/09/09/us/politics/09fear1/04fear1-articleLarge.jpg?quality=75\&auto=webp\&disable=upscale}

\href{https://www.nytimes3xbfgragh.onion/by/jeremy-w-peters}{\includegraphics{https://static01.graylady3jvrrxbe.onion/images/2018/11/06/multimedia/author-jeremy-w-peters/author-jeremy-w-peters-thumbLarge.png}}

By \href{https://www.nytimes3xbfgragh.onion/by/jeremy-w-peters}{Jeremy
W. Peters}

\begin{itemize}
\item
  Sept. 8, 2020
\item
  \begin{itemize}
  \item
  \item
  \item
  \item
  \item
  \item
  \end{itemize}
\end{itemize}

By the time Republicans were done with Sharice Davids in 2018, she
barely recognized herself. In ads that blanketed her suburban Kansas
City district during her congressional race, she was portrayed as ``the
candidate of the liberal mob,'' an enemy of the police, a threat to
children, and an ally of ``radical left-wing protesters.''

As flabbergasted as she was by the strategy then, she said she was
surprised Republicans were at it again, only this time in the
presidential election.

``It didn't work last time,'' said Ms. Davids, who
\href{https://www.nytimes3xbfgragh.onion/elections/results/kansas-house-district-3}{won}
her race by 10 points and is favored to be re-elected to a second term
in November. As a former mixed martial arts fighter who learned the
importance of developing new techniques in combat, she said her
opponents' attacks seem stale. ``I haven't seen any evolution. The skill
set looks the same.''

\href{https://www.nytimes3xbfgragh.onion/interactive/2020/us/elections/donald-trump.html}{President
Trump} is using a fear-based playbook that is as familiar to him as it
is questionable in actually helping Republicans get elected in recent
years. Some of the players have changed --- instead of MS-13 gang
members and migrant caravans, now there are rioters and looters --- but
the target audience and themes are the same: suburban communities that
he claims Democrats won't keep safe. The president is even reusing
phrases and imagery from 2018, with slogans like ``jobs not mobs'' and
ads showing Democratic politicians and liberal figures kneeling during
the national anthem.

Democrats can point to the 41 House seats they picked up in 2018 to show
that the Republican strategy did not work then, and that voters were
more concerned about health care than havoc. Even Republicans say there
is no solid evidence in their polling that proves the president's
tactics are helping him today.

But behind their confidence, Democrats acknowledge a real risk that Mr.
Trump and Republicans could benefit by casting former Vice President
\href{https://www.nytimes3xbfgragh.onion/interactive/2020/us/elections/joe-biden.html}{Joseph
R. Biden Jr.} and the Democratic Party as indifferent to the violence
and unrest that has shaken cities across the country, especially in the
Midwestern suburbs in Wisconsin and Minnesota where it is not so distant
and abstract.

And as the president made clear during a news conference on Monday, he
is trying to blame his opponents for far more than that, with unfounded
claims that ``radical socialist Democrats'' would ``immediately collapse
the economy'' and cause ``countless deaths from suicide, substance
abuse, depression, heart disease'' by keeping coronavirus lockdowns in
place.

Fear --- over crime, street violence, funding cuts to police departments
or economic security --- resonates with suburban voters of all
demographics, said Dan Sena, a Democratic strategist who oversaw the
Democratic Congressional Campaign Committee's strategy to win the House
in 2018. ``Democrats need to be in a place where they acknowledge the
fear that comes with that and put the issue to bed. Then people are
willing to listen to you,'' Mr. Sena said. ``But if you ignore the
problem, then you open the door for the Republican strategy to work.''

Republicans point to Minnesota's First Congressional District as an
example of how the playbook can work. The Democratic candidate in 2018,
Dan Feehan, narrowly lost after a scorching ad campaign. Ads from
Republicans and their outside groups depicted him alongside images of a
kneeling Colin Kaepernick and grimacing Hispanic men covered in tattoos,
meant to evoke stereotypes of hardened criminals.

\includegraphics{https://static01.graylady3jvrrxbe.onion/images/2020/09/09/us/politics/09fear2/merlin_147794934_acf00e4a-5af1-4ab5-8496-c9f3f4322e0c-articleLarge.jpg?quality=75\&auto=webp\&disable=upscale}

When Mr. Trump stopped to campaign in the district last month --- he
lost Minnesota in 2016 by only 44,000 votes and sees it within reach
again this year --- he claimed that Mr. Biden would ``abolish the
suburbs'' and ``pass legislation gutting every single police department
in America.''

Nick Frentz, a Democratic state senator who represents Mankato, Minn.,
the city about 80 miles south of Minneapolis where Mr. Trump spoke, said
he was proud to list endorsements from local law enforcement agencies at
the top of his campaign materials in his last election. Mr. Frentz said
that Republicans cannot be allowed to conflate the movement to raise
awareness about racial inequalities in policing with the anarchists who
have exploited it.

``What I'm encouraging people to do is get out in front of it,'' Mr.
Frentz said of his fellow Democrats and the issue of public safety,
which is especially prominent in his district given its closeness to the
city that became the epicenter for demonstrations against racial
injustice this summer after a
\href{https://www.nytimes3xbfgragh.onion/2020/07/18/us/derek-chauvin-george-floyd.html}{Minneapolis
police officer} was seen on video kneeling on the neck of George Floyd
until he died.

It has also erupted as a point of division in the fight for control of
the State Senate this year, which Republicans control by just two votes.
Mr. Frentz pointed to
\href{https://www.minnpost.com/greater-minnesota/2020/09/minnesota-republicans-are-deploying-their-own-law-and-order-strategy-in-legislative-races/}{campaign
mail} from Republicans that claims Democrats want to ``defund the police
and destroy the rule of law'' next to a picture of a building in flames.
``That's false,'' he said. ``We have 110 Democrats in the state
legislature --- not a one calling to defund police departments.''

The suburbs helped propel Democratic gains in 2018, suggesting that if
voters were motivated by fear, it wasn't gangs and migrant invasions
they were worried about. Concerns about crime then and now are real, Mr.
Sena said, ``But not any scarier than the idea of someone taking away
their health care.''

Republicans believe otherwise. The looting and property destruction that
damaged
\href{https://www.startribune.com/minneapolis-st-paul-buildings-are-damaged-looted-after-george-floyd-protests-riots/569930671/}{more
than 1,500} buildings and businesses across the Twin Cities and spread
to other major metropolitan areas provided instant grist for Mr. Trump,
Republicans and conservative media. Ignoring the fact that these bursts
of violence were relatively isolated amid the mass demonstrations that
drew millions of Americans from Whitefish, Mont., to Miami --- and aware
that many Democrats were loath to be seen as critical of the broader
movement for racial justice --- the president and his allies focused on
the unrest often to the exclusion of anything else.

And then they accused Democrats of failing to condemn the unrest.

Republican strategists said their research showed a vulnerability for
Mr. Biden with swing voters: the belief, advanced frequently and
misleadingly by Mr. Trump, that Mr. Biden is overly susceptible to
influence from the far left.

What followed was some of the bleakest and dystopian messaging of the
campaign so far, with Trump campaign ads featuring fictitious unanswered
calls to 911 because of cuts to public safety. An ad that went out in
July \href{https://www.youtube.com/watch?v=moZOrq0qL3Q}{shows a man
prying open the door} to an elderly woman's home with a crowbar. When
she reaches 911, a recording tells her, ``Leave a message and we'll get
back to you as soon as we can.'' The man approaches, and the camera cuts
away to a shot of the phone lying on the ground.

Corry Bliss, a Republican strategist who oversaw the political group
that made many of the ads targeting Democrats as soft on crime in 2018
and is consulting on G.O.P. races this year, said the strategy is
clearly hitting a nerve this time --- with voters and Mr. Biden.

``The most telling thing is not what the Republican Party is doing, but
what Joe Biden and the Democrats are doing. And they are clearly scared
to death that they're out of step with the American people,'' Mr. Bliss
said.

Image

Joseph R. Biden Jr. condemned rioting and looting during a speech in
Pittsburgh on Monday.Credit...Amr Alfiky/The New York Times

Since the phrase ``defund the police'' first became a mantra among some
progressive activists, Mr. Biden
\href{https://www.nytimes3xbfgragh.onion/2020/06/08/us/politics/biden-defund-the-police.html}{has
distanced himself from it}, saying that while he supports the need for
overhauling policing practices, he opposes across-the-board cuts to law
enforcement budgets. So do
\href{https://www.nytimes3xbfgragh.onion/2020/06/26/us/politics/defund-police-protests-democrats.html}{most
voters} and Democratic Party leaders. But the pressure on the Biden
campaign to explain in a more detailed way how Mr. Biden would approach
public safety as president, and specifically what he thought about the
\href{https://www.nytimes3xbfgragh.onion/2020/08/24/us/kenosha-police-shooting.html}{riots
that followed the police shooting} late last month in Kenosha, Wis.,
that left a Black man partially paralyzed, was evident last week.

Mr. Biden gave a
\href{https://www.nytimes3xbfgragh.onion/2020/08/31/us/politics/biden-speech-trump.html}{speech
on the issue}, which his campaign quickly turned into an ad. ``I want to
make it absolutely clear,''
\href{https://www.youtube.com/watch?v=LgHXJ3rdOn0\&feature=youtu.be}{he
says in the ad}, an apparent acknowledgment that there are people who
believe he has not been. ``Rioting is not protesting. Looting is not
protesting. It's lawlessness, plain and simple. And those who do it
should be prosecuted.''

In Minnesota, the Trump campaign started running a new ad that opens
with a jarring juxtaposition. ``Lawless criminals terrorized
Minneapolis. Joe Biden takes a knee,'' the announcer says as an image of
Mr. Biden kneeling during a meeting at a Black church appears,
superimposed over video of a burning building.

``It started here, sadly,'' said Jason Lewis, a candidate for United
States Senate who is challenging the Democratic incumbent, Senator Tina
Smith. ``But the Democrats have overreached. And because of their
initial refusal to condemn the riots and to stand down and to not do
something about public safety, they own this.''

Mr. Lewis's campaign conducted a poll last week showing him within two
points of his opponent and Mr. Trump down three points.

But Mr. Trump's gravitational pull is everything in Republican politics,
as Mr. Lewis acknowledged. In 2018, he lost his suburban Minneapolis
House seat to a Democrat. ``With this guy, you're in for a penny, you're
in for a pound,'' he said. And as much as some voters might be appalled
at the street violence, Mr. Trump is taking the wrong approach to
winning them over, some Republicans said.

``He actually is talking about issues that people care about, but he's
using language that makes it less effective,'' said Frank Luntz, a
veteran adviser to Republican campaigns who has been critical of the
president.

``The problem with `law and order,' if you ask voters they will tell you
they think of cops hitting protesters over the head, and nobody wants
that. Trump is using the language of 1968, and it's 2020,'' Mr. Luntz
said.

With more Democrats speaking the language of law enforcement,
Republicans may find their approach even less effective. From Kansas,
Ms. Davids spoke of being raised by a mother who served in the Army and
then worked for a time in law enforcement. She said she has attended
rallies for racial justice and sat down with police leaders. Somewhere
in between, she said, is where most Americans are.

``I do think that there are a lot of people who really want us to have
the conversation about racial justice and who really want us to be
thoughtful about how we allocate our resources to police,'' she said.

\hypertarget{our-2020-election-guide}{%
\section{Our 2020 Election Guide}\label{our-2020-election-guide}}

Updated ~Sept. 8, 2020

\begin{center}\rule{0.5\linewidth}{\linethickness}\end{center}

\begin{itemize}
\item ~
  \hypertarget{the-latest}{%
  \subsection{The Latest}\label{the-latest}}

  \begin{itemize}
  \item
    President Trump and his party are using a playbook that aims to
    alarm people about crime in their backyards. It didn't work in 2018,
    but
    \href{https://www.nytimes3xbfgragh.onion/2020/09/08/us/politics/trump-republicans-fear-strategy.html?action=click\&pgtype=Article\&state=default\&region=BELOW_MAIN_CONTENT\&context=storylines_guide}{both
    parties think it could resonate more this year}.
  \end{itemize}
\item ~
  \hypertarget{how-to-win-270}{%
  \subsection{How to Win 270}\label{how-to-win-270}}

  \begin{itemize}
  \item
    Joe Biden and Donald Trump need 270 electoral votes to reach the
    White House. Try building
    \href{https://www.nytimes3xbfgragh.onion/interactive/2020/us/elections/election-states-biden-trump.html?action=click\&pgtype=Article\&state=default\&region=BELOW_MAIN_CONTENT\&context=storylines_guide}{your
    own coalition of battleground states}~to see potential outcomes.
  \end{itemize}
\item ~
  \hypertarget{voting-by-mail}{%
  \subsection{Voting by Mail}\label{voting-by-mail}}

  \begin{itemize}
  \item
    Will you have enough time to vote by mail in your state? Yes, but
    it's risky to procrastinate.
    \href{https://www.nytimes3xbfgragh.onion/interactive/2020/08/31/us/politics/vote-by-mail-deadlines.html?action=click\&pgtype=Article\&state=default\&region=BELOW_MAIN_CONTENT\&context=storylines_guide}{Check
    your state's deadline.}
  \item
    \href{https://www.nytimes3xbfgragh.onion/interactive/2020/us/elections/joe-biden.html?action=click\&pgtype=Article\&state=default\&region=BELOW_MAIN_CONTENT\&context=storylines_guide}{}

    \hypertarget{joe-biden}{%
    \section{Joe Biden}\label{joe-biden}}

    \hypertarget{democrat}{%
    \subsection{Democrat}\label{democrat}}

    \href{https://www.nytimes3xbfgragh.onion/interactive/2020/us/elections/donald-trump.html?action=click\&pgtype=Article\&state=default\&region=BELOW_MAIN_CONTENT\&context=storylines_guide}{}

    \hypertarget{donald-trump}{%
    \section{Donald Trump}\label{donald-trump}}

    \hypertarget{republican}{%
    \subsection{Republican}\label{republican}}
  \end{itemize}
\item
  \hypertarget{keep-up-with-our-coverage}{%
  \subsection{Keep Up With Our
  Coverage}\label{keep-up-with-our-coverage}}

  \begin{itemize}
  \item
    Get an
    \href{https://www.nytimes3xbfgragh.onion/newsletters/politics?action=click\&pgtype=Article\&state=default\&region=BELOW_MAIN_CONTENT\&context=storylines_guide}{email}~recapping
    the day's news
  \item
    Download our mobile app on
    \href{https://apps.apple.com/us/app/nytimes/id284862083?ls=1\&mat_click_id=5c79ae7455014fd1bd66b5610c05b8f2-20191112-16948\&referrer=mat_click_id\%3D5c79ae7455014fd1bd66b5610c05b8f2-20191112-16948\%26link_click_id\%3D722930677036718082}{iOS}~and
    \href{http://a.localytics.com/android?id=com.nytimes.android\&referrer=utm_source\%3Dother_nyt_mobile_web\%26utm_medium\%3DWeb\%2520page\%26utm_term\%3DGeneral\%2520Mobile\%2520Page\%26utm_campaign\%3DNYT\%2520Mobile\%2520General\%2520Page}{Android}~and
    turn on Breaking News and Politics alerts
  \end{itemize}
\end{itemize}

Advertisement

\protect\hyperlink{after-bottom}{Continue reading the main story}

\hypertarget{site-index}{%
\subsection{Site Index}\label{site-index}}

\hypertarget{site-information-navigation}{%
\subsection{Site Information
Navigation}\label{site-information-navigation}}

\begin{itemize}
\tightlist
\item
  \href{https://help.nytimes3xbfgragh.onion/hc/en-us/articles/115014792127-Copyright-notice}{©~2020~The
  New York Times Company}
\end{itemize}

\begin{itemize}
\tightlist
\item
  \href{https://www.nytco.com/}{NYTCo}
\item
  \href{https://help.nytimes3xbfgragh.onion/hc/en-us/articles/115015385887-Contact-Us}{Contact
  Us}
\item
  \href{https://www.nytco.com/careers/}{Work with us}
\item
  \href{https://nytmediakit.com/}{Advertise}
\item
  \href{http://www.tbrandstudio.com/}{T Brand Studio}
\item
  \href{https://www.nytimes3xbfgragh.onion/privacy/cookie-policy\#how-do-i-manage-trackers}{Your
  Ad Choices}
\item
  \href{https://www.nytimes3xbfgragh.onion/privacy}{Privacy}
\item
  \href{https://help.nytimes3xbfgragh.onion/hc/en-us/articles/115014893428-Terms-of-service}{Terms
  of Service}
\item
  \href{https://help.nytimes3xbfgragh.onion/hc/en-us/articles/115014893968-Terms-of-sale}{Terms
  of Sale}
\item
  \href{https://spiderbites.nytimes3xbfgragh.onion}{Site Map}
\item
  \href{https://help.nytimes3xbfgragh.onion/hc/en-us}{Help}
\item
  \href{https://www.nytimes3xbfgragh.onion/subscription?campaignId=37WXW}{Subscriptions}
\end{itemize}
