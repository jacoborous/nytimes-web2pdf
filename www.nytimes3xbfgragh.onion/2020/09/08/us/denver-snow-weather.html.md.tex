Sections

SEARCH

\protect\hyperlink{site-content}{Skip to
content}\protect\hyperlink{site-index}{Skip to site index}

\href{https://www.nytimes3xbfgragh.onion/section/us}{U.S.}

\href{https://myaccount.nytimes3xbfgragh.onion/auth/login?response_type=cookie\&client_id=vi}{}

\href{https://www.nytimes3xbfgragh.onion/section/todayspaper}{Today's
Paper}

\href{/section/us}{U.S.}\textbar{}Crazy Weather in Colorado: Roasting
Yesterday, Snowing Today

\url{https://nyti.ms/3haJrwZ}

\begin{itemize}
\item
\item
\item
\item
\item
\end{itemize}

Advertisement

\protect\hyperlink{after-top}{Continue reading the main story}

Supported by

\protect\hyperlink{after-sponsor}{Continue reading the main story}

\hypertarget{crazy-weather-in-colorado-roasting-yesterday-snowing-today}{%
\section{Crazy Weather in Colorado: Roasting Yesterday, Snowing
Today}\label{crazy-weather-in-colorado-roasting-yesterday-snowing-today}}

Temperatures plunged more than 50 degrees overnight as an Arctic air
mass drove into the Denver area, bringing a very early winter storm on
Tuesday.

\includegraphics{https://static01.graylady3jvrrxbe.onion/images/2020/09/08/us/08DENVER/08DENVER-videoSixteenByNine3000.jpg}

By \href{https://www.nytimes3xbfgragh.onion/by/jack-healy}{Jack Healy}

\begin{itemize}
\item
  Sept. 8, 2020
\item
  \begin{itemize}
  \item
  \item
  \item
  \item
  \item
  \end{itemize}
\end{itemize}

BOULDER, Colo. --- Wait, what happened to fall?

On Monday, the region around Denver was sweltering in summer heat, the
scorched skies thick with haze, smoke and ash from a wildfire roaring
through the dried-out forests near Rocky Mountain National Park. By
Tuesday morning, there was snow on the ground, and temperatures had
plunged more than 50 degrees Fahrenheit.

``We switched from summer to winter in a day,'' said David Barjenbruch,
a senior forecaster at the National Weather Service in Boulder.

Outside his office, an inch or so of snow was already sticking to
hillsides and tree branches on Tuesday morning, the start of what was
expected to be a daylong snowstorm, dropping more than 12 inches in the
foothills and mountains and three to six inches around Denver.

Mr. Barjenbruch said the cold air mass had rolled in from above the
Arctic Circle, traveling rapidly south along the spine of the Rocky
Mountains. Some of Colorado's ski resorts, which have been preparing for
a socially distanced ski season, were expected to get an early dump,
though probably not enough to last until they open around Thanksgiving.
Live cameras showed that mountain passes were already a blur of white.

Across Denver, people were hauling potted herbs and flowers indoors and
wrapping their bushes in burlap and plastic. Mr. Barjenbruch said one of
the biggest threats posed by the storm was that overloaded tree
branches, still leafed out for summer, could snap and tumble onto power
lines.

Forecasters and fire crews were hoping that the snow might damp the
Cameron Peak wildfire in Northern Colorado, a blaze that has exploded to
more than 102,000 acres and forced a round of evacuations on Monday.

Sheriff Justin Smith of Larimer County said that while the stark respite
from a record-setting string of 90-degree days and punishing drought
across Colorado was welcome, it was ``certainly not going to stop this
fire,'' The Colorado Sun
\href{https://coloradosun.com/2020/09/07/cameron-peak-fire-update-structures/}{reported}.

It remains to be seen whether conditions will bounce back to hot, windy
and dry after the storm retreats to the north and northwest during the
week. Skies are expected to clear in Denver by Thursday, with
temperatures climbing back into the more seasonable 70s and 80s for the
weekend.

Still, Mr. Barjenbruch said, the snow that had already begun to fall on
the fire Tuesday was good news.

``It's going to hang on trees and give the fire no fuel to burn, and
give firefighters a chance to catch up,'' he said. ``This is the best
thing that could've happened for this fire.''

Advertisement

\protect\hyperlink{after-bottom}{Continue reading the main story}

\hypertarget{site-index}{%
\subsection{Site Index}\label{site-index}}

\hypertarget{site-information-navigation}{%
\subsection{Site Information
Navigation}\label{site-information-navigation}}

\begin{itemize}
\tightlist
\item
  \href{https://help.nytimes3xbfgragh.onion/hc/en-us/articles/115014792127-Copyright-notice}{©~2020~The
  New York Times Company}
\end{itemize}

\begin{itemize}
\tightlist
\item
  \href{https://www.nytco.com/}{NYTCo}
\item
  \href{https://help.nytimes3xbfgragh.onion/hc/en-us/articles/115015385887-Contact-Us}{Contact
  Us}
\item
  \href{https://www.nytco.com/careers/}{Work with us}
\item
  \href{https://nytmediakit.com/}{Advertise}
\item
  \href{http://www.tbrandstudio.com/}{T Brand Studio}
\item
  \href{https://www.nytimes3xbfgragh.onion/privacy/cookie-policy\#how-do-i-manage-trackers}{Your
  Ad Choices}
\item
  \href{https://www.nytimes3xbfgragh.onion/privacy}{Privacy}
\item
  \href{https://help.nytimes3xbfgragh.onion/hc/en-us/articles/115014893428-Terms-of-service}{Terms
  of Service}
\item
  \href{https://help.nytimes3xbfgragh.onion/hc/en-us/articles/115014893968-Terms-of-sale}{Terms
  of Sale}
\item
  \href{https://spiderbites.nytimes3xbfgragh.onion}{Site Map}
\item
  \href{https://help.nytimes3xbfgragh.onion/hc/en-us}{Help}
\item
  \href{https://www.nytimes3xbfgragh.onion/subscription?campaignId=37WXW}{Subscriptions}
\end{itemize}
