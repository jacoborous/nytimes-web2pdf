Sections

SEARCH

\protect\hyperlink{site-content}{Skip to
content}\protect\hyperlink{site-index}{Skip to site index}

\href{https://www.nytimes3xbfgragh.onion/section/us}{U.S.}

\href{https://myaccount.nytimes3xbfgragh.onion/auth/login?response_type=cookie\&client_id=vi}{}

\href{https://www.nytimes3xbfgragh.onion/section/todayspaper}{Today's
Paper}

\href{/section/us}{U.S.}\textbar{}A Police Officer Shot a 13-Year-Old
With Autism in Salt Lake City

\url{https://nyti.ms/2GyvhZN}

\begin{itemize}
\item
\item
\item
\item
\item
\end{itemize}

\hypertarget{race-and-america}{%
\subsubsection{\texorpdfstring{\href{https://www.nytimes3xbfgragh.onion/news-event/george-floyd-protests-minneapolis-new-york-los-angeles?name=styln-george-floyd\&region=TOP_BANNER\&block=storyline_menu_recirc\&action=click\&pgtype=Article\&impression_id=ac5c10d0-f27d-11ea-84fb-6965c80d8485\&variant=undefined}{Race
and America}}{Race and America}}\label{race-and-america}}

\begin{itemize}
\tightlist
\item
  \href{https://www.nytimes3xbfgragh.onion/2020/09/04/nyregion/rochester-police-daniel-prude.html?name=styln-george-floyd\&region=TOP_BANNER\&block=storyline_menu_recirc\&action=click\&pgtype=Article\&impression_id=ac5c37e0-f27d-11ea-84fb-6965c80d8485\&variant=undefined}{What
  Happened in Rochester, N.Y.}
\item
  \href{https://www.nytimes3xbfgragh.onion/2020/09/01/us/politics/trump-fact-check-protests.html?name=styln-george-floyd\&region=TOP_BANNER\&block=storyline_menu_recirc\&action=click\&pgtype=Article\&impression_id=ac5c37e1-f27d-11ea-84fb-6965c80d8485\&variant=undefined}{Trump
  Fact Check}
\item
  \href{https://www.nytimes3xbfgragh.onion/2020/08/30/us/portland-shooting-explained.html?name=styln-george-floyd\&region=TOP_BANNER\&block=storyline_menu_recirc\&action=click\&pgtype=Article\&impression_id=ac5c37e2-f27d-11ea-84fb-6965c80d8485\&variant=undefined}{Portland
  Shooting}
\item
  \href{https://www.nytimes3xbfgragh.onion/2020/08/30/us/breonna-taylor-police-killing.html?name=styln-george-floyd\&region=TOP_BANNER\&block=storyline_menu_recirc\&action=click\&pgtype=Article\&impression_id=ac5c37e3-f27d-11ea-84fb-6965c80d8485\&variant=undefined}{Breonna
  Taylor's Life and Death}
\end{itemize}

Advertisement

\protect\hyperlink{after-top}{Continue reading the main story}

Supported by

\protect\hyperlink{after-sponsor}{Continue reading the main story}

\hypertarget{a-police-officer-shot-a-13-year-old-with-autism-in-salt-lake-city}{%
\section{A Police Officer Shot a 13-Year-Old With Autism in Salt Lake
City}\label{a-police-officer-shot-a-13-year-old-with-autism-in-salt-lake-city}}

The authorities said the officer shot the boy while responding to a call
about a ``violent psych issue.'' The teenager, Linden Cameron, was
fleeing and did not have a weapon, his mother said.

By \href{https://www.nytimes3xbfgragh.onion/by/jacey-fortin}{Jacey
Fortin}

\begin{itemize}
\item
  Sept. 8, 2020
\item
  \begin{itemize}
  \item
  \item
  \item
  \item
  \item
  \end{itemize}
\end{itemize}

A Salt Lake City police officer shot a 13-year-old boy with autism on
Friday night, prompting an investigation and raising concerns about
officers' use of force in a city that has been reckoning with protests
and police accountability.

The officer fired at the boy while responding to a call about a
``violent psych issue,'' Sgt. Keith Horrocks of the Salt Lake City
Police Department
\href{https://kutv.com/news/local/developing-news-shooting-in-salt-lake-city}{told
reporters} on Saturday morning.

``In this case it was a juvenile that was having a mental episode, a
psychological episode, and had made threats to some folks with a
weapon,'' Sergeant Horrocks said, adding that the officer had fired his
gun ``during a short foot pursuit.''

The boy's mother, Golda Barton, identified her son to local news
reporters as Linden Cameron. She said that he did not have a weapon, and
that she had called the police to get help and possibly take him to a
hospital.

``I said, `Look, he's unarmed, he doesn't have anything,''' she told
KUTV in Salt Lake City. ``He just gets mad and he starts yelling and
screaming. He's a kid. He's trying to get attention. He doesn't know how
to regulate.''

She said that as her son was running away, she heard a series of bangs
and did not immediately know whether Linden had been killed.

``They're supposed to come out and be able to de-escalate a situation
using the most minimal force possible,'' said Ms. Barton, who did not
immediately respond to a request for an interview on Tuesday.

Sergeant Horrocks said that the boy was in ``serious condition'' when he
was taken to a hospital on Friday. He said that there was ``no
indication'' that Linden had a weapon but added that an investigation
was still underway.

The shooting is being investigated by officers in Utah who are not
members of the Salt Lake City Police Department. The police in Salt Lake
City are expected to release body camera footage of the episode by Sept.
21.

``I know that these things are very difficult for the community as a
whole, and there is a process in terms of looking at what happened and
investigating it,'' said Detective Greg Wilking, a spokesman for the
Police Department. He declined to name the officer who fired a weapon.

``Good, bad or in between, that process will be carried out, and we will
make the body-worn camera available and hold our officers accountable if
that's what needs to happen,'' he said.

The episode raised concerns about
\href{https://www.nytimes3xbfgragh.onion/2020/02/27/well/family/autism-special-needs-police.html}{how
the police handle interactions with people with autism} or who are
dealing with mental illness.

According to a 2017 study from the
\href{https://drexel.edu/AutismInstitute/}{Autism Institute} at Drexel
University, an estimated
\href{https://www.ncbi.nlm.nih.gov/pubmed/27844248}{one in five
teenagers with autism are stopped and questioned by the police by the
time they turn 21}. And according to
\href{https://www.chop.edu/news/chop-researchers-present-new-findings-2019-international-society-autism-research-annual-meeting}{research
by the Children's Hospital of Philadelphia}, people with disabilities,
including those on the autism spectrum, are disproportionately injured
in interactions with the police and are five times more likely to be
incarcerated than people in the general population.

``There are plenty of systemic biases against autistic and mentally ill
people'' in Utah, said Whitney Lee Geertsen, the founding director of
\href{https://www.facebookcorewwwi.onion/neurodiverseutah/}{Neurodiverse
U}tah, a group that promotes autism acceptance and self-advocacy. She
called for more police training on neurodiversity and said that the
officers involved in the shooting on Friday should be fired.

The Salt Lake City police have repeatedly come under criticism this
year.

In July, Gov. Gary Herbert of Utah
\href{https://www.nytimes3xbfgragh.onion/2020/07/10/us/utah-state-of-emergency.html}{declared
a state of emergency} in response to protests in Salt Lake City after
the authorities said that the fatal police shooting of a 22-year-old
man, Bernardo Palacios-Carbajal, in May was justified.

And in August, the authorities
\href{https://www.nytimes3xbfgragh.onion/2020/08/13/us/salt-lake-city-police-dog.html}{suspended
the use of police dogs in arrests} after one bit a Black man, Jeffery
Ryans, 36. He was on one knee with his hands in the air after officers
were called to his house after someone heard him arguing with his wife
in April,
\href{https://www.sltrib.com/news/2020/08/11/salt-lake-city-police-dog/}{The
Salt Lake Tribune reported}.

Mayor Erin Mendenhall said that a police officer, who was not named, was
suspended from duty pending an investigation. Ms. Mendenhall said she
was ``disturbed'' by what she saw in a body camera video of the episode,
in which a dog could be seen lunging at Mr. Ryans and biting him as he
pleaded for the police to stop.

\includegraphics{https://static01.graylady3jvrrxbe.onion/images/2020/09/08/multimedia/08xp-slc-shooting2/merlin_175277634_4f28c633-02d7-45b2-96ca-4a5d697564b2-articleLarge.jpg?quality=75\&auto=webp\&disable=upscale}

The mayor said there would be ``a thorough review of the breakdown in
communication to ensure that it does not happen again.''

Last month, Ms. Mendenhall
\href{https://www.slc.gov/mayor/2020/08/03/mayor-mendenhall-slcpd-chief-mike-brown-announce-substantive-policy-changes/}{signed
an executive order} that aimed to restrict the use of force by the
police and promote de-escalation tactics. It was to take effect no later
than Saturday, the day after Linden was shot.

On Tuesday, Mr. Wilking, the police department spokesman, said the
adoption of those changes was still underway. He added that all of the
department's officers had been trained in crisis intervention, which
focuses on interactions with people who have disabilities or are
experiencing mental health emergencies.

In her interview with KUTV, Ms. Barton asked why the police had not used
a Taser or rubber bullets instead of a gun. ``He's a small child,'' she
added. ``Why don't you just tackle him?''

In a statement on Tuesday, Ms. Mendenhall said she was ``thankful this
young boy is alive and no one else was injured.''

``No matter the circumstances, what happened on Friday night is a
tragedy and I expect this investigation to be handled swiftly and
transparently for the sake of everyone involved,'' she said.

Alan Yuhas contributed reporting. Alain Delaquérière contributed
research.

Advertisement

\protect\hyperlink{after-bottom}{Continue reading the main story}

\hypertarget{site-index}{%
\subsection{Site Index}\label{site-index}}

\hypertarget{site-information-navigation}{%
\subsection{Site Information
Navigation}\label{site-information-navigation}}

\begin{itemize}
\tightlist
\item
  \href{https://help.nytimes3xbfgragh.onion/hc/en-us/articles/115014792127-Copyright-notice}{©~2020~The
  New York Times Company}
\end{itemize}

\begin{itemize}
\tightlist
\item
  \href{https://www.nytco.com/}{NYTCo}
\item
  \href{https://help.nytimes3xbfgragh.onion/hc/en-us/articles/115015385887-Contact-Us}{Contact
  Us}
\item
  \href{https://www.nytco.com/careers/}{Work with us}
\item
  \href{https://nytmediakit.com/}{Advertise}
\item
  \href{http://www.tbrandstudio.com/}{T Brand Studio}
\item
  \href{https://www.nytimes3xbfgragh.onion/privacy/cookie-policy\#how-do-i-manage-trackers}{Your
  Ad Choices}
\item
  \href{https://www.nytimes3xbfgragh.onion/privacy}{Privacy}
\item
  \href{https://help.nytimes3xbfgragh.onion/hc/en-us/articles/115014893428-Terms-of-service}{Terms
  of Service}
\item
  \href{https://help.nytimes3xbfgragh.onion/hc/en-us/articles/115014893968-Terms-of-sale}{Terms
  of Sale}
\item
  \href{https://spiderbites.nytimes3xbfgragh.onion}{Site Map}
\item
  \href{https://help.nytimes3xbfgragh.onion/hc/en-us}{Help}
\item
  \href{https://www.nytimes3xbfgragh.onion/subscription?campaignId=37WXW}{Subscriptions}
\end{itemize}
