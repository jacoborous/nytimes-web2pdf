Sections

SEARCH

\protect\hyperlink{site-content}{Skip to
content}\protect\hyperlink{site-index}{Skip to site index}

\href{https://www.nytimes3xbfgragh.onion/section/your-money}{Your Money}

\href{https://myaccount.nytimes3xbfgragh.onion/auth/login?response_type=cookie\&client_id=vi}{}

\href{https://www.nytimes3xbfgragh.onion/section/todayspaper}{Today's
Paper}

\href{/section/your-money}{Your Money}\textbar{}Teen Stock Trading Seems
Dangerous. It Doesn't Have to Be.

\url{https://nyti.ms/3bQXoin}

\begin{itemize}
\item
\item
\item
\item
\item
\end{itemize}

Advertisement

\protect\hyperlink{after-top}{Continue reading the main story}

Supported by

\protect\hyperlink{after-sponsor}{Continue reading the main story}

Your Money

\hypertarget{teen-stock-trading-seems-dangerous-it-doesnt-have-to-be}{%
\section{Teen Stock Trading Seems Dangerous. It Doesn't Have to
Be.}\label{teen-stock-trading-seems-dangerous-it-doesnt-have-to-be}}

Gunslinging young investors are making stock ownership seem like a
terrible idea for novices. But owning equities, with limits and
guardrails, can teach kids plenty.

\includegraphics{https://static01.graylady3jvrrxbe.onion/images/2020/09/12/business/11money/11money-articleLarge.jpg?quality=75\&auto=webp\&disable=upscale}

\href{https://www.nytimes3xbfgragh.onion/by/ron-lieber}{\includegraphics{https://static01.graylady3jvrrxbe.onion/images/2018/10/22/multimedia/author-ron-lieber/author-ron-lieber-thumbLarge.png}}

By \href{https://www.nytimes3xbfgragh.onion/by/ron-lieber}{Ron Lieber}

\begin{itemize}
\item
  Sept. 12, 2020, 3:00 a.m. ET
\item
  \begin{itemize}
  \item
  \item
  \item
  \item
  \item
  \end{itemize}
\end{itemize}

This year,
\href{https://www.nytimes3xbfgragh.onion/2020/06/14/business/sports-gamblers-stocks-virus.html}{Robinhood}
and its
\href{https://www.nytimes3xbfgragh.onion/2020/06/14/business/sports-gamblers-stocks-virus.html}{millions}
of younger-than-average customers have found themselves at the center of
attention that is both comic and tragic.

``Robinhood: `We now allow teenagers with their parents' credit cards to
trade stocks''' went the April headline on the satire site
\href{https://thestonkmarket.com/robinhood-we-now-allow-teenagers-with-their-parents-credit-cards-to-trade-stocks/}{Stonk
Market}. Two months later, the headlines were far more sobering, after a
20-year-old
\href{https://www.forbes.com/sites/sergeiklebnikov/2020/06/17/20-year-old-robinhood-customer-dies-by-suicide-after-seeing-a-730000-negative-balance/\#46e6932c1638}{killed
himself}, leaving behind a note about his negative Robinhood balance.

Robinhood's customers trade fast --- often in particularly volatile
types of investments --- and can lose lots quickly, as my colleague
Nathaniel Popper
\href{https://www.nytimes3xbfgragh.onion/2020/07/08/technology/robinhood-risky-trading.html}{reported}
in July. Customers have gone to its headquarters to complain, and the
company has installed bulletproof glass.

All of this activity frightens many parents, and with good reason. But
owning just a few shares of a company's stock is something else
entirely, and it need not lead to ruin or ingrain bad investing habits.
It may actually be the way to build good ones.

Learning about financial risk can be challenging for children whose
families have never faced much economic hardship. Nevertheless, it's a
crucial lesson.

The best way for most people to save enough for retirement is to start
early, invest most of the money in stocks early on and hang on tight for
half a century. Watching with dread or exhilaration as balances fall
when the market swoons or rise when it booms can lead to poor decisions.
The earlier someone experiences that volatility and learns how to react
to it, the better.

So what are we talking about when we talk about risk? In his lucid new
book,
``\href{https://www.collaborativefund.com/blog/book-the-psychology-of-money/}{The
Psychology of Money},'' Morgan Housel explains it like so: Any goal
worth chasing in life will almost always come with odds of success that
are less than 100 percent. ``Risk is just what happens when you end up
on the unfortunate side of that equation,'' he writes.

The challenge for any investor is figuring out whether any such failure
is due to bad luck or to poor skill. Solving for that equation --- and,
hopefully, developing some humility as a result --- is a lifelong
pursuit.

Why stocks for teenagers, then? They offer regular score-keeping and the
possibility of a bit of real pain when the stakes are lower than they
are when managing larger amounts of money. Erect the proper guardrails,
and stocks can teach the emerging adults in your life valuable lessons.

\hypertarget{the-right-account}{%
\subsection{The Right Account}\label{the-right-account}}

If you're under 18, you can't have your own brokerage account and trade
without supervision.

It may be tempting to simply open a regular account and trade with your
kids, but a better option may be a custodial account, which an adult
sets up for a person who is not yet 18. It may come with
\href{https://www.irs.gov/taxtopics/tc553}{lower taxes} on any gains,
although the overall balance is subject to the calculations of college
financial aid examiners.

Ask questions about trading commissions, account fees and any minimum
balance requirements. Also, inquire about whether you can buy fractional
shares of individual stocks that may have high prices for even a single
share.

\hypertarget{the-right-rules}{%
\subsection{The Right Rules}\label{the-right-rules}}

There are some guidelines that you as a parent, relative or mentor ought
to set. Stick to basics for the first few years, which means no short
sales, options or use of debt to buy on margin.

Then there are the firms' rules, which adults sometimes ignore. Charles
Schwab, Fidelity and TD Ameritrade were pretty much unanimous in this
refrain: Don't give kids the account passwords so that they can trade on
their own. And if you do, don't come running to us for help if they make
some gonzo bet that doesn't work out.

Robinhood, which does not offer custodial accounts, doesn't want anyone
handing out passwords, either. A spokesman declined to comment on how
often it needs to shut down accounts because people under 18 have
managed to trade anyhow.

\hypertarget{skin-in-the-game}{%
\subsection{Skin in the Game}\label{skin-in-the-game}}

Like so many newbie investors in the 1990s, I was set on a straight path
by columns from The Wall Street Journal's
\href{https://www.deseret.com/1995/6/11/19176239/index-funds-are-excellent-if-you-know-what-you-re-doing}{Jonathan
Clements}, who used his own children as guinea pigs in delightful ways.

In an interview this week, he reminded me of one failed test, where he
doled out a bit of money and then held a mutual-fund-picking contest.
His son lost to both his dad and his sister, but he didn't seem to care
or learn all that much from the experience.

``He was handed a bunch of chips and told to go off and play,'' said Mr.
Clements, who is the author of
``\href{https://humbledollar.com/book/how-to-think-about-money/}{How to
Think About Money}'' and now edits
\href{https://humbledollar.com/about/}{Humbledollar.com} ``It doesn't
feel like losing money if you come away empty-handed at the end of the
evening.''

Best, then, to have kids invest money they have earned --- so they can
recall the hours of toil it took to assemble their little all.

\hypertarget{set-a-goal}{%
\subsection{Set a Goal}\label{set-a-goal}}

In a spirited
\href{https://twitter.com/christine_benz/status/1273626396267614209}{Twitter
exchange} and follow-up
\href{https://www.morningstar.com/articles/988805/theres-got-to-be-a-better-way}{article}
in June, Morningstar's director of personal finance, Christine Benz,
expressed serious reservations about buying individual stocks if you're
a new investor. People who invest in them, after all, tend not to earn
as much over time as those who just put their money in a mutual or
exchange-traded fund that owns scores of individual stocks. Why not have
novices invest in index funds from the get-go?

Well, that can be boring. Also, individual stocks get teenagers thinking
about larger economic forces: Why is this company performing better than
another? And the dizzying losses that are more likely with individual
stocks can teach a valuable early lesson.

When Ms. Benz and I chatted this week, we agreed on this: Any stock
investment must begin with defining the point of the exercise. And that
depends on the teenager, too.

``What really got me stoked was achieving a goal,'' Ms. Benz said. A
shorter-term objective like saving for a bicycle probably would have
motivated her to take less investment risk, not more. That would have
had her avoiding individual stocks.

\hypertarget{challenge-their-choices}{%
\subsection{Challenge Their Choices}\label{challenge-their-choices}}

Collin Roberts, 17, and his teammates at Maclay School in Tallahassee,
Fla., were co-champions of the
\href{https://kwhs.wharton.upenn.edu/news/2020-investment-competition-global-finale-ends-tie-first-place/}{Wharton
Global High School Investment Competition} this year. But when he wants
to buy a stock, he reports to an investment committee of one: his
father.

\href{https://harvestinvestmentadvisors.com/}{David Roberts,} a
financial adviser who once had to answer to the overlords at Northern
Trust when he was a mutual fund manager, still puts Collin through his
own paces, Collin said.

``If it's not a good stock, I'll usually realize it as I'm pitching
it,'' he said. ``When I'm explaining it, I'm exposing myself to the
problems.''

\hypertarget{the-upper-limits}{%
\subsection{The Upper Limits}\label{the-upper-limits}}

One downside of teen stock ownership is the upside.

Big gains can make you feel invincible. It is a real danger now, given
how well some technology stocks have performed of late.

Mr. Clements has a suggestion. Set a limit on gains in any stock. Once
past that point, sell enough to capture all the winnings and put them in
a basic index fund.

Then you have an experiment cooking, where the money that you have left
in the stock faces off against the new fund. That fund will zigzag far
less than the stock, most likely. Hopefully it ends up as the better
performer, reinforcing the lesson that diversifying your investments is
a smart move and creates less anxiety, too.

But even if single-stock gunslinging does net a windfall for a young
investor, that exposure to mutual funds will count for something. After
all, they make up the backbone of workplace retirement plans, and those
funds are the best routes to long-term financial security for most
investors. Owning a bit of stock should help young investors know what
to do by the time they enter the working world.

``With any luck, it will be the trigger that excites them about
investing and turns them into someone investing in the stock market for
the rest of their lives,'' Mr. Clements said. ``If you can do that, the
payoff is huge.''

Advertisement

\protect\hyperlink{after-bottom}{Continue reading the main story}

\hypertarget{site-index}{%
\subsection{Site Index}\label{site-index}}

\hypertarget{site-information-navigation}{%
\subsection{Site Information
Navigation}\label{site-information-navigation}}

\begin{itemize}
\tightlist
\item
  \href{https://help.nytimes3xbfgragh.onion/hc/en-us/articles/115014792127-Copyright-notice}{©~2020~The
  New York Times Company}
\end{itemize}

\begin{itemize}
\tightlist
\item
  \href{https://www.nytco.com/}{NYTCo}
\item
  \href{https://help.nytimes3xbfgragh.onion/hc/en-us/articles/115015385887-Contact-Us}{Contact
  Us}
\item
  \href{https://www.nytco.com/careers/}{Work with us}
\item
  \href{https://nytmediakit.com/}{Advertise}
\item
  \href{http://www.tbrandstudio.com/}{T Brand Studio}
\item
  \href{https://www.nytimes3xbfgragh.onion/privacy/cookie-policy\#how-do-i-manage-trackers}{Your
  Ad Choices}
\item
  \href{https://www.nytimes3xbfgragh.onion/privacy}{Privacy}
\item
  \href{https://help.nytimes3xbfgragh.onion/hc/en-us/articles/115014893428-Terms-of-service}{Terms
  of Service}
\item
  \href{https://help.nytimes3xbfgragh.onion/hc/en-us/articles/115014893968-Terms-of-sale}{Terms
  of Sale}
\item
  \href{https://spiderbites.nytimes3xbfgragh.onion}{Site Map}
\item
  \href{https://help.nytimes3xbfgragh.onion/hc/en-us}{Help}
\item
  \href{https://www.nytimes3xbfgragh.onion/subscription?campaignId=37WXW}{Subscriptions}
\end{itemize}
