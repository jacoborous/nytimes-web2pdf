Sections

SEARCH

\protect\hyperlink{site-content}{Skip to
content}\protect\hyperlink{site-index}{Skip to site index}

\href{https://www.nytimes3xbfgragh.onion/section/health}{Health}

\href{https://myaccount.nytimes3xbfgragh.onion/auth/login?response_type=cookie\&client_id=vi}{}

\href{https://www.nytimes3xbfgragh.onion/section/todayspaper}{Today's
Paper}

\href{/section/health}{Health}\textbar{}Now It's Not Safe at Home
Either. Wildfires Bring Ashen Air Into the House.

\url{https://nyti.ms/3bTCu2j}

\begin{itemize}
\item
\item
\item
\item
\item
\end{itemize}

Advertisement

\protect\hyperlink{after-top}{Continue reading the main story}

Supported by

\protect\hyperlink{after-sponsor}{Continue reading the main story}

\hypertarget{now-its-not-safe-at-home-either-wildfires-bring-ashen-air-into-the-house}{%
\section{Now It's Not Safe at Home Either. Wildfires Bring Ashen Air
Into the
House.}\label{now-its-not-safe-at-home-either-wildfires-bring-ashen-air-into-the-house}}

Our reporter, a San Francisco resident, describes family life amid a
pandemic and a natural disaster.

\includegraphics{https://static01.graylady3jvrrxbe.onion/images/2020/09/11/science/00SANFRAN-ESSAY1/merlin_176818071_c52a134f-6afe-42a3-ae5e-7e3f89581ce4-articleLarge.jpg?quality=75\&auto=webp\&disable=upscale}

By \href{https://www.nytimes3xbfgragh.onion/by/matt-richtel}{Matt
Richtel}

\begin{itemize}
\item
  Sept. 12, 2020, 11:18 a.m. ET
\item
  \begin{itemize}
  \item
  \item
  \item
  \item
  \item
  \end{itemize}
\end{itemize}

SAN FRANCISCO --- The new thing we do here when we get up in the
morning, even before the tooth brushing and the coffee making, is to
look at the sky. Then we look at the internet to see if our eyes
deceive.

``Purple again,'' I said to my wife this morning. Not the sky. That was
the color of soot, like a child had taken dirty fingers and rubbed them
all over the horizon. Purple is the color on the air quality chart. It
means that we've hit ``very unhealthy,'' our air filled with microscopic
particles that, speaking of children, are dangerous for them to breathe
into their soft pink lungs. And not so great for those of us who have a
few miles on our lungs already.

During the coronavirus pandemic, our last refuge had been to stay inside
the house, but when things go this purple this persistently, the trouble
seeps inside. Thanks to rampant wildfires, our at-home air filter has
started telling us that things have turned unhealthy in our home --- the
bad air is managing to sneak in, even through closed windows and doors.

So we've taken to passing our one air purifier from room to room so our
two children can do SOTG (school on the go) without getting SOOT (soot
in the bloodstream). We clean each room, then rotate the device, and I
trail to maintain the obnoxious optimism that is my hallmark and
fatherly duty. But you can tell things are bad when you start reaching
for comparisons, like: Well, we could be in Flanders in 1918. (Maybe
that rose tint to my glasses is actually ash.)

In actuality, I don't have to reach back to Belgium during World War I
to know things could be worse. We could be in the Portland suburbs or
lots of other places in the Pacific Northwest, circa right now. There,
the ash in the sky comes with rampant blazes that are creating actual
refugees, meaning people who are running from death with whatever they
can carry.

\includegraphics{https://static01.graylady3jvrrxbe.onion/images/2020/09/12/science/12SANFRAN-ESSAY2/12SANFRAN-ESSAY2-articleLarge.jpg?quality=75\&auto=webp\&disable=upscale}

So yes, we are privileged: roof over the head, freezer full of meat and
crisper stocked with vegetables. My wife and I remain employed and no
one we are close to has died from that terrible virus.

That said, I have had a migraine three days running from the poor air
(or self-pity, or both). My wife is a neurologist who specializes in
treating migraines, and she says that it's supposed to help when you sit
in darkness. But I can tell you that on Wednesday morning --- when we
woke up and looked at the sky and it was the orange-black of Halloween
--- the all-day darkness did little to calm the headache.

I'm a science reporter, and it's hard not to see what's happening now as
a science story, with Covid-19 taking advantage of population density
and other modern factors to hop and skip across the globe and from cough
to nose and lung to lung, and for wildfire to take advantage of our
human spread into the urban/wild nexus and turn our Manifest Destiny
into so much hay. Mother nature tidies up, you might say, which is a
really clinical and unsympathetic way to see things, to the point of
being fatalistic.

What my family wants is not some ``context'' but to go outside and play.
Or even inside and play.

``Dad, can I come out of my room after school?'' my son, Milo, 12, asked
me this afternoon when he poked his head into my home office. He'd been
under orders to stay in his bedroom, with SOTG and his turn with the air
purifier.

``I'll check the air,'' I told him.

I looked outside, and it was yellow-gray like a smoker's teeth. I looked
on the internet and it was purple still, worse than this morning, the
air quality index now reading 228, a higher level of ``very unhealthy.''

This weekend was supposed to be a relief, the start again of some
socially distanced sports, including baseball and tennis practice for my
son, tennis for me, maybe a family bike ride. Unlikely now. I'm hoping
just to look up in the sky, and on the internet, and see only red
(merely unhealthy) or even yellow. That would be moderate.

Advertisement

\protect\hyperlink{after-bottom}{Continue reading the main story}

\hypertarget{site-index}{%
\subsection{Site Index}\label{site-index}}

\hypertarget{site-information-navigation}{%
\subsection{Site Information
Navigation}\label{site-information-navigation}}

\begin{itemize}
\tightlist
\item
  \href{https://help.nytimes3xbfgragh.onion/hc/en-us/articles/115014792127-Copyright-notice}{©~2020~The
  New York Times Company}
\end{itemize}

\begin{itemize}
\tightlist
\item
  \href{https://www.nytco.com/}{NYTCo}
\item
  \href{https://help.nytimes3xbfgragh.onion/hc/en-us/articles/115015385887-Contact-Us}{Contact
  Us}
\item
  \href{https://www.nytco.com/careers/}{Work with us}
\item
  \href{https://nytmediakit.com/}{Advertise}
\item
  \href{http://www.tbrandstudio.com/}{T Brand Studio}
\item
  \href{https://www.nytimes3xbfgragh.onion/privacy/cookie-policy\#how-do-i-manage-trackers}{Your
  Ad Choices}
\item
  \href{https://www.nytimes3xbfgragh.onion/privacy}{Privacy}
\item
  \href{https://help.nytimes3xbfgragh.onion/hc/en-us/articles/115014893428-Terms-of-service}{Terms
  of Service}
\item
  \href{https://help.nytimes3xbfgragh.onion/hc/en-us/articles/115014893968-Terms-of-sale}{Terms
  of Sale}
\item
  \href{https://spiderbites.nytimes3xbfgragh.onion}{Site Map}
\item
  \href{https://help.nytimes3xbfgragh.onion/hc/en-us}{Help}
\item
  \href{https://www.nytimes3xbfgragh.onion/subscription?campaignId=37WXW}{Subscriptions}
\end{itemize}
