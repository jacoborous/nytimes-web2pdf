Sections

SEARCH

\protect\hyperlink{site-content}{Skip to
content}\protect\hyperlink{site-index}{Skip to site index}

\href{https://www.nytimes3xbfgragh.onion/section/business}{Business}

\href{https://myaccount.nytimes3xbfgragh.onion/auth/login?response_type=cookie\&client_id=vi}{}

\href{https://www.nytimes3xbfgragh.onion/section/todayspaper}{Today's
Paper}

\href{/section/business}{Business}\textbar{}Terence Conran, Designer and
Retail Magnate, Is Dead at 88

\url{https://nyti.ms/3bPPjuB}

\begin{itemize}
\item
\item
\item
\item
\item
\end{itemize}

Advertisement

\protect\hyperlink{after-top}{Continue reading the main story}

Supported by

\protect\hyperlink{after-sponsor}{Continue reading the main story}

\hypertarget{terence-conran-designer-and-retail-magnate-is-dead-at-88}{%
\section{Terence Conran, Designer and Retail Magnate, Is Dead at
88}\label{terence-conran-designer-and-retail-magnate-is-dead-at-88}}

An entrepreneur of mercurial moods and missionary zeal, he created an
empire to market his designs and later opened restaurants in London,
Paris and New York.

\includegraphics{https://static01.graylady3jvrrxbe.onion/images/2020/09/13/obituaries/12conran-image1/merlin_51253654_e5a8c7df-f0e9-4d67-8e12-24bab8ff83ac-articleLarge.jpg?quality=75\&auto=webp\&disable=upscale}

By \href{https://www.nytimes3xbfgragh.onion/by/robert-d-mcfadden}{Robert
D. McFadden}

\begin{itemize}
\item
  Sept. 12, 2020Updated 2:40 p.m. ET
\item
  \begin{itemize}
  \item
  \item
  \item
  \item
  \item
  \end{itemize}
\end{itemize}

Terence Conran, a London designer and retailing magnate who eased the
gloom of postwar British austerity with stylish home furnishings
affordable on a teacher's salary, and then suffered financial reverses
before reinventing himself as an international restaurateur and doyen of
modern design, died on Saturday at his country home in Berkshire,
England. He was 88.

Image

Mr. Conran and his second wife, the novelist Shirley Pearce, in
1955.Credit...Express/Hulton Archive, via Getty Images

His family confirmed the death in a statement, without specifying the
cause.

Blind in one eye since childhood, Mr. Conran was an entrepreneur of
mercurial moods and missionary zeal who created an empire to market his
designs, stores known in Europe as Habitat and in America as Conran
Shops. After his business declined, he opened restaurants in London,
Paris and New York --- notably Guastavino's, a dining cathedral under
the tiled terra-cotta arches of the Queensboro Bridge in Manhattan.

He wrote scores of books on design, cooking and other subjects; turned a
London warehouse riverfront into a fashionable South Bank commercial
development; founded the Design Museum, Britain's only museum for
contemporary products and architectural designs; and was knighted by
Queen Elizabeth II. He also married four times, had five children and
collected wildflowers, butterflies, old master paintings and Bugatti
pedal cars.

Image

Among Mr. Conran's designs was this one for a detergent
container.Credit...Design Museum Book of 20th-Century Design

Detractors called him a cynical self-promoter who sold simplistic ideas
to the masses, like ``democratized luxury,'' and struck it rich with a
sure thing: the inevitable desire of Britons to climb from grinding
wartime privations into a consumer class that could afford to replace
the threadbare old sofa with something seen as ``modern'' and in ``good
taste.''

But admirers said he tried, with remarkable success, to revolutionize
the sensibilities of a rising British middle class, offering not just
better food but an idea of what a sunlit breakfast on Sunday should be;
not just mod touches for the drab suburban semidetached but a taste of
la dolce vita: Scandinavian furniture, Italian lighting, French
cookware, Bauhaus-style modular shelving and splashes of Pop Art on the
walls.

In a career that spanned six decades, he had only one actual job: At 19,
he worked briefly for an architect who helped design the 1951 Festival
of Britain, a national exposition intended to give Britons a sense of
recovery from the war. It also gave him a frank look at a people weary
of shortages, and a glimpse of the future of commercial design.

``They came along in their dreary wartime mackintoshes, gas-mask cases
filled with Spam sandwiches, and found bright cafes, music, flowers,
modern furniture and a spirit of something that none of them had ever
experienced in their lives,'' Mr. Conran
\href{https://www.telegraph.co.uk/lifestyle/interiors/8750145/The-taste-maker-interview-with-Terence-Conran.html}{told
The Daily Telegraph}, the British newspaper, in 2011.

Over the next decade, he designed simple furniture and sold it in an
arcade in Piccadilly; opened his first restaurant, a sandwich-and-salad
bar called Soup Kitchen that had one of London's first espresso makers;
and created new lines of fabrics and moderately priced, functional home
furnishings.

In 1964, he opened his first Habitat store in Chelsea. Its staff had
uniforms by Mary Quant and hairstyles by Vidal Sassoon.

By the late 1980s, after acquiring other chains, he owned 900 stores in
Britain, Europe, Japan and America, selling furniture, housewares and
clothing. His company, the Storehouse Group, had 35,000 employees and
billions in revenues.

But overexpansion --- including additions to the upscale Butler's Wharf
on the Thames, where he installed his Design Museum in 1989 and lived in
a glass penthouse --- corroded his empire. So did his failure to
integrate interests reaching into publishing, office products,
architecture and real estate. He resigned as chairman in 1990.
Storehouse was dismantled, and Habitat was taken over by Ikea, the
Swedish furniture giant.

Mr. Conran kept some Conran Shops and recovered in the 1990s, opening
many theatrically dazzling restaurants, including Le Pont de la Tour and
Mezzo in London, Alcazar in Paris and Berns Salonger in Stockholm. In
New York, he and his partner, Joel Kissin, opened Guastavino's in 2000
under the Queensboro Bridge. The site was dramatic but out of the way,
and a few years later became a catered event space.

\includegraphics{https://static01.graylady3jvrrxbe.onion/images/2020/09/13/obituaries/12conran-image5/12conran-image5-articleLarge-v2.jpg?quality=75\&auto=webp\&disable=upscale}

In 2005, he was named the most influential restaurateur in Britain by
CatererSearch, the website of Caterer and Hotelkeeper magazine, and his
resurrected fortune was estimated at more than \$100 million.

Image

Mr. Conran in 2009 in the Terence Conran Suite at Boundary in the
Shoreditch district of London, a project consisting of three
restaurants, rooms and suites.Credit...Jonathan Player for The New York
Times

Terence Orby Conran was born on Oct. 4, 1931, in the London suburb
Kingston upon Thames, to Gerard and Christina (Halstead) Conran. His
father was a businessman. His mother, who had a taste for art, nurtured
Terence's creative talents. When he was 13, his left eye was permanently
blinded by a sliver of metal that flew up from a lathe he was using.

He attended Bryanston, a private school in Dorset, and the Central
School of Arts and Crafts in London. He did not graduate, but a teacher
there, the sculptor Eduardo Paolozzi, became a lifelong friend and
mentor.

His first marriage, at the age of 19 to the architect Brenda Davison,
lasted six months. He and his second wife, the novelist Shirley Pearce,
had two children, Sebastian and Jasper, designers who held various
professional and executive positions with their father's enterprises
over the years; they were divorced in 1962. He and his third wife, the
food writer Caroline Herbert, had three children, Edmund (known as Ned),
Tom and Sophie, and were divorced in 1996. He married his fourth wife,
Victoria Davis, in 2000.

He is survived by his wife and his children, as well as 14
grandchildren, one great-grandchild and a sister, Priscilla, a designer
and restaurateur.

Image

Mr. Conran and his wife, Victoria, at an opening at the Saatchi Gallery
in London in 2010.Credit...Press Association, via Associated Press

Mr. Conran became a disciple of Elizabeth David, whose books
Europeanized British cooking. Besides design and cooking, his own books
explored home furnishings, textiles, gardening and other subjects. An
authorized biography, ``Terence Conran,'' by Nicholas Ind, was published
in 1995.

In addition to his 145-acre estate in Berkshire, Barton Court, Mr.
Conran had an apartment in London.

From 2003 to 2011, Mr. Conran was provost of the Royal College of Art in
London. Besides his 1983 knighthood, a title he said he used only to
make reservations, his honors included the Minerva Medal, the highest
award of the Chartered Society of Designers, and the Prince Phillip
Designers Prize for lifetime achievement.

Alex Marshall contributed reporting from London.

Advertisement

\protect\hyperlink{after-bottom}{Continue reading the main story}

\hypertarget{site-index}{%
\subsection{Site Index}\label{site-index}}

\hypertarget{site-information-navigation}{%
\subsection{Site Information
Navigation}\label{site-information-navigation}}

\begin{itemize}
\tightlist
\item
  \href{https://help.nytimes3xbfgragh.onion/hc/en-us/articles/115014792127-Copyright-notice}{©~2020~The
  New York Times Company}
\end{itemize}

\begin{itemize}
\tightlist
\item
  \href{https://www.nytco.com/}{NYTCo}
\item
  \href{https://help.nytimes3xbfgragh.onion/hc/en-us/articles/115015385887-Contact-Us}{Contact
  Us}
\item
  \href{https://www.nytco.com/careers/}{Work with us}
\item
  \href{https://nytmediakit.com/}{Advertise}
\item
  \href{http://www.tbrandstudio.com/}{T Brand Studio}
\item
  \href{https://www.nytimes3xbfgragh.onion/privacy/cookie-policy\#how-do-i-manage-trackers}{Your
  Ad Choices}
\item
  \href{https://www.nytimes3xbfgragh.onion/privacy}{Privacy}
\item
  \href{https://help.nytimes3xbfgragh.onion/hc/en-us/articles/115014893428-Terms-of-service}{Terms
  of Service}
\item
  \href{https://help.nytimes3xbfgragh.onion/hc/en-us/articles/115014893968-Terms-of-sale}{Terms
  of Sale}
\item
  \href{https://spiderbites.nytimes3xbfgragh.onion}{Site Map}
\item
  \href{https://help.nytimes3xbfgragh.onion/hc/en-us}{Help}
\item
  \href{https://www.nytimes3xbfgragh.onion/subscription?campaignId=37WXW}{Subscriptions}
\end{itemize}
