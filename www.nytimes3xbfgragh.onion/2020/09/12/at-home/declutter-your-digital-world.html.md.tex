Sections

SEARCH

\protect\hyperlink{site-content}{Skip to
content}\protect\hyperlink{site-index}{Skip to site index}

\href{https://www.nytimes3xbfgragh.onion/spotlight/at-home}{At Home}

\href{https://myaccount.nytimes3xbfgragh.onion/auth/login?response_type=cookie\&client_id=vi}{}

\href{https://www.nytimes3xbfgragh.onion/section/todayspaper}{Today's
Paper}

\href{/spotlight/at-home}{At Home}\textbar{}How to Declutter Your
Digital World

\url{https://nyti.ms/3hoRo1H}

\begin{itemize}
\item
\item
\item
\item
\item
\end{itemize}

\href{https://www.nytimes3xbfgragh.onion/spotlight/at-home?action=click\&pgtype=Article\&state=default\&region=TOP_BANNER\&context=at_home_menu}{At
Home}

\begin{itemize}
\tightlist
\item
  \href{https://www.nytimes3xbfgragh.onion/2020/09/07/travel/route-66.html?action=click\&pgtype=Article\&state=default\&region=TOP_BANNER\&context=at_home_menu}{Cruise
  Along: Route 66}
\item
  \href{https://www.nytimes3xbfgragh.onion/2020/09/04/dining/sheet-pan-chicken.html?action=click\&pgtype=Article\&state=default\&region=TOP_BANNER\&context=at_home_menu}{Roast:
  Chicken With Plums}
\item
  \href{https://www.nytimes3xbfgragh.onion/2020/09/04/arts/television/dark-shadows-stream.html?action=click\&pgtype=Article\&state=default\&region=TOP_BANNER\&context=at_home_menu}{Watch:
  Dark Shadows}
\item
  \href{https://www.nytimes3xbfgragh.onion/interactive/2020/at-home/even-more-reporters-editors-diaries-lists-recommendations.html?action=click\&pgtype=Article\&state=default\&region=TOP_BANNER\&context=at_home_menu}{Explore:
  Reporters' Google Docs}
\end{itemize}

Advertisement

\protect\hyperlink{after-top}{Continue reading the main story}

Supported by

\protect\hyperlink{after-sponsor}{Continue reading the main story}

\hypertarget{how-to-declutter-your-digital-world}{%
\section{How to Declutter Your Digital
World}\label{how-to-declutter-your-digital-world}}

If you're overwhelmed from telecommuting for months, here are ways to
step away from your devices and, just maybe, get to inbox zero.

\includegraphics{https://static01.graylady3jvrrxbe.onion/images/2020/09/13/multimedia/13ah-DigitalLife/13ah-DigitalLife-articleLarge.jpg?quality=75\&auto=webp\&disable=upscale}

\href{https://www.nytimes3xbfgragh.onion/by/sara-aridi}{\includegraphics{https://static01.graylady3jvrrxbe.onion/images/2019/04/30/multimedia/author-sara-aridi/author-sara-aridi-thumbLarge.png}}

By \href{https://www.nytimes3xbfgragh.onion/by/sara-aridi}{Sara Aridi}

\begin{itemize}
\item
  Sept. 12, 2020, 1:00 p.m. ET
\item
  \begin{itemize}
  \item
  \item
  \item
  \item
  \item
  \end{itemize}
\end{itemize}

Working remotely may have eliminated your commute and allowed you to
spend the day in your pajamas, but it also means you're most likely
bombarded with digital communication every second of the day --- from
personal and professional emails crowding your inboxes to push
notifications reminding you of every news development to the nonstop
viral allure of Twitter and Instagram.

If you are suffering from tech fatigue, or simply trying to become more
productive online, here are steps you can take to organize your digital
landscape.

\hypertarget{create-separation}{%
\subsection{Create separation.}\label{create-separation}}

Cal Newport, a computer science professor at Georgetown University who
writes about the intersection of technology and culture, said many
people succumb to what he calls the list/reactive method: They instantly
react to communication --- texts, emails, Slack messages --- while
occasionally trying to make progress on their work. One moment they're
responding to an email from their child's teacher, the next they're
jumping on a conference call --- blurring the line between the
professional and personal.

``It blends together the lives completely,'' Dr. Newport said. ``You're
never not working. You always feel behind.''

To avoid that cycle, set a fixed digital schedule that clearly dictates
when you are working, when you are attending to your family and when you
are unwinding. Deal with communications concerning the different parts
of your life only during those times. Put aside blocks of time to check
personal text messages. And only go over the day's headlines in the
morning so you don't casually check the news during work hours.

``In our current moment, to not look at any news seems like it would be
a betrayal of your civic responsibility,'' Dr. Newport said. ``But on
the other hand, to look at news all the time is a betrayal of your
sanity.''

\hypertarget{set-expectations}{%
\subsection{Set expectations.}\label{set-expectations}}

Talk to your colleagues --- or, if you're a teacher, your students ---
about when you are available to answer them.

``Set expectations for everyone involved,'' said Lynette O'Keefe, the
director of research and innovation at the
\href{https://onlinelearningconsortium.org/}{Online Learning
Consortium}, a nonprofit that offers digital teaching guidance to
educators. That can help reduce the volume of messages you receive and
make clear to people that your schedule may not align with theirs.
Educators, for instance, should let their students know whether they
respond to emails after hours or not.

``We're expected to be available a lot more, which is perhaps ironic,''
said Ioana Literat, a communication professor at Teachers College,
Columbia University. ``We have so many more responsibilities and our
lives are so chaotic now.''

With that in mind, share how you prefer to be reached. If you are
telecommuting with young children at home and worry that they may Zoom
bomb your meetings, explain to your boss that you prefer audio calls.
Educators teaching online should find out how their students prefer to
communicate and try to meet them halfway, Dr. O'Keefe said. The chat app
\href{https://www.remind.com/}{Remind}, for example, lets teachers
receive messages by text, email or push notifications.

You may feel obligated to instantly answer every email, Twitter message
and Slack message that comes your way. But Diane Bailey, a communication
professor at Cornell University, says it's important to remember that
``what's an interruption for us is typically help for somebody else.''
Some requests are urgent; others are not. Before stopping what you're
doing to lend a hand, think about when you can make time to help others
and when you need to focus on helping yourself. If, for instance, you
need an hour to meet a deadline, close your email inbox, and don't check
it until your job is done.

\hypertarget{assume-control-of-your-inbox}{%
\subsection{Assume control of your
inbox.}\label{assume-control-of-your-inbox}}

One of the simplest ways to clear out your inbox is to unsubscribe from
mailing lists. Both Gmail and Apple's Mail app notify users if an email
is from a mailing list with the option to unsubscribe with a single
click. Use it.

You can also sort --- and limit --- emails by filtering them by the
sender, recipient or subject line. Say you receive a weekly progress
report that is good to have in your back pocket but doesn't need to be
read as soon as it arrives. You can create a filter that will
automatically mark it as read, send it to your archives or give it a
certain label. For your personal inbox, consider creating labels for
bills or appointment reminders, so they don't get lost in the mix.

Then, consider whether you would be more productive if you consolidated
your personal and professional emails in one inbox. If you worry about
missing important notes from either and constantly toggle between the
two, import them under a single address. You can do this on Gmail using
the mail fetcher option, or on Outlook by creating aliases that send and
receive emails from different accounts. Both systems also have features
that can automatically forward all your emails from one account to
another.

If you collaborate with a large team and feel that long email threads
often get in the way of the task at hand, brainstorm an efficient work
flow. Maybe that means dropping ideas into a shared Google doc or
holding weekly meetings to go over specific goals. Having a structured
process ``substantially reduces the number of simultaneous, asynchronous
back and forth conversations happening,'' Dr. Newport said.

To avoid wasting time emailing back and forth to schedule meetings, use
a shared calendar --- like \href{https://youcanbook.me/}{YouCanBookMe},
\href{https://calendly.com/}{Calendly} or
\href{https://x.ai/?utm_source=zapier.com\&utm_medium=referral\&utm_campaign=zapier}{x.ia}
--- where colleagues can see your availability and book slots
accordingly. If you frequently set up meetings with people outside your
organization, those tools can be integrated with Google and Outlook
calendars, so you don't have to switch between different platforms.

Another timesaver: Rather than type up the same response to common
questions or requests, save a template so you can quickly fire it off
when needed. That, Dr. O'Keefe said, tells recipients, ``I see you, I
hear you, I'm interested in responding to you --- but here's when it
will happen.''

Finally, you will never become an inbox zero person if you treat your
email like a to-do list. It's common to leave messages unread and use
them as reminders to get to certain tasks. The thinking goes: ``If I
need to do it, it's in my inbox. And if I want to take something off my
plate, I'll just send an email about it to someone else,'' Dr. Newport
said. ``That \emph{is} a task management system. It's just a terrible
one.''

Instead, he suggests creating a separate ``space of obligations.'' Use
online tools like \href{https://trello.com/en-US}{Trello},
\href{https://flow-e.com/}{Flow-e} or
\href{https://asana.com/guide/get-started/begin/adding-assigning-tasks}{Asana}
to create task boards that organize your responsibilities according to
urgency and progress. If those aren't for you, Gmail has a task feature
embedded in the calendar app that lets you create digital to-do lists,
while Outlook has a similar feature called To Do. (Its classic task tool
is being phased out.) Or, simply use a pen and paper to outline your
day's priorities.

``The key thing here is low friction,'' Dr. Newport said. ``Get things
written down out of your head.''

\hypertarget{remember-youre-in-charge}{%
\subsection{Remember, you're in
charge.}\label{remember-youre-in-charge}}

There are small lifestyle changes you can make to tune out when needed.
Dr. O'Keefe recommends taking time to examine each of your digital tools
and ask, ``How does this fit in my life?''

If you don't absolutely have to be on call 24/7, snooze professional
email and chat notifications once you sign off for the day. If you have
a smart watch that syncs to your inbox and phone, take it off after
you're done working.

Smartphones are so versatile --- we use them to work, connect and even
\href{https://www.comscore.com/Insights/Press-Releases/2020/8/Comscore-Captures-Notable-Rise-of-Mobile-Vehicle-Shopping}{buy
cars} --- but remember that you're in control of how you use them. Dr.
Literat decided to stop reading and watching shows on her phone, so she
bought a Kindle and started turning on the television more.

If you're guilty of endlessly scrolling on Instagram, Twitter or TikTok,
make it a habit of unfollowing accounts that don't add much value to
your life. Apply the same thinking to your apps --- if you don't use
them at least once a month, get rid of them. With the iPhone's screen
time feature, you can see how much time you spend on your phone every
week and create a schedule to limit your app and call usage.

The bottom line is that you have agency over how often you check your
emails and feeds. It's probably best not to ignore messages from your
boss or students. But with social media, Dr. Bailey said it boils down
to, ``Who will get mad if we're not looking when they want us to be
looking?''

Advertisement

\protect\hyperlink{after-bottom}{Continue reading the main story}

\hypertarget{site-index}{%
\subsection{Site Index}\label{site-index}}

\hypertarget{site-information-navigation}{%
\subsection{Site Information
Navigation}\label{site-information-navigation}}

\begin{itemize}
\tightlist
\item
  \href{https://help.nytimes3xbfgragh.onion/hc/en-us/articles/115014792127-Copyright-notice}{©~2020~The
  New York Times Company}
\end{itemize}

\begin{itemize}
\tightlist
\item
  \href{https://www.nytco.com/}{NYTCo}
\item
  \href{https://help.nytimes3xbfgragh.onion/hc/en-us/articles/115015385887-Contact-Us}{Contact
  Us}
\item
  \href{https://www.nytco.com/careers/}{Work with us}
\item
  \href{https://nytmediakit.com/}{Advertise}
\item
  \href{http://www.tbrandstudio.com/}{T Brand Studio}
\item
  \href{https://www.nytimes3xbfgragh.onion/privacy/cookie-policy\#how-do-i-manage-trackers}{Your
  Ad Choices}
\item
  \href{https://www.nytimes3xbfgragh.onion/privacy}{Privacy}
\item
  \href{https://help.nytimes3xbfgragh.onion/hc/en-us/articles/115014893428-Terms-of-service}{Terms
  of Service}
\item
  \href{https://help.nytimes3xbfgragh.onion/hc/en-us/articles/115014893968-Terms-of-sale}{Terms
  of Sale}
\item
  \href{https://spiderbites.nytimes3xbfgragh.onion}{Site Map}
\item
  \href{https://help.nytimes3xbfgragh.onion/hc/en-us}{Help}
\item
  \href{https://www.nytimes3xbfgragh.onion/subscription?campaignId=37WXW}{Subscriptions}
\end{itemize}
