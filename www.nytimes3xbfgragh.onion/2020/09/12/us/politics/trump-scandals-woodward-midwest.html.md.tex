Sections

SEARCH

\protect\hyperlink{site-content}{Skip to
content}\protect\hyperlink{site-index}{Skip to site index}

\href{https://www.nytimes3xbfgragh.onion/section/politics}{Politics}

\href{https://myaccount.nytimes3xbfgragh.onion/auth/login?response_type=cookie\&client_id=vi}{}

\href{https://www.nytimes3xbfgragh.onion/section/todayspaper}{Today's
Paper}

\href{/section/politics}{Politics}\textbar{}A Big Florida Poll, Nevada
Tightens, Trump on Defense: This Week in the 2020 Race

\begin{itemize}
\item
\item
\item
\item
\item
\end{itemize}

\begin{itemize}
\item
  \href{https://www.nytimes3xbfgragh.onion/live/2020/09/11/us/trump-vs-biden?action=click\&pgtype=Article\&state=default\&region=TOP_BANNER\&context=storylines_menu}{Election
  Updates}
\item
  \href{https://www.nytimes3xbfgragh.onion/interactive/2020/us/elections/election-states-biden-trump.html?action=click\&pgtype=Article\&state=default\&region=TOP_BANNER\&context=storylines_menu}{Paths
  to 270}
\item
  \href{https://www.nytimes3xbfgragh.onion/interactive/2019/us/elections/2020-presidential-election-calendar.html?action=click\&pgtype=Article\&state=default\&region=TOP_BANNER\&context=storylines_menu}{Key
  Dates}
\item
  \href{https://www.nytimes3xbfgragh.onion/interactive/2020/08/31/us/politics/vote-by-mail-deadlines.html?action=click\&pgtype=Article\&state=default\&region=TOP_BANNER\&context=storylines_menu}{Voting
  by Mail}
\item
  \href{https://www.nytimes3xbfgragh.onion/newsletters/politics?action=click\&pgtype=Article\&state=default\&region=TOP_BANNER\&context=storylines_menu}{Politics
  Newsletter}
\end{itemize}

Advertisement

\protect\hyperlink{after-top}{Continue reading the main story}

Supported by

\protect\hyperlink{after-sponsor}{Continue reading the main story}

\hypertarget{a-big-florida-poll-nevada-tightens-trump-on-defense-this-week-in-the-2020-race}{%
\section{A Big Florida Poll, Nevada Tightens, Trump on Defense: This
Week in the 2020
Race}\label{a-big-florida-poll-nevada-tightens-trump-on-defense-this-week-in-the-2020-race}}

As new polling continues to give clues toward the states in play for the
presidential campaign, President Trump was once again at the center of
two story lines that consumed the news.

\includegraphics{https://static01.graylady3jvrrxbe.onion/images/2020/09/12/us/politics/12MOMENTS1/merlin_176737893_33eb7553-a45d-4180-99bc-a891427c903c-articleLarge.jpg?quality=75\&auto=webp\&disable=upscale}

\href{https://www.nytimes3xbfgragh.onion/by/annie-karni}{\includegraphics{https://static01.graylady3jvrrxbe.onion/images/2019/02/05/multimedia/author-annie-karni/author-annie-karni-thumbLarge.png}}\href{https://www.nytimes3xbfgragh.onion/by/astead-w-herndon}{\includegraphics{https://static01.graylady3jvrrxbe.onion/images/2018/09/14/us/author-head-astead/author-head-astead-thumbLarge-v2.png}}

By \href{https://www.nytimes3xbfgragh.onion/by/annie-karni}{Annie Karni}
and \href{https://www.nytimes3xbfgragh.onion/by/astead-w-herndon}{Astead
W. Herndon}

\begin{itemize}
\item
  Sept. 12, 2020, 3:00 a.m. ET
\item
  \begin{itemize}
  \item
  \item
  \item
  \item
  \item
  \end{itemize}
\end{itemize}

\emph{Welcome to our weekly analysis of the state of the 2020 campaign.}

\hypertarget{the-week-in-numbers}{%
\subsection{The week in numbers}\label{the-week-in-numbers}}

\begin{itemize}
\tightlist
\item
  In good news for \textbf{President Trump}, the
  \href{https://cookpolitical.com/analysis/national/national-politics/electoral-college-rating-changes-florida-and-nevada-shift-right}{Cook
  Political Report made two changes to its elections forecast}, moving
  Florida from ``Lean Democrat'' to ``Toss Up,'' and moving Nevada from
  ``Likely Democrat'' to ``Lean Democrat.'' Trump advisers view Florida
  in particular as must-win. The shifts reflect Mr. Biden's potential
  weakness with Latino voters, and Trump's poll numbers stabilizing
  after months of protests after the killing of George Floyd.
\end{itemize}

\begin{itemize}
\item
  The Biden campaign continues to dominate the airwaves, spending
  \textbf{\$32 million} on broadcast television over the past week,
  while the Trump campaign spent only about \textbf{\$10 million}.
  Spending is nearly even on Facebook, as the Biden campaign spent
  \textbf{\$3.7 million} over the past week while the Trump team spent
  \textbf{\$3.2 million} on the platform.
\item
  A
  \href{https://www.monmouth.edu/polling-institute/reports/monmouthpoll_US_091020/}{Monmouth
  University poll} released this week showed Biden holding a
  \textbf{seven-point lead} over Trump among likely voters nationwide.
  Among all registered voters, just \textbf{37 percent} said they were
  certain they would vote for Trump, versus \textbf{43 percent} who were
  sure they would be voting for Biden.
\item
  But an
  \href{http://maristpoll.marist.edu/wp-content/uploads/2020/09/NBC-News_Marist-Poll_FL-Likely-Voters_NOS-and-Tables_2020090712221.pdf\#page=3}{NBC
  News/Marist College survey} of Florida offered some rare positive news
  for the president on the polling front: He and Biden were \textbf{tied
  at 48 percent} each among likely voters in the state, with Trump
  supported by \textbf{50 percent} of Latino voters (albeit a
  particularly hard demographic to accurately poll).
\end{itemize}

\hypertarget{catch-me-up}{%
\subsection{Catch me up}\label{catch-me-up}}

For the president, the week began with him defending himself against a
report in The Atlantic and ended with him defending himself against a
report by the veteran journalist Bob Woodward.

Both story lines --- one about his alleged disrespect for the military,
the other, about purposefully playing down the deadly nature of the
coronavirus --- threatened to undermine his standing with voters whose
support he is counting on, especially servicemembers and seniors. Mr.
Trump, himself, was once again the story, less than 55 days away from
the election --- a time when
\href{https://www.nytimes3xbfgragh.onion/2020/09/10/us/politics/trump-campaign-virus-woodward.html}{veteran
political strategists said} the person who the race is a referendum on
is frequently the person who is losing.

\href{https://www.nytimes3xbfgragh.onion/interactive/2020/us/elections/joe-biden.html}{Mr.
Biden}, the Democratic presidential nominee and former vice president,
tried to capitalize on the negative news reports during an appearance in
Michigan, where he blamed the president's handling of the pandemic for
the ongoing recession. In contrast, Mr. Trump, also in Michigan, tried
to push a message about a great American comeback, complete with the
revival of packed, old-school Trump rallies he's now holding regularly
at airport hangars in battleground states.

Nothing sticks to this president, but with just weeks left before
Election Day, every negative news cycle counts a little bit more. Here's
how this one played out.

\hypertarget{trump-and-woodward}{%
\subsection{Trump and Woodward}\label{trump-and-woodward}}

\includegraphics{https://static01.graylady3jvrrxbe.onion/images/2020/09/12/us/politics/12MOMENTS2/merlin_176810472_3bd78561-258d-45aa-a10d-0882a3253b09-articleLarge.jpg?quality=75\&auto=webp\&disable=upscale}

For Mr. Woodward's first book about the Trump presidency, Mr. Trump did
not participate in the project and (are you sitting down?) there was no
plan from the White House communications department in 2018 to try to
shape the narrative. That left senior officials freelancing, in an
effort to preserve their own reputations, and others speaking to Mr.
Woodward simply out of fear that they would be the only ones who didn't.

For Mr. Woodward's second book, Mr. Trump seems to have overcorrected,
this time participating in 18 freewheeling on-the-record sessions with
the author. ``I gave him some time,'' Mr. Trump told Sean Hannity, the
Fox News host, earlier this week. ``But, as usual, with the books he
writes, that didn't work out too well, perhaps.'' Why did he do it?

\begin{itemize}
\item
  \textbf{Mr. Trump thinks he can charm anyone.} His desire to speak at
  length with Mr. Woodward underscores what has always been the reality
  of Mr. Trump's relationship with the news media, despite shouts of
  ``fake news.'' Mr. Trump loves talking to journalists --- especially
  famous ones --- and is driven, in large part, by his desire to earn
  positive coverage from the establishment.
\item
  \textbf{But he may have been the one charmed,} by Mr. Woodward's
  status (even if he hasn't read his books).
\item
  \textbf{And he doesn't seem to care.} Unlike other authors who have
  written unflattering accounts of the Trump White House, Mr. Woodward
  has yet to receive the book pre-sales bump that typically comes after
  the president denounces an author and their work on Twitter. Mr. Trump
  seems resigned to the fact that he got played, perhaps because most of
  the damaging content appears to come straight from the president's own
  mouth. Instead of denouncing Mr. Woodward, Mr. Trump is defending
  himself.
\end{itemize}

\hypertarget{how-bidens-campaign-responds-to-trumps-scandals}{%
\subsection{How Biden's campaign responds to Trump's
scandals}\label{how-bidens-campaign-responds-to-trumps-scandals}}

Image

Joesph R. Biden Jr. spoke to reporters in Shanksville, Pa., on Sept.
11.Credit...Amr Alfiky/The New York Times

In recent weeks, Mr. Biden has faced a challenge familiar to Hillary
Clinton --- how to weaponize Mr. Trump's scandals. New revelations about
the president's conduct have dominated headlines and cable news chyrons,
including his disparaging comments about members of the military
reported in The Atlantic, as has the book by Mr. Woodward and another by
Michael Cohen, Mr. Trump's former personal lawyer and confidante.

But making those stories last is hard, and breaking through to voters is
even harder. Here's how Mr. Biden is trying:

\begin{itemize}
\item
  \textbf{Dispatch surrogates, not the candidate:} Following the release
  of The Atlantic article and highlights from Mr. Woodward's book, Mr.
  Biden's campaign held a media conference call with high-profile
  surrogates, including Senators Tammy Duckworth of Illinois and Sherrod
  Brown of Ohio. In doing so, the campaign sought to prolong a news
  cycle on damaging information to its opponent. Biden advisers also
  continue a tack they've pursued since Mr. Biden became the nominee:
  While his primary campaign was focused on Mr. Trump and electability,
  his general election strategy has often left attacking the president
  to others.
\item
  \textbf{Focus on the virus:} When Mr. Biden does target Mr. Trump, it
  has usually been on issues with which he feels most comfortable. He
  has tried to make this election a referendum on how Mr. Trump has
  handled the pandemic, and has weaponized new information that bolsters
  his argument that the administration shirked its responsibility. But
  the campaign has stayed away from the more gossip-driven elements that
  animate Mr. Trump's opponents on social media. Books like the one
  written by Mr. Trump's niece, Mary Trump, and Mr. Cohen's account of
  his time with the president have rarely found their way into Mr.
  Biden's campaign messaging.
\item
  \textbf{Presidential contrast:} Unlike Mrs. Clinton, who was dealing
  with the possibility of Mr. Trump becoming president, Mr. Biden is
  dealing with the reality. And as the scandals have continued into his
  administration, Democrats believe that voters who were willing to take
  a chance on Mr. Trump changing in office are now ready for a course
  correction. This is another element of how the Biden campaign seeks to
  use Mr. Trump's words against him, by arguing that Mr. Biden would
  bring calm and stability to the White House, rather than the stream of
  norm-busting headlines.
\end{itemize}

\hypertarget{both-campaigns-agree-the-midwest-is-best}{%
\subsection{Both campaigns agree: The Midwest is
best}\label{both-campaigns-agree-the-midwest-is-best}}

Image

Mr. Biden met local residents during a campaign stop in
Detroit.Credit...Amr Alfiky/The New York Times

With both candidates in Michigan this week, and top surrogates including
Donald Trump Jr. and Jill Biden in Minnesota, the travel was a sign of
how much attention both campaigns are paying to the Midwest. The intense
interest in Michigan, Minnesota, Wisconsin and Pennsylvania has dwarfed
other regions. There are many ways to get to 270 electoral votes, but
here's why Mr. Trump and Mr. Biden think this campaign will be won in
the country's industrial center.

\begin{itemize}
\item
  \textbf{White working-class:} Mr. Trump and Mr. Biden are both figures
  who have staked their appeal on having a specific connection with
  white working-class voters, a demographic that was not enthusiastic
  about Mrs. Clinton's candidacy. Advisers to Mr. Biden believe that's a
  population their candidate is better set to succeed with, and states
  with industrial backgrounds like Wisconsin and Pennsylvania are a good
  way to test that appeal.
\item
  \textbf{Black voters:} There are more Latino voters in the general
  election, but Democrats and Republicans have probably spent more time
  focusing on Black voters in this election than any other minority
  group. Mr. Biden has leaned on his personal connection with former
  President Barack Obama, and Republicans have pitched Democrats as
  irresponsible stewards of Black urban communities. More than other
  battleground states like Florida or out West, the industrial states
  have cities with Black turnout that could determine the statewide
  totals. These include places like Milwaukee, Philadelphia and
  Cleveland.
\item
  \textbf{Mr. Biden's campaign is not expanding the map:} Early in the
  race, some Democratic operatives pleaded with the Biden campaign to
  expand the traditional battleground map and invest in states like
  Texas and Georgia that have had demographic shifts beneficial to
  Democrats. However, if the candidate's travel schedule is any
  indication, the campaign is focusing efforts on the traditional
  battlegrounds --- for now. Mr. Biden's campaign just announced another
  Midwest trip, to Minnesota, in the coming week. It shows a willingness
  to defend states Mrs. Clinton won in 2016 over expanding the map to
  new states that have long proved fool's gold for the party.
\end{itemize}

\hypertarget{what-you-might-have-missed}{%
\subsection{What you might have
missed}\label{what-you-might-have-missed}}

\begin{itemize}
\item
  A week after Mr. Trump suggested that voters in North Carolina should
  cast two ballots --- one by mail and another at the polls --- the
  authorities in Georgia are
  \href{https://www.nytimes3xbfgragh.onion/2020/09/08/us/politics/georgia-double-voting.html}{threatening
  criminal action against 1,000 Georgia voters who did just that}.
\item
  Intentionally voting twice in a federal election could result in a
  \$10,000 fine and up to five years in prison. In the past, few people
  have done it,
  \href{https://www.nytimes3xbfgragh.onion/article/voting-twice.html}{or
  have had to pay the full fine}.
\item
  The Trump campaign has spent over \$800 million of their \$1.1 billion
  budget since July 2019. Some say the money was blown on small costs,
  \href{https://www.nytimes3xbfgragh.onion/2020/09/07/us/politics/trump-election-campaign-fundraising.html}{like
  buying \$110,000 worth of magnetic pouches} that are used to store
  cellphones during fund-raisers so that donors could not secretly
  record Mr. Trump.
\end{itemize}

\hypertarget{our-2020-election-guide}{%
\section{Our 2020 Election Guide}\label{our-2020-election-guide}}

Updated ~Sept. 11, 2020

\begin{center}\rule{0.5\linewidth}{\linethickness}\end{center}

\begin{itemize}
\item ~
  \hypertarget{the-latest}{%
  \subsection{The Latest}\label{the-latest}}

  \begin{itemize}
  \item
    Joe Biden and President Trump put
    \href{https://www.nytimes3xbfgragh.onion/2020/09/11/us/politics/shanksville-trump-biden.html?action=click\&pgtype=Article\&state=default\&region=BELOW_MAIN_CONTENT\&context=storylines_guide}{hostilities
    on hold today to travel to ground zero and then to Shanksville, Pa.,
    where they separately honored 9/11 victims}.
  \end{itemize}
\item ~
  \hypertarget{how-to-win-270}{%
  \subsection{How to Win 270}\label{how-to-win-270}}

  \begin{itemize}
  \item
    Joe Biden and Donald Trump need 270 electoral votes to reach the
    White House. Try building
    \href{https://www.nytimes3xbfgragh.onion/interactive/2020/us/elections/election-states-biden-trump.html?action=click\&pgtype=Article\&state=default\&region=BELOW_MAIN_CONTENT\&context=storylines_guide}{your
    own coalition of battleground states}~to see potential outcomes.
  \end{itemize}
\item ~
  \hypertarget{voting-by-mail}{%
  \subsection{Voting by Mail}\label{voting-by-mail}}

  \begin{itemize}
  \item
    Will you have enough time to vote by mail in your state? Yes, but
    it's risky to procrastinate.
    \href{https://www.nytimes3xbfgragh.onion/interactive/2020/08/31/us/politics/vote-by-mail-deadlines.html?action=click\&pgtype=Article\&state=default\&region=BELOW_MAIN_CONTENT\&context=storylines_guide}{Check
    your state's deadline.}
  \item
    \href{https://www.nytimes3xbfgragh.onion/interactive/2020/us/elections/joe-biden.html?action=click\&pgtype=Article\&state=default\&region=BELOW_MAIN_CONTENT\&context=storylines_guide}{}

    \hypertarget{joe-biden}{%
    \section{Joe Biden}\label{joe-biden}}

    \hypertarget{democrat}{%
    \subsection{Democrat}\label{democrat}}

    \href{https://www.nytimes3xbfgragh.onion/interactive/2020/us/elections/donald-trump.html?action=click\&pgtype=Article\&state=default\&region=BELOW_MAIN_CONTENT\&context=storylines_guide}{}

    \hypertarget{donald-trump}{%
    \section{Donald Trump}\label{donald-trump}}

    \hypertarget{republican}{%
    \subsection{Republican}\label{republican}}
  \end{itemize}
\item
  \hypertarget{keep-up-with-our-coverage}{%
  \subsection{Keep Up With Our
  Coverage}\label{keep-up-with-our-coverage}}

  \begin{itemize}
  \item
    Get an
    \href{https://www.nytimes3xbfgragh.onion/newsletters/politics?action=click\&pgtype=Article\&state=default\&region=BELOW_MAIN_CONTENT\&context=storylines_guide}{email}~recapping
    the day's news
  \item
    Download our mobile app on
    \href{https://apps.apple.com/us/app/nytimes/id284862083?ls=1\&mat_click_id=5c79ae7455014fd1bd66b5610c05b8f2-20191112-16948\&referrer=mat_click_id\%3D5c79ae7455014fd1bd66b5610c05b8f2-20191112-16948\%26link_click_id\%3D722930677036718082}{iOS}~and
    \href{http://a.localytics.com/android?id=com.nytimes.android\&referrer=utm_source\%3Dother_nyt_mobile_web\%26utm_medium\%3DWeb\%2520page\%26utm_term\%3DGeneral\%2520Mobile\%2520Page\%26utm_campaign\%3DNYT\%2520Mobile\%2520General\%2520Page}{Android}~and
    turn on Breaking News and Politics alerts
  \end{itemize}
\end{itemize}

Advertisement

\protect\hyperlink{after-bottom}{Continue reading the main story}

\hypertarget{site-index}{%
\subsection{Site Index}\label{site-index}}

\hypertarget{site-information-navigation}{%
\subsection{Site Information
Navigation}\label{site-information-navigation}}

\begin{itemize}
\tightlist
\item
  \href{https://help.nytimes3xbfgragh.onion/hc/en-us/articles/115014792127-Copyright-notice}{©~2020~The
  New York Times Company}
\end{itemize}

\begin{itemize}
\tightlist
\item
  \href{https://www.nytco.com/}{NYTCo}
\item
  \href{https://help.nytimes3xbfgragh.onion/hc/en-us/articles/115015385887-Contact-Us}{Contact
  Us}
\item
  \href{https://www.nytco.com/careers/}{Work with us}
\item
  \href{https://nytmediakit.com/}{Advertise}
\item
  \href{http://www.tbrandstudio.com/}{T Brand Studio}
\item
  \href{https://www.nytimes3xbfgragh.onion/privacy/cookie-policy\#how-do-i-manage-trackers}{Your
  Ad Choices}
\item
  \href{https://www.nytimes3xbfgragh.onion/privacy}{Privacy}
\item
  \href{https://help.nytimes3xbfgragh.onion/hc/en-us/articles/115014893428-Terms-of-service}{Terms
  of Service}
\item
  \href{https://help.nytimes3xbfgragh.onion/hc/en-us/articles/115014893968-Terms-of-sale}{Terms
  of Sale}
\item
  \href{https://spiderbites.nytimes3xbfgragh.onion}{Site Map}
\item
  \href{https://help.nytimes3xbfgragh.onion/hc/en-us}{Help}
\item
  \href{https://www.nytimes3xbfgragh.onion/subscription?campaignId=37WXW}{Subscriptions}
\end{itemize}
