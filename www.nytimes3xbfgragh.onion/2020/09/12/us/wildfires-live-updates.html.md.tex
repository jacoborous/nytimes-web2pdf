Sections

SEARCH

\protect\hyperlink{site-content}{Skip to
content}\protect\hyperlink{site-index}{Skip to site index}

\href{https://www.nytimes3xbfgragh.onion/section/us}{U.S.}

\href{https://myaccount.nytimes3xbfgragh.onion/auth/login?response_type=cookie\&client_id=vi}{}

\href{https://www.nytimes3xbfgragh.onion/section/todayspaper}{Today's
Paper}

\href{/section/us}{U.S.}\textbar{}Wildfires Live Updates: Dozens Missing
as States Look to Weather for Relief

\url{https://nyti.ms/3c3pNlL}

\begin{itemize}
\item
\item
\item
\item
\item
\item
\end{itemize}

\hypertarget{wildfires-in-the-west}{%
\subsubsection{\texorpdfstring{\href{https://www.nytimes3xbfgragh.onion/spotlight/california-wildfires?name=styln-california-wildfires\&region=TOP_BANNER\&block=storyline_menu_recirc\&action=click\&pgtype=Article\&impression_id=7e9f47d0-f52e-11ea-aca2-5fcae7f3caf4\&variant=undefined}{Wildfires
in the West}}{Wildfires in the West}}\label{wildfires-in-the-west}}

\begin{itemize}
\tightlist
\item
  \href{https://www.nytimes3xbfgragh.onion/2020/09/12/us/wildfires-live-updates.html?name=styln-california-wildfires\&region=TOP_BANNER\&block=storyline_menu_recirc\&action=click\&pgtype=Article\&impression_id=7e9f47d1-f52e-11ea-aca2-5fcae7f3caf4\&variant=undefined}{Fires
  Updates}
\item
  \href{https://www.nytimes3xbfgragh.onion/interactive/2020/us/fires-map-tracker.html?name=styln-california-wildfires\&region=TOP_BANNER\&block=storyline_menu_recirc\&action=click\&pgtype=Article\&impression_id=7e9f47d2-f52e-11ea-aca2-5fcae7f3caf4\&variant=undefined}{Maps
  of the Fires}
\item
  \href{https://www.nytimes3xbfgragh.onion/article/wildfires-photos-california-oregon-washington-state.html?name=styln-california-wildfires\&region=TOP_BANNER\&block=storyline_menu_recirc\&action=click\&pgtype=Article\&impression_id=7e9f47d3-f52e-11ea-aca2-5fcae7f3caf4\&variant=undefined}{Photos}
\item
  \href{https://www.nytimes3xbfgragh.onion/2020/09/10/us/climate-change-california-wildfires.html?name=styln-california-wildfires\&region=TOP_BANNER\&block=storyline_menu_recirc\&action=click\&pgtype=Article\&impression_id=7e9f47d4-f52e-11ea-aca2-5fcae7f3caf4\&variant=undefined}{A
  Climate Reckoning}
\item
  \href{https://www.nytimes3xbfgragh.onion/article/wildfires-california-oregon-washington.html?name=styln-california-wildfires\&region=TOP_BANNER\&block=storyline_menu_recirc\&action=click\&pgtype=Article\&impression_id=7e9f47d5-f52e-11ea-aca2-5fcae7f3caf4\&variant=undefined}{Answers
  to Your Questions}
\item
  \href{https://www.nytimes3xbfgragh.onion/2020/09/09/us/california-wildfires.html?name=styln-california-wildfires\&region=TOP_BANNER\&block=storyline_menu_recirc\&action=click\&pgtype=Article\&impression_id=7e9f47d6-f52e-11ea-aca2-5fcae7f3caf4\&variant=undefined}{Newsletter}
\end{itemize}

Advertisement

\protect\hyperlink{after-top}{Continue reading the main story}

Supported by

\protect\hyperlink{after-sponsor}{Continue reading the main story}

LIVE UPDATES

Updated~

Sept. 12, 2020, 2:53 p.m. ET

Sept. 12, 2020, 2:53 p.m. ET

\hypertarget{wildfires-live-updates-dozens-missing-as-states-look-to-weather-for-relief}{%
\section{Wildfires Live Updates: Dozens Missing as States Look to
Weather for
Relief}\label{wildfires-live-updates-dozens-missing-as-states-look-to-weather-for-relief}}

President Trump will visit California on Monday to be briefed about
blazes that have burned more than three million acres. Tens of thousands
of people have evacuated in Oregon.

Right Now

Oregon, where crews have struggled to contain wildfires, put its state
fire marshal on paid administrative leave.

\hypertarget{heres-what-you-need-to-know}{%
\subsubsection{Here's what you need to
know:}\label{heres-what-you-need-to-know}}

\begin{itemize}
\tightlist
\item
  \protect\hyperlink{link-f3961ff}{President Trump will visit California
  on Monday after destructive fires.}
\item
  \protect\hyperlink{link-7e503ae9}{Shifting weather may improve
  firefighting conditions on the West Coast.}
\item
  \protect\hyperlink{link-5e4c548d}{Oregon's fire marshal is temporarily
  replaced as firefighters battle blazes.}
\item
  \protect\hyperlink{link-7a0c7fbe}{False rumors are complicating the
  fight against the fires around Portland.}
\item
  \protect\hyperlink{link-18416023}{A man is charged with arson in a
  southern Oregon blaze.}
\item
  \protect\hyperlink{link-5b86b2c4}{`This is a fathomless loss': Some
  searches for the missing end in tragedy.}
\item
  \protect\hyperlink{link-7a6a9ef4}{Wildfire smoke is dangerous to your
  health. Here's how to protect yourself.}
\end{itemize}

\includegraphics{https://static01.graylady3jvrrxbe.onion/images/2020/09/12/us/12fires-briefing-lede/12fires-briefing-lede-videoSixteenByNine3000.jpg}

\subsection{}

President Trump will visit California on Monday after destructive fires.

Dozens of people remain missing in Oregon as wildfires that have torched
millions of acres across the West continue to burn on Saturday, the
death toll rising to 17 and smoke choking residents in cities far from
the fires.

With the blazes still spreading and many homes destroyed, Oregon's
director of emergency management said this week that the state feared a
``mass fatality incident.'' Three additional deaths in the state were
announced after his warning.

But as residents prepared for more pain, they also hoped that changing
weather might help them this weekend. Doug Grafe, chief of Fire
Protection for the Oregon Department of Forestry, said that the strong
winds that had spread the fires had dissipated, and that cooler
temperatures and higher humidity would help fire crews fight the blazes.

Oregon, Washington and California are all under assault from a wildfire
season of historic proportions, with the firefighting effort compounded
by the coronavirus pandemic and
\href{https://www.nytimes3xbfgragh.onion/2020/09/10/us/wildfires-misinformation-arson-activists.html}{misinformation
online}. President Trump will visit McClellan Park, Calif., on Monday to
be briefed on the wildfires, the White House announced.

Senators Ron Wyden and Jeff Merkley, both Democrats of Oregon, and
Representative Greg Walden, a Republican whose district includes the
site of the Almeda Fire, visited an evacuee relief site in Central Point
on Saturday morning. The congressmen were scheduled to tour the towns of
Phoenix and Talent, which were devastated by the blaze.

\includegraphics{https://static01.graylady3jvrrxbe.onion/images/2020/09/11/us/politics/Screen-Shot-2020-09-11-at-4/Screen-Shot-2020-09-11-at-4-videoSixteenByNineJumbo1600.png}

The fires in Oregon have burned more than 1 million acres --- a larger
area than Rhode Island --- and the state's air quality ranks among the
worst in the world. Tens of thousands of people have already been
evacuated, and about 500,000 are in areas that may be ordered to flee.

``Almost anywhere in the state you can feel this right now,'' Gov. Kate
Brown said.

In Washington, where fires have burned more than 626,000 acres this
week, Gov. Jay Inslee said the state was suffering ``a cataclysmic
event.''

California has seen more than 3.1 million acres go up in flames, about
26 times as much as had burned at this point last year, and officials
warn that more fires are likely. One of the fire complexes burning this
week became the largest in the state's history this week, having burned
across 747,000 acres.

``It's just something we've never seen in our lifetime,'' Gov. Gavin
Newsom said on Friday, standing amid charred trees and a yellow haze of
smoke left by the raging fires.

\emph{{[}Sign up}
\href{https://www.nytimes3xbfgragh.onion/newsletters/california-today}{\emph{for
California Today}}\emph{, our daily newsletter from the Golden
State.{]}}

\href{https://www.nytimes3xbfgragh.onion/interactive/2020/us/fires-map-tracker.html}{}

\includegraphics{https://static01.graylady3jvrrxbe.onion/images/2020/09/11/us/fires-map-tracker-1599839565497/fires-map-tracker-1599839565497-articleLarge.png}

\hypertarget{california-oregon-and-washington-fire-tracking-maps}{%
\subsection{California, Oregon and Washington Fire Tracking
Maps}\label{california-oregon-and-washington-fire-tracking-maps}}

Maps showing air quality and where major fires are burning in the
Western states.

\hypertarget{-1}{%
\subsection{}\label{-1}}

Shifting weather may improve firefighting conditions on the West Coast.

Dry, windy conditions that have worsened fires and vexed firefighters
this season were expected to continue in parts of the West on Saturday,
but officials say they are hopeful that wetter weather in the coming
days may allow them to contain the fires.

The National Weather Service issued a ``red flag'' warning for
\href{https://www.wrh.noaa.gov/map/?\&zoom=7\&scroll_zoom=true\&center=42.28950073090457,-123.72802734375\&basemap=OpenStreetMap\&boundaries=true,true\&hazard=true\&hazard_type=hi-all\&hazard_opacity=60}{parts}
of southern Oregon and Northern California through the weekend and said
that a combination of gusty winds and low humidity could allow for ``a
significant spread of new and existing fires.''

Dense smoke has also made firefighting harder, grounding planes used to
drop retardants meant to slow the fires.

``The air's not good for doing anything right now,'' Mark Bertuccelli, a
pilot with the United States Forest Service
\href{https://abc30.com/creek-fire-air-drops-firefight-pilot-smoke-firefighter-from-above/6417975/}{told
KFSN-TV in Fresno}.

Yet California's statewide fire agency, Cal Fire, said fire conditions
have improved and there could be good news ahead as the air gets more
humid.

Forecasters in the National Weather Service's Portland office were
similarly optimistic, saying that rain could fall in parts of Oregon and
Washington early next week and that the smoke could clear as air from
over the ocean blows over the states.

\hypertarget{-2}{%
\subsection{}\label{-2}}

Oregon's fire marshal is temporarily replaced as firefighters battle
blazes.

The Oregon State Police said on Saturday that the state fire marshal had
been replaced by a deputy, an abrupt change in leadership that came as
crews struggled to contain wildfires that have consumed more than 1
million acres.

Jim Walker, who has served as state fire marshal since 2014, was placed
on paid administrative leave on Friday night, the State Police said.
Mariana Ruiz-Temple, who has worked for the agency since 1995, was
appointed as acting fire marshal.

In Oregon, the state fire marshal is responsible for commanding the
state fire service, advising the governor, and overseeing code
enforcement and inspections.

A spokesman for the State Police said it was ``conducting an internal
personnel investigation'' in conjunction with Mr. Walker's leave. Travis
Hampton, the State Police superintendent, did not explain in a prepared
statement why Mr. Walker was replaced.

``Mariana is assuming this position as Oregon is in an unprecedented
crisis which demands an urgent response,'' Superintendent Hampton said.
``This response and the circumstances necessitated a leadership
change.''

\hypertarget{-3}{%
\subsection{}\label{-3}}

False rumors are complicating the fight against the fires around
Portland.

\includegraphics{https://static01.graylady3jvrrxbe.onion/images/2020/09/11/us/11FIRES-BRIEFING-rumors2/merlin_176834859_604e8de9-36fe-4f66-969d-3ae87f7954f5-articleLarge.jpg?quality=75\&auto=webp\&disable=upscale}

As misinformation about the origin of the wildfires in Oregon continued
to spread on social media, a civilian blockade popped up outside
Portland, with residents setting up checkpoints and stopping cars coming
into the area, the Multnomah County Sheriff's Office said.

``While we understand their intent is to keep the community safe, it is
never legal to block a public roadway or force other citizens to stop,''
\href{https://twitter.com/MultCoSO/status/1304631505314930688}{the
Sheriff's Office posted on Twitter} late Friday. ``Please report
suspicious activity to us and do not take action yourselves.''

Widely discredited rumors claiming that left-wing activists had set
fires have spread on Facebook and Twitter, prompting some residents to
contemplate dismissing evacuation orders, saying they want to protect
their homes.

As a Level 3 evacuation this week urged people to ``leave now,'' an
eerie stillness fell over Molalla, an old timber town of 9,000 an hour's
drive south of Portland, and the holdout residents girded themselves for
two threats. One was the very real 130,000-acre Riverside Fire burning
just east of town. The other was the imagined invasion of left-wing mobs
and arsonists that multiple
\href{https://www.nytimes3xbfgragh.onion/2020/09/10/us/antifa-wildfires.html}{law
enforcement agencies have sought to refute}.

Residents who remained hosed down their roofs and soaked their lawns.
They organized go-bags of baby supplies and clothes, just in case. They
scouted for unfamiliar cars on the roads.

``I'm protecting my city,'' Troy McNeeley said as he stood in front of
the 900-square foot home he shares with his son, his son's partner and
several cats. ``If I see people doing crap, I'm going to hurt them.''

The rumors played into some conservative residents'
\href{https://www.nytimes3xbfgragh.onion/2020/09/05/us/portland-political-chasm-protests-unrest.html}{fears
and anger} over months of protests in Portland, where left-wing and
right-wing groups
\href{https://www.nytimes3xbfgragh.onion/2020/08/22/us/portland-protests.html}{have
occasionally clashed}.

On Wednesday, the police in Portland warned protesters about lighting
fires --- a seemingly innocuous public-safety message that was followed
by waves of rumor about arsonists and mayhem. Sheriff's offices and fire
departments already coping with wildfires that have consumed 900,000
acres were flooded with phone calls.

``We are inundated with questions about things that are FAKE stories,''
the Jackson County Sheriff's Office in Medford posted on Facebook. ``One
example is a story circulating that varies about what group is involved
as to setting fires and arrests being made. THIS IS NOT TRUE!''

\hypertarget{-4}{%
\subsection{}\label{-4}}

A man is charged with arson in a southern Oregon blaze.

Image

A man was charged with arson on Friday after residents said he had
started a blaze during the Almeda Fire, which destroyed much of Phoenix,
Ore.Credit...Gillian Flaccus/Associated Press

Authorities in southern Oregon charged a 41-year-old man with starting
part of one of this year's most destructive fires, saying he lit the
fire in a small Oregon town as a larger blaze moved toward the area.

The Jackson County Sheriff's Office said the Almeda Fire started around
11 a.m. on Tuesday in Ashland, Ore., and then began spreading north
along Interstate 5. Around 5 p.m., residents reported that a man had
started a fire in Phoenix, a town of about 4,500 people just north of
Ashland that was under orders to evacuate, authorities said.

The Sheriff's Office said police officers discovered Michael Jarrod
Bakkela at the scene, denying that he started the large fire nearby.
Police arrested him on a parole violation.

On Friday, the Jackson County district attorney charged him with arson,
criminal mischief and reckless endangering.

Mr. Bakkela, who could not be reached, had not yet been appointed a
lawyer, said Beth Heckert, the county's district attorney. He was
scheduled to be arraigned in court on Monday, she said.

Mike Moran, a public information officer for the Jackson County sheriff,
said Mr. Bakkela had a criminal record and was well known to local law
enforcement. A news release from the Sheriff's Office described him as
``a local transient.''

While many wildfires on the West Coast this year have burned through
remote areas and parts of rural communities, the Almeda Fire hit a
series of towns along the freeway in southern Oregon, destroying an
estimated 500 homes and 100 businesses. Mr. Moran said authorities were
still investigating the fire's initial point of origin in Ashland. He
said that they suspected arson there, too, and that they found the
remains of one man near the fire's start.

\hypertarget{-5}{%
\subsection{}\label{-5}}

`This is a fathomless loss': Some searches for the missing end in
tragedy.

As the blazes rage across California, Oregon and Washington, family and
friends are desperately searched for missing loved ones who remained
unaccounted for.

Zygy Roe-Zurz, whose family lives in Berry Creek, Calif., said he had
been waiting for days for news from his mother, his aunt and his uncle.
On Thursday, he learned that his aunt was killed as the Bear Fire ripped
through the community, and that his mother remained missing. Authorities
told the family that Mr. Roe-Zurz's uncle was likely dead as well, he
said.

``I feel barren --- this is a fathomless loss and I will never be the
same,'' said Mr. Roe-Zurz, 37, who is in Arkansas and last spoke to his
mother on Tuesday night, before the flames intensified. ``This cruel
fire took everything.''

He said that his family members staying at the property in Berry Creek
had been under the impression that the fire was getting under control,
but that the situation changed dramatically as the Bear Fire jumped an
astonishing 230,000 acres overnight Tuesday into Wednesday.

``It's pretty much a nightmare scenario,'' Mr. Roe-Zurz said. ``I'm
devastated.''

There was better news for other families who found out that loved ones
they believed to be missing were found safe on Thursday.

Katy Carmel said her daughter, Natalie Anderson, had been on a camping
trip with her boyfriend near the McKenzie Bridge east of Eugene, Ore.
But when the Holiday Farm Fire broke out on Monday evening, Ms. Carmel
could no longer reach Ms. Anderson.

Ms. Carmel could not sleep, fearing the worst. Days passed and the
anxiety built. On Thursday, authorities notified the families that both
Ms. Anderson and her boyfriend, Enmanuel Rodriguez, were safe and
evacuated.

Ms. Carmel said she was relieved to hear the news, but added, ``I'll be
better once she's actually home.''

\hypertarget{-6}{%
\subsection{}\label{-6}}

Wildfire smoke is dangerous to your health. Here's how to protect
yourself.

Image

Heavy smoke conditions in Clackamas County, Ore., on
Thursday.Credit...Kristina Barker for The New York Times

Smoke from wildfires, which can include toxic substances from burned
buildings, has been linked to serious health problems.

Studies have shown that when waves of smoke hit,
\href{https://insights.ovid.com/epidemiology/epide/2017/01/000/wildfire-specific-fine-particulate-matter-risk/13/00001648}{the
rate of hospital visits rises} and many of the additional patients
experience
\href{https://www.ncbi.nlm.nih.gov/pmc/articles/PMC6015400/}{respiratory
problems, heart attacks and strokes}.

If you cannot leave an area that has high levels of smoke, the Centers
for Disease Control and Prevention recommend
\href{https://www3.epa.gov/airnow/smoke_fires/prepare-for-fire-season-508.pdf}{limiting
exposure} by staying indoors with windows and doors closed and running
air-conditioners in recirculation mode so that outside air is not drawn
into your home.

Portable air purifiers are also recommended, though, like
air-conditioners, they require electricity. If utilities cut off power,
\href{https://www.nytimes3xbfgragh.onion/2020/08/18/us/california-blackouts.html}{as
has happened in California}, those options are limited.

Experts say it is especially important to avoid cigarettes. They also
recommend avoiding strenuous outdoor activities when the air is bad.
When outside, well-fitted N95 masks are also recommended, though they
are in short supply because of the coronavirus pandemic.

Reporting was contributed by Davey Alba, Tim Arango, Mike Baker,
Nicholas Bogel-Burroughs, Maria Cramer, Kate Conger, Jill Cowan, Richard
Fausset, Marie Fazio, Christopher Flavelle, Thomas Fuller, Jack Healy,
Annie Karni, Giulia McDonnell Nieto del Rio, Jack Nicas, Bryan Pietsch,
John Schwartz, Will Wright and Alan Yuhas.

Advertisement

\protect\hyperlink{after-bottom}{Continue reading the main story}

\hypertarget{site-index}{%
\subsection{Site Index}\label{site-index}}

\hypertarget{site-information-navigation}{%
\subsection{Site Information
Navigation}\label{site-information-navigation}}

\begin{itemize}
\tightlist
\item
  \href{https://help.nytimes3xbfgragh.onion/hc/en-us/articles/115014792127-Copyright-notice}{©~2020~The
  New York Times Company}
\end{itemize}

\begin{itemize}
\tightlist
\item
  \href{https://www.nytco.com/}{NYTCo}
\item
  \href{https://help.nytimes3xbfgragh.onion/hc/en-us/articles/115015385887-Contact-Us}{Contact
  Us}
\item
  \href{https://www.nytco.com/careers/}{Work with us}
\item
  \href{https://nytmediakit.com/}{Advertise}
\item
  \href{http://www.tbrandstudio.com/}{T Brand Studio}
\item
  \href{https://www.nytimes3xbfgragh.onion/privacy/cookie-policy\#how-do-i-manage-trackers}{Your
  Ad Choices}
\item
  \href{https://www.nytimes3xbfgragh.onion/privacy}{Privacy}
\item
  \href{https://help.nytimes3xbfgragh.onion/hc/en-us/articles/115014893428-Terms-of-service}{Terms
  of Service}
\item
  \href{https://help.nytimes3xbfgragh.onion/hc/en-us/articles/115014893968-Terms-of-sale}{Terms
  of Sale}
\item
  \href{https://spiderbites.nytimes3xbfgragh.onion}{Site Map}
\item
  \href{https://help.nytimes3xbfgragh.onion/hc/en-us}{Help}
\item
  \href{https://www.nytimes3xbfgragh.onion/subscription?campaignId=37WXW}{Subscriptions}
\end{itemize}
