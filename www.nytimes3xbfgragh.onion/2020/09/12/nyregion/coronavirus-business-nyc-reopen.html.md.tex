Sections

SEARCH

\protect\hyperlink{site-content}{Skip to
content}\protect\hyperlink{site-index}{Skip to site index}

\href{https://www.nytimes3xbfgragh.onion/section/nyregion}{New York}

\href{https://myaccount.nytimes3xbfgragh.onion/auth/login?response_type=cookie\&client_id=vi}{}

\href{https://www.nytimes3xbfgragh.onion/section/todayspaper}{Today's
Paper}

\href{/section/nyregion}{New York}\textbar{}Inside the Clash Between
Powerful Business Leaders and N.Y.C.'s Mayor

\begin{itemize}
\item
\item
\item
\item
\item
\item
\end{itemize}

\hypertarget{the-coronavirus-outbreak}{%
\subsubsection{\texorpdfstring{\href{https://www.nytimes3xbfgragh.onion/news-event/coronavirus?name=styln-coronavirus-national\&region=TOP_BANNER\&block=storyline_menu_recirc\&action=click\&pgtype=Article\&impression_id=49c370a0-f52d-11ea-bb3a-a9cd15cea7c9\&variant=undefined}{The
Coronavirus
Outbreak}}{The Coronavirus Outbreak}}\label{the-coronavirus-outbreak}}

\begin{itemize}
\tightlist
\item
  live\href{https://www.nytimes3xbfgragh.onion/2020/09/12/world/covid-19-coronavirus.html?name=styln-coronavirus-national\&region=TOP_BANNER\&block=storyline_menu_recirc\&action=click\&pgtype=Article\&impression_id=49c397b0-f52d-11ea-bb3a-a9cd15cea7c9\&variant=undefined}{Latest
  Updates}
\item
  \href{https://www.nytimes3xbfgragh.onion/interactive/2020/us/coronavirus-us-cases.html?name=styln-coronavirus-national\&region=TOP_BANNER\&block=storyline_menu_recirc\&action=click\&pgtype=Article\&impression_id=49c397b1-f52d-11ea-bb3a-a9cd15cea7c9\&variant=undefined}{Maps
  and Cases}
\item
  \href{https://www.nytimes3xbfgragh.onion/interactive/2020/science/coronavirus-vaccine-tracker.html?name=styln-coronavirus-national\&region=TOP_BANNER\&block=storyline_menu_recirc\&action=click\&pgtype=Article\&impression_id=49c397b2-f52d-11ea-bb3a-a9cd15cea7c9\&variant=undefined}{Vaccine
  Tracker}
\item
  \href{https://www.nytimes3xbfgragh.onion/2020/09/10/us/politics/fda-coronavirus-vaccine.html?name=styln-coronavirus-national\&region=TOP_BANNER\&block=storyline_menu_recirc\&action=click\&pgtype=Article\&impression_id=49c397b3-f52d-11ea-bb3a-a9cd15cea7c9\&variant=undefined}{F.D.A.
  Regulators' Self-Defense}
\item
  \href{https://www.nytimes3xbfgragh.onion/2020/09/09/upshot/coronavirus-surprise-test-fees.html?name=styln-coronavirus-national\&region=TOP_BANNER\&block=storyline_menu_recirc\&action=click\&pgtype=Article\&impression_id=49c397b4-f52d-11ea-bb3a-a9cd15cea7c9\&variant=undefined}{Surprise
  Test Fees}
\end{itemize}

Advertisement

\protect\hyperlink{after-top}{Continue reading the main story}

Supported by

\protect\hyperlink{after-sponsor}{Continue reading the main story}

\hypertarget{inside-the-clash-between-powerful-business-leaders-and-nycs-mayor}{%
\section{Inside the Clash Between Powerful Business Leaders and N.Y.C.'s
Mayor}\label{inside-the-clash-between-powerful-business-leaders-and-nycs-mayor}}

The tensions burst into the open when 163 executives joined to criticize
Bill de Blasio's leadership. Others think their portrait of the city is
overly bleak.

\includegraphics{https://static01.graylady3jvrrxbe.onion/images/2020/09/13/nyregion/13NYVIRUSMAYOR-print/merlin_174842076_98d1840b-4c2d-4c04-bdb6-353f523dee56-articleLarge.jpg?quality=75\&auto=webp\&disable=upscale}

\href{https://www.nytimes3xbfgragh.onion/by/j-david-goodman}{\includegraphics{https://static01.graylady3jvrrxbe.onion/images/2018/07/18/nyregion/author-j-david-goodman/author-j-david-goodman-thumbLarge.png}}\href{https://www.nytimes3xbfgragh.onion/by/emma-g-fitzsimmons}{\includegraphics{https://static01.graylady3jvrrxbe.onion/images/2018/07/18/multimedia/author-emma-g-fitzsimmons/author-emma-g-fitzsimmons-thumbLarge.png}}\href{https://www.nytimes3xbfgragh.onion/by/jeffery-c-mays}{\includegraphics{https://static01.graylady3jvrrxbe.onion/images/2018/07/18/multimedia/author-jeffery-c-mays/author-jeffery-c-mays-thumbLarge.png}}

By \href{https://www.nytimes3xbfgragh.onion/by/j-david-goodman}{J. David
Goodman},
\href{https://www.nytimes3xbfgragh.onion/by/emma-g-fitzsimmons}{Emma G.
Fitzsimmons} and
\href{https://www.nytimes3xbfgragh.onion/by/jeffery-c-mays}{Jeffery C.
Mays}

\begin{itemize}
\item
  Sept. 12, 2020Updated 10:29 a.m. ET
\item
  \begin{itemize}
  \item
  \item
  \item
  \item
  \item
  \item
  \end{itemize}
\end{itemize}

With **** conditions decaying in New York City neighborhoods and
business districts, a powerful corporate executive traveled to Gracie
Mansion in July to meet with Mayor Bill de Blasio. He briefed the mayor
on a plan --- prepared by 14 consulting firms --- for how City Hall
could work with business leaders to overcome the pandemic downturn.

Mr. de Blasio appeared supportive. The executive, Steven R. Swartz, head
of the Hearst media conglomerate, left feeling hopeful, as he later told
others from the Partnership for New York City, a top business group.

But weeks then went by, and the corporate leaders began feeling that Mr.
de Blasio was ignoring their concerns.

On Thursday,
\href{https://www.nytimes3xbfgragh.onion/2020/09/10/nyregion/de-blasio-economy-coronavirus.html}{they
struck back in the form of an open letter} that publicly upbraided the
mayor for neglecting ``public safety, cleanliness and other
quality-of-life issues,'' which they said had led to ``widespread
anxiety'' among New Yorkers.

\href{https://pfnyc.org/news/letter-to-mayor-bill-de-blasio-from-nyc-business-leaders/}{The
letter} was signed by 163 chief executives and leaders, a striking array
from some of the biggest companies in New York City, including Goldman
Sachs, JetBlue, Mastercard, Morgan Stanley, Pfizer and Warby Parker, as
well as from top law firms and real estate developers. They called on
the mayor to take ``immediate action to restore essential services.''

From the start of his mayoralty in 2014, Mayor de Blasio has
\href{https://www.nytimes3xbfgragh.onion/2017/08/16/nyregion/bill-de-blasio-wall-street.html}{prided
himself on championing the working class and spurning the city's
business elite}, drawing a sharp contrast with his predecessor, Michael
R. Bloomberg, a billionaire with close ties to corporate leaders. That
antagonistic posture rankled many top executives, but they mostly kept
their criticisms private during the economic boom years that
characterized most of Mr. de Blasio's tenure.

Faced with a pandemic and its devastating economic consequences, the
mayor and the city's top business leaders now have little record of
working together to draw upon. The letter from the chief executives
underscores how the years of distrust are creating new obstacles for
what Mr. de Blasio had hoped would be the start of the city's
``rebirth.''

The relationship is further complicated because Mr. de Blasio is
term-limited. He leaves office at the end of 2021, so both sides have
less incentive to patch things up.

For now, the companies say they need Mr. de Blasio's leadership to help
persuade workers that it is safe to return to the office and assure them
that any quality-of-life problems that may have worsened during the
pandemic will be addressed. The city needs the companies to return to
help begin to restore the huge loss in tax revenue.

\includegraphics{https://static01.graylady3jvrrxbe.onion/images/2020/09/11/nyregion/11NYVIRUSMAYOR8/merlin_161007951_077e29ba-7b1d-450f-aae4-e3c066ceec89-articleLarge.jpg?quality=75\&auto=webp\&disable=upscale}

``It's all a chicken-and-egg problem. Until the people come back, the
streets aren't safe. If the streets aren't safe, the people don't come
back,'' said Kathryn Wylde, the president of the Partnership for New
York City, which sent the letter. ``So somebody's got to break the
egg.''

\hypertarget{latest-updates-the-coronavirus-outbreak}{%
\section{\texorpdfstring{\href{https://www.nytimes3xbfgragh.onion/2020/09/11/world/covid-19-coronavirus.html?action=click\&pgtype=Article\&state=default\&region=MAIN_CONTENT_1\&context=storylines_live_updates}{Latest
Updates: The Coronavirus
Outbreak}}{Latest Updates: The Coronavirus Outbreak}}\label{latest-updates-the-coronavirus-outbreak}}

Updated 2020-09-12T12:04:20.515Z

\begin{itemize}
\tightlist
\item
  \href{https://www.nytimes3xbfgragh.onion/2020/09/11/world/covid-19-coronavirus.html?action=click\&pgtype=Article\&state=default\&region=MAIN_CONTENT_1\&context=storylines_live_updates\#link-dfb8a16}{Fauci
  cautions the virus could disrupt life in the U.S. until `maybe even
  towards the end of 2021.'}
\item
  \href{https://www.nytimes3xbfgragh.onion/2020/09/11/world/covid-19-coronavirus.html?action=click\&pgtype=Article\&state=default\&region=MAIN_CONTENT_1\&context=storylines_live_updates\#link-7104d154}{From
  Asia to Africa, China promotes its vaccine candidates to win friends.}
\item
  \href{https://www.nytimes3xbfgragh.onion/2020/09/11/world/covid-19-coronavirus.html?action=click\&pgtype=Article\&state=default\&region=MAIN_CONTENT_1\&context=storylines_live_updates\#link-393ad215}{The
  other way the virus will kill: hunger.}
\end{itemize}

\href{https://www.nytimes3xbfgragh.onion/2020/09/11/world/covid-19-coronavirus.html?action=click\&pgtype=Article\&state=default\&region=MAIN_CONTENT_1\&context=storylines_live_updates}{See
more updates}

More live coverage:
\href{https://www.nytimes3xbfgragh.onion/live/2020/09/11/business/stock-market-today-coronavirus?action=click\&pgtype=Article\&state=default\&region=MAIN_CONTENT_1\&context=storylines_live_updates}{Markets}

To some extent, both the mayor and the business leaders are grappling
more with perceptions than reality.

The pandemic has
\href{https://www.nytimes3xbfgragh.onion/interactive/2020/nyregion/new-york-city-coronavirus-cases.html}{killed
more than 23,000 people in New York City} and clearly presents unique
threats to the city's future. But New York City has rebounded before,
notably after the fiscal crisis of the 1970s and the attacks of Sept.
11. And the city has, for the moment, defied predictions and largely
contained its outbreak, successfully ramping up testing and contact
tracing while maintaining some of the lowest rates of positive test
results in the country.

Moreover, the disorder described in the letter is not prevalent on most
streets, where New Yorkers dine comfortably outside at night.

Shootings have increased to a worrisome degree in some neighborhoods,
but in general, crime is nowhere near as bad as it was in the early
1990s. (Roughly the same number of people have been shot so far this
year
\href{https://www1.nyc.gov/assets/nypd/downloads/pdf/crime_statistics/cs-en-us-city.pdf}{as
at this point in 2010}, during the middle of the Bloomberg
administration.)

Some business leaders privately expressed concern that the public nature
of the letter, and its suggestion of rampant disorder, could be
counterproductive because it suggested that conditions were far worse
than they actually are.

The report sent to the mayor in July outlined a number of potential
partnerships between private business and the city government, and
called for, among other things, flexibility in building codes,
community-based child care and public-private partnerships to provide
free Wi-Fi access to students through the school year.

On its face, many of the ideas appeared aimed at appealing to Mr. de
Blasio, with proposals to support small and minority-owned businesses
and to improve online learning.

The report, entitled ``A Call to Action and Collaboration,'' was less
pointed on the subject of disorderly streets than Thursday's letter,
referring only obliquely to the need for ``trust that the urban
environment is healthy, secure and welcoming'' in order to ``attract and
retain talent.''

At least publicly, Mr. de Blasio and his senior aides did not lash back
at the business leaders in response to the letter. Mr. de Blasio urged
the business community to lobby the federal and state governments for
more aid to help patch the city's big budget deficits caused by the
outbreak.

``We need these leaders to join the fight to move the city forward,''
\href{https://twitter.com/NYCMayor/status/1304161101458833409}{Mr. de
Blasio said}.

Privately, de Blasio aides expressed frustration that the corporate
executives were not pressing state and federal officials to do more to
help the city.

Still, at roughly the same time, and in a very different setting, a
similar message of discontent was being delivered to the de Blasio
administration by a coalition of neighborhood business groups and local
chambers of commerce from around the city.

In a 90-minute video meeting on Thursday with top city officials,
including three deputy mayors and the police commissioner, the small
business groups urged action on open-air drug use, drug sales, illegal
vending and homelessness, according to two people who attended the
meeting.

The group had been organized with the help of the Real Estate Board of
New York and included at least one labor leader, Gary LaBarbera of the
Building and Construction Trades Council of Greater New York.

``I live in Harlem, and the trash is not being picked up,'' said the
president of the Manhattan Chamber of Commerce, Jessica Walker, who
signed the letter and took part in the video meeting. ``This is being
felt all over the city, and we want to make sure it doesn't get too
far.''

The local business groups said that the city could inspire some
confidence if it began encouraging its own sprawling work force to
return to the office.

Image

The city's fiscal crisis, which was brought on by the pandemic, has led
to its cutting back on some municipal services, including
sanitation.Credit...OK McCausland for The New York Times

The one-two punch from New York City's business class was not
coordinated, according to two people involved in its planning. But it
highlighted the long road to recovery for the city, and raised questions
about whether a mayor with a little over a year to go in his tenure can
dismiss these complaints as just coming from the wealthy.

``We do not make decisions based on the wealthy few,'' the mayor said in
August, when asked about rich New Yorkers possibly abandoning the city.
``That's not how it works around here anymore.''

Bill Neidhardt, the mayor's press secretary, said Mr. de Blasio shares
the concerns of business leaders about cuts to city services and
genuinely wants their help with a solution.

For City Hall, that has meant help in seeking funds from Washington and
long-term borrowing authority from Albany, in the hope of forestalling
layoffs of city workers on Oct 1. Many of the signatories of the letter
have good relationships with state leaders, including Gov. Andrew M.
Cuomo. (Mr. Cuomo and Mr. de Blasio, both Democrats, have had
\href{https://www.nytimes3xbfgragh.onion/2018/04/22/nyregion/cuomo-deblasio-feud-nyc.html}{a
long and fraught relationship}.)

Image

Mayor Bill de Blasio has been dismissive of the concerns of the rich,
saying last month that he does not ``make decisions based on the wealthy
few.''Credit...Frank Franklin Ii/Associated Press

``There is a simple message --- help us get long-term borrowing,'' Mr.
Neidhardt said. ``For the people who have the ear of lawmakers and
decision makers, we desperately need your help.''

Ms. Wylde said the group does not support the long-term borrowing that
the mayor and his advisers favor. ``They think the problem is money. The
problem is not money. The problem is uniting the city around a practical
plan for recovery,'' she said.

The letter reflected frustration among business leaders after the
publication by the partnership of
\href{https://pfnyc.org/wp-content/uploads/2020/07/actionandcollaboration.pdf}{the
report that was presented to Mr. de Blasio}at Gracie Mansion in July.
``It didn't seem to result in any action,'' Ms. Wylde said.

Then in August, Ms. Wylde said she became alarmed when a survey of her
members showed, surprisingly, that even as the pandemic appeared to be
largely under control in New York City --- with roughly 1 percent of
tests coming back positive ---
\href{https://pfnyc.org/news/return-to-office-survey-released-from-partnership-for-new-york-city/}{fewer
major businesses were planning to return} to their offices than had been
planning to do so in May.

While state and city guidelines permit offices to be filled to
half-capacity, neither the mayor nor the governor, fearful of a new
outbreak, have pushed hard for office workers to return. Most buildings
are below 10 percent occupancy.

The letter began to circulate later that month among partnership members
and attracted signatures quickly, including from sports organizations
like the National Basketball Association, and even past donors to the
mayor, such as Steven Rattner, a Wall Street financier and a former
adviser to the Obama administration.

But Ms. Wylde said she waited to publish it until after Labor Day, in
part because of concern among some members, who had spent the pandemic
outside the city, that they would be criticized for weighing in on New
York's future from afar.

``They felt it was unseemly to be writing from the Hamptons,'' she said.

Still, the letter set off howls among some Democrats and progressives
who saw the call to address quality-of-life issues --- and the
\href{https://twitter.com/CommissBratton/status/1304162872738578433?s=20}{support
it garnered} from the city's former police commissioner, William J.
Bratton --- as an endorsement of punitive policing to deal with problems
of poverty, drug addiction and homelessness.

Mark Treyger, a city councilman from Brooklyn, said the letter reeked of
``chutzpah'' and the executives could instead be offering to pay more in
taxes. ``There are folks worried about facing eviction --- as opposed to
executives worried about a gum wrapper on the sidewalk,'' he said.

Image

 Even as the pandemic appeared to be largely under control in New York
City, major businesses have been reluctant to send workers back to their
offices.Credit...Victor J. Blue for The New York Times

Several of the signatories have been antagonists of Mr. de Blasio in the
past, including John Catsimatidis, a billionaire Republican grocery
magnate who has run for mayor before and has
\href{https://nypost.com/2020/08/01/catsimatidis-says-hed-spend-100-million-to-win-nyc-mayoral-race/}{toyed
with a run in 2021}. Others, like Blair W. Effron, of the investment
bank Centerview Partners, have been supporters.

``It's not about being a Democrat or a Republican right now; it's about
being a New Yorker who loves New York,'' said Mr. Catsimatidis, who has
returned to his Manhattan office after several months in the Hamptons.
``These people are going to be moving their companies.''

Lisa Sorin, president of the Bronx Chamber of Commerce, said she signed
the letter as a last resort. The Bronx is struggling to address street
homelessness, graffiti and illegal vendors, she said. Commercial
corridors are filthier than ever.

``You can only be but so patient. We are in unprecedented times,'' said
Ms. Sorin. ``That explains why we needed to stand on the rooftop and
scream.''

Kathryn Garcia, the city's sanitation commissioner who is leaving her
post to explore a run for mayor, said she understood the anxiety voiced
in the letter. The budget cuts to her department had taken a toll on New
Yorkers, she said, especially those who spend time at parks where bins
are overflowing with trash.

``It's one of the few things that New Yorkers are allowed to do,'' she
said, ``and we've made it unpleasant.''

Advertisement

\protect\hyperlink{after-bottom}{Continue reading the main story}

\hypertarget{site-index}{%
\subsection{Site Index}\label{site-index}}

\hypertarget{site-information-navigation}{%
\subsection{Site Information
Navigation}\label{site-information-navigation}}

\begin{itemize}
\tightlist
\item
  \href{https://help.nytimes3xbfgragh.onion/hc/en-us/articles/115014792127-Copyright-notice}{©~2020~The
  New York Times Company}
\end{itemize}

\begin{itemize}
\tightlist
\item
  \href{https://www.nytco.com/}{NYTCo}
\item
  \href{https://help.nytimes3xbfgragh.onion/hc/en-us/articles/115015385887-Contact-Us}{Contact
  Us}
\item
  \href{https://www.nytco.com/careers/}{Work with us}
\item
  \href{https://nytmediakit.com/}{Advertise}
\item
  \href{http://www.tbrandstudio.com/}{T Brand Studio}
\item
  \href{https://www.nytimes3xbfgragh.onion/privacy/cookie-policy\#how-do-i-manage-trackers}{Your
  Ad Choices}
\item
  \href{https://www.nytimes3xbfgragh.onion/privacy}{Privacy}
\item
  \href{https://help.nytimes3xbfgragh.onion/hc/en-us/articles/115014893428-Terms-of-service}{Terms
  of Service}
\item
  \href{https://help.nytimes3xbfgragh.onion/hc/en-us/articles/115014893968-Terms-of-sale}{Terms
  of Sale}
\item
  \href{https://spiderbites.nytimes3xbfgragh.onion}{Site Map}
\item
  \href{https://help.nytimes3xbfgragh.onion/hc/en-us}{Help}
\item
  \href{https://www.nytimes3xbfgragh.onion/subscription?campaignId=37WXW}{Subscriptions}
\end{itemize}
