Sections

SEARCH

\protect\hyperlink{site-content}{Skip to
content}\protect\hyperlink{site-index}{Skip to site index}

\href{https://myaccount.nytimes3xbfgragh.onion/auth/login?response_type=cookie\&client_id=vi}{}

\href{https://www.nytimes3xbfgragh.onion/section/todayspaper}{Today's
Paper}

\href{/section/upshot}{The Upshot}\textbar{}Coronavirus Tests Are
Supposed to Be Free. The Surprise Bills Come Anyway.

\url{https://nyti.ms/3jUZAYY}

\begin{itemize}
\item
\item
\item
\item
\item
\item
\end{itemize}

\hypertarget{the-coronavirus-outbreak}{%
\subsubsection{\texorpdfstring{\href{https://www.nytimes3xbfgragh.onion/news-event/coronavirus?name=styln-coronavirus-national\&region=TOP_BANNER\&block=storyline_menu_recirc\&action=click\&pgtype=Article\&impression_id=819d6b70-f2a3-11ea-8800-53e543aa1720\&variant=undefined}{The
Coronavirus
Outbreak}}{The Coronavirus Outbreak}}\label{the-coronavirus-outbreak}}

\begin{itemize}
\tightlist
\item
  live\href{https://www.nytimes3xbfgragh.onion/2020/09/09/world/covid-19-coronavirus.html?name=styln-coronavirus-national\&region=TOP_BANNER\&block=storyline_menu_recirc\&action=click\&pgtype=Article\&impression_id=819d9280-f2a3-11ea-8800-53e543aa1720\&variant=undefined}{Latest
  Updates}
\item
  \href{https://www.nytimes3xbfgragh.onion/interactive/2020/us/coronavirus-us-cases.html?name=styln-coronavirus-national\&region=TOP_BANNER\&block=storyline_menu_recirc\&action=click\&pgtype=Article\&impression_id=819d9281-f2a3-11ea-8800-53e543aa1720\&variant=undefined}{Maps
  and Cases}
\item
  \href{https://www.nytimes3xbfgragh.onion/interactive/2020/science/coronavirus-vaccine-tracker.html?name=styln-coronavirus-national\&region=TOP_BANNER\&block=storyline_menu_recirc\&action=click\&pgtype=Article\&impression_id=819d9282-f2a3-11ea-8800-53e543aa1720\&variant=undefined}{Vaccine
  Tracker}
\item
  \href{https://www.nytimes3xbfgragh.onion/2020/09/02/your-money/eviction-moratorium-covid.html?name=styln-coronavirus-national\&region=TOP_BANNER\&block=storyline_menu_recirc\&action=click\&pgtype=Article\&impression_id=819d9283-f2a3-11ea-8800-53e543aa1720\&variant=undefined}{Eviction
  Moratorium}
\item
  \href{https://www.nytimes3xbfgragh.onion/interactive/2020/09/02/magazine/food-insecurity-hunger-us.html?name=styln-coronavirus-national\&region=TOP_BANNER\&block=storyline_menu_recirc\&action=click\&pgtype=Article\&impression_id=819d9284-f2a3-11ea-8800-53e543aa1720\&variant=undefined}{American
  Hunger}
\end{itemize}

Advertisement

\protect\hyperlink{after-top}{Continue reading the main story}

Upshot

Supported by

\protect\hyperlink{after-sponsor}{Continue reading the main story}

\hypertarget{coronavirus-tests-are-supposed-to-be-free-the-surprise-bills-come-anyway}{%
\section{Coronavirus Tests Are Supposed to Be Free. The Surprise Bills
Come
Anyway.}\label{coronavirus-tests-are-supposed-to-be-free-the-surprise-bills-come-anyway}}

Congress sought to ensure that patients would not face costs connected
to the virus. But rules are not always being followed.

\includegraphics{https://static01.graylady3jvrrxbe.onion/images/2020/09/02/business/00up-virus-test-fees1/merlin_175997958_da5d0759-6976-4981-8ad7-bd82e0d9dba9-articleLarge.jpg?quality=75\&auto=webp\&disable=upscale}

\href{https://www.nytimes3xbfgragh.onion/by/sarah-kliff}{\includegraphics{https://static01.graylady3jvrrxbe.onion/images/2020/08/25/reader-center/author-sarah-kliff/author-sarah-kliff-thumbLarge.png}}

By \href{https://www.nytimes3xbfgragh.onion/by/sarah-kliff}{Sarah Kliff}

\begin{itemize}
\item
  Sept. 9, 2020, 5:00 a.m. ET
\item
  \begin{itemize}
  \item
  \item
  \item
  \item
  \item
  \item
  \end{itemize}
\end{itemize}

\emph{The New York Times is investigating the costs associated with
testing and treatment for the coronavirus and how the pandemic is
changing health care in America. You can read more about the project and
submit your medical bills}
\href{http://www.nytimes3xbfgragh.onion/costofcare}{\emph{here}}\emph{.}

Sarah Goldstone got a coronavirus test in Massachusetts after her health
insurer said it was ``waiving cost sharing for Covid-19 testing-related
visits.''

Amanda Bowes, a health policy analyst in Maryland, got hers because she
knew a new federal law should make coronavirus testing free for insured
patients like her.

Kelly Daisley had one after seeing New York City's ads offering free
tests. ``Do it for them,'' says one bus shelter ad near her home,
showing a happy family.

All three were surprised when their health insurers said that they were
responsible for a significant chunk of their bills --- in Ms. Daisley's
case, as much as \$2,718.

``I had seen so many commercials saying there is testing everywhere,
it's free, you don't need insurance,'' said Ms. Daisley, 47, who was
tested at an urgent care center three blocks from her Brooklyn
apartment. ``If I had to pay it off, it would clear out my savings.''

For months, Americans have been told not to worry about the costs of
coronavirus tests, which are crucial to stopping the pandemic's spread.
``It is critical that Americans have peace of mind knowing that cost
won't be a barrier to testing during this national public health
emergency,'' Medicare's administrator, Seema Verma, said in April.

Congress passed laws requiring insurers to pay for tests, and the Trump
administration created a program to cover the
\href{https://www.nytimes3xbfgragh.onion/2020/08/29/health/Covid-obamacare-uninsured.html}{bills
of the uninsured}. Cities and states set up no-cost testing sites.

\includegraphics{https://static01.graylady3jvrrxbe.onion/images/2020/09/02/business/00up-virus-test-fees2/merlin_176499048_f7b62fce-612f-425f-a43a-06f715cb747e-articleLarge.jpg?quality=75\&auto=webp\&disable=upscale}

Patients,~whether with or without insurance, are beginning to find holes
in those new coverage programs. Nationwide, people have been hit with
unexpected fees and denied claims related to coronavirus tests,
according to dozens of bills that The New York Times has reviewed.
Insurers have told these patients they could owe from a few dollars to
thousands.

These patients responded to a Times request for medical bills related to
coronavirus testing and treatment, allowing us to identify previously
unreported patterns in medical billing.

\hypertarget{latest-updates-the-coronavirus-outbreak}{%
\section{\texorpdfstring{\href{https://www.nytimes3xbfgragh.onion/2020/09/09/world/covid-19-coronavirus.html?action=click\&pgtype=Article\&state=default\&region=MAIN_CONTENT_1\&context=storylines_live_updates}{Latest
Updates: The Coronavirus
Outbreak}}{Latest Updates: The Coronavirus Outbreak}}\label{latest-updates-the-coronavirus-outbreak}}

Updated 2020-09-09T13:35:31.017Z

\begin{itemize}
\tightlist
\item
  \href{https://www.nytimes3xbfgragh.onion/2020/09/09/world/covid-19-coronavirus.html?action=click\&pgtype=Article\&state=default\&region=MAIN_CONTENT_1\&context=storylines_live_updates\#link-70cea8bb}{As
  drugmakers pledge to thoroughly vet a vaccine, one company pauses its
  trials for a safety review.}
\item
  \href{https://www.nytimes3xbfgragh.onion/2020/09/09/world/covid-19-coronavirus.html?action=click\&pgtype=Article\&state=default\&region=MAIN_CONTENT_1\&context=storylines_live_updates\#link-780eaa2f}{Britain
  is expected to ban gatherings of more than six people.}
\item
  \href{https://www.nytimes3xbfgragh.onion/2020/09/09/world/covid-19-coronavirus.html?action=click\&pgtype=Article\&state=default\&region=MAIN_CONTENT_1\&context=storylines_live_updates\#link-11cec4c0}{Quarantine
  breakdowns at colleges in the U.S. are leaving some at risk.}
\end{itemize}

\href{https://www.nytimes3xbfgragh.onion/2020/09/09/world/covid-19-coronavirus.html?action=click\&pgtype=Article\&state=default\&region=MAIN_CONTENT_1\&context=storylines_live_updates}{See
more updates}

More live coverage:
\href{https://www.nytimes3xbfgragh.onion/live/2020/09/09/business/stock-market-today-coronavirus?action=click\&pgtype=Article\&state=default\&region=MAIN_CONTENT_1\&context=storylines_live_updates}{Markets}

They are not alone. About 2.4 percent of coronavirus tests billed to
insurers leave the patient responsible for some portion of payment,
according to the health data firm Castlight. With
\href{https://covidtracking.com/}{77 million tests} performed so far, it
could add up to hundreds of thousands of Americans who receive
unexpected bills.

\hypertarget{help-shape-our-reporting}{%
\subsection{Help Shape Our Reporting}\label{help-shape-our-reporting}}

``Whether it's through legislative action or public statements, Congress
has made it really clear that there shouldn't be cost sharing for
Covid-19 testing,'' said Julie Khani, president of the American Clinical
Laboratory Association. ``In practice, that's not really the case.''

In some cases, the charges appear to violate new federal laws that aim
to make coronavirus tests free for privately insured patients. In other
cases, insurers are interpreting gray areas in these new rules in ways
that work in their favor.

When asked about these charges, health plans say they are doing their
best to follow the rules and cover all costs related to the testing.
``If a claim is submitted with the proper coding to demonstrate that a
test was given to diagnose Covid-19, or that a service was delivered to
treat Covid-19, generally the claims for those tests and services are
being covered at no cost to the patient,'' said Kristine Grow, a
spokeswoman for America's Health Insurance Plans.

The insurers faulted the complexity of American medical billing, which
can sometimes make it hard to tell when a coronavirus test is provided.
Insurers can't know to cover a claim differently if hospitals and doctor
offices don't use the right codes.

\subsection{}

\hypertarget{the-cost-of-care}{%
\subsubsection{\texorpdfstring{\href{https://www.nytimes3xbfgragh.onion/spotlight/new-york-shuttered}{The
Cost of Care}}{The Cost of Care}}\label{the-cost-of-care}}

We are examining how Americans are grappling with the costs of health
care during the Covid-19 pandemic.

The new rules that Congress wrote midyear do not slot neatly into
insurers' billing systems. One health insurer said it was having to
manually revise each claim for a coronavirus test, deleting the charges
one by one.

``This is legitimately a confusing area, and the coding is all
evolving,'' said Christen Linke Young, a Brookings Institution fellow
who helped write federal billing rules during her time at the Department
of Health and Human Services. ``Even if you're an insurer or provider
operating in good faith, and think you know what the rules are, figuring
out how to identify relevant claims is a hurdle.''

Congress has legislated twice on coronavirus test billing. The
\href{https://www.congress.gov/bill/116th-congress/house-bill/6201/text}{Families
First Coronavirus Response Act}, passed in March, told insurers they
could not charge co-payments or apply deductibles to coronavirus tests
and other ``items and services furnished'' during the doctor visit. The
rules apply to tests both to detect the disease and to those for
antibodies.

The \href{https://home.treasury.gov/policy-issues/cares}{CARES Act}
built on those protections. It created rules for how to handle
out-of-network coronavirus tests, telling insurers those, too, had to be
covered at no cost to the patient. Insurers estimate that about 10
percent of coronavirus tests have been billed out of network so far, and
that those tests tend to be more expensive than those in network.

Image

Sanitizing chairs at a Covid-19 testing location in Los Angeles in
July.Credit...Jessica Pons for The New York Times

Patients' bills suggest that the rules aren't always being followed.
Insurers have, for example, applied co-payments and deductibles to the
tests, claim documents show.

Ms. Bowes, from Maryland, was especially surprised to be charged a \$50
co-payment for a coronavirus test at an urgent care center. She knew
from her work as a health policy analyst for the National Association of
Attorneys General that this wasn't supposed to happen.

``I was really shocked when I got the bill,'' she said. ``It felt wrong,
and I was angry especially because we were being billed before even
receiving our results.'' After protesting the fee to her insurer, the
charge was reversed and covered.

\href{https://www.nytimes3xbfgragh.onion/news-event/coronavirus?action=click\&pgtype=Article\&state=default\&region=MAIN_CONTENT_3\&context=storylines_faq}{}

\hypertarget{the-coronavirus-outbreak-}{%
\subsubsection{The Coronavirus Outbreak
›}\label{the-coronavirus-outbreak-}}

\hypertarget{frequently-asked-questions}{%
\paragraph{Frequently Asked
Questions}\label{frequently-asked-questions}}

Updated September 4, 2020

\begin{itemize}
\item ~
  \hypertarget{what-are-the-symptoms-of-coronavirus}{%
  \paragraph{What are the symptoms of
  coronavirus?}\label{what-are-the-symptoms-of-coronavirus}}

  \begin{itemize}
  \tightlist
  \item
    In the beginning, the coronavirus
    \href{https://www.nytimes3xbfgragh.onion/article/coronavirus-facts-history.html?action=click\&pgtype=Article\&state=default\&region=MAIN_CONTENT_3\&context=storylines_faq\#link-6817bab5}{seemed
    like it was primarily a respiratory illness}~--- many patients had
    fever and chills, were weak and tired, and coughed a lot, though
    some people don't show many symptoms at all. Those who seemed
    sickest had pneumonia or acute respiratory distress syndrome and
    received supplemental oxygen. By now, doctors have identified many
    more symptoms and syndromes. In April,
    \href{https://www.nytimes3xbfgragh.onion/2020/04/27/health/coronavirus-symptoms-cdc.html?action=click\&pgtype=Article\&state=default\&region=MAIN_CONTENT_3\&context=storylines_faq}{the
    C.D.C. added to the list of early signs}~sore throat, fever, chills
    and muscle aches. Gastrointestinal upset, such as diarrhea and
    nausea, has also been observed. Another telltale sign of infection
    may be a sudden, profound diminution of one's
    \href{https://www.nytimes3xbfgragh.onion/2020/03/22/health/coronavirus-symptoms-smell-taste.html?action=click\&pgtype=Article\&state=default\&region=MAIN_CONTENT_3\&context=storylines_faq}{sense
    of smell and taste.}~Teenagers and young adults in some cases have
    developed painful red and purple lesions on their fingers and toes
    --- nicknamed ``Covid toe'' --- but few other serious symptoms.
  \end{itemize}
\item ~
  \hypertarget{why-is-it-safer-to-spend-time-together-outside}{%
  \paragraph{Why is it safer to spend time together
  outside?}\label{why-is-it-safer-to-spend-time-together-outside}}

  \begin{itemize}
  \tightlist
  \item
    \href{https://www.nytimes3xbfgragh.onion/2020/05/15/us/coronavirus-what-to-do-outside.html?action=click\&pgtype=Article\&state=default\&region=MAIN_CONTENT_3\&context=storylines_faq}{Outdoor
    gatherings}~lower risk because wind disperses viral droplets, and
    sunlight can kill some of the virus. Open spaces prevent the virus
    from building up in concentrated amounts and being inhaled, which
    can happen when infected people exhale in a confined space for long
    stretches of time, said Dr. Julian W. Tang, a virologist at the
    University of Leicester.
  \end{itemize}
\item ~
  \hypertarget{why-does-standing-six-feet-away-from-others-help}{%
  \paragraph{Why does standing six feet away from others
  help?}\label{why-does-standing-six-feet-away-from-others-help}}

  \begin{itemize}
  \tightlist
  \item
    The coronavirus spreads primarily through droplets from your mouth
    and nose, especially when you cough or sneeze. The C.D.C., one of
    the organizations using that measure,
    \href{https://www.nytimes3xbfgragh.onion/2020/04/14/health/coronavirus-six-feet.html?action=click\&pgtype=Article\&state=default\&region=MAIN_CONTENT_3\&context=storylines_faq}{bases
    its recommendation of six feet}~on the idea that most large droplets
    that people expel when they cough or sneeze will fall to the ground
    within six feet. But six feet has never been a magic number that
    guarantees complete protection. Sneezes, for instance, can launch
    droplets a lot farther than six feet,
    \href{https://jamanetwork.com/journals/jama/fullarticle/2763852}{according
    to a recent study}. It's a rule of thumb: You should be safest
    standing six feet apart outside, especially when it's windy. But
    keep a mask on at all times, even when you think you're far enough
    apart.
  \end{itemize}
\item ~
  \hypertarget{i-have-antibodies-am-i-now-immune}{%
  \paragraph{I have antibodies. Am I now
  immune?}\label{i-have-antibodies-am-i-now-immune}}

  \begin{itemize}
  \tightlist
  \item
    As of right
    now,\href{https://www.nytimes3xbfgragh.onion/2020/07/22/health/covid-antibodies-herd-immunity.html?action=click\&pgtype=Article\&state=default\&region=MAIN_CONTENT_3\&context=storylines_faq}{~that
    seems likely, for at least several months.}~There have been
    frightening accounts of people suffering what seems to be a second
    bout of Covid-19. But experts say these patients may have a
    drawn-out course of infection, with the virus taking a slow toll
    weeks to months after initial exposure.~People infected with the
    coronavirus typically
    \href{https://www.nature.com/articles/s41586-020-2456-9}{produce}~immune
    molecules called antibodies, which are
    \href{https://www.nytimes3xbfgragh.onion/2020/05/07/health/coronavirus-antibody-prevalence.html?action=click\&pgtype=Article\&state=default\&region=MAIN_CONTENT_3\&context=storylines_faq}{protective
    proteins made in response to an
    infection}\href{https://www.nytimes3xbfgragh.onion/2020/05/07/health/coronavirus-antibody-prevalence.html?action=click\&pgtype=Article\&state=default\&region=MAIN_CONTENT_3\&context=storylines_faq}{.
    These antibodies may}~last in the body
    \href{https://www.nature.com/articles/s41591-020-0965-6}{only two to
    three months}, which may seem worrisome, but that's~perfectly normal
    after an acute infection subsides, said Dr. Michael Mina, an
    immunologist at Harvard University. It may be possible to get the
    coronavirus again, but it's highly unlikely that it would be
    possible in a short window of time from initial infection or make
    people sicker the second time.
  \end{itemize}
\item ~
  \hypertarget{what-are-my-rights-if-i-am-worried-about-going-back-to-work}{%
  \paragraph{What are my rights if I am worried about going back to
  work?}\label{what-are-my-rights-if-i-am-worried-about-going-back-to-work}}

  \begin{itemize}
  \tightlist
  \item
    Employers have to provide
    \href{https://www.osha.gov/SLTC/covid-19/standards.html}{a safe
    workplace}~with policies that protect everyone equally.
    \href{https://www.nytimes3xbfgragh.onion/article/coronavirus-money-unemployment.html?action=click\&pgtype=Article\&state=default\&region=MAIN_CONTENT_3\&context=storylines_faq}{And
    if one of your co-workers tests positive for the coronavirus, the
    C.D.C.}~has said that
    \href{https://www.cdc.gov/coronavirus/2019-ncov/community/guidance-business-response.html}{employers
    should tell their employees}~-\/- without giving you the sick
    employee's name -\/- that they may have been exposed to the virus.
  \end{itemize}
\end{itemize}

Some patients found that insurance covered the test but denied payment
for other services that went with it: another billing decision that
could violate federal law.

One mother in California was surprised that her daughter's coronavirus
test was fully covered but that a \$49 ``after hours'' fee was not ---
the clinic said it provided tests only in the evening, so as to not
infect other patients.

Insurers have told some patients they are responsible for out-of-network
charges, even though federal law appears to require insurers to at least
partly cover them. This includes Ms. Goldstone from Massachusetts, who
went for a test after experiencing mild coronavirus symptoms.

UnitedHealth paid \$160 for her coronavirus test, but denied the \$250
doctor visit that went with it, stating that her plan did not come with
out-of-network benefits.

``It's upsetting and demoralizing,'' said Ms. Goldstone, a musician who
has been largely out of work since the start of the pandemic. ``I've
spent months being careful with my finances, and already pay \$266 a
month for insurance.''

Federal law requires insurers to pay for any doctor visit associated
with a coronavirus test, specifically noting that visits to urgent care
centers are included. It is silent, however, on how much an insurer must
pay to an out-of-network facility --- although most experts agree a
health plan would need to pay something rather than deny the fee.

``They shouldn't be able to do that,'' Ms. Young of Brookings said.
``But I have sympathy for them and their claims system. It probably has
rules that are saying: This person doesn't have out-of-network
coverage.''

UnitedHealth said it denied the charge because of how the urgent care
center did its billing: It divided the coronavirus test and the visit
into separate claims. After an inquiry from The Times, the insurer said
it would reverse the bill and review how the urgent care center billed
for its services.

``UnitedHealthcare is waiving cost share for Covid-19 testing, in
accordance with state and federal guidelines, including the test Ms.
Goldstone received,'' a spokeswoman, Maria Gordon Shydlo, said. ``She
will not be responsible for the costs.''

Image

A bus stop advertisement announcing free testing in Brooklyn. For some
people, it hasn't been free.~Credit...Maridelis Morales Rosado for The
New York Times

Other bills present murkier situations. Ms. Daisley, from Brooklyn, had
coronavirus diagnostic and antibody tests last month. She was surprised
when she logged into her health insurance portal and saw four claims
associated with her tests: one for each test, one for the doctor visit,
and one for other tests she didn't realize were being ordered.

Her insurance covered the visit and the diagnostic test. But it paid
nothing for an antibody test and the other lab services, which were both
sent off to out-of-network providers.

Experts say federal law requires the insurer to cover the antibody test
in full, even out of network. But the rules around the other tests are
less clear: The law states that insurers must cover services related to
obtaining a coronavirus test but doesn't identify what type of care
makes the cut. Some providers seem to tack on unrelated lab tests.
Patients at a drive-through coronavirus testing site in Texas, for
example, were
\href{https://www.nytimes3xbfgragh.onion/2020/06/29/upshot/coronavirus-tests-unpredictable-prices.html}{unknowingly
tested for sexually transmitted diseases}. Without clear federal
guidance, insurers are left to sift through charges to decide what is
related to coronavirus and what isn't.

Initially, Ms. Daisley was left with more than \$2,000 to pay to
out-of-network labs: \$210 for the antibody test and \$2,508 for the
other lab services. Her health plan, Anthem, denied the larger charge
because her health benefits do not cover out-of-network care.

The insurer covered the charges after The Times inquired. ``Seeing as
Ms. Daisley was unaware the treating provider would send her samples to
multiple out-of-network labs for what she understood was related to
Covid testing, Anthem is covering the costs of the outstanding claims,''
a spokeswoman for Anthem, Leslie Porras, said.

Advertisement

\protect\hyperlink{after-bottom}{Continue reading the main story}

\hypertarget{site-index}{%
\subsection{Site Index}\label{site-index}}

\hypertarget{site-information-navigation}{%
\subsection{Site Information
Navigation}\label{site-information-navigation}}

\begin{itemize}
\tightlist
\item
  \href{https://help.nytimes3xbfgragh.onion/hc/en-us/articles/115014792127-Copyright-notice}{©~2020~The
  New York Times Company}
\end{itemize}

\begin{itemize}
\tightlist
\item
  \href{https://www.nytco.com/}{NYTCo}
\item
  \href{https://help.nytimes3xbfgragh.onion/hc/en-us/articles/115015385887-Contact-Us}{Contact
  Us}
\item
  \href{https://www.nytco.com/careers/}{Work with us}
\item
  \href{https://nytmediakit.com/}{Advertise}
\item
  \href{http://www.tbrandstudio.com/}{T Brand Studio}
\item
  \href{https://www.nytimes3xbfgragh.onion/privacy/cookie-policy\#how-do-i-manage-trackers}{Your
  Ad Choices}
\item
  \href{https://www.nytimes3xbfgragh.onion/privacy}{Privacy}
\item
  \href{https://help.nytimes3xbfgragh.onion/hc/en-us/articles/115014893428-Terms-of-service}{Terms
  of Service}
\item
  \href{https://help.nytimes3xbfgragh.onion/hc/en-us/articles/115014893968-Terms-of-sale}{Terms
  of Sale}
\item
  \href{https://spiderbites.nytimes3xbfgragh.onion}{Site Map}
\item
  \href{https://help.nytimes3xbfgragh.onion/hc/en-us}{Help}
\item
  \href{https://www.nytimes3xbfgragh.onion/subscription?campaignId=37WXW}{Subscriptions}
\end{itemize}
