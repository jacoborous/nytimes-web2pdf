Sections

SEARCH

\protect\hyperlink{site-content}{Skip to
content}\protect\hyperlink{site-index}{Skip to site index}

\href{https://www.nytimes3xbfgragh.onion/section/sports/tennis}{Tennis}

\href{https://myaccount.nytimes3xbfgragh.onion/auth/login?response_type=cookie\&client_id=vi}{}

\href{https://www.nytimes3xbfgragh.onion/section/todayspaper}{Today's
Paper}

\href{/section/sports/tennis}{Tennis}\textbar{}2020 U.S. Open: What to
Watch on Wednesday

\url{https://nyti.ms/3heEyD9}

\begin{itemize}
\item
\item
\item
\item
\item
\end{itemize}

Advertisement

\protect\hyperlink{after-top}{Continue reading the main story}

Supported by

\protect\hyperlink{after-sponsor}{Continue reading the main story}

\hypertarget{2020-us-open-what-to-watch-on-wednesday}{%
\section{2020 U.S. Open: What to Watch on
Wednesday}\label{2020-us-open-what-to-watch-on-wednesday}}

Serena Williams and Daniil Medvedev, two of last year's finalists, will
try to secure spots in the semifinals.

\includegraphics{https://static01.graylady3jvrrxbe.onion/images/2020/09/09/sports/09watch-serena/merlin_176703906_c484bf57-9c14-4951-a0ca-745cb1107621-articleLarge.jpg?quality=75\&auto=webp\&disable=upscale}

By Max Gendler

\begin{itemize}
\item
  Sept. 9, 2020, 5:00 a.m. ET
\item
  \begin{itemize}
  \item
  \item
  \item
  \item
  \item
  \end{itemize}
\end{itemize}

\textbf{How to watch:} From noon to 6 p.m. Eastern time on ESPN; 7 to 11
p.m. on ESPN2; and streaming on the ESPN app.

As the United States Open inches closer to its final weekend, the
matches are now taking place on just two courts. All of today's singles
quarterfinals will be at Arthur Ashe Stadium, and the lone doubles
semifinal at Louis Armstrong Stadium. With so many top players facing
off, it should be an interesting day.

\hypertarget{here-are-some-matches-to-keep-an-eye-on}{%
\subsection{Here are some matches to keep an eye
on.}\label{here-are-some-matches-to-keep-an-eye-on}}

\emph{Because of the number of matches cycling through courts, the times
for individual matchups are at best a guess and are certain to fluctuate
based on the times at which earlier play is completed. All times are
Eastern.}

Arthur Ashe Stadium \textbar{} Noon

\hypertarget{serena-williams-vs-tsvetana-pironkova}{%
\subsubsection{\texorpdfstring{\textbf{Serena Williams vs. Tsvetana
Pironkova}}{Serena Williams vs. Tsvetana Pironkova}}\label{serena-williams-vs-tsvetana-pironkova}}

Williams, a 23-time major singles champion, on Monday won her 100th
match at Arthur Ashe Stadium. It's a testament to her dominance and
longevity that the next best performer at Arthur Ashe is Roger Federer,
with 77 victories. Although the rabid fans that usually accompany such
wins are not present this year, the change appears to have had almost no
effect on Williams.

``This is my job,'' she said after her win over Maria Sakkari on Monday.
``This is what I wake up to do. This is what I train to do 365 days of
the year.''

That resolve was clear in her past two matches. Against Sakkari and
Sloane Stephens, she dropped a set to an opponent who knew she needed to
play her best tennis in order to unseat Williams. But with aggressive
returns and tenacious rallies, Williams refused to allow either of them
the opportunity to win.

Pironkova, in her first professional tournament since Wimbledon in 2017,
has reached the quarterfinals in style. Though the unseeded Pironkova
has been a dark horse for the average viewer, she has gained two
comprehensive victories over seeded players: Garbiñe Muguruza and Donna
Vekic. This is only Pironkova's third Grand Slam quarterfinal appearance
out of 50 appearances, and she has reached the round of 16 at the U.S.
Open just once before.

Pironkova prefers a faster court, but it will be tough for her to outgun
Williams today. The experience that Williams has been building --- not
just throughout her career, but also in the past few months of
competition in preparation for the U.S. Open --- will probably allow her
to adjust to her fellow veteran's style of play.

\includegraphics{https://static01.graylady3jvrrxbe.onion/images/2020/09/09/sports/09watch-daniil/merlin_176708559_aa5639ad-a4ff-45c6-850b-67747c21b729-articleLarge.jpg?quality=75\&auto=webp\&disable=upscale}

Arthur Ashe Stadium \textbar{} 2 p.m.

\hypertarget{andrey-rublev-vs-daniil-medvedev}{%
\subsubsection{\texorpdfstring{\textbf{Andrey Rublev vs. Daniil
Medvedev}}{Andrey Rublev vs. Daniil Medvedev}}\label{andrey-rublev-vs-daniil-medvedev}}

In the men's singles draw ahead of the U.S. Open, the quarter with last
year's runner-up, Daniil Medvedev, stood out as easily the most packed
with talent. Medvedev was joined by two quarterfinalists from last year
--- Matteo Berrettini and Grigor Dimitrov --- and promising youngsters,
such as Frances Tiafoe and Rublev.

In the early rounds, Medvedev and Rublev made it clear that they both
had the potential to reach the final this year. Rublev has dropped only
one set, against the big serve of Berrettini, but ultimately became the
first player to break Berrettini's serve as he secured a four-set
victory. Medvedev has not dropped a set, and has lost only seven games
on average in each match.

Both Russians possess unique games. Rublev hits with a very open stance,
a technique that has recently grown in popularity, but that is generally
reserved for situations in which a player is on the run and needs to
play defensively. Rublev uses this open stance even when in control of
the point, which can make it difficult to tell where he intends to hit.

Medvedev, an extremely lanky player, has an erratic and transfixing
game. He seems able to hit the ball in almost any way, depending on the
position he finds himself in. He appears equally comfortable to shape up
in perfect form, or to hit a winner by swinging his arm wildly over his
head.

This matchup could leave viewers transfixed, making them ask just how
the modern game of tennis works. There is no simple answer anymore.

Arthur Ashe Stadium \textbar{} 7 p.m.

\hypertarget{victoria-azarenka-vs-elise-mertens}{%
\subsubsection{\texorpdfstring{\textbf{Victoria Azarenka vs. Elise
Mertens}}{Victoria Azarenka vs. Elise Mertens}}\label{victoria-azarenka-vs-elise-mertens}}

There were times during Azarenka's round-of-16 matchup against Karolina
Muchova that one could see just how mentally tough the Belarusian
veteran is. After battling through the first set, and being kept at bay
by the 20th seed, Azarenka adjusted, finding weaknesses in Muchova's
game and exploiting them with almost no fuss to charge ahead and win the
next two sets.

Azarenka, the champion at the Western \& Southern Open last month, has
shown this ability time and again in the past few weeks.

For Mertens, the 16th seed, this match will be a tough task. Even though
Mertens has yet to drop a set through the first four rounds and she
upset the second seed, Sofia Kenin, an in-form Azarenka poses a
different set of complications. The two have met in doubles a few times,
including in the 2019 U.S. Open doubles final, but there, the challenges
posed to opponents tend to be about placement, not power.

Mertens is an exceptionally consistent player. That works well against
players who can be forced into mistakes, but with the way Azarenka has
been playing, it's unlikely that Mertens will be able to withstand an
onslaught of groundstrokes for long enough to find that one mistake.

Arthur Ashe Stadium \textbar{} 9 p.m.

\hypertarget{alex-de-minaur-vs-dominic-thiem}{%
\subsubsection{\texorpdfstring{\textbf{Alex de Minaur vs. Dominic
Thiem}}{Alex de Minaur vs. Dominic Thiem}}\label{alex-de-minaur-vs-dominic-thiem}}

Thiem, the world No. 3 and the highest seed left in the men's
competition, is primarily a defensive player. Having reached the final
at three major events, only to be foiled by one of the ``Big Three,''
Thiem has developed a reputation as a champion in waiting. Now that he
is one of the favorites, it will be interesting to see how he handles
the pressure.

Thiem prefers slow clay courts and has some tendencies that don't seem
to fit a hard-court match. He returns from deep behind the baseline,
plays high looping balls during points to reset himself and almost never
comes to net. These strategies, for an average player, would be a
distinct disadvantage on faster surfaces. Yet Thiem manages to chase
balls down and to keep himself in points. Tonight, he will play another
resolute defender.

After his round-of-16 victory over Vasek Pospisil, de Minaur, the 21st
seed, was asked about his defensive style of play, which is predicated
on his speed.

``Well, if I could definitely blast people off the court, then trust me,
I would rather do that,'' he replied wryly. ``This running thing gets
tiring, that's for sure.''

To watch him, you could imagine that de Minaur, known as Speed Demon, is
never tired. It can be discouraging for any opponent to hit a nearly
perfect drop shot, only to see the wiry 21-year-old chase it down, then
immediately pop up to chase the following shot.

Often, when two defensive players meet, there can be a lull in energy,
since they are used to redirecting the pace that is being directed into
the corners of their court. In this case, you will see an exception.
Both players are capable of creating their own pace, and it will be
interesting to see how they moderate their shotmaking to create problems
for the opponent without compromising their own strengths.

Image

Asia Muhammad, left, and Taylor Townsend celebrating their win in New
Zealand in January.Credit...Phil Walter/Getty Images

\hypertarget{an-interesting-doubles-match}{%
\subsection{An interesting doubles
match:}\label{an-interesting-doubles-match}}

Asia Muhammad/Taylor Townsend vs. Xu Yifan/Nicole Melichar, Louis
Armstrong Stadium \textbar{} Noon

This will be the first appearance in a Grand Slam semifinal for Muhammad
and Townsend. Throughout the tournament, they have seemed remarkably
calm, letting each match stand on its own, without heeding the usually
worrisome context of a Grand Slam appearance. Xu and Melichar, who began
playing together this year, have a bit more experience with deep runs.
Melichar has one Grand Slam doubles title, and Xu made it to the finals
at Wimbledon last year. Each pair has shown stoicism, but the deeper the
run, the more difficult that can be to maintain. Today's match may end
up being a test of nerves, more than a measure of physical abilities.

Advertisement

\protect\hyperlink{after-bottom}{Continue reading the main story}

\hypertarget{site-index}{%
\subsection{Site Index}\label{site-index}}

\hypertarget{site-information-navigation}{%
\subsection{Site Information
Navigation}\label{site-information-navigation}}

\begin{itemize}
\tightlist
\item
  \href{https://help.nytimes3xbfgragh.onion/hc/en-us/articles/115014792127-Copyright-notice}{©~2020~The
  New York Times Company}
\end{itemize}

\begin{itemize}
\tightlist
\item
  \href{https://www.nytco.com/}{NYTCo}
\item
  \href{https://help.nytimes3xbfgragh.onion/hc/en-us/articles/115015385887-Contact-Us}{Contact
  Us}
\item
  \href{https://www.nytco.com/careers/}{Work with us}
\item
  \href{https://nytmediakit.com/}{Advertise}
\item
  \href{http://www.tbrandstudio.com/}{T Brand Studio}
\item
  \href{https://www.nytimes3xbfgragh.onion/privacy/cookie-policy\#how-do-i-manage-trackers}{Your
  Ad Choices}
\item
  \href{https://www.nytimes3xbfgragh.onion/privacy}{Privacy}
\item
  \href{https://help.nytimes3xbfgragh.onion/hc/en-us/articles/115014893428-Terms-of-service}{Terms
  of Service}
\item
  \href{https://help.nytimes3xbfgragh.onion/hc/en-us/articles/115014893968-Terms-of-sale}{Terms
  of Sale}
\item
  \href{https://spiderbites.nytimes3xbfgragh.onion}{Site Map}
\item
  \href{https://help.nytimes3xbfgragh.onion/hc/en-us}{Help}
\item
  \href{https://www.nytimes3xbfgragh.onion/subscription?campaignId=37WXW}{Subscriptions}
\end{itemize}
