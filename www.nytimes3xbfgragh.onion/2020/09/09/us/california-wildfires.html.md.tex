Sections

SEARCH

\protect\hyperlink{site-content}{Skip to
content}\protect\hyperlink{site-index}{Skip to site index}

\href{https://www.nytimes3xbfgragh.onion/section/us}{U.S.}

\href{https://myaccount.nytimes3xbfgragh.onion/auth/login?response_type=cookie\&client_id=vi}{}

\href{https://www.nytimes3xbfgragh.onion/section/todayspaper}{Today's
Paper}

\href{/section/us}{U.S.}\textbar{}Newsom Vows to `Face Climate Change
Head On' in California

\url{https://nyti.ms/3hkMBOU}

\begin{itemize}
\item
\item
\item
\item
\item
\end{itemize}

Advertisement

\protect\hyperlink{after-top}{Continue reading the main story}

Supported by

\protect\hyperlink{after-sponsor}{Continue reading the main story}

California Today

\hypertarget{newsom-vows-to-face-climate-change-head-on-in-california}{%
\section{Newsom Vows to `Face Climate Change Head On' in
California}\label{newsom-vows-to-face-climate-change-head-on-in-california}}

Wednesday: Another update on the raging wildfires. Also: More counties
can ease restrictions; and a look at school district-supported learning
pods.

\href{https://www.nytimes3xbfgragh.onion/by/jill-cowan}{\includegraphics{https://static01.graylady3jvrrxbe.onion/images/2018/12/10/multimedia/author-jill-cowan/author-jill-cowan-thumbLarge.png}}\href{https://www.nytimes3xbfgragh.onion/by/marie-tae-mcdermott}{\includegraphics{https://static01.graylady3jvrrxbe.onion/images/2018/11/26/multimedia/author-marie-tae-mcdermott/author-marie-tae-mcdermott-thumbLarge.png}}

By \href{https://www.nytimes3xbfgragh.onion/by/jill-cowan}{Jill Cowan}
and
\href{https://www.nytimes3xbfgragh.onion/by/marie-tae-mcdermott}{Marie
Tae McDermott}

\begin{itemize}
\item
  Sept. 9, 2020
\item
  \begin{itemize}
  \item
  \item
  \item
  \item
  \item
  \end{itemize}
\end{itemize}

\includegraphics{https://static01.graylady3jvrrxbe.onion/images/2020/09/09/us/09firecatoday/merlin_176712765_26d582e9-ba14-43e6-a7a3-0d337637a2e4-articleLarge.jpg?quality=75\&auto=webp\&disable=upscale}

\emph{Good morning.}

\textbf{First, here's an update on the wildfires burning across the
state:}

After another weekend plagued by compounding disasters in California,
Gov. Gavin Newsom on Tuesday pointed to his burning state as clear
evidence that climate change --- and its most extreme manifestations ---
is a major driver of the fires' scale and severity.

And even as the Trump administration has pushed to roll back
regulations, Mr. Newsom said the state would push ahead with its
efforts.

``In the absence of federal leadership, California will continue to
lead,'' he said. ``The more we push into this space, more partners will
be forthcoming.''

\emph{{[}Read more about how}
\href{https://www.nytimes3xbfgragh.onion/2019/10/14/us/santa-ana-winds-diablo-saddleridge-fire-blackouts.html}{\emph{California's
``demonized'' winds}} \emph{shape wildfire season.{]}}

Still, the picture he laid out of millions of burned acres, thousands of
homes and buildings destroyed, all with the looming threat of dangerous
\href{https://www.nytimes3xbfgragh.onion/2019/10/14/us/santa-ana-winds-diablo-saddleridge-fire-blackouts.html}{Santa
Ana and Diablo winds} in coming days, wasn't encouraging.

Mr. Newsom said that 7,606
\href{https://www.nytimes3xbfgragh.onion/2020/09/09/us/fires-washington-california-oregon-malden.html}{fires}
have burned 2.3 million acres in California this year. That's a record
in modern history.

Although last year's was a less active
\href{https://www.nytimes3xbfgragh.onion/article/wildfires-california-oregon-washington.html}{fire}
season ---
\href{https://www.nytimes3xbfgragh.onion/2019/03/01/us/california-today-extreme-weather-forecast.html}{it
followed a rainy winter} --- the governor noted that by this time last
year, just 118,000 acres had burned in 4,927 fires.

\emph{{[}Read the}
\href{https://www.nytimes3xbfgragh.onion/2020/09/08/us/wildfires-live-updates.html?name=styln-california-wildfires\&region=TOP_BANNER\&block=storyline_menu_recirc\&action=click\&pgtype=Article\&impression_id=69d920f1-f259-11ea-bef8-f7a505c4b1e2\&variant=1_Show}{\emph{latest
updates}} \emph{on the fires.{]}}

And that's just in California; wildfires were
\href{https://www.nytimes3xbfgragh.onion/2020/09/08/us/wildfires-live-updates.html?name=styln-california-wildfires\&region=TOP_BANNER\&block=storyline_menu_recirc\&action=click\&pgtype=Article\&impression_id=68a81b00-f259-11ea-867a-a3448642d022\&variant=1_Show}{burning
across the West}. And that's just the early part of what is likely to be
a highly active and dangerous fire season.

As of Tuesday, Mr. Newsom said firefighters from not just the United
States but other countries were battling major fires, including the
Creek Fire, which prompted a dramatic rescue of hundreds of campers.

\emph{{[}Read}
\href{https://www.nytimes3xbfgragh.onion/2020/09/08/us/california-wildfires-helicopter-rescue.html?referringSource=articleShare}{\emph{about
the rescue here}}\emph{.{]}}

While he noted that a vast majority of the state's fires had been
sparked by people --- on purpose or inadvertently --- Mr. Newsom said
the weekend's extraordinary heat made fighting fires and preventing new
ones especially difficult.

``Never have I felt more of a sense of obligation and a sense of purpose
to maintain California's leadership not only nationally but
internationally to face climate change head on,'' he said.

Pacific Gas \& Electric on Monday started cutting power to 170,000
customers in an effort to
\href{https://www.nytimes3xbfgragh.onion/2020/09/08/us/wildfires-live-updates.html?name=styln-california-wildfires\&region=TOP_BANNER\&block=storyline_menu_recirc\&action=click\&pgtype=Article\&impression_id=69d920f1-f259-11ea-bef8-f7a505c4b1e2\&variant=1_Show\#link-ae8e448}{prevent
its equipment from sparking additional blazes}.

\emph{(This article is part of the}
\href{https://www.nytimes3xbfgragh.onion/column/california-today}{\emph{California
Today}} \emph{newsletter.}
\href{https://www.nytimes3xbfgragh.onion/newsletters/california-today}{\emph{Sign
up}} \emph{to get it by email.)}

\begin{center}\rule{0.5\linewidth}{\linethickness}\end{center}

\hypertarget{a-quick-update-on-reopening}{%
\subsection{A quick update on
reopening}\label{a-quick-update-on-reopening}}

The governor on Tuesday announced the first handful of counties to move
from the most restrictive of the
\href{https://www.nytimes3xbfgragh.onion/2020/08/31/us/california-coronavirus-reopening.html}{state's
new reopening tiers} into a tier that allows more businesses to operate
indoors at reduced capacity, including restaurants, gyms and houses of
worship.

\emph{{[}Read about the new,}
\href{https://www.nytimes3xbfgragh.onion/2020/08/31/us/california-coronavirus-reopening.html}{\emph{color-coded
tier system}}\emph{.{]}}

The counties that were moved \href{https://covid19.ca.gov/}{to the less
restrictive red tier from the purple included} Amador, Orange, Placer,
Santa Clara and Santa Cruz.

Although the news was encouraging, it seemed to depart from Mr. Newsom's
emphasis a little more than a week ago when he unveiled the new system
that counties would be
\href{https://www.sfchronicle.com/politics/article/California-s-new-rules-for-coronavirus-15522578.php}{required
to spend at least three weeks} in each tier before being allowed to
further ease restrictions.

\begin{center}\rule{0.5\linewidth}{\linethickness}\end{center}

\hypertarget{as-kids-head-back-to-school-a-look-at-learning-pods}{%
\subsection{As kids head back to school, a look at learning
pods}\label{as-kids-head-back-to-school-a-look-at-learning-pods}}

Image

Students attending online classes at the Boys \& Girls Club of Hollywood
in Los Angeles. The facility provides free services for children whose
parents must leave home to work.Credit...Jae C. Hong/Associated Press

\textbf{Now, as millions of kids around the}
\textbf{\href{https://www.nytimes3xbfgragh.onion/spotlight/schools-reopening?}{country
return to school}} \textbf{--- virtually and not --- my colleague}
\textbf{\href{https://www.nytimes3xbfgragh.onion/by/marie-tae-mcdermott}{Marie
Tae McDermott}} \textbf{looked into a question about school pods.}

Buffy Kinstle, a reader in San Francisco, asked us: ``I've noticed a
surge in local families seeking to create micro schools and pandemic
mini pods, in some cases with plans to hire private teachers to manage
their kids' educational and child care needs. What are school districts
doing to protect public access to free education for all, support
working families who cannot afford private small-group education, and
retain qualified educators?''

My colleague Abby Goodnough
\href{https://www.nytimes3xbfgragh.onion/2020/08/14/us/covid-schools-learning-pods.html}{wrote
about how small-group education is on the rise across the country}.
While many wealthy families are hiring private educators to teach their
children, some families are priced out of pods and virtual learning
alternatives, leading to a
\href{https://www.nytimes3xbfgragh.onion/2020/06/05/us/coronavirus-education-lost-learning.html}{widening
education gap} for students.

The solution, some say, is for school districts to provide small-group
learning to families for free. Alpine Union School District, a small
community in eastern San Diego County, was one of the first in the state
to offer families this option, according to the district's
superintendent, Richard Newman.

\emph{{[}Read all of our}
\href{https://www.nytimes3xbfgragh.onion/spotlight/schools-reopening?}{\emph{coverage
of schools}} \emph{reopening.{]}}

Alpine's learning pods started a little over two weeks ago. Parents take
turns supervising a fixed cohort of 12 students. The students receive
virtual instruction from their teachers, and one teacher assigned to the
pod provides face-to-face support. The district provides the physical
location, technology, curriculum and facilitates safety protocols.

The California Department of Public Health
\href{https://www.cdph.ca.gov/Programs/CID/DCDC/Pages/COVID-19/small-groups-child-youth.aspx}{released
a set of health guidelines for facilitating groups of students}, which
Dr. Newman said his learning pods met. Alpine is also relatively small,
with only 1,700 students from kindergarten through **** eighth grade.

On a larger scale, San Francisco is transforming recreation facilities,
libraries and community centers into learning hubs, where some 6,000
students will be able to go daily to complete their online schoolwork
and engage in programming and outdoor play.

In light of the
\href{https://messaging-custom-newsletters.nytimes3xbfgragh.onion/template/oakv2?campaign_id=49\&emc=edit_ca_20200831\&instance_id=21774\&nl=california-today\&productCode=CA\&regi_id=68519573\&segment_id=37216\&te=1\&uri=nyt\%3A\%2F\%2Fnewsletter\%2Fef87c1d8-ddb3-57ab-8f46-f3d9b9d9b92c\&user_id=ac4ca2966d3316b2e2f6ffbb6a584192}{state's
new reopening rules}, Mayor London Breed said in a news conference that
the learning hubs would open as planned in mid-September and that
enrollment was nearly full.

``Even when we provided them with devices and internet service, they are
still falling further behind,'' Ms. Breed said. ``It is so important we
opened these learning hubs, and we are almost at capacity.''

Still, Liana Chavarín, a single parent and director of the
\href{http://berkeleyforestschool.org}{Berkeley Forest School},
expressed frustration over the lack of options currently available to
families. Her school district **** in Berkeley has a program that offers
families a more flexible schedule, but the program is already full.

``If a single parent like myself doesn't have the option to work from
home and my family doesn't have the option to enroll in the district's
independent study program, that leaves me with no choice but to unenroll
my child,'' she said.

Ms. Chavarín decided to incorporate the
\href{http://www.berkeleyforestschool.org/roots-and-waters-collective}{Roots
\& Waters Collective}, a cohort of elementary school-age children, into
her school. Many of her students come from low-income households or are
children of essential workers, she said. **** Because students and
teachers at her forest school spend their days outdoors in small groups,
they are able to learn and explore at a safe distance from one another.
The school offers tuition on a sliding scale.

In Alpine, Dr. Newman said he planned on welcoming students back to
campus in the coming weeks in a hybrid learning model that combines
virtual and in-person classes. However, because students will be
attending campus only on certain days, the learning pods will stay in
place.

Dr. Newman said that single parents, parents of children with special
needs and families with means have all signed on to use the pods.

``I think this is a good example of how we might change education down
the road.'' he said. The question now is: ``How do we provide more
opportunities for students to learn, parents to be engaged and use the
resources of a school district?''

\emph{\textbf{Have a question about how the pandemic is changing daily
life in California?}}
\textbf{\href{https://nl.nytimes3xbfgragh.onion/f/newsletter/RYd0nNn_JZ_1cV4f3uy2gg~~/AAAAAQA~/RgRhIlGqP0TgaHR0cHM6Ly93d3cubnl0aW1lcy5jb20vMjAyMC8wNi8xNy91cy9jb3JvbmF2aXJ1cy1jYWxpZm9ybmlhLWxpZmUuaHRtbD9jYW1wYWlnbl9pZD00OSZlbWM9ZWRpdF9jYV8yMDIwMDgyMSZpbnN0YW5jZV9pZD0yMTQ4OSZubD1jYWxpZm9ybmlhLXRvZGF5JnJlZ2lfaWQ9Njg1MTk1NzMmc2VnbWVudF9pZD0zNjY5NiZ0ZT0xJnVzZXJfaWQ9YWM0Y2EyOTY2ZDMzMTZiMmUyZjZmZmJiNmE1ODQxOTJXA255dEIKACuqzD9f34_zJFIbbWFyaWUubWNkZXJtb3R0QG55dGltZXMuY29tWAQAAAAA}{\emph{Click
here to submit.}}}

\begin{center}\rule{0.5\linewidth}{\linethickness}\end{center}

\emph{California Today goes live at 6:30 a.m. Pacific time weekdays.
Tell us what you want to see:}
\href{mailto:CAtoday@NYTimes.com}{\emph{CAtoday@NYTimes.com}}\emph{.
Were you forwarded this email?}
\href{https://www.nytimes3xbfgragh.onion/newsletters/california-today?module=inline}{\emph{Sign
up for California Today here}} \emph{and}
\href{https://www.nytimes3xbfgragh.onion/column/california-today}{\emph{read
every edition online here}}\emph{.}

\emph{Jill Cowan grew up in Orange County, went to school at U.C.
Berkeley and has reported all over the state, including the Bay Area,
Bakersfield and Los Angeles --- but she always wants to see more. Follow
along here or on}
\href{https://twitter.com/JillCowan}{\emph{Twitter}}\emph{.}

\emph{California Today is edited by Julie Bloom, who grew up in Los
Angeles and graduated from U.C. Berkeley.}

Advertisement

\protect\hyperlink{after-bottom}{Continue reading the main story}

\hypertarget{site-index}{%
\subsection{Site Index}\label{site-index}}

\hypertarget{site-information-navigation}{%
\subsection{Site Information
Navigation}\label{site-information-navigation}}

\begin{itemize}
\tightlist
\item
  \href{https://help.nytimes3xbfgragh.onion/hc/en-us/articles/115014792127-Copyright-notice}{©~2020~The
  New York Times Company}
\end{itemize}

\begin{itemize}
\tightlist
\item
  \href{https://www.nytco.com/}{NYTCo}
\item
  \href{https://help.nytimes3xbfgragh.onion/hc/en-us/articles/115015385887-Contact-Us}{Contact
  Us}
\item
  \href{https://www.nytco.com/careers/}{Work with us}
\item
  \href{https://nytmediakit.com/}{Advertise}
\item
  \href{http://www.tbrandstudio.com/}{T Brand Studio}
\item
  \href{https://www.nytimes3xbfgragh.onion/privacy/cookie-policy\#how-do-i-manage-trackers}{Your
  Ad Choices}
\item
  \href{https://www.nytimes3xbfgragh.onion/privacy}{Privacy}
\item
  \href{https://help.nytimes3xbfgragh.onion/hc/en-us/articles/115014893428-Terms-of-service}{Terms
  of Service}
\item
  \href{https://help.nytimes3xbfgragh.onion/hc/en-us/articles/115014893968-Terms-of-sale}{Terms
  of Sale}
\item
  \href{https://spiderbites.nytimes3xbfgragh.onion}{Site Map}
\item
  \href{https://help.nytimes3xbfgragh.onion/hc/en-us}{Help}
\item
  \href{https://www.nytimes3xbfgragh.onion/subscription?campaignId=37WXW}{Subscriptions}
\end{itemize}
