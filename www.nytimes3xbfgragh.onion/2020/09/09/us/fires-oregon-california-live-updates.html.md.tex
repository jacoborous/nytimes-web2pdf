Sections

SEARCH

\protect\hyperlink{site-content}{Skip to
content}\protect\hyperlink{site-index}{Skip to site index}

\href{https://www.nytimes3xbfgragh.onion/section/us}{U.S.}

\href{https://myaccount.nytimes3xbfgragh.onion/auth/login?response_type=cookie\&client_id=vi}{}

\href{https://www.nytimes3xbfgragh.onion/section/todayspaper}{Today's
Paper}

\href{/section/us}{U.S.}\textbar{}7 People Die in West Coast Wildfires

\url{https://nyti.ms/3m8nQZx}

\begin{itemize}
\item
\item
\item
\item
\item
\end{itemize}

\hypertarget{wildfires-in-the-west}{%
\subsubsection{\texorpdfstring{\href{https://www.nytimes3xbfgragh.onion/spotlight/california-wildfires?name=styln-california-wildfires\&region=TOP_BANNER\&block=storyline_menu_recirc\&action=click\&pgtype=Article\&impression_id=805822c0-f52b-11ea-bf06-e35e969503ac\&variant=undefined}{Wildfires
in the West}}{Wildfires in the West}}\label{wildfires-in-the-west}}

\begin{itemize}
\tightlist
\item
  live\href{https://www.nytimes3xbfgragh.onion/2020/09/12/us/wildfires-live-updates.html?name=styln-california-wildfires\&region=TOP_BANNER\&block=storyline_menu_recirc\&action=click\&pgtype=Article\&impression_id=805849d0-f52b-11ea-bf06-e35e969503ac\&variant=undefined}{Fires
  Updates}
\item
  \href{https://www.nytimes3xbfgragh.onion/interactive/2020/us/fires-map-tracker.html?name=styln-california-wildfires\&region=TOP_BANNER\&block=storyline_menu_recirc\&action=click\&pgtype=Article\&impression_id=805849d1-f52b-11ea-bf06-e35e969503ac\&variant=undefined}{Maps
  of the Fires}
\item
  \href{https://www.nytimes3xbfgragh.onion/article/wildfires-photos-california-oregon-washington-state.html?name=styln-california-wildfires\&region=TOP_BANNER\&block=storyline_menu_recirc\&action=click\&pgtype=Article\&impression_id=805849d2-f52b-11ea-bf06-e35e969503ac\&variant=undefined}{Photos}
\item
  \href{https://www.nytimes3xbfgragh.onion/2020/09/10/us/climate-change-california-wildfires.html?name=styln-california-wildfires\&region=TOP_BANNER\&block=storyline_menu_recirc\&action=click\&pgtype=Article\&impression_id=805849d3-f52b-11ea-bf06-e35e969503ac\&variant=undefined}{A
  Climate Reckoning}
\item
  \href{https://www.nytimes3xbfgragh.onion/article/wildfires-california-oregon-washington.html?name=styln-california-wildfires\&region=TOP_BANNER\&block=storyline_menu_recirc\&action=click\&pgtype=Article\&impression_id=805849d4-f52b-11ea-bf06-e35e969503ac\&variant=undefined}{Answers
  to Your Questions}
\item
  \href{https://www.nytimes3xbfgragh.onion/2020/09/09/us/california-wildfires.html?name=styln-california-wildfires\&region=TOP_BANNER\&block=storyline_menu_recirc\&action=click\&pgtype=Article\&impression_id=805870e0-f52b-11ea-bf06-e35e969503ac\&variant=undefined}{Newsletter}
\end{itemize}

Advertisement

\protect\hyperlink{after-top}{Continue reading the main story}

Supported by

\protect\hyperlink{after-sponsor}{Continue reading the main story}

\hypertarget{7-people-die-in-west-coast-wildfires}{%
\section{7 People Die in West Coast
Wildfires}\label{7-people-die-in-west-coast-wildfires}}

Strong winds were likely to continue to propel the extraordinary number
of fires burning in California, Oregon and Washington.

Published Sept. 9, 2020Updated Sept. 10, 2020

\begin{itemize}
\item
\item
\item
\item
\item
\end{itemize}

\hypertarget{heres-what-you-need-to-know}{%
\subsubsection{Here's what you need to
know:}\label{heres-what-you-need-to-know}}

\begin{itemize}
\tightlist
\item
  \protect\hyperlink{link-1a2d0777}{The death toll rises as wildfires
  spread at an astonishing rate.}
\item
  \protect\hyperlink{link-6e18e4e0}{Hundreds of homes were destroyed in
  Oregon, officials said.}
\item
  \protect\hyperlink{link-2a88e094}{The sun rose, yet the sky stayed
  ominously dark in Northern California.}
\item
  \protect\hyperlink{link-150005eb}{`I can't believe this devastation':
  A blaze sweeps through a small Washington town.}
\item
  \protect\hyperlink{link-64853c8a}{Helicopter pilots say rescue
  missions were the `toughest flying' they've ever done.}
\item
  \protect\hyperlink{link-27c03528}{There is a strong link between
  California's wildfires and climate change, experts say.}
\end{itemize}

\includegraphics{https://static01.graylady3jvrrxbe.onion/images/2020/09/10/us/10fires-briefing-copy/10fires-briefing-copy-videoSixteenByNine3000-v4.jpg}

\hypertarget{the-death-toll-rises-as-wildfires-spread-at-an-astonishing-rate}{%
\subsection{The death toll rises as wildfires spread at an astonishing
rate.}\label{the-death-toll-rises-as-wildfires-spread-at-an-astonishing-rate}}

At least seven people were pronounced dead in
\href{https://www.nytimes3xbfgragh.onion/2020/09/10/us/wildfires-live-updates.html}{wildfires}
in Washington, Oregon and California on Wednesday, with blazes being
driven by high winds and fueled by recent heat waves.

A 1-year-old boy was killed in the Cold Springs Fire in northern
Washington, one person was killed near Ashland, Ore., two victims were
discovered in a vehicle east of Salem, Ore., and three people were found
dead in Butte County, Calif., according to the county sheriff's offices.

The devastation of deaths and scorched homes came during unprecedented
fire seasons for several states in the Pacific Northwest. In Northern
California, the fast-moving
\href{https://www.nytimes3xbfgragh.onion/2020/09/10/us/wildfires-help-ca-or-wa.html}{Bear
Fire} created apocalyptic scenes as smoke-filled air settled over the
Bay Area and produced an ominous orange glow. The blaze forced thousands
of people to flee their homes.

\href{https://www.nytimes3xbfgragh.onion/interactive/2020/us/fires-map-tracker.html}{}

\includegraphics{https://static01.graylady3jvrrxbe.onion/images/2020/09/11/us/fires-map-tracker-1599839565497/fires-map-tracker-1599839565497-articleLarge.png}

\hypertarget{california-oregon-and-washington-fire-tracking-maps}{%
\subsection{California, Oregon and Washington Fire Tracking
Maps}\label{california-oregon-and-washington-fire-tracking-maps}}

Maps showing air quality and where major fires are burning in the
Western states.

Fires appeared even more destructive in Oregon, where officials said a
wildfire driven by 45-mile-per-hour wind gusts tore through two towns,
destroying more than a thousand homes and raising fears that some people
had not been able to escape.

``We expect to see a great deal of loss, both in structures and in human
lives,'' Gov. Kate Brown of Oregon said. ``This could be the greatest
loss of human lives and property due to wildfire in our state's
history.''

In one town, Phoenix, Mayor Chris Lux estimated that 1,000 homes may
have been lost. In nearby Talent, hundreds more homes were incinerated.

``Everything is completely gone,'' said Sandra Spelliscy, Talent's city
manager.

California's wildfire season is already the most severe in modern
history, measured by acres burned. More than 2.5 million acres of land
have burned in the state this year, nearly 20 times what had burned at
this time last year.

In Washington, Gov. Jay Inslee said that 480,000 acres had burned across
the state this week, more than almost every recent fire season. Nearly
all of the homes and municipal buildings --- including the post office
and the fire station --- in the small town of Malden burned to the
ground.

\emph{{[}Sign up}
\href{https://www.nytimes3xbfgragh.onion/newsletters/california-today}{\emph{for
California Today}}\emph{, our daily newsletter from the Golden
State.{]}}

\hypertarget{hundreds-of-homes-were-destroyed-in-oregon-officials-said}{%
\subsection{Hundreds of homes were destroyed in Oregon, officials
said.}\label{hundreds-of-homes-were-destroyed-in-oregon-officials-said}}

\includegraphics{https://static01.graylady3jvrrxbe.onion/images/2020/09/09/us/09fires-briefing-ore/merlin_176781735_2e5a97ff-d6b5-4f63-a71e-ddf66553bd54-articleLarge.jpg?quality=75\&auto=webp\&disable=upscale}

About 35 wildfires fueled by hot, dry winds have burned more than
300,000 acres across Oregon, causing widespread evacuations and possibly
destroying entire communities.

Gov. Kate Brown said at a news briefing on Wednesday that some of the
towns that have been ``substantially destroyed'' include Detroit, in
central Oregon; Blue River and Vida, east of Eugene; and Phoenix and
Talent, in the state's southwest. Many residents have been rescued, some
even pulled from rivers to safety, Ms. Brown said.

On Wednesday evening, Sheriff Joe Kast of Marion County said crews had
found two people dead in a vehicle from a wildfire east of Salem. He
said searchers fear that they could find more bodies as rescue efforts
continue. Sheriff Nathan Sickler of Jackson County said one fatality was
identified near the start of a fire in the Ashland area.

Chris Luz, the mayor of Phoenix, a town of about 7,000, estimated that
the area may have lost some 1,000 homes and apartment units. He said
that the downtown area was decimated, with many businesses lost, and
that the fires continued to smolder on Wednesday.

Mr. Luz said the fire had rushed into town propelled by winds of about
45 miles per hour, leaving residents with little time to evacuate. Some
people reported on social media that they were unable to get back to
their homes to get their pets. Officials had not found anyone who died
in the fire, but Mr. Luz worried that some people may not have gotten
out in time.

``It's just devastating,'' Mr. Luz said.

Hundreds of homes and other buildings were wiped out in the nearby town
of Talent, the city manager, Sandra Spelliscy, said.

Ms. Spelliscy said residents there also had little time to evacuate,
forcing them to leave belongings behind. She said the evacuation was
complicated by traffic that had been diverted off Interstate 5 when that
highway closed. But she said police officers and other crews worked to
get people out of the city, so she was hopeful that everyone had managed
to escape in time.

That blaze, known as the Almeda Fire, was also encroaching on the city
of Medford, home to about 80,000 people.
\href{https://twitter.com/rajmathai/status/1303565121097064449}{Videos}
posted on social media showed flaming hillsides and clouds of smoke
approaching the city's neighborhoods.

Although winds had subsided on Wednesday, the gusts were still
problematic, fire officials noted at the briefing, particularly the
winds pushing blazes forward on the west slope of the Cascades.

Officials could not provide a count for fatalities or missing people
because they have not been able to reach some of the areas hardest hit
by the fires, they said, adding that they expect the numbers to rise
over the next couple of days.

``The worst fire conditions in three decades persist,'' Ms. Brown said.

\hypertarget{the-sun-rose-yet-the-sky-stayed-ominously-dark-in-northern-california}{%
\subsection{The sun rose, yet the sky stayed ominously dark in Northern
California.}\label{the-sun-rose-yet-the-sky-stayed-ominously-dark-in-northern-california}}

\includegraphics{https://static01.graylady3jvrrxbe.onion/images/2020/09/09/us/09fires-sky01alt/09fires-sky01alt-videoSixteenByNine3000.jpg}

Hours after sunrise on Wednesday residents of the San Francisco Bay Area
waited for daylight. Instead they got only the faintest suggestion that
somewhere above the smoky skies the sun had indeed risen.

Some called it a nuclear winter. Cars kept their headlights on. Offices
towers in San Francisco, where the smoke is mixing with fog, were
illuminated as if in the middle of the night.

Across Northern California huge plumes of smoke from a fire that blasted
through the foothills of the Sierra Nevada sent giant plumes of smoke
high into the atmosphere, blotting out the sun.

The Bear Fire added to the smoke already pumped into the atmosphere by
the more than 20 large fires burning across California. Craig Shoemaker,
a meteorologist with the National Weather Service in Sacramento, said
the massive volume of smoke rose up to 40,000 feet overnight.

``We have a huge cloud of ash and ice,'' he said, adding that it
resembled thunderstorm clouds.

Fires are essentially creating their own weather, Mr. Shoemaker said.

``Without the smoke it would be a clear day,'' he said. ``This is all
generated from the fires.''

The Bear Fire grew overnight at an astonishing rate of a thousand acres
every half-hour as it bore down on communities surrounding Oroville. It
was burning in some of the same areas as the Camp Fire in 2018, which
destroyed the town of Paradise.

On Wednesday evening, Sheriff Kory L. Honea of Butte County said in a
news conference that crews had found three people dead in the county ---
two of them in the same location.

\hypertarget{i-cant-believe-this-devastation-a-blaze-sweeps-through-a-small-washington-town}{%
\subsection{`I can't believe this devastation': A blaze sweeps through a
small Washington
town.}\label{i-cant-believe-this-devastation-a-blaze-sweeps-through-a-small-washington-town}}

Image

Homes were destroyed by wildfire in Malden, Wash.Credit...Jesse
Tinsley/The Spokesman-Review, via Associated Press

The wildfires that ripped through eastern and central Washington this
week devastated communities, killing a 1-year-old and leaving the boy's
parents with third-degree burns.

\href{https://www.nytimes3xbfgragh.onion/spotlight/california-wildfires}{Wildfires
in the West ›}

\hypertarget{live-updates}{%
\subsection{\texorpdfstring{\href{https://www.nytimes3xbfgragh.onion/2020/09/12/us/wildfires-live-updates.html}{Live
Updates}}{Live Updates}}\label{live-updates}}

Updated~

Sept. 12, 2020, 2:53 p.m. ET

\begin{itemize}
\tightlist
\item
  \href{https://www.nytimes3xbfgragh.onion/2020/09/12/us/wildfires-live-updates.html\#link-f3961ff}{President
  Trump will visit California on Monday after destructive fires.}
\item
  \href{https://www.nytimes3xbfgragh.onion/2020/09/12/us/wildfires-live-updates.html\#link-7e503ae9}{Shifting
  weather may improve firefighting conditions on the West Coast.}
\item
  \href{https://www.nytimes3xbfgragh.onion/2020/09/12/us/wildfires-live-updates.html\#link-5e4c548d}{Oregon's
  fire marshal is temporarily replaced as firefighters battle blazes.}
\end{itemize}

Among the hardest-hit places was the old railroad town of Malden, where
deputies rushed through the streets and screamed for residents to flee
as the flames roared toward town. By Tuesday afternoon, most of the
town's homes were destroyed, along with City Hall, the post office, the
library and the fire station.

``I've seen this kind of loss before, dozens of times,'' said Royle
Hehr, a resident who used to run a flood and fire restoration business
in Arizona. ``I've worked with people who lost everything. I can't
believe this devastation.''

On Wednesday, volunteers handed out doughnuts and bottled water.
Portable toilets and hand-washing stations were set up as wispy tails of
smoke from smoldering debris --- homes, outbuildings, trees, vegetation
and power poles --- corkscrewed into the late-summer skies.

Four miles down the two-lane county road, three or four large grain
bins, filled with recently harvested wheat, continued to burn. One had
split open, its commodity ablaze on the ground like sawdust logs.

In northern Washington, a 1-year-old boy was killed in the Cold Springs
Fire after the child and his parents attempted to flee their property,
the Okanogan County Sheriff's Office said. The family was found along
the bank of the Columbia River on Wednesday morning, and the parents
were flown to a hospital in Seattle with third-degree burns.

``It's an extreme tragedy for any loss of life,'' Sheriff Tony Hawley
said.

\hypertarget{helicopter-pilots-say-rescue-missions-were-the-toughest-flying-theyve-ever-done}{%
\subsection{Helicopter pilots say rescue missions were the `toughest
flying' they've ever
done.}\label{helicopter-pilots-say-rescue-missions-were-the-toughest-flying-theyve-ever-done}}

Image

Dozens of people were evacuated by a California National Guard
helicopter on Saturday after being trapped by the Creek Fire in the
Sierra National Forest.Credit...California National Guard, via
EPA/Shutterstock

The California National Guard is routinely called to help with
search-and-rescue operations on land and at sea, but members of the
Guard say they have seen nothing like this.

In a scene that played out multiple times over the weekend and into
Tuesday afternoon, the National Guard airlifted hundreds of civilians
out of the Sierra National Forest, their exits trapped by a dense ring
of fire.

Pilots involved in the rescues said it was the most harrowing flying
they have done in their careers. Crew members became nauseated from the
smoke. They flew up a valley in strong winds, surpassing ridgelines
illuminated by fire. They contemplated turning back.

As of noon on Tuesday, 362 people and at least 16 dogs had been
evacuated by air from burning forests of cedar and ponderosa pine. The
Creek Fire, which ignited on Friday evening, had burned 143,929 acres
--- five times the size of San Francisco --- and was still raging out of
control. It is one of more than 20 wildfires in California.

``Every piece of vegetation as far as you could see around that lake was
on fire,'' Chief Warrant Officer Kipp Goding, the pilot of a Blackhawk
helicopter, said in a briefing.

``I've been flying for 25 years,'' he said, removing a cloth mask to
speak. ``We get occasionally shot at overseas during missions. It's
definitely by far the toughest flying that I've ever done,'' he said of
the rescue missions in California.

Chief Warrant Officer Joseph Rosamond, the pilot of the Chinook, said in
an interview on Tuesday that as someone born and raised in the state,
the fires were particularly affecting.

``It's really sad that California has to go through all these disasters
--- it seems like one after another,'' he said. Over the past four
years, the state has suffered fires, flooding, mudslides and an
earthquake on the edge of the desert.

``As a citizen of California it gets really draining,'' he said.

\hypertarget{there-is-a-strong-link-between-californias-wildfires-and-climate-change-experts-say}{%
\subsection{There is a strong link between California's wildfires and
climate change, experts
say.}\label{there-is-a-strong-link-between-californias-wildfires-and-climate-change-experts-say}}

\includegraphics{https://static01.graylady3jvrrxbe.onion/images/2020/09/08/us/08fire-vid02/08fire-vid02-videoSixteenByNine3000.jpg}

While California's climate has always made the state prone to fires,
\href{https://agupubs.onlinelibrary.wiley.com/doi/full/10.1029/2019EF001210}{the
link between human-caused climate change and bigger fires is
inextricable}, said Park Williams, a bioclimatologist at Columbia
University's Lamont-Doherty Earth Observatory. ``This climate change
connection is straightforward: Warmer temperatures dry out fuels,'' he
said. ``In areas with abundant and very dry fuels, all you need is a
spark.''

``In pretty much every single way, a perfect recipe for fire is just
kind of written in California,'' Dr. Williams said. ``Nature creates the
perfect conditions for fire, as long as people are there to start the
fires. But then climate change, in a few different ways, seems to also
load the dice toward more fire in the future.''

Even if the conditions are right for a wildfire, you still need
something or someone to ignite it. Sometimes the trigger is nature, like
the unusual lightning strikes that set off the L.N.U. Lightning Complex
fires in August, but more often than not humans are responsible, said
Nina S. Oakley, a research scientist at the Scripps Institution of
Oceanography.

Whether it is downed power lines or the fire ignited last weekend by
smoke-generating fireworks
\href{https://www.nytimes3xbfgragh.onion/2020/09/07/us/gender-reveal-party-wildfire.html}{as
part of a gender-reveal party}, humans tend to play a part --- and not
just in the initial trigger of a blaze, she said.

``You also have the human contribution to wildfire,'' which includes the
warming that has been caused by greenhouse gas emissions and the
accompanying increased drying, as well as forest policies that involved
suppressing fires instead of letting some burn, leaving fuel in place.
Those factors, she said, are ``contributing to creating a situation
favorable to wildfire.''

Gov. Gavin Newsom, who has often held up California as an example of the
consequences of climate change, said on Tuesday that he had ``no
patience for climate change deniers.''

``Never have I felt more of a sense of obligation and a sense of purpose
to maintain California's leadership not only nationally but
internationally to face climate change head on,'' he said.

Reporting was contributed by Mike Baker, Nicholas Bogel-Burroughs, Coral
Davenport, Thomas Fuller, Giulia McDonnell Nieto del Rio, Sarah Mervosh,
John Schwartz, Jeanna Smialek, Lucy Tompkins and Will Wright

Advertisement

\protect\hyperlink{after-bottom}{Continue reading the main story}

\hypertarget{site-index}{%
\subsection{Site Index}\label{site-index}}

\hypertarget{site-information-navigation}{%
\subsection{Site Information
Navigation}\label{site-information-navigation}}

\begin{itemize}
\tightlist
\item
  \href{https://help.nytimes3xbfgragh.onion/hc/en-us/articles/115014792127-Copyright-notice}{©~2020~The
  New York Times Company}
\end{itemize}

\begin{itemize}
\tightlist
\item
  \href{https://www.nytco.com/}{NYTCo}
\item
  \href{https://help.nytimes3xbfgragh.onion/hc/en-us/articles/115015385887-Contact-Us}{Contact
  Us}
\item
  \href{https://www.nytco.com/careers/}{Work with us}
\item
  \href{https://nytmediakit.com/}{Advertise}
\item
  \href{http://www.tbrandstudio.com/}{T Brand Studio}
\item
  \href{https://www.nytimes3xbfgragh.onion/privacy/cookie-policy\#how-do-i-manage-trackers}{Your
  Ad Choices}
\item
  \href{https://www.nytimes3xbfgragh.onion/privacy}{Privacy}
\item
  \href{https://help.nytimes3xbfgragh.onion/hc/en-us/articles/115014893428-Terms-of-service}{Terms
  of Service}
\item
  \href{https://help.nytimes3xbfgragh.onion/hc/en-us/articles/115014893968-Terms-of-sale}{Terms
  of Sale}
\item
  \href{https://spiderbites.nytimes3xbfgragh.onion}{Site Map}
\item
  \href{https://help.nytimes3xbfgragh.onion/hc/en-us}{Help}
\item
  \href{https://www.nytimes3xbfgragh.onion/subscription?campaignId=37WXW}{Subscriptions}
\end{itemize}
