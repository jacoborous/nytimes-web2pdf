Sections

SEARCH

\protect\hyperlink{site-content}{Skip to
content}\protect\hyperlink{site-index}{Skip to site index}

\href{https://www.nytimes3xbfgragh.onion/section/technology}{Technology}

\href{https://myaccount.nytimes3xbfgragh.onion/auth/login?response_type=cookie\&client_id=vi}{}

\href{https://www.nytimes3xbfgragh.onion/section/todayspaper}{Today's
Paper}

\href{/section/technology}{Technology}\textbar{}Facebook May Be Ordered
to Change Data Practices in Europe

\url{https://nyti.ms/2Zi2LlR}

\begin{itemize}
\item
\item
\item
\item
\item
\end{itemize}

Advertisement

\protect\hyperlink{after-top}{Continue reading the main story}

Supported by

\protect\hyperlink{after-sponsor}{Continue reading the main story}

\hypertarget{facebook-may-be-ordered-to-change-data-practices-in-europe}{%
\section{Facebook May Be Ordered to Change Data Practices in
Europe}\label{facebook-may-be-ordered-to-change-data-practices-in-europe}}

Irish regulators have started an inquiry into Facebook's movement of
data on European users to the United States.

\includegraphics{https://static01.graylady3jvrrxbe.onion/images/2020/09/09/business/09FBDATA/09FBDATA-articleLarge.jpg?quality=75\&auto=webp\&disable=upscale}

By \href{https://www.nytimes3xbfgragh.onion/by/adam-satariano}{Adam
Satariano}

\begin{itemize}
\item
  Sept. 9, 2020
\item
  \begin{itemize}
  \item
  \item
  \item
  \item
  \item
  \end{itemize}
\end{itemize}

Facebook is facing the prospect of not being able to move data about its
European users to the United States, after European regulators raised
concerns that such transfers do not adequately protect the information
from American government surveillance.

The social network
\href{https://about.fb.com/news/2020/09/securing-the-long-term-stability-of-cross-border-data-flows/}{said
on Wednesday} that the Irish Data Protection Commission had begun an
inquiry into its movement of data on European users to the United
States. The Irish regulator oversees Facebook's data practices in Europe
and can fine it up to 4 percent of its global revenue for breaking
European data protection laws.

The Silicon Valley company may now have to overhaul its operations to
keep data on Europeans stored within the European Union, an immensely
complicated task given the way that Facebook moves data among data
centers around the world.

The inquiry,
\href{https://www.wsj.com/articles/ireland-to-order-facebook-to-stop-sending-user-data-to-u-s-11599671980?mod=djemalertNEWS}{earlier
reported} by The Wall Street Journal, is the first major fallout of a
European Union high court decision in July that invalidated a key
trans-Atlantic agreement called Privacy Shield. That agreement between
the United States and European Union had allowed businesses to send data
between the two regions, but the court struck it down, saying Europeans
did not have sufficient protections from American spy agencies.

The ruling affects thousands of businesses, but Facebook's data-sharing
practices have been under particular scrutiny by European authorities.
Facebook had argued that the court allowed certain kinds of legal
contracts to continue transferring data, but Irish regulators disagreed
and said those arrangements were invalid.

Facebook has until later this month to respond to Ireland's complaint,
then the Irish regulator will make a final decision toward the end of
the year. Facebook could challenge that judgment in court.

``A lack of safe, secure and legal international data transfers would
damage the economy and hamper the growth of data-driven businesses in
the E.U., just as we seek a recovery from Covid-19,'' Nick Clegg,
Facebook's vice president of global affairs, said of the moves. ``The
impact would be felt by businesses large and small, across multiple
sectors.''

Ireland's Data Protection Commission declined to comment.

Facebook's experience will be closely watched by other major tech
companies, like Google, that also depend on transferring data between
the United States and Europe.

American and European officials have expressed a desire to work out a
new data-sharing agreement. But legal experts have said complying with
the European court ruling will require substantive changes to American
surveillance laws to give Europeans added privacy protections.

Advertisement

\protect\hyperlink{after-bottom}{Continue reading the main story}

\hypertarget{site-index}{%
\subsection{Site Index}\label{site-index}}

\hypertarget{site-information-navigation}{%
\subsection{Site Information
Navigation}\label{site-information-navigation}}

\begin{itemize}
\tightlist
\item
  \href{https://help.nytimes3xbfgragh.onion/hc/en-us/articles/115014792127-Copyright-notice}{©~2020~The
  New York Times Company}
\end{itemize}

\begin{itemize}
\tightlist
\item
  \href{https://www.nytco.com/}{NYTCo}
\item
  \href{https://help.nytimes3xbfgragh.onion/hc/en-us/articles/115015385887-Contact-Us}{Contact
  Us}
\item
  \href{https://www.nytco.com/careers/}{Work with us}
\item
  \href{https://nytmediakit.com/}{Advertise}
\item
  \href{http://www.tbrandstudio.com/}{T Brand Studio}
\item
  \href{https://www.nytimes3xbfgragh.onion/privacy/cookie-policy\#how-do-i-manage-trackers}{Your
  Ad Choices}
\item
  \href{https://www.nytimes3xbfgragh.onion/privacy}{Privacy}
\item
  \href{https://help.nytimes3xbfgragh.onion/hc/en-us/articles/115014893428-Terms-of-service}{Terms
  of Service}
\item
  \href{https://help.nytimes3xbfgragh.onion/hc/en-us/articles/115014893968-Terms-of-sale}{Terms
  of Sale}
\item
  \href{https://spiderbites.nytimes3xbfgragh.onion}{Site Map}
\item
  \href{https://help.nytimes3xbfgragh.onion/hc/en-us}{Help}
\item
  \href{https://www.nytimes3xbfgragh.onion/subscription?campaignId=37WXW}{Subscriptions}
\end{itemize}
