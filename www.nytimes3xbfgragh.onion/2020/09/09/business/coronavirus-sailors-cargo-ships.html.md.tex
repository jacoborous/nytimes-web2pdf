Sections

SEARCH

\protect\hyperlink{site-content}{Skip to
content}\protect\hyperlink{site-index}{Skip to site index}

\href{https://www.nytimes3xbfgragh.onion/section/business}{Business}

\href{https://myaccount.nytimes3xbfgragh.onion/auth/login?response_type=cookie\&client_id=vi}{}

\href{https://www.nytimes3xbfgragh.onion/section/todayspaper}{Today's
Paper}

\href{/section/business}{Business}\textbar{}Trapped by Pandemic, Ships'
Crews Fight Exhaustion and Despair

\url{https://nyti.ms/3m84pQW}

\begin{itemize}
\item
\item
\item
\item
\item
\end{itemize}

\hypertarget{the-coronavirus-outbreak}{%
\subsubsection{\texorpdfstring{\href{https://www.nytimes3xbfgragh.onion/news-event/coronavirus?name=styln-coronavirus-markets\&region=TOP_BANNER\&block=storyline_menu_recirc\&action=click\&pgtype=Article\&impression_id=0407ed00-f282-11ea-b183-fd6e653e4bb2\&variant=undefined}{The
Coronavirus
Outbreak}}{The Coronavirus Outbreak}}\label{the-coronavirus-outbreak}}

\begin{itemize}
\tightlist
\item
  live\href{https://www.nytimes3xbfgragh.onion/2020/09/09/world/covid-19-coronavirus.html?name=styln-coronavirus-markets\&region=TOP_BANNER\&block=storyline_menu_recirc\&action=click\&pgtype=Article\&impression_id=04081410-f282-11ea-b183-fd6e653e4bb2\&variant=undefined}{Latest
  Updates}
\item
  \href{https://www.nytimes3xbfgragh.onion/interactive/2020/us/coronavirus-us-cases.html?name=styln-coronavirus-markets\&region=TOP_BANNER\&block=storyline_menu_recirc\&action=click\&pgtype=Article\&impression_id=04081411-f282-11ea-b183-fd6e653e4bb2\&variant=undefined}{Maps
  and Cases}
\item
  \href{https://www.nytimes3xbfgragh.onion/interactive/2020/science/coronavirus-vaccine-tracker.html?name=styln-coronavirus-markets\&region=TOP_BANNER\&block=storyline_menu_recirc\&action=click\&pgtype=Article\&impression_id=04081412-f282-11ea-b183-fd6e653e4bb2\&variant=undefined}{Vaccine
  Tracker}
\item
  \href{https://www.nytimes3xbfgragh.onion/2020/09/02/your-money/eviction-moratorium-covid.html?name=styln-coronavirus-markets\&region=TOP_BANNER\&block=storyline_menu_recirc\&action=click\&pgtype=Article\&impression_id=04081413-f282-11ea-b183-fd6e653e4bb2\&variant=undefined}{Eviction
  Moratorium}
\item
  \href{https://www.nytimes3xbfgragh.onion/interactive/2020/09/02/magazine/food-insecurity-hunger-us.html?name=styln-coronavirus-markets\&region=TOP_BANNER\&block=storyline_menu_recirc\&action=click\&pgtype=Article\&impression_id=04081414-f282-11ea-b183-fd6e653e4bb2\&variant=undefined}{American
  Hunger}
\end{itemize}

Advertisement

\protect\hyperlink{after-top}{Continue reading the main story}

Supported by

\protect\hyperlink{after-sponsor}{Continue reading the main story}

\hypertarget{trapped-by-pandemic-ships-crews-fight-exhaustion-and-despair}{%
\section{Trapped by Pandemic, Ships' Crews Fight Exhaustion and
Despair}\label{trapped-by-pandemic-ships-crews-fight-exhaustion-and-despair}}

When borders closed, seafarers on ships around the world suddenly had no
way home. Half a year later, there's no solution in sight.

\includegraphics{https://static01.graylady3jvrrxbe.onion/images/2020/09/10/world/10VIRUS-SEAFARERS1/merlin_176516718_b45cbcf5-12d5-4776-9175-7b271ec1faca-articleLarge.jpg?quality=75\&auto=webp\&disable=upscale}

By Aurora Almendral

\begin{itemize}
\item
  Sept. 9, 2020, 3:28 a.m. ET
\item
  \begin{itemize}
  \item
  \item
  \item
  \item
  \item
  \end{itemize}
\end{itemize}

BANGKOK --- Ralph Santillan, a merchant seaman from the Philippines,
hasn't had shore leave in half a year. It has been 18 months since he
reported for duty on his ship, which hauls corn, barley and other
commodities around the world. It has been even longer since he saw his
wife and son.

``There's nothing I can do,'' Mr. Santillan said late last month from
his ship, a 965-foot bulk carrier off South Korea. ``I have to leave to
God whatever might happen here.''

His time on the ship, where he spends long days chipping rust off the
deck or cleaning out cargo holds, was supposed to have ended in
February, after an 11-month stint --- the maximum length for a
seafarer's contract.

But the Covid-19 pandemic led countries to start closing borders and
\href{https://www.nytimes3xbfgragh.onion/2020/03/25/world/europe/coronavirus-ship-crews-trapped.html}{refusing
to let sailors come ashore}. For cargo ships around the world, the
process known as crew change, in which seamen like Mr. Santillan are
replaced by new ones as their contracts expire, ground nearly to a halt.

In June, the United Nations called the situation a
``\href{https://www.un.org/sg/en/content/sg/statement/2020-06-12/statement-attributable-the-spokesman-for-the-secretary-general-the-repatriation-of-seafarers}{growing
humanitarian and safety crisis}.'' And there is still no solution in
sight.

Last month, the International Transport Workers' Federation, a
seafarers' union, estimated that 300,000 of the 1.2 million crew members
at sea were essentially stranded on their ships, working past the
expiration of their original contracts and fighting isolation,
uncertainty and fatigue.

``This floating population, many of which have been at sea for over a
year, are reaching the end of their tether,'' Guy Platten, secretary
general of the International Chamber of Shipping, which represents
shipowners, said on Friday. ``If governments do not act quickly and
decisively to facilitate the transfer of crews and ease restrictions
around air travel, we face the very real situation of a slowdown in
global trade.''

\includegraphics{https://static01.graylady3jvrrxbe.onion/images/2020/09/04/world/00seafarers-2/merlin_176516679_9898ad34-2216-4aa9-a270-f4262b86e3c6-articleLarge.jpg?quality=75\&auto=webp\&disable=upscale}

Some crew members have begun refusing to work, forcing ships to stay in
port. And many in the shipping industry fear that the stress and
exhaustion will lead to accidents, perhaps disastrous ones.

``Owners made their contract so short for a reason,'' said Joost Mes,
the director of Avior Marine, a maritime recruitment agency in Manila.
``The consequences are coming closer, and the margins of safety are
getting less.''

\hypertarget{latest-updates-the-coronavirus-outbreak-and-the-economy}{%
\section{\texorpdfstring{\href{https://www.nytimes3xbfgragh.onion/live/2020/09/08/business/stock-market-today-coronavirus?action=click\&pgtype=Article\&state=default\&region=MAIN_CONTENT_1\&context=storylines_live_updates}{Latest
Updates: The Coronavirus Outbreak and the
Economy}}{Latest Updates: The Coronavirus Outbreak and the Economy}}\label{latest-updates-the-coronavirus-outbreak-and-the-economy}}

\href{https://www.nytimes3xbfgragh.onion/live/2020/09/08/business/stock-market-today-coronavirus?action=click\&pgtype=Article\&state=default\&region=MAIN_CONTENT_1\&context=storylines_live_updates\#the-latest-under-armour-announces-layoffs-and-lubys-will-liquidate}{12h
ago}

\href{https://www.nytimes3xbfgragh.onion/live/2020/09/08/business/stock-market-today-coronavirus?action=click\&pgtype=Article\&state=default\&region=MAIN_CONTENT_1\&context=storylines_live_updates\#the-latest-under-armour-announces-layoffs-and-lubys-will-liquidate}{The
latest: Under Armour announces layoffs, and Luby's will liquidate.}

\href{https://www.nytimes3xbfgragh.onion/live/2020/09/08/business/stock-market-today-coronavirus?action=click\&pgtype=Article\&state=default\&region=MAIN_CONTENT_1\&context=storylines_live_updates\#lululemon-reports-a-quarterly-profit-as-consumers-flock-to-yoga-pants}{13h
ago}

\href{https://www.nytimes3xbfgragh.onion/live/2020/09/08/business/stock-market-today-coronavirus?action=click\&pgtype=Article\&state=default\&region=MAIN_CONTENT_1\&context=storylines_live_updates\#lululemon-reports-a-quarterly-profit-as-consumers-flock-to-yoga-pants}{Lululemon
reports a quarterly profit as consumers flock to yoga pants.}

\href{https://www.nytimes3xbfgragh.onion/live/2020/09/08/business/stock-market-today-coronavirus?action=click\&pgtype=Article\&state=default\&region=MAIN_CONTENT_1\&context=storylines_live_updates\#the-work-from-home-challenge-for-employees-of-color}{15h
ago}

\href{https://www.nytimes3xbfgragh.onion/live/2020/09/08/business/stock-market-today-coronavirus?action=click\&pgtype=Article\&state=default\&region=MAIN_CONTENT_1\&context=storylines_live_updates\#the-work-from-home-challenge-for-employees-of-color}{The
work-from-home challenge for employees of color.}

\href{https://www.nytimes3xbfgragh.onion/live/2020/09/08/business/stock-market-today-coronavirus?action=click\&pgtype=Article\&state=default\&region=MAIN_CONTENT_1\&context=storylines_live_updates}{See
more updates}

More live coverage:
\href{https://www.nytimes3xbfgragh.onion/2020/09/09/world/covid-19-coronavirus.html?action=click\&pgtype=Article\&state=default\&region=MAIN_CONTENT_1\&context=storylines_live_updates}{Global}

Seafarers have to stay vigilant. Standing in the wrong spot on deck, or
missing a step on a long, narrow ladder, could mean injury or death. A
distracted watch officer could miss an approaching vessel until it is
too late.

``I can see the fatigue and stress in their faces,'' Mr. Santillan said
in July from his ship, referring to the five men who worked with him on
the deck. ``I'm sure they can see it on my face.'' He said they
sometimes worked 23-hour days to meet their schedules.

Three of the 20 crew members on a
\href{https://www.nytimes3xbfgragh.onion/2020/08/18/world/africa/captain-mauritius-oil-spill-arrested.html}{bulk
carrier} that ran aground off Mauritius in late July,
\href{https://www.nytimes3xbfgragh.onion/2020/08/28/us/mauritius-dolphin-deaths.html?searchResultPosition=1}{spilling
1,000 tons of oil into the pristine waters}, were on extended contracts,
according to
\href{https://lloydslist.maritimeintelligence.informa.com/daily-briefing/2020/08-august/daily-briefing-august-18-2020}{Lloyd's
List}, a maritime intelligence company. The cause of the accident has
not been determined, but the seafarers' union said it pointed to the
potential consequences of having an overworked crew. Two of the ships'
officers have been charged with unsafe navigation.

In a June survey by the seafarers' union, many crew members on extended
contracts said exhaustion was affecting their ability to focus. Some
compared themselves to prisoners or slaves, according to the survey, and
some said they had considered suicide.

Members of one crew had to shave their heads after running out of
shampoo because no one could go ashore for provisions, according to the
survey. Another ship's captain had to pull the tooth of a seafarer who
could not go ashore to see a dentist, a shipping company executive said.

``If someone is hurt, there is no hospital,'' said Burcu Akceken, the
chief officer of a chemical tanker that was anchored off Dakar, Senegal,
who is from Turkey.

Image

Burcu Akceken, the chief officer on a chemical tanker, said prohibitions
on letting seafarers ashore made the work more dangerous. ``If someone
is hurt, there is no hospital,'' she said.~~Credit...via Burcu Akceken

Many stranded crew members said governments should do more to
accommodate crew changes. ``Ports and countries want the cargo, but when
it comes to the crew who are bringing the cargo to them, they are not
helping us,'' said Nilesh Mukherjee, the chief officer on a tanker
carrying liquid petroleum gas, who is from India.

Even in normal times, replacing a crew member involves complex
logistics, said Frederick Kenney, director of legal and external affairs
at the International Maritime Organization, a U.N. agency that oversees
global shipping.

Leaving a ship, and getting home, requires more than just disembarking.
It usually involves multiple border crossings, flights with at least one
connection, and a slew of certificates, specialized visas and
immigration stamps. A crew member's replacement has to go through the
same steps.

Every step in that procedure is ``broken'' because of the pandemic, with
flights limited, border controls tightened and many consulates closed,
according to Mr. Kenney. While some countries have found ways around the
problem, ``the rate of progress is not keeping up with the growing
backlog of seafarers,'' he said last week.

Some ports have exempted crew members from border restrictions, then
backtracked after seafarers, arriving from their home countries to
report for duty on a ship, were found to have Covid-19.

Hong Kong exempted sea as well as airline crews from a 14-day quarantine
requirement, but it
\href{https://www.maritime-executive.com/article/hong-kong-suspends-crew-changes-except-for-cargo-ships-in-port}{changed
those rules in July}, after the exemptions were blamed for a surge in
case numbers. In Singapore, too, protocols were tightened after
seafarers tested positive for the virus on arrival.

Mr. Platten, of the International Chamber of Shipping, said that if the
crisis continued, vessels would inevitably stop sailing. ``It's not
going to be suddenly, tomorrow, that they're all going to stop,'' he
said. ``It'll be a gradual creeping up on this, and that's a real worry
for the global supply chain.''

Some ships have already been idled, at least temporarily, because
seafarers refused to keep working. Under international maritime law, an
undermanned ship cannot sail.

The departure of the Ben Rinnes, chartered to haul soy for Cargill from
Geelong, Australia, was delayed last month after five seafarers demanded
to be sent home; at least one had been working for 17 months. Cargill
said that as a charterer, it did not manage crew changes, but that it
had been involved in discussions that led to the crew members'
departure. In a statement, it said it joined the union's call for
``immediate government action to ensure seafarers can be repatriated.''

Another ship was idled in the Australian port of Fremantle because
seafarers stopped working, and there were at least two similar cases in
which crew members were allowed to disembark in Panama.

While some seafarers have extended their contracts out of a sense of
duty, or because they feared being blacklisted if they didn't, others
have accused captains or employers of intimidation. The Australian
maritime authorities detained the cargo ship Unison Jasper last month
over accusations that its Burmese crew had been abused and forced to
sign contract extensions.

Image

A vessel pulled away in May after receiving fuel from Ms. Akceken's
tanker off the coast of Senegal.~Credit...Burcu Akceken

Mr. Santillan, who boarded his ship in March 2019, was near the end of
his contract when the pandemic hit. After a monthlong voyage from Brazil
to Singapore, which was supposed to be his last stop, he was told that
his flight home to the Philippines had been canceled.

It wasn't clear to him who was responsible --- the airline, his employer
or the Philippine government, which, because of the pandemic, was
letting only a few of its many overseas workers back into the country
each day.

But border restrictions meant that Mr. Santillan wasn't allowed onshore.
And with no one to replace him, the ship would be unable to sail if he
stopped working.

Fearing he'd be blacklisted if he did so, Mr. Santillan signed a new
contract. Since then, he said, his captain has told him at least three
times that he would be allowed to leave, but it hasn't happened.

He and the rest of the crew try to keep one another's spirits up, but
their list of diversions is grimly short: Go to the gym, belt out some
songs on the karaoke machine, or buy internet credits and scroll through
Facebook, looking for
\href{https://www.facebookcorewwwi.onion/watch/?v=274567883836011}{something
to laugh at}. Mr. Santillan has watched ``Pirates of the Caribbean'' so
many times that he has memorized it --- a point of exasperation, not
pride.

He still has chocolates that he bought in Brazil for his wife and their
young son, but they have passed their expiration date. His son, who was
a week old when Mr. Santillan left the Philippines, is now walking and
talking.

Mr. Santillan said he had to resist thinking about his family while
working.

``Missing someone is not allowed,'' he said. ``For you to focus on work,
you can't think about them. Your body is heavier when you miss
someone.''

Advertisement

\protect\hyperlink{after-bottom}{Continue reading the main story}

\hypertarget{site-index}{%
\subsection{Site Index}\label{site-index}}

\hypertarget{site-information-navigation}{%
\subsection{Site Information
Navigation}\label{site-information-navigation}}

\begin{itemize}
\tightlist
\item
  \href{https://help.nytimes3xbfgragh.onion/hc/en-us/articles/115014792127-Copyright-notice}{©~2020~The
  New York Times Company}
\end{itemize}

\begin{itemize}
\tightlist
\item
  \href{https://www.nytco.com/}{NYTCo}
\item
  \href{https://help.nytimes3xbfgragh.onion/hc/en-us/articles/115015385887-Contact-Us}{Contact
  Us}
\item
  \href{https://www.nytco.com/careers/}{Work with us}
\item
  \href{https://nytmediakit.com/}{Advertise}
\item
  \href{http://www.tbrandstudio.com/}{T Brand Studio}
\item
  \href{https://www.nytimes3xbfgragh.onion/privacy/cookie-policy\#how-do-i-manage-trackers}{Your
  Ad Choices}
\item
  \href{https://www.nytimes3xbfgragh.onion/privacy}{Privacy}
\item
  \href{https://help.nytimes3xbfgragh.onion/hc/en-us/articles/115014893428-Terms-of-service}{Terms
  of Service}
\item
  \href{https://help.nytimes3xbfgragh.onion/hc/en-us/articles/115014893968-Terms-of-sale}{Terms
  of Sale}
\item
  \href{https://spiderbites.nytimes3xbfgragh.onion}{Site Map}
\item
  \href{https://help.nytimes3xbfgragh.onion/hc/en-us}{Help}
\item
  \href{https://www.nytimes3xbfgragh.onion/subscription?campaignId=37WXW}{Subscriptions}
\end{itemize}
