Sections

SEARCH

\protect\hyperlink{site-content}{Skip to
content}\protect\hyperlink{site-index}{Skip to site index}

\href{https://www.nytimes3xbfgragh.onion/section/nyregion}{New York}

\href{https://myaccount.nytimes3xbfgragh.onion/auth/login?response_type=cookie\&client_id=vi}{}

\href{https://www.nytimes3xbfgragh.onion/section/todayspaper}{Today's
Paper}

\href{/section/nyregion}{New York}\textbar{}N.Y.C. to Allow Indoor
Dining, in Milestone on Recovery From Pandemic

\url{https://nyti.ms/33amWmY}

\begin{itemize}
\item
\item
\item
\item
\item
\item
\end{itemize}

\hypertarget{the-coronavirus-outbreak}{%
\subsubsection{\texorpdfstring{\href{https://www.nytimes3xbfgragh.onion/news-event/coronavirus?name=styln-coronavirus-national\&region=TOP_BANNER\&block=storyline_menu_recirc\&action=click\&pgtype=Article\&impression_id=faa237d0-f52d-11ea-8b73-3bde8420cfaf\&variant=undefined}{The
Coronavirus
Outbreak}}{The Coronavirus Outbreak}}\label{the-coronavirus-outbreak}}

\begin{itemize}
\tightlist
\item
  live\href{https://www.nytimes3xbfgragh.onion/2020/09/12/world/covid-19-coronavirus.html?name=styln-coronavirus-national\&region=TOP_BANNER\&block=storyline_menu_recirc\&action=click\&pgtype=Article\&impression_id=faa237d1-f52d-11ea-8b73-3bde8420cfaf\&variant=undefined}{Latest
  Updates}
\item
  \href{https://www.nytimes3xbfgragh.onion/interactive/2020/us/coronavirus-us-cases.html?name=styln-coronavirus-national\&region=TOP_BANNER\&block=storyline_menu_recirc\&action=click\&pgtype=Article\&impression_id=faa237d2-f52d-11ea-8b73-3bde8420cfaf\&variant=undefined}{Maps
  and Cases}
\item
  \href{https://www.nytimes3xbfgragh.onion/interactive/2020/science/coronavirus-vaccine-tracker.html?name=styln-coronavirus-national\&region=TOP_BANNER\&block=storyline_menu_recirc\&action=click\&pgtype=Article\&impression_id=faa237d3-f52d-11ea-8b73-3bde8420cfaf\&variant=undefined}{Vaccine
  Tracker}
\item
  \href{https://www.nytimes3xbfgragh.onion/2020/09/10/us/politics/fda-coronavirus-vaccine.html?name=styln-coronavirus-national\&region=TOP_BANNER\&block=storyline_menu_recirc\&action=click\&pgtype=Article\&impression_id=faa25ee0-f52d-11ea-8b73-3bde8420cfaf\&variant=undefined}{F.D.A.
  Regulators' Self-Defense}
\item
  \href{https://www.nytimes3xbfgragh.onion/2020/09/09/upshot/coronavirus-surprise-test-fees.html?name=styln-coronavirus-national\&region=TOP_BANNER\&block=storyline_menu_recirc\&action=click\&pgtype=Article\&impression_id=faa25ee1-f52d-11ea-8b73-3bde8420cfaf\&variant=undefined}{Surprise
  Test Fees}
\end{itemize}

Advertisement

\protect\hyperlink{after-top}{Continue reading the main story}

Supported by

\protect\hyperlink{after-sponsor}{Continue reading the main story}

\hypertarget{nyc-to-allow-indoor-dining-in-milestone-on-recovery-from-pandemic}{%
\section{N.Y.C. to Allow Indoor Dining, in Milestone on Recovery From
Pandemic}\label{nyc-to-allow-indoor-dining-in-milestone-on-recovery-from-pandemic}}

Gov. Andrew M. Cuomo announced that restaurants can open for indoor
service at 25 percent capacity, starting on Sept. 30.

\includegraphics{https://static01.graylady3jvrrxbe.onion/images/2020/09/09/nyregion/NYVIRUS-DINING2/merlin_176693760_3315f45c-be9d-41af-bd61-b45b482c4fea-articleLarge.jpg?quality=75\&auto=webp\&disable=upscale}

\href{https://www.nytimes3xbfgragh.onion/by/jesse-mckinley}{\includegraphics{https://static01.graylady3jvrrxbe.onion/images/2018/02/20/multimedia/author-jesse-mckinley/author-jesse-mckinley-thumbLarge.jpg}}\href{https://www.nytimes3xbfgragh.onion/by/sharon-otterman}{\includegraphics{https://static01.graylady3jvrrxbe.onion/images/2018/06/14/multimedia/author-sharon-otterman/author-sharon-otterman-thumbLarge.png}}\href{https://www.nytimes3xbfgragh.onion/by/joseph-goldstein}{\includegraphics{https://static01.graylady3jvrrxbe.onion/images/2018/07/16/multimedia/author-joseph-goldstein/author-joseph-goldstein-thumbLarge.png}}

By \href{https://www.nytimes3xbfgragh.onion/by/jesse-mckinley}{Jesse
McKinley},
\href{https://www.nytimes3xbfgragh.onion/by/sharon-otterman}{Sharon
Otterman} and
\href{https://www.nytimes3xbfgragh.onion/by/joseph-goldstein}{Joseph
Goldstein}

\begin{itemize}
\item
  Published Sept. 9, 2020Updated Sept. 11, 2020
\item
  \begin{itemize}
  \item
  \item
  \item
  \item
  \item
  \item
  \end{itemize}
\end{itemize}

Gov. Andrew M. Cuomo announced on Wednesday that the prohibition on
\href{https://www.nytimes3xbfgragh.onion/2020/09/10/nyregion/nyc-outdoor-dining-homeless.html}{indoor
dining} in New York City would be lifted on Sept. 30, giving a boost to
the city's recovery from the pandemic and ending its status as one of
the few places in the nation with a ban.

The governor's decision to allow restaurants to have indoor dining at 25
percent capacity will be a major milestone in the coronavirus crisis in
New York City, signaling to tourists and residents that the city is
slowly returning to normal.

Still, the reopening will likely not be enough to save some restaurants,
battered by the combination of the coronavirus, the economic crisis and
the reluctance of many Americans to socialize in proximity to others,
particularly as colder weather limits
\href{https://www.nytimes3xbfgragh.onion/2020/09/10/nyregion/nyc-outdoor-dining-homeless.html}{outdoor
dining}.

The move came more than two months after the governor and Mayor Bill de
Blasio
\href{https://www.nytimes3xbfgragh.onion/2020/07/01/nyregion/indoor-dining-coronavirus-nyc.html}{halted
a plan to permit indoor dining} at restaurants, citing worries about a
resurgence of the coronavirus, which has killed more than 30,000 people
in New York State. Even so, the reopening announcement left some
concerns still percolating at City Hall, as Mr. de Blasio had favored a
longer lag between a planned reopening of schools and the city's indoor
dining.

But with the infection rate in the state stabilized at under 1 percent
for more than a month, the governor said he would ease some restrictions
at a time of desperation and frustration for the city's restaurant
industry, which has watched some venerable and venerated eateries fail
and losses mount for others even as neighboring states and regions
reopened their dining rooms.

``Because compliance has gotten so much better, we can now take the next
step,'' the governor said.

Under the governor's plan, restaurants would be permitted to use a
quarter of their indoor tables just as the fall weather is likely to put
a chill on outdoor service, which began in June and has allowed many
restaurants to make game attempts at staying afloat.

On Wednesday, many of those manning kitchens --- from greasy spoons to
fine dining --- voiced measured relief that they would soon be able to
welcome patrons inside, as well as rehire some staff, as the city
grapples with
\href{https://labor.ny.gov/stats/pressreleases/pruistat.shtm}{an
unemployment rate of nearly 20 percent}.

``With 25 percent plus outdoor, I can hire back 50 percent,'' said
Jean-Georges Vongerichten, the celebrity chef whose flagship restaurant,
Jean-Georges, is in Columbus Circle.

The move carries substantial financial and health risks for New York,
the one-time epicenter of the pandemic, and still home to some of the
most stringent regulations regarding the disease, including a 14-day
quarantine for visitors from dozens of other states.

New York City, a world capital of dining and entertainment, remains in a
defensive crouch, with Broadway shuttered, movie theaters and clubs
closed and many cultural institutions operating under strict capacity
restrictions.

\hypertarget{latest-updates-the-coronavirus-outbreak}{%
\section{\texorpdfstring{\href{https://www.nytimes3xbfgragh.onion/2020/09/11/world/covid-19-coronavirus.html?action=click\&pgtype=Article\&state=default\&region=MAIN_CONTENT_1\&context=storylines_live_updates}{Latest
Updates: The Coronavirus
Outbreak}}{Latest Updates: The Coronavirus Outbreak}}\label{latest-updates-the-coronavirus-outbreak}}

Updated 2020-09-12T12:04:20.515Z

\begin{itemize}
\tightlist
\item
  \href{https://www.nytimes3xbfgragh.onion/2020/09/11/world/covid-19-coronavirus.html?action=click\&pgtype=Article\&state=default\&region=MAIN_CONTENT_1\&context=storylines_live_updates\#link-dfb8a16}{Fauci
  cautions the virus could disrupt life in the U.S. until `maybe even
  towards the end of 2021.'}
\item
  \href{https://www.nytimes3xbfgragh.onion/2020/09/11/world/covid-19-coronavirus.html?action=click\&pgtype=Article\&state=default\&region=MAIN_CONTENT_1\&context=storylines_live_updates\#link-7104d154}{From
  Asia to Africa, China promotes its vaccine candidates to win friends.}
\item
  \href{https://www.nytimes3xbfgragh.onion/2020/09/11/world/covid-19-coronavirus.html?action=click\&pgtype=Article\&state=default\&region=MAIN_CONTENT_1\&context=storylines_live_updates\#link-393ad215}{The
  other way the virus will kill: hunger.}
\end{itemize}

\href{https://www.nytimes3xbfgragh.onion/2020/09/11/world/covid-19-coronavirus.html?action=click\&pgtype=Article\&state=default\&region=MAIN_CONTENT_1\&context=storylines_live_updates}{See
more updates}

More live coverage:
\href{https://www.nytimes3xbfgragh.onion/live/2020/09/11/business/stock-market-today-coronavirus?action=click\&pgtype=Article\&state=default\&region=MAIN_CONTENT_1\&context=storylines_live_updates}{Markets}

The caution seems warranted: State officials said about 10 percent of
coronavirus clusters outside of New York City have been tied to bars and
restaurants, a source of infection second only to large gatherings. Bars
and restaurants have also fueled outbreaks nationwide and across the
world.

The reopening will also come shortly after New York City's public
schools --- the nation's largest school system --- will welcome students
back inside on Sept. 21, another complex and fraught experiment that has
already been delayed once by Mr. de Blasio.

Local officials, wary of returning to the nightmare of March and April,
when the disease was killing hundreds of people a day, were reluctant to
do anything that would put the city in peril, and were prioritizing
reopening schools rather than indoor dining.

The timing of Mr. Cuomo's announcement, which came after months of
debate at a state and city level, caught some by surprise, coming just
hours after Mr. de Blasio suggested a decision on dining was still being
fine-tuned.

One plan under discussion at City Hall called for putting indoor dining
off until schools had been reopened for at least a month and there had
been no substantial uptick in the positivity rate, according to one city
official familiar with the plan. But the announcement by the governor,
who has authority over such decisions because of expanded powers granted
to him during the crisis, rendered that approach moot.

Under a compromise, both sides would reassess indoor dining should the
infection rate in the city go past 2 percent, according to one City Hall
official.

Mr. Cuomo, a third-term Democrat who has frequently sparred with Mr. de
Blasio, had acknowledged that he was under intense pressure from
restaurateurs to reopen indoor dining spaces, noting on Wednesday that
``a restaurant is not just the restaurant owner, a restaurant is the
kitchen staff, the wait staff.''

``Restaurants also pose a possible risk," he said, adding, ``But there
is also a great economic loss when they don't operate.''

The combination of students in schools and patrons in dining rooms will
likely heighten anxiety about the possibility of a second wave of
coronavirus infections for public health officials and experts, who said
that reopening both in such quick succession comes with a substantial
risk.

``I would like to understand the extent to which school reopening is
contributing to a bump, or a spike, or lack thereof before moving on to
another significant contributor to new cases,'' said Denis Nash, an
epidemiology professor at the CUNY School of Public Health, who has in
the past worked for the New York City Department of Health and Mental
Hygiene.

He added, ``If it's all going on at the same time, it makes it difficult
to tease apart and know what's driving any bumps we may see,'' he said,
referring to possible upticks in cases.

Mr. Cuomo said that state would evaluate infection rates and other data
after the Sept. 30 reopening, with an eye toward increasing capacity to
50 percent by Nov. 1, perhaps sooner. At the same time, he also warned
that any spike in infections could lead to sudden closures via an
``emergency pause button.''

Even as other parts of the state reopened their dining rooms in late
spring, the city's indoor tables remained closed. For weeks, many
restaurant owners had complained about a split reality at the city
border --- on one side indoor dining, on the other, a ban. Some sued,
questioning why there was a continuing bar on indoor dining in the city
even as virus levels in both the suburbs and the city were virtually the
same.

Nor will the allowance be a panacea. Even with indoor dining permitted,
not all restaurants may do it, since many are concerned about safety,
and some have spaces so small that capacity restrictions may not make it
worthwhile.

``I know I can make it work at 50 percent, but the expenses of getting
it up and running, versus the revenue, my gut tells me that it will not
work at 25 percent,'' said Eric Ripert, the chef and an owner of Le
Bernardin, on West 51st Street.

There is also the question of whether diners will feel comfortable
returning to indoor spaces: A
\href{https://scri.siena.edu/2020/09/02/62-say-completely-opening-schools-runs-too-great-a-risk-32-disagree/}{recent
poll} by the Siena College Research Institute found that 58 percent of
New Yorkers, and 65 percent of city residents, said they were still not
comfortable with dining indoors in a restaurant. More than 70 percent of
both city and state residents also said they were not yet comfortable
having a drink at a bar.

Without a government relief package specifically for restaurants, about
64 percent of New York restaurants said they are likely to close by the
end of the year, according to a recent
\href{https://www.nysra.org/uploads/1/2/1/3/121352550/state_restaurant_association_survey_results_090320.pdf}{survey}
from the New York State Restaurant Association.

``I think at the end of September, you are going to see a lot of
for-rent signs on New York City restaurants,'' said Cindy Smith, an
owner of the Mermaid Inn, which has already permanently closed its
original East Village location, but is hanging on for now at its three
other Manhattan locations. Only one is open for outdoor dining.

Under the governor's
\href{https://forward.ny.gov/nyc-indoor-dining}{plan}, restaurants will
be required to check customers' temperatures and collect contact
information for one person in each party. Diners will be required to
wear face coverings when not seated, and bar service will not be
allowed. Closing time will be midnight.

Russell Jackson, the chef and owner of Reverence, a tasting menu
restaurant on Frederick Douglass Boulevard in Harlem, said he has been
staying alive with takeout and delivery, which were still allowed after
Mr. Cuomo ordered bars and restaurants to close in March.

``As this is going to settle in that we're all going back to work, I
know all of us are going to start freaking out,'' Mr. Jackson said. ``We
shifted this restaurant to be a completely different animal over the
past six months. Now we have to change all the protocols. It's going to
be an enormous adjustment.''

Mr. Cuomo said that the city will provide 400 code enforcers, which
could include New York City police officers, in addition to oversight
from the New York State Liquor Authority, New York State Police and
other agencies. But he also invited diners to anonymously report
restaurants where they believe the 25 percent capacity is being
violated.

On Wednesday, Mr. Cuomo acknowledged that the infection rate might
increase with ``more people at work, more people at schools, more
restaurants, more people on the street,'' though he said there was no
``hard number'' that would force the state to retreat.

``We'll just watch it,'' he said, ``and see what we hear.''

Reporting contributed by Julia Moskin, Pete Wells, Dana Rubinstein, and
Luis Ferré-Sadurní.

Advertisement

\protect\hyperlink{after-bottom}{Continue reading the main story}

\hypertarget{site-index}{%
\subsection{Site Index}\label{site-index}}

\hypertarget{site-information-navigation}{%
\subsection{Site Information
Navigation}\label{site-information-navigation}}

\begin{itemize}
\tightlist
\item
  \href{https://help.nytimes3xbfgragh.onion/hc/en-us/articles/115014792127-Copyright-notice}{©~2020~The
  New York Times Company}
\end{itemize}

\begin{itemize}
\tightlist
\item
  \href{https://www.nytco.com/}{NYTCo}
\item
  \href{https://help.nytimes3xbfgragh.onion/hc/en-us/articles/115015385887-Contact-Us}{Contact
  Us}
\item
  \href{https://www.nytco.com/careers/}{Work with us}
\item
  \href{https://nytmediakit.com/}{Advertise}
\item
  \href{http://www.tbrandstudio.com/}{T Brand Studio}
\item
  \href{https://www.nytimes3xbfgragh.onion/privacy/cookie-policy\#how-do-i-manage-trackers}{Your
  Ad Choices}
\item
  \href{https://www.nytimes3xbfgragh.onion/privacy}{Privacy}
\item
  \href{https://help.nytimes3xbfgragh.onion/hc/en-us/articles/115014893428-Terms-of-service}{Terms
  of Service}
\item
  \href{https://help.nytimes3xbfgragh.onion/hc/en-us/articles/115014893968-Terms-of-sale}{Terms
  of Sale}
\item
  \href{https://spiderbites.nytimes3xbfgragh.onion}{Site Map}
\item
  \href{https://help.nytimes3xbfgragh.onion/hc/en-us}{Help}
\item
  \href{https://www.nytimes3xbfgragh.onion/subscription?campaignId=37WXW}{Subscriptions}
\end{itemize}
