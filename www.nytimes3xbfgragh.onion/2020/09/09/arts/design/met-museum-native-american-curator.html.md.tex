Sections

SEARCH

\protect\hyperlink{site-content}{Skip to
content}\protect\hyperlink{site-index}{Skip to site index}

\href{https://www.nytimes3xbfgragh.onion/section/arts/design}{Art \&
Design}

\href{https://myaccount.nytimes3xbfgragh.onion/auth/login?response_type=cookie\&client_id=vi}{}

\href{https://www.nytimes3xbfgragh.onion/section/todayspaper}{Today's
Paper}

\href{/section/arts/design}{Art \& Design}\textbar{}The Met Hires Its
First Full-Time Native American Curator

\url{https://nyti.ms/35jU5zd}

\begin{itemize}
\item
\item
\item
\item
\item
\end{itemize}

Advertisement

\protect\hyperlink{after-top}{Continue reading the main story}

Supported by

\protect\hyperlink{after-sponsor}{Continue reading the main story}

\hypertarget{the-met-hires-its-first-full-time-native-american-curator}{%
\section{The Met Hires Its First Full-Time Native American
Curator}\label{the-met-hires-its-first-full-time-native-american-curator}}

Patricia Marroquin Norby, most recently of the National Museum of the
American Indian in New York, will soon join the Metropolitan Museum of
Art.

\includegraphics{https://static01.graylady3jvrrxbe.onion/images/2020/09/10/arts/09met-item/09met-item-articleLarge.jpg?quality=75\&auto=webp\&disable=upscale}

\href{https://www.nytimes3xbfgragh.onion/by/sarah-bahr}{\includegraphics{https://static01.graylady3jvrrxbe.onion/images/2020/08/14/reader-center/author-sarah-bahr/author-sarah-bahr-thumbLarge.png}}

By \href{https://www.nytimes3xbfgragh.onion/by/sarah-bahr}{Sarah Bahr}

\begin{itemize}
\item
  Sept. 9, 2020
\item
  \begin{itemize}
  \item
  \item
  \item
  \item
  \item
  \end{itemize}
\end{itemize}

For the first time in its
\href{https://www.nytimes3xbfgragh.onion/2020/08/27/arts/design/met-museum-reopens-anniversary.html}{150-year
history}, the Metropolitan Museum of Art has hired a full-time Native
American curator: Patricia Marroquin Norby.

Dr. Norby --- who is of Purépecha heritage, an Indigenous population
that primarily lives in Michoacán, Mexico ---
\href{https://slack-redir.net/link?url=https\%3A\%2F\%2Fwww.metmuseum.org\%2Fpress\%2Fnews\%2F2020\%2Fpatricia-marroquin-norby}{will
assume the role} of associate curator of Native American art on Monday.
She most recently served as senior executive and assistant director of
the National Museum of the American Indian in New York.

In a statement, Max Hollein, the Met's director, said of Dr. Norby: ``We
look forward to supporting her scholarship and programmatic
collaborations with colleagues across the Met as well as with Indigenous
communities throughout the region and continent.''

Before coming to the National Museum of the American Indian, which is
part of the Smithsonian Institution, Dr. Norby was the director of the
D'Arcy McNickle Center for American Indian and Indigenous Studies at the
Newberry, a research library in Chicago. She also worked as an assistant
professor of American Indian studies at the University of Wisconsin-Eau
Claire. She earned a Ph.D. from the University of Minnesota, Twin
Cities, in American studies, with a specialization in Native American
art history and visual culture.

``This is a time of significant evolution for the museum,'' Dr. Norby
said in a statement. ``I look forward to being part of this critical
shift in the presentation of Native American art.''

The Met had been seeking to fill the position since last September.

For most of the museum's history, work by Native American artists has
been displayed in the galleries of Africa, Oceania and the Americas. The
museum staged a Native American art exhibition in its American Wing in
2018, but the show faced pushback from the Association on American
Indian Affairs, an advocacy group that claimed the museum had not
satisfactorily consulted with tribal representatives before the show
opened.

\href{https://www.indian-affairs.org/uploads/8/7/3/8/87380358/2018-10-29_met_pr.pdf}{In
a statement}, the group said that a majority of the items in the show
were not art, but ``sacred ceremonial objects, cultural patrimony and
burial objects,'' though it did not point to any specific item. A
spokeswoman for the Met
\href{https://www.theartnewspaper.com/news/native-american-group-denounces-met-s-exhibition-of-indigenous-objects}{told
The Art Newspaper} at the time that the allegations were without merit
and that it had ``engaged regularly and repeatedly with tribal leaders
in many Native communities throughout the country.''

Advertisement

\protect\hyperlink{after-bottom}{Continue reading the main story}

\hypertarget{site-index}{%
\subsection{Site Index}\label{site-index}}

\hypertarget{site-information-navigation}{%
\subsection{Site Information
Navigation}\label{site-information-navigation}}

\begin{itemize}
\tightlist
\item
  \href{https://help.nytimes3xbfgragh.onion/hc/en-us/articles/115014792127-Copyright-notice}{©~2020~The
  New York Times Company}
\end{itemize}

\begin{itemize}
\tightlist
\item
  \href{https://www.nytco.com/}{NYTCo}
\item
  \href{https://help.nytimes3xbfgragh.onion/hc/en-us/articles/115015385887-Contact-Us}{Contact
  Us}
\item
  \href{https://www.nytco.com/careers/}{Work with us}
\item
  \href{https://nytmediakit.com/}{Advertise}
\item
  \href{http://www.tbrandstudio.com/}{T Brand Studio}
\item
  \href{https://www.nytimes3xbfgragh.onion/privacy/cookie-policy\#how-do-i-manage-trackers}{Your
  Ad Choices}
\item
  \href{https://www.nytimes3xbfgragh.onion/privacy}{Privacy}
\item
  \href{https://help.nytimes3xbfgragh.onion/hc/en-us/articles/115014893428-Terms-of-service}{Terms
  of Service}
\item
  \href{https://help.nytimes3xbfgragh.onion/hc/en-us/articles/115014893968-Terms-of-sale}{Terms
  of Sale}
\item
  \href{https://spiderbites.nytimes3xbfgragh.onion}{Site Map}
\item
  \href{https://help.nytimes3xbfgragh.onion/hc/en-us}{Help}
\item
  \href{https://www.nytimes3xbfgragh.onion/subscription?campaignId=37WXW}{Subscriptions}
\end{itemize}
