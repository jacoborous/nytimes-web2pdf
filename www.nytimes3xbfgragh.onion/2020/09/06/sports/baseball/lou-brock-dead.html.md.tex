Sections

SEARCH

\protect\hyperlink{site-content}{Skip to
content}\protect\hyperlink{site-index}{Skip to site index}

\href{https://www.nytimes3xbfgragh.onion/section/sports/baseball}{Baseball}

\href{https://myaccount.nytimes3xbfgragh.onion/auth/login?response_type=cookie\&client_id=vi}{}

\href{https://www.nytimes3xbfgragh.onion/section/todayspaper}{Today's
Paper}

\href{/section/sports/baseball}{Baseball}\textbar{}Lou Brock, Baseball
Hall of Famer Known for Stealing Bases, Dies at 81

\url{https://nyti.ms/3h3hKWU}

\begin{itemize}
\item
\item
\item
\item
\item
\end{itemize}

Advertisement

\protect\hyperlink{after-top}{Continue reading the main story}

Supported by

\protect\hyperlink{after-sponsor}{Continue reading the main story}

\hypertarget{lou-brock-baseball-hall-of-famer-known-for-stealing-bases-dies-at-81}{%
\section{Lou Brock, Baseball Hall of Famer Known for Stealing Bases,
Dies at
81}\label{lou-brock-baseball-hall-of-famer-known-for-stealing-bases-dies-at-81}}

The son of sharecroppers, Brock attended a one-room schoolhouse, but was
inspired by possibilities beyond the poverty and segregation of the
rural South.

\includegraphics{https://static01.graylady3jvrrxbe.onion/images/2020/09/08/obituaries/06brock3-obit-print/00brock-lou-toppix-articleLarge.jpg?quality=75\&auto=webp\&disable=upscale}

By
\href{https://www.nytimes3xbfgragh.onion/by/richard-goldstein}{Richard
Goldstein}

\begin{itemize}
\item
  Sept. 6, 2020
\item
  \begin{itemize}
  \item
  \item
  \item
  \item
  \item
  \end{itemize}
\end{itemize}

Lou Brock, the St. Louis Cardinals' Hall of Fame outfielder who in a
career spanning two decades became the greatest base-stealer the major
leagues had ever known when he eclipsed the single-season and career
records for steals, died on Sunday. He was 81.

Dick Zitzmann, Brock's agent, confirmed the death to The Associated
Press, but did not provide any details. Brock began receiving treatment
for multiple myeloma, a type of blood cancer, in 2017. His left leg was
amputated in 2015 as a result of a diabetes-related infection.

On June 15, 1964, a floundering Cardinals team traded one of the
National League's leading pitchers for an outfielder who had failed to
live up to his promise. That deal, sending the right-hander Ernie
Broglio to the Chicago Cubs for Brock as the centerpiece of a six-player
swap, became one of the most one-sided trades in baseball history, but
hardly in the way that many envisioned.

Broglio won only seven games for the Cubs over the next two and a half
seasons, then retired. Brock, sought by Cardinals Manager Johnny Keane
for his largely untapped speed, helped take St. Louis to the 1964 World
Series championship and went on to turn around games year after year
with his feet and his bat.

\includegraphics{https://static01.graylady3jvrrxbe.onion/images/2020/09/07/obituaries/06brock2-obit/merlin_176675379_e8e5ef2a-fd84-4784-ba04-0584935fb1b9-articleLarge.jpg?quality=75\&auto=webp\&disable=upscale}

Brock's 118 stolen bases in 1974 eclipsed Maury Wills's single-season
record of 104, set in 1962, and his 938 career steals broke Ty Cobb's
mark of 892.

He led the National League in steals eight times. Although Rickey
Henderson would break Brock's stolen-base records, Brock's luster
remained undimmed. A left-handed batter, he had 3,023 hits and hit .300
eight times. He helped propel the Cardinals to three pennants and two
World Series championships. He was elected to the Baseball Hall of Fame
in 1985.

Louis Clark Brock was born on June 18, 1939, in El Dorado, Ark., and
grew up in Collinston, La., in a family of sharecroppers who picked
cotton. He attended a one-room schoolhouse, but at the age of 9 he was
inspired by possibilities beyond the poverty and segregation of the
rural South.

He was listening one night to a feed from the St. Louis radio station
KMOX. Harry Caray was broadcasting a game between the Cardinals and
Jackie Robinson's Brooklyn Dodgers, the summer after Robinson broke the
major leagues' color barrier, a time when, as Brock put it, ``Jim Crow
was king.''

``I was searching the dial of an old Philco radio,'' Brock recalled. and
when he heard about Robinson, ``I felt pride in being alive. The
baseball field was my fantasy of what life offered.''

As a boy, Brock never played organized baseball. Instead of a ball and
bat, he swatted rocks with tree branches. But he received an academic
scholarship to Southern University in Baton Rouge, La., and played
baseball there, catching the attention of Buck O'Neil, the longtime
Negro leagues player and manager, who was scouting for the Cubs.

Image

Brock slid into second base as he set the record for stolen bases at 893
in 1977. ``You know before you steal a base that you've got nine guys
out there in different uniforms,'' he once said. ``You're alone in a sea
of enemies.''Credit...Lennox McLendon/Associated Press

The Cubs' organization signed Brock in August 1960, and he made his
major league debut late in the '61 season. But two summers later, he was
batting only .251 and struggling with the Wrigley Field sun as the Cubs'
right fielder. He was considered perhaps the fastest man in the league,
but the Cubs were reluctant to turn him loose on the basepaths.

At the 1964 trade deadline, the Cardinals gambled by trading for Brock,
hoping that his speed would provide the missing element in an impressive
lineup featuring Ken Boyer, Bill White, Curt Flood, Dick Groat and Tim
McCarver.

``I thought it was a dumb trade,'' the Cardinals' future Hall of Fame
pitcher Bob Gibson was quoted by The St. Louis Post-Dispatch as saying.
``I didn't know how good Lou would be. No one knew. I didn't even
remember facing him. I heard it and thought, `For who? How could you
trade Broglio for that?'''

Keane told Brock he wanted him to steal bases, but Brock regarded
himself as primarily a power hitter and had his doubts. Keane's
confidence in him nonetheless inspired Brock, who was put in left field,
replacing the retired Stan Musial, one of baseball's greatest hitters.

Playing in 103 games for the 1964 Cardinals, Brock hit .348, stole 33
bases and scored 81 runs. The Cardinals overtook the Philadelphia
Phillies in the season's final week to win the pennant, then defeated
the Yankees in a seven-game World Series.

Image

Brock at bat in a game against the Mets at Shea Stadium in 1979. He had
3,023 hits in his career and hit .300 eight times.Credit...Richard
Drew/Associated Press

Brock's Cardinals defeated the Boston Red Sox in the 1967 World Series
and won another pennant the next year, but lost to the Detroit Tigers in
the Series.

For Brock, base stealing required a certain bravado.

``You know before you steal a base that you've got nine guys out there
in different uniforms,'' he once said. ``You're alone in a sea of
enemies. The only way you can hold your own is by arrogance, the ability
to stand before the crowd. Every time you get thrown out, you've got to
believe that somebody owes you four or five steals.''

Brock retired after the 1979 season with a career batting average of
.293 to complement his base-stealing superlatives. He hit 149 home runs
and scored 1,610 runs. He later pursued business ventures in St. Louis
and worked as an instructor in the Cardinals' organization. The team
retired his No. 20, and a statue honoring him stands outside Busch
Stadium.

Brock's survivors include his third wife, Jacqueline, a
special-education teacher whom he married in 1996; his son, Lou Jr., and
his daughter, Wanda, from his first marriage, to Katie Hay; three
stepchildren; and two granddaughters, according to St. Louis Public
Radio. His first two marriages ended in divorce.

For all his natural speed, Brock was also a student of baseball and an
innovator in pursuing the art of stealing bases, using technology to
``synchronize your movement with the pitcher's movement.'' Late in the
'64 season, he obtained a movie camera and began filming pitchers as
they took their set position, threw to first base and threw to the
plate, hoping to discover tendencies that might give him an edge.

Image

In 2017, Brock attended the 50th-anniversary celebration of the
Cardinals' 1967 World Series victory.Credit...Jeff Roberson/Associated
Press

Brock's ingenuity wasn't appreciated by at least one pitcher, as David
Halberstam related in his book ``October 1964'':

``One day he was filming Don Drysdale, as tough a pitcher as existed in
the league.

```What the hell are you doing with that camera, Brock?'

```Just taking home movies,' said Brock.

```I don't want to be in your goddamn movies, Brock,' Drysdale said, and
threw at him the next time he was up.''

Advertisement

\protect\hyperlink{after-bottom}{Continue reading the main story}

\hypertarget{site-index}{%
\subsection{Site Index}\label{site-index}}

\hypertarget{site-information-navigation}{%
\subsection{Site Information
Navigation}\label{site-information-navigation}}

\begin{itemize}
\tightlist
\item
  \href{https://help.nytimes3xbfgragh.onion/hc/en-us/articles/115014792127-Copyright-notice}{©~2020~The
  New York Times Company}
\end{itemize}

\begin{itemize}
\tightlist
\item
  \href{https://www.nytco.com/}{NYTCo}
\item
  \href{https://help.nytimes3xbfgragh.onion/hc/en-us/articles/115015385887-Contact-Us}{Contact
  Us}
\item
  \href{https://www.nytco.com/careers/}{Work with us}
\item
  \href{https://nytmediakit.com/}{Advertise}
\item
  \href{http://www.tbrandstudio.com/}{T Brand Studio}
\item
  \href{https://www.nytimes3xbfgragh.onion/privacy/cookie-policy\#how-do-i-manage-trackers}{Your
  Ad Choices}
\item
  \href{https://www.nytimes3xbfgragh.onion/privacy}{Privacy}
\item
  \href{https://help.nytimes3xbfgragh.onion/hc/en-us/articles/115014893428-Terms-of-service}{Terms
  of Service}
\item
  \href{https://help.nytimes3xbfgragh.onion/hc/en-us/articles/115014893968-Terms-of-sale}{Terms
  of Sale}
\item
  \href{https://spiderbites.nytimes3xbfgragh.onion}{Site Map}
\item
  \href{https://help.nytimes3xbfgragh.onion/hc/en-us}{Help}
\item
  \href{https://www.nytimes3xbfgragh.onion/subscription?campaignId=37WXW}{Subscriptions}
\end{itemize}
