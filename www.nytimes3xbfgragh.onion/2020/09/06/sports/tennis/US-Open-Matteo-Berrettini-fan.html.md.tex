Sections

SEARCH

\protect\hyperlink{site-content}{Skip to
content}\protect\hyperlink{site-index}{Skip to site index}

\href{https://www.nytimes3xbfgragh.onion/section/sports/tennis}{Tennis}

\href{https://myaccount.nytimes3xbfgragh.onion/auth/login?response_type=cookie\&client_id=vi}{}

\href{https://www.nytimes3xbfgragh.onion/section/todayspaper}{Today's
Paper}

\href{/section/sports/tennis}{Tennis}\textbar{}With Fans Barred From the
U.S. Open, One Gets as Close as He Can

\url{https://nyti.ms/3i8Oj6U}

\begin{itemize}
\item
\item
\item
\item
\item
\end{itemize}

Advertisement

\protect\hyperlink{after-top}{Continue reading the main story}

Supported by

\protect\hyperlink{after-sponsor}{Continue reading the main story}

\hypertarget{with-fans-barred-from-the-us-open-one-gets-as-close-as-he-can}{%
\section{With Fans Barred From the U.S. Open, One Gets as Close as He
Can}\label{with-fans-barred-from-the-us-open-one-gets-as-close-as-he-can}}

Giovanni Bartocci has had a 2020-type year, including enduring a fire
that closed down his East Village restaurant, but he is still there to
cheer the tennis player Matteo Berrettini.

\includegraphics{https://static01.graylady3jvrrxbe.onion/images/2020/09/06/sports/06usopen-buddy/merlin_176629617_451358f5-310b-4b00-8c14-5ccee954a956-articleLarge.jpg?quality=75\&auto=webp\&disable=upscale}

\href{https://www.nytimes3xbfgragh.onion/by/david-waldstein}{\includegraphics{https://static01.graylady3jvrrxbe.onion/images/2018/02/20/multimedia/author-david-waldstein/author-david-waldstein-thumbLarge.jpg}}

By \href{https://www.nytimes3xbfgragh.onion/by/david-waldstein}{David
Waldstein}

\begin{itemize}
\item
  Sept. 6, 2020
\item
  \begin{itemize}
  \item
  \item
  \item
  \item
  \item
  \end{itemize}
\end{itemize}

If only the founders of modern tennis could have witnessed the scene
outside the gates of the United States Open on Saturday. They might not
have recognized the trappings of their once-genteel sport.

For there was Giovanni Bartocci, 41, a self-described ``crazy Italian
guy,'' decorated in tattoos and piercings, his long hair perched in a
bun above his shaggy beard and sunglasses, screaming, ``Daje, Matte,''
or ``Go, Matte,'' into a bullhorn.

Just over the fence, Matteo Berrettini,
\href{https://www.atptour.com/en/rankings/singles}{the eighth-ranked
men's tennis player} in the world, could hear his friend shouting
encouragement in their shared Romanesco dialect as Berrettini completed
a straight-sets victory over Casper Ruud.

``Nobody didn't hear him,'' Berrettini said. ``It was pretty loud
today.''

As the United States Open enters its second week without fans in
attendance because of the coronavirus pandemic, Bartocci, a restaurateur
who met his own 2020-style misfortune even before the pandemic was
declared, refuses to sit at home and watch on television.

Every time Berrettini plays on a court within earshot of a booming
voice, Bartocci will be there, as he was on Saturday, pacing nervously
while following the points on the giant scoreboard perched above the
Court 17 stadium. Bartocci cannot see inside, but he knows Berrettini
hears him.

``Because of that kid,'' Bartocci said, gesturing over the fence, where
Berrettini was beating Ruud, ``I was able to live the last seven months
without a job. What he did for my restaurant, that is why I am here to
support him no matter what.''

\includegraphics{https://static01.graylady3jvrrxbe.onion/images/2020/09/06/sports/06usopen-buddy-3/merlin_176542875_d0a3f7ca-fd83-4166-8b7b-643a06a8a181-articleLarge.jpg?quality=75\&auto=webp\&disable=upscale}

A year ago, Bartocci and Berrettini were the toast of the U.S. Open ---
Berrettini as the breakout star reaching the semifinals before he lost
to the eventual champion, Rafael Nadal. Bartocci shared the stage,
sitting prominently in Berrettini's box during most of the matches.

The gregarious owner of an Italian restaurant called Via Della Pace in
the East Village of Manhattan, Bartocci certainly cut a unique figure
for a tennis fan, and the television cameras could barely turn away from
him.

When Berrettini was asked in an interview after an early victory where
he would celebrate, he revealed that he was headed to Via Della Pace, as
many of the Italian players did. That was before a fire shut down the
restaurant in February.

``It was the Italian players' lounge,'' said Corrado Tschabuschnig, the
manager for Berrettini and several other players.

Within hours of Berrettini's on-court comment last year, Bartocci
received a call from his business partner, Marco Ventura, who reported
that a crowd was gathering at the restaurant, some with oversize tennis
balls hoping to get an autograph from Berrettini and the other Italian
players. And because of all of the publicity, which mushroomed as
Berrettini advanced through the draw, the restaurant more than doubled
its business in the ensuing months.

But on the afternoon of Feb. 10, Bartocci was at the New York Open in
Long Island, a guest of the Italian player Paolo Lorenzi, when he
received another urgent call from Ventura: A fire had broken out in the
building that housed the restaurant. It was bad.

\hypertarget{sports-and-the-virus}{%
\subsubsection{Sports and the Virus}\label{sports-and-the-virus}}

\paragraph{}

Updated Sept. 4, 2020

Here's what's happening as the world of sports slowly comes back to
life:

\begin{itemize}
\item
  \begin{itemize}
  \tightlist
  \item
    The 146th running of the Kentucky Derby, which was moved to Saturday
    from May 2, will have
    \href{https://www.nytimes3xbfgragh.onion/2020/09/04/sports/horse-racing/kentucky-derby-odds-picks.html?action=click\&pgtype=Article\&state=default\&region=MAIN_CONTENT_2\&context=storylines_keepup}{no
    spectators present because of the coronavirus pandemic}.
  \item
    The coronavirus pandemic has had an
    \href{https://www.nytimes3xbfgragh.onion/2020/09/03/sports/ncaafootball/high-school-football-coronavirus-pandemic.html?action=click\&pgtype=Article\&state=default\&region=MAIN_CONTENT_2\&context=storylines_keepup}{uneven
    impact on high school football}~across the United States.
  \item
    The
    \href{https://www.nytimes3xbfgragh.onion/2020/09/02/sports/ncaafootball/coronavirus-cal-athletics-season.html?action=click\&pgtype=Article\&state=default\&region=MAIN_CONTENT_2\&context=storylines_keepup}{most
    complicated puzzle in sports is the return of college
    athletics}~during a pandemic. The University of California, Berkeley
    is allowing The Times an inside look at their journey's ups and
    downs.
  \end{itemize}
\end{itemize}

Bartocci hopped on his Harley-Davidson motorcycle and rushed through the
rain to Manhattan. When he got there, the block was swarmed by
firefighters, trucks and hoses. The fire had started behind a wall,
Bartocci said, and although there were no flames inside the restaurant,
the firefighters had to rip through the kitchen to find the source.
Everything in it was destroyed, and the restaurant was forced to close,
at least in that location.

Bartocci said that by the time the investigation was completed and
claims were filed, the pandemic had been declared. He said the insurance
company told him that because of the pandemic he would have had to close
anyway, and its payout was not enough to reopen the restaurant.

What's more, Bartocci has an E2 visa for investors, which would have
renewed automatically as long as the business was open. But now it is
set to expire in October, and he may have to leave the United States.

``That place was everything to me,'' he said. ``I would rather my house
burned down. What is a house, a place to sleep? The restaurant was my
life. But you know, I am lucky. So many people, they lose more than me.
I have my health, and my friends and family are OK, too. I will be OK.''

Image

The many faces of Giovanni Bartocci, the gregarious restaurateur in the
East Village who has befriended many of the Italian tennis
players.Credit...Calla Kessler for The New York Times

After word of the fire spread in the Italian tennis community, players
reached out to Bartocci to express sympathy. They included Fabio
Fognini, the Italian star who once argued with the restaurateur during a
match because he was supporting Fognini's opponent. (Bartocci told him,
in Italian, ``Shut up and play.'')

Berrettini hinted that he, and perhaps other players, might try to help
Bartocci and Ventura reopen the restaurant, maybe in another location.

``I'm really sorry because he had really bad luck, twice,'' Berrettini
said. ``For the fire and the virus.''

Bartocci shrugs. His support for Berrettini is about their friendship,
and for what Berrettini did for him in the past, he said. When
Berrettini qualified for the year-end championships in London, he
invited Bartocci, who of course caused a stir in that city, too, for his
vociferous support of his friend (although he did not bring the bullhorn
to the O2 Arena).

Bartocci salvaged that from the restaurant. He had used it for songs and
chants during games played by his beloved Lazio soccer team. The
restaurant also served as headquarters for the New York Lazio supporters
club.

Bartocci brings the same kind of passion to tennis, as if he is
screaming for Lazio at the Stadio Olimpico in Rome.

He took the megaphone to Lorenzi's first-round loss to Brandon Nakashima
on Court 8 last Monday. Locked outside the gates, Bartocci dragged a
police barrier to a tree, climbed up and perched on a limb as he watched
a corner of the court from there.

Berrettini's first match last week was in Louis Armstrong Stadium, but
there is no access for the public outside that stadium. Court 17 is the
perfect spot, and on Saturday, another Italian tennis fan arrived just
before the match began.

Alessandro Artoni, 25, a shipping executive from Mantua, Italy, who now
lives in Baltimore, came to New York for the weekend. He recognized
Bartocci from last year's tournament and introduced himself.

``Everyone in Italy knows Giovanni,'' Artoni said.

Next up for Berrettini is 10th-seeded Andrey Rublev on Monday afternoon
in the fourth round. But as the tournament progresses, most of the
matches are funneled into the bigger stadiums and this one is scheduled
for Armstrong. But Bartocci has surrendered.

``Hey,'' he asked a reporter. ``If Matteo makes it to the final, can
they put it on Court 17?''

Advertisement

\protect\hyperlink{after-bottom}{Continue reading the main story}

\hypertarget{site-index}{%
\subsection{Site Index}\label{site-index}}

\hypertarget{site-information-navigation}{%
\subsection{Site Information
Navigation}\label{site-information-navigation}}

\begin{itemize}
\tightlist
\item
  \href{https://help.nytimes3xbfgragh.onion/hc/en-us/articles/115014792127-Copyright-notice}{©~2020~The
  New York Times Company}
\end{itemize}

\begin{itemize}
\tightlist
\item
  \href{https://www.nytco.com/}{NYTCo}
\item
  \href{https://help.nytimes3xbfgragh.onion/hc/en-us/articles/115015385887-Contact-Us}{Contact
  Us}
\item
  \href{https://www.nytco.com/careers/}{Work with us}
\item
  \href{https://nytmediakit.com/}{Advertise}
\item
  \href{http://www.tbrandstudio.com/}{T Brand Studio}
\item
  \href{https://www.nytimes3xbfgragh.onion/privacy/cookie-policy\#how-do-i-manage-trackers}{Your
  Ad Choices}
\item
  \href{https://www.nytimes3xbfgragh.onion/privacy}{Privacy}
\item
  \href{https://help.nytimes3xbfgragh.onion/hc/en-us/articles/115014893428-Terms-of-service}{Terms
  of Service}
\item
  \href{https://help.nytimes3xbfgragh.onion/hc/en-us/articles/115014893968-Terms-of-sale}{Terms
  of Sale}
\item
  \href{https://spiderbites.nytimes3xbfgragh.onion}{Site Map}
\item
  \href{https://help.nytimes3xbfgragh.onion/hc/en-us}{Help}
\item
  \href{https://www.nytimes3xbfgragh.onion/subscription?campaignId=37WXW}{Subscriptions}
\end{itemize}
