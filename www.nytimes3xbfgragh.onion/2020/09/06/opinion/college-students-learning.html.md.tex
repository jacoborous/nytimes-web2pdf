Sections

SEARCH

\protect\hyperlink{site-content}{Skip to
content}\protect\hyperlink{site-index}{Skip to site index}

\href{https://myaccount.nytimes3xbfgragh.onion/auth/login?response_type=cookie\&client_id=vi}{}

\href{https://www.nytimes3xbfgragh.onion/section/todayspaper}{Today's
Paper}

\href{/section/opinion}{Opinion}\textbar{}The Best Reason to Go to
College

\url{https://nyti.ms/3lUYj6e}

\begin{itemize}
\item
\item
\item
\item
\item
\end{itemize}

Advertisement

\protect\hyperlink{after-top}{Continue reading the main story}

\href{/section/opinion}{Opinion}

Supported by

\protect\hyperlink{after-sponsor}{Continue reading the main story}

\hypertarget{the-best-reason-to-go-to-college}{%
\section{The Best Reason to Go to
College}\label{the-best-reason-to-go-to-college}}

It's the same as it ever was: To learn that the world is more than the
issues that divide us.

By Pico Iyer

Mr. Iyer is an author.

\begin{itemize}
\item
  Sept. 6, 2020
\item
  \begin{itemize}
  \item
  \item
  \item
  \item
  \item
  \end{itemize}
\end{itemize}

\includegraphics{https://static01.graylady3jvrrxbe.onion/images/2020/09/07/opinion/07Iyer-op-ed-print/merlin_176099325_df47ced8-bcea-4481-9cf6-1b226b267d8a-articleLarge.jpg?quality=75\&auto=webp\&disable=upscale}

As colleges throughout the United States reopen, facing a weird new
landscape of empty rooms and scattered classmates, it's easy to wonder
what these traditional places of learning still have to teach the rest
of us. Long before the pandemic, campuses were in the news not so much
for opening young minds as for closing down discussions and less for
encouraging humanity than for promoting ideologies.

Upon my own return to a university classroom, in the spring of 2019,
after a hiatus of 37 years, I imagined that my tastes and values, my
very language, might seem out-of-date to many of the students I was
instructing, and I'm sure they did. I suspected that these teenagers
would be much less concerned with books than I and my old classmates
were, and I was right. I assumed that as a writer who had been
crisscrossing the globe for 45 years, I'd have wisdom about travel to
impart, and I was wrong: Thanks in part to their generous and
well-endowed university, the 16 undergraduates in front of me spent the
first class speaking of recent trips they'd taken to Nauru and
Kyrgyzstan and Hongpo, among other places I'd barely heard of.

In almost every way, the young at this elite university seemed brighter,
more mature, more reliable and infinitely more globally aware than I and
my pals had been in our radically less diverse day. But the most
beautiful surprise was to see how deeply many of them had absorbed
lessons not to be found in any textbook. Picking up a campus newspaper
one day, I found an article by the person I'd foolishly taken to be our
class clown. He went to Mass every Sunday, he wrote, precisely because
he had no religious commitment. He wanted to learn about perspectives
other than the ones he knew. He admired the discipline and sense of
order encouraged by such a practice, which he felt he might lack
otherwise. He'd been startled by the open-mindedness of a devout
roommate, with whom he used to argue through the night. If someone of
religious faith could be so responsive to other positions, he wrote,
should not a secular liberal aspire to the same?

I realized, as I read the piece, that I had little to teach such
students in a class ostensibly about exploring cultures different from
our own. More deeply, I was impressed by how imaginatively a young
person was addressing the central problem of the times: the fact we're
all united mostly by our divisiveness. Whether in the context of climate
change or the right to life --- let alone the ethics of trying to
protect others from a killer virus by simply wearing a mask --- more and
more of us refuse ever to cross party lines. And in an age of social
media, when we all imagine we can best capture the world's attention by
shouting as loudly as possible, there's every incentive to take the most
extreme --- and polarizing --- position around.

Our institutions are not going to solve this; they (and the unwisdom of
crowds) are often the problem. As the wise Franciscan priest Richard
Rohr points out, the only thing more dangerous than individual ego is
group ego. That's one reason I, driving around blue-state Santa Barbara,
Calif., try to listen to Fox News --- I can get plenty of the other side
from my friends. It's also why I, though not a Christian, seek out the
clarity of Richard Rohr. We're caught up in an addiction to
simplifications for which the only medicine lies within. We need to be
reminded that not to be right doesn't always mean you're wrong. And that
to be terribly wronged does not mean you're innocent. The world deals in
black-or-whites no more than a hurricane or a virus does.

It's hardly surprising that so many citizens, unable to find wisdom in
the political sphere (which, almost by definition, thrives on
either/ors), look to religious figures for a more inclusive vision. Pope
Francis, in Wim Wenders's glorious documentary ``A Man of His Word,''
stresses the importance of not imposing our views on others and never
thinking in terms of simplistic us-versus-thems: Would God, Francis
asks, love Gandhi any less than he does a priest or a nun simply because
the Mahatma wasn't a Christian? The Dalai Lama, for his part, points out
that to be pro-Tibetan is not to be anti-Chinese, not least because
Tibet and China will always be neighbors; the welfare of either depends
on the other. He begins his days by praying for the health of his
``Chinese brothers and sisters.''

Traveling across Japan with the Dalai Lama a year before the pandemic, I
heard him say often that after watching the planet up close as a leader
of his people for what was then 79 years, he felt the world was
suffering through an ``emotional crisis.'' The cure, he said, was
``emotional disarmament.'' What he meant by the striking phrase was that
we can see beyond panic and rage and confusion only by using our minds,
and that part of the mind that doesn't deal in binaries. Emotional
disarmament might prove even more feasible than the nuclear type,
insofar as most of us can reform our minds more easily than we can move
a huge and intractable government. By opening our minds, we begin to
change the world.

Religion itself, of course, can be as sectarian as the enmities it
deplores, which is why the Dalai Lama, one of the world's most visible
religious figures, published a book titled ``Beyond Religion.'' It's why
he puts much of his faith in science, whose laws and discoveries lie
beyond human divisions and apply equally to believer and nonbeliever,
Muslim and Jew. Yet the same wisdom was apparent to me in 16 students
who seemed ready to look beyond convenient dogma and dehumanizing
abstraction.

One of them, a sunny and very personable gay athlete, was an unabashed
supporter of Donald Trump (whatever, he asserted, the president might
say about gay rights). When I handed out an excerpt from Barack Obama's
``Dreams From my Father'' for our group to read and discuss, I was
properly apprehensive.

The minute we assembled the following week, up shot the hand of the
passionate Trumpite. He'd been stunned, he said, by the intelligence,
the eloquence and the subtlety of ``President Obama,'' as he
respectfully called him. ``I don't agree with many of his positions,''
he said, ``and I wouldn't vote for him.'' But how could he not be swayed
by the humanity of the man's command of the word and the power of his
prose? He'd been so impressed that after completing the 20-page
assignment, he'd spent the weekend going through the entire 442-page
book.

Of all the many things I learned in that classroom, perhaps that was the
most valuable. If someone barely of voting age could open his mind so
expansively, how could I and others a generation or two older continue
acting like preschoolers? We alone among the animals, the Dalai Lama
regularly points out, enjoy reasoning minds, the capacity to see beyond
reflex. The best reason to go to school, even if you're a so-called
teacher, is to find out how much you don't know.

Pico Iyer is the author of 15 books, most recently the companion works
``Autumn Light'' and ``A Beginner's Guide to Japan.''

\emph{The Times is committed to publishing}
\href{https://www.nytimes3xbfgragh.onion/2019/01/31/opinion/letters/letters-to-editor-new-york-times-women.html}{\emph{a
diversity of letters}} \emph{to the editor. We'd like to hear what you
think about this or any of our articles. Here are some}
\href{https://help.nytimes3xbfgragh.onion/hc/en-us/articles/115014925288-How-to-submit-a-letter-to-the-editor}{\emph{tips}}\emph{.
And here's our email:}
\href{mailto:letters@NYTimes.com}{\emph{letters@NYTimes.com}}\emph{.}

\emph{Follow The New York Times Opinion section on}
\href{https://www.facebookcorewwwi.onion/nytopinion}{\emph{Facebook}}\emph{,}
\href{http://twitter.com/NYTOpinion}{\emph{Twitter (@NYTopinion)}}
\emph{and}
\href{https://www.instagram.com/nytopinion/}{\emph{Instagram}}\emph{.}

Advertisement

\protect\hyperlink{after-bottom}{Continue reading the main story}

\hypertarget{site-index}{%
\subsection{Site Index}\label{site-index}}

\hypertarget{site-information-navigation}{%
\subsection{Site Information
Navigation}\label{site-information-navigation}}

\begin{itemize}
\tightlist
\item
  \href{https://help.nytimes3xbfgragh.onion/hc/en-us/articles/115014792127-Copyright-notice}{©~2020~The
  New York Times Company}
\end{itemize}

\begin{itemize}
\tightlist
\item
  \href{https://www.nytco.com/}{NYTCo}
\item
  \href{https://help.nytimes3xbfgragh.onion/hc/en-us/articles/115015385887-Contact-Us}{Contact
  Us}
\item
  \href{https://www.nytco.com/careers/}{Work with us}
\item
  \href{https://nytmediakit.com/}{Advertise}
\item
  \href{http://www.tbrandstudio.com/}{T Brand Studio}
\item
  \href{https://www.nytimes3xbfgragh.onion/privacy/cookie-policy\#how-do-i-manage-trackers}{Your
  Ad Choices}
\item
  \href{https://www.nytimes3xbfgragh.onion/privacy}{Privacy}
\item
  \href{https://help.nytimes3xbfgragh.onion/hc/en-us/articles/115014893428-Terms-of-service}{Terms
  of Service}
\item
  \href{https://help.nytimes3xbfgragh.onion/hc/en-us/articles/115014893968-Terms-of-sale}{Terms
  of Sale}
\item
  \href{https://spiderbites.nytimes3xbfgragh.onion}{Site Map}
\item
  \href{https://help.nytimes3xbfgragh.onion/hc/en-us}{Help}
\item
  \href{https://www.nytimes3xbfgragh.onion/subscription?campaignId=37WXW}{Subscriptions}
\end{itemize}
