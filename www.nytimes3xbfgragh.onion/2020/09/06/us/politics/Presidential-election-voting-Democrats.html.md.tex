Sections

SEARCH

\protect\hyperlink{site-content}{Skip to
content}\protect\hyperlink{site-index}{Skip to site index}

\href{https://www.nytimes3xbfgragh.onion/section/politics}{Politics}

\href{https://myaccount.nytimes3xbfgragh.onion/auth/login?response_type=cookie\&client_id=vi}{}

\href{https://www.nytimes3xbfgragh.onion/section/todayspaper}{Today's
Paper}

\href{/section/politics}{Politics}\textbar{}Democrats Belatedly Launch
Operation to Share Information on Voters

\url{https://nyti.ms/3haWrTq}

\begin{itemize}
\item
\item
\item
\item
\item
\end{itemize}

\begin{itemize}
\item
  \href{https://www.nytimes3xbfgragh.onion/interactive/2020/09/08/us/elections/results-new-hampshire-primary-elections.html?action=click\&pgtype=Article\&state=default\&region=TOP_BANNER\&context=storylines_menu}{New
  Hampshire Results}
\item
  \href{https://www.nytimes3xbfgragh.onion/live/2020/09/08/us/trump-vs-biden?action=click\&pgtype=Article\&state=default\&region=TOP_BANNER\&context=storylines_menu}{Election
  Updates}
\item
  \href{https://www.nytimes3xbfgragh.onion/interactive/2020/us/elections/election-states-biden-trump.html?action=click\&pgtype=Article\&state=default\&region=TOP_BANNER\&context=storylines_menu}{Paths
  to 270}
\item
  \href{https://www.nytimes3xbfgragh.onion/interactive/2020/08/31/us/politics/vote-by-mail-deadlines.html?action=click\&pgtype=Article\&state=default\&region=TOP_BANNER\&context=storylines_menu}{Voting
  by Mail}
\item
  \href{https://www.nytimes3xbfgragh.onion/interactive/2019/us/elections/2020-presidential-election-calendar.html?action=click\&pgtype=Article\&state=default\&region=TOP_BANNER\&context=storylines_menu}{Key
  Dates}
\item
  \href{https://www.nytimes3xbfgragh.onion/newsletters/politics?action=click\&pgtype=Article\&state=default\&region=TOP_BANNER\&context=storylines_menu}{Politics
  Newsletter}
\end{itemize}

Advertisement

\protect\hyperlink{after-top}{Continue reading the main story}

Supported by

\protect\hyperlink{after-sponsor}{Continue reading the main story}

\hypertarget{democrats-belatedly-launch-operation-to-share-information-on-voters}{%
\section{Democrats Belatedly Launch Operation to Share Information on
Voters}\label{democrats-belatedly-launch-operation-to-share-information-on-voters}}

Democrats have been far behind Republicans on compiling and sharing
information that can be used by campaigns, state parties and super PACs.

\href{https://www.nytimes3xbfgragh.onion/by/reid-j-epstein}{\includegraphics{https://static01.graylady3jvrrxbe.onion/images/2019/06/25/reader-center/author-reid-epstein/9e877853d8234217b58e5762253aa771-thumbLarge.png}}

By \href{https://www.nytimes3xbfgragh.onion/by/reid-j-epstein}{Reid J.
Epstein}

\begin{itemize}
\item
  Sept. 6, 2020
\item
  \begin{itemize}
  \item
  \item
  \item
  \item
  \item
  \end{itemize}
\end{itemize}

\includegraphics{https://static01.graylady3jvrrxbe.onion/images/2020/09/06/lens/06dems-hillary/06dems-hillary-articleLarge.jpg?quality=75\&auto=webp\&disable=upscale}

WASHINGTON --- When Hillary Clinton lost the 2016 election, she blamed
Russian interference and the former F.B.I. director James Comey's
eleventh-hour resurrection of her emails for her defeat.

But she also lashed out at something that got far fewer headlines: the
Democratic National Committee's failure to keep up with Republicans in
the data arms race.

Now, with less than two months remaining before the 2020 election, the
party has started the Democratic Data Exchange, a legally independent
entity that allows campaigns, state parties, super PACs and other
independent groups that are forbidden to coordinate with each other to
share information on individual voters.

Democratic officials involved in the new data program say the system
will help them narrow what had been a yawning gap between their party
and Republicans, who started a similar independent data operation ahead
of the 2016 election. Campaigns and supportive independent groups will
now have a full, and nearly real-time, view into which voters have been
contacted by other Democratic organizations and how those voters feel
about candidates.

Access to that information can be critical, particularly in battleground
states, where the contest between President Trump and former Vice
President Joseph R. Biden Jr. is expected to be close. In Michigan,
Pennsylvania and Wisconsin --- the three states that swung the 2016
election to Mr. Trump --- the president won by a total of just 77,000
votes.

Along with the 50 state parties and the Democratic National Committee,
the exchange includes several dozen Democratic outside groups and super
PACs, including Priorities USA, Senate Majority PAC, House Majority PAC,
Emily's List, major labor and environmental organizations and Everytown
for Gun Safety, the gun-control group that is largely funded by Michael
R. Bloomberg.

The Democratic National Committee chairman, Tom Perez, spent much of
2018 quietly cajoling his party's state chairmen, who were reluctant to
relinquish control of their data to a clearinghouse outside the control
of the national committee, to sign on to the exchange.
\href{https://www.nytimes3xbfgragh.onion/2019/02/14/us/politics/on-politics-democratic-data-sharing-2020.html}{He
and others argued} that an independent data agency was necessary if the
party hoped to begin narrowing the Republicans' voter data advantage.

Image

Tom Perez, Chairman of the Democratic National Committee, spoke in
Washington last month on the 57th anniversary of the Rev. Dr. Martin
Luther King Jr.'s historic march.~Credit...Pool photo by Michael M
Santiago

``I mean, I walked into a Radio Shack, and that was a challenge,'' Mr.
Perez said of the party's data operation upon his arrival as chairman.
``We really needed to get into the --- not just the 21st century, but
the mid-21st century.''

Mrs. Clinton, following her 2016 defeat, was particularly focused on the
data gap between Democrats and Republicans. In March 2017, just weeks
after he had won election to become party chairman, Mr. Perez visited
Mrs. Clinton in her Chappaqua, N.Y., home and received a briefing about
what her plans to rebuild the party's data infrastructure would have
been.

That July, Mrs. Clinton publicly criticized the data she had received
from the party while running for president as ``mediocre to poor,
nonexistent, wrong.'' Clinton campaign veterans still privately seethe
at President Barack Obama's aides for allowing the D.N.C. to atrophy
during his second term, leaving her campaign at a substantial
disadvantage.

The election of ``2016 was the ultimate moment of frustration because
you have more and more resources going into these elections, and the
entire progressive Democratic ecosystem was not on the same page,'' said
Stephanie Schriock, the president of Emily's List, an organization that
backs Democratic women running for office.

A demonstration of the Democratic Data Exchange conducted last week for
The New York Times showed a dashboard that allows campaigns, state
parties and independent organizations to sort voters based on categories
including whom they support for a particular office --- from president
to state legislative seats and local offices --- and what their comfort
level is with voting by mail and whether they trust the Postal Service.

While campaign officials have long had access to that sort of
information, Democrats have never before been able to share it across
the party's archipelago of allied groups. And in a year in which
Democratic campaigns are eschewing door-knocking amid the coronavirus
pandemic, information on individual voters views is a valuable
commodity.

``The thing that is most predictive of if you're going to vote in this
election or vote for this candidate,'' said Lindsey Schuh Cortes, the
exchange's chief executive, ``is voter contact data where you're
literally asking somebody: `Do you support this candidate? Are you going
to turn out to vote?' This data is really helpful at the tactical level
to figure out: `Who am I contacting? Who am I not contacting?''

Ms. Schuh Cortes said Democratic campaigns and organizations would also
use the exchange to conserve their resources.

``If you were going to send the 900th text message to somebody, you
probably don't have to send that 900th text message,'' she said. ``The
other 899 probably did the job.''

The exchange has gone live in 13 states --- 11 presidential
battlegrounds, plus Colorado and Missouri --- and expects to be online
in 10 more states, which include many of those with competitive Senate
races, by the end of this week.

It works like a cooperative. State parties and the outside groups
involved have each paid a membership fee, but to receive more data from
the exchange, they must first input their own information about voters.

While the system is built for advancing Democrats in general elections,
it may be useful --- and create tensions --- in future primary contests.
Many of the Democratic organizations involved in the exchange are major
players in Democratic primaries for House and Senate seats, and most
Democratic state parties take vows of neutrality in intraparty contests.

Democratic officials said that each organization that added its data
into the exchange would be allowed to determine whether its specific
information could be used in both primaries and general elections or be
limited only to general elections.

Democrats already have evidence of how crucial the data can be. When
party officials did a test run of their system last year for the
Kentucky governor's race, three weeks before the election, the data
exchange produced a list of 14,483 voters who supported Andy Beshear,
the Democratic candidate, that the Kentucky Democratic Party did not
already have in its own voter database.

The party went on to contact them all, said Mary Nishimuta, the
executive director of the Kentucky Democrats, and 9,587 of them voted.
\href{https://www.nytimes3xbfgragh.onion/interactive/2019/11/05/us/elections/results-kentucky-general-elections.html}{Mr.
Beshear won the election} by 5,086 votes.

That sort of information had been available to Republicans but not to
Mrs. Clinton in 2016. Adhering to Federal Election Commission rules
barring coordination between candidates and independent expenditure
groups, Mrs. Clinton's top aides were unaware of what supportive
organizations were doing on her behalf and were unable to calibrate
their decisions and strategy based on what others had learned about
voters.

``I remember in 2016 going to a post-mortem meeting with a bunch of
allied groups and them listing off everything they were doing, and I
didn't know any of it,'' Robby Mook, Mrs. Clinton's campaign manager,
said in an interview last week.

The Biden campaign and its top allies will use the exchange not just to
find information about specific voters but to determine what types of
voters other supportive groups are targeting --- information that can
guide decisions on how to avoid duplication of outreach efforts.

``You had independent groups and the Clinton campaign doing voter
registration, often in same location at the same time,'' said Guy Cecil,
the chairman of Priorities USA, the largest Democratic super PAC. ``We
had people knocking on the same doors and making phone calls to the same
households.''

\hypertarget{our-2020-election-guide}{%
\section{Our 2020 Election Guide}\label{our-2020-election-guide}}

Updated ~Sept. 8, 2020

\begin{center}\rule{0.5\linewidth}{\linethickness}\end{center}

\begin{itemize}
\item ~
  \hypertarget{the-latest}{%
  \subsection{The Latest}\label{the-latest}}

  \begin{itemize}
  \item
    President Trump and his party are using a playbook that aims to
    alarm people about crime in their backyards. It didn't work in 2018,
    but
    \href{https://www.nytimes3xbfgragh.onion/2020/09/08/us/politics/trump-republicans-fear-strategy.html?action=click\&pgtype=Article\&state=default\&region=BELOW_MAIN_CONTENT\&context=storylines_guide}{both
    parties think it could resonate more this year}.
  \end{itemize}
\item ~
  \hypertarget{how-to-win-270}{%
  \subsection{How to Win 270}\label{how-to-win-270}}

  \begin{itemize}
  \item
    Joe Biden and Donald Trump need 270 electoral votes to reach the
    White House. Try building
    \href{https://www.nytimes3xbfgragh.onion/interactive/2020/us/elections/election-states-biden-trump.html?action=click\&pgtype=Article\&state=default\&region=BELOW_MAIN_CONTENT\&context=storylines_guide}{your
    own coalition of battleground states}~to see potential outcomes.
  \end{itemize}
\item ~
  \hypertarget{voting-by-mail}{%
  \subsection{Voting by Mail}\label{voting-by-mail}}

  \begin{itemize}
  \item
    Will you have enough time to vote by mail in your state? Yes, but
    it's risky to procrastinate.
    \href{https://www.nytimes3xbfgragh.onion/interactive/2020/08/31/us/politics/vote-by-mail-deadlines.html?action=click\&pgtype=Article\&state=default\&region=BELOW_MAIN_CONTENT\&context=storylines_guide}{Check
    your state's deadline.}
  \item
    \href{https://www.nytimes3xbfgragh.onion/interactive/2020/us/elections/joe-biden.html?action=click\&pgtype=Article\&state=default\&region=BELOW_MAIN_CONTENT\&context=storylines_guide}{}

    \hypertarget{joe-biden}{%
    \section{Joe Biden}\label{joe-biden}}

    \hypertarget{democrat}{%
    \subsection{Democrat}\label{democrat}}

    \href{https://www.nytimes3xbfgragh.onion/interactive/2020/us/elections/donald-trump.html?action=click\&pgtype=Article\&state=default\&region=BELOW_MAIN_CONTENT\&context=storylines_guide}{}

    \hypertarget{donald-trump}{%
    \section{Donald Trump}\label{donald-trump}}

    \hypertarget{republican}{%
    \subsection{Republican}\label{republican}}
  \end{itemize}
\item
  \hypertarget{keep-up-with-our-coverage}{%
  \subsection{Keep Up With Our
  Coverage}\label{keep-up-with-our-coverage}}

  \begin{itemize}
  \item
    Get an
    \href{https://www.nytimes3xbfgragh.onion/newsletters/politics?action=click\&pgtype=Article\&state=default\&region=BELOW_MAIN_CONTENT\&context=storylines_guide}{email}~recapping
    the day's news
  \item
    Download our mobile app on
    \href{https://apps.apple.com/us/app/nytimes/id284862083?ls=1\&mat_click_id=5c79ae7455014fd1bd66b5610c05b8f2-20191112-16948\&referrer=mat_click_id\%3D5c79ae7455014fd1bd66b5610c05b8f2-20191112-16948\%26link_click_id\%3D722930677036718082}{iOS}~and
    \href{http://a.localytics.com/android?id=com.nytimes.android\&referrer=utm_source\%3Dother_nyt_mobile_web\%26utm_medium\%3DWeb\%2520page\%26utm_term\%3DGeneral\%2520Mobile\%2520Page\%26utm_campaign\%3DNYT\%2520Mobile\%2520General\%2520Page}{Android}~and
    turn on Breaking News and Politics alerts
  \end{itemize}
\end{itemize}

Advertisement

\protect\hyperlink{after-bottom}{Continue reading the main story}

\hypertarget{site-index}{%
\subsection{Site Index}\label{site-index}}

\hypertarget{site-information-navigation}{%
\subsection{Site Information
Navigation}\label{site-information-navigation}}

\begin{itemize}
\tightlist
\item
  \href{https://help.nytimes3xbfgragh.onion/hc/en-us/articles/115014792127-Copyright-notice}{©~2020~The
  New York Times Company}
\end{itemize}

\begin{itemize}
\tightlist
\item
  \href{https://www.nytco.com/}{NYTCo}
\item
  \href{https://help.nytimes3xbfgragh.onion/hc/en-us/articles/115015385887-Contact-Us}{Contact
  Us}
\item
  \href{https://www.nytco.com/careers/}{Work with us}
\item
  \href{https://nytmediakit.com/}{Advertise}
\item
  \href{http://www.tbrandstudio.com/}{T Brand Studio}
\item
  \href{https://www.nytimes3xbfgragh.onion/privacy/cookie-policy\#how-do-i-manage-trackers}{Your
  Ad Choices}
\item
  \href{https://www.nytimes3xbfgragh.onion/privacy}{Privacy}
\item
  \href{https://help.nytimes3xbfgragh.onion/hc/en-us/articles/115014893428-Terms-of-service}{Terms
  of Service}
\item
  \href{https://help.nytimes3xbfgragh.onion/hc/en-us/articles/115014893968-Terms-of-sale}{Terms
  of Sale}
\item
  \href{https://spiderbites.nytimes3xbfgragh.onion}{Site Map}
\item
  \href{https://help.nytimes3xbfgragh.onion/hc/en-us}{Help}
\item
  \href{https://www.nytimes3xbfgragh.onion/subscription?campaignId=37WXW}{Subscriptions}
\end{itemize}
