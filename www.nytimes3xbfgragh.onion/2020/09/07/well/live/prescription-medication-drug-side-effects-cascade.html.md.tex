Sections

SEARCH

\protect\hyperlink{site-content}{Skip to
content}\protect\hyperlink{site-index}{Skip to site index}

\href{https://www.nytimes3xbfgragh.onion/section/well/live}{Live}

\href{https://myaccount.nytimes3xbfgragh.onion/auth/login?response_type=cookie\&client_id=vi}{}

\href{https://www.nytimes3xbfgragh.onion/section/todayspaper}{Today's
Paper}

\href{/section/well/live}{Live}\textbar{}The Risks of the Prescribing
Cascade

\url{https://nyti.ms/3lYmy3q}

\begin{itemize}
\item
\item
\item
\item
\item
\item
\end{itemize}

\href{https://www.nytimes3xbfgragh.onion/spotlight/at-home?action=click\&pgtype=Article\&state=default\&region=TOP_BANNER\&context=at_home_menu}{At
Home}

\begin{itemize}
\tightlist
\item
  \href{https://www.nytimes3xbfgragh.onion/2020/09/07/travel/route-66.html?action=click\&pgtype=Article\&state=default\&region=TOP_BANNER\&context=at_home_menu}{Cruise
  Along: Route 66}
\item
  \href{https://www.nytimes3xbfgragh.onion/2020/09/04/dining/sheet-pan-chicken.html?action=click\&pgtype=Article\&state=default\&region=TOP_BANNER\&context=at_home_menu}{Roast:
  Chicken With Plums}
\item
  \href{https://www.nytimes3xbfgragh.onion/2020/09/04/arts/television/dark-shadows-stream.html?action=click\&pgtype=Article\&state=default\&region=TOP_BANNER\&context=at_home_menu}{Watch:
  Dark Shadows}
\item
  \href{https://www.nytimes3xbfgragh.onion/interactive/2020/at-home/even-more-reporters-editors-diaries-lists-recommendations.html?action=click\&pgtype=Article\&state=default\&region=TOP_BANNER\&context=at_home_menu}{Explore:
  Reporters' Google Docs}
\end{itemize}

Advertisement

\protect\hyperlink{after-top}{Continue reading the main story}

Supported by

\protect\hyperlink{after-sponsor}{Continue reading the main story}

Personal Health

\hypertarget{the-risks-of-the-prescribing-cascade}{%
\section{The Risks of the Prescribing
Cascade}\label{the-risks-of-the-prescribing-cascade}}

The problem occurs when drug-induced side effects are viewed as a new
ailment and treated with yet another drug that can cause still other
side effects.

\includegraphics{https://static01.graylady3jvrrxbe.onion/images/2020/09/08/science/08BRODY-PRESCRIBINGCASCADE/08BRODY-PRESCRIBINGCASCADE-articleLarge.jpg?quality=75\&auto=webp\&disable=upscale}

\href{https://www.nytimes3xbfgragh.onion/by/jane-e-brody}{\includegraphics{https://static01.graylady3jvrrxbe.onion/images/2018/06/12/multimedia/jane-e-brody/jane-e-brody-thumbLarge.png}}

By \href{https://www.nytimes3xbfgragh.onion/by/jane-e-brody}{Jane E.
Brody}

\begin{itemize}
\item
  Published Sept. 7, 2020Updated Sept. 10, 2020
\item
  \begin{itemize}
  \item
  \item
  \item
  \item
  \item
  \item
  \end{itemize}
\end{itemize}

The medical mistakes that befell the 87-year-old mother of a North
Carolina pharmacist should not happen to anyone, and my hope is that
this column will keep you and your loved ones from experiencing similar,
all-too-common mishaps.

As the pharmacist, Kim H. DeRhodes of Charlotte, N.C., recalled, it all
began when her mother went to the emergency room two weeks after a fall
because she had lingering pain in her back and buttocks. Told she had
sciatica, the elderly woman was prescribed prednisone and a muscle
relaxant. Three days later, she became delirious, returned to the E.R.,
was admitted to the hospital, and was discharged two days later when her
drug-induced delirium resolved.

A few weeks later, stomach pain prompted a third trip to the E.R. and a
prescription for an antibiotic and proton-pump inhibitor. Within a
month, she developed severe diarrhea lasting several days. Back to the
E.R., and this time she was given a prescription for dicyclomine to
relieve intestinal spasms, which triggered another bout of delirium and
three more days in the hospital. She was discharged after lab tests and
imaging studies revealed nothing abnormal.

``Review of my mother's case highlights separate but associated
problems: likely misdiagnosis and inappropriate prescribing of
medications,''
\href{https://jamanetwork.com/journals/jamainternalmedicine/article-abstract/2732693}{Ms.
DeRhodes wrote in JAMA Internal Medicine}. ``Diagnostic errors led to
the use of prescription drugs that were not indicated and caused my
mother further harm. The muscle relaxer and prednisone led to her first
incidence of delirium. Prednisone likely led to the gastrointestinal
issues, and the antibiotic likely led to the diarrhea, which led to the
prescribing of dicyclomine, which led to the second incidence of
delirium.''

The doctors who wrote the woman's prescriptions apparently never
consulted \href{https://www.aafp.org/afp/2020/0101/p56.html}{the Beers
Criteria}, a list created by the American Geriatrics Society of drugs
often unsafe for the elderly.

In short, Ms. DeRhodes's mother was a victim of two medical problems
that are too often overlooked by examining doctors and unrecognized by
families. The first is giving an 87-year-old medications known to be
unsafe for the elderly; the second is a costly and often frightening
medically induced condition called ``a prescribing cascade'' that starts
with drug-induced side effects which are then viewed as a new ailment
and treated with yet another drug or drugs that can cause still other
side effects.

I'd like to think that none of this would have happened if instead of
going to the E.R. the older woman had seen her primary care doctor. But
experts told me that no matter where patients are treated, they are not
immune to getting caught in a prescribing cascade. The problem also can
happen to people who self-treat with over-the-counter or herbal
remedies. Nor is it limited to the elderly; young people can also become
victims of a prescribing cascade, Ms. DeRhodes said.

``Doctors are often taught to think of everything as a new problem,''
Dr. Timothy Anderson, internist at Beth Israel Deaconess Medical Center
in Boston, said. ``They have to start thinking about whether the patient
is on medication and whether the medication is the problem.''

``Doctors are very good at prescribing but not so good at
deprescribing,'' Ms. DeRhodes said. ``And a lot of times patients are
given a prescription without first trying something else.''

A popular treatment for high blood pressure, which afflicts a huge
proportion of older people, is a
\href{https://jamanetwork.com/journals/jamainternalmedicine/article-abstract/2761267}{common
precipitant of the prescribing cascade}, Dr. Anderson said.

He cited a
\href{https://jamanetwork.com/journals/jamainternalmedicine/article-abstract/2761272}{Canadian
study of 41,000 older adults with hypertension}who were prescribed drugs
called calcium channel blockers. Within a year after treatment began,
nearly one person in 10 was given a diuretic to treat leg swelling
caused by the first drug. Many were inappropriately prescribed a
so-called loop diuretic that Dr. Anderson said can result in
dehydration, kidney problems, lightheadedness and falls.

Type 2 diabetes is another common condition in which medications are
often improperly prescribed to treat drug-induced side effects, said
Lisa M. McCarthy, doctor of pharmacy at the University of Toronto who
directed the Canadian study. Recognizing a side effect for what it is
can be hampered when the effect doesn't happen for weeks or even months
after a drug is started. While patients taking opioids for pain may
readily recognize constipation as a consequence, Dr. McCarthy said that
over time, patients taking metformin for diabetes can develop diarrhea
and may self-treat with loperamide, which in turn can cause dizziness
and confusion.

Dr. Paula Rochon, geriatrician at Women's College Hospital in Ontario,
said patients taking a drug called a cholinesterase inhibitor to treat
early dementia can develop urinary incontinence, which is then treated
with another drug that can worsen the patient's confusion.

Complicating matters is the large number of drugs some people take.
``Older adults frequently take many medications, with two-fifths taking
five or more,'' Dr. Anderson
\href{https://jamanetwork.com/journals/jamainternalmedicine/article-abstract/2761267}{wrote
in JAMA Internal Medicine}. In cases of polypharmacy, as this is called,
it can be hard to determine which, if any, of the drugs a person is
taking is the cause of the current symptom.

Dr. Rochon emphasized that a prescribing cascade can happen to anybody.
She said, ``Everyone needs to consider the possibility every time a drug
is prescribed.''

Before accepting a prescription, she recommended that patients or their
caregivers should ask the doctor a series of questions, starting with
``Am I experiencing a symptom that could be a side effect of a drug I'm
taking?'' Follow-up questions should include:

Is this new drug being used to treat a side effect?

Is there a safer drug available than the one I'm taking?

Could I take a lower dose of the prescribed drug?

Most important, Dr. Rochon said, patients should ask ``Do I need to take
this drug at all?''

Patients and doctors alike often overlook or resist alternatives to
medication that may be more challenging to adopt than swallowing a pill.
For example, among well-established nondrug remedies for hypertension
are weight loss, increasing physical activity, consuming less salt and
other sources of sodium, and eating more potassium-rich foods like
bananas and cantaloupe.

For some patients, frequent use of a nonsteroidal anti-inflammatory drug
sold over-the-counter, like ibuprofen or naproxen, is
\href{https://www.thelancet.com/journals/lancet/article/PIIS0140-6736(17)31188-1/fulltext}{responsible
for their elevated blood pressure}.

The risk of getting caught in a prescribing cascade is increased when
patients are prescribed medications by more than one provider. It's up
to patients to be sure every doctor they consult is given an up-to-date
list of every drug they take, whether prescription or over-the-counter,
as well as nondrug remedies and dietary supplements. Dr. Rochon
recommended that patients maintain an up-to-date list of when and why
they started every new drug, along with its dose and frequency, and show
that list to the doctor as well.

Advertisement

\protect\hyperlink{after-bottom}{Continue reading the main story}

\hypertarget{site-index}{%
\subsection{Site Index}\label{site-index}}

\hypertarget{site-information-navigation}{%
\subsection{Site Information
Navigation}\label{site-information-navigation}}

\begin{itemize}
\tightlist
\item
  \href{https://help.nytimes3xbfgragh.onion/hc/en-us/articles/115014792127-Copyright-notice}{©~2020~The
  New York Times Company}
\end{itemize}

\begin{itemize}
\tightlist
\item
  \href{https://www.nytco.com/}{NYTCo}
\item
  \href{https://help.nytimes3xbfgragh.onion/hc/en-us/articles/115015385887-Contact-Us}{Contact
  Us}
\item
  \href{https://www.nytco.com/careers/}{Work with us}
\item
  \href{https://nytmediakit.com/}{Advertise}
\item
  \href{http://www.tbrandstudio.com/}{T Brand Studio}
\item
  \href{https://www.nytimes3xbfgragh.onion/privacy/cookie-policy\#how-do-i-manage-trackers}{Your
  Ad Choices}
\item
  \href{https://www.nytimes3xbfgragh.onion/privacy}{Privacy}
\item
  \href{https://help.nytimes3xbfgragh.onion/hc/en-us/articles/115014893428-Terms-of-service}{Terms
  of Service}
\item
  \href{https://help.nytimes3xbfgragh.onion/hc/en-us/articles/115014893968-Terms-of-sale}{Terms
  of Sale}
\item
  \href{https://spiderbites.nytimes3xbfgragh.onion}{Site Map}
\item
  \href{https://help.nytimes3xbfgragh.onion/hc/en-us}{Help}
\item
  \href{https://www.nytimes3xbfgragh.onion/subscription?campaignId=37WXW}{Subscriptions}
\end{itemize}
