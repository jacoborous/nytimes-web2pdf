Sections

SEARCH

\protect\hyperlink{site-content}{Skip to
content}\protect\hyperlink{site-index}{Skip to site index}

\href{/section/world/europe}{Europe}\textbar{}London's Bridges Really
Are Falling Down

\url{https://nyti.ms/338Kecz}

\begin{itemize}
\item
\item
\item
\item
\item
\end{itemize}

\includegraphics{https://static01.graylady3jvrrxbe.onion/images/2020/09/07/world/07LONDON-BRIDGES1/merlin_176618325_790730df-9505-413b-b8f5-5594a813a05f-articleLarge.jpg?quality=75\&auto=webp\&disable=upscale}

\hypertarget{londons-bridges-really-are-falling-down}{%
\section{London's Bridges Really Are Falling
Down}\label{londons-bridges-really-are-falling-down}}

Three major crossings on the Thames are closed to cars --- one of them
considered too dangerous even to walk across. Even the landmark Tower
Bridge was recently shut for two days.

Tower Bridge, recently shut for two days by a drawbridge problem, seen
from beneath London Bridge, closed to cars during
repairs.Credit...Andrew Testa for The New York Times

Supported by

\protect\hyperlink{after-sponsor}{Continue reading the main story}

\href{https://www.nytimes3xbfgragh.onion/by/mark-landler}{\includegraphics{https://static01.graylady3jvrrxbe.onion/images/2019/10/22/reader-center/author-mark-landler/author-mark-landler-thumbLarge-v3.png}}

By \href{https://www.nytimes3xbfgragh.onion/by/mark-landler}{Mark
Landler}

\begin{itemize}
\item
  Published Sept. 7, 2020Updated Sept. 8, 2020, 3:58 a.m. ET
\item
  \begin{itemize}
  \item
  \item
  \item
  \item
  \item
  \end{itemize}
\end{itemize}

LONDON --- One by one, they stepped forward to tell their stories.
Children suddenly forced to travel two hours each way to school.
Pensioners whose weekly doctors' appointments have turned into arduous,
half-day treks. Shopkeepers whose businesses have been crippled by the
disappearance of commuters.

All because Hammersmith Bridge, a majestic but badly corroded
19th-century suspension bridge that connects the district of Barnes with
much of London, was closed last month for safety reasons.

``Now, I need to wake up at quarter past 6, every day, six days a
week,'' said Aston Jenkins, 10, drawing sympathetic groans from the
frustrated, if exceedingly polite, crowd protesting recently at the
bridge. ``I can't cope with that.''

While Hammersmith Bridge's structural problems are particularly dire, it
is far from the only London bridge that is crumbling. Two major
crossings in the city center, Vauxhall Bridge and London Bridge, are
closed to car traffic while they receive urgent repairs. Tower Bridge,
the very symbol of London, was closed for two days last month after a
mechanical glitch jammed its drawbridge open.

It fell to a young schoolgirl --- outfitted in a red cardigan and
patent-leather Mary Janes, and brandishing a placard with angry pink
letters --- to make the inevitable point: ``London Bridges are falling
down!''

\includegraphics{https://static01.graylady3jvrrxbe.onion/images/2020/09/07/world/07LONDON-BRIDGES2/merlin_176618388_c40c53ef-a77f-4671-bc31-184e62629321-articleLarge.jpg?quality=75\&auto=webp\&disable=upscale}

Philip Englefield, a professional magician who lives in Barnes, pointed
out that when
\href{https://www.nytimes3xbfgragh.onion/interactive/2018/09/06/world/europe/genoa-italy-bridge.html}{a
suspension bridge collapsed in Genoa, Italy}, in 2018, killing 43
people, the Italians worked tirelessly, even as the country battled the
coronavirus pandemic, to build a replacement. It
\href{https://www.nytimes3xbfgragh.onion/2020/08/03/world/europe/genoa-italy-new-bridge.html}{was
inaugurated last month}.

``Why can't we do that?'' Mr. Englefield asked the crowd, as a gentle
rain further dampened their spirits. ``For goodness' sake, this is
England.''

It turns out that is precisely the problem: Hammersmith Bridge is an apt
metaphor for all the ways the country has changed after a decade of
economic austerity, years of political wars over Brexit, and months of
lockdown to combat the pandemic, the last of which has decimated
already-stressed public finances.

Like other London roads and bridges, Hammersmith Bridge had been
neglected for decades. Fully repairing it would cost an estimated 141
million pounds (\$187 million) --- funds that neither Hammersmith \&
Fulham Council, which owns the bridge, nor London's transportation
authority, which depends on it, currently have.

Transport for London, which runs the subway and bus system and some
major roads, has already had to negotiate a nearly £2 billion bailout
from the government to make up for a shortfall in revenue after
ridership plummeted during the lockdown. Except for rush hour, London's
subways are still largely ghost trains.

Hammersmith has appealed for help to Prime Minister Boris Johnson. But
he won election by promising to spend money on marquee projects like a
\$130 billion-plus high-speed railway, not a cast-iron relic of Queen
Victoria's reign.

He also wants to spread the wealth to Britain's economically challenged
Midlands and North, not rescue a leafy, affluent enclave of London,
where professionals commute from gracious Regency villas to jobs in the
City and students practice on the manicured playing fields of the elite
St. Paul's School.

Image

Pedestrians crossing Vauxhall Bridge, west of Parliament. It is closed
to most motor traffic while it undergoes emergency repairs.
~Credit...Andrew Testa for The New York Times

``The national government is afraid of spending money in London because
it would threaten its `leveling up' agenda,'' said Tony Travers, an
expert in urban affairs at the London School of Economics. ``Promising
to build shiny things for the future is more attractive than fixing road
surfaces or mending bridges.''

It doesn't help that the member of Parliament from Mr. Johnson's
Conservatives who represented the district that encompasses Barnes, Zac
Goldsmith, lost his seat in the last election. Mr. Goldsmith, a
well-connected friend of Mr. Johnson's, had pledged to fix the bridge
during his campaign. His successor, Sarah Olney, from the centrist
Liberal Democrats, said she could not get any cabinet ministers to
answer her letters pleading for help.

Michael White, a former political editor at The Guardian who lives on
the north bank of the Thames, pointed out a problem of asymmetry:
Barnes, on the southern side, needs the bridge more than Hammersmith, on
the northern side, because scores of its commuters cross it every day to
reach the nearest Tube station. There is less traffic in the opposite
direction, which makes an expensive repair job politically hard to sell
for officials in less well-off Hammersmith.

Still, the Labour Party leader of the council, Stephan Cowan, insisted
that Hammersmith was fully committed to fixing the bridge --- if it can
find a financial lifeline. He credited the council with averting a
potential calamity by hiring engineers to inspect the bridge in 2014.
They found a web of tiny fractures in its cast-iron pedestals, evidence
of untold years of corrosion.

In April 2019, the authorities closed the bridge to cars, but left it
open to pedestrians and cyclists. Then, after a recent heat wave,
inspectors discovered that the fractures had widened. Because cast iron
is more brittle than steel, those changes raised the danger that the
pedestals could shatter, plunging the bridge into the Thames. The
council immediately closed the bridge to everyone.

``If we hadn't done the comprehensive integrity review,'' Mr. Cowan
said, ``I genuinely believe we could have had a catastrophe.''

Image

Paddling along the Thames past London Bridge, a traffic artery for the
financial district that is also partly closed for
repairs.Credit...Andrew Testa for The New York Times

Not only is the bridge, and the footpath under it, off limits, the Port
of London has banned boats from sailing underneath it. That will disrupt
the annual \href{https://www.theboatrace.org/}{Boat Race} between Oxford
and Cambridge universities, since by custom, the rowers cover a 4.2-mile
stretch of the Thames that rounds the bend at Barnes, where revelers
line up under the bridge's swooping cables.

In a letter to the prime minister last month, Mr. Cowan appealed to Mr.
Johnson's sense of history. What a ``terrible metaphor'' it would be, he
said, to allow a pioneering example of 19th-century engineering ``to
simply crumble away in the middle of the Thames, at the heart of our
capital city.'' In truth, he said, the bridge's unusual design has long
made it vulnerable to structural problems, and its cast-iron
construction has made it much harder and costlier to fix.

The bridge narrowly escaped destruction in 1996 when the Irish
Republican Army planted two powerful plastic explosives underneath
\href{https://www.nytimes3xbfgragh.onion/1996/04/26/world/london-bombs-were-huge-police-say.html}{that
failed to detonate}. Four years later, another I.R.A. faction
successfully exploded a bomb under the bridge, forcing it to close for
repairs for two years.

Residents could face a similar or even longer wait this time. Even
stopgap fixes are costly: Stabilizing the bridge enough so that people
could walk across it and boats could pass under it would cost £46
million, Mr. Cowan said. Building a temporary bridge for pedestrians and
cyclists would cost £27 million and take six to nine months.

In the meantime, the locals are floating other solutions, like starting
a ferry service or running shuttle buses. Some, like Toby Gordon-Smith,
have resorted to roundabout routes across other bridges. (There are more
than a dozen road or pedestrian crossings between Hammersmith Bridge and
Tower Bridge.) Mr. Gordon-Smith, 46, who uses a wheelchair, said he
chose to live in a riverfront apartment in Barnes because he could wheel
himself across the bridge to his office in Hammersmith --- 10 minutes
door to door.

Image

Hammersmith Bridge, a Victorian-era gem, is closed even to pedestrians
for fear that it might collapse.Credit...Andrew Testa for The New York
Times

``This is an important place for me to live, to be able to access my
work, to be able to access the rest of London,'' he said.

For older people who came to the rally, the fragility of London's
bridges is more than just grist for a nursery rhyme. Christopher Morcom,
81, recalled that in 1967, an American entrepreneur, Robert McCulloch,
bought the crumbling London Bridge, dismantled it, and transported it
stone by stone to
\href{https://www.google.co.uk/travel/things-to-do/see-all?g2lb=2502548\%2C4258168\%2C4270442\%2C4306835\%2C4308226\%2C4317915\%2C4322823\%2C4328159\%2C4371335\%2C4401769\%2C4403882\%2C4419364\%2C4424916\%2C4425793\%2C4427776\%2C4432284\%2C4270859\%2C4284970\%2C4291517\%2C4412693\&hl=en\&gl=uk\&un=1\&dest_mid=\%2Fm\%2F0qpxs\&dest_state_type=sattd\&dest_src=ts\&sa=X\&ved=2ahUKEwiCir6h5NbrAhWS2aQKHSXaA_4Q69EBKAAwCnoECAQQCw\#ttdm=34.459092_-114.343583_12\&ttdmf=\%252Fm\%252F0dlk69g}{Lake
Havasu City, Ariz.,} where it now sits as a tourist attraction in the
desert. (The London Bridge currently undergoing work is a replacement
for that 19th-century version.)

It all gave Mr. Morcom the germ of an idea. ``I don't know whether this
old bridge is reparable,'' he said, gesturing to Hammersmith Bridge.
``Maybe we should sell it to the president of the United States.''

Advertisement

\protect\hyperlink{after-bottom}{Continue reading the main story}

\hypertarget{site-index}{%
\subsection{Site Index}\label{site-index}}

\hypertarget{site-information-navigation}{%
\subsection{Site Information
Navigation}\label{site-information-navigation}}

\begin{itemize}
\tightlist
\item
  \href{https://help.nytimes3xbfgragh.onion/hc/en-us/articles/115014792127-Copyright-notice}{©~2020~The
  New York Times Company}
\end{itemize}

\begin{itemize}
\tightlist
\item
  \href{https://www.nytco.com/}{NYTCo}
\item
  \href{https://help.nytimes3xbfgragh.onion/hc/en-us/articles/115015385887-Contact-Us}{Contact
  Us}
\item
  \href{https://www.nytco.com/careers/}{Work with us}
\item
  \href{https://nytmediakit.com/}{Advertise}
\item
  \href{http://www.tbrandstudio.com/}{T Brand Studio}
\item
  \href{https://www.nytimes3xbfgragh.onion/privacy/cookie-policy\#how-do-i-manage-trackers}{Your
  Ad Choices}
\item
  \href{https://www.nytimes3xbfgragh.onion/privacy}{Privacy}
\item
  \href{https://help.nytimes3xbfgragh.onion/hc/en-us/articles/115014893428-Terms-of-service}{Terms
  of Service}
\item
  \href{https://help.nytimes3xbfgragh.onion/hc/en-us/articles/115014893968-Terms-of-sale}{Terms
  of Sale}
\item
  \href{https://spiderbites.nytimes3xbfgragh.onion}{Site Map}
\item
  \href{https://help.nytimes3xbfgragh.onion/hc/en-us}{Help}
\item
  \href{https://www.nytimes3xbfgragh.onion/subscription?campaignId=37WXW}{Subscriptions}
\end{itemize}
