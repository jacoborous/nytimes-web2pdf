Sections

SEARCH

\protect\hyperlink{site-content}{Skip to
content}\protect\hyperlink{site-index}{Skip to site index}

\href{https://myaccount.nytimes3xbfgragh.onion/auth/login?response_type=cookie\&client_id=vi}{}

\href{https://www.nytimes3xbfgragh.onion/section/todayspaper}{Today's
Paper}

\href{/section/opinion}{Opinion}\textbar{}How to Fix New York's \$5
Billion Budget Crisis

\url{https://nyti.ms/3jS91IP}

\begin{itemize}
\item
\item
\item
\item
\item
\end{itemize}

Advertisement

\protect\hyperlink{after-top}{Continue reading the main story}

\href{/section/opinion}{Opinion}

Supported by

\protect\hyperlink{after-sponsor}{Continue reading the main story}

\hypertarget{how-to-fix-new-yorks-5-billion-budget-crisis}{%
\section{How to Fix New York's \$5 Billion Budget
Crisis}\label{how-to-fix-new-yorks-5-billion-budget-crisis}}

Before he borrows, Mayor de Blasio needs to make significant cuts to
avoid greater pain later.

By
\href{https://www.nytimes3xbfgragh.onion/interactive/opinion/editorialboard.html}{The
Editorial Board}

The editorial board is a group of opinion journalists whose views are
informed by expertise, research, debate and certain longstanding ****
\href{https://www.nytimes3xbfgragh.onion/interactive/2018/opinion/editorialboard.html}{values}.
It is separate from the newsroom.

\begin{itemize}
\item
  Sept. 7, 2020
\item
  \begin{itemize}
  \item
  \item
  \item
  \item
  \item
  \end{itemize}
\end{itemize}

\includegraphics{https://static01.graylady3jvrrxbe.onion/images/2020/09/07/opinion/07budget_editorial/07budget_editorial-articleLarge.jpg?quality=75\&auto=webp\&disable=upscale}

New York is facing the nightmare scenario that its political leaders
have feared since the 1970s, when the city nearly went bankrupt. It is
staring down a budget hole of more than \$5 billion, along with hard
questions about how to fill it.

Mayor Bill de Blasio has asked the State Legislature to give him the
authority to borrow the \$5 billion. Without it, the mayor says, he'll
be forced to lay off or furlough
\href{https://www.nytimes3xbfgragh.onion/2020/06/24/nyregion/budget-layoffs-nyc-mta-coronavirus.html}{22,000}
city workers.

In the end, New York may have to borrow some money until revenues
recover and the city is back on its feet. Like other areas of the
country hard-hit by the coronavirus, the city and state will also need
more federal aid in the months and years ahead.

Borrowing should be a last resort since it increases the cost of every
dollar the city spends, though it can be an appropriate strategy to get
through a short-term crisis. But borrowing to meet operating expenses is
especially hazardous. Cities that do so over and over again are at
greater risk of the kind of bankruptcy faced by New York in the late
1970s and Detroit in 2013.

Any borrowing should be accompanied by a clear plan for how the city
would use the funds to stabilize its finances and what it is doing to
ensure the government is the right size for New York's needs and
revenues. ``Otherwise, we're using borrowing to prop up spending we
can't afford,'' said Andrew Rein, president of the Citizens Budget
Commission, a nonpartisan watchdog group. ``We'll make our kids pay our
bills.''

Before Mr. de Blasio adds billions to the city's debt sheet --- or lays
off thousands of workers --- he needs to find savings.

It won't be easy. The city's budget grew under Mr. de Blasio, to \$92
billion last year from
\href{https://ibo.nyc.ny.us/RevenueSpending/citywide-summary.html}{about
\$73 billion} in 2014, his first year in office. Complicating matters,
the mayor has hired
\href{https://www.nytimes3xbfgragh.onion/2017/06/15/nyregion/high-number-city-employees-bill-deblasio.html}{tens
of thousands of employees} over his tenure, adding significantly to the
city's pension and retirement obligations.

To make cuts without slashing vital services or laying off workers, the
mayor will have to be creative, make unpopular decisions and demand
serious cost-saving measures from nearly every city agency and,
crucially, the municipal unions.

One way to begin is with a far stricter
\href{https://www.nytimes3xbfgragh.onion/2020/04/16/nyregion/nyc-budget-coronavirus.html}{hiring
freeze}. Every year, some 20,000 city workers leave their jobs or
retire. Yet as of June, the work force was reduced by only about 800
from the year before, according to the Citizens Budget Commission. If
the city hired about 7,500 fewer workers each year, it could save an
additional \$750 million annually, the commission has said.

The mayor will need to do something he has rarely been able to: ask the
labor unions to share in the sacrifice. The Citizens Budget Commission
found that the city could save nearly a quarter of a billion dollars in
the first year alone, rising to a saving of \$750 million annually after
several years.

Nicole Gelinas, a senior fellow at the Manhattan Institute, a
conservative think tank, said she determined that a \$150,000 salary cap
on the city's nonunion work force could save New York \$200 million
every year.

The mayor must also get serious about enforcing overtime caps, which
\href{https://cbcny.org/research/overboard-ot}{have been blown} for
years by city agencies. As just one example, this year's budget calls
for overtime pay at the Police Department to be reduced by
\href{https://www.nytimes3xbfgragh.onion/2020/07/01/opinion/new-york-city-budget-nypd.html}{\$350
million}, a commitment that should be kept.

There are other cuts to be made. Scott Stringer, the city comptroller,
has urged the mayor to demand that agencies come to the table with
greater savings, an exercise former Mayor Michael Bloomberg turned to
frequently that can help force agencies to become more efficient.

Laura Feyer, a spokeswoman for Mayor de Blasio, said the city had
already shrunk its budget significantly. ``We can't just cut our way out
of this Covid-19-induced budget hole,'' she said in a email. ``We are
not asking for borrowing to avoid making hard choices. We're going to
continue having discussions with unions to avert as much pain as
possible, but we all agree long-term borrowing is the best solution.''

The uncertainty around New York's financial position has some calling
for the state's Financial Control Board, created in the mid-1970s to
oversee the city through its fiscal crisis, to take the reins of New
York's finances. That is premature and should be avoided if at all
possible. New York's mayor and City Council, duly elected by New York
voters, are far more accountable to residents than a panel in Albany.
The city should be given a real chance at managing this crisis.

To make it through, the city needs Mr. de Blasio to act swiftly and
forcefully to make tough cuts that will save the city greater pain
later. With any luck, this lame-duck mayor is still up to the task.

\emph{The Times is committed to publishing}
\href{https://www.nytimes3xbfgragh.onion/2019/01/31/opinion/letters/letters-to-editor-new-york-times-women.html}{\emph{a
diversity of letters}} \emph{to the editor. We'd like to hear what you
think about this or any of our articles. Here are some}
\href{https://help.nytimes3xbfgragh.onion/hc/en-us/articles/115014925288-How-to-submit-a-letter-to-the-editor}{\emph{tips}}\emph{.
And here's our email:}
\href{mailto:letters@NYTimes.com}{\emph{letters@NYTimes.com}}\emph{.}

\emph{Follow The New York Times Opinion section on}
\href{https://www.facebookcorewwwi.onion/nytopinion}{\emph{Facebook}}\emph{,}
\href{http://twitter.com/NYTOpinion}{\emph{Twitter (@NYTopinion)}}
\emph{and}
\href{https://www.instagram.com/nytopinion/}{\emph{Instagram}}\emph{.}

Advertisement

\protect\hyperlink{after-bottom}{Continue reading the main story}

\hypertarget{site-index}{%
\subsection{Site Index}\label{site-index}}

\hypertarget{site-information-navigation}{%
\subsection{Site Information
Navigation}\label{site-information-navigation}}

\begin{itemize}
\tightlist
\item
  \href{https://help.nytimes3xbfgragh.onion/hc/en-us/articles/115014792127-Copyright-notice}{©~2020~The
  New York Times Company}
\end{itemize}

\begin{itemize}
\tightlist
\item
  \href{https://www.nytco.com/}{NYTCo}
\item
  \href{https://help.nytimes3xbfgragh.onion/hc/en-us/articles/115015385887-Contact-Us}{Contact
  Us}
\item
  \href{https://www.nytco.com/careers/}{Work with us}
\item
  \href{https://nytmediakit.com/}{Advertise}
\item
  \href{http://www.tbrandstudio.com/}{T Brand Studio}
\item
  \href{https://www.nytimes3xbfgragh.onion/privacy/cookie-policy\#how-do-i-manage-trackers}{Your
  Ad Choices}
\item
  \href{https://www.nytimes3xbfgragh.onion/privacy}{Privacy}
\item
  \href{https://help.nytimes3xbfgragh.onion/hc/en-us/articles/115014893428-Terms-of-service}{Terms
  of Service}
\item
  \href{https://help.nytimes3xbfgragh.onion/hc/en-us/articles/115014893968-Terms-of-sale}{Terms
  of Sale}
\item
  \href{https://spiderbites.nytimes3xbfgragh.onion}{Site Map}
\item
  \href{https://help.nytimes3xbfgragh.onion/hc/en-us}{Help}
\item
  \href{https://www.nytimes3xbfgragh.onion/subscription?campaignId=37WXW}{Subscriptions}
\end{itemize}
