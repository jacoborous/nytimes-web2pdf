Sections

SEARCH

\protect\hyperlink{site-content}{Skip to
content}\protect\hyperlink{site-index}{Skip to site index}

\href{/section/opinion}{Opinion}\textbar{}A 150,000-Bird Orchestra in
the Sky

\url{https://nyti.ms/35dRB5a}

\begin{itemize}
\item
\item
\item
\item
\item
\end{itemize}

\includegraphics{https://static01.graylady3jvrrxbe.onion/images/2020/09/07/opinion/07renkl1/merlin_176374272_1e313f20-440a-4509-96b2-cecb16ac74d6-articleLarge.jpg?quality=75\&auto=webp\&disable=upscale}

\href{/section/opinion}{Opinion}

\hypertarget{a-150000-bird-orchestra-in-the-sky}{%
\section{A 150,000-Bird Orchestra in the
Sky}\label{a-150000-bird-orchestra-in-the-sky}}

A huge flock of purple martins is using Nashville as a staging ground
for the fall migration --- and bringing music back to the city's
shuttered symphony center.

Purple martins roosting in trees in Nashville, Tenn.Credit...William
DeShazer for The New York Times

Supported by

\protect\hyperlink{after-sponsor}{Continue reading the main story}

\href{https://www.nytimes3xbfgragh.onion/by/margaret-renkl}{\includegraphics{https://static01.graylady3jvrrxbe.onion/images/2017/04/08/opinion/margaret-renkl/margaret-renkl-thumbLarge-v2.png}}

By \href{https://www.nytimes3xbfgragh.onion/by/margaret-renkl}{Margaret
Renkl}

Contributing Opinion Writer

\begin{itemize}
\item
  Sept. 7, 2020
\item
  \begin{itemize}
  \item
  \item
  \item
  \item
  \item
  \end{itemize}
\end{itemize}

NASHVILLE --- At first they circle high in the evening sky. But as night
descends, they, too, begin to descend, bird by bird, one at a time, and
then all in a rush:
\href{https://www.instagram.com/p/CEfrnK0BMwn/?igshid=rgygk0ym4lzz}{150,000
purple martins swirling together}, each bird calling to the others in
the failing light as they sweep past the tops of buildings in the heart
of downtown Nashville. To anyone watching from the ground, they look
like one great airborne beast, one unmistakable, singular mind.

Their music grows louder and louder as the circles tighten and the birds
swing lower and lower, settling in the branches of sidewalk trees, or
swerving to take off again as new waves of birds dip down. They circle
the building and return. They lift off, circle, reverse, settle, lift
off again. Again and again and again, until finally it is dark. Their
chittering voices fall silent. Their rustling wings fall still.

It is not like Hitchcock: Watching these birds is nothing at all like
watching crows and sea gulls and sparrows attack the characters in ``The
Birds,'' Alfred Hitchcock's classic horror film. The purple martins that
have been gathering here the past few weeks are merely doing what purple
martins always do this time of year: flocking together to fatten up on
insects before making the long flight to South America, where they will
spend the winter.

That's not to say the birds aren't causing problems. The place where
they have chosen to roost this time is Nashville's Schermerhorn Symphony
Center, which was already having a terrible year. With all scheduled
programming canceled or postponed by the pandemic and so much of the
symphony budget based on ticket sales, the organization had no choice
but to furlough all the musicians and most of the staff and hope for
better days. What the Nashville Symphony got instead was a plaza full of
bird droppings and elm trees so burdened by the weight of 150,000 birds
alighting in them night after night that whole limbs are now bent and
hanging limp.

\includegraphics{https://static01.graylady3jvrrxbe.onion/images/2020/09/07/opinion/07renkl2/merlin_176374176_7362f4d3-e07d-40de-8da5-e184f381349c-articleLarge.jpg?quality=75\&auto=webp\&disable=upscale}

The folks at the Schermerhorn at first assumed the birds roosting in
their trees were starlings. Downtown Nashville is home to a large number
of European starlings that live here year-round, and they have been a
nuisance in years past. It's easy to mistake a flock of purple martins
for a flock of starlings, especially when actual starlings join the
martin flock from time to time.

Starlings are an invasive species, introduced during the early 1890s by
Shakespeare enthusiasts determined to bring to the United States every
bird ever mentioned in Shakespeare. All 200 million starlings now living
in North America are descended from a few dozen birds unwisely released
into Central Park during the late 19th century. Thanks to the Migratory
Bird Treaty Act, it is against the law to kill native songbirds. It is
perfectly legal to kill starlings.

The transcendently beautiful Schermerhorn is built of limestone, which
is highly porous. ``The sheer amount of bird poop was causing a massive
amount of damage,'' my old friend Jonathan Marx, the interim chief
operating officer of the Nashville Symphony, said when I called him to
ask about the purple martins. ``But we never had any intention of
killing the birds. We just wanted them to move on.'' The plan was to
disperse them by fogging the trees with grapeseed oil.

Purple martins have been roosting in the Nashville area for years --- at
least since 1996, according to Melinda Welton, the conservation policy
co-chair of the Tennessee Ornithological Society --- though always
before in much smaller numbers. Among birders, word quickly got around
that the purple martins had settled in at the Schermerhorn this year,
and in far, far greater numbers than ever before. ``It's a pretty
remarkable roost --- definitely one of the larger ones in the country,''
Joe Siegrist, the president and chief executive of the Purple Martin
Conservation Association, said on the phone last week.

Which is why Kim Bailey, Kim Matthews, John Noel, Anne Paine, Ms. Welton
and Mary Glynn Williamson went into action as soon as Mr. Noel noticed a
pest control truck on the symphony plaza. It was, as Mr. Marx put it,
``a collision of people who are taking care of their property with
people who are staring in awe and wonder at the birds.''

Purple martins are already in trouble from virtually every angle
imaginable. Climate change has intensified hurricane season, making the
fall migration even more perilous. Deforestation has destroyed the
birds' natural nesting sites, and aggressive nonnative species like
starlings and house sparrows have claimed most of those that remain.
Like other swallows, purple martins are insectivores, but pesticides
have made food scarce. One reason the birds chose Nashville as their
migration staging ground may be its proximity to the insect-rich
Cumberland River.

That night, while Ms. Bailey, who works as a staff naturalist at the
Warner Parks Nature Center, explained to the exterminators that purple
martins are a federally protected species, others in the group starting
calling and texting and messaging everyone they could think of who might
be able to help: News Channel 5, the mayor's office, the Tennessee
Wildlife Resources Agency, and local conservation nonprofits like the
Tennessee Wildlife Federation, the Nature Conservancy in Tennessee, and
the Nashville Wildlife Conservation Center.

Image

Purple martins have been roosting in the Nashville area for years,
though always before in much smaller numbers.~Credit...William DeShazer
for The New York Times

Those folks reached others, who in turn contacted others still. With
phones ringing and emails flying and social media on fire, the
exterminators hastily decamped. The group stayed put, Ms. Bailey told me
in an email, until they received assurances from a T.W.R.A. officer that
he had contacted the pest control company and the truck would not be
returning that night.

And now, like a flock of purple martins, this story veers in an
unexpected direction. A tale of conflict becomes instead the story of
human beings who listened to one another and then came up with a plan
that benefits everyone involved, and the birds most of all.

Mr. Marx heard from a number of conservation groups that evening and
others the following day. Each time he explained that the symphony staff
had no idea they were hosting purple martins and, now that they knew the
truth, would never harm or harass the birds. But he also pointed out
that the flock had already caused significant property damage: The cost
of power-washing the front of the building alone is at least \$10,000,
and that's not even addressing the rest of the building or the damage to
the trees.

``As soon as we heard that, we started trying to think of ways in which
we could work together,'' Terry Cook, the state director for the Nature
Conservancy in Tennessee, told me. ``One, we wanted to mitigate the
current impact of the roost, but, two, we wanted to think about
long-term opportunities to either make the site less preferable to
purple martins in future years or to embrace this as a unique Nashville
event.''

Within hours, the Tennessee Wildlife Federation and the Nature
Conservancy in Tennessee had joined forces to start
\href{https://act.tnwf.org/a/purple-martins}{a fund-raising campaign} to
help with cleanup costs. ``In the conservation community, we felt like
we needed to rally around this problem so the symphony wouldn't have to
carry this burden alone,'' said the Tennessee Wildlife Federation's
Kendall McCarter, who hosts a nesting colony of purple martins in his
own yard every year. ``Especially right now, when they're in a very
difficult place because of Covid.''

The initial campaign to pay for power washing the Schermerhorn's facade
was fully funded within hours, but the appeal is ongoing, and any extra
money it raises will be used to treat damage to the trees, to replace
trees that can't be saved, and to help with costs that arise during
future purple martin migrations. Because the birds, which seem to prefer
well-lighted roosts, will most likely be back.

In one way of looking at it, this rescue operation mimics the long
relationship between human beings and purple martins themselves: Even as
we are responsible for the birds' troubles, we are also responsible for
their survival. The population east of the Rocky Mountains, where 98
percent of all purple martins live, ``is completely reliant on people
putting up bird houses for them to reproduce in,'' said Mr. Siegrist.
``If people didn't do that, the bird would go extinct in the majority of
its range. Each one of those birds putting on that spectacular display
in downtown Nashville exists because people cared enough to put up a
bird house. Each one of those birds came from somebody's backyard.''

``We're so thankful to have community partners who are willing to help
us deal with this completely unexpected situation,'' said Mr. Marx,
``because we need to be putting our focus on
\href{https://www.nashvillesymphony.org/contribute/donate-now/}{the
fund-raising that's going to allow us to bring our musicians back to
work}. This is a time when so many people are under so many forms of
duress, but one thing we know is that music is one of those things that
brings people together.''

Until then, this collaboration between naturalists and the symphony is,
for everyone involved, a happy ending at a time when people are
desperate for happy endings. ``I'm so excited about how it's been
handled there in Nashville,'' Mr. Siegrist said. ``I think it can be a
blueprint for other communities.''

I find myself dreaming of a time when the musicians of the Nashville
Symphony are back in that beautiful space, perhaps even playing a sunset
concert, the doors of the Schermerhorn thrown wide to the music of
purple martins swooping down from the sky. What a glorious sound that
would be, after this year of silence and fear. What a gift to gather
together and hear that music --- the music our own species makes and the
music of the birds. Both at once.

\emph{You can donate to the campaign to pay for purple-martin cleanup at
this} \href{https://act.tnwf.org/a/purple-martins}{\emph{Tennessee
Wildlife Foundation website}}\emph{, and support the Nashville
Symphony,}
\href{https://www.nashvillesymphony.org/contribute/donate-now/}{\emph{here}}\emph{.}

Margaret Renkl is a contributing opinion writer who covers flora, fauna,
politics and culture in the American South. She is the author of the
book ``\href{https://milkweed.org/book/late-migrations}{Late Migrations:
A Natural History of Love and Loss}.''

\emph{The Times is committed to publishing}
\href{https://www.nytimes3xbfgragh.onion/2019/01/31/opinion/letters/letters-to-editor-new-york-times-women.html}{\emph{a
diversity of letters}} \emph{to the editor. We'd like to hear what you
think about this or any of our articles. Here are some}
\href{https://help.nytimes3xbfgragh.onion/hc/en-us/articles/115014925288-How-to-submit-a-letter-to-the-editor}{\emph{tips}}\emph{.
And here's our email:}
\href{mailto:letters@NYTimes.com}{\emph{letters@NYTimes.com}}\emph{.}

\emph{Follow The New York Times Opinion section on}
\href{https://www.facebookcorewwwi.onion/nytopinion}{\emph{Facebook}}\emph{,}
\href{http://twitter.com/NYTOpinion}{\emph{Twitter (@NYTopinion)}}
\emph{and}
\href{https://www.instagram.com/nytopinion/}{\emph{Instagram}}\emph{.}

Advertisement

\protect\hyperlink{after-bottom}{Continue reading the main story}

\hypertarget{site-index}{%
\subsection{Site Index}\label{site-index}}

\hypertarget{site-information-navigation}{%
\subsection{Site Information
Navigation}\label{site-information-navigation}}

\begin{itemize}
\tightlist
\item
  \href{https://help.nytimes3xbfgragh.onion/hc/en-us/articles/115014792127-Copyright-notice}{©~2020~The
  New York Times Company}
\end{itemize}

\begin{itemize}
\tightlist
\item
  \href{https://www.nytco.com/}{NYTCo}
\item
  \href{https://help.nytimes3xbfgragh.onion/hc/en-us/articles/115015385887-Contact-Us}{Contact
  Us}
\item
  \href{https://www.nytco.com/careers/}{Work with us}
\item
  \href{https://nytmediakit.com/}{Advertise}
\item
  \href{http://www.tbrandstudio.com/}{T Brand Studio}
\item
  \href{https://www.nytimes3xbfgragh.onion/privacy/cookie-policy\#how-do-i-manage-trackers}{Your
  Ad Choices}
\item
  \href{https://www.nytimes3xbfgragh.onion/privacy}{Privacy}
\item
  \href{https://help.nytimes3xbfgragh.onion/hc/en-us/articles/115014893428-Terms-of-service}{Terms
  of Service}
\item
  \href{https://help.nytimes3xbfgragh.onion/hc/en-us/articles/115014893968-Terms-of-sale}{Terms
  of Sale}
\item
  \href{https://spiderbites.nytimes3xbfgragh.onion}{Site Map}
\item
  \href{https://help.nytimes3xbfgragh.onion/hc/en-us}{Help}
\item
  \href{https://www.nytimes3xbfgragh.onion/subscription?campaignId=37WXW}{Subscriptions}
\end{itemize}
