Sections

SEARCH

\protect\hyperlink{site-content}{Skip to
content}\protect\hyperlink{site-index}{Skip to site index}

\href{https://www.nytimes3xbfgragh.onion/section/fashion}{Fashion}

\href{https://myaccount.nytimes3xbfgragh.onion/auth/login?response_type=cookie\&client_id=vi}{}

\href{https://www.nytimes3xbfgragh.onion/section/todayspaper}{Today's
Paper}

\href{/section/fashion}{Fashion}\textbar{}Eleanor Jacobs, 91, Dies; a
Force Behind the Earth Shoe Phenomenon

\url{https://nyti.ms/3jNxZZR}

\begin{itemize}
\item
\item
\item
\item
\item
\end{itemize}

Advertisement

\protect\hyperlink{after-top}{Continue reading the main story}

Supported by

\protect\hyperlink{after-sponsor}{Continue reading the main story}

\hypertarget{eleanor-jacobs-91-dies-a-force-behind-the-earth-shoe-phenomenon}{%
\section{Eleanor Jacobs, 91, Dies; a Force Behind the Earth Shoe
Phenomenon}\label{eleanor-jacobs-91-dies-a-force-behind-the-earth-shoe-phenomenon}}

After discovering a pair in Copenhagen that eased her back pain, she and
her husband rebranded the funny-looking shoe and sold millions in the
U.S.

\includegraphics{https://static01.graylady3jvrrxbe.onion/images/2020/09/08/obituaries/03Jacobs1-print/03Jacobs1-articleLarge.jpg?quality=75\&auto=webp\&disable=upscale}

\href{https://www.nytimes3xbfgragh.onion/by/richard-sandomir}{\includegraphics{https://static01.graylady3jvrrxbe.onion/images/2018/12/10/multimedia/author-richard-sandomir/author-richard-sandomir-thumbLarge.png}}

By \href{https://www.nytimes3xbfgragh.onion/by/richard-sandomir}{Richard
Sandomir}

\begin{itemize}
\item
  Sept. 5, 2020
\item
  \begin{itemize}
  \item
  \item
  \item
  \item
  \item
  \end{itemize}
\end{itemize}

Eleanor Jacobs, who stumbled upon a pair of odd-looking shoes in Denmark
and, with her husband, Raymond, created a short-lived phenomenon by
selling them in the United States as Earth Shoes, died on Aug. 25 at her
home in Litchfield, Conn. She was 91.

Her daughter Susan Jacobs said the cause was congestive heart failure.

In 1969, while she and her husband were vacationing in Denmark, Ms.
Jacobs's chronic back pain increased from all the walking she had been
doing. She found unexpected salvation with a pair of negative-heel shoes
she found at a store in Copenhagen.

Originally called
\href{http://foottalk.blogspot.com/2005/07/down-to-earth.html}{Anne
Kalso}Minus-Heels, named after the Danish yoga instructor who had
designed them, they featured a wide toe box and a sole that was thicker
in the front than in the back.

As the family continued to Norway, Ms. Jacobs was so pleased with her
new purchase that Mr. Jacobs, who was a commercial photographer at the
time, suggested they try to sell the shoes in the United States.

When they called Ms. Kalso, she was happy that the Jacobses were not in
the shoe business --- she had rejected earlier offers from conventional
shoe manufacturers --- and that Mr. Jacobs wanted to market the shoes as
part of a back-to-nature movement.

On April 22 the next year, the couple opened a store devoted to their
homely import in their street-level brownstone near Union Square in
Manhattan. Later that morning, they noticed a stream of young people,
many of them hippies, passing their store toward Union Square Park, and
Ms. Jacobs asked one of them what was going on.

``Hey, man, it's Earth Day! There is a love-in-down the street in the
park,'' she recalled a young man telling her when she wrote about that
day for the Litchfield Enquirer in 2008. ``Come and join us.''

The Jacobses did not join them. Instead, they renamed the shoes.

``I turned to look at my husband, who had a eureka look on his formerly
worried face,'' Ms. Jacobs wrote. ```Ellie, that's it,' he screamed at
me. `We'll call them Earth Shoes.'''

He scribbled the name in black crayon on a piece of cardboard and placed
it in the store window. Customers quickly began buying the shoes.

\includegraphics{https://static01.graylady3jvrrxbe.onion/images/2020/09/08/obituaries/03Jacobs5-print/merlin_176541939_ddd9b0a3-f9fb-4962-97db-a8f185506c92-articleLarge.jpg?quality=75\&auto=webp\&disable=upscale}

It was the start of an exciting but brief business odyssey for the
couple, who were not footwear experts and had never run a business.
After making a deal with Ms. Kalso for North American distribution
rights (which would expand to world rights, except in Denmark), they
licensed and supplied more than 100 stores. Demand grew so quickly that
the Jacobses opened a factory in 1973 in Middleboro, Mass., to augment
their Danish imports. In all, they would sell millions of pairs of Earth
Shoes.

Despite its peculiar look, the Earth Shoe became popular beyond the
counterculture set, largely because of the help it was said to offer to
aching backs and feet (a point of disagreement among podiatrists and
other foot experts). The shoes were featured in The Whole Earth Catalog
and on Time magazine billboards in Grand Central Terminal.

``The ugly duckling shoe seems to have caught on like `The
Exorcist,'''\href{https://www.nytimes3xbfgragh.onion/1974/02/25/archives/shoes-that-make-you-waddle-like-a-duckand-they-sell.html}{The
Times reported in 1974.} ``And indeed, the line at the cash counter on
Monday would have been envied by many theater managers. Army parkas and
bluejeans cued up with mink and Brooks Brothers tweeds to pay for their
shoes and walked out happily with purchases in burlap bags."

The Times also reported that a chauffeur pulled up at the Jacobses'
store on East 17th Street one day, carrying a penciled outline of the
actor Walter Matthau's feet. He was shooting a film in Manhattan, and
his feet were hurting. Six pairs were sent back to him. He bought two.

Around that time, the Jacobses added new styles, including an athletic
shoe and a hiking boot, to their staples of walking shoe and sandals.
Ms. Jacobs said they needed to appeal to a widening customer base.

``The people who are buying our shoes are no longer leftover flower
children from the '60s,'' she told The Washington Post in 1975.
``They're worn by a cross section of American people now. So we had to
develop a new line of shoe with that in mind.''

Image

Ms. Jacobs, left, with Anne Kalso, the Danish yoga instructor who
designed the shoe that Ms. Jacobs and her husband would market as the
Earth Shoe.Credit...Martha Holmes

Eleanor Cohen was born on July 25, 1929, in the Bronx to immigrants from
Lodz, Poland. Her father, Samuel, worked in the garment industry; her
mother, Mary (Praw) Cohen, was a homemaker.

After graduating from high school in 1946, Eleanor worked as a secretary
at Teachers College at Columbia University for a few years before moving
on to the J. Walter Thompson advertising agency, where she was a
secretary and copywriter in training until 1958. During that time, she
also began to study painting with the Russian-born artist Evsa Model.
She married Mr. Jacobs in 1955.

Over the next decade, she raised their two daughters while Raymond built
his career as
a\href{https://www.raymondjacobsphotography.com/raymond-jacobs-biography}{fine
art and commercial photographer, whose work}appeared in Fortune,
Esquire, Harper's Bazaar, Redbook and Ladies' Home Journal, as well as
in advertising campaigns for Campbell's Soup, Pan Am and Johnson \&
Johnson. She often accompanied her husband on his assignments.

By 1977, the Jacobses' seven-year Earth Shoe sojourn was over. Hurt by
knockoffs and by questions about the shoe's purported health benefits,
the company was burdened by debt and cash flow problems, and filed for
Chapter 11 bankruptcy protection. (A new company, Earth Inc.,
\href{https://earthshoes.com/pages/earth-kalso-grounded-heel-shoes}{revived
what it calls Earth Kalso Shoes}in 2001.)

The bank that had lent the company money ``didn't trust us, nor our
flower children, nor our unconventional marketing strategy despite our
steady growth and profits,'' Ms. Jacobs wrote in The Litchfield Enquirer
in 2000.

In addition to her daughter Susan, Ms. Jacobs is survived by another
daughter, Laura Pavlick, and two grandchildren.
\href{https://www.nytimes3xbfgragh.onion/1993/03/20/obituaries/raymond-jacobs-69-co-founder-of-earth-shoe-company-in-1970-s.html}{Her
husband died in 1993.}

While running the Earth Shoe business, Ms. Jacobs studied art history at
New York University's Gallatin School of Individualized Study (she
received her bachelor's degree in 1979). The Jacobses opened the Art
Appreciation Gallery in the former Earth Shoe space in 1978. She later
worked as an administrator at Sotheby's and Hirschl \& Adler Galleries,
before becoming an art consultant.

She also kept a decades-rich supply of Earth Shoes --- sandals, oxfords
and shearling-lined winter boots --- that she wore nearly to the end of
her life.

Advertisement

\protect\hyperlink{after-bottom}{Continue reading the main story}

\hypertarget{site-index}{%
\subsection{Site Index}\label{site-index}}

\hypertarget{site-information-navigation}{%
\subsection{Site Information
Navigation}\label{site-information-navigation}}

\begin{itemize}
\tightlist
\item
  \href{https://help.nytimes3xbfgragh.onion/hc/en-us/articles/115014792127-Copyright-notice}{©~2020~The
  New York Times Company}
\end{itemize}

\begin{itemize}
\tightlist
\item
  \href{https://www.nytco.com/}{NYTCo}
\item
  \href{https://help.nytimes3xbfgragh.onion/hc/en-us/articles/115015385887-Contact-Us}{Contact
  Us}
\item
  \href{https://www.nytco.com/careers/}{Work with us}
\item
  \href{https://nytmediakit.com/}{Advertise}
\item
  \href{http://www.tbrandstudio.com/}{T Brand Studio}
\item
  \href{https://www.nytimes3xbfgragh.onion/privacy/cookie-policy\#how-do-i-manage-trackers}{Your
  Ad Choices}
\item
  \href{https://www.nytimes3xbfgragh.onion/privacy}{Privacy}
\item
  \href{https://help.nytimes3xbfgragh.onion/hc/en-us/articles/115014893428-Terms-of-service}{Terms
  of Service}
\item
  \href{https://help.nytimes3xbfgragh.onion/hc/en-us/articles/115014893968-Terms-of-sale}{Terms
  of Sale}
\item
  \href{https://spiderbites.nytimes3xbfgragh.onion}{Site Map}
\item
  \href{https://help.nytimes3xbfgragh.onion/hc/en-us}{Help}
\item
  \href{https://www.nytimes3xbfgragh.onion/subscription?campaignId=37WXW}{Subscriptions}
\end{itemize}
