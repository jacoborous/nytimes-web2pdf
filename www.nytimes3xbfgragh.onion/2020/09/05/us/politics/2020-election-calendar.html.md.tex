Sections

SEARCH

\protect\hyperlink{site-content}{Skip to
content}\protect\hyperlink{site-index}{Skip to site index}

\href{https://www.nytimes3xbfgragh.onion/section/politics}{Politics}

\href{https://myaccount.nytimes3xbfgragh.onion/auth/login?response_type=cookie\&client_id=vi}{}

\href{https://www.nytimes3xbfgragh.onion/section/todayspaper}{Today's
Paper}

\href{/section/politics}{Politics}\textbar{}A Big 2020 Campaign
Question: Wait, What Day Is It?

\url{https://nyti.ms/3jOBi2T}

\begin{itemize}
\item
\item
\item
\item
\item
\end{itemize}

\begin{itemize}
\item
  \href{https://www.nytimes3xbfgragh.onion/interactive/2020/09/08/us/elections/results-new-hampshire-primary-elections.html?action=click\&pgtype=Article\&state=default\&region=TOP_BANNER\&context=storylines_menu}{New
  Hampshire Results}
\item
  \href{https://www.nytimes3xbfgragh.onion/live/2020/09/08/us/trump-vs-biden?action=click\&pgtype=Article\&state=default\&region=TOP_BANNER\&context=storylines_menu}{Election
  Updates}
\item
  \href{https://www.nytimes3xbfgragh.onion/interactive/2020/us/elections/election-states-biden-trump.html?action=click\&pgtype=Article\&state=default\&region=TOP_BANNER\&context=storylines_menu}{Paths
  to 270}
\item
  \href{https://www.nytimes3xbfgragh.onion/interactive/2020/08/31/us/politics/vote-by-mail-deadlines.html?action=click\&pgtype=Article\&state=default\&region=TOP_BANNER\&context=storylines_menu}{Voting
  by Mail}
\item
  \href{https://www.nytimes3xbfgragh.onion/interactive/2019/us/elections/2020-presidential-election-calendar.html?action=click\&pgtype=Article\&state=default\&region=TOP_BANNER\&context=storylines_menu}{Key
  Dates}
\item
  \href{https://www.nytimes3xbfgragh.onion/newsletters/politics?action=click\&pgtype=Article\&state=default\&region=TOP_BANNER\&context=storylines_menu}{Politics
  Newsletter}
\end{itemize}

Advertisement

\protect\hyperlink{after-top}{Continue reading the main story}

Supported by

\protect\hyperlink{after-sponsor}{Continue reading the main story}

Political Memo

\hypertarget{a-big-2020-campaign-question-wait-what-day-is-it}{%
\section{A Big 2020 Campaign Question: Wait, What Day Is
It?}\label{a-big-2020-campaign-question-wait-what-day-is-it}}

Time is marching on, toward Election Day. Somehow.

\includegraphics{https://static01.graylady3jvrrxbe.onion/images/2020/09/05/us/politics/05calendar-promo/05calendar-promo-threeByTwoMediumAt2X.jpg}

\href{https://www.nytimes3xbfgragh.onion/by/sarah-lyall}{\includegraphics{https://static01.graylady3jvrrxbe.onion/images/2018/02/20/multimedia/author-sarah-lyall/author-sarah-lyall-thumbLarge.jpg}}

By \href{https://www.nytimes3xbfgragh.onion/by/sarah-lyall}{Sarah Lyall}

\begin{itemize}
\item
  Sept. 5, 2020
\item
  \begin{itemize}
  \item
  \item
  \item
  \item
  \item
  \end{itemize}
\end{itemize}

What day is it today? Is it:

1. Sunday

2. The same day as it was yesterday, for all intents and purposes

3. Another dimension, a dimension not only of sight and sound but of
mind

4. Some number of days before Election Day, though to be honest, I'm not
sure how many

These last six months have so often felt like an endless loop of horror
--- pandemic followed by economic collapse followed by social upheaval
followed by natural disaster, repeat --- but Americans of all political
persuasions have at least had Election Day to look forward to. Decide
this thing, one way or another! Put us out of our existential misery.

Nothing is normal in this most distressing of years, though, not even
Election Day, if by ``normal'' we mean that everyone votes at basically
the same time --- on Nov. 3 --- and then we all stay up watching TV, or
some TV equivalent, until the winner is declared later that night.

It is true that some voters will actually cast their ballots in person
that day, and that various states (as well as the N.B.A., and
\href{https://www.nytimes3xbfgragh.onion/2020/08/24/us/politics/lebron-james-poll-workers.html}{LeBron
James} with his new group More Than a Vote) are trying to bolster
on-the-day voting by, for instance, recruiting new poll workers and
\href{https://www.nytimes3xbfgragh.onion/2020/08/28/sports/basketball/nba-playoffs-resume.html}{turning
some arenas} into polling places.

Then there are those voters who will cast their ballots in person at a
polling place, but not on Election Day. Rules vary from state to state.
(Please try to keep up.) People in some states might even find
themselves voting before the first
\href{https://www.nytimes3xbfgragh.onion/2020/09/02/business/media/trump-biden-debate-moderators.html}{presidential
debate}.
\href{https://www.npr.org/2020/09/04/909597279/voting-season-begins-north-carolina-mails-out-first-ballots}{Early
voting in Minnesota, for instance, starts on Sept. 18} --- less than two
weeks from now.

But many people, or most people (your guess is as good as anyone's; a
\href{https://www.opinium.com/resource-center/biden-leads-trump-by-wide-margin-in-august/}{recent
poll} put it at 39 percent of Americans) plan to vote by mail this year.
North Carolina
\href{https://www.nytimes3xbfgragh.onion/interactive/2020/08/31/us/politics/vote-by-mail-deadlines.html}{began
mailing its ballots} on Friday. A
\href{https://www.nytimes3xbfgragh.onion/interactive/2020/08/11/us/politics/vote-by-mail-us-states.html}{New
York Times analysis} found that at least three-quarters of all American
voters, roughly 80 million people, will be eligible to receive ballots
in the mail for 2020.

Whether the votes themselves will be received, or counted, on time by
the states is another matter entirely.

For months now, even as many states had to reschedule their primaries
because of the pandemic,
\href{https://www.nytimes3xbfgragh.onion/2020/06/24/us/politics/november-2020-election-day-results.html}{officials
have been planning for} --- and warning about --- the high probability
of a delay in November's results.

``What is going to be very different this year is election night,'' said
Elaine C. Kamarck, a senior fellow at the Brookings Institution and the
author of ``Primary Politics: Everything You Need to Know About How
America Nominates Its Presidential Candidates.'' ``We might not know the
answer to the presidential election for weeks, and that is going to
confuse people.''

Yes, it is.

Also confusing: what the
\href{https://www.nytimes3xbfgragh.onion/2020/08/15/us/post-office-vote-by-mail.html}{recent
upheavals} at the U.S. Postal Service, an institution whose effective
functioning no longer seems to be one of those things Americans depend
on as part of our national consensus, mean for the already-bewildering
\href{https://projects.fivethirtyeight.com/how-to-vote-2020/}{patchwork
of dates and regulations} governing mail-in voting in each state.

Over the summer, a number of head-spinning social media posts declared
that Election Day is actually taking place on October 20 this year,
given the mail troubles. That is
\href{https://www.politifact.com/factchecks/2020/jul/28/facebook-posts/how-early-should-you-send-your-mail-ballot-make-su/}{not
true}, according to the fact-checking site PolitiFact, which
investigated the claim with the Postal Service.

Or, not exactly true.

``The Postal Service recommends that domestic, nonmilitary voters mail
their ballots at least one week prior to their state's due date to allow
for timely receipt by election officials,'' a postal official said.
``The Postal Service also recommends that voters contact local election
officials for information about deadlines.''

In July, the Postal Service took out its fog machine and sprayed even
more confusion over the issue,
\href{https://www.nytimes3xbfgragh.onion/2020/08/15/us/elections/the-postal-service-warns-states-it-may-not-meet-mail-in-ballot-deadlines.html}{sending
ominous warning letters to states} urging them to require that voters
request ballots at least 15 days before the election.

By any objective measure, of course --- by the rules of the calendar and
the laws of science --- the progression of time today, in the seventh
month of the coronavirus pandemic, is exactly the same as it has always
been. It may feel like today is some fake day outside the normal
calendar --- March 3000, or, if you prefer, June 666 --- but it is not.
The year is moving forward at a constant rate. The seasons are changing;
the days are passing. An hour is still an hour.

It's our perceptions that are discombobulated, Ellen Braaten, assistant
professor of psychology at Harvard Medical School,
\href{https://www.statnews.com/2020/08/17/pandemic-stress-alters-your-perception-of-time/}{pointed
out recently}in STAT News. Our routines are awry, and things we depend
on to happen at a particular time have not happened. We are disturbed
and battered by the bewildering barrage of events, even as our own lives
seem to be stuck in stressed-out states of animated suspension. ``If
we're really anxious, we might experience time as slowing down,'' Ms.
Braaten said. ``Anxiety is one of those things that can actually make
time feel like it's going on forever.''

Some things are happening at the same time as before, but not in the
same way. Memorial Day didn't signal the start of summer as many people
usually experience it. Labor Day won't usher in a normal, new-beginning
fall. School feels more like ``school.''

Other events have happened, but at new times. Sports are being played in
truncated seasons on jury-rigged schedules --- is it really the 2020
Olympics if it takes place in 2021? --- with cardboard cutouts instead
of people in the stands. A return to a sense of normalcy in the calendar
seems to depend on a vaccine against Covid-19, but despite researchers'
efforts to accelerate a process that typically takes years, we don't
know when we will get one.

``Everything is topsy-turvy right now,'' said Michael N. Shadlen, a
professor of neuroscience at Columbia University. ``It's extremely
disorienting when your calendar gets messed up --- as if someone planted
you in the middle of the Sahara with no GPS.''

If this were a regular election season, like 2016, which now feels like
it was 400 years ago, Ms. Kamarck would have spent much of the summer on
the road. She would have traveled to Milwaukee for the Democratic
National Convention, which would have been held in July, as regularly
scheduled, and to Charlotte, N.C., in August for the Republican National
Convention. She would have mingled with crowds of strangers and talked
to them about their political views.

But she did none of those things. She stayed home, watching the
conventions from afar and experiencing for herself the weirdness of the
passage of time, in politics as in life.

``It's very, very slow,'' Ms. Kamarck said. ``We don't have the usual
markers. People aren't rushing out to do school shopping. They're not
rushing out for new clothes.''

In a normal year, we would be on the verge of a ratcheting up of
down-to-the-wire campaign activity. But activity is not really activity
this year either, unless it is done by phone or computer. ``Politically,
the campaigns aren't out there on the ground,'' Ms. Kamarck said. ``For
people in politics, we simply aren't traveling.''

It seems almost beside the point now to mention something that President
Trump said a little over a month ago, because the president's way of
flooding the zone of incredulity worsens the sense that time is
stretching and snapping back on us.

But here he was, adding to our sense of temporal and democratic
instability with an
\href{https://twitter.com/realDonaldTrump/status/1288818160389558273?ref_src=twsrc\%5Etfw\%7Ctwcamp\%5Etweetembed\%7Ctwterm\%5E1288818160389558273\%7Ctwgr\%5Eshare_3\&ref_url=https\%3A\%2F\%2Fwww.npr.org\%2F2020\%2F07\%2F31\%2F897724197\%2Fciting-election-delay-tweet-influential-trump-ally-now-demands-his-re-impeachmen}{incendiary
tweet} at the end of July that
\href{https://www.nytimes3xbfgragh.onion/2020/07/31/us/politics/trump-tweet-democracy.html}{denounced
mail-in voting}, saying without evidence that ``2020 will be the most
INACCURATE AND FRAUDULENT Election in history'' and suggesting that we
``delay the Election.''

(Note to citizens: the president can do a lot of things. But he cannot
do that.)

\hypertarget{our-2020-election-guide}{%
\section{Our 2020 Election Guide}\label{our-2020-election-guide}}

Updated ~Sept. 8, 2020

\begin{center}\rule{0.5\linewidth}{\linethickness}\end{center}

\begin{itemize}
\item ~
  \hypertarget{the-latest}{%
  \subsection{The Latest}\label{the-latest}}

  \begin{itemize}
  \item
    President Trump and his party are using a playbook that aims to
    alarm people about crime in their backyards. It didn't work in 2018,
    but
    \href{https://www.nytimes3xbfgragh.onion/2020/09/08/us/politics/trump-republicans-fear-strategy.html?action=click\&pgtype=Article\&state=default\&region=BELOW_MAIN_CONTENT\&context=storylines_guide}{both
    parties think it could resonate more this year}.
  \end{itemize}
\item ~
  \hypertarget{how-to-win-270}{%
  \subsection{How to Win 270}\label{how-to-win-270}}

  \begin{itemize}
  \item
    Joe Biden and Donald Trump need 270 electoral votes to reach the
    White House. Try building
    \href{https://www.nytimes3xbfgragh.onion/interactive/2020/us/elections/election-states-biden-trump.html?action=click\&pgtype=Article\&state=default\&region=BELOW_MAIN_CONTENT\&context=storylines_guide}{your
    own coalition of battleground states}~to see potential outcomes.
  \end{itemize}
\item ~
  \hypertarget{voting-by-mail}{%
  \subsection{Voting by Mail}\label{voting-by-mail}}

  \begin{itemize}
  \item
    Will you have enough time to vote by mail in your state? Yes, but
    it's risky to procrastinate.
    \href{https://www.nytimes3xbfgragh.onion/interactive/2020/08/31/us/politics/vote-by-mail-deadlines.html?action=click\&pgtype=Article\&state=default\&region=BELOW_MAIN_CONTENT\&context=storylines_guide}{Check
    your state's deadline.}
  \item
    \href{https://www.nytimes3xbfgragh.onion/interactive/2020/us/elections/joe-biden.html?action=click\&pgtype=Article\&state=default\&region=BELOW_MAIN_CONTENT\&context=storylines_guide}{}

    \hypertarget{joe-biden}{%
    \section{Joe Biden}\label{joe-biden}}

    \hypertarget{democrat}{%
    \subsection{Democrat}\label{democrat}}

    \href{https://www.nytimes3xbfgragh.onion/interactive/2020/us/elections/donald-trump.html?action=click\&pgtype=Article\&state=default\&region=BELOW_MAIN_CONTENT\&context=storylines_guide}{}

    \hypertarget{donald-trump}{%
    \section{Donald Trump}\label{donald-trump}}

    \hypertarget{republican}{%
    \subsection{Republican}\label{republican}}
  \end{itemize}
\item
  \hypertarget{keep-up-with-our-coverage}{%
  \subsection{Keep Up With Our
  Coverage}\label{keep-up-with-our-coverage}}

  \begin{itemize}
  \item
    Get an
    \href{https://www.nytimes3xbfgragh.onion/newsletters/politics?action=click\&pgtype=Article\&state=default\&region=BELOW_MAIN_CONTENT\&context=storylines_guide}{email}~recapping
    the day's news
  \item
    Download our mobile app on
    \href{https://apps.apple.com/us/app/nytimes/id284862083?ls=1\&mat_click_id=5c79ae7455014fd1bd66b5610c05b8f2-20191112-16948\&referrer=mat_click_id\%3D5c79ae7455014fd1bd66b5610c05b8f2-20191112-16948\%26link_click_id\%3D722930677036718082}{iOS}~and
    \href{http://a.localytics.com/android?id=com.nytimes.android\&referrer=utm_source\%3Dother_nyt_mobile_web\%26utm_medium\%3DWeb\%2520page\%26utm_term\%3DGeneral\%2520Mobile\%2520Page\%26utm_campaign\%3DNYT\%2520Mobile\%2520General\%2520Page}{Android}~and
    turn on Breaking News and Politics alerts
  \end{itemize}
\end{itemize}

Advertisement

\protect\hyperlink{after-bottom}{Continue reading the main story}

\hypertarget{site-index}{%
\subsection{Site Index}\label{site-index}}

\hypertarget{site-information-navigation}{%
\subsection{Site Information
Navigation}\label{site-information-navigation}}

\begin{itemize}
\tightlist
\item
  \href{https://help.nytimes3xbfgragh.onion/hc/en-us/articles/115014792127-Copyright-notice}{©~2020~The
  New York Times Company}
\end{itemize}

\begin{itemize}
\tightlist
\item
  \href{https://www.nytco.com/}{NYTCo}
\item
  \href{https://help.nytimes3xbfgragh.onion/hc/en-us/articles/115015385887-Contact-Us}{Contact
  Us}
\item
  \href{https://www.nytco.com/careers/}{Work with us}
\item
  \href{https://nytmediakit.com/}{Advertise}
\item
  \href{http://www.tbrandstudio.com/}{T Brand Studio}
\item
  \href{https://www.nytimes3xbfgragh.onion/privacy/cookie-policy\#how-do-i-manage-trackers}{Your
  Ad Choices}
\item
  \href{https://www.nytimes3xbfgragh.onion/privacy}{Privacy}
\item
  \href{https://help.nytimes3xbfgragh.onion/hc/en-us/articles/115014893428-Terms-of-service}{Terms
  of Service}
\item
  \href{https://help.nytimes3xbfgragh.onion/hc/en-us/articles/115014893968-Terms-of-sale}{Terms
  of Sale}
\item
  \href{https://spiderbites.nytimes3xbfgragh.onion}{Site Map}
\item
  \href{https://help.nytimes3xbfgragh.onion/hc/en-us}{Help}
\item
  \href{https://www.nytimes3xbfgragh.onion/subscription?campaignId=37WXW}{Subscriptions}
\end{itemize}
