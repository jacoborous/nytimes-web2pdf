Sections

SEARCH

\protect\hyperlink{site-content}{Skip to
content}\protect\hyperlink{site-index}{Skip to site index}

\href{/section/us}{U.S.}\textbar{}100 Days of Protest: A Chasm Grows
Between Portland and the Rest of Oregon

\url{https://nyti.ms/3lR9EnK}

\begin{itemize}
\item
\item
\item
\item
\item
\end{itemize}

\hypertarget{race-and-america}{%
\subsubsection{\texorpdfstring{\href{https://www.nytimes3xbfgragh.onion/news-event/george-floyd-protests-minneapolis-new-york-los-angeles?name=styln-george-floyd\&region=TOP_BANNER\&block=storyline_menu_recirc\&action=click\&pgtype=Article\&impression_id=f546fb10-f52b-11ea-9829-2300db455254\&variant=undefined}{Race
and America}}{Race and America}}\label{race-and-america}}

\begin{itemize}
\tightlist
\item
  \href{https://www.nytimes3xbfgragh.onion/2020/09/11/us/black-police-chiefs-reform.html?name=styln-george-floyd\&region=TOP_BANNER\&block=storyline_menu_recirc\&action=click\&pgtype=Article\&impression_id=f546fb11-f52b-11ea-9829-2300db455254\&variant=undefined}{Black
  Police Chiefs}
\item
  \href{https://www.nytimes3xbfgragh.onion/2020/09/04/nyregion/rochester-police-daniel-prude.html?name=styln-george-floyd\&region=TOP_BANNER\&block=storyline_menu_recirc\&action=click\&pgtype=Article\&impression_id=f5472220-f52b-11ea-9829-2300db455254\&variant=undefined}{What
  Happened in Rochester, N.Y.}
\item
  \href{https://www.nytimes3xbfgragh.onion/2020/08/30/us/portland-shooting-explained.html?name=styln-george-floyd\&region=TOP_BANNER\&block=storyline_menu_recirc\&action=click\&pgtype=Article\&impression_id=f5472221-f52b-11ea-9829-2300db455254\&variant=undefined}{Portland
  Shooting}
\item
  \href{https://www.nytimes3xbfgragh.onion/2020/08/30/us/breonna-taylor-police-killing.html?name=styln-george-floyd\&region=TOP_BANNER\&block=storyline_menu_recirc\&action=click\&pgtype=Article\&impression_id=f5472222-f52b-11ea-9829-2300db455254\&variant=undefined}{Breonna
  Taylor's Life and Death}
\end{itemize}

\includegraphics{https://static01.graylady3jvrrxbe.onion/images/2020/09/04/us/00portland-geography01/merlin_176578275_bc1632c4-2c37-4630-9530-32f602979627-articleLarge.jpg?quality=75\&auto=webp\&disable=upscale}

\hypertarget{100-days-of-protest-a-chasm-grows-between-portland-and-the-rest-of-oregon}{%
\section{100 Days of Protest: A Chasm Grows Between Portland and the
Rest of
Oregon}\label{100-days-of-protest-a-chasm-grows-between-portland-and-the-rest-of-oregon}}

The proximity of the left and the right in Oregon has created a dynamic
of fear, mistrust and anger.

Farmland in the town of Gresham, Ore., 15 miles from downtown Portland,
Ore.Credit...Mason Trinca for The New York Times

Supported by

\protect\hyperlink{after-sponsor}{Continue reading the main story}

\href{https://www.nytimes3xbfgragh.onion/by/thomas-fuller}{\includegraphics{https://static01.graylady3jvrrxbe.onion/images/2018/06/12/multimedia/author-thomas-fuller/author-thomas-fuller-thumbLarge.png}}

By \href{https://www.nytimes3xbfgragh.onion/by/thomas-fuller}{Thomas
Fuller}

\begin{itemize}
\item
  Published Sept. 5, 2020Updated Sept. 11, 2020
\item
  \begin{itemize}
  \item
  \item
  \item
  \item
  \item
  \end{itemize}
\end{itemize}

SANDY, Ore. --- Trucks carrying bales of hay, horse paddocks and
Christmas tree farms --- drive a few miles out of Portland and the
suburbs quickly give way to rural Oregon.

Barely a half-hour from the Portland streets where racial justice
protesters on Saturday were marking 100 consecutive days of tempestuous,
sometimes violent, demonstrations, there are plenty of communities where
people dismiss the protesters as lawless hooligans.

``Portland is an island in Oregon,'' said Stan Pulliam, the mayor of
Sandy, a more conservative town of 10,000 people about 30 miles
southeast of Portland that feeds off the economic dynamism of Oregon's
largest city but also strives to be separate from it. ``We are scared to
death that what's happening in Portland will ever come out to where we
live.''

The rural-urban divide is a reality writ large across much of the
nation, a crucial dynamic as the Nov. 3 election approaches. But the
proximity of left and right in Oregon, both moderates and extremists,
has created a dynamic of fear, mistrust and anger that feeds the
conflicts in the streets in ways that it has not in other states.

At Rapid Fire Arms, a gun shop along the main road in Sandy, the owner,
Brian Coleman, has sold 4.5 million rounds of ammunition since March,
when the arrival of the pandemic drove up sales. Demand for guns and
ammunition soared even further, he said, when the protests in Portland
turned violent in the weeks after George Floyd died in police custody in
Minneapolis.

``There's panic buying every once in a while but nowhere near like
this,'' Mr. Coleman said at the entrance of his shop, fortified with
steel bars. ``There's such a massive rush, people are taking anything
they can get.''

Mr. Coleman, who has sold thousands of guns this year, estimates that 70
percent of customers in recent months are first-time gun buyers.

In the town of Gresham, 15 miles from the urban canyons of downtown
Portland, Bonnie Johnson, a member of a Republican precinct committee,
is on a waiting list for her first firearm, a Smith \& Wesson revolver.

``I didn't even want a gun,'' said Ms. Johnson who grew up in the
neighboring town of Boring. ``But when you see all that's going on in
Portland, it scares you.''

Ms. Johnson took part in a flag-waving demonstration on Wednesday
evening, joining a group of 50 or so people, many of them wearing hats
and T-shirts in support of President Trump. They gathered at the Gresham
civic center to show their patriotism and mourn the death of Aaron J.
Danielson, a supporter of the far-right group Patriot Prayer who was
\href{https://www.nytimes3xbfgragh.onion/2020/08/30/us/portland-trump-rally-shooting.html}{shot
on Saturday} amid clashes between protesters from the right and left.

As a line of people beside Ms. Johnson waved American flags on a
sidewalk, passing motorists honked in support, or in some cases raised a
middle finger and shouted insults.

The ideological divide between Portland and its environs can be stark.
Conservative groups outside Portland have held demonstrations in support
of the police. Protesters in Portland have called for police forces to
be abolished altogether.

\includegraphics{https://static01.graylady3jvrrxbe.onion/images/2020/09/06/us/06portland-geography-print2/merlin_176578215_6ab8e2de-2d43-4cf4-baae-9c87e9058068-articleLarge.jpg?quality=75\&auto=webp\&disable=upscale}

Image

A flag-waving demonstration in Gresham this week.Credit...Mason Trinca
for The New York Times

Mr. Pulliam, the Sandy mayor, whose post is nonpartisan but who is
registered as a Republican, says he is dismayed that the clash between
left and right, while highly emotionally charged, is vague in its
prescriptions.

``Neither movement has asked our leaders for any kind of concrete
action,'' he said.

The Portland protests began in reaction to the killing of Mr. Floyd in
May but came to represent a more general campaign for racial justice and
opposition to the
\href{https://www.nytimes3xbfgragh.onion/2020/07/25/us/portland-federal-legal-jurisdiction-courts.html}{presence
of federal agents} in the city. Conservatives in the Portland area say
the authorities have allowed protesters to hijack the downtown. When
they visit, they say, they feel unwelcome and have been harassed.

Rebecca Crymer moved to the Portland area two years ago and although she
leans conservative she says she was never particularly interested in
politics.

In late August she was walking through the protests in Portland wearing
a Captain America T-shirt. She said she was called a Nazi and followed
by a man who threatened to throw dog feces at her. On Wednesday, Ms.
Crymer, who grew up in a military family, joined the flag-waving
demonstration in Gresham.

``I'm a normal person and I don't have extremist views,'' Ms. Crymer
said. ``Normal people should be able to feel like they can fly an
American flag and not get hunted down for it.''

Conservatives who have lived in the Portland area for decades say they
increasingly feel like strangers in their own state when they visit the
city.

Similar to the Far North of California, a conservative area where
residents
\href{https://www.nytimes3xbfgragh.onion/2017/07/02/us/california-far-north-identity-conservative.html}{feel
vastly outnumbered} in that state's legislature, communities outside
Portland often complain that laws and regulations are drafted to suit
the city and then imposed on the rest of the state.

Portland and its surrounding areas make up around 60 percent of the
state's population of four million people.

Image

A home in Gresham, Ore., was decorated with Trump signs. The rural-urban
divide is a reality writ large across much of the nation, a crucial
dynamic as the Nov. 3 election approaches.Credit...Mason Trinca for The
New York Times

``Conservatives feel angry and they feel treated unjustly and I guess
the word is oppressed,'' said Mr. James Buchal, chairman of the
Multnomah County Republican Party, the county that includes Portland. He
is also a lawyer representing Joey Gibson, leader of the far-right
protest group Patriot Prayer, who is being sued by the owner of a
Portland bar. The bar owner has accused Mr. Gibson of instigating a
brawl last year and disrupting business.

David Peterson del Mar, a professor at Portland State University and an
author of a book on Oregon's history, says rural-urban tensions have
existed for decades but they have become heightened as Portland expanded
in economic power and population --- and became more liberal.
Portlanders, he said, are often insensitive to a sense of hopelessness
in the poorer, rural parts of the state.

``There's tremendous distrust,'' he said. ``I think that's something
that white liberals don't get. Whether we realize it or not we exude a
lot of arrogance.''

As the cost of living in Portland has soared in recent years city
residents have moved to the suburbs, helping transform politically
conservative areas into shades of purple.

In Clackamas County, southeast of Portland and which includes Sandy,
Hillary Clinton won 50 percent of the vote in 2016, defeating Donald J.
Trump by seven points.

Ms. Clinton carried Oregon because of her strength in Portland. But the
state's electoral map was a sea of red with blue blotches in Portland
and the Willamette Valley.

Those conservatives near Portland are often in the awkward position of
mistrusting the city but relying on it for their livelihoods. In Sandy,
where Mr. Trump won 54 percent of the vote in 2016, around two-thirds of
residents commute to the city for work, according to the mayor.

``The outer areas hate Portland,'' said Mr. Coleman, the gun shop owner
in Sandy. And sometimes, he added, ``Portland hates Portland.''

Image

``Portland is an island in Oregon,'' said Stan Pulliam, the mayor of
Sandy, Ore.Credit...Mason Trinca for The New York Times

Image

People swam in the Sandy River in Sandy, a conservative town of 10,000
people about 30 miles southeast of Portland.Credit...Mason Trinca for
The New York Times

With political passions running so high, even liberals in Portland can
have a complicated relationship with their city. Disputes within and
among the left can be as spirited as between the left and the right.
Protesters have alternated their ire between Mr. Trump and their mayor,
Ted Wheeler, a moderate Democrat.

Mr. Peterson del Mar pokes fun at the ``tyranny of small differences''
in Portland.

``Sometimes you'll hear these debates, `Well, it's great that you have a
Black Lives Matter sign but you have one that is corporate-made and it
should be hand-lettered,''' he said.

Portlanders are proud of their racial justice campaign, which
\href{https://www.nytimes3xbfgragh.onion/2020/07/24/us/portland-oregon-protests-white-race.html}{enjoys
widespread support} and at various times has drawn crowds of many
thousands. But after 100 days, even among those who say they strongly
support the cause, there is an element of fatigue.

Many Portlanders are also bemused when friends from out of state or
abroad called to ask if they were OK. Or when they hear Mr. Trump refer
to Portland as a ``city that's falling apart.''

On most nights the protests have been very circumscribed, limited to a
few square blocks. For those who live in the tree-lined residential
neighborhoods or in the hills with sweeping views onto the snowy majesty
of Mount Hood, the protests can seem like a distant distraction.

``The protests have very little to do with our daily lives,'' said
Joseph Anthony, a tax preparer in the city. ``They are just the right
size for a television screen.''

The nightly protests come at a difficult time for many American cities.
Some downtown streets in Portland that would be bustling with tourists
and office workers are empty because of the coronavirus. Many
restaurants are closed --- some of them possibly for good. And
homelessness persists, with some blocks hosting more people without
homes than anyone else on the sidewalks.

How is Portland doing? These days it's a question loaded with
ideological baggage, and for some of those outside the city, the answer
is clear.

``Portland is a garbage dump,'' said Ms. Johnson, the resident of
Gresham, a half-hour away. ``I don't want anything to do with it.''

Advertisement

\protect\hyperlink{after-bottom}{Continue reading the main story}

\hypertarget{site-index}{%
\subsection{Site Index}\label{site-index}}

\hypertarget{site-information-navigation}{%
\subsection{Site Information
Navigation}\label{site-information-navigation}}

\begin{itemize}
\tightlist
\item
  \href{https://help.nytimes3xbfgragh.onion/hc/en-us/articles/115014792127-Copyright-notice}{©~2020~The
  New York Times Company}
\end{itemize}

\begin{itemize}
\tightlist
\item
  \href{https://www.nytco.com/}{NYTCo}
\item
  \href{https://help.nytimes3xbfgragh.onion/hc/en-us/articles/115015385887-Contact-Us}{Contact
  Us}
\item
  \href{https://www.nytco.com/careers/}{Work with us}
\item
  \href{https://nytmediakit.com/}{Advertise}
\item
  \href{http://www.tbrandstudio.com/}{T Brand Studio}
\item
  \href{https://www.nytimes3xbfgragh.onion/privacy/cookie-policy\#how-do-i-manage-trackers}{Your
  Ad Choices}
\item
  \href{https://www.nytimes3xbfgragh.onion/privacy}{Privacy}
\item
  \href{https://help.nytimes3xbfgragh.onion/hc/en-us/articles/115014893428-Terms-of-service}{Terms
  of Service}
\item
  \href{https://help.nytimes3xbfgragh.onion/hc/en-us/articles/115014893968-Terms-of-sale}{Terms
  of Sale}
\item
  \href{https://spiderbites.nytimes3xbfgragh.onion}{Site Map}
\item
  \href{https://help.nytimes3xbfgragh.onion/hc/en-us}{Help}
\item
  \href{https://www.nytimes3xbfgragh.onion/subscription?campaignId=37WXW}{Subscriptions}
\end{itemize}
