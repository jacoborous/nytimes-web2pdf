Sections

SEARCH

\protect\hyperlink{site-content}{Skip to
content}\protect\hyperlink{site-index}{Skip to site index}

\href{https://www.nytimes3xbfgragh.onion/spotlight/at-home}{At Home}

\href{https://myaccount.nytimes3xbfgragh.onion/auth/login?response_type=cookie\&client_id=vi}{}

\href{https://www.nytimes3xbfgragh.onion/section/todayspaper}{Today's
Paper}

\href{/spotlight/at-home}{At Home}\textbar{}7 Sculpture Gardens that
Merge Art With the Landscape

\url{https://nyti.ms/3i8686c}

\begin{itemize}
\item
\item
\item
\item
\item
\end{itemize}

\href{https://www.nytimes3xbfgragh.onion/spotlight/at-home?action=click\&pgtype=Article\&state=default\&region=TOP_BANNER\&context=at_home_menu}{At
Home}

\begin{itemize}
\tightlist
\item
  \href{https://www.nytimes3xbfgragh.onion/2020/09/07/travel/route-66.html?action=click\&pgtype=Article\&state=default\&region=TOP_BANNER\&context=at_home_menu}{Cruise
  Along: Route 66}
\item
  \href{https://www.nytimes3xbfgragh.onion/2020/09/04/dining/sheet-pan-chicken.html?action=click\&pgtype=Article\&state=default\&region=TOP_BANNER\&context=at_home_menu}{Roast:
  Chicken With Plums}
\item
  \href{https://www.nytimes3xbfgragh.onion/2020/09/04/arts/television/dark-shadows-stream.html?action=click\&pgtype=Article\&state=default\&region=TOP_BANNER\&context=at_home_menu}{Watch:
  Dark Shadows}
\item
  \href{https://www.nytimes3xbfgragh.onion/interactive/2020/at-home/even-more-reporters-editors-diaries-lists-recommendations.html?action=click\&pgtype=Article\&state=default\&region=TOP_BANNER\&context=at_home_menu}{Explore:
  Reporters' Google Docs}
\end{itemize}

Advertisement

\protect\hyperlink{after-top}{Continue reading the main story}

Supported by

\protect\hyperlink{after-sponsor}{Continue reading the main story}

\hypertarget{7-sculpture-gardens-that-merge-art-with-the-landscape}{%
\section{7 Sculpture Gardens that Merge Art With the
Landscape}\label{7-sculpture-gardens-that-merge-art-with-the-landscape}}

This fall, getting culture outdoors is even more restorative than usual.
Here are some options across the country.

\includegraphics{https://static01.graylady3jvrrxbe.onion/images/2020/09/06/multimedia/06ah-sculpturegarden1/merlin_176442309_a90cdecc-c66a-4f65-9d70-b7a939152c2b-articleLarge.jpg?quality=75\&auto=webp\&disable=upscale}

By
\href{https://www.nytimes3xbfgragh.onion/by/thessaly-la-force}{Thessaly
La Force}

\begin{itemize}
\item
  Published Sept. 5, 2020Updated Sept. 8, 2020
\item
  \begin{itemize}
  \item
  \item
  \item
  \item
  \item
  \end{itemize}
\end{itemize}

Many museums and galleries across the country have cautiously begun to
reopen in recent weeks, offering a chance for the culture-starved to
enjoy a moment of reprieve with their favorite works of art. Still, the
lines can be long, and timed ticketing limits a more impulsive visit.

These seven sculpture gardens or outdoor art spaces --- ranging from
world-class art collections to more hidden and eccentric destinations
--- are especially appealing beginning this month, when the weather is
ideal for strolling outside and the fall programming and curatorial
programs (some of them delayed from closings this summer) begin in
earnest.

\hypertarget{grounds-for-sculpture}{%
\subsection{\texorpdfstring{\href{https://www.groundsforsculpture.org/}{Grounds
for Sculpture}}{Grounds for Sculpture}}\label{grounds-for-sculpture}}

Hamilton, N.J.

This 42-acre park and museum was founded in 1992 by the American artist
Seward Johnson, with the hope of promoting a better understanding of
contemporary sculpture. Close to 300 works by artists such as Beverly
Pepper, Kiki Smith, Anthony Caro, Magdalena Abakanowicz and Autin Wright
populate the grounds, where natural woodlands, ponds and bamboo groves
are set alongside paved terraces, pergolas and courtyards where the
occasional peacock may make an appearance. Included, of course, are
several of Johnson's own pop-art-inflected, larger-than-life figures.
Families with children under 12 can purchase an ArtBox --- a beginner's
sculpture kit --- in advance of their visit. Don't miss the recently
installed show ``Rebirth,'' composed of six works made from steel
elevator cables by the Taiwanese sculptor Kang Muxiang.

\hypertarget{storm-king-art-center}{%
\subsection{\texorpdfstring{\href{https://stormking.org/}{Storm King Art
Center}}{Storm King Art Center}}\label{storm-king-art-center}}

Cornwall, N.Y.

Named after Storm King Mountain located along the Hudson River and built
on what now encompasses 500 acres, this open-air museum is home to some
of the best contemporary and mid-20th-century sculpture. Works from
artists such as Alexander Calder, Richard Serra and Louise Bourgeois
have been carefully installed in relation to the landscape, where at
every turn you encounter stunning vistas, especially as leaves begin to
change. A visitor favorite is Maya Lin's 2009 ``Storm King Wavefield''
--- seven undulating rows sculpted into the land itself. Part of the
appeal of Storm King is its excellently curated exhibition program. This
fall there are two new outdoor works inspired by the local landscape: an
installation of Kiki Smith's large-scale flag textiles and Martha
Tuttle's cairns. Visitors must book ahead online. Timed-entry tickets
are released in two-week blocks every Wednesday; \$20 each for the first
two people in a car. The museum is offering free admission this year to
frontline medical workers, active military and their families and
others.

\includegraphics{https://static01.graylady3jvrrxbe.onion/images/2020/09/06/multimedia/06ah-sculpturegarden2/merlin_176527371_902ca061-939d-4eff-a2e6-b580228422e8-articleLarge.jpg?quality=75\&auto=webp\&disable=upscale}

\hypertarget{the-minneapolis-sculpture-garden}{%
\subsection{\texorpdfstring{\href{https://walkerart.org/visit/garden}{The
Minneapolis Sculpture
Garden}}{The Minneapolis Sculpture Garden}}\label{the-minneapolis-sculpture-garden}}

Minneapolis

Built in 1988, and one of the country's premier sculpture gardens, the
Minneapolis Sculpture Garden is open to the public from 6 a.m. to
midnight, with more than 60 works of art across 19 campus acres --- all
free. Though most visitors are drawn to the garden's inaugural
centerpiece, ``Spoonbridge and Cherry,'' by Claes Oldenburg and Coosje
van Bruggen, there are many other important works of art from artists
like Sol Lewitt, Eva Rothschild and James Turrell. In 2017, the gardens
underwent an extensive renovation. Pieces from contemporary artists such
as Theaster Gates and Katherina Fritsch were added, and a former wetland
was restored and planted with native flora to help feed essential and
imperiled pollinators such as monarch butterflies and bees. The garden
is an ongoing collaboration between the city's parks department and the
Walker Art Center, which, through a ticketed and timed system, is also
now open to the public.

\hypertarget{the-clark-art-institute}{%
\subsection{\texorpdfstring{\href{https://www.clarkart.edu/}{The Clark
Art
Institute}}{The Clark Art Institute}}\label{the-clark-art-institute}}

Williamstown, Mass.

Three hours north of New York City in the bucolic Berkshires sits the
Clark, whose indoor collection of French Impressionism and 19th-century
academic painting is matched only by its expansive 140-acre campus and
recent outstanding architectural additions from Tadao Ando and Annabelle
Selldorf. The Clark is one of the three great museums in the area ---
the other two being MASS MoCA and the Williams College Museum.
``Ground/work,'' the Clark's first outdoor exhibition, opens on Oct. 5,
featuring newly commissioned work from the artists Kelly Akashi, Nairy
Baghramian, Eva LeWitt, Jennie C. Jones, Analia Saban and Haegue Yang.
Each was asked to create work in dialogue with the Clark's natural
terrain.

\hypertarget{crystal-bridges-museum-of-american-art}{%
\subsection{\texorpdfstring{\href{https://crystalbridges.org/}{Crystal
Bridges Museum of American
Art}}{Crystal Bridges Museum of American Art}}\label{crystal-bridges-museum-of-american-art}}

Bentonville, Ark.

Located in the northwestern corner of Arkansas and opened in 2011,
Crystal Bridges was developed largely by Alice Walton, daughter of Sam
Walton of the Walmart fortune. The museum, which was designed by the
Israeli architect Moshe Safdie, has amassed an impressive collection of
American art that includes works by Jasper Johns, Ruth Asawa, Georgia
O'Keeffe, Josef and Anni Albers and many others. Parts of it are now
open to the public (although the Frank Lloyd Wright House on its campus
remains closed), with timed tickets and walk-up entry. One hundred and
twenty acres of Ozark forest surround the museum, offering five miles of
walking and biking trails, with artwork throughout, including pieces by
Carol Bove, Deborah Butterfield and R. Buckminster Fuller. Every Friday,
the museum offers a ticketed, free and family-friendly concert,
performance or art activity.

Image

``Accurate Figure,'' by Tony Cragg, at the Nasher Sculpture
Center.Credit...Artists Rights Society (ARS), New York/VG Bild-Kunst,
Bonn; Carolyn Brown

\hypertarget{nasher-sculpture-center}{%
\subsection{\texorpdfstring{\href{https://www.nashersculpturecenter.org/}{Nasher
Sculpture
Center}}{Nasher Sculpture Center}}\label{nasher-sculpture-center}}

Dallas

This museum, built by the larger-than-life Dallas philanthropist and
real estate developer Raymond Nasher and his wife, Patsy, and located in
the downtown arts district, brings together a vast collection of
sculptural masterpieces, with works from the French sculptor Auguste
Rodin, as well as Pablo Picasso, Ellsworth Kelly, Barbara Hepworth and
many others. The museum, which was designed by Renzo Piano in 1997, is
now admitting visitors, with a timed ticket system. The 1.5-acre garden
designed by the landscape architect Peter Walker adds to the serenity of
the place. Works on view outdoors this month include Willem de Kooning's
``Seated Woman'' and Hepworth's ``Squares With Two Circles (Monolith)''
as well as pieces by Joan Miró and Mark di Suvero.

\hypertarget{california-scenario}{%
\subsection{\texorpdfstring{\href{https://www.southcoastplaza.com/stories/2016/12/noguchi-garden/}{California
Scenario}}{California Scenario}}\label{california-scenario}}

Costa Mesa, Calif.

The sculptor Isamu Noguchi is perhaps most closely associated with his
museum in Queens (reopening this month), which he created to preserve
the site-specific nature of his work, ranging from lamps to coffee
tables to monumental sculpture. In 1979, the Segerstrom family --- one
of Southern California's leading patrons of the arts --- commissioned
the prolific Japanese-American artist to create a sculpture garden for
the South Coast Plaza shopping mall, built on top of the lima bean ranch
from which the Segerstroms originally made their fortune. Partly a work
of abstract land-art and partly a riff on a Zen rock garden, California
Scenario --- open and free to the public --- is a geometric tribute to
the area's landscape.

Advertisement

\protect\hyperlink{after-bottom}{Continue reading the main story}

\hypertarget{site-index}{%
\subsection{Site Index}\label{site-index}}

\hypertarget{site-information-navigation}{%
\subsection{Site Information
Navigation}\label{site-information-navigation}}

\begin{itemize}
\tightlist
\item
  \href{https://help.nytimes3xbfgragh.onion/hc/en-us/articles/115014792127-Copyright-notice}{©~2020~The
  New York Times Company}
\end{itemize}

\begin{itemize}
\tightlist
\item
  \href{https://www.nytco.com/}{NYTCo}
\item
  \href{https://help.nytimes3xbfgragh.onion/hc/en-us/articles/115015385887-Contact-Us}{Contact
  Us}
\item
  \href{https://www.nytco.com/careers/}{Work with us}
\item
  \href{https://nytmediakit.com/}{Advertise}
\item
  \href{http://www.tbrandstudio.com/}{T Brand Studio}
\item
  \href{https://www.nytimes3xbfgragh.onion/privacy/cookie-policy\#how-do-i-manage-trackers}{Your
  Ad Choices}
\item
  \href{https://www.nytimes3xbfgragh.onion/privacy}{Privacy}
\item
  \href{https://help.nytimes3xbfgragh.onion/hc/en-us/articles/115014893428-Terms-of-service}{Terms
  of Service}
\item
  \href{https://help.nytimes3xbfgragh.onion/hc/en-us/articles/115014893968-Terms-of-sale}{Terms
  of Sale}
\item
  \href{https://spiderbites.nytimes3xbfgragh.onion}{Site Map}
\item
  \href{https://help.nytimes3xbfgragh.onion/hc/en-us}{Help}
\item
  \href{https://www.nytimes3xbfgragh.onion/subscription?campaignId=37WXW}{Subscriptions}
\end{itemize}
