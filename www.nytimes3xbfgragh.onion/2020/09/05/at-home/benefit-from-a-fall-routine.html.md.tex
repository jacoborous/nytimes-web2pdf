Sections

SEARCH

\protect\hyperlink{site-content}{Skip to
content}\protect\hyperlink{site-index}{Skip to site index}

\href{https://www.nytimes3xbfgragh.onion/spotlight/at-home}{At Home}

\href{https://myaccount.nytimes3xbfgragh.onion/auth/login?response_type=cookie\&client_id=vi}{}

\href{https://www.nytimes3xbfgragh.onion/section/todayspaper}{Today's
Paper}

\href{/spotlight/at-home}{At Home}\textbar{}Benefit From a Fall Routine

\url{https://nyti.ms/331uMPG}

\begin{itemize}
\item
\item
\item
\item
\item
\end{itemize}

\href{https://www.nytimes3xbfgragh.onion/spotlight/at-home?action=click\&pgtype=Article\&state=default\&region=TOP_BANNER\&context=at_home_menu}{At
Home}

\begin{itemize}
\tightlist
\item
  \href{https://www.nytimes3xbfgragh.onion/2020/09/07/travel/route-66.html?action=click\&pgtype=Article\&state=default\&region=TOP_BANNER\&context=at_home_menu}{Cruise
  Along: Route 66}
\item
  \href{https://www.nytimes3xbfgragh.onion/2020/09/04/dining/sheet-pan-chicken.html?action=click\&pgtype=Article\&state=default\&region=TOP_BANNER\&context=at_home_menu}{Roast:
  Chicken With Plums}
\item
  \href{https://www.nytimes3xbfgragh.onion/2020/09/04/arts/television/dark-shadows-stream.html?action=click\&pgtype=Article\&state=default\&region=TOP_BANNER\&context=at_home_menu}{Watch:
  Dark Shadows}
\item
  \href{https://www.nytimes3xbfgragh.onion/interactive/2020/at-home/even-more-reporters-editors-diaries-lists-recommendations.html?action=click\&pgtype=Article\&state=default\&region=TOP_BANNER\&context=at_home_menu}{Explore:
  Reporters' Google Docs}
\end{itemize}

Advertisement

\protect\hyperlink{after-top}{Continue reading the main story}

Supported by

\protect\hyperlink{after-sponsor}{Continue reading the main story}

\hypertarget{benefit-from-a-fall-routine}{%
\section{Benefit From a Fall
Routine}\label{benefit-from-a-fall-routine}}

Your days don't need to look the same every single day, but recognizing
and honoring a schedule brings plenty of mental and physical benefits.

\includegraphics{https://static01.graylady3jvrrxbe.onion/images/2020/09/06/multimedia/06AH-fallroutines/06AH-fallroutines-articleLarge.jpg?quality=75\&auto=webp\&disable=upscale}

By
\href{https://www.nytimes3xbfgragh.onion/by/alexandra-e--petri}{Alexandra
E. Petri}

\begin{itemize}
\item
  Sept. 5, 2020
\item
  \begin{itemize}
  \item
  \item
  \item
  \item
  \item
  \end{itemize}
\end{itemize}

As the pandemic continues and the school year starts up, some people
might be craving some regularity and structure. Having a routine not
only guides you through your days, but it also brings mental and
physical benefits, too, whether by adding exercise to your day, aiding
in getting better sleep, helping children feel more secure or providing
a sense of control during such an uncertain time.

Here are some tips on how to establish, re-establish and maintain a
routine.

\hypertarget{start-small}{%
\subsection{Start small.}\label{start-small}}

You've done it before: Created routines that look as if you're competing
for a spot in the Productivity Olympics. Instead, whether as an
individual or as a family, focus first on incremental, achievable goals.

``Set yourself up for little wins,'' said Dilan Gomih, a
\href{https://www.dilangomih.com/}{fitness instructor and life coach}
based in New York City. Rather than telling yourself you're going to
start waking up every day for 6 a.m. workouts, think about starting with
a few workouts a week and gradually increasing them. For families, some
experts suggest starting breakfast routines before kids begin their
school days, whether that's kicking off the day over bowls of cereal, or
using a chore wheel to keep track of daily tasks.

It's also important to understand why you're craving routine, said
Chanel Dokun, an Atlanta-based life planner and
\href{http://www.chaneldokun.com}{founder of LifePlan NYC}. ``Ask
yourself, `What do I want to experience or feel on a day-to-day
basis?''' she said. Most people tend to focus on tasks, but Ms. Dokun
suggests focusing instead on the things that fuel or replenish you.

In response to an email to At Home readers asking about their schedules,
a reader named Julia Zhou wrote that her routines ``distinguish each day
and foster a sense of progress.''

``There are a lot of little constants in my days that I hadn't paid much
attention to before,'' she wrote. ``I am realizing now that they add
comfort and richness to each day.''

\hypertarget{get-organized}{%
\subsection{Get organized.}\label{get-organized}}

If you're the type who prefers pen and paper, consider a small notebook
or diary to carry around with you, detailing daily goals you'd like to
accomplish. For those who prefer going digital, Google calendars, Asana
boards or self-care apps can help you visualize the routine you want to
develop and stay organized as you get into the swing of it.

How you use these tools varies on what works best for you. For some,
that might mean scheduling things to fit in certain hours: getting in a
7 a.m. workout, taking a break at 12:30 for lunch and a walk around the
block; for others, it might mean creating a list of things you want to
accomplish each week.

You can also turn to those around you to hold you accountable.

``Everyday my grandma will call my little sister and me to do back
exercises in the living room,'' Ms. Zhou wrote. ``Sometimes we scoff
that we have to do them at all, but this little routine does anchor each
day.''

With kids, a great way to create accountability is to do what some
teachers do: Fill up a jar with small rocks or coins, or add stickers to
a chart for each accomplished task. If they have completed daily tasks
or chores, then perhaps your children can be rewarded by picking the
movie for movie night, or choosing the dessert.

``There's a reason we see teachers in kindergarten through third grade
using this,'' said Corinn Cross, a pediatrician and spokeswoman for the
American Academy of Pediatrics. ``It's motivating.''

\hypertarget{work-with-whats-there}{%
\subsection{Work with what's there.}\label{work-with-whats-there}}

Sleep is what frames our days. ``Our day starts when we wake up, and
ends when we fall asleep at night,'' said Jennifer Martin, a professor
of medicine at the David Geffen School of Medicine at the University of
California, Los Angeles. She suggests figuring out a wake-up time that
works for you, and sticking to it six or seven days a week. ``This might
mean getting up a little earlier than you want on the weekends, but it
will set you up for good sleep the next night,'' she said. Keep a
consistent bedtime, too.

Meals can also provide a framework. ``Understanding your own individual
eating schedule can optimize the way we structure our day by letting us
know what times would be best to do other activities --- like
exercising, sleeping, or work,'' said May Zhu, founder of
\href{http://www.nutritionhappens.com/}{Nutrition Happens}, a
nutritional counseling service.

Cooking is also a great family activity, Ms. Zhu added. ``By blocking
out a set time to cook before a meal, you have the opportunity to get
the kids involved, and give them a sense of control, to learn more about
food and making healthy choices,'' she said. Experts also suggest using
dinner as a way to structure screen time for kids, like agreeing to one
show before or after dinner.

\hypertarget{ask-yourself-is-this-working}{%
\subsection{Ask yourself, `Is this
working?'}\label{ask-yourself-is-this-working}}

While you might be great at creating a schedule for yourself, you need
to consider whether it is actually working.

``Check in and reflect on the routine that you've established,'' Ms.
Dokun said. ``That's the one piece that people often miss.'' It may take
as long as three weeks to really develop a rhythm. If you still haven't
hit your stride by then, it's time to make adjustments. ``This kind of
`failure' is helpful in illuminating your true beliefs about your rhythm
and revealing blind spots that keep you from being successful,'' she
said.

And remember, while this is a stressful time for parents, kids are also
dealing with pandemic-induced anxiety. ``A lot of times we think kids
will bounce back, so whether it's over dinner or putting kids to bed,
remember to schedule in that 10 minutes in your head that's device-free
to check in, especially as school is getting started,'' Dr. Cross said.

\hypertarget{dont-forget-the-importance-of-flexibility-and-kindness}{%
\subsection{Don't forget the importance of flexibility and
kindness.}\label{dont-forget-the-importance-of-flexibility-and-kindness}}

All the experts agree that among the most important aspects of creating
and establishing a routine is remaining flexible and being kind to
yourself. ``Routines don't mean you have to have every hour of every
single day planned and then, if something goes wrong, it's like Murphy's
law and the dominoes are all going to topple,'' Ms. Gomih said.
``Routines are just guideposts.''

R. Lynn Barnett, another Times reader, wrote: ``Sometimes the change in
routine works out well. When I couldn't walk at night, I walked in the
morning, and I was able to see a neighbor whom I hadn't seen in a while.
We caught up, talking from driveway to driveway.''

Advertisement

\protect\hyperlink{after-bottom}{Continue reading the main story}

\hypertarget{site-index}{%
\subsection{Site Index}\label{site-index}}

\hypertarget{site-information-navigation}{%
\subsection{Site Information
Navigation}\label{site-information-navigation}}

\begin{itemize}
\tightlist
\item
  \href{https://help.nytimes3xbfgragh.onion/hc/en-us/articles/115014792127-Copyright-notice}{©~2020~The
  New York Times Company}
\end{itemize}

\begin{itemize}
\tightlist
\item
  \href{https://www.nytco.com/}{NYTCo}
\item
  \href{https://help.nytimes3xbfgragh.onion/hc/en-us/articles/115015385887-Contact-Us}{Contact
  Us}
\item
  \href{https://www.nytco.com/careers/}{Work with us}
\item
  \href{https://nytmediakit.com/}{Advertise}
\item
  \href{http://www.tbrandstudio.com/}{T Brand Studio}
\item
  \href{https://www.nytimes3xbfgragh.onion/privacy/cookie-policy\#how-do-i-manage-trackers}{Your
  Ad Choices}
\item
  \href{https://www.nytimes3xbfgragh.onion/privacy}{Privacy}
\item
  \href{https://help.nytimes3xbfgragh.onion/hc/en-us/articles/115014893428-Terms-of-service}{Terms
  of Service}
\item
  \href{https://help.nytimes3xbfgragh.onion/hc/en-us/articles/115014893968-Terms-of-sale}{Terms
  of Sale}
\item
  \href{https://spiderbites.nytimes3xbfgragh.onion}{Site Map}
\item
  \href{https://help.nytimes3xbfgragh.onion/hc/en-us}{Help}
\item
  \href{https://www.nytimes3xbfgragh.onion/subscription?campaignId=37WXW}{Subscriptions}
\end{itemize}
