Sections

SEARCH

\protect\hyperlink{site-content}{Skip to
content}\protect\hyperlink{site-index}{Skip to site index}

\href{https://www.nytimes3xbfgragh.onion/section/books/review}{Book
Review}

\href{https://myaccount.nytimes3xbfgragh.onion/auth/login?response_type=cookie\&client_id=vi}{}

\href{https://www.nytimes3xbfgragh.onion/section/todayspaper}{Today's
Paper}

\href{/section/books/review}{Book Review}\textbar{}Two Books Wonder: How
Long Until You Fall in Love With a Robot?

\url{https://nyti.ms/33g4zgp}

\begin{itemize}
\item
\item
\item
\item
\item
\item
\end{itemize}

Advertisement

\protect\hyperlink{after-top}{Continue reading the main story}

Supported by

\protect\hyperlink{after-sponsor}{Continue reading the main story}

Nonfiction

\hypertarget{two-books-wonder-how-long-until-you-fall-in-love-with-a-robot}{%
\section{Two Books Wonder: How Long Until You Fall in Love With a
Robot?}\label{two-books-wonder-how-long-until-you-fall-in-love-with-a-robot}}

\includegraphics{https://static01.graylady3jvrrxbe.onion/images/2020/09/10/books/review/10Hess/merlin_145348320_c91fd8fb-7c7a-483e-92ce-58e5e6fcbf01-articleLarge.jpg?quality=75\&auto=webp\&disable=upscale}

By \href{https://www.nytimes3xbfgragh.onion/by/amanda-hess}{Amanda Hess}

\begin{itemize}
\item
  Sept. 11, 2020
\item
  \begin{itemize}
  \item
  \item
  \item
  \item
  \item
  \item
  \end{itemize}
\end{itemize}

\textbf{WORK MATE MARRY LOVE}\\
\textbf{How Machines Shape Our Human Destiny}\\
By Debora L. Spar

\textbf{SEX ROBOTS AND VEGAN MEAT}\\
\textbf{Adventures at the Frontier of Birth, Food, Sex, and Death}\\
By Jenny Kleeman

``Science fiction is not about the future,'' the sci-fi novelist Samuel
R. Delany wrote in 1984. The future ``is only a writerly convention,''
he continued, one that ``sets up a rich and complex dialogue with the
reader's here and now.'' That is a useful way of understanding all the
many pop nonfiction books that speculate about the technologies of the
future, and attempt to divine their effects on human beings. Their
predictions depend on how well they interpret the present.

One such interpreter is Debora L. Spar, the dean of Harvard Business
School Online, who writes at the intersection of tech and gender. In her
new book, ``Work Mate Marry Love,'' she considers an emerging wave of
innovations that she believes could upend how we experience
relationships, reproduction, gender expression and death. ``We will fall
in love with nonhuman beings,'' Spar predicts in the book's opening
pages, ``and find ways to extend our human lives into something that
begins to approximate forever.'' Spar argues that new technologies spark
shifts in the most intimate of human affairs, often in unexpected ways.
She casts this as a causal relationship, one imbued with a sense of
inevitability. The book's subtitle, ``How Machines Shape Our Human
Destiny,'' gives the machines the agency.

Spar spends the first half of the book looking backward, tracing how
monogamy sprang from the plow, how the steam engine pried open a gender
divide at work and home, and how the dishwasher set the stage for the
second-wave feminist movement. When she is excavating this history, Spar
carves convincing paths through mountains of academic and historical
records. But as she veers into the present and pitches toward the
future, her visions become murkier and her trail of evidence grows
faint. She combs other popular nonfiction works, like Hanna Rosin's
``The End of Men,'' for bits of reporting; she performs a vicarious
exercise in online dating, creating an account on Match.com; she
transcribes encounters with a couple of transgender people she has met.

\includegraphics{https://static01.graylady3jvrrxbe.onion/images/2020/06/30/books/review/Hess1-2/Hess1-2-articleLarge.jpg?quality=75\&auto=webp\&disable=upscale}

I don't know if Spar is right about what the future holds, but her view
of the present reveals some limitations. She can treat decade-old
technologies as if they are vexing new developments. She describes
Grindr, a queer dating app launched in 2009, as ``a strange but powerful
confluence of gay men and mobile phones.'' Tinder, which followed in
2012, ``is a scary place,'' she writes, misapprehending its purpose as
``just about sex --- straightforward, fairly anonymous, often
pretty-near-instantaneous sex.'' Her section on developments in hormone
therapy and their impact on gender expression mostly emphasizes her own
experience straining to parse these changes, which she does not appear
to have yet mastered. She writes that the ``words are confusing ---
\emph{transgender? transsexual? transvestite?''} and refers to trans
people in tortured constructions like ``the shes-who-would-be-hes and
hes-who-would-be-shes.''

Though Spar is interested in the cultural side of technology, the strict
determinism of her argument --- tech begets culture --- can flatten the
complex interplay of these forces. Spar plugs so many innovations and
social shifts into her thesis, jumping from in vitro gametogenesis to
hormone therapy to sex robots, that her insights can feel programmatic,
and at times strangely dehumanizing. ```Trans' is also the perfect segue
--- the transition, literally --- between the changing worlds of
reproduction and robotics,'' she writes. ``Because if we can love across
gender and sex, if we can harness technology to build bodies that defy
reproductive logic, then we can build bodies and intimacies that cross
species as well.'' Drawing an analogy between trans people and machines
built for comfort and sex does not offer serious insight, but it fits
within a book that seems interested in human experience only insofar as
it relates to some tool.

In ``Sex Robots and Vegan Meat,'' the journalist Jenny Kleeman looks
toward the future from a very different vantage point. Like Spar,
Kleeman is interested in technologies of sex, reproduction, gender and
death; she also takes on food, investigating the cluster of start-ups
that are engineering artificial meat. Kleeman approaches the future as a
reporter firmly grounded in the present; her method is to journey to the
frontier and take a long look around.

The book leans heavily on conversations between the writer and various
figures with stakes in the game --- the people who are inventing
technologies, consuming them, promoting them and campaigning against
them. Among them are a sex robot maker who ``wants to be respected as an
artist,'' a doctor dubbed ``the Elon Musk of suicide'' who wants to
engineer the perfect death, and a Silicon Valley bro trying to grow a
chicken nugget in a lab.

Kleeman is keenly interested in the practical realities of these
inventions --- how they look and move and taste --- and she intuits that
claims about products representing ``the future'' often mean that the
products do not actually work, at least not yet. As she hops from
start-up to start-up, she fends off publicists and pulls back the
curtain on their bits of futuristic theater. A sex robot prototype
presented to Kleeman is grotesque, with ``a Medusa of wires bursting out
the back''; the nugget ``has the texture of the most low-grade processed
food I could ever imagine.'' In Kleeman's telling, futuristic
technologies are not accidents of history that drive unexpected social
changes. They are designed to fit the worldviews of specific kinds of
people --- men, mostly --- and are fueled by hubris, spin and private
equity.

Image

Kleeman's exploration of the frontier sometimes leads her into the
weeds. Her section on reproductive technology takes a long detour
through Men Going Their Own Way, a ghoulish anonymous message board for
straight men who have sworn off women and are also --- perhaps not
unrelated --- largely uninteresting. But more often, Kleeman's capacious
curiosity opens up a kaleidoscopic view of an issue.

Her final subject, death, brings her to a doctor who creates
technologies for assisted suicide, and his network of salespeople and
acolytes. At first glance, his project appears targeted at giving people
the choice to die with dignity, providing instructions for constructing
D.I.Y. suicide devices. But Kleeman's narrative culminates with him
garishly debuting a janky 3-D-printed luxury suicide pod called the
Sarco (short for sarcophagus --- after you kill yourself, you can be
buried in it). The product seems unlikely to ease the pain of the
terminally ill any time soon, but it fulfills the doctor's desire for
self-promotion instantaneously.

When Kleeman does venture to speculate --- that sex robot inventors are
essentially creating slaves that could compromise human empathy, or that
artificial wombs risk snatching reproductive control from women, eroding
abortion rights --- her insights feel earned. Perhaps that's because she
seems less invested in predicting the future than she is in questioning
the people who are so obsessed with shaping it. Kleeman recognizes that
technology has the power to shape human life, of course, but she is also
interested in interrogating that power, and understanding who exactly
gets to wield it. She sees in the future what Delany did: the stage for
a ``rich and complex dialogue'' with the ``here and now.''

Advertisement

\protect\hyperlink{after-bottom}{Continue reading the main story}

\hypertarget{site-index}{%
\subsection{Site Index}\label{site-index}}

\hypertarget{site-information-navigation}{%
\subsection{Site Information
Navigation}\label{site-information-navigation}}

\begin{itemize}
\tightlist
\item
  \href{https://help.nytimes3xbfgragh.onion/hc/en-us/articles/115014792127-Copyright-notice}{©~2020~The
  New York Times Company}
\end{itemize}

\begin{itemize}
\tightlist
\item
  \href{https://www.nytco.com/}{NYTCo}
\item
  \href{https://help.nytimes3xbfgragh.onion/hc/en-us/articles/115015385887-Contact-Us}{Contact
  Us}
\item
  \href{https://www.nytco.com/careers/}{Work with us}
\item
  \href{https://nytmediakit.com/}{Advertise}
\item
  \href{http://www.tbrandstudio.com/}{T Brand Studio}
\item
  \href{https://www.nytimes3xbfgragh.onion/privacy/cookie-policy\#how-do-i-manage-trackers}{Your
  Ad Choices}
\item
  \href{https://www.nytimes3xbfgragh.onion/privacy}{Privacy}
\item
  \href{https://help.nytimes3xbfgragh.onion/hc/en-us/articles/115014893428-Terms-of-service}{Terms
  of Service}
\item
  \href{https://help.nytimes3xbfgragh.onion/hc/en-us/articles/115014893968-Terms-of-sale}{Terms
  of Sale}
\item
  \href{https://spiderbites.nytimes3xbfgragh.onion}{Site Map}
\item
  \href{https://help.nytimes3xbfgragh.onion/hc/en-us}{Help}
\item
  \href{https://www.nytimes3xbfgragh.onion/subscription?campaignId=37WXW}{Subscriptions}
\end{itemize}
