Sections

SEARCH

\protect\hyperlink{site-content}{Skip to
content}\protect\hyperlink{site-index}{Skip to site index}

\href{https://www.nytimes3xbfgragh.onion/section/books/review}{Book
Review}

\href{https://myaccount.nytimes3xbfgragh.onion/auth/login?response_type=cookie\&client_id=vi}{}

\href{https://www.nytimes3xbfgragh.onion/section/todayspaper}{Today's
Paper}

\href{/section/books/review}{Book Review}\textbar{}Bodies of Evidence

\url{https://nyti.ms/2Rf0inI}

\begin{itemize}
\item
\item
\item
\item
\item
\end{itemize}

Advertisement

\protect\hyperlink{after-top}{Continue reading the main story}

Supported by

\protect\hyperlink{after-sponsor}{Continue reading the main story}

\href{/column/crime}{Crime}

\hypertarget{bodies-of-evidence}{%
\section{Bodies of Evidence}\label{bodies-of-evidence}}

\includegraphics{https://static01.graylady3jvrrxbe.onion/images/2020/09/20/books/review/20Crime/20Crime-articleLarge.jpg?quality=75\&auto=webp\&disable=upscale}

By Marilyn Stasio

\begin{itemize}
\item
  Sept. 11, 2020
\item
  \begin{itemize}
  \item
  \item
  \item
  \item
  \item
  \end{itemize}
\end{itemize}

What's a wobble? you ask. In Victorian England, it was a grueling,
six-day speedwalking race, a fad that also became popular in America.
This eccentric sport inspired Peter Lovesey's first murder mystery,
``\href{https://www.theguardian.com/science/blog/2012/jun/19/sports-doping-victorian-style}{Wobble
to Death},'' originally published in 1970.

To mark the 50th anniversary of the novel, Lovesey has written a
companion piece, \textbf{THE FINISHER (Soho Crime, 353 pp., \$27.95),}
about another footrace and another murder that becomes a baffling case
for his enduring series detective, Peter Diamond of the Bath police.
That classically designed English city might not seem like the ideal
setting for a marathon. But Lovesey knows his city intimately, and once
the 5,000 or so entrants are off and running --- along with ``the
pirates, pantomime horses, fairies, carrots, bananas, spacemen,
dinosaurs'' and other costumed entrants running for the fun of it --- he
treats us to lovingly detailed descriptions of the civic highlights. The
runners he follows are an interesting cross section of the citizenry,
from Maeve Kelly, a primary schoolteacher whose drab life gets its
sparkle after she takes up running, to an illegal immigrant named Spiro,
who escapes virtual slavery in his native Albania only to find himself
running for his life in Bath.

Lovesey is careful to remind us that Bath holds hidden secrets behind
its gracious Georgian architecture. ``It offers unrivaled facilities for
getting rid of unwanted corpses,'' he says. ``Beneath the creamy,
sun-kissed squares, crescents and terraces is a rat-infested underworld
undreamed of by most visitors.'' This is the secret world traveled by
``the Finisher,'' a murderer who leaves no traces because of his
intimate knowledge of the city's ``dark, dank warren of cellars, vaults,
culverts, sewers and drains.'' The light and dark imagery is a fixture
of Lovesey's Bath novels, in which life is lived on many levels, some in
full sunshine and others buried in shadow.

It's the middle of winter in Matt Goldman's \textbf{DEAD WEST (Forge,
320 pp., \$26.99),} and Minnesota temperatures are below zero. That's
reason enough for Goldman's engaging private eye, Nils Shapiro, to jump
on an assignment that will take him to warm and sunny Los Angeles. The
job seems like a no-brainer; all he has to do is reassure his client
that her grandson, Ebben Mayer, hasn't blown his \$50 million
inheritance on something stupid, like the movie business. But when he
gets to California, Shapiro finds Ebben mourning the suspicious death of
his fiancée and deeply invested in a film.

Always a clever writer, Goldman is flat-out hilarious when he's
satirizing Hollywood hustlers and their outlandish projects. In this
town, he notes, even a Russian gangster has a screenplay in his trunk.
(``Russian tragedy. Will make you cry.'') Individually, these agents,
managers, writers, producers, et al. are endearingly awful; but
observing them in groups --- pitching ideas, taking meetings and
poisoning their best friends --- is too funny for words.

Let's say you want to isolate a group of people from the outside world.
Where would you put them? A ship at sea would work.
\href{https://www.nytimes3xbfgragh.onion/2019/07/17/books/review/the-whisper-man-alex-north-turn-of-the-key-ruth-ware-adrian-mckinty-chain.html}{So
would a mansion in the Scottish
Highlands}.\href{https://www.nytimes3xbfgragh.onion/2016/08/07/books/review/inside-the-list.html}{Ruth
Ware} has already used those settings, so she sets the bar higher in
\textbf{ONE BY ONE (Scout Press, 372 pp., \$27.99)} by burying her
principal players in an Alpine chalet beneath an avalanche.

``There is a definite gilded quality to this group,'' Ware writes.
They're an insufferable lot for the most part, especially when they're
fighting over stock shares, so that lovely avalanche can't arrive soon
enough. One awed observer sees ``what looks like a wall of snow coming
down. But not a wall --- that implies something solid. This is something
else. A boiling mass that is air and ice and earth all rolled
together.''

Happily, most of these twits know how to ski, so there are stunning
scenes on the mountain as, one by one, they fall off cliffs, plunge into
ravines and tumble into snowbanks. Readers will recognize the obvious
homage to Agatha Christie's ``And Then There Were None,'' but with
enough ingenious twists to make this whodunit another triumph for Ware.

My idea of a good psychological suspense story is one that messes with
your head. No cheap thrills, just lots of disorienting plot twists that
have you doubting your own mental faculties. \textbf{AN INCONVENIENT
WOMAN (Scarlet, 312 pp., \$25.95),} a remarkably polished first mystery
by Stéphanie Buelens, succeeds at these mind games with a taut plot
about Claire Fontaine, a woman who can't convince anyone that her
ex-husband, Simon, had abused her daughter and caused her death. But in
her determination to keep him from doing the same thing to another
woman's daughter, Claire devotes herself to stalking him. Another strong
woman enters the story when Simon hires Sloan Wilson, a ``sin eater''
who makes her clients' problems go away.

So, what's the story here? Is Simon a monster? Is Claire delusional,
``the madwoman in the attic''? And will the marvelous Sloan Wilson come
back to fix more broken souls? One hopes.

Advertisement

\protect\hyperlink{after-bottom}{Continue reading the main story}

\hypertarget{site-index}{%
\subsection{Site Index}\label{site-index}}

\hypertarget{site-information-navigation}{%
\subsection{Site Information
Navigation}\label{site-information-navigation}}

\begin{itemize}
\tightlist
\item
  \href{https://help.nytimes3xbfgragh.onion/hc/en-us/articles/115014792127-Copyright-notice}{©~2020~The
  New York Times Company}
\end{itemize}

\begin{itemize}
\tightlist
\item
  \href{https://www.nytco.com/}{NYTCo}
\item
  \href{https://help.nytimes3xbfgragh.onion/hc/en-us/articles/115015385887-Contact-Us}{Contact
  Us}
\item
  \href{https://www.nytco.com/careers/}{Work with us}
\item
  \href{https://nytmediakit.com/}{Advertise}
\item
  \href{http://www.tbrandstudio.com/}{T Brand Studio}
\item
  \href{https://www.nytimes3xbfgragh.onion/privacy/cookie-policy\#how-do-i-manage-trackers}{Your
  Ad Choices}
\item
  \href{https://www.nytimes3xbfgragh.onion/privacy}{Privacy}
\item
  \href{https://help.nytimes3xbfgragh.onion/hc/en-us/articles/115014893428-Terms-of-service}{Terms
  of Service}
\item
  \href{https://help.nytimes3xbfgragh.onion/hc/en-us/articles/115014893968-Terms-of-sale}{Terms
  of Sale}
\item
  \href{https://spiderbites.nytimes3xbfgragh.onion}{Site Map}
\item
  \href{https://help.nytimes3xbfgragh.onion/hc/en-us}{Help}
\item
  \href{https://www.nytimes3xbfgragh.onion/subscription?campaignId=37WXW}{Subscriptions}
\end{itemize}
