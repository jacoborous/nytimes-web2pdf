Sections

SEARCH

\protect\hyperlink{site-content}{Skip to
content}\protect\hyperlink{site-index}{Skip to site index}

\href{https://www.nytimes3xbfgragh.onion/section/politics}{Politics}

\href{https://myaccount.nytimes3xbfgragh.onion/auth/login?response_type=cookie\&client_id=vi}{}

\href{https://www.nytimes3xbfgragh.onion/section/todayspaper}{Today's
Paper}

\href{/section/politics}{Politics}\textbar{}Democrats Gained These
Voters in 2018, and Biden Needs to Keep Them Now

\url{https://nyti.ms/3bLJvlP}

\begin{itemize}
\item
\item
\item
\item
\item
\end{itemize}

\begin{itemize}
\item
  \href{https://www.nytimes3xbfgragh.onion/2020/09/12/us/politics/biden-trump-poll-wisconsin-minnesota.html?action=click\&pgtype=Article\&state=default\&region=TOP_BANNER\&context=storylines_menu}{New
  York Times Poll}
\item
  \href{https://www.nytimes3xbfgragh.onion/interactive/2020/us/elections/election-states-biden-trump.html?action=click\&pgtype=Article\&state=default\&region=TOP_BANNER\&context=storylines_menu}{Paths
  to 270}
\item
  \href{https://www.nytimes3xbfgragh.onion/interactive/2019/us/elections/2020-presidential-election-calendar.html?action=click\&pgtype=Article\&state=default\&region=TOP_BANNER\&context=storylines_menu}{Voting
  Deadlines}
\item
  \href{https://www.nytimes3xbfgragh.onion/interactive/2020/08/31/us/politics/vote-by-mail-deadlines.html?action=click\&pgtype=Article\&state=default\&region=TOP_BANNER\&context=storylines_menu}{Voting
  by Mail}
\item
  \href{https://www.nytimes3xbfgragh.onion/newsletters/politics?action=click\&pgtype=Article\&state=default\&region=TOP_BANNER\&context=storylines_menu}{Politics
  Newsletter}
\end{itemize}

Advertisement

\protect\hyperlink{after-top}{Continue reading the main story}

Supported by

\protect\hyperlink{after-sponsor}{Continue reading the main story}

Poll Watch

\hypertarget{democrats-gained-these-voters-in-2018-and-biden-needs-to-keep-them-now}{%
\section{Democrats Gained These Voters in 2018, and Biden Needs to Keep
Them
Now}\label{democrats-gained-these-voters-in-2018-and-biden-needs-to-keep-them-now}}

An authoritative new study on 2018 midterm voters shows how despite his
strong position, Joe Biden is lagging behind the rates at which some key
demographics broke for the Democrats two years ago.

\includegraphics{https://static01.graylady3jvrrxbe.onion/images/2020/09/11/us/politics/11pollwatch-newsletter/merlin_169981998_8538b3a3-62d9-4cf5-bf8d-a9f390ea3dca-articleLarge.jpg?quality=75\&auto=webp\&disable=upscale}

\href{https://www.nytimes3xbfgragh.onion/by/giovanni-russonello}{\includegraphics{https://static01.graylady3jvrrxbe.onion/images/2019/04/03/multimedia/author-giovanni-russonello/author-giovanni-russonello-thumbLarge.png}}

By
\href{https://www.nytimes3xbfgragh.onion/by/giovanni-russonello}{Giovanni
Russonello}

\begin{itemize}
\item
  Sept. 11, 2020
\item
  \begin{itemize}
  \item
  \item
  \item
  \item
  \item
  \end{itemize}
\end{itemize}

\emph{Welcome to}
\href{https://www.nytimes3xbfgragh.onion/column/poll-watch}{\emph{Poll
Watch}}\emph{, our weekly look at}
\href{https://www.nytimes3xbfgragh.onion/interactive/2020/us/elections/democratic-polls.html}{\emph{polling
data}} \emph{and survey research on the candidates, voters and issues
that will shape the}
\href{https://www.nytimes3xbfgragh.onion/news-event/2020-election}{\emph{2020
election}}\emph{.}

\begin{center}\rule{0.5\linewidth}{\linethickness}\end{center}

Two years ago, a flood of anti-Trump sentiment helped flip the House of
Representatives blue, with Democratic candidates winning more votes than
Republicans by nearly 10 percentage points nationwide --- a record
margin for a party that had been in the minority.

Because the 2018 election was largely seen as a referendum on President
Trump, Democratic strategists are looking to carry those gains forward.
Indeed, national and swing-state polls continue to show Joseph R. Biden
Jr. with a steady lead --- particularly in the suburban areas where
Democrats made some of their biggest gains in the midterms.

But Mr. Biden may not be able to count on the same level of support that
Democratic candidates received in 2018. Some of the groups that swung
hardest in Democrats' direction that year have been slow to warm to Mr.
Biden. Compared with an
\href{https://www.pewresearch.org/methods/2020/09/08/democrats-made-gains-from-multiple-sources-in-2018-midterm-victories/}{authoritative
study of the 2018 midterm electorate} released this week by the Pew
Research Center, recent polls show the party's presidential nominee
lagging behind the rates at which certain key demographics broke for the
Democrats two years ago.

To conduct the study, Pew used its American Trends Panel, which tracks a
nationally representative sample of Americans and allows researchers to
re-contact the same voters over time. Because of its large sample size
and because it used a method called voter validation --- checking
panelists' responses against publicly available voter files to confirm
that they participated when they said they did --- Pew's study is
considered more reliable than the national exit polls, which are
conducted quickly on the day of the election and undergo minimal
adjustments afterward.

Midterm elections after a new president has taken office always tend to
be tough for the president's party. Yet midterm voters also tend to skew
slightly more affluent and conservative than general-election voters. So
the surge in Democratic votes across the board, particularly among key
groups, appears to tee up Mr. Biden for a strong showing.

Latinos, white suburbanites and young voters swung especially hard in
Democrats' favor in 2018, as the Pew study reflected, sometimes even
more starkly than the exit polls. Here's a look at what the Pew study
tells us about those groups, and at where things stand now.

\hypertarget{young-voters}{%
\subsection{Young voters}\label{young-voters}}

Youth enthusiasm and participation ran low in the 2016 general election,
but voters under 30 grew heavily involved in 2018 --- doubling their
participation rate from the previous midterms, according to an analysis
by the United States Elections Project at the University of Florida. No
other age group jumped by as much.

In 2016, most young voters
\href{https://www.monmouth.edu/polling-institute/documents/monmouthpoll_us_110716.pdf/}{saw
neither candidate} in a positive light, and 14 percent of them expressed
their displeasure by voting third party, a far higher number than for
older voters. But as they surged to the polls in 2018, those under 30
picked Democratic House candidates by an enormous 49-point margin.
(That's way more than the 35-point advantage that
\href{https://www.cnn.com/election/2018/exit-polls}{national exit polls}
from 2018 had indicated.)

Scott Keeter, a senior survey adviser at Pew who helped assemble the
report, noted that people under 30 accounted for more than one-third of
2018 voters who had not cast ballots in 2016. ``That's a fairly striking
figure,'' he said. ``And they were already a good group for Hillary
Clinton, but they became even more Democratic in 2018.''

Yet Mr. Biden is not particularly popular among young people: His
favorability rating is five points in the negative among likely voters
under 35, according to the latest
\href{https://poll.qu.edu/images/polling/us/us09022020_udmp37.pdf}{Quinnipiac
University poll}. Still, young voters appear to be even less fond of Mr.
Trump --- and uninterested in sitting out another presidential election.

Mr. Biden leads Mr. Trump by 63 to 27 percent among voters under 35,
according to Quinnipiac; that's better than Mrs. Clinton's 58-28 margin
among the youngest voters in 2016, according to Pew, and it suggests
that those who didn't vote or who cast third-party ballots in 2016 may
be wary of doing so again.

\hypertarget{hispanic-voters}{%
\subsection{Hispanic voters}\label{hispanic-voters}}

Hispanic turnout two years ago jumped by 74 percent from the 2014
midterms --- more than for any other major racial or ethnic group ---
according to voting data
\href{http://www.electproject.org/home/voter-turnout/demographics}{compiled
by the Elections Project}. (The growth rate for Black and white voters
was about half that.) These voters favored Democrats by a whopping 47
points in 2018, with just a quarter of Hispanic voters casting ballots
for Republicans, the Pew study shows. That was a nine-point gain on Mrs.
Clinton's margin in 2016.

Mr. Biden is up on Mr. Trump by anywhere from 20 to
\href{https://www.suffolk.edu/-/media/suffolk/documents/academics/research-at-suffolk/suprc/polls/national/2020/9_4_2020_tables_pdftxt_1.pdf?la=en\&hash=D3CA68E8C4C8511FBE7BE4A26D60C717C859A38B}{32
points} among Hispanic voters nationwide, according to recent polls.
That is considerably weaker than Mrs. Clinton's advantage, and far below
the Democrats' wider lead in 2018.

Partly because the Hispanic population is so diverse --- with regard to
nation of origin, racial identity and political thought, among much else
--- and because phone surveys rarely have a Hispanic sample of much more
than 100 people, this group can be hard to accurately poll. Pollsters
have long considered exit polling of Hispanic voters to be notably
problematic.

That is part of what is so valuable about the Pew report, which affirms,
using validated voters, how Hispanic voters actually cast ballots ---
though it doesn't break down the Hispanic population into more specific
demographic categories.

``There's all sorts of signals coming from the polling world,'' Mr.
Keeter said, ``that make it hard to know how enthusiastic the Latino
vote is for Biden and ultimately how much they're going to turn out for
him.''

\hypertarget{suburban-voters}{%
\subsection{Suburban voters}\label{suburban-voters}}

Suburban voters were crucial to Democratic victories in many House
districts that flipped from red to blue in 2018, and Pew's results
reflected that trend nationwide. While the suburbs over all broke just
barely for Mr. Trump in 2016, these increasingly diverse areas of the
country swung Democratic by seven points in 2018, Pew found.

Among white suburbanites in particular there was still a slight
Republican tilt in the midterms, but these voters chose Democratic
candidates 47 percent of the time, according to Pew. That was up from
the 38 percent who voted for Mrs. Clinton.

This is one area in which Mr. Biden is well positioned to carry the
gains of 2018 forward --- and even exceed them. An
\href{http://maristpoll.marist.edu/wp-content/uploads/2020/08/NPR_PBS-NewsHour_Marist-Poll_USA-NOS-and-TABLES_202008121039-1.pdf\#page=3}{NPR/PBS
NewsHour/Marist College poll} last month showed Mr. Biden winning in the
suburbs by a whopping 25 points. A separate
\href{http://maristpoll.marist.edu/wp-content/uploads/2020/09/NBC-News_Marist-Poll_PA-Likely-Voters_NOS-and-Tables_202009081145.pdf\#page=3}{Marist
poll} this week of Pennsylvania found Mr. Biden leading in the suburbs
by 19 points

\hypertarget{who-got-more-involved-in-2018}{%
\subsection{Who got more involved in
2018}\label{who-got-more-involved-in-2018}}

In a sign of just how heavily the anti-Trump winds were blowing in the
midterms, when you look specifically at new voters and 2016 abstainers
(those who didn't vote for either Mr. Trump or Mrs. Clinton), Democrats
won even in rural areas, a Trump stronghold, by 19 points, according to
Pew.

Over all, those who sat on the sidelines in 2016 but went to vote in
2018 tended to be liberal or moderate, and they were more likely to be
Black or Hispanic than those who had voted for a major candidate in
2016. That reflects the lack of enthusiasm felt both on the left and
among voters of color in the last presidential election.

The bounce-back of nonwhite voter participation in 2018 also serves as a
reminder to Democratic strategists that Barack Obama's absence from the
ballot in 2016 was not the only factor driving down participation from
Black and other nonwhite voters.

The challenge confronting Mr. Biden will be to convince many of those
who sat out in 2016, but two years later felt compelled to provide a
check to Mr. Trump, that the former vice president is worth voting
Democratic again for, even amid a pandemic.

\hypertarget{our-2020-election-guide}{%
\section{Our 2020 Election Guide}\label{our-2020-election-guide}}

Updated ~Sept. 12, 2020

\begin{center}\rule{0.5\linewidth}{\linethickness}\end{center}

\begin{itemize}
\item ~
  \hypertarget{the-latest}{%
  \subsection{The Latest}\label{the-latest}}

  \begin{itemize}
  \item
    President Trump has failed to erase Joseph R. Biden Jr.'s lead
    across a set of key swing states,
    \href{https://www.nytimes3xbfgragh.onion/2020/09/12/us/politics/biden-trump-poll-wisconsin-minnesota.html?action=click\&pgtype=Article\&state=default\&region=BELOW_MAIN_CONTENT\&context=storylines_guide}{according
    to a poll}~conducted by The Times and Siena College.
  \end{itemize}
\item ~
  \hypertarget{paths-to-270}{%
  \subsection{Paths to 270}\label{paths-to-270}}

  \begin{itemize}
  \item
    Joe Biden and Donald Trump need 270 electoral votes to reach the
    White House. Try building
    \href{https://www.nytimes3xbfgragh.onion/interactive/2020/us/elections/election-states-biden-trump.html?action=click\&pgtype=Article\&state=default\&region=BELOW_MAIN_CONTENT\&context=storylines_guide}{your
    own coalition of battleground states}~to see potential outcomes.
  \end{itemize}
\item ~
  \hypertarget{voting-deadlines}{%
  \subsection{Voting Deadlines}\label{voting-deadlines}}

  \begin{itemize}
  \item
    Early voting for the presidential election starts in September~in
    some states. Take a look at
    \href{https://www.nytimes3xbfgragh.onion/interactive/2019/us/elections/2020-presidential-election-calendar.html?action=click\&pgtype=Article\&state=default\&region=BELOW_MAIN_CONTENT\&context=storylines_guide}{key
    dates}\href{https://www.nytimes3xbfgragh.onion/interactive/2019/us/elections/2020-presidential-election-calendar.html?action=click\&pgtype=Article\&state=default\&region=BELOW_MAIN_CONTENT\&context=storylines_guide}{where
    you
    liv}\href{https://www.nytimes3xbfgragh.onion/interactive/2019/us/elections/2020-presidential-election-calendar.html?action=click\&pgtype=Article\&state=default\&region=BELOW_MAIN_CONTENT\&context=storylines_guide}{e}.
    If you're voting by
    mail,~\href{https://www.nytimes3xbfgragh.onion/interactive/2020/08/31/us/politics/vote-by-mail-deadlines.html?action=click\&pgtype=Article\&state=default\&region=BELOW_MAIN_CONTENT\&context=storylines_guide}{it's
    risky to procrastinate}.
  \item
    \href{https://www.nytimes3xbfgragh.onion/interactive/2020/us/elections/joe-biden.html?action=click\&pgtype=Article\&state=default\&region=BELOW_MAIN_CONTENT\&context=storylines_guide}{}

    \hypertarget{joe-biden}{%
    \section{Joe Biden}\label{joe-biden}}

    \hypertarget{democrat}{%
    \subsection{Democrat}\label{democrat}}

    \href{https://www.nytimes3xbfgragh.onion/interactive/2020/us/elections/donald-trump.html?action=click\&pgtype=Article\&state=default\&region=BELOW_MAIN_CONTENT\&context=storylines_guide}{}

    \hypertarget{donald-trump}{%
    \section{Donald Trump}\label{donald-trump}}

    \hypertarget{republican}{%
    \subsection{Republican}\label{republican}}
  \end{itemize}
\item
  \hypertarget{keep-up-with-our-coverage}{%
  \subsection{Keep Up With Our
  Coverage}\label{keep-up-with-our-coverage}}

  \begin{itemize}
  \item
    Get an
    \href{https://www.nytimes3xbfgragh.onion/newsletters/politics?action=click\&pgtype=Article\&state=default\&region=BELOW_MAIN_CONTENT\&context=storylines_guide}{email}~recapping
    the day's news
  \item
    Download our mobile app on
    \href{https://apps.apple.com/us/app/nytimes/id284862083?ls=1\&mat_click_id=5c79ae7455014fd1bd66b5610c05b8f2-20191112-16948\&referrer=mat_click_id\%3D5c79ae7455014fd1bd66b5610c05b8f2-20191112-16948\%26link_click_id\%3D722930677036718082}{iOS}~and
    \href{http://a.localytics.com/android?id=com.nytimes.android\&referrer=utm_source\%3Dother_nyt_mobile_web\%26utm_medium\%3DWeb\%2520page\%26utm_term\%3DGeneral\%2520Mobile\%2520Page\%26utm_campaign\%3DNYT\%2520Mobile\%2520General\%2520Page}{Android}~and
    turn on Breaking News and Politics alerts
  \end{itemize}
\end{itemize}

Advertisement

\protect\hyperlink{after-bottom}{Continue reading the main story}

\hypertarget{site-index}{%
\subsection{Site Index}\label{site-index}}

\hypertarget{site-information-navigation}{%
\subsection{Site Information
Navigation}\label{site-information-navigation}}

\begin{itemize}
\tightlist
\item
  \href{https://help.nytimes3xbfgragh.onion/hc/en-us/articles/115014792127-Copyright-notice}{©~2020~The
  New York Times Company}
\end{itemize}

\begin{itemize}
\tightlist
\item
  \href{https://www.nytco.com/}{NYTCo}
\item
  \href{https://help.nytimes3xbfgragh.onion/hc/en-us/articles/115015385887-Contact-Us}{Contact
  Us}
\item
  \href{https://www.nytco.com/careers/}{Work with us}
\item
  \href{https://nytmediakit.com/}{Advertise}
\item
  \href{http://www.tbrandstudio.com/}{T Brand Studio}
\item
  \href{https://www.nytimes3xbfgragh.onion/privacy/cookie-policy\#how-do-i-manage-trackers}{Your
  Ad Choices}
\item
  \href{https://www.nytimes3xbfgragh.onion/privacy}{Privacy}
\item
  \href{https://help.nytimes3xbfgragh.onion/hc/en-us/articles/115014893428-Terms-of-service}{Terms
  of Service}
\item
  \href{https://help.nytimes3xbfgragh.onion/hc/en-us/articles/115014893968-Terms-of-sale}{Terms
  of Sale}
\item
  \href{https://spiderbites.nytimes3xbfgragh.onion}{Site Map}
\item
  \href{https://help.nytimes3xbfgragh.onion/hc/en-us}{Help}
\item
  \href{https://www.nytimes3xbfgragh.onion/subscription?campaignId=37WXW}{Subscriptions}
\end{itemize}
