Sections

SEARCH

\protect\hyperlink{site-content}{Skip to
content}\protect\hyperlink{site-index}{Skip to site index}

\href{https://www.nytimes3xbfgragh.onion/section/politics}{Politics}

\href{https://myaccount.nytimes3xbfgragh.onion/auth/login?response_type=cookie\&client_id=vi}{}

\href{https://www.nytimes3xbfgragh.onion/section/todayspaper}{Today's
Paper}

\href{/section/politics}{Politics}\textbar{}Veterans Fortify the Ranks
of Militias Aligned With Trump's Views

\url{https://nyti.ms/32kcjP2}

\begin{itemize}
\item
\item
\item
\item
\item
\end{itemize}

Advertisement

\protect\hyperlink{after-top}{Continue reading the main story}

Supported by

\protect\hyperlink{after-sponsor}{Continue reading the main story}

\hypertarget{veterans-fortify-the-ranks-of-militias-aligned-with-trumps-views}{%
\section{Veterans Fortify the Ranks of Militias Aligned With Trump's
Views}\label{veterans-fortify-the-ranks-of-militias-aligned-with-trumps-views}}

The vast majority of veterans do not join militias, but some
fast-growing militias have many veterans among their ranks.

\includegraphics{https://static01.graylady3jvrrxbe.onion/images/2020/09/10/us/politics/10dc-unrest-militias1/merlin_174937320_6efcc5d5-cb90-4396-a481-cc280e95bc97-articleLarge.jpg?quality=75\&auto=webp\&disable=upscale}

By
\href{https://www.nytimes3xbfgragh.onion/by/jennifer-steinhauer}{Jennifer
Steinhauer}

\begin{itemize}
\item
  Sept. 11, 2020
\item
  \begin{itemize}
  \item
  \item
  \item
  \item
  \item
  \end{itemize}
\end{itemize}

WASHINGTON --- Emboldened by President Trump's campaign platform of law
and order, militia groups have bolstered their strength before Election
Day by attracting military veterans who bring weapons and tactical
skills viewed as important to the organizations.

The role of veterans in the newly proliferating militia groups --- which
sometimes are steeped in racism and other times steeped simply in
antigovernment zealotry --- has increased over the last decade, said a
dozen experts on law enforcement, domestic terrorism and extremist
groups.

Although only a small fraction of the nation's 20 million veterans joins
militia groups, experts in domestic terrorism and law enforcement
analysts estimate that veterans and active-duty members of the military
may now make up at least 25 percent of militia rosters. These experts
estimate that there are some 15,000 to 20,000 active militia members in
around 300 groups.

But gauging the size of these groups is difficult and imprecise, because
much of their membership is limited to online participation. The
estimates are based on samplings of militia member data gleaned from
social media profiles, blogs, online forums, militia publications,
interviews, assessments from watchdog groups and news reports.

At least four recently formed violent organizations were founded by
military veterans, and many high-profile episodes stemming from militia
groups ---
\href{https://www.nytimes3xbfgragh.onion/2020/06/16/us/steven-carrillo-air-force-boogaloo.html}{the
killing of a federal security officer in May} in Oakland, Calif.,
\href{https://apnews.com/6223153093f08fa910c4ab445771b773}{a thwarted
plan} to incite violence at a recent demonstration in Las Vegas and
\href{https://www.propublica.org/article/atomwaffen-division-hate-group-active-duty-military}{the
violence during a 2017 protest} in Charlottesville, Va. --- involved
veterans.

Underscoring how the threat of violent domestic groups is rising with
limited official oversight, the top leaders of the Department of
Homeland Security directed agency analysts to play down threats from
white supremacist groups, according to
\href{https://int.graylady3jvrrxbe.onion/data/documenttools/homeland-security-whistleblower/0819ec9ee29306a5/full.pdf}{a
whistle-blower complaint} released on Wednesday.

While militias and other paramilitary groups have been historically
hostile toward the federal government regardless of the party in power,
many have turned their animus in recent months toward Black Lives Matter
activists as well as local leaders who enforced restrictions to combat
the coronavirus. A notable example was in Michigan, where
\href{https://www.theguardian.com/world/2020/apr/17/far-right-coronavirus-protests-restrictions}{protesters,
some armed, stormed the statehouse} this spring in opposition to
pandemic rules. Some have begun adopting the language Mr. Trump uses to
preemptively cast doubt on the outcome of an election.

Militias have historically risen after periods of war, said Kathleen
Belew, an assistant professor of history at the University of Chicago
and author of ``Bring the War Home: The White Power Movement and
Paramilitary America.''

``We have seen veterans and active-duty members being recruited because
they have operational skills that are useful,'' Ms. Belew said. She
described the estimates of how many veterans had been drawn to the
movement as ``deeply concerning.''

It is an issue that federal agencies have largely avoided. ``The V.A.
has no authority to enact or enforce policies regarding veterans'
memberships in any organizations,'' said Christina Noel, a spokeswoman
for the Department of Veterans Affairs.

One of the larger groups,
\href{https://www.adl.org/resources/profiles/the-oath-keepers}{the Oath
Keepers}, makes recruiting veterans and law enforcement officers central
to its mission.

``As a country we have spent so long at war overseas that a small
percentage of veterans, but a percentage nonetheless, has warmed them to
the idea that the way to deal with political conflict is to engage in
armed struggle,'' said Devin Burghart, the executive director of the
Institute for Research and Education on Human Rights, a Seattle-based
research center on far-right groups. ``This is a dangerous indicator of
where things may go.''

\includegraphics{https://static01.graylady3jvrrxbe.onion/images/2020/09/10/us/politics/10dc-unrest-militias/10dc-unrest-militias-articleLarge.jpg?quality=75\&auto=webp\&disable=upscale}

From the years after the Vietnam War to the mid-1990s, a small flurry of
militia groups cropped up around the United States.

\href{https://www.splcenter.org/fighting-hate/extremist-files/individual/frazier-glenn-miller}{Frazier
Glenn Miller}, a former Army master sergeant who served two tours in
Vietnam, created the White Patriot Party in the 1980s. Decades later, he
was sentenced to death for killing three people outside a Jewish
community center in Overland Park, Kan. In 1995,
\href{https://www.biography.com/crime-figure/timothy-mcveigh}{Timothy J.
McVeigh}, a former Army soldier who belonged to a right-wing survivalist
group based in Michigan, blew up a federal building in Oklahoma City,
killing 168 people, including 19 children. Mr. McVeigh promoted the
works of
\href{https://www.nytimes3xbfgragh.onion/2002/07/24/us/william-pierce-69-neo-nazi-leader-dies.html}{William
Pierce}, who ran a white supremacist group that
\href{https://www.nytimes3xbfgragh.onion/1995/12/21/us/for-most-gi-s-only-few-hints-of-hate-groups.html}{once
posted a recruiting notice} on a billboard outside Fort Bragg, N.C.

But beginning in 2009, antagonism toward the presidency of Barack Obama,
combined with a new crop of post-Sept. 11 veterans, fueled exponential
growth in militia groups.

While the military strictly forbids its active-duty personnel from
participating in hate groups, it is silent on militias and the role of
veterans who have left service.

``Veterans are often looked at for their paramilitary skills, their
ability to survive in the field as well as leadership skills,'' said
Daryl Johnson, a former senior terrorism analyst at the Department of
Homeland Security. ``They are proficient with weapons, which they often
own.''

While many veterans who are deployed overseas return filled with
gratitude to be back in the United States, others return with very
different views, informed by their work in countries whose political
systems they despise and fearful that such ideologies could infiltrate
their own country.

``You see overseas how things can go wrong,'' Mr. Johnson said. Fear of
communism, Islamic law and Marxism permeate some veterans' thinking.
``They take experiences they have had overseas and transport them to the
homeland and think there are all these threats,'' he said.

In 2009, the Department of Homeland Security released an
\href{https://fas.org/irp/eprint/rightwing.pdf}{intelligence assessment}
warning that returning veterans who faced trouble reintegrating could
``lead to the potential emergence of terrorist groups or lone wolf
extremists capable of carrying out violent attacks.''

The report led to such an outcry from conservatives and one prominent
veterans organization that the department
\href{https://thecaucus.blogs.nytimes3xbfgragh.onion/2009/04/16/extremist-report-draws-criticism-prompts-apology/}{deep-sixed
it}. ``We used the term `disgruntled' so that terminology was
insensitive,'' said Mr. Johnson, who helped prepare the report. ``We
were trying to show they were susceptible to recruitment because of
skills they learned. That is a glaring truth no matter who is
offended.''

That same year, the F.B.I. did its
\href{https://www.wsj.com/articles/SB123992665198727459}{own
investigation} of extremist groups with a focus on veterans from Iraq
and Afghanistan.

The Obama years were a growth period for these groups, many of them
loosely tied to the Tea Party movement. Most notable was the Oath
Keepers, formed in 2009 with a core notion that its members should
continue to honor the oaths they took in the military and law
enforcement agencies to defend the country, via their efforts in a
militia.

Stewart Rhodes, a former Army paratrooper who served as a staff member
for Ron Paul, then a Republican representative of Texas, ``formed the
group to encourage current and former military and law enforcement
members to honor their oath against tyranny,'' said Sam Jackson, an
assistant professor at the University at Albany who has written a book
on the group. ``But the focus of threats has changed to be antifa and
Black Lives Matter and others on the left.''

Image

Clashes during the Unite the Right rally in Charlottesville in
2017.Credit...Edu Bayer for The New York Times

The movement has accelerated during Mr. Trump's time in office. In 2015,
Brandon Russell, a member of the Florida Army National Guard, formed the
Atomwaffen Division, a neo-Nazi group. One of its members,
\href{https://www.marinecorpstimes.com/news/your-marine-corps/2019/09/04/the-neo-nazi-boot-inside-one-marines-descent-into-extremism/}{Vasillios
Pistolis}, a private at the time, participated in the ``Unite the
Right'' rally in Charlottesville, bragging on social media about
injuring people. (He was
\href{https://www.marinecorpstimes.com/news/your-marine-corps/2019/09/04/the-neo-nazi-boot-inside-one-marines-descent-into-extremism/}{later
kicked out} of the Marines.)

After that rally in 2017, Joffre Cross III, a former private in the 82nd
Airborne Division at Fort Bragg and a member of the newly formed Patriot
Front, was charged with multiple weapons felonies.

\href{https://www.theatlantic.com/technology/archive/2020/07/american-boogaloo-meme-or-terrorist-movement/613843/}{The
``boogaloo'' movement}, a loose network of right-leaning, antigovernment
groups that seeks to bring about a second civil war to overthrow the
government, has been around since 2012, when it was largely an online
movement.

In June, Daniel Austin Dunn, a former Marine,
\href{https://int.graylady3jvrrxbe.onion/data/documenttools/dunn-indictment/a97279ae6811af57/full.pdf}{was
indicted in Texas} for making violent threats toward police officers on
Facebook and Twitter posts, in which he associated himself with
boogaloos. The authorities found a cache of weapons at his house. This
year,
\href{https://www.militarytimes.com/news/your-military/2020/06/04/army-reservist-navy-and-air-force-vets-plotted-to-terrorize-vegas-protests-prosecutors-charge/}{the
F.B.I. arrested an Army reservist and two veterans} with ties to the
movement for planning to incite violence at a Las Vegas protest. An
\href{https://www.airforcetimes.com/news/your-air-force/2020/06/16/air-force-sergeant-charged-in-killing-of-federal-officer-in-california/}{active-duty
airman} affiliated with the group was also charged with killing a
federal officer in Oakland.

A small number of veterans have joined ranks with left-leaning groups or
groups not associated with the political right. A sniper who shot a
dozen Dallas police officers in 2016,
\href{https://www.nbcnews.com/storyline/dallas-police-ambush/dallas-shooter-micah-xavier-johnson-was-army-veteran-n606101}{killing
five,} was an Army veteran.

The man law enforcement officials believe shot and killed a right-wing
activist in Portland, Ore., last month was an antifa supporter and a
veteran; he was
killed\href{https://www.nytimes3xbfgragh.onion/2020/09/03/us/michael-reinoehl-arrest-portland-shooting.html}{last
week}by the police. But veterans with far-left views are small in number
and tend to act outside any organized force --- the antifa movement, for
example, lacks the structure and leadership of a militia --- according
to experts in the field.

Many groups have proclaimed themselves as enforcers of Trump
administration policies, and more recently, as protectors of businesses
in cities with protests, often antagonizing those protesters. The
confrontations with protesters have also dovetailed with actions to
protest
\href{https://www.nytimes3xbfgragh.onion/2020/05/03/us/coronavirus-extremists.html}{coronavirus
containment measures}, often with a side of conspiracy theories to
generate new member interest.

Image

A member of the Three Percenters during a militia training exercise near
Jackson, Ga., in 2016.Credit...Kevin D. Liles for The New York Times

A well-known group, the
\href{https://www.adl.org/resources/backgrounders/three-percenters}{Three
Percenters}, focuses on anti-immigrant activities and targets leftists
like members of antifa. A leader of a chapter in Georgia,
\href{https://www.nytimes3xbfgragh.onion/2016/11/05/us/a-militia-gets-battle-ready-for-a-gun-grabbing-clinton-presidency.html}{Chris
Hill}, is a Marine veteran who leads regular field training exercises.

The United Constitutional Patriots, a militia that patrols the
southwestern border with Mexican,
\href{https://www.nytimes3xbfgragh.onion/2019/04/23/us/new-mexico-militia-border.html}{has
also attracted veterans}.

``The militia movement traditionally hated the federal government,''
said Heidi Beirich, a co-founder of the Global Project Against Hate and
Extremism. ``This has completely changed with Trump.''

As they have inserted themselves in cities with large protests, the
groups have found themselves sometimes welcomed by local law
enforcement. ``We have militia groups that are inserting themselves into
cities to, from their perspective, to fill a vacuum of law
enforcement,'' said Seth G. Jones, a senior adviser at the Center for
Strategic and International Studies. ``But they are doing things outside
of the law to take law and order into their own hands.''

Mike Martinez, the police chief of Arroyo Grande, Calif., said the
militias were a concern. ``Many militias have their own ideology,'' he
said. ``Some are not pro-law enforcement, so it is always important for
us to be aware.''

The end of the Trump era would not spell the end to militias, the
experts agreed. ``In the immediate aftermath of an election, I don't see
this ebbing,'' Mr. Jones said. ``In fact my concern is there will be a
range of organizations that don't support the legitimacy of a Biden
president and that administration will have to think about how to disarm
militias. That will be a dangerous situation.''

Seamus Hughes contributed reporting

Advertisement

\protect\hyperlink{after-bottom}{Continue reading the main story}

\hypertarget{site-index}{%
\subsection{Site Index}\label{site-index}}

\hypertarget{site-information-navigation}{%
\subsection{Site Information
Navigation}\label{site-information-navigation}}

\begin{itemize}
\tightlist
\item
  \href{https://help.nytimes3xbfgragh.onion/hc/en-us/articles/115014792127-Copyright-notice}{©~2020~The
  New York Times Company}
\end{itemize}

\begin{itemize}
\tightlist
\item
  \href{https://www.nytco.com/}{NYTCo}
\item
  \href{https://help.nytimes3xbfgragh.onion/hc/en-us/articles/115015385887-Contact-Us}{Contact
  Us}
\item
  \href{https://www.nytco.com/careers/}{Work with us}
\item
  \href{https://nytmediakit.com/}{Advertise}
\item
  \href{http://www.tbrandstudio.com/}{T Brand Studio}
\item
  \href{https://www.nytimes3xbfgragh.onion/privacy/cookie-policy\#how-do-i-manage-trackers}{Your
  Ad Choices}
\item
  \href{https://www.nytimes3xbfgragh.onion/privacy}{Privacy}
\item
  \href{https://help.nytimes3xbfgragh.onion/hc/en-us/articles/115014893428-Terms-of-service}{Terms
  of Service}
\item
  \href{https://help.nytimes3xbfgragh.onion/hc/en-us/articles/115014893968-Terms-of-sale}{Terms
  of Sale}
\item
  \href{https://spiderbites.nytimes3xbfgragh.onion}{Site Map}
\item
  \href{https://help.nytimes3xbfgragh.onion/hc/en-us}{Help}
\item
  \href{https://www.nytimes3xbfgragh.onion/subscription?campaignId=37WXW}{Subscriptions}
\end{itemize}
