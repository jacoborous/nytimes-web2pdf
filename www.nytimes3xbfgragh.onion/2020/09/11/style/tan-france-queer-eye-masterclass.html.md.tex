\href{/section/style}{Style}\textbar{}Back to School With Tan France

\url{https://nyti.ms/2FikXoA}

\begin{itemize}
\item
\item
\item
\item
\item
\item
\end{itemize}

\href{https://www.nytimes3xbfgragh.onion/spotlight/at-home?action=click\&pgtype=Article\&state=default\&region=TOP_BANNER\&context=at_home_menu}{At
Home}

\begin{itemize}
\tightlist
\item
  \href{https://www.nytimes3xbfgragh.onion/2020/09/07/travel/route-66.html?action=click\&pgtype=Article\&state=default\&region=TOP_BANNER\&context=at_home_menu}{Cruise
  Along: Route 66}
\item
  \href{https://www.nytimes3xbfgragh.onion/2020/09/04/dining/sheet-pan-chicken.html?action=click\&pgtype=Article\&state=default\&region=TOP_BANNER\&context=at_home_menu}{Roast:
  Chicken With Plums}
\item
  \href{https://www.nytimes3xbfgragh.onion/2020/09/04/arts/television/dark-shadows-stream.html?action=click\&pgtype=Article\&state=default\&region=TOP_BANNER\&context=at_home_menu}{Watch:
  Dark Shadows}
\item
  \href{https://www.nytimes3xbfgragh.onion/interactive/2020/at-home/even-more-reporters-editors-diaries-lists-recommendations.html?action=click\&pgtype=Article\&state=default\&region=TOP_BANNER\&context=at_home_menu}{Explore:
  Reporters' Google Docs}
\end{itemize}

\includegraphics{https://static01.graylady3jvrrxbe.onion/images/2020/09/13/fashion/10TANFRANCE3/merlin_176755773_27b5494b-472d-4358-8090-6bec1d67f78d-articleLarge.jpg?quality=75\&auto=webp\&disable=upscale}

Sections

\protect\hyperlink{site-content}{Skip to
content}\protect\hyperlink{site-index}{Skip to site index}

\hypertarget{back-to-school-with-tan-france}{%
\section{Back to School With Tan
France}\label{back-to-school-with-tan-france}}

The ``Queer Eye'' star has a lot to teach us about dressing.

Credit...Phylicia J. L. Munn for The New York Times

Supported by

\protect\hyperlink{after-sponsor}{Continue reading the main story}

By Brennan Carley

\begin{itemize}
\item
  Sept. 11, 2020
\item
  \begin{itemize}
  \item
  \item
  \item
  \item
  \item
  \item
  \end{itemize}
\end{itemize}

Three years ago at an Embassy Suites in Glendale, Calif.,
\href{https://www.nytimes3xbfgragh.onion/2018/03/06/style/antoni-porowski-queer-eye-cooking.html}{Antoni
Porowski} spotted his soon-to-be co-star Tan France in the hotel's gym.
Mr. France was wearing tailored gym shorts and a crisp Nike shirt tucked
neatly into his waistband. His hair was perfectly coifed and he was
smiling as he StairMastered without breaking a sweat.

An hour later, Mr. France locked eyes with Mr. Porowski, who was wearing
a robe in the hotel's dining room.

``It was a chic robe,'' Mr. Porowski said. ``But Tan came down in those
gym clothes, and he made a comment: `Oh, showing up to breakfast wearing
a \emph{bathrobe}.' I was like, `Yeah, I'm in a hotel. It's what you're
supposed to do.' He was sweet about it, but there was also a hint of
judgment.''

That delicate balance led Mr. France, 37, to be cast alongside Mr.
Porowski, Karamo Brown, Bobby Berk and
\href{https://www.nytimes3xbfgragh.onion/2019/09/21/style/jonathan-van-ness-hiv-memoir.html}{Jonathan
Van Ness} as the style expert on Netflix's ``Queer Eye.'' The show,
which released its fifth season in June, follows the experts as they
travel in search of ``heroes'' whose lives could be improved by simple
tips, a haircut, some positive self-talk, major home renovations and, of
course, shopping.

\includegraphics{https://static01.graylady3jvrrxbe.onion/images/2020/09/13/fashion/10TANFRANCE2/merlin_173200725_46426648-ff42-4407-ba6a-7056822a8ee0-articleLarge.jpg?quality=75\&auto=webp\&disable=upscale}

Mr. France has spun his newfound fame into ample opportunity --- a 2019
memoir (``Naturally Tan''), a Netflix competition series co-hosted by
Alexa Chung (``Next in Fashion''), and a web series in which he styles
famous comedians (``Dressing Funny'') --- and now, the debut of his
MasterClass. The subscription service allows viewers access to a
collection of talks hosted by a high-caliber roster of celebrities and
notable public figures, at a price point of \$180 per year.

When organizers came calling nine months ago, Mr. France balked.
``Initially I thought, `Oh, no, I can't, because what on earth am I
going to talk about?''' he said. ``I had never considered myself a
master at anything.'' But he realized teaching a class could spare him
the ``nearly 800 daily'' DMs he gets on Instagram asking for style
advice from fans and celebrities. ``I thought, `Wouldn't it be nice if I
could just say: Watch my MasterClass?'''

Mr. France's lessons are for all genders, body types and degrees of
interest in personal style. The course introduces the idea of a capsule
wardrobe meant for every closet, including (mostly neutral) items like a
leather jacket, a suit, knitwear, white sneakers and simple T-shirts. It
advises on the fit of clothing (skinny jeans: not \emph{just} for the
societally agreed-upon ``skinny'' among us!). And from there, Mr. France
said, you've opened up the door to thousands of combinations; anything
beyond the initial wardrobe (brighter colors, ``trendier'' designs) are
personalized sprinkles on top of your newfound sense of style.

\hypertarget{dont-do-this-i-dont-like-that}{%
\subsection{`Don't Do This. I Don't Like
That.'}\label{dont-do-this-i-dont-like-that}}

Growing up the youngest of five siblings in northern England, Mr. France
first developed a love for style while visiting his grandfather's denim
factory, where he'd press patterns of Minnie Mouse onto jean jackets and
wear them around the house.

``I'm from a Muslim family, and we wear modest clothing,'' he said. ``We
do not show much of our bodies. We don't wear tight clothes. But I
wanted to feel sexy and desirable. I was sick of blending in and feeling
invisible. So I started dressing the way I wanted to dress.''

He worked for companies like Zara and Selfridges, where he learned how
to properly order product and run department stores, before creating his
own line of stylish but modest clothing geared toward Mormon women at
age 26. Quickly, fast-fashion brands started knocking off his designs,
so Mr. France decided to beat them to the punch, starting a second line
that buoyed his business. ``I've seen the reward,'' he said. ``I've seen
that the way I put an outfit together dictates my mood and the way
people view me.''

Though he was living out of a West Hollywood hotel when we connected
this month, Mr. France normally splits his time between Utah (where he
met his husband, Rob France, on vacation in Salt Lake City in 2008) and,
soon, a nearly-refurbished home in the Hollywood Hills he purchased
earlier this year.

But moving to Tinseltown doesn't mean he's comfortably settled into
fame. Before ``Queer Eye,'' he'd planned to retire from his apparel
brands to focus on his family. If anything, even when he's out getting
coffee, ``I feel the pressure to make sure I look like a style star,''
he said. ``And nobody can keep up with that all day, every day.''

In his memoir, Mr. France details a hectic, high-stakes stretch of his
life --- five years ago, at the apex of his design work --- during which
he nearly overworked himself into the ground. ``I began to feel
suicidal,'' he wrote. ``Every day on my drive to or from work, I would
start to fantasize about driving into oncoming traffic. I cried in my
car every day, thinking I just wanted to take the easy way out.''

Image

``I don't do anything just because the audience might love it,'' Mr.
France said of his role on ``Queer Eye''. ``I'm doing it because I want
the hero to love it.Credit...Phylicia J. L. Munn for The New York Times

Instead, with his husband's help, he sold his share in his businesses,
planning to rely on his savings and consulting work to provide for their
future children. Shortly thereafter, ``Queer Eye'' producers asked him
to audition for the show. After several rounds of casting calls and
chemistry tests, Mr. France was hired, and quickly decamped to Atlanta
for a 16-episode, two-season shoot.

His role is also one of the most clearly defined, based less on
platitudes and more on replicable tips and visible transformations.
Audiences have, perhaps unfairly, over-distilled Mr. France to his
onscreen love of patterned shirts (``out of 20 outfits, only four of
them were floral'' he wrote last year) and, of course, the often
imitated, never properly duplicated French tuck. ``I don't do anything
just because the audience might love it,'' he said. ``I'm doing it
because I want the hero to love it.''

In his five seasons on ``Queer Eye,'' his candor and warmth have made
him a favorite among fans and the show's heroes alike. ``He's a little
judgmental, like all of us are, but not in a way that would make anybody
feel bad,'' Mr. Porowski said. ``If he thinks something is ridiculous
--- like wearing sweatpants with holes --- he's going to tell you. He's
not the type who's going to be like, `OK, well if that works for you.'
It's like, `No, do better. You can do better. You have a responsibility
to do better.'''

``He's just a caring, loving sweetheart of a human being,'' Pete
Davidson said. He's given the ``Saturday Night Live'' star confidence in
what to wear. ``I'm a shy person,'' Mr. Davidson said, ``and Tan has
helped me try things out of my comfort zone.''

Mr. France is quick to dismantle any preconceived notions about success
and stardom for anyone who asks. ``I'm very Asian --- we don't know how
to filter the things that we desperately want to say,'' he said with a
laugh. ``The things I say are very frank: `Don't do this. I don't like
that. Do this instead.' If you don't agree, so be it. This is just my
opinion.''

But when filming the first season of ``Queer Eye,'' he feared that any
missteps --- a hero mis-measuring him or herself before the shoot, a
store refusing to permit filming at the last-minute, a tailor not
altering quickly enough for the final reveal --- would reflect poorly on
him.

``Season 1 and 2 were really hard because I was so new,'' he said.
``Then the show took off, and it gave me more power. Now I'm very happy
saying, `If I don't want to do something, I'm not doing it.'''

Image

Mr. France has spun his newfound fame into ample opportunity, including
a Netflix competition series co-hosted by Alexa Chung ``Next in
Fashion.''Credit...Netflix

\hypertarget{speaking-up}{%
\subsection{Speaking Up}\label{speaking-up}}

With five seasons under his belt and a 2020 Emmy nomination for
outstanding host (an honor he shares with his four co-stars), Mr. France
said his power and opinion are both respected on the show and off. That
newfound sense of authority has also allowed him to publicly engage in
the political conversation for the first time as a celebrity, joining
Jill Biden with his castmates for a fund-raiser for a
\href{https://www.nytimes3xbfgragh.onion/interactive/2020/us/elections/joe-biden.html}{Joseph
R. Biden Jr.} in late August.

``I am a representative of so many things that have been so negatively
portrayed over the last three and half years,'' he said during the
event. ``We want to be respected. Muslims want to be respected, people
of color want to be respected, the Black community wants to be
respected, gay people want to be respected, trans people want to be
respected.''

It's not that he hadn't wanted to speak up before this summer; he very
much did. But he couldn't. When ``Queer Eye'' took off, Mr. France had
just begun the process of getting his U.S. citizenship. ``I really
struggled,'' he said. ``Because people from the L.G.B.T.Q. community to
people from the Muslim community and the South Asian community were
frustrated, saying, `Why aren't you using your platform for more than
just styling tips?'''

``We have a person in power who is a monster and vindictive --- the most
vindictive leader I've ever known in the free world,'' he said. ``But I
was warned by my attorney: `I know that you have things to say, but do
not risk your citizenship by saying something stupid, publicly or
aggressively.'''

For someone paid to offer their opinion necessary, self-preservative
silence stung. ``So, so much was at stake and nobody seemed to
understand. It was a constant question of, `Aren't you going to use your
voice?' I \emph{can't} use my voice,'' he said. ``I'm not willing to
lose everything I've worked for.''

Three months ago, Mr. France officially became a U.S. citizen (he
maintains dual citizenship with his native England) and began making up
for lost time, encouraging his nearly four million Instagram followers
to vote while supporting Mr. Biden's candidacy. That came at a cost. ``I
had a mass unfollowing,'' he said. Thousands of followers (16,000 to be
exact) fled within hours. ``It was smart that they left,'' he said,
``because now I'm getting more involved.''

``Queer Eye'' was one week into filming a new season in Texas when the
pandemic caused production to shut down in March, forcing Mr. France and
his husband to return home, where he estimates he only spent a combined
20 days total between press commitments and shooting last year.

``For me, having an opportunity to reset has been shockingly
beautiful,'' he said of the unexpected time off. ``I thought I could
just keep going and it didn't matter because I loved work. That wasn't
sustainable long-term, so this has actually shown me that my life will
change after this.''

What that change looks like is yet unknown. But Mr. France also knows he
can walk away from it all like he did once before. ``I didn't get into
show business for the money,'' he said. ``I took this job because I had
an agenda. I needed people to see my people --- Muslims, gays,
Pakistanis, immigrants --- as real people, not just characters on a TV
show.''

``So I'm in a very unique position, whereas long as it's great and as
long as it's fun, I will continue,'' Mr. France said. ``But as soon as I
stop enjoying this, I'm more than happy to go and get back to my life.''

Advertisement

\protect\hyperlink{after-bottom}{Continue reading the main story}

\hypertarget{site-index}{%
\subsection{Site Index}\label{site-index}}

\hypertarget{site-information-navigation}{%
\subsection{Site Information
Navigation}\label{site-information-navigation}}

\begin{itemize}
\tightlist
\item
  \href{https://help.nytimes3xbfgragh.onion/hc/en-us/articles/115014792127-Copyright-notice}{©~2020~The
  New York Times Company}
\end{itemize}

\begin{itemize}
\tightlist
\item
  \href{https://www.nytco.com/}{NYTCo}
\item
  \href{https://help.nytimes3xbfgragh.onion/hc/en-us/articles/115015385887-Contact-Us}{Contact
  Us}
\item
  \href{https://www.nytco.com/careers/}{Work with us}
\item
  \href{https://nytmediakit.com/}{Advertise}
\item
  \href{http://www.tbrandstudio.com/}{T Brand Studio}
\item
  \href{https://www.nytimes3xbfgragh.onion/privacy/cookie-policy\#how-do-i-manage-trackers}{Your
  Ad Choices}
\item
  \href{https://www.nytimes3xbfgragh.onion/privacy}{Privacy}
\item
  \href{https://help.nytimes3xbfgragh.onion/hc/en-us/articles/115014893428-Terms-of-service}{Terms
  of Service}
\item
  \href{https://help.nytimes3xbfgragh.onion/hc/en-us/articles/115014893968-Terms-of-sale}{Terms
  of Sale}
\item
  \href{https://spiderbites.nytimes3xbfgragh.onion}{Site Map}
\item
  \href{https://help.nytimes3xbfgragh.onion/hc/en-us}{Help}
\item
  \href{https://www.nytimes3xbfgragh.onion/subscription?campaignId=37WXW}{Subscriptions}
\end{itemize}
