Sections

SEARCH

\protect\hyperlink{site-content}{Skip to
content}\protect\hyperlink{site-index}{Skip to site index}

\href{https://myaccount.nytimes3xbfgragh.onion/auth/login?response_type=cookie\&client_id=vi}{}

\href{https://www.nytimes3xbfgragh.onion/section/todayspaper}{Today's
Paper}

Finding a Place for Third-Culture Kids in the Culture

\url{https://nyti.ms/3bO144F}

\begin{itemize}
\item
\item
\item
\item
\item
\end{itemize}

Advertisement

\protect\hyperlink{after-top}{Continue reading the main story}

Supported by

\protect\hyperlink{after-sponsor}{Continue reading the main story}

\hypertarget{finding-a-place-for-third-culture-kids-in-the-culture}{%
\section{Finding a Place for Third-Culture Kids in the
Culture}\label{finding-a-place-for-third-culture-kids-in-the-culture}}

In his new HBO series, the filmmaker Luca Guadagnino revisits a timeless
yet timely question: What does it mean to be from everywhere and nowhere
at once?

\includegraphics{https://static01.graylady3jvrrxbe.onion/images/2020/09/11/t-magazine/entertainment/luca-slide-OCJ2/luca-slide-OCJ2-articleLarge.jpg?quality=75\&auto=webp\&disable=upscale}

By Noor Brara

\begin{itemize}
\item
  Sept. 11, 2020
\item
  \begin{itemize}
  \item
  \item
  \item
  \item
  \item
  \end{itemize}
\end{itemize}

On a blanched, sun-baked afternoon, two teenagers, a boy and a girl,
wander into a grocery store to pick up lunch. Fraser is a recent
transplant from New York, and Britney a new friend who has lived her
life evenly between South Korea, Germany and Italy, though you'd never
know it by her American drawl or the pop music she blares through her
headphones. To the viewer, the scene presents like quotidian life in the
United States --- but for the fact that it takes place in Veneto, Italy,
on a military base where families work and attend school, their children
running off every evening to dance and drink by the cerulean sea
alongside their friends from town with whom they scheme and share
secrets, whispered in fluent Italian. In a few years, many of them will
ready themselves for a move --- to another home on another military base
in another country, with a supermarket configured to look exactly like
this one. ``They look the same so you don't feel lost,'' Britney tells
Fraser. ``Do you ever feel lost?'' he asks. She shrugs.

The idea that a sense of belonging is challenged by the straddling of
cultures is hardly a revelation; nearly every maker whose back story was
shaped by more than one place has arrived at some version of that
conclusion. But rarely do we hear the stories of so-called
``third-culture kids'' and the private, nomadic worlds in which they are
raised, marked by a certain shared disorientation and the sense that
home is everywhere and nowhere at once. It's for this reason that the
Italian director
\href{https://www.nytimes3xbfgragh.onion/2018/09/19/t-magazine/luca-guadagnino-interior-design-lake-como-house.html}{Luca
Guadagnino} will attempt to unpack one iteration of this experience ---
through Fraser, Britney and their five best friends --- in
``\href{https://www.nytimes3xbfgragh.onion/2020/09/10/arts/television/luca-guadagnino.html}{We
Are Who We Are},'' an eight-part series premiering this September on HBO
that pulls back the curtain on the experiences of the children of
military families abroad and other third-culture kids like them, whose
place in the world now feels both more tenuous and important than ever
before.

\includegraphics{https://static01.graylady3jvrrxbe.onion/images/2020/09/11/t-magazine/entertainment/luca-slide-NX7J/luca-slide-NX7J-articleLarge.jpg?quality=75\&auto=webp\&disable=upscale}

Coined by the American sociologist Ruth Useem in the 1950s, the term
``\href{https://www.nytimes3xbfgragh.onion/2018/10/09/smarter-living/the-edit-moving-alot.html}{third-culture
kid}'' was conceived for expatriate children who spend their formative
years overseas, shaped by the multicultural, peripatetic spheres of
their parents, many of whom are diplomats, military members or others
working in foreign service. They relocate frequently and enroll their
children in international schools, exposing them to miniature realms
cultivated by peers from nations far and wide, whose customs, languages
and mores coalesce, birthing hybrid or ``third'' cultures that are
globe-spanning, diverse, highly empathic and oftentimes difficult to
translate outside these environments.

Perhaps because this life is characteristically slippery, it's struggled
to become clearly defined in the culture, even in fictional stories,
suited though they are to crafting imagined worlds. Ironically, while
most TCKs cite the ability to relate to nearly everyone, their own
narratives suffer a relatability problem, perhaps because their youthful
experiences, relegated wholly to remembrance and recollection, are in
many ways too singular and strange-seeming to others. Still, there are
characters that have managed to catch hold, the complexities of their
placelessness often anchored to more universal quandaries: Elio Perlman,
played by Timothée Chalamet in Guadagnino's 2017 film adaptation of
André Aciman's
``\href{https://www.nytimes3xbfgragh.onion/2017/11/22/movies/call-me-by-your-name-review-armie-hammer.html}{Call
Me By Your Name}'' is one such example; a trilingual adolescent reared
in the university orbit between the United States and Northern Italy ---
his father is from the former, his mother the latter --- he casts his
American and European identities on and off with a kind of begrudging
ease, lording them over his father's visiting graduate student, Oliver
(Armie Hammer), on some days, while on others he's consumed by a sort of
languid estrangement from everyone around him, retreating into himself.
Though the story is propelled forward by the unfurling of muffled desire
and fleeting boyhood, it's hard not to notice how a defined cultural
identity --- or lack thereof --- inevitably underscores Elio's
coming-of-age, as he pursues different versions of himself in different
relationships: in English with Oliver, in French and Italian with his
girlfriend Marzia and in all three with his parents, code-switching in
what feels like a futile attempt to stitch together facets of a
fractured self.

Image

Timothée Chalamet in ``Call Me By Your Name.''Credit...Sony Pictures
Classics, via Associated Press

Of course, how Elio conveys this onscreen may have more to do with
Guadagnino himself, who has long constructed his complex, layered
characters partly in his own image. ``That's me,'' he says immediately
over Zoom in August, when I read off Useem's definition of a ****
third-culture kid. ``I was born in Palermo, and moved almost right away
to Ethiopia. I spent the first six years of my life there. Then we went
to Rome, then Palermo again and then back to Rome, then to Milan and to
London. I feel the most important aspect of being a filmmaker is to be
really aware of what forms you as much as what's in front of you. So, I
always try to keep in mind what I could have been experiencing during my
youth in all these places through the prism of these complex stories I
tell.''

If asked, any third-culture kid will tell you that shape-shifting ---
rousing one of the many selves stacked within you to best suit the place
you're in --- becomes a necessary survival skill, a sort of feigned
fitting in that allows you to relate something of yourself to nearly
everyone you meet. As someone raised between New York and the diplobrat
bubble of an international school in New Delhi, India, where friends
would come and go every few years, I became adept at calibrating myself
to find **** the points of connection between us, able to relate equally
to someone from South Korea, Iceland, Japan, Italy or Jamaica, in many
cases more so than to other Indian Americans whose lives, at least on
paper, read closer to my own. And because our **** stories couldn't be
gleaned from our outward appearances, accents or possessions, we all
came humble to the table, open and permeable and ready to barter the
surfaces of our souls: our learnings, our languages, our cuisines, our
clothing.

While all of this contributed, certainly, to feeling perennially adrift
(according to multiple studies by Useem and others, much as they may
try, adult TCKs never wholly repatriate culturally), it blotted the
sensation of feeling like we'd ``grown up at an angle to everywhere and
everyone,'' as the writer Pico Iyer --- of Indian parentage, raised
between England and California, who now lives between the latter and
Japan --- told me during a recent phone conversation. In his own work,
Iyer has spent a lifetime examining this feeling and others that result
from cultural crisscrossing, both out in the world in
``\href{https://archive.nytimes3xbfgragh.onion/www.nytimes3xbfgragh.onion/books/97/08/10/bookend/9026.html}{Video
Night in Kathmandu}\emph{,}'' ** a 1988 collection of essays which
examines the unlikely cultural points at which East and West meet across
Asia \emph{---} Japan's affinity for baseball, say, or the Philippines'
obsession with country and western music --- and then in
``\href{https://www.penguinrandomhouse.com/books/85777/the-global-soul-by-pico-iyer/}{The
Global Soul},'' written twelve years later, which studied, conversely,
the crisscrossings that take place within. Iyer found peace in accepting
that belonging had little to do with geography, but rather a collection
of personal interests, ideas and relationships accumulated over time.
``Growing up with three cultures around or inside me, I felt that I
could define myself by my passions, not my passport,'' he says. ``In
some ways, I would never be Indian or English or Californian, and that
was quite freeing, though people may always define me by my skin color
or accent. But also, because I didn't have that external way of defining
myself, I had to be really rigorous and directed in grounding myself
internally, through my values and loyalties and to the people I hold
closest to me.''

Image

Others have found freedom in the same, becoming natural shape-shifters
whose value systems transcend borders to instill a sense of home. The
most famous example is probably Barack Obama, whose 1995 memoir, **
``\href{https://www.nytimes3xbfgragh.onion/2009/01/19/books/19read.html}{Dreams
From My Father: A Story of Race and Inheritance,}'' ** whirls through
Jakarta, Seattle, Kenya and Hawaii with unsparing analysis of what it
means to belong to multiple worlds and therefore to none of them, but to
find, later, that refuge lies in the space between all of them --- and
in the ability to unite not just your worlds but others', too. As much
as the third-culture experience is clouded by the fog of liminality,
it's informed also by the ability to define oneself on one's own terms,
difficult as that endeavor may be in the face of increasing scrutiny
toward globalism and those formed by it.

The presentation of this --- dazzling and dressed up --- is what makes
``We Are Who We Are'' ** thrilling to watch. Its characters come alive
in the blur, filling in one another's spaces and dancing over questions
of home, while bragging about where they've been, their exchanges
captured in shimmering, slow-motion interludes scored to original music,
the silky synth pop of Blood Orange. And while the show takes place in
the run-up to the 2016 election, its politics remain a quiet drumbeat in
the offing, its spotlight focused wholly on all the ways by which
differences are, in fact, paradoxically harmonious when everyone is
otherized. In fashioning themselves to evade traditional modes of
identification (culturally, politically, sexually and through gender),
these characters build their own castles in the sky. ``When you grow up
this way, there is a feeling of being lost, but to be lost is also to be
open,'' Guadagnino says. ``It reminds us of our empathy, and of what we
share if we were only to try and find it.''

This may be the ultimate lesson of third-culture kids' stories. In the
late \href{https://www.nytimes3xbfgragh.onion/spotlight/kobe}{Kobe
Bryant}'s 2018 book ``The Mamba Mentality,'' which offers a glimpse into
his childhood years in Reggio Emilia, Italy, he discusses the importance
of having learned how to navigate a new culture with compassion. Though
he eventually settled down in America --- becoming not only one of its
sports heroes, but **** one of its cultural icons, too --- he continued
to make frequent trips back to Italy, where he'd speak the sort of
Italian that boasted a native European bravado, a casual swagger that
rode along his perfect pronunciation. And when he died in Los Angeles,
he died in Reggio Emilia, too, where they mourned a version of him
America never knew, except for the Italian names he had chosen for his
daughters: Gianna, Natalia, Bianka and Capri.

Image

Of course, not all depictions of third-culture life have been so
uplifting. Occasionally, too, these characters **** are written to be
spoofed and ridiculed, assigned snobbish attitudes and superiority
complexes. Without proper context, it can appear as if they need too
much and require a sort of excess to keep them perpetually moving,
making it hard to divorce third-culture life from that of overt wealth
and privilege, or an indifference to local customs. In the 2018 Netflix
show
``\href{https://www.nytimes3xbfgragh.onion/2019/01/24/arts/television/penn-badgley-you-netflix.html}{You},''
the model-actress Hari Nef portrays Blythe, a third-culture poet prodigy
whose parents worked for the state department and raised her between
Papua New Guinea and Tokyo. When the central character, Beck --- a
timid, hopeful writer played by Elizabeth Lail --- meets her, she looks
her up and down and smirks before asking, ``Jersey, right?'' and runs
off to take a call from her grandparents in Swedish. In the
third-culture writer Stephanie LaCava's forthcoming novel,
``\href{https://mitpress.mit.edu/books/superrationals}{The
Superrationals},'' which dives into the torrid waters of the
international art world, the protagonist Mathilde, raised between the
U.S. and France, is ridiculed relentlessly by ``the girls,'' a catty
clique of gallery insiders who dislike her for all the ways in which
she's different (``What \emph{is} that name?'' they ask. ``Is she even
French? She's so pretentious''). And in 2010's ** ``Sidewalks,'' a
razor-sharp collection of essays about the failures of finding home in
lived experiences and written ones alike,
\href{https://www.nytimes3xbfgragh.onion/2019/02/07/arts/valeria-luiselli-lost-children-archive.html}{Valeria
Luiselli} --- the author of the 2019 novel ``Lost Children Archive'' and
the daughter of a Mexican diplomat formed by an upbringing in Costa
Rica, South Korea, India and South Africa --- sarcastically comments on
her own selection of Mexico as ``her country,'' driven mostly by
cynicism and ``a sort of spiritual laziness than an authentic act of
faith.'' She admits she's never felt true allegiance to anywhere she's
lived, knowing only that she must continue roaming.

But all these stories, of course, predate the precarious state we find
ourselves in today, when borders are clamping down in domino effect,
driven in part by the Covid-19 pandemic, itself a case against globalism
and the speed at which interconnectedness can burn it all down,
imperiling not only our ability to travel but limiting those who find
selfhood in marginal spaces, whose stories underscore the urgency of
seeing the world as one. And while internationalism deserves
examination, what we stand to lose without it is our ability to lift one
another up, to find each other in the in-between. One might look to
Kamala Harris --- who, born to Jamaican and Indian parents, often
discusses her ability to consider multiple sides --- or Obama before
her. Such voices, with their chameleonic stories and sensibilities, help
locate the light in the dark.

Advertisement

\protect\hyperlink{after-bottom}{Continue reading the main story}

\hypertarget{site-index}{%
\subsection{Site Index}\label{site-index}}

\hypertarget{site-information-navigation}{%
\subsection{Site Information
Navigation}\label{site-information-navigation}}

\begin{itemize}
\tightlist
\item
  \href{https://help.nytimes3xbfgragh.onion/hc/en-us/articles/115014792127-Copyright-notice}{©~2020~The
  New York Times Company}
\end{itemize}

\begin{itemize}
\tightlist
\item
  \href{https://www.nytco.com/}{NYTCo}
\item
  \href{https://help.nytimes3xbfgragh.onion/hc/en-us/articles/115015385887-Contact-Us}{Contact
  Us}
\item
  \href{https://www.nytco.com/careers/}{Work with us}
\item
  \href{https://nytmediakit.com/}{Advertise}
\item
  \href{http://www.tbrandstudio.com/}{T Brand Studio}
\item
  \href{https://www.nytimes3xbfgragh.onion/privacy/cookie-policy\#how-do-i-manage-trackers}{Your
  Ad Choices}
\item
  \href{https://www.nytimes3xbfgragh.onion/privacy}{Privacy}
\item
  \href{https://help.nytimes3xbfgragh.onion/hc/en-us/articles/115014893428-Terms-of-service}{Terms
  of Service}
\item
  \href{https://help.nytimes3xbfgragh.onion/hc/en-us/articles/115014893968-Terms-of-sale}{Terms
  of Sale}
\item
  \href{https://spiderbites.nytimes3xbfgragh.onion}{Site Map}
\item
  \href{https://help.nytimes3xbfgragh.onion/hc/en-us}{Help}
\item
  \href{https://www.nytimes3xbfgragh.onion/subscription?campaignId=37WXW}{Subscriptions}
\end{itemize}
