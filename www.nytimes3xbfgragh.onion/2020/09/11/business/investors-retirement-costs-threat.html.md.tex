Sections

SEARCH

\protect\hyperlink{site-content}{Skip to
content}\protect\hyperlink{site-index}{Skip to site index}

\href{https://www.nytimes3xbfgragh.onion/section/business}{Business}

\href{https://myaccount.nytimes3xbfgragh.onion/auth/login?response_type=cookie\&client_id=vi}{}

\href{https://www.nytimes3xbfgragh.onion/section/todayspaper}{Today's
Paper}

\href{/section/business}{Business}\textbar{}Even the Threat of a Tougher
Rule on Financial Advice Has Helped Investors

\url{https://nyti.ms/3m9093s}

\begin{itemize}
\item
\item
\item
\item
\item
\end{itemize}

Advertisement

\protect\hyperlink{after-top}{Continue reading the main story}

Supported by

\protect\hyperlink{after-sponsor}{Continue reading the main story}

strategies

\hypertarget{even-the-threat-of-a-tougher-rule-on-financial-advice-has-helped-investors}{%
\section{Even the Threat of a Tougher Rule on Financial Advice Has
Helped
Investors}\label{even-the-threat-of-a-tougher-rule-on-financial-advice-has-helped-investors}}

The Obama administration's efforts to require firms to truly work in the
interests of investors has already given people better choices, a new
study says.

\includegraphics{https://static01.graylady3jvrrxbe.onion/images/2020/09/13/business/11Strategies-illo/11Strategies-illo-articleLarge.jpg?quality=75\&auto=webp\&disable=upscale}

\href{https://www.nytimes3xbfgragh.onion/by/jeff-sommer}{\includegraphics{https://static01.graylady3jvrrxbe.onion/images/2018/02/20/multimedia/author-jeff-sommer/author-jeff-sommer-thumbLarge.jpg}}

By \href{https://www.nytimes3xbfgragh.onion/by/jeff-sommer}{Jeff Sommer}

\begin{itemize}
\item
  Sept. 11, 2020
\item
  \begin{itemize}
  \item
  \item
  \item
  \item
  \item
  \end{itemize}
\end{itemize}

A battle over exactly how investors should be treated when they get
financial advice has been underway for years. The Obama administration
pushed for stricter investor protections, while the Trump administration
has been putting looser rules into effect.

But here's the thing: The tougher Obama rules, which were never fully
put into effect, have already done some good.

That's the import of an innovative new study published by the
\href{https://www.hbs.edu/faculty/Publication\%20Files/WP21-018rev8-28-20_d683a777-2c59-4ce5-859d-6baad6af7860.pdf}{Harvard
Business School} and the
\href{https://www.nber.org/papers/w27577\#:~:text=Our\%20results\%20indicate\%20that\%20variable,when\%20dealing\%20with\%20retirement\%20accounts.}{National
Bureau of Economic Research}. It has a technical title --- ``Conflicting
Interests and the Effect of Fiduciary Duty --- Evidence from Variable
Annuities,'' but contains nuggets that are worthy of wider attention.
Its authors are
\href{https://www.hbs.edu/faculty/Pages/profile.aspx?facId=774513}{Mark
Egan}, a Harvard Business School professor;
\href{https://www.stern.nyu.edu/faculty/bio/shan-ge}{Shan Ge}, a
professor at New York University's Leonard N. Stern School of Business;
and \href{https://scholar.harvard.edu/johnnytang/home}{Johnny Tang}, an
economics graduate student at Harvard.

The study says that the threat of stricter regulations alone has
improved at least some of the behavior of brokers and financial service
firms. The improvement began while Barack Obama was still president and
it has continued, despite the deregulatory approach of the Trump
administration. Perhaps because stricter regulations could return in a
new administration, that shift in the financial services industry hasn't
abated yet, Professor Egan said in an interview.

``The interesting thing is that these effects have persisted even though
the fiduciary rule was never enforced,'' Professor Egan said. ``We seem
to be in a holding pattern right now, with these changes holding steady,
given uncertainty over how the rules themselves might change over the
next few years.''

Those changes center on the so-called ``fiduciary rule,'' which would
have banned conflicts of interest in the advice given to people
investing for retirement.

The rule's history is a bit tangled: The Department of Labor proposed
the rule in 2015 and
\href{https://www.nytimes3xbfgragh.onion/2016/04/07/your-money/new-rules-for-retirement-accounts-financial-advisers.html}{formally
announced it in 2016}. But a suit filed by several financial industry
groups, which were represented by
\href{https://www.nytimes3xbfgragh.onion/2020/08/21/business/labor-department-proposal-retirement-planning.html}{Eugene
Scalia}, who is now Labor secretary, succeeded
in\href{https://www.nytimes3xbfgragh.onion/2018/06/22/your-money/fiduciary-rule-dies.html}{dislodging}
the rule before it was ever put in place.

Separate protections for nonretirement accounts also were considered by
the Obama administration but were never put into effect by the
Securities and Exchange Commission.
\href{https://www.nytimes3xbfgragh.onion/2020/07/16/your-money/fiduciary-duty-investments-best-interest.html\#:~:text=A\%20Securities\%20and\%20Exchange\%20Commission\%20rule\%20that\%20took\%20effect\%20on,protections\%20than\%20the\%20rule\%20delivers.}{In
June} the S.E.C. instituted what it calls a ``best interest'' rule that
makes it easier to charge for advice that may lead you to make
investments that aren't the cheapest or the best.

Now, at the Labor Department, Mr. Scalia is
\href{https://www.nytimes3xbfgragh.onion/2020/08/21/business/labor-department-proposal-retirement-planning.html}{moving
rapidly} to change the rules for retirement accounts and to impose new
regulations that would be weaker than those intended by the original
fiduciary rule.

Amid all of these regulatory battles, you may think that investor
protection under the fiduciary rule is a lost cause. But that's not the
case.

The study found something startling: The financial services industry
began to respond to the fiduciary rule's requirements --- even though
the rule has never gone into effect.

The study dealt specifically with a limited part of the financial
marketplace but it has broader implications. The researchers focused on
the sale of variable annuities, which are
\href{https://www.nytimes3xbfgragh.onion/2015/06/20/your-money/variable-annuities-with-guaranteed-income-riders-require-careful-scrutiny.html}{notorious}
for frequently carrying high fees and onerous restrictions that generate
hefty profits for insurers and brokers while eating away at the
long-term returns of retirement investors.

\href{https://www.nytimes3xbfgragh.onion/2019/07/01/business/rolling-stones-social-security-retirement.html}{Annuities}
can be a valuable part of a retirement portfolio when they are low-cost
and transparent. Simple immediate annuities or deferred annuities can
provide steady, guaranteed income for many years. And Social Security,
the most
\href{https://www.nytimes3xbfgragh.onion/2020/05/29/business/Social-Security-benefits-shortfall-coronavirus.html}{valuable}
asset most Americans have in retirement, is an annuity --- one that is
backed by the government.

But variable annuities are different. They are complex products, which
can rise and fall based on the value of underlying investments in the
stock market. Brokers who sell them to investors have sometimes received
substantial commissions of more than 10 percent from insurers, and these
annuities frequently contain clauses and restrictions that can cost
investors dearly, the S.E.C.
\href{https://www.sec.gov/investor/pubs/sec-guide-to-variable-annuities.pdf}{warns}.

It is difficult to argue that high-expense variable annuities are the
best possible option for most investors, though they are a fine source
of income for individual brokers. In other words, a high-commission
variable annuity is the kind of investment that a fiduciary --- someone
who puts a client's interest above their own --- might not recommend.
Such annuities are associated with a high frequency of consumer
\href{https://brokercheck.finra.org/}{complaints} to the Financial
Industry Regulatory Authority. That's why the researchers shined a
spotlight on them.

From 2015 to 2016, a year in which the proposed fiduciary
\href{https://www.nytimes3xbfgragh.onion/2015/04/08/opinion/successful-investing-for-the-long-haul.html}{rule}
was in the news, many financial services companies began revising
practices that might run afoul of it. They adjusted their sales
practices or restructured their product lines, the study found, reducing
compensation incentives that encouraged brokers to steer investors to
costly products. The paper cited several companies, including Voya, Axa
Group, Aegon and Lincoln Financial.

In that one year, the report said, sales of high-expense variable
annuities fell by 52 percent, while sales of lower-expense annuities
surged. What's more, there was no indication that investors were
receiving less advice, as the
\href{https://www.sifma.org/resources/news/sifma-submits-comments-and-new-evidence-of-the-dol-fiduciary-rules-negative-impact-on-retirement-savers/}{industry
groups} that have fought the fiduciary rule have frequently claimed
would happen. ``Based on our structural model estimates, investor
welfare improved as a result of the fiduciary rule under conservative
assumptions,'' the paper concluded.

Professor Egan said he has been monitoring recent sales data and has
found that ``as far as the latest numbers show, the industry is still
holding back, and the effect of the fiduciary rule can still be seen ---
even though the rule hasn't been put into place.''

It is possible, of course, that the market itself is responsible for
some of the continuing improvement. After all, the fiduciary rule has
been widely debated since 2015 and has focused public attention on the
advantages of low-cost, no frills investments that leave more money in
the pockets of consumers and less in the coffers of big corporations. No
doubt, many consumers are seeking simple, economical options like index
funds and staying away from higher-cost variable annuities.

But as Professor Egan points out, knowledge in a broad population is
always ``asymmetrical,'' and financial predators will find opportunity
for outsize profits if regulators are shackled or sleeping.

I'm all for the creativity unleashed by the profit motive in a market
economy. But one way or another, society must impose some limits.
Especially when it comes to the money that working people have salted
away for retirement, it strikes me as cruel and uncivilized to permit
exploitation. That, in a nutshell, is what government is for,
particularly amid the immense suffering of a pandemic.

Protecting Social Security is one thing that can be done. Another is
bringing attention to shoddy practices that hurt savers. Shining a
bright light can help, the new paper suggests, even if strong
regulations and active regulators are really needed to deal with the
problem.

Advertisement

\protect\hyperlink{after-bottom}{Continue reading the main story}

\hypertarget{site-index}{%
\subsection{Site Index}\label{site-index}}

\hypertarget{site-information-navigation}{%
\subsection{Site Information
Navigation}\label{site-information-navigation}}

\begin{itemize}
\tightlist
\item
  \href{https://help.nytimes3xbfgragh.onion/hc/en-us/articles/115014792127-Copyright-notice}{©~2020~The
  New York Times Company}
\end{itemize}

\begin{itemize}
\tightlist
\item
  \href{https://www.nytco.com/}{NYTCo}
\item
  \href{https://help.nytimes3xbfgragh.onion/hc/en-us/articles/115015385887-Contact-Us}{Contact
  Us}
\item
  \href{https://www.nytco.com/careers/}{Work with us}
\item
  \href{https://nytmediakit.com/}{Advertise}
\item
  \href{http://www.tbrandstudio.com/}{T Brand Studio}
\item
  \href{https://www.nytimes3xbfgragh.onion/privacy/cookie-policy\#how-do-i-manage-trackers}{Your
  Ad Choices}
\item
  \href{https://www.nytimes3xbfgragh.onion/privacy}{Privacy}
\item
  \href{https://help.nytimes3xbfgragh.onion/hc/en-us/articles/115014893428-Terms-of-service}{Terms
  of Service}
\item
  \href{https://help.nytimes3xbfgragh.onion/hc/en-us/articles/115014893968-Terms-of-sale}{Terms
  of Sale}
\item
  \href{https://spiderbites.nytimes3xbfgragh.onion}{Site Map}
\item
  \href{https://help.nytimes3xbfgragh.onion/hc/en-us}{Help}
\item
  \href{https://www.nytimes3xbfgragh.onion/subscription?campaignId=37WXW}{Subscriptions}
\end{itemize}
