Sections

SEARCH

\protect\hyperlink{site-content}{Skip to
content}\protect\hyperlink{site-index}{Skip to site index}

\href{https://www.nytimes3xbfgragh.onion/section/climate}{Climate}

\href{https://myaccount.nytimes3xbfgragh.onion/auth/login?response_type=cookie\&client_id=vi}{}

\href{https://www.nytimes3xbfgragh.onion/section/todayspaper}{Today's
Paper}

\href{/section/climate}{Climate}\textbar{}California's Air Quality Is
Poor. Here's How to Protect Yourself.

\url{https://nyti.ms/35pRsvD}

\begin{itemize}
\item
\item
\item
\item
\item
\item
\end{itemize}

\hypertarget{wildfires-in-the-west}{%
\subsubsection{\texorpdfstring{\href{https://www.nytimes3xbfgragh.onion/spotlight/california-wildfires?name=styln-california-wildfires\&region=TOP_BANNER\&block=storyline_menu_recirc\&action=click\&pgtype=Article\&impression_id=55b37860-f52d-11ea-8682-bfbfb46aa2bd\&variant=undefined}{Wildfires
in the West}}{Wildfires in the West}}\label{wildfires-in-the-west}}

\begin{itemize}
\tightlist
\item
  live\href{https://www.nytimes3xbfgragh.onion/2020/09/12/us/wildfires-live-updates.html?name=styln-california-wildfires\&region=TOP_BANNER\&block=storyline_menu_recirc\&action=click\&pgtype=Article\&impression_id=55b37861-f52d-11ea-8682-bfbfb46aa2bd\&variant=undefined}{Fires
  Updates}
\item
  \href{https://www.nytimes3xbfgragh.onion/interactive/2020/us/fires-map-tracker.html?name=styln-california-wildfires\&region=TOP_BANNER\&block=storyline_menu_recirc\&action=click\&pgtype=Article\&impression_id=55b37862-f52d-11ea-8682-bfbfb46aa2bd\&variant=undefined}{Maps
  of the Fires}
\item
  \href{https://www.nytimes3xbfgragh.onion/article/wildfires-photos-california-oregon-washington-state.html?name=styln-california-wildfires\&region=TOP_BANNER\&block=storyline_menu_recirc\&action=click\&pgtype=Article\&impression_id=55b37863-f52d-11ea-8682-bfbfb46aa2bd\&variant=undefined}{Photos}
\item
  \href{https://www.nytimes3xbfgragh.onion/2020/09/10/us/climate-change-california-wildfires.html?name=styln-california-wildfires\&region=TOP_BANNER\&block=storyline_menu_recirc\&action=click\&pgtype=Article\&impression_id=55b37864-f52d-11ea-8682-bfbfb46aa2bd\&variant=undefined}{A
  Climate Reckoning}
\item
  \href{https://www.nytimes3xbfgragh.onion/article/wildfires-california-oregon-washington.html?name=styln-california-wildfires\&region=TOP_BANNER\&block=storyline_menu_recirc\&action=click\&pgtype=Article\&impression_id=55b37865-f52d-11ea-8682-bfbfb46aa2bd\&variant=undefined}{Answers
  to Your Questions}
\item
  \href{https://www.nytimes3xbfgragh.onion/2020/09/09/us/california-wildfires.html?name=styln-california-wildfires\&region=TOP_BANNER\&block=storyline_menu_recirc\&action=click\&pgtype=Article\&impression_id=55b37866-f52d-11ea-8682-bfbfb46aa2bd\&variant=undefined}{Newsletter}
\end{itemize}

Advertisement

\protect\hyperlink{after-top}{Continue reading the main story}

Supported by

\protect\hyperlink{after-sponsor}{Continue reading the main story}

\hypertarget{californias-air-quality-is-poor-heres-how-to-protect-yourself}{%
\section{California's Air Quality Is Poor. Here's How to Protect
Yourself.}\label{californias-air-quality-is-poor-heres-how-to-protect-yourself}}

Wildfire smoke spreads misery, including health problems, far beyond
fire zones. We have key facts and tips.

\includegraphics{https://static01.graylady3jvrrxbe.onion/images/2020/09/12/climate/12CLI-AIRQUALITY1/merlin_176779317_691c556f-d7aa-4b09-82c0-7d94f66502c0-articleLarge.jpg?quality=75\&auto=webp\&disable=upscale}

\href{https://www.nytimes3xbfgragh.onion/by/nicole-perlroth}{\includegraphics{https://static01.graylady3jvrrxbe.onion/images/2018/02/20/multimedia/author-nicole-perlroth/author-nicole-perlroth-thumbLarge.jpg}}\href{https://www.nytimes3xbfgragh.onion/by/john-schwartz}{\includegraphics{https://static01.graylady3jvrrxbe.onion/images/2018/02/16/multimedia/author-john-schwartz/author-john-schwartz-thumbLarge.jpg}}

By \href{https://www.nytimes3xbfgragh.onion/by/nicole-perlroth}{Nicole
Perlroth} and
\href{https://www.nytimes3xbfgragh.onion/by/john-schwartz}{John
Schwartz}

\begin{itemize}
\item
  Sept. 11, 2020
\item
  \begin{itemize}
  \item
  \item
  \item
  \item
  \item
  \item
  \end{itemize}
\end{itemize}

SAN FRANCISCO --- Jennifer Krasner's 4-year-old daughter had been
coughing for days. Ms. Krasner and her family live 20 minutes north of
San Francisco, in Mill Valley, Calif., not close to any fire but
wreathed in smoke nonetheless, with her house and car dusted with ash.

``I had to get her tested for Covid because she's been coughing so
much,'' she said Thursday, ``but it turned out her lungs were just
irritated from all the smoke.''

Across San Francisco Bay to the southeast, in Alameda, Monica Chellam's
daughter, also 4, asked Wednesday why it was so dark. ``I told her the
sun was blocked by smoke,'' Ms. Chellam said.

``She turned to me and asked, `Is this how the dinosaurs died?'''

Children aren't the only ones coughing. And they're not the only ones
with questions about the smoke that is spreading misery around the West.
Here are some key facts and tips on what you can do.

\hypertarget{how-much-can-smoke-affect-your-health}{%
\subsubsection{How much can smoke affect your
health?}\label{how-much-can-smoke-affect-your-health}}

The health effects of wildfire smoke are not fully understood, and the
particles differ in some ways from other air pollution, which has been
shown to cause disease. But wildfire smoke, which can include toxic
substances from burned buildings, has been linked to serious health
problems.

``When this is happening people's health is suffering,'' said Sarah
Henderson, senior scientist in environmental health services at the
British Columbia Center for Disease Control. ``There is no doubt.''

Studies have shown that, when waves of smoke hit,
\href{https://insights.ovid.com/epidemiology/epide/2017/01/000/wildfire-specific-fine-particulate-matter-risk/13/00001648}{the
rate of hospital visits rises} and many of the additional patients
experience
\href{https://www.ncbi.nlm.nih.gov/pmc/articles/PMC6015400/}{respiratory
problems, heart attacks and strokes}.

Dr. Henderson said smoke exposure could have lifelong health
implications for babies, though she said more research on the question
was needed. ``This may do damage to the developing lungs that they may
never recover from,'' she said.

The risks are greater for people of color, who tend to live in areas
already exposed to high levels of particulate pollution. According to a
\href{https://academic.oup.com/aje/article/186/6/730/3836014}{2017
study}, older Black people are three times more likely to be
hospitalized for respiratory conditions because of smoke.

Francesca Dominici, a professor of biostatistics at Harvard and an
author of the study, said, ``Underrepresented minorities are
experiencing a much higher health burden from pollution and wildfire
smoke and, now, Covid.''

The coronavirus pandemic, which has also hit people of color
disproportionately, adds further problems. The Centers for Disease
Control and Prevention has warned that ``people with Covid-19 are at
\href{https://www.cdc.gov/disasters/covid-19/wildfire_smoke_covid-19.html}{increased
risk from wildfire smoke during the pandemic}.''

\includegraphics{https://static01.graylady3jvrrxbe.onion/images/2020/09/11/climate/11CLI-AIRQUALITY3/merlin_176128734_c7e73023-c09e-4d42-a27a-5adc34f3b919-articleLarge.jpg?quality=75\&auto=webp\&disable=upscale}

And the health effects of wildfire smoke don't go away when skies clear.
A
\href{https://www.sciencedirect.com/science/article/pii/S0160412019326935}{recent
study on Montana residents} suggested a long tail for wildfire smoke
exposure.

Erin Landguth, an associate professor in the school of public and
community health science at the University of Montana and the lead
author on the study, said research had shown that ``after bad fire
seasons, one would expect to see three to five times worse flu seasons''
months later. The study's findings, she added, fit what is already known
about pollution and disease.

\href{https://www.nytimes3xbfgragh.onion/spotlight/california-wildfires}{Wildfires
in the West ›}

\hypertarget{live-updates}{%
\subsection{\texorpdfstring{\href{https://www.nytimes3xbfgragh.onion/2020/09/12/us/wildfires-live-updates.html}{Live
Updates}}{Live Updates}}\label{live-updates}}

Updated~

Sept. 12, 2020, 2:53 p.m. ET

\begin{itemize}
\tightlist
\item
  \href{https://www.nytimes3xbfgragh.onion/2020/09/12/us/wildfires-live-updates.html\#link-f3961ff}{President
  Trump will visit California on Monday after destructive fires.}
\item
  \href{https://www.nytimes3xbfgragh.onion/2020/09/12/us/wildfires-live-updates.html\#link-7e503ae9}{Shifting
  weather may improve firefighting conditions on the West Coast.}
\item
  \href{https://www.nytimes3xbfgragh.onion/2020/09/12/us/wildfires-live-updates.html\#link-5e4c548d}{Oregon's
  fire marshal is temporarily replaced as firefighters battle blazes.}
\end{itemize}

``Decades of research have shown that elevated air pollution exposure is
associated with a number of adverse health impacts, including
compromised immune systems,'' Dr. Landguth said.

\hypertarget{whats-the-climate-connection}{%
\subsubsection{What's the climate
connection?}\label{whats-the-climate-connection}}

The underlying causes of the rising fire risks in the American West are
complex. They include past forestry practices that created abundant fuel
for fires and the expansion of communities up to the edges of
forestlands.

Underlying all of that, however, is climate change, which warms and
dries out the vegetation fuel so that a spark --- whether from
\href{https://www.nytimes3xbfgragh.onion/interactive/2019/03/18/business/pge-california-wildfires.html}{downed
power lines},
\href{https://www.nytimes3xbfgragh.onion/2020/08/24/us/california-fires.html}{lightning}
or even
\href{https://www.nytimes3xbfgragh.onion/2020/09/07/us/gender-reveal-party-wildfire.html}{a
gender-reveal party gone terribly wrong} --- can lead to a vast scorched
landscape.

Even with the most aggressive effort to fight global warming, the
inherent lag time in the climate system means that worsening fires and
their health effects will be with us for decades. With less vigorous
action, the effects of warming will become even more disastrous. ``Into
the climate future, we're just going to keep seeing situations that set
new records,'' Dr. Henderson said.

Image

The Grizzly Creek Fire in Colorado in August.Credit...Kelsey Brunner/The
Aspen Times, via Associated Press

Daniel Swain, a climate scientist with the Institute of the Environment
and Sustainability at the University of California, Los Angeles, said
that many of today's fires, even with a measure of containment, ``are
going to be going for weeks, if not months, and are going to be
generating smoke for weeks, if not months.''

Normally, Dr. Swain said, what finally extinguishes the fires are autumn
rains and snowfall, which historically come in October or November.
However, he added, ``recently, it's been coming later than that,'' and
climate change, again, appears to be part of the reason.

\hypertarget{can-you-protect-yourself}{%
\subsubsection{Can you protect
yourself?}\label{can-you-protect-yourself}}

The C.D.C. recommends
\href{https://www3.epa.gov/airnow/smoke_fires/prepare-for-fire-season-508.pdf}{limiting
exposure to smoke} by staying indoors with windows and doors closed and
running air-conditioners in recirculation mode so that outside air isn't
drawn into your home.

Portable air purifiers are also recommended, though, like
air-conditioners, they require electricity. If utilities cut off power,
\href{https://www.nytimes3xbfgragh.onion/2020/08/18/us/california-blackouts.html}{as
has happened in California}, those options are limited.

If you do have power, avoid frying food, which can increase indoor
smoke.

Experts say it is especially important to avoid cigarettes. They also
recommend avoiding strenuous outdoor activities such as exercising or
mowing the lawn when the air is bad. When outside, well-fitted N95 masks
are also recommended, though they are in short supply because of the
pandemic.

Some do-it-yourself options are available, Dr. Henderson said, noting
that masks made from different layers of fabrics, ``particularly tightly
woven cotton and silk together,'' can provide ``pretty good filtration''
if they are fitted closely to the face.

Asked the best way to protect yourself in an area shrouded in smoke, Dr.
Dominici said the question was a difficult one. Leaving the area until
the smoke clears might be the safest option, but many people don't have
the ability to move or the luxury of choice about whether to work
outside.

Ultimately, she said, there's only so much individuals can do.

``This is why I think it's a strong responsibility of the government''
to respond to the crises caused by climate change, she said.
``Unfortunately, we're heading in the opposite direction.''

Advertisement

\protect\hyperlink{after-bottom}{Continue reading the main story}

\hypertarget{site-index}{%
\subsection{Site Index}\label{site-index}}

\hypertarget{site-information-navigation}{%
\subsection{Site Information
Navigation}\label{site-information-navigation}}

\begin{itemize}
\tightlist
\item
  \href{https://help.nytimes3xbfgragh.onion/hc/en-us/articles/115014792127-Copyright-notice}{©~2020~The
  New York Times Company}
\end{itemize}

\begin{itemize}
\tightlist
\item
  \href{https://www.nytco.com/}{NYTCo}
\item
  \href{https://help.nytimes3xbfgragh.onion/hc/en-us/articles/115015385887-Contact-Us}{Contact
  Us}
\item
  \href{https://www.nytco.com/careers/}{Work with us}
\item
  \href{https://nytmediakit.com/}{Advertise}
\item
  \href{http://www.tbrandstudio.com/}{T Brand Studio}
\item
  \href{https://www.nytimes3xbfgragh.onion/privacy/cookie-policy\#how-do-i-manage-trackers}{Your
  Ad Choices}
\item
  \href{https://www.nytimes3xbfgragh.onion/privacy}{Privacy}
\item
  \href{https://help.nytimes3xbfgragh.onion/hc/en-us/articles/115014893428-Terms-of-service}{Terms
  of Service}
\item
  \href{https://help.nytimes3xbfgragh.onion/hc/en-us/articles/115014893968-Terms-of-sale}{Terms
  of Sale}
\item
  \href{https://spiderbites.nytimes3xbfgragh.onion}{Site Map}
\item
  \href{https://help.nytimes3xbfgragh.onion/hc/en-us}{Help}
\item
  \href{https://www.nytimes3xbfgragh.onion/subscription?campaignId=37WXW}{Subscriptions}
\end{itemize}
