Sections

SEARCH

\protect\hyperlink{site-content}{Skip to
content}\protect\hyperlink{site-index}{Skip to site index}

\href{https://www.nytimes3xbfgragh.onion/section/nyregion}{New York}

\href{https://myaccount.nytimes3xbfgragh.onion/auth/login?response_type=cookie\&client_id=vi}{}

\href{https://www.nytimes3xbfgragh.onion/section/todayspaper}{Today's
Paper}

\href{/section/nyregion}{New York}\textbar{}How Paul Rudnick, Humorist,
Spends His Sundays

\url{https://nyti.ms/2ZnZVvy}

\begin{itemize}
\item
\item
\item
\item
\item
\end{itemize}

\href{https://www.nytimes3xbfgragh.onion/spotlight/at-home?action=click\&pgtype=Article\&state=default\&region=TOP_BANNER\&context=at_home_menu}{At
Home}

\begin{itemize}
\tightlist
\item
  \href{https://www.nytimes3xbfgragh.onion/2020/09/07/travel/route-66.html?action=click\&pgtype=Article\&state=default\&region=TOP_BANNER\&context=at_home_menu}{Cruise
  Along: Route 66}
\item
  \href{https://www.nytimes3xbfgragh.onion/2020/09/04/dining/sheet-pan-chicken.html?action=click\&pgtype=Article\&state=default\&region=TOP_BANNER\&context=at_home_menu}{Roast:
  Chicken With Plums}
\item
  \href{https://www.nytimes3xbfgragh.onion/2020/09/04/arts/television/dark-shadows-stream.html?action=click\&pgtype=Article\&state=default\&region=TOP_BANNER\&context=at_home_menu}{Watch:
  Dark Shadows}
\item
  \href{https://www.nytimes3xbfgragh.onion/interactive/2020/at-home/even-more-reporters-editors-diaries-lists-recommendations.html?action=click\&pgtype=Article\&state=default\&region=TOP_BANNER\&context=at_home_menu}{Explore:
  Reporters' Google Docs}
\end{itemize}

Advertisement

\protect\hyperlink{after-top}{Continue reading the main story}

Supported by

\protect\hyperlink{after-sponsor}{Continue reading the main story}

\hypertarget{how-paul-rudnick-humorist-spends-his-sundays}{%
\section{How Paul Rudnick, Humorist, Spends His
Sundays}\label{how-paul-rudnick-humorist-spends-his-sundays}}

In his new socially distanced HBO satire, ``Coastal Elites,'' five New
Yorkers and Angelenos handle the curveballs of 2020.

\includegraphics{https://static01.graylady3jvrrxbe.onion/images/2020/09/11/nyregion/11nyvirus-routine1/merlin_176410227_9f45f474-c56b-4a33-8ab0-692fe67ece9c-articleLarge.jpg?quality=75\&auto=webp\&disable=upscale}

By \href{https://www.nytimes3xbfgragh.onion/by/paige-darrah}{Paige
Darrah}

\begin{itemize}
\item
  Sept. 11, 2020
\item
  \begin{itemize}
  \item
  \item
  \item
  \item
  \item
  \end{itemize}
\end{itemize}

In 1993, Paul Rudnick made his mark as a comedic writer who took on
serious issues with
``\href{https://www.nytimes3xbfgragh.onion/1993/01/21/theater/review-theater-laughs-that-mask-the-fears-of-gay-manhattan.html?searchResultPosition=7}{Jeffrey},''
a play that captured the strained social tenor of AIDS-ravaged
Manhattan. At first, producers refused to stage the production, but it
eventually became an award-winning Off Broadway hit, followed by a movie
version, starring Patrick Stewart.

Mr. Rudnick's newest project,
``\href{https://www.hbo.com/specials/coastal-elites}{Coastal Elites},''
a series of monologues, was also intended for the stage. But when
theaters shut down in March, he turned it into a comedy special for HBO.
Streaming this weekend, the star-studded event (including Bette Midler
and Issa Rae) will feature various ``elites'' speaking to others via
screens as they navigate the pandemic and the current political
landscape. They Zoom with therapists and friends, or record YouTube
segments. At one point, a mindfulness coach has a breakdown while
filming a meditation video.

``There isn't anything funny about death or terrible illnesses, but
human beings are funny,'' Mr. Rudnick said. ``And we were able to
assemble an extraordinary cast because theater people like to work but
nobody can right now.''

Mr. Rudnick, 62, also a novelist, screenwriter and regular contributor
to \href{https://www.newyorker.com/contributors/paul-rudnick}{The New
Yorker}, spends much of his time in the West Village, in an apartment he
has owned for 25 years, with his partner, Dr. John Raftis, 67, who
specializes in rehabilitative medicine. They also have an apartment near
City Hall and a home in Port Jefferson, Long Island.

\includegraphics{https://static01.graylady3jvrrxbe.onion/images/2020/09/11/nyregion/11nyvirus-routine2/11nyvirus-routine2-articleLarge.jpg?quality=75\&auto=webp\&disable=upscale}

\textbf{WORK FROM HOME} My happiest Sunday in a very long time was in
July --- our final day of filming: 6 a.m., spread out with all my Zoom
and \href{https://qtakehd.com/}{QTake} screens, texting with the
director Jay Roach, while we watched Bette go from comic bliss to total
tears. These days, entertainment occurs with a distilled efficiency you
didn't get on a movie set with tons of people around. You're rarely just
sitting around waiting anymore. But it's sad there are no brunches, and
very little gossip, usually show-business plasma.

\textbf{CARBS} On a more usual Sunday, John gets up around 7; I stumble
out of bed by 8. Before the pandemic, we were big
\href{https://www.ihop.com/en/restaurants-new-york-ny/235-e-14th-st--237-3357}{IHOP}
guys. Sometimes you feel you have to defend it --- and I will to the
death because I was raised in New Jersey. But we haven't become big
outdoor dining guys, so I'll just heat up a Thomas's
\href{https://thomasbreads.com/products/corn-toast-r-cakes}{corn
Toast-R-Cake}. It's like a large poker chip made out of cornbread.

Image

``We walk past IHOP with a sense of lingering affection.'' Mr. Rudnick
and his longtime partner, Dr. John Raftis, are unapologetic
fans.Credit...Brittainy Newman for The New York Times

\textbf{PERSONAL WEATHER REPORT} I can't claim to understand this, but
John is a weather junkie. Saves the log from our New York Times every
Sunday going back decades, archived on our bookshelves. You can ask him,
what was the weather like in 1982? It's like living with Wikipedia.

\textbf{TO EACH HIS OWN} One secret of any lasting relationship is to
not try to change each other. John knew from our first date 27 years ago
that I have the strangest diet of almost any human. So John cooks for
himself with balsamic vinegars and other things I would never go near.
We have our separate ends of the table.

\textbf{THE DINING ROOM TABLE} You look at John's end of the table: a
plate, utensils, stemware. My end usually looks more like a child's
birthday party: sprinkles, frosting, paper goods. This has never
bothered me in the slightest, and I think some people have a secret envy
because I can subsist pretty much on crap. For example, I go through
very extreme Pringles phases. Not the flavored varieties, though. Your
classic Pringle. I love that it's a food that stacks. I went through a
period where I saved all those canisters because I had this idea of
creating furniture out of them. And still feel that's a valid --- and
eco-friendly --- idea.

Image

He can subsist on food on the lower end of the nutritional
scale.Credit...Brittainy Newman for The New York Times

\textbf{SERENITY LATER} I'm a big Gothic guy. Visual artists I know ---
painters, sculptors, filmmakers --- tend to work in serene backdrops. In
big white rooms without any clutter. I'm the opposite. I like to look up
and see lots of color and distractions. Gothic is real good for that.
Gnomes and knights and dragons staring right back at you. I come from a
big family of shoppers, whom I paid tribute to in my novel
\href{https://www.amazon.com/Ill-Take-Paul-Rudnick/dp/034536225X/ref=tmm_mmp_swatch_0?_encoding=UTF8\&qid=1598973991\&sr=1-1}{``I'll
Take It.''} But I've only been buying surgical gloves and books lately.
I've had this apartment for 25 years; at this point it's overstuffed
with gothic everything.

\textbf{THE PROCESS} In a way, everybody's on this freelance schedule
now where --- as with most writers --- you're never sure if you're
allowed a weekend. My first and most emotional draft is always longhand
in big yellow legal pads, lying on the living room couch with a big pen.
Just grabbing Frosted Flakes out of the box and water out of the big
Poland Spring. Then I think, ``OK, it might be lunchtime.'' But I just
keep going.

Image

``Getting to watch architecture happen'' --- like the Little Island
project --- ``is part of what I love about living
here.''Credit...Brittainy Newman for The New York Times

\textbf{FRESH AIR} I miss going to the gym. Though I do run \emph{to}
and \emph{around} the gym every day. The mile-long stretch up the Hudson
to Chelsea Piers centers me, even though I don't go in. It's big and
above ground --- New York gyms can sometimes feel subterranean. It's
been fascinating passing \href{https://littleisland.org/}{Little Island}
that Barry Diller's putting in the river there --- it's being landscaped
with enormous trees now. Getting to watch architecture happen is part of
what I love about living here.
\href{https://timesmachine.nytimes3xbfgragh.onion/timesmachine/1967/05/07/83598670.html?pageNumber=386}{Jefferson
Market Library} on Sixth Avenue at Christopher Street is my favorite
building. There is a big Victorian Gothic tower and turret; it's an
inspiring little Harry Potter corner. I walk by there when I'm getting
Gristedes groceries.

Image

Mr. Rudnick on the ``inspiring little Harry Potter corner'' near his
favorite building,~the Jefferson Market Library.Credit...Brittainy
Newman for The New York Times

\textbf{WARDROBE} By the time evening rolls around, we're ready for
Netflix. Wish this were only true during the pandemic, but I tend to go
to bed in whatever I wore that day. The great American uniform: ancient
navy blue J. Crew sweatpants and white socks. That's where we all are
right now --- the line between sleepwear and workwear, erased. John has
insomnia, but before I conk out at 1 a.m. we'll also watch a little
``\href{https://www.hgtv.com/shows/house-hunters}{House Hunters}'' on
HGTV, which acts as the ideal narcotic.

Sunday Routine readers can follow Paul Rudnick on Twitter:
@PaulRudnickNY.

Advertisement

\protect\hyperlink{after-bottom}{Continue reading the main story}

\hypertarget{site-index}{%
\subsection{Site Index}\label{site-index}}

\hypertarget{site-information-navigation}{%
\subsection{Site Information
Navigation}\label{site-information-navigation}}

\begin{itemize}
\tightlist
\item
  \href{https://help.nytimes3xbfgragh.onion/hc/en-us/articles/115014792127-Copyright-notice}{©~2020~The
  New York Times Company}
\end{itemize}

\begin{itemize}
\tightlist
\item
  \href{https://www.nytco.com/}{NYTCo}
\item
  \href{https://help.nytimes3xbfgragh.onion/hc/en-us/articles/115015385887-Contact-Us}{Contact
  Us}
\item
  \href{https://www.nytco.com/careers/}{Work with us}
\item
  \href{https://nytmediakit.com/}{Advertise}
\item
  \href{http://www.tbrandstudio.com/}{T Brand Studio}
\item
  \href{https://www.nytimes3xbfgragh.onion/privacy/cookie-policy\#how-do-i-manage-trackers}{Your
  Ad Choices}
\item
  \href{https://www.nytimes3xbfgragh.onion/privacy}{Privacy}
\item
  \href{https://help.nytimes3xbfgragh.onion/hc/en-us/articles/115014893428-Terms-of-service}{Terms
  of Service}
\item
  \href{https://help.nytimes3xbfgragh.onion/hc/en-us/articles/115014893968-Terms-of-sale}{Terms
  of Sale}
\item
  \href{https://spiderbites.nytimes3xbfgragh.onion}{Site Map}
\item
  \href{https://help.nytimes3xbfgragh.onion/hc/en-us}{Help}
\item
  \href{https://www.nytimes3xbfgragh.onion/subscription?campaignId=37WXW}{Subscriptions}
\end{itemize}
