\href{/section/arts/television}{Television}\textbar{}Larry Wilmore
Doesn't Miss Late Night, but He's Back Anyway

\url{https://nyti.ms/3ma1hnq}

\begin{itemize}
\item
\item
\item
\item
\item
\item
\end{itemize}

\includegraphics{https://static01.graylady3jvrrxbe.onion/images/2020/09/13/arts/13wilmore1/merlin_176511975_e91c2bb7-d17c-4dca-b284-df0c6422a6c9-articleLarge.jpg?quality=75\&auto=webp\&disable=upscale}

Sections

\protect\hyperlink{site-content}{Skip to
content}\protect\hyperlink{site-index}{Skip to site index}

\hypertarget{larry-wilmore-doesnt-miss-late-night-but-hes-back-anyway}{%
\section{Larry Wilmore Doesn't Miss Late Night, but He's Back
Anyway}\label{larry-wilmore-doesnt-miss-late-night-but-hes-back-anyway}}

Four years after the abrupt end of ``The Nightly Show,'' Wilmore has a
new topical comedy series on Peacock.

A prolific producer and commentator, Larry Wilmore has yet to prove that
he can thrive as a talk show host.Credit...Joyce Kim for The New York
Times

Supported by

\protect\hyperlink{after-sponsor}{Continue reading the main story}

\href{https://www.nytimes3xbfgragh.onion/by/dave-itzkoff}{\includegraphics{https://static01.graylady3jvrrxbe.onion/images/2018/02/16/multimedia/author-dave-itzkoff/author-dave-itzkoff-thumbLarge.jpg}}

By \href{https://www.nytimes3xbfgragh.onion/by/dave-itzkoff}{Dave
Itzkoff}

\begin{itemize}
\item
  Sept. 11, 2020
\item
  \begin{itemize}
  \item
  \item
  \item
  \item
  \item
  \item
  \end{itemize}
\end{itemize}

Back when \href{https://www.thelarrywilmore.com/}{Larry Wilmore} was
about to introduce
\href{http://www.cc.com/shows/the-nightly-show-with-larry-wilmore}{``The
Nightly Show,''} his short-lived late-night series on Comedy Central, he
saw a tweet from an angry prospective viewer who wished failure on the
host and his lousy show, which at that point had not yet aired a single
episode.

Recalling his own reaction at the time, Wilmore said he thought to
himself that he would at least like the chance to be terrible before
being dismissed as terrible.

``It's not even on yet, so how do you know?'' he said. ``You might be
right, but let me do it first.''

That crabby electronic dispatch was prophetic, though: ``The Nightly
Show,'' which was intended as a companion program for Jon Stewart's
\href{http://www.cc.com/shows/the-daily-show-with-jon-stewart}{``Daily
Show''} and a replacement for Stephen Colbert's
\href{http://www.cc.com/shows/the-colbert-report}{``Colbert Report,''}
lasted less than two years before its cancellation in August 2016.

For Wilmore, 58, a prolific producer, host and commentator, it remains
one of the less successful entries on his lengthy résumé, which includes
writing credits on comedies like
\href{https://www.youtube.com/watch?v=_QhuBIkPXn0}{``In Living Color''}
and \href{https://www.youtube.com/watch?v=r-UhYRXYG8c}{``Sister,
Sister,''} as well as
\href{https://www.youtube.com/watch?v=-_FYMH4OUUk\&t=8s}{``The Bernie
Mac Show,''} which he created.

He has become a mentor of other emerging talents and helped shows like
\href{https://abc.com/shows/blackish}{``black-ish''} and
\href{https://www.hbo.com/insecure}{``Insecure''} find their footing,
though he has yet to prove that he can be the center of his own TV
program.

Four years after the ``The Nightly Show,'' Wilmore shrugged off its
demise with the nonchalance of a veteran who knows not to get too
attached to any particular opportunity. Comparing himself to a
basketball player, he said, ``That missed shot is forgotten, and I'm
shooting again.''

Now, without really having to campaign for it, Wilmore finds himself
returning to the arena of topical TV comedy. On Friday he will once
again host his own late-night series, called ``Wilmore,'' for NBC's
Peacock streaming service.

\includegraphics{https://static01.graylady3jvrrxbe.onion/images/2020/09/13/arts/13WILMORE8/13WILMORE8-articleLarge.jpg?quality=75\&auto=webp\&disable=upscale}

It is a weekly program with a mission as simple as its title: to allow
Wilmore to riff on the coming presidential election, other news events
and whatever else he is panicking about in a given moment, and to
interview guests he finds funny or interesting.

``Wilmore'' is not necessarily his attempt to find closure after ``The
Nightly Show,'' to prove that he can do the job or to see himself on
television again. What it represents to him, Wilmore said, is the latest
step in a continuing journey to figure out what he wants to say and to
find the best place for him to say it.

``Being on camera isn't that important,'' Wilmore said in a recent Zoom
conversation. ``It's fun, but when it went away, I didn't miss it.''

``I don't do this for the attention,'' he added, ``I do this for the
expression.''

On an afternoon in late August, Wilmore was speaking from a sparsely
decorated office at his home in Los Angeles. Despite the imminent debut
of his show, he was an unhurried yarn-spinner who, when his interviewer
least expected it, started flexing
\href{https://www.youtube.com/watch?v=dhj5rZk9tmg\&t=4m29s}{his skills
as a sleight-of-hand magician} and began making coins and playing cards
disappear and reappear out of thin air.

He cautioned that, at its outset, ``Wilmore'' would be equally modest in
its production values: He'll be hosting from a basic studio with no
audience while his guests join in remotely.

``We have zero budget,'' Wilmore said with a chuckle. ``I almost owe
them money at this point.''

The new show grew out of
\href{https://deadline.com/2019/05/larry-wilmore-overall-deal-universal-television-1202624582/}{an
overall deal that Wilmore made last year} with Universal Television and
his ongoing conversations with executives there, during which he would
drop hints that he wanted to oversee a comedic election town-hall
special, similar to
\href{https://www.sho.com/video/17698/behind-the-scenes-larry-wilmore}{one
he did for Showtime in 2012}.

This past spring, Wilmore appeared on Peacock in a celebrity
fund-raising program, anchoring remote segments in which
\href{https://www.youtube.com/watch?v=dhj5rZk9tmg\&t=1m36s}{he debated
Sean Hayes} on whether or not almond milk should be considered milk, and
learned some potentially offensive Mandarin Chinese slang
\href{https://www.youtube.com/watch?v=dhj5rZk9tmg\&t=3m00s}{from his
daughter, Lauren}.

Dan Shear, who is Peacock's executive vice president of comedy
development, said that those segments had been persuasive indicators
that Wilmore ``needed to have a place in the cultural conversation ---
with everything that was going on in the world, it just felt really
important to have him on the air right now.''

Shear said that Wilmore's inauspicious history at ``The Nightly Show''
was by no means a strike against the host and had actually made viewers
more eager to see him again.

``It's a well-known fact that he hadn't been on the air during the 2016
election,'' Shear said. ``That felt like such a loss for the audience
that he wasn't there through that.''

In May, after the police killing of George Floyd and the wave of
protests that followed, Peacock asked Wilmore if he wanted to address
viewers at that moment.

But Wilmore demurred, feeling that the time wasn't right: ``People were
so upset and they didn't know what to do with those emotions,'' he
explained. ``Who am I to just go out and talk about this?''

Image

``I don't do this for the attention,'' Wilmore said of his on-camera
hosting and commentary. ``I do this for the expression.''Credit...Joyce
Kim for The New York Times

Wilmore's instincts have generally served him well since the start of
his writing career in the early 1990s, when he took inspiration from
Black creators who were producing their own shows, like Keenen Ivory
Wayans (``In Living Color'') and Yvette Lee Bowser
\href{https://www.youtube.com/watch?v=jnD2mdaxUAo}{(``Living Single''}).

Though Hollywood offered creative heroes to admire, Wilmore said that
opportunities for writers of color were limited by prevailing attitudes
in the industry.

``If you were Black you couldn't work on a white show, but if you were
white, you could definitely work on a Black show,'' he said. ``It was so
condescending.''

Even so, Wilmore said that he tended to be overly picky about the
projects he chose for himself and turned down jobs if he felt they did
not fulfill a particular need or urgency within him.

Describing his thought process, Wilmore said: ``Can I say the thing I
want to say in this? If I can't, I'm just somebody saying nothing, and
I'm not good enough to be another empty voice.''

His forte, he said, has been coming in at the start of a new show and
helping to populate it with memorable scenarios and characters --- even
\href{https://www.youtube.com/watch?v=W6ZBfqkIAAc}{Smokey, the crack
addict} he devised for ``The PJs,'' the animated series he created with
Eddie Murphy and Steve Tompkins.

``I said you've got to have a crackhead in this,'' Wilmore recalled.
``I'm very proud of it. I live for that stuff.''

Wilmore has also helped advance the careers of creative partners like
\href{https://www.nytimes3xbfgragh.onion/2015/08/09/magazine/the-misadventures-of-issa-rae.html?searchResultPosition=6}{Issa
Rae}, who stars on ``Insecure'' and created that HBO series with him,
and who started working with Wilmore after a fraught and unproductive
development process at ABC.

At that time, Rae said, ``I was creatively broken and very fragile and
didn't have the confidence in my voice.'' But when she began her
collaboration with Wilmore, she said, ``he had such a calming,
personable demeanor and asked the smartest questions.''

Over many conversations and meals, Rae said, she confessed countless
personal details to Wilmore about failed relationships and about gripe
sessions she had with female friends
\href{https://www.youtube.com/watch?v=r2DN-eYmRI0}{regarding their
anatomy}, much of which was woven into the ``Insecure'' pilot. ``I was
like wow, I've been duped, but in the best way,'' she said.

And when Wilmore was approached about hosting what became ``The Nightly
Show'' --- in the midst of his development duties on ``Insecure'' and
``black-ish'' --- Rae knew she couldn't stand in his way.

``I was absolutely devastated, but I had to be understanding,'' she
said. ``You can't be mad at someone who's doing his dream job.''

Image

Wilmore, a prolific comedy producer, has become a mentor of emerging
talents and helped develop shows like Kenya Barris's
``black-ish.''Credit...Patrick Wymore/ABC

Image

Issa Rae, right, with Yvonne Orji in ``Insecure,'' said Wilmore ``had
such a calming, personable demeanor and asked the smartest
questions.''Credit...John P. Johnson/HBO

At ``The Nightly Show,'' Wilmore said, he knew he would be fighting to
overcome the lofty expectations set by Colbert, his predecessor in the
time slot, who had created a seminal work of political and media satire
with ``The Colbert Report'' before he left to host ``The Late Show'' on
CBS.

Wilmore said that he had sensed Comedy Central wanted a similar show
from him,
\href{http://www.cc.com/video-clips/hgeu2x/the-nightly-show-with-larry-wilmore-keep-it-100---larry-s-forced-to-play-favorites}{with
repeatable franchise elements}, ``something that had more form to it,
that seemed formulaic.''

But he wanted to make something more malleable: ``I'm interested in
keeping it 100 percent real, and whatever comes out of that expression
can be on the show,'' he said. ``I'd rather keep a conversation going
that might not be as funny, but if I'm just doing some silly bit, that
doesn't make sense.''

Wilmore was also comfortable sharing his spotlight with colleagues like
\href{https://www.nytimes3xbfgragh.onion/2017/10/08/arts/television/on-the-rundown-robin-thede-is-filling-a-void-in-late-night-talk.html?searchResultPosition=2}{Robin
Thede}, his ``Nightly Show'' head writer, who was one of several staff
members who often appeared on camera.

``He set us all up for success and he was intentional about it,'' said
Thede, who went on to host her own BET late-night series, ``The
Rundown,'' and to create and star in HBO's
\href{https://www.nytimes3xbfgragh.onion/2019/07/31/arts/television/a-black-lady-sketch-show-hbo.html?searchResultPosition=1}{``A
Black Lady Sketch Show.''}

``He said to me, `I'm here to help you win,' when I was on \emph{his}
show,'' Thede added. ``Other people only want to bring in people who
don't challenge their way of thinking. He revels in smart brains ---
that's his happy spot.''

But as ratings for ``The Nightly Show'' declined --- particularly after
\href{https://www.nytimes3xbfgragh.onion/2015/08/07/arts/television/jon-stewart-signs-off-from-daily-show-with-wit-and-sincerity.html}{Stewart
left ``The Daily Show'' in August 2015} --- Wilmore could tell that
Comedy Central had soured on him, he said. ``There was a certain point
where they didn't even talk to us.'' (Comedy Central declined to
comment.)

At its cancellation, ``The Nightly Show'' was drawing about 776,000
viewers a night, far below the average audience of 1.7 million viewers
that ``The Colbert Report'' attracted in its final year. (Since then,
Comedy Central has fared no better with shows hosted by
\href{https://www.nytimes3xbfgragh.onion/2017/09/15/arts/jordan-klepper-wants-to-be-a-colbert-for-the-breitbart-era.html}{Jordan
Klepper} and by
\href{https://www.nytimes3xbfgragh.onion/2019/07/22/arts/television/david-spade-is-back-on-tv-even-though-he-never-left.html}{David
Spade}, each of which lasted less than a year.)

Wilmore said he held no lingering grudges against the network but
admitted that he found a certain pleasure in the fact that Comedy
Central still had not found a hit program to follow ``The Daily Show.''

``My schadenfreude is full every day,'' Wilmore said. ``Every single day
I have a cup of that in the morning.''

Image

``I'm interested in keeping it 100 percent real, and whatever comes out
of that expression can be on the show,'' Wilmore said.Credit...Joyce Kim
for The New York Times

In the time since ``The Nightly Show'' ended, Wilmore has produced and
developed other projects for broadcast and streaming networks and has
hosted a podcast,
\href{https://www.theringer.com/larry-wilmore-black-on-air}{``Black on
the Air,''} for The Ringer.

Wilmore plans to continue ``Black on the Air,'' which mixes personal
monologues with his interviews of celebrities, politicians and
journalists, while he hosts his Peacock series. He said the podcast had
provided him with a crucial education in conducting long-form interviews
and allowed him to reach places he could not get to in his late-night
comedy round tables.

``I've learned so much about just having a conversation without needing
to turn it into entertainment, being actively interested in what the
other person's saying and not just waiting to ask your questions,'' he
said.

But he does not necessarily see ``The Nightly Show'' as a lesson to be
learned from or a skid to steer out of as he figures out ``Wilmore.''

``As a producer,'' he said, ``I can only make a show what it has to be.
It's this conversation you're having with your audience that tells you
what a show has to be.''

Unlike with his ``Nightly Show'' tenure, Wilmore is proclaiming at the
outset of his Peacock show that it is a limited-run series, planned for
11 episodes that will continue through the end of November.

``Is it going to get picked up? No,'' he said. ``This is going to be
done, and then we'll sit down at the right time and say, Is this
something we want to do as a permanent thing?''

This time around, Wilmore acknowledged that he will be more of a known
quantity than he was at the start of ``The Nightly Show,'' a status that
comes with both advantages and disadvantages.

He fully expects to be criticized by audience members who will complain
that ``Wilmore'' isn't ``The Nightly Show'' --- a program that he
couldn't get them to embrace in sufficient numbers when it was on the
air.

Imagining himself addressing these detractors, Wilmore said, ``Guys,
each time I try something new, trust me, you're going to object to it
because it's new. We haven't seen it yet.''

Advertisement

\protect\hyperlink{after-bottom}{Continue reading the main story}

\hypertarget{site-index}{%
\subsection{Site Index}\label{site-index}}

\hypertarget{site-information-navigation}{%
\subsection{Site Information
Navigation}\label{site-information-navigation}}

\begin{itemize}
\tightlist
\item
  \href{https://help.nytimes3xbfgragh.onion/hc/en-us/articles/115014792127-Copyright-notice}{©~2020~The
  New York Times Company}
\end{itemize}

\begin{itemize}
\tightlist
\item
  \href{https://www.nytco.com/}{NYTCo}
\item
  \href{https://help.nytimes3xbfgragh.onion/hc/en-us/articles/115015385887-Contact-Us}{Contact
  Us}
\item
  \href{https://www.nytco.com/careers/}{Work with us}
\item
  \href{https://nytmediakit.com/}{Advertise}
\item
  \href{http://www.tbrandstudio.com/}{T Brand Studio}
\item
  \href{https://www.nytimes3xbfgragh.onion/privacy/cookie-policy\#how-do-i-manage-trackers}{Your
  Ad Choices}
\item
  \href{https://www.nytimes3xbfgragh.onion/privacy}{Privacy}
\item
  \href{https://help.nytimes3xbfgragh.onion/hc/en-us/articles/115014893428-Terms-of-service}{Terms
  of Service}
\item
  \href{https://help.nytimes3xbfgragh.onion/hc/en-us/articles/115014893968-Terms-of-sale}{Terms
  of Sale}
\item
  \href{https://spiderbites.nytimes3xbfgragh.onion}{Site Map}
\item
  \href{https://help.nytimes3xbfgragh.onion/hc/en-us}{Help}
\item
  \href{https://www.nytimes3xbfgragh.onion/subscription?campaignId=37WXW}{Subscriptions}
\end{itemize}
