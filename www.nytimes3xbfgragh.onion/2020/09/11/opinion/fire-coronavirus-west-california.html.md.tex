Sections

SEARCH

\protect\hyperlink{site-content}{Skip to
content}\protect\hyperlink{site-index}{Skip to site index}

\href{https://myaccount.nytimes3xbfgragh.onion/auth/login?response_type=cookie\&client_id=vi}{}

\href{https://www.nytimes3xbfgragh.onion/section/todayspaper}{Today's
Paper}

\href{/section/opinion}{Opinion}\textbar{}I've Never Seen the American
West in Such Deep Distress

\url{https://nyti.ms/33xQE5B}

\begin{itemize}
\item
\item
\item
\item
\item
\item
\end{itemize}

Advertisement

\protect\hyperlink{after-top}{Continue reading the main story}

\href{/section/opinion}{Opinion}

Supported by

\protect\hyperlink{after-sponsor}{Continue reading the main story}

\hypertarget{ive-never-seen-the-american-west-in-such-deep-distress}{%
\section{I've Never Seen the American West in Such Deep
Distress}\label{ive-never-seen-the-american-west-in-such-deep-distress}}

We're choking on smoke and staring out at Martian-red skies in a world
becoming uninhabitable.

\href{https://www.nytimes3xbfgragh.onion/by/timothy-egan}{\includegraphics{https://static01.graylady3jvrrxbe.onion/images/2018/05/11/opinion/timothy-egan/timothy-egan-thumbLarge.png}}

By \href{https://www.nytimes3xbfgragh.onion/by/timothy-egan}{Timothy
Egan}

Contributing Opinion Writer

\begin{itemize}
\item
  Sept. 11, 2020
\item
  \begin{itemize}
  \item
  \item
  \item
  \item
  \item
  \item
  \end{itemize}
\end{itemize}

\includegraphics{https://static01.graylady3jvrrxbe.onion/images/2020/09/11/opinion/11eganWeb/merlin_176095812_2b44f108-a6fd-4d3f-8727-7399d748fa4e-articleLarge.jpg?quality=75\&auto=webp\&disable=upscale}

The open road in the Big Empty part of the American West has always been
therapeutic. Vacant skies, horizons that stretch to infinity, country
without clutter. The soul needs to roam, too.

After six months of confinement, I was a caged bird gnawing at the bars.
Ahead were mountains beyond mountains, rivers that hustled out of tight
canyons, and winds strong enough to knock a prairie chicken down.

Alas, my map was obsolete. The West of 2020 is very sick. Like much of
the country, we Westerners are at one another's throats, struggling to
put our lives back together under a madman for a president. But unlike
the rest of the country, we're also choking on smoke and staring out at
Martian-red skies in a world becoming uninhabitable.

My map should have included hot spots of the coronavirus and wildfire. I
spent as much time checking an air quality index app as the weather
forecast. And the live-free-or-die ethos of tumbledown towns defying
mask orders turned many a curious detour into a perilous proposition.

Even the historical markers, commemorating wagon trains in trespass over
Native land, rivers dammed for oligarchs of industry and agriculture,
rail lines built on migrant labor, seemed out of sync and out of time.

I left Puget Sound with the sun burnishing Mount Rainier's glaciers, a
string of bluebird days in the contrails of the season. But I no sooner
crested the Cascades than the smoke of the arid interior blotted out the
way ahead, a harbinger of a week when the West would blow up.

About
\href{https://twitter.com/GovInslee/status/1303459243853451264}{330,000
acres} of the Evergreen State burned on Monday --- more land consumed by
fire in a single day than all the acreage of an entire typical season in
Washington.

\href{https://www.opb.org/news/article/yakima-county-covid-19-hot-agriculture/}{Yakima
Valley}, ripe with Christmas ornament apples and pinch-me peaches, was
monochrome gray, in fierce battle with runaway flames. But it's also one
of the hardest-hit areas in the country for Covid-19. This year, all
that beautiful fruit is picked at a terrible cost, in lives and
sickness, to people living in cramped, temporary quarters.

Then, I went across the mighty Columbia, the river of the West, and
along the Snake, formerly two of the most crowded salmon highways in the
world, now held in the harness of hydroelectric dams. Some of the feeder
streams --- the Umatilla, the Grand Ronde, the Malheur --- looked anemic
and infirm.

Oregon held California's smoke, and many of its recent refugees. A
record 2.5 million acres have burned in the Golden State this year, and
the fire season has only just begun.

``I have no patience for climate change deniers,'' said
\href{https://www.theguardian.com/world/video/2020/sep/09/california-governor-wildfires-no-patience-climate-change-deniers-video}{Gov.
Gavin Newsom} of California, a state with 150 million dead trees and
temperatures that recently
\href{https://www.latimes.com/california/story/2020-09-06/southern-california-weather-forecast-sunday-los-angeles-record-breaking-heat}{reached
121 degrees} in Los Angeles County.

Meanwhile, the world's most dangerous climate change denier continued to
spout gibberish. ``You gotta clean your floors, you gotta clean your
forests,'' said
\href{https://www.politico.com/states/california/story/2020/08/20/trump-blames-california-for-wildfires-tells-state-you-gotta-clean-your-floors-1311059}{President
Trump, scolding California}.

That's like telling people to drain their wading pools in advance of a
hurricane. Nearly 48 percent of the land in California is federally
owned. Those are \emph{his} floors. And this West in distress is made
sicker by \emph{his} defiance of the globe's existential threat. If ash
were falling on his hair, he'd be more alert.

We followed a road along the old Oregon Trail into Idaho, then picked up
parts of the southern branch into Utah. The historical markers note that
immigrants recruited by Mormons pushed and
\href{https://www.wyohistory.org/encyclopedia/journey-martins-cove-mormon-handcart-tragedy-1856}{pulled
wooden handcarts}, essentially large wheelbarrows, across the
continent's midsection. It was insane, leading to many deaths.

I'd always marveled at those who walked thousands of miles to grab off a
piece of dry turf to call their own. But this time around I wondered
more about the people whose land was being taken. The Shoshone, Bannock
and Northern Paiutes lived well without having to push 300-pound carts
over the Continental Divide.

I'd never seen southern Wyoming in such a bad way. The sky was white
with heat, and then blue-white with smoke, the endless beige tableau of
the land littered with the detritus of oil, coal and gas extraction. We
saw one fire go off like a nuclear bomb.

Here is another bit of insanity in the hellscape of this season:
Wyoming's desperate effort to hold on to its earth-killing coal plants
is a contributing cause to all the climate-change fires.

An unrelated thought: How come Wyoming, with a falling population of
567,000, has two United States senators, while Washington, D.C., with
more than 700,000 people, has none?

Colorado's skies were blood red, another Rocky Mountain sigh, as we came
under the cloud of the Cameron Peak Fire, one of the 10 largest in state
history, all of them coming since 2002.

The authorities urged everyone to stay indoors. My parked car, in
Boulder, took on a coat of falling ash.
\href{https://www.nytimes3xbfgragh.onion/2020/09/08/us/denver-snow-weather.html}{Overnight,
temperatures dropped 50 degrees}, and by morning snow was falling on
cedars and muffling some of the fires along the Front Range.

Back home, an
\href{https://www.seattletimes.com/seattle-news/environment/orca-tahlequah-is-a-mother-again/}{endangered
orca named Tahlequah}, who had captured the world's attention when she
carried her dead baby for 17 days in 2018, gave birth to a healthy calf.
New life in the Salish Sea, fresh snow on the Flatirons; it was enough
of a hint that nature can make things right, if only we give it a
chance.

\emph{The Times is committed to publishing}
\href{https://www.nytimes3xbfgragh.onion/2019/01/31/opinion/letters/letters-to-editor-new-york-times-women.html}{\emph{a
diversity of letters}} \emph{to the editor. We'd like to hear what you
think about this or any of our articles. Here are some}
\href{https://help.nytimes3xbfgragh.onion/hc/en-us/articles/115014925288-How-to-submit-a-letter-to-the-editor}{\emph{tips}}\emph{.
And here's our email:}
\href{mailto:letters@NYTimes.com}{\emph{letters@NYTimes.com}}\emph{.}

\emph{Follow The New York Times Opinion section on}
\href{https://www.facebookcorewwwi.onion/nytopinion}{\emph{Facebook}}\emph{,}
\href{http://twitter.com/NYTOpinion}{\emph{Twitter (@NYTopinion)}}
\emph{and}
\href{https://www.instagram.com/nytopinion/}{\emph{Instagram}}\emph{.}

Timothy Egan (@nytegan) is a contributing opinion writer who covers the
environment, the American West and politics. He is a winner of the
National Book Award and author, most recently, of ``A Pilgrimage to
Eternity.''

Advertisement

\protect\hyperlink{after-bottom}{Continue reading the main story}

\hypertarget{site-index}{%
\subsection{Site Index}\label{site-index}}

\hypertarget{site-information-navigation}{%
\subsection{Site Information
Navigation}\label{site-information-navigation}}

\begin{itemize}
\tightlist
\item
  \href{https://help.nytimes3xbfgragh.onion/hc/en-us/articles/115014792127-Copyright-notice}{©~2020~The
  New York Times Company}
\end{itemize}

\begin{itemize}
\tightlist
\item
  \href{https://www.nytco.com/}{NYTCo}
\item
  \href{https://help.nytimes3xbfgragh.onion/hc/en-us/articles/115015385887-Contact-Us}{Contact
  Us}
\item
  \href{https://www.nytco.com/careers/}{Work with us}
\item
  \href{https://nytmediakit.com/}{Advertise}
\item
  \href{http://www.tbrandstudio.com/}{T Brand Studio}
\item
  \href{https://www.nytimes3xbfgragh.onion/privacy/cookie-policy\#how-do-i-manage-trackers}{Your
  Ad Choices}
\item
  \href{https://www.nytimes3xbfgragh.onion/privacy}{Privacy}
\item
  \href{https://help.nytimes3xbfgragh.onion/hc/en-us/articles/115014893428-Terms-of-service}{Terms
  of Service}
\item
  \href{https://help.nytimes3xbfgragh.onion/hc/en-us/articles/115014893968-Terms-of-sale}{Terms
  of Sale}
\item
  \href{https://spiderbites.nytimes3xbfgragh.onion}{Site Map}
\item
  \href{https://help.nytimes3xbfgragh.onion/hc/en-us}{Help}
\item
  \href{https://www.nytimes3xbfgragh.onion/subscription?campaignId=37WXW}{Subscriptions}
\end{itemize}
