Sections

SEARCH

\protect\hyperlink{site-content}{Skip to
content}\protect\hyperlink{site-index}{Skip to site index}

\href{https://myaccount.nytimes3xbfgragh.onion/auth/login?response_type=cookie\&client_id=vi}{}

\href{https://www.nytimes3xbfgragh.onion/section/todayspaper}{Today's
Paper}

\href{/section/opinion}{Opinion}\textbar{}How to Win the Latino Vote

\url{https://nyti.ms/35uW5EL}

\begin{itemize}
\item
\item
\item
\item
\item
\end{itemize}

Advertisement

\protect\hyperlink{after-top}{Continue reading the main story}

transcript

Back to The Argument

bars

0:00/0:00

-0:00

transcript

\hypertarget{how-to-win-the-latino-vote}{%
\subsection{How to Win the Latino
Vote}\label{how-to-win-the-latino-vote}}

\hypertarget{with-isvett-verde-chuck-rocha-and-linda-chavez}{%
\subsubsection{With Isvett Verde, Chuck Rocha and Linda
Chavez}\label{with-isvett-verde-chuck-rocha-and-linda-chavez}}

\hypertarget{a-podcast-debate-over-the-political-future-of-the-growing-demographic-of-voters}{%
\paragraph{A podcast debate over the political future of the growing
demographic of
voters.}\label{a-podcast-debate-over-the-political-future-of-the-growing-demographic-of-voters}}

Friday, September 11th, 2020

\begin{itemize}
\item
  chuck rocha\\
  I'm Chuck Rocha.
\item
  linda chavez\\
  I'm Linda Chavez.
\item
  isvett verde\\
  I'm Isvett Verde, and this is ``The Argument.'' {[}THEME MUSIC
  PLAYS{]}

  OK, so we're not ``The Argument hosts you're used to. I'm Isvett, a
  writer and editor in Opinion focusing on the Latino experience in the
  United States. Michelle and Ross let me jump into the host seat today
  for a special episode focused on the Latino vote. It's one of the
  fastest-growing demographics of voters in the United States. But in
  the Trump era, which party makes a better pitch to Latinos? I'm
  honored to be joined today by two of Opinion's contributors. Linda
  Chavez is a senior fellow at the Niskanen Center and the director of
  the Becoming American Institute, and she was one of the
  highest-ranking women in Reagan's White House. Linda, welcome to ``The
  Argument.''
\item
  linda chavez\\
  Thank you.
\item
  isvett verde\\
  And joining me and Linda is Chuck Rocha. He was a senior campaign
  advisor to Bernie Sanders. He's also the founder of Solidarity
  Strategies, a political consulting firm that supports progressive
  candidates. And while you may not be able to see his cowboy hat,
  you'll no doubt hear the Texas in his voice. Chuck, thank you for
  joining us today.
\item
  chuck rocha\\
  Thanks for having me.
\item
  isvett verde\\
  So there's often talk of this thing called a ``Latino voting bloc.''
  Why don't we just start with an easy question? Is there such a thing
  as a Latino voting bloc?
\item
  chuck rocha\\
  Well, I would say that there's definitely a bloc, but I would say that
  that bloc is multicultural, multidynamic, multigenerational, and I'd
  even say multicolored. People who think about our vote think about us
  all being Mexican, think about us all speaking like I do as a man from
  Texas, and they just don't realize how different and how many shades
  of the rainbow there is. Not only with the color of our skin, but
  where we come from and our countries of origin and I would say, for
  someone like me, how many generations your family has been here in
  America. For many examples, I like to use the one that's my favorite
  is I'm a third-generation Texan. It's been a long time since my folks
  lived in Guanajuato, Mexico. And so the way that my family, my son,
  and even my twin grandsons view the world --- all Latinos and all
  Latino voters at some point in their lives --- could not be more
  different than the experience maybe that a Cuban or a Dominican or
  Puerto Rican. All of these different cultures make our voting blocks
  so unique. So we are a block. There are just so many pieces of that
  block.
\item
  linda chavez\\
  I know that usually on these kinds of podcasts we're supposed to
  disagree with each other, but I couldn't agree more with Chuck on that
  point. Hispanics are very diverse. Chuck talked about being a
  third-generation Texan. Well, my family came over in 1601. Pedro Duran
  y Chavez came from Sevilla and ended up in Santa Fe, New Mexico. He
  was part of the original expedition that founded New Mexico. My family
  has lived on soil that is now part of the United States going back 10
  or 11 generations. And so whenever people tell me to go back where I
  come from, it's a little hard. And by the way, I don't think they mean
  Spain. {[}CHUCK LAUGHS{]}
\item
  isvett verde\\
  I think between the three of us, we span the Latino American
  experience. You know I came from Cuba in a boat in 1980 and grew up in
  Miami surrounded by Cubans, in a way feeling like I'd never left the
  island. And our stories are an example of how disparate the Latino
  American experience can be in this country. All right, so now we know
  that Latinos are a diverse bloc, and I'm curious. What do we talk
  about when we say Latino issues? What are they to you? Does such a
  thing exist?
\item
  linda chavez\\
  Well, I think there are some issues that obviously have more
  importance in the Latino community. Immigration is certainly one of
  them, but when you ask Latino voters themselves what issues matter to
  them, immigration is not usually number one on that list. There are
  issues like the economy, which I think cut through. And probably if
  you were to look at public-opinion polls of Latino voters, the economy
  tends to be very high up on that list and may, in fact, surpass things
  like immigration, which you might think would have more salience in
  the Hispanic community given that about half of the adult Hispanic
  population in the United States is, in fact, foreign born.
\item
  chuck rocha\\
  I'd break that down a little bit differently of issues. What does it
  mean to be a Latino issue? I think there's a set of Latino issues for
  Latinos. I think there's a set of issues for Latino voters. The
  problem is how white pollsters ask the question, and that is what is
  the most important issue for you this election cycle? And to like what
  Linda said is normally that's the jobs and the economy. Latino voters
  in that sense will act a lot like a white voter. But I just find that
  there is a nuance between an emotional issue that a Latino will tell
  you is not their most important issue to go vote today. It will always
  be jobs, the economy, the coronavirus, education, something, whatever
  that particular voter feels about. But if you want to get to the
  emotion, you'll notice when Linda did her intro --- not to pick on
  Linda --- but she talked about the pride that she felt in her family,
  and going back that many years has been instilled in her to
  regurgitate that story in such a prideful way because, just like I
  talked about my family, it's such a prideful thing, right? So when you
  want to talk about an emotional issue, immigration will never be in
  the top two or three of a Latino voters. But if you want to talk about
  tapping into an emotion, tap into what Isvett said about getting on a
  boat, coming here, getting with your family and moving from anywhere.
  Like, that's an emotional trauma or emotional beautiful story or both
  all wrapped up.
\item
  isvett verde\\
  Chuck, you raised a great point about this mix of pride and trauma. I
  think for me personally, I experience it that way, right, the sense of
  being proud of being Latina but also the trauma of immigration, of
  leaving the country that you're from and building roots in a different
  one. And now that we've sorted out what issues matter to Latino
  Americans, I have a feeling you two disagree about the role politics
  can play in meeting those issues. So, Linda, I'm going to guess you
  don't think Bernie Sanders is the way. Am I right?
\item
  linda chavez\\
  I don't think Bernie Sanders is the way. That's absolutely true. But I
  certainly don't think Donald Trump is the way either.
\item
  isvett verde\\
  Chuck, you turned out Latino voters in big numbers in Nevada,
  Colorado, and Texas. Why do you think voters connected with ``Tío
  Bernie``?
\item
  chuck rocha\\
  It's not science, and it's not hard. And I wrote this book ``Tío
  Bernie'' to explain to everyone in our community if you want our vote
  and if you want to come into our community and get our vote, you
  should spend resources and come have a conversation with our
  community. And I wanted to prove once and for all after 31 years of
  doing this if I could do it with Bernie Sanders that anybody could do
  it with another candidate if they do it in the way that I would
  outline, which is spend a lot of money to go have conversations about
  what you're going to do and your vision for what you want to do to
  make Latinos' lives better. My argument is have that conversation with
  the voter. Let's let the voter decide and lay out your best case. And
  some of those will align with Republicans. Some of them will align
  with Democrats. And some of them are not sure who and how they align,
  but they want to hear information on how their value set --- back to
  your first question on who we are and where we come from. So someone
  who just came here from Venezuela who had to get red and get out of a
  country where they were being forced out because of a brutal dictator
  looks at that country and issue set way differently than a
  third-generation Texas growing up with a bunch of white people in East
  Texas who sound like an old Mexican redneck like I do. It's just how
  we bring in that information. What I say is I welcome that debate, but
  come to our community. The Bernie Sanders story in ``Tío Bernie''
  showed how I spent \$15 million in a primary with a Democratic
  socialist in six states, and what do you get in return? Fifty percent,
  60 percent, 70 percent of the vote because we invested. So it's
  disheartening for me sometimes that the Democrats still haven't
  learned that lesson when we're allocating resources, but I have to
  continue to have to beat that drum or I wouldn't be doing my job.
\item
  linda chavez\\
  Well, it's a very interesting story that Chuck tells. I was having a
  discussion with another political friend of mine recently, and he
  talked about the white vote and particularly the white suburban vote
  against Trump. And the way he put it was if you get a white suburban
  voter to vote for Joe Biden, it's really two votes because that person
  would have voted for Trump. So that would be one vote, and now he's
  voting for Biden, and that would be another as well. So I think that's
  part of the reason you see that overspending in the white community.
  But one of the problems is that the Latino vote is not a homogeneous
  vote. If you look at voting patterns, for example, in the African
  American community, about 10 percent or less of African Americans vote
  for the Republican candidate in presidential elections, but that is
  not true for Latino voters. Latino voters have in past elections,
  going back really to the 1970s, 1972, Richard Nixon got almost a third
  of Mexican American votes in that election, and you've seen both
  Ronald Reagan and George W. Bush get northward of 40 percent of that
  vote. So I think what, unfortunately I think, Joe Biden is not doing
  is paying enough attention to the fact that he could lose a sizable
  chunk of the Latino vote. And in doing that, he could harm himself and
  harm his chances. Arizona is now a swing state, and if Joe Biden is
  not able to get an overwhelming support in that community, it's going
  to go the way it has in most elections, and that is Republican. So I
  do think that it's possible to reach Latino voters. I, frankly, am
  astonished when I see the level of support for Donald Trump in the
  Latino community, when I see that anywhere from 20 percent to 30
  percent --- some polls show him even above 30 percent in Latino
  support. To me, it's shocking because Donald Trump is absolutely the
  most anti-immigrant, anti-Hispanic political candidate that we have
  seen, I think certainly in my lifetime.
\item
  isvett verde\\
  Yeah, what is that about? I mean, I'm totally fascinated by this. If
  anyone of you can explain this to me, I would love to hear why you
  think that is.
\item
  chuck rocha\\
  I can.
\item
  isvett verde\\
  Because ---
\item
  chuck rocha\\
  I can.
\item
  isvett verde\\
  --- his support has remained pretty steady, right? It was like 28
  percent in 2016 to around 22 percent now. That's pretty incredible.
\item
  chuck rocha\\
  Yeah, and I think I can explain it. Let me just do that right now for
  all you listening out there because I have been dumbfounded by it as
  well, but the explanation is pretty clear if you know what's happening
  on the underbelly of how these campaigns are run. Now, you have a lot
  of things that are happening. Donald Trump has 1,000 percent name ID.
  So when the primary was over, you had what --- I have a Bernie
  hangover because I just spent \$15 million telling everybody why
  Bernie Sanders was the second coming of Cesar Chavez. So that ends in
  the late spring, and we get out. And it's something I don't want to
  live through again and think about it shutting down because it will
  make my eye twitch. But then you had the corona. So the corona there
  jumps up on the scene. Everybody had to shut down their campaigns.
  Everybody was told to go home. Everybody's told not to go to work, not
  to go to school. And what you had every day at 5:00 was Donald Trump
  talking about the coronavirus and what he was going to do. Now, we all
  know that it was crazy. But what would happen every night --- and
  y'all both know this --- is that it would be covered on Univision and
  Telemundo. So he's getting out there every evening talking. Now, he
  ain't got to win all the vote. He ain't even got to win a majority of
  the vote. But just him on with no other narrative out there but just
  him talking about what he was going to do, maybe you lose a point
  here, two points here, three points there. So if you take Bernie
  Sanders kick-butt operation, if you take then the coronavirus and the
  monopoly of all the prime-time news coverage, then you do see him pick
  up 2, 3, 5, up to about 10 points of support, right? That's how his
  support has grown to the point where it is where everybody's like, how
  can that be? Well, boys and girls, for all you taking notes at home,
  that's how it can be.
\item
  isvett verde\\
  Yeah, and something like four points can make a huge difference in
  some of these battleground states, right?
\item
  linda chavez\\
  Well, I think that's absolutely right. I have very mixed feelings
  about this. I am still a conservative on a whole variety of issues. I
  will agree more with a Trump administration than I will with a Biden
  administration. It's just the facts. But what I don't understand is
  why the Biden folks are not paying attention to the Chuck Rochas of
  the world because I am going to vote against Donald Trump because I
  think he's a threat to democracy. I think that it has nothing to do
  with where --- I would probably prefer the appointees to the Supreme
  Court that a second Trump term would produce more than I would a Biden
  administration. Tax policy I'm probably going to be more--- there are
  a lot of reasons why I would agree with Trump. But I think Trump is a
  true threat to Democratic institutions. So I think that's very
  worrisome to me, and the fact that the Biden campaign seems to be so
  late in realizing this could end up having a dramatic effect on the
  election.
\item
  isvett verde\\
  I kind of want to ask the question, I'm curious if you think the
  Republican Party you joined even exists anymore. How do Latinos fit
  into Trump's Republican Party? Do they fit?
\item
  linda chavez\\
  I don't think conservatives fit into Trump's party, quite frankly, and
  I don't know what's going to happen to the Republican Party. {[}ISVETT
  LAUGHS{]} A lot is going to depend on this election. If it's a rout
  --- if Donald Trump goes down in flames and loses the election and
  it's not a squeaker, it's a definitive loss for him, if the United
  States Senate flips and if the Democrats are able to hold on to their
  majority in the House and not lose many seats in the House, then I
  think the Republican Party is going to go through some handwringing,
  and they're going to have to think about the future, and they're going
  to have to decide what that future is going to be. But I don't know
  that I have a lot of faith that the Republican Party which I joined
  --- which, by the way, I didn't join until 1985. I didn't become a
  Republican until I had already joined the Reagan administration. I
  voted twice for Ronald Reagan as a registered Democrat. But I don't
  know that the party of Ronald Reagan and George W. Bush is going to
  return. It's not clear to me given what we've seen that Trump has done
  to the Republican base, and that seems to make all the difference in
  the world, what he's been able to do in his invigorating populism and
  reaching out to some disaffected Democrats who are now Republicans.
  His base in lower-middle-class white America, that base may remain and
  remain within the Republican Party, and it may transform the
  Republican Party in the 21st century.
\item
  chuck rocha\\
  People forget --- and I'll say this before Linda probably says it ---
  is that people forget that George W. Bush got 44 percent of the Latino
  vote, and he did that because he ran commercials. He went to the
  community. And let me stop right there. When I say go to the
  community, I don't mean have a Zoom meeting, fly into a Lulac chapter
  and meet with 15 leaders. I mean go to the community and spend some
  money. What does that mean, Chuck? Oh, let me tell you what that
  means. That means buy TV advertisement, radio advertisement, and
  digital advertisement, that local paper down at the market that your
  grandmother goes and gets in Spanish every Friday because it's got the
  sales coupons in it. Advertise in that paper. I mean you spend money
  talking to brown people like you spend money talking to white people.
  That's point A. Point B is that Joe Biden is not doing a bad job
  reaching out to Latinos. He's doing the same job that every other
  Democratic presidential general election candidate has done to try to
  court our vote. And what I'm getting at there is that that's wrong.
  They all have good intentions. In my mind, Joe Biden cares about the
  Latino vote. He cares about Latino staff. He cares about running a
  good operation, and a lot of his friends, a lot of my friends work on
  this campaign. The problem is and the whole reason I wrote the book
  ``Tío Bernie'' was to say there's a different way to do this. When we
  set up the Bernie Sanders campaign, we didn't have a Latino
  department, and we dominated the vote because we made the Latino
  outreach integrated into the overall campaign. We didn't say, oh,
  Linda's going to be over this part of the cabinet because she's a
  Latina. No, we're putting her there because she's talented and can do
  the job. She just happens to be a Latina. That's the formula to
  success that I want to see Joe Biden and all of them do. I'm not going
  to throw rocks at Joe Biden's campaign. I think that they're doing a
  good job. It's just that it can be done so much better.
\item
  isvett verde\\
  Yeah, I mean, I don't think I've ever seen voters turn out as
  passionately as they did for Bernie, at least Latino voters. I think
  while other presidents have been able to capture a significant
  percentage of that vote, I don't think I saw voters feel as passionate
  for a candidate as they did for Bernie and not even Julián Castro, who
  actually was Latino.
\item
  chuck rocha\\
  It's the old adage if a tree falls in the woods, right? Julián had a
  great message, and he would have been wonderful. But he had, again, no
  money to go tell people who he was and what he stood for.
\item
  linda chavez\\
  Could I could I raise another issue here? And it's an uncomfortable
  one, but I think, as Latinos, we need to talk about it, and that is
  Latinos and their attitudes towards African Americans. I think that
  the Black Lives Matter protests that we're seeing in the streets,
  certainly some of the rioting that we've seen in certain places may
  end up rebounding to the benefit of Donald Trump in certain parts of
  the Latino community because --- and I've written about this for years
  --- this notion that Blacks and Latinos are all on the same
  wavelength, that we are one big, happy community of disadvantaged
  people and our goals are the same, that may be true demographically.
  That may be truer demographically. But I think that we have to
  recognize that there is competition and there is conflict and there is
  prejudice that works against the whole notion of this united front of
  Black and Latino voters.
\item
  isvett verde\\
  So I want to get into this idea of the support that he has which is so
  surprising between Black and Latino men. We're seeing some polls that
  he has enough support that it can also make a difference. Is it
  because of his machista message? Is that appealing? Is it that simple?
  I don't know. I'm eager to hear what you think about that.
\item
  linda chavez\\
  Well, I do think that the kind of macho image is something that
  probably benefits him. Again, he's sort of standing up to things that
  they don't necessarily like. And while there certainly is demographic
  overlap and there should be some commonality between Black and Latino
  voters --- and there is, I mean, in large part. But as Chuck has made
  the point, Joe Biden is going to win a majority of Latino voters. The
  question is how big a majority is he going to win? And the size of
  that majority could make the difference in some important states. It's
  not going to make a difference in California, but it could make a
  difference in Texas, and it definitely will make a difference in
  Arizona. So one of the things that I think Trump is doing right now
  that I think maybe Biden could use to his advantage in appealing to
  Latino men is the way in which Trump talks about the military, the
  whole you know losers and suckers argument that we heard this week
  that supposedly Donald Trump referred to fallen servicemen, people who
  had died in World War I as losers and suckers, and we've seen that
  kind of language. I think there is a latent patriotism within the
  Latino community. A lot of Hispanic men have served in the military.
  And I think when Trump starts badmouthing the US military and bad
  mouthing the courage that it takes to go and defend your country, I
  think that is an opening that Biden should exploit. But as Chuck says,
  he has to recognize that he has to target that message. He has to go
  in there with enough resources and with a message tailored to that
  community that it's going to have some impact. {[}MUSIC PLAYING{]}
\item
  isvett verde\\
  Let's take a quick break, and we'll be right back.

  OK, we're back. So I think we've had too much agreement for a podcast
  called ``The Argument,'' so let's get into where I think you may
  disagree. What policies do you think best speak to Latino concerns,
  Linda?
\item
  linda chavez\\
  Well, I think the economy is important, and I do believe that
  Democrats who are much more enamored of taxes and redistributing
  wealth may actually not be doing a favor to Hispanics who want to
  start their own businesses, who are entrepreneurial. I think education
  is another arena. I think that school choice gives an opportunity for
  people who live in communities where public schools are not doing well
  an opportunity to be able to choose their own schools outside the
  public-school system. There's a growing number, for example, of
  Latinos who are evangelical, and evangelical parents may want to send
  their kids to a religious-based school. And I think that having the
  opportunity to be able to choose the school your children go to and to
  be able to allow your own money to be used through a voucher or some
  other means, a tax credit, is something that can appeal. Again, it's
  not to the entire Hispanic community, but certainly it is to a segment
  of it.
\item
  chuck rocha\\
  Well, I would think about policies a little bit differently,
  obviously, working for Bernie Sanders. People were not shocked to know
  that Medicare for All was the number-one issue with Latinos through
  that entire process before we got to corona. Medicare for All was
  hugely popular with Latinos because they are underinsured, so that is
  just a big part of what their lexicon is. Thinking about small
  businessmen, thinking about people who want to --- and Latinos were
  very aspirational in all of our polling and things I've seen over
  generations, and they do want to create and they do create lots of
  wealth around being entrepreneurs, but they also understood the
  message of the two-tiered system that we live in. So talking about the
  haves and the have-nots, not that you don't want to have the haves.
  I'm a small businessman in Washington, D.C. I pay \$0.46 for every
  dollar that comes into my firm. Now how, Chuck, would you say is that
  possible? `Cause this tax thing is a very big deal for me. I'm at the
  highest tax bracket because I make a couple hundred thousand dollars a
  year, so let's call that 38 percent. And then I live in D.C., so
  there's another \$0.10 popped on that. You go from 38 percent to 48
  percent there pretty quickly. My problem is is that the folks who have
  money and lawyers and income and wealth, a la Republicans, figure out
  a way to get around them having to pay 48 percent. So they're paying
  GE 10 percent or Bezos no tax while my tax dollars are used to prop up
  these fellas. So Latinos get that part of the messaging. And it's just
  like you can be aspirational, but also understand that the system is
  rigged.
\item
  isvett verde\\
  And so what do you think of Chuck's point on taxes, Linda?
\item
  linda chavez\\
  Well, I think he is right that one of the reasons you don't
  necessarily raise in the aggregate more money when you raise taxes is
  that people who are very smart and who don't want to pay taxes have
  other people they hire to figure out how not to pay those taxes. And
  so the kind of loopholes --- I mean, there is no such thing as a flat
  tax. And even when I was in the Reagan administration, I worked on the
  1986 tax-reform bill, and originally there were going to be basically
  three tax brackets, and most deductions were going to be eliminated,
  and it was pretty much going to be if you fall into this bracket,
  you're going to pay this amount in taxes. Then the lobbyists got
  working, and everybody from the Knights of Columbus, who wanted to
  make sure that the deduction for charitable gifts were was not
  eliminated, to the housing industry didn't want to see the housing
  mortgage check go by the wayside. So Chuck's right. Effective tax
  rates are very different than the marginal tax rates.
\item
  chuck rocha\\
  We need to remember that Latinos are younger, that we're just younger
  overall for the listeners out there. Like when the average Latino in
  this country is less than 30 years old --- and that's just the fact
  that we are younger demographically. Now, the Latino voter is a little
  bit older than that because guess what? Only older voters go and vote.
  And what we found during the Bernie Sanders campaign is that these
  younger Latinos --- is this surprising? {[}LAUGHING{]} I'm thinking
  about my 30-year-old son --- is that their view of policy is much
  different than their parents' who have a mortgage, who have a car
  payment, who's figuring out how, if you come from a point of privilege
  like I do and I see my privilege, of trying to take care of my
  grandkids' college education someday hopefully. So guess what? He's a
  little more liberal than even his very liberal father because he views
  the lens of public policy through a different scope. And so I think we
  need to recognize that that's very different in the Latino population
  to a Latino voter who Linda may be talking about who's older Cuban,
  older Mexican, who's just more Catholic, more conservative. So there's
  more inroads for Republicans there. That 44 percent George Bush got
  was a way different demographic than what you're going to see in 2010
  because the population is growing at such a fast rate, so it brings
  that average number of how old it is younger.
\item
  isvett verde\\
  I think that's such an excellent point. You even see that with the
  Cuban community in Miami, right? Younger Cubans have very different
  perspectives from their parents on politics, on policy, on the way the
  direction the country is going in, and they tend to skew more
  progressive. And I think you're seeing that that vote is actually
  shifting because of that. I also want to get back to this Medicare for
  All question. Linda, do you think Medicare for All is something that
  you'd support?
\item
  linda chavez\\
  Well, I have to tell you, this is probably one of the few areas over
  the last 30 years or so where my viewpoint has shifted somewhat. I
  used to be very adamantly opposed to government-run, single-payer-type
  systems. I thought it would, in fact, make care more accessible to
  more people, but I also thought it would lower the quality of care. I
  now think that the system is so broken --- and certainly the pandemic
  has given us insight into what that means --- that we have to do
  something to fix our medical system, and that does mean a much more
  broad-based system of care that's successful to more people. I have
  always believed that what people really worry about is not so much the
  visit to the doctor when you've got the sniffles but the catastrophic
  event. So I was always in favor of treating health insurance much like
  we do home insurance, that if your house burns down, you get
  reimbursed by the insurance company, and you build a new house. But if
  your furnace breaks, that's on you. You have to pay for that. I still
  think that perhaps there is some system we could come up with. It
  would be absolute universal availability for catastrophic insurance
  but then give more choice to the kinds of routine care. And I think if
  there was more competition, you might, in fact, see providers coming
  up with more willingness to lower their fees if those fees were paid
  directly instead of through insurance companies. But, look, it is a
  complicated system. I don't think anybody has the answer. I don't
  think Medicare for All, as somebody--- probably the only person on
  this podcast who's actually on Medicare will tell you Medicare is not
  all that it is cracked up to be and can be not as good as private
  insurance. I've had both, and Medicare is often inefficient. It often
  makes you go through hoops to get testing and other things done that
  you should have the ability to do. So I don't think Medicare for All
  is necessarily the answer, but we do need to tackle this problem.
\item
  isvett verde\\
  I also think that while we think Medicare for all isn't the answer, we
  think of Medicare in its current form, right?
\item
  linda chavez\\
  Correct.
\item
  isvett verde\\
  If we actually funded and were intentional about Medicare, maybe that
  could be a way forward, right? When you think about the pandemic,
  which has turned everything on its head, but it's impacted Black and
  brown people disproportionately. Our communities die in far, far
  higher numbers. A lot of them just didn't have access to health care.
  So you're sick. You just, you don't go to the hospital, and that's a
  problem. And also I think that Trump is playing into this fear that
  people that come from countries like Venezuela and Cuba that
  socialized health care is the worst thing that can happen, but those
  were very dysfunctional governments, and that's not an example of what
  socialized health care can look like. So what are both your
  expectations for this election? You're in very different political
  circles. What are you hearing or seeing? Unfortunately, Walter Mercado
  is no longer with us, so we're on our own. Tell me, what do you see in
  the stars?
\item
  chuck rocha\\
  A, I just finished doing a Walter Mercado piece of mail in North
  Carolina. Just so you know, when I say cultural competency, baby,
  that's what I'm talking about. {[}ISVETT LAUGHS{]} So I made a Walter
  piece of mail and sent it to every Latino in North Carolina, not from
  me but from this great group of community-based Latino organizations
  called N.C. Poder. I think that they give me the most hope is these
  great local groups of Latinos who are reaching out in a nonpartisan
  way, some of them partisan. But like N.C. Poder, LUCHA in Arizona,
  like these groups that are tied to the community who aren't only
  registering Latinos and turning them out--- they're also helping
  provide government service and helping them with citizenship and
  helping them with things. So I do have a great hope. It would be good
  for me if Donald Trump really just barely lost because then I could
  tie Donald Trump around the neck of every commonsense Republican like
  Linda that I know and just say, here, you've got to wear this for at
  least 20 more years {[}LINDA LAUGHS{]} when she don't want nothing to
  do with it just so it helps me win more elections. That's being very
  self-serving. Let me be clear. But if he gets blown out and we win
  back the House and we win a majority of over 50 in the Senate, then
  people will look at me then that these Democrats, who I agree with
  about 75 percent of what they're doing, have to deliver. And then it
  becomes a whole new heap of responsibility for the senior-most Latino
  Democratic operative in D.C. or in the country, who I am, to say,
  look, yeah, I helped elect a lot of these fools, but now they're
  acting like a fool. And now we all got to go hold them accountable and
  remind them every day that it's easier to hold a crazy Democrat
  accountable than it is a crazy Republican because we can go beat a
  crazy Democrat in a primary. A.O.C. proved that. So that's going to be
  kind of my thinking of long-term strategy. I don't want to be the dog
  who caught the car. You know what I mean?
\item
  isvett verde\\
  Do you think someone like A.O.C. is the future of the party?
\item
  chuck rocha\\
  I think she's a future of part of the party. I think that's what makes
  us a unique thing. I used A.O.C. in that Bernie thing. People think,
  well, she's liberal, Chuck, and she don't represent this and that.
  Well, what she is is she is a smart, beautiful, charismatic Latina who
  every grandmother saw their granddaughter being. My granddaughter
  could be a congresswoman. So I didn't talk about her policies as much
  as I talked about her aspiration to be able to overcome the two-party
  system and kind of jump in, and I used that with grandmothers in Iowa,
  Latina grandmothers. So she's part of the future of the party for
  sure.
\item
  isvette verde\\
  Linda, you're leaning in. Tell me what you think.
\item
  linda chavez\\
  Well, as you might imagine, I have very mixed emotions. I do care a
  great deal about immigration. I've spent the last 10 years of my life
  almost exclusively focused on the immigration issue. And if Donald
  Trump is reelected, I see a very dismal future in terms of that issue.
  I don't think we will see immigration reform. I have said for years
  that the way to stop illegal immigration is to give people who want to
  come here to work and who are going to contribute to our society a way
  to come here legally, but that means expanding legal immigration. And,
  of course, Trump is all about reducing it. So I don't want to see a
  narrow victory for the Democrats. And I know this is going to sound
  bizarre since I am still a registered Republican, but I think because
  I am a conservative, because I do hold many things about the
  Republican Party dear, I want to see a transformed Republican Party. I
  want to see the Republican Party revert to the free-market,
  free-trade, entrepreneurial Republican Party that welcomed with open
  arms people as long as they came to the United States, letting people
  come here in greater numbers legally. So I'd like to see not the kind
  of narrow victory that Chuck suggests. But I hope that A.O.C. is not
  the future of Latinos in politics, Latinas in politics in America. I'd
  much rather see the kind of more moderate folks like Susana Martinez,
  who was governor of New Mexico. I'd like to see those kinds of people
  coming to the fore. It would be great to me if the Republicans could
  get back to getting 44 percent, 45 percent, maybe 50 percent,
  51percent of Latino votes but only with a candidate who speaks to the
  kinds of principles that I think the Republican Party once stood for
  and, unfortunately, in the era of Trump has not stood for.
\item
  chuck rocha\\
  If you set her --- if you set A.O.C.`s policies to the side, whether
  we agree or we disagree --- and I agree with all of her policies, for
  the most part. If we think about her story, though, of how this young
  woman was a waitress, got involved with a community-based
  organization, and took on the second- or the third-most-powerful
  Democrat in the U.S. Congress and out organized him in a Democratic
  primary, that's what I'm about.
\item
  linda chavez\\
  She certainly has charisma, and I have to tell you when she was first
  elected and many of my Republican Facebook friends were all posting
  the video of her dancing on the rooftop and they were doing it in a
  way to demean her and to act as if she wasn't a serious person, I
  watched it and I thought this is absolutely delightful. And she is, as
  you say, Chuck, as a person in terms of the kind of charisma she
  brings, she is appealing. I just happen to disagree with virtually
  everything she stands for, not just in domestic policy but probably
  more importantly on foreign policy.
\item
  chuck rocha\\
  There you go. Y'all got your disagreement right there. That's what we
  came here for, ``The Argument.'' There we go.
\item
  linda chavez\\
  I'm still a right winger when it comes to foreign policy.
\item
  chuck rocha\\
  There we go.
\item
  isvett verde\\
  {[}LAUGHS{]} Linda, do you think --- looking forward, do you think the
  Republican Party will be able to separate itself from Trump's virulent
  xenophobia against Latinos? He launched a campaign referring to
  Mexicans as rapists. He characterized caravans as invasions, and his
  party has stood by him. Can they come back from that, or is that
  really what the party thinks of Latinos?
\item
  linda chavez\\
  Well, it's interesting. Prior to 2015 when Trump launched his
  campaign, I was living in Colorado. I was out speaking to Republican
  groups every week, and I was bringing a proimmigration message to
  them. I was out talking to them about the facts, the fact that illegal
  immigration, even at that time, was down. There had been many more
  people coming 15 years earlier than there were in 2015. I talked to
  them about the assimilation of Hispanics in the United States. I wrote
  frequently about the way in which Hispanics have fallen into the
  pattern that virtually all immigrants have, and that is to start on
  the bottom rungs but to rapidly move up. Second-generation Hispanics
  were English speakers. By third generation, some --- in fact, many
  were English monolingual speakers. They had lost the ability to speak
  Spanish. And I was getting a very receptive audience in Republican
  circles, in conservative circles. That changed on a dime with Donald
  Trump. And one of the things that I think happened is that was for the
  last 30, 40 years, maybe longer in the United States, it was impolite
  to say prejudiced things. If you had prejudiced feelings, certainly in
  mixed company you kept them to yourself. You didn't speak them out
  loud. Well, suddenly Donald Trump gave permission for people to take
  their prejudices and pushed them to the front and engage in
  discussions and talk that was mean, that was racist, and he gave them
  permission to do that. I've always know you may have pulled back the
  rock and revealed the squirmy things underneath it. I'd like to put
  the rock back because I do think there is some value in simply making
  it not permissible in polite company to express views that are
  hateful. And with Trump gone, maybe some of that will come back.
\item
  isvett verde\\
  I'm curious because you mentioned that you did change parties, and so
  why did you ---
\item
  linda chavez\\
  Become a Republican?
\item
  isvette verde\\
  --- change --- why did you--- yes, exactly. {[}LAUGHS{]}
\item
  linda chavez\\
  Well, I mean, it's funny. My second book was called an unlikely
  conservative. And one of the reviewers of my book noted the fact that
  I talked about this transformation that had taken place. And I think
  it was a woman, and she said there was no transformation. She was
  always a conservative, even though she may have been a Democrat. And I
  think, for the most part, that's true. I think the Democratic Party of
  my youth, my father's Democratic Party was a very different party than
  the Democratic Party of the early 1970s. It was really George McGovern
  who changed the party in ways that I found unacceptable. Foreign
  policy is a very big deal to me. I am a Cold War anticommunist, so I
  saw Jimmy Carter as basically ceding much of the world to the Soviet
  sphere during his tenure. I saw the map of the free world began to
  change and Latin America, Africa, other places moving into the more
  autocratic and totalitarian left, and that was the reason that I first
  voted for Ronald Reagan. It was solely on foreign-policy issues. I
  only became more conservative on economic issues, as you might expect,
  as I started earning more money and started paying more taxes. But
  foreign policy is still a very big factor in my vote, and it's one of
  the reasons I couldn't support Bernie Sanders. And if Bernie Sanders
  had won the nomination, I would not vote for Trump, but I certainly
  would not have voted for Bernie Sanders.
\item
  isvette verde\\
  And, Chuck, what makes you a progressive rather than a conservative?
\item
  chuck rocha\\
  I think growing up and not having much. I think understanding the
  concept of my vision of what the government should do to stand up for
  its citizens, that the power not be concentrated with just the wealthy
  of the country who gets to make the law and hire the lobbyists and
  have the tax attorneys. It's just these basic things to where I just
  know that I'm not going to align with a party that doesn't speak to my
  personal values, and that's just where I've always stood. I wrote in
  my book --- it's funny--- of how did you get involved in politics? And
  I write about the NAFTA proposition --- and people like to throw that
  in my face that it was Bill Clinton, a Democratic president, and it
  was. And that factory I worked in in East Texas and watching how it
  devastated that whole little community in East Texas when that tire
  factory went away. I think that you know this, Isvett, that I'm one of
  the few professionals in this city who never went to college. I've
  lived a real life, and I've just been so blessed and so lucky to be
  able to do the things that I've done and be in the Oval Office and get
  to run presidential campaigns.
\item
  linda chavez\\
  Let me just sort of get in here because people don't necessarily know
  my story, and Chuck and I probably come from backgrounds that are more
  similar than you might expect. My father was a house painter with a
  ninth-grade education. My mother worked primarily in restaurants until
  I got to be a teenager, and then she worked in department stores. I
  ended up going to college quite by accident. I was working in a beauty
  salon. I was a receptionist. And a girlfriend came by to go to lunch
  and walked over to the University of Colorado Denver center and
  decided she wanted to enroll in some classes. And I thought, well,
  maybe I'll do that too. And I met my husband the first semester, and
  my husband comes from an upper-middle-class Jewish family, son of a
  doctor and a college dance professor, and the rest is history. We've
  been married for 53 years. But my background is also working class,
  and I never forget that. And I know that much of what I've been able
  to accomplish has been by virtue of luck and being in the right place
  at the right time. A lot of it is hard work, and I certainly have
  worked hard, but a lot of it is really luck.
\item
  isvett verde\\
  I actually may cry because this is all just so lovely to be on here
  today and talking to you all and hearing about our stories and our
  perspectives and our community. I mean, I started kindergarten not
  even speaking English. I went to community college, and now I'm here
  talking to you all. And it's a testament to the promise of this
  country, and I hope that that promise isn't lost and that the future
  will look different, that our country can come back after these past
  few years and these horrid past few months.
\item
  linda chavez\\
  Well, I will just say that I think you're absolutely right, and the
  future of the country is at stake in this election. The direction that
  we're going to go for the future is very much in the hands of the
  American voters. And I hope that Latinos who have, over the history of
  our community, have not voted as much as many other groups, I hope
  that they get out, I hope they register, and I hope they vote.
\item
  isvette verde\\
  And with that, Linda Chavez, Chuck Rocha, thank you both so much for
  helping me add a little sázon to ``The Argument'' today. This was so
  much fun.
\item
  linda chavez\\
  Thank you, Isvett. Great to be with you.
\item
  chuck rocha\\
  Thank you, Isvett. Appreciate it.
\item
  isvett verde\\
  And that's our show this week. Thank you for listening. If you have an
  election question that you want to hear debated on ``The Argument,''
  leave us a voicemail at 347-915-4324. You can also email it to us at
  \href{mailto:argument@NYTimes.com}{\nolinkurl{argument@NYTimes.com}}.
  ``The Argument'' is a production of The New York Times Opinion
  section. The team includes Phoebe Lett, Kristin Lin, Paula Szuchman,
  Isaac Jones, Kathy Tu, Alison Bruzek, and Vishakha Darbha. Don't
  worry, dear ``Argument'' listeners, Ross and Michelle will be back in
  your ears next Friday.

  {[}LAUGHING{]} Bienvenidos al ``Argumento.'' {[}LAUGHS{]}
\end{itemize}

\href{https://www.nytimes3xbfgragh.onion/column/the-argument}{\includegraphics{https://static01.graylady3jvrrxbe.onion/images/2018/10/03/opinion/the-argument-album-art/the-argument-album-art-square320-v3.png}The
Argument}Subscribe:

\begin{itemize}
\tightlist
\item
  \href{https://itunes.apple.com/us/podcast/id1438024613}{Apple
  Podcasts}
\item
  \href{https://www.google.com/podcasts?feed=aHR0cHM6Ly9yc3MuYXJ0MTkuY29tL3RoZS1hcmd1bWVudA\%3D\%3D}{Google
  Podcasts}
\end{itemize}

\hypertarget{how-to-win-the-latino-vote-1}{%
\section{How to Win the Latino
Vote}\label{how-to-win-the-latino-vote-1}}

\hypertarget{a-podcast-debate-over-the-political-future-of-the-growing-demographic-of-voters-1}{%
\subsection{A podcast debate over the political future of the growing
demographic of
voters.}\label{a-podcast-debate-over-the-political-future-of-the-growing-demographic-of-voters-1}}

With Isvett Verde, Chuck Rocha and Linda Chavez

Transcript

transcript

Back to The Argument

bars

0:00/0:00

-0:00

transcript

\hypertarget{how-to-win-the-latino-vote-2}{%
\subsection{How to Win the Latino
Vote}\label{how-to-win-the-latino-vote-2}}

\hypertarget{with-isvett-verde-chuck-rocha-and-linda-chavez-1}{%
\subsubsection{With Isvett Verde, Chuck Rocha and Linda
Chavez}\label{with-isvett-verde-chuck-rocha-and-linda-chavez-1}}

\hypertarget{a-podcast-debate-over-the-political-future-of-the-growing-demographic-of-voters-2}{%
\paragraph{A podcast debate over the political future of the growing
demographic of
voters.}\label{a-podcast-debate-over-the-political-future-of-the-growing-demographic-of-voters-2}}

Friday, September 11th, 2020

\begin{itemize}
\item
  chuck rocha\\
  I'm Chuck Rocha.
\item
  linda chavez\\
  I'm Linda Chavez.
\item
  isvett verde\\
  I'm Isvett Verde, and this is ``The Argument.'' {[}THEME MUSIC
  PLAYS{]}

  OK, so we're not ``The Argument hosts you're used to. I'm Isvett, a
  writer and editor in Opinion focusing on the Latino experience in the
  United States. Michelle and Ross let me jump into the host seat today
  for a special episode focused on the Latino vote. It's one of the
  fastest-growing demographics of voters in the United States. But in
  the Trump era, which party makes a better pitch to Latinos? I'm
  honored to be joined today by two of Opinion's contributors. Linda
  Chavez is a senior fellow at the Niskanen Center and the director of
  the Becoming American Institute, and she was one of the
  highest-ranking women in Reagan's White House. Linda, welcome to ``The
  Argument.''
\item
  linda chavez\\
  Thank you.
\item
  isvett verde\\
  And joining me and Linda is Chuck Rocha. He was a senior campaign
  advisor to Bernie Sanders. He's also the founder of Solidarity
  Strategies, a political consulting firm that supports progressive
  candidates. And while you may not be able to see his cowboy hat,
  you'll no doubt hear the Texas in his voice. Chuck, thank you for
  joining us today.
\item
  chuck rocha\\
  Thanks for having me.
\item
  isvett verde\\
  So there's often talk of this thing called a ``Latino voting bloc.''
  Why don't we just start with an easy question? Is there such a thing
  as a Latino voting bloc?
\item
  chuck rocha\\
  Well, I would say that there's definitely a bloc, but I would say that
  that bloc is multicultural, multidynamic, multigenerational, and I'd
  even say multicolored. People who think about our vote think about us
  all being Mexican, think about us all speaking like I do as a man from
  Texas, and they just don't realize how different and how many shades
  of the rainbow there is. Not only with the color of our skin, but
  where we come from and our countries of origin and I would say, for
  someone like me, how many generations your family has been here in
  America. For many examples, I like to use the one that's my favorite
  is I'm a third-generation Texan. It's been a long time since my folks
  lived in Guanajuato, Mexico. And so the way that my family, my son,
  and even my twin grandsons view the world --- all Latinos and all
  Latino voters at some point in their lives --- could not be more
  different than the experience maybe that a Cuban or a Dominican or
  Puerto Rican. All of these different cultures make our voting blocks
  so unique. So we are a block. There are just so many pieces of that
  block.
\item
  linda chavez\\
  I know that usually on these kinds of podcasts we're supposed to
  disagree with each other, but I couldn't agree more with Chuck on that
  point. Hispanics are very diverse. Chuck talked about being a
  third-generation Texan. Well, my family came over in 1601. Pedro Duran
  y Chavez came from Sevilla and ended up in Santa Fe, New Mexico. He
  was part of the original expedition that founded New Mexico. My family
  has lived on soil that is now part of the United States going back 10
  or 11 generations. And so whenever people tell me to go back where I
  come from, it's a little hard. And by the way, I don't think they mean
  Spain. {[}CHUCK LAUGHS{]}
\item
  isvett verde\\
  I think between the three of us, we span the Latino American
  experience. You know I came from Cuba in a boat in 1980 and grew up in
  Miami surrounded by Cubans, in a way feeling like I'd never left the
  island. And our stories are an example of how disparate the Latino
  American experience can be in this country. All right, so now we know
  that Latinos are a diverse bloc, and I'm curious. What do we talk
  about when we say Latino issues? What are they to you? Does such a
  thing exist?
\item
  linda chavez\\
  Well, I think there are some issues that obviously have more
  importance in the Latino community. Immigration is certainly one of
  them, but when you ask Latino voters themselves what issues matter to
  them, immigration is not usually number one on that list. There are
  issues like the economy, which I think cut through. And probably if
  you were to look at public-opinion polls of Latino voters, the economy
  tends to be very high up on that list and may, in fact, surpass things
  like immigration, which you might think would have more salience in
  the Hispanic community given that about half of the adult Hispanic
  population in the United States is, in fact, foreign born.
\item
  chuck rocha\\
  I'd break that down a little bit differently of issues. What does it
  mean to be a Latino issue? I think there's a set of Latino issues for
  Latinos. I think there's a set of issues for Latino voters. The
  problem is how white pollsters ask the question, and that is what is
  the most important issue for you this election cycle? And to like what
  Linda said is normally that's the jobs and the economy. Latino voters
  in that sense will act a lot like a white voter. But I just find that
  there is a nuance between an emotional issue that a Latino will tell
  you is not their most important issue to go vote today. It will always
  be jobs, the economy, the coronavirus, education, something, whatever
  that particular voter feels about. But if you want to get to the
  emotion, you'll notice when Linda did her intro --- not to pick on
  Linda --- but she talked about the pride that she felt in her family,
  and going back that many years has been instilled in her to
  regurgitate that story in such a prideful way because, just like I
  talked about my family, it's such a prideful thing, right? So when you
  want to talk about an emotional issue, immigration will never be in
  the top two or three of a Latino voters. But if you want to talk about
  tapping into an emotion, tap into what Isvett said about getting on a
  boat, coming here, getting with your family and moving from anywhere.
  Like, that's an emotional trauma or emotional beautiful story or both
  all wrapped up.
\item
  isvett verde\\
  Chuck, you raised a great point about this mix of pride and trauma. I
  think for me personally, I experience it that way, right, the sense of
  being proud of being Latina but also the trauma of immigration, of
  leaving the country that you're from and building roots in a different
  one. And now that we've sorted out what issues matter to Latino
  Americans, I have a feeling you two disagree about the role politics
  can play in meeting those issues. So, Linda, I'm going to guess you
  don't think Bernie Sanders is the way. Am I right?
\item
  linda chavez\\
  I don't think Bernie Sanders is the way. That's absolutely true. But I
  certainly don't think Donald Trump is the way either.
\item
  isvett verde\\
  Chuck, you turned out Latino voters in big numbers in Nevada,
  Colorado, and Texas. Why do you think voters connected with ``Tío
  Bernie``?
\item
  chuck rocha\\
  It's not science, and it's not hard. And I wrote this book ``Tío
  Bernie'' to explain to everyone in our community if you want our vote
  and if you want to come into our community and get our vote, you
  should spend resources and come have a conversation with our
  community. And I wanted to prove once and for all after 31 years of
  doing this if I could do it with Bernie Sanders that anybody could do
  it with another candidate if they do it in the way that I would
  outline, which is spend a lot of money to go have conversations about
  what you're going to do and your vision for what you want to do to
  make Latinos' lives better. My argument is have that conversation with
  the voter. Let's let the voter decide and lay out your best case. And
  some of those will align with Republicans. Some of them will align
  with Democrats. And some of them are not sure who and how they align,
  but they want to hear information on how their value set --- back to
  your first question on who we are and where we come from. So someone
  who just came here from Venezuela who had to get red and get out of a
  country where they were being forced out because of a brutal dictator
  looks at that country and issue set way differently than a
  third-generation Texas growing up with a bunch of white people in East
  Texas who sound like an old Mexican redneck like I do. It's just how
  we bring in that information. What I say is I welcome that debate, but
  come to our community. The Bernie Sanders story in ``Tío Bernie''
  showed how I spent \$15 million in a primary with a Democratic
  socialist in six states, and what do you get in return? Fifty percent,
  60 percent, 70 percent of the vote because we invested. So it's
  disheartening for me sometimes that the Democrats still haven't
  learned that lesson when we're allocating resources, but I have to
  continue to have to beat that drum or I wouldn't be doing my job.
\item
  linda chavez\\
  Well, it's a very interesting story that Chuck tells. I was having a
  discussion with another political friend of mine recently, and he
  talked about the white vote and particularly the white suburban vote
  against Trump. And the way he put it was if you get a white suburban
  voter to vote for Joe Biden, it's really two votes because that person
  would have voted for Trump. So that would be one vote, and now he's
  voting for Biden, and that would be another as well. So I think that's
  part of the reason you see that overspending in the white community.
  But one of the problems is that the Latino vote is not a homogeneous
  vote. If you look at voting patterns, for example, in the African
  American community, about 10 percent or less of African Americans vote
  for the Republican candidate in presidential elections, but that is
  not true for Latino voters. Latino voters have in past elections,
  going back really to the 1970s, 1972, Richard Nixon got almost a third
  of Mexican American votes in that election, and you've seen both
  Ronald Reagan and George W. Bush get northward of 40 percent of that
  vote. So I think what, unfortunately I think, Joe Biden is not doing
  is paying enough attention to the fact that he could lose a sizable
  chunk of the Latino vote. And in doing that, he could harm himself and
  harm his chances. Arizona is now a swing state, and if Joe Biden is
  not able to get an overwhelming support in that community, it's going
  to go the way it has in most elections, and that is Republican. So I
  do think that it's possible to reach Latino voters. I, frankly, am
  astonished when I see the level of support for Donald Trump in the
  Latino community, when I see that anywhere from 20 percent to 30
  percent --- some polls show him even above 30 percent in Latino
  support. To me, it's shocking because Donald Trump is absolutely the
  most anti-immigrant, anti-Hispanic political candidate that we have
  seen, I think certainly in my lifetime.
\item
  isvett verde\\
  Yeah, what is that about? I mean, I'm totally fascinated by this. If
  anyone of you can explain this to me, I would love to hear why you
  think that is.
\item
  chuck rocha\\
  I can.
\item
  isvett verde\\
  Because ---
\item
  chuck rocha\\
  I can.
\item
  isvett verde\\
  --- his support has remained pretty steady, right? It was like 28
  percent in 2016 to around 22 percent now. That's pretty incredible.
\item
  chuck rocha\\
  Yeah, and I think I can explain it. Let me just do that right now for
  all you listening out there because I have been dumbfounded by it as
  well, but the explanation is pretty clear if you know what's happening
  on the underbelly of how these campaigns are run. Now, you have a lot
  of things that are happening. Donald Trump has 1,000 percent name ID.
  So when the primary was over, you had what --- I have a Bernie
  hangover because I just spent \$15 million telling everybody why
  Bernie Sanders was the second coming of Cesar Chavez. So that ends in
  the late spring, and we get out. And it's something I don't want to
  live through again and think about it shutting down because it will
  make my eye twitch. But then you had the corona. So the corona there
  jumps up on the scene. Everybody had to shut down their campaigns.
  Everybody was told to go home. Everybody's told not to go to work, not
  to go to school. And what you had every day at 5:00 was Donald Trump
  talking about the coronavirus and what he was going to do. Now, we all
  know that it was crazy. But what would happen every night --- and
  y'all both know this --- is that it would be covered on Univision and
  Telemundo. So he's getting out there every evening talking. Now, he
  ain't got to win all the vote. He ain't even got to win a majority of
  the vote. But just him on with no other narrative out there but just
  him talking about what he was going to do, maybe you lose a point
  here, two points here, three points there. So if you take Bernie
  Sanders kick-butt operation, if you take then the coronavirus and the
  monopoly of all the prime-time news coverage, then you do see him pick
  up 2, 3, 5, up to about 10 points of support, right? That's how his
  support has grown to the point where it is where everybody's like, how
  can that be? Well, boys and girls, for all you taking notes at home,
  that's how it can be.
\item
  isvett verde\\
  Yeah, and something like four points can make a huge difference in
  some of these battleground states, right?
\item
  linda chavez\\
  Well, I think that's absolutely right. I have very mixed feelings
  about this. I am still a conservative on a whole variety of issues. I
  will agree more with a Trump administration than I will with a Biden
  administration. It's just the facts. But what I don't understand is
  why the Biden folks are not paying attention to the Chuck Rochas of
  the world because I am going to vote against Donald Trump because I
  think he's a threat to democracy. I think that it has nothing to do
  with where --- I would probably prefer the appointees to the Supreme
  Court that a second Trump term would produce more than I would a Biden
  administration. Tax policy I'm probably going to be more--- there are
  a lot of reasons why I would agree with Trump. But I think Trump is a
  true threat to Democratic institutions. So I think that's very
  worrisome to me, and the fact that the Biden campaign seems to be so
  late in realizing this could end up having a dramatic effect on the
  election.
\item
  isvett verde\\
  I kind of want to ask the question, I'm curious if you think the
  Republican Party you joined even exists anymore. How do Latinos fit
  into Trump's Republican Party? Do they fit?
\item
  linda chavez\\
  I don't think conservatives fit into Trump's party, quite frankly, and
  I don't know what's going to happen to the Republican Party. {[}ISVETT
  LAUGHS{]} A lot is going to depend on this election. If it's a rout
  --- if Donald Trump goes down in flames and loses the election and
  it's not a squeaker, it's a definitive loss for him, if the United
  States Senate flips and if the Democrats are able to hold on to their
  majority in the House and not lose many seats in the House, then I
  think the Republican Party is going to go through some handwringing,
  and they're going to have to think about the future, and they're going
  to have to decide what that future is going to be. But I don't know
  that I have a lot of faith that the Republican Party which I joined
  --- which, by the way, I didn't join until 1985. I didn't become a
  Republican until I had already joined the Reagan administration. I
  voted twice for Ronald Reagan as a registered Democrat. But I don't
  know that the party of Ronald Reagan and George W. Bush is going to
  return. It's not clear to me given what we've seen that Trump has done
  to the Republican base, and that seems to make all the difference in
  the world, what he's been able to do in his invigorating populism and
  reaching out to some disaffected Democrats who are now Republicans.
  His base in lower-middle-class white America, that base may remain and
  remain within the Republican Party, and it may transform the
  Republican Party in the 21st century.
\item
  chuck rocha\\
  People forget --- and I'll say this before Linda probably says it ---
  is that people forget that George W. Bush got 44 percent of the Latino
  vote, and he did that because he ran commercials. He went to the
  community. And let me stop right there. When I say go to the
  community, I don't mean have a Zoom meeting, fly into a Lulac chapter
  and meet with 15 leaders. I mean go to the community and spend some
  money. What does that mean, Chuck? Oh, let me tell you what that
  means. That means buy TV advertisement, radio advertisement, and
  digital advertisement, that local paper down at the market that your
  grandmother goes and gets in Spanish every Friday because it's got the
  sales coupons in it. Advertise in that paper. I mean you spend money
  talking to brown people like you spend money talking to white people.
  That's point A. Point B is that Joe Biden is not doing a bad job
  reaching out to Latinos. He's doing the same job that every other
  Democratic presidential general election candidate has done to try to
  court our vote. And what I'm getting at there is that that's wrong.
  They all have good intentions. In my mind, Joe Biden cares about the
  Latino vote. He cares about Latino staff. He cares about running a
  good operation, and a lot of his friends, a lot of my friends work on
  this campaign. The problem is and the whole reason I wrote the book
  ``Tío Bernie'' was to say there's a different way to do this. When we
  set up the Bernie Sanders campaign, we didn't have a Latino
  department, and we dominated the vote because we made the Latino
  outreach integrated into the overall campaign. We didn't say, oh,
  Linda's going to be over this part of the cabinet because she's a
  Latina. No, we're putting her there because she's talented and can do
  the job. She just happens to be a Latina. That's the formula to
  success that I want to see Joe Biden and all of them do. I'm not going
  to throw rocks at Joe Biden's campaign. I think that they're doing a
  good job. It's just that it can be done so much better.
\item
  isvett verde\\
  Yeah, I mean, I don't think I've ever seen voters turn out as
  passionately as they did for Bernie, at least Latino voters. I think
  while other presidents have been able to capture a significant
  percentage of that vote, I don't think I saw voters feel as passionate
  for a candidate as they did for Bernie and not even Julián Castro, who
  actually was Latino.
\item
  chuck rocha\\
  It's the old adage if a tree falls in the woods, right? Julián had a
  great message, and he would have been wonderful. But he had, again, no
  money to go tell people who he was and what he stood for.
\item
  linda chavez\\
  Could I could I raise another issue here? And it's an uncomfortable
  one, but I think, as Latinos, we need to talk about it, and that is
  Latinos and their attitudes towards African Americans. I think that
  the Black Lives Matter protests that we're seeing in the streets,
  certainly some of the rioting that we've seen in certain places may
  end up rebounding to the benefit of Donald Trump in certain parts of
  the Latino community because --- and I've written about this for years
  --- this notion that Blacks and Latinos are all on the same
  wavelength, that we are one big, happy community of disadvantaged
  people and our goals are the same, that may be true demographically.
  That may be truer demographically. But I think that we have to
  recognize that there is competition and there is conflict and there is
  prejudice that works against the whole notion of this united front of
  Black and Latino voters.
\item
  isvett verde\\
  So I want to get into this idea of the support that he has which is so
  surprising between Black and Latino men. We're seeing some polls that
  he has enough support that it can also make a difference. Is it
  because of his machista message? Is that appealing? Is it that simple?
  I don't know. I'm eager to hear what you think about that.
\item
  linda chavez\\
  Well, I do think that the kind of macho image is something that
  probably benefits him. Again, he's sort of standing up to things that
  they don't necessarily like. And while there certainly is demographic
  overlap and there should be some commonality between Black and Latino
  voters --- and there is, I mean, in large part. But as Chuck has made
  the point, Joe Biden is going to win a majority of Latino voters. The
  question is how big a majority is he going to win? And the size of
  that majority could make the difference in some important states. It's
  not going to make a difference in California, but it could make a
  difference in Texas, and it definitely will make a difference in
  Arizona. So one of the things that I think Trump is doing right now
  that I think maybe Biden could use to his advantage in appealing to
  Latino men is the way in which Trump talks about the military, the
  whole you know losers and suckers argument that we heard this week
  that supposedly Donald Trump referred to fallen servicemen, people who
  had died in World War I as losers and suckers, and we've seen that
  kind of language. I think there is a latent patriotism within the
  Latino community. A lot of Hispanic men have served in the military.
  And I think when Trump starts badmouthing the US military and bad
  mouthing the courage that it takes to go and defend your country, I
  think that is an opening that Biden should exploit. But as Chuck says,
  he has to recognize that he has to target that message. He has to go
  in there with enough resources and with a message tailored to that
  community that it's going to have some impact. {[}MUSIC PLAYING{]}
\item
  isvett verde\\
  Let's take a quick break, and we'll be right back.

  OK, we're back. So I think we've had too much agreement for a podcast
  called ``The Argument,'' so let's get into where I think you may
  disagree. What policies do you think best speak to Latino concerns,
  Linda?
\item
  linda chavez\\
  Well, I think the economy is important, and I do believe that
  Democrats who are much more enamored of taxes and redistributing
  wealth may actually not be doing a favor to Hispanics who want to
  start their own businesses, who are entrepreneurial. I think education
  is another arena. I think that school choice gives an opportunity for
  people who live in communities where public schools are not doing well
  an opportunity to be able to choose their own schools outside the
  public-school system. There's a growing number, for example, of
  Latinos who are evangelical, and evangelical parents may want to send
  their kids to a religious-based school. And I think that having the
  opportunity to be able to choose the school your children go to and to
  be able to allow your own money to be used through a voucher or some
  other means, a tax credit, is something that can appeal. Again, it's
  not to the entire Hispanic community, but certainly it is to a segment
  of it.
\item
  chuck rocha\\
  Well, I would think about policies a little bit differently,
  obviously, working for Bernie Sanders. People were not shocked to know
  that Medicare for All was the number-one issue with Latinos through
  that entire process before we got to corona. Medicare for All was
  hugely popular with Latinos because they are underinsured, so that is
  just a big part of what their lexicon is. Thinking about small
  businessmen, thinking about people who want to --- and Latinos were
  very aspirational in all of our polling and things I've seen over
  generations, and they do want to create and they do create lots of
  wealth around being entrepreneurs, but they also understood the
  message of the two-tiered system that we live in. So talking about the
  haves and the have-nots, not that you don't want to have the haves.
  I'm a small businessman in Washington, D.C. I pay \$0.46 for every
  dollar that comes into my firm. Now how, Chuck, would you say is that
  possible? `Cause this tax thing is a very big deal for me. I'm at the
  highest tax bracket because I make a couple hundred thousand dollars a
  year, so let's call that 38 percent. And then I live in D.C., so
  there's another \$0.10 popped on that. You go from 38 percent to 48
  percent there pretty quickly. My problem is is that the folks who have
  money and lawyers and income and wealth, a la Republicans, figure out
  a way to get around them having to pay 48 percent. So they're paying
  GE 10 percent or Bezos no tax while my tax dollars are used to prop up
  these fellas. So Latinos get that part of the messaging. And it's just
  like you can be aspirational, but also understand that the system is
  rigged.
\item
  isvett verde\\
  And so what do you think of Chuck's point on taxes, Linda?
\item
  linda chavez\\
  Well, I think he is right that one of the reasons you don't
  necessarily raise in the aggregate more money when you raise taxes is
  that people who are very smart and who don't want to pay taxes have
  other people they hire to figure out how not to pay those taxes. And
  so the kind of loopholes --- I mean, there is no such thing as a flat
  tax. And even when I was in the Reagan administration, I worked on the
  1986 tax-reform bill, and originally there were going to be basically
  three tax brackets, and most deductions were going to be eliminated,
  and it was pretty much going to be if you fall into this bracket,
  you're going to pay this amount in taxes. Then the lobbyists got
  working, and everybody from the Knights of Columbus, who wanted to
  make sure that the deduction for charitable gifts were was not
  eliminated, to the housing industry didn't want to see the housing
  mortgage check go by the wayside. So Chuck's right. Effective tax
  rates are very different than the marginal tax rates.
\item
  chuck rocha\\
  We need to remember that Latinos are younger, that we're just younger
  overall for the listeners out there. Like when the average Latino in
  this country is less than 30 years old --- and that's just the fact
  that we are younger demographically. Now, the Latino voter is a little
  bit older than that because guess what? Only older voters go and vote.
  And what we found during the Bernie Sanders campaign is that these
  younger Latinos --- is this surprising? {[}LAUGHING{]} I'm thinking
  about my 30-year-old son --- is that their view of policy is much
  different than their parents' who have a mortgage, who have a car
  payment, who's figuring out how, if you come from a point of privilege
  like I do and I see my privilege, of trying to take care of my
  grandkids' college education someday hopefully. So guess what? He's a
  little more liberal than even his very liberal father because he views
  the lens of public policy through a different scope. And so I think we
  need to recognize that that's very different in the Latino population
  to a Latino voter who Linda may be talking about who's older Cuban,
  older Mexican, who's just more Catholic, more conservative. So there's
  more inroads for Republicans there. That 44 percent George Bush got
  was a way different demographic than what you're going to see in 2010
  because the population is growing at such a fast rate, so it brings
  that average number of how old it is younger.
\item
  isvett verde\\
  I think that's such an excellent point. You even see that with the
  Cuban community in Miami, right? Younger Cubans have very different
  perspectives from their parents on politics, on policy, on the way the
  direction the country is going in, and they tend to skew more
  progressive. And I think you're seeing that that vote is actually
  shifting because of that. I also want to get back to this Medicare for
  All question. Linda, do you think Medicare for All is something that
  you'd support?
\item
  linda chavez\\
  Well, I have to tell you, this is probably one of the few areas over
  the last 30 years or so where my viewpoint has shifted somewhat. I
  used to be very adamantly opposed to government-run, single-payer-type
  systems. I thought it would, in fact, make care more accessible to
  more people, but I also thought it would lower the quality of care. I
  now think that the system is so broken --- and certainly the pandemic
  has given us insight into what that means --- that we have to do
  something to fix our medical system, and that does mean a much more
  broad-based system of care that's successful to more people. I have
  always believed that what people really worry about is not so much the
  visit to the doctor when you've got the sniffles but the catastrophic
  event. So I was always in favor of treating health insurance much like
  we do home insurance, that if your house burns down, you get
  reimbursed by the insurance company, and you build a new house. But if
  your furnace breaks, that's on you. You have to pay for that. I still
  think that perhaps there is some system we could come up with. It
  would be absolute universal availability for catastrophic insurance
  but then give more choice to the kinds of routine care. And I think if
  there was more competition, you might, in fact, see providers coming
  up with more willingness to lower their fees if those fees were paid
  directly instead of through insurance companies. But, look, it is a
  complicated system. I don't think anybody has the answer. I don't
  think Medicare for All, as somebody--- probably the only person on
  this podcast who's actually on Medicare will tell you Medicare is not
  all that it is cracked up to be and can be not as good as private
  insurance. I've had both, and Medicare is often inefficient. It often
  makes you go through hoops to get testing and other things done that
  you should have the ability to do. So I don't think Medicare for All
  is necessarily the answer, but we do need to tackle this problem.
\item
  isvett verde\\
  I also think that while we think Medicare for all isn't the answer, we
  think of Medicare in its current form, right?
\item
  linda chavez\\
  Correct.
\item
  isvett verde\\
  If we actually funded and were intentional about Medicare, maybe that
  could be a way forward, right? When you think about the pandemic,
  which has turned everything on its head, but it's impacted Black and
  brown people disproportionately. Our communities die in far, far
  higher numbers. A lot of them just didn't have access to health care.
  So you're sick. You just, you don't go to the hospital, and that's a
  problem. And also I think that Trump is playing into this fear that
  people that come from countries like Venezuela and Cuba that
  socialized health care is the worst thing that can happen, but those
  were very dysfunctional governments, and that's not an example of what
  socialized health care can look like. So what are both your
  expectations for this election? You're in very different political
  circles. What are you hearing or seeing? Unfortunately, Walter Mercado
  is no longer with us, so we're on our own. Tell me, what do you see in
  the stars?
\item
  chuck rocha\\
  A, I just finished doing a Walter Mercado piece of mail in North
  Carolina. Just so you know, when I say cultural competency, baby,
  that's what I'm talking about. {[}ISVETT LAUGHS{]} So I made a Walter
  piece of mail and sent it to every Latino in North Carolina, not from
  me but from this great group of community-based Latino organizations
  called N.C. Poder. I think that they give me the most hope is these
  great local groups of Latinos who are reaching out in a nonpartisan
  way, some of them partisan. But like N.C. Poder, LUCHA in Arizona,
  like these groups that are tied to the community who aren't only
  registering Latinos and turning them out--- they're also helping
  provide government service and helping them with citizenship and
  helping them with things. So I do have a great hope. It would be good
  for me if Donald Trump really just barely lost because then I could
  tie Donald Trump around the neck of every commonsense Republican like
  Linda that I know and just say, here, you've got to wear this for at
  least 20 more years {[}LINDA LAUGHS{]} when she don't want nothing to
  do with it just so it helps me win more elections. That's being very
  self-serving. Let me be clear. But if he gets blown out and we win
  back the House and we win a majority of over 50 in the Senate, then
  people will look at me then that these Democrats, who I agree with
  about 75 percent of what they're doing, have to deliver. And then it
  becomes a whole new heap of responsibility for the senior-most Latino
  Democratic operative in D.C. or in the country, who I am, to say,
  look, yeah, I helped elect a lot of these fools, but now they're
  acting like a fool. And now we all got to go hold them accountable and
  remind them every day that it's easier to hold a crazy Democrat
  accountable than it is a crazy Republican because we can go beat a
  crazy Democrat in a primary. A.O.C. proved that. So that's going to be
  kind of my thinking of long-term strategy. I don't want to be the dog
  who caught the car. You know what I mean?
\item
  isvett verde\\
  Do you think someone like A.O.C. is the future of the party?
\item
  chuck rocha\\
  I think she's a future of part of the party. I think that's what makes
  us a unique thing. I used A.O.C. in that Bernie thing. People think,
  well, she's liberal, Chuck, and she don't represent this and that.
  Well, what she is is she is a smart, beautiful, charismatic Latina who
  every grandmother saw their granddaughter being. My granddaughter
  could be a congresswoman. So I didn't talk about her policies as much
  as I talked about her aspiration to be able to overcome the two-party
  system and kind of jump in, and I used that with grandmothers in Iowa,
  Latina grandmothers. So she's part of the future of the party for
  sure.
\item
  isvette verde\\
  Linda, you're leaning in. Tell me what you think.
\item
  linda chavez\\
  Well, as you might imagine, I have very mixed emotions. I do care a
  great deal about immigration. I've spent the last 10 years of my life
  almost exclusively focused on the immigration issue. And if Donald
  Trump is reelected, I see a very dismal future in terms of that issue.
  I don't think we will see immigration reform. I have said for years
  that the way to stop illegal immigration is to give people who want to
  come here to work and who are going to contribute to our society a way
  to come here legally, but that means expanding legal immigration. And,
  of course, Trump is all about reducing it. So I don't want to see a
  narrow victory for the Democrats. And I know this is going to sound
  bizarre since I am still a registered Republican, but I think because
  I am a conservative, because I do hold many things about the
  Republican Party dear, I want to see a transformed Republican Party. I
  want to see the Republican Party revert to the free-market,
  free-trade, entrepreneurial Republican Party that welcomed with open
  arms people as long as they came to the United States, letting people
  come here in greater numbers legally. So I'd like to see not the kind
  of narrow victory that Chuck suggests. But I hope that A.O.C. is not
  the future of Latinos in politics, Latinas in politics in America. I'd
  much rather see the kind of more moderate folks like Susana Martinez,
  who was governor of New Mexico. I'd like to see those kinds of people
  coming to the fore. It would be great to me if the Republicans could
  get back to getting 44 percent, 45 percent, maybe 50 percent,
  51percent of Latino votes but only with a candidate who speaks to the
  kinds of principles that I think the Republican Party once stood for
  and, unfortunately, in the era of Trump has not stood for.
\item
  chuck rocha\\
  If you set her --- if you set A.O.C.`s policies to the side, whether
  we agree or we disagree --- and I agree with all of her policies, for
  the most part. If we think about her story, though, of how this young
  woman was a waitress, got involved with a community-based
  organization, and took on the second- or the third-most-powerful
  Democrat in the U.S. Congress and out organized him in a Democratic
  primary, that's what I'm about.
\item
  linda chavez\\
  She certainly has charisma, and I have to tell you when she was first
  elected and many of my Republican Facebook friends were all posting
  the video of her dancing on the rooftop and they were doing it in a
  way to demean her and to act as if she wasn't a serious person, I
  watched it and I thought this is absolutely delightful. And she is, as
  you say, Chuck, as a person in terms of the kind of charisma she
  brings, she is appealing. I just happen to disagree with virtually
  everything she stands for, not just in domestic policy but probably
  more importantly on foreign policy.
\item
  chuck rocha\\
  There you go. Y'all got your disagreement right there. That's what we
  came here for, ``The Argument.'' There we go.
\item
  linda chavez\\
  I'm still a right winger when it comes to foreign policy.
\item
  chuck rocha\\
  There we go.
\item
  isvett verde\\
  {[}LAUGHS{]} Linda, do you think --- looking forward, do you think the
  Republican Party will be able to separate itself from Trump's virulent
  xenophobia against Latinos? He launched a campaign referring to
  Mexicans as rapists. He characterized caravans as invasions, and his
  party has stood by him. Can they come back from that, or is that
  really what the party thinks of Latinos?
\item
  linda chavez\\
  Well, it's interesting. Prior to 2015 when Trump launched his
  campaign, I was living in Colorado. I was out speaking to Republican
  groups every week, and I was bringing a proimmigration message to
  them. I was out talking to them about the facts, the fact that illegal
  immigration, even at that time, was down. There had been many more
  people coming 15 years earlier than there were in 2015. I talked to
  them about the assimilation of Hispanics in the United States. I wrote
  frequently about the way in which Hispanics have fallen into the
  pattern that virtually all immigrants have, and that is to start on
  the bottom rungs but to rapidly move up. Second-generation Hispanics
  were English speakers. By third generation, some --- in fact, many
  were English monolingual speakers. They had lost the ability to speak
  Spanish. And I was getting a very receptive audience in Republican
  circles, in conservative circles. That changed on a dime with Donald
  Trump. And one of the things that I think happened is that was for the
  last 30, 40 years, maybe longer in the United States, it was impolite
  to say prejudiced things. If you had prejudiced feelings, certainly in
  mixed company you kept them to yourself. You didn't speak them out
  loud. Well, suddenly Donald Trump gave permission for people to take
  their prejudices and pushed them to the front and engage in
  discussions and talk that was mean, that was racist, and he gave them
  permission to do that. I've always know you may have pulled back the
  rock and revealed the squirmy things underneath it. I'd like to put
  the rock back because I do think there is some value in simply making
  it not permissible in polite company to express views that are
  hateful. And with Trump gone, maybe some of that will come back.
\item
  isvett verde\\
  I'm curious because you mentioned that you did change parties, and so
  why did you ---
\item
  linda chavez\\
  Become a Republican?
\item
  isvette verde\\
  --- change --- why did you--- yes, exactly. {[}LAUGHS{]}
\item
  linda chavez\\
  Well, I mean, it's funny. My second book was called an unlikely
  conservative. And one of the reviewers of my book noted the fact that
  I talked about this transformation that had taken place. And I think
  it was a woman, and she said there was no transformation. She was
  always a conservative, even though she may have been a Democrat. And I
  think, for the most part, that's true. I think the Democratic Party of
  my youth, my father's Democratic Party was a very different party than
  the Democratic Party of the early 1970s. It was really George McGovern
  who changed the party in ways that I found unacceptable. Foreign
  policy is a very big deal to me. I am a Cold War anticommunist, so I
  saw Jimmy Carter as basically ceding much of the world to the Soviet
  sphere during his tenure. I saw the map of the free world began to
  change and Latin America, Africa, other places moving into the more
  autocratic and totalitarian left, and that was the reason that I first
  voted for Ronald Reagan. It was solely on foreign-policy issues. I
  only became more conservative on economic issues, as you might expect,
  as I started earning more money and started paying more taxes. But
  foreign policy is still a very big factor in my vote, and it's one of
  the reasons I couldn't support Bernie Sanders. And if Bernie Sanders
  had won the nomination, I would not vote for Trump, but I certainly
  would not have voted for Bernie Sanders.
\item
  isvette verde\\
  And, Chuck, what makes you a progressive rather than a conservative?
\item
  chuck rocha\\
  I think growing up and not having much. I think understanding the
  concept of my vision of what the government should do to stand up for
  its citizens, that the power not be concentrated with just the wealthy
  of the country who gets to make the law and hire the lobbyists and
  have the tax attorneys. It's just these basic things to where I just
  know that I'm not going to align with a party that doesn't speak to my
  personal values, and that's just where I've always stood. I wrote in
  my book --- it's funny--- of how did you get involved in politics? And
  I write about the NAFTA proposition --- and people like to throw that
  in my face that it was Bill Clinton, a Democratic president, and it
  was. And that factory I worked in in East Texas and watching how it
  devastated that whole little community in East Texas when that tire
  factory went away. I think that you know this, Isvett, that I'm one of
  the few professionals in this city who never went to college. I've
  lived a real life, and I've just been so blessed and so lucky to be
  able to do the things that I've done and be in the Oval Office and get
  to run presidential campaigns.
\item
  linda chavez\\
  Let me just sort of get in here because people don't necessarily know
  my story, and Chuck and I probably come from backgrounds that are more
  similar than you might expect. My father was a house painter with a
  ninth-grade education. My mother worked primarily in restaurants until
  I got to be a teenager, and then she worked in department stores. I
  ended up going to college quite by accident. I was working in a beauty
  salon. I was a receptionist. And a girlfriend came by to go to lunch
  and walked over to the University of Colorado Denver center and
  decided she wanted to enroll in some classes. And I thought, well,
  maybe I'll do that too. And I met my husband the first semester, and
  my husband comes from an upper-middle-class Jewish family, son of a
  doctor and a college dance professor, and the rest is history. We've
  been married for 53 years. But my background is also working class,
  and I never forget that. And I know that much of what I've been able
  to accomplish has been by virtue of luck and being in the right place
  at the right time. A lot of it is hard work, and I certainly have
  worked hard, but a lot of it is really luck.
\item
  isvett verde\\
  I actually may cry because this is all just so lovely to be on here
  today and talking to you all and hearing about our stories and our
  perspectives and our community. I mean, I started kindergarten not
  even speaking English. I went to community college, and now I'm here
  talking to you all. And it's a testament to the promise of this
  country, and I hope that that promise isn't lost and that the future
  will look different, that our country can come back after these past
  few years and these horrid past few months.
\item
  linda chavez\\
  Well, I will just say that I think you're absolutely right, and the
  future of the country is at stake in this election. The direction that
  we're going to go for the future is very much in the hands of the
  American voters. And I hope that Latinos who have, over the history of
  our community, have not voted as much as many other groups, I hope
  that they get out, I hope they register, and I hope they vote.
\item
  isvette verde\\
  And with that, Linda Chavez, Chuck Rocha, thank you both so much for
  helping me add a little sázon to ``The Argument'' today. This was so
  much fun.
\item
  linda chavez\\
  Thank you, Isvett. Great to be with you.
\item
  chuck rocha\\
  Thank you, Isvett. Appreciate it.
\item
  isvett verde\\
  And that's our show this week. Thank you for listening. If you have an
  election question that you want to hear debated on ``The Argument,''
  leave us a voicemail at 347-915-4324. You can also email it to us at
  \href{mailto:argument@NYTimes.com}{\nolinkurl{argument@NYTimes.com}}.
  ``The Argument'' is a production of The New York Times Opinion
  section. The team includes Phoebe Lett, Kristin Lin, Paula Szuchman,
  Isaac Jones, Kathy Tu, Alison Bruzek, and Vishakha Darbha. Don't
  worry, dear ``Argument'' listeners, Ross and Michelle will be back in
  your ears next Friday.

  {[}LAUGHING{]} Bienvenidos al ``Argumento.'' {[}LAUGHS{]}
\end{itemize}

Previous

More episodes ofThe Argument

\href{https://www.nytimes3xbfgragh.onion/2020/09/11/opinion/the-argument-latino-2020-vote.html?action=click\&module=audio-series-bar\&region=header\&pgtype=Article}{\includegraphics{https://static01.graylady3jvrrxbe.onion/images/2020/09/12/opinion/10argumentWeb/10argumentWeb-thumbLarge-v2.jpg}}

September 11, 2020How to Win the Latino Vote

\href{https://www.nytimes3xbfgragh.onion/2020/09/03/opinion/the-argument-trump-biden-kenosha-portland.html?action=click\&module=audio-series-bar\&region=header\&pgtype=Article}{\includegraphics{https://static01.graylady3jvrrxbe.onion/images/2020/09/05/opinion/03argumentWeb/03argumentWeb-thumbLarge.jpg}}

September 3, 2020Is `American Carnage' Campaign Gold?

\href{https://www.nytimes3xbfgragh.onion/2020/08/27/opinion/the-argument-republican-convention-trump.html?action=click\&module=audio-series-bar\&region=header\&pgtype=Article}{\includegraphics{https://static01.graylady3jvrrxbe.onion/images/2020/08/28/opinion/27argument-ninetytwo1-print/27argument-ninetytwo1-thumbLarge.jpg}}

August 27, 2020Can the Republicans Sell a Whole New Trump?

\href{https://www.nytimes3xbfgragh.onion/2020/08/20/opinion/the-argument-democratic-convention-biden.html?action=click\&module=audio-series-bar\&region=header\&pgtype=Article}{\includegraphics{https://static01.graylady3jvrrxbe.onion/images/2020/08/20/opinion/20argument-ninetyone1/20argument-ninetyone1-thumbLarge.jpg}}

August 20, 2020What Biden Must Do

\href{https://www.nytimes3xbfgragh.onion/2020/08/13/opinion/the-argument-coronavirus-catholic-covid.html?action=click\&module=audio-series-bar\&region=header\&pgtype=Article}{\includegraphics{https://static01.graylady3jvrrxbe.onion/images/2020/08/13/opinion/13argument1/merlin_173532477_02e02102-92e6-4f5a-82bf-5394265f898b-thumbLarge.jpg}}

August 13, 2020Is Individualism America's Religion?

\href{https://www.nytimes3xbfgragh.onion/2020/08/06/opinion/the-argument-trump-coronavirus-election.html?action=click\&module=audio-series-bar\&region=header\&pgtype=Article}{\includegraphics{https://static01.graylady3jvrrxbe.onion/images/2020/08/06/opinion/06argSub/06argSub-thumbLarge.jpg}}

August 6, 2020Trump Supporters Make Their Case for 2020

\href{https://www.nytimes3xbfgragh.onion/2020/07/30/opinion/the-argument-authoritarianism-anne-applebaum.html?action=click\&module=audio-series-bar\&region=header\&pgtype=Article}{\includegraphics{https://static01.graylady3jvrrxbe.onion/images/2020/07/31/opinion/30argumentWeb-print/30argumentWeb-thumbLarge.jpg}}

July 30, 2020When Conservatives Fall for Demagogues

\href{https://www.nytimes3xbfgragh.onion/2020/07/23/opinion/the-argument-israel-palestinian.html?action=click\&module=audio-series-bar\&region=header\&pgtype=Article}{\includegraphics{https://static01.graylady3jvrrxbe.onion/images/2020/07/25/opinion/25audio/21argumentWeb-thumbLarge.jpg}}

July 23, 2020The Case for a One-State Solution

\href{https://www.nytimes3xbfgragh.onion/2020/07/16/opinion/the-argument-tammy-duckworth.html?action=click\&module=audio-series-bar\&region=header\&pgtype=Article}{\includegraphics{https://static01.graylady3jvrrxbe.onion/images/2020/07/17/opinion/16argumentWeb-print/16argumentWeb-thumbLarge.jpg}}

July 16, 2020A Conversation With Tammy Duckworth

\href{https://www.nytimes3xbfgragh.onion/2020/07/09/opinion/is-trumps-fate-sealed.html?action=click\&module=audio-series-bar\&region=header\&pgtype=Article}{\includegraphics{https://static01.graylady3jvrrxbe.onion/images/2020/07/10/opinion/10a2_audio/09argument1-thumbLarge.jpg}}

July 9, 2020Is Trump's Fate Sealed?

\href{https://www.nytimes3xbfgragh.onion/2020/07/02/opinion/the-argument-protest-statue-revolution.html?action=click\&module=audio-series-bar\&region=header\&pgtype=Article}{\includegraphics{https://static01.graylady3jvrrxbe.onion/images/2020/07/05/opinion/02argument-eightyfive1/02argument-eightyfive1-thumbLarge.jpg}}

July 2, 2020Whose Statue Must Fall?

\href{https://www.nytimes3xbfgragh.onion/2020/06/25/opinion/the-argument-biden-vice-president-supreme-court.html?action=click\&module=audio-series-bar\&region=header\&pgtype=Article}{\includegraphics{https://static01.graylady3jvrrxbe.onion/images/2020/06/28/opinion/25argument-eightyfour1/25argument-eightyfour1-thumbLarge.jpg}}

June 25, 2020Place Your Bets on Biden's V.P.

\href{https://www.nytimes3xbfgragh.onion/column/the-argument}{See All
Episodes ofThe Argument}

Next

Sept. 11, 2020

\begin{itemize}
\item
\item
\item
\item
\item
\end{itemize}

\emph{\textbf{Listen and subscribe to ``The Argument'' from your mobile
device:}}

\textbf{\href{https://itunes.apple.com/us/podcast/the-argument/id1438024613?mt=2}{\emph{Apple
Podcasts}}} \emph{\textbf{\textbar{}}}
\textbf{\href{https://open.spotify.com/show/6bmhSFLKtApYClEuSH8q42}{\emph{Spotify}}}
\emph{\textbf{\textbar{}}}
\textbf{\href{https://play.google.com/music/m/Idxib4hsg3yviao4gtym76knjjy?t=The_Argument}{\emph{Google
Play}}} \emph{\textbf{\textbar{}}}
\textbf{\href{https://radiopublic.com/the-argument-Wdbepr}{\emph{RadioPublic}}}
\emph{\textbf{\textbar{}}}
\textbf{\href{https://www.stitcher.com/podcast/the-new-york-times/the-argument}{\emph{Stitcher}}}
\emph{\textbf{\textbar{}}}
\textbf{\href{https://rss.art19.com/the-argument}{\emph{RSS Feed}}}

In a special episode of ``The Argument,'' Opinion editor and writer
Isvett Verde hosts a round table on the Latino vote in the 2020
election. Isvett welcomes Chuck Rocha, a senior campaign adviser to
Bernie Sanders, and Linda Chavez, director of the Becoming American
Institute and a senior fellow at the Niskanen Center. Together, they
debunk the myth of a monolithic ``Latino voting bloc,'' explain Latino
support for President Trump and discuss the role of Latinos in the
future of both parties. Linda describes going from being one of the
highest-ranking women in the Reagan White House to not recognizing her
party in the Trump era. And Chuck explains how Sanders was able to
excite Latino voters like no other candidate.

\includegraphics{https://static01.graylady3jvrrxbe.onion/images/2020/09/12/opinion/10argumentWeb/10argumentWeb-articleLarge-v2.jpg?quality=75\&auto=webp\&disable=upscale}

\begin{center}\rule{0.5\linewidth}{\linethickness}\end{center}

\textbf{Background Reading:}

\begin{itemize}
\item
  Isvett on
  \href{https://www.nytimes3xbfgragh.onion/2019/11/05/opinion/walter-mercado.html}{Walter
  Mercado's optimism}, and a
  \href{https://www.nytimes3xbfgragh.onion/2016/11/28/opinion/what-castros-death-means-for-a-child-of-mariel.html}{Cuban
  expat's reaction to the death of Fidel Castro}
\item
  Chuck on the
  \href{https://www.nytimes3xbfgragh.onion/2020/06/24/opinion/2020-campaigns-diversity.html}{work
  of people of color on political campaigns} and
  \href{https://www.nytimes3xbfgragh.onion/2020/08/18/opinion/joe-biden-kamala-harris-latino-vote.html?searchResultPosition=1}{why
  Latinos are Joe Biden's secret untapped weapon}
\item
  Linda's
  \href{https://www.nytimes3xbfgragh.onion/2020/08/25/opinion/rnc-best-worst-night-1.html?searchResultPosition=4}{analysis}
  \href{https://www.nytimes3xbfgragh.onion/2020/08/26/opinion/rnc-best-worst-night-2.html?searchResultPosition=3}{of
  the}
  \href{https://www.nytimes3xbfgragh.onion/2020/08/27/opinion/rnc-best-worst-night-3.html?searchResultPosition=2}{Republican}
  \href{https://www.nytimes3xbfgragh.onion/2020/08/28/opinion/rnc-best-worst-trump-night-4.html?searchResultPosition=1}{convention},
  on
  \href{https://www.nytimes3xbfgragh.onion/2005/11/17/opinion/fellow-republicans-open-your-doors.html}{immigration
  reform for the Republican Party} and on the
  \href{https://www.nytimes3xbfgragh.onion/2006/03/30/opinion/30iht-edchavez.html?searchResultPosition=260}{American
  dreams of Latino Americans}
\end{itemize}

\begin{center}\rule{0.5\linewidth}{\linethickness}\end{center}

\textbf{How to listen to ``The Argument'':}

\emph{Press play or read the transcript (found by midday Thursday above
the center teal eye) at the top of this page, or tune in on}
\href{https://itunes.apple.com/us/podcast/the-argument/id1438024613?mt=2}{\emph{iTunes}}\emph{,}
\href{https://play.google.com/music/listen?u=0\#/ps/Idxib4hsg3yviao4gtym76knjjy}{\emph{Google
Play}}\emph{,}
\href{https://open.spotify.com/episode/5fIsHqqunLBwoxPSUUSGre?si=Rz5D9VnlRFKdGMu8ixzBOw}{\emph{Spotify}}\emph{,}
\href{https://www.stitcher.com/podcast/the-new-york-times/the-argument}{\emph{Stitcher}}
\emph{or your preferred podcast listening app. Tell us what you think
at} \href{mailto:argument@NYTimes.com}{\emph{argument@NYTimes.com.}}

\begin{center}\rule{0.5\linewidth}{\linethickness}\end{center}

``The Argument'' is a production of The New York Times Opinion section.
The team includes Alison Bruzek, Phoebe Lett, Vishakha Darbha, Kathy Tu,
Kristin Lin, Paula Szuchman and Isaac Jones. Theme by Allison
Leyton-Brown.

Advertisement

\protect\hyperlink{after-bottom}{Continue reading the main story}

\hypertarget{site-index}{%
\subsection{Site Index}\label{site-index}}

\hypertarget{site-information-navigation}{%
\subsection{Site Information
Navigation}\label{site-information-navigation}}

\begin{itemize}
\tightlist
\item
  \href{https://help.nytimes3xbfgragh.onion/hc/en-us/articles/115014792127-Copyright-notice}{©~2020~The
  New York Times Company}
\end{itemize}

\begin{itemize}
\tightlist
\item
  \href{https://www.nytco.com/}{NYTCo}
\item
  \href{https://help.nytimes3xbfgragh.onion/hc/en-us/articles/115015385887-Contact-Us}{Contact
  Us}
\item
  \href{https://www.nytco.com/careers/}{Work with us}
\item
  \href{https://nytmediakit.com/}{Advertise}
\item
  \href{http://www.tbrandstudio.com/}{T Brand Studio}
\item
  \href{https://www.nytimes3xbfgragh.onion/privacy/cookie-policy\#how-do-i-manage-trackers}{Your
  Ad Choices}
\item
  \href{https://www.nytimes3xbfgragh.onion/privacy}{Privacy}
\item
  \href{https://help.nytimes3xbfgragh.onion/hc/en-us/articles/115014893428-Terms-of-service}{Terms
  of Service}
\item
  \href{https://help.nytimes3xbfgragh.onion/hc/en-us/articles/115014893968-Terms-of-sale}{Terms
  of Sale}
\item
  \href{https://spiderbites.nytimes3xbfgragh.onion}{Site Map}
\item
  \href{https://help.nytimes3xbfgragh.onion/hc/en-us}{Help}
\item
  \href{https://www.nytimes3xbfgragh.onion/subscription?campaignId=37WXW}{Subscriptions}
\end{itemize}
