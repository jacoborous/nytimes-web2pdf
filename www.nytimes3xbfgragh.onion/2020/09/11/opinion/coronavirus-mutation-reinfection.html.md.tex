Sections

SEARCH

\protect\hyperlink{site-content}{Skip to
content}\protect\hyperlink{site-index}{Skip to site index}

\href{https://myaccount.nytimes3xbfgragh.onion/auth/login?response_type=cookie\&client_id=vi}{}

\href{https://www.nytimes3xbfgragh.onion/section/todayspaper}{Today's
Paper}

\href{/section/opinion}{Opinion}\textbar{}The Coronavirus Is Mutating,
and That's Fine (So Far)

\url{https://nyti.ms/3m98rIB}

\begin{itemize}
\item
\item
\item
\item
\item
\end{itemize}

Advertisement

\protect\hyperlink{after-top}{Continue reading the main story}

\href{/section/opinion}{Opinion}

Supported by

\protect\hyperlink{after-sponsor}{Continue reading the main story}

\hypertarget{the-coronavirus-is-mutating-and-thats-fine-so-far}{%
\section{The Coronavirus Is Mutating, and That's Fine (So
Far)}\label{the-coronavirus-is-mutating-and-thats-fine-so-far}}

SARS-CoV-2 has been slowly changing in small ways, without getting more
dangerous.

By Edward Holmes

Dr. Holmes is an evolutionary virologist.

\begin{itemize}
\item
  Sept. 11, 2020
\item
  \begin{itemize}
  \item
  \item
  \item
  \item
  \item
  \end{itemize}
\end{itemize}

\includegraphics{https://static01.graylady3jvrrxbe.onion/images/2020/09/10/opinion/10Holmes/10Holmes-articleLarge.jpg?quality=75\&auto=webp\&disable=upscale}

SYDNEY, Australia --- No doubt you have read the recent headlines about
patients who recovered from Covid-19 only to be infected with SARS-CoV-2
again later --- purportedly by a different
``\href{https://www.sciencedaily.com/releases/2020/08/200803105246.htm}{strain}''
of the virus.

In late August came news alerts about the world's first
``\href{https://www.nytimes3xbfgragh.onion/2020/08/24/health/coronavirus-reinfection.html}{documented}''
or
``\href{https://www.physiciansweekly.com/worlds-first-confirmed-case-of-covid-19-reinfection-reported-in-hong-kong/}{confirmed}''
case of reinfection with SARS-CoV-2: a man from Hong Kong, diagnosed in
March, had contracted
``\href{https://www.infectioncontroltoday.com/view/covid-19-reinfection-first-instance-recorded-in-hong-kong}{a
new virus}'' circulating in Western Europe this summer. The very next
day, news broke that
\href{https://www.nytimes3xbfgragh.onion/2020/08/25/world/covid-19-coronavirus.html\#link-f61c099}{two
people in Europe} also appeared to have been reinfected.

After that, it was
\href{https://www.nbcnews.com/health/health-news/covid-19-reinfection-reported-nevada-patient-researchers-say-n1238679}{stories}
about the
\href{https://www.foxnews.com/health/first-coronavirus-reinfection-case-us-nevada}{first
American case of the kind} --- involving a patient in Nevada said to
have suffered worse symptoms the second time around. The preprint study
(not peer-reviewed) on which those reports were based
\href{https://papers.ssrn.com/sol3/papers.cfm?abstract_id=3681489}{seems
to no longer be available.}

All this talk about new, perhaps more virulent, forms of SARS-CoV-2 is
unnecessarily sparking fear and sowing confusion.

Let's consider the evidence and the science.

For one thing, isolated cases of reinfection also happen with other
viruses. That fact is not
\href{https://www.nytimes3xbfgragh.onion/2020/07/31/opinion/coronavirus-antibodies-immunity.html?searchResultPosition=3}{necessarily
alarming}. Reinfection usually tells us something only about how the
human immune system works. It is not, on the face of it, evidence that a
virus has mutated in ways that make it more dangerous.

For another thing, viruses routinely mutate --- and most of these
changes are bad for the virus or even fatal,
\href{https://www.ncbi.nlm.nih.gov/pmc/articles/PMC420405/}{according to
some studies}. (A minority of mutations are neutral, and only a tiny
minority beneficial.) The word ``mutation'' may sound ominous, but it is
a humdrum fact of viral life and its implications most often aren't
nefarious for humans.

And yes, SARS-CoV-2 is mutating, too. So what?

The real question is this: Has it become more virulent or more
infectious than it was when it was first detected in Wuhan, central
China, in December? The evidence suggests that it has not.

Like the viruses that give us influenza or measles, SARS-CoV-2 has a
genetic code made up of RNA, or ribonucleic acid. But RNA is highly
mutable, and since SARS-CoV-2 infects us by using our body's cells to
replicate itself again and again, every time its genome is copied, an
error might creep in.

Most mutations are quickly lost, either by chance or because they damage
some part of the virus's main functions. Only a small proportion end up
spreading widely or lasting. Mutation may be the fuel of evolution but,
especially for an RNA virus, it also is just business as usual.

RNA viruses tend to evolve rapidly --- about
\href{https://journals.plos.org/plosbiology/article?id=10.1371/journal.pbio.3000003}{a
million times faster than human genes}. Yet if SARS-CoV-2 stands out
among them, it is for evolving more slowly than many: about
\href{https://www.biorxiv.org/content/10.1101/2020.05.04.077735v1.abstract}{five
times less rapidly than the influenza viruses}, for example.

According to \href{https://nextstrain.org/}{Nextstrain}, an open-source
project that tracks the evolution of pathogens in real time, and
\href{https://www.sciencemag.org/news/2020/07/pandemic-virus-slowly-mutating-it-getting-more-dangerous}{other
sources}, SARS-CoV-2 is accumulating an average of about two mutations
per month --- which means that the forms of the virus circulating today
are only about 15 mutations or so different from the first version
traced to the outbreak in Wuhan.

This is a tiny number considering that the SARS-CoV-2's genome is about
30,000 nucleotides long. And it means, too, that the versions of the
virus today are roughly 99.95 percent the same as the Wuhan original.
For an RNA virus, SARS-CoV-2 is in the slow lane of evolution.

(So talk about SARS-CoV-2 having developed into however many different
``strains'' is
\href{https://www.virology.ws/2020/05/07/there-is-one-and-only-one-strain-of-sars-cov-2/}{misleading}.
Scientists tend to reserve the word for versions of a virus that differ
in major biological ways. SARS-CoV-2's different forms are very similar;
better to call them ``variants.'')

The coronavirus's sluggish pace of mutation is good news for us: A virus
that evolved more rapidly would have a greater chance of outrunning any
vaccines or drugs developed to counter it.

That said, have even the small mutations so far changed SARS-CoV-2 in
any important ways?

For example, has it become more deadly?

To my knowledge, there is to date no evidence that SARS-CoV-2 has become
more virulent or more lethal --- nor, for that matter, that it has
become less so.

For example, a recent
\href{https://www.medrxiv.org/content/10.1101/2020.07.31.20166082v2}{preprint
paper} (not yet peer-reviewed) by Erik Volz, of the faculty of medicine
at Imperial College in London, and numerous colleagues at other
institutions --- including members of the Covid-19 Genomics UK
Consortium --- which analyzed 25,000 whole genome SARS-CoV-2 sequences
collected in the United Kingdom, found that one particular mutation in
the virus, known as D614G, had not increased mortality in patients.

What about infectivity?

There has
\href{https://www.nytimes3xbfgragh.onion/2020/06/12/science/coronavirus-mutation-genetics-spike.html}{been}
\href{https://medium.com/microbial-instincts/what-the-d614g-mutation-means-for-covid-19-spread-fatality-treatment-and-vaccine-7dda1c066f0d}{much}
\href{https://www.businessinsider.com.au/new-coronavirus-strain-infectious-does-not-make-people-sicker-study-2020-7?r=US\&IR=T}{discussion}
over whether the D614G mutation --- which affects the so-called spike
protein of the virus --- has made SARS-CoV-2 more infectious.

The spike protein sits on the surface of the coronavirus, and it matters
because it's the part of the virus that attaches to the host's cells.
``D614G'' is shorthand for a change at position 614 of the spike
protein, from an aspartic acid (D) to a glycine amino acid (G). (The
technical literature refers to ``D614'' as the earlier configuration and
``G614'' as the later one.)

The D614G mutation, which probably initially arose in China, first
appeared to become more and more frequent in the outbreak in northern
Italy in February. The G614 form of the virus has since spread all over
the world and has become the dominant variant.

The D614G mutation does seem to have increased the infectivity of the
coronavirus ---
\href{https://www.scripps.edu/news-and-events/press-room/2020/20200611-choe-farzan-sars-cov-2-spike-protein.html}{at
least in cells grown in laboratories}, according to
\href{https://www.cell.com/cell/pdf/S0092-8674(20)30820-5.pdf}{a recent
paper by the computational biologist Bette Korber and others} published
in the journal Cell.

Apparently based partly on this and
\href{https://www.scripps.edu/news-and-events/press-room/2020/20200611-choe-farzan-sars-cov-2-spike-protein.html}{other
studies},
\href{https://www.todayonline.com/world/10-times-more-infectious-d614g-coronavirus-strain-detected-malaysia}{health
authorities} in
\href{https://timesofindia.indiatimes.com/india/ten-times-more-infectious-than-coronavirus-all-you-need-to-know-about-d614g/articleshow/77586073.cms}{various
countries} have claimed that the G614 form of the coronavirus may be 10
times more infectious than the version first detected in Wuhan.

But as
\href{https://www.cell.com/cell/fulltext/S0092-8674(20)30817-5?_returnURL=https\%3A\%2F\%2Flinkinghub.elsevier.com\%2Fretrieve\%2Fpii\%2FS0092867420308175\%3Fshowall\%3Dtrue}{some
epidemiologists have warned}, it is difficult, not to mention unwise, to
extrapolate from lab results to explain how the virus actually spreads
in a real population.

I do not believe that the evolution of SARS-CoV-2 is what's driving the
virus's continued spread. The coronavirus remains good at propagating
itself because most of us still are susceptible to it; we are not
immune, and it can still find new hosts to infect relatively easily.

In the same issue of Cell that published the Korber paper, the viral
epidemiologist Nathan Grubaugh and colleagues argued that the ``increase
in the frequency of G614 could be explained by chance and the
epidemiology of the pandemic.''

I agree.

In other words: The next time you compare different outbreaks and start
wondering or worrying about the variations, assume first that those
variations have to do with conditions on the ground, rather than
anything about the virus itself, like a new mutation.

Consider, for example, the wave of SARS-CoV-2 infections that has hit
Australia since June. While there has been a major outbreak in the state
of Victoria (peaking \href{https://www.covid19data.com.au/victoria}{at
around 700 cases per day}), the one in the state of New South Wales has
been minor so far (with
\href{https://www.covid19data.com.au/transmission-sources-states}{a
daily case count usually around 10}) --- yet both have been caused by
the same variant of the coronavirus, the one with the D614G mutation.

The precise reasons for these differences are still being investigated,
but one may be, simply, that because
\href{https://www.theguardian.com/australia-news/2020/aug/28/victoria-covid-hotel-quarantine-inquiry-hears-guest-escaped-to-lobby-as-security-guard-was-on-his-phone}{the
outbreak hit Victoria first}, the health authorities of New South Wales
had more time to prepare.

Mortality rates, too,
\href{https://www.worldometers.info/coronavirus/}{differ between
locations}, and in some places the virus may appear to kill more people.
But again, these variations probably say less about the virus than about
differences in how the disease is being treated, or where the virus has
spread mostly among vulnerable populations, like
\href{https://www.theguardian.com/australia-news/2020/sep/07/revealed-more-than-40-of-victorian-coronavirus-aged-care-deaths-were-residents-in-just-10-homes}{people
in nursing homes}.

What's more, even if the D614G mutation does increase the virus's
infectivity in humans, that fact probably doesn't have any major
implications for our prospects of developing an effective vaccine. The
mutation does affect the spike protein, but
\href{https://www.pnas.org/content/early/2020/08/28/2008281117}{not the
part of it that the neutralizing antibodies of the human immune system
target} when the body defends itself against infection.

Viruses mutate constantly; SARS-CoV-2 is no different. And it's
essential that we continue to monitor when and how, and with what
effects, it is evolving.

Whether SARS-CoV-2 is becoming more infectious or more deadly are
important questions, all the more so because it doesn't look like this
virus will be eradicated any time soon. More likely, it will become a
pathogen endemic in humans, as everyday as influenza.

For now, though, SARS-CoV-2 essentially is the same virus that emerged
in December. Sure, it has mutated, but not, so far, in ways that **
should change how scientists think about how to tackle it --- and not in
ways that should worry you.

Edward Holmes is an evolutionary virologist at the University of Sydney.

\emph{The Times is committed to publishing}
\href{https://www.nytimes3xbfgragh.onion/2019/01/31/opinion/letters/letters-to-editor-new-york-times-women.html}{\emph{a
diversity of letters}} \emph{to the editor. We'd like to hear what you
think about this or any of our articles. Here are some}
\href{https://help.nytimes3xbfgragh.onion/hc/en-us/articles/115014925288-How-to-submit-a-letter-to-the-editor}{\emph{tips}}\emph{.
And here's our email:}
\href{mailto:letters@NYTimes.com}{\emph{letters@NYTimes.com}}\emph{.}

\emph{Follow The New York Times Opinion section on}
\href{https://www.facebookcorewwwi.onion/nytopinion}{\emph{Facebook}}\emph{,}
\href{http://twitter.com/NYTOpinion}{\emph{Twitter (@NYTopinion)}}
\emph{and}
\href{https://www.instagram.com/nytopinion/}{\emph{Instagram}}\emph{.}

Advertisement

\protect\hyperlink{after-bottom}{Continue reading the main story}

\hypertarget{site-index}{%
\subsection{Site Index}\label{site-index}}

\hypertarget{site-information-navigation}{%
\subsection{Site Information
Navigation}\label{site-information-navigation}}

\begin{itemize}
\tightlist
\item
  \href{https://help.nytimes3xbfgragh.onion/hc/en-us/articles/115014792127-Copyright-notice}{©~2020~The
  New York Times Company}
\end{itemize}

\begin{itemize}
\tightlist
\item
  \href{https://www.nytco.com/}{NYTCo}
\item
  \href{https://help.nytimes3xbfgragh.onion/hc/en-us/articles/115015385887-Contact-Us}{Contact
  Us}
\item
  \href{https://www.nytco.com/careers/}{Work with us}
\item
  \href{https://nytmediakit.com/}{Advertise}
\item
  \href{http://www.tbrandstudio.com/}{T Brand Studio}
\item
  \href{https://www.nytimes3xbfgragh.onion/privacy/cookie-policy\#how-do-i-manage-trackers}{Your
  Ad Choices}
\item
  \href{https://www.nytimes3xbfgragh.onion/privacy}{Privacy}
\item
  \href{https://help.nytimes3xbfgragh.onion/hc/en-us/articles/115014893428-Terms-of-service}{Terms
  of Service}
\item
  \href{https://help.nytimes3xbfgragh.onion/hc/en-us/articles/115014893968-Terms-of-sale}{Terms
  of Sale}
\item
  \href{https://spiderbites.nytimes3xbfgragh.onion}{Site Map}
\item
  \href{https://help.nytimes3xbfgragh.onion/hc/en-us}{Help}
\item
  \href{https://www.nytimes3xbfgragh.onion/subscription?campaignId=37WXW}{Subscriptions}
\end{itemize}
