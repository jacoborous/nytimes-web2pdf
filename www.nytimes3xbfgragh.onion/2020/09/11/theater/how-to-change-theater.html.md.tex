Sections

SEARCH

\protect\hyperlink{site-content}{Skip to
content}\protect\hyperlink{site-index}{Skip to site index}

\href{https://www.nytimes3xbfgragh.onion/section/theater}{Theater}

\href{https://myaccount.nytimes3xbfgragh.onion/auth/login?response_type=cookie\&client_id=vi}{}

\href{https://www.nytimes3xbfgragh.onion/section/todayspaper}{Today's
Paper}

\href{/section/theater}{Theater}\textbar{}How to Birth a New American
Theater

\url{https://nyti.ms/3io5P7p}

\begin{itemize}
\item
\item
\item
\item
\item
\item
\end{itemize}

\href{https://www.nytimes3xbfgragh.onion/spotlight/at-home?action=click\&pgtype=Article\&state=default\&region=TOP_BANNER\&context=at_home_menu}{At
Home}

\begin{itemize}
\tightlist
\item
  \href{https://www.nytimes3xbfgragh.onion/2020/09/07/travel/route-66.html?action=click\&pgtype=Article\&state=default\&region=TOP_BANNER\&context=at_home_menu}{Cruise
  Along: Route 66}
\item
  \href{https://www.nytimes3xbfgragh.onion/2020/09/04/dining/sheet-pan-chicken.html?action=click\&pgtype=Article\&state=default\&region=TOP_BANNER\&context=at_home_menu}{Roast:
  Chicken With Plums}
\item
  \href{https://www.nytimes3xbfgragh.onion/2020/09/04/arts/television/dark-shadows-stream.html?action=click\&pgtype=Article\&state=default\&region=TOP_BANNER\&context=at_home_menu}{Watch:
  Dark Shadows}
\item
  \href{https://www.nytimes3xbfgragh.onion/interactive/2020/at-home/even-more-reporters-editors-diaries-lists-recommendations.html?action=click\&pgtype=Article\&state=default\&region=TOP_BANNER\&context=at_home_menu}{Explore:
  Reporters' Google Docs}
\end{itemize}

Advertisement

\protect\hyperlink{after-top}{Continue reading the main story}

Supported by

\protect\hyperlink{after-sponsor}{Continue reading the main story}

\hypertarget{how-to-birth-a-new-american-theater}{%
\section{How to Birth a New American
Theater}\label{how-to-birth-a-new-american-theater}}

Six months dark. Thousands of artists out of work. Could this disaster
have a surprise ending? Five critics on what must change, onstage and
off.

\includegraphics{https://static01.graylady3jvrrxbe.onion/images/2020/09/13/arts/13theater-change-cover/13theater-change-cover-articleLarge.jpg?quality=75\&auto=webp\&disable=upscale}

By \href{https://www.nytimes3xbfgragh.onion/by/jesse-green}{Jesse
Green}, \href{https://www.nytimes3xbfgragh.onion/by/maya-phillips}{Maya
Phillips},
\href{https://www.nytimes3xbfgragh.onion/by/laura-collins-hughes}{Laura
Collins-Hughes},
\href{https://www.nytimes3xbfgragh.onion/by/elisabeth-vincentelli}{Elisabeth
Vincentelli} and
\href{https://www.nytimes3xbfgragh.onion/by/alexis-soloski}{Alexis
Soloski}

\begin{itemize}
\item
  Sept. 11, 2020
\item
  \begin{itemize}
  \item
  \item
  \item
  \item
  \item
  \item
  \end{itemize}
\end{itemize}

When New York City shut down on March 12, its theaters were preparing
for a busy spring season: big names on Broadway, Tony Awards mania,
millions of dollars in sales and of course a smattering of thoughtful,
important plays on smaller stages.

That's all gone.

A cause for grieving, yes: especially for the thousands of artists out
of work. Playwrights awaiting their breakthroughs no less than producers
awaiting their windfalls instantly faced a future that had literally
gone dark.

But what if the end of the business-as-usual party were actually the
start of a new dream of what theater could be in New York --- and by
extension in the rest of the country? It's not as if the shotgun
marriage of art and industry that for decades decided what and whom we
see onstage had produced an equitable, or even a sensible, result.

Just the opposite, as the Black Lives Matter movement and cultural
offshoots like \href{https://www.weseeyouwat.com/}{We See You White
American Theater} have pointed out. The racist assumptions, lordly
practices and bad compromises that have favored some voices and
squelched others at every level of production amount to what Jamil Jude,
artistic director of \href{https://truecolorstheatre.org/}{True Colors
Theater Company} in Atlanta, has called ``a gross case of malpractice.''

And then there is the garden-variety malpractice of an industry
perpetually at odds with itself. As the increased violence against Black
Americans has laid bare the inequities of creative access, the collapse
of the economy has forced us to notice just how badly organized the
business part of show business has been.

Things clearly had to change --- and with the enforced pause of the
pandemic, the opportunity has now arrived in the nick of time. If ever
there was a need, and a moment, to fix the theater, this is it.

So for the six-month anniversary of the shutdown, The New York Times
asked its theater critics --- as well as dozens of people who make
theater every day --- what those fixes might look like.

Some of their ideas are pie-in-the-sky. (Profit-sharing?) Some are
small-bore. (No more couches onstage!) None taken alone, or even all
together, will effect an immediate, magical change to full equity,
inclusion and financial stability. And even the biggest, best
innovations will be difficult to sell in an environment that lacks
concerted vocal leadership from those in power. It may be up to artists
themselves, working from the ground up, to make change happen.

But it's worth noting that the American theater has remade itself during
disaster before. The Depression led to a flourishing of socially
conscious (and often government funded) drama that produced a golden age
of playwriting. In the aftermath of World War II, the regional theater
movement arose to make the art form more responsive to local audiences
and less fixated on profit.

Likewise, in the six months since theaters went dark, we have already
seen that theater can arise from the ashes of the world's (and its own)
failures. In some ways it has even thrived. Artists in their lockdown
apartments, whether next door in New York or anywhere in the world, have
been creating new work online and delivering it to anyone who wants to
watch it. This new ecology of all-access production has reminded many of
us that the human need to make and share stories, not just to sell them,
is immortal.

Even so, especially at moments of great change, it needs to be midwifed.
As the actor and playwright Nikkole Salter has said of this moment,
``Ask women who have given natural birth: There is a time to breathe and
a time to push.''

This is a time to push. And here are some ways to start. JESSE GREEN

\begin{center}\rule{0.5\linewidth}{\linethickness}\end{center}

\hypertarget{theater-must-open-up-the-canon-of-classics}{%
\subsection{Theater Must Open Up the Canon of
Classics}\label{theater-must-open-up-the-canon-of-classics}}

Too many playwrights who emerged from the Harlem Renaissance and the
Black Arts Movement have been forgotten. They deserve a second look. By
Maya Phillips

\includegraphics{https://static01.graylady3jvrrxbe.onion/images/2020/09/13/arts/13change-package/13change-package-articleLarge.jpg?quality=75\&auto=webp\&disable=upscale}

Class, it's time to review the syllabus. Shakespeare, Ibsen, Williams,
Miller, Pinter: If the history books have taught us anything, it is that
theater loves a singular image of brilliance --- and that image is often
of a white man.

To build a new theater, we need to break open this canon, making room
for people of color to be studied in classrooms and thus, eventually,
take their place on contemporary stages.

We have, and will surely see again, the plays of
\href{https://www.nytimes3xbfgragh.onion/2017/01/11/theater/what-august-wilson-means-now.html?searchResultPosition=1}{August
Wilson} and Lorraine Hansberry. We are well-acquainted with Suzan-Lori
Parks and have just met
\href{https://www.nytimes3xbfgragh.onion/2018/11/28/theater/jeremy-o-harris-slave-play.html?searchResultPosition=19}{Jeremy
O. Harris}. But to expect them to represent the whole history of Black
theater is lazy and ignorant.

Embrace Hansberry and Wilson and Parks and Harris, but consider them in
a long, rich and largely unknown historical context.

Three points on the timeline: In 1821, William Alexander Brown opened
the African Theater, the first Black theater in New York City, and two
years later his play ``The Drama of King Shotaway'' was presented there.
It's considered the first work by a Black playwright produced in this
country.

In 1896, George Walker and Bert Williams were the first Black performers
on Broadway in ``The Gold Bug.''

In 1916, ``Rachel,'' by Angelina Weld Grimké, became the century's first
full-length play written by a Black playwright and acted and produced by
Black people.

I knew of Grimké as a noted
\href{https://poets.org/poet/angelina-weld-grimke}{Harlem Renaissance
poet}, but not as a dramatist. Is that because her work was billed as a
``race play'' and derided as too political?

``Rachel'' --- about a bright young Black woman who becomes
disillusioned with the injustice African-Americans encounter and decides
she'll never bring children into this unjust world --- is worth
revisiting now, for its lively dialogue, advanced sexual politics and
stubborn portrayal of racism.

There are countless others ready for their close-up. New York theaters
have recently presented work by
\href{https://www.nytimes3xbfgragh.onion/2018/01/10/theater/adrienne-kennedy-playwright-still-quiet-still-bold-still-furious.html}{Adrienne
Kennedy}, including a brand-new play, and the Roundabout Theater Company
promises to stage ``Trouble in Mind'' by
\href{https://www.nytimes3xbfgragh.onion/1994/08/19/obituaries/alice-childress-77-a-novelist-drew-themes-from-black-life.html}{Alice
Childress} on Broadway when theaters open again.

I want to hear from May Miller and
\href{https://www.nytimes3xbfgragh.onion/1972/03/19/archives/bullins-its-not-the-play-i-wrote-bullins-its-not-the-play-i-wrote.html}{Ed
Bullins},
\href{https://www.nytimes3xbfgragh.onion/1998/05/01/arts/louis-peterson-76-playwright-who-opened-doors-for-blacks.html}{Louis
Peterson} and
\href{https://www.nytimes3xbfgragh.onion/1996/06/13/arts/lonne-elder-69-pioneering-playwright-dies.html}{Lonne
Elder III} and Eulalie Spence --- playwrights of the Harlem Renaissance
and Black Arts Movement, moments in American cultural history marked by
pride in self-presentation. (Several of them I learned about only
through research; I, too, need to expand my education.)

The Black Arts figures were central to the tradition of activist art
from the 1960s and '70s. Agitprop gets a bad rap, but it was a powerful
tool of protest against the Vietnam War. So if radical times demand
radical means of expression, why not revive the incendiary dramas of
Amiri Baraka? Or look further back, to the political plays of the Harlem
Renaissance poet
\href{https://www.georgiaencyclopedia.org/articles/arts-culture/georgia-douglas-johnson-ca-1877-1966}{Georgia
Douglas Johnson,}who wrote fiercely about lynching?

We need to look forward, too. Contemporary playwrights of color are
plentiful in the pipeline, and they are getting commissions. But they
need more than residencies and promises of consideration; they need
productions.

Once Covid has left us, let's see theaters deliver full seasons of work
by people of color, and not just fill a slot. Let's keep track of the
commendable
\href{https://www.lct.org/explore/blog/lct-announces-beaumont-new-play-commission-program/}{promise
just made by Lincoln Center Theater} --- commissioning writers of color
for shows aimed at its big, potentially lucrative Broadway house, not
one of the smaller spaces.

``The Negro is already in the theater and has been there for a long
time; but his presence there is not yet thoroughly normal,'' W.E.B.
DuBois wrote. ``His audience is mainly a white audience and the Negro
actor has, for a long time, been asked to entertain this more or less
alien group.''

That was 1926. Things haven't much changed for Black artists, nor for
Latinx and Asian and Native American ones, and every other nonwhite
group.

In this time of turbulence, we must rally for a theater that rises to
the full force of the moment.

\textbf{While we're at it:} Schedule more
``\href{https://www.nytimes3xbfgragh.onion/2019/12/03/style/self-care/at-black-out-performances-the-power-of-healing-through-community.html}{Black
out'' nights} --- discounted performances exclusively for people of
color, as Harris arranged for ``Slave Play.'' This will help make
theater welcome, and accessible, to audiences that rarely get to see
people like themselves onstage.

\begin{center}\rule{0.5\linewidth}{\linethickness}\end{center}

\hypertarget{theater-must-embrace-streaming-to-grow-audiences}{%
\subsection{Theater Must Embrace Streaming to Grow
Audiences}\label{theater-must-embrace-streaming-to-grow-audiences}}

Experiments in lockdown have made live performance far more accessible,
reaching new fans all over the world. There's no going back. By Jesse
Green

Image

Clockwise from top left: Jay O. Sanders, Maryann Plunkett,~ Sally
Murphy, Stephen Kunken and Laila Robins in the Zoom play "And So We Came
Forth,'' the second in a trilogy by Richard Nelson.~Credit...Sara
Krulwich/The New York Times

Streamed theater was supposed to be a tourniquet: an emergency measure
to stop the industry from bleeding out while the pandemic made in-person
performance impossible.

But something totally unexpected happened.
\href{https://www.nytimes3xbfgragh.onion/2020/06/17/theater/review-state-vs-natasha-banina.html}{Zoom
plays},
\href{https://www.nytimes3xbfgragh.onion/2020/08/10/theater/review-weston-playhouse-one-room.html}{Instagram
monologues}, \href{https://www.youtube.com/watch?v=YeOqzQedwHE}{YouTube
shorts} and other hybrids started blossoming on their own terms --- and
with a few huge advantages.

Those advantages are so important that they need to be part of the new
normal. When live theater finally returns, the streamed kind, far from
disappearing, must continue in parallel.

Fairness alone demands it. The low-cost, high-impact, huge-reach format
allows artists who could barely get past the gatekeepers before to
establish themselves on a nearly equal footing with long-ensconced
figures.

The same goes for audiences. Anyone with a computer can now see almost
anything, regardless of where they live, what their schedule is and
whether they have disabilities or differences that physical theaters too
often fail to address. A subtler barrier has also been removed, so that
people who don't fit the traditional audience profile --- which is
whiter, older and wealthier than the general population --- need no
longer feel unwelcome.

Can we really dream of retracting that access?

Access works the other way, too. Theaters anywhere can now play
everywhere, achieving viewership numbers they never dreamed of. Richard
Nelson's
``\href{https://www.nytimes3xbfgragh.onion/2020/04/30/theater/what-do-we-need-to-talk-about-review.html}{What
Do We Need to Talk About?}'' --- produced on Zoom for an eight-week run
--- was seen by almost 80,000 people. It would have taken a year to
accommodate that many people at The Public Theater, Nelson's home base.
Far smaller companies have also seen their audience numbers soar.

Of course, some people do not consider streamed theater to be theater at
all. That's true to the extent you define the form as a gathering in
real space of performers and viewers. But let's recall that intimacy is
always an illusion. Actors aren't really eyeing everyone in the
1,500-seat Winter Garden Theater --- they just seem to be. Newly created
virtual productions, especially live ones, are already mastering
\href{https://theatreforone.com/nowplaying}{technological and dramatic
workarounds} that replicate or improve on the intimacy of in-person
work. A mega-example: ``Hamilton.''

The real problem is (what else?) money. Most online productions have
been free or fund-raisers: good for the viewers and the organizations
that benefit, not for the already cash-strapped artists who even in
regular times are so often underpaid.

But abandoning a format that promises a more democratic reach cannot be
the right solution.

So let's make it work. Let's use those huge and diverse viewership
figures to stimulate support; surely so many eyeballs will make the
format attractive to governmental, philanthropic and even commercial
funders. Add in modest ticket prices --- less than the cost of a movie
--- and you have the makings of an economy that provides decent pay to
artists while engaging a much wider audience and broadening interest in
the theater as a whole --- including the traditional commercial theater.

Best of all, this multitrack system might finally uncork the pipeline
for new work, letting flow the full diversity of what's out there
waiting. So let's make streaming theater something more than a
tourniquet. Let's make it a flag.

\textbf{And while we're at it:} Get the unions and the rights-holders
organizations together to hammer out a plan that will finally make
Lincoln Center's incredible
\href{https://www.nypl.org/about/divisions/theatre-film-and-tape-archive}{Theater
on Film and Tape Archive} available online to everyone.

\begin{center}\rule{0.5\linewidth}{\linethickness}\end{center}

\hypertarget{theater-must-be-affordable-to-essential-workers}{%
\subsection{Theater Must Be Affordable to Essential
Workers}\label{theater-must-be-affordable-to-essential-workers}}

There is no diversifying the theater without diversifying the audience.
The people who keep our society moving must not be priced out. By Laura
Collins-Hughes

Image

Credit...Lincoln Agnew

In theater, there is theory and there is practice, and in fundamental
ways the two often don't meet.

Listen for a moment to
\href{https://www.nytimes3xbfgragh.onion/2020/01/09/theater/jamie-lloyd-director-cyrano.html}{the
British director Jamie Lloyd}, who made a splash on Broadway last season
with his glamorous, gutting revival of
``\href{https://www.nytimes3xbfgragh.onion/2019/09/05/theater/betrayal-review-tom-hiddleston.html}{Betrayal}.''

``Theater is about trying to connect on a very deep level to another
human being,'' he said in a New York Times interview in December.
``We're trying to learn about who we are and make sense of our
existence, make sense of our relationships with each other.

``And that can't be an experience that only certain people have,'' he
argued. ``Everybody needs to have that experience --- or have it
available to them, anyway.''

That first bit is the theory. The second bit is what Lloyd has put into
practice in London, where his namesake company's West End productions
have become known for movie-star leads and a robust
\href{https://www.theplayhousetheatre.co.uk/the-seagull}{ticket-accessibility
program}. Last season, it offered young people, job seekers and,
startlingly, key workers --- we would call them essential workers ---
15,000 tickets at 15 British pounds apiece. That's about \$20.

For Broadway and much of the American theater, affordable tickets for
essential workers is an idea emphatically worth adopting. It's not only
the right thing to do; it's the smart thing. There is no diversifying
the theater without diversifying the audience, and this would be a
meaningful step in that direction.

In the 2018-19 Broadway season, theatergoers' average household income
was \$261,000, according to
\href{https://www.broadwayleague.com/research/research-reports/}{research
by the Broadway League}. Among those 25 and older, 81 percent were
college graduates, and 41 percent had a graduate degree. They spent an
average of \$145.60 on a ticket.

To a lot of people, no matter how much they love theater, that is an
impossible price point --- and there is a real danger that the cost will
only climb in an industry financially wounded by the pandemic. But
becoming even more exclusive would be an act of grievous self-harm.

Corporate sponsorship enabled the Jamie Lloyd Company's program. Off
Broadway, foundation funding has been the bedrock of Signature Theater
Company's longstanding
\href{https://www.signaturetheatre.org/Support/Ticket-Initiative.aspx}{ticket
initiative}, which last season priced all seats at \$35 for the first
five weeks of each show.

Yes, the economic landscape has changed. But despite all the pain that
this crisis has caused, wealthy foundations, corporations and
individuals are still out there to be tapped.

If we have learned anything positive during the shutdown about how our
society functions, it's that essential workers keep it moving, even when
the rest of us have to hibernate for the greater good. When Broadway
reopens, they cannot be left out.

True, there are ways to score cheap tickets, at least to shows that
aren't selling well. But established schemes are built on the assumption
that people have the disposable time to camp out in line, or the
flexibility to drop everything if they win the digital lottery for that
night.

That can work great for students and tourists --- but if you have a
tight schedule to juggle, or child care to arrange, you're out of luck.
Which is one reason audiences at so many theaters look the way they
look.

It's time to switch that up, and take care of the workers who take care
of us.

\textbf{And while we're at it:} How about banning the secondary ticket
market? Even when it isn't counterfeit, a \$1,000 ticket is a needless
obscenity.

\begin{center}\rule{0.5\linewidth}{\linethickness}\end{center}

\hypertarget{theater-profits-must-be-redistributed}{%
\subsection{Theater Profits Must Be
Redistributed}\label{theater-profits-must-be-redistributed}}

If pro sports can do it, why not Broadway? Pooled resources and
partnerships could bolster Off Broadway nonprofits and the artists loyal
to them. By Elisabeth Vincentelli

Image

The lead producer Jeffrey Seller, center, accepting the 2016~ best
musical Tony Award for ``Hamilton.''Credit...Sara Krulwich/The New York
Times

Profit-making theaters need to subsidize their nonprofit cousins. How to
pull off such a radical idea? Look to professional sports.

Nobody would ever describe the National Football League, Major League
Baseball or the National Basketball Association as a bunch of
red-flag-waving socialists. Yet --- please sit down for this, theater
folks --- they split and pool and redistribute large parts of their
revenues evenly among their teams to ensure that smaller franchises have
a chance against wealthier ones.

Now, theater companies compete for audiences rather than wins over each
other, but the industry could still look at the redistribution concept
for inspiration. The N.F.L. pools revenues from national broadcast
deals, so let's imagine a similar setup in which Broadway shows are
regularly filmed and sold to streaming services. (Forget cannibalizing
ticket sales; the opposite tends to happen when a major screen
adaptation comes out.) Some of the proceeds would then be funneled into
a pool made up of smaller companies and institutions.

Broadway, after all, already uses the nonprofit and Off and Off Off
sectors as a kind of farm system, where underpaid talent develops its
skills, a bit like the way the N.F.L. exploits college players. A
revenue pool would help assuage the unfairness baked into such a system.

Example: The producers of ``Hamilton''
\href{https://www.nytimes3xbfgragh.onion/2020/06/25/movies/hamilton-movie-disney-streaming.html}{spent
under \$10 million to shoot their musical}, then sold the resulting film
to Disney for about \$75 million. (An exceptional figure, I know, and
unlikely to be matched.) With revenue sharing, the producers are
reimbursed for expenses, the creative team and actors earn nice
paychecks, and the rest goes into the pool.

Another way to fill that pot: Create a luxury tax, as there is in
baseball. ``Teams are allowed to spend a certain amount on player
payroll and they have to pay a tax for every dollar they spend above
that,'' Andrew Zimbalist, a Smith College economist who specializes in
sports, patiently explained to me on the phone.

So if you're a producer who wants to cast Hugh Jackman or Bette Midler
in your show, and inevitably raise your ticket prices accordingly, you
pay a luxury tax, with the money then shared among a variety of
recipients.

Imagine what could be done with profits distributed from those pools.

Off and Off Off companies could apply for membership in the pool, the
way they would apply to a grant. Membership could last, say, three
years. Funds could help pay actors a living wage, create a group health
care plan for artists, or subsidize lower-cost tickets for students and
underserved communities.

All these would help underwrite successful Off and even Off Off Broadway
productions that, for whatever reason, might not work on Broadway. Many
of those shows, needing to make way for the next offering in a
nonprofit's season, are faced with either closing while there is still
demand or undertaking a risky, costly Broadway transfer, a situation
that makes no sense.

An influx of pooled money could help companies nurture playwrights by
paying them real money for longer stretches of time. Broadway is not the
end goal for daring incubators like, say, Soho Rep, but they still play
a vital role in the ecosystem and need financial support.

Or take the Roundabout Theater Company, which can program young writers
in its black-box theater, move them up to its Off Broadway venue, then
graduate them to one of its Broadway houses --- with access to pooled
money, it would not need to bring in outside producers for that last
step.

Dedicating pooled money to that leap would create support and incentives
for the same company to maintain exclusive partnerships, and ideally
boost risk-taking from both institutions and playwrights.

\textbf{And while we're at it:} ** Let's take a hard look at the pay
ratio between nonprofit theater leaders, whose salaries seem to be
climbing, and their rank and file employees.

\begin{center}\rule{0.5\linewidth}{\linethickness}\end{center}

\hypertarget{theater-etiquette-must-be-relaxed}{%
\subsection{Theater Etiquette Must Be
Relaxed}\label{theater-etiquette-must-be-relaxed}}

Sit-down-and-shut-up practices are a recent invention. How about making
the experience more inviting? By Alexis Soloski

Image

Adrienne Warren leads the audience in song after the curtain call at
``Tina: The Tina Turner Musical.''~Credit...Sara Krulwich/The New York
Times

In two decades of professional theater going, here are some things I
have been shushed for: coughing, unwrapping a cough drop, reading a
Playbill, writing a note and checking texts when I had left a baby at
home with a fever.

To attend a play is to commit, bodily, to a communal experience. But
contemporary norms ask audience members to pretend our bodies aren't
really there. We are compelled to participate only in approved ways at
approved times --- whooping during a curtain call, say. Behave otherwise
and risk some killjoy in the next row hissing at you to put a sock in
it. What's communal about that?

Who knows what orchestras and balconies will look like when indoor
entertainment returns. But however theaters organize their interiors,
it's time to rethink how people fill those spaces. We should adjust the
compact between performer and audience and the relationships among
spectators, moving from a model where audience behavior is policed
toward one in which engagement --- in various forms --- is celebrated.

Maybe it helps to know that these sit-down-and-shut-up practices are a
recent invention. Ancient Greeks wept and beat their breasts.
Elizabethans ate and drank and threw stuff. Even into the 20th century,
vaudeville and music hall anticipated and rewarded a responsive public.

But in the mid-19th century, when bourgeois theater rebranded as a
civilizing tool and lighting technologies improved, new standards
emerged. ``Audiences were deliberately retrained in these newly imagined
correct modes of behavior --- sitting down being silent,'' said Kirsty
Sedgman, the author of ``The Reasonable Audience: Theatre Etiquette,
Behaviour Policing, and the Live Performance Experience.'' This
retraining had obviously classist roots and arguably racist ones, too.

Still, it remains the norm in most Western theaters. Speak up, dress
down and, as
\href{https://broadwaydirect.com/theater-etiquette-dos-donts-attending-broadway/}{online}
\href{https://www.townandcountrymag.com/leisure/arts-and-culture/a29402504/broadway-theater-new-etiquette-rules/}{etiquette}
\href{https://www.playbill.com/article/15-pieces-of-theatre-etiquette}{guides}
will tell you, you disrupt the experience for performers and other
patrons. Don't believe the guides? Just ask
\href{https://www.nytimes3xbfgragh.onion/2015/07/10/theater/hold-the-phone-its-patti-lupone.html}{Patti
LuPone}.

I would draw a distinction between behaviors that engage with a work
(snapping, clapping), those that don't (Tweeting) and those that don't
or shouldn't affect others (wearing flip-flops). Some conduct is
obviously intolerable ---
\href{https://www.theguardian.com/stage/2015/jul/07/broadway-hand-to-god-cellphone-charge-fake-outlet}{charging
a cellphone} from an onstage outlet,
\href{https://pagesix.com/2019/09/16/did-a-woman-orgasm-during-tom-hiddlestons-broadway-show/?_ga=2.199376814.1633250828.1568634805-1265972161.1561034448}{masturbating
(allegedly) during ``Betrayal.''} But we have created a culture that
shames even benign participation --- laughing, crying, taking a selfie
as the lights go down. A compulsive rule follower, I've still been
scolded hilariously often, sometimes for doing my job, sometimes for
having a sore throat. (This was pre-Covid; coughing mattered less.)

The playwright Dominique Morisseau has included a program insert at her
plays, reminding audiences: ``This can be church for some of us, and
\href{about:blank}{testifying is allowed},'' she writes.

Susan Bennett, the author of
\href{https://books.google.com/books/about/Theatre_Audiences.html?id=3vqRbjkvvr4C}{``Theater
Audiences: A Theory of Production and Reception,''} has suggested that
theaters look at theme parks, interactive museums, sporting events and
even video games to imagine other forms of engagement. Closer to home,
theaters can adopt some of the ways that immersive shows allow audience
members more freedom of movement and behavior.

Here's another thought: theater for kids. I've logged a lot of hours at
children's theater over the past six years, mostly as a civilian. During
these plays, kids wriggle, they giggle, they chat and stamp and sing
along. Performers manage. So do kids and parents. The show goes on.

When theater returns, we should demand more tolerance of other people's
pleasure. Sedgman playfully suggests that instead of offering a handful
of ``relaxed'' performances, designed to accommodate audiences who
aren't necessarily neurotypical, theaters could instead make most
performances relaxed and designate a few performances as ``uptight.''

There's a lot of talk now about accessibility and ways to make theaters
available to people of different abilities, races, ages and buying
power. One idea? Acknowledge that not everyone enjoys themselves at the
same volume and in the same way. Let's make delight and not obedience
the new normal.

\textbf{And while we're at it:} If we allow eating and drinking,
couldn't we make concessions much, much better and not quite so
soul-crushingly expensive?

Advertisement

\protect\hyperlink{after-bottom}{Continue reading the main story}

\hypertarget{site-index}{%
\subsection{Site Index}\label{site-index}}

\hypertarget{site-information-navigation}{%
\subsection{Site Information
Navigation}\label{site-information-navigation}}

\begin{itemize}
\tightlist
\item
  \href{https://help.nytimes3xbfgragh.onion/hc/en-us/articles/115014792127-Copyright-notice}{©~2020~The
  New York Times Company}
\end{itemize}

\begin{itemize}
\tightlist
\item
  \href{https://www.nytco.com/}{NYTCo}
\item
  \href{https://help.nytimes3xbfgragh.onion/hc/en-us/articles/115015385887-Contact-Us}{Contact
  Us}
\item
  \href{https://www.nytco.com/careers/}{Work with us}
\item
  \href{https://nytmediakit.com/}{Advertise}
\item
  \href{http://www.tbrandstudio.com/}{T Brand Studio}
\item
  \href{https://www.nytimes3xbfgragh.onion/privacy/cookie-policy\#how-do-i-manage-trackers}{Your
  Ad Choices}
\item
  \href{https://www.nytimes3xbfgragh.onion/privacy}{Privacy}
\item
  \href{https://help.nytimes3xbfgragh.onion/hc/en-us/articles/115014893428-Terms-of-service}{Terms
  of Service}
\item
  \href{https://help.nytimes3xbfgragh.onion/hc/en-us/articles/115014893968-Terms-of-sale}{Terms
  of Sale}
\item
  \href{https://spiderbites.nytimes3xbfgragh.onion}{Site Map}
\item
  \href{https://help.nytimes3xbfgragh.onion/hc/en-us}{Help}
\item
  \href{https://www.nytimes3xbfgragh.onion/subscription?campaignId=37WXW}{Subscriptions}
\end{itemize}
