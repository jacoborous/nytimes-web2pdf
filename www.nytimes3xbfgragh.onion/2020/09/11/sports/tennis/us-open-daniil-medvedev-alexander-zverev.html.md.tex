Sections

SEARCH

\protect\hyperlink{site-content}{Skip to
content}\protect\hyperlink{site-index}{Skip to site index}

\href{https://www.nytimes3xbfgragh.onion/section/sports/tennis}{Tennis}

\href{https://myaccount.nytimes3xbfgragh.onion/auth/login?response_type=cookie\&client_id=vi}{}

\href{https://www.nytimes3xbfgragh.onion/section/todayspaper}{Today's
Paper}

\href{/section/sports/tennis}{Tennis}\textbar{}For U.S. Open Contenders
Medvedev and Zverev, a Journey From Russia Was a Big Step

\url{https://nyti.ms/3bOA58R}

\begin{itemize}
\item
\item
\item
\item
\item
\item
\end{itemize}

Advertisement

\protect\hyperlink{after-top}{Continue reading the main story}

Supported by

\protect\hyperlink{after-sponsor}{Continue reading the main story}

\hypertarget{for-us-open-contenders-medvedev-and-zverev-a-journey-from-russia-was-a-big-step}{%
\section{For U.S. Open Contenders Medvedev and Zverev, a Journey From
Russia Was a Big
Step}\label{for-us-open-contenders-medvedev-and-zverev-a-journey-from-russia-was-a-big-step}}

Daniil Medvedev and Alexander Zverev are part of a talented final four,
all seeking their first Grand Slam title.

\includegraphics{https://static01.graylady3jvrrxbe.onion/images/2020/09/11/sports/10usopen-russia2-print/10usopen-russia2-print-articleLarge-v2.jpg?quality=75\&auto=webp\&disable=upscale}

\href{https://www.nytimes3xbfgragh.onion/by/christopher-clarey}{\includegraphics{https://static01.graylady3jvrrxbe.onion/images/2018/09/10/multimedia/author-christopher-clarey/author-christopher-clarey-thumbLarge.png}}

By
\href{https://www.nytimes3xbfgragh.onion/by/christopher-clarey}{Christopher
Clarey}

\begin{itemize}
\item
  Sept. 11, 2020
\item
  \begin{itemize}
  \item
  \item
  \item
  \item
  \item
  \item
  \end{itemize}
\end{itemize}

Two tennis stories that began in Russia will continue in New York on
Friday, as Daniil Medvedev and Alexander Zverev take turns trying to
reach the United States Open men's final.

Both are trilingual and 6-foot-6, which was once considered too tall for
a great tennis player. Both clearly have the power and potential to win
multiple Grand Slam titles, and this title, for a change, is there for
the younger generation's taking, with none of the Big Three --- Novak
Djokovic, Roger Federer or Rafael Nadal --- blocking the path.

Zverev, 23, the German son of Russian professional tennis players, will
face Pablo Carreño Busta of Spain, 29, in the first semifinal. Medvedev,
24, raised in Moscow but now married and based in Monaco, will then take
on Dominic Thiem, 27, the big-swinging Austrian who is the highest seed
remaining, at No. 2. None of the four has won a Grand Slam tournament.

``The most important thing is to not get ahead of yourself,'' said
Gilles Cervara, Medvedev's coach. ``We all know what it means to not
have those three guys in the tournament, but we cannot stay fixated on
it. It's all about going little by little, step by step.''

\includegraphics{https://static01.graylady3jvrrxbe.onion/images/2020/09/11/sports/11usopen-russia-med/merlin_176804280_105909f7-63ca-48ba-935d-6427dd414382-articleLarge.jpg?quality=75\&auto=webp\&disable=upscale}

It took many steps to bring Medvedev and Zverev to the brink, and in
both cases, leaving Russia was a big one.

Zverev's parents, Alexander Sr. and Irina, met in Sochi, the Russian
resort city and tennis hub on the Black Sea. It was also the home of
\href{https://www.atptour.com/en/news/atp-heritage-kafelnikov-no-1-fedex-atp-rankings}{Yevgeny
Kafelnikov}, the first Russian man to be ranked No. 1 in the world in
tennis, and it was where Maria Sharapova, a future women's No. 1, spent
her very early childhood before leaving with her father, Yuri, for the
tennis academies of Florida, with less than \$1,000 in cash in Yuri's
pocket.

Alexander Sr., an attacking player, came of age in the Soviet Union,
where tennis was long viewed suspiciously as a bourgeois pursuit and
where leading players often found it difficult to leave for
international events. Stars did emerge, like Olga Morozova and Alex
Metreveli, who both reached Wimbledon finals in the 1970s, and later
Natasha Zvereva and Andrei Chesnokov, who both broke into the top 10 in
singles and had an ongoing and risky tussle with Soviet authorities over
how much of their prize money they could retain.

Alexander Sr. peaked at No. 175, and Irina, seven years younger, peaked
at No. 380. Their oldest son, Mischa, was born in Russia, but the family
moved to Germany in 1991, and Alexander, nicknamed Sascha, was born in
Hamburg in 1997.

The brothers became professionals with radically different playing
styles.
\href{https://www.atptour.com/en/players/mischa-zverev/z168/overview}{Mischa,
ranked as high as 25 in 2017}, is one of the few pure
serve-and-volleyers on tour, while Alexander, who has
been\href{https://www.atptour.com/en/players/alexander-zverev/z355/overview}{ranked
as high as No. 3}, is a more conventional attacking baseliner with a
huge serve and a potent, fluid two-handed backhand.

``I'm not surprised at all the Zverevs raised two top players,''
Kafelnikov said in a telephone interview from Moscow. ``They knew what
they had faced in Russia before, and they knew exactly what they wanted
to give to their kids. They wanted the kids to be professional tennis
players and be successful. Mischa is playing a little bit like his
father's style. Sascha is quite a bit different, with the big
groundstrokes.''

Kafelnikov, 46, was the most successful Russian player after the Soviet
Union broke up in 1991. He initially scrambled to find training bases
and stability with his coach, Anatoly Lepeshin, a former head of the
Soviet junior program who once worked with Alexander Sr. But while other
talented Russian players were unable to find the funding to continue
their careers, Kafelnikov, a smooth and flat-hitting baseliner,
persevered to become the first Russian man to win a Grand Slam singles
title, at the French Open in 1996, and then won the Australian Open in
1999.

He later joined forces with Marat Safin, Russia's other big post-breakup
men's tennis star, to win the Davis Cup for the first time for Russia in
2002.

Image

Yevgeny Kafelnikov was inducted into the International Tennis Hall of
Fame~ last summer.Credit...Brian Snyder/Reuters

Safin, a swashbuckling figure who broke rackets almost as often as he
broke serve, also reached No. 1 and won two Grand Slam singles
titles:\href{https://www.nytimes3xbfgragh.onion/2000/09/11/sports/tennis-us-open-in-the-final-youth-serves.html}{the
2000 U.S. Open}, where he shocked Pete Sampras in straight sets in the
final, and the 2005 Australian Open.

Like Sharapova and another future star of the women's game, Anna
Kournikova, Safin left Russia to develop his game. He did so in part
because of the lack of academy-style facilities at home and the brutal
winter weather. Safin's parents were coaches and former players, as
well, but he went to Valencia, Spain, to train at age 14 and was joined
by his younger sister,
\href{https://www.nytimes3xbfgragh.onion/2010/08/29/sports/tennis/29safina.html}{Dinara,
who also became No. 1.}

``We could not rely on anybody but ourselves,'' Kafelnikov said. ``My
parents were not that rich, and I knew if I didn't succeed in tennis, I
would have nothing to do. And that's what was driving so many of us. The
life we had in the Soviet Union wasn't great, and when it broke up
everyone was kind of left alone. The choice was simple. Will you do the
hard work or not?''

Svetlana Kuznetsova, the 2004 U.S. Open women's singles champion, went
to Spain, too, before becoming a key part of the big wave of Russian
women who entered the elite in the 2000s --- including Sharapova,
Safina, Anastasia Myskina, Elena Dementieva and Vera Zvonareva.

Russian women's tennis has yet to scale such heights in recent years,
although Zvonareva, 36, has reached the women's doubles final of this
U.S. Open with a German partner, Laura Siegemund.

Image

Vera Zvonareva will play in the women's doubles final on
Friday.Credit...Jason Szenes/EPA, via Shutterstock

But the men, after a fallow period, are now resurgent with Medvedev;
Karen Khachanov, 24, and Andrey Rublev, 22.

Medvedev, a shape-shifting tactician ranked No. 5, is the leader. He
pushed Nadal to five sets before losing last year's classic U.S. Open
final, and he defeated Rublev, his good friend and boyhood rival, in the
quarterfinals on Wednesday.

``I play a little bit more counterattack,'' Medvedev said. ``Maybe
seeing what my opponent does, then deciding how I'm going to play.
Andrey is different. He tries to dictate his game with the forehand, go
for the shots. He doesn't really care what the opponent does. He just
cares about himself, so it's a different strategy. But I think what is a
similarity is that, starting from juniors, we always tried to get
better. We always pushed each other.''

Individual success in tennis so often begins with a talented group of
youngsters who can feed off each other: Consider the great American
men's generation of Sampras, Andre Agassi, Jim Courier and Michael
Chang, all of whom won Grand Slam singles titles.

Khachanov and Rublev both went to Barcelona as teenagers to train at the
4 Slam Academy. Medvedev headed in the same direction in his teens but
ended up on the Côte d'Azur, based in Monaco and training in Cannes,
France, at a small academy co-founded by Cervara.

``He is someone who is very open to new things,,'' Cervara said. ``He
has a mentality that embraces diversity.''

He quickly learned French, and is a fast learner in general, though he
is still waiting for his first Grand Slam singles title. It could come
very soon, although Thiem is a major obstacle and Zverev could be
another one.

``They have not yet accomplished what me and Marat did in our young
careers,'' Kafelnikov said of Medvedev and the other rising Russians.
``Marat won his first Slam at 20. I did it when I was 22. So those guys
have not come to our level yet, but I'm hoping that through the years
they will have better success than Marat and I had.''

Advertisement

\protect\hyperlink{after-bottom}{Continue reading the main story}

\hypertarget{site-index}{%
\subsection{Site Index}\label{site-index}}

\hypertarget{site-information-navigation}{%
\subsection{Site Information
Navigation}\label{site-information-navigation}}

\begin{itemize}
\tightlist
\item
  \href{https://help.nytimes3xbfgragh.onion/hc/en-us/articles/115014792127-Copyright-notice}{©~2020~The
  New York Times Company}
\end{itemize}

\begin{itemize}
\tightlist
\item
  \href{https://www.nytco.com/}{NYTCo}
\item
  \href{https://help.nytimes3xbfgragh.onion/hc/en-us/articles/115015385887-Contact-Us}{Contact
  Us}
\item
  \href{https://www.nytco.com/careers/}{Work with us}
\item
  \href{https://nytmediakit.com/}{Advertise}
\item
  \href{http://www.tbrandstudio.com/}{T Brand Studio}
\item
  \href{https://www.nytimes3xbfgragh.onion/privacy/cookie-policy\#how-do-i-manage-trackers}{Your
  Ad Choices}
\item
  \href{https://www.nytimes3xbfgragh.onion/privacy}{Privacy}
\item
  \href{https://help.nytimes3xbfgragh.onion/hc/en-us/articles/115014893428-Terms-of-service}{Terms
  of Service}
\item
  \href{https://help.nytimes3xbfgragh.onion/hc/en-us/articles/115014893968-Terms-of-sale}{Terms
  of Sale}
\item
  \href{https://spiderbites.nytimes3xbfgragh.onion}{Site Map}
\item
  \href{https://help.nytimes3xbfgragh.onion/hc/en-us}{Help}
\item
  \href{https://www.nytimes3xbfgragh.onion/subscription?campaignId=37WXW}{Subscriptions}
\end{itemize}
