Sections

SEARCH

\protect\hyperlink{site-content}{Skip to
content}\protect\hyperlink{site-index}{Skip to site index}

\href{https://www.nytimes3xbfgragh.onion/section/sports/soccer}{Soccer}

\href{https://myaccount.nytimes3xbfgragh.onion/auth/login?response_type=cookie\&client_id=vi}{}

\href{https://www.nytimes3xbfgragh.onion/section/todayspaper}{Today's
Paper}

\href{/section/sports/soccer}{Soccer}\textbar{}Six Questions Looming
Over the New Season in Europe

\url{https://nyti.ms/33eIO0o}

\begin{itemize}
\item
\item
\item
\item
\item
\end{itemize}

Advertisement

\protect\hyperlink{after-top}{Continue reading the main story}

Supported by

\protect\hyperlink{after-sponsor}{Continue reading the main story}

On Soccer

\hypertarget{six-questions-looming-over-the-new-season-in-europe}{%
\section{Six Questions Looming Over the New Season in
Europe}\label{six-questions-looming-over-the-new-season-in-europe}}

Soccer's new calculus must account for infection rates, empty stadiums
and a compressed schedule certain to exhaust everyone involved.

\includegraphics{https://static01.graylady3jvrrxbe.onion/images/2020/09/12/sports/11onsoccer-print/merlin_176083806_d5947872-f542-46c8-8398-ce7c057e27f7-articleLarge.jpg?quality=75\&auto=webp\&disable=upscale}

\href{https://www.nytimes3xbfgragh.onion/by/rory-smith}{\includegraphics{https://static01.graylady3jvrrxbe.onion/images/2019/08/23/sports/Rory-Smith-better/Rory-Smith-thumbLarge.png}}

By \href{https://www.nytimes3xbfgragh.onion/by/rory-smith}{Rory Smith}

\begin{itemize}
\item
  Sept. 11, 2020
\item
  \begin{itemize}
  \item
  \item
  \item
  \item
  \item
  \end{itemize}
\end{itemize}

An hour or so after the fireworks had finished, long after the smoke had
cleared and the lights had dimmed, a handful of Bayern Munich players
returned to the field at the Stadium of Light in Lisbon. In near
darkness, Serge Gnabry, Joshua Kimmich and David Alaba sat down on the
turf. At last, the strangest season was over. Now was the time to rest
and to reflect.

Or it should have been, at any rate. Yet even before Bayern collected
the Champions League trophy, the new season was already underway. A
peppering of domestic leagues had started across Europe. The early
rounds of the next Champions League and Europa League were already being
played. Other teams had long since returned to preseason training.

This feels like the weekend that the 2020-21 season starts: the opening
of the new \href{https://www.premierleague.com/}{Premier League} and
\href{https://www.laliga.com/en-GB/laliga-santander/results}{La Liga}
campaigns, with
\href{http://www.legaseriea.it/en/serie-a/fixture-and-results}{Serie A}
and the
\href{https://www.bundesliga.com/en/bundesliga/matchday}{Bundesliga}
scheduled to return in a few days' time. The reality, though, is
different: Soccer never really went away.

This is still, though, a watershed moment. The restarts of the
continent's major competitions in May and June felt, at the time,
somehow novel, isolated events, make-do exceptions on the road back to
normality.

Now, though, players and fans alike can see that this is how it is going
to be for the foreseeable future: stadiums empty or open at only a
fraction of their capacities, plans interrupted by the coronavirus, the
ever-present sense that another shutdown could be on the horizon. For
Gnabry, Kimmich and Alaba, and for everyone else, the strangest season
may just be starting.

\includegraphics{https://static01.graylady3jvrrxbe.onion/images/2020/09/11/sports/11onsoccer-liverpool/merlin_175608393_c28d5d81-1836-4b75-9708-b91f55a2ca36-articleLarge.jpg?quality=75\&auto=webp\&disable=upscale}

\hypertarget{the-soccer-season-is-starting-what-if-it-cant-finish}{%
\subsection{The Soccer Season Is Starting. What if It Can't
Finish?}\label{the-soccer-season-is-starting-what-if-it-cant-finish}}

The problem, the clubs of the Premier League quickly realized this
spring, was that there was no rule for this. The competition's
\href{https://resources.premierleague.com/premierleague/document/2020/09/11/dc7e76c1-f78d-45a2-be4a-4c6bc33368fa/2020-21-PL-Handbook-110920.pdf}{handbook}
stretches to hundreds of pages, but as executives pored over it and a
dozen or so appendices in March and April, they saw that not one of them
addressed what happened if a season could not be finished.

Six months later, they admit that the ``rule book did not adequately
deal with the situation,'' as the league's chief executive, Richard
Masters, put it this week. In a meeting of all 20 teams last week, the
hope was that an agreement could be reached that would --- in his words
--- ``add more certainty.''

It did not quite work like that. England's clubs agreed that ``finishing
the season is the No. 1 priority.'' Playing games behind closed doors is
``now enshrined as one of the things you would have to go through before
you reached curtailment.'' But beyond that, there is no plan for what
will happen if coronavirus cases spike again and the league cannot
continue. ``The issue of a cutoff point, or a number of matches to be
played for a season to be valid, was not agreed,'' Masters said. The
same is true in Spain, where La Liga has no blueprint for a worst-case
scenario.

As Europe grapples with the virus's lingering presence and localized
lockdowns and international quarantines become more commonplace, then,
the soccer season opens with a backdrop of uncertainty. Games are
already underway in France and Scotland. England and Spain start this
weekend. Germany's season begins next Friday and Italy's a day after
that. After six months of conversations, what happens if they cannot
finish their schedules is anyone's guess.

Image

Masks and social distancing kept the Premier League virus free for
weeks. But will that last?Credit...Oli Scarff/Pool, via Reuters

\hypertarget{how-will-teams-manage-outbreaks}{%
\subsection{How Will Teams Manage
Outbreaks?}\label{how-will-teams-manage-outbreaks}}

Europe managed to finish last season thanks, in no small part, to the
willingness of the players who comprise the competition to put their
lives on hold for a few weeks. Premier League players, for example, were
told that they would be under
\href{https://www.nytimes3xbfgragh.onion/2020/08/24/sports/soccer/champions-league-premier-league-virus.html}{as
much scrutiny as special forces troops} to ensure plans for the restart
were not thrown into doubt by a virus outbreak.

The game's authorities accept that while such an approach could work for
six weeks, it is not particularly realistic over the nine-month span of
a full season. Indeed, in the short break between campaigns, a raft of
players have contracted coronavirus.
\href{https://www.nytimes3xbfgragh.onion/2020/09/02/sports/soccer/neymar-PSG-coronavirus.html}{Paris
St.-Germain alone} has reported seven cases, including Neymar and Kylian
Mbappé.

In March, of course, it was confirmation that Callum Hudson-Odoi, the
Chelsea forward, and Mikel Arteta, the Arsenal coach, had contracted the
virus that essentially forced the Premier League to shut down. That is
no longer the automatic response; in France, the authorities have said
that teams with four or more players who return positive tests will see
their games postponed.

That does not mean, though, that the course of the season will not be
influenced by the virus. Teams that register significant outbreaks will
face daunting schedules to try to make up lost ground, and individual
players who test positive will lose at least two weeks of training
during their period of isolation, and then require time to be brought
back to full speed, affecting their coaches' plans and their teams'
hopes.

Image

France's top league opened its new season with fans in the stands are
varying degrees of mask discipline.Credit...Pascal Rossignol/Reuters

\hypertarget{will-fans-be-back-and-will-they-stay-back}{%
\subsection{Will Fans Be Back? And Will They Stay
Back?}\label{will-fans-be-back-and-will-they-stay-back}}

Some already are: in France, a number of stadiums have permitted a small
percentage of fans to return over the course of the first two games of
the season. Others are hopeful: RB Leipzig has received permission to
host 8,500 fans at its opening game in the Bundesliga next weekend.

UEFA, European soccer's governing body, had hoped to show its member
associations the way by staging the European Super Cup --- between
Bayern Munich and Sevilla in Budapest, Hungary, on Sept. 24 --- in front
of around 13,000 fans, though that has since been complicated by
Hungary's decision to close its borders.

Progress elsewhere is slow. Spain does not expect fans to return in
numbers until a coronavirus vaccine is available, and England's plans
for test events with 2,500 fans in attendance --- ahead of a possible
broader return to stadiums in October --- have been altered or canceled
because of a rise in case numbers in recent weeks.

At this point, certainly, the idea that stadiums will be even vaguely
recognizable before the latter part of the season --- at the absolute
earliest --- seems remote. Much of the season will be held either
entirely, or largely, without fans in attendance, leaving clubs across
Europe facing a massive shortfall in revenue.

Image

In drawing up a revised calender, UEFA and its president, Aleksander
Ceferin, made sure to carve out room for games important to the
organization's priorities.Credit...Harold Cunningham/Agence
France-Presse --- Getty Images

\hypertarget{how-will-they-fit-in-all-the-games}{%
\subsection{How Will They Fit In All the
Games?}\label{how-will-they-fit-in-all-the-games}}

With difficulty. During the long, tense negotiations over how to restart
the season, the UEFA president, Aleksander Ceferin, was struck by how
smoothly soccer's various squabbling factions came together in extremis.
The good of the game, he said, was paramount; red lines that had been
sources of tension for years suddenly faded away.

A quick glimpse at
\href{https://www.uefa.com/insideuefa/about-uefa/news/025c-0f8e787ef28c-879e44a21e77-1000--updated-uefa-competitions-calendar/}{the
fixture calendar} for the next nine months is enough to suggest that
self-interest did not lay dormant for long. UEFA has made sure to find
space for its Nations League project, as well as playing all six
Champions League group stage games in the space of eight weeks before
Christmas.

More impressively, various national associations have managed to squeeze
in a couple of friendlies --- England against New Zealand, anyone? ---
too. Only minor concessions to things like exhaustion or burnout have
materialized: The Bundesliga will have a two-week winter break, rather
than the traditional month, and England's Football Association has
abandoned replays in the F.A. Cup.

This season, then, is likely to be the survival of the fittest. The
winter, in particular, will be arduous and endless in equal measure. The
teams that have the physical conditioning --- and, more significantly,
can afford the depth of resources --- to cope with its demands are
likely to be those that emerge on top.

That should reduce the (already minimal) likelihood of surprise
contenders challenging the established elite in domestic leagues; it may
also give Bayern Munich a better-than-average shot of retaining the
Champions League: suddenly, Germany's 34-game league campaign has the
look of a distinct advantage. It may also mean that the continent's
stars will be gasping for air by the time they arrive at the European
Championships next summer, after 13 months of almost constant soccer.

Image

The American midfielder Weston McKennie joined Juventus from
Schalke.Credit...Alessandro Di Marco/EPA, via Shutterstock

Image

Andrea Pirlo's first coaching task is a big one: winning a 10th straight
Italian title.Credit...Massimo Pinca/Reuters

\hypertarget{so-where-is-the-entertainment}{%
\subsection{So Where Is the
Entertainment?}\label{so-where-is-the-entertainment}}

While the compressed schedule and the lingering threat of the
coronavirus are the season's overarching themes, there is no shortage of
subplots.

Italy has the look of the most intriguing title race: a Juventus team
coached by a novice, Andrea Pirlo, but boasting the timeworn talents of
Cristiano Ronaldo and, likely, Luis Suárez going for a 10th Scudetto **
in a row, with the Internazionale of Antonio Conte, Romelu Lukaku and
Achraf Hakimi, the summer's best signing, standing in the way.

It is a similar plotline in Scotland, where Celtic stands on the brink
of a 10th consecutive championship, too, something that is all but
unthinkable to Rangers, its nemesis. Bayern Munich is on eight straight
in Germany (there is a theme here, and it is not one that bodes well for
European soccer as a whole) and Borussia Dortmund will need Erling
Haaland to maintain the ruthlessness that marked
\href{https://www.nytimes3xbfgragh.onion/2020/02/18/sports/soccer/dortmund-psg-haaland.html}{his
first few months in the Bundesliga} to stop that becoming nine.

The French season, already underway, is more intriguing. Thomas Tuchel's
P.S.G. lost its opener to Lens on Thursday night, giving heart to its
two likeliest contenders, Marseille and Lyon, the latter fresh from its
run to the Champions League semifinals and yet to be plucked bare by the
continent's predators.

The Premier League looks more evenly poised than in some time.
Manchester City has a title to regain from Liverpool, and Pep Guardiola,
its coach, has a point to prove. And as Liverpool must find some other
sense of purpose after ending its 30-year wait for a domestic title,
Manchester United, Arsenal and Tottenham all ended the season on an
upward curve.

Everton has added stardust, in the form of James Rodríguez, and Marcelo
Bielsa's Leeds United adds intrigue, but it is Chelsea that is most
transformed. Frank Lampard, its rookie coach, was tasked with bringing
through a new generation last year. This time, after spending \$250
million on the likes of Kai Havertz and Timo Werner, expectations,
internal and external, will be far weightier.

And then, of course, there is the story that dominated the late summer,
and the one that could yet prove to be the most defining of this
curious, compact season.

\hypertarget{is-this-the-last-dance-for-lionel-messi}{%
\subsection{Is This the Last Dance for Lionel
Messi?}\label{is-this-the-last-dance-for-lionel-messi}}

Image

Lionel Messi returned to training at Barcelona on Tuesday. How long he
keeps coming is anyone's guess.Credit...Lluis Gene/Agence France-Presse
--- Getty Images

Deceived and dejected, Lionel Messi returned to the Barcelona fold in
the first week of September. It was a strange kind of triumph for the
club: its crown jewel, a player who towers over its history, confessing
that he had wanted to leave for a year, shredding the reputations of its
administrators, expressing how little he believed in its approach,
admitting that he has only stayed
\href{https://www.nytimes3xbfgragh.onion/2020/09/04/sports/soccer/lionel-messi-barcelona.html}{because
he could not leave}.

He has vowed that the circumstances will not see him simply go through
the motions in the final year of his contract; he is too driven, too
competitive for that. There is a temptation to believe that it sets up a
Last Dance scenario, with Messi reviving Barcelona one more time before
heading into the sunset in Manchester or Paris.

For that to happen, though, Messi would have to be proved wrong:
Barcelona would have to demonstrate that it does have some idea of how
it wants to return to the summit, some plan in place, some grand vision
of what the future looks like. For all that Ronald Koeman, its new
coach, is self-confident enough to impose himself at Camp Nou, it feels
a distant prospect.

More likely is a far sadder denouement than anyone could have expected,
even a year ago, one far bleaker than Messi deserves, but somehow
fitting for the world we find ourselves in: a season played out in front
of empty stands, the greatest of all time watching the ticking of the
clock, the games coming so thick and fast that nobody has chance to
catch their breath, the virus such a threat that nobody --- not even him
--- has chance to say goodbye.

Advertisement

\protect\hyperlink{after-bottom}{Continue reading the main story}

\hypertarget{site-index}{%
\subsection{Site Index}\label{site-index}}

\hypertarget{site-information-navigation}{%
\subsection{Site Information
Navigation}\label{site-information-navigation}}

\begin{itemize}
\tightlist
\item
  \href{https://help.nytimes3xbfgragh.onion/hc/en-us/articles/115014792127-Copyright-notice}{©~2020~The
  New York Times Company}
\end{itemize}

\begin{itemize}
\tightlist
\item
  \href{https://www.nytco.com/}{NYTCo}
\item
  \href{https://help.nytimes3xbfgragh.onion/hc/en-us/articles/115015385887-Contact-Us}{Contact
  Us}
\item
  \href{https://www.nytco.com/careers/}{Work with us}
\item
  \href{https://nytmediakit.com/}{Advertise}
\item
  \href{http://www.tbrandstudio.com/}{T Brand Studio}
\item
  \href{https://www.nytimes3xbfgragh.onion/privacy/cookie-policy\#how-do-i-manage-trackers}{Your
  Ad Choices}
\item
  \href{https://www.nytimes3xbfgragh.onion/privacy}{Privacy}
\item
  \href{https://help.nytimes3xbfgragh.onion/hc/en-us/articles/115014893428-Terms-of-service}{Terms
  of Service}
\item
  \href{https://help.nytimes3xbfgragh.onion/hc/en-us/articles/115014893968-Terms-of-sale}{Terms
  of Sale}
\item
  \href{https://spiderbites.nytimes3xbfgragh.onion}{Site Map}
\item
  \href{https://help.nytimes3xbfgragh.onion/hc/en-us}{Help}
\item
  \href{https://www.nytimes3xbfgragh.onion/subscription?campaignId=37WXW}{Subscriptions}
\end{itemize}
