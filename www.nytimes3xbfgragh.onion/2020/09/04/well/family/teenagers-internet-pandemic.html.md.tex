Sections

SEARCH

\protect\hyperlink{site-content}{Skip to
content}\protect\hyperlink{site-index}{Skip to site index}

\href{https://www.nytimes3xbfgragh.onion/section/well/family}{Family}

\href{https://myaccount.nytimes3xbfgragh.onion/auth/login?response_type=cookie\&client_id=vi}{}

\href{https://www.nytimes3xbfgragh.onion/section/todayspaper}{Today's
Paper}

\href{/section/well/family}{Family}\textbar{}When My Dad Turned Off the
Internet

\url{https://nyti.ms/3i1MkBi}

\begin{itemize}
\item
\item
\item
\item
\item
\item
\end{itemize}

\href{https://www.nytimes3xbfgragh.onion/spotlight/at-home?action=click\&pgtype=Article\&state=default\&region=TOP_BANNER\&context=at_home_menu}{At
Home}

\begin{itemize}
\tightlist
\item
  \href{https://www.nytimes3xbfgragh.onion/2020/09/07/travel/route-66.html?action=click\&pgtype=Article\&state=default\&region=TOP_BANNER\&context=at_home_menu}{Cruise
  Along: Route 66}
\item
  \href{https://www.nytimes3xbfgragh.onion/2020/09/04/dining/sheet-pan-chicken.html?action=click\&pgtype=Article\&state=default\&region=TOP_BANNER\&context=at_home_menu}{Roast:
  Chicken With Plums}
\item
  \href{https://www.nytimes3xbfgragh.onion/2020/09/04/arts/television/dark-shadows-stream.html?action=click\&pgtype=Article\&state=default\&region=TOP_BANNER\&context=at_home_menu}{Watch:
  Dark Shadows}
\item
  \href{https://www.nytimes3xbfgragh.onion/interactive/2020/at-home/even-more-reporters-editors-diaries-lists-recommendations.html?action=click\&pgtype=Article\&state=default\&region=TOP_BANNER\&context=at_home_menu}{Explore:
  Reporters' Google Docs}
\end{itemize}

Advertisement

\protect\hyperlink{after-top}{Continue reading the main story}

Supported by

\protect\hyperlink{after-sponsor}{Continue reading the main story}

Ties

\hypertarget{when-my-dad-turned-off-the-internet}{%
\section{When My Dad Turned Off the
Internet}\label{when-my-dad-turned-off-the-internet}}

I'm 16. When my parents were my age, they didn't have the internet and
they didn't have a pandemic.

\includegraphics{https://static01.graylady3jvrrxbe.onion/images/2020/09/02/science/TIES-INTERNET/TIES-INTERNET-articleLarge.jpg?quality=75\&auto=webp\&disable=upscale}

By Zoya Aziz

\begin{itemize}
\item
  Sept. 4, 2020
\item
  \begin{itemize}
  \item
  \item
  \item
  \item
  \item
  \item
  \end{itemize}
\end{itemize}

The pandemic tension in my house reached its peak this summer the night
my dad switched off the Wi-Fi. He claimed that ``You spend too much time
on your phones and not enough time with the family! I never had the
internet when I was your age; we used to play on the streets.''

With all due respect, when he was my age --- 16 --- it was the 1980s,
and the world was not in the middle of a pandemic. My parents are
doctors who have both had the virus and are well aware of the impact of
this pandemic. Dad is just saying what many parents say because they are
unsettled by how much time my generation spends online. But teenagers
are wired to be social, and right now the internet is one of the few
places we can safely socialize.

Later that evening I sneaked downstairs with my 8-year-old brother to
switch the router back on. But it was being heavily guarded --- it was
in our dad's study, where he was working. The next day I awoke early and
immediately checked to see if we were back online. No, we weren't. In
fact, the whole box was gone! My dad had taken it to work with him that
morning. I was speechless. Who turns off the internet? Apparently quite
a lot of you, a search of Twitter told me later, once I had internet
access again.

But in the moment, we refused to accept defeat. At first we were in
denial, and could not believe that the box was not in the house. We
searched under the beds, behind the TV and even in the bathroom. There
was no sign of it. In ordinary circumstances when a pandemic was not
raging across the world and schools were all open, this could have been
more bearable. Perhaps. But in lockdown, with no school and all events
canceled in Bristol, England, where we live, it seemed as if my
connection to the outside world had been severed abruptly.

Time felt infinite. When I am watching Netflix, time seems to accelerate
and before I know it, the hours have flown by. But with the internet
gone, time became my worst enemy.

I looked elsewhere for inspiration. My dad said that as a child he had
played on the streets all day. I took out my bike for a ride. It was
hot, and none of my friends were around, so I soon went back inside. I
realized then just how reliant I was on technology. I had been using my
phone or laptop for reading, watching movies, playing games and talking
to friends. Without the internet, I could barely do anything that I
normally would do in a day.

At first I was angry, as my dad had taken away my only level of
connection to the outside world. I had a Zoom meeting with my friends
later that day which I knew I would not be able to attend. I worried
that my friends would think I was ignoring them, but I was later able to
explain what had happened --- much to their amusement. To make matters
worse, my cellphone data had run out just before the internet was
switched off. So I tried to connect to the neighbor's Wi-Fi, but it was
password protected. After a few aimless hours I started reading actual
books. It was better than I had anticipated. Fortunately I have many
books. When Dad came home that evening the router was not with him. He
had left it at work.

Though my dad's parenting stance was rather authoritarian, he had a
valid point. My brothers and I were spending far too much time hiding
away in our rooms as if we were self-isolating from the family. We had
always done that to some extent, but far more so in lockdown. In the
absence of technology, we went for walks, baked cakes and cycled
together. My mom taught me how to cook some of my favorite meals from
recipes handed down from my grandmother. But when it came to baking we
had a few disasters. I made a misshapen, overly sweet cake which no one
ate, not even my little brother.

The one upside was that it gave my family something to laugh about
together, which helped us appreciate each other.

And then after one week, without any big announcement, my dad switched
the internet back on. I didn't immediately go running upstairs to check
my phone. The brief time without the internet had changed me: I realized
that I wasn't really missing as much as I thought I was.

Despite the lessons I learned from this experience, a part of me wishes
that my dad had taken a different approach in encouraging us to spend
more time as a family. He made the decision unilaterally, before asking
us kids why we spent so much time cooped up in our rooms. I felt as if
he could not comprehend the reality of how the pandemic was affecting my
life. My parents were going out to work and had little spare time,
unlike us who were spending a lot of time at home. I was initially upset
and disappointed. I wished he had spoken to me first and given me the
chance to make decisions with them. But perhaps he had a point. Would I
have listened otherwise?

My daily life differs so much from my parents' younger years, and it's
hard to imagine they were ever teenagers. But they still have the
capacity to understand and learn from me as I do from them. They too
have learned that their role as confidants is invaluable for me and my
siblings and that talking to us and asking us questions (but not too
many!) is beneficial.

Although the traditional questions like ``How was your day at school?''
or ``What did you eat for lunch?'' no longer apply, I actually still
appreciate my parents' questions. Asking what we are doing to spend our
time in lockdown, helping us structure our days or writing a list of
goals has really helped me. I have developed a better rapport with them
and our relationship is stronger.

Although I'm really glad to have the internet back, I realize that our
conflict was never about the internet. It was a chance for my parents to
remind me and my brothers to appreciate human connections and strike a
balance in our lives.

And as my siblings and I head back to a new school year with new,
different routines, I think it also helped my parents appreciate that
being a kid in 2020 is not the same as when they were kids.

Before they would frequently urge me to ``Come downstairs and spend time
with the family!'' and ask me ``What are you doing in your room?'' But
now, since lockdown and especially our internet shutdown, my parents are
working on respecting my autonomy and understanding that sometimes I
need space and time alone, far from the chaos and drama of today's
world.

\emph{Zoya Aziz is a high school student in Bristol, England.}

Advertisement

\protect\hyperlink{after-bottom}{Continue reading the main story}

\hypertarget{site-index}{%
\subsection{Site Index}\label{site-index}}

\hypertarget{site-information-navigation}{%
\subsection{Site Information
Navigation}\label{site-information-navigation}}

\begin{itemize}
\tightlist
\item
  \href{https://help.nytimes3xbfgragh.onion/hc/en-us/articles/115014792127-Copyright-notice}{©~2020~The
  New York Times Company}
\end{itemize}

\begin{itemize}
\tightlist
\item
  \href{https://www.nytco.com/}{NYTCo}
\item
  \href{https://help.nytimes3xbfgragh.onion/hc/en-us/articles/115015385887-Contact-Us}{Contact
  Us}
\item
  \href{https://www.nytco.com/careers/}{Work with us}
\item
  \href{https://nytmediakit.com/}{Advertise}
\item
  \href{http://www.tbrandstudio.com/}{T Brand Studio}
\item
  \href{https://www.nytimes3xbfgragh.onion/privacy/cookie-policy\#how-do-i-manage-trackers}{Your
  Ad Choices}
\item
  \href{https://www.nytimes3xbfgragh.onion/privacy}{Privacy}
\item
  \href{https://help.nytimes3xbfgragh.onion/hc/en-us/articles/115014893428-Terms-of-service}{Terms
  of Service}
\item
  \href{https://help.nytimes3xbfgragh.onion/hc/en-us/articles/115014893968-Terms-of-sale}{Terms
  of Sale}
\item
  \href{https://spiderbites.nytimes3xbfgragh.onion}{Site Map}
\item
  \href{https://help.nytimes3xbfgragh.onion/hc/en-us}{Help}
\item
  \href{https://www.nytimes3xbfgragh.onion/subscription?campaignId=37WXW}{Subscriptions}
\end{itemize}
