\href{/section/food}{Food}\textbar{}The Elements of Wok Hei, and How to
Capture Them at Home

\url{https://nyti.ms/3lTOchM}

\begin{itemize}
\item
\item
\item
\item
\item
\item
\end{itemize}

\href{https://www.nytimes3xbfgragh.onion/spotlight/at-home?action=click\&pgtype=Article\&state=default\&region=TOP_BANNER\&context=at_home_menu}{At
Home}

\begin{itemize}
\tightlist
\item
  \href{https://www.nytimes3xbfgragh.onion/2020/09/07/travel/route-66.html?action=click\&pgtype=Article\&state=default\&region=TOP_BANNER\&context=at_home_menu}{Cruise
  Along: Route 66}
\item
  \href{https://www.nytimes3xbfgragh.onion/2020/09/04/dining/sheet-pan-chicken.html?action=click\&pgtype=Article\&state=default\&region=TOP_BANNER\&context=at_home_menu}{Roast:
  Chicken With Plums}
\item
  \href{https://www.nytimes3xbfgragh.onion/2020/09/04/arts/television/dark-shadows-stream.html?action=click\&pgtype=Article\&state=default\&region=TOP_BANNER\&context=at_home_menu}{Watch:
  Dark Shadows}
\item
  \href{https://www.nytimes3xbfgragh.onion/interactive/2020/at-home/even-more-reporters-editors-diaries-lists-recommendations.html?action=click\&pgtype=Article\&state=default\&region=TOP_BANNER\&context=at_home_menu}{Explore:
  Reporters' Google Docs}
\end{itemize}

\includegraphics{https://static01.graylady3jvrrxbe.onion/images/2020/09/04/dining/04kenji7/merlin_176535075_9d5b5e5b-970f-4c24-9899-fe8751d6f3a5-articleLarge.jpg?quality=75\&auto=webp\&disable=upscale}

Sections

\protect\hyperlink{site-content}{Skip to
content}\protect\hyperlink{site-index}{Skip to site index}

\hypertarget{the-elements-of-wok-hei-and-how-to-capture-them-at-home}{%
\section{The Elements of Wok Hei, and How to Capture Them at
Home}\label{the-elements-of-wok-hei-and-how-to-capture-them-at-home}}

The elusive smoky flavors and aromas of stir-fry can be achieved in your
home kitchen. J. Kenji López-Alt shows you how.

One of the secrets to achieving wok hei in a home kitchen is a
blowtorch.Credit...David Malosh for The New York Times. Food Stylist:
Simon Andrews.

Supported by

\protect\hyperlink{after-sponsor}{Continue reading the main story}

By \href{https://www.nytimes3xbfgragh.onion/by/j-kenji-lopez-alt}{J.
Kenji López-Alt}

\begin{itemize}
\item
  Published Sept. 4, 2020Updated Sept. 5, 2020
\item
  \begin{itemize}
  \item
  \item
  \item
  \item
  \item
  \item
  \end{itemize}
\end{itemize}

``I like the food here,'' my dad would unfailingly say to me as he
pulled open the aluminum-framed, oil-smudged glass door at
\href{https://www.nytimes3xbfgragh.onion/2002/10/09/dining/25-and-under-cantonese-old-timer-changes-boroughs-not-bean-sauce.html}{Sun
Lok Kee}, a Mott Street stalwart that served beef chow fun and other
Cantonese classics at any hour of the day until it burned down in 2002.
``It has that nice smoky flavor.'' My family moved to New York in the
early '80s, when I was 4 years old, and those stir-fries from Sun Lok
Kee, with their savory char and smoky aroma, are among my first and
fondest taste memories.

Wok hei is the Cantonese name for that aroma (literally ``wok energy''
or ``wok breath''). My dad has always been a wok hei fiend, first
scouring the streets of Chinatown and later the suburbs of Boston for
smoky clams in black bean sauce, fire-kissed stir-fried greens, beef
chow fun that almost tastes grilled, or noodles that are singed just
right.

As a professional cook and recipe developer, I've spent a good 15 years
attempting to identify the elements that go into creating wok hei, so
that I might capture that flavor in a basic home kitchen. Last year, as
I researched and tested for a book on wok cooking, I finally got there.
True wok hei flavor right in my own kitchen.

Of course, some folks would say wok hei is incongruous with home
cooking. ``That smoky flavor'' is the kind of matter-of-fact
flourish-free description my dad excels at, but it's by no means the
agreed-upon definition. In her 2004 book, ``The Breath of a Wok,'' Grace
Young identifies wok hei as ``when a wok breathes energy into a
stir-fry, giving foods a unique concentrated flavor and aroma.'' In
``The Chinese Kitchen,'' Eileen Yin-Fey Lo says it's when ``the proper
amount of fire is made to curl up around the bowl of the wok to cook
foods precisely to that point of optimum flavor.''

\includegraphics{https://static01.graylady3jvrrxbe.onion/images/2020/09/09/dining/04kenji5/04kenji5-articleLarge-v2.jpg?quality=75\&auto=webp\&disable=upscale}

When I asked my friends Steph Li and Chris Thomas, the couple in
Shenzhen, China, behind the popular YouTube channel
\href{https://www.youtube.com/channel/UC54SLBnD5k5U3Q6N__UjbAw}{Chinese
Cooking Demystified}, their response jumped between descriptive and
evocative. ``Wok hei is this ethereal thing,'' Steph said. It's ``that
taste of the first bite of a hot restaurant stir-fry. It's got that
taste of the restaurant oil, the slightly deeper restaurant browning,
the heavier restaurant seasoning.

``Seeing home cooks outside of China being obsessed about wok hei has
always been kind of bewildering to me,'' she added.

She has a point. Most folks in China don't have restaurant-style
equipment at home, and even the concept of wok hei is not widely known
outside of the Cantonese regions of Southeastern China. But it is
perhaps because most Chinese food in America has its earliest roots in
Cantonese cuisine that American diners so strongly associate good
Chinese food with that flavor. (According to Andrew Smith's ``Eating
History,'' there were five Chinese restaurants in San Francisco by 1850,
started by Cantonese immigrants who arrived during the Gold Rush.)

But what \emph{is} the flavor? Where does it come from? What's so
different about cooking in a restaurant? In my own testing, I've managed
to narrow it down to a few key elements.

Most of these traits are intrinsic to woks, particularly those made of
carbon steel, a material that, like cast iron, can be seasoned to a
jet-black, nonstick coating and unlike cast iron, can be cast or
hammered thin and light enough to make tossing food a possibility.

When I was a test cook at Cook's Illustrated, I conducted a number of
blind taste tests, stir-frying noodles, beef and vegetables in
Western-style skillets (the magazine's recommended method at the time)
side by side with a nonstick wok, and my own well-seasoned carbon steel
wok. The carbon steel wok unanimously won those taste tests, producing
flavors that tasters described as ``grilled'' or ``caramelized.''

As Lan Lam has written in a
\href{https://www.cooksillustrated.com/articles/2189-the-science-of-stir-frying-in-a-wok}{more
recent article from Cook's Illustrated}, this has to do with the
chemical interactions between the food and the layers of polymerized
oils on the surface of a seasoned wok. It's also tied to the unique
action of stir-frying in a wok. ``Modernist Cuisine'' describes how when
a morsel is tossed up through the heavy cloud of steam that forms above
a hot wok, that steam condenses on the surface of the food, a process
that ``deposits formidable amounts of latent energy that rapidly heats
the food.'' It then drops back down onto the hot surface of the wok
where that surface moisture is re-vaporized, and the cycle repeats. A
wok allows you to constantly toss food through its own vapors, speeding
up its cooking, which concentrates flavor and promotes the development
of new flavor compounds through the Maillard reaction better than a flat
skillet can.

The wide rim of a wok factors into another key element of wok hei. In
her 2010 book, ``Stir-Frying to Sky's Edge,'' Ms. Young emphasizes the
importance of adding soy sauce and other liquids around the perimeter of
the wok, so as not to decrease the temperature of the searing zone in
the center, which can cause meat and vegetables to steam rather than
sizzle. This advice is common among Chinese chefs and cookbook authors.

But as I watched Sichuan chef Wang Gang splash soy sauce around the
perimeter of a wok full of
\href{https://www.youtube.com/watch?v=EvE5cYNXufY}{home-style egg fried
rice on his YouTube channel}, I noticed something: how rapidly it
sizzles and sputters. I was reminded of a Mexican cooking technique I
learned from the late chef
\href{https://www.nytimes3xbfgragh.onion/2008/04/23/dining/23mini.html}{David
Sterling} at his home in the Yucatán: As he tipped fresh salsa into a
ripping-hot saucepan, it superheated in an instantaneous steamy sputter,
giving it a richer color and smoky undertones. Could this concept of a
seared sauce also be a factor in wok hei flavor?

To test this, I made two identical batches of lo mein, changing only the
manner in which I finished them. For the first, I finished by splashing
two tablespoons of soy sauce around the perimeter of the wok, while
simultaneously splashing two tablespoons of water into the center of the
wok. For the second, I swapped the water and soy sauce. (Adding water to
the test ensured that both batches would experience the same cool-down
effect of liquid added directly to the center, while only one would
develop seared soy sauce flavors.)

The difference was stark. Adding soy sauce to the center of the wok left
the noodles with a raw soy sauce flavor, while drizzling it around the
hot edges of the wok created smoky flavors reminiscent of grilled meat.
I've since found that adding a small splash of oil to the perimeter of
the wok before adding the soy sauce will prevent the soy sauce from
ending up caked onto the side of the wok.

Given that it's the shape of a wok that allows for tossing through steam
and searing sauces, is there any hope for capturing wok hei in a Western
skillet? Fortunately, yes there is, and it has to do with the final,
most important contributor to wok hei: burnt oil.

Watch a Cantonese restaurant chef in action and you'll see that, just as
the author Ms. Lo described, flames will lick up the back of the wok,
sometimes even spreading down into the wok itself. This happens because
as food gets tossed through the hot zone behind the wok, tiny droplets
of aerosolized oil will ignite and flare up. That singed oil then leaves
small, sooty deposits on the food as it gets tossed through the smoke.
It's this flavor --- the same flavor that develops as a hamburger drips
fat onto red hot coals below it --- that I most strongly associate with
wok hei.

The problem is that Cantonese restaurant wok ranges, which can output
200,000 B.T.U.s per hour or more, are an order of magnitude more
powerful than even the most powerful home burner. It's that massive jet
of flame that makes igniting vaporized fat possible, but I've found a
reasonable workaround.

If I can't bring my food to the flame, why not bring the flame to my
food? A camping-style fuel tank, along with a brazing head (such as the
Iwatani Pro butane torch head or
\href{https://www.bernzomatic.com/Products/Hand-Torches/Instant-On-Off/TS8000}{Bernzomatic
TS8000} propane torch head) that I point directly at the food inside a
wok for a few brief moments as I toss, can lend that vaporized oil
flavor.

In practice, this technique, which my colleagues at Serious Eats
independently arrived at and cleverly coined ``torch hei,'' is tricky.
Even with two hands, stir-frying takes practice. To simplify things,
I've found that transferring the stir-fried food to a rimmed baking
sheet, spreading it in a single layer, and giving it a few leisurely
passes with the torch before returning it to the wok for final saucing
and garnishing is a simple and effective workaround.

Image

A quick pass of the torch over the ingredients before they are returned
to the wok is an effective wok hei workaround.Credit...David Malosh for
The New York Times. Food Stylist: Simon Andrews.

The torch hei method combined with a skillet is good enough to capture
some of the magic of a great Cantonese restaurant. When I add a carbon
steel wok and some seared soy sauce to the mix, I can almost hear my dad
pushing open my kitchen door as he says to me, ``I like the food here.''

Recipes:
\textbf{\href{https://cooking.nytimes3xbfgragh.onion/recipes/1021379-smoky-lo-mein-with-shiitake-and-vegetables}{Smoky
Lo Mein With Shiitake and Vegetables}} \textbar{}
\textbf{\href{https://cooking.nytimes3xbfgragh.onion/recipes/1021378-smoky-stir-fried-greens}{Smoky
Stir-Fried Greens}}

\emph{{[}Watch J. Kenji López-Alt make the}
\href{https://youtu.be/iac_idcz6XE}{\emph{smoky lo mein}} \emph{and the}
\href{https://youtu.be/hcGRskPjQcU}{\emph{stir-fried greens}}\emph{.{]}}

\emph{Follow} \href{https://twitter.com/nytfood}{\emph{NYT Food on
Twitter}} \emph{and}
\href{https://www.instagram.com/nytcooking/}{\emph{NYT Cooking on
Instagram}}\emph{,}
\href{https://www.facebookcorewwwi.onion/nytcooking/}{\emph{Facebook}}\emph{,}
\href{https://www.youtube.com/nytcooking}{\emph{YouTube}} \emph{and}
\href{https://www.pinterest.com/nytcooking/}{\emph{Pinterest}}\emph{.}
\href{https://www.nytimes3xbfgragh.onion/newsletters/cooking}{\emph{Get
regular updates from NYT Cooking, with recipe suggestions, cooking tips
and shopping advice}}\emph{.}

Advertisement

\protect\hyperlink{after-bottom}{Continue reading the main story}

\hypertarget{site-index}{%
\subsection{Site Index}\label{site-index}}

\hypertarget{site-information-navigation}{%
\subsection{Site Information
Navigation}\label{site-information-navigation}}

\begin{itemize}
\tightlist
\item
  \href{https://help.nytimes3xbfgragh.onion/hc/en-us/articles/115014792127-Copyright-notice}{©~2020~The
  New York Times Company}
\end{itemize}

\begin{itemize}
\tightlist
\item
  \href{https://www.nytco.com/}{NYTCo}
\item
  \href{https://help.nytimes3xbfgragh.onion/hc/en-us/articles/115015385887-Contact-Us}{Contact
  Us}
\item
  \href{https://www.nytco.com/careers/}{Work with us}
\item
  \href{https://nytmediakit.com/}{Advertise}
\item
  \href{http://www.tbrandstudio.com/}{T Brand Studio}
\item
  \href{https://www.nytimes3xbfgragh.onion/privacy/cookie-policy\#how-do-i-manage-trackers}{Your
  Ad Choices}
\item
  \href{https://www.nytimes3xbfgragh.onion/privacy}{Privacy}
\item
  \href{https://help.nytimes3xbfgragh.onion/hc/en-us/articles/115014893428-Terms-of-service}{Terms
  of Service}
\item
  \href{https://help.nytimes3xbfgragh.onion/hc/en-us/articles/115014893968-Terms-of-sale}{Terms
  of Sale}
\item
  \href{https://spiderbites.nytimes3xbfgragh.onion}{Site Map}
\item
  \href{https://help.nytimes3xbfgragh.onion/hc/en-us}{Help}
\item
  \href{https://www.nytimes3xbfgragh.onion/subscription?campaignId=37WXW}{Subscriptions}
\end{itemize}
