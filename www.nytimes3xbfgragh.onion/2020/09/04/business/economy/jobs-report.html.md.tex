Sections

SEARCH

\protect\hyperlink{site-content}{Skip to
content}\protect\hyperlink{site-index}{Skip to site index}

\href{https://www.nytimes3xbfgragh.onion/section/business/economy}{Economy}

\href{https://myaccount.nytimes3xbfgragh.onion/auth/login?response_type=cookie\&client_id=vi}{}

\href{https://www.nytimes3xbfgragh.onion/section/todayspaper}{Today's
Paper}

\href{/section/business/economy}{Economy}\textbar{}Job Growth Slackens,
Signaling Vulnerability of Recovery

\url{https://nyti.ms/31Xjc8K}

\begin{itemize}
\item
\item
\item
\item
\item
\end{itemize}

\hypertarget{the-coronavirus-outbreak}{%
\subsubsection{\texorpdfstring{\href{https://www.nytimes3xbfgragh.onion/news-event/coronavirus?name=styln-coronavirus-markets\&region=TOP_BANNER\&block=storyline_menu_recirc\&action=click\&pgtype=Article\&impression_id=3ef949b0-f4c9-11ea-8e77-775acd3382e0\&variant=undefined}{The
Coronavirus
Outbreak}}{The Coronavirus Outbreak}}\label{the-coronavirus-outbreak}}

\begin{itemize}
\tightlist
\item
  live\href{https://www.nytimes3xbfgragh.onion/2020/09/11/world/covid-19-coronavirus.html?name=styln-coronavirus-markets\&region=TOP_BANNER\&block=storyline_menu_recirc\&action=click\&pgtype=Article\&impression_id=3ef970c0-f4c9-11ea-8e77-775acd3382e0\&variant=undefined}{Latest
  Updates}
\item
  \href{https://www.nytimes3xbfgragh.onion/interactive/2020/us/coronavirus-us-cases.html?name=styln-coronavirus-markets\&region=TOP_BANNER\&block=storyline_menu_recirc\&action=click\&pgtype=Article\&impression_id=3ef970c1-f4c9-11ea-8e77-775acd3382e0\&variant=undefined}{Maps
  and Cases}
\item
  \href{https://www.nytimes3xbfgragh.onion/interactive/2020/science/coronavirus-vaccine-tracker.html?name=styln-coronavirus-markets\&region=TOP_BANNER\&block=storyline_menu_recirc\&action=click\&pgtype=Article\&impression_id=3ef970c2-f4c9-11ea-8e77-775acd3382e0\&variant=undefined}{Vaccine
  Tracker}
\item
  \href{https://www.nytimes3xbfgragh.onion/2020/09/10/us/politics/fda-coronavirus-vaccine.html?name=styln-coronavirus-markets\&region=TOP_BANNER\&block=storyline_menu_recirc\&action=click\&pgtype=Article\&impression_id=3ef970c3-f4c9-11ea-8e77-775acd3382e0\&variant=undefined}{F.D.A.
  Regulators' Self-Defense}
\item
  \href{https://www.nytimes3xbfgragh.onion/2020/09/09/upshot/coronavirus-surprise-test-fees.html?name=styln-coronavirus-markets\&region=TOP_BANNER\&block=storyline_menu_recirc\&action=click\&pgtype=Article\&impression_id=3ef970c4-f4c9-11ea-8e77-775acd3382e0\&variant=undefined}{Surprise
  Test Fees}
\end{itemize}

Advertisement

\protect\hyperlink{after-top}{Continue reading the main story}

Supported by

\protect\hyperlink{after-sponsor}{Continue reading the main story}

\hypertarget{job-growth-slackens-signaling-vulnerability-of-recovery}{%
\section{Job Growth Slackens, Signaling Vulnerability of
Recovery}\label{job-growth-slackens-signaling-vulnerability-of-recovery}}

Unemployment fell to 8.4\% in August, but the gain of 1.4 million jobs
was the weakest in months. The end of federal aid programs is casting a
shadow.

\includegraphics{https://static01.graylady3jvrrxbe.onion/images/2020/09/04/business/04virus-jobs1a/merlin_176482278_45bc8977-f4ef-4d4c-bd9d-545433d740e2-articleLarge.jpg?quality=75\&auto=webp\&disable=upscale}

\href{https://www.nytimes3xbfgragh.onion/by/ben-casselman}{\includegraphics{https://static01.graylady3jvrrxbe.onion/images/2018/11/09/multimedia/author-ben-casselman/author-ben-casselman-thumbLarge.png}}

By \href{https://www.nytimes3xbfgragh.onion/by/ben-casselman}{Ben
Casselman}

\begin{itemize}
\item
  Sept. 4, 2020
\item
  \begin{itemize}
  \item
  \item
  \item
  \item
  \item
  \end{itemize}
\end{itemize}

Job growth slowed further last month, the latest sign that the economy's
spring momentum has faded --- and a warning that the recovery could go
into reverse this fall without further government support.

U.S. employers added 1.4 million jobs in August, the
\href{https://www.bls.gov/news.release/empsit.nr0.htm}{Labor Department
said Friday}, down from the gains in the three previous months. The
slowdown would have been more pronounced without the hiring of nearly a
quarter-million temporary census workers.

The report held some good news: The unemployment rate fell by more than
expected, to 8.4 percent. In April --- when joblessness was the highest
since the Great Depression --- forecasters at the
\href{https://www.cbo.gov/publication/56335}{Congressional Budget
Office} said unemployment would remain in the double digits well into
next year.

But an increasing number of people reported in the Labor Department's
August survey that they had lost their jobs permanently, rather than
being temporarily laid off or furloughed --- a sign that the crisis is
doing lasting damage.

``There's a fragility in the numbers,'' said Diane Swonk, chief
economist at the accounting firm Grant Thornton. ``There are cracks in
the underlying foundation.''

\hypertarget{jobs-remain-far-below-pre-pandemic-levels}{%
\subsubsection{Jobs remain far below pre-pandemic
levels}\label{jobs-remain-far-below-pre-pandemic-levels}}

\hypertarget{cumulative-change-in-all-jobs-since-august-2016}{%
\paragraph{Cumulative change in all jobs since August
2016}\label{cumulative-change-in-all-jobs-since-august-2016}}

By Ella Koeze·Data is seasonally adjusted.·Source: Bureau of Labor
Statistics

Those cracks are appearing as trillions of dollars in federal spending,
which helped sustain many households and businesses early in the
pandemic, are drying up.

A \$600 weekly federal supplement to unemployment benefits expired in
July; a
\href{https://www.nytimes3xbfgragh.onion/article/stimulus-unemployment-payment-benefit.html?action=click\&module=RelatedLinks\&pgtype=Article}{\$300-a-week
replacement}, announced by President Trump last month, has been slow to
kick in and will last for only a few weeks. The government's marquee
business relief effort, the Paycheck Protection Program, ended in
August.

The August jobs data was collected early in the month, and might not
reflect the full impact of the loss of benefits, economists warn. That
calendar quirk could have political ramifications, easing pressure on
Congress to agree on a new round of emergency spending.

``If the labor market data continue to hold, if we don't see a big
destruction to consumer spending on the back of the loss of the
unemployment benefits, that reduces the sense of urgency that something
needs to be done prior to the election,'' said Michelle Meyer, head of
U.S. economics for Bank of America.

\hypertarget{latest-updates-the-coronavirus-outbreak-and-the-economy}{%
\section{\texorpdfstring{\href{https://www.nytimes3xbfgragh.onion/live/2020/09/11/business/stock-market-today-coronavirus?action=click\&pgtype=Article\&state=default\&region=MAIN_CONTENT_1\&context=storylines_live_updates}{Latest
Updates: The Coronavirus Outbreak and the
Economy}}{Latest Updates: The Coronavirus Outbreak and the Economy}}\label{latest-updates-the-coronavirus-outbreak-and-the-economy}}

\href{https://www.nytimes3xbfgragh.onion/live/2020/09/11/business/stock-market-today-coronavirus?action=click\&pgtype=Article\&state=default\&region=MAIN_CONTENT_1\&context=storylines_live_updates\#the-nyse-may-move-its-data-center-out-of-new-jersey-in-response-to-a-proposed-tax}{11h
ago}

\href{https://www.nytimes3xbfgragh.onion/live/2020/09/11/business/stock-market-today-coronavirus?action=click\&pgtype=Article\&state=default\&region=MAIN_CONTENT_1\&context=storylines_live_updates\#the-nyse-may-move-its-data-center-out-of-new-jersey-in-response-to-a-proposed-tax}{The
N.Y.S.E. may move its data center out of New Jersey in response to a
proposed tax.}

\href{https://www.nytimes3xbfgragh.onion/live/2020/09/11/business/stock-market-today-coronavirus?action=click\&pgtype=Article\&state=default\&region=MAIN_CONTENT_1\&context=storylines_live_updates\#the-federal-budget-deficit-hit-3-trillion-as-of-august}{13h
ago}

\href{https://www.nytimes3xbfgragh.onion/live/2020/09/11/business/stock-market-today-coronavirus?action=click\&pgtype=Article\&state=default\&region=MAIN_CONTENT_1\&context=storylines_live_updates\#the-federal-budget-deficit-hit-3-trillion-as-of-august}{The
federal budget deficit hit \$3 trillion as of August.}

\href{https://www.nytimes3xbfgragh.onion/live/2020/09/11/business/stock-market-today-coronavirus?action=click\&pgtype=Article\&state=default\&region=MAIN_CONTENT_1\&context=storylines_live_updates\#warner-bros-pushes-the-release-of-wonder-woman-1984-to-christmas}{14h
ago}

\href{https://www.nytimes3xbfgragh.onion/live/2020/09/11/business/stock-market-today-coronavirus?action=click\&pgtype=Article\&state=default\&region=MAIN_CONTENT_1\&context=storylines_live_updates\#warner-bros-pushes-the-release-of-wonder-woman-1984-to-christmas}{Warner
Bros. pushes the release of `Wonder Woman 1984' to Christmas.}

\href{https://www.nytimes3xbfgragh.onion/live/2020/09/11/business/stock-market-today-coronavirus?action=click\&pgtype=Article\&state=default\&region=MAIN_CONTENT_1\&context=storylines_live_updates}{See
more updates}

More live coverage:
\href{https://www.nytimes3xbfgragh.onion/2020/09/11/world/covid-19-coronavirus.html?action=click\&pgtype=Article\&state=default\&region=MAIN_CONTENT_1\&context=storylines_live_updates}{Global}

Economists warn that would set the stage for a big drop in spending in
the fall, leading to more job losses and a
\href{https://www.nytimes3xbfgragh.onion/2020/09/01/business/economy/small-businesses-coronavirus.html}{wave
of small-business failures}. Corporations including American Airlines
have announced they are
\href{https://www.nytimes3xbfgragh.onion/2020/08/25/business/american-airline-furlough-19000.html}{laying
off more workers} or, as in the case of the department store stalwart
Lord \& Taylor,
\href{https://www.nytimes3xbfgragh.onion/aponline/2020/08/28/business/ap-lord-taylor-going-out-of-business.html}{going
out of business}.

Applications for unemployment benefits
\href{https://www.nytimes3xbfgragh.onion/2020/09/03/business/economy/unemployment-claims.html}{rose
last week}, and data from \href{https://joinhomebase.com/data}{Homebase}
--- which provides time-management software to small businesses ---
shows that the number of people working has declined since early August.
Economists say those figures suggest that job growth could turn flat or
negative in the fall.

``Federal spending was meant to be a bridge,'' said Beth Ann Bovino,
chief U.S. economist for S\&P Global. ``Well, it looks like the ravine
has widened and the bridge is halfway built, so there are a lot of
people stranded.''

\hypertarget{unemployment-rate}{%
\subsubsection{Unemployment rate}\label{unemployment-rate}}

By Ella Koeze·Unemployment rates are seasonally adjusted.·Source: Bureau
of Labor Statistics

All told, less than half of the 22 million jobs lost early in the
pandemic have been recovered. But the unemployment rate has fallen much
faster than most forecasters expected, from 10.2 percent in July and
14.7 percent in April. And the labor force grew in August, an indication
that jobless workers are not yet giving up their searches as many did
during the last recession a decade ago. Some sectors that were dealt a
blow by the pandemic, such as the retail industry, continued to post
strong job gains.

Mr. Trump cheered the report
\href{https://twitter.com/realDonaldTrump/status/1301868845678952448}{on
Twitter}. ``Great Jobs Numbers,'' he declared, highlighting the
unemployment rate's decline into single digits as ``faster and deeper
than thought possible.''

His Democratic opponent, former Vice President Joseph R. Biden Jr., said
in Wilmington, Del., that ``any job added back is positive.'' But when
working people are asked ``how do they feel about the economy coming
back,'' he said, ``you'll find they don't feel it.''

Economists said the slowdown was a worrying sign that the low-hanging
fruit of the recovery --- the rehiring of millions of furloughed
restaurant, hotel and entertainment workers --- could be largely gone.

Just 174,000 jobs were added last month in leisure and hospitality, a
disappointing gain for an industry that lost more than eight million to
the pandemic and has recovered only half. And as companies reopen, many
are discovering that with demand still weak, they don't need or can't
afford as many workers as before the pandemic.

Marcus Hotels, which operates more than a dozen hotels, mostly in the
Midwest, began reopening its properties in June and has brought back
about 60 percent of the nearly 4,000 employees it had before the
pandemic. But in recent weeks, it has begun permanently laying off many
of the employees who remained on furlough.

``We held out as long as we could, waiting to see what was going to
happen,'' said Michael Evans, the president of Marcus, who added that it
had paid benefits for employees as long as they were furloughed.

Mr. Evans said that he thought the hotel business would bounce back
eventually, but that it could take years to return to its previous
level. And even when it does, he said, Marcus will probably not need as
many workers.

``As we've planned for the reopenings, we re-evaluated our entire
business model,'' he said. ``If business were back to normal right now,
we would still operate more efficiently.''

\hypertarget{some-industries-are-approaching-pre-pandemic-employment-but-leisure-and-hospitality-jobs-are-lagging-far-behind}{%
\subsubsection{Some industries are approaching pre-pandemic employment,
but leisure and hospitality jobs are lagging far
behind}\label{some-industries-are-approaching-pre-pandemic-employment-but-leisure-and-hospitality-jobs-are-lagging-far-behind}}

\hypertarget{cumulative-change-in-jobs-since-august-2016-by-industry}{%
\paragraph{Cumulative change in jobs since August 2016, by
industry}\label{cumulative-change-in-jobs-since-august-2016-by-industry}}

By Ella Koeze·Data is seasonally adjusted.·Source: Bureau of Labor
Statistics

Decisions like the ones at Marcus mean that workers like Kara Hanley
could have a hard time getting back to work.

Ms. Hanley, 23, was furloughed in March from her job as a room-service
supervisor at a Marriott hotel in Orlando, Fla. At first, she expected
the furlough to last a few weeks. Then, as weeks passed and she didn't
get a call, she thought she might be waiting until late summer, or
perhaps Christmas.

When the phone rang on Sept. 1, she assumed that Marriott was calling to
tell her it was finally time to return to work. Instead, she learned she
was being laid off, effective Sept. 18.

``I really didn't think that I was going to get let go,'' she said. ``I
just kept thinking, when they need me, they're going to call me back.''

The emergency spending programs that Congress passed last spring were
meant to help prevent that kind of lasting economic harm. Forgivable
loans to small businesses were intended to help avert bankruptcies and
layoffs. Enhanced unemployment benefits were intended not just to
prevent hunger and homelessness among the jobless, but also to minimize
the cascade of economic damage that would follow.

Those programs, despite a rocky start, were largely successful. A wave
of foreclosures has yet to occur. Consumer spending rebounded strongly
in May and June, and companies began recalling furloughed workers. But
without a new round of aid, much of that progress could be lost, with
lasting economic consequences.

``I am more concerned about where the economy is now than I was in
April,'' said Martha Gimbel, an economist and labor market expert at
Schmidt Futures, a philanthropic initiative. ``In April, it was fixable.
We're just letting the scars build up now.''

H. Brandon Williams was supposed to spend this month celebrating the
third anniversary of \href{https://www.wearefishscale.com/}{FishScale},
his restaurant in Washington, D.C. Instead, he is trying to keep his
business afloat.

\includegraphics{https://static01.graylady3jvrrxbe.onion/images/2020/09/04/business/04virus-jobs3/merlin_176482281_f1357167-f17a-4d10-9e79-35a246d3cafa-articleLarge.jpg?quality=75\&auto=webp\&disable=upscale}

The restaurant, which specializes in burgers made from sustainably
caught wild fish, survived the initial blow from the pandemic, which
wiped out many other Black-owned small businesses.

But the business isn't out of the woods. Nearby Howard University
\href{https://home.howard.edu/reopen}{recently announced} that it would
shift its undergraduate classes online for the fall semester and close
its dormitories. FishScale also relied on revenue from sales at a
seasonal farmers' market, which didn't open this year.

Now, with the federal unemployment supplement and other aid programs
gone, Mr. Williams, 39, notices customers pinching their pennies ---
which is forcing him to do the same. He has cut back to three employees
from six and has won rent concessions from his landlord that he said
should get him through the end of the year. He isn't sure what will
happen after that.

``We're still at that area where we could go either way,'' Mr. Williams
said.

Julia Pollak, a labor economist for the employment site ZipRecruiter,
said many businesses were facing similar decisions heading into the
winter season, which is a challenge for many small businesses in the
best of times.

``There are many companies that after a summer of gathering way too few
acorns are going into a hibernation that may not sustain them,'' she
said.

Widespread business failures, Ms. Pollak said, ``could have a cascading
effect on those local economies.'' That is especially true of Black
neighborhoods that often struggle to draw investment from large
corporations.

Mr. Williams said he wanted to stay in business not only for himself but
also for his community. ``There are a lot of people who couldn't get a
job if it weren't for Black-owned businesses,'' he said. ``I want young
boys and girls to look and see somebody doing something that's out of
the box.''

Gillian Friedman contributed reporting.

Advertisement

\protect\hyperlink{after-bottom}{Continue reading the main story}

\hypertarget{site-index}{%
\subsection{Site Index}\label{site-index}}

\hypertarget{site-information-navigation}{%
\subsection{Site Information
Navigation}\label{site-information-navigation}}

\begin{itemize}
\tightlist
\item
  \href{https://help.nytimes3xbfgragh.onion/hc/en-us/articles/115014792127-Copyright-notice}{©~2020~The
  New York Times Company}
\end{itemize}

\begin{itemize}
\tightlist
\item
  \href{https://www.nytco.com/}{NYTCo}
\item
  \href{https://help.nytimes3xbfgragh.onion/hc/en-us/articles/115015385887-Contact-Us}{Contact
  Us}
\item
  \href{https://www.nytco.com/careers/}{Work with us}
\item
  \href{https://nytmediakit.com/}{Advertise}
\item
  \href{http://www.tbrandstudio.com/}{T Brand Studio}
\item
  \href{https://www.nytimes3xbfgragh.onion/privacy/cookie-policy\#how-do-i-manage-trackers}{Your
  Ad Choices}
\item
  \href{https://www.nytimes3xbfgragh.onion/privacy}{Privacy}
\item
  \href{https://help.nytimes3xbfgragh.onion/hc/en-us/articles/115014893428-Terms-of-service}{Terms
  of Service}
\item
  \href{https://help.nytimes3xbfgragh.onion/hc/en-us/articles/115014893968-Terms-of-sale}{Terms
  of Sale}
\item
  \href{https://spiderbites.nytimes3xbfgragh.onion}{Site Map}
\item
  \href{https://help.nytimes3xbfgragh.onion/hc/en-us}{Help}
\item
  \href{https://www.nytimes3xbfgragh.onion/subscription?campaignId=37WXW}{Subscriptions}
\end{itemize}
