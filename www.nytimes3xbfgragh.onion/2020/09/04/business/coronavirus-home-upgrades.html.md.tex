Sections

SEARCH

\protect\hyperlink{site-content}{Skip to
content}\protect\hyperlink{site-index}{Skip to site index}

\href{https://www.nytimes3xbfgragh.onion/section/business}{Business}

\href{https://myaccount.nytimes3xbfgragh.onion/auth/login?response_type=cookie\&client_id=vi}{}

\href{https://www.nytimes3xbfgragh.onion/section/todayspaper}{Today's
Paper}

\href{/section/business}{Business}\textbar{}They're Stuck at Home, So
They're Making Home a Sanctuary

\url{https://nyti.ms/35brGem}

\begin{itemize}
\item
\item
\item
\item
\item
\end{itemize}

\href{https://www.nytimes3xbfgragh.onion/spotlight/at-home?action=click\&pgtype=Article\&state=default\&region=TOP_BANNER\&context=at_home_menu}{At
Home}

\begin{itemize}
\tightlist
\item
  \href{https://www.nytimes3xbfgragh.onion/2020/09/07/travel/route-66.html?action=click\&pgtype=Article\&state=default\&region=TOP_BANNER\&context=at_home_menu}{Cruise
  Along: Route 66}
\item
  \href{https://www.nytimes3xbfgragh.onion/2020/09/04/dining/sheet-pan-chicken.html?action=click\&pgtype=Article\&state=default\&region=TOP_BANNER\&context=at_home_menu}{Roast:
  Chicken With Plums}
\item
  \href{https://www.nytimes3xbfgragh.onion/2020/09/04/arts/television/dark-shadows-stream.html?action=click\&pgtype=Article\&state=default\&region=TOP_BANNER\&context=at_home_menu}{Watch:
  Dark Shadows}
\item
  \href{https://www.nytimes3xbfgragh.onion/interactive/2020/at-home/even-more-reporters-editors-diaries-lists-recommendations.html?action=click\&pgtype=Article\&state=default\&region=TOP_BANNER\&context=at_home_menu}{Explore:
  Reporters' Google Docs}
\end{itemize}

Advertisement

\protect\hyperlink{after-top}{Continue reading the main story}

Supported by

\protect\hyperlink{after-sponsor}{Continue reading the main story}

\hypertarget{theyre-stuck-at-home-so-theyre-making-home-a-sanctuary}{%
\section{They're Stuck at Home, So They're Making Home a
Sanctuary}\label{theyre-stuck-at-home-so-theyre-making-home-a-sanctuary}}

With few places to go or reasons to spend, those lucky enough to be
employed remotely are upgrading their surroundings. The impulse comes
with guilt.

\includegraphics{https://static01.graylady3jvrrxbe.onion/images/2020/09/03/multimedia/00xp-virus-nesting-pix1/00xp-virus-nesting-pix1-articleLarge.jpg?quality=75\&auto=webp\&disable=upscale}

By \href{https://www.nytimes3xbfgragh.onion/by/maria-cramer}{Maria
Cramer} and
\href{https://www.nytimes3xbfgragh.onion/by/aimee-ortiz}{Aimee Ortiz}

\begin{itemize}
\item
  Sept. 4, 2020
\item
  \begin{itemize}
  \item
  \item
  \item
  \item
  \item
  \end{itemize}
\end{itemize}

In March, the threadbare couch in the living room was merely an
annoyance, but Megan Barney, a book publicist in Cambridge, Mass., was
not ready to spend hundreds of dollars to replace it.

By August, after six months of working from home, Ms. Barney could not
stand looking at what had become an unsightly beige monstrosity. It had
to go.

Ms. Barney, 26, and her husband, a research scientist, ordered a blue
sectional, which arrived last month, joining a collection of other
household amenities that the couple has splurged on since the pandemic
began.

Cuisinart pots and Crate \& Barrel pans. A cocktail shaker and martini
glasses. New dishes.

``If I'm going to be here,'' Ms. Barney said, ``I want it to be as
comfortable as possible and as calming as possible.''

With limited restaurant options, even fewer travel options and little
reason to spend money on nice clothes for the office, those fortunate
enough to have kept their jobs during the pandemic
\href{https://www.forbes.com/sites/elizabethfazzare/2020/06/27/retailers-report-that-pandemic-shoppers-are-buying-more-home-and-design-products/\#673de1824436}{are
using their disposable income to upgrade their pandemic headquarters}.

They are buying
\href{https://www.jpost.com/special-content/bamboo-material-could-improve-hygiene-during-coronavirus-636461}{bamboo-linen
sheets},
\href{https://www.usatoday.com/story/tech/2020/07/20/coronavirus-effect-big-tvs-have-helped-some-homes-navigate-pandemic/5432582002/}{big-screen
TVs},
\href{https://www.nytimes3xbfgragh.onion/wirecutter/reviews/best-blender/}{high-end
blenders} and new
\href{https://www.kmbc.com/article/with-more-people-home-during-pandemic-furniture-stores-are-busy/33383497}{furniture}.

And some of them feel guilty being able to buy freely when so many other
people are unemployed. Shouldn't they be giving money to charity, or
\href{https://www.washingtonpost.com/lifestyle/home/the-big-pandemic-clean-out-clearing-the-junk-out-of-your-home-while-stuck-there/2020/08/04/230d71d2-c868-11ea-a99f-3bbdffb1af38_story.html}{decluttering
their lives}?

``I definitely feel weird,'' said Ms. Barney, who has donated and also
tried to help friends who were laid off. ``I also just feel weird
generally about having a job because I don't necessarily feel special or
better than any of my friends who have lost their jobs.''

But experts warn against being too hard on yourself in a time of great
anxiety.

It may feel indulgent to splurge on your household now, but it's
perfectly normal, even healthy, said Asia Wong, a social worker and
director of counseling and health services at Loyola University New
Orleans.

``Think of this as an amplified nesting response,'' she said. ``Yes,
we're looking for ways to make home feel more entertaining and vibrant.
But we're also looking for ways to feel safer and more cozy.''

\hypertarget{keeping-up-with-the-joneses-on-zoom}{%
\subsection{Keeping up with the Joneses on
Zoom.}\label{keeping-up-with-the-joneses-on-zoom}}

But let's not kid ourselves. We're also looking to impress others,
experts said.

The
\href{https://www.idealhome.co.uk/news/how-instagram-makes-us-feel-about-homes-208263}{pressure
to have a home pretty enough for Instagram} was already intense before
the pandemic. As video conferences and virtual learning have opened up
people's living spaces to more outside scrutiny, that pressure has only
grown, said Annetta Grant, an assistant professor of marketing at
Bucknell University in Pennsylvania.

``Colleagues who may never have been to your home, see the inside of
your home,'' Professor Grant said. ``Kids carry their classmates and
teacher through the house to show off a book or toy.''

What's more, we are seeing the inside of others' homes, which may be
bigger or better decorated and equipped than ours, increasing the desire
to make our home feel as luxurious, she added.

``That may prompt them to do the renovation they've been thinking about
doing just a little more quickly,'' Professor Grant said.

\includegraphics{https://static01.graylady3jvrrxbe.onion/images/2020/09/03/multimedia/00xp-virus-nesting-pix2/00xp-virus-nesting-pix2-articleLarge.jpg?quality=75\&auto=webp\&disable=upscale}

Reenat Sinay, a journalist in Manhattan, said she had felt pangs of envy
during Zoom chats with friends who live in cheaper parts of the country
and can afford large, airy houses or with colleagues living in
meticulously decorated apartments.

``There is one woman on my team who is engaged and has this gorgeous
apartment with this awesome balcony,'' Ms. Sinay, 32, said. ``She always
Zooms from there and I'm always like, `Damn.'''

Ms. Sinay, who described herself as thrifty, said the pandemic had
pushed her to indulge. She has bought a Dutch oven and roasting pans to
make better meals and glass-encased scented candles.

After her two roommates moved out in March, the apartment felt empty.
Ms. Sinay bought a plant, a
\href{https://www.thespruce.com/zz-zanzibar-gem-plant-profile-4796783}{Zanzibar
Gem} she named Margot-Anaïs. She said she had begun scouring the
internet for the ``perfect throw pillow'' that doesn't cost \$100.

``I feel like I've let myself buy things that I wouldn't,'' Ms. Sinay
said. ``I feel like: `OK, is this going to make me feel better? Is this
going to brighten my day when I'm sitting here by myself and lonely?
Probably.' So I should get it.''

\hypertarget{splurging-to-feel-normal-again}{%
\subsection{Splurging to feel normal
again.}\label{splurging-to-feel-normal-again}}

Meg Casey, 38, a lawyer in Nashville, said she and her husband, a
doctor, loved going to the movies before the pandemic.

To recreate the lost experience, they bought a movie projector with a
large white screen, then built a fire pit in the backyard, where her
family has watched ``Star Wars,'' ``The SpongeBob SquarePants Movie''
and ``lots of Scooby-Doo,'' she said.

``It almost feels like normal life,'' Ms. Casey said.

It is the kind of investment that many consumers have made in recent
months, buying
\href{https://www.nytimes3xbfgragh.onion/2020/08/04/style/outdoor-camping-gear-pools-backordered.html}{kayaks,
pools, outdoor patio heaters and trampolines} to liven up their
backyards and soften the blow of lost vacations.

Not everyone feels comfortable splurging. Louise Dunlap, 82, a retired
writing teacher in Oakland, Calif., said she had used pandemic savings
to donate to organizations seeking to return land to Indigenous people.

``I just don't think that buying things is going to stabilize our
world,'' Ms. Dunlap said.

She said she had indulged a little, however, buying kippers at the
grocery store and was considering buying a printer and a new chair for
her desk.

Ali Besharat, a marketing professor and co-director of the Consumer
Insights and Business Innovation Center at the University of Denver,
said many people were also saving money or paying down debt. In April,
consumers reported saving 33 percent of their income, up from an average
of 7 percent, he said.

But not everyone has that luxury, he said.

Unemployment filings
\href{https://www.nytimes3xbfgragh.onion/2020/08/27/business/economy/unemployment-claims.html}{remained
high in August}, job growth continued to slow down throughout the summer
and the
\href{https://www.nytimes3xbfgragh.onion/2020/08/07/business/economy/housing-economy-eviction-renters.html}{threat
of mass evictions} is expected to loom. Even lower-income people
fortunate enough to keep their jobs have been unable to save their
money, much less buy household luxuries, Professor Besharat said.

``They didn't notice a difference because pre-pandemic they were
paycheck to paycheck,'' he said. ``They kept their jobs, but they never
had a nest egg. They already were not spending money on traveling.''

Alisa Thiede, 43, a high school teacher who lives in Moorestown, N.J.,
said she bought a trampoline for her 13-year-old daughter, who has
epilepsy and learning disabilities. Ms. Thiede also bought an adult
stroller large enough to take her daughter when she goes on runs.

``Most of the things that I have purchased was stuff to try to make
lockdown more manageable'' Ms. Thiede said.

She has also paid off debt, she said.

``It is hard as a single mom with a special-needs kid to stay in front
of expenses,'' Ms. Thiede said. ``While my credit card debts aren't wild
trips to wherever --- it's just like you need to fix your car, or you
need a new mattress --- you're creating debt for pretty small things. So
it definitely has felt to good to pay off credit cards.''

\hypertarget{a-cozy-nest-or-a-cozy-doomsday-bunker}{%
\subsection{A cozy nest, or a cozy doomsday
bunker?}\label{a-cozy-nest-or-a-cozy-doomsday-bunker}}

Michelle Iorio, 46, closed on a new house in Belmar, N.J., in February.
When the pandemic shut down the state the next month, she sped up her
move.

She promptly donated her furniture to Habitat for Humanity and spent
hours on Wayfair, Overstock, Amazon and Walmart, looking for inexpensive
furniture and home goods. She invested in patio furniture so she could
invite friends over for socially distanced visits outside.

``All of this is nicer furniture than I had,'' Ms. Iorio said. ``Is it
top of the line? No, but it's definitely a heck of a lot nicer than what
I had.''

Ms. Iorio, who has lupus, said it quickly became clear that her new
house was about to become a haven from a world that was especially
dangerous for her and others with compromised immune systems. Her doctor
had all but ordered her to move into the new house quickly, she said.

``He told me get into self-quarantine right away,'' Ms. Iorio said. She
moved in March 15.

``And I pretty much haven't left,'' she added.

Ms. Wong, the social worker, said people should be vigilant that the
comfortable spaces they are creating do not become a permanent retreat
from the world.

``Are you building yourself a nice little nest?'' Ms. Wong said. ``Or
are you trying to make this place as palatable as possible because
you're never getting out of there?''

Ms. Wong added: ``There is a difference between `I like it here' and `I
hate it out there.'''

Advertisement

\protect\hyperlink{after-bottom}{Continue reading the main story}

\hypertarget{site-index}{%
\subsection{Site Index}\label{site-index}}

\hypertarget{site-information-navigation}{%
\subsection{Site Information
Navigation}\label{site-information-navigation}}

\begin{itemize}
\tightlist
\item
  \href{https://help.nytimes3xbfgragh.onion/hc/en-us/articles/115014792127-Copyright-notice}{©~2020~The
  New York Times Company}
\end{itemize}

\begin{itemize}
\tightlist
\item
  \href{https://www.nytco.com/}{NYTCo}
\item
  \href{https://help.nytimes3xbfgragh.onion/hc/en-us/articles/115015385887-Contact-Us}{Contact
  Us}
\item
  \href{https://www.nytco.com/careers/}{Work with us}
\item
  \href{https://nytmediakit.com/}{Advertise}
\item
  \href{http://www.tbrandstudio.com/}{T Brand Studio}
\item
  \href{https://www.nytimes3xbfgragh.onion/privacy/cookie-policy\#how-do-i-manage-trackers}{Your
  Ad Choices}
\item
  \href{https://www.nytimes3xbfgragh.onion/privacy}{Privacy}
\item
  \href{https://help.nytimes3xbfgragh.onion/hc/en-us/articles/115014893428-Terms-of-service}{Terms
  of Service}
\item
  \href{https://help.nytimes3xbfgragh.onion/hc/en-us/articles/115014893968-Terms-of-sale}{Terms
  of Sale}
\item
  \href{https://spiderbites.nytimes3xbfgragh.onion}{Site Map}
\item
  \href{https://help.nytimes3xbfgragh.onion/hc/en-us}{Help}
\item
  \href{https://www.nytimes3xbfgragh.onion/subscription?campaignId=37WXW}{Subscriptions}
\end{itemize}
