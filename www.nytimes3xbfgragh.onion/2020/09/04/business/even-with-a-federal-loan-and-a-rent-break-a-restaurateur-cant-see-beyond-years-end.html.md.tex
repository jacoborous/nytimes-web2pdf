Sections

SEARCH

\protect\hyperlink{site-content}{Skip to
content}\protect\hyperlink{site-index}{Skip to site index}

\href{https://www.nytimes3xbfgragh.onion/section/business}{Business}

\href{https://myaccount.nytimes3xbfgragh.onion/auth/login?response_type=cookie\&client_id=vi}{}

\href{https://www.nytimes3xbfgragh.onion/section/todayspaper}{Today's
Paper}

\href{/section/business}{Business}\textbar{}Even with a federal loan and
a rent break, a restaurateur can't see beyond year's end.

\url{https://nyti.ms/31Xp02e}

\begin{itemize}
\item
\item
\item
\item
\item
\end{itemize}

Advertisement

\protect\hyperlink{after-top}{Continue reading the main story}

Supported by

\protect\hyperlink{after-sponsor}{Continue reading the main story}

\hypertarget{even-with-a-federal-loan-and-a-rent-break-a-restaurateur-cant-see-beyond-years-end}{%
\section{Even with a federal loan and a rent break, a restaurateur can't
see beyond year's
end.}\label{even-with-a-federal-loan-and-a-rent-break-a-restaurateur-cant-see-beyond-years-end}}

\includegraphics{https://static01.graylady3jvrrxbe.onion/images/2020/09/04/business/04markets-brf-fishscale/merlin_176482227_18128dff-a5c5-4209-bf67-d91f303d999e-articleLarge.jpg?quality=75\&auto=webp\&disable=upscale}

\href{https://www.nytimes3xbfgragh.onion/by/ben-casselman}{\includegraphics{https://static01.graylady3jvrrxbe.onion/images/2018/11/09/multimedia/author-ben-casselman/author-ben-casselman-thumbLarge.png}}

By \href{https://www.nytimes3xbfgragh.onion/by/ben-casselman}{Ben
Casselman}

\begin{itemize}
\item
  Sept. 4, 2020
\item
  \begin{itemize}
  \item
  \item
  \item
  \item
  \item
  \end{itemize}
\end{itemize}

H. Brandon Williams was supposed to spend this month celebrating the
third anniversary of \href{https://www.wearefishscale.com/}{FishScale},
his restaurant in Washington, D.C. Instead, he is trying to keep his
business afloat.

The restaurant, which specializes in burgers made from sustainably
caught wild fish, survived the initial blow dealt by the pandemic, which
wiped out many other Black-owned small businesses. FishScale already
relied heavily on takeout orders, which made the adjustment to
pandemic-era restrictions comparatively easy, and Mr. Williams was able
to obtain an emergency loan under the federal Paycheck Protection
Program.

But the business isn't out of the woods. Nearby Howard University
\href{https://home.howard.edu/reopen}{recently announced} that it would
shift its undergraduate classes online for the fall semester and close
its dormitories. A seasonal farmers' market, which provides additional
revenue most summers, didn't open this year.

Now, with the federal unemployment supplement and other aid programs
gone, Mr. Williams, 39, notices customers pinching their pennies ---
which is forcing him to do the same. He has cut back to three employees
from six and has won rent concessions from his landlord, which he said
should get him through the end of the year. He isn't sure what will
happen after that.

``We're still at that area where we could go either way,'' Mr. Williams
said.

Julia Pollak, a labor economist for the employment site ZipRecruiter,
said many businesses are facing similar decisions heading into a winter
season that is a challenge for many small businesses in the best of
times.

``There are many companies that after a summer of gathering way too few
acorns are going into a hibernation that may not sustain them,'' she
said.

Widespread business failures, Ms. Pollak said, ``could have a cascading
effect on those local economies.'' That is especially true of Black
neighborhoods that often struggle to draw investment from large
corporations.

Mr. Williams said he wanted to stay in business not only for himself but
also for his community. ``There are a lot of people who couldn't get a
job if it weren't for Black-owned businesses,'' he said. ``I want young
boys and girls to look and see somebody doing something that's out of
the box.''

Advertisement

\protect\hyperlink{after-bottom}{Continue reading the main story}

\hypertarget{site-index}{%
\subsection{Site Index}\label{site-index}}

\hypertarget{site-information-navigation}{%
\subsection{Site Information
Navigation}\label{site-information-navigation}}

\begin{itemize}
\tightlist
\item
  \href{https://help.nytimes3xbfgragh.onion/hc/en-us/articles/115014792127-Copyright-notice}{©~2020~The
  New York Times Company}
\end{itemize}

\begin{itemize}
\tightlist
\item
  \href{https://www.nytco.com/}{NYTCo}
\item
  \href{https://help.nytimes3xbfgragh.onion/hc/en-us/articles/115015385887-Contact-Us}{Contact
  Us}
\item
  \href{https://www.nytco.com/careers/}{Work with us}
\item
  \href{https://nytmediakit.com/}{Advertise}
\item
  \href{http://www.tbrandstudio.com/}{T Brand Studio}
\item
  \href{https://www.nytimes3xbfgragh.onion/privacy/cookie-policy\#how-do-i-manage-trackers}{Your
  Ad Choices}
\item
  \href{https://www.nytimes3xbfgragh.onion/privacy}{Privacy}
\item
  \href{https://help.nytimes3xbfgragh.onion/hc/en-us/articles/115014893428-Terms-of-service}{Terms
  of Service}
\item
  \href{https://help.nytimes3xbfgragh.onion/hc/en-us/articles/115014893968-Terms-of-sale}{Terms
  of Sale}
\item
  \href{https://spiderbites.nytimes3xbfgragh.onion}{Site Map}
\item
  \href{https://help.nytimes3xbfgragh.onion/hc/en-us}{Help}
\item
  \href{https://www.nytimes3xbfgragh.onion/subscription?campaignId=37WXW}{Subscriptions}
\end{itemize}
