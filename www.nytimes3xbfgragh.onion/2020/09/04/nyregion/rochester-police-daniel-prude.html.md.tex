Sections

SEARCH

\protect\hyperlink{site-content}{Skip to
content}\protect\hyperlink{site-index}{Skip to site index}

\href{https://www.nytimes3xbfgragh.onion/section/nyregion}{New York}

\href{https://myaccount.nytimes3xbfgragh.onion/auth/login?response_type=cookie\&client_id=vi}{}

\href{https://www.nytimes3xbfgragh.onion/section/todayspaper}{Today's
Paper}

\href{/section/nyregion}{New York}\textbar{}Daniel Prude's Death: Police
Silence and Accusations of a Cover-Up

\url{https://nyti.ms/2Z3h7GD}

\begin{itemize}
\item
\item
\item
\item
\item
\end{itemize}

\hypertarget{race-and-america}{%
\subsubsection{\texorpdfstring{\href{https://www.nytimes3xbfgragh.onion/news-event/george-floyd-protests-minneapolis-new-york-los-angeles?name=styln-george-floyd\&region=TOP_BANNER\&block=storyline_menu_recirc\&action=click\&pgtype=Article\&impression_id=e8adb670-f277-11ea-882a-edf7a22bd498\&variant=undefined}{Race
and America}}{Race and America}}\label{race-and-america}}

\begin{itemize}
\tightlist
\item
  \href{https://www.nytimes3xbfgragh.onion/2020/09/04/nyregion/rochester-police-daniel-prude.html?name=styln-george-floyd\&region=TOP_BANNER\&block=storyline_menu_recirc\&action=click\&pgtype=Article\&impression_id=e8adb671-f277-11ea-882a-edf7a22bd498\&variant=undefined}{What
  Happened in Rochester, N.Y.}
\item
  \href{https://www.nytimes3xbfgragh.onion/2020/09/01/us/politics/trump-fact-check-protests.html?name=styln-george-floyd\&region=TOP_BANNER\&block=storyline_menu_recirc\&action=click\&pgtype=Article\&impression_id=e8adb672-f277-11ea-882a-edf7a22bd498\&variant=undefined}{Trump
  Fact Check}
\item
  \href{https://www.nytimes3xbfgragh.onion/2020/08/30/us/portland-shooting-explained.html?name=styln-george-floyd\&region=TOP_BANNER\&block=storyline_menu_recirc\&action=click\&pgtype=Article\&impression_id=e8adb673-f277-11ea-882a-edf7a22bd498\&variant=undefined}{Portland
  Shooting}
\item
  \href{https://www.nytimes3xbfgragh.onion/2020/08/30/us/breonna-taylor-police-killing.html?name=styln-george-floyd\&region=TOP_BANNER\&block=storyline_menu_recirc\&action=click\&pgtype=Article\&impression_id=e8addd80-f277-11ea-882a-edf7a22bd498\&variant=undefined}{Breonna
  Taylor's Life and Death}
\end{itemize}

Advertisement

\protect\hyperlink{after-top}{Continue reading the main story}

Supported by

\protect\hyperlink{after-sponsor}{Continue reading the main story}

\hypertarget{daniel-prudes-death-police-silence-and-accusations-of-a-cover-up}{%
\section{Daniel Prude's Death: Police Silence and Accusations of a
Cover-Up}\label{daniel-prudes-death-police-silence-and-accusations-of-a-cover-up}}

The police in Rochester, N.Y., treated the death of Mr. Prude as a drug
overdose. The case drew scrutiny when footage showed that he had been
pinned down and hooded by officers.

\includegraphics{https://static01.graylady3jvrrxbe.onion/images/2020/09/04/nyregion/04rochester/04rochester-articleLarge.jpg?quality=75\&auto=webp\&disable=upscale}

By \href{https://www.nytimes3xbfgragh.onion/by/michael-wilson}{Michael
Wilson},
\href{https://www.nytimes3xbfgragh.onion/by/jesse-mckinley}{Jesse
McKinley},
\href{https://www.nytimes3xbfgragh.onion/by/luis-ferre-sadurni}{Luis
Ferré-Sadurní},
\href{https://www.nytimes3xbfgragh.onion/by/troy-closson}{Troy Closson}
and \href{https://www.nytimes3xbfgragh.onion/by/sarah-maslin-nir}{Sarah
Maslin Nir}

\begin{itemize}
\item
  Published Sept. 4, 2020Updated Sept. 8, 2020
\item
  \begin{itemize}
  \item
  \item
  \item
  \item
  \item
  \end{itemize}
\end{itemize}

In the minutes after Daniel Prude's heart briefly stopped during a
struggle with officers who had
\href{https://www.nytimes3xbfgragh.onion/2020/09/03/nyregion/daniel-prude-police-rochester.html}{pulled
a hood over his head}, an unofficial police narrative took hold: He had
suffered a drug overdose.

That account hardened when the
\href{https://www.nytimes3xbfgragh.onion/2020/09/08/nyregion/rochester-police-chief-resigns-prude.html}{police
chief in Rochester, N.Y.,} told the mayor that a man in custody was in
the hospital after taking PCP, or angel dust.

Mr. Prude died a week later, on March 30, but the Rochester police
department offered no public comment in response, continuing to treat
his death as an overdose. An extended period of silence followed,
beginning in Rochester and leading all the way to the state capital. It
ended this week with deepening scrutiny of the long-overlooked case and
accusations from Mr. Prude's family that the city and state have engaged
in a cover-up.

Mr. Prude's death drew national attention on Wednesday after his family
released police body camera footage, obtained through a public records
request, that showed Mr. Prude surrounded by police officers, naked,
handcuffed and held facedown in the street, wearing the hood.

He was under the influence of PCP, and having a psychotic episode. Mr.
Prude seemed lucid at times but at other points asked for money or a
gun. He did not resist arrest and was unarmed.

The Rochester police chief this week denied that he and the department
had misled the public. But an examination by The New York Times of the
official response to Mr. Prude's death shows that police and city
officials in Rochester withheld information about their handling of the
case.

When Mr. Prude died, the police issued no news release. No publicity
followed a county autopsy in mid-April that determined that the manner
of death was homicide, and found that Mr. Prude, 41, had suffered
``complications of asphyxia,'' with the PCP a contributing factor.

An internal investigation in late April quickly cleared the officers
involved of any wrongdoing; its findings were never disclosed.
(\href{https://www.nytimes3xbfgragh.onion/2020/09/03/nyregion/daniel-prude-police-rochester.html}{The
seven officers involved were suspended only on Thursday}.)

Despite a national debate over race and law enforcement, state officials
made little effort to bring attention to Mr. Prude's death. The state
attorney general's office did not disclose that it was investigating the
case until this week, though it said it does not generally announce such
inquiries until they are concluded.

Gov. Andrew M. Cuomo's only public mention of the case before this week
came on July 15, three months after a county medical examiner had ruled
it a homicide, when the governor issued an order
\href{https://www.governor.ny.gov/news/no-14735-amendment-executive-order-147-special-prosecutor-investigate-and-prosecute-matters}{formally
declaring that the attorney general had jurisdiction}.

Image

Mr. PrudeCredit...via Prude Family

And even as the nation erupted in outrage over the death of George
Floyd, a Black man who died in Minneapolis after a police officer held a
knee to his neck, New Yorkers were kept in the dark about another death
in police custody.

Days after Mr. Floyd's death, state prosecutors asked a City of
Rochester lawyer to withhold body camera footage from the public because
releasing such evidence would interfere with the office's investigation,
the mayor's office said on Thursday.

The attorney general's office denied this, noting that the city and the
Rochester police department were ``free to move forward with their own
investigation.'' Rochester city officials, however, repeated their
assertion on Friday.

A combination of factors may have ultimately caused Mr. Prude's death,
according to the medical examiner's report. But the release by Mr.
Prude's family of officers' body camera footage from that night ---
showing a naked Black man, handcuffed and hooded in the falling snow ---
radically complicated the police narrative.

His death added another name to the list of Black people, including Mr.
Floyd and Breonna Taylor, who lost their lives after police encounters,
leading to unrest in the nation's streets and providing a potent issue
in the 2020 presidential election.

Protesters have taken to the streets of Rochester every night since the
video was made public on Wednesday. They have grown in intensity, with
the police using pepper spray to disperse crowds on Thursday.

On Friday night, what began as a peaceful rally at Martin Luther King
Jr. Memorial Park took a violent turn. Protesters marching past
restaurants
\href{https://twitter.com/ScooterCasterNY/status/1302060108898357257}{overturned
tables and threw furniture and bottles} as diners scattered. Police
officers in riot gear responded to the chaotic scene with pepper spray
and orders to disperse.

Later, as protesters paused at the intersection of East Avenue and
Alexander Street at about 1:20 a.m. Saturday, two cars drove into the
crowd, knocking at least two people to the ground. In videos shared on
Twitter, the driver of at least one car can be seen spraying
demonstrators with chemicals and racing away.

Events in the case began on March 22, when Mr. Prude, a father of five
from Chicago, arrived at his brother's home in Rochester. He had been
kicked off an Amtrak train in Buffalo, about 75 miles to the west, and
he appeared deeply troubled, seeming to hallucinate and trying to hurt
himself by jumping down a flight of stairs, family members said.

His brother, Joe Prude, had him hospitalized that very day for
evaluation, but the hospital released him hours later.

\includegraphics{https://static01.graylady3jvrrxbe.onion/images/2020/09/04/nyregion/04rochester-03/04rochester-03-articleLarge.jpg?quality=75\&auto=webp\&disable=upscale}

In the early hours of March 23, Mr. Prude ran from his brother's home in
just a tank top and long underwear on a freezing night. A short time
later, officers responded to a 911 call of a naked man who was ranting
in the streets and telling at least one passing stranger that he had the
coronavirus.

Officers arrived and handcuffed Mr. Prude without incident as he sat in
the street. Between prayers and profanities, and demands for money or a
cigarette --- or a gun --- he spat on the street, ignoring officers'
commands to stop, the footage shows. An officer draped a mesh hood ---
\href{https://www.nytimes3xbfgragh.onion/2020/09/03/nyregion/spit-hoods-police.html}{known
as a spit sock} --- over his head, and he became more agitated,
attempting to rise to his feet.

Officers pinned him to the ground, and his pleas turned into muffled
gurgles inside the hood, then stopped altogether.

An ambulance arrived, and a paramedic performed chest compressions. Mr.
Prude was in cardiac distress --- or ``coded,'' in medical jargon. One
of the paramedics stood up and addressed the officers.

``So, PCP can cause what we call `excited delirium,''' she said, as
recorded on an officer's body camera. ``I guarantee you that's how he
coded. It's not you guys' fault. You've got to keep yourselves safe.''

No one disagreed.

Image

Mayor Lovely Warren of Rochester said the city's police chief did not
inform her of the actions taken to restrain Mr. Prude. The chief this
week denied that he or the Police Department had engaged in a
cover-up.Credit...Joshua Rashaad McFadden for The New York Times

Later that day, the Rochester police chief, La'Ron Singletary, told
Lovely Warren, the mayor, ``that Mr. Prude had an apparent drug overdose
while in custody,'' Ms. Warren said in a statement this week. ``Chief
Singletary never informed Mayor Warren of the actions his officers took
to forcibly restrain Mr. Prude,'' the statement said.

The Rochester Police Department did not return repeated calls and emails
for comment on Friday. Chief Singletary insisted on Wednesday that the
police were not hiding anything.

``I know that there is a rhetoric that is out there that this is a
cover-up,'' the chief said. ``This is not a cover-up.''

While Mr. Prude was in the hospital, his brother --- forbidden to visit
because of concerns over the coronavirus --- could find out little about
his condition.

He called a lawyer's office, and on April 3, the family filed an open
records request with the Rochester Police Department seeking footage
from the body cameras and a request that any evidence in the encounter
with Mr. Prude be preserved, according to the lawyer, Elliot Shields.

Letitia James, the state attorney general, learned of the case on April
16, after her office was informed by the county district attorney of the
preliminary autopsy report that showed asphyxiation was a cause of
death. Ms. James's office took over the case by April 21 and began an
investigation.

The Rochester police were already closing their own internal
investigation, interviewing the officers involved and reviewing their
body camera footage. It concluded on April 27: ``Based upon the
investigation, the officers' actions and conduct displayed when dealing
with Prude appear to be appropriate and consistent with their
training.''

The case remained under wraps in May, even as the country erupted in
protests. On June 4, a little more than a week after the death of Mr.
Floyd, a top official in Ms. James's
office\href{https://twitter.com/WHEC_AHyman/status/1301713197452361728}{asked
city officials} not to release body camera footage so as not to
``interfere with the attorney general's ongoing investigation,'' the
Rochester mayor's office said this week.

``The city complied with the attorney general's office request,'' said
Justin Roj, a city spokesman.

Image

Demonstrators in Rochester protested Mr. Prude's death through the night
on Thursday.Credit...Joshua Rashaad McFadden for The New York Times

Ms. James's office bluntly denied that.

``There was never a request from the attorney general's office to the
City of Rochester corporation counsel to withhold information about the
events surrounding the death of Daniel Prude, plain and simple,'' a
spokeswoman for the office said in a statement.

Mr. Shields, the Prude family's lawyer, was still awaiting the videos he
had requested from the Rochester police when he and family members,
including Mr. Prude's father, were invited to view those same videos in
the attorney general's office on July 31. The footage was horrifying,
Mr. Shields said.

``It was honestly the most difficult thing I've ever done as an
attorney,'' he said. ``I thought the father was going to die of an
asthma attack. He had the inhaler out. He couldn't breathe. It was
horrible.''

On Aug. 4, the Rochester corporation counsel reviewed the camera footage
before releasing it to Mr. Shields and the Prude family, and shared it
with the mayor for the first time. It was the first time she learned
officers had struggled with Mr. Prude, more than four months after that
night, she said this week.

The release of the videos on Wednesday brought the first public comments
from Mr. Cuomo and Ms. James.

Image

Demonstrations have been a daily occurrence in the city since the video
was released to the public on Wednesday.Credit...Joshua Rashaad McFadden
for The New York Times

The state's only other acknowledgment of the investigation into Mr.
Prude's death came on July 15, when Mr. Cuomo's order was posted on a
\href{https://www.governor.ny.gov/news/no-14735-amendment-executive-order-147-special-prosecutor-investigate-and-prosecute-matters}{state
website,} but not publicized.

The attorney general's office has defended its actions and the pace of
investigation, noting that such probes --- authorized by
\href{https://www.governor.ny.gov/sites/governor.ny.gov/files/atoms/old-files//EO147.pdf}{a
2015 executive order} by Governor Cuomo --- sometimes take more than a
year to complete and involve extensive interviews, examination of police
and medical records and reviews of police protocol. As of Friday, the
seven suspended officers still had not agreed to be questioned by Ms.
James's office, according to a person with knowledge of the
investigation.

On Thursday, Mr. Cuomo asked for the case ``to be concluded ‎as
expeditiously as possible'' and urged the Rochester Police Department to
cooperate.

In
\href{https://www.facebookcorewwwi.onion/watch/live/?v=997967440651864\&ref=search}{a
news conference} on Thursday afternoon, Rochester's mayor, Ms. Warren,
said she was personally and professionally offended that her police
chief had failed ``to fully and accurately inform me of what occurred
with Mr. Prude.''

But Ms. Warren, a lawyer, also blamed herself, saying that she had
approached the case --- and the investigation --- in ``the mind-set of
an attorney, and not necessarily the mind-set of a human being.''

``What I saw in that video,'' the mayor said, ``was a man who needed
help, a man who needed compassion, a man who needed humanity, a man who
we should have respected, a man who was in crisis.

``Our response to him,'' she concluded, ``was wrong.''

Joshua Rashaad McFadden contributed reporting.

Advertisement

\protect\hyperlink{after-bottom}{Continue reading the main story}

\hypertarget{site-index}{%
\subsection{Site Index}\label{site-index}}

\hypertarget{site-information-navigation}{%
\subsection{Site Information
Navigation}\label{site-information-navigation}}

\begin{itemize}
\tightlist
\item
  \href{https://help.nytimes3xbfgragh.onion/hc/en-us/articles/115014792127-Copyright-notice}{©~2020~The
  New York Times Company}
\end{itemize}

\begin{itemize}
\tightlist
\item
  \href{https://www.nytco.com/}{NYTCo}
\item
  \href{https://help.nytimes3xbfgragh.onion/hc/en-us/articles/115015385887-Contact-Us}{Contact
  Us}
\item
  \href{https://www.nytco.com/careers/}{Work with us}
\item
  \href{https://nytmediakit.com/}{Advertise}
\item
  \href{http://www.tbrandstudio.com/}{T Brand Studio}
\item
  \href{https://www.nytimes3xbfgragh.onion/privacy/cookie-policy\#how-do-i-manage-trackers}{Your
  Ad Choices}
\item
  \href{https://www.nytimes3xbfgragh.onion/privacy}{Privacy}
\item
  \href{https://help.nytimes3xbfgragh.onion/hc/en-us/articles/115014893428-Terms-of-service}{Terms
  of Service}
\item
  \href{https://help.nytimes3xbfgragh.onion/hc/en-us/articles/115014893968-Terms-of-sale}{Terms
  of Sale}
\item
  \href{https://spiderbites.nytimes3xbfgragh.onion}{Site Map}
\item
  \href{https://help.nytimes3xbfgragh.onion/hc/en-us}{Help}
\item
  \href{https://www.nytimes3xbfgragh.onion/subscription?campaignId=37WXW}{Subscriptions}
\end{itemize}
