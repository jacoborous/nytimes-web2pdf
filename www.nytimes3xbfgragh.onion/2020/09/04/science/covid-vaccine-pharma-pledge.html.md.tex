Sections

SEARCH

\protect\hyperlink{site-content}{Skip to
content}\protect\hyperlink{site-index}{Skip to site index}

\href{https://www.nytimes3xbfgragh.onion/section/science}{Science}

\href{https://myaccount.nytimes3xbfgragh.onion/auth/login?response_type=cookie\&client_id=vi}{}

\href{https://www.nytimes3xbfgragh.onion/section/todayspaper}{Today's
Paper}

\href{/section/science}{Science}\textbar{}Pharma Companies Plan Joint
Pledge on Vaccine Safety

\url{https://nyti.ms/2EYXatG}

\begin{itemize}
\item
\item
\item
\item
\item
\end{itemize}

Advertisement

\protect\hyperlink{after-top}{Continue reading the main story}

Supported by

\protect\hyperlink{after-sponsor}{Continue reading the main story}

\hypertarget{pharma-companies-plan-joint-pledge-on-vaccine-safety}{%
\section{Pharma Companies Plan Joint Pledge on Vaccine
Safety}\label{pharma-companies-plan-joint-pledge-on-vaccine-safety}}

The statement is meant to reassure the public that the companies will
not seek a premature approval of vaccines under pressure from the Trump
administration.

By \href{https://www.nytimes3xbfgragh.onion/by/katie-thomas}{Katie
Thomas}, \href{https://www.nytimes3xbfgragh.onion/by/noah-weiland}{Noah
Weiland} and
\href{https://www.nytimes3xbfgragh.onion/by/sharon-lafraniere}{Sharon
LaFraniere}

\begin{itemize}
\item
  Sept. 4, 2020
\item
  \begin{itemize}
  \item
  \item
  \item
  \item
  \item
  \end{itemize}
\end{itemize}

\includegraphics{https://static01.graylady3jvrrxbe.onion/images/2020/09/05/lens/04JPvaccine-print/04vaccine-2pic-articleLarge.jpg?quality=75\&auto=webp\&disable=upscale}

A group of drug companies competing with one another to be among the
first to develop coronavirus vaccines are planning to pledge early next
week that they will not release any vaccines that do not follow rigorous
efficacy and safety standards, according to representatives of three of
the companies.

The statement, which has not yet been finalized, is meant to reassure
the public that the companies will not seek a premature approval of
vaccines under political pressure from the Trump administration.
President Trump has pushed for a vaccine to be available by October ---
just before the presidential election --- and a growing number of
scientists, regulators and public health experts have expressed concern
over what they see as a pattern of political arm-twisting by the Trump
administration in its efforts to combat the virus.

The companies' joint statement was planned for early next week, but it
may be released before then after its existence was made public on
Friday by The
\href{https://www.wsj.com/articles/covid-19-vaccine-developers-prepare-joint-pledge-on-safety-standards-11599257729?mod=hp_lead_pos2}{Wall
Street Journal}. The manufacturers that are said to have signed the
letter include Pfizer, Moderna, Johnson \& Johnson, GlaxoSmithKline and
Sanofi.

The pharmaceutical companies are not the only ones pushing back. Senior
regulators at the Food and Drug Administration have been discussing
making their own joint public statement about the need to rely on proven
science, according to two senior administration officials, a move that
would breach their usual reticence as civil servants.

Scientists have been rushing at record speed to develop a vaccine that
could end the pandemic,
\href{https://www.nytimes3xbfgragh.onion/interactive/2020/us/coronavirus-us-cases.html}{which
has taken nearly 190,000 lives} and infected more than six million
people in the United States. Three companies --- Moderna, Pfizer and
AstraZeneca --- are testing their candidates in late-stage clinical
trials.

Pfizer's chief executive said this week that the company could see
results as early as October, but the others have said only that they
plan to release a vaccine by the end of the year.

Public health experts have applauded the companies' rapid development of
a vaccine, and early results have been promising. But in recent weeks,
they have grown worried as Mr. Trump and his allies have begun talking
about a vaccine that could be ready before the election on Nov. 3.

Image

A lab technician sorted blood samples for a Covid-19 vaccination study
at the Research Centers of America in Hollywood, Fla., last
month.Credit...Chandan Khanna/Agence France-Presse --- Getty Images

Even as companies are competing to be the first to bring a coronavirus
vaccine to market, they must navigate perilous political terrain. If
they are among the first to bring a successful vaccine to market, they
could earn major profits and help rehabilitate the image of an industry
battered by rising drug prices.

But if a vaccine turns out to have dangerous side effects for some
people, the fallout could be catastrophic, damaging their corporate
reputations, putting their broader portfolio of products at risk and
broadly undermining trust in vaccines, one of the great public health
advances in human history.

In tweets and public comments, Mr. Trump has explicitly tied his
re-election fortunes to a vaccine, an idea detailed last week at the
Republican National Convention, where promotional videos featured the
administration's efforts to fund and develop one in its crash program
called Operation Warp Speed.

Trump campaign advisers have privately called a pre-election vaccine
``the holy grail.''

Also last week, the Centers for Disease Control and Prevention
\href{https://www.nytimes3xbfgragh.onion/2020/09/02/health/covid-19-vaccine-cdc-plans.html}{sent
letters to public health agencies} around the country asking them to
prepare for the possibility that a vaccine could be ready by late
October or early November.

And on Friday, even as federal health officials had been tempering
expectations about when a vaccine would be ready, Mr. Trump said one
would ``probably'' be ready in October. Even the companies can't see the
results while the trials are underway, but he promised, ``You are going
to see results that are shockingly good.''

Just the day before, Dr. Moncef Slaoui, the top scientist on Operation
Warp Speed, warned in
\href{https://www.npr.org/2020/09/03/909312697/operation-warp-speed-top-adviser-on-the-status-of-a-coronavirus-vaccine}{an
interview with National Public Radio} that the chance of successful
vaccine results by October was ``very, very low.''

Scientists within the federal government and outside of it say they are
dismayed by what they see as meddling by the Trump administration in the
federal pandemic response, from the president's
\href{https://www.nytimes3xbfgragh.onion/2020/06/15/health/fda-hydroxychloroquine-malaria.html}{misguided
promotion of hydroxychloroquine} as a treatment and his
\href{https://www.nytimes3xbfgragh.onion/2020/08/23/us/politics/fda-plasma-coronavirus.html}{exaggeration}
of the benefits of convalescent plasma to the C.D.C.'s
\href{https://www.nytimes3xbfgragh.onion/2020/08/26/us/politics/coronavirus-testing-trump-cdc.html}{changing
guidance on who should be tested}.

Several top health officials have made it explicit in recent weeks that
they would rather quit than be co-opted by the White House in approving
a vaccine.

Dr. Slaoui
\href{https://www.sciencemag.org/news/2020/09/leader-us-vaccine-push-says-he-ll-quit-if-politics-trumps-science-approval-process}{told}
Science magazine on Thursday that he would ``immediately resign if there
is undue interference in this process,'' though he said there had been
none so far.

In a conference call last month, Dr. Peter Marks, who heads the F.D.A.
division that approves new vaccines and treatments, made the same pledge
to members of a vaccine working group at the National Institutes of
Health.

``If something is not safe enough and effective enough for my family,
there's no way I'm going to stand by and see it given to the rest of the
country,'' Dr. Marks said in an interview on Thursday. ``It's not going
to happen under my watch.''

Dr. Stephen M. Hahn, the F.D.A. commissioner, has repeatedly said his
decisions are based on scientific data alone. He has
\href{https://www.healthaffairs.org/do/10.1377/hblog20200814.996612/full/}{publicly
committed} to vet any vaccine approval through an advisory committee of
outside experts, who typically review clinical trial data before a new
vaccine is approved.

Senior F.D.A. officials, including political appointees, have been
frustrated by comments Mr. Trump has made in recent weeks about his
efforts to speed along approvals, giving the impression the White House
is playing a regulatory role.

At his news conference Friday, Mr. Trump said he had just spoken to the
head of Pfizer, describing him as a ``great guy'' whose company is a
leader in the race to develop a vaccine.

\href{https://slack-redir.net/link?url=https\%3A\%2F\%2Ftwitter.com\%2Fthehill\%2Fstatus\%2F1301926138818768896}{In
a separate appearance} on Friday, the president said pharmaceutical
companies had told him that ``if this was a more typical kind of
president, getting these approvals would take two or three years.''

With vaccines, Mr. Trump's ability to influence the approval process has
its limits. While a government agency, such as the C.D.C., can request
vaccine approval, requests typically come from the drug makers.

Companies have separately sought to underscore their commitment to
rigorous scientific review.

On Monday, AstraZeneca's chief executive, Pascal Soriot,
\href{https://www.astrazeneca.com/content/astraz/media-centre/press-releases/2020/astrazenecas-scientific-and-social-commitment-for-covid-19-vaccine.html}{released
a statement} acknowledging recent questions about the speed of vaccine
development. ``I want to reiterate my commitment that we are putting
science and the interest of society at the heart of our work,'' he said.
``We are moving quickly but without cutting corners.''

And on Friday, Moderna's chief executive, Stéphane Bancel,
\href{https://www.cnbc.com/2020/09/04/moderna-slows-coronavirus-vaccine-trial-t-to-ensure-minority-representation-ceo-says.html}{told
CNBC} that the company was slowing enrollment in its trials to include
more people from groups at high risk for Covid-19. ``I would rather we
have higher diverse participants and take one extra week,'' Mr. Bancel
said in the interview.

Michael D. Shear contributed reporting.

Advertisement

\protect\hyperlink{after-bottom}{Continue reading the main story}

\hypertarget{site-index}{%
\subsection{Site Index}\label{site-index}}

\hypertarget{site-information-navigation}{%
\subsection{Site Information
Navigation}\label{site-information-navigation}}

\begin{itemize}
\tightlist
\item
  \href{https://help.nytimes3xbfgragh.onion/hc/en-us/articles/115014792127-Copyright-notice}{©~2020~The
  New York Times Company}
\end{itemize}

\begin{itemize}
\tightlist
\item
  \href{https://www.nytco.com/}{NYTCo}
\item
  \href{https://help.nytimes3xbfgragh.onion/hc/en-us/articles/115015385887-Contact-Us}{Contact
  Us}
\item
  \href{https://www.nytco.com/careers/}{Work with us}
\item
  \href{https://nytmediakit.com/}{Advertise}
\item
  \href{http://www.tbrandstudio.com/}{T Brand Studio}
\item
  \href{https://www.nytimes3xbfgragh.onion/privacy/cookie-policy\#how-do-i-manage-trackers}{Your
  Ad Choices}
\item
  \href{https://www.nytimes3xbfgragh.onion/privacy}{Privacy}
\item
  \href{https://help.nytimes3xbfgragh.onion/hc/en-us/articles/115014893428-Terms-of-service}{Terms
  of Service}
\item
  \href{https://help.nytimes3xbfgragh.onion/hc/en-us/articles/115014893968-Terms-of-sale}{Terms
  of Sale}
\item
  \href{https://spiderbites.nytimes3xbfgragh.onion}{Site Map}
\item
  \href{https://help.nytimes3xbfgragh.onion/hc/en-us}{Help}
\item
  \href{https://www.nytimes3xbfgragh.onion/subscription?campaignId=37WXW}{Subscriptions}
\end{itemize}
