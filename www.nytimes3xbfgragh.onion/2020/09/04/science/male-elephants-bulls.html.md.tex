Sections

SEARCH

\protect\hyperlink{site-content}{Skip to
content}\protect\hyperlink{site-index}{Skip to site index}

\href{https://www.nytimes3xbfgragh.onion/section/science}{Science}

\href{https://myaccount.nytimes3xbfgragh.onion/auth/login?response_type=cookie\&client_id=vi}{}

\href{https://www.nytimes3xbfgragh.onion/section/todayspaper}{Today's
Paper}

\href{/section/science}{Science}\textbar{}Old Male Elephants: Don't
Count Them Out

\url{https://nyti.ms/3gYJU5d}

\begin{itemize}
\item
\item
\item
\item
\item
\end{itemize}

Advertisement

\protect\hyperlink{after-top}{Continue reading the main story}

Supported by

\protect\hyperlink{after-sponsor}{Continue reading the main story}

Trilobites

\hypertarget{old-male-elephants-dont-count-them-out}{%
\section{Old Male Elephants: Don't Count Them
Out}\label{old-male-elephants-dont-count-them-out}}

New research challenges the assumption that bulls become redundant in
elephant society after breeding.

\includegraphics{https://static01.graylady3jvrrxbe.onion/images/2020/09/04/science/04TB-ELEPHANTS/04TB-ELEPHANTS-articleLarge.jpg?quality=75\&auto=webp\&disable=upscale}

By Rachel Nuwer

\begin{itemize}
\item
  Sept. 4, 2020
\item
  \begin{itemize}
  \item
  \item
  \item
  \item
  \item
  \end{itemize}
\end{itemize}

For years, scientists assumed that when it came to elephants on the
African savanna, matriarchs were the only leaders. Old males, on the
other hand, were seen as solitary loners whose social contribution ended
at breeding. New evidence suggests that male elephants do have social
lives, and that
\href{https://www.nature.com/articles/s41598-020-70682-y}{older males
may act as leaders} for younger ones.

``These findings build on support that's slowly being revealed about the
importance of old bulls,'' said Connie Allen, a doctoral researcher in
animal behavior at the University of Exeter in England, and lead author
of a study published on Thursday in Scientific Reports. ``This is the
first study that concretely shows older bulls in a leadership role.''

Little research has focused on male elephants --- which can live 60-plus
years --- because
\href{https://www.nytimes3xbfgragh.onion/2016/03/18/world/an-african-elephant-named-morgan-strays-into-a-war-zone-and-survives.html}{males
tend to roam across vast distances}. This makes them difficult to track
and observe. A few studies have hinted, though, that there is more to
males than assumed.

For example, from 1992 to 1997, young orphaned male elephants that had
been introduced to Pilanesberg National Park in South Africa began
coming into premature musth, a temporary state of heightened aggression
and sexual activity. When females rejected the adolescents' advances,
the young males took their aggression out on white rhinos, killing more
than 40. Seeking a solution, researchers
\href{https://www.nature.com/articles/35044191?proof=true}{introduced
six older male elephants} to the park. The younger males' musth
subsided, and the rhino killing stopped.

``Older males perhaps control aggressive behavior in younger ones,'' Ms.
Allen said.

Scientists have also found that male elephants of all ages prefer to
have the
\href{https://www.idausa.org/wp-content/uploads/2013/05/EvansHarris-male-sociality2-2.pdf}{oldest
males as their closest neighbors}, possibly because of opportunities to
learn, and that younger males seem to
\href{https://journals.plos.org/plosone/article?id=10.1371/journal.pone.0031382}{learn
how to raid crops} from older males.

Ms. Allen and her colleagues based their study in Botswana's
Makgadikgadi Pans National Park, where 98 percent of the elephants are
males. No one knows why the park's sex ratio is so skewed, although
environmental pressure might be pushing males to gather on the fringes
of the country's more northerly elephant stronghold.

The researchers set up camera traps along seven pathways where elephants
routinely traveled alone or in a single file line. In thousands of hours
of footage, they were able to identify more than 1,000 individual male
elephants, whose age they estimated according to body characteristics.

Their analysis revealed that adolescent males almost always traveled
with other males, and that older bulls were most likely to be found at
the head of the line. The elephants otherwise did not follow a strict
age-related pecking order, suggesting that older males are acting as
leaders.

Because of their larger tusks,
\href{https://www.nytimes3xbfgragh.onion/2017/12/04/science/elephants-lions-africa-hunting.html}{older
bulls tend to be targeted by both poachers and trophy hunters}. Trophy
hunters have justified this by pointing to the assumption that male
elephants are redundant for a population. The new findings, however,
``underline the dire consequences of removing the oldest, largest males
from elephant populations,'' said Karen McComb, an animal behavior
scientist at the University of Sussex in England, who was not involved
in the study.

In the absence of older bulls, younger ones would probably still be able
to navigate a habitat by using well-worn paths to water, said Chris
Thouless, head of research at Save the Elephants, a conservation
organization. But older animals could be crucial to more critical
knowledge, such as where to find water during a prolonged drought, or
how to evade poachers. ``That's much more difficult to test, but more
directly related to the question of the value of age,'' Dr. Thouless
said.

Graeme Shannon, a zoologist at Bangor University in Wales, who was also
not involved in the study, added that the findings do not establish
whether older males are actively leading the group or just tolerating
the younger males. But they might still be imparting ``crucial social
knowledge,'' he said, such as ``how to behave and succeed.''

``There remains lots of questions to answer, but this research provides
further evidence of the importance of these older male elephants in
elephant society,'' Dr. Shannon said.

Advertisement

\protect\hyperlink{after-bottom}{Continue reading the main story}

\hypertarget{site-index}{%
\subsection{Site Index}\label{site-index}}

\hypertarget{site-information-navigation}{%
\subsection{Site Information
Navigation}\label{site-information-navigation}}

\begin{itemize}
\tightlist
\item
  \href{https://help.nytimes3xbfgragh.onion/hc/en-us/articles/115014792127-Copyright-notice}{©~2020~The
  New York Times Company}
\end{itemize}

\begin{itemize}
\tightlist
\item
  \href{https://www.nytco.com/}{NYTCo}
\item
  \href{https://help.nytimes3xbfgragh.onion/hc/en-us/articles/115015385887-Contact-Us}{Contact
  Us}
\item
  \href{https://www.nytco.com/careers/}{Work with us}
\item
  \href{https://nytmediakit.com/}{Advertise}
\item
  \href{http://www.tbrandstudio.com/}{T Brand Studio}
\item
  \href{https://www.nytimes3xbfgragh.onion/privacy/cookie-policy\#how-do-i-manage-trackers}{Your
  Ad Choices}
\item
  \href{https://www.nytimes3xbfgragh.onion/privacy}{Privacy}
\item
  \href{https://help.nytimes3xbfgragh.onion/hc/en-us/articles/115014893428-Terms-of-service}{Terms
  of Service}
\item
  \href{https://help.nytimes3xbfgragh.onion/hc/en-us/articles/115014893968-Terms-of-sale}{Terms
  of Sale}
\item
  \href{https://spiderbites.nytimes3xbfgragh.onion}{Site Map}
\item
  \href{https://help.nytimes3xbfgragh.onion/hc/en-us}{Help}
\item
  \href{https://www.nytimes3xbfgragh.onion/subscription?campaignId=37WXW}{Subscriptions}
\end{itemize}
