Sections

SEARCH

\protect\hyperlink{site-content}{Skip to
content}\protect\hyperlink{site-index}{Skip to site index}

\href{https://www.nytimes3xbfgragh.onion/section/us}{U.S.}

\href{https://myaccount.nytimes3xbfgragh.onion/auth/login?response_type=cookie\&client_id=vi}{}

\href{https://www.nytimes3xbfgragh.onion/section/todayspaper}{Today's
Paper}

\href{/section/us}{U.S.}\textbar{}Bar and Medical Exam Delays Keep
Graduates in Limbo

\url{https://nyti.ms/3jO2AX0}

\begin{itemize}
\item
\item
\item
\item
\item
\end{itemize}

\hypertarget{school-reopenings}{%
\subsubsection{\texorpdfstring{\href{https://www.nytimes3xbfgragh.onion/spotlight/schools-reopening?name=styln-coronavirus-schools-reopening\&region=TOP_BANNER\&block=storyline_menu_recirc\&action=click\&pgtype=Article\&impression_id=73af9050-f281-11ea-aef1-df8e69b7388b\&variant=undefined}{School
Reopenings}}{School Reopenings}}\label{school-reopenings}}

\begin{itemize}
\tightlist
\item
  \href{https://www.nytimes3xbfgragh.onion/2020/09/04/us/bar-exam-coronavirus.html?name=styln-coronavirus-schools-reopening\&region=TOP_BANNER\&block=storyline_menu_recirc\&action=click\&pgtype=Article\&impression_id=73af9051-f281-11ea-aef1-df8e69b7388b\&variant=undefined}{Delayed
  Licensing Exams}
\item
  \href{https://www.nytimes3xbfgragh.onion/2020/09/08/upshot/children-testing-shortfalls-virus.html?name=styln-coronavirus-schools-reopening\&region=TOP_BANNER\&block=storyline_menu_recirc\&action=click\&pgtype=Article\&impression_id=73af9052-f281-11ea-aef1-df8e69b7388b\&variant=undefined}{Limited
  Testing for Children}
\item
  \href{https://www.nytimes3xbfgragh.onion/2020/09/01/world/schools-reopen-globe-students.html?name=styln-coronavirus-schools-reopening\&region=TOP_BANNER\&block=storyline_menu_recirc\&action=click\&pgtype=Article\&impression_id=73af9053-f281-11ea-aef1-df8e69b7388b\&variant=undefined}{School
  Around the World}
\item
  \href{https://www.nytimes3xbfgragh.onion/interactive/2020/us/covid-college-cases-tracker.html?name=styln-coronavirus-schools-reopening\&region=TOP_BANNER\&block=storyline_menu_recirc\&action=click\&pgtype=Article\&impression_id=73afb760-f281-11ea-aef1-df8e69b7388b\&variant=undefined}{Tracking
  College Cases}
\end{itemize}

Advertisement

\protect\hyperlink{after-top}{Continue reading the main story}

Supported by

\protect\hyperlink{after-sponsor}{Continue reading the main story}

\hypertarget{bar-and-medical-exam-delays-keep-graduates-in-limbo}{%
\section{Bar and Medical Exam Delays Keep Graduates in
Limbo}\label{bar-and-medical-exam-delays-keep-graduates-in-limbo}}

Many recent graduates can't practice their professions without passing a
licensing exam, but those tests have been disrupted for months by the
coronavirus pandemic.

\includegraphics{https://static01.graylady3jvrrxbe.onion/images/2020/09/04/us/04virus-licensing01/merlin_175576743_51158857-00d1-4255-827f-96af5a8b7cad-articleLarge.jpg?quality=75\&auto=webp\&disable=upscale}

By \href{https://www.nytimes3xbfgragh.onion/by/emma-goldberg}{Emma
Goldberg}

\begin{itemize}
\item
  Sept. 4, 2020
\item
  \begin{itemize}
  \item
  \item
  \item
  \item
  \item
  \end{itemize}
\end{itemize}

John Molera thought the anxiety of taking the bar exam was over after he
finished his test in late July. Then, a few hours later, he got an email
from a classmate: One of the roughly 30 people sitting in the room with
him for the past two days --- where Mr. Molera recalled many people
lifting masks for snacks and water breaks --- had the coronavirus.

Mr. Molera, 25, is among roughly 70,000 recent law school graduates who
take the bar each year, along with thousands of others who sit for
licensing exams in social work, engineering, surgery and other fields,
typically alongside hundreds of other students squeezed into crowded
rooms at universities or testing centers.

Many of those exams were postponed and testing sites shut down in the
spring as the coronavirus spread across the country, forcing recent
graduates to delay the start of their careers.

Some tests moved online --- often with scheduling problems and even
computer glitches. Other states continued to offer them in person,
raising concern about the possible spread of the virus at testing
centers like the one where Mr. Molera took his exam at the University of
Denver.

The chaos and confusion are helping fuel efforts in some states to
eliminate the bar and other licensing exams, which are seen by some
critics as unnecessary and antiquated, while administrators defend them
as a needed protection for the public.

The test to get a law license is administered by each state's bar
association under the authority of its highest court, typically in
February and July. This year, 23 states went ahead with in-person exams
in July, while at least 20 postponed them until the fall --- with some
offering a new remote option in October.

Jena Speiser, 26, graduated from New York University's law school in the
spring, when the city was still the center of the virus in the United
States. She said the New York State Bar Association encouraged her
classmates and her to apply to take the exam in another state because an
in-person test in July seemed unlikely.

She planned to apply in Massachusetts but then tested positive for the
coronavirus and was bedridden for weeks. She was hospitalized in June
and missed the deadline to apply for the online bar exam New York will
hold in October. Her next opportunity is not until February, and she
cannot practice law until then.

``I have \$300,000 in loans, and I have no idea how I'll start paying
them off,'' Ms. Speiser said. ``I can't work, so I can't get health
insurance. The whole time I was sick I was like, `What if I have to go
to the hospital again?'''

Her anxieties are shared by many recent graduates.

``I need to squeeze the bar exam between feeding my kids and worrying
about their mental and physical health,'' said Leslie Caraballo, 42, who
graduated in the spring and had planned to study for the bar while her
young children were at school and summer camp. ``Everyone's like, `Ruth
Bader Ginsburg did it, she had her ailing husband and children.' But
it's virtually impossible.''

The New York State Bar Association said the move to an online exam in
October was the best way to balance the needs of recent graduates with
the ``integrity of the profession.''

``We understand the pain faced by this year's graduating law school
students,'' Scott Karson, the president of the bar association, wrote in
an email. ``With thousands of graduates hoping to practice in New York
and the unprecedented challenges wrought by Covid-19, there is no easy
solution.''

Medical students have confronted similar uncertainty because of delays
in their licensing exams, which are needed before they can become
residents at hospitals. The United States Medical Licensing Examination
temporarily suspended its exams in late March after the private company
administering them, Prometric, shut down in-person testing sites.

Some medical students said they learned of the postponements from
Prometric less than 48 hours before their tests. The licensing
organization later said it was ``disappointed'' with Prometric's failure
to communicate with students.

``I was trying to study for the most important exam of my life not
knowing if or when it was going to happen,'' said Sirpi Nackeeran, 26, a
third-year medical student at the University of Miami. ``There's only so
many study materials, and you want to time it perfectly.''

Prometric testing sites have begun reopening and administered some board
exams in May and June. While some students have called for an online
version of the test, the Federation of State Medical Boards has
resisted.

\includegraphics{https://static01.graylady3jvrrxbe.onion/images/2020/09/04/us/04virus-licensing02/merlin_175541799_9edc761b-394c-4579-975e-a52268668e1d-articleLarge.jpg?quality=75\&auto=webp\&disable=upscale}

``You're relying on people's internet connection,'' said David Johnson,
the organization's chief assessment officer. ``If you're talking about
sitting in front of a computer taking a high-stakes test for six hours,
that's something to be worried about.''

\href{https://www.nytimes3xbfgragh.onion/spotlight/schools-reopening?action=click\&pgtype=Article\&state=default\&region=MAIN_CONTENT_3\&context=storylines_keepup}{}

\hypertarget{school-reopenings-}{%
\subsubsection{School Reopenings ›}\label{school-reopenings-}}

\hypertarget{back-to-school}{%
\paragraph{Back to School}\label{back-to-school}}

Updated Sept. 8, 2020

The latest on how schools are reopening amid the pandemic.

\begin{itemize}
\item
  \begin{itemize}
  \tightlist
  \item
    The first day of school is an annual rite of passage. But this year,
    it looks very different for tens of millions of students.
    \href{https://www.nytimes3xbfgragh.onion/2020/09/05/us/virtual-return-to-school-covid.html?action=click\&pgtype=Article\&state=default\&region=MAIN_CONTENT_3\&context=storylines_keepup}{We
    talked to some about their hopes and fears}.
  \item
    Coronavirus cases
    \href{https://www.nytimes3xbfgragh.onion/2020/09/06/us/colleges-coronavirus-students.html?action=click\&pgtype=Article\&state=default\&region=MAIN_CONTENT_3\&context=storylines_keepup}{are
    spiking in America's college towns}, leading to concern that young
    people who are infected will contribute to a spread of the virus.
  \item
    A growing number of Catholic schools across the country are
    \href{https://www.nytimes3xbfgragh.onion/2020/09/05/us/catholic-school-closings.html?action=click\&pgtype=Article\&state=default\&region=MAIN_CONTENT_3\&context=storylines_keepup}{shutting
    down forever during the coronavirus pandemic}, citing insurmountable
    financial pressure.
  \item
    The magazine's Ethicist columnist answers a question from a
    spokesperson at a major university:
    \href{https://www.nytimes3xbfgragh.onion/2020/09/08/magazine/university-reopening-safety-ethics.html?action=click\&pgtype=Article\&state=default\&region=MAIN_CONTENT_3\&context=storylines_keepup}{Can
    I promote a reopening plan I have doubts about}?
  \end{itemize}
\end{itemize}

The American Board of Surgery tried to make the switch online,
administering its annual licensing exam virtually in July. The exam
crashed for many test takers, forcing the board to cancel all results.

After months of study, Ayesha Lovick, 33, had prepared for the
eight-hour test, which costs \$1,850, by borrowing a laptop from a
friend and arranging to stay at a neighbor's empty apartment. The
morning after the test, she woke to an email from the board informing
her that her results had been nullified. She was recently informed by
the board that the test will next be offered in April, though she is
contracted to start a new job in October.

``I've spent my entire life working toward becoming a surgeon, and this
was the last test,'' Dr. Lovick said. ``My employer expects that I'll
have taken board exams before I start work. Now my job will have to make
accommodations for the new test date.''

Online testing brings an extra set of challenges for people with
disabilities. All jurisdictions offering the bar exam online are using
artificial-intelligence-driven software that captures audio and video
recordings of test takers in order to prevent cheating. But these
systems sometimes flag excessive motion, including fidgeting, since
\href{https://www.insidehighered.com/news/2020/05/11/online-proctoring-surging-during-covid-19}{movement
can suggest the test taker is communicating with somebody else} in the
room. The monitoring systems have raised concerns among test-takers who
are disabled.

Tara Roslin, director of research for the National Disabled Law Students
Association, plans to take the bar online in October but worries about
the proctoring software because she has Ehlers-Danlos syndrome, which
forces her to self-massage to manage her symptoms.

The disruptions and delays have further fueled a movement in some states
to disband licensing tests, particularly the bar exam, and allow people
to practice their professions with diplomas from accredited
institutions.

Many law students, lawmakers and even law school deans
\href{https://www.nytimes3xbfgragh.onion/2015/03/20/business/dealbook/bar-exam-the-standard-to-become-a-lawyer-comes-under-fire.html}{question
the time and expense devoted to the bar exam}. They say its focus on
rote memorization is not a useful standard for admission to the
profession. Five states --- Louisiana, Oregon, Utah, Washington and
Wisconsin ---~have announced they will permit students who graduated
from accredited law schools to practice as lawyers without taking the
exam.

But the National Conference of Bar Examiners, a nonprofit group that
produces material for the exam and scores it, says licensing exams
maintain an important role.

``A bar exam and C.P.A. exam and physician licensing exam are there for
public protection,'' Judith Gundersen, the group's president, said. ``It
tells the public that these lawyers can competently represent people.''

Ms. Speiser, the N.Y.U. graduate, continues to wait for February and her
next opportunity to take the bar. To start paying back her student
loans, she has been earning money by babysitting.

``People plan their lives around this exam,'' she said. ``Now, on top of
the stress of the pandemic, we're unable to make money. Every single day
I'm panicking.''

Advertisement

\protect\hyperlink{after-bottom}{Continue reading the main story}

\hypertarget{site-index}{%
\subsection{Site Index}\label{site-index}}

\hypertarget{site-information-navigation}{%
\subsection{Site Information
Navigation}\label{site-information-navigation}}

\begin{itemize}
\tightlist
\item
  \href{https://help.nytimes3xbfgragh.onion/hc/en-us/articles/115014792127-Copyright-notice}{©~2020~The
  New York Times Company}
\end{itemize}

\begin{itemize}
\tightlist
\item
  \href{https://www.nytco.com/}{NYTCo}
\item
  \href{https://help.nytimes3xbfgragh.onion/hc/en-us/articles/115015385887-Contact-Us}{Contact
  Us}
\item
  \href{https://www.nytco.com/careers/}{Work with us}
\item
  \href{https://nytmediakit.com/}{Advertise}
\item
  \href{http://www.tbrandstudio.com/}{T Brand Studio}
\item
  \href{https://www.nytimes3xbfgragh.onion/privacy/cookie-policy\#how-do-i-manage-trackers}{Your
  Ad Choices}
\item
  \href{https://www.nytimes3xbfgragh.onion/privacy}{Privacy}
\item
  \href{https://help.nytimes3xbfgragh.onion/hc/en-us/articles/115014893428-Terms-of-service}{Terms
  of Service}
\item
  \href{https://help.nytimes3xbfgragh.onion/hc/en-us/articles/115014893968-Terms-of-sale}{Terms
  of Sale}
\item
  \href{https://spiderbites.nytimes3xbfgragh.onion}{Site Map}
\item
  \href{https://help.nytimes3xbfgragh.onion/hc/en-us}{Help}
\item
  \href{https://www.nytimes3xbfgragh.onion/subscription?campaignId=37WXW}{Subscriptions}
\end{itemize}
