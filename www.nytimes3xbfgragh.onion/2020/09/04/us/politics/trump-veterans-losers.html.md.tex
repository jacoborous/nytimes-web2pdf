Sections

SEARCH

\protect\hyperlink{site-content}{Skip to
content}\protect\hyperlink{site-index}{Skip to site index}

\href{https://www.nytimes3xbfgragh.onion/section/politics}{Politics}

\href{https://myaccount.nytimes3xbfgragh.onion/auth/login?response_type=cookie\&client_id=vi}{}

\href{https://www.nytimes3xbfgragh.onion/section/todayspaper}{Today's
Paper}

\href{/section/politics}{Politics}\textbar{}Trump Faces Uproar Over
Reported Remarks Disparaging Fallen Soldiers

\url{https://nyti.ms/3gREvNm}

\begin{itemize}
\item
\item
\item
\item
\item
\item
\end{itemize}

\begin{itemize}
\item
  \href{https://www.nytimes3xbfgragh.onion/live/2020/09/08/us/trump-vs-biden?action=click\&pgtype=Article\&state=default\&region=TOP_BANNER\&context=storylines_menu}{Election
  Updates}
\item
  \href{https://www.nytimes3xbfgragh.onion/interactive/2020/us/elections/election-states-biden-trump.html?action=click\&pgtype=Article\&state=default\&region=TOP_BANNER\&context=storylines_menu}{Paths
  to 270}
\item
  \href{https://www.nytimes3xbfgragh.onion/interactive/2020/08/31/us/politics/vote-by-mail-deadlines.html?action=click\&pgtype=Article\&state=default\&region=TOP_BANNER\&context=storylines_menu}{Voting
  by Mail}
\item
  \href{https://www.nytimes3xbfgragh.onion/interactive/2019/us/elections/2020-presidential-election-calendar.html?action=click\&pgtype=Article\&state=default\&region=TOP_BANNER\&context=storylines_menu}{Key
  Dates}
\item
  \href{https://www.nytimes3xbfgragh.onion/newsletters/politics?action=click\&pgtype=Article\&state=default\&region=TOP_BANNER\&context=storylines_menu}{Politics
  Newsletter}
\end{itemize}

Advertisement

\protect\hyperlink{after-top}{Continue reading the main story}

Supported by

\protect\hyperlink{after-sponsor}{Continue reading the main story}

\hypertarget{trump-faces-uproar-over-reported-remarks-disparaging-fallen-soldiers}{%
\section{Trump Faces Uproar Over Reported Remarks Disparaging Fallen
Soldiers}\label{trump-faces-uproar-over-reported-remarks-disparaging-fallen-soldiers}}

A report in The Atlantic said the president called troops killed in
combat ``losers'' and ``suckers.'' He strenuously denied it, but some
close to him said it was in keeping with other private comments he has
made disparaging soldiers.

\includegraphics{https://static01.graylady3jvrrxbe.onion/images/2020/09/04/us/04dc-trump-military1/merlin_176592819_131b4765-15a4-4d5c-81cd-a77954c879cd-articleLarge.jpg?quality=75\&auto=webp\&disable=upscale}

\href{https://www.nytimes3xbfgragh.onion/by/peter-baker}{\includegraphics{https://static01.graylady3jvrrxbe.onion/images/2018/06/13/multimedia/peter-baker/peter-baker-thumbLarge-v2.png}}\href{https://www.nytimes3xbfgragh.onion/by/maggie-haberman}{\includegraphics{https://static01.graylady3jvrrxbe.onion/images/2018/07/12/multimedia/author-maggie-haberman/author-maggie-haberman-thumbLarge.png}}

By \href{https://www.nytimes3xbfgragh.onion/by/peter-baker}{Peter Baker}
and \href{https://www.nytimes3xbfgragh.onion/by/maggie-haberman}{Maggie
Haberman}

\begin{itemize}
\item
  Sept. 4, 2020
\item
  \begin{itemize}
  \item
  \item
  \item
  \item
  \item
  \item
  \end{itemize}
\end{itemize}

WASHINGTON ---
\href{https://www.nytimes3xbfgragh.onion/interactive/2020/us/elections/donald-trump.html}{President
Trump} confronted a political crisis on Friday that could undercut badly
needed support in the military community for his re-election campaign as
he sought to dispute a report that he privately referred to American
soldiers killed in combat as ``losers'' and ``suckers.''

Mr. Trump, who has long portrayed himself as a champion of the armed
forces and has boasted of rebuilding a military depleted after years of
overseas wars, came under intense fire from Democrats and other
opponents who said
\href{https://www.theatlantic.com/politics/archive/2020/09/trump-americans-who-died-at-war-are-losers-and-suckers/615997/}{a
report in The Atlantic} demonstrated his actual contempt for those who
serve their country in uniform.

The president's foes organized conference calls, blasted out statements,
flocked to television studios and quickly posted advertising online
calling attention to the reported comments. At a news conference, former
Vice President
\href{https://www.nytimes3xbfgragh.onion/interactive/2020/us/elections/joe-biden.html}{Joseph
R. Biden Jr.}, the Democratic presidential nominee, grew emotional as he
said that his son Beau Biden,
\href{https://www.nytimes3xbfgragh.onion/2015/05/31/us/politics/joseph-r-biden-iii-vice-presidents-son-beau-dies-at-46.html}{who
died of brain cancer in 2015}, ``wasn't a sucker'' for serving in the
Army in Iraq.

``How would you feel if you had a kid in Afghanistan right now?'' Mr.
Biden said. ``How would you feel if you lost a son, daughter, husband,
wife? How would you feel, for real?''

Mr. Biden called the reported comments ``disgusting,'' ``sick,
``deplorable,'' ``un-American'' and ``absolutely damnable,'' adding that
he was closer to losing his temper than at any point during the
campaign. ``I've just never been as disappointed in my whole career with
a leader that I've worked with, president or otherwise.''

Mr. Trump denied that he made the remarks repeatedly over the course of
the day and rallied current and former aides who backed him up on the
record. ``It's a fake story and it's a disgrace that they're allowed to
do it,'' he told reporters in the Oval Office, insisting that he
respected the troops. ``To me, they're heroes,'' he said. ``It's even
hard to believe how they could do it. And I say that, the level of
bravery, and to me, they're absolute heroes.''

But he railed against one former military officer,
\href{https://www.nytimes3xbfgragh.onion/2020/09/04/us/politics/kelly-trump.html}{John
F. Kelly}, a retired four-star Marine general who served as his White
House chief of staff at the time of the reported episode and whom he
seemed to blame for the article. ``Didn't do a good job, had no
temperament and ultimately he was petered out,'' Mr. Trump said when
asked about Mr. Kelly on Friday evening. ``He was exhausted. This man
was totally exhausted. He wasn't even able to function in the last
number of months.''

The furor came at a time of rising tension between the commander in
chief and the military leadership over
\href{https://www.nytimes3xbfgragh.onion/2020/06/05/us/politics/protests-milley-trump.html}{his
use of troops against protesters on American streets}, his refusal to
\href{https://www.nytimes3xbfgragh.onion/2020/06/10/us/politics/trump-rejects-renaming-military-bases.html}{rename
bases named for Confederate generals} and his
\href{https://www.nytimes3xbfgragh.onion/2019/11/15/us/trump-pardons.html}{clemency
for accused and convicted war criminals}. A new
\href{https://www.militarytimes.com/news/pentagon-congress/2020/08/31/as-trumps-popularity-slips-in-latest-military-times-poll-more-troops-say-theyll-vote-for-biden/}{poll
by The Military Times} showed Mr. Biden leading Mr. Trump with 41
percent to 37 percent among active-duty troops, a stark departure from
the military's longstanding support for Republicans and a danger sign
for the president.

Recognizing that, the president sought to smooth over friction with some
in the military by abruptly reversing course on Friday afternoon and
announcing that his administration
\href{https://www.nytimes3xbfgragh.onion/2020/09/04/us/politics/stars-and-stripes-trump-military.html}{would
not be closing Stars and Stripes}, the venerable military newspaper, by
the end of the month after all. ``It will continue to be a wonderful
source of information to our Great Military!''
\href{https://twitter.com/realDonaldTrump/status/1301968873487564802}{he
wrote on Twitter}.

While current and former officials contacted on Friday could not confirm
some of the specifics in The Atlantic's account, they did verify that
Mr. Trump resisted supporting an official funeral and lowering flags
after the death of Senator John McCain of Arizona, a Vietnam War hero
whose military service he had disparaged. And Mr. Trump's assertion on
Friday that
\href{https://twitter.com/realDonaldTrump/status/1301710819349204992}{``I
never called John a loser''} was belied
\href{https://www.youtube.com/watch?v=7k1ajHAeXMU}{by video} and
\href{https://twitter.com/realDonaldTrump/status/622522682245033984}{Twitter}
recording him doing just that in 2015.

Moreover, people familiar with Mr. Trump's private conversations say he
has long scorned those who served in Vietnam as being too dumb to have
gotten out of it, as he did through
\href{https://www.nytimes3xbfgragh.onion/2018/12/26/us/politics/trump-vietnam-draft-exemption.html}{a
medical diagnosis of bone spurs in his heels}. At other times, according
to those familiar with the remarks, Mr. Trump has expressed bewilderment
that people choose military service over making money.

Some also recalled him asking why the United States should be so
interested in finding captured soldiers, a comment made in the context
of Mr. McCain, who was a prisoner of war in Vietnam. Another former
official said Mr. Trump often expressed discomfort around people who had
been injured, although he has held events with wounded veterans.

John R. Bolton, the president's former national security adviser who has
broken with him and called him unfit for office, said he was on the trip
in question and never heard Mr. Trump make those remarks. ``I didn't
hear that,'' Mr. Bolton said in an interview. ``I'm not saying he didn't
say them later in the day or another time, but I was there for that
discussion.''

The president privately raged about The Atlantic's article on Friday
morning, and advisers were panicked about how to counter it. They feared
it was the beginning of a constant drip of negative stories from
disenchanted former officials that could sway voters. While Mr. Trump
demanded that allies knock down the article, aides recognized that few
senior military officers were willing to openly defend the president.

The potential for damage was clear by 9:04 a.m., barely 15 hours after
the article was published, when VoteVets, a liberal veterans
organization that has long been critical of Mr. Trump,
\href{https://twitter.com/votevets/status/1301868835998400515}{released
an online ad} featuring the parents of troops slain in Iraq and
Afghanistan, each one declaring that their son or stepson was not a
``loser'' or ``sucker.''

The report by The Atlantic's editor in chief, Jeffrey Goldberg, said
that Mr. Trump decided against visiting a cemetery for American soldiers
killed in World War I during a 2018 visit to France because the rain
would have mussed his hair and because he did not deem it important to
honor the war dead.

The article cited ``four people with firsthand knowledge of the
discussion that day,'' but did not name them. During a conversation with
senior officials that day, according to the magazine, Mr. Trump said:
``Why should I go to that cemetery? It's filled with losers.'' On the
same trip, the article said, he referred to American Marines slain in
combat at Belleau Wood as ``suckers'' for getting killed.

The article also said that Mr. Trump resisted honoring Mr. McCain after
\href{https://www.nytimes3xbfgragh.onion/2018/08/25/obituaries/john-mccain-dead.html}{the
senator's death in August 2018}. ``We're not going to support that
loser's funeral,'' the article quotes Mr. Trump telling his staff. He
became furious at seeing flags lowered to half-staff. ``What the fuck
are we doing that for? Guy was a fucking loser,'' the president told
aides, according to the article.

\includegraphics{https://static01.graylady3jvrrxbe.onion/images/2020/09/04/us/politics/04dc-trump-military2/merlin_164230518_9f1935e7-40c1-4e67-9f88-f83ef03d4401-articleLarge.jpg?quality=75\&auto=webp\&disable=upscale}

Mr. Trump's trip to France in November 2018 came at a fraught moment.
Republicans had just
\href{https://www.nytimes3xbfgragh.onion/2018/11/06/us/politics/midterm-elections-results.html}{lost
the House in midterm elections} when he flew to Paris to attend a
ceremony marking the 100th anniversary of the end of World War I, and he
was vexed at President Emmanuel Macron of France over a security
disagreement.

But it was Mr. Trump's failure to go through with a
\href{https://www.abmc.gov/aisne-marne\#.W-WNEXpKg_U}{planned visit to
the Aisne-Marne American Cemetery} at the foot of the hill where the
Battle of Belleau Wood was fought that drew the most attention. Aides at
the time cited the rain in canceling a helicopter flight, but his
absence went over badly in Europe and in the United States. The
president did pay respects to the war dead the next day at the Suresnes
American Cemetery outside Paris.

Image

John F. Kelly, left, the White House chief of staff at the time, and
Gen. Joseph F. Dunford Jr., the chairman of the Joint Chiefs of Staff at
the time, attending a ceremony at the Aisne-Marne American Cemetery in
France in November 2018, after Mr. Trump declined to
appear.Credit...Francois Mori/Associated Press

At the time of the visit to France, advisers were blunt in confiding
that Mr. Trump was in a foul mood and quizzing aides about whether he
should replace Mr. Kelly.

Several White House officials at the time said the decision not to take
Marine One to the Belleau Wood cemetery was made by Zachary Fuentes, a
close aide to Mr. Kelly, without consulting the president's military
aide. Others argued that a trip by road would have taken too long, at
roughly two hours.

Administration officials said then that Mr. Fuentes had assured Mr.
Trump it was fine to miss the visit. Mr. Kelly traveled to the cemetery
himself in the president's place along with Gen. Joseph F. Dunford Jr.,
then the chairman of the Joint Chiefs of Staff. Mr. Kelly did not
respond to messages on Friday.

Speaking with reporters next to Air Force One on Thursday night after
returning from a campaign rally, Mr. Trump insisted that weather, not
disrespect, forced the flight to be scrapped and that a motorcade would
have had to wind its way through congested areas of Paris. ``The Secret
Service told me, `You can't do it,''' he said. ``I said, `I have to do
it. I want to be there.' They said, `You can't do it.'''

\href{https://www.nytimes3xbfgragh.onion/news-event/2020-election}{Election
2020 ›}

\hypertarget{live-updates}{%
\subsection{\texorpdfstring{\href{https://www.nytimes3xbfgragh.onion/live/2020/09/08/us/trump-vs-biden}{Live
Updates}}{Live Updates}}\label{live-updates}}

\href{https://www.nytimes3xbfgragh.onion/live/2020/09/08/us/trump-vs-biden\#kathryn-garcia-nycs-sanitation-commissioner-resigns-to-mull-a-run-for-mayor}{}

Sept. 8, 2020, 11:10 a.m. ET

\href{https://www.nytimes3xbfgragh.onion/live/2020/09/08/us/trump-vs-biden\#kathryn-garcia-nycs-sanitation-commissioner-resigns-to-mull-a-run-for-mayor}{Kathryn
Garcia, N.Y.C.'s sanitation commissioner, resigns to mull a run for
mayor.}\href{https://www.nytimes3xbfgragh.onion/live/2020/09/08/us/trump-vs-biden\#a-top-house-democrat-calls-for-the-suspension-of-postmaster-general-louis-dejoy-over-campaign-finance-allegations}{}

Sept. 8, 2020, 10:00 a.m. ET

\href{https://www.nytimes3xbfgragh.onion/live/2020/09/08/us/trump-vs-biden\#a-top-house-democrat-calls-for-the-suspension-of-postmaster-general-louis-dejoy-over-campaign-finance-allegations}{A
top House Democrat calls for the suspension of Postmaster General Louis
DeJoy over campaign finance
allegations.}\href{https://www.nytimes3xbfgragh.onion/live/2020/09/08/us/trump-vs-biden\#in-a-closely-watched-new-hampshire-primary-democrats-will-pick-a-challenger-to-governor-sununu}{}

Sept. 8, 2020, 9:37 a.m. ET

\href{https://www.nytimes3xbfgragh.onion/live/2020/09/08/us/trump-vs-biden\#in-a-closely-watched-new-hampshire-primary-democrats-will-pick-a-challenger-to-governor-sununu}{In
a closely watched New Hampshire primary, Democrats will pick a
challenger to Governor Sununu.}

More than a half-dozen current and former aides to Mr. Trump backed him
up with Twitter messages and statements disputing that part of the
Atlantic article. ``I was actually there and one of the people part of
the discussion --- this never happened,''
\href{https://twitter.com/SarahHuckabee/status/1301702348071460864}{wrote
Sarah Huckabee Sanders}, who was then the White House press secretary.
``This is not even close to being factually accurate,''
\href{https://twitter.com/JordanKarem1/status/1301656144713265158}{added
Jordan Karem}, the president's personal aide at the time.

For the White House, it was a full-court defense. ``I've never heard the
president use the language that assertively is said in that article
about him calling military suckers and losers,'' Secretary of State Mike
Pompeo told Fox News. Defense Secretary Mark T. Esper released a
statement saying that Mr. Trump ``has the highest respect and admiration
for our nation's military members, veterans and families.'' Even Melania
Trump weighed in, asserting that the
\href{https://twitter.com/FLOTUS/status/1302011438647701504}{``story is
not true.''}

Mr. Bolton said he was in the room at the ambassador's residence when
Mr. Trump arrived and Mr. Kelly told him that the helicopter trip had to
be canceled. A two-hour motorcade would have put him too far away from
Air Force One and the most capable communications array a president
needs in case of an emergency, per usual protocol, Mr. Bolton said. ``It
was a straight weather call,'' he said.

While Mr. Bolton said he did not hear the president disparage troops, he
added that Mr. Trump did not protest the decision, as he now says he
did. ``He didn't say, `This is terrible, I have to go out to the
veterans,''' Mr. Bolton said. ``He accepted it, and that was pretty much
the end of it.''

Mr. Bolton added that the reported comments were not out of character
for the president. ``I haven't heard anybody yet react to say, `That's
not the Donald Trump I know,''' he said.

The president's reported remarks about Mr. McCain, in fact, were
consistent with his public comments. In 2015, Mr. Trump famously
\href{https://www.nytimes3xbfgragh.onion/2015/07/19/us/politics/trump-belittles-mccains-war-record.html}{mocked
the senator's military service} and five and a half years in captivity
in Vietnam. \href{https://www.youtube.com/watch?v=7k1ajHAeXMU}{``He's
not a war hero,''} Mr. Trump said. ``He was a war hero because he was
captured. I like people who weren't captured.'' Mr. Trump first mocked
Mr. McCain for getting captured during an interview in 1999.

Mr. McCain remained a thorn in Mr. Trump's side after he won the
presidency,
\href{https://www.nytimes3xbfgragh.onion/2017/07/28/us/john-mccains-real-return.html}{blocking
an effort} to overturn President Barack Obama's health care program, a
vote Mr. Trump never forgave. When Mr. McCain died, aides said at the
time that the president had to be shamed into lowering the flags and
\href{https://www.nytimes3xbfgragh.onion/2018/09/01/us/politics/john-mccain-funeral.html}{he
was not invited to the funeral}.

Mr. Trump insisted on Thursday night and on Friday that he respected Mr.
McCain even though they disagreed. ``I was never a fan. I will admit
that openly,'' he said. But ``we lowered the flags,'' sent a military
jet to Arizona to pick up the casket and approved a ``first-class,
triple-A funeral,'' he added.

``All of that had to be approved by the president,'' the president said.
``I approved it without hesitation, without complaint.''

A former senior administration official on Friday disputed Mr. Trump's
assertion that he lowered the flags for Mr. McCain without complaint.
Miles Taylor, who was chief of staff at the Department of Homeland
Security at the time, said he got calls from the White House unhappy
that the department had ordered flags lowered. ``The president is upset,
this has gone out too soon and he doesn't want it to happen,'' he quoted
a White House aide telling him.

``I was then asked, `Would you guys be able to rescind the directive?'''
Mr. Taylor said in an interview. He said he resisted, and ultimately
White House aides pushed Mr. Trump to keep the flags lowered. But it was
made clear that the president ``won't want them down, and he's angry.''
Mr. Taylor, who recently endorsed Mr. Biden, said that he found the
episode ``pretty astounding and disgusting.''

The president's animosity with Mr. McCain had its roots in a dispute
over a development project in 1996, when the senator opposed a federal
loan guarantee that Mr. Trump sought for a West Side project in
Manhattan. But he is not the only military figure to come under Mr.
Trump's critical gaze. During his first presidential campaign, he
publicly dismissed the commanders fighting the Islamic State, saying,
``I know more about ISIS than the generals do.''

During a meeting at the Pentagon in 2017, he berated top generals. ``I
wouldn't go to war with you people,'' Mr. Trump told them, according to
``A Very Stable Genius'' by Philip Rucker and Carol Leonnig. ``You're a
bunch of dopes and babies.''

Peter Baker reported from Washington, and Maggie Haberman from New York.
Annie Karni, Helene Cooper and Eric Schmitt contributed reporting from
Washington.

\hypertarget{our-2020-election-guide}{%
\section{Our 2020 Election Guide}\label{our-2020-election-guide}}

Updated ~Sept. 8, 2020

\begin{center}\rule{0.5\linewidth}{\linethickness}\end{center}

\begin{itemize}
\item ~
  \hypertarget{the-latest}{%
  \subsection{The Latest}\label{the-latest}}

  \begin{itemize}
  \item
    The campaign
    \href{https://www.nytimes3xbfgragh.onion/live/2020/09/08/us/trump-vs-biden?action=click\&pgtype=Article\&state=default\&region=BELOW_MAIN_CONTENT\&context=storylines_guide}{shifts
    to a higher gear this week}, with President Trump set to visit
    Florida and North Carolina today and Joseph R. Biden heading to
    Michigan tomorrow.
  \end{itemize}
\item ~
  \hypertarget{how-to-win-270}{%
  \subsection{How to Win 270}\label{how-to-win-270}}

  \begin{itemize}
  \item
    Joe Biden and Donald Trump need 270 electoral votes to reach the
    White House. Try building
    \href{https://www.nytimes3xbfgragh.onion/interactive/2020/us/elections/election-states-biden-trump.html?action=click\&pgtype=Article\&state=default\&region=BELOW_MAIN_CONTENT\&context=storylines_guide}{your
    own coalition of battleground states}~to see potential outcomes.
  \end{itemize}
\item ~
  \hypertarget{voting-by-mail}{%
  \subsection{Voting by Mail}\label{voting-by-mail}}

  \begin{itemize}
  \item
    Will you have enough time to vote by mail in your state? Yes, but
    it's risky to procrastinate.
    \href{https://www.nytimes3xbfgragh.onion/interactive/2020/08/31/us/politics/vote-by-mail-deadlines.html?action=click\&pgtype=Article\&state=default\&region=BELOW_MAIN_CONTENT\&context=storylines_guide}{Check
    your state's deadline.}
  \item
    \href{https://www.nytimes3xbfgragh.onion/interactive/2020/us/elections/joe-biden.html?action=click\&pgtype=Article\&state=default\&region=BELOW_MAIN_CONTENT\&context=storylines_guide}{}

    \hypertarget{joe-biden}{%
    \section{Joe Biden}\label{joe-biden}}

    \hypertarget{democrat}{%
    \subsection{Democrat}\label{democrat}}

    \href{https://www.nytimes3xbfgragh.onion/interactive/2020/us/elections/donald-trump.html?action=click\&pgtype=Article\&state=default\&region=BELOW_MAIN_CONTENT\&context=storylines_guide}{}

    \hypertarget{donald-trump}{%
    \section{Donald Trump}\label{donald-trump}}

    \hypertarget{republican}{%
    \subsection{Republican}\label{republican}}
  \end{itemize}
\item
  \hypertarget{keep-up-with-our-coverage}{%
  \subsection{Keep Up With Our
  Coverage}\label{keep-up-with-our-coverage}}

  \begin{itemize}
  \item
    Get an
    \href{https://www.nytimes3xbfgragh.onion/newsletters/politics?action=click\&pgtype=Article\&state=default\&region=BELOW_MAIN_CONTENT\&context=storylines_guide}{email}~recapping
    the day's news
  \item
    Download our mobile app on
    \href{https://apps.apple.com/us/app/nytimes/id284862083?ls=1\&mat_click_id=5c79ae7455014fd1bd66b5610c05b8f2-20191112-16948\&referrer=mat_click_id\%3D5c79ae7455014fd1bd66b5610c05b8f2-20191112-16948\%26link_click_id\%3D722930677036718082}{iOS}~and
    \href{http://a.localytics.com/android?id=com.nytimes.android\&referrer=utm_source\%3Dother_nyt_mobile_web\%26utm_medium\%3DWeb\%2520page\%26utm_term\%3DGeneral\%2520Mobile\%2520Page\%26utm_campaign\%3DNYT\%2520Mobile\%2520General\%2520Page}{Android}~and
    turn on Breaking News and Politics alerts
  \end{itemize}
\end{itemize}

Advertisement

\protect\hyperlink{after-bottom}{Continue reading the main story}

\hypertarget{site-index}{%
\subsection{Site Index}\label{site-index}}

\hypertarget{site-information-navigation}{%
\subsection{Site Information
Navigation}\label{site-information-navigation}}

\begin{itemize}
\tightlist
\item
  \href{https://help.nytimes3xbfgragh.onion/hc/en-us/articles/115014792127-Copyright-notice}{©~2020~The
  New York Times Company}
\end{itemize}

\begin{itemize}
\tightlist
\item
  \href{https://www.nytco.com/}{NYTCo}
\item
  \href{https://help.nytimes3xbfgragh.onion/hc/en-us/articles/115015385887-Contact-Us}{Contact
  Us}
\item
  \href{https://www.nytco.com/careers/}{Work with us}
\item
  \href{https://nytmediakit.com/}{Advertise}
\item
  \href{http://www.tbrandstudio.com/}{T Brand Studio}
\item
  \href{https://www.nytimes3xbfgragh.onion/privacy/cookie-policy\#how-do-i-manage-trackers}{Your
  Ad Choices}
\item
  \href{https://www.nytimes3xbfgragh.onion/privacy}{Privacy}
\item
  \href{https://help.nytimes3xbfgragh.onion/hc/en-us/articles/115014893428-Terms-of-service}{Terms
  of Service}
\item
  \href{https://help.nytimes3xbfgragh.onion/hc/en-us/articles/115014893968-Terms-of-sale}{Terms
  of Sale}
\item
  \href{https://spiderbites.nytimes3xbfgragh.onion}{Site Map}
\item
  \href{https://help.nytimes3xbfgragh.onion/hc/en-us}{Help}
\item
  \href{https://www.nytimes3xbfgragh.onion/subscription?campaignId=37WXW}{Subscriptions}
\end{itemize}
