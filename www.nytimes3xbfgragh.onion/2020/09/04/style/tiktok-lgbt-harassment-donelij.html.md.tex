Sections

SEARCH

\protect\hyperlink{site-content}{Skip to
content}\protect\hyperlink{site-index}{Skip to site index}

\href{https://www.nytimes3xbfgragh.onion/section/style}{Style}

\href{https://myaccount.nytimes3xbfgragh.onion/auth/login?response_type=cookie\&client_id=vi}{}

\href{https://www.nytimes3xbfgragh.onion/section/todayspaper}{Today's
Paper}

\href{/section/style}{Style}\textbar{}Are Influencers Responsible for
the Behavior of Their Followers?

\url{https://nyti.ms/2DANl4A}

\begin{itemize}
\item
\item
\item
\item
\item
\end{itemize}

Advertisement

\protect\hyperlink{after-top}{Continue reading the main story}

Supported by

\protect\hyperlink{after-sponsor}{Continue reading the main story}

\hypertarget{are-influencers-responsible-for-the-behavior-of-their-followers}{%
\section{Are Influencers Responsible for the Behavior of Their
Followers?}\label{are-influencers-responsible-for-the-behavior-of-their-followers}}

Fan armies are harassing gay and trans people on TikTok.

\href{https://www.nytimes3xbfgragh.onion/by/taylor-lorenz}{\includegraphics{https://static01.graylady3jvrrxbe.onion/images/2020/03/18/reader-center/author-taylor-lorenz/author-taylor-lorenz-thumbLarge.png}}

By \href{https://www.nytimes3xbfgragh.onion/by/taylor-lorenz}{Taylor
Lorenz}

\begin{itemize}
\item
  Sept. 4, 2020
\item
  \begin{itemize}
  \item
  \item
  \item
  \item
  \item
  \end{itemize}
\end{itemize}

\includegraphics{https://static01.graylady3jvrrxbe.onion/images/2020/09/04/style/chris-tiktok/oakImage-1599246651127-articleLarge.jpg?quality=75\&auto=webp\&disable=upscale}

Chris smiles widely as a video of Miso Chan, a transgender creator
presenting as a young woman with pink hair, plays next to him. A song
with the lyrics ``Now I know what's real and what is fake,'' loops in
the background.

Then Miso Chan rips off their wig, pulls tissues out of their shirt and
is revealed to present as male. Chris's face drops to a wide-eyed
deadpan expression.

The video has been viewed over 1.7 million times --- and its comment
section is filled with cruel remarks.

``This is why I have trust issues nowadays,'' one person replied.
``Imagine what it's parents thought,'' said another.

Chris, who is 17 and uses the name @Donelij online, has grown an
enormous audience on TikTok. Before his account was banned on Tuesday he
had amassed more than 2.5 million followers. Another one of his accounts
had more than 2.2 million followers, but, on Friday afternoon, both of
Chris's backup TikTok accounts were banned as well. (``This account was
banned due to multiple community guideline violations,'' read a banner
atop the account.)

``It's stressful,'' he said by phone. ``That's how I got my money.''

Chris's account is known in the TikTok community as a reaction account,
an account where someone creates reaction videos that appear alongside
other TikToks. Reaction and commentary TikTok channels are a
\href{https://www.nytimes3xbfgragh.onion/2020/07/02/style/tati-devin-tiktok.html}{growing
niche} on the platform and have been taking off in recent months.

Nearly all of Chris's videos follow the same format: a video loops to
his right, he smiles, sometimes gives a thumbs up, then something
happens in the video and his smile drops. The majority of Chris's videos
are reactions to anodyne moments. In one, his smile drops when a man
slams a brick of tofu in his own face; in another it's when cockroaches
appear onscreen.

Some of his videos, however, feature reactions to LGBTQ creators. He has
a shocked expression when men put on skirts, when a man sucks on a
straw, or when trans people reveal transformations over time.

Even though he never says a word, thousands of people have called them
out for being homophobic. Young gay and trans TikTokers who have been
featured in Chris's reactions report they have suffered vicious
harassment from commenters and in messages. Some have deleted their
accounts.

Chris said he had no ill intent with his videos, and said, before his
accounts were banned, that he would stop duetting members of the LGBTQ
community.

``I want people to know I'm not homophobic or transphobic,'' he said.
``The facial expressions I make in my videos are jokes. I myself didn't
think of them as offensive, but I can see how people would take it that
way and I don't want them to feel that way.''

Image

A screenshot from Chris's TikTok from late July.

Rob Anderson, 32, a TikTok creator who is a gay man, ended up in the
cross hairs after calling attention to Chris's videos in a video of his
own. Immediately after posting, Mr. Anderson was inundated with a stream
of gay slurs, death threats and threats to his family.

``It's a vicious, intense relentless form of harassment and it's
endless,'' said Mr. Anderson. ``These people go through all of your
social channels, find any information about you --- they sent gay slurs
to my agent. It doesn't stop, it doesn't go away one day and leave.''

Any creator who speaks out against Chris's content inevitably receives a
wave of threats and harassment. After Toby Phillips, 20, posted a
soft-spoken '90s-style PSA video asking Chris to take responsibility for
his fans' behavior, he too suffered for it.

Mr. Phillips was met with death threats and doxxing attempts.

As the controversy began trending on Thursday, popular creators began to
weigh in, including the children of several politicians and celebrities.

Cisco Hernandez, a 15-year-old high school student in San Diego, was
dismayed to see that big creators he followed were expressing support
for homophobic content. He posted a video about his disappointment and
was immediately met with an onslaught of harassment himself. Mr.
Hernandez is not a full-time TikTok creator. He has a very small
following and was shocked by the vitriol he encountered. ``People DMed
me telling me to kill myself, a lot of other things,'' including
homophobic slurs, he said. Mr. Hernandez identifies as bisexual.

Other large TikTok creators began issuing threats to LGBTQ creators for
speaking out. In an Instagram Live on Thursday night,
\href{https://www.tiktok.com/@freekdagemini?lang=en}{Freek Da Gemini}, a
21-year-old TikTok creator with more than 750,000 followers, issued
threats to a 17-year-old, calling him a slew of homophobic slurs. ``I'm
saying slurs, yeah I said a slur,'' he said on his livestream, adding
that the young man could expect to have ``blood leaking out of his
head'' if he encountered him in person.

Chris's fans say that critiques of his content are overblown. ``This has
to be the worst display of cancel culture that I've ever seen,'' a
YouTube drama commentator known as Relex said in a
\href{https://www.youtube.com/watch?v=T3s0uaOfpPo\&feature=youtu.be}{video}
on Thursday. ``He's getting canceled for the facial expressions he's
making in his videos. How are you getting offended by someone who
doesn't even talk?''

Many flooded the comment sections of LGBTQ creators with the snowflake
emoji.

Chris said that he had been on the receiving end of harassment himself
as anger about his content built on the app. ``I get called the N-word,
monkey, in my DMs, people saying they're going to kill me when they see
me,'' he said. Many TikTokers have encouraged their fans to flood
Chris's comments with abusive messages, others have attacked members of
his family and have tried to get him thrown out of school.

He also said he was frustrated with his fans' behavior. He said that
he's spoken out several times on his TikTok account asking people to
stop making homophobic comments, but that ultimately he can't control
the millions of people who follow him, and that they interpret his pleas
to ``spread positivity'' as a covert directive to attack.

``A percentage of my followers are trolls and I feel like they do this
because they like getting adrenaline on the internet,'' he said. ``I'll
just keep on trying to get my fans to change their ways. I'm trying my
best to get my fans to stop doing what they're doing.''

Many members of the LGBTQ community respond that, whether Chris intended
to or not, he built an audience of homophobic followers by posting
homophobic content and it's now his responsibility to manage that
audience.

``They say Chris is not responsible for his followers,'' said Mr.
Anderson. ``To that I say: Of course he is. He's cultivated this group
of people with this content. The people who follow him are the people
who enjoy the content he's putting out and his content is clearly
anti-gay and homophobic. If you have a large following, it's your
responsibility to make sure people aren't getting hurt by what you
post.''

In this experience, many young creators are realizing how inadequate
TikTok can be at protecting users from harassment. Hateful videos
snowball quickly as they gain traction on the app's all-powerful ``For
You'' page. There's no mechanism to mass report or mass block a user's
entire following. Comment controls are limited and glitchy; the
reporting process can be slow. Often, harassment campaigns are
instigated with a wink and a nod.

``We are committed to promoting a safe and positive app environment for
our users. Our Community Guidelines outline behavior that is not
acceptable on the platform, and we take action against behavior that
violates those policies, including by removing content or accounts. We
also offer a number of features to help users control their online
experience, including options to report inappropriate content, limit and
filter comments and block users,'' a spokesperson for TikTok said in a
statement.

``I'm 32, I'm not on this app to have drama with teenagers,'' Mr.
Anderson said, ``I'm confident about who I am. But there are queer
people who are 15 or 16, exploring their gender. They are just trying to
explore who they are.''

Advertisement

\protect\hyperlink{after-bottom}{Continue reading the main story}

\hypertarget{site-index}{%
\subsection{Site Index}\label{site-index}}

\hypertarget{site-information-navigation}{%
\subsection{Site Information
Navigation}\label{site-information-navigation}}

\begin{itemize}
\tightlist
\item
  \href{https://help.nytimes3xbfgragh.onion/hc/en-us/articles/115014792127-Copyright-notice}{©~2020~The
  New York Times Company}
\end{itemize}

\begin{itemize}
\tightlist
\item
  \href{https://www.nytco.com/}{NYTCo}
\item
  \href{https://help.nytimes3xbfgragh.onion/hc/en-us/articles/115015385887-Contact-Us}{Contact
  Us}
\item
  \href{https://www.nytco.com/careers/}{Work with us}
\item
  \href{https://nytmediakit.com/}{Advertise}
\item
  \href{http://www.tbrandstudio.com/}{T Brand Studio}
\item
  \href{https://www.nytimes3xbfgragh.onion/privacy/cookie-policy\#how-do-i-manage-trackers}{Your
  Ad Choices}
\item
  \href{https://www.nytimes3xbfgragh.onion/privacy}{Privacy}
\item
  \href{https://help.nytimes3xbfgragh.onion/hc/en-us/articles/115014893428-Terms-of-service}{Terms
  of Service}
\item
  \href{https://help.nytimes3xbfgragh.onion/hc/en-us/articles/115014893968-Terms-of-sale}{Terms
  of Sale}
\item
  \href{https://spiderbites.nytimes3xbfgragh.onion}{Site Map}
\item
  \href{https://help.nytimes3xbfgragh.onion/hc/en-us}{Help}
\item
  \href{https://www.nytimes3xbfgragh.onion/subscription?campaignId=37WXW}{Subscriptions}
\end{itemize}
