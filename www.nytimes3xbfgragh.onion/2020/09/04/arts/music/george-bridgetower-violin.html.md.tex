Sections

SEARCH

\protect\hyperlink{site-content}{Skip to
content}\protect\hyperlink{site-index}{Skip to site index}

\href{https://www.nytimes3xbfgragh.onion/section/arts/music}{Music}

\href{https://myaccount.nytimes3xbfgragh.onion/auth/login?response_type=cookie\&client_id=vi}{}

\href{https://www.nytimes3xbfgragh.onion/section/todayspaper}{Today's
Paper}

\href{/section/arts/music}{Music}\textbar{}The Black Violinist Who
Inspired Beethoven

\url{https://nyti.ms/3gZInMb}

\begin{itemize}
\item
\item
\item
\item
\item
\end{itemize}

\href{https://www.nytimes3xbfgragh.onion/spotlight/at-home?action=click\&pgtype=Article\&state=default\&region=TOP_BANNER\&context=at_home_menu}{At
Home}

\begin{itemize}
\tightlist
\item
  \href{https://www.nytimes3xbfgragh.onion/2020/09/07/travel/route-66.html?action=click\&pgtype=Article\&state=default\&region=TOP_BANNER\&context=at_home_menu}{Cruise
  Along: Route 66}
\item
  \href{https://www.nytimes3xbfgragh.onion/2020/09/04/dining/sheet-pan-chicken.html?action=click\&pgtype=Article\&state=default\&region=TOP_BANNER\&context=at_home_menu}{Roast:
  Chicken With Plums}
\item
  \href{https://www.nytimes3xbfgragh.onion/2020/09/04/arts/television/dark-shadows-stream.html?action=click\&pgtype=Article\&state=default\&region=TOP_BANNER\&context=at_home_menu}{Watch:
  Dark Shadows}
\item
  \href{https://www.nytimes3xbfgragh.onion/interactive/2020/at-home/even-more-reporters-editors-diaries-lists-recommendations.html?action=click\&pgtype=Article\&state=default\&region=TOP_BANNER\&context=at_home_menu}{Explore:
  Reporters' Google Docs}
\end{itemize}

Advertisement

\protect\hyperlink{after-top}{Continue reading the main story}

Supported by

\protect\hyperlink{after-sponsor}{Continue reading the main story}

\hypertarget{the-black-violinist-who-inspired-beethoven}{%
\section{The Black Violinist Who Inspired
Beethoven}\label{the-black-violinist-who-inspired-beethoven}}

George Bridgetower, the original dedicatee of the ``Kreutzer'' Sonata,
was a charismatic prodigy but faded into history.

\includegraphics{https://static01.graylady3jvrrxbe.onion/images/2020/09/06/arts/06bridgetower-1NEW/06bridgetower-1NEW-articleLarge.jpg?quality=75\&auto=webp\&disable=upscale}

By Patricia Morrisroe

\begin{itemize}
\item
  Sept. 4, 2020
\item
  \begin{itemize}
  \item
  \item
  \item
  \item
  \item
  \end{itemize}
\end{itemize}

Six months after Beethoven contemplated suicide, confessing his despair
over his increasing deafness in the 1802 document known as the
Heiligenstadt Testament, he was carousing in taverns with a charismatic
new comrade, George Polgreen Bridgetower. This biracial violinist had
recently arrived in Vienna, and inspired one of Beethoven's most famous
and passionate pieces, the ``Kreutzer'' Sonata.

Beethoven even dedicated the sonata to Bridgetower. But the irritable
composer --- who would later remove the dedication to Napoleon from his
Third Symphony --- eventually took it back.

While Napoleon didn't need Beethoven to secure his place in history,
this snub reduced Bridgetower to near obscurity. Though his name was
included in Anton Schindler's 1840 biography of Beethoven, he was
described inaccurately as ``an American sea captain.'' Like so many
Black artists prominent in their lifetimes, he has been largely
forgotten by a history that belongs to those who control the narrative.

Bridgetower was born on Aug. 13, 1778, in eastern Poland, and christened
Hieronymus Hyppolitus de Augustus. His father, Joanis Fredericus de
Augustus, was of African descent; his mother, Maria Schmid, was
German-Polish, making Bridgetower what was then known as a mulatto, a
person of mixed race. (The poet Rita Dove's
\href{https://www.nytimes3xbfgragh.onion/2009/04/03/books/03dove.html}{2008
book ``Sonata Mulattica,''} an imagined chronicle of Bridgetower's life,
has helped raise his profile a bit in recent years.)

Bridgetower's father --- who took the name Frederick, and sometimes went
by others --- was the driving force behind his son's career. Handsome,
charming and fluent in multiple languages, Frederick was a natural
storyteller with a flair for promotion; he claimed that his father had
been an African prince unofficially adopted by a Dutch sea captain, was
promised diamonds and gold dust, and then sold into slavery, surviving a
shipwreck in the process. The father married an African woman and wound
up in Barbados, where Frederick was born; the name Bridgetower was
likely derived from the island's capital, Bridgetown.

It's unclear how Frederick wound up in Poland, but the historian William
Hart wrote in
\href{https://www.researchgate.net/publication/319710845_New_Light_on_George_Bridgtower}{a
2017 article in The Musical Times} that young Bridgetowers's godparents
were members of the noble Radziwill family; Frederick, and possibly his
wife, may have been in their service. The couple and their son soon
moved to Austria, where Frederick, known as ``the Moor,'' worked as a
page to Prince Nikolaus Esterhazy. The music-loving prince maintained
his own orchestra at his palace in Eisenstadt, where Haydn was court
composer. (George Bridgetower was later touted as a pupil of Haydn's,
but it's unclear if he ever studied with the master.)

Bridgetower's public debut was long thought to have taken place in Paris
in 1789. But Mr. Hart discovered an advertisement in a Frankfurt
newspaper promoting a concert by ``Hieronymus August Bridgetown,'' the
``son of a Moor,'' in April 1786, when the boy would have been just
seven. It noted that he had already played for Emperor Joseph II.

The Bridgetowns, as they were then known, lived for a time in Mainz, an
important musical center, where Maria gave birth to another son, who
would later become a cellist. Frederick, leaving his wife and younger
child behind, took on tour his elder son, who, billed as a ``young Negro
of the Colonies,'' performed a violin concerto by Giornovichi in the
prominent Concert Spirituel series in Paris in 1789.

``His talent, as genuine as it is precocious, is one of the best replies
one can give to the philosophers who wish to deprive those of his nation
and his color the faculty of distinguishing themselves in the arts,''
said a review in Le Mercure de France.

After several more concerts in Paris, including one attended by Thomas
Jefferson, the Bridgetowers --- as they then called themselves --- left
for England, where the family created a sensation.

With Oriental-inspired clothing in vogue, Frederick played up his
presumed exoticism by wearing flowing Turkish robes. Everyone wanted to
meet this ``African prince'' and his prodigy --- whose name had now
become George. By the fall of 1789, Frederick had arranged for his son
to play before King George III and Queen Charlotte, as well as the
Prince of Wales, later George IV.

George induced ``general astonishment'' playing in Bath, according to
the Bath Morning Post. At 11, he made his London debut with a
Giornovichi concerto between the first two parts of Handel's
``Messiah*.*'' He and his father were often at Carlton House, the town
residence of the Prince of Wales, who organized regular chamber
concerts. On June 2, 1790, the prince sponsored a benefit concert for
Bridgetower and another young artist at the Hanover Square Rooms, the
premier concert venue for fashionable society.

Until then, Frederick had skillfully managed his son's career. But his
behavior turned increasingly self-destructive. At a masquerade attended
by the prince, Frederick dressed as a caricature of a Black slave,
advocating for abolition; this was certainly a worthy cause, but the
stunt served to alienate the elites whose favor he had taken pains to
cultivate. During a performance of ``Messiah,'' ** he shouted for a
repeat of the ``Hallelujah'' chorus, and, after a struggle, was thrown
out of the theater. There were reports of excessive drinking and
womanizing.

Charlotte Papendiek, a lady-in-waiting to Queen Charlotte and a prolific
journal keeper, wrote that Frederick gambled away his son's money and
treated him so brutally that George sought refuge with the Prince of
Wales at Carlton House. Frederick was committed to an asylum before
being sent back to Germany by the prince, who took 12-year old George
under his protection.

The prince gave him the opportunity to learn from the finest musicians
in London. He studied composition, theory and piano with Thomas Attwood
and violin with both François-Hippolyte Barthélémon and Giornovichi. He
formed a close relationship with Giovanni Battista Viotti, a violinist
and composer whose confident, daring style would influence his own.

Over the next decade, Bridgetower would play in nearly 50 public
concerts with leading orchestras and musicians, including Haydn and the
double-bass virtuoso Domenico Dragonetti. He was the first violinist of
the Prince of Wales's band; the organist and composer Samuel Wesley
wrote that Bridgetower was ``justly ranked with the very first masters
of the violin.''

After visiting his ailing mother in Dresden, Bridgetower arrived in
Vienna in early April 1803. He had been invited by Prince Lobkowitz, one
of Beethoven's patrons, to play that composer's quartets.

\includegraphics{https://static01.graylady3jvrrxbe.onion/images/2020/09/06/arts/06bridgetower-2/06bridgetower-2-articleLarge.jpg?quality=75\&auto=webp\&disable=upscale}

Beethoven and Bridgetower formed an instant bond. The composer, then 32,
may have recognized himself in the 24-year-old violinist. Beethoven had
been nicknamed the Spaniard for his swarthy complexion, and engravings
of the two men show a marked resemblance. They also had in common
abusive fathers with vested interests in their careers, as well as the
ability to thrill audiences with their astonishing talents.

After hearing Bridgetower play, Beethoven not only agreed to participate
in a concert for him at the Augarten, but also decided to write
something for them to perform together. He had already started sketching
out the first two movements of a violin sonata, to accompany a
previously discarded finale. He now began to compose with Bridgetower in
mind, as the two men stayed up nights drinking and acting like
teenagers. Though Bridgetower was described as melancholic, he could
also be high-spirited and ribald. He brought out Beethoven's
freewheeling, bawdy side.

The concert had been planned for May 22, 1803, but since the sonata
wasn't ready, it was postponed until the 24th. At 4:30 that morning,
Beethoven instructed his pupil, Ferdinand Ries, to copy out the first
two movements for the violinist. Ries managed only the first, and the
piano part was still in sketch form. Beethoven and Bridgetower took the
stage for the morning concert, having never rehearsed the piece.
Bridgetower was sight-reading.

Beethoven had given Bridgetower an opening solo that began with an
explosive declaration, moving into a fiery, sensual dialogue. At one
point, Bridgetower surprised Beethoven by imitating and then expanding
on a short piano cadenza in the first movement. Beethoven, jumping up,
hugged him, crying, ``My dear boy! Once more!''

After the performance, Beethoven presented Bridgetower his tuning fork
and wrote a dedication on the score: ``Sonata mulattica composta per il
mulatto Brischdauer, gran pazzo e compositore mulattico'' (``Mulatto
sonata composed for the mulatto Bridgetower, great lunatic and mulatto
composer'').

Tolstoy wrote about the unsettling first movement in his novella ``The
Kreutzer Sonata,'' whose protagonist, after hearing his wife play the
piece with her violin teacher, stabs her to death in a jealous rage.
Beethoven didn't do anything that extreme, but after Bridgetower made a
rude comment about a woman Beethoven admired, the two men quarreled and
Beethoven took back the dedication.

Image

The ``Kreutzer'' Sonata ended up dedicated to someone who disliked and
never played it.Credit...DEA/A. DAGLI ORTI, via De Agostini, via Getty
Images

When the sonata was published, it instead bore the name of the French
violinist Rudolphe Kreutzer. Beethoven had been thinking of moving to
Paris, and dedicating the piece to Kreutzer was a calculated political
move. What Beethoven didn't know was that Kreutzer disliked his music;
Kreutzer described the sonata as ``outrageously unintelligible'' and
never played it.

Bridgetower returned to London and continued to perform, enjoying the
patronage of the Prince of Wales. On May 23, 1805, he participated in a
concert in the Hanover Rooms, along with his brother, who played a
Romberg cello concerto. Their father had also come back to England,
where he was arrested and thrown in jail for vagrancy.

In 1811, Bridgetower received a master's degree in music from Cambridge
University and became a member of the Royal Philharmonic Society. Five
years later, he married Mary Leake, the daughter of a prosperous cotton
manufacturer; they had two daughters. One died in infancy, and he grew
estranged from the other. He and his wife separated in 1824.

Little is known about Bridgetower's later years; at some point, he seems
to have stopped performing, making his living as a piano teacher in Rome
and Paris. In an 1847 letter to Madame de Fauché, a fellow musician, he
makes a joking but telling reference to his biracial identity: ``If the
bearer of this letter is fortunate to find you, favor me by having your
message conveyed to him who is not fair enough to be `my tiger,' nor
`dark enough' to be `my Friday,' but is my long-tried honest Caliban.''
The allusion to the half-human, half-beast character in Shakespeare's
``Tempest'' ** is a poignant one: When his island is suddenly occupied,
Caliban is enslaved.

Bridgetower died on Feb. 29, 1860, in a house on a small back street in
south London; he was buried at Kensal Green Cemetery. The death
certificate identifies him as a ``gentleman.'' By then, Beethoven had
been gone for 32 years.

It's unknown if Bridgetower ever played the ``Kreutzer'' Sonata again,
or if he was in contact with Beethoven after their rift. All we know is
that on May 24, 1803, two brilliant performers dazzled a crowd with
their high-wire virtuosity. One of them entered history.

Patricia Morrisoe is the author of the novel ``The Woman in the
Moonlight.''

Advertisement

\protect\hyperlink{after-bottom}{Continue reading the main story}

\hypertarget{site-index}{%
\subsection{Site Index}\label{site-index}}

\hypertarget{site-information-navigation}{%
\subsection{Site Information
Navigation}\label{site-information-navigation}}

\begin{itemize}
\tightlist
\item
  \href{https://help.nytimes3xbfgragh.onion/hc/en-us/articles/115014792127-Copyright-notice}{©~2020~The
  New York Times Company}
\end{itemize}

\begin{itemize}
\tightlist
\item
  \href{https://www.nytco.com/}{NYTCo}
\item
  \href{https://help.nytimes3xbfgragh.onion/hc/en-us/articles/115015385887-Contact-Us}{Contact
  Us}
\item
  \href{https://www.nytco.com/careers/}{Work with us}
\item
  \href{https://nytmediakit.com/}{Advertise}
\item
  \href{http://www.tbrandstudio.com/}{T Brand Studio}
\item
  \href{https://www.nytimes3xbfgragh.onion/privacy/cookie-policy\#how-do-i-manage-trackers}{Your
  Ad Choices}
\item
  \href{https://www.nytimes3xbfgragh.onion/privacy}{Privacy}
\item
  \href{https://help.nytimes3xbfgragh.onion/hc/en-us/articles/115014893428-Terms-of-service}{Terms
  of Service}
\item
  \href{https://help.nytimes3xbfgragh.onion/hc/en-us/articles/115014893968-Terms-of-sale}{Terms
  of Sale}
\item
  \href{https://spiderbites.nytimes3xbfgragh.onion}{Site Map}
\item
  \href{https://help.nytimes3xbfgragh.onion/hc/en-us}{Help}
\item
  \href{https://www.nytimes3xbfgragh.onion/subscription?campaignId=37WXW}{Subscriptions}
\end{itemize}
