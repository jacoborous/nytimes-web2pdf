\href{/section/arts}{Arts}\textbar{}This Labor Day, These Workers Are
Trying to Stay Afloat

\url{https://nyti.ms/3bs62E8}

\begin{itemize}
\item
\item
\item
\item
\item
\end{itemize}

\href{https://www.nytimes3xbfgragh.onion/spotlight/at-home?action=click\&pgtype=Article\&state=default\&region=TOP_BANNER\&context=at_home_menu}{At
Home}

\begin{itemize}
\tightlist
\item
  \href{https://www.nytimes3xbfgragh.onion/2020/09/07/travel/route-66.html?action=click\&pgtype=Article\&state=default\&region=TOP_BANNER\&context=at_home_menu}{Cruise
  Along: Route 66}
\item
  \href{https://www.nytimes3xbfgragh.onion/2020/09/04/dining/sheet-pan-chicken.html?action=click\&pgtype=Article\&state=default\&region=TOP_BANNER\&context=at_home_menu}{Roast:
  Chicken With Plums}
\item
  \href{https://www.nytimes3xbfgragh.onion/2020/09/04/arts/television/dark-shadows-stream.html?action=click\&pgtype=Article\&state=default\&region=TOP_BANNER\&context=at_home_menu}{Watch:
  Dark Shadows}
\item
  \href{https://www.nytimes3xbfgragh.onion/interactive/2020/at-home/even-more-reporters-editors-diaries-lists-recommendations.html?action=click\&pgtype=Article\&state=default\&region=TOP_BANNER\&context=at_home_menu}{Explore:
  Reporters' Google Docs}
\end{itemize}

\includegraphics{https://static01.graylady3jvrrxbe.onion/images/2020/09/04/arts/03laborday-workers-14/merlin_176419530_2be6aa70-2f52-46a2-ac1d-87cd9c37d29c-articleLarge.jpg?quality=75\&auto=webp\&disable=upscale}

Sections

\protect\hyperlink{site-content}{Skip to
content}\protect\hyperlink{site-index}{Skip to site index}

\hypertarget{this-labor-day-these-workers-are-trying-to-stay-afloat}{%
\section{This Labor Day, These Workers Are Trying to Stay
Afloat}\label{this-labor-day-these-workers-are-trying-to-stay-afloat}}

The coronavirus pandemic has brought various hardships. An artist,
bookseller, comedian and five others share their stories of how they are
coping with all the uncertainty.

``April was dark'': On the rooftop of Baseera Khan's apartment building
in Crown Heights, Brooklyn. The artist not only suffered from the
coronavirus, but also struggled financially after exhibitions were put
on hold and teaching gigs evaporated.Credit...Maridelis Morales Rosado
for The New York Times

Supported by

\protect\hyperlink{after-sponsor}{Continue reading the main story}

This Labor Day Weekend is unlike any other in recent memory: Broadway
and most of the city's live performance venues have been closed for
nearly six months; museums have only recently begun to reopen; and
unemployment reached alarming levels in April.

The coronavirus pandemic has ravaged all parts of the economy, and
culture workers are among the hardest hit. Yet some have managed to keep
their jobs --- and even thrive --- while others are still struggling or
have pivoted to new roles. Now more than ever we wanted to reflect on
those people who have devoted their lives to the arts in some fashion.
They've soldiered on, amid the shutdown, to create some semblance of
normalcy --- whether recommending a book, securing the grounds of an
iconic cultural institution or entertaining children. Here are some of
their stories. --- \emph{Nicole Herrington, Weekend Arts Editor}

\hypertarget{the-shoemaker}{%
\subsection{The Shoemaker}\label{the-shoemaker}}

\includegraphics{https://static01.graylady3jvrrxbe.onion/images/2020/09/04/arts/04labordayworkers-1/merlin_176297934_9ba61330-e8f9-4a3e-b63e-954cdce08340-articleLarge.jpg?quality=75\&auto=webp\&disable=upscale}

\textbf{Phil LaDuca, 65, founder and creator of LaDuca Shoes in
Manhattan}

Before the pandemic halted live performance, LaDuca Shoes was, in the
words of its creator, Phil LaDuca, ``ready to explode on the scene.''

His flexible character shoes, which grace --- and save --- the prized
feet of Rockettes as well as Broadway dancers and actors, were to be
seen in an array of productions this year, among them ``The Eternals,''
the Marvel film whose release date has been moved to next year, and
Andrew Lloyd Webber's ``Cinderella'' in London (pushed to next March).
Also on the horizon? ``The Music Man,'' with Hugh Jackman and Sutton
Foster, on Broadway.

Speaking recently from Los Angeles, where he lives with his wife, Fon,
his voice was a mix of wonderment and dismay. The day that Broadway
shows were shuttered, the couple were in New York for the opening of
\href{https://www.nytimes3xbfgragh.onion/2020/02/27/theater/six-broadway.html}{the
new musical ``Six.''} He had designed the boots for the six actresses
who play the ill-fated wives of Henry VIII.

Image

``I went through a two-week period of funk and depression,'' Phil LaDuca
said. ``And then I said, no, we're going to fight.''Credit...Michelle
Groskopf for The New York Times

``We were two hours away from the red carpet for opening night when I
received the call,'' he said. ``Fon and I were desperate to get out of
New York --- we heard rumors that they were going to stop flights. I was
doing my Kurt Russell-Snake impersonation from `Escape From New York.'''

But while LaDuca did indeed escape, he hasn't stopped working. In the
early days of the quarantine, LaDuca Shoes offered dance classes on
Instagram Live. ``I always want to stay positive,'' he said. ``We did it
out of love for the community.''

He added: ``I went through a two-week period of funk and depression. And
then I said, no, we're going to fight. Let's not sit on the sidelines!''

The classes ended in August --- dancers, he said, have many more options
now --- and the shop, on 45th Street between 10th and 11th Avenues, has
reopened for fittings and pickups. This summer brought a significant
change at LaDuca Shoes: the LaDuca Palette, which features four new
colors to accommodate more skin tones. He said the inspiration came from
the Black Lives Matter movement, following the killing of George Floyd.
He said it made him reflect on his own white privilege.

LaDuca, who arrived in New York from the South Side of Chicago in 1979
with a pair of suitcases and \$800 in his wallet, had no help in the
creation of his footwear and shop. The movement made him question
``would a Black kid from the South Side of Chicago have had the same
opportunities as me?'' he said. ``My answer is no. It made me realize I
wasn't doing enough.''

And LaDuca's aim is to do everything he can for dancers. He was one
himself. A year after he moved to New York, LaDuca was dancing on
Broadway in Agnes de Mille's ``Brigadoon.'' Multiple injuries --- his
own --- led him to start LaDuca Shoes about 20 years ago.

Image

Credit...Michelle Groskopf for The New York Times

Image

Credit...Michelle Groskopf for The New York Times

His favorite story involves a young woman from North Carolina who was in
town for a Radio City Rockettes intensive. She left the shop with the
Roxie, a hard-sole tap shoe LaDuca designed for the Rockettes, and the
Alexis, ``because you need to bevel,'' he said, referring to the
signature pose. ``The Rockette bevel is what it's all about.''

She returned a year later --- and announced that she was a Rockette.
``That, to me, is what LaDuca shoes \emph{is,''} he said. ``I'm just
concerned with how I'm going to keep evolving and helping dancers.''
\emph{GIA KOURLAS}

\hypertarget{the-bookseller}{%
\subsection{The Bookseller}\label{the-bookseller}}

Image

James Fugate, a co-owner of Eso Won Books in Los Angeles. He says this
Labor Day closes out a summer that saw the greatest period of growth in
the store's 30-year history.Credit...Erik Carter for The New York Times

\textbf{James Fugate, 65, co-owner of Eso Won Books in Los Angeles}

It was the second week of the shutdown in March, and James Fugate, the
co-owner of Eso Won Books in the Leimert Park neighborhood of South Los
Angeles, was recovering from the flu. He thought it could have been
Covid-19, but he couldn't get a test to confirm. Paul Coates, the
founder of Black Classic Press (and father of Ta-Nehisi), had organized
a Zoom conference for hundreds of Black booksellers and publishers to
talk about how they were going to keep their businesses alive in the
pandemic. Fugate joined, but turned off his audio so no one could hear
him coughing.

The event spurred a flurry of marketing for readers around the country
to support Black-owned bookstores, especially now that they couldn't
shop in person. ``Suddenly we started getting 25 orders a day, and that
really helped,'' Fugate said in a masked interview in late August, in
the back office of his store on Degnan Boulevard.

But that boost was nothing compared to what was to come. On June 3, a
week and a half after the police killing of George Floyd, Fugate posted
a notice on Eso Won's website, apologizing for having fallen behind on
filling the ``overwhelming amount of orders'' they'd received that week.
``We never anticipated anything like this,'' he wrote.

Image

``We've always had a cross-section of Los Angeles come to our store,''
James Fugate said. ``Our growth is in part due to our long
record.''Credit...Erik Carter for The New York Times

Soon they were receiving 400 orders in a single day; ``I thought, it's
going to take two days for me to get all those done,'' he said. By the
next day, there were 1,200 more. ``Oh, no, no, no,'' he thought,
``there's no way.''

In Los Angeles, Black-owned bookstores are few and far between. Besides
Eso Won, Fugate said he knew of only two others: Malik Books in nearby
Baldwin Hills, and Reparations Club in Mid-City --- both predominantly
African-American neighborhoods. ``We've always had a cross-section of
Los Angeles come to our store,'' he said. ``Our growth is in part due to
our long record.''

Most foot traffic still consists of South Los Angeles locals, but lately
the majority of pickup orders are being placed by white customers from
all over the county. ``There are some knuckleheads around here who are
upset,'' he said. ``They're like, `Why are all these white people coming
here?'''

This Labor Day closes out a summer that saw the greatest period of
growth in the store's 30-year history. Fugate has been a bookseller all
his life; he discovered his passion for the industry as a teenager, in
copies of Publishers Weekly borrowed from his local library in Detroit.
Now in his 60s, Fugate still processes and fulfills every single online
order himself. (``Sometime in June I said, `Stop answering the phones.
We cannot.''')

But Fugate is grateful for the stress, and for his customers, who've by
and large been understanding of the inevitable delays, and even lost
transactions. He said he's ``disturbed'' to hear other bookstore owners
complaining about the increased demand. ``If you went from doing 20
orders a day to 200 orders a day,'' he said, ``just be happy.''

He also credits the American Booksellers Association for having
developed a bulk ordering program to help stores like Eso Won handle
higher sales volumes. Not all Black booksellers feel welcome ``around
all those white people'' in the A.B.A., he said. ``But to me, you want
to be part of that bookselling community.''

Image

Credit...Erik Carter for The New York Times

Image

Credit...Erik Carter for The New York Times

Since reopening Eso Won's doors on June 1, Fugate and the bookshop's
co-owner, Tom Hamilton, have been worried about the virus; but Hamilton
lets in only a few customers at a time, and asks that they wear masks
and use hand sanitizer before entering. Store hours are now noon to 4
p.m., Monday through Saturday, but the owners are there much longer:
from about 7 a.m. (sometimes earlier) until 5 p.m.

``Since the pandemic, it's like nothing we've ever seen,'' he said. ``We
have so many books, but they all seem to be selling.

``I think that's going to continue for some time,'' he added, referring
to both the steady flux of orders in his inbox and the civil rights
movement that is at least partly fueling them. ``This isn't going to
stop.'' \emph{LAUREN CHRISTENSEN}

\hypertarget{the-artist}{%
\subsection{The Artist}\label{the-artist}}

Image

On the rooftop of Baseera Khan's apartment building in Crown Heights,
Brooklyn. The artist fell ill with the coronavirus this
spring.Credit...Maridelis Morales Rosado for The New York Times

\textbf{Baseera Khan, 40, a Brooklyn-based artist who will be in
residency at the Kitchen this fall}

The artist Baseera Khan decided to film a cooking series for Instagram
called ``Apocalypse Cooking'' in the days after New York began
sheltering in place.

The videos were decidedly tongue-in-cheek, almost parodying Instagram as
a medium (``look at all my toilet paper,'' ``I still have cute nails''),
that provided viewers with easy-to-follow recipes.

```You don't need fancy things to make fancy food' was the theme,'' Khan
said in a recent interview.

On March 26, Khan, who uses the pronoun ``they,'' did a live cooking
session on BRIC Brooklyn's Instagram page. And then, that day, Khan
started feeling the symptoms of Covid-19. Immediately after filming for
BRIC, they started getting the chills.

``So April was dark for me,'' Khan said, and much of their life came to
a halt.

Almost overnight, the four art shows that Khan --- who uses a range of
materials to create installation, collage, sound and performance art ---
had lined up for the rest of the year, including one at the
\href{https://atlantacontemporary.org/press/atlanta-contemporary-announces-decade-by-baseera-khan}{Atlanta
Contemporary} in Georgia and another at the Wexner Center for the Arts
in Ohio, were canceled, postponed or left in limbo. And Khan said
teaching positions evaporated. (In addition to teaching at the summer
Masters of Fine Arts program at the School of Visual Arts, Khan also
taught last year at Virginia Commonwealth University.)

Image

The artist with the sculpture titled ``Seat 23 Blue with
Trimmings.''Credit...Maridelis Morales Rosado for The New York Times

Image

Coping with the coronavirus, isolation and delayed exhibitions ``was all
a really horrific experience,'' Khan says.Credit...Maridelis Morales
Rosado for The New York Times

With barely any money coming in and no health insurance, Khan resorted
to rationing food. ``I'd wake up in the morning and have coffee and a
banana,'' they said. ``And then for a snack, I would have tea and an
apple. I would wait until around six to make a proper meal. Then I would
just go to sleep really early so I didn't have to be hungry.''

Every day, Khan said they would also try to register for unemployment
benefits, calling the state labor department again and again with little
success.

It wasn't until June 1 that they were finally able to register.

``It was all a really horrific experience,'' Khan recalled. ``But then,
at the same time, it felt familiar. Being an artist, you're either
inundated with people and social activity or you're very alone. So
there's an oil-and-water to it. It did feel familiar to isolate myself,
to be honest.''

In the summer months, Khan's circumstances slowly started turning
around.

This fall the New Orleans Museum of Art will screen a film of their
performance last year at the University of Albany. Titled ``Braidrage,''
the work featured Khan scaling a rock- climbing wall that had been
constructed with resin casts of the artist's body parts as the ``rocks''
and a floor-to-ceiling thick black braid as the rope.

Khan also started selling prints of their work on Instagram. And, on
Sept. 8, Khan will begin
\href{https://onscreen.thekitchen.org/baseera-khan}{a six week residency
at the Kitchen} in New York, where they will draw on their experiences
in isolation and suffering from Covid-19 to create artwork in the form
of a TV show.

``I'm doing the best I can in terms of not worrying,'' they said. ``I
don't know any other way.'' \emph{ALISHA HARIDASANI GUPTA}

\hypertarget{the-security-team}{%
\subsection{The Security Team}\label{the-security-team}}

Image

Elrige Shelton, left, and Jervin Archibald at Lincoln Center. ``I
couldn't walk 10 feet without someone asking me a question,'' Archibald
said of the times before the coronavirus shut down. ``Now no one wants
to talk.''Credit...Maridelis Morales Rosado for The New York Times

\textbf{Elrige Shelton, 58, and Jervin Archibald, 46, chiefs of security
at Lincoln Center in Manhattan}

For the two chiefs of security at Lincoln Center for the Performing
Arts, one constant in their lives before the pandemic was daily
conversations with audience members as they patrolled the sprawling
16-acre campus.

Elrige Shelton, the late-shift security chief, remembers frequently
greeting an elderly man who was a regular at the summer dance sessions
known as Midsummer Night Swing in Damrosch Park. The man would always
show up with a much younger dance partner, he recalled.

Shelton's counterpart during the daytime shift, Jervin Archibald, can
picture an older woman to whom he always said ``good morning'' when she
was on her way to work. She would greet him back: ``Good morning, have a
good day!'' He never learned her name, but she was a constant who
disappeared when the pandemic shut down New York City in March.

There were the Lincoln Center patrons who would stop by just to chat
about the plays or operas that they planned to see.

``Sometimes people just need someone to talk to,'' Archibald said. ``And
you're there to give them an ear.''

And then there were the harried ticket holders desperate to get
directions for a theater before showtime.

``I couldn't walk 10 feet without someone asking me a question,'' he
said. ``Now no one wants to talk.''

The Metropolitan Opera House and Lincoln Center Theater haven't hosted
formal performances for more than five months, but the chiefs have been
on duty the entire time, making sure the campus and its remaining
population is safe.

Image

Archibald, left, and Shelton in the South Plaza of Lincoln Center. The
emptiness was hard for Shelton, who was used to monitoring crowds of
several thousand before showtime and afterward. Now Lincoln Center has
reopened the outdoor space to the public. Credit...Maridelis Morales
Rosado for The New York Times

Archibald, 46, who has worked in Lincoln Center's security division for
more than half of his life, starts his shift at 7 a.m., calling roll and
getting his staff up to speed before heading out to patrol the campus.

Shelton, 58, who has worked there for more than 25 years, takes over at
3 p.m. and stays until 11 p.m., when --- during typical times ---
audience members are usually streaming out of the various theaters after
a ballet or concert or play.

Before the pandemic, Shelton took an hour-and-a-half bus ride from his
hometown in Pennsylvania to Manhattan; now, the bus no longer runs, so
he drives instead. In March, when the pandemic was first bearing down on
the city, he remembers that much of his job was making sure his security
staff felt safe.

``At first it was a little frightening,'' he said. ``We didn't want the
staff to be fearful of coming to work.''

Then, he had to usher them through a difficult furlough period. As of
July 1, out of Lincoln Center's roughly 400 full-time employees, about
30 percent were on furlough.

For months, Lincoln Center's campus was entirely shut down. Neighbors
were no longer permitted to stroll around or sit by the fountain. The
emptiness was hard for Shelton, who was used to monitoring crowds of
several thousands before showtime and afterward.

Archibald, whose father worked as a Lincoln Center security guard for 14
years, drives in every day from his home in Brooklyn. Earlier this
summer, people who lived nearby told him that they were desperate for
the grounds to open again.

``I miss the patrons, I miss my co-workers,'' he said. ``I'm hoping for
the day that we can all come back.''

Then, in the middle of July, a partial reopening: The outdoor space
opened to the public. Instead of theatergoers circulating on the campus,
the outdoor areas are much more family oriented now, Archibald said.
There are people walking their dogs, children riding bikes, older
couples strolling through the plaza. For a couple of weeks, playlists
curated by Lincoln Center employees and institutions like New York City
Ballet played from the outdoor speakers on the plazas.

Image

Shelton, the late-shift security chief, drives in from Pennsylvania.
Archibald, who works the daytime shift, commutes from
Brooklyn.~Credit...Maridelis Morales Rosado for The New York Times

While the chiefs are still missing their regular patrons and colleagues,
there are new people --- and animals --- who are becoming familiar
passers-by. Shelton said that he has gotten to know a huge German
shepherd named Mr. Pancake. He often sees an elderly couple who told him
how they miss the New York Philharmonic concerts.

``I consider myself a people person,'' Shelton said. ``So seeing the
people come back, it relaxes me.'' \emph{JULIA JACOBS}

\hypertarget{the-theater-educator}{%
\subsection{The Theater Educator}\label{the-theater-educator}}

Image

Caitlyn McCain, artistic associate at New York City Children's Theater,
in her neighborhood, Ditmas Park, Brooklyn.Credit...Maridelis Morales
Rosado for The New York Times

\textbf{Caitlyn McCain, 23, artistic associate at New York City
Children's Theater in Manhattan}

She has enjoyed the heroine's spotlight in plays like Shakespeare's
``Cymbeline'' and ``As You Like It.'' But lately
\href{https://www.caitlynmccain.com/}{Caitlyn McCain}'s most visible
role is not onstage but online, where she is simply Miss Caitlyn, the
effervescent host of Creative Clubhouse Stories at
\href{https://nycchildrenstheater.org/}{New York City Children's
Theater}.

``Covid definitely had me shift my mind-set a bit,'' McCain said,
``although I've always seen acting and my work with young people in
applied theater and education as kind of equally important.''

When the city went into lockdown in March, McCain was appearing in
``\href{https://nycchildrenstheater.org/education/multi-sensory-musical/}{Five},''
a touring musical from this company. She was also its temporary gala
associate, helping plan a fund-raiser that was soon canceled. Instead,
she began to work with Nicole Hogsett, director of marketing and
audience development, on ``how we were pivoting our programming,''
McCain said.

The result was
\href{https://nycchildrenstheater.org/shows-and-programs/creative-clubhouse/}{Creative
Clubhouse}, a web page of performances, reading recommendations, craft
projects, singalongs and games. McCain, who had been an education
apprentice at the theater while an undergraduate at New York University,
proposed also developing
\href{https://nycchildrenstheater.org/shows-and-programs/creative-clubhouse-stories-caitlyn/}{Creative
Clubhouse Stories}, a series of livestreaming book-based classes for
ages 3 to 8.

Image

McCain is the effervescent host of Creative Clubhouse Stories at New
York City Children's Theater.Credit...via New York City Children's
Theater

``Covid-related things is how we started,'' she said. ``We chose books
that explored boredom, anger, anxiety --- really big things that were
probably coming up for a lot of little ones.''

In the classes, Miss Caitlyn reads a picture book, introduces a related
activity and discusses the theme. After hearing
``\href{https://www.kirkusreviews.com/book-reviews/tom-percival/ravis-roar/}{Ravi's
Roar},'' Tom Percival's story about a boy who has trouble controlling
his inner (and here, literal) tiger, ``one little human raises their
hand and says, `It makes me angry that I can't see my friends,''' McCain
recalled. This led to helpful reflections on how Ravi tamed his own
beast.

Although the story series is on hiatus until October, Creative Clubhouse
features videos about each book. It also offers
\href{https://www.kirkusreviews.com/book-reviews/tom-percival/ravis-roar/}{Start
the Conversation}, a resource for helping children deal with challenging
subjects. The first installment, online now, covers race, racism and
Black Lives Matter. It includes McCain demonstrating a stress-reducing
exercise, as well as two videos she helped create: ``You Matter,'' for
families of color, centers on
\href{https://www.nytimes3xbfgragh.onion/2020/06/05/us/talking-to-kids-about-racism.html}{Christian
Robinson}'s children's
\href{https://www.theartoffun.com/you-matter}{book of the same title}.
``Black Lives Matter,'' for white families, focuses on
\href{https://www.nytimes3xbfgragh.onion/2020/05/16/well/family/children-books-how-to-be-upset.html}{Anastasia
Higginbotham}'s ``\href{https://www.dottirpress.com/not-my-idea}{Not My
Idea: A Book About Whiteness}.''

``I wasn't really interested in exploring `We all matter' and `We're all
welcome here,''' McCain said. ``That's been done.'' Instead, each video
addresses specific concerns.

Image

Making time for a cartwheel: McCain has been demonstrating a range of
stress-reducing exercises for young children as part of streaming events
for New York City Children's Theater.Credit...Maridelis Morales Rosado
for The New York Times

As artistic associate, a title she assumed on July 1, McCain is also
developing digital programming connected to the theater's virtual
season. In October, the company will stream a 2015 performance of
``\href{https://nycchildrenstheater.org/shows-and-programs/band-angels-2005/}{A
Band of Angels},'' Myla Churchill's
\href{https://www.nytimes3xbfgragh.onion/2005/02/04/arts/fighting-a-war-on-two-fronts.html}{adaptation
of Deborah Hopkinson's book} about a girl's encounter with her enslaved
ancestor and the founding of the
\href{http://fiskjubileesingers.org/}{Fisk Jubilee Singers} at Fisk
University.

McCain said her goal was to give audiences ``the ability to adopt
another person's point of view through their imaginations.''
\emph{LAUREL GRAEBER}

\hypertarget{the-coding-expert}{%
\subsection{The Coding Expert}\label{the-coding-expert}}

Image

Andy Carluccio and his team work out of his basement in Springfield, Va.
``We think of ourselves sort of like virtual plumbers, routing audio and
video across the internet,'' he said of working to bring livestreams to
computers across the country (and world).Credit...Nichelle Dailey for
The New York Times

\textbf{Andy Carluccio, 22, lead developer at Liminal Entertainment
Technologies in Springfield, Va.}

At the beginning of March, Andy Carluccio assumed that when he graduated
from the University of Virginia he would pay his dues at a software
company and only later try to find a job that combined computer science
and theater, his twin passions.

``I thought the kind of work I'm doing now would be the kind of work I'd
maybe do in five years,'' he said in an interview.

But by the end of the month --- after the pandemic forced performing
arts venues across the country to shutter --- Carluccio had been
recruited by his mentor \href{http://www.eamonnsgarden.com/}{Eamonn
Farrell} to contribute his coding expertise to an online production of
Caryl Churchill's play ``Mad Forest''
\href{https://fishercenter.bard.edu/events/spring-mainstage-2020}{at
Bard College}. He was also plugging away at his undergraduate thesis.

Along with Farrell\href{http://www.eamonnsgarden.com/}{,} who was
working as the show's video designer, Carluccio was charged with
reconciling the creative vision of the production's director, Ashley
Tata, with the capabilities of Zoom's teleconferencing software. The
\href{https://fishercenter.bard.edu/events/the-making-of-mad-forest/}{success
of that endeavor} in April (the
\href{https://www.tfana.org/current-season/digital-programming/madforest}{production
was reprised} in May
\href{https://www.nytimes3xbfgragh.onion/2020/05/24/theater/mad-forest-livestream.html}{to
critical acclaim}) and the skyrocketing demand for streaming content
convinced Carluccio that he didn't need to complete an office job
apprenticeship before striking out on his own.

``I saw the writing on the wall that this was my chance to jump in and
do it now,'' he said. ``I figured I wasn't even going to wait for the
conferral of my degree.'' By May, Carluccio and his business partners,
Jonathan Kokotajlo and Nolan Jacobs-Walker, had Liminal Entertainment
Technology up and running.

Image

Andy Carluccio, left, and Jonathan Kokotajlo at work on Tectonic Theater
Project's ``Las Aventuras de Juan Planchard,'' which will premiere on
YouTube on Oct. 6.Credit...Nichelle Dailey for The New York Times

In the few short months since its founding, the company has facilitated
hundreds of livestreams for houses of worship, high school and
university groups, children's theater organizations, Off Broadway
productions and arts festivals from Carluccio's basement in Springfield,
Va. He is modest about his team's role in these projects: ``We think of
ourselves sort of like virtual plumbers, routing audio and video across
the internet from remote performers and technicians back to me to mix
into a final livestream for the audience to enjoy.''

During the spring it was not a foregone conclusion that theatermakers
would be able to successfully pivot from live performance to creating
compelling digital content. There were serious technical hurdles to
clear and relatively few resources that artists and producers could draw
on to make the switch. Some despaired, choosing to try to wait out the
public health crisis.

``I created this company because I was saddened to see both community
and professional theaters cancel their shows and close their doors
because they felt the task of performing online was an insurmountable
challenge,'' Carluccio explained. ``I wanted to do whatever I could to
make sure these works could continue and perhaps even thrive during a
time when performing arts felt most needed but least accessible.''

The
\href{https://www.nytimes3xbfgragh.onion/2020/07/08/theater/streaming-theater-experiments.html}{rapid
evolution} that online theater has undergone in just a few months has
convinced the young technologist that the hybrid art form is here to
stay. ``There are things that can be done by weaving together
collaborators and designers over decentralized internet that I think has
opened up design possibilities and audience interactions and scale in a
way that we're not going to want to turn off once we're finally able to
gather in person again.'' \emph{PETER LIBBEY}

\hypertarget{the-comedian}{%
\subsection{The Comedian}\label{the-comedian}}

Image

``It's better than nothing,'' Kerryn Feehan (in East River Park) said of
performing at outdoor shows around the city. At the height of the
shutdown, she told jokes to ``Brady Bunch heads'' on
Zoom.Credit...Maridelis Morales Rosado for The New York Times

\textbf{Kerryn Feehan, 37, a stand-up comedian in New York City}

Before the pandemic emptied the streets of New York,
\href{https://twitter.com/KFreehams}{Kerryn Feehan} had her hands full.
She was working as a writer for Paramount Network and performing
stand-up comedy several nights a week.

After the city went into lockdown, adjusting to life without live comedy
``was really depressing,'' Feehan said in a Zoom interview. She kept
busy with fitness and created a show on Instagram called
``\href{https://www.instagram.com/p/CEO17H2gAr6/}{Cooking With
Kerryn.''}

``I made toast for the first seven episodes,'' she said. ``It's
stupid.''

Her first return to stand-up had its setbacks. In the early days of the
health crisis, she participated in Zoom shows hosted by local clubs,
telling jokes to ``Brady Bunch heads'' and viewers casually eating
noodles on the other end of the screen. ``Those were --- interesting,''
Feehan said.

Image

Back in her groove: Adjusting to life without live comedy ``was really
depressing,'' Feehan said.Credit...Maridelis Morales Rosado for The New
York Times

Image

Yet adjusting to outdoor performances can be quite humbling, especially
for veteran comedians now ``performing on Orchard Street while a garbage
truck goes by.''Credit...Maridelis Morales Rosado for The New York Times

She telecommuted for a few months before losing her job in May. Then she
applied for unemployment benefits and took on side gigs.

Things started looking up soon after, though, as coronavirus cases in
New York dropped and the city gradually began to reopen. Outdoor shows
were cropping up in parks and on sidewalks, and Feehan was back onstage
--- figuratively.

Now she is performing almost nightly, with clubs like
\href{https://www.nytimes3xbfgragh.onion/2020/08/26/arts/television/live-comedy-new-york.html}{Stand
Up NY} (which holds performances in city parks) and an outdoor, pop-up
comedy show called
\href{https://www.instagram.com/takeitoutsidecomedy/}{Take It Outside}.

``While the conditions aren't ideal, it's better than nothing,'' Feehan
said.

Microphones often need to be sanitized, if performers don't have their
own, and there's more pressure on the hosts to keep audiences engaged in
between sets. The reactions from the crowd have also changed.

Watching a comedy show in a dimly lit club, where the spotlight is on
the performer, allows the audience members more freedom to laugh at
jokes others in the room may find insensitive or vulgar, Feehan said.
``You laugh with abandon,'' she said. ``You don't filter yourself.''

Remove the shroud of darkness, and people become more apprehensive.
``That's a challenge that has existed with every outdoor show, whether
it's during the day or at night,'' Feehan said.

Image

Feehan at the amphitheater in East River Park in Manhattan. She's
worried about this winter, when outdoor shows won't work. ``The worst
thing you can do is get rid of comedy shows,'' she said, calling the art
form public therapy.Credit...Maridelis Morales Rosado for The New York
Times

The adjustments can be quite humbling for veteran comedians,
particularly those who have HBO and Netflix comedy specials under their
belts and are now ``performing on Orchard Street while a garbage truck
goes by,'' Feehan said.

Naturally, there is no shortage of coronavirus jokes. Feehan would like
to move on from them, but for now, that is all life has to offer. (One
for the road: ``My dad is an essential customer at Home Depot,'' Feehan
said. ``He's never leaving.'')

Though she has consistent shows lined up, Feehan is aware that winter
doesn't bode too well for outdoor comedy. She hopes the city will find a
way to accommodate socially distanced, indoor performances --- whatever
that may look like.

``The worst thing you can do is get rid of comedy shows,'' she said,
calling the art form public therapy that encourages people to laugh at
the strange times we live in.

``Comics,'' Feehan cheekily said, ``are the frontline workers.''
\emph{SARA ARIDI}

Advertisement

\protect\hyperlink{after-bottom}{Continue reading the main story}

\hypertarget{site-index}{%
\subsection{Site Index}\label{site-index}}

\hypertarget{site-information-navigation}{%
\subsection{Site Information
Navigation}\label{site-information-navigation}}

\begin{itemize}
\tightlist
\item
  \href{https://help.nytimes3xbfgragh.onion/hc/en-us/articles/115014792127-Copyright-notice}{©~2020~The
  New York Times Company}
\end{itemize}

\begin{itemize}
\tightlist
\item
  \href{https://www.nytco.com/}{NYTCo}
\item
  \href{https://help.nytimes3xbfgragh.onion/hc/en-us/articles/115015385887-Contact-Us}{Contact
  Us}
\item
  \href{https://www.nytco.com/careers/}{Work with us}
\item
  \href{https://nytmediakit.com/}{Advertise}
\item
  \href{http://www.tbrandstudio.com/}{T Brand Studio}
\item
  \href{https://www.nytimes3xbfgragh.onion/privacy/cookie-policy\#how-do-i-manage-trackers}{Your
  Ad Choices}
\item
  \href{https://www.nytimes3xbfgragh.onion/privacy}{Privacy}
\item
  \href{https://help.nytimes3xbfgragh.onion/hc/en-us/articles/115014893428-Terms-of-service}{Terms
  of Service}
\item
  \href{https://help.nytimes3xbfgragh.onion/hc/en-us/articles/115014893968-Terms-of-sale}{Terms
  of Sale}
\item
  \href{https://spiderbites.nytimes3xbfgragh.onion}{Site Map}
\item
  \href{https://help.nytimes3xbfgragh.onion/hc/en-us}{Help}
\item
  \href{https://www.nytimes3xbfgragh.onion/subscription?campaignId=37WXW}{Subscriptions}
\end{itemize}
