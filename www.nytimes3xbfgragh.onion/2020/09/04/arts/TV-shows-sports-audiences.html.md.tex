\href{/section/arts}{Arts}\textbar{}The Shows Must Go On. But They
Aren't the Same Without You.

\url{https://nyti.ms/2Z3tK4g}

\begin{itemize}
\item
\item
\item
\item
\item
\item
\end{itemize}

\href{https://www.nytimes3xbfgragh.onion/spotlight/at-home?action=click\&pgtype=Article\&state=default\&region=TOP_BANNER\&context=at_home_menu}{At
Home}

\begin{itemize}
\tightlist
\item
  \href{https://www.nytimes3xbfgragh.onion/2020/09/07/travel/route-66.html?action=click\&pgtype=Article\&state=default\&region=TOP_BANNER\&context=at_home_menu}{Cruise
  Along: Route 66}
\item
  \href{https://www.nytimes3xbfgragh.onion/2020/09/04/dining/sheet-pan-chicken.html?action=click\&pgtype=Article\&state=default\&region=TOP_BANNER\&context=at_home_menu}{Roast:
  Chicken With Plums}
\item
  \href{https://www.nytimes3xbfgragh.onion/2020/09/04/arts/television/dark-shadows-stream.html?action=click\&pgtype=Article\&state=default\&region=TOP_BANNER\&context=at_home_menu}{Watch:
  Dark Shadows}
\item
  \href{https://www.nytimes3xbfgragh.onion/interactive/2020/at-home/even-more-reporters-editors-diaries-lists-recommendations.html?action=click\&pgtype=Article\&state=default\&region=TOP_BANNER\&context=at_home_menu}{Explore:
  Reporters' Google Docs}
\end{itemize}

\includegraphics{https://static01.graylady3jvrrxbe.onion/images/2020/09/06/arts/06audiences1/merlin_176303781_b368770f-66e2-4970-96d5-e8cb3858dfd3-articleLarge.jpg?quality=75\&auto=webp\&disable=upscale}

Sections

\protect\hyperlink{site-content}{Skip to
content}\protect\hyperlink{site-index}{Skip to site index}

The Great ReadCritic's Notebook

\hypertarget{the-shows-must-go-on-but-they-arent-the-same-without-you}{%
\section{The Shows Must Go On. But They Aren't the Same Without
You.}\label{the-shows-must-go-on-but-they-arent-the-same-without-you}}

The sudden absence of live audiences has upended the worlds of sports,
comedy and politics. What do we lose when the crowd doesn't show?

The crowd has been compared to an electric spark, a dance partner, an
intoxicant and a character in and of itself. Now it is
gone.Credit...Margeaux Walter for The New York Times; Photographs via
Getty Images; ABC; Harpo Productions; NBC

Supported by

\protect\hyperlink{after-sponsor}{Continue reading the main story}

\href{https://www.nytimes3xbfgragh.onion/by/amanda-hess}{\includegraphics{https://static01.graylady3jvrrxbe.onion/images/2018/02/16/multimedia/author-amanda-hess/author-amanda-hess-thumbLarge-v2.png}}

By \href{https://www.nytimes3xbfgragh.onion/by/amanda-hess}{Amanda Hess}

\begin{itemize}
\item
  Published Sept. 4, 2020Updated Sept. 5, 2020
\item
  \begin{itemize}
  \item
  \item
  \item
  \item
  \item
  \item
  \end{itemize}
\end{itemize}

They used to arrive before dawn. Hundreds of them came to scream and
leap and wave posters scrawled with the names of their hometowns as they
vied to be caught on camera among the ``Today'' show crowd. ``People
dream about coming to 30 Rockefeller Plaza,'' Hoda Kotb, the show's
co-anchor, told me recently over Zoom after a show. But for the last
several months, Kotb has heard the eerie sound of her own footsteps as
she heads into the studio and slips behind the anchor desk, where she
perches at a socially distanced remove from her co-stars and broadcasts
in front of a ghostly plaza. One morning, she spied some movement
outside the window --- it was a nurse in scrubs, lugging a rolling
suitcase --- and Kotb was so hungry for a taste of audience connection,
``I literally held my phone number on a white piece of paper to the
glass,'' she said. ``I was like, `Call me and tell me where you're
from!'''

Since the coronavirus swept across the United States, morning-show
anchors have kept bantering, late-night hosts have kept joking and
politicians have kept stumping. It's the audiences that have not showed.
Their sudden disappearance has spotlighted the mythical, almost
mystical, role they play in popular entertainment. The crowd has been
compared to an electric spark, a dance partner, an intoxicant and a
character in and of itself. It is said to hold great power over
professional performers, messing with their heads and triggering
hormonal surges in their glands. The crowd lends a democratic sheen to
an event, legitimizing the performer's skill and authenticating the show
as real. If the crowd laughs, the joke was funny. If it boos, the call
was bad. The crowd is, as Kotb put it, ``the juice.'' And for now, it is
gone.

This has proved to be a vexing experience for the entertainers of
America. When ``The View'' first banished its studio audience, in March,
Whoopi Goldberg
\href{https://www.thecut.com/2020/03/whoopi-goldberg-the-view-no-audience.html}{cried}
``Welcome to `The View'! Welcome to `The View'!'' again and again into
silence, as cameras swept an expanse of empty seats. Before he sealed
himself into the N.B.A. bubble at Disney World, LeBron James could not
conceive of the game without a crowd,
\href{https://twitter.com/bleacherreport/status/1236193321053081600?lang=en}{saying}:
``If I show up to an arena and there ain't no fans there? I ain't
playing.'' When even A-list celebrities seem bored enough to appear at
events hosted on videoconferencing software, it is the crowd that has
stepped into the role of the withholding diva. A long-anticipated
reunion of ``Friends'' is on indefinite hold, not for David Schwimmer or
Jennifer Aniston but for the anonymous audience members tasked with
observing them: ``We cannot do it without them,'' Marta Kauffman, the
show's co-creator, has
\href{https://news.yahoo.com/friends-reunion-special-audience-delay-131824866.html}{said}.

\includegraphics{https://static01.graylady3jvrrxbe.onion/images/2020/09/06/arts/06audience7/06audience7-articleLarge.jpg?quality=75\&auto=webp\&disable=upscale}

So longing are the shows for their crowds that they have grasped for
imitations. The ``Today'' show has erected a
``\href{https://www.today.com/allday/join-today-s-virtual-plaza-t180466}{virtual
plaza}'' and enlisted performers of
\href{https://www.today.com/music-series}{its once-outdoor music series}
to surprise superfans at home. In a masterwork of artifice, American
ballparks and European soccer stadiums have piped in the
\href{https://www.espn.com/mlb/story/_/id/29477550/mlb-parks-use-crowd-noise-mlb-show-games}{crowd
roars}originally created for
\href{https://www.nytimes3xbfgragh.onion/2020/06/16/sports/coronavirus-stadium-fans-crowd-noise.html}{video
games}. Many baseball teams have put literal stand-ins in the seats,
arranging stiff cardboard cutouts of fans in macabre tableaus; at one
game, the Washington Nationals outfielder Adam Eaton caught a fly ball
and
\href{http://dcsportsking.com/2020/08/17/watch-adam-eaton-give-baseball-to-cutout-baby/}{offered
it} up to the frozen visage of a cardboard baby cradled in her ersatz
mother's lap in the right field stands. And for the Video Music Awards
last weekend, MTV crafted an orgy of simulation, stitching together
uncanny C.G.I. fans and fake crowd buzz into a dystopian New York
cityscape.

The ultimate audiences for sports, politics, talk shows and award
presentations are not found inside arenas or convention halls or studios
--- they are watching from home, slack on the couch, absorbing ads and
paying for cable and streaming packages. In normal times, the live crowd
mounts a performance for the remote audience. But this summer, without
our stand-ins to guide us, we home viewers confront a void. The pretense
of the crowd always provided the true audience a bit of cover; we could
vicariously ride its emotions, feeding off its energy, absorbing its
delight and its outrage, even as we sat quietly alone at home. But now
we are directly implicated in the show itself.

\textbf{The television experience} was largely designed to replicate
live performances --- to transport their spontaneous thrills into the
remote home. In his book
``\href{https://auslander.lmc.gatech.edu/liveness-performance-in-a-mediatized-culture-1999/}{Liveness:
Performance in a Mediatized Culture},'' Philip Auslander, a professor of
performance studies at the Georgia Institute of Technology, traces how
TV borrowed the storytelling conventions of the theater: it was styled
as an immediate event, with the viewer positioned at the scene of the
action, as if watching from the lip of the stage or the sideline of the
court. The classic three-camera setup mimicked the movement of the
audience's roving eye, perhaps aided with a pair of opera glasses. And
even as TV absorbed more cinematic elements, playing with shifting
perspectives and transpositions of time, it also built up conventions
that simulate the feeling of liveness: recorded laugh tracks and cuts to
the ``live studio audience,'' where the crowd of spectators is vetted
for entrance, warmed up by producers and cued to applaud. And all that
prompts the home audience to feel invested in the show. ``Maybe even
more than the performance, we identify with the audience,'' Auslander
said.

Image

Fans are said to hold great power over professional performers, messing
with their heads and triggering hormonal surges in their
glands.Credit...Margeaux Walter for The New York Times; Photograph via
Getty Images

Even as the internet disrupts these conventions again, giving rise to an
era of scattered, disembodied crowd collaboration, the figure of the
audience is pulled along for the ride.
\href{https://www.vox.com/culture/2017/6/15/15804082/greatest-reaction-gifs-supa-hot-fire-blinking-white-guy}{Reaction
GIFs} are sliced from evocative scenes and shared to perform the
emotions of online spectators. Often the GIFs are culled from actual
television audiences, like the anonymous woman
\href{https://tenor.com/view/wendy-williams-gif-6115300}{nodding
approvingly} in ``The Wendy Williams Show'' crowd, or Chrissy Teigen's
\href{https://giphy.com/gifs/chrissy-nancy-odell-mEqMknMZWh1Fm}{strained,
awkward cry face}caught during a scan of the 2015 Golden Globes audience
as her husband, John Legend, won a statuette. Teigen has since emerged
as our reigning celebrity spectator; she walks among the stars but is
positioned as their observer and judge, laughing and cringing as she
goes. Much of the thrill of the modern award show is produced by
celebrities reacting to other celebrities, from Martin Scorsese blankly
processing an
\href{https://www.cinemablend.com/news/2489906/the-best-celebrity-reaction-shots-during-eminems-surprise-oscar-performance}{Eminem
performance} to
\href{https://www.popsugar.com/celebrity/Meryl-Streep-Shouting-2018-Oscars-44640382}{Meryl
Streep hollering} at the stage.

The sudden absence of the crowd is scrambling entrenched media
narratives. In the N.B.A., the fans are imbued with the power to
influence the refs, psych out free-throw shooters and generally mess
with players' heads. But the crowdless bubble has called the home-court
advantage into question and subdued the typical whiplash drama of the
playoffs, where rival teams jet back and forth across the country,
playing for adoring, then hostile, then adoring crowds; already, it's
clear that
\href{https://www.wsj.com/articles/nba-bubble-shooting-soccer-empty-stadiums-11596539693}{players
are shooting better} without the fans. A player in a typical series
faces ``the full gamut of emotions'' from the crowd, Greg Anthony, the
former Knicks point guard and Turner Sports announcer, told me from
inside the bubble. The newly muted atmosphere ``could change the entire
course of how the playoffs play out,'' he said.

Image

Many baseball teams, including the Los Angeles Dodgers, have opted to
fill out their seats with literal stand-ins, arranging for stiff
cardboard cutouts of fans.Credit...Harry How/Getty Images

On late-night comedy shows, the laughter has died. In March, Samantha
Bee's weekly TBS show, ``Full Frontal,'' began filming in her backyard
in upstate New York. ``When I do the show in front of a live studio
audience, it's a very communal experience,'' Bee said. ``We're in it
together.'' Making the crowd laugh feels ``intoxicating,'' she said.
Now, her jokes are met with chirping birds and buzzing cicadas, which
``Full Frontal'' preserves as background noise. Her only audience is
another seasoned comedian --- her husband, Jason Jones --- and their
three children, who make for a tough crowd. ``I'm at my most
self-conscious when they're watching,'' Bee said of her kids. ``They do
not think I have any comedic ability.'' The ``Full Frontal'' staff has
coped with the dead space by filling it with more jokes. Said Bee:
``We're just packing more and more into the show.''

Meanwhile, politics is getting more serious. Crowdless stump speeches
are cut short --- Joe Biden's was
\href{https://tennesseestar.com/2020/08/22/joe-biden-gives-shortest-dnc-acceptance-speech-in-decades/}{the
shortest Democratic National Convention acceptance speech} in recent
memory --- and trimmed of jokes and broad applause lines tailored to
fire up the base, said David Litt, a former speechwriter for Barack
Obama. The speeches are forced to be subtler and more sincere. Normal
campaign seasons amass political crowds so large that even a slightly
amusing observation can prompt an outsize reaction. ``You could write a
joke --- not even a hilarious line, but a warm introductory line --- and
if one-third of the people in attendance thought it was funny, that
would be 10,000 people laughing,'' Litt said. For some politicians, that
feedback is the very point of the political performance. ``I think it's
one of the reasons President Trump is so desperate to get in front of
live crowds anywhere,'' Litt said. Without the validation of the roaring
crowd, ``You have to be able to say something and just trust that it
will sound good.''

In politics, the crowd functions as a visual and rhetorical metaphor for
democracy itself, even if --- as is typical at the Democratic and
Republican national conventions --- it is actually assembled from a
curated crew of delegates and party die-hards. A crowd also opens the
opportunity for performed dissent: At the 2016 D.N.C., some Bernie
Sanders delegates
\href{https://time.com/4425475/dnc-bernie-sanders-protest-walkout-convention/}{staged
a}walkout; a few
donned\href{https://theintercept.com/2016/07/25/on-day-one-of-the-democratic-convention-the-boos-have-it/}{green
Robin Hood hats} and stuck
\href{https://www.businessinsider.com/sanders-delegates-occupy-media-tents-dnc-2016-7}{duct
tape over their mouths}. But at this year's convention, any protests
were preemptively blocked. Biden spoke live to a silent, darkened
Delaware auditorium, then turned to a giant screen featuring a grid of
selected supporters clapping to their webcams. The only hiccup was the
video feed of one couple in the middle, who stared blankly to the side,
as if they had missed their cue to convert from true spectators of the
speech into performers of spectating.

Image

At the Democratic National Convention, Joe Biden spoke live to a silent,
darkened Delaware auditorium, then turned to a giant screen featuring a
grid of selected supporters clapping to their webcams.Credit...Andrew
Harnik/Associated Press

The crowd also offers an imprimatur of an authentic political
performance. Sean Hannity of Fox News painted Biden's performance as a
``dull, boring speech to an empty room,'' but some right-wing observers
cast it in a more sinister light: A
\href{https://twitter.com/johncardillo/status/1297267264715096065}{conspiracy
theory} swept the conservative media accusing the D.N.C. of recording
the event, perhaps even doctoring it. It was a ludicrous claim, but it
could gain purchase because the only witnesses to the speech were
campaign insiders and journalists. A crowd is the visual assurance that
something really happened.

And in our current moment, it has taken on an even more perverse valence
--- as a veil thrown over a deadly pandemic. Trump's R.N.C. did, in
fact, feature many recorded speeches, and many more delivered to a
hollow auditorium in Washington. But on the final night, Trump
\href{https://www.nytimes3xbfgragh.onion/2020/08/28/us/politics/trump-convention-speech-white-house.html}{gathered
a crowd of 1,500} mostly unmasked people on the White House lawn to
receive him, flouting
Washington\href{https://coronavirus.dc.gov/phasetwo}{rules} banning
large gatherings. While the Democrats have a political and scientific
explanation for mounting an abnormal convention --- it holds a mirror to
the devastation wreaked by the coronavirus under the Trump
administration --- the Republicans are invested in projecting the image
that everything is fine. Trump's crowd capably played the part of a
satisfied public. ``Four more years,'' it chanted.

Image

Samantha Bee's ``Full Frontal'' now tapes in her backyard in upstate New
York, where her jokes are met with chirping birds and buzzing
cicadas.Credit...TBS

\textbf{Several years ago}, a friend and I attended the Video Music
Awards as members of the audience. As we filed into Madison Square
Garden, we were swept into a stream of thousands of ticket holders,
corralled through glaring white corridors and shunted up escalators into
upper-deck seats. The crowd evinced the docile resignation of workers
reporting to a factory floor. On a faraway stage, Britney Spears and
Rihanna and Ariana Grande appeared as brief glimpses of distant
wildlife. We watched them on video screens instead. It felt less like an
experience than an assignment: We had done a satisfactory job of
creating the image of a packed house.

The last few months have cracked an opportunity for a new kind of crowd
relationship, one not predicated on such rote exercises of theatrical
feedback. Our new era of ``live'' performance requires something not
exactly like cinema, not quite like television, but something more like
the internet. Traditional entertainers now feel in direct competition
with internet stars, who are preternaturally skilled at performing
one-sided conversations to unfeeling camera lenses, then riding waves of
online reactions that spin off in unexpected directions. The best
internet videos carry a frisson of intimacy and spontaneity; they seem
crafted not to please the crowd but to connect with one individual,
millions and millions of times.

YouTube and TikTok and Instagram, which have made sensations of people
shooting videos alone in their bedrooms, are the ideal platforms for a
pandemic that mandates private viewing. The must-see live show of the
summer is Verzuz, a
\href{https://www.nytimes3xbfgragh.onion/2020/05/07/arts/music/hip-hop-instagram-coronavirus.html}{D.J.
battle reimagined for Instagram}, in which players like Jill Scott and
Erykah Badu take turns vibing to their own greatest hits in a feat of
synchronized isolation. It's hard to find that on television, even as
the medium grasps for a more online sensibility. A rash of reunion
specials that assumed the aesthetics of the Zoom grid had all the
excitement of a staff meeting. The socially distant conventions bore the
emotional sterility of a telethon. But some glimmers of interest have
emerged.

Biden, hardly a master of new media technologies, nevertheless delivered
a convention speech directly to the camera that functioned as a more
intimate appeal, helping to underscore his bid to be presidential
empath. Watching professional comedians
\href{https://mashable.com/video/jimmy-fallon-home-monologue-heckled-daughters/}{fail
to delight their children} has the indelible stamp of a viral video, one
that keys into the online mood better than any topical punchline. And
last month, professional athletes pulled off a remarkable show ---
\href{https://www.nytimes3xbfgragh.onion/2020/08/27/sports/basketball/kenosha-nba-protests-players-boycott.html}{they
stopped playing} in protest of the police shooting of Jacob Blake. In a
normal season, with thousands of anxious paying fans looming above them,
would they have had the nerve to walk away? The strike worked both as
protest and performance because it was pitched not to the die-hards in
the stands but to the complacent television audience flipping through
the channels. Its message spoke directly to viewers at home: Get off the
couch.

Image

The convention of the live studio audience, where the crowd of
spectators is tightly controlled, warmed up by producers and cued to
applaud, prompts the home audience to feel invested in the show.
Credit...Margeaux Walter for The New York Times; Photograph via Harpo
Productions

Advertisement

\protect\hyperlink{after-bottom}{Continue reading the main story}

\hypertarget{site-index}{%
\subsection{Site Index}\label{site-index}}

\hypertarget{site-information-navigation}{%
\subsection{Site Information
Navigation}\label{site-information-navigation}}

\begin{itemize}
\tightlist
\item
  \href{https://help.nytimes3xbfgragh.onion/hc/en-us/articles/115014792127-Copyright-notice}{©~2020~The
  New York Times Company}
\end{itemize}

\begin{itemize}
\tightlist
\item
  \href{https://www.nytco.com/}{NYTCo}
\item
  \href{https://help.nytimes3xbfgragh.onion/hc/en-us/articles/115015385887-Contact-Us}{Contact
  Us}
\item
  \href{https://www.nytco.com/careers/}{Work with us}
\item
  \href{https://nytmediakit.com/}{Advertise}
\item
  \href{http://www.tbrandstudio.com/}{T Brand Studio}
\item
  \href{https://www.nytimes3xbfgragh.onion/privacy/cookie-policy\#how-do-i-manage-trackers}{Your
  Ad Choices}
\item
  \href{https://www.nytimes3xbfgragh.onion/privacy}{Privacy}
\item
  \href{https://help.nytimes3xbfgragh.onion/hc/en-us/articles/115014893428-Terms-of-service}{Terms
  of Service}
\item
  \href{https://help.nytimes3xbfgragh.onion/hc/en-us/articles/115014893968-Terms-of-sale}{Terms
  of Sale}
\item
  \href{https://spiderbites.nytimes3xbfgragh.onion}{Site Map}
\item
  \href{https://help.nytimes3xbfgragh.onion/hc/en-us}{Help}
\item
  \href{https://www.nytimes3xbfgragh.onion/subscription?campaignId=37WXW}{Subscriptions}
\end{itemize}
