Sections

SEARCH

\protect\hyperlink{site-content}{Skip to
content}\protect\hyperlink{site-index}{Skip to site index}

\href{https://www.nytimes3xbfgragh.onion/section/climate}{Climate}

\href{https://myaccount.nytimes3xbfgragh.onion/auth/login?response_type=cookie\&client_id=vi}{}

\href{https://www.nytimes3xbfgragh.onion/section/todayspaper}{Today's
Paper}

\href{/section/climate}{Climate}\textbar{}Americans Back Tough Limits on
Building in Fire and Flood Zones

\url{https://nyti.ms/3lYQY5N}

\begin{itemize}
\item
\item
\item
\item
\item
\end{itemize}

\hypertarget{climate-and-environment}{%
\subsubsection{\texorpdfstring{\href{https://www.nytimes3xbfgragh.onion/section/climate?name=styln-climate\&region=TOP_BANNER\&block=storyline_menu_recirc\&action=click\&pgtype=Article\&impression_id=72e5d120-f4c5-11ea-8643-75abe867ee48\&variant=undefined}{Climate
and
Environment}}{Climate and Environment}}\label{climate-and-environment}}

\begin{itemize}
\tightlist
\item
  \href{https://www.nytimes3xbfgragh.onion/2020/09/08/climate/california-wildfires-climate.html?name=styln-climate\&region=TOP_BANNER\&block=storyline_menu_recirc\&action=click\&pgtype=Article\&impression_id=72e5d121-f4c5-11ea-8643-75abe867ee48\&variant=undefined}{Wildfires}
\item
  \href{https://www.nytimes3xbfgragh.onion/interactive/2020/climate/trump-environment-rollbacks.html?name=styln-climate\&region=TOP_BANNER\&block=storyline_menu_recirc\&action=click\&pgtype=Article\&impression_id=72e5f830-f4c5-11ea-8643-75abe867ee48\&variant=undefined}{Trump's
  Changes}
\item
  \href{https://www.nytimes3xbfgragh.onion/interactive/2020/04/19/climate/climate-crash-course-1.html?name=styln-climate\&region=TOP_BANNER\&block=storyline_menu_recirc\&action=click\&pgtype=Article\&impression_id=72e5f831-f4c5-11ea-8643-75abe867ee48\&variant=undefined}{Climate
  101}
\item
  \href{https://www.nytimes3xbfgragh.onion/interactive/2020/08/24/climate/racism-redlining-cities-global-warming.html?name=styln-climate\&region=TOP_BANNER\&block=storyline_menu_recirc\&action=click\&pgtype=Article\&impression_id=72e5f832-f4c5-11ea-8643-75abe867ee48\&variant=undefined}{Environmental
  Racism}
\end{itemize}

Advertisement

\protect\hyperlink{after-top}{Continue reading the main story}

Supported by

\protect\hyperlink{after-sponsor}{Continue reading the main story}

\hypertarget{americans-back-tough-limits-on-building-in-fire-and-flood-zones}{%
\section{Americans Back Tough Limits on Building in Fire and Flood
Zones}\label{americans-back-tough-limits-on-building-in-fire-and-flood-zones}}

A majority support outright bans on construction in disaster-prone
areas, as well as paying people to move, researchers say --- policies
that local governments have been reluctant to adopt.

\includegraphics{https://static01.graylady3jvrrxbe.onion/images/2020/09/04/climate/04CLI-RESTRICTIONS1b/merlin_176348559_5f51b6b2-0754-49bf-9fe9-0a5a0d78bdde-articleLarge.jpg?quality=75\&auto=webp\&disable=upscale}

\href{https://www.nytimes3xbfgragh.onion/by/christopher-flavelle}{\includegraphics{https://static01.graylady3jvrrxbe.onion/images/2019/06/28/climate/author-chris-flavelle/author-chris-flavelle-thumbLarge-v3.png}}

By
\href{https://www.nytimes3xbfgragh.onion/by/christopher-flavelle}{Christopher
Flavelle}

\begin{itemize}
\item
  Sept. 4, 2020
\item
  \begin{itemize}
  \item
  \item
  \item
  \item
  \item
  \end{itemize}
\end{itemize}

WASHINGTON --- Americans support far more aggressive government
regulation to fight the effects of climate change than elected officials
have been willing to pursue so far,
\href{https://www.rff.org/publications/reports/climateinsights2020-natural-disasters/}{new
research shows}, including outright bans on building in flood- or
fire-prone areas --- a level of restrictiveness almost unheard-of in the
United States.

The findings suggest that the public's appetite for government action to
prepare for global warming is shifting as natural disasters worsen.

Eighty-four percent of respondents, including 73 percent of Republicans,
supported mandatory building codes in risky areas, and 57 percent
supported making it illegal to build in those areas. More than half of
respondents favored
\href{https://www.nytimes3xbfgragh.onion/2020/08/26/climate/flooding-relocation-managed-retreat.html}{paying
people to move}, including three-quarters of Democrats.

But while the findings show bipartisan support, more stringent
restrictions have been generally opposed by local officials, who cite
the cost they would impose on the economy. ``There's a disconnect
between public preference and public policy,'' said Jon A. Krosnick, a
professor of communication, political science and psychology at Stanford
University who led the project.

As global greenhouse gas emissions continue to rise, decisions about
where and how to build have become increasingly important. If local
governments continue to allow homes to go up in places most exposed to
climate change, such as coastlines, floodplains or
\href{https://www.nytimes3xbfgragh.onion/2020/09/10/climate/wildfires-climate-policy.html}{fire-prone
wilderness}, experts say, it will make generations of current and future
residents more vulnerable to worsening hurricanes, floods, wildfires and
other disasters.

Yet those long-term concerns have typically been outweighed by the
demand for new homes, and the jobs and tax revenue that come with them.
In many coastal states, the most flood-prone areas have seen the
\href{https://www.nytimes3xbfgragh.onion/2019/07/31/climate/climate-change-new-homes-flooding.html}{highest
rates} of home construction since 2010, a study last year found. And in
California and elsewhere, officials continue to approve development in
areas
\href{https://www.sandiegouniontribune.com/news/environment/story/2019-05-25/san-diegos-latest-backcountry-development-to-be-built-where-california-suffered-one-of-its-most-historic-wildfires}{hit
by fires}.

``Some of the most vulnerable land also ends up being some of the
highest-priced land,'' said Otis Rolley, senior vice president at the
Rockefeller Foundation and former North America managing director for
100 Resilient Cities, an initiative that worked with cities to better
withstand shocks from climate change and other challenges. ``There's a
lot of pressure on elected officials.''

\href{\%3Ca\%20href=\%22https://www.nytimes3xbfgragh.onion/section/climate?action=click\&pgtype=Article\&state=default\&region=MAIN_CONTENT_1\&context=storylines_keepup\%22\%3Ehttps://www.nytimes3xbfgragh.onion/section/climate?action=click\&pgtype=Article\&state=default\&region=MAIN_CONTENT_1\&context=storylines_keepup\%3C/a\%3E}{}

\hypertarget{climate-and-environment-}{%
\subsubsection{Climate and Environment
›}\label{climate-and-environment-}}

\hypertarget{keep-up-on-the-latest-climate-news}{%
\paragraph{Keep Up on the Latest Climate
News}\label{keep-up-on-the-latest-climate-news}}

Updated Sept. 10, 2020

Here's what you need to know this week:

\begin{itemize}
\item
  \begin{itemize}
  \tightlist
  \item
    The
    \href{https://www.nytimes3xbfgragh.onion/2020/09/10/climate/wildfires-climate-policy.html?action=click\&pgtype=Article\&state=default\&region=MAIN_CONTENT_1\&context=storylines_keepup}{wildfires
    scorching the West}~highlight the urgency of rethinking fire
    management policies, as climate change threatens to make things
    worse.
  \item
    Americans back
    \href{https://www.nytimes3xbfgragh.onion/2020/09/04/climate/flood-fire-building-restrictions.html?action=click\&pgtype=Article\&state=default\&region=MAIN_CONTENT_1\&context=storylines_keepup}{tough
    limits on building in fire and flood zones}, new research shows.
  \item
    The Trump administration has relaxed Obama-era rules limiting the
    release of
    \href{https://www.nytimes3xbfgragh.onion/2020/08/31/climate/trump-coal-plants.html?action=click\&pgtype=Article\&state=default\&region=MAIN_CONTENT_1\&context=storylines_keepup}{toxic
    waste from coal plants}.
  \end{itemize}
\end{itemize}

A wave of disasters has pushed some cities and counties to
\href{https://www.nytimes3xbfgragh.onion/2019/11/19/climate/climate-real-estate-developers.html}{limit}
where they build. The new survey --- a joint project of Stanford;
Resources for the Future, a Washington research group; and ReconMR, a
survey research company --- asked whether governments should require
that new buildings in risky areas ``need to be made in a way that
doesn't get damaged easily by floods.''

The support among Republican respondents was notable considering that
fewer than one-third of Republican voters say global warming is a
\href{https://www.pewresearch.org/global/2020/04/13/americans-see-spread-of-disease-as-top-international-threat-along-with-terrorism-nuclear-weapons-cyberattacks/}{major
threat} to the United States, according to a Pew Research Center survey
from March, and despite the party's general aversion to new regulations.

There was even greater support for construction requirements in
fire-prone areas, with 87 percent of respondents favoring them,
including 79 percent of Republicans. ``It's clear that people want
this,'' said Ray Kopp, who worked on the project as vice president for
research and policy engagement at Resources for the Future.

That public support is at odds with actual policies in most of the
country. Just one-third of local jurisdictions around the United States
have adopted disaster-resistant provisions into their building codes for
homes and businesses, according to research by the Federal Alliance for
Safe Homes, an advocacy group based in Florida.

The lack of tougher codes reflects the influence of home builders and
developers on local officials who oppose tougher restrictions, said
Leslie Chapman-Henderson, president and chief executive officer of the
organization. ``They are really well organized, and that's what they
advocate for,'' she said.

\includegraphics{https://static01.graylady3jvrrxbe.onion/images/2020/09/04/climate/04CLI-RESTRICTIONS2/merlin_176186988_3ae2440c-bcf9-42c9-8cbe-479a4df33f35-articleLarge.jpg?quality=75\&auto=webp\&disable=upscale}

Chuck Fowke, chairman of the National Association of Home Builders, said
the requirements already in effect around the country were enough. New
rules could ``not only curtail homeownership and significantly hinder
housing affordability,'' he said in a statement, but ``also can severely
impact state and local economies.''

A more aggressive measure than mandatory building codes is prohibiting
development entirely in vulnerable places, which almost no jurisdictions
have done, said Larry Larson, senior policy adviser for the Association
of State Floodplain Managers. He said cities and counties allowed
building in flood-prone areas in part because they know the federal
government will pay most of the cost to rebuild after a disaster.

``Locals can allow development and get all the taxes from development,
and when the flooding or other natural disaster happens, the cost is too
often picked up by the federal taxpayer,'' Mr. Larson said.

The National Association of Counties, which represents local
governments, said its members must weigh environmental issues along with
economic ones. ``Both are important,'' Paul Guequierre, a spokesman,
said.

If local governments follow public opinion and impose new restrictions
on development, it's important that they consider the effects of those
changes on poorer communities, including communities of color, said R.
Jisung Park, an assistant professor of public policy at the University
of California, Los Angeles, who focuses on climate adaptation.

While many vulnerable areas have wealthy residents drawn to the scenery,
others are home to low-income families, including minorities, who can't
afford to live elsewhere, Dr. Park said. Development restrictions that
increase costs could hurt those communities, he added, even if they
reduce disasters in the future.

One approach would be for governments to make it more expensive to live
in vulnerable neighborhoods, but subsidize low-income residents who want
to move, Dr. Park. Doing both ``is certainly possible,'' he said.

The survey shows support for that approach. Asked whether governments
should offer people money to move their homes away from risky areas, 59
percent of respondents said yes, including 46 percent of Republicans.

Getting governments to do more to protect against climate change might
be easier than it seems, Ms. Chapman-Henderson said. She recalled a home
builder who said to her: ``No one has ever stormed city hall demanding a
stronger building code. But the day they do, they'll get it.''

Advertisement

\protect\hyperlink{after-bottom}{Continue reading the main story}

\hypertarget{site-index}{%
\subsection{Site Index}\label{site-index}}

\hypertarget{site-information-navigation}{%
\subsection{Site Information
Navigation}\label{site-information-navigation}}

\begin{itemize}
\tightlist
\item
  \href{https://help.nytimes3xbfgragh.onion/hc/en-us/articles/115014792127-Copyright-notice}{©~2020~The
  New York Times Company}
\end{itemize}

\begin{itemize}
\tightlist
\item
  \href{https://www.nytco.com/}{NYTCo}
\item
  \href{https://help.nytimes3xbfgragh.onion/hc/en-us/articles/115015385887-Contact-Us}{Contact
  Us}
\item
  \href{https://www.nytco.com/careers/}{Work with us}
\item
  \href{https://nytmediakit.com/}{Advertise}
\item
  \href{http://www.tbrandstudio.com/}{T Brand Studio}
\item
  \href{https://www.nytimes3xbfgragh.onion/privacy/cookie-policy\#how-do-i-manage-trackers}{Your
  Ad Choices}
\item
  \href{https://www.nytimes3xbfgragh.onion/privacy}{Privacy}
\item
  \href{https://help.nytimes3xbfgragh.onion/hc/en-us/articles/115014893428-Terms-of-service}{Terms
  of Service}
\item
  \href{https://help.nytimes3xbfgragh.onion/hc/en-us/articles/115014893968-Terms-of-sale}{Terms
  of Sale}
\item
  \href{https://spiderbites.nytimes3xbfgragh.onion}{Site Map}
\item
  \href{https://help.nytimes3xbfgragh.onion/hc/en-us}{Help}
\item
  \href{https://www.nytimes3xbfgragh.onion/subscription?campaignId=37WXW}{Subscriptions}
\end{itemize}
