Sections

SEARCH

\protect\hyperlink{site-content}{Skip to
content}\protect\hyperlink{site-index}{Skip to site index}

\href{https://www.nytimes3xbfgragh.onion/section/politics}{Politics}

\href{https://myaccount.nytimes3xbfgragh.onion/auth/login?response_type=cookie\&client_id=vi}{}

\href{https://www.nytimes3xbfgragh.onion/section/todayspaper}{Today's
Paper}

\href{/section/politics}{Politics}\textbar{}Health Care Powered
Democratic Wins in 2018. The Party Hopes for a Repeat.

\url{https://nyti.ms/32S9gww}

\begin{itemize}
\item
\item
\item
\item
\item
\end{itemize}

\begin{itemize}
\item
  \href{https://www.nytimes3xbfgragh.onion/interactive/2020/09/08/us/elections/results-new-hampshire-primary-elections.html?action=click\&pgtype=Article\&state=default\&region=TOP_BANNER\&context=storylines_menu}{New
  Hampshire Results}
\item
  \href{https://www.nytimes3xbfgragh.onion/live/2020/09/08/us/trump-vs-biden?action=click\&pgtype=Article\&state=default\&region=TOP_BANNER\&context=storylines_menu}{Election
  Updates}
\item
  \href{https://www.nytimes3xbfgragh.onion/interactive/2020/us/elections/election-states-biden-trump.html?action=click\&pgtype=Article\&state=default\&region=TOP_BANNER\&context=storylines_menu}{Paths
  to 270}
\item
  \href{https://www.nytimes3xbfgragh.onion/interactive/2020/08/31/us/politics/vote-by-mail-deadlines.html?action=click\&pgtype=Article\&state=default\&region=TOP_BANNER\&context=storylines_menu}{Voting
  by Mail}
\item
  \href{https://www.nytimes3xbfgragh.onion/interactive/2019/us/elections/2020-presidential-election-calendar.html?action=click\&pgtype=Article\&state=default\&region=TOP_BANNER\&context=storylines_menu}{Key
  Dates}
\item
  \href{https://www.nytimes3xbfgragh.onion/newsletters/politics?action=click\&pgtype=Article\&state=default\&region=TOP_BANNER\&context=storylines_menu}{Politics
  Newsletter}
\end{itemize}

Advertisement

\protect\hyperlink{after-top}{Continue reading the main story}

Supported by

\protect\hyperlink{after-sponsor}{Continue reading the main story}

\hypertarget{health-care-powered-democratic-wins-in-2018-the-party-hopes-for-a-repeat}{%
\section{Health Care Powered Democratic Wins in 2018. The Party Hopes
for a
Repeat.}\label{health-care-powered-democratic-wins-in-2018-the-party-hopes-for-a-repeat}}

With the pandemic highlighting the importance of health coverage,
Democrats are leaning into health care as a campaign issue. Republicans
appear to be on shakier footing.

\includegraphics{https://static01.graylady3jvrrxbe.onion/images/2020/09/02/us/politics/02healthcare2/merlin_173927892_8c340c0b-1700-40f6-9ef1-93bf20a98d1f-articleLarge.jpg?quality=75\&auto=webp\&disable=upscale}

\href{https://www.nytimes3xbfgragh.onion/by/thomas-kaplan}{\includegraphics{https://static01.graylady3jvrrxbe.onion/images/2019/08/28/reader-center/author-thomas-kaplan/author-thomas-kaplan-thumbLarge-v2.png}}

By \href{https://www.nytimes3xbfgragh.onion/by/thomas-kaplan}{Thomas
Kaplan}

\begin{itemize}
\item
  Sept. 2, 2020
\item
  \begin{itemize}
  \item
  \item
  \item
  \item
  \item
  \end{itemize}
\end{itemize}

At the Democratic National Convention, viewers heard from an Arizona man
whose young son was born with a congenital heart defect, a Wisconsin
woman with an autoimmune disease and cancer survivors from several
states. Their stories highlighted the importance of health care --- and
the protections provided by the Affordable Care Act.

When Republicans held their convention last week, they had little to say
about their own vision for America's health care system. Obamacare, for
years a punching bag for the party, went
\href{https://www.nytimes3xbfgragh.onion/2020/08/28/upshot/republican-convention-obamacare.html}{almost
entirely unmentioned}. When the phrase ``health care'' was spoken, it
was often in the service of attacking Democrats over health care for
undocumented immigrants.

Those dueling approaches to discussing health policy offered a preview
of what to expect as the two parties, and their presidential nominees,
make their closing arguments on one of the most critical issues to many
voters --- one whose importance has been underscored by the coronavirus
pandemic, which has killed
\href{https://www.nytimes3xbfgragh.onion/interactive/2020/us/coronavirus-us-cases.html}{at
least 184,000 people} in the United States.

Democrats are once again trying to capitalize on an issue that was key
to their success in the 2018 midterm elections. And Republicans are once
again vulnerable: Three years after failing to repeal and replace the
Affordable Care Act, the party still has not coalesced around a plan for
the future of America's health care system.

``No one could quite figure out what `replace' was,'' said Adam Brandon,
the president of FreedomWorks, a conservative advocacy group, as he
recalled the party's struggle to repeal the law known as Obamacare after
\href{https://www.nytimes3xbfgragh.onion/interactive/2020/us/elections/donald-trump.html}{President
Trump}'s election. ``That's where the problem was, and we never
recovered from it. And we still haven't recovered from it today.''

Mr. Brandon cited the successful health care message employed by
Democrats in 2018, with its emphasis on protecting people with
pre-existing medical conditions, and likened it to a football team that
calls a running play that proves successful --- and then keeps calling
the same play. ``If I'm the Democrats,'' he said, ``I just keep handing
the ball off on pre-existing conditions until Republicans prove they can
stop that.''

It's a playbook that down-ballot Democrats and their allies are using in
campaign ads.

``Tell Thom Tillis: Stop cutting health care,'' the narrator says in
several recent ads attacking Mr. Tillis, a Republican senator up for
re-election in North Carolina. And an ad from Amy McGrath, the
Democratic challenger to Senator Mitch McConnell of Kentucky, says of
her opponent: ``Even during a pandemic, with people out of work, he's
trying to take away health care.''

With two months until Election Day, health care is looming as a major
weak spot for Republicans, including incumbent senators in close races
that could determine which party wins the majority. In a
\href{https://static.foxnews.com/foxnews.com/content/uploads/2020/08/Fox_August-9-12-2020_Complete_National_Topline_August-13-Release.pdf}{Fox
News poll} in August, 53 percent of voters disapproved of the way Mr.
Trump was handling health care, while
\href{https://www.nytimes3xbfgragh.onion/interactive/2020/us/elections/joe-biden.html}{Joseph
R. Biden Jr.} led Mr. Trump by a 15-point margin when voters were asked
whom they trusted to do a better job on health care.

The Trump administration is also asking the Supreme Court to
\href{https://www.nytimes3xbfgragh.onion/2020/06/26/us/politics/obamacare-trump-administration-supreme-court.html}{overturn
the Affordable Care Act}, a course of action that offers Democrats a
political piñata to swing at in the weeks leading up to Election Day.
And that is in addition to the matter of how Mr. Trump has handled the
pandemic, a front-and-center issue that is shaping up to be a major
liability for Republicans.

``We are in a health crisis; they clearly do not want to talk about
that,'' said Kathleen Sebelius, who served as secretary of health and
human services for President Barack Obama when the Affordable Care Act
was enacted. She said that Republicans were trying to topple the health
law at a time when ``people are terrified about losing their health
insurance.''

The programming at the Democratic convention reminded viewers of Mr.
Biden's own painful brushes with the health care system, including when
a car accident in 1972 killed his first wife and daughter and left his
two young sons hospitalized. Decades later, in 2015, his elder son, Beau
Biden,
\href{https://www.nytimes3xbfgragh.onion/2015/05/31/us/politics/joseph-r-biden-iii-vice-presidents-son-beau-dies-at-46.html}{died
of brain cancer}.

\includegraphics{https://static01.graylady3jvrrxbe.onion/images/2020/09/02/us/politics/02healthcare/merlin_175963251_37a5da73-8488-4d15-8bdd-f3e94c79d9d5-articleLarge.jpg?quality=75\&auto=webp\&disable=upscale}

``This is my promise to you: When I'm president, I will take care of
your health care coverage and your family the same way I would my own,''
Mr. Biden said during a segment on health care. In his speech accepting
the Democratic nomination, he warned that if Mr. Trump is re-elected,
``the assault on the Affordable Care Act will continue until it's
destroyed.''

Mr. Biden's victory in the Democratic primary has also left his party's
candidates less susceptible to being attacked on health care than they
might have been if a more progressive candidate had prevailed. Mr.
Biden, who was vice president when the Affordable Care Act was passed,
does not support ``Medicare for all,'' a government-run health insurance
system under which private insurance would be eliminated; instead, he
wants to build on the health law by offering a government plan known as
a public option. Had a supporter of Medicare for all, like Senator
Bernie Sanders of Vermont, been atop the Democratic ticket, Republicans
would have been able to use the issue of private insurance as a cudgel.

The Trump campaign is nevertheless mounting an attack in that vein,
arguing that a public option would open the door to a larger government
role in health care in the future. Mr. Biden's proposal, the campaign
said in August, ``will ultimately kill families' private health plans
and pave the way for Bernie Sanders's socialist single-payer system.''

``Who has private health insurance here?'' Mr. Trump asked the crowd at
a rally in New Hampshire last week. ``And you love it, right? You love
it. It's luxury, it's good, it's beautiful and you have the greatest ---
you're going to lose it. Hate to tell you. Under his plan, you're going
to lose your private health care.''

Lanhee J. Chen, a fellow at the Hoover Institution who was policy
director for Mitt Romney's 2012 presidential campaign, said the critique
of Mr. Biden on health care reinforced the Republican argument that Mr.
Biden ``is a Trojan horse for bad progressive outcomes.''

``The fact that Biden is for a public option is enough to usher in the
possibility that he could also be for Medicare for all at some point, or
that Medicare for all would be the likely outcome,'' he said. ``If
Republicans repeat enough, `Trojan horse,' `single payer,' that's going
to leave a mark.''

But even if that critique sticks with some voters, the Republican Party
will still have to overcome its recent history of missteps on health
care.

In 2016, Mr. Trump stood onstage at his party's convention and offered
an unequivocal pledge: ``We will repeal and replace disastrous
Obamacare.'' But that goal proved elusive. In 2017, after considerable
difficulty, Republicans in the House
\href{https://www.nytimes3xbfgragh.onion/2017/05/04/us/politics/health-care-bill-vote.html}{passed
a bill} to repeal and replace the health law, a step that Mr. Trump
celebrated in the Rose Garden, but Republicans in the Senate
\href{https://www.nytimes3xbfgragh.onion/2017/07/27/us/politics/obamacare-partial-repeal-senate-republicans-revolt.html}{could
not agree} on their own measure. As part of the tax overhaul passed
later that year, Mr. Trump and Republicans did manage to scrap one piece
of the Affordable Care Act, eliminating the penalty for people who
choose not to buy health insurance.

Despite the failure to repeal the law, the Trump administration has
taken
\href{https://www.nytimes3xbfgragh.onion/interactive/2017/10/12/us/trump-undermine-obamacare.html}{a
number of steps to undermine it}, and the administration is now asking
the Supreme Court to overturn it. Oral arguments are scheduled for one
week after Election Day. The administration's legal position has handed
Democrats a simple and powerful talking point: that Mr. Trump is
threatening the health care of millions of Americans in the midst of a
pandemic.

``Instead of crushing the virus, he's trying to crush the Affordable
Care Act,'' Speaker Nancy Pelosi of California said last week.

Image

The White House press secretary, Kayleigh McEnany, spoke of President
Trump's support for people with pre-existing conditions.Credit...Pete
Marovich for The New York Times

At the Republican convention, Obamacare had seemingly escaped the
vocabulary of Mr. Trump and other party leaders. The four nights of
convention programming included only scattered references to health care
issues, with speakers touching on matters like the opioid crisis,
efforts to make hospital prices public and a law signed by Mr. Trump
that allows seriously ill patients to seek access to experimental drugs.
Kayleigh McEnany, the White House press secretary, spoke of
\href{https://www.nytimes3xbfgragh.onion/live/2020/08/26/us/rnc-convention-election/kayleigh-mcenany-tells-a-personal-story-in-a-bid-to-make-trump-more-appealing-to-women}{undergoing
a preventive double mastectomy} and testified to Mr. Trump's support for
people with pre-existing conditions, though his record is squarely at
odds with that characterization.

The lack of a crystal clear message on health care was not a surprise,
because Mr. Trump has been anything but clear on the subject.

Mr. Trump and his campaign have tried to put a focus on the cost of
prescription drugs, with a
\href{https://www.nytimes3xbfgragh.onion/2020/07/24/us/politics/trump-drug-prices-coronavirus.html}{series
of executive orders} and a television ad that proclaims Mr. Trump is
``standing up to the drug companies.'' (One of those orders still has
not been
\href{https://www.nytimes3xbfgragh.onion/2020/08/24/us/politics/trump-drug-prices.html}{released
publicly}.)

But more than three years into his presidency, Mr. Trump's plans for the
future of the overall health care system remain a question mark. In
July, he said, ``We're signing a health care plan within two weeks, a
full and complete health care plan.'' Two weeks later, no health care
plan had materialized. Mr. Trump then offered a new time frame,
promising ``a tremendous health care plan'' before the end of August and
adding that it was ``just about completed.''

The end of August came and went without any plan. Asked for an update, a
White House spokeswoman, Sarah Matthews, cited Mr. Trump's executive
orders on prescription drugs and promised ``more action to come in the
coming weeks,'' without providing any specifics.

Mr. Trump has also teased an executive order protecting people with
pre-existing conditions, a curious undertaking given that the Affordable
Care Act already protects people with pre-existing conditions.

And when he addressed Republican delegates gathered in Charlotte, N.C.,
last week, he declared that ``we knocked out Obamacare.''

There was only one problem: Obamacare still exists.

\hypertarget{our-2020-election-guide}{%
\section{Our 2020 Election Guide}\label{our-2020-election-guide}}

Updated ~Sept. 8, 2020

\begin{center}\rule{0.5\linewidth}{\linethickness}\end{center}

\begin{itemize}
\item ~
  \hypertarget{the-latest}{%
  \subsection{The Latest}\label{the-latest}}

  \begin{itemize}
  \item
    President Trump and his party are using a playbook that aims to
    alarm people about crime in their backyards. It didn't work in 2018,
    but
    \href{https://www.nytimes3xbfgragh.onion/2020/09/08/us/politics/trump-republicans-fear-strategy.html?action=click\&pgtype=Article\&state=default\&region=BELOW_MAIN_CONTENT\&context=storylines_guide}{both
    parties think it could resonate more this year}.
  \end{itemize}
\item ~
  \hypertarget{how-to-win-270}{%
  \subsection{How to Win 270}\label{how-to-win-270}}

  \begin{itemize}
  \item
    Joe Biden and Donald Trump need 270 electoral votes to reach the
    White House. Try building
    \href{https://www.nytimes3xbfgragh.onion/interactive/2020/us/elections/election-states-biden-trump.html?action=click\&pgtype=Article\&state=default\&region=BELOW_MAIN_CONTENT\&context=storylines_guide}{your
    own coalition of battleground states}~to see potential outcomes.
  \end{itemize}
\item ~
  \hypertarget{voting-by-mail}{%
  \subsection{Voting by Mail}\label{voting-by-mail}}

  \begin{itemize}
  \item
    Will you have enough time to vote by mail in your state? Yes, but
    it's risky to procrastinate.
    \href{https://www.nytimes3xbfgragh.onion/interactive/2020/08/31/us/politics/vote-by-mail-deadlines.html?action=click\&pgtype=Article\&state=default\&region=BELOW_MAIN_CONTENT\&context=storylines_guide}{Check
    your state's deadline.}
  \item
    \href{https://www.nytimes3xbfgragh.onion/interactive/2020/us/elections/joe-biden.html?action=click\&pgtype=Article\&state=default\&region=BELOW_MAIN_CONTENT\&context=storylines_guide}{}

    \hypertarget{joe-biden}{%
    \section{Joe Biden}\label{joe-biden}}

    \hypertarget{democrat}{%
    \subsection{Democrat}\label{democrat}}

    \href{https://www.nytimes3xbfgragh.onion/interactive/2020/us/elections/donald-trump.html?action=click\&pgtype=Article\&state=default\&region=BELOW_MAIN_CONTENT\&context=storylines_guide}{}

    \hypertarget{donald-trump}{%
    \section{Donald Trump}\label{donald-trump}}

    \hypertarget{republican}{%
    \subsection{Republican}\label{republican}}
  \end{itemize}
\item
  \hypertarget{keep-up-with-our-coverage}{%
  \subsection{Keep Up With Our
  Coverage}\label{keep-up-with-our-coverage}}

  \begin{itemize}
  \item
    Get an
    \href{https://www.nytimes3xbfgragh.onion/newsletters/politics?action=click\&pgtype=Article\&state=default\&region=BELOW_MAIN_CONTENT\&context=storylines_guide}{email}~recapping
    the day's news
  \item
    Download our mobile app on
    \href{https://apps.apple.com/us/app/nytimes/id284862083?ls=1\&mat_click_id=5c79ae7455014fd1bd66b5610c05b8f2-20191112-16948\&referrer=mat_click_id\%3D5c79ae7455014fd1bd66b5610c05b8f2-20191112-16948\%26link_click_id\%3D722930677036718082}{iOS}~and
    \href{http://a.localytics.com/android?id=com.nytimes.android\&referrer=utm_source\%3Dother_nyt_mobile_web\%26utm_medium\%3DWeb\%2520page\%26utm_term\%3DGeneral\%2520Mobile\%2520Page\%26utm_campaign\%3DNYT\%2520Mobile\%2520General\%2520Page}{Android}~and
    turn on Breaking News and Politics alerts
  \end{itemize}
\end{itemize}

Advertisement

\protect\hyperlink{after-bottom}{Continue reading the main story}

\hypertarget{site-index}{%
\subsection{Site Index}\label{site-index}}

\hypertarget{site-information-navigation}{%
\subsection{Site Information
Navigation}\label{site-information-navigation}}

\begin{itemize}
\tightlist
\item
  \href{https://help.nytimes3xbfgragh.onion/hc/en-us/articles/115014792127-Copyright-notice}{©~2020~The
  New York Times Company}
\end{itemize}

\begin{itemize}
\tightlist
\item
  \href{https://www.nytco.com/}{NYTCo}
\item
  \href{https://help.nytimes3xbfgragh.onion/hc/en-us/articles/115015385887-Contact-Us}{Contact
  Us}
\item
  \href{https://www.nytco.com/careers/}{Work with us}
\item
  \href{https://nytmediakit.com/}{Advertise}
\item
  \href{http://www.tbrandstudio.com/}{T Brand Studio}
\item
  \href{https://www.nytimes3xbfgragh.onion/privacy/cookie-policy\#how-do-i-manage-trackers}{Your
  Ad Choices}
\item
  \href{https://www.nytimes3xbfgragh.onion/privacy}{Privacy}
\item
  \href{https://help.nytimes3xbfgragh.onion/hc/en-us/articles/115014893428-Terms-of-service}{Terms
  of Service}
\item
  \href{https://help.nytimes3xbfgragh.onion/hc/en-us/articles/115014893968-Terms-of-sale}{Terms
  of Sale}
\item
  \href{https://spiderbites.nytimes3xbfgragh.onion}{Site Map}
\item
  \href{https://help.nytimes3xbfgragh.onion/hc/en-us}{Help}
\item
  \href{https://www.nytimes3xbfgragh.onion/subscription?campaignId=37WXW}{Subscriptions}
\end{itemize}
