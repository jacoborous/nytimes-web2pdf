How Hunger Persists in a Rich Country Like America

\url{https://nyti.ms/2EL1Jb4}

\begin{itemize}
\item
\item
\item
\item
\item
\item
\end{itemize}

\hypertarget{the-coronavirus-outbreak}{%
\subsubsection{\texorpdfstring{\href{https://www.nytimes3xbfgragh.onion/news-event/coronavirus?name=styln-coronavirus-national\&region=TOP_BANNER\&block=storyline_menu_recirc\&action=click\&pgtype=Article\&impression_id=79274710-f1c5-11ea-9d0b-75131b3353e3\&variant=undefined}{The
Coronavirus
Outbreak}}{The Coronavirus Outbreak}}\label{the-coronavirus-outbreak}}

\begin{itemize}
\tightlist
\item
  live\href{https://www.nytimes3xbfgragh.onion/2020/09/08/world/covid-19-coronavirus.html?name=styln-coronavirus-national\&region=TOP_BANNER\&block=storyline_menu_recirc\&action=click\&pgtype=Article\&impression_id=79274711-f1c5-11ea-9d0b-75131b3353e3\&variant=undefined}{Latest
  Updates}
\item
  \href{https://www.nytimes3xbfgragh.onion/interactive/2020/us/coronavirus-us-cases.html?name=styln-coronavirus-national\&region=TOP_BANNER\&block=storyline_menu_recirc\&action=click\&pgtype=Article\&impression_id=79274712-f1c5-11ea-9d0b-75131b3353e3\&variant=undefined}{Maps
  and Cases}
\item
  \href{https://www.nytimes3xbfgragh.onion/interactive/2020/science/coronavirus-vaccine-tracker.html?name=styln-coronavirus-national\&region=TOP_BANNER\&block=storyline_menu_recirc\&action=click\&pgtype=Article\&impression_id=79274713-f1c5-11ea-9d0b-75131b3353e3\&variant=undefined}{Vaccine
  Tracker}
\item
  \href{https://www.nytimes3xbfgragh.onion/2020/09/02/your-money/eviction-moratorium-covid.html?name=styln-coronavirus-national\&region=TOP_BANNER\&block=storyline_menu_recirc\&action=click\&pgtype=Article\&impression_id=79274714-f1c5-11ea-9d0b-75131b3353e3\&variant=undefined}{Eviction
  Moratorium}
\item
  \href{https://www.nytimes3xbfgragh.onion/interactive/2020/09/02/magazine/food-insecurity-hunger-us.html?name=styln-coronavirus-national\&region=TOP_BANNER\&block=storyline_menu_recirc\&action=click\&pgtype=Article\&impression_id=79274715-f1c5-11ea-9d0b-75131b3353e3\&variant=undefined}{American
  Hunger}
\end{itemize}

\includegraphics{https://static01.graylady3jvrrxbe.onion/images/2020/09/06/magazine/06mag-hunger-intro-09/06mag-hunger-intro-09-articleLarge-v2.jpg?quality=75\&auto=webp\&disable=upscale}

Sections

\protect\hyperlink{site-content}{Skip to
content}\protect\hyperlink{site-index}{Skip to site index}

\hypertarget{how-hunger-persists-in-a-rich-country-like-america}{%
\section{How Hunger Persists in a Rich Country Like
America}\label{how-hunger-persists-in-a-rich-country-like-america}}

Beyond the pandemic emergency, there is a food crisis hidden in plain
sight: Millions struggling for years to feed their families.

Barbara Broomall (right) and her children sharing a room in a
hotel-turned-shelter in Menands, N.Y.Credit...Brenda Ann Kenneally for
The New York Times

Supported by

\protect\hyperlink{after-sponsor}{Continue reading the main story}

By Adrian Nicole LeBlanc

\begin{itemize}
\item
  Sept. 2, 2020
\item
  \begin{itemize}
  \item
  \item
  \item
  \item
  \item
  \item
  \end{itemize}
\end{itemize}

In March, the photographer Brenda Ann Kenneally was visiting Troy, N.Y.,
when the coronavirus pandemic hit the East Coast. She grew up in the
area, bouncing among friends and group homes after her mother kicked her
out when she was 12. Kenneally has spent decades immersed in the
intimate lives of a group of upstate families who share her legacy,
using images to explore the way economic forces ravage people's lives
for generations. ``I knew every single layer of disadvantage they lived
on a daily basis would be exacerbated by Covid,'' she says.

As April unfolded, Kenneally checked on friends in crowded apartments
and shelters in and around Troy, and for weeks she was the only person
wearing a mask. Stressors were so common --- evictions, unemployment,
isolation --- that Covid-19 hadn't yet struck many of them as
particularly significant. But as jobs continued to disappear in New York
and around the country, Kenneally knew that millions of Americans were
now being thrown into the kind of precarity that the people she knew had
long endured. ``It was the moment to connect the root causes of all the
things that people could be shamed for with what you see in front of the
camera,'' she says. ``The situations that define a life of scarcity were
becoming democratized.''

She returned to Queens, where she lives, packed up her pull camper and
enlisted Rafael Gonzalez, the father of her 26-year-old son, beginning
what would become a 92-day trip across the country documenting food
insecurity. She and Gonzalez met as homeless teenagers working for a
carnival, so they knew the road.

\hypertarget{hunger-embed}{%
\subsection{Hunger Embed}\label{hunger-embed}}

\href{https://www.nytimes3xbfgragh.onion/interactive/2020/09/02/magazine/food-insecurity-hunger-us.html}{}

\hypertarget{america-at-hungers-edge}{%
\section{AMERICA AT HUNGER'S EDGE}\label{america-at-hungers-edge}}

See all the
photographs\includegraphics{https://static01.graylady3jvrrxbe.onion/newsgraphics/2020/08/17/hunger/assets/images/tiny_arrow_white.svg}

\includegraphics{https://static01.graylady3jvrrxbe.onion/images/2020/09/01/magazine/food-insecurity-hunger-us/food-insecurity-hunger-us-threeByTwoMediumAt2X.jpg}

Brenda Ann Kenneally for The New York Times

The highways were quiet as they headed north. They visited Salvation
Armys and food pantries in Canandaigua, Utica and Buffalo. Kenneally
knocked on car windows and walked the lines. Every postindustrial town
and city they passed through looked abandoned. ``You couldn't tell if
they had been closed down because of globalization three decades ago or
Covid,'' she says.

They made a brief return to Troy before heading west. On Mother's Day,
Kenneally joined the Stocklas family for a meal. Family members had
pooled their benefits from SNAP (the federal Supplemental Nutrition
Assistance Program --- food stamps) to buy the food. ``I've been
watching the Stocklases struggle to put dinner on the table for 15
years,'' Kenneally says. Kenneally's mother was aided by food stamps
after her father left, and she used federal food benefits, too,
especially when raising her own son as a single mother.

Then Kenneally and Gonzalez left for Pennsylvania. From there to Gary,
Ind., to Chicago, charities and nonprofits were adjusting to new safety
protocols to deliver food. The Salvation Army has more than 7,600
centers of operation, and many waived eligibility requirements. Food
banks distribute to local pantries, which in turn get groceries directly
to people or to organizations that serve hot meals. Volunteers --- many
of them senior citizens --- were now at risk, so staff was scrambling to
find help while converting to curbside pickup. In Parma, Ohio, the
school district, like so many across the country, had essentially become
its own food bank. In Memphis, a woman became a distribution point for
her condominium complex, giving away boxed lunches that she retrieved
from her niece's school.

Image

Jasmine Williams, 2, lives in Gary, Ind. During the pandemic, she and
her family have had to rely on food from the Salvation Army and meals
provided by the school district.Credit...Brenda Ann Kenneally for The
New York Times

Image

Luckas Manaseri, 12, made apple bread using the chopped-up fruit from a
week's worth of school lunches.Credit...Brenda Ann Kenneally for The New
York Times

Precarity wasn't new to Kenneally, but what was striking now was the
astonishing scale. Lines at food banks stretched to hundreds of cars,
some carrying people who had never sought food assistance before. In
Houston, Catholic Charities was providing food to as many as 2,000
people every six hours. The Mamie George Community Center there gave out
567,000 pounds of food in 2019; between March 18, 2020, and July 6, when
Kenneally arrived, the M.G.C.C. had already distributed 528,437 pounds.

At a time when the heat and the fear were rising, when Americans were
urged to keep distance from loved ones, when protesters --- outdoors ---
were risking their safety, strangers let Kenneally and Gonzalez, wearing
masks, into their homes and kitchens to watch them prepare their food
and eat. ``They understood that telling their food-struggle story now
and even pre-Covid was important,'' Kenneally says. She moved in close,
photographing this ordinary intimacy under extraordinary circumstances.
``I want you to feel like you are there, to go in there, to be
vulnerable and to honor the fact that these people are making themselves
vulnerable.''

\includegraphics{https://static01.graylady3jvrrxbe.onion/images/2020/09/06/magazine/06mag-hunger-intro-05/06mag-hunger-intro-05-articleLarge.jpg?quality=75\&auto=webp\&disable=upscale}

\textbf{In 1936, Dorothea Lange} took what would become a world-famous
photograph of 32-year-old Florence Owens Thompson, in Nipomo, Calif. It
was early March, and Lange was speeding home to Berkeley, where she
lived. She glimpsed a handwritten sign that read ``Pea-Pickers Camp,''
but at first she drove right past it.

She had spent that bitter cold February following migrant workers who
had fled the Dust Bowl and were following crops. Many were starving. By
1936, thousands were flooding into California every month, and police
officers were stationed at the state's borders to turn back anyone
deemed a ``transient.'' Lange was taking photographs for the
Resettlement Administration, a New Deal agency that would soon change
its name to the Farm Security Administration, which relocated poor urban
and rural people into government-planned communities. The government's
goal was to educate voters who hadn't been so hard hit by the Great
Depression and didn't know much about the degrees of suffering in their
midst.

\hypertarget{latest-updates-the-coronavirus-outbreak}{%
\section{\texorpdfstring{\href{https://www.nytimes3xbfgragh.onion/2020/09/08/world/covid-19-coronavirus.html?action=click\&pgtype=Article\&state=default\&region=MAIN_CONTENT_1\&context=storylines_live_updates}{Latest
Updates: The Coronavirus
Outbreak}}{Latest Updates: The Coronavirus Outbreak}}\label{latest-updates-the-coronavirus-outbreak}}

Updated 2020-09-08T11:04:36.368Z

\begin{itemize}
\tightlist
\item
  \href{https://www.nytimes3xbfgragh.onion/2020/09/08/world/covid-19-coronavirus.html?action=click\&pgtype=Article\&state=default\&region=MAIN_CONTENT_1\&context=storylines_live_updates\#link-4a77847f}{As
  senators return to Washington, an impasse over a virus relief package
  looms.}
\item
  \href{https://www.nytimes3xbfgragh.onion/2020/09/08/world/covid-19-coronavirus.html?action=click\&pgtype=Article\&state=default\&region=MAIN_CONTENT_1\&context=storylines_live_updates\#link-679303d7}{Nine
  drugmakers pledge to thoroughly vet any coronavirus vaccine.}
\item
  \href{https://www.nytimes3xbfgragh.onion/2020/09/08/world/covid-19-coronavirus.html?action=click\&pgtype=Article\&state=default\&region=MAIN_CONTENT_1\&context=storylines_live_updates\#link-1c973131}{`The
  lockdown killed my father': Farmer suicides add to India's virus
  misery.}
\end{itemize}

\href{https://www.nytimes3xbfgragh.onion/2020/09/08/world/covid-19-coronavirus.html?action=click\&pgtype=Article\&state=default\&region=MAIN_CONTENT_1\&context=storylines_live_updates}{See
more updates}

More live coverage:
\href{https://www.nytimes3xbfgragh.onion/live/2020/09/08/business/stock-market-today-coronavirus?action=click\&pgtype=Article\&state=default\&region=MAIN_CONTENT_1\&context=storylines_live_updates}{Markets}

After Lange passed the pea-pickers sign, she drove for another 20 miles,
wanting to believe, she later wrote in an article, that she already had
enough evidence of hardship, but she had an argument with herself:
``Dorothea, how about that camp back there? \ldots{} Nobody could ask
this of you, now could they? \ldots{} To turn back certainly is not
necessary. Isn't this just one more of the same?'' She turned back to
see for herself.

Image

Dorothea Lange's ``Migrant Mother,'' photographed in Nipomo, Calif., in
1936.Credit...Library of Congress

She found Thompson and three of her children huddled under a tattered,
dirty tent. ``She said that they had been living on frozen vegetables
from the surrounding fields, and birds that the children killed,'' Lange
wrote. ``She seemed to know that my pictures might help her, and so she
helped me.''

Lange's photographs of Thompson ran in The San Francisco News shortly
afterward. The public reaction to the image of an attractive mother and
her daughters was immediate: letters of concern, calls to action,
donations. The government assembled 20,000 pounds of emergency food, but
by the time it was shipped to that particular migrant camp, the woman
had already packed up her seven hungry children and pressed on. The
image, which eventually came to be titled ``Migrant Mother,'' circulated
widely and increased popular support for the New Deal programs that
evolved into what remains of our social safety net today. Until 1978,
her name --- and that she was of Cherokee descent --- remained unknown.

Our treatment of hunger as an emergency, rather than a symptom of
systemic inequities, has long informed our response to it, and as a
result, government programs have been designed to alleviate each peak
rather than to address the factors that produce them. ``Hunger becoming
public is the start of a struggle, but it's only the beginning of what's
required for change,'' says Laurie B. Green, an associate professor of
history at the University of Texas at Austin, whose research looks at
the moment in the 1960s when public health commissions, politicians and
the media ``discovered'' hunger.

The severing of hunger from its socioeconomic context minimized the
relationship between the restructuring of land, labor and industrial
farming and its effect on diets and access to healthful food. Federal
surplus-commodity programs grew out of the Great Depression, providing
hungry people with leftover staples like flour, rice and lard. But their
priority was to subsidize white farmers; the starchy diet did little to
alleviate malnutrition. In the early 1960s, some areas began to offer
food stamps instead. But because the coupons needed to be purchased
every month, and values were set by local counties, they were
inaccessible to the poorest --- especially Southern Black residents ---
who were now unable to get any food at all. Activists like Fannie Lou
Hamer organized against the program. The purchase requirement remained
in place until 1977.

The first food bank opened in 1967. That December, Look magazine
published photographs by Al Clayton, part of an exposé about a destitute
family living in a windowless shack on no more than ``coffee, flour and
an inch of rice in a cellophane bag.'' The next year, a CBS documentary,
``Hunger in America,'' featured a baby in an American hospital crib
dying of starvation onscreen. Public pressure led to legislation that
improved access to food stamps and created the Special Supplemental
Nutrition Program for Women, Infants and Children (WIC) in 1972.

Image

An image from Al Clayton's exploration of poverty in the South in
1967.Credit...The Estate of Al Clayton

But those programs weren't designed to eliminate need. WIC limits the
age of child recipients; SNAP meets roughly two-thirds of a household's
food needs, and recipients run out of food by the end of the month.
``Our whole safety net is based on the premise that all able-bodied
adults can get a job, and every kind of assistance is temporary,'' says
the Princeton sociologist Kathryn Edin.

\href{https://www.nytimes3xbfgragh.onion/news-event/coronavirus?action=click\&pgtype=Article\&state=default\&region=MAIN_CONTENT_3\&context=storylines_faq}{}

\hypertarget{the-coronavirus-outbreak-}{%
\subsubsection{The Coronavirus Outbreak
›}\label{the-coronavirus-outbreak-}}

\hypertarget{frequently-asked-questions}{%
\paragraph{Frequently Asked
Questions}\label{frequently-asked-questions}}

Updated September 4, 2020

\begin{itemize}
\item ~
  \hypertarget{what-are-the-symptoms-of-coronavirus}{%
  \paragraph{What are the symptoms of
  coronavirus?}\label{what-are-the-symptoms-of-coronavirus}}

  \begin{itemize}
  \tightlist
  \item
    In the beginning, the coronavirus
    \href{https://www.nytimes3xbfgragh.onion/article/coronavirus-facts-history.html?action=click\&pgtype=Article\&state=default\&region=MAIN_CONTENT_3\&context=storylines_faq\#link-6817bab5}{seemed
    like it was primarily a respiratory illness}~--- many patients had
    fever and chills, were weak and tired, and coughed a lot, though
    some people don't show many symptoms at all. Those who seemed
    sickest had pneumonia or acute respiratory distress syndrome and
    received supplemental oxygen. By now, doctors have identified many
    more symptoms and syndromes. In April,
    \href{https://www.nytimes3xbfgragh.onion/2020/04/27/health/coronavirus-symptoms-cdc.html?action=click\&pgtype=Article\&state=default\&region=MAIN_CONTENT_3\&context=storylines_faq}{the
    C.D.C. added to the list of early signs}~sore throat, fever, chills
    and muscle aches. Gastrointestinal upset, such as diarrhea and
    nausea, has also been observed. Another telltale sign of infection
    may be a sudden, profound diminution of one's
    \href{https://www.nytimes3xbfgragh.onion/2020/03/22/health/coronavirus-symptoms-smell-taste.html?action=click\&pgtype=Article\&state=default\&region=MAIN_CONTENT_3\&context=storylines_faq}{sense
    of smell and taste.}~Teenagers and young adults in some cases have
    developed painful red and purple lesions on their fingers and toes
    --- nicknamed ``Covid toe'' --- but few other serious symptoms.
  \end{itemize}
\item ~
  \hypertarget{why-is-it-safer-to-spend-time-together-outside}{%
  \paragraph{Why is it safer to spend time together
  outside?}\label{why-is-it-safer-to-spend-time-together-outside}}

  \begin{itemize}
  \tightlist
  \item
    \href{https://www.nytimes3xbfgragh.onion/2020/05/15/us/coronavirus-what-to-do-outside.html?action=click\&pgtype=Article\&state=default\&region=MAIN_CONTENT_3\&context=storylines_faq}{Outdoor
    gatherings}~lower risk because wind disperses viral droplets, and
    sunlight can kill some of the virus. Open spaces prevent the virus
    from building up in concentrated amounts and being inhaled, which
    can happen when infected people exhale in a confined space for long
    stretches of time, said Dr. Julian W. Tang, a virologist at the
    University of Leicester.
  \end{itemize}
\item ~
  \hypertarget{why-does-standing-six-feet-away-from-others-help}{%
  \paragraph{Why does standing six feet away from others
  help?}\label{why-does-standing-six-feet-away-from-others-help}}

  \begin{itemize}
  \tightlist
  \item
    The coronavirus spreads primarily through droplets from your mouth
    and nose, especially when you cough or sneeze. The C.D.C., one of
    the organizations using that measure,
    \href{https://www.nytimes3xbfgragh.onion/2020/04/14/health/coronavirus-six-feet.html?action=click\&pgtype=Article\&state=default\&region=MAIN_CONTENT_3\&context=storylines_faq}{bases
    its recommendation of six feet}~on the idea that most large droplets
    that people expel when they cough or sneeze will fall to the ground
    within six feet. But six feet has never been a magic number that
    guarantees complete protection. Sneezes, for instance, can launch
    droplets a lot farther than six feet,
    \href{https://jamanetwork.com/journals/jama/fullarticle/2763852}{according
    to a recent study}. It's a rule of thumb: You should be safest
    standing six feet apart outside, especially when it's windy. But
    keep a mask on at all times, even when you think you're far enough
    apart.
  \end{itemize}
\item ~
  \hypertarget{i-have-antibodies-am-i-now-immune}{%
  \paragraph{I have antibodies. Am I now
  immune?}\label{i-have-antibodies-am-i-now-immune}}

  \begin{itemize}
  \tightlist
  \item
    As of right
    now,\href{https://www.nytimes3xbfgragh.onion/2020/07/22/health/covid-antibodies-herd-immunity.html?action=click\&pgtype=Article\&state=default\&region=MAIN_CONTENT_3\&context=storylines_faq}{~that
    seems likely, for at least several months.}~There have been
    frightening accounts of people suffering what seems to be a second
    bout of Covid-19. But experts say these patients may have a
    drawn-out course of infection, with the virus taking a slow toll
    weeks to months after initial exposure.~People infected with the
    coronavirus typically
    \href{https://www.nature.com/articles/s41586-020-2456-9}{produce}~immune
    molecules called antibodies, which are
    \href{https://www.nytimes3xbfgragh.onion/2020/05/07/health/coronavirus-antibody-prevalence.html?action=click\&pgtype=Article\&state=default\&region=MAIN_CONTENT_3\&context=storylines_faq}{protective
    proteins made in response to an
    infection}\href{https://www.nytimes3xbfgragh.onion/2020/05/07/health/coronavirus-antibody-prevalence.html?action=click\&pgtype=Article\&state=default\&region=MAIN_CONTENT_3\&context=storylines_faq}{.
    These antibodies may}~last in the body
    \href{https://www.nature.com/articles/s41591-020-0965-6}{only two to
    three months}, which may seem worrisome, but that's~perfectly normal
    after an acute infection subsides, said Dr. Michael Mina, an
    immunologist at Harvard University. It may be possible to get the
    coronavirus again, but it's highly unlikely that it would be
    possible in a short window of time from initial infection or make
    people sicker the second time.
  \end{itemize}
\item ~
  \hypertarget{what-are-my-rights-if-i-am-worried-about-going-back-to-work}{%
  \paragraph{What are my rights if I am worried about going back to
  work?}\label{what-are-my-rights-if-i-am-worried-about-going-back-to-work}}

  \begin{itemize}
  \tightlist
  \item
    Employers have to provide
    \href{https://www.osha.gov/SLTC/covid-19/standards.html}{a safe
    workplace}~with policies that protect everyone equally.
    \href{https://www.nytimes3xbfgragh.onion/article/coronavirus-money-unemployment.html?action=click\&pgtype=Article\&state=default\&region=MAIN_CONTENT_3\&context=storylines_faq}{And
    if one of your co-workers tests positive for the coronavirus, the
    C.D.C.}~has said that
    \href{https://www.cdc.gov/coronavirus/2019-ncov/community/guidance-business-response.html}{employers
    should tell their employees}~-\/- without giving you the sick
    employee's name -\/- that they may have been exposed to the virus.
  \end{itemize}
\end{itemize}

In the 1980s, in response to cuts to food benefits during the Reagan
administration, hunger was discovered again with commissions and
reports. The underlying problem was bound up with the increasingly
punishing nature of the American economy, especially for people of
color. Food banks were supposed to fill in the gaps. But today more than
37 million Americans are food insecure, according to the U.S.D.A. ``We
call it an emergency food system, but it's a 50-year emergency,'' says
Noreen Springstead, executive director of WhyHunger, which supports
grass-roots organizations that approach food insecurity systemically.

\textbf{Food insecurity no longer} looks like a skinny mother in a tent
or children with rickets and kwashiorkor; it looks like fast food at the
end of the month when SNAP runs out, or rural ``food deserts,'' where
few food banks reach. Its legacy is diabetes, high blood pressure and
obesity.

The pandemic has revealed the fragility of a highly centralized
industrial food system and has given us a glimpse of the tenuous lives
of the workers who farm, process, deliver and ring up the food we need.
It also has shown, as Springstead points out, just ``how close people
are to the edge of the abyss. They can't keep their apartment and can't
pay for their groceries; they are one paycheck away from, `What am I
going to do?'''

Image

Gesma Mohamed, a single mother with three small children, works 10-hour
night shifts processing returned packages at a warehouse. ``Every time I
come home, the kids say, `Mama, mama, we hungry!' ''Credit...Brenda Ann
Kenneally for The New York Times

Image

Zubaidah Abdulshukur, 40, and her husband, who works at a recycling
plant, receive \$1,000 a month in food stamps for their six children;
when the benefits run out, she picks up donations from a food truck
around the corner.Credit...Brenda Ann Kenneally for The New York Times

Programs created to help the poorest Americans now supplement the
working poor. More than half of all SNAP recipients work. The pandemic
has heightened food insecurity. The Salvation Army reported an 84
percent increase since last year in the number of boxes handed out at
their drive-through pantries. Meals on Wheels has seen a 47 percent
increase in the number of people it serves. In addition to federal
subsidies, food banks rely on private donations, which historically
decline during economic downturns. Corporate donors are selling more of
the food they would ordinarily donate because it's no longer expiring on
the shelves.

Even before the pandemic, food insecurity was entangled with
unaffordable housing, health care costs, unreliable transportation. In
Troy, before she traveled across the country, Kenneally met with her
friend Barbara Broomall. Three days before the eviction moratorium,
federal marshals put Broomall, her three children and their belongings
on the street. With the pandemic lockdown, it became clear that her only
option was a room in the Schuyler Inn, a homeless shelter that was once
a hotel. Broomall and her son both received S.S.I. for mental health
issues, and the \$1,457 rent ate up the checks. She had no car to reach
her children's schools to collect the food they were distributing,
though before the school kitchens were up and running, they were
offering only snacks --- Ritz crackers, chips, granola bars --- so it
wasn't worth bus fare. The Schuyler Inn didn't provide Wi-Fi, so her
daughter tried to connect to her schoolwork in a Burger King parking
lot.

Image

Rosy Romero, 26, and her three children: Alexander, 7; Azly, 2; and
Cristina, 9 months. Romero, who lost her babysitting job when the
pandemic hit, gets food from her church and a food
pantry.Credit...Brenda Ann Kenneally for The New York Times

If they go on for too long, temporary solutions become permanent. Food
banks become bureaucracies; hotels meant to hold the overflow of
shelters, like the Schuyler Inn, become homes. Public schools, which
have never reconciled their hours with the actual schedules of working
people, become essential hubs for entire communities.

On July 16, toward the end of her travels, Kenneally pulled up to the
fields of Hatch, N.M. Teodula Portillo, 47, had been up since 4 a.m. She
had allowed her teenage sons 20 more minutes to sleep and didn't wake
her 11-year-old twin daughters because they cannot work legally until
they are 12. By 5:30 a.m., Portillo and her boys were bent over picking
onions, for which they are paid by the bushel. Employers are required to
pay minimum wage only for certain tasks that are part of agricultural
work. Portillo receives SNAP intermittently --- if she earns too much,
they are not eligible. Kenneally knelt on the dirt and began shooting,
some 900 miles from Nipomo, where Lange took her iconic photograph,
which helped Americans discover the hunger that both she and Kenneally
knew too much about. The attachment to this discovering is as persistent
as the underlying social problems --- which to this day remain ignored.

Image

Rene Lopez, 46, helped open a food bank on the Pascua Yaqui reservation
in Tucson last year. ``I know how it is on the reservation,'' said
Lopez, who is half Native American. ``When you're out there, it's far,
there's no grocery stores nearby.''Credit...Brenda Ann Kenneally for The
New York Times

\begin{center}\rule{0.5\linewidth}{\linethickness}\end{center}

\textbf{Adrian Nicole LeBlanc}, an independent journalist and MacArthur
fellow, was embedded in an assisted-living facility as Kenneally began
her trip for this issue. They have worked together since 2003.
\textbf{Brenda Ann Kenneally} is a multimedia journalist who, over 30
years, has produced participatory media projects with families from her
home community, including ``Upstate Girls: Unraveling Collar City.'' She
is currently assembling a multimedia autobiography, charting her
experience from being a disenfranchised youth to becoming a Guggenheim
fellow and frequent contributor to the magazine.

\hypertarget{hunger-embed-1}{%
\subsection{Hunger Embed}\label{hunger-embed-1}}

\href{https://www.nytimes3xbfgragh.onion/interactive/2020/09/02/magazine/food-insecurity-hunger-us.html}{}

\hypertarget{america-at-hungers-edge-1}{%
\section{AMERICA AT HUNGER'S EDGE}\label{america-at-hungers-edge-1}}

See all the
photographs\includegraphics{https://static01.graylady3jvrrxbe.onion/newsgraphics/2020/08/17/hunger/assets/images/tiny_arrow_white.svg}

\includegraphics{https://static01.graylady3jvrrxbe.onion/images/2020/09/01/magazine/food-insecurity-hunger-us/food-insecurity-hunger-us-threeByTwoMediumAt2X.jpg}

Brenda Ann Kenneally for The New York Times

Advertisement

\protect\hyperlink{after-bottom}{Continue reading the main story}

\hypertarget{site-index}{%
\subsection{Site Index}\label{site-index}}

\hypertarget{site-information-navigation}{%
\subsection{Site Information
Navigation}\label{site-information-navigation}}

\begin{itemize}
\tightlist
\item
  \href{https://help.nytimes3xbfgragh.onion/hc/en-us/articles/115014792127-Copyright-notice}{©~2020~The
  New York Times Company}
\end{itemize}

\begin{itemize}
\tightlist
\item
  \href{https://www.nytco.com/}{NYTCo}
\item
  \href{https://help.nytimes3xbfgragh.onion/hc/en-us/articles/115015385887-Contact-Us}{Contact
  Us}
\item
  \href{https://www.nytco.com/careers/}{Work with us}
\item
  \href{https://nytmediakit.com/}{Advertise}
\item
  \href{http://www.tbrandstudio.com/}{T Brand Studio}
\item
  \href{https://www.nytimes3xbfgragh.onion/privacy/cookie-policy\#how-do-i-manage-trackers}{Your
  Ad Choices}
\item
  \href{https://www.nytimes3xbfgragh.onion/privacy}{Privacy}
\item
  \href{https://help.nytimes3xbfgragh.onion/hc/en-us/articles/115014893428-Terms-of-service}{Terms
  of Service}
\item
  \href{https://help.nytimes3xbfgragh.onion/hc/en-us/articles/115014893968-Terms-of-sale}{Terms
  of Sale}
\item
  \href{https://spiderbites.nytimes3xbfgragh.onion}{Site Map}
\item
  \href{https://help.nytimes3xbfgragh.onion/hc/en-us}{Help}
\item
  \href{https://www.nytimes3xbfgragh.onion/subscription?campaignId=37WXW}{Subscriptions}
\end{itemize}
