Sections

SEARCH

\protect\hyperlink{site-content}{Skip to
content}\protect\hyperlink{site-index}{Skip to site index}

\href{https://www.nytimes3xbfgragh.onion/section/health}{Health}

\href{https://myaccount.nytimes3xbfgragh.onion/auth/login?response_type=cookie\&client_id=vi}{}

\href{https://www.nytimes3xbfgragh.onion/section/todayspaper}{Today's
Paper}

\href{/section/health}{Health}\textbar{}2 College Students Dreamed Up an
A.L.S. Treatment. The Results Are In.

\url{https://nyti.ms/3lM6JMY}

\begin{itemize}
\item
\item
\item
\item
\item
\item
\end{itemize}

Advertisement

\protect\hyperlink{after-top}{Continue reading the main story}

Supported by

\protect\hyperlink{after-sponsor}{Continue reading the main story}

\hypertarget{2-college-students-dreamed-up-an-als-treatment-the-results-are-in}{%
\section{2 College Students Dreamed Up an A.L.S. Treatment. The Results
Are
In.}\label{2-college-students-dreamed-up-an-als-treatment-the-results-are-in}}

A study of their therapy and clinical trials of other experimental
treatments are offering glimmers of hope that paralysis from the
disorder can be slowed.

\includegraphics{https://static01.graylady3jvrrxbe.onion/images/2020/09/08/science/02ALS1/merlin_176266437_3cc8b3fd-fdbd-4abd-bfed-b7904b1055bf-articleLarge.jpg?quality=75\&auto=webp\&disable=upscale}

\href{https://www.nytimes3xbfgragh.onion/by/pam-belluck}{\includegraphics{https://static01.graylady3jvrrxbe.onion/images/2018/02/16/multimedia/author-pam-belluck/author-pam-belluck-thumbLarge-v2.png}}

By \href{https://www.nytimes3xbfgragh.onion/by/pam-belluck}{Pam Belluck}

\begin{itemize}
\item
  Published Sept. 2, 2020Updated Sept. 4, 2020
\item
  \begin{itemize}
  \item
  \item
  \item
  \item
  \item
  \item
  \end{itemize}
\end{itemize}

Seven years ago, Joshua Cohen, then a junior at Brown University
majoring in biomedical engineering, was captivated by the question of
why people develop brain disorders. ``How does a neuron die?'' he
wondered.

After poring over scientific studies, he sketched out his ideas for a
way to treat them. ``I was sitting in my dorm room and I had kind of
written out the research on these crazy-looking diagrams,'' he recalled.

A \href{http://www.nejm.org/doi/full/10.1056/NEJMoa1916945}{study
published on Wednesday} in the New England Journal of Medicine reported
that the experimental treatment he and another Brown student, Justin
Klee, conceived might hold promise for slowing progression of
amyotrophic lateral sclerosis, the ruthless disease that robs people of
their ability to move, speak, eat and ultimately breathe.

More than 50 clinical trials over 25 years have failed to find effective
treatments for A.L.S., also called Lou Gehrig's disease, which often
causes death within two to five years. But now, scientific advances and
an influx of funding are driving clinical trials for many potential
therapies, generating hope and intense discussion among patients,
doctors and researchers.

The new study reported that a two-drug combination slowed progression of
A.L.S. paralysis by about six weeks over about six months, approximately
25 percent more than a placebo. On average, patients on a placebo
declined in 18 weeks to a level that patients receiving the treatment
didn't reach until 24 weeks, said the principal investigator, Dr.
Sabrina Paganoni, a neuromuscular medicine specialist at Massachusetts
General Hospital's Healey \& AMG Center for A.L.S.

``It's such a terrible disease and as you can imagine, for the folks who
have it or the family members, it's just desperation that something's
going to work,'' said Dr. Walter Koroshetz, director of the National
Institute of Neurological Disorders and Stroke, who wasn't involved in
the new study. ``Any kind of slowing of progression for a patient with
A.L.S. might be valuable even though it's not a big effect.''

He and other experts were careful not to overstate the results and noted
that the drug wasn't shown to improve patients' condition or halt
decline. The study evaluated safety and efficacy in a Phase 2 trial with
137 participants, not as large and long as many Phase 3 trials often
required for regulatory approval. Experts and the authors themselves
said further trials were necessary.

Still, doctors and advocates said the relentlessness of the illness and
the availability of only two approved A.L.S. medications, neither
significantly effective, gives urgency to finding additional treatments.
The A.L.S. Association, an advocacy group, said that since the study
found the drug to be safe and patients can die waiting for other trials,
it should be made available to people with the disease as soon as
possible.

``That can mean the difference between being able to feed yourself
versus being fed or not needing a wheelchair versus needing a
wheelchair, and if we can delay that level of disability, that's a big
deal for our community,'' said Neil Thakur, chief mission officer of the
association, which helped finance the study.

The association will urge the Food and Drug Administration to grant
approval as soon as the company applies for it, and then require
rigorous follow-up studies. The group will also urge the company,
Amylyx, a Massachusetts start-up the students founded, to seek the
agency's permission to provide the drug for compassionate use while it
is still being evaluated.

\includegraphics{https://static01.graylady3jvrrxbe.onion/images/2020/09/08/science/02ALS2/merlin_176250909_004ee643-67c5-46d2-9bf0-cb1b35d4ae03-articleLarge.jpg?quality=75\&auto=webp\&disable=upscale}

A.L.S., the most common motor neuron disorder, diagnosed in
\href{https://alsnewstoday.com/how-common-is-als/\#:~:text=The\%20average\%20incidence\%20rate\%20of,ratio\%20of\%201.5\%20to\%201).}{about
6,000 people} worldwide each year, has drawn greater attention of late,
bolstered by prominent people with the disease, like
\href{https://www.nytimes3xbfgragh.onion/2018/03/14/obituaries/stephen-hawking-dead.html}{Stephen
Hawking}, the astrophysicist who died in 2018; Steve Gleason, a former
professional football player; and Ady Barkan, a health care activist who
used a computer-generated voice at this year's
\href{https://www.youtube.com/watch?v=XV7xSzXyaT8}{Democratic National
Convention}because he can no longer speak.

There is now
\href{https://alsnewstoday.com/2020/06/03/legislation-seeks-early-access-to-promising-treatments-for-als-similar-disorders/}{legislation
in Congress to accelerate A.L.S. therapy access} and
\href{https://www.nih.gov/news-events/news-releases/nih-announces-new-transformative-research-award-program-als}{a
\$25 million federal research program}. The
\href{https://www.washingtonpost.com/news/to-your-health/wp/2015/08/19/scientists-are-crediting-the-ice-bucket-challenge-for-breakthroughs-in-research/}{Ice
Bucket Challenge}, a 2014 fund-raising juggernaut featuring celebrities
and others dumping icy water on their heads, generated about \$220
million. More than 20 treatments are being tested, including stem cells,
immunotherapy and genetic therapies for the 10 percent of cases caused
by known mutations. Results from other trials are expected soon.

``This is a really exciting time,'' said Dr. Robert Miller, director of
clinical research at Forbes Norris MDA/A.L.S. Research Center at
California Pacific Medical Center, who is involved in several trials,
but not the new study.

Most of the study's participants were already taking one or both of the
approved A.L.S. medications:
\href{https://www.nytimes3xbfgragh.onion/1995/09/19/science/committee-endorses-new-drug-for-als.html}{riluzole},
which can extend survival by several months, and
\href{https://alsnewstoday.com/2017/05/08/fda-approves-radicava-first-new-als-therapy-in-20-years-and-cause-for-hope/}{edaravone},
which can slow progression by about 33 percent. It's possible the new
drug, AMX0035, provided additional benefit. Dr. Merit Cudkowicz, the
Healey Center's director and the study's senior author, said she
envisioned the new drug combination would be taken alongside existing
medications.

The study is the first clinical trial supported by Ice Bucket Challenge
money to publish results, said the A.L.S. Association. Amylyx financed
the bulk of the study and agreed to use a percentage of income from
sales of the drug to repay 150 percent of the association's grant to
fund more research.

Mr. Cohen's idea in 2013 was that a combination of taurursodiol, a
supplement, and sodium phenylbutyrate, a medication for a pediatric urea
disorder, could safeguard neurons by preventing dysfunction of two
structures in cells, mitochondria and the endoplasmic reticulum.

He quickly involved Mr. Klee, a senior neuroscience major who was a
fraternity brother and fellow player on the university's club tennis
team. Over cheap sparkling wine, ``we both said `let's start a
company,''' Mr. Klee said. ``We had no idea what we were doing.''

They heard skepticism from several experts they consulted until they met
with Rudolph Tanzi, a prominent Alzheimer's expert who had belonged to
their fraternity.

Image

First draft notes and diagrams of Mr. Cohen's and Mr. Klee's research
plans. The two met as undergraduates at Brown University.Credit...Cody
O'Loughlin for The New York Times

Dr. Tanzi told them to test whether the drug combination protected rat
neurons from a bleach-like chemical that kills them. With \$8,000 from a
university grant, their parents (two of whom are physicians) and
savings, they hired a professional lab, which found that their
combination salvaged 90 percent of neurons, Dr. Tanzi said.

``'That's impossible,''' he said he told them, urging more tests, which
showed 95 percent of neurons were saved.

``Guys, you got something here,'' Dr. Tanzi told them. He became an
Amylyx co-founder and leads its scientific advisory board.

The combination was christened AMX0035 because 3 and 5 are the favorite
numbers of Mr. Cohen's fiancée. During YMCA basketball sessions with Dr.
Tanzi, they discussed trying it for Alzheimer's. But investors weren't
interested.

Dr. Tanzi introduced the young men to Dr. Cudkowicz, who had once
studied sodium phenylbutyrate and convinced them to test it for A.L.S.
It's now also in an Alzheimer's trial.

The A.L.S. study, called Centaur, conducted across the country by
leading A.L.S. researchers, involved patients who developed symptoms
within 18 months before the trial and were affected in at least three
body regions, generally signs of fast-progressing disease. Two-thirds
received AMX0035, a bitter-tasting powder they mixed with water to drink
or ingest through a feeding tube twice daily.

The primary goal was slowing decline on a
\href{https://www.mdcalc.com/revised-amyotrophic-lateral-sclerosis-functional-rating-scale-alsfrs-r\#use-cases}{48-point
A.L.S. scale} rating 12 physical abilities, including walking, speech,
swallowing, dressing, handwriting and breathing. Over 24 weeks, patients
on placebo declined 2.32 points more than those taking the drug
combination. Fine motor skills benefited most.

``The data that we see here indicates there may be some beneficial
effect but it doesn't look like what you'd call a home run,'' Dr.
Koroshetz said.

Some patients experienced gastrointestinal side effects like nausea and
diarrhea, but after three weeks those effects largely subsided, and
overall, the drug was safe, researchers said.

In most secondary measures, including muscle strength, respiratory
ability and whether patients were hospitalized, AMX0035 appeared better
than placebo, although it wasn't statistically significant. Another
measure, a biomarker of neurodegeneration, didn't seem significantly
affected. A few patients died in both groups, but experts said
identifying the impact on mortality would require evaluation over a
longer period.

``This is very encouraging,'' said Dr. Neil Shneider, director of the
Eleanor and Lou Gehrig A.L.S. Center at Columbia University, who was not
involved. ``The question is, is the effect on function sustained beyond
the six-month trial period and does it have an effect on survival?''

Researchers said they would soon publish longer-term data because most
participants opted to take the drug combination after the trial, and
some have now taken it for over two years.

Experts were torn about whether F.D.A. approval should be granted, since
Phase 3 results are often required.

``From my heart, I'd say we are so desperate for meaningful treatment
for A.L.S. that something that looks as promising as this might well be
approved,'' Dr. Miller said. ``From my head, I'd say it could be chance.
We've seen that before where Phase 2 looked really good.''

Dr. Shneider noted that some patients have already been obtaining one or
both components from Europe or Asia and taking it themselves. ``There'll
be a lot of interest from patients and families to get out this drug,''
he said.

But experts also said that making the drug available soon might make it
difficult to recruit patients for subsequent trials. And insurers may
not cover drugs approved based on Phase 2 results, Dr. Koroshetz said.
Some patients have had difficulty getting insurance coverage for
edaravone, which costs about \$148,000 a year and was approved after a
Phase 3 trial of the same size and duration as Centaur. Amylyx officials
declined to provide a price estimate for their treatment.

In interviews, two trial participants said they believed AMX0035 was
beneficial. Given the unpredictable trajectory of the disease, they said
any specific effects were hard to describe. Neither knows if they
received the drug or placebo during the trial, but they've received the
treatment since.

Image

Mr. Teal, with his wife, Lauren, and dogs Lucy and Jack, at home. He
reports no side effects from taking AMX0035 and believes it may have
eased cramping in his neck, abdomen and legs.Credit...Aileen Perilla for
The New York Times

Mike Teal, 52, of Tallahassee, Fla., began having symptoms in 2016 and
has taken the drug since at least the spring of 2018, when his trial
ended. Soon after, he also started edaravone.

He currently has limited speech, needs a feeding tube, often uses a
wheelchair and requires a breathing machine every few hours. Last year,
he had to stop working at the gift and accessories store he owns with
his wife, Lauren.

He said he's had no negative side effects and believes the drug may have
eased cramps in his neck, abdomen and legs.

``I'm confident it has slowed my progression,'' he wrote in an email.
``But it's difficult to measure.''

Jeff Derby, 61, a retired forest products company manager in Cloverdale,
British Columbia, said that when he was diagnosed in July 2018, doctors
described his disease as relatively slow-progressing. He thinks his
decline has become more gradual in the 18 months he's been taking the
drug since his trial ended. Mr. Derby, who also takes the two approved
medications, said weakness in his left hand isn't worsening as quickly.

``I think AMX0035 will ultimately be part of a treatment cocktail like
there is for other diseases where you'll take three, four or five
different things, and as a group, they will help slow the progression to
the point where you can live a somewhat normal life,'' he said.

Advertisement

\protect\hyperlink{after-bottom}{Continue reading the main story}

\hypertarget{site-index}{%
\subsection{Site Index}\label{site-index}}

\hypertarget{site-information-navigation}{%
\subsection{Site Information
Navigation}\label{site-information-navigation}}

\begin{itemize}
\tightlist
\item
  \href{https://help.nytimes3xbfgragh.onion/hc/en-us/articles/115014792127-Copyright-notice}{©~2020~The
  New York Times Company}
\end{itemize}

\begin{itemize}
\tightlist
\item
  \href{https://www.nytco.com/}{NYTCo}
\item
  \href{https://help.nytimes3xbfgragh.onion/hc/en-us/articles/115015385887-Contact-Us}{Contact
  Us}
\item
  \href{https://www.nytco.com/careers/}{Work with us}
\item
  \href{https://nytmediakit.com/}{Advertise}
\item
  \href{http://www.tbrandstudio.com/}{T Brand Studio}
\item
  \href{https://www.nytimes3xbfgragh.onion/privacy/cookie-policy\#how-do-i-manage-trackers}{Your
  Ad Choices}
\item
  \href{https://www.nytimes3xbfgragh.onion/privacy}{Privacy}
\item
  \href{https://help.nytimes3xbfgragh.onion/hc/en-us/articles/115014893428-Terms-of-service}{Terms
  of Service}
\item
  \href{https://help.nytimes3xbfgragh.onion/hc/en-us/articles/115014893968-Terms-of-sale}{Terms
  of Sale}
\item
  \href{https://spiderbites.nytimes3xbfgragh.onion}{Site Map}
\item
  \href{https://help.nytimes3xbfgragh.onion/hc/en-us}{Help}
\item
  \href{https://www.nytimes3xbfgragh.onion/subscription?campaignId=37WXW}{Subscriptions}
\end{itemize}
