Sections

SEARCH

\protect\hyperlink{site-content}{Skip to
content}\protect\hyperlink{site-index}{Skip to site index}

\href{https://www.nytimes3xbfgragh.onion/section/books}{Books}

\href{https://myaccount.nytimes3xbfgragh.onion/auth/login?response_type=cookie\&client_id=vi}{}

\href{https://www.nytimes3xbfgragh.onion/section/todayspaper}{Today's
Paper}

\href{/section/books}{Books}\textbar{}`Vanguard' Spotlights the Black
Women Who Fought for the Vote

\url{https://nyti.ms/32U39YA}

\begin{itemize}
\item
\item
\item
\item
\item
\end{itemize}

Advertisement

\protect\hyperlink{after-top}{Continue reading the main story}

Supported by

\protect\hyperlink{after-sponsor}{Continue reading the main story}

\href{/column/books-of-the-times}{Books of The Times}

\hypertarget{vanguard-spotlights-the-black-women-who-fought-for-the-vote}{%
\section{`Vanguard' Spotlights the Black Women Who Fought for the
Vote}\label{vanguard-spotlights-the-black-women-who-fought-for-the-vote}}

By \href{https://www.nytimes3xbfgragh.onion/by/jennifer-szalai}{Jennifer
Szalai}

\begin{itemize}
\item
  Sept. 2, 2020
\item
  \begin{itemize}
  \item
  \item
  \item
  \item
  \item
  \end{itemize}
\end{itemize}

\includegraphics{https://static01.graylady3jvrrxbe.onion/images/2020/09/03/books/02BOOKMARTHAJONES1/02BOOKMARTHAJONES1-articleLarge.png?quality=75\&auto=webp\&disable=upscale}

Buy Book ▾

\begin{itemize}
\tightlist
\item
  \href{https://www.amazon.com/gp/search?index=books\&tag=NYTBSREV-20\&field-keywords=Vanguard+Martha+S.+Jones}{Amazon}
\item
  \href{https://du-gae-books-dot-nyt-du-prd.appspot.com/buy?title=Vanguard\&author=Martha+S.+Jones}{Apple
  Books}
\item
  \href{https://www.anrdoezrs.net/click-7990613-11819508?url=https\%3A\%2F\%2Fwww.barnesandnoble.com\%2Fw\%2F\%3Fean\%3D9781541618619}{Barnes
  and Noble}
\item
  \href{https://www.anrdoezrs.net/click-7990613-35140?url=https\%3A\%2F\%2Fwww.booksamillion.com\%2Fp\%2FVanguard\%2FMartha\%2BS.\%2BJones\%2F9781541618619}{Books-A-Million}
\item
  \href{https://bookshop.org/a/3546/9781541618619}{Bookshop}
\item
  \href{https://www.indiebound.org/book/9781541618619?aff=NYT}{Indiebound}
\end{itemize}

When you purchase an independently reviewed book through our site, we
earn an affiliate commission.

Crisis and upheaval can often spur a growing interest in reconsidering
the myths of the past; triumphalist tales of a march toward national
greatness sound increasingly hollow next to the irrepressible weight of
reality.

As Americans mark the centennial of the 19th Amendment, which states
that a citizen's suffrage can't be denied ``on account of sex,'' the
long-held gloss that it guarantees women the right to vote has come
under pointed scrutiny. Scholars like
\href{https://www.nytimes3xbfgragh.onion/2019/01/04/obituaries/rosalyn-terborg-penn-dead.html}{Rosalyn
Terborg-Penn} and Lisa Tetrault have already shown how this history is
in fact more vexed and exclusionary than the popular narrative allows.
It's a truth that feels especially immediate now, when Americans face an
election during a pandemic with the Postal Service under attack and
\href{https://www.nytimes3xbfgragh.onion/2018/09/12/books/review-one-person-no-vote-carol-anderson.html}{without
the full protections of the Voting Rights Act}.

In ``Vanguard: How Black Women Broke Barriers, Won the Vote, and
Insisted on Equality For All,'' the historian Martha S. Jones writes
about the 19th Amendment in a chapter she simply calls ``Amendment.''
The title is appropriately minimalist and matter-of-fact. ``For Black
women, ratification of the 19th Amendment was not a guarantee of the
vote, but it was a clarifying moment,'' she says. Jim Crow made voting
in the South as fraught, dangerous and generally impossible for Black
women as it had long been for Black men. What happened in 1920 wasn't a
grand finale but an inflection point; there was still too much work to
do. ``Black women,'' Jones writes, ``were the new keepers of voting
rights in the United States.''

Jones has written an elegant and expansive history of Black women who
sought to build political power where they could. Instead of beginning
with the Seneca Falls Convention of 1848 --- where women gathered to
draft a fiery declaration of rights and the only Black person whose
presence was recorded was Frederick Douglass --- Jones opens a couple of
decades earlier, with Jarena Lee, the first woman authorized to preach
by the African Methodist Episcopal Church.

Women like Lee at the time could expect some digs at their reputation,
even from the more enlightened men in their community. As the editors of
the African-American weekly Freedom's Journal put it: ``A woman, in a
passion, is disgusting to her friends.'' One minister compared women
preachers to male alcoholics: Both, he said, neglected their household
obligations. With the so-called colored convention movement that began
in the 1830s, women were welcomed to public life, but mainly as
helpmeets. ``These men encouraged women's work, but not their
leadership,'' Jones writes.

Image

Martha S. Jones, whose new book is ``Vanguard: How Black Women Broke
Barriers, Won the Vote, and Insisted on Equality For
All.''Credit...Johns Hopkins University

Jones recounts how Lee and others cultivated their own ``power of
persuasion,'' whether they chose the pulpit, podium or pen. She includes
some ``firsts'' like Lee, but in a sense ``Vanguard'' is a rebuke to our
fixation on firsts. Jones is just as interested in everything these
women made possible --- not just the trails they blazed, but the
journeys they took, and what came after.

Suffrage may have been one goal, but there were more immediately
pressing concerns. An issue that keeps coming up for the women in
Jones's book is transportation --- or, as Jones says, ``traveling while
Black.'' When traversing the country to speak or preach, Black women
often faced impositions on their freedom to move.

The poet and abolitionist Frances Ellen Watkins Harper recalled the
insults thrown her way, the demands that she give up her seat on the
train or disembark entirely when trying to navigate the lecture circuit.
Speaking before the American Equal Rights Association in 1866, Harper
brought up what Jones calls ``the terror of the ladies' car.'' ``You
white women speak of rights,'' Harper said. ``I speak of wrongs.''
Getting the ballot could never be the panacea some suffragists made it
out to be as long as ``there exists this brutal element in society which
tramples upon the feeble and treads down the weak.''

``Vanguard'' includes a number of such iconic moments: Ida B. Wells
marching with her Illinois state delegation in the 1913 suffragist
parade, in defiance of white organizers who told Black women they would
have to march in an all-Black assembly at the back; Fannie Lou Hamer at
the 1964 Democratic National Convention, recalling how a vicious beating
in a Mississippi jail left her with permanent kidney damage and
blindness in one eye.

But Jones also introduces us to formidable women who haven't been
enshrined in popular memory, like Maggie Hood-Banks, a bishop's daughter
who combined forceful moral suasion with a sly wit. Hood-Banks would
take familiar lines and torque them to her argument's advantage. In
1900, at a conference for A.M.E. Zion Church, she made a play on the
language of the notorious Dred Scott decision when she declared: ``For
centuries woman was considered inferior to man, and in view of this fact
had no rights man was bound to respect.'' She also warned that
churchwomen were ``getting very tired of `taxation without
representation.'''

Jones is an assiduous scholar and an absorbing writer, turning to the
archives to unearth the stories of Black women who worked alongside
white suffragists only to be marginalized, in what often amounted to a
``dirty compromise with white supremacy.'' In a conversation with other
historians last year, when the subject of impending celebrations for the
19th Amendment came up, Jones warned against yielding to the ``tug of
mythmaking and sanitization that these sort of rituals require.''
Occasionally ``Vanguard'' slips into the kind of sweeping register that
invited Jones's skepticism, with refrains about Black women working to
``serve all humanity,'' or reaching for ``cures for what ailed all
humanity.'' But for the most part she allows the history to unfurl with
all of its twists and complexity.

In the book's introduction, she writes movingly about the women in her
own family, including her grandmother Susie, who arrived with her
husband and four children in Greensboro, N.C., in 1926. Susie, like her
mother and grandmother before her, was a woman ``of learning, status and
enough savvy to navigate the maze that led to the ballot box.'' But
Jones was never able to find the records that would tell her whether
Susie exercised her newly won right to vote. Still, as Jones recalled in
\href{https://www.nytimes3xbfgragh.onion/2020/08/14/us/suffrage-segregation-voting-black-women-19th-amendment.html}{a
recent essay for The Times}: ``For my grandmother, the 19th Amendment
was only a starting place.''

Advertisement

\protect\hyperlink{after-bottom}{Continue reading the main story}

\hypertarget{site-index}{%
\subsection{Site Index}\label{site-index}}

\hypertarget{site-information-navigation}{%
\subsection{Site Information
Navigation}\label{site-information-navigation}}

\begin{itemize}
\tightlist
\item
  \href{https://help.nytimes3xbfgragh.onion/hc/en-us/articles/115014792127-Copyright-notice}{©~2020~The
  New York Times Company}
\end{itemize}

\begin{itemize}
\tightlist
\item
  \href{https://www.nytco.com/}{NYTCo}
\item
  \href{https://help.nytimes3xbfgragh.onion/hc/en-us/articles/115015385887-Contact-Us}{Contact
  Us}
\item
  \href{https://www.nytco.com/careers/}{Work with us}
\item
  \href{https://nytmediakit.com/}{Advertise}
\item
  \href{http://www.tbrandstudio.com/}{T Brand Studio}
\item
  \href{https://www.nytimes3xbfgragh.onion/privacy/cookie-policy\#how-do-i-manage-trackers}{Your
  Ad Choices}
\item
  \href{https://www.nytimes3xbfgragh.onion/privacy}{Privacy}
\item
  \href{https://help.nytimes3xbfgragh.onion/hc/en-us/articles/115014893428-Terms-of-service}{Terms
  of Service}
\item
  \href{https://help.nytimes3xbfgragh.onion/hc/en-us/articles/115014893968-Terms-of-sale}{Terms
  of Sale}
\item
  \href{https://spiderbites.nytimes3xbfgragh.onion}{Site Map}
\item
  \href{https://help.nytimes3xbfgragh.onion/hc/en-us}{Help}
\item
  \href{https://www.nytimes3xbfgragh.onion/subscription?campaignId=37WXW}{Subscriptions}
\end{itemize}
