Sections

SEARCH

\protect\hyperlink{site-content}{Skip to
content}\protect\hyperlink{site-index}{Skip to site index}

\href{https://www.nytimes3xbfgragh.onion/section/business}{Business}

\href{https://myaccount.nytimes3xbfgragh.onion/auth/login?response_type=cookie\&client_id=vi}{}

\href{https://www.nytimes3xbfgragh.onion/section/todayspaper}{Today's
Paper}

\href{/section/business}{Business}\textbar{}Inflation Is Higher Than the
Numbers Say

\url{https://nyti.ms/3lKQ3FY}

\begin{itemize}
\item
\item
\item
\item
\item
\end{itemize}

\hypertarget{the-coronavirus-outbreak}{%
\subsubsection{\texorpdfstring{\href{https://www.nytimes3xbfgragh.onion/news-event/coronavirus?name=styln-coronavirus-markets\&region=TOP_BANNER\&block=storyline_menu_recirc\&action=click\&pgtype=Article\&impression_id=25791370-f1c3-11ea-a0c5-d9bb321ace9a\&variant=undefined}{The
Coronavirus
Outbreak}}{The Coronavirus Outbreak}}\label{the-coronavirus-outbreak}}

\begin{itemize}
\tightlist
\item
  live\href{https://www.nytimes3xbfgragh.onion/2020/09/08/world/covid-19-coronavirus.html?name=styln-coronavirus-markets\&region=TOP_BANNER\&block=storyline_menu_recirc\&action=click\&pgtype=Article\&impression_id=25791371-f1c3-11ea-a0c5-d9bb321ace9a\&variant=undefined}{Latest
  Updates}
\item
  \href{https://www.nytimes3xbfgragh.onion/interactive/2020/us/coronavirus-us-cases.html?name=styln-coronavirus-markets\&region=TOP_BANNER\&block=storyline_menu_recirc\&action=click\&pgtype=Article\&impression_id=25793a80-f1c3-11ea-a0c5-d9bb321ace9a\&variant=undefined}{Maps
  and Cases}
\item
  \href{https://www.nytimes3xbfgragh.onion/interactive/2020/science/coronavirus-vaccine-tracker.html?name=styln-coronavirus-markets\&region=TOP_BANNER\&block=storyline_menu_recirc\&action=click\&pgtype=Article\&impression_id=25793a81-f1c3-11ea-a0c5-d9bb321ace9a\&variant=undefined}{Vaccine
  Tracker}
\item
  \href{https://www.nytimes3xbfgragh.onion/2020/09/02/your-money/eviction-moratorium-covid.html?name=styln-coronavirus-markets\&region=TOP_BANNER\&block=storyline_menu_recirc\&action=click\&pgtype=Article\&impression_id=25793a82-f1c3-11ea-a0c5-d9bb321ace9a\&variant=undefined}{Eviction
  Moratorium}
\item
  \href{https://www.nytimes3xbfgragh.onion/interactive/2020/09/02/magazine/food-insecurity-hunger-us.html?name=styln-coronavirus-markets\&region=TOP_BANNER\&block=storyline_menu_recirc\&action=click\&pgtype=Article\&impression_id=25793a83-f1c3-11ea-a0c5-d9bb321ace9a\&variant=undefined}{American
  Hunger}
\end{itemize}

Advertisement

\protect\hyperlink{after-top}{Continue reading the main story}

Supported by

\protect\hyperlink{after-sponsor}{Continue reading the main story}

\hypertarget{inflation-is-higher-than-the-numbers-say}{%
\section{Inflation Is Higher Than the Numbers
Say}\label{inflation-is-higher-than-the-numbers-say}}

While government statistics say inflation is low, the reality is that
the cost of living has risen during the pandemic, especially for poorer
Americans.

\includegraphics{https://static01.graylady3jvrrxbe.onion/images/2020/09/06/business/01Virus-View-01/01Virus-View-01-articleLarge-v2.jpg?quality=75\&auto=webp\&disable=upscale}

By Justin Wolfers

\begin{itemize}
\item
  Sept. 2, 2020
\item
  \begin{itemize}
  \item
  \item
  \item
  \item
  \item
  \end{itemize}
\end{itemize}

The latest inflation statistics say prices have risen by only 1 percent
over the past year. But there's something wrong with those numbers
because the pandemic has made economic life more expensive in ways the
official bean counters aren't capturing.

This distortion has led other economic statistics to paint an
artificially rosy picture of our current situation. The problem is that
measures like real output, real wages and poverty are calculated using
inflation adjustments that don't reflect the higher cost of living
during a pandemic. This might help explain why measured poverty has
fallen even as lines at food banks have grown.

The government's approach to measuring inflation is straightforward
enough. The Bureau of Labor Statistics tracks the price of a basket of
goods and services that is intended to represent average American
patterns. The inflation rate is the monthly percentage change in that
price.

But no economic statistic can perfectly track the cost of living. In
normal times, the government's numbers are thought to overstate the true
rate of inflation. But the pandemic has upended the economy in ways that
have reversed these biases, so that the official statistics are now an
underestimate.

\hypertarget{people-are-buying-more-of-those-goods-whose-prices-are-rising-the-fastest}{%
\subsection{People are buying more of those goods whose prices are
rising the
fastest.}\label{people-are-buying-more-of-those-goods-whose-prices-are-rising-the-fastest}}

The Consumer Price Index tracks the cost of a fixed basket of goods, but
people constantly change what they buy. This ``substitution bias''
usually leads inflation statistics to overstate changes in the cost of
living, because people tend to substitute lower-cost alternatives when
prices rise.

But since the coronavirus hit, people are buying more of the essentials,
like groceries, forcing their prices up. And they're buying fewer
airline tickets and less gasoline and clothing, pushing those prices
down.

Alberto Cavallo, an economist at Harvard Business School, has mined
credit and debit card data and found that these changing buying patterns
are especially important for low-income households, which devote a
larger share of their spending to food.

\hypertarget{latest-updates-the-coronavirus-outbreak-and-the-economy}{%
\section{\texorpdfstring{\href{https://www.nytimes3xbfgragh.onion/live/2020/09/08/business/stock-market-today-coronavirus?action=click\&pgtype=Article\&state=default\&region=MAIN_CONTENT_1\&context=storylines_live_updates}{Latest
Updates: The Coronavirus Outbreak and the
Economy}}{Latest Updates: The Coronavirus Outbreak and the Economy}}\label{latest-updates-the-coronavirus-outbreak-and-the-economy}}

\href{https://www.nytimes3xbfgragh.onion/live/2020/09/08/business/stock-market-today-coronavirus?action=click\&pgtype=Article\&state=default\&region=MAIN_CONTENT_1\&context=storylines_live_updates\#elon-musk-says-the-new-electric-vw-is-pretty-good-for-a-non-sporty-car-that-is}{8m
ago}

\href{https://www.nytimes3xbfgragh.onion/live/2020/09/08/business/stock-market-today-coronavirus?action=click\&pgtype=Article\&state=default\&region=MAIN_CONTENT_1\&context=storylines_live_updates\#elon-musk-says-the-new-electric-vw-is-pretty-good-for-a-non-sporty-car-that-is}{Elon
Musk says the new electric VW is `pretty good. For a `non-sporty' car,
that is.}

\href{https://www.nytimes3xbfgragh.onion/live/2020/09/08/business/stock-market-today-coronavirus?action=click\&pgtype=Article\&state=default\&region=MAIN_CONTENT_1\&context=storylines_live_updates\#wage-violations-have-spiked-as-low-paid-workers-become-more-vulnerable-study-says}{17m
ago}

\href{https://www.nytimes3xbfgragh.onion/live/2020/09/08/business/stock-market-today-coronavirus?action=click\&pgtype=Article\&state=default\&region=MAIN_CONTENT_1\&context=storylines_live_updates\#wage-violations-have-spiked-as-low-paid-workers-become-more-vulnerable-study-says}{Wage
violations have spiked as low-paid workers become more vulnerable, study
says.}

\href{https://www.nytimes3xbfgragh.onion/live/2020/09/08/business/stock-market-today-coronavirus?action=click\&pgtype=Article\&state=default\&region=MAIN_CONTENT_1\&context=storylines_live_updates\#european-stocks-slide-with-worries-over-us-china-trade-and-airlines}{17m
ago}

\href{https://www.nytimes3xbfgragh.onion/live/2020/09/08/business/stock-market-today-coronavirus?action=click\&pgtype=Article\&state=default\&region=MAIN_CONTENT_1\&context=storylines_live_updates\#european-stocks-slide-with-worries-over-us-china-trade-and-airlines}{European
stocks slide, with worries over U.S.-China trade and airlines.}

\href{https://www.nytimes3xbfgragh.onion/live/2020/09/08/business/stock-market-today-coronavirus?action=click\&pgtype=Article\&state=default\&region=MAIN_CONTENT_1\&context=storylines_live_updates}{See
more updates}

More live coverage:
\href{https://www.nytimes3xbfgragh.onion/2020/09/08/world/covid-19-coronavirus.html?action=click\&pgtype=Article\&state=default\&region=MAIN_CONTENT_1\&context=storylines_live_updates}{Global}

While poorer people often try to keep the cost of living down by buying
whichever brand happens to be discounted that week, the pandemic appears
to have reduced the number of discounts.

\hypertarget{the-pandemic-has-changed-where-and-how-people-shop}{%
\subsection{The pandemic has changed where and how people
shop.}\label{the-pandemic-has-changed-where-and-how-people-shop}}

\includegraphics{https://static01.graylady3jvrrxbe.onion/images/2020/09/01/business/01Virus-View-02/merlin_171539790_7db42bdb-edcd-4699-87fd-8135908a9688-articleLarge.jpg?quality=75\&auto=webp\&disable=upscale}

The price that Costco charges for Cheerios has not changed much, but
during the pandemic my family is visiting Costco less and relying on
Instacart instead, which charges a premium for delivery.

The Cheerios are identical --- Instacart even picks them up from Costco!
--- but Instacart charges a premium over Costco's low prices. While my
weekly box of Cheerios has become more expensive, the government
statistics infer that if neither the price of Cheerios at Costco nor the
price of Cheerios from Instacart has changed, there must be no
Cheerios-related inflation. My wallet disagrees.

In general, the risks associated with in-person shopping have led many
people to shop around less, or to switch to more expensive online or
delivery options, and they're often also adding a healthy tip. As a
result, the cost of living for many families has risen in ways that the
Consumer Price Index fails to capture. (By contrast, in normal times
people tend to flock to cheaper outlets, so this ``outlet bias'' usually
leads official statistics to understate inflation.)

\hypertarget{the-quality-of-many-services-has-gotten-worse}{%
\subsection{The quality of many services has gotten
worse.}\label{the-quality-of-many-services-has-gotten-worse}}

Typically, many businesses add new features, or improve the quality of
their products. Government statisticians try to make adjustments for how
the changing quality of goods affects the price. It's hard to account
for every change, so these unmeasured quality improvements often lead
the statistical authorities to overstate how fast the cost of living is
rising.

But the pandemic has forced many businesses to switch to producing
lower-quality products. That restaurant meal you might have enjoyed with
table service and mood lighting is now offered in a foam container to
eat at your kitchen counter. Your therapist might be available over
Zoom, but is it really an adequate substitute? And few colleges are
offering discounts this year, even as they're moving to largely online
instruction, which my students tell me is a vastly inferior product.

Even as the price tags on these products haven't changed much, their
quality has declined, which is a hidden form of price increase ignored
by the official inflation numbers.

\hypertarget{variety-has-decreased}{%
\subsection{Variety has decreased.}\label{variety-has-decreased}}

We're living in a time of shortages. Bikes, dumbbells, bread makers (or
even just yeast), camping equipment or Nintendos are hard to find at any
price.

While you might view this as an effective price hike (to infinity and
beyond!), the government statisticians don't see it this way. They
impute the price of what a sold-out good would have been if available,
effectively ignoring the inflationary costs of pandemic-fueled
shortages. The problem is that fewer choices give you fewer options for
achieving a given quality of life.

In normal times, businesses continually introduce new varieties that the
inflation numbers fail to capture, creating a ``variety bias,'' which
overstates the true rise in the cost of living. But the pandemic has led
to a sharp reduction in the number of goods and services available ---
including child care --- which effectively amounts to a hidden increase
in the cost of living.

\hypertarget{the-true-inflation-rate-has-risen}{%
\subsection{The true inflation rate has
risen.}\label{the-true-inflation-rate-has-risen}}

Put these biases together and it appears that the cost of living has
risen substantially faster than the official inflation numbers. These
numbers directly affect people's lives, because Social Security benefits
and other programs are tied to inflation numbers that aren't keeping
pace with the higher cost of living.

Peter Klenow, a Stanford economist, told me by email that much of this
distortion was ``likely to be transitory.'' Once the pandemic recedes,
he said, ``most of the lost varieties will eventually be regained,'' and
``spending patterns will largely revert'' to their earlier patterns.

But this temporary shift matters because it changes the diagnosis of our
current economic ills. If inflation were falling --- as the official
statistics suggest --- that would be read as evidence of insufficient
demand, as people cut back on their spending. But if inflation is really
rising, that suggests that supply side disruptions are a bigger problem
than is widely appreciated. The source of these supply disruptions is in
plain sight: It's the virus.

This diagnosis says that fiscal and monetary policies to bolster demand
will be most effective when paired with effective public health measures
that make it safe for suppliers to get back to business. The payoff to
beating the bug isn't just that it'll save lives. It may also save the
economy.

Justin Wolfers is a professor of economics and public policy at the
University of Michigan and a host of the ``Think Like an Economist''
podcast. Follow him on Twitter:
\href{https://twitter.com/JustinWolfers?ref_src=twsrc\%5Egoogle\%7Ctwcamp\%5Eserp\%7Ctwgr\%5Eauthor}{@justinwolfers}

Advertisement

\protect\hyperlink{after-bottom}{Continue reading the main story}

\hypertarget{site-index}{%
\subsection{Site Index}\label{site-index}}

\hypertarget{site-information-navigation}{%
\subsection{Site Information
Navigation}\label{site-information-navigation}}

\begin{itemize}
\tightlist
\item
  \href{https://help.nytimes3xbfgragh.onion/hc/en-us/articles/115014792127-Copyright-notice}{©~2020~The
  New York Times Company}
\end{itemize}

\begin{itemize}
\tightlist
\item
  \href{https://www.nytco.com/}{NYTCo}
\item
  \href{https://help.nytimes3xbfgragh.onion/hc/en-us/articles/115015385887-Contact-Us}{Contact
  Us}
\item
  \href{https://www.nytco.com/careers/}{Work with us}
\item
  \href{https://nytmediakit.com/}{Advertise}
\item
  \href{http://www.tbrandstudio.com/}{T Brand Studio}
\item
  \href{https://www.nytimes3xbfgragh.onion/privacy/cookie-policy\#how-do-i-manage-trackers}{Your
  Ad Choices}
\item
  \href{https://www.nytimes3xbfgragh.onion/privacy}{Privacy}
\item
  \href{https://help.nytimes3xbfgragh.onion/hc/en-us/articles/115014893428-Terms-of-service}{Terms
  of Service}
\item
  \href{https://help.nytimes3xbfgragh.onion/hc/en-us/articles/115014893968-Terms-of-sale}{Terms
  of Sale}
\item
  \href{https://spiderbites.nytimes3xbfgragh.onion}{Site Map}
\item
  \href{https://help.nytimes3xbfgragh.onion/hc/en-us}{Help}
\item
  \href{https://www.nytimes3xbfgragh.onion/subscription?campaignId=37WXW}{Subscriptions}
\end{itemize}
