Sections

SEARCH

\protect\hyperlink{site-content}{Skip to
content}\protect\hyperlink{site-index}{Skip to site index}

\href{https://www.nytimes3xbfgragh.onion/section/well/move}{Move}

\href{https://myaccount.nytimes3xbfgragh.onion/auth/login?response_type=cookie\&client_id=vi}{}

\href{https://www.nytimes3xbfgragh.onion/section/todayspaper}{Today's
Paper}

\href{/section/well/move}{Move}\textbar{}For Successful Aging, Pick Up
the Pace or Mix It Up

\url{https://nyti.ms/32Ngkuu}

\begin{itemize}
\item
\item
\item
\item
\item
\item
\end{itemize}

\href{https://www.nytimes3xbfgragh.onion/spotlight/at-home?action=click\&pgtype=Article\&state=default\&region=TOP_BANNER\&context=at_home_menu}{At
Home}

\begin{itemize}
\tightlist
\item
  \href{https://www.nytimes3xbfgragh.onion/2020/09/07/travel/route-66.html?action=click\&pgtype=Article\&state=default\&region=TOP_BANNER\&context=at_home_menu}{Cruise
  Along: Route 66}
\item
  \href{https://www.nytimes3xbfgragh.onion/2020/09/04/dining/sheet-pan-chicken.html?action=click\&pgtype=Article\&state=default\&region=TOP_BANNER\&context=at_home_menu}{Roast:
  Chicken With Plums}
\item
  \href{https://www.nytimes3xbfgragh.onion/2020/09/04/arts/television/dark-shadows-stream.html?action=click\&pgtype=Article\&state=default\&region=TOP_BANNER\&context=at_home_menu}{Watch:
  Dark Shadows}
\item
  \href{https://www.nytimes3xbfgragh.onion/interactive/2020/at-home/even-more-reporters-editors-diaries-lists-recommendations.html?action=click\&pgtype=Article\&state=default\&region=TOP_BANNER\&context=at_home_menu}{Explore:
  Reporters' Google Docs}
\end{itemize}

Advertisement

\protect\hyperlink{after-top}{Continue reading the main story}

Supported by

\protect\hyperlink{after-sponsor}{Continue reading the main story}

Phys Ed

\hypertarget{for-successful-aging-pick-up-the-pace-or-mix-it-up}{%
\section{For Successful Aging, Pick Up the Pace or Mix It
Up}\label{for-successful-aging-pick-up-the-pace-or-mix-it-up}}

Older people who cycled for exercise walked more efficiently than people
whose primary exercise is placid walking.

\includegraphics{https://static01.graylady3jvrrxbe.onion/images/2020/09/08/well/physed-cycle-walk/physed-cycle-walk-articleLarge-v2.jpg?quality=75\&auto=webp\&disable=upscale}

\href{https://www.nytimes3xbfgragh.onion/by/gretchen-reynolds}{\includegraphics{https://static01.graylady3jvrrxbe.onion/images/2019/03/18/multimedia/author-gretchen-reynolds/author-gretchen-reynolds-thumbLarge.png}}

By
\href{https://www.nytimes3xbfgragh.onion/by/gretchen-reynolds}{Gretchen
Reynolds}

\begin{itemize}
\item
  Published Sept. 2, 2020Updated Sept. 3, 2020
\item
  \begin{itemize}
  \item
  \item
  \item
  \item
  \item
  \item
  \end{itemize}
\end{itemize}

If we want to walk easily and well as we age, we may need to do more
than just stroll now, according to an eye-opening new study of older
walkers and cyclists. The study finds that people who often cycle for
exercise can walk more efficiently than people whose primary exercise is
placid walking, even if everyone works out for the same amount of time.

The results, which build on earlier work involving walkers and runners,
raise important questions about whether gentle exercise, by itself, is
enough for our well-being or if we might need, at least sometimes, to
add oomph to our workouts.

In general, exercise science shows that doing something --- anything ---
physical is much better for our health and longevity than doing nothing.
A raft of epidemiological studies indicate that if men and women start
moving just enough to lift themselves out of the group of people who are
the most sedentary, they get greater reductions in their risks for
chronic diseases and premature death than if a marathon runner crams in
a few additional, weekly miles.

But most of us are not completely inactive or in constant motion and,
for us, many questions remain about the ideal mix of duration, intensity
or type of exercise to elevate our fitness and health. Can we get away
with an occasional amble around the block? Or should we keep going for a
longer period of time? And is it important to intentionally get out of
breath on occasion?

Intrigued by those concerns, a group of exercise scientists at Humboldt
State University in California and the University of Colorado at Boulder
began to wonder recently about walking and whether it might tell us
something about workouts and ideal intensities.

In general, most of us can walk from the time we are small and probably
expect to continue to be able to walk for most of our lives. But past
biomechanics studies show that people tend to become physiologically
inefficient walkers with age, using more oxygen to walk at the same pace
as younger people. In practical terms, this rising inefficiency would
make walking feel harder and more tiring, perhaps prompting older people
to walk less, sit more and potentially become frail.

The researchers speculated that exercise might maintain walking
efficiency in older people, although what type of exercise was not
clear.
So,\href{https://journals.plos.org/plosone/article?id=10.1371/journal.pone.0113471}{for
a study published in 2014 in PLoS One,} they invited healthy walkers and
runners who were 65 or older to the lab and asked them to walk on a
treadmill at various speeds while wearing headgear to measure their
oxygen consumption.

They then compared the runners' and walkers' efficiency and crosschecked
those results against similar data from earlier experiments with
sedentary college students and retirees. It turned out that older
runners were quite efficient walkers, using about the same amount of
oxygen to walk as young people. But the older walkers had lost a step,
physiologically, requiring about 7 to 10 percent more oxygen to walk at
the same pace as the runners or the students. Their efficiency matched
that of the older men and women who rarely exercised at all.

Now, for the \href{https://pubmed.ncbi.nlm.nih.gov/32723930/}{new study,
which was published in July in the Journal of Aging and Physical
Activity,} the researchers set out to see if a different exercise, in
this case cycling, might likewise affect the ease of walking. They
recruited older riders and walkers and asked them how strenuously they
felt they worked out, on a scale of 1 to 3, from easy to tiring. The
walkers' reported intensity hovered at just under 2, while the
cyclists', as a group, neared 3. The researchers also brought in a group
of healthy young people as a control.

Everyone then walked on a treadmill at paces ranging up to about 4 miles
per hour while the researchers tracked their oxygen consumption. And, as
with the runners, the older cyclists walked well, their efficiency
matching that of the young people. But the older walkers' efficiency was
as much as 17 percent lower.

In effect, walking for exercise seemed not to have ``supplied sufficient
physical stimulus'' to maintain people's ability to walk easily as they
aged, says Justus Ortega, a professor at Humboldt State University who
co-authored both studies. Running and cycling were associated with more
efficient walking than regular walking was.

The studies did not delve into how cycling or running might have
affected people's walking efficiency. But Dr. Ortega says he and his
colleagues suspect that the more demanding exertions boosted the health
and function of mitochondria inside muscle cells in ways that gentler
walking did not. Mitochondria affect how cells make and utilize energy.
Healthier mitochondria should contribute to more efficient movement.

Of course, these studies were single snapshots of people's lives, and do
not show that running or cycling directly caused people to be efficient
walkers, only that the activities were related. They also did not look
at middle-aged people and whether different types of exercise then might
affect how well people walk later.

But Dr. Ortega says he believes the studies' findings can be both
cautionary and encouraging, suggesting that, while any physical activity
is worthwhile, pushing yourself a bit now might yield lasting benefits
for health and mobility. So, if you currently stroll for exercise, he
says, perhaps consider cycling or jogging sometimes, too, if possible.
Or add hills to your usual walking route, or, at least for a block or
three, pick up the pace.

Advertisement

\protect\hyperlink{after-bottom}{Continue reading the main story}

\hypertarget{site-index}{%
\subsection{Site Index}\label{site-index}}

\hypertarget{site-information-navigation}{%
\subsection{Site Information
Navigation}\label{site-information-navigation}}

\begin{itemize}
\tightlist
\item
  \href{https://help.nytimes3xbfgragh.onion/hc/en-us/articles/115014792127-Copyright-notice}{©~2020~The
  New York Times Company}
\end{itemize}

\begin{itemize}
\tightlist
\item
  \href{https://www.nytco.com/}{NYTCo}
\item
  \href{https://help.nytimes3xbfgragh.onion/hc/en-us/articles/115015385887-Contact-Us}{Contact
  Us}
\item
  \href{https://www.nytco.com/careers/}{Work with us}
\item
  \href{https://nytmediakit.com/}{Advertise}
\item
  \href{http://www.tbrandstudio.com/}{T Brand Studio}
\item
  \href{https://www.nytimes3xbfgragh.onion/privacy/cookie-policy\#how-do-i-manage-trackers}{Your
  Ad Choices}
\item
  \href{https://www.nytimes3xbfgragh.onion/privacy}{Privacy}
\item
  \href{https://help.nytimes3xbfgragh.onion/hc/en-us/articles/115014893428-Terms-of-service}{Terms
  of Service}
\item
  \href{https://help.nytimes3xbfgragh.onion/hc/en-us/articles/115014893968-Terms-of-sale}{Terms
  of Sale}
\item
  \href{https://spiderbites.nytimes3xbfgragh.onion}{Site Map}
\item
  \href{https://help.nytimes3xbfgragh.onion/hc/en-us}{Help}
\item
  \href{https://www.nytimes3xbfgragh.onion/subscription?campaignId=37WXW}{Subscriptions}
\end{itemize}
