Sections

SEARCH

\protect\hyperlink{site-content}{Skip to
content}\protect\hyperlink{site-index}{Skip to site index}

\href{https://www.nytimes3xbfgragh.onion/section/nyregion}{New York}

\href{https://myaccount.nytimes3xbfgragh.onion/auth/login?response_type=cookie\&client_id=vi}{}

\href{https://www.nytimes3xbfgragh.onion/section/todayspaper}{Today's
Paper}

\href{/section/nyregion}{New York}\textbar{}`Nobody Likes Snitching':
How Rules Against Parties Are Dividing Campuses

\url{https://nyti.ms/3jCPex0}

\begin{itemize}
\item
\item
\item
\item
\item
\item
\end{itemize}

\hypertarget{school-reopenings}{%
\subsubsection{\texorpdfstring{\href{https://www.nytimes3xbfgragh.onion/spotlight/schools-reopening?name=styln-coronavirus-schools-reopening\&region=TOP_BANNER\&block=storyline_menu_recirc\&action=click\&pgtype=Article\&impression_id=3bd86cc0-f27b-11ea-9a21-295489aede86\&variant=undefined}{School
Reopenings}}{School Reopenings}}\label{school-reopenings}}

\begin{itemize}
\tightlist
\item
  \href{https://www.nytimes3xbfgragh.onion/2020/09/04/us/bar-exam-coronavirus.html?name=styln-coronavirus-schools-reopening\&region=TOP_BANNER\&block=storyline_menu_recirc\&action=click\&pgtype=Article\&impression_id=3bd893d0-f27b-11ea-9a21-295489aede86\&variant=undefined}{Delayed
  Licensing Exams}
\item
  \href{https://www.nytimes3xbfgragh.onion/2020/09/08/upshot/children-testing-shortfalls-virus.html?name=styln-coronavirus-schools-reopening\&region=TOP_BANNER\&block=storyline_menu_recirc\&action=click\&pgtype=Article\&impression_id=3bd893d1-f27b-11ea-9a21-295489aede86\&variant=undefined}{Limited
  Testing for Children}
\item
  \href{https://www.nytimes3xbfgragh.onion/2020/09/01/world/schools-reopen-globe-students.html?name=styln-coronavirus-schools-reopening\&region=TOP_BANNER\&block=storyline_menu_recirc\&action=click\&pgtype=Article\&impression_id=3bd893d2-f27b-11ea-9a21-295489aede86\&variant=undefined}{School
  Around the World}
\item
  \href{https://www.nytimes3xbfgragh.onion/interactive/2020/us/covid-college-cases-tracker.html?name=styln-coronavirus-schools-reopening\&region=TOP_BANNER\&block=storyline_menu_recirc\&action=click\&pgtype=Article\&impression_id=3bd893d3-f27b-11ea-9a21-295489aede86\&variant=undefined}{Tracking
  College Cases}
\end{itemize}

Advertisement

\protect\hyperlink{after-top}{Continue reading the main story}

Supported by

\protect\hyperlink{after-sponsor}{Continue reading the main story}

\hypertarget{nobody-likes-snitching-how-rules-against-parties-are-dividing-campuses}{%
\section{`Nobody Likes Snitching': How Rules Against Parties Are
Dividing
Campuses}\label{nobody-likes-snitching-how-rules-against-parties-are-dividing-campuses}}

As colleges reopen despite the pandemic, students must decide whether
they are willing to blow the whistle on their classmates.

\includegraphics{https://static01.graylady3jvrrxbe.onion/images/2020/08/27/nyregion/nyvirus-snitches1/nyvirus-snitches1-articleLarge.png?quality=75\&auto=webp\&disable=upscale}

By \href{https://www.nytimes3xbfgragh.onion/by/troy-closson}{Troy
Closson}

\begin{itemize}
\item
  Published Sept. 2, 2020Updated Sept. 4, 2020
\item
  \begin{itemize}
  \item
  \item
  \item
  \item
  \item
  \item
  \end{itemize}
\end{itemize}

It looked to be a typical college party: a small group of students
crammed in a kitchenette, cheering on as a shirtless guy arm-wrestled a
laughing young woman. No one wore masks.

The scene was posted on Snapchat by one of the partygoers, a first-year
student at Cornell University, along with a selfie with a mocking
caption: ``The people who slide up saying `you're not social distancing'
are the ones that wouldn't have been invited anyway.''

The response was swift and severe. Within days, an online petition was
created demanding that the student's admission to Cornell be revoked,
and in the week since, the petition has collected more than 3,500
signatures.

``Cornell University is attempting to take the biggest feat of allowing
all students back on campus. This cannot be done without immense safety
precautions taken and the compliance of every student,'' a group calling
itself the Concerned Student Coalition
\href{https://www.change.org/p/cornell-university-de-densify-cornell-s-ithaca-campus-by-rescinding-jessica-zhang-24-acceptance}{wrote
in the petition}. ``We need to hold these students accountable for their
actions.''

The situation at Cornell underscores a deeper tension on campuses all
over the country as
\href{https://www.chronicle.com/article/heres-a-list-of-colleges-plans-for-reopening-in-the-fall/?cid2=gen_login_refresh\&cid=gen_sign_in}{about
1,100 colleges embark} on the huge experiment of reopening in a
pandemic. Students, returning to school after months of isolation, are
not only being asked to fully reimagine what their college social lives
look like, but also to assume active roles as the front line against an
outbreak at their schools by policing campus safety.

``Nobody likes snitching --- it's not comfortable,'' said Melissa
Montejo, a sophomore at Cornell who signed the petition. ``I really am
not one to go around and tell people what to do, but for me, this was
troubling. Three months of being careful and not engaging in problematic
behavior is worth saving a life.''

Jessica Zhang, the student who posted the party scene to Snapchat, said
she had apologized and met with Cornell officials. Neither Ms. Zhang nor
Cornell would say whether she was disciplined.

In recent weeks, the coronavirus outbreak has spread swiftly on college
campuses. The New York Times has
\href{https://www.nytimes3xbfgragh.onion/interactive/2020/us/covid-college-cases-tracker.html}{tracked
thousands of cases} that were linked to returning students. Several
schools, including the
\href{https://www.nytimes3xbfgragh.onion/2020/08/17/us/unc-chapel-hill-covid.html}{University
of North Carolina at Chapel Hill},
\href{https://www.nytimes3xbfgragh.onion/2020/08/18/us/notre-dame-coronavirus.html}{Notre
Dame} and the
\href{https://www.nbcnewyork.com/news/local/state-shuts-down-suny-oneonta-for-2-weeks-after-105-test-positive-for-virus/2593007/}{State
University of New York College at Oneonta}, suspended in-person classes
after more than 100 students at each campus tested positive, often
following large parties.

As a result, growing numbers of college officials are realizing that
there are limits to what they can monitor on their own --- and are
calling on students to help.

Colgate University sent students a memo encouraging them to report
classmates who violate social-distancing guidelines and to include names
so action could be taken. Similar instructions were sent out at schools
across the country from the University of Colorado Boulder to the
University of Pennsylvania. Yale University and some other colleges have
\href{https://your.yale.edu/work-yale/financial-management/university-auditing-quality-assurance/yale-university-hotline/hotline}{hotlines
in place} for reports of risky activity.

It's an extraordinary situation, and students face a quandary: Report
parties to campus officials? Or keep quiet and hope for the best? As one
freshman said at Hunter College, which has a dorm open even though
classes are remote this semester: ``I don't know if I'd want to narc on
people I'm trying to become friends with.''

For those in the middle of it, the choice is not as simple as they might
have expected.

``Before coming here, I remember thinking `Yeah, I'll definitely report
people if they're going to parties,''' said Kyle Duran, a freshman at
Binghamton University in upstate New York. But after spending just a few
days on campus, Mr. Duran had second thoughts. ``It's a lot harder to
want to when you're living and going to class with everyone.''

Some faculty members at schools have
\href{https://www.nytimes3xbfgragh.onion/2020/08/12/opinion/coronavirus-college-reopening.html}{warned
against asking students to police their peers}. They have said doing so
could disrupt student life when classmates are pitted against one other,
particularly when the consequences for breaking the rules can be harsh.

SUNY Plattsburgh, for example, placed
\href{https://www.plattsburgh.edu/news/news-archive/president-suspends-students-for-reported-covid-violations.html}{43
students on interim suspension last week} after a large outdoor
gathering.
\href{https://www.maristcircle.com/home/2020/8/22/fifteen-students-suspended-following-off-campus-party-wednesday}{Fifteen
others at Marist College}, a small liberal arts school in Poughkeepsie,
N.Y., were recently sent home for not following rules at an off-campus
party, while at Ohio State University,
\href{https://www.thelantern.com/2020/08/more-than-200-students-on-interim-suspension-after-weekend-parties/}{more
than 200 have been suspended} for similar reasons.

Ariana Rebello, a freshman at Hofstra University on Long Island, said
hearing about those punishments at other schools has dissuaded her from
attending parties, but also from reporting her classmates. ``I don't
think I could bring myself to snitch. I just wouldn't associate with
them,'' she said.

\includegraphics{https://static01.graylady3jvrrxbe.onion/images/2020/08/27/nyregion/00nyvirus-snitches1/00nyvirus-snitches1-articleLarge.jpg?quality=75\&auto=webp\&disable=upscale}

In states with high virus counts, many administrators said they worried
that college parties could accelerate an all-but-inevitable
\href{https://www.nytimes3xbfgragh.onion/interactive/2020/us/covid-college-cases-tracker.html?action=click\&module=Top\%20Stories\&pgtype=Homepage}{rise
of clusters on their campuses}. But in the New York metropolitan area,
which has largely continued to stem its own outbreak, the concerns carry
a different weight.

Some epidemiologists said they feared that college parties and large
social gatherings could lead to a resurgence of the virus in places like
the New York region that have kept case counts low.

``The biggest concern is that you are going to have newly infected
people leave these parties and disperse back into their communities,''
said Dr. Stephen Thomas, an infectious disease specialist at SUNY
Upstate Medical University. ``It's that they're going to be sources for
continuing to spread the virus and it's going to reverse the work that
has already been done.''

Many students say they have more self-interested reasons to report their
peers. On TikTok and other social media platforms, videos have gone
viral in which students say ``snitching'' on their classmates would be
an easy choice because of how much it costs to attend their colleges.

But for others like Cambria Kelley, a first-year graduate student at New
York University, the issue is more personal. Ms. Kelley, who is from
California, said several members of her family contracted the illness
over the last few months, including her grandmother who died in July.

N.Y.U. has asked students to
\href{https://www.nyu.edu/life/safety-health-wellness/coronavirus-information/safety-and-health/coronavirus-testing/student-coronavirus-testing.html}{``politely
urge'' their classmates} to wear masks and socially distance and to
report those who violate that advice to school officials. And despite
the friendships she may form with her classmates, Ms. Kelley said she
will still feel an obligation to do so, keeping her family in mind.

``If it was bad enough, I wouldn't hesitate to report them,'' Ms. Kelley
said. ``I'm not going to be having my life put at risk because people
decided to be selfish. These rules are for the good of everyone here.''

Some students, however, said deciding whether to report classmates
involves a different calculation.

\href{https://www.nytimes3xbfgragh.onion/spotlight/schools-reopening?action=click\&pgtype=Article\&state=default\&region=MAIN_CONTENT_3\&context=storylines_keepup}{}

\hypertarget{school-reopenings-}{%
\subsubsection{School Reopenings ›}\label{school-reopenings-}}

\hypertarget{back-to-school}{%
\paragraph{Back to School}\label{back-to-school}}

Updated Sept. 8, 2020

The latest on how schools are reopening amid the pandemic.

\begin{itemize}
\item
  \begin{itemize}
  \tightlist
  \item
    The first day of school is an annual rite of passage. But this year,
    it looks very different for tens of millions of students.
    \href{https://www.nytimes3xbfgragh.onion/2020/09/05/us/virtual-return-to-school-covid.html?action=click\&pgtype=Article\&state=default\&region=MAIN_CONTENT_3\&context=storylines_keepup}{We
    talked to some about their hopes and fears}.
  \item
    Coronavirus cases
    \href{https://www.nytimes3xbfgragh.onion/2020/09/06/us/colleges-coronavirus-students.html?action=click\&pgtype=Article\&state=default\&region=MAIN_CONTENT_3\&context=storylines_keepup}{are
    spiking in America's college towns}, leading to concern that young
    people who are infected will contribute to a spread of the virus.
  \item
    A growing number of Catholic schools across the country are
    \href{https://www.nytimes3xbfgragh.onion/2020/09/05/us/catholic-school-closings.html?action=click\&pgtype=Article\&state=default\&region=MAIN_CONTENT_3\&context=storylines_keepup}{shutting
    down forever during the coronavirus pandemic}, citing insurmountable
    financial pressure.
  \item
    The magazine's Ethicist columnist answers a question from a
    spokesperson at a major university:
    \href{https://www.nytimes3xbfgragh.onion/2020/09/08/magazine/university-reopening-safety-ethics.html?action=click\&pgtype=Article\&state=default\&region=MAIN_CONTENT_3\&context=storylines_keepup}{Can
    I promote a reopening plan I have doubts about}?
  \end{itemize}
\end{itemize}

As a national conversation erupted on the role of police in cities
following the killing of George Floyd, groups at schools including
Vassar College, Stony Brook University and Columbia University called on
their institutions to
\href{https://www.insidehighered.com/news/2020/06/05/students-demand-universities-break-ties-local-police-few-have}{rethink
their relationships with campus and local police}.

Now they are wrestling with the prospect of relying on those departments
to disperse and crack down on large gatherings.

Image

In her senior year at Syracuse University, Maggie Peng fears students'
parties could shut down the campus. But she does not believe reporting
them is her best option.Credit...Heather Ainsworth for The New York
Times

Maggie Peng, a senior at Syracuse University, said she plans to approach
friends one-on-one to have conversations about the risks of partying.
She even wrote a lengthy message in a Facebook group for first-year
students urging them to take social distancing more seriously after a
\href{http://dailyorange.com/2020/08/videos-surface-showing-100-syracuse-university-students-gathering-on-campus-not-social-distancing/}{large
outdoor gathering that drew hundreds}.

But Ms. Peng said shifting the responsibility for keeping one another
safe from individual students to campus police was troubling. She and
other students at the school said that since large social events
occurred two weeks ago, they have noticed more officers than usual
monitoring residential spaces and common areas.

Ms. Peng said the role of campus security in enforcing rules would
dissuade her from reporting her classmates.

``It just doesn't make sense to rely on campus police for enforcing
these rules,'' Ms. Peng said. ``Especially when there has been so much
tension, it's hard to want them involved. Anything bad that happens will
usually involve students of color.''

As the first weeks back on campuses shift into the more regular pace of
the fall, the question remains: Just how long will students, many of
whom arrived in New York State to a
\href{https://www.nytimes3xbfgragh.onion/2020/08/18/nyregion/college-reopening-quarantine-coronavirus.html}{mandatory
two-week quarantine}, continue to follow their schools'
physical-distancing guidelines?

Some officials at large universities in particular worry parties,
especially those off campus, could slip by undetected.

For now, the students who are caught attending them face harsh
punishments as a warning to others.

Ms. Zhang, the first-year student at Cornell who posted videos of a
party to Snapchat, said in a recent interview that she deeply regrets
both the ``lapse in judgment'' she made by attending the gathering and
her ``insensitive comments'' afterward.

``Incoming freshmen come in with heavy expectations, we all want to find
our people,'' Ms. Zhang said. ``I'm not proud of those posts, they show
me at my worst.''

Since her posts spread on social media, Ms. Zhang said she has received
hundreds of threats, with much of the heaviest criticism coming from
classmates.

As classes at Cornell begin this week, there may soon be an additional
layer to complicated tensions students face when reporting one another:
Ms. Zhang could end up sitting in the same row of her first college
lectures as those who called for her expulsion.

Advertisement

\protect\hyperlink{after-bottom}{Continue reading the main story}

\hypertarget{site-index}{%
\subsection{Site Index}\label{site-index}}

\hypertarget{site-information-navigation}{%
\subsection{Site Information
Navigation}\label{site-information-navigation}}

\begin{itemize}
\tightlist
\item
  \href{https://help.nytimes3xbfgragh.onion/hc/en-us/articles/115014792127-Copyright-notice}{©~2020~The
  New York Times Company}
\end{itemize}

\begin{itemize}
\tightlist
\item
  \href{https://www.nytco.com/}{NYTCo}
\item
  \href{https://help.nytimes3xbfgragh.onion/hc/en-us/articles/115015385887-Contact-Us}{Contact
  Us}
\item
  \href{https://www.nytco.com/careers/}{Work with us}
\item
  \href{https://nytmediakit.com/}{Advertise}
\item
  \href{http://www.tbrandstudio.com/}{T Brand Studio}
\item
  \href{https://www.nytimes3xbfgragh.onion/privacy/cookie-policy\#how-do-i-manage-trackers}{Your
  Ad Choices}
\item
  \href{https://www.nytimes3xbfgragh.onion/privacy}{Privacy}
\item
  \href{https://help.nytimes3xbfgragh.onion/hc/en-us/articles/115014893428-Terms-of-service}{Terms
  of Service}
\item
  \href{https://help.nytimes3xbfgragh.onion/hc/en-us/articles/115014893968-Terms-of-sale}{Terms
  of Sale}
\item
  \href{https://spiderbites.nytimes3xbfgragh.onion}{Site Map}
\item
  \href{https://help.nytimes3xbfgragh.onion/hc/en-us}{Help}
\item
  \href{https://www.nytimes3xbfgragh.onion/subscription?campaignId=37WXW}{Subscriptions}
\end{itemize}
