Sections

SEARCH

\protect\hyperlink{site-content}{Skip to
content}\protect\hyperlink{site-index}{Skip to site index}

\href{https://www.nytimes3xbfgragh.onion/section/nyregion}{New York}

\href{https://myaccount.nytimes3xbfgragh.onion/auth/login?response_type=cookie\&client_id=vi}{}

\href{https://www.nytimes3xbfgragh.onion/section/todayspaper}{Today's
Paper}

\href{/section/nyregion}{New York}\textbar{}After 5 Months, N.Y.C. Gyms
Reopen. Here's What to Expect.

\url{https://nyti.ms/32Qeg4O}

\begin{itemize}
\item
\item
\item
\item
\item
\end{itemize}

\hypertarget{the-coronavirus-outbreak}{%
\subsubsection{\texorpdfstring{\href{https://www.nytimes3xbfgragh.onion/news-event/coronavirus?name=styln-coronavirus-national\&region=TOP_BANNER\&block=storyline_menu_recirc\&action=click\&pgtype=Article\&impression_id=a796d5e0-f2a4-11ea-9745-87ca264dc7a7\&variant=undefined}{The
Coronavirus
Outbreak}}{The Coronavirus Outbreak}}\label{the-coronavirus-outbreak}}

\begin{itemize}
\tightlist
\item
  live\href{https://www.nytimes3xbfgragh.onion/2020/09/09/world/covid-19-coronavirus.html?name=styln-coronavirus-national\&region=TOP_BANNER\&block=storyline_menu_recirc\&action=click\&pgtype=Article\&impression_id=a796d5e1-f2a4-11ea-9745-87ca264dc7a7\&variant=undefined}{Latest
  Updates}
\item
  \href{https://www.nytimes3xbfgragh.onion/interactive/2020/us/coronavirus-us-cases.html?name=styln-coronavirus-national\&region=TOP_BANNER\&block=storyline_menu_recirc\&action=click\&pgtype=Article\&impression_id=a796fcf0-f2a4-11ea-9745-87ca264dc7a7\&variant=undefined}{Maps
  and Cases}
\item
  \href{https://www.nytimes3xbfgragh.onion/interactive/2020/science/coronavirus-vaccine-tracker.html?name=styln-coronavirus-national\&region=TOP_BANNER\&block=storyline_menu_recirc\&action=click\&pgtype=Article\&impression_id=a796fcf1-f2a4-11ea-9745-87ca264dc7a7\&variant=undefined}{Vaccine
  Tracker}
\item
  \href{https://www.nytimes3xbfgragh.onion/2020/09/02/your-money/eviction-moratorium-covid.html?name=styln-coronavirus-national\&region=TOP_BANNER\&block=storyline_menu_recirc\&action=click\&pgtype=Article\&impression_id=a796fcf2-f2a4-11ea-9745-87ca264dc7a7\&variant=undefined}{Eviction
  Moratorium}
\item
  \href{https://www.nytimes3xbfgragh.onion/2020/09/09/upshot/coronavirus-surprise-test-fees.html?name=styln-coronavirus-national\&region=TOP_BANNER\&block=storyline_menu_recirc\&action=click\&pgtype=Article\&impression_id=a796fcf3-f2a4-11ea-9745-87ca264dc7a7\&variant=undefined}{Surprise
  Test Fees}
\end{itemize}

Advertisement

\protect\hyperlink{after-top}{Continue reading the main story}

Supported by

\protect\hyperlink{after-sponsor}{Continue reading the main story}

\hypertarget{after-5-months-nyc-gyms-reopen-heres-what-to-expect}{%
\section{After 5 Months, N.Y.C. Gyms Reopen. Here's What to
Expect.}\label{after-5-months-nyc-gyms-reopen-heres-what-to-expect}}

Expect less crowds, but no group activities like spin classes or yoga.

\includegraphics{https://static01.graylady3jvrrxbe.onion/images/2020/09/02/nyregion/02nyvirus-gyms1/02nyvirus-gyms1-articleLarge.jpg?quality=75\&auto=webp\&disable=upscale}

\href{https://www.nytimes3xbfgragh.onion/by/daniel-e-slotnik}{\includegraphics{https://static01.graylady3jvrrxbe.onion/images/2018/07/12/multimedia/author-daniel-e-slotnik/author-daniel-e-slotnik-thumbLarge.png}}

By \href{https://www.nytimes3xbfgragh.onion/by/daniel-e-slotnik}{Daniel
E. Slotnik}

\begin{itemize}
\item
  Sept. 2, 2020
\item
  \begin{itemize}
  \item
  \item
  \item
  \item
  \item
  \end{itemize}
\end{itemize}

Most New Yorkers have lived without communal workouts since mid-March,
when Gov. Andrew M. Cuomo closed gyms in the state to prevent the spread
of the coronavirus.

Gyms seem an intuitively high-risk environment, and New York waited
months until it allowed the 2,000 or so gyms in the state to reopen.
They were finally
\href{https://www.nytimes3xbfgragh.onion/2020/08/17/nyregion/nyc-gyms-reopening.html}{given
the go-ahead on Aug. 24}, along with museums, aquariums and bowling
alleys.

The city waited until Wednesday to allow gyms to reopen, so that
officials had more time to conduct inspections, which are taking place
as the city also tries to
\href{https://www.nytimes3xbfgragh.onion/2020/09/01/nyregion/schools-open-coronavirus-nyc.html}{reopen
schools}.

Gyms may be a sanctuary of sorts and a healthy outlet for stress when
everyone in the city could use a release, but they remained closed as
retail stores and restaurants reopened --- at least for outdoor dining
--- and the rate of positive virus tests in the city continued to hover
around 1 percent.

The reopening of gyms signals yet another step toward a return to
normalcy, even though the pandemic is still a threat and many New
Yorkers may not feel safe using them.

State officials said that they had tracked coronavirus infections
connected to gyms in Hawaii and South Korea as case studies to inform
their policies.

\hypertarget{latest-updates-the-coronavirus-outbreak}{%
\section{\texorpdfstring{\href{https://www.nytimes3xbfgragh.onion/2020/09/09/world/covid-19-coronavirus.html?action=click\&pgtype=Article\&state=default\&region=MAIN_CONTENT_1\&context=storylines_live_updates}{Latest
Updates: The Coronavirus
Outbreak}}{Latest Updates: The Coronavirus Outbreak}}\label{latest-updates-the-coronavirus-outbreak}}

Updated 2020-09-09T13:51:03.112Z

\begin{itemize}
\tightlist
\item
  \href{https://www.nytimes3xbfgragh.onion/2020/09/09/world/covid-19-coronavirus.html?action=click\&pgtype=Article\&state=default\&region=MAIN_CONTENT_1\&context=storylines_live_updates\#link-70cea8bb}{As
  drugmakers pledge to thoroughly vet a vaccine, one company pauses its
  trials for a safety review.}
\item
  \href{https://www.nytimes3xbfgragh.onion/2020/09/09/world/covid-19-coronavirus.html?action=click\&pgtype=Article\&state=default\&region=MAIN_CONTENT_1\&context=storylines_live_updates\#link-780eaa2f}{Britain
  is expected to ban gatherings of more than six people.}
\item
  \href{https://www.nytimes3xbfgragh.onion/2020/09/09/world/covid-19-coronavirus.html?action=click\&pgtype=Article\&state=default\&region=MAIN_CONTENT_1\&context=storylines_live_updates\#link-11cec4c0}{Quarantine
  breakdowns at colleges in the U.S. are leaving some at risk.}
\end{itemize}

\href{https://www.nytimes3xbfgragh.onion/2020/09/09/world/covid-19-coronavirus.html?action=click\&pgtype=Article\&state=default\&region=MAIN_CONTENT_1\&context=storylines_live_updates}{See
more updates}

More live coverage:
\href{https://www.nytimes3xbfgragh.onion/live/2020/09/09/business/stock-market-today-coronavirus?action=click\&pgtype=Article\&state=default\&region=MAIN_CONTENT_1\&context=storylines_live_updates}{Markets}

But while gyms are coming back, in the city and in most states, the
scene at your local fitness center will be considerably different than
it was before the start of the pandemic. Here is what to expect.

\hypertarget{free-weights-are-available-but-studios-are-not}{%
\subsubsection{Free weights are available, but studios are
not}\label{free-weights-are-available-but-studios-are-not}}

Gyms offering weights and exercise machines are allowed to reopen, but
many other exercise facilities in the city have not yet been given the
go-ahead. Indoor pools and places that only offer group fitness classes,
like spinning, Zumba, yoga and Pilates, are not allowed to reopen
because the city sees those activities as higher risk.

One-on-one sessions with personal trainers or yoga teachers are allowed.

``We are certainly going to take a pretty strict stance in the name of
preserving our low level of infection,'' Mayor Bill de Blasio said on
NY1's ``Inside City Hall'' last week, adding, ``We want to see jobs come
back, we want to see amenities for people. We also have to make sure
it's done safely.''

Many fitness studio owners, who have started offering classes outside
and teaching video sessions during the pandemic, believe the strict
regulations that are keeping their businesses closed are unfair.

Amanda Freeman, who owns \href{https://sltnyc.com/}{a dozen SLT studios}
in New York City and 14 outside of it, said that she thinks her
low-impact workout routine is actually less dangerous than workouts at a
conventional gym.

``We're even more controlled, it's by appointment only, there are very
limited spots,'' Ms. Freeman said. ``I don't understand why we're worse
than they are.''

\hypertarget{expect-fewer-lines-for-machines-and-far-more-cleaning}{%
\subsubsection{Expect fewer lines for machines, and far more
cleaning}\label{expect-fewer-lines-for-machines-and-far-more-cleaning}}

\includegraphics{https://static01.graylady3jvrrxbe.onion/images/2020/09/02/nyregion/02nyvirus-gyms2/02nyvirus-gyms2-articleLarge.jpg?quality=75\&auto=webp\&disable=upscale}

Gyms may be open, but don't expect capacity crowds.
\href{https://www.governor.ny.gov/sites/governor.ny.gov/files/atoms/files/Gyms_and_Fitness_Centers_Detailed_Guidelines.pdf}{State
regulations} greatly limit the number of attendees, the distance between
machines and many other factors.

Gyms may only fill up to 33 percent capacity. Gym-goers must wear face
coverings, pass a health screening and maintain at least six feet of
separation at all times. Equipment must be cleaned after each use, and
signage detailing safe practices must be prominently displayed. Gyms
need to install expensive filtration systems, if they do not already
have them.

Different gyms are approaching the regulations in different ways.
\href{https://www.newyorksportsclubs.com/page/nysc-club-health-updates}{New
York Sports Club} and
\href{https://www.equinox.com/page/theequinoxstandard}{Equinox}
locations will operate under adjusted hours, and Equinox and
\href{https://info.crunch.com/member-updates}{Crunch gyms will use}
electrostatic disinfectant sprayers as an additional cleaning method.

These measures may not make for the most sociable workout, but they
should help protect you from infection. State officials said that there
was not a specific metric for whether gyms would need to close again,
but that they would monitor the situation.

\href{https://www.nytimes3xbfgragh.onion/news-event/coronavirus?action=click\&pgtype=Article\&state=default\&region=MAIN_CONTENT_3\&context=storylines_faq}{}

\hypertarget{the-coronavirus-outbreak-}{%
\subsubsection{The Coronavirus Outbreak
›}\label{the-coronavirus-outbreak-}}

\hypertarget{frequently-asked-questions}{%
\paragraph{Frequently Asked
Questions}\label{frequently-asked-questions}}

Updated September 4, 2020

\begin{itemize}
\item ~
  \hypertarget{what-are-the-symptoms-of-coronavirus}{%
  \paragraph{What are the symptoms of
  coronavirus?}\label{what-are-the-symptoms-of-coronavirus}}

  \begin{itemize}
  \tightlist
  \item
    In the beginning, the coronavirus
    \href{https://www.nytimes3xbfgragh.onion/article/coronavirus-facts-history.html?action=click\&pgtype=Article\&state=default\&region=MAIN_CONTENT_3\&context=storylines_faq\#link-6817bab5}{seemed
    like it was primarily a respiratory illness}~--- many patients had
    fever and chills, were weak and tired, and coughed a lot, though
    some people don't show many symptoms at all. Those who seemed
    sickest had pneumonia or acute respiratory distress syndrome and
    received supplemental oxygen. By now, doctors have identified many
    more symptoms and syndromes. In April,
    \href{https://www.nytimes3xbfgragh.onion/2020/04/27/health/coronavirus-symptoms-cdc.html?action=click\&pgtype=Article\&state=default\&region=MAIN_CONTENT_3\&context=storylines_faq}{the
    C.D.C. added to the list of early signs}~sore throat, fever, chills
    and muscle aches. Gastrointestinal upset, such as diarrhea and
    nausea, has also been observed. Another telltale sign of infection
    may be a sudden, profound diminution of one's
    \href{https://www.nytimes3xbfgragh.onion/2020/03/22/health/coronavirus-symptoms-smell-taste.html?action=click\&pgtype=Article\&state=default\&region=MAIN_CONTENT_3\&context=storylines_faq}{sense
    of smell and taste.}~Teenagers and young adults in some cases have
    developed painful red and purple lesions on their fingers and toes
    --- nicknamed ``Covid toe'' --- but few other serious symptoms.
  \end{itemize}
\item ~
  \hypertarget{why-is-it-safer-to-spend-time-together-outside}{%
  \paragraph{Why is it safer to spend time together
  outside?}\label{why-is-it-safer-to-spend-time-together-outside}}

  \begin{itemize}
  \tightlist
  \item
    \href{https://www.nytimes3xbfgragh.onion/2020/05/15/us/coronavirus-what-to-do-outside.html?action=click\&pgtype=Article\&state=default\&region=MAIN_CONTENT_3\&context=storylines_faq}{Outdoor
    gatherings}~lower risk because wind disperses viral droplets, and
    sunlight can kill some of the virus. Open spaces prevent the virus
    from building up in concentrated amounts and being inhaled, which
    can happen when infected people exhale in a confined space for long
    stretches of time, said Dr. Julian W. Tang, a virologist at the
    University of Leicester.
  \end{itemize}
\item ~
  \hypertarget{why-does-standing-six-feet-away-from-others-help}{%
  \paragraph{Why does standing six feet away from others
  help?}\label{why-does-standing-six-feet-away-from-others-help}}

  \begin{itemize}
  \tightlist
  \item
    The coronavirus spreads primarily through droplets from your mouth
    and nose, especially when you cough or sneeze. The C.D.C., one of
    the organizations using that measure,
    \href{https://www.nytimes3xbfgragh.onion/2020/04/14/health/coronavirus-six-feet.html?action=click\&pgtype=Article\&state=default\&region=MAIN_CONTENT_3\&context=storylines_faq}{bases
    its recommendation of six feet}~on the idea that most large droplets
    that people expel when they cough or sneeze will fall to the ground
    within six feet. But six feet has never been a magic number that
    guarantees complete protection. Sneezes, for instance, can launch
    droplets a lot farther than six feet,
    \href{https://jamanetwork.com/journals/jama/fullarticle/2763852}{according
    to a recent study}. It's a rule of thumb: You should be safest
    standing six feet apart outside, especially when it's windy. But
    keep a mask on at all times, even when you think you're far enough
    apart.
  \end{itemize}
\item ~
  \hypertarget{i-have-antibodies-am-i-now-immune}{%
  \paragraph{I have antibodies. Am I now
  immune?}\label{i-have-antibodies-am-i-now-immune}}

  \begin{itemize}
  \tightlist
  \item
    As of right
    now,\href{https://www.nytimes3xbfgragh.onion/2020/07/22/health/covid-antibodies-herd-immunity.html?action=click\&pgtype=Article\&state=default\&region=MAIN_CONTENT_3\&context=storylines_faq}{~that
    seems likely, for at least several months.}~There have been
    frightening accounts of people suffering what seems to be a second
    bout of Covid-19. But experts say these patients may have a
    drawn-out course of infection, with the virus taking a slow toll
    weeks to months after initial exposure.~People infected with the
    coronavirus typically
    \href{https://www.nature.com/articles/s41586-020-2456-9}{produce}~immune
    molecules called antibodies, which are
    \href{https://www.nytimes3xbfgragh.onion/2020/05/07/health/coronavirus-antibody-prevalence.html?action=click\&pgtype=Article\&state=default\&region=MAIN_CONTENT_3\&context=storylines_faq}{protective
    proteins made in response to an
    infection}\href{https://www.nytimes3xbfgragh.onion/2020/05/07/health/coronavirus-antibody-prevalence.html?action=click\&pgtype=Article\&state=default\&region=MAIN_CONTENT_3\&context=storylines_faq}{.
    These antibodies may}~last in the body
    \href{https://www.nature.com/articles/s41591-020-0965-6}{only two to
    three months}, which may seem worrisome, but that's~perfectly normal
    after an acute infection subsides, said Dr. Michael Mina, an
    immunologist at Harvard University. It may be possible to get the
    coronavirus again, but it's highly unlikely that it would be
    possible in a short window of time from initial infection or make
    people sicker the second time.
  \end{itemize}
\item ~
  \hypertarget{what-are-my-rights-if-i-am-worried-about-going-back-to-work}{%
  \paragraph{What are my rights if I am worried about going back to
  work?}\label{what-are-my-rights-if-i-am-worried-about-going-back-to-work}}

  \begin{itemize}
  \tightlist
  \item
    Employers have to provide
    \href{https://www.osha.gov/SLTC/covid-19/standards.html}{a safe
    workplace}~with policies that protect everyone equally.
    \href{https://www.nytimes3xbfgragh.onion/article/coronavirus-money-unemployment.html?action=click\&pgtype=Article\&state=default\&region=MAIN_CONTENT_3\&context=storylines_faq}{And
    if one of your co-workers tests positive for the coronavirus, the
    C.D.C.}~has said that
    \href{https://www.cdc.gov/coronavirus/2019-ncov/community/guidance-business-response.html}{employers
    should tell their employees}~-\/- without giving you the sick
    employee's name -\/- that they may have been exposed to the virus.
  \end{itemize}
\end{itemize}

Many experts agree that even with the regulations, fitness centers could
still be risky.

``Gyms can be high risk depending on the setup --- from crowding, to
close contact, a lack of masks, adequate ventilation, high-touch
surfaces and a need for disinfection,'' said Saskia Popescu, an
epidemiologist at George Mason University. ``The challenge is that
people really need to be spaced out and wearing masks the entire time,
and this includes the locker rooms.''

\hypertarget{health-departments-will-inspect-clubs}{%
\subsubsection{Health departments will inspect
clubs}\label{health-departments-will-inspect-clubs}}

State regulations are to be enforced by local health departments, who
are responsible for inspecting gyms within two weeks of their reopening.
People from other agencies, like fire marshals or building inspectors,
can also carry out inspections. Different health departments around the
state work out the specifics of their inspections.

\href{https://www1.nyc.gov/assets/doh/downloads/pdf/covid/businesses/covid-19-reopening-gyms.pdf}{New
York City's inspections} will be carried out virtually, with inspectors
communicating with a gym owner on a video call to make certain that
their precautions are up to code.

Gyms that do not meet the state requirements --- say by not having a
strong enough filtration system, insufficient personal protective
equipment, signage or cleaning supplies --- will have to close and will
be fined up to \$10,000 if they remain open without complying with state
guidelines.

\hypertarget{the-future-for-fitness-in-the-city-is-unclear}{%
\subsubsection{The future for fitness in the city is
unclear}\label{the-future-for-fitness-in-the-city-is-unclear}}

Reopening is sure to reinvigorate frustrated fitness fans, but the
future for the workout industry in the city remains uncertain.

Reopening a gym can cost more than \$20,000, especially with expensive
upgrades to a location's air filtration system. The added costs come
after months without revenue, when many gym owners find themselves deep
in debt, and will have to reopen to only a third of their normal
customer base.

There is also the risk that people may be fearful of returning to
fitness centers, or may have grown to prefer working out outside, or
forgoing exercise altogether.

``Exercise is habitual, the longer you change a habit the harder it is
to get back,'' said Ms. Freeman, the studio owner.

Several gym owners said that they had heard a wide range of responses
from their clients, but that many were enthusiastic to get back to their
gyms.

One gym-goer, Mika Sneddon, said she had worked out a lot less with her
personal trainer,
\href{https://www.lcbfitnessnyc.com/meetlunelle}{Lunelle Belhomme},
since gyms closed. She said she was eager to return to an atmosphere
that inspired her to work out, and that she thought gyms as a whole
would find a way to bounce back.

``Our ability to adapt is much greater than some people think,'' Ms.
Sneddon said. ``I think it's just a matter of accepting that maybe
things might look different.''

Advertisement

\protect\hyperlink{after-bottom}{Continue reading the main story}

\hypertarget{site-index}{%
\subsection{Site Index}\label{site-index}}

\hypertarget{site-information-navigation}{%
\subsection{Site Information
Navigation}\label{site-information-navigation}}

\begin{itemize}
\tightlist
\item
  \href{https://help.nytimes3xbfgragh.onion/hc/en-us/articles/115014792127-Copyright-notice}{©~2020~The
  New York Times Company}
\end{itemize}

\begin{itemize}
\tightlist
\item
  \href{https://www.nytco.com/}{NYTCo}
\item
  \href{https://help.nytimes3xbfgragh.onion/hc/en-us/articles/115015385887-Contact-Us}{Contact
  Us}
\item
  \href{https://www.nytco.com/careers/}{Work with us}
\item
  \href{https://nytmediakit.com/}{Advertise}
\item
  \href{http://www.tbrandstudio.com/}{T Brand Studio}
\item
  \href{https://www.nytimes3xbfgragh.onion/privacy/cookie-policy\#how-do-i-manage-trackers}{Your
  Ad Choices}
\item
  \href{https://www.nytimes3xbfgragh.onion/privacy}{Privacy}
\item
  \href{https://help.nytimes3xbfgragh.onion/hc/en-us/articles/115014893428-Terms-of-service}{Terms
  of Service}
\item
  \href{https://help.nytimes3xbfgragh.onion/hc/en-us/articles/115014893968-Terms-of-sale}{Terms
  of Sale}
\item
  \href{https://spiderbites.nytimes3xbfgragh.onion}{Site Map}
\item
  \href{https://help.nytimes3xbfgragh.onion/hc/en-us}{Help}
\item
  \href{https://www.nytimes3xbfgragh.onion/subscription?campaignId=37WXW}{Subscriptions}
\end{itemize}
