Sections

SEARCH

\protect\hyperlink{site-content}{Skip to
content}\protect\hyperlink{site-index}{Skip to site index}

\href{https://www.nytimes3xbfgragh.onion/section/science}{Science}

\href{https://myaccount.nytimes3xbfgragh.onion/auth/login?response_type=cookie\&client_id=vi}{}

\href{https://www.nytimes3xbfgragh.onion/section/todayspaper}{Today's
Paper}

\href{/section/science}{Science}\textbar{}Melting Glaciers Are Filling
Unstable Lakes. And They're Growing.

\url{https://nyti.ms/3lGKMz3}

\begin{itemize}
\item
\item
\item
\item
\item
\item
\end{itemize}

Advertisement

\protect\hyperlink{after-top}{Continue reading the main story}

Supported by

\protect\hyperlink{after-sponsor}{Continue reading the main story}

Trilobites

\hypertarget{melting-glaciers-are-filling-unstable-lakes-and-theyre-growing}{%
\section{Melting Glaciers Are Filling Unstable Lakes. And They're
Growing.}\label{melting-glaciers-are-filling-unstable-lakes-and-theyre-growing}}

A census of the world's glacial lakes shows there are more than there
used to be, and their water volume is growing.

\includegraphics{https://static01.graylady3jvrrxbe.onion/images/2020/09/08/science/02TB-GLACIALLAKES1/02TB-GLACIALLAKES1-articleLarge.jpg?quality=75\&auto=webp\&disable=upscale}

By Katherine Kornei

\begin{itemize}
\item
  Sept. 2, 2020
\item
  \begin{itemize}
  \item
  \item
  \item
  \item
  \item
  \item
  \end{itemize}
\end{itemize}

Nearly freezing and often an otherworldly shade of blue, glacial lakes
form as glaciers melt and retreat. These lakes are a source of drinking
and irrigation water for many communities. But they can turn deadly in
an instant when the rocks that hold them in place shift and send
torrents of water coursing downstream.

Now, researchers have compiled the first global database of glacial
lakes and found that they increased in volume by nearly 50 percent over
the last few decades. That growth, largely fueled by climate change,
means that such floods will likely strike more frequently in the future,
the team concluded in a paper
\href{https://www.nature.com/articles/s41558-020-0855-4}{published
Monday} in Nature Climate Change.

Dan Shugar, a geomorphologist at the University of Calgary, and his
colleagues did not set out to take a global census of glacial lakes.
They had originally planned to focus on only a few dozen concentrated in
the \href{https://nsidc.org/data/highmountainasia}{Himalayas and
neighboring mountain ranges in East and South Asia}. But when the team
finished writing computer programs to automatically identify and outline
water in satellite images, they realized they could easily expand their
study to include most of the world's glacial lakes.

``It wasn't that much of a bigger leap,'' Dr. Shugar said.

The researchers collected more than 250,000
\href{https://landsat.gsfc.nasa.gov}{Landsat} images of the Earth's
surface and fed that satellite imagery into
\href{https://earthengine.google.com}{Google Earth Engine}, a platform
for analyzing large Earth science data sets, to assemble the most
complete glacial lake inventory to date.

``We mapped almost the whole world,'' Dr. Shugar said.

\includegraphics{https://static01.graylady3jvrrxbe.onion/images/2020/09/02/science/02TB-GLACIALLAKES2/02TB-GLACIALLAKES2-articleLarge.jpg?quality=75\&auto=webp\&disable=upscale}

This study demonstrates cloud computing's capabilities, said David
Rounce, a glaciologist at Carnegie Mellon University who was not
involved in the research. ``Being able to churn through over 200,000
images is really remarkable.''

The global coverage also makes it possible to pick out large-scale
patterns and regional differences that other studies might miss, said
Kristen Cook, a geologist at the GFZ German Research Centre for
Geosciences, who also was not part of the research team.

Dr. Shugar and his collaborators measured how the number and size of
glacial lakes evolved from 1990 through 2018. The team found that the
number of lakes increased to over 14,300 from roughly 9,400, an uptick
of more than 50 percent. The volume of water in the lakes also tended to
swell over time, with an increase of about 50 percent.

Lakes at high latitudes exhibited the fastest growth, the researchers
found. That makes sense, Dr. Shugar and his colleagues proposed, because
climate change is
\href{https://www.nytimes3xbfgragh.onion/2018/12/11/climate/arctic-warming.html?searchResultPosition=1}{warming
the Arctic faster} than other parts of the world.

All this growth is troubling, Dr. Shugar and his research team members
suggest, because glacial lakes, by their very nature, can pose
significant danger to downstream communities.

Some glacial lakes sit in bowl-shaped depressions bordered by glacial
moraine, the often unstable rocky rubble left behind by a retreating
glacier. When moraine collapses, glacial lake water can course downslope
in an outburst flood.

These events, which have occurred from Nepal to Peru to Iceland,
\href{https://eos.org/features/the-dangers-of-glacial-lake-floods-pioneering-and-capitulation}{can
be devastating}. ``They are a very real threat in many parts of the
world,'' Dr. Shugar said.

Some countries have made significant investments to mitigate the risk of
such floods. In 2016, Nepalese officials
\href{https://kathmandupost.com/miscellaneous/2016/12/03/dammed}{lowered
the water level in Imja Lake}, a glacial lake near Mt. Everest, by more
than 11 feet.

This global census can help identify other lakes in need of monitoring
or remediation, Dr. Shugar said. ``We hope that it allows governments to
see where the hot spots might be for glacial lakes growing in the
future.''

Advertisement

\protect\hyperlink{after-bottom}{Continue reading the main story}

\hypertarget{site-index}{%
\subsection{Site Index}\label{site-index}}

\hypertarget{site-information-navigation}{%
\subsection{Site Information
Navigation}\label{site-information-navigation}}

\begin{itemize}
\tightlist
\item
  \href{https://help.nytimes3xbfgragh.onion/hc/en-us/articles/115014792127-Copyright-notice}{©~2020~The
  New York Times Company}
\end{itemize}

\begin{itemize}
\tightlist
\item
  \href{https://www.nytco.com/}{NYTCo}
\item
  \href{https://help.nytimes3xbfgragh.onion/hc/en-us/articles/115015385887-Contact-Us}{Contact
  Us}
\item
  \href{https://www.nytco.com/careers/}{Work with us}
\item
  \href{https://nytmediakit.com/}{Advertise}
\item
  \href{http://www.tbrandstudio.com/}{T Brand Studio}
\item
  \href{https://www.nytimes3xbfgragh.onion/privacy/cookie-policy\#how-do-i-manage-trackers}{Your
  Ad Choices}
\item
  \href{https://www.nytimes3xbfgragh.onion/privacy}{Privacy}
\item
  \href{https://help.nytimes3xbfgragh.onion/hc/en-us/articles/115014893428-Terms-of-service}{Terms
  of Service}
\item
  \href{https://help.nytimes3xbfgragh.onion/hc/en-us/articles/115014893968-Terms-of-sale}{Terms
  of Sale}
\item
  \href{https://spiderbites.nytimes3xbfgragh.onion}{Site Map}
\item
  \href{https://help.nytimes3xbfgragh.onion/hc/en-us}{Help}
\item
  \href{https://www.nytimes3xbfgragh.onion/subscription?campaignId=37WXW}{Subscriptions}
\end{itemize}
