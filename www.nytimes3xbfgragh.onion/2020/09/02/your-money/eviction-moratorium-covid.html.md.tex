Sections

SEARCH

\protect\hyperlink{site-content}{Skip to
content}\protect\hyperlink{site-index}{Skip to site index}

\href{https://www.nytimes3xbfgragh.onion/section/your-money}{Your Money}

\href{https://myaccount.nytimes3xbfgragh.onion/auth/login?response_type=cookie\&client_id=vi}{}

\href{https://www.nytimes3xbfgragh.onion/section/todayspaper}{Today's
Paper}

\href{/section/your-money}{Your Money}\textbar{}The New Eviction
Moratorium: What You Need to Know

\url{https://nyti.ms/31Rrsr5}

\begin{itemize}
\item
\item
\item
\item
\item
\end{itemize}

\hypertarget{the-coronavirus-outbreak}{%
\subsubsection{\texorpdfstring{\href{https://www.nytimes3xbfgragh.onion/news-event/coronavirus?name=styln-coronavirus-national\&region=TOP_BANNER\&block=storyline_menu_recirc\&action=click\&pgtype=Article\&impression_id=7c097730-f52d-11ea-84d7-f589d7b1ff3e\&variant=undefined}{The
Coronavirus
Outbreak}}{The Coronavirus Outbreak}}\label{the-coronavirus-outbreak}}

\begin{itemize}
\tightlist
\item
  live\href{https://www.nytimes3xbfgragh.onion/2020/09/12/world/covid-19-coronavirus.html?name=styln-coronavirus-national\&region=TOP_BANNER\&block=storyline_menu_recirc\&action=click\&pgtype=Article\&impression_id=7c099e40-f52d-11ea-84d7-f589d7b1ff3e\&variant=undefined}{Latest
  Updates}
\item
  \href{https://www.nytimes3xbfgragh.onion/interactive/2020/us/coronavirus-us-cases.html?name=styln-coronavirus-national\&region=TOP_BANNER\&block=storyline_menu_recirc\&action=click\&pgtype=Article\&impression_id=7c099e41-f52d-11ea-84d7-f589d7b1ff3e\&variant=undefined}{Maps
  and Cases}
\item
  \href{https://www.nytimes3xbfgragh.onion/interactive/2020/science/coronavirus-vaccine-tracker.html?name=styln-coronavirus-national\&region=TOP_BANNER\&block=storyline_menu_recirc\&action=click\&pgtype=Article\&impression_id=7c099e42-f52d-11ea-84d7-f589d7b1ff3e\&variant=undefined}{Vaccine
  Tracker}
\item
  \href{https://www.nytimes3xbfgragh.onion/2020/09/10/us/politics/fda-coronavirus-vaccine.html?name=styln-coronavirus-national\&region=TOP_BANNER\&block=storyline_menu_recirc\&action=click\&pgtype=Article\&impression_id=7c099e43-f52d-11ea-84d7-f589d7b1ff3e\&variant=undefined}{F.D.A.
  Regulators' Self-Defense}
\item
  \href{https://www.nytimes3xbfgragh.onion/2020/09/09/upshot/coronavirus-surprise-test-fees.html?name=styln-coronavirus-national\&region=TOP_BANNER\&block=storyline_menu_recirc\&action=click\&pgtype=Article\&impression_id=7c099e44-f52d-11ea-84d7-f589d7b1ff3e\&variant=undefined}{Surprise
  Test Fees}
\end{itemize}

Advertisement

\protect\hyperlink{after-top}{Continue reading the main story}

Supported by

\protect\hyperlink{after-sponsor}{Continue reading the main story}

\hypertarget{the-new-eviction-moratorium-what-you-need-to-know}{%
\section{The New Eviction Moratorium: What You Need to
Know}\label{the-new-eviction-moratorium-what-you-need-to-know}}

A Trump administration order could allow many renters to avoid eviction
through Dec. 31. We answer renters' questions here.

\includegraphics{https://static01.graylady3jvrrxbe.onion/images/2020/09/02/business/02evictionfaq/merlin_133423170_63034e6f-846e-497e-b581-c7558039fe20-articleLarge.jpg?quality=75\&auto=webp\&disable=upscale}

\href{https://www.nytimes3xbfgragh.onion/by/ron-lieber}{\includegraphics{https://static01.graylady3jvrrxbe.onion/images/2018/10/22/multimedia/author-ron-lieber/author-ron-lieber-thumbLarge.png}}

By \href{https://www.nytimes3xbfgragh.onion/by/ron-lieber}{Ron Lieber}

\begin{itemize}
\item
  Published Sept. 2, 2020Updated Sept. 3, 2020
\item
  \begin{itemize}
  \item
  \item
  \item
  \item
  \item
  \end{itemize}
\end{itemize}

\href{https://www.nytimes3xbfgragh.onion/es/2020/09/02/espanol/negocios/desalojos-trump.html}{Leer
en español}

The Trump administration has announced an
\href{https://s3.amazonaws.com/public-inspection.federalregister.gov/2020-19654.pdf}{order}
to suspend the possibility of eviction for millions of renters who have
suffered financially because of the coronavirus pandemic. The
\href{https://www.nytimes3xbfgragh.onion/2020/09/01/business/eviction-moratorium-order.html}{Centers
for Disease Control and Prevention} said the order was an emergency
action, which it is entitled to take under the law.

Here are the answers to questions that renters may have about the order,
which is more expansive than the now-expired moratorium that was part of
the virus relief package this spring. We will add to this list as we
learn more. Please email your questions to
\href{mailto:hubforhelp@NYTimes.com}{\nolinkurl{hubforhelp@NYTimes.com}}.

\textbf{Who is eligible?}

You must meet a five-pronged test.

\begin{itemize}
\item
  You need to have used your ``best efforts'' to obtain any and all
  forms of government rental assistance.
\item
  You can't ``expect'' to earn more than \$99,000 in 2020, or \$198,000
  if you're married and filing a joint tax return. If you don't qualify
  that way, you could still be eligible if you did not need to report
  any income at all to the federal government in 2019 or if you received
  a stimulus check this year.
\item
  You must be experiencing a ``substantial'' loss of household income, a
  layoff or ``extraordinary'' out-of-pocket medical expenses (which the
  order defines as any unreimbursed expense likely to exceed 7.5 percent
  of your adjusted gross income this year).
\item
  You have to be making your best efforts to make ``timely'' partial
  payments that are as close to the full amount due as ``circumstances
  may permit,'' taking into account other nondiscretionary expenses.
\item
  Eviction would ``likely'' lead to either homelessness or your having
  to move to a place that was more expensive or where you could get sick
  from being close to others.
\end{itemize}

\textbf{A lot of that is pretty subjective. If it's a close call, who
decides?}

Landlords who disagree with renters' self-assessments could try to evict
nonpaying tenants by arguing that they are not a ``covered person''
within the order's scope and dare them to fight back legally. Then it
could be up to a housing court judge to decide if a renter is eligible
or if the landlord can, in fact, evict.

\textbf{How do I prove to my landlord that I'm eligible?}

You can use the
\href{https://www.cdc.gov/coronavirus/2019-ncov/downloads/declaration-form.pdf}{declaration
form} that the C.D.C. published on its website.

Soon after the order appeared, the Legal Innovation and Technology lab
at Suffolk University Law School created an
\href{https://massaccess.suffolklitlab.org/housing/\#CDC}{interactive
tool} that can help people determine if they are eligible. It can also
generate a declaration to give to a landlord.

\textbf{The sample declaration form does not say anything about whether
I need to prove my hardship to my landlord. Should I attach bank
statements or other documents?}

No, not to the declaration --- at least not at first. The way the order
is written means you need not lay out specifics in your declaration,
said Emily Benfer, a visiting professor of law at Wake Forest
University.

If the landlord challenges your initial assessment, however, you should
provide ``reasonable'' specifics to prove your eligibility, according to
senior administration officials who helped write the order.

\textbf{Who should make a declaration?}

The order says every adult who is on the lease should draft and sign a
separate declaration.

\hypertarget{latest-updates-the-coronavirus-outbreak}{%
\section{\texorpdfstring{\href{https://www.nytimes3xbfgragh.onion/2020/09/11/world/covid-19-coronavirus.html?action=click\&pgtype=Article\&state=default\&region=MAIN_CONTENT_1\&context=storylines_live_updates}{Latest
Updates: The Coronavirus
Outbreak}}{Latest Updates: The Coronavirus Outbreak}}\label{latest-updates-the-coronavirus-outbreak}}

Updated 2020-09-12T12:04:20.515Z

\begin{itemize}
\tightlist
\item
  \href{https://www.nytimes3xbfgragh.onion/2020/09/11/world/covid-19-coronavirus.html?action=click\&pgtype=Article\&state=default\&region=MAIN_CONTENT_1\&context=storylines_live_updates\#link-dfb8a16}{Fauci
  cautions the virus could disrupt life in the U.S. until `maybe even
  towards the end of 2021.'}
\item
  \href{https://www.nytimes3xbfgragh.onion/2020/09/11/world/covid-19-coronavirus.html?action=click\&pgtype=Article\&state=default\&region=MAIN_CONTENT_1\&context=storylines_live_updates\#link-7104d154}{From
  Asia to Africa, China promotes its vaccine candidates to win friends.}
\item
  \href{https://www.nytimes3xbfgragh.onion/2020/09/11/world/covid-19-coronavirus.html?action=click\&pgtype=Article\&state=default\&region=MAIN_CONTENT_1\&context=storylines_live_updates\#link-393ad215}{The
  other way the virus will kill: hunger.}
\end{itemize}

\href{https://www.nytimes3xbfgragh.onion/2020/09/11/world/covid-19-coronavirus.html?action=click\&pgtype=Article\&state=default\&region=MAIN_CONTENT_1\&context=storylines_live_updates}{See
more updates}

More live coverage:
\href{https://www.nytimes3xbfgragh.onion/live/2020/09/11/business/stock-market-today-coronavirus?action=click\&pgtype=Article\&state=default\&region=MAIN_CONTENT_1\&context=storylines_live_updates}{Markets}

\textbf{I have a roommate. How do the rules work for us?}

The order does not deal with roommates directly, but the officials
clarified that the income cap was \$99,000 per roommate. As for who
should pay what if just one person can't pay in full, the specifics may
depend on the terms of the lease, any written agreement between you and
your roommate, and applicable state or local law.

Eric Dunn, director of litigation for the National Housing Law Project,
said it was possible that housing court judges would interpret the order
expansively in this context. For example, consider a scenario where one
roommate would become homeless if evicted but the other could move in
with parents in an uncrowded home. In that instance, he said, the second
roommate could not truthfully sign the declaration.

So would only the first roommate receive protection from the moratorium?
``This would be an absurd result, and regulations should be interpreted
to avoid absurd results,'' Mr. Dunn said. He predicted that courts would
dismiss eviction cases filed against tenant households where at least
one member has signed a declaration.

\textbf{I'm in a pretty bad way. Can I stretch the truth some?}

You shouldn't. The order makes a point of noting that the declaration
``is sworn testimony, meaning that you can be prosecuted, go to jail or
pay a fine if you lie, mislead or omit important information.''

\textbf{What do I do with the declarations once they are done?}

Email, send or hand them to the landlord in a way that allows you to get
proof that the landlord received them. That way, there will be no
question as to whether you did what you were supposed to do. Make sure
you keep a copy for yourself.

\textbf{Then what?}

Keep paying as much as you can. Otherwise, you risk failing the
eligibility test, which says you should be trying to make partial
payments to the best of your ability.

\textbf{Can the landlord still evict me for reasons other than
nonpayment?}

Yes. All the usual rules about criminal behavior or disruptions or
destruction of property still apply. And it's possible that a landlord
will look hard for some other reason to start the eviction process, so
it's wise to follow every term of the lease, as well as any other
building or property rule.

Amy Woolard, a lawyer and policy coordinator for the Legal Aid Justice
Center in Charlottesville, Va., warned of one issue that she and her
colleagues frequently see cited in eviction cases: people not on the
lease who are living at the property. This could be an issue if you're
hosting guests --- like a family member who has already been evicted
elsewhere.

\textbf{Will interest or penalties accrue if I don't pay the rent in
full?}

The order does not prevent landlords from charging fees, penalties or
interest ``under the terms of any applicable contract.'' Nor does it
place any restrictions on how high they can go. Check your lease to see
if there is any mention of such charges.

\textbf{Will I have to pay everything I owe all at once in January?}

You might. The order specifically mentions this possibility. And the
National Rental Home Council, a trade group for landlords who own
single-family properties, said in a statement Wednesday that ``once the
moratorium expires, renters will owe back rent for several months.''

\textbf{Does the order halt evictions that are already in process?}

Yes, according to administration officials.

\textbf{Does the order apply to every landlord and every residential
renter in the country?}

No. Aside from the income caps, your local rules may apply instead. If
you're in a state, territory or tribal area that
\href{https://evictionlab.org/covid-eviction-policies/}{already has} a
moratorium in place that provides the same or better level of
protection, then that more local action will take its place. Local
jurisdictions are also still free to impose stronger restrictions than
the federal order. California's
\href{https://www.gov.ca.gov/2020/08/31/governor-newsom-signs-statewide-covid-19-tenant-and-landlord-protection-legislation/}{moratorium}
goes through the end of January, for example.

The federal moratorium doesn't apply in American Samoa, though it will
if it reports its first coronavirus cases.

\textbf{I'm living in a motel right now. Does the order apply to those
properties?}

No. The order specifically excludes hotels and motels.

\textbf{What about Airbnb rentals and other similar properties?}

The order excludes any ``guesthouse rented to a temporary guest or
seasonal tenant as defined under the laws of the state, territorial,
tribal or local jurisdiction.''

\textbf{What if my landlord sends me an eviction notice anyway?}

Seek counsel. You can search for a low- or no-cost legal assistance
office near you via the Legal Services Corporation's online
\href{https://www.lsc.gov/what-legal-aid/find-legal-aid}{map}. Just
Shelter, a tenant advocacy group, also offers
\href{https://justshelter.org/community-resources/}{information} on
local organizations that can help renters.

A lawyer can also help if a landlord tries a different approach. For
instance, a landlord might try to sue in small claims court over partial
payments, without filing an eviction notice that might be illegal under
the order, Mr. Dunn said.

\textbf{Does the order specify the size of the penalties that landlords
may be subject to?}

Yes. An individual landlord could be subject to a fine up to \$100,000
if no death (say from someone getting sick after eviction) results from
the violation, or one year in jail, or both. If a death occurs, the fine
rises to no more than \$250,000. If it's an organization in violation,
the fines are \$200,000 or \$500,000.

\textbf{Is the order legal?}

The White House and the C.D.C. think so. It is possible that landlord
industry groups or others will sue to stop it, in which case it will be
up to the courts to decide.

\textbf{Could some local housing judges simply ignore the order?}

Lawyers on the ground say they would not be surprised to see that in
smaller jurisdictions. ``Then it would be up to the tenant to scrape
together enough resources to try to file in federal court or seek an
injunction from another authority in their state's judicial system,''
said Rebecca Maurer, a lawyer in Cleveland.

\textbf{When does the order take effect, and how long does it last?}

It takes effect as soon as it is published in the Federal Register. The
order says that will happen on Sept. 4. The order applies through Dec.
31, and it's possible that it could be extended.

\textbf{I'm dizzy from all of the various local, state and federal
orders. Is this the last of them?}

Maybe not. Congress could pass a new relief package that would supersede
this order.

Matthew Goldstein contributed reporting.

Advertisement

\protect\hyperlink{after-bottom}{Continue reading the main story}

\hypertarget{site-index}{%
\subsection{Site Index}\label{site-index}}

\hypertarget{site-information-navigation}{%
\subsection{Site Information
Navigation}\label{site-information-navigation}}

\begin{itemize}
\tightlist
\item
  \href{https://help.nytimes3xbfgragh.onion/hc/en-us/articles/115014792127-Copyright-notice}{©~2020~The
  New York Times Company}
\end{itemize}

\begin{itemize}
\tightlist
\item
  \href{https://www.nytco.com/}{NYTCo}
\item
  \href{https://help.nytimes3xbfgragh.onion/hc/en-us/articles/115015385887-Contact-Us}{Contact
  Us}
\item
  \href{https://www.nytco.com/careers/}{Work with us}
\item
  \href{https://nytmediakit.com/}{Advertise}
\item
  \href{http://www.tbrandstudio.com/}{T Brand Studio}
\item
  \href{https://www.nytimes3xbfgragh.onion/privacy/cookie-policy\#how-do-i-manage-trackers}{Your
  Ad Choices}
\item
  \href{https://www.nytimes3xbfgragh.onion/privacy}{Privacy}
\item
  \href{https://help.nytimes3xbfgragh.onion/hc/en-us/articles/115014893428-Terms-of-service}{Terms
  of Service}
\item
  \href{https://help.nytimes3xbfgragh.onion/hc/en-us/articles/115014893968-Terms-of-sale}{Terms
  of Sale}
\item
  \href{https://spiderbites.nytimes3xbfgragh.onion}{Site Map}
\item
  \href{https://help.nytimes3xbfgragh.onion/hc/en-us}{Help}
\item
  \href{https://www.nytimes3xbfgragh.onion/subscription?campaignId=37WXW}{Subscriptions}
\end{itemize}
