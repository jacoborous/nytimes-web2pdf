Sections

SEARCH

\protect\hyperlink{site-content}{Skip to
content}\protect\hyperlink{site-index}{Skip to site index}

\href{https://www.nytimes3xbfgragh.onion/section/food}{Food}

\href{https://myaccount.nytimes3xbfgragh.onion/auth/login?response_type=cookie\&client_id=vi}{}

\href{https://www.nytimes3xbfgragh.onion/section/todayspaper}{Today's
Paper}

\href{/section/food}{Food}\textbar{}You're Going to Love This BLT Tart

\url{https://nyti.ms/32S7oUy}

\begin{itemize}
\item
\item
\item
\item
\item
\item
\end{itemize}

\href{https://www.nytimes3xbfgragh.onion/spotlight/at-home?action=click\&pgtype=Article\&state=default\&region=TOP_BANNER\&context=at_home_menu}{At
Home}

\begin{itemize}
\tightlist
\item
  \href{https://www.nytimes3xbfgragh.onion/2020/09/07/travel/route-66.html?action=click\&pgtype=Article\&state=default\&region=TOP_BANNER\&context=at_home_menu}{Cruise
  Along: Route 66}
\item
  \href{https://www.nytimes3xbfgragh.onion/2020/09/04/dining/sheet-pan-chicken.html?action=click\&pgtype=Article\&state=default\&region=TOP_BANNER\&context=at_home_menu}{Roast:
  Chicken With Plums}
\item
  \href{https://www.nytimes3xbfgragh.onion/2020/09/04/arts/television/dark-shadows-stream.html?action=click\&pgtype=Article\&state=default\&region=TOP_BANNER\&context=at_home_menu}{Watch:
  Dark Shadows}
\item
  \href{https://www.nytimes3xbfgragh.onion/interactive/2020/at-home/even-more-reporters-editors-diaries-lists-recommendations.html?action=click\&pgtype=Article\&state=default\&region=TOP_BANNER\&context=at_home_menu}{Explore:
  Reporters' Google Docs}
\end{itemize}

Advertisement

\protect\hyperlink{after-top}{Continue reading the main story}

Supported by

\protect\hyperlink{after-sponsor}{Continue reading the main story}

From The Pantry

\hypertarget{youre-going-to-love-this-blt-tart}{%
\section{You're Going to Love This BLT
Tart}\label{youre-going-to-love-this-blt-tart}}

This recipe showcases the best parts of the classic sandwich in a
vibrant end-of-summer tart.

\includegraphics{https://static01.graylady3jvrrxbe.onion/images/2020/09/02/dining/02pantry/02pantry-articleLarge.jpg?quality=75\&auto=webp\&disable=upscale}

\href{https://www.nytimes3xbfgragh.onion/by/melissa-clark}{\includegraphics{https://static01.graylady3jvrrxbe.onion/images/2018/06/21/multimedia/author-melissa-clark/author-melissa-clark-thumbLarge.png}}

By \href{https://www.nytimes3xbfgragh.onion/by/melissa-clark}{Melissa
Clark}

\begin{itemize}
\item
  Sept. 2, 2020
\item
  \begin{itemize}
  \item
  \item
  \item
  \item
  \item
  \item
  \end{itemize}
\end{itemize}

I may eat my weight in BLT sandwiches all tomato season long, but
there's still room for variations on the theme. Take, for example, this
crunchy BLT tart with ripe, end-of-the-summer tomatoes, crumbled candied
bacon and a thicket of baby lettuces greening the top.

This recipe was inspired by an excellent
\href{https://cooking.nytimes3xbfgragh.onion/recipes/1020373-roasted-tomato-tart-with-ricotta-and-pesto}{roasted
tomato tart} created by my colleague Alexa Weibel. In her version,
dollops of fresh ricotta and drizzles of pesto enrich a base of sliced
tomatoes nestled in puff pastry.

I substituted bacon and lettuce for the ricotta and pesto to make a tart
with a sweeter, porkier flavor profile that's still just as juicy and
buttery.

Using purchased puff pastry keeps thing easy. Draining the tomatoes for
20 minutes before assembling the tart keeps the pastry from developing
soggy bottom syndrome, which means it will get nice and crisp in the
oven.

First though, you'll need to track down and defrost one package of puff
pastry. I used a 14-ounce packet, but if yours is slightly larger or
smaller, that's fine, too; anything between 12 ounces and one pound will
work. If you have a choice, go for the bigger package because more puff
pastry is always better.

Then there's the candied bacon, which is worth making all by itself.
Sweet, peppery and crunchy-chewy, it's just as perfect for brunch as it
is glistening on top of this tart.

To make it, heat your oven to 375 degrees and line a rimmed baking pan
with parchment paper. (You can skip the parchment, but it makes clean up
much easier.) Spread 8 ounces bacon strips in the pan, and sprinkle them
with 2 tablespoons light brown sugar and some freshly ground black
pepper. (Don't use thick cut bacon here, it never gets quite as crisp as
the thinner kind.) Bake until bubbling and deeply browned, 20 to 25
minutes. Transfer the cooked bacon to a wire rack until it's cool enough
to crumble, and raise the oven temperature to 400 degrees.

While the bacon is baking and cooling, drain the tomatoes. Slice up one
pound of small, thick-skinned tomatoes, preferably in a rainbow of
colors. Plum and grape tomatoes work well here because they have more
flesh and less pastry-soaking juice. Lay the slices out on a clean
kitchen towel or a double layer of paper towels and season them with
salt and pepper. Let them drain until the bacon is ready.

Place the puff pastry on a piece of parchment paper and roll it into a
10-by-13-inch rectangle. Transfer pastry and parchment to a baking
sheet.

Prick the pastry all over with a fork, leaving a 1/2-inch border. Within
the border, spread 2 tablespoons crème fraîche, sour cream, mascarpone,
mayonnaise or Greek yogurt over the pastry. Sprinkle with 2 tablespoons
Parmesan.

Lay the drained tomatoes over the cheese. Crumble the bacon on top and
sprinkle with another tablespoon or so of Parmesan.

Bake until the pastry is deeply golden all over, 30 to 35 minutes.

Transfer the tart to a wire rack for at least 10 minutes to cool
slightly. Just before serving, scatter 1 to 2 cups baby lettuces over
the top (or use a combination of baby lettuce mixed with herbs like
basil and mint). Drizzle with good olive oil and sprinkle with flaky sea
salt and serve immediately, before the lettuces wilt.

This tart is at its best within 2 or 3 hours of baking, but I will admit
to polishing off leftovers cold from the fridge the next day. The pastry
had gotten soft and the lettuces, a little wilted. But even so, it was a
satisfying taste of summer in all of its evanescent, tomato-rich glory.

\emph{Follow} \href{https://twitter.com/nytfood}{\emph{NYT Food on
Twitter}} \emph{and}
\href{https://www.instagram.com/nytcooking/}{\emph{NYT Cooking on
Instagram}}\emph{,}
\href{https://www.facebookcorewwwi.onion/nytcooking/}{\emph{Facebook}}\emph{,}
\href{https://www.youtube.com/nytcooking}{\emph{YouTube}} \emph{and}
\href{https://www.pinterest.com/nytcooking/}{\emph{Pinterest}}\emph{.}
\href{https://www.nytimes3xbfgragh.onion/newsletters/cooking}{\emph{Get
regular updates from NYT Cooking, with recipe suggestions, cooking tips
and shopping advice}}\emph{.}

Advertisement

\protect\hyperlink{after-bottom}{Continue reading the main story}

\hypertarget{site-index}{%
\subsection{Site Index}\label{site-index}}

\hypertarget{site-information-navigation}{%
\subsection{Site Information
Navigation}\label{site-information-navigation}}

\begin{itemize}
\tightlist
\item
  \href{https://help.nytimes3xbfgragh.onion/hc/en-us/articles/115014792127-Copyright-notice}{©~2020~The
  New York Times Company}
\end{itemize}

\begin{itemize}
\tightlist
\item
  \href{https://www.nytco.com/}{NYTCo}
\item
  \href{https://help.nytimes3xbfgragh.onion/hc/en-us/articles/115015385887-Contact-Us}{Contact
  Us}
\item
  \href{https://www.nytco.com/careers/}{Work with us}
\item
  \href{https://nytmediakit.com/}{Advertise}
\item
  \href{http://www.tbrandstudio.com/}{T Brand Studio}
\item
  \href{https://www.nytimes3xbfgragh.onion/privacy/cookie-policy\#how-do-i-manage-trackers}{Your
  Ad Choices}
\item
  \href{https://www.nytimes3xbfgragh.onion/privacy}{Privacy}
\item
  \href{https://help.nytimes3xbfgragh.onion/hc/en-us/articles/115014893428-Terms-of-service}{Terms
  of Service}
\item
  \href{https://help.nytimes3xbfgragh.onion/hc/en-us/articles/115014893968-Terms-of-sale}{Terms
  of Sale}
\item
  \href{https://spiderbites.nytimes3xbfgragh.onion}{Site Map}
\item
  \href{https://help.nytimes3xbfgragh.onion/hc/en-us}{Help}
\item
  \href{https://www.nytimes3xbfgragh.onion/subscription?campaignId=37WXW}{Subscriptions}
\end{itemize}
