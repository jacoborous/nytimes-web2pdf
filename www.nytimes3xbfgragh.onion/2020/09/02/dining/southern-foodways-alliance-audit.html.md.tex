Sections

SEARCH

\protect\hyperlink{site-content}{Skip to
content}\protect\hyperlink{site-index}{Skip to site index}

\href{https://www.nytimes3xbfgragh.onion/section/food}{Food}

\href{https://myaccount.nytimes3xbfgragh.onion/auth/login?response_type=cookie\&client_id=vi}{}

\href{https://www.nytimes3xbfgragh.onion/section/todayspaper}{Today's
Paper}

\href{/section/food}{Food}\textbar{}Its Leader Under Fire, a Southern
Food Group Vows to Examine Racism

\url{https://nyti.ms/31P54ys}

\begin{itemize}
\item
\item
\item
\item
\item
\item
\end{itemize}

Advertisement

\protect\hyperlink{after-top}{Continue reading the main story}

Supported by

\protect\hyperlink{after-sponsor}{Continue reading the main story}

\hypertarget{its-leader-under-fire-a-southern-food-group-vows-to-examine-racism}{%
\section{Its Leader Under Fire, a Southern Food Group Vows to Examine
Racism}\label{its-leader-under-fire-a-southern-food-group-vows-to-examine-racism}}

The University of Mississippi will commission an outside audit of the
Southern Foodways Alliance, looking into its record on diversity, after
calls for the group's white director to resign.

\includegraphics{https://static01.graylady3jvrrxbe.onion/images/2020/09/02/dining/02southernfood2/merlin_121650458_0946d3da-279a-44c2-8b46-1a551f15816a-articleLarge.jpg?quality=75\&auto=webp\&disable=upscale}

\href{https://www.nytimes3xbfgragh.onion/by/kim-severson}{\includegraphics{https://static01.graylady3jvrrxbe.onion/images/2018/06/13/multimedia/author-kim-severson/author-kim-severson-thumbLarge.jpg}}

By \href{https://www.nytimes3xbfgragh.onion/by/kim-severson}{Kim
Severson}

\begin{itemize}
\item
  Sept. 2, 2020
\item
  \begin{itemize}
  \item
  \item
  \item
  \item
  \item
  \item
  \end{itemize}
\end{itemize}

Two months ago, several women and people of color who had worked for and
supported the \href{https://www.southernfoodways.org/}{Southern Foodways
Alliance}, an association dedicated to the study and preservation of the
region's food and to healing its tortured racist history, called on John
T. Edge, the influential white man who heads the alliance,
\href{https://www.nytimes3xbfgragh.onion/2020/06/29/dining/john-t-edge-southern-foodways-alliance.html}{to
step aside}.

On Tuesday night, a committee from the University of Mississippi, where
the organization is based, gave them an answer in the form of
\href{https://southernstudies.olemiss.edu/cssc-sfa-joint-statement/}{a
1,500-word statement}.

It mentioned Mr. Edge once, but did not address his employment. Rather,
it called for an examination, including an outside audit, of how
institutional racism and patriarchy affect both the university's
\href{https://southernstudies.olemiss.edu/}{Center for the Study of
Southern Culture} --- under which the Southern Foodways Alliance
operates --- and the alliance itself.

It noted an ``urgent need'' for the alliance to assemble a more diverse
leadership team and staff, and called for a deeper examination of the
relationship between the center and the alliance, which for 20 years has
operated in a loose partnership with that more established academic
body.

``A recurring pattern in the feedback we've received is that structural
change is needed, and needed urgently,'' the report said. ``We agree.''

The 12-person committee, selected in part by Kathryn B. McKee, the
director of the Center for the Study of Southern Culture, included
faculty and staff members of the center and members of the alliance
staff and board. (Mr. Edge is not on the panel.)

The panel's statement was met with skepticism from people who, in
letters to the alliance board and the university, had described an
organization and leader whose work had been essential to the alliance
but was now standing in the way of change.

``It's a Band-Aid on a gunshot wound,'' said
\href{https://www.nytimes3xbfgragh.onion/2017/06/06/dining/chef-asha-gomez-india.html}{Asha
Gomez}, an Atlanta-area chef who has been one of Mr. Edge's critics.

``It's filled with really strong and thoughtful intention, but the more
I look at it there's not really a clear action in this,'' said
\href{https://www.southernfoodways.org/interview/ronni-lundy/}{Ronni
Lundy}, the Appalachian-food scholar and a founder of the alliance, who
in June delivered a letter calling for Mr. Edge to step aside. ``There
is no answer to the pointed question, which is people asking for John T.
to step down.''

\includegraphics{https://static01.graylady3jvrrxbe.onion/images/2020/09/02/dining/02southernfood/merlin_157416978_e549153d-9b09-48f1-87c2-e7438a970784-articleLarge.jpg?quality=75\&auto=webp\&disable=upscale}

Her letter to Mr. Edge and the alliance came in June, days after the
chef
\href{https://www.nytimes3xbfgragh.onion/interactive/2019/07/16/dining/black-chefs-restaurants-food.html}{Tunde
Wey} asked Mr. Edge to step down and cede his position to a Black woman.
Mr. Wey made the request during a James Beard Foundation webinar that
was something of a reprise of a 2016 column he and Mr. Edge wrote in The
Oxford American, titled ``Who Owns Southern Food?''

That article ran as an installment of Mr. Edge's regular column, and he
shared it with Mr. Wey, who is Black, as a device to explore white
privilege and its impact on Southern food culture.

After the webinar and Ms. Lundy's letter, former staff members and
Southern cooks who had worked with the alliance came forward asking for
Mr. Edge's resignation.

Mr. Edge did not respond Wednesday to a request to comment. But in
earlier interviews, he said he had embarked on a long-term plan to
better endow the organization and pay for his replacement. Part of that
eventually will come from
\href{https://news.olemiss.edu/gift-helps-ensure-success-for-the-southern-foodways-alliance/}{a
\$1 million donation} to the John T. Edge Southern Foodways Alliance
Director Endowment that a retired California couple pledged earlier this
summer.

The statement the alliance issued on Tuesday calls for an audit of the
organization by an ``external agency,'' yet to be chosen, to examine the
ways institutional racism and patriarchy have shaped --- intentionally
or not --- the structure and programming of both the alliance and
center.

``The coming reassessment is not symbolic,'' the statement says. ``It is
not review for the sake of review, or dialogue for the sake of
discussion. It is a deliberate effort to identify and implement a set of
concrete, actionable recommendations that will address what we see as
valid concerns about diversity, equity, and inclusion in the composition
of S.F.A.'s paid staff; the leadership of S.F.A.'s general operations
and year-to-year programming; and the structure of S.F.A.'s
leadership.''

The criticism has been difficult for the alliance and its supporters,
many of whom have energetically defended Mr. Edge and the work the
alliance has done over two decades. Mr. Edge, 57, has been instrumental
in creating an organization that offers a powerful stage for cooks,
writers and academics who gather each fall for its symposium in Oxford,
Miss. The alliance also has a prolific media arm that collects oral
histories, produces documentaries and publishes scholarly food articles
from the South.

Mr. Edge himself became a prolific author and a media star, whose
writing on Southern culture have appeared in several publications,
including The New York Times. He also has a television show on ESPN
called ``True South.''

The statement suggests that the alliance hire new people over the next
several years to diversify its staff and redistribute decision-making
power. But it notes that the university faces a pandemic-driven economic
crisis, which includes a hiring freeze for the current academic year.

\emph{Follow} \href{https://twitter.com/nytfood}{\emph{NYT Food on
Twitter}} \emph{and}
\href{https://www.instagram.com/nytcooking/}{\emph{NYT Cooking on
Instagram}}\emph{,}
\href{https://www.facebookcorewwwi.onion/nytcooking/}{\emph{Facebook}}\emph{,}
\href{https://www.youtube.com/nytcooking}{\emph{YouTube}} \emph{and}
\href{https://www.pinterest.com/nytcooking/}{\emph{Pinterest}}\emph{.}
\href{https://www.nytimes3xbfgragh.onion/newsletters/cooking}{\emph{Get
regular updates from NYT Cooking, with recipe suggestions, cooking tips
and shopping advice}}\emph{.}

Advertisement

\protect\hyperlink{after-bottom}{Continue reading the main story}

\hypertarget{site-index}{%
\subsection{Site Index}\label{site-index}}

\hypertarget{site-information-navigation}{%
\subsection{Site Information
Navigation}\label{site-information-navigation}}

\begin{itemize}
\tightlist
\item
  \href{https://help.nytimes3xbfgragh.onion/hc/en-us/articles/115014792127-Copyright-notice}{©~2020~The
  New York Times Company}
\end{itemize}

\begin{itemize}
\tightlist
\item
  \href{https://www.nytco.com/}{NYTCo}
\item
  \href{https://help.nytimes3xbfgragh.onion/hc/en-us/articles/115015385887-Contact-Us}{Contact
  Us}
\item
  \href{https://www.nytco.com/careers/}{Work with us}
\item
  \href{https://nytmediakit.com/}{Advertise}
\item
  \href{http://www.tbrandstudio.com/}{T Brand Studio}
\item
  \href{https://www.nytimes3xbfgragh.onion/privacy/cookie-policy\#how-do-i-manage-trackers}{Your
  Ad Choices}
\item
  \href{https://www.nytimes3xbfgragh.onion/privacy}{Privacy}
\item
  \href{https://help.nytimes3xbfgragh.onion/hc/en-us/articles/115014893428-Terms-of-service}{Terms
  of Service}
\item
  \href{https://help.nytimes3xbfgragh.onion/hc/en-us/articles/115014893968-Terms-of-sale}{Terms
  of Sale}
\item
  \href{https://spiderbites.nytimes3xbfgragh.onion}{Site Map}
\item
  \href{https://help.nytimes3xbfgragh.onion/hc/en-us}{Help}
\item
  \href{https://www.nytimes3xbfgragh.onion/subscription?campaignId=37WXW}{Subscriptions}
\end{itemize}
