Sections

SEARCH

\protect\hyperlink{site-content}{Skip to
content}\protect\hyperlink{site-index}{Skip to site index}

\href{https://myaccount.nytimes3xbfgragh.onion/auth/login?response_type=cookie\&client_id=vi}{}

\href{https://www.nytimes3xbfgragh.onion/section/todayspaper}{Today's
Paper}

\href{/section/opinion}{Opinion}\textbar{}Mark Zuckerberg Is the Most
Powerful Unelected Man in America

\url{https://nyti.ms/32PDTTq}

\begin{itemize}
\item
\item
\item
\item
\item
\item
\end{itemize}

Advertisement

\protect\hyperlink{after-top}{Continue reading the main story}

\href{/section/opinion}{Opinion}

Supported by

\protect\hyperlink{after-sponsor}{Continue reading the main story}

\hypertarget{mark-zuckerberg-is-the-most-powerful-unelected-man-in-america}{%
\section{Mark Zuckerberg Is the Most Powerful Unelected Man in
America}\label{mark-zuckerberg-is-the-most-powerful-unelected-man-in-america}}

Facebook is too big for democracy.

\href{https://www.nytimes3xbfgragh.onion/by/charlie-warzel}{\includegraphics{https://static01.graylady3jvrrxbe.onion/images/2019/03/15/opinion/charlie-warzel/charlie-warzel-thumbLarge-v3.png}}

By \href{https://www.nytimes3xbfgragh.onion/by/charlie-warzel}{Charlie
Warzel}

Opinion writer at large.

\begin{itemize}
\item
  Sept. 3, 2020
\item
  \begin{itemize}
  \item
  \item
  \item
  \item
  \item
  \item
  \end{itemize}
\end{itemize}

\includegraphics{https://static01.graylady3jvrrxbe.onion/images/2020/09/05/opinion/05warzel-print-editorial/merlin_163199223_19e1a0ef-0e61-4a38-ba44-b80b96922dcb-articleLarge.jpg?quality=75\&auto=webp\&disable=upscale}

On Thursday, Facebook's chief executive, Mark Zuckerberg,
\href{https://www.nytimes3xbfgragh.onion/2020/09/03/technology/facebook-election-chaos-november.html}{announced
the company's} ``New Steps to Protect the U.S. Elections.'' They include
blocking new political ads in the week leading up to Election Day and
attaching labels to posts containing misinformation, specifically
related to the coronavirus and posts from politicians declaring victory
before all the results are counted.

One can ---
\href{https://www.washingtonpost.com/politics/2020/09/03/facebook-made-change-that-could-help-control-election-misinformation-it-isnt-ban-ads/}{and
many will}--- debate just how effective these measures will be at
preventing election night chaos during a pandemic. (So far Facebook's
``misleading post'' labels are
\href{https://twitter.com/sheeraf/status/1301539720199979013}{vague to
the point} of causing additional confusion for voters. Similarly,
blocking new political ads one week out from the vote ignores the vast
amounts of disinformation Americans are subjected to year after year.)
But what seems beyond debate is just how deeply Facebook has woven
itself into the fabric of democracy.

Reading Mr. Zuckerberg's election security blog post reminded me of a
line from
\href{https://nymag.com/intelligencer/2017/10/does-even-mark-zuckerberg-know-what-facebook-is.html}{a
seminal 2017 article by the journalist Max Read}. Three years ago, Mr.
Read was struck by a similar pledge from Mr. Zuckerberg to ``ensure the
integrity'' of the German elections. The commitment was admirable, he
wrote, but also a tacit admission of Facebook's immense power. ``It's a
declaration that Facebook is assuming a level of power at once of the
state and beyond it, as a sovereign, self-regulating, suprastate entity
within which states themselves operate.''

That power is consolidated in the decisions of its chief executive, who
has voting control over the company. Here's how Facebook's co-founder
\href{https://www.nytimes3xbfgragh.onion/2019/05/09/opinion/sunday/chris-hughes-facebook-zuckerberg.html}{Chris
Hughes described} Mr. Zuckerberg's iron grip on the company in The Times
last year:

\begin{quote}
Mark's influence is staggering, far beyond that of anyone else in the
private sector or in government. He controls three core communications
platforms ---
\href{https://www.nytimes3xbfgragh.onion/2019/05/09/business/facebook-response-chris-hughes.html}{Facebook},
Instagram and WhatsApp --- that billions of people use every day.
Facebook's board works more like an advisory committee than an overseer,
because Mark
\href{https://www.vox.com/technology/2018/11/19/18099011/mark-zuckerberg-facebook-stock-nyt-wsj}{controls
around 60 percent of voting shares}. Mark alone can decide how to
configure Facebook's algorithms to determine what people see in their
News Feeds, what privacy settings they can use and even which messages
get delivered. He sets the rules for how to distinguish violent and
incendiary speech from the merely offensive, and he can choose to shut
down a competitor by acquiring, blocking or copying it.
\end{quote}

If Mr. Hughes's description feels hyperbolic, it may be because such a
consolidation of power is actually hard to comprehend.

``I think we underestimate Facebook's power constantly,'' Siva
Vaidhyanathan, a professor of media studies at the University of
Virginia, told me. ``It's really hard for human beings to picture in
their head the actual size and influence of the platform. Something like
one out of three people use the thing --- it's like nothing we've
encountered in human history. And I'm not sure Mark Zuckerberg is even
willing to contemplate his influence. I'm not sure he'd ever sleep if he
ever thought about how much power he has.''

Facebook's power is now self perpetuating. This week provided a great
example. On Tuesday, Facebook and other platforms revealed
\href{https://www.nytimes3xbfgragh.onion/2020/09/01/technology/facebook-russia-disinformation-election.html}{a
covert operation run by the Kremlin-backed Internet Research Agency} to
sow division ahead of the presidential election by setting up a network
of fake user accounts and websites. This time, though, the agency hired
unwitting American freelance journalists to create the content.

There's a grim circle-of-life quality to this news. Facebook's
unprecedented growth and commandeering of the digital advertising market
---~alongside Google and others --- helped accelerate the collapse of
journalism's broken business models. This led to consolidation,
publications shuttering and layoffs of journalists everywhere.
Facebook's news dominance and mercurial distribution algorithms led to a
rise of hyperpartisan pages and websites to fill the gaps and capitalize
on the platform's ability to monetize engagement, which in turn led to a
glut of viral misinformation and disinformation that Facebook has been
unable (or perhaps unwilling) to adequately police.

This free-for-all has made Facebook the platform of choice for political
manipulation. Those bad actors are now hiring and exploiting the very
freelance journalists displaced by the collapse of the media industry
that Facebook helped accelerate. Eventually, Facebook takes action to
remove the bad actors, assuring the country of its commitment to
democracy and cementing its role as a protector of free and fair
elections.

Facebook wins in every direction. Its size and power creates
instability, the answer to which, according to Facebook, is to give the
company additional authority.

This cycle is unsustainable. This summer has shown that the platform has
been a prime vector for the most destabilizing forces in American life.
It has helped
\href{https://www.nytimes3xbfgragh.onion/2020/08/15/opinion/qanon-marjorie-greene-congress.html}{supercharge
conspiracies} around the dangerous QAnon movement. It has provided
organization for, and
\href{https://www.theverge.com/2020/8/26/21403004/facebook-kenosha-militia-groups-shooting-blm-protest}{amplified
calls} to action from, militia movements, which have been linked to
deaths in U.S. cities at protests. Its moderation policies have failed
to catch blatant rule violations around voter disenfranchisement, and
the conspiracy theories that go viral on the platform have found their
way, time and again,
\href{https://www.nbcnews.com/politics/politics-news/trump-s-plane-loaded-thugs-rumor-matches-months-old-facebook-n1238962}{to
President Trump's mouth}.

Facebook employees seem to understand the situation is untenable and are
speaking out internally against Mr. Zuckerberg's leadership. ``He seems
truly incapable of taking personal responsibility for decisions and
actions at Facebook,'' one Facebook employee
\href{https://www.buzzfeednews.com/article/ryanmac/facebook-employees-slam-zuckerberg-kenosha-militia-shooting}{told
BuzzFeed News last week} after a company meeting in response to the
violence in Kenosha, Wis.

With just two months to go before the election, the nation's focus is on
the integrity of the electoral process. With the president threatening
to undermine the results of the election, the stakes could not be
higher. As Mr. Zuckerberg
\href{https://about.fb.com/news/2020/09/additional-steps-to-protect-the-us-elections/}{wrote}
on Thursday, ``We all have a responsibility to protect our democracy.''

But what does it say that one of those institutions charged with
protecting democracy is, itself, structured more like a dictatorship?

``Facebook had grown too big, and its users too complacent, for
democracy,'' Mr. Read concluded at the end of his 2017 piece. His words
feel prescient today as Facebook, unchecked and unregulated by
governments, positions itself as a primary line of defense to protect
those institutions.

At first, Mr. Zuckerberg's recent election pledge might feel comforting
(Somebody! Doing something!). But his plan is an admission of a great
power that should make Americans uncomfortable. In our quest to fend off
a would-be strongman's power grab in one realm, we ought not allow a
stronger man's power grab in another.

\emph{The Times is committed to publishing}
\href{https://www.nytimes3xbfgragh.onion/2019/01/31/opinion/letters/letters-to-editor-new-york-times-women.html}{\emph{a
diversity of letters}} \emph{to the editor. We'd like to hear what you
think about this or any of our articles. Here are some}
\href{https://help.nytimes3xbfgragh.onion/hc/en-us/articles/115014925288-How-to-submit-a-letter-to-the-editor}{\emph{tips}}\emph{.
And here's our
email:}\href{mailto:letters@NYTimes.com}{\emph{letters@NYTimes.com}}\emph{.}

\emph{Follow The New York Times Opinion section on}
\href{https://www.facebookcorewwwi.onion/nytopinion}{\emph{Facebook}}\emph{,}
\href{http://twitter.com/NYTOpinion}{\emph{Twitter (@NYTopinion)}}
\emph{and}
\href{https://www.instagram.com/nytopinion/}{\emph{Instagram}}\emph{.}

Advertisement

\protect\hyperlink{after-bottom}{Continue reading the main story}

\hypertarget{site-index}{%
\subsection{Site Index}\label{site-index}}

\hypertarget{site-information-navigation}{%
\subsection{Site Information
Navigation}\label{site-information-navigation}}

\begin{itemize}
\tightlist
\item
  \href{https://help.nytimes3xbfgragh.onion/hc/en-us/articles/115014792127-Copyright-notice}{©~2020~The
  New York Times Company}
\end{itemize}

\begin{itemize}
\tightlist
\item
  \href{https://www.nytco.com/}{NYTCo}
\item
  \href{https://help.nytimes3xbfgragh.onion/hc/en-us/articles/115015385887-Contact-Us}{Contact
  Us}
\item
  \href{https://www.nytco.com/careers/}{Work with us}
\item
  \href{https://nytmediakit.com/}{Advertise}
\item
  \href{http://www.tbrandstudio.com/}{T Brand Studio}
\item
  \href{https://www.nytimes3xbfgragh.onion/privacy/cookie-policy\#how-do-i-manage-trackers}{Your
  Ad Choices}
\item
  \href{https://www.nytimes3xbfgragh.onion/privacy}{Privacy}
\item
  \href{https://help.nytimes3xbfgragh.onion/hc/en-us/articles/115014893428-Terms-of-service}{Terms
  of Service}
\item
  \href{https://help.nytimes3xbfgragh.onion/hc/en-us/articles/115014893968-Terms-of-sale}{Terms
  of Sale}
\item
  \href{https://spiderbites.nytimes3xbfgragh.onion}{Site Map}
\item
  \href{https://help.nytimes3xbfgragh.onion/hc/en-us}{Help}
\item
  \href{https://www.nytimes3xbfgragh.onion/subscription?campaignId=37WXW}{Subscriptions}
\end{itemize}
