Sections

SEARCH

\protect\hyperlink{site-content}{Skip to
content}\protect\hyperlink{site-index}{Skip to site index}

\href{https://myaccount.nytimes3xbfgragh.onion/auth/login?response_type=cookie\&client_id=vi}{}

\href{https://www.nytimes3xbfgragh.onion/section/todayspaper}{Today's
Paper}

\href{/section/opinion}{Opinion}\textbar{}Is `American Carnage' Campaign
Gold?

\url{https://nyti.ms/3lKpzUT}

\begin{itemize}
\item
\item
\item
\item
\item
\end{itemize}

Advertisement

\protect\hyperlink{after-top}{Continue reading the main story}

transcript

Back to The Argument

bars

0:00/0:00

-0:00

transcript

\hypertarget{is-american-carnage-campaign-gold}{%
\subsection{Is `American Carnage' Campaign
Gold?}\label{is-american-carnage-campaign-gold}}

\hypertarget{with-frank-bruni-ross-douthat-and-michelle-goldberg}{%
\subsubsection{With Frank Bruni, Ross Douthat and Michelle
Goldberg}\label{with-frank-bruni-ross-douthat-and-michelle-goldberg}}

\hypertarget{what-is-the-political-fallout-of-urban-violence-and-is-the-media-just-buying-into-it}{%
\paragraph{What is the political fallout of urban violence? And is the
media just buying into
it?}\label{what-is-the-political-fallout-of-urban-violence-and-is-the-media-just-buying-into-it}}

Thursday, September 3rd, 2020

\begin{itemize}
\item
  michelle goldberg\\
  I'm Michelle Goldberg.
\item
  ross douthat\\
  I'm Ross Douthat.
\item
  frank bruni\\
  I'm Frank Bruni. And this is ``The Argument.'' {[}THEME MUSIC{]}
  Protests against police brutality have been reignited after the
  shooting of Jacob Blake. And the Trump campaign sees an opportunity.
  As counter-protests turn lethal and the president continues to stoke
  division, what's the political fallout of violence in American cities?
  How did it become a cudgel against Joe Biden and the Democrats and not
  an indictment of the man actually in charge?

  So guys, this is our Peaches \& Herb moment. We're reunited. And
  speaking for myself, it feels so good.
\item
  michelle goldberg\\
  I know, and for the last time.
\item
  ross douthat\\
  It's wonderful to be back with you guys, even if Frank is poised to
  abandon us. And I really missed you guys, you know? Sure, I was
  sitting on a beach in Maine watching the time of the world go by,
  watching my children play. But really, I was thinking about arguing
  about politics the whole time. {[}LAUGHTER{]}
\item
  frank bruni\\
  You were looking at your lovely little children and thinking of
  Michelle and me, of course.
\item
  ross douthat\\
  Of course. I'm like, which one is more Frank, and which one is more
  Michelle? And my wife was like, what are you talking about?
  {[}LAUGHTER{]}
\item
  frank bruni\\
  So let's get right into our topic this week because it is a big one.
  Almost two weeks ago, in Kenosha, Wisconsin, a 29-year-old Black man
  named Jacob Blake was shot at close range in the back by a white
  police officer in front of Blake's three sons. Video of the shooting
  spread fast, instantly rekindling protests against police brutality.
  Two days into the protests, a 17-year-old, reportedly a Trump
  supporter, shot protesters in Kenosha, killing two people. Then, in
  Portland, Oregon, a caravan of Trump supporters were met with
  counter-protesters. And amid the skirmishes that broke out, a member
  of the far right group Patriot Prayer was shot and killed. Meanwhile,
  Trump eggs on this division with all the subtlety of a military
  parade. He tweets praise for his supporters, whom he calls patriots,
  and links the Black Lives Matter protests to violent anarchy and the
  cities' failure to quell that unrest to the Democratic Party as a
  whole. So let's start with a question we've asked before on this show.
  And that is, can we blame Trump for this escalation of violence
  between Americans?
\item
  michelle goldberg\\
  I don't think there's any question that Trump has incited white
  vigilantism, white supremacy. He has been egging on kind of violent
  thuggery from his supporters since his campaign in 2015 and 2016. He
  has repeatedly lauded people who either threaten or take violent
  action against those he considers his enemies. He just recently
  praised Kyle Rittenhouse, who has been charged with murder after
  shooting two protesters in Kenosha. And something I would point out is
  that this is an argument that I have been making for years and years
  now. It's an argument that people like the Southern Poverty Law
  Center, the Anti-Defamation League has made. In my column this week, I
  talked to Elizabeth Neumann, who was part of Trump's D.H.S. until
  April, where she was a high-level counterterrorism official. And she
  also makes this argument. She says that a huge problem that people
  working on counterterrorism in the United States have to deal with is
  the growing emboldening of white supremacy and violent white
  nationalism, and that Trump has both encouraged them, made them feel
  like they have his tacit encouragement, and handicapped government
  efforts to combat it. So there's just no question at this point that
  Trump has emboldened the violent right. And then the other piece of
  that is that you're seeing some of the protests against police
  brutality also give way to violence. I mean, obviously, the proximate
  cause of that is police brutality. But I also think part of it is that
  you're starting to see left-wing people show up to these protests with
  guns, which I think is a direct response to the fact that so many
  right-wing people are showing up to these protests with guns.
\item
  frank bruni\\
  Ross, what's your reaction to that?
\item
  ross douthat\\
  I mean, I've thought all along that Trump bears responsibility for the
  general mood in American politics. I think that under a different
  president, you would have less violence in the streets. I think
  Trump's habit of sort of winking at white nationalist supporters in
  various ways probably does have some emboldening effect. I'm not sure
  it's really the only place I'd lay the stress, though, right now,
  since we're living through a wave of riots that's probably the most
  serious in this country since LA, certainly in the early 1990s. And as
  Michelle says, the proximate cause of the riots is responses to police
  brutality. But I think it's pretty clear that a mostly white, sort of
  anarchist fringe has fastened onto the protests and used them as a
  pretext for maintaining a simmering level of aggressive property crime
  and destruction in a few American cities over the last few months. And
  it's a problem independent of Trump that those cities --- Portland,
  Oregon being the main exampl e--- haven't figured out a way to get the
  violent side of those protests under control. And I think that it's
  totally reasonable to have a conversation about this that isn't just
  about Trump and is about what is the response if you're the mayor of a
  liberal city right now, and your business districts are getting
  burned? What is your response if you're the mayor of Kenosha,
  Wisconsin, and your business districts are getting burned? What's the
  response if immigrant-owned small businesses are getting torched? And
  that can't just be a conversation about how Trump is bad. I do think
  Trump is bad. There's some other stuff going on here that's mostly
  action happening on the left, I would say.
\item
  michelle goldberg\\
  Can I just interject? Because I actually--- I mean, I agree with Ross
  to a certain extent that these questions about how Democratic mayors
  should be handling violent unrest is not just a rehash of the way that
  Trump has unleashed evil spirits in the United States. But I don't
  think you can totally separate all of this and say we're just living
  through a wave of urban riots and not violent white supremacy, or not
  right-wing incitement for a second civil war, when if you look at the
  fact that even some of the most shocking acts of violence to come out
  of the wave of urban unrest that's followed George Floyd's killing has
  been by the far right. It's really, really telling that when Mike
  Pence needed an example of a really shocking killing that happened at
  one of these riots, he spoke about the murder of a federal officer
  named Dave Patrick Underwood in Oakland. And he made it seem as if
  this was an outgrowth of the Black Lives Matter protests. He didn't
  mention that the person charged in this killing was part of the
  Boogaloo boys, which is this bizarre, extremely online sort of militia
  movement that wants to spark a second civil war. So I agree that
  there's a conversation that's kind of separate from the white
  nationalist conversation. But I don't think you can take right-wing
  incitement, provocateurs out of the violence that's happening all over
  this country right now.
\item
  frank bruni\\
  Ross, in terms of how Trump does or doesn't fit into all of this, let
  me ask the question in a different way. If we had these police
  shootings under a different president, do you think we would see the
  aftermath playing out the way it is now?
\item
  ross douthat\\
  Not exactly. I mean, I think the reason the aftermath has played out
  the way it has is the pandemic, and the lockdowns, and the suspension
  of ordinary American life. I think that if you'd had the shootings and
  the protests in a situation where more people were at work, and at
  school, and cities were more normalized, you would have had some
  protests, maybe some days and nights of riots in a few places. But you
  wouldn't have had both this sort of national movement aspect of it and
  the kind of sustained protest-cum-violent stuff that you've had in a
  few places. And obviously, Trump plays a role in how America responded
  to the pandemic and so on. But I think the weird suspension of normal
  life has played a bigger role in creating the overall culture here
  than necessarily Trump himself. What also might have played out
  differently is you might have different responses from blue state
  governors and liberal mayors to the situation because they wouldn't be
  afraid of being seen as siding with Trump. I think there's a certain
  kind of political pressure that having Trump as president brings to
  bear on liberals in government confronting rallies and protests, where
  doing anything that seems to feed into Trump's narrative is something
  you can't do.
\item
  frank bruni\\
  So you think they're being less, for lack of a better adjective, stern
  with what's going on in their cities because they want to distinguish
  themselves from Trump and sort of defy him?
\item
  ross douthat\\
  Yeah, I think there's probably more hesitancy about whether it's
  calling in the National Guard, which obviously has been done in a lot
  of places, or instituting curfews, or just being seen as condemning
  the violent fringe of the protest too harshly. I mean, we had that
  whole phase of the CHAZ Zone, the sort of separatist enclave in
  Seattle, where Trump was raging about it. So the mayor of Seattle,
  because Trump is ranting, feels compelled to go on Twitter and talk
  about how peaceful this all is and how it's like we're just hanging
  out and having another summer of love here in Seattle. And then
  eventually, lo and behold, some people get killed, and the CHAZ Zone
  has to shut down. And then in the aftermath, you get reporting,
  including in our newspaper, that says, actually, this was kind of a
  hellscape if you owned a business in this area, right? And I think
  that that kind of thing would have played out very differently under
  both a Democratic president and a non-ranting Republican president. So
  that's, I think, a small case study. But I think you have similar case
  studies around the country. It would be easier for liberal mayors and
  governors to take both tougher measures and use tougher rhetoric if
  they weren't afraid of being seen to give aid and comfort to Trump's
  anti-B.L.M., anti-Antifa narrative.
\item
  frank bruni\\
  I should make sure our listeners are aware--- we're recording this on
  Tuesday morning. And I think that's important because, later today,
  after we finish recording, Donald Trump is going to be in Kenosha,
  Wisconsin. He was asked, I think, by the governor and by others not to
  come. But that is not something that is ever going to deter our
  president. Michelle, at this point in time, with the election two
  months away, does Trump want this violence?
\item
  michelle goldberg\\
  Absolutely he wants it. There was a quote from Kellyanne Conway last
  week that Biden has repeated a bunch of times for obvious reasons
  because it was extremely telling and revealing. She said on Fox, ``The
  more chaos, and anarchy, and vandalism, and violence reigns, the
  better it is for the very clear choice on who's best on public safety,
  and law and order.'' You don't invite this ridiculous, gun-waving, St.
  Louis couple to speak at the R.N.C. and say that protesters are going
  to invade the suburbs if you don't think that this wave of violence is
  good for you. And I think that that is still a very open question,
  right? There has been a few hints that some of this might be working
  in Trump's favor. But after the freak out that showed the polls
  tightening after the Republican convention. They're now back about to
  where they were. There is a kind of a chicken-and-egg question about
  whether this works, right? Because I mean, it has been sort of insane
  to watch this 17-year-old kid, a Blue Lives Matter obsessive, someone
  who, according to BuzzFeed, was at a Trump rally in January, you know,
  takes his gun, drives to protest, kills two people, is charged with
  murder. And in the next couple of days, the media narrative is, is
  Kenosha terrible for Biden? How are the Democrats going to deal with
  this, right? And it's sort of the soft bigotry of no expectations for
  Republicans, right? Nobody even expects Trump to gesture towards the
  middle. Nobody expects Trump to try to calm things down, whereas
  people do have these expectations of the Democrats. And so, although I
  think it's unclear whether kind of sparking a second civil war in the
  United States is going to be a net positive for the incumbent
  president, I think that Trump thinks it is. And I think that's what
  he's trying to do.
\item
  frank bruni\\
  Do you agree with that, Ross? Or what do you think the political
  fallout of this is going to be?
\item
  ross douthat\\
  I mean, I think there is a sort of soft bigotry of no expectations
  around Trump in that we know what we get from Trump. We know that he
  is not capable of making certain politically obvious moves that a
  normal president would make in these circumstances. And of course, it
  would benefit Trump politically if he were more likely to condemn
  people to his right. We just are aware at this point that that's not
  something he's going to do. And so the focus is on the more
  unpredictable question of how does Joe Biden handle any of this.
  Kenosha became a subject of media debate, one, because it was a really
  severe wave of vandalism and arson that happened in a swing state, in
  a region that Trump has to win and that he won, to everyone's
  surprise, last time, right? So it's not just the riots. It's the
  location of the riots and so on. It happened at a moment when you
  could see public opinions seeming to turn a bit in general surveys
  about protests and Black Lives Matter. And there was this initial
  surge of public support for the protests. And as time has gone on, and
  the Portland stuff has continued, and so on, that support has
  diminished somewhat. So there was sort of an intersection of another
  night of severe riots with that turn in public opinion. And then
  Rittenhouse himself is, as far as we can tell, not a white
  supremacist, not a Boogaloo boy, just an idiot, basically, who, at 17,
  decided to go and protect car dealerships from rioters and ended up
  shooting people in a context that was not him wading into a crowd
  firing. It was him basically getting his ass kicked, as far as anyone
  could tell.
\item
  frank bruni\\
  OK, but nonetheless, what do you make, Ross, about the fact that the
  right has turned Kyle Rittenhouse into a hero? Tucker Carlson has been
  raving positively about him on air. Ann Coulter said she wants Kyle
  Rittenhouse as her president, I guess thereby admitting that Donald
  Trump isn't doing such a great job. Looking for the silver lining in
  that comment.
\item
  ross douthat\\
  Well, Ann Coulter has been arguing that Trump is doing a terrible job
  for a long time.
\item
  frank bruni\\
  She wants her wall. She wants her wall. She wants her wall.
\item
  michelle goldberg\\
  Right.
\item
  ross douthat\\
  And if you watch Tucker Carlson closely, the fact that he thinks Trump
  is a failure is a pretty strong subtext of his show night to night.
\item
  frank bruni\\
  OK, but let's go back to Kyle. What do you make, Ross, about the fact
  that elements of the right, and Ann Coulter ---
\item
  michelle goldberg\\
  And Trump.
\item
  frank bruni\\
  Tucker Carlson are there --- yeah, and Trump himself, are lionizing
  this 17-year-old, who went with a gun into an inflammatory situation?
\item
  ross douthat\\
  Because the right thinks that the people in charge of American cities
  have essentially retreated and abdicated their responsibility to
  protect peoples, businesses from rioters. And so people are, as
  happened with Bernie Goetz, famously, in the 1980s, people are you
  know are celebrating this kid who, again, as far as I can tell, was an
  idiot who is culpable for people's deaths, whether he's guilty of
  murder or not. But they're celebrating him for going and trying to
  effectively defend the things that the police weren't defending.
\item
  frank bruni\\
  But that celebration is sick, no?
\item
  michelle goldberg\\
  Well, it's dangerous, right? Because it's going --- it sends a message
  to other dumb ---
\item
  ross douthat\\
  I mean, I don't agree. Again, I think what Kyle Rittenhouse did was
  moronic at best, and people are dead because he, at best, was a moron.
  And I'm not celebrating it. I think that when you have an abdication
  of sort of civil authority to protect people's property and sometimes
  lives, then you're going to get a mood that's sympathetic to
  vigilantism, yeah. I don't think that's remotely surprising.
\item
  frank bruni\\
  No, no, well, that's not--- granted, Ross, that's not a celebration.
  That sounds a little bit to me like it's at least in the zip code of a
  justification.
\item
  ross douthat\\
  I mean, I think that if Kyle Rittenhouse had not been a 17-year-old
  teenager from another state, but had been a small business owner, who
  was armed when people came into their business and started setting it
  on fire, then it would have been justified, yeah.
\item
  michelle goldberg\\
  Right. But that's if it had been a totally different situation, right?
\item
  frank bruni\\
  {[}LAUGHING{]} Exactly. Exactly. Thank you, Michelle.
\item
  ross douthat\\
  Well, yeah, yes. But I'm saying that there is a reason that, in
  situations where people are going around burning businesses, that
  people get sympathetic to vigilantism. And it's dumb to be sympathetic
  to the kind of vigilantism that Kyle Rittenhouse embodied. But just
  saying we can't --- it's sick to celebrate vigilantism in a climate of
  people's businesses being torched, then --- I mean, people's
  businesses getting torched is sick, right? Some of the people doing
  things in these riots, not just the people shooting cops and things
  like that, but people torching somebody's livelihood, that's sick too,
  right?
\item
  michelle goldberg\\
  Well, I don't know. I mean, to me, it sounds like I could make the
  same argument, but I won't, that, say, in a climate where police are
  shooting people and the government is not protecting them, then it
  makes sense that people will riot as --- what is the Martin Luther
  King line? A riot is the voice of the unheard. I could construct a
  similar justification for rioting and looting, but I won't do it. And
  it sounds like, to me, that's what you're doing, basically, kind of
  desperate times call for desperate measures.
\item
  ross douthat\\
  Why don't we achieve consensus here and say that I will say that
  crossing state lines with a gun to try and interject yourself into a
  gravely chaotic situation, whatever mitigating circumstances there
  are, that is both a stupid and a wicked thing to do. It's also wicked
  to burn somebody's business, right?
\item
  frank bruni\\
  Oh, yes, of course. Of course.
\item
  michelle goldberg\\
  I think it's less wicked to burn someone's business than to kill
  someone.
\item
  ross douthat\\
  OK, is it less wicked to burn someone's business than to take your gun
  and try and stand guard around someone's business? What Kyle
  Rittenhouse intended to do, which was stand guard around someone's car
  dealership that might have been torched, is that more wicked than
  torching a car dealership? Less wicked? Wash? What do you think?
\item
  michelle goldberg\\
  I think it's more dangerous.
\item
  frank bruni\\
  But beyond these gradations, which, to me, are almost getting a little
  silly, is when you enter a situation like that, when you walk into a
  room that is strewn with fuel with a match, bad things are going to
  happen. I mean, this is a volatile situation that your presence is
  going to do nothing but render more volatile. But the reason I dwelled
  on it, Ross, is ---
\item
  ross douthat\\
  No, but look. But Frank, the conservative --- the thing that you're
  reacting to so strongly from conservatives here is rooted in the fact
  that you're describing the burning building in these sort of abstract
  terms like it's a situation, or there's things on fire. But in fact,
  people are setting those fires. People are torching those businesses.
  People are performing actions that are wrong.
\item
  michelle goldberg\\
  No, and it is --- I mean, it's striking to me to listen --- Ross, the
  degree of emotion that I can tell you feel about this, where, to me,
  it is kind of bad. I feel immensely sorry for some of these small
  business owners who are not implicated in police violence. But it
  doesn't bring up the same emotion in me as seeing police shoot unarmed
  civilians or police ---
\item
  ross douthat\\
  OK, but the police shooting unarmed civilians is obviously worse than
  burning a building. But we're having an argument about the idiot kid
  who tries to protect the car dealership, right?
\item
  frank bruni\\
  No, I'm actually trying not to have an argument --- I'm trying not to
  have an argument about the kid. I'm trying to have an argument, or not
  even an argument, about the disparate reactions to what's going on. So
  I started--- we went down this Kyle Rittenhouse rabbit hole when I was
  asking why people on the right were celebrating him, were lionizing,
  were going well beyond justifying him, including the president, right?
  And earlier, you made a reference, Ross, to will Joe Biden condemn
  certain actions on the left? Well, he has now in Pittsburgh yesterday.
  We're recording on Tuesday. In Pittsburgh on Monday, he said, these
  protests that get out of hand, the sooner --- we have --- what I'm
  focused on here is the real imbalance between what Joe Biden has said
  and will say and I think will continue to say and what Donald Trump
  won't say and isn't saying. And it's not just Donald Trump. This has
  spread throughout the Republican Party. There was a fascinating
  interview on Sunday that Dana Bash of CNN did with Senator Ron Johnson
  from Wisconsin, where she asked him if he would condemn the violence.
  And he kept dancing around the words because he was clearly so worried
  that if he seemed to be condemning Kyle Rittenhouse and people who'd
  run to those burning stores that you're talking about, Ross, that he
  would somehow alienate his political constituency. I just think that's
  weird, dangerous, and unhealthy.
\item
  ross douthat\\
  My strong impression is that we have passed through a months-long
  period in which Democratic politicians and much of the mainstream
  press have wanted to insist that riots have not been as bad as riots
  have actually been because they are supportive of the cause of the
  protests that the riots have been attached to. That's my impression of
  the story of the last few months.
\item
  frank bruni\\
  I will grant you, Ross, the riots have been horrible. And I think
  sometimes, there's a big element of truth to what you're saying. And
  part of the challenge right now for Joe Biden is, I think, a lot of
  people feel that they're not hearing a stern enough condemnation, that
  people are not admitting fully that people gave protesters too much of
  a pass in terms of gathering amid a pandemic and all of that.
\item
  ross douthat\\
  Well, I don't think that's. That's a separate issue.
\item
  michelle goldberg\\
  I would like some empirical understanding of how bad the riots have
  been versus how bad they've been presented in the mainstream media.
  Because my impression, being back here in New York City, is that
  there's this presentation of New York and other urban areas as this
  sort of crime-ridden hellscape, when, in reality, being back in New
  York, I wouldn't really know any of this was going on if I didn't see
  it on television or on the internet. It's much more contained, at
  least here, than you would understand from reading mainstream
  newspapers, from watching cable news, from looking at it on the
  internet. I feel like I hear similar things from people in Portland,
  who will say, you know, there's this idea that the city is on fire.
  But if you're out of a few-block radius, you're barely even aware of
  it. And so it's just not clear to me that the scope of the riots has
  been underplayed.
\item
  ross douthat\\
  I think the scope of the riots have now been overplayed on the right.
  But I think that in our world of media, I think there was a period of
  a couple months when the scale of the damage in places like
  Minneapolis, especially, was underplayed. And also, there was a spasm
  of looting in Chicago a few weeks ago. And it was bad enough that they
  raised the bridges around Chicago to try and keep people from entering
  or leaving downtown in order to contain it. And it got only coverage,
  as far as I can tell, from local media and right-wing media. And you
  know, I cited the example of the CHAZ stuff in Seattle. I think what
  was --- everything that was wrong there got a lot more coverage after
  the fact, after it had been shut down. I guess all I'm saying is that
  I think --- not to be all both sides here. But I think both sides are
  engaged in a kind of dangerous denial of parts of reality.
\item
  michelle goldberg\\
  So can I say something to that? I mean, I do think that left-wing
  social media has had a really deleterious effect in that there is such
  stigma in seeming to call out, quote-unquote, ``bad protesters'' or
  make distinctions between good and bad protesters, or say commonsense
  things about the political impact of unrest. If you look at the firing
  of David Shor, which is something that maybe took up more oxygen in
  some of these debates than you would think the firing of a data
  analyst would usually receive. And it was because this data analyst
  for a progressive consulting firm, who was fired from tweeting out a
  study by a Black Princeton professor about how riots in 1968 helped
  boost Richard Nixon's vote share. So I think this bleeds over a little
  bit. And Democratic politicians, to the extent that they are too
  online, might be hampered in making what seem like commonsense points
  to a lot of their constituents. But at the same time, I don't buy the
  idea that this stuff has not been widely covered. There's also an
  element of just journalists --- there are --- some of our very brave
  colleagues are traveling and are reporting and kind of risking their
  health to do it. But I think fewer people are doing it than would have
  done in the past because of the pandemic. So there's a bit of a fog of
  war element to try and figure out what is going on in CHAZ. But the
  reason that we know how bad CHAZ got is because of a big story in
  ``The New York Times.''
\item
  ross douthat\\
  I think there's truth to that. I also, though, think that there has
  been a lot --- precisely because Donald Trump keeps talking about
  Antifa, there has been a lot of pressure to say like, oh Antifa, that
  just means people who are against fascism, or Antifa, that doesn't
  really exist, or it only exists in Tom Cotton's fervent imagination.
\item
  michelle goldberg\\
  But it doesn't exist in the way that they talk about it as like ---
  you hear Laura Logan talking about how they're getting marching
  orders. And there's this rumor that Donald Trump spread of black-clad
  people on an airplane. I mean, I know these people. I've reported on
  these people enough to know that, yeah, you have kind of black bloc
  idiots in every single metropolitan area in the country who are always
  trying to hijack protests and break shit. And that's something that
  precedes this current moment. There's overlap but also a distinction
  between the black bloc and Antifa, if we really want to go down that
  road. They're not entirely the same thing. But there has been this
  attempt to basically turn kind of black bloc idiots into a
  well-organized terrorist threat. That's ridiculous.
\item
  ross douthat\\
  I think it's wrong to see them as a well-organized terrorist threat. I
  think, as someone who's spent much of the last three years in
  arguments with people on my right, saying, look, you Antifa, this is
  not really a real thing. This is just a bunch of idiots cosplaying. I
  think we've established that those people are capable of doing
  substantial amounts of property damage and creating extremely
  dangerous environments in American cities, when given the opportunity
  afforded by a pandemic, Donald Trump, and mass protests. I would say
  that I underestimated the capacities of those idiots to sustain arson,
  damage, looting, and violence over a multi-month period. I think
  they've done a fairly impressive job in a lot of American cities of
  doing that. And that doesn't make them the next ISIS. It doesn't make
  them an existential threat to the US. It doesn't make Donald Trump
  right in his fantasies about black-clad Antifa ninjas on a plane. But
  it's still kind of a big story, right?
\item
  michelle goldberg\\
  I don't know. I think it's really unclear. Look, there's clearly these
  kids who've been dreaming of--- I don't know --- some sort of
  overthrow of the system for their whole lives and think that this is
  their big shot. And you kind of burn down enough coffee shops, and the
  next thing you know, you have the end of capitalism. {[}LAUGHTER{]}
  But it's very unclear---
\item
  ross douthat\\
  Which is true. I mean, we should concede that that's true. There is a
  moment at which capitalism will end.
\item
  michelle goldberg\\
  But it's very unclear to me how much of this is Antifa, how much of it
  is just opportunists that see looting going on and want in on it, how
  much of it is other sorts of protesters. And then there is --- I think
  that people on the left can overplay the role of right-wing
  provocateurs, but they're also part of the mix. If you want to talk
  about what happened in Minneapolis, there was that black-clad guy that
  really kicked off a lot of the arson and property damage. And you
  obviously can't blame him for all of it. But I think people can look
  back and say that's when it started. And we do now know, or at least
  he's been arrested and charged as being a white supremacist
  provocateur. So I think that there is still a fair degree of confusion
  about the precise makeup and the precise breakdown of responsibility
  for what's happening right now. {[}MUSIC PLAYING{]}
\item
  frank bruni\\
  Let's take a quick break, and we'll be right back.

  {[}MUSIC PLAYING{]} And we're back. So the election is exactly two
  months away. And this issue, clearly, is going to linger throughout
  it. Michelle, you said earlier you think Trump wants this violence. Do
  you think, at the end of the day, it is going to help him on November
  3, or I should say, beforehand as well, since a lot of the voting this
  year is going to be mail-in?
\item
  michelle goldberg\\
  I mean, I should say that because I am so terrified of what's going to
  happen, I fear in my darkest moments that it will. I don't see how you
  can experience the 2016 election and not think that white backlash is
  an incredibly powerful force in American politics. But I think that
  might be more my emotions talking than anything empirical. There was
  one poll that showed some tightening. There's been some anecdotal
  reporting, including in our newspaper. But overall, the polls are
  really, really stable. In the 1968 analogy, Donald Trump is not
  Richard Nixon. He's L.B.J., right, in that it's kind of hard to make
  the case that you need to re-elect me president to stop the violence
  and chaos that has happened while I'm president.
\item
  frank bruni\\
  Ross, what do you think the political fallout of this is going to be?
  How do you see this coloring the presidential race?
\item
  ross douthat\\
  I mean, so I wrote a column weeks ago now, where I tried to imagine
  how Trump could possibly come back. This was when Biden was up by an
  average of 10, I think, in a lot of polls. And the scenario I spun out
  was basically that Trump needed COVID infections to drop dramatically,
  that he needed some of the early herd immunity theories to be true.
  And he probably needed the violent fringe of the protests to become
  much more salient going into the election. A very mild version of that
  is happening. COVID infection rates have fallen, not as far as we
  would like them, but they have fallen. And the violent fringe of the
  protests has gotten somewhat more salient. I think it's understandable
  that liberals would be worried. And I think it's reasonable to worry
  in the sense that--- the incumbency point that Michelle makes is, I
  think, partially right. But there is another reality, which actually,
  David Shor, the data analyst Michelle mentioned in the first segment,
  who was fired, famously, for some of his tweeting about the politics
  of riots, he makes this argument about issues salience, that voters,
  for better or worse, trust one party more than the other on a
  particular issue. And so if you raise the salience of that issue,
  who's got the policies that technically poll best might be less
  important than just the fact that the issue is salient. So if you
  raise the issue salience of health care, the policy details may not
  matter as much as the fact that voters tend to trust Democrats more on
  health care. If that issue becomes more salient, it's good for the
  Democrats. And so if voters trust Republicans more on fighting crime
  and controlling urban riots, then even if Donald Trump doesn't have a
  10-point plan to stop the urban riots, the more you raise the salience
  of that issue, the more potential advantages Republicans have. So
  that, I think, is one way of arguing that what's happened in the last
  few weeks around these issues should make Democrats worried. That
  being said, Trump is not--- I mean, we were saying this in the first
  segment. But Trump is running against Joe Biden. Joe Biden is a
  famously moderate Democrat with a tough-on-crime past who's, to put it
  charitably, a little bit past his prime as a political communicator,
  but still is perfectly capable of giving a speech where he says, riots
  are bad, right? And Trump is not really capable of sustaining a
  Richard-Nixon-type, claiming-the-center strategy. And to the extent
  that that's the basic political reality of the race, I think things
  would need to get a lot better with the coronavirus and a lot worse
  with the riots before full Democratic panic would be justified.
\item
  frank bruni\\
  So you said Biden is perfectly capable of giving a speech decrying the
  violence. He gave that speech on Monday in Pittsburgh. Michelle, did
  you think he accomplished what he needed to? Do you think that speech
  was on the mark?
\item
  michelle goldberg\\
  Yeah, I mean, it's a difficult thing for me to evaluate because
  there's sort of like what I want Biden to do because I think it will
  help him win versus what I believe. So what I believe is that the real
  problem here is police violence. And that is what you have to tackle
  first. Do I believe that it is good for Biden to decry these riots and
  to show wavering suburbanites that he cares about urban unrest and
  that he's going to keep them safe, even though I believe that they are
  already 100 percent safe from this phenomenon? Yes, I do. So to me,
  the speech was great. And I thought it was great also because it
  opened up the obvious question. Yes, I condemn violence. I condemn
  left-wing extremism. Your turn. Why won't you do it? If we just had a
  whole news cycle about is this bad for Biden, why won't Biden condemn
  the riots, will he condemn them strongly enough --- it's frustrating
  that it seems unlikely that we're not going to have a similar news
  cycle about Trump, again, because nobody expects anything of him.
\item
  ross douthat\\
  Except that the news cycle is always about how Trump is bad. Why won't
  Biden do this thing that he needs to do to make sure this bad man is
  beaten? I mean, that's the dynamic, at least of CNN. It's not the
  dynamic everywhere. But it's a pretty common media dynamic. And Trump
  does get--- after Charlottesville that Trump had some of his worst
  polling numbers. When Trump is seen as not condemning right-wing
  extremists, white supremacists, and so on, it shows up in the polls.
  And the Trump campaign --- other people have made this point, maybe
  Josh Barro or somebody. But they keep setting up really easy tests for
  Joe Biden to pass. They're like, oh, Joe Biden, he won't leave his
  basement. He can't speak. He just mumbles. And then he gives a normal,
  pretty good convention speech. And it's like, OK, pass that test. And
  then they're like, well, Joe Biden won't condemn the riots. And so Joe
  Biden gives a speech condemning the riots. And now, the bar will be
  Joe Biden won't specifically condemn Antifa. And as you can tell from
  the first segment, I think Joe Biden should specifically condemn
  Antifa. Think he could go further in his specific attacks to the left.
  But I think if that seems politically necessary, he obviously will do
  that.
\item
  frank bruni\\
  He did something else really smart, I think, in his remarks on Monday.
  And I think he said it several times. It was simple. But he certainly
  said it once loud and clear. And like I said, I think multiple times,
  where he kind of said to voters, you know me. You know me. And that
  was bigger than just about the violence and whether he condemns the
  violence. Obviously, whether they're trying to pin him into a corner
  by being not sufficiently condemning of the violence, whether they're
  trying to do it in other ways, they're saying that--- Trump and his
  surrogates and his enablers, they're all saying that Joe Biden is a
  hostage of the left or will be a hostage of the left. And I just think
  it's very effective when he speaks plainly and simply and trades on
  what is his greatest strength, which is that he has been around a long
  time, and he is not scary to Middle America. And I just really,
  really, I thought, was smart, and it really kind of rang out when he
  said, I believe multiple times, in his remarks to voters, you know me.
\item
  michelle goldberg\\
  Well, and he said, do I look like a radical socialist to you? Like,
  come on.
\item
  ross douthat\\
  Yeah, I think that's an effective line. I mean, I don't want to --- I
  think there was a little bit of over-praising the Biden speech maybe
  from journalists who were panicked by a few of those post-convention
  polls. And Biden has lost a step. That's just not really disputable.
  He is not as vigorous a politician as he was eight or 10 years ago.
  And that's a weakness. And he sort of wanders verbally more than he
  used to. He's not a dynamic presidential candidate at all. But that's
  probably OK for the kind of race he's trying to win.
\item
  frank bruni\\
  It may actually be to his advantage, given how exhausting Donald Trump
  is to most Americans.
\item
  michelle goldberg\\
  I don't know. I worry about it a lot. The Democrats haven't won with a
  candidate over 55 in a really long time. You know, LBJ seems like an
  ancient person, but he was, I think, 55 when he became president. We
  don't have a great record of, in either party, electing kind of old
  Senate warhorses. And I don't know. It definitely scares me. I mean,
  it's why I didn't want him to be the nominee, even though I now think
  he has some strengths.
\item
  frank bruni\\
  But wait a second, Michelle. You wanted Elizabeth Warren, and she's 71
  now. So ---
\item
  ross douthat\\
  But she's a vigorous 71, Frank. I mean, she is. She's more vigorous.
\item
  frank bruni\\
  OK, I will admit she's a much more vigorous 71 than Joe is at 77. But
  I mean, she breaks that paradigm of Democrats winning with the
  youngest candidate. I'm just---
\item
  michelle goldberg\\
  Yeah, but, you know, 71 in woman years.
\item
  frank bruni\\
  Fair enough. Fair enough.
\item
  michelle goldberg\\
  You know. {[}LAUGHS{]}
\item
  ross douthat\\
  I mean, but I think that is what Michelle is saying is that the danger
  for Biden is that he doesn't turn out young voters at quite the rates
  that a more --- either a more dynamic or more left-wing candidate
  would have, and that his age and wobbliness becomes a problem in the
  event that, if there were sustained rioting and the politics of law
  and order just stay incredibly salient in states like Wisconsin and
  Minnesota, then the Democratic candidate needs to --- he needs to
  project some toughness, some sense that if it's necessary to sort of
  bang heads --- again, not of criminals, but of mayors, of going after
  his own governors or his own mayors and so on, that he can do that.
  And a little more 1980s Biden would probably be better for this
  campaign.
\item
  frank bruni\\
  Is one of the dangers for Biden us? And by that, I mean the media. And
  what I'm talking about is I kind of notice in this most recent thing
  with urban --- with the violence and the protests as a good example.
  You know, Donald Trump spotlights an issue, takes some position on it,
  says something provocative. And the media immediately begins writing,
  how is Joe Biden going to respond? And I see a lot of people making
  the correct comment that Biden isn't controlling or seizing the
  narrative. But I'm not sure we allow him to. I feel like, for all of
  our hand-wringing and soul searching and self-examination about the
  way we covered Trump, I still think we let him define the event of a
  given 24, 72-hour news cycle. And then it becomes how will Joe Biden
  respond. And I'm not sure how Biden breaks out of that if, in fact,
  it's a pattern that we've established. Michelle, am I seeing--- are
  you seeing this too? Or---
\item
  michelle goldberg\\
  I think that's right. I mean, and again, I think there's a bit of a
  chicken and an egg thing, like I said before, where it's hard to say,
  is the media responding to a genuine shift in the electorate or the
  polling? Or is the media creating a narrative because it's sort of
  that time in the race when you need a new story besides Biden is
  inevitable and Trump is self destructing. But I definitely do think
  that --- again, the media, I think it both sort of follows Trump's
  lead on defining the issues and then lets him off the hook so that
  there's not a --- you don't see a lot of reporting. And I do think
  that Ross is right in that, because I think a lot of journalists do
  fear for the republic, so they're not sort of worried about what Trump
  is doing wrong. There's not a sense of what does Trump have to do to
  right the ship. That piece Jamelle wrote is the kind of piece that you
  see about Biden all the time and you basically never see about Trump.
  You never see people sort of suggesting that Trump might be losing the
  center. You never see people dwelling on Trump's refusal to disavow
  the far right. And so I do think that there is a little bit of chronic
  both-sidesing that I think was a big, big problem for Hillary Clinton.
  I think that the media vastly inflated the salience of email server
  management as an issue. And I think voters took a cue from that, that
  this is a really big deal. And I think that they could be doing the
  same thing over again.
\item
  ross douthat\\
  Yeah, well, but it's different, right? Biden is not --- there's a lot
  of coverage of, is Biden doing enough to win? Is Biden going to blow
  it and so on? There's very little critical scrutiny of Biden's actual
  record. I mean, we're not getting a range of stories that are --- the
  stories that liberals feared we'd get, that are like, well, what about
  the Biden family's overseas business dealings? I mean, there was a
  bunch of stories about Biden not being left enough during the primary
  campaign. But there's --- I don't know. I mean, I feel like there's
  very little, let's say, of the media echoing the Trump campaign's
  narrative about Biden, right? The Trump campaign's narrative about
  Biden is that he is an embodiment of a failed American establishment
  that outsourced American jobs overseas and led us into disastrous wars
  in the Middle East. And that's kind of true. {[}LAUGHS{]} But nobody
  in the media is echoing that line. Nobody in the media is
  re-excavating Biden's Iraq war vote or things like that in the way
  that the media did, I think, play into Trump's ``crooked Hillary''
  narrative in 2016. I think all of the Biden stuff is about --- right
  now, at least, we've still got a little ways to go --- is just about
  is he doing enough. And they aren't trying to like grab the narrative
  with huge policy speeches and big ideas. I mean, I don't know if you
  can completely blame the media for what is, in fact, choices that the
  Biden campaign is making, which is to run a --- it's a very cautious,
  front-porch-style campaign. I mean, Biden's convention speech was--- I
  mean, it was actually really distinctive. It was really short. It had
  almost no policy in it. It was just a personal appeal. Like I'm Joe
  Biden. You know who I am. I'm a good guy. Donald Trump's terrible.
  He's messed up the coronavirus. Vote for me. That's the Biden
  strategy. And it is vulnerable, but it's vulnerable to external events
  like the coronavirus diminishing and riots getting worse.
\item
  frank bruni\\
  No, but that's an excellent point. It's impossible to seize the
  narrative if, in fact, you're protecting a lead. And that has been the
  Biden strategy. It factors into why Kamala Harris was the
  vice-presidential choice. I think there has been an assumption that
  he's ahead, that if you just kind of extrapolate forward far enough,
  he ends up ahead, he ends up winning. And let's not do anything wrong.
  Let's not take any big risks. Let's protect our lead. I think that's a
  big part of the dynamic. And I think that's what you're referring to
  Ross, right?
\item
  ross douthat\\
  Yeah. And he does --- well, and the other little thing is that,
  because there's an assumption that Trump can win the electoral college
  even if he loses the popular vote by 2 or 3 points, there's also a
  weird uncertainty in the media narrative about how much is Biden
  really winning by. If he is up by 8 points, that's huge by modern
  presidential campaign standards. On the other hand, it's only a
  5-point lead from an electoral college point of view. So there's a
  certain kind of whip sawing and uncertainty in the media narrative
  there, too, I think.
\item
  frank bruni\\
  And the media is, once again, despite its promises every four years
  not to do so, covering the horse race more than anything else. Here we
  are talking about polling margins and all of that. I'm raising my hand
  and saying, guilty as charged, you know?
\item
  michelle goldberg\\
  No, but I don't think there's anything to be guilty of. I mean, we're
  talking about is there going to be a recognizable American republic
  three months from now. The question of kind of who's ahead and who's
  behind in that existential contest is a big one and it's actually way
  more important than anything in Biden's policy platform.
\item
  ross douthat\\
  Without necessarily agreeing with that argument, {[}BRUNI LAUGHS{]} I
  will say that it's also true that Trump is not rolling out policy
  ideas regularly either. I mean, there is nothing but --- with Biden
  running a cautious, low-policy campaign, and Trump running a
  scattershot, demagogic, low-policy campaign --- I mean, in 2016, Trump
  had more policy. He had an actual sort of critique of the neoliberal
  establishment that was really important to his closing argument in
  2016. And that sort of resurfaced in the Republican convention. But
  it's not really there. Trump's not really running on policy. So what
  else does the media have to write about except the horse race?
\item
  frank bruni\\
  Well, this is a great place to end because it's not often that all
  three of us are in perfect alignment, and on the fact that this is not
  a policy election, I think we have total, total agreement, right?
\item
  michelle goldberg\\
  Yes, absolutely.
\item
  ross douthat\\
  And I guess agreement is also a good place to end your wonderful stint
  as part of our show here, Frank. And we will miss you terribly. But
  since this is your final ``Argument'' episode, we, of course, have to
  turn to you for the recommendation. So what will you give us as a
  parting gift as you say goodbye?
\item
  frank bruni\\
  Well, since it's my last recommendation ever, at least on the show ---
  I hope I have recommendations in my life. {[}LAUGHS{]}
\item
  ross douthat\\
  You're going to do your own podcast that's just recommendations.
\item
  frank bruni\\
  I have, over the last five years, become this strange and stubborn
  evangelist for the short story that is one of my favorite short
  stories ever called ``In the Cemetery Where Al Jolson is Buried.'' Has
  either of you ever read it?
\item
  michelle goldberg\\
  No.
\item
  ross douthat\\
  No.
\item
  frank bruni\\
  It's by the famed, accomplished, terrific short story writer, Amy
  Hempel. It's title does not prepare you for what it's about. It's
  basically about a woman watching a friend die. And yet it manages to
  be jovial to the point of jocular until the final paragraphs. And it
  is the story that just does this amazing thing, where you're kind of
  rolling along with it. You're feeling little pinpricks of grief and
  sorrow. And then in the last couple of sentences, it just tears your
  heart straight out of your chest. Now, you're thinking, why am I
  recommending this? I'm one of those people who, when I'm feeling a
  little blue, I like to take the feeling all the way down until it
  comes back over the top. I get a glass of wine. I listen to Billie
  Holiday. And I read ``In the Cemetery Where Al Jolson is Buried'' by
  Amy Hempel. If you Google it, you will find the entire story online.
  And you can read it without buying anything. When I read it that way,
  I did not own any Amy Hempel book. I felt like I owed her for the
  intense pleasure. And so my recommendation is read the story online.
  And if you're as taken with it as I am, reward Amy Hempel, and do the
  right thing by buying a copy of ``The Collected Stories of Amy
  Hempel,'' in which there are many, many other gems. But the peak, the
  Everest is ``In the Cemetery Where Al Jolson is Buried.''
\item
  michelle goldberg\\
  I'm also a wallower, so I'll do that.
\item
  frank bruni\\
  You'll love this.
\item
  ross douthat\\
  And Frank, I guess, bringing us all the way down to the bottom so we
  can come back up is a good metaphor or synecdoche or some literary
  form for how you're departing our show. {[}LAUGHTER{]} So---
\item
  frank bruni\\
  Or I was actually going to say for the Trump years. He's bringing all
  the way to the bottom so we can come back up the top. Yeah!
\item
  ross douthat\\
  Well, we'll save that for ---
\item
  michelle goldberg\\
  The worse the better.
\item
  ross douthat\\
  We'll save that for December. But Frank, thank you again for being
  with us over these totally insane months. And we will miss you
  terribly.
\item
  michelle goldberg\\
  Yes, thank you, Frank.
\item
  frank bruni\\
  Thank you guys. I'll miss you. And I'll miss the listeners for ``The
  Argument.'' I thank them for bearing with me.
\item
  ross douthat\\
  And may the road always rise to meet your feet and, you know, the rest
  of an Irish proverb. {[}LAUGHTER{]}
\item
  frank bruni\\
  I'm on the road, Ross. I'm on the road. {[}THEME MUSIC{]}
\item
  ross douthat\\
  Now, just one more thing before we go this week. We're going to start
  releasing this podcast on Fridays rather than on Thursdays in an
  effort to catch more of the week's news as the election heats up. So
  starting next week, we'll be talking to you and arguing with each
  other on Fridays instead. And with that, that's our show this week.
  Thank you for listening. The team includes Phoebe Lett, Vishakha
  Darbha, Kristin Lin, Isaac Jones, and Paula Szuchman. Special thanks
  to Kathy Tu. ``The Argument'' will be back in your feed next Friday.

  Wait, what are peaches and herbs?
\item
  frank bruni\\
  Oh, Ross.
\item
  michelle goldberg\\
  Oh, Ross!
\item
  frank bruni\\
  Peaches \& Herb, it's an R\&B duo that did the song ``Reunited,''
  which is a prom staple.
\item
  ross douthat\\
  Oh, man.
\item
  frank bruni\\
  You obviously never went --- you didn't go to your prom, did you?
\item
  ross douthat\\
  I didn't. We're really going to excavate some pretty some pretty dark
  territory here if we go down that road. {[}LAUGHTER{]}
\end{itemize}

\href{https://www.nytimes3xbfgragh.onion/column/the-argument}{\includegraphics{https://static01.graylady3jvrrxbe.onion/images/2018/10/03/opinion/the-argument-album-art/the-argument-album-art-square320-v3.png}The
Argument}Subscribe:

\begin{itemize}
\tightlist
\item
  \href{https://itunes.apple.com/us/podcast/id1438024613}{Apple
  Podcasts}
\item
  \href{https://www.google.com/podcasts?feed=aHR0cHM6Ly9yc3MuYXJ0MTkuY29tL3RoZS1hcmd1bWVudA\%3D\%3D}{Google
  Podcasts}
\end{itemize}

\hypertarget{is-american-carnage-campaign-gold-1}{%
\section{Is `American Carnage' Campaign
Gold?}\label{is-american-carnage-campaign-gold-1}}

\hypertarget{what-is-the-political-fallout-of-urban-violence-and-is-the-media-just-buying-into-it-1}{%
\subsection{What is the political fallout of urban violence? And is the
media just buying into
it?}\label{what-is-the-political-fallout-of-urban-violence-and-is-the-media-just-buying-into-it-1}}

With Frank Bruni, Ross Douthat and Michelle Goldberg

Transcript

transcript

Back to The Argument

bars

0:00/0:00

-0:00

transcript

\hypertarget{is-american-carnage-campaign-gold-2}{%
\subsection{Is `American Carnage' Campaign
Gold?}\label{is-american-carnage-campaign-gold-2}}

\hypertarget{with-frank-bruni-ross-douthat-and-michelle-goldberg-1}{%
\subsubsection{With Frank Bruni, Ross Douthat and Michelle
Goldberg}\label{with-frank-bruni-ross-douthat-and-michelle-goldberg-1}}

\hypertarget{what-is-the-political-fallout-of-urban-violence-and-is-the-media-just-buying-into-it-2}{%
\paragraph{What is the political fallout of urban violence? And is the
media just buying into
it?}\label{what-is-the-political-fallout-of-urban-violence-and-is-the-media-just-buying-into-it-2}}

Thursday, September 3rd, 2020

\begin{itemize}
\item
  michelle goldberg\\
  I'm Michelle Goldberg.
\item
  ross douthat\\
  I'm Ross Douthat.
\item
  frank bruni\\
  I'm Frank Bruni. And this is ``The Argument.'' {[}THEME MUSIC{]}
  Protests against police brutality have been reignited after the
  shooting of Jacob Blake. And the Trump campaign sees an opportunity.
  As counter-protests turn lethal and the president continues to stoke
  division, what's the political fallout of violence in American cities?
  How did it become a cudgel against Joe Biden and the Democrats and not
  an indictment of the man actually in charge?

  So guys, this is our Peaches \& Herb moment. We're reunited. And
  speaking for myself, it feels so good.
\item
  michelle goldberg\\
  I know, and for the last time.
\item
  ross douthat\\
  It's wonderful to be back with you guys, even if Frank is poised to
  abandon us. And I really missed you guys, you know? Sure, I was
  sitting on a beach in Maine watching the time of the world go by,
  watching my children play. But really, I was thinking about arguing
  about politics the whole time. {[}LAUGHTER{]}
\item
  frank bruni\\
  You were looking at your lovely little children and thinking of
  Michelle and me, of course.
\item
  ross douthat\\
  Of course. I'm like, which one is more Frank, and which one is more
  Michelle? And my wife was like, what are you talking about?
  {[}LAUGHTER{]}
\item
  frank bruni\\
  So let's get right into our topic this week because it is a big one.
  Almost two weeks ago, in Kenosha, Wisconsin, a 29-year-old Black man
  named Jacob Blake was shot at close range in the back by a white
  police officer in front of Blake's three sons. Video of the shooting
  spread fast, instantly rekindling protests against police brutality.
  Two days into the protests, a 17-year-old, reportedly a Trump
  supporter, shot protesters in Kenosha, killing two people. Then, in
  Portland, Oregon, a caravan of Trump supporters were met with
  counter-protesters. And amid the skirmishes that broke out, a member
  of the far right group Patriot Prayer was shot and killed. Meanwhile,
  Trump eggs on this division with all the subtlety of a military
  parade. He tweets praise for his supporters, whom he calls patriots,
  and links the Black Lives Matter protests to violent anarchy and the
  cities' failure to quell that unrest to the Democratic Party as a
  whole. So let's start with a question we've asked before on this show.
  And that is, can we blame Trump for this escalation of violence
  between Americans?
\item
  michelle goldberg\\
  I don't think there's any question that Trump has incited white
  vigilantism, white supremacy. He has been egging on kind of violent
  thuggery from his supporters since his campaign in 2015 and 2016. He
  has repeatedly lauded people who either threaten or take violent
  action against those he considers his enemies. He just recently
  praised Kyle Rittenhouse, who has been charged with murder after
  shooting two protesters in Kenosha. And something I would point out is
  that this is an argument that I have been making for years and years
  now. It's an argument that people like the Southern Poverty Law
  Center, the Anti-Defamation League has made. In my column this week, I
  talked to Elizabeth Neumann, who was part of Trump's D.H.S. until
  April, where she was a high-level counterterrorism official. And she
  also makes this argument. She says that a huge problem that people
  working on counterterrorism in the United States have to deal with is
  the growing emboldening of white supremacy and violent white
  nationalism, and that Trump has both encouraged them, made them feel
  like they have his tacit encouragement, and handicapped government
  efforts to combat it. So there's just no question at this point that
  Trump has emboldened the violent right. And then the other piece of
  that is that you're seeing some of the protests against police
  brutality also give way to violence. I mean, obviously, the proximate
  cause of that is police brutality. But I also think part of it is that
  you're starting to see left-wing people show up to these protests with
  guns, which I think is a direct response to the fact that so many
  right-wing people are showing up to these protests with guns.
\item
  frank bruni\\
  Ross, what's your reaction to that?
\item
  ross douthat\\
  I mean, I've thought all along that Trump bears responsibility for the
  general mood in American politics. I think that under a different
  president, you would have less violence in the streets. I think
  Trump's habit of sort of winking at white nationalist supporters in
  various ways probably does have some emboldening effect. I'm not sure
  it's really the only place I'd lay the stress, though, right now,
  since we're living through a wave of riots that's probably the most
  serious in this country since LA, certainly in the early 1990s. And as
  Michelle says, the proximate cause of the riots is responses to police
  brutality. But I think it's pretty clear that a mostly white, sort of
  anarchist fringe has fastened onto the protests and used them as a
  pretext for maintaining a simmering level of aggressive property crime
  and destruction in a few American cities over the last few months. And
  it's a problem independent of Trump that those cities --- Portland,
  Oregon being the main exampl e--- haven't figured out a way to get the
  violent side of those protests under control. And I think that it's
  totally reasonable to have a conversation about this that isn't just
  about Trump and is about what is the response if you're the mayor of a
  liberal city right now, and your business districts are getting
  burned? What is your response if you're the mayor of Kenosha,
  Wisconsin, and your business districts are getting burned? What's the
  response if immigrant-owned small businesses are getting torched? And
  that can't just be a conversation about how Trump is bad. I do think
  Trump is bad. There's some other stuff going on here that's mostly
  action happening on the left, I would say.
\item
  michelle goldberg\\
  Can I just interject? Because I actually--- I mean, I agree with Ross
  to a certain extent that these questions about how Democratic mayors
  should be handling violent unrest is not just a rehash of the way that
  Trump has unleashed evil spirits in the United States. But I don't
  think you can totally separate all of this and say we're just living
  through a wave of urban riots and not violent white supremacy, or not
  right-wing incitement for a second civil war, when if you look at the
  fact that even some of the most shocking acts of violence to come out
  of the wave of urban unrest that's followed George Floyd's killing has
  been by the far right. It's really, really telling that when Mike
  Pence needed an example of a really shocking killing that happened at
  one of these riots, he spoke about the murder of a federal officer
  named Dave Patrick Underwood in Oakland. And he made it seem as if
  this was an outgrowth of the Black Lives Matter protests. He didn't
  mention that the person charged in this killing was part of the
  Boogaloo boys, which is this bizarre, extremely online sort of militia
  movement that wants to spark a second civil war. So I agree that
  there's a conversation that's kind of separate from the white
  nationalist conversation. But I don't think you can take right-wing
  incitement, provocateurs out of the violence that's happening all over
  this country right now.
\item
  frank bruni\\
  Ross, in terms of how Trump does or doesn't fit into all of this, let
  me ask the question in a different way. If we had these police
  shootings under a different president, do you think we would see the
  aftermath playing out the way it is now?
\item
  ross douthat\\
  Not exactly. I mean, I think the reason the aftermath has played out
  the way it has is the pandemic, and the lockdowns, and the suspension
  of ordinary American life. I think that if you'd had the shootings and
  the protests in a situation where more people were at work, and at
  school, and cities were more normalized, you would have had some
  protests, maybe some days and nights of riots in a few places. But you
  wouldn't have had both this sort of national movement aspect of it and
  the kind of sustained protest-cum-violent stuff that you've had in a
  few places. And obviously, Trump plays a role in how America responded
  to the pandemic and so on. But I think the weird suspension of normal
  life has played a bigger role in creating the overall culture here
  than necessarily Trump himself. What also might have played out
  differently is you might have different responses from blue state
  governors and liberal mayors to the situation because they wouldn't be
  afraid of being seen as siding with Trump. I think there's a certain
  kind of political pressure that having Trump as president brings to
  bear on liberals in government confronting rallies and protests, where
  doing anything that seems to feed into Trump's narrative is something
  you can't do.
\item
  frank bruni\\
  So you think they're being less, for lack of a better adjective, stern
  with what's going on in their cities because they want to distinguish
  themselves from Trump and sort of defy him?
\item
  ross douthat\\
  Yeah, I think there's probably more hesitancy about whether it's
  calling in the National Guard, which obviously has been done in a lot
  of places, or instituting curfews, or just being seen as condemning
  the violent fringe of the protest too harshly. I mean, we had that
  whole phase of the CHAZ Zone, the sort of separatist enclave in
  Seattle, where Trump was raging about it. So the mayor of Seattle,
  because Trump is ranting, feels compelled to go on Twitter and talk
  about how peaceful this all is and how it's like we're just hanging
  out and having another summer of love here in Seattle. And then
  eventually, lo and behold, some people get killed, and the CHAZ Zone
  has to shut down. And then in the aftermath, you get reporting,
  including in our newspaper, that says, actually, this was kind of a
  hellscape if you owned a business in this area, right? And I think
  that that kind of thing would have played out very differently under
  both a Democratic president and a non-ranting Republican president. So
  that's, I think, a small case study. But I think you have similar case
  studies around the country. It would be easier for liberal mayors and
  governors to take both tougher measures and use tougher rhetoric if
  they weren't afraid of being seen to give aid and comfort to Trump's
  anti-B.L.M., anti-Antifa narrative.
\item
  frank bruni\\
  I should make sure our listeners are aware--- we're recording this on
  Tuesday morning. And I think that's important because, later today,
  after we finish recording, Donald Trump is going to be in Kenosha,
  Wisconsin. He was asked, I think, by the governor and by others not to
  come. But that is not something that is ever going to deter our
  president. Michelle, at this point in time, with the election two
  months away, does Trump want this violence?
\item
  michelle goldberg\\
  Absolutely he wants it. There was a quote from Kellyanne Conway last
  week that Biden has repeated a bunch of times for obvious reasons
  because it was extremely telling and revealing. She said on Fox, ``The
  more chaos, and anarchy, and vandalism, and violence reigns, the
  better it is for the very clear choice on who's best on public safety,
  and law and order.'' You don't invite this ridiculous, gun-waving, St.
  Louis couple to speak at the R.N.C. and say that protesters are going
  to invade the suburbs if you don't think that this wave of violence is
  good for you. And I think that that is still a very open question,
  right? There has been a few hints that some of this might be working
  in Trump's favor. But after the freak out that showed the polls
  tightening after the Republican convention. They're now back about to
  where they were. There is a kind of a chicken-and-egg question about
  whether this works, right? Because I mean, it has been sort of insane
  to watch this 17-year-old kid, a Blue Lives Matter obsessive, someone
  who, according to BuzzFeed, was at a Trump rally in January, you know,
  takes his gun, drives to protest, kills two people, is charged with
  murder. And in the next couple of days, the media narrative is, is
  Kenosha terrible for Biden? How are the Democrats going to deal with
  this, right? And it's sort of the soft bigotry of no expectations for
  Republicans, right? Nobody even expects Trump to gesture towards the
  middle. Nobody expects Trump to try to calm things down, whereas
  people do have these expectations of the Democrats. And so, although I
  think it's unclear whether kind of sparking a second civil war in the
  United States is going to be a net positive for the incumbent
  president, I think that Trump thinks it is. And I think that's what
  he's trying to do.
\item
  frank bruni\\
  Do you agree with that, Ross? Or what do you think the political
  fallout of this is going to be?
\item
  ross douthat\\
  I mean, I think there is a sort of soft bigotry of no expectations
  around Trump in that we know what we get from Trump. We know that he
  is not capable of making certain politically obvious moves that a
  normal president would make in these circumstances. And of course, it
  would benefit Trump politically if he were more likely to condemn
  people to his right. We just are aware at this point that that's not
  something he's going to do. And so the focus is on the more
  unpredictable question of how does Joe Biden handle any of this.
  Kenosha became a subject of media debate, one, because it was a really
  severe wave of vandalism and arson that happened in a swing state, in
  a region that Trump has to win and that he won, to everyone's
  surprise, last time, right? So it's not just the riots. It's the
  location of the riots and so on. It happened at a moment when you
  could see public opinions seeming to turn a bit in general surveys
  about protests and Black Lives Matter. And there was this initial
  surge of public support for the protests. And as time has gone on, and
  the Portland stuff has continued, and so on, that support has
  diminished somewhat. So there was sort of an intersection of another
  night of severe riots with that turn in public opinion. And then
  Rittenhouse himself is, as far as we can tell, not a white
  supremacist, not a Boogaloo boy, just an idiot, basically, who, at 17,
  decided to go and protect car dealerships from rioters and ended up
  shooting people in a context that was not him wading into a crowd
  firing. It was him basically getting his ass kicked, as far as anyone
  could tell.
\item
  frank bruni\\
  OK, but nonetheless, what do you make, Ross, about the fact that the
  right has turned Kyle Rittenhouse into a hero? Tucker Carlson has been
  raving positively about him on air. Ann Coulter said she wants Kyle
  Rittenhouse as her president, I guess thereby admitting that Donald
  Trump isn't doing such a great job. Looking for the silver lining in
  that comment.
\item
  ross douthat\\
  Well, Ann Coulter has been arguing that Trump is doing a terrible job
  for a long time.
\item
  frank bruni\\
  She wants her wall. She wants her wall. She wants her wall.
\item
  michelle goldberg\\
  Right.
\item
  ross douthat\\
  And if you watch Tucker Carlson closely, the fact that he thinks Trump
  is a failure is a pretty strong subtext of his show night to night.
\item
  frank bruni\\
  OK, but let's go back to Kyle. What do you make, Ross, about the fact
  that elements of the right, and Ann Coulter ---
\item
  michelle goldberg\\
  And Trump.
\item
  frank bruni\\
  Tucker Carlson are there --- yeah, and Trump himself, are lionizing
  this 17-year-old, who went with a gun into an inflammatory situation?
\item
  ross douthat\\
  Because the right thinks that the people in charge of American cities
  have essentially retreated and abdicated their responsibility to
  protect peoples, businesses from rioters. And so people are, as
  happened with Bernie Goetz, famously, in the 1980s, people are you
  know are celebrating this kid who, again, as far as I can tell, was an
  idiot who is culpable for people's deaths, whether he's guilty of
  murder or not. But they're celebrating him for going and trying to
  effectively defend the things that the police weren't defending.
\item
  frank bruni\\
  But that celebration is sick, no?
\item
  michelle goldberg\\
  Well, it's dangerous, right? Because it's going --- it sends a message
  to other dumb ---
\item
  ross douthat\\
  I mean, I don't agree. Again, I think what Kyle Rittenhouse did was
  moronic at best, and people are dead because he, at best, was a moron.
  And I'm not celebrating it. I think that when you have an abdication
  of sort of civil authority to protect people's property and sometimes
  lives, then you're going to get a mood that's sympathetic to
  vigilantism, yeah. I don't think that's remotely surprising.
\item
  frank bruni\\
  No, no, well, that's not--- granted, Ross, that's not a celebration.
  That sounds a little bit to me like it's at least in the zip code of a
  justification.
\item
  ross douthat\\
  I mean, I think that if Kyle Rittenhouse had not been a 17-year-old
  teenager from another state, but had been a small business owner, who
  was armed when people came into their business and started setting it
  on fire, then it would have been justified, yeah.
\item
  michelle goldberg\\
  Right. But that's if it had been a totally different situation, right?
\item
  frank bruni\\
  {[}LAUGHING{]} Exactly. Exactly. Thank you, Michelle.
\item
  ross douthat\\
  Well, yeah, yes. But I'm saying that there is a reason that, in
  situations where people are going around burning businesses, that
  people get sympathetic to vigilantism. And it's dumb to be sympathetic
  to the kind of vigilantism that Kyle Rittenhouse embodied. But just
  saying we can't --- it's sick to celebrate vigilantism in a climate of
  people's businesses being torched, then --- I mean, people's
  businesses getting torched is sick, right? Some of the people doing
  things in these riots, not just the people shooting cops and things
  like that, but people torching somebody's livelihood, that's sick too,
  right?
\item
  michelle goldberg\\
  Well, I don't know. I mean, to me, it sounds like I could make the
  same argument, but I won't, that, say, in a climate where police are
  shooting people and the government is not protecting them, then it
  makes sense that people will riot as --- what is the Martin Luther
  King line? A riot is the voice of the unheard. I could construct a
  similar justification for rioting and looting, but I won't do it. And
  it sounds like, to me, that's what you're doing, basically, kind of
  desperate times call for desperate measures.
\item
  ross douthat\\
  Why don't we achieve consensus here and say that I will say that
  crossing state lines with a gun to try and interject yourself into a
  gravely chaotic situation, whatever mitigating circumstances there
  are, that is both a stupid and a wicked thing to do. It's also wicked
  to burn somebody's business, right?
\item
  frank bruni\\
  Oh, yes, of course. Of course.
\item
  michelle goldberg\\
  I think it's less wicked to burn someone's business than to kill
  someone.
\item
  ross douthat\\
  OK, is it less wicked to burn someone's business than to take your gun
  and try and stand guard around someone's business? What Kyle
  Rittenhouse intended to do, which was stand guard around someone's car
  dealership that might have been torched, is that more wicked than
  torching a car dealership? Less wicked? Wash? What do you think?
\item
  michelle goldberg\\
  I think it's more dangerous.
\item
  frank bruni\\
  But beyond these gradations, which, to me, are almost getting a little
  silly, is when you enter a situation like that, when you walk into a
  room that is strewn with fuel with a match, bad things are going to
  happen. I mean, this is a volatile situation that your presence is
  going to do nothing but render more volatile. But the reason I dwelled
  on it, Ross, is ---
\item
  ross douthat\\
  No, but look. But Frank, the conservative --- the thing that you're
  reacting to so strongly from conservatives here is rooted in the fact
  that you're describing the burning building in these sort of abstract
  terms like it's a situation, or there's things on fire. But in fact,
  people are setting those fires. People are torching those businesses.
  People are performing actions that are wrong.
\item
  michelle goldberg\\
  No, and it is --- I mean, it's striking to me to listen --- Ross, the
  degree of emotion that I can tell you feel about this, where, to me,
  it is kind of bad. I feel immensely sorry for some of these small
  business owners who are not implicated in police violence. But it
  doesn't bring up the same emotion in me as seeing police shoot unarmed
  civilians or police ---
\item
  ross douthat\\
  OK, but the police shooting unarmed civilians is obviously worse than
  burning a building. But we're having an argument about the idiot kid
  who tries to protect the car dealership, right?
\item
  frank bruni\\
  No, I'm actually trying not to have an argument --- I'm trying not to
  have an argument about the kid. I'm trying to have an argument, or not
  even an argument, about the disparate reactions to what's going on. So
  I started--- we went down this Kyle Rittenhouse rabbit hole when I was
  asking why people on the right were celebrating him, were lionizing,
  were going well beyond justifying him, including the president, right?
  And earlier, you made a reference, Ross, to will Joe Biden condemn
  certain actions on the left? Well, he has now in Pittsburgh yesterday.
  We're recording on Tuesday. In Pittsburgh on Monday, he said, these
  protests that get out of hand, the sooner --- we have --- what I'm
  focused on here is the real imbalance between what Joe Biden has said
  and will say and I think will continue to say and what Donald Trump
  won't say and isn't saying. And it's not just Donald Trump. This has
  spread throughout the Republican Party. There was a fascinating
  interview on Sunday that Dana Bash of CNN did with Senator Ron Johnson
  from Wisconsin, where she asked him if he would condemn the violence.
  And he kept dancing around the words because he was clearly so worried
  that if he seemed to be condemning Kyle Rittenhouse and people who'd
  run to those burning stores that you're talking about, Ross, that he
  would somehow alienate his political constituency. I just think that's
  weird, dangerous, and unhealthy.
\item
  ross douthat\\
  My strong impression is that we have passed through a months-long
  period in which Democratic politicians and much of the mainstream
  press have wanted to insist that riots have not been as bad as riots
  have actually been because they are supportive of the cause of the
  protests that the riots have been attached to. That's my impression of
  the story of the last few months.
\item
  frank bruni\\
  I will grant you, Ross, the riots have been horrible. And I think
  sometimes, there's a big element of truth to what you're saying. And
  part of the challenge right now for Joe Biden is, I think, a lot of
  people feel that they're not hearing a stern enough condemnation, that
  people are not admitting fully that people gave protesters too much of
  a pass in terms of gathering amid a pandemic and all of that.
\item
  ross douthat\\
  Well, I don't think that's. That's a separate issue.
\item
  michelle goldberg\\
  I would like some empirical understanding of how bad the riots have
  been versus how bad they've been presented in the mainstream media.
  Because my impression, being back here in New York City, is that
  there's this presentation of New York and other urban areas as this
  sort of crime-ridden hellscape, when, in reality, being back in New
  York, I wouldn't really know any of this was going on if I didn't see
  it on television or on the internet. It's much more contained, at
  least here, than you would understand from reading mainstream
  newspapers, from watching cable news, from looking at it on the
  internet. I feel like I hear similar things from people in Portland,
  who will say, you know, there's this idea that the city is on fire.
  But if you're out of a few-block radius, you're barely even aware of
  it. And so it's just not clear to me that the scope of the riots has
  been underplayed.
\item
  ross douthat\\
  I think the scope of the riots have now been overplayed on the right.
  But I think that in our world of media, I think there was a period of
  a couple months when the scale of the damage in places like
  Minneapolis, especially, was underplayed. And also, there was a spasm
  of looting in Chicago a few weeks ago. And it was bad enough that they
  raised the bridges around Chicago to try and keep people from entering
  or leaving downtown in order to contain it. And it got only coverage,
  as far as I can tell, from local media and right-wing media. And you
  know, I cited the example of the CHAZ stuff in Seattle. I think what
  was --- everything that was wrong there got a lot more coverage after
  the fact, after it had been shut down. I guess all I'm saying is that
  I think --- not to be all both sides here. But I think both sides are
  engaged in a kind of dangerous denial of parts of reality.
\item
  michelle goldberg\\
  So can I say something to that? I mean, I do think that left-wing
  social media has had a really deleterious effect in that there is such
  stigma in seeming to call out, quote-unquote, ``bad protesters'' or
  make distinctions between good and bad protesters, or say commonsense
  things about the political impact of unrest. If you look at the firing
  of David Shor, which is something that maybe took up more oxygen in
  some of these debates than you would think the firing of a data
  analyst would usually receive. And it was because this data analyst
  for a progressive consulting firm, who was fired from tweeting out a
  study by a Black Princeton professor about how riots in 1968 helped
  boost Richard Nixon's vote share. So I think this bleeds over a little
  bit. And Democratic politicians, to the extent that they are too
  online, might be hampered in making what seem like commonsense points
  to a lot of their constituents. But at the same time, I don't buy the
  idea that this stuff has not been widely covered. There's also an
  element of just journalists --- there are --- some of our very brave
  colleagues are traveling and are reporting and kind of risking their
  health to do it. But I think fewer people are doing it than would have
  done in the past because of the pandemic. So there's a bit of a fog of
  war element to try and figure out what is going on in CHAZ. But the
  reason that we know how bad CHAZ got is because of a big story in
  ``The New York Times.''
\item
  ross douthat\\
  I think there's truth to that. I also, though, think that there has
  been a lot --- precisely because Donald Trump keeps talking about
  Antifa, there has been a lot of pressure to say like, oh Antifa, that
  just means people who are against fascism, or Antifa, that doesn't
  really exist, or it only exists in Tom Cotton's fervent imagination.
\item
  michelle goldberg\\
  But it doesn't exist in the way that they talk about it as like ---
  you hear Laura Logan talking about how they're getting marching
  orders. And there's this rumor that Donald Trump spread of black-clad
  people on an airplane. I mean, I know these people. I've reported on
  these people enough to know that, yeah, you have kind of black bloc
  idiots in every single metropolitan area in the country who are always
  trying to hijack protests and break shit. And that's something that
  precedes this current moment. There's overlap but also a distinction
  between the black bloc and Antifa, if we really want to go down that
  road. They're not entirely the same thing. But there has been this
  attempt to basically turn kind of black bloc idiots into a
  well-organized terrorist threat. That's ridiculous.
\item
  ross douthat\\
  I think it's wrong to see them as a well-organized terrorist threat. I
  think, as someone who's spent much of the last three years in
  arguments with people on my right, saying, look, you Antifa, this is
  not really a real thing. This is just a bunch of idiots cosplaying. I
  think we've established that those people are capable of doing
  substantial amounts of property damage and creating extremely
  dangerous environments in American cities, when given the opportunity
  afforded by a pandemic, Donald Trump, and mass protests. I would say
  that I underestimated the capacities of those idiots to sustain arson,
  damage, looting, and violence over a multi-month period. I think
  they've done a fairly impressive job in a lot of American cities of
  doing that. And that doesn't make them the next ISIS. It doesn't make
  them an existential threat to the US. It doesn't make Donald Trump
  right in his fantasies about black-clad Antifa ninjas on a plane. But
  it's still kind of a big story, right?
\item
  michelle goldberg\\
  I don't know. I think it's really unclear. Look, there's clearly these
  kids who've been dreaming of--- I don't know --- some sort of
  overthrow of the system for their whole lives and think that this is
  their big shot. And you kind of burn down enough coffee shops, and the
  next thing you know, you have the end of capitalism. {[}LAUGHTER{]}
  But it's very unclear---
\item
  ross douthat\\
  Which is true. I mean, we should concede that that's true. There is a
  moment at which capitalism will end.
\item
  michelle goldberg\\
  But it's very unclear to me how much of this is Antifa, how much of it
  is just opportunists that see looting going on and want in on it, how
  much of it is other sorts of protesters. And then there is --- I think
  that people on the left can overplay the role of right-wing
  provocateurs, but they're also part of the mix. If you want to talk
  about what happened in Minneapolis, there was that black-clad guy that
  really kicked off a lot of the arson and property damage. And you
  obviously can't blame him for all of it. But I think people can look
  back and say that's when it started. And we do now know, or at least
  he's been arrested and charged as being a white supremacist
  provocateur. So I think that there is still a fair degree of confusion
  about the precise makeup and the precise breakdown of responsibility
  for what's happening right now. {[}MUSIC PLAYING{]}
\item
  frank bruni\\
  Let's take a quick break, and we'll be right back.

  {[}MUSIC PLAYING{]} And we're back. So the election is exactly two
  months away. And this issue, clearly, is going to linger throughout
  it. Michelle, you said earlier you think Trump wants this violence. Do
  you think, at the end of the day, it is going to help him on November
  3, or I should say, beforehand as well, since a lot of the voting this
  year is going to be mail-in?
\item
  michelle goldberg\\
  I mean, I should say that because I am so terrified of what's going to
  happen, I fear in my darkest moments that it will. I don't see how you
  can experience the 2016 election and not think that white backlash is
  an incredibly powerful force in American politics. But I think that
  might be more my emotions talking than anything empirical. There was
  one poll that showed some tightening. There's been some anecdotal
  reporting, including in our newspaper. But overall, the polls are
  really, really stable. In the 1968 analogy, Donald Trump is not
  Richard Nixon. He's L.B.J., right, in that it's kind of hard to make
  the case that you need to re-elect me president to stop the violence
  and chaos that has happened while I'm president.
\item
  frank bruni\\
  Ross, what do you think the political fallout of this is going to be?
  How do you see this coloring the presidential race?
\item
  ross douthat\\
  I mean, so I wrote a column weeks ago now, where I tried to imagine
  how Trump could possibly come back. This was when Biden was up by an
  average of 10, I think, in a lot of polls. And the scenario I spun out
  was basically that Trump needed COVID infections to drop dramatically,
  that he needed some of the early herd immunity theories to be true.
  And he probably needed the violent fringe of the protests to become
  much more salient going into the election. A very mild version of that
  is happening. COVID infection rates have fallen, not as far as we
  would like them, but they have fallen. And the violent fringe of the
  protests has gotten somewhat more salient. I think it's understandable
  that liberals would be worried. And I think it's reasonable to worry
  in the sense that--- the incumbency point that Michelle makes is, I
  think, partially right. But there is another reality, which actually,
  David Shor, the data analyst Michelle mentioned in the first segment,
  who was fired, famously, for some of his tweeting about the politics
  of riots, he makes this argument about issues salience, that voters,
  for better or worse, trust one party more than the other on a
  particular issue. And so if you raise the salience of that issue,
  who's got the policies that technically poll best might be less
  important than just the fact that the issue is salient. So if you
  raise the issue salience of health care, the policy details may not
  matter as much as the fact that voters tend to trust Democrats more on
  health care. If that issue becomes more salient, it's good for the
  Democrats. And so if voters trust Republicans more on fighting crime
  and controlling urban riots, then even if Donald Trump doesn't have a
  10-point plan to stop the urban riots, the more you raise the salience
  of that issue, the more potential advantages Republicans have. So
  that, I think, is one way of arguing that what's happened in the last
  few weeks around these issues should make Democrats worried. That
  being said, Trump is not--- I mean, we were saying this in the first
  segment. But Trump is running against Joe Biden. Joe Biden is a
  famously moderate Democrat with a tough-on-crime past who's, to put it
  charitably, a little bit past his prime as a political communicator,
  but still is perfectly capable of giving a speech where he says, riots
  are bad, right? And Trump is not really capable of sustaining a
  Richard-Nixon-type, claiming-the-center strategy. And to the extent
  that that's the basic political reality of the race, I think things
  would need to get a lot better with the coronavirus and a lot worse
  with the riots before full Democratic panic would be justified.
\item
  frank bruni\\
  So you said Biden is perfectly capable of giving a speech decrying the
  violence. He gave that speech on Monday in Pittsburgh. Michelle, did
  you think he accomplished what he needed to? Do you think that speech
  was on the mark?
\item
  michelle goldberg\\
  Yeah, I mean, it's a difficult thing for me to evaluate because
  there's sort of like what I want Biden to do because I think it will
  help him win versus what I believe. So what I believe is that the real
  problem here is police violence. And that is what you have to tackle
  first. Do I believe that it is good for Biden to decry these riots and
  to show wavering suburbanites that he cares about urban unrest and
  that he's going to keep them safe, even though I believe that they are
  already 100 percent safe from this phenomenon? Yes, I do. So to me,
  the speech was great. And I thought it was great also because it
  opened up the obvious question. Yes, I condemn violence. I condemn
  left-wing extremism. Your turn. Why won't you do it? If we just had a
  whole news cycle about is this bad for Biden, why won't Biden condemn
  the riots, will he condemn them strongly enough --- it's frustrating
  that it seems unlikely that we're not going to have a similar news
  cycle about Trump, again, because nobody expects anything of him.
\item
  ross douthat\\
  Except that the news cycle is always about how Trump is bad. Why won't
  Biden do this thing that he needs to do to make sure this bad man is
  beaten? I mean, that's the dynamic, at least of CNN. It's not the
  dynamic everywhere. But it's a pretty common media dynamic. And Trump
  does get--- after Charlottesville that Trump had some of his worst
  polling numbers. When Trump is seen as not condemning right-wing
  extremists, white supremacists, and so on, it shows up in the polls.
  And the Trump campaign --- other people have made this point, maybe
  Josh Barro or somebody. But they keep setting up really easy tests for
  Joe Biden to pass. They're like, oh, Joe Biden, he won't leave his
  basement. He can't speak. He just mumbles. And then he gives a normal,
  pretty good convention speech. And it's like, OK, pass that test. And
  then they're like, well, Joe Biden won't condemn the riots. And so Joe
  Biden gives a speech condemning the riots. And now, the bar will be
  Joe Biden won't specifically condemn Antifa. And as you can tell from
  the first segment, I think Joe Biden should specifically condemn
  Antifa. Think he could go further in his specific attacks to the left.
  But I think if that seems politically necessary, he obviously will do
  that.
\item
  frank bruni\\
  He did something else really smart, I think, in his remarks on Monday.
  And I think he said it several times. It was simple. But he certainly
  said it once loud and clear. And like I said, I think multiple times,
  where he kind of said to voters, you know me. You know me. And that
  was bigger than just about the violence and whether he condemns the
  violence. Obviously, whether they're trying to pin him into a corner
  by being not sufficiently condemning of the violence, whether they're
  trying to do it in other ways, they're saying that--- Trump and his
  surrogates and his enablers, they're all saying that Joe Biden is a
  hostage of the left or will be a hostage of the left. And I just think
  it's very effective when he speaks plainly and simply and trades on
  what is his greatest strength, which is that he has been around a long
  time, and he is not scary to Middle America. And I just really,
  really, I thought, was smart, and it really kind of rang out when he
  said, I believe multiple times, in his remarks to voters, you know me.
\item
  michelle goldberg\\
  Well, and he said, do I look like a radical socialist to you? Like,
  come on.
\item
  ross douthat\\
  Yeah, I think that's an effective line. I mean, I don't want to --- I
  think there was a little bit of over-praising the Biden speech maybe
  from journalists who were panicked by a few of those post-convention
  polls. And Biden has lost a step. That's just not really disputable.
  He is not as vigorous a politician as he was eight or 10 years ago.
  And that's a weakness. And he sort of wanders verbally more than he
  used to. He's not a dynamic presidential candidate at all. But that's
  probably OK for the kind of race he's trying to win.
\item
  frank bruni\\
  It may actually be to his advantage, given how exhausting Donald Trump
  is to most Americans.
\item
  michelle goldberg\\
  I don't know. I worry about it a lot. The Democrats haven't won with a
  candidate over 55 in a really long time. You know, LBJ seems like an
  ancient person, but he was, I think, 55 when he became president. We
  don't have a great record of, in either party, electing kind of old
  Senate warhorses. And I don't know. It definitely scares me. I mean,
  it's why I didn't want him to be the nominee, even though I now think
  he has some strengths.
\item
  frank bruni\\
  But wait a second, Michelle. You wanted Elizabeth Warren, and she's 71
  now. So ---
\item
  ross douthat\\
  But she's a vigorous 71, Frank. I mean, she is. She's more vigorous.
\item
  frank bruni\\
  OK, I will admit she's a much more vigorous 71 than Joe is at 77. But
  I mean, she breaks that paradigm of Democrats winning with the
  youngest candidate. I'm just---
\item
  michelle goldberg\\
  Yeah, but, you know, 71 in woman years.
\item
  frank bruni\\
  Fair enough. Fair enough.
\item
  michelle goldberg\\
  You know. {[}LAUGHS{]}
\item
  ross douthat\\
  I mean, but I think that is what Michelle is saying is that the danger
  for Biden is that he doesn't turn out young voters at quite the rates
  that a more --- either a more dynamic or more left-wing candidate
  would have, and that his age and wobbliness becomes a problem in the
  event that, if there were sustained rioting and the politics of law
  and order just stay incredibly salient in states like Wisconsin and
  Minnesota, then the Democratic candidate needs to --- he needs to
  project some toughness, some sense that if it's necessary to sort of
  bang heads --- again, not of criminals, but of mayors, of going after
  his own governors or his own mayors and so on, that he can do that.
  And a little more 1980s Biden would probably be better for this
  campaign.
\item
  frank bruni\\
  Is one of the dangers for Biden us? And by that, I mean the media. And
  what I'm talking about is I kind of notice in this most recent thing
  with urban --- with the violence and the protests as a good example.
  You know, Donald Trump spotlights an issue, takes some position on it,
  says something provocative. And the media immediately begins writing,
  how is Joe Biden going to respond? And I see a lot of people making
  the correct comment that Biden isn't controlling or seizing the
  narrative. But I'm not sure we allow him to. I feel like, for all of
  our hand-wringing and soul searching and self-examination about the
  way we covered Trump, I still think we let him define the event of a
  given 24, 72-hour news cycle. And then it becomes how will Joe Biden
  respond. And I'm not sure how Biden breaks out of that if, in fact,
  it's a pattern that we've established. Michelle, am I seeing--- are
  you seeing this too? Or---
\item
  michelle goldberg\\
  I think that's right. I mean, and again, I think there's a bit of a
  chicken and an egg thing, like I said before, where it's hard to say,
  is the media responding to a genuine shift in the electorate or the
  polling? Or is the media creating a narrative because it's sort of
  that time in the race when you need a new story besides Biden is
  inevitable and Trump is self destructing. But I definitely do think
  that --- again, the media, I think it both sort of follows Trump's
  lead on defining the issues and then lets him off the hook so that
  there's not a --- you don't see a lot of reporting. And I do think
  that Ross is right in that, because I think a lot of journalists do
  fear for the republic, so they're not sort of worried about what Trump
  is doing wrong. There's not a sense of what does Trump have to do to
  right the ship. That piece Jamelle wrote is the kind of piece that you
  see about Biden all the time and you basically never see about Trump.
  You never see people sort of suggesting that Trump might be losing the
  center. You never see people dwelling on Trump's refusal to disavow
  the far right. And so I do think that there is a little bit of chronic
  both-sidesing that I think was a big, big problem for Hillary Clinton.
  I think that the media vastly inflated the salience of email server
  management as an issue. And I think voters took a cue from that, that
  this is a really big deal. And I think that they could be doing the
  same thing over again.
\item
  ross douthat\\
  Yeah, well, but it's different, right? Biden is not --- there's a lot
  of coverage of, is Biden doing enough to win? Is Biden going to blow
  it and so on? There's very little critical scrutiny of Biden's actual
  record. I mean, we're not getting a range of stories that are --- the
  stories that liberals feared we'd get, that are like, well, what about
  the Biden family's overseas business dealings? I mean, there was a
  bunch of stories about Biden not being left enough during the primary
  campaign. But there's --- I don't know. I mean, I feel like there's
  very little, let's say, of the media echoing the Trump campaign's
  narrative about Biden, right? The Trump campaign's narrative about
  Biden is that he is an embodiment of a failed American establishment
  that outsourced American jobs overseas and led us into disastrous wars
  in the Middle East. And that's kind of true. {[}LAUGHS{]} But nobody
  in the media is echoing that line. Nobody in the media is
  re-excavating Biden's Iraq war vote or things like that in the way
  that the media did, I think, play into Trump's ``crooked Hillary''
  narrative in 2016. I think all of the Biden stuff is about --- right
  now, at least, we've still got a little ways to go --- is just about
  is he doing enough. And they aren't trying to like grab the narrative
  with huge policy speeches and big ideas. I mean, I don't know if you
  can completely blame the media for what is, in fact, choices that the
  Biden campaign is making, which is to run a --- it's a very cautious,
  front-porch-style campaign. I mean, Biden's convention speech was--- I
  mean, it was actually really distinctive. It was really short. It had
  almost no policy in it. It was just a personal appeal. Like I'm Joe
  Biden. You know who I am. I'm a good guy. Donald Trump's terrible.
  He's messed up the coronavirus. Vote for me. That's the Biden
  strategy. And it is vulnerable, but it's vulnerable to external events
  like the coronavirus diminishing and riots getting worse.
\item
  frank bruni\\
  No, but that's an excellent point. It's impossible to seize the
  narrative if, in fact, you're protecting a lead. And that has been the
  Biden strategy. It factors into why Kamala Harris was the
  vice-presidential choice. I think there has been an assumption that
  he's ahead, that if you just kind of extrapolate forward far enough,
  he ends up ahead, he ends up winning. And let's not do anything wrong.
  Let's not take any big risks. Let's protect our lead. I think that's a
  big part of the dynamic. And I think that's what you're referring to
  Ross, right?
\item
  ross douthat\\
  Yeah. And he does --- well, and the other little thing is that,
  because there's an assumption that Trump can win the electoral college
  even if he loses the popular vote by 2 or 3 points, there's also a
  weird uncertainty in the media narrative about how much is Biden
  really winning by. If he is up by 8 points, that's huge by modern
  presidential campaign standards. On the other hand, it's only a
  5-point lead from an electoral college point of view. So there's a
  certain kind of whip sawing and uncertainty in the media narrative
  there, too, I think.
\item
  frank bruni\\
  And the media is, once again, despite its promises every four years
  not to do so, covering the horse race more than anything else. Here we
  are talking about polling margins and all of that. I'm raising my hand
  and saying, guilty as charged, you know?
\item
  michelle goldberg\\
  No, but I don't think there's anything to be guilty of. I mean, we're
  talking about is there going to be a recognizable American republic
  three months from now. The question of kind of who's ahead and who's
  behind in that existential contest is a big one and it's actually way
  more important than anything in Biden's policy platform.
\item
  ross douthat\\
  Without necessarily agreeing with that argument, {[}BRUNI LAUGHS{]} I
  will say that it's also true that Trump is not rolling out policy
  ideas regularly either. I mean, there is nothing but --- with Biden
  running a cautious, low-policy campaign, and Trump running a
  scattershot, demagogic, low-policy campaign --- I mean, in 2016, Trump
  had more policy. He had an actual sort of critique of the neoliberal
  establishment that was really important to his closing argument in
  2016. And that sort of resurfaced in the Republican convention. But
  it's not really there. Trump's not really running on policy. So what
  else does the media have to write about except the horse race?
\item
  frank bruni\\
  Well, this is a great place to end because it's not often that all
  three of us are in perfect alignment, and on the fact that this is not
  a policy election, I think we have total, total agreement, right?
\item
  michelle goldberg\\
  Yes, absolutely.
\item
  ross douthat\\
  And I guess agreement is also a good place to end your wonderful stint
  as part of our show here, Frank. And we will miss you terribly. But
  since this is your final ``Argument'' episode, we, of course, have to
  turn to you for the recommendation. So what will you give us as a
  parting gift as you say goodbye?
\item
  frank bruni\\
  Well, since it's my last recommendation ever, at least on the show ---
  I hope I have recommendations in my life. {[}LAUGHS{]}
\item
  ross douthat\\
  You're going to do your own podcast that's just recommendations.
\item
  frank bruni\\
  I have, over the last five years, become this strange and stubborn
  evangelist for the short story that is one of my favorite short
  stories ever called ``In the Cemetery Where Al Jolson is Buried.'' Has
  either of you ever read it?
\item
  michelle goldberg\\
  No.
\item
  ross douthat\\
  No.
\item
  frank bruni\\
  It's by the famed, accomplished, terrific short story writer, Amy
  Hempel. It's title does not prepare you for what it's about. It's
  basically about a woman watching a friend die. And yet it manages to
  be jovial to the point of jocular until the final paragraphs. And it
  is the story that just does this amazing thing, where you're kind of
  rolling along with it. You're feeling little pinpricks of grief and
  sorrow. And then in the last couple of sentences, it just tears your
  heart straight out of your chest. Now, you're thinking, why am I
  recommending this? I'm one of those people who, when I'm feeling a
  little blue, I like to take the feeling all the way down until it
  comes back over the top. I get a glass of wine. I listen to Billie
  Holiday. And I read ``In the Cemetery Where Al Jolson is Buried'' by
  Amy Hempel. If you Google it, you will find the entire story online.
  And you can read it without buying anything. When I read it that way,
  I did not own any Amy Hempel book. I felt like I owed her for the
  intense pleasure. And so my recommendation is read the story online.
  And if you're as taken with it as I am, reward Amy Hempel, and do the
  right thing by buying a copy of ``The Collected Stories of Amy
  Hempel,'' in which there are many, many other gems. But the peak, the
  Everest is ``In the Cemetery Where Al Jolson is Buried.''
\item
  michelle goldberg\\
  I'm also a wallower, so I'll do that.
\item
  frank bruni\\
  You'll love this.
\item
  ross douthat\\
  And Frank, I guess, bringing us all the way down to the bottom so we
  can come back up is a good metaphor or synecdoche or some literary
  form for how you're departing our show. {[}LAUGHTER{]} So---
\item
  frank bruni\\
  Or I was actually going to say for the Trump years. He's bringing all
  the way to the bottom so we can come back up the top. Yeah!
\item
  ross douthat\\
  Well, we'll save that for ---
\item
  michelle goldberg\\
  The worse the better.
\item
  ross douthat\\
  We'll save that for December. But Frank, thank you again for being
  with us over these totally insane months. And we will miss you
  terribly.
\item
  michelle goldberg\\
  Yes, thank you, Frank.
\item
  frank bruni\\
  Thank you guys. I'll miss you. And I'll miss the listeners for ``The
  Argument.'' I thank them for bearing with me.
\item
  ross douthat\\
  And may the road always rise to meet your feet and, you know, the rest
  of an Irish proverb. {[}LAUGHTER{]}
\item
  frank bruni\\
  I'm on the road, Ross. I'm on the road. {[}THEME MUSIC{]}
\item
  ross douthat\\
  Now, just one more thing before we go this week. We're going to start
  releasing this podcast on Fridays rather than on Thursdays in an
  effort to catch more of the week's news as the election heats up. So
  starting next week, we'll be talking to you and arguing with each
  other on Fridays instead. And with that, that's our show this week.
  Thank you for listening. The team includes Phoebe Lett, Vishakha
  Darbha, Kristin Lin, Isaac Jones, and Paula Szuchman. Special thanks
  to Kathy Tu. ``The Argument'' will be back in your feed next Friday.

  Wait, what are peaches and herbs?
\item
  frank bruni\\
  Oh, Ross.
\item
  michelle goldberg\\
  Oh, Ross!
\item
  frank bruni\\
  Peaches \& Herb, it's an R\&B duo that did the song ``Reunited,''
  which is a prom staple.
\item
  ross douthat\\
  Oh, man.
\item
  frank bruni\\
  You obviously never went --- you didn't go to your prom, did you?
\item
  ross douthat\\
  I didn't. We're really going to excavate some pretty some pretty dark
  territory here if we go down that road. {[}LAUGHTER{]}
\end{itemize}

Previous

More episodes ofThe Argument

\href{https://www.nytimes3xbfgragh.onion/2020/09/11/opinion/the-argument-latino-2020-vote.html?action=click\&module=audio-series-bar\&region=header\&pgtype=Article}{\includegraphics{https://static01.graylady3jvrrxbe.onion/images/2020/09/12/opinion/10argumentWeb/10argumentWeb-thumbLarge-v2.jpg}}

September 11, 2020How to Win the Latino Vote

\href{https://www.nytimes3xbfgragh.onion/2020/09/03/opinion/the-argument-trump-biden-kenosha-portland.html?action=click\&module=audio-series-bar\&region=header\&pgtype=Article}{\includegraphics{https://static01.graylady3jvrrxbe.onion/images/2020/09/05/opinion/03argumentWeb/03argumentWeb-thumbLarge.jpg}}

September 3, 2020Is `American Carnage' Campaign Gold?

\href{https://www.nytimes3xbfgragh.onion/2020/08/27/opinion/the-argument-republican-convention-trump.html?action=click\&module=audio-series-bar\&region=header\&pgtype=Article}{\includegraphics{https://static01.graylady3jvrrxbe.onion/images/2020/08/28/opinion/27argument-ninetytwo1-print/27argument-ninetytwo1-thumbLarge.jpg}}

August 27, 2020Can the Republicans Sell a Whole New Trump?

\href{https://www.nytimes3xbfgragh.onion/2020/08/20/opinion/the-argument-democratic-convention-biden.html?action=click\&module=audio-series-bar\&region=header\&pgtype=Article}{\includegraphics{https://static01.graylady3jvrrxbe.onion/images/2020/08/20/opinion/20argument-ninetyone1/20argument-ninetyone1-thumbLarge.jpg}}

August 20, 2020What Biden Must Do

\href{https://www.nytimes3xbfgragh.onion/2020/08/13/opinion/the-argument-coronavirus-catholic-covid.html?action=click\&module=audio-series-bar\&region=header\&pgtype=Article}{\includegraphics{https://static01.graylady3jvrrxbe.onion/images/2020/08/13/opinion/13argument1/merlin_173532477_02e02102-92e6-4f5a-82bf-5394265f898b-thumbLarge.jpg}}

August 13, 2020Is Individualism America's Religion?

\href{https://www.nytimes3xbfgragh.onion/2020/08/06/opinion/the-argument-trump-coronavirus-election.html?action=click\&module=audio-series-bar\&region=header\&pgtype=Article}{\includegraphics{https://static01.graylady3jvrrxbe.onion/images/2020/08/06/opinion/06argSub/06argSub-thumbLarge.jpg}}

August 6, 2020Trump Supporters Make Their Case for 2020

\href{https://www.nytimes3xbfgragh.onion/2020/07/30/opinion/the-argument-authoritarianism-anne-applebaum.html?action=click\&module=audio-series-bar\&region=header\&pgtype=Article}{\includegraphics{https://static01.graylady3jvrrxbe.onion/images/2020/07/31/opinion/30argumentWeb-print/30argumentWeb-thumbLarge.jpg}}

July 30, 2020When Conservatives Fall for Demagogues

\href{https://www.nytimes3xbfgragh.onion/2020/07/23/opinion/the-argument-israel-palestinian.html?action=click\&module=audio-series-bar\&region=header\&pgtype=Article}{\includegraphics{https://static01.graylady3jvrrxbe.onion/images/2020/07/25/opinion/25audio/21argumentWeb-thumbLarge.jpg}}

July 23, 2020The Case for a One-State Solution

\href{https://www.nytimes3xbfgragh.onion/2020/07/16/opinion/the-argument-tammy-duckworth.html?action=click\&module=audio-series-bar\&region=header\&pgtype=Article}{\includegraphics{https://static01.graylady3jvrrxbe.onion/images/2020/07/17/opinion/16argumentWeb-print/16argumentWeb-thumbLarge.jpg}}

July 16, 2020A Conversation With Tammy Duckworth

\href{https://www.nytimes3xbfgragh.onion/2020/07/09/opinion/is-trumps-fate-sealed.html?action=click\&module=audio-series-bar\&region=header\&pgtype=Article}{\includegraphics{https://static01.graylady3jvrrxbe.onion/images/2020/07/10/opinion/10a2_audio/09argument1-thumbLarge.jpg}}

July 9, 2020Is Trump's Fate Sealed?

\href{https://www.nytimes3xbfgragh.onion/2020/07/02/opinion/the-argument-protest-statue-revolution.html?action=click\&module=audio-series-bar\&region=header\&pgtype=Article}{\includegraphics{https://static01.graylady3jvrrxbe.onion/images/2020/07/05/opinion/02argument-eightyfive1/02argument-eightyfive1-thumbLarge.jpg}}

July 2, 2020Whose Statue Must Fall?

\href{https://www.nytimes3xbfgragh.onion/2020/06/25/opinion/the-argument-biden-vice-president-supreme-court.html?action=click\&module=audio-series-bar\&region=header\&pgtype=Article}{\includegraphics{https://static01.graylady3jvrrxbe.onion/images/2020/06/28/opinion/25argument-eightyfour1/25argument-eightyfour1-thumbLarge.jpg}}

June 25, 2020Place Your Bets on Biden's V.P.

\href{https://www.nytimes3xbfgragh.onion/column/the-argument}{See All
Episodes ofThe Argument}

Next

Sept. 3, 2020

\begin{itemize}
\item
\item
\item
\item
\item
\end{itemize}

\emph{\textbf{Listen and subscribe to ``The Argument'' from your mobile
device:}}

\textbf{\href{https://itunes.apple.com/us/podcast/the-argument/id1438024613?mt=2}{\emph{Apple
Podcasts}}} \emph{\textbf{\textbar{}}}
\textbf{\href{https://open.spotify.com/show/6bmhSFLKtApYClEuSH8q42}{\emph{Spotify}}}
\emph{\textbf{\textbar{}}}
\textbf{\href{https://play.google.com/music/m/Idxib4hsg3yviao4gtym76knjjy?t=The_Argument}{\emph{Google
Play}}} \emph{\textbf{\textbar{}}}
\textbf{\href{https://radiopublic.com/the-argument-Wdbepr}{\emph{RadioPublic}}}
\emph{\textbf{\textbar{}}}
\textbf{\href{https://www.stitcher.com/podcast/the-new-york-times/the-argument}{\emph{Stitcher}}}
\emph{\textbf{\textbar{}}}
\textbf{\href{https://rss.art19.com/the-argument}{\emph{RSS Feed}}}

The Trump re-election strategy has revealed itself: Cast American cities
as hotbeds of chaos, and place the blame entirely on the Democratic
Party. Yet why is unrest being seen as a weakness for Joe Biden, and not
the man in charge? Is the media unthinkingly accepting a Republican
narrative?

This week on the podcast, Frank, Michelle and Ross argue about the
protests and counterprotests in Portland, Ore., and Kenosha, Wis., and
disagree over the politicization of the clashes. They debate the lines
between vigilantism and rioting and discuss the role coverage plays in
the perception of the violence.

And as a fitting parting gift on his final episode of ``The Argument,''
Frank recommends a short story that goes from jocular to chest-gripping
grief in just 10 pages.

\includegraphics{https://static01.graylady3jvrrxbe.onion/images/2020/09/05/opinion/03argumentWeb/merlin_176044272_ee454184-96e5-4851-8554-8932321aa633-articleLarge.jpg?quality=75\&auto=webp\&disable=upscale}

\begin{center}\rule{0.5\linewidth}{\linethickness}\end{center}

\textbf{Background Reading:}

\begin{itemize}
\item
  Ross on what it would take for a
  \href{https://www.nytimes3xbfgragh.onion/2020/07/21/opinion/trump-polls-election-2020.html}{Trump
  comeback} and the
  \href{https://www.nytimes3xbfgragh.onion/2020/05/30/opinion/sunday/riots-george-floyd.html}{case
  against riots}
\item
  Michelle on
  \href{https://www.nytimes3xbfgragh.onion/2020/08/31/opinion/trump-biden-violence.html}{Trump's
  re-election strategy of urging mayhem} and
  \href{https://www.nytimes3xbfgragh.onion/2020/07/02/opinion/trump-racism-2020-election.html}{white
  grievance}
\item
  Frank on the
  \href{https://www.nytimes3xbfgragh.onion/2020/08/26/opinion/trump-melania-rnc.html}{Republican
  convention} and
  \href{https://www.nytimes3xbfgragh.onion/2020/08/28/opinion/sunday/trump-rnc-speech.html}{Trump's
  performance}
\item
  Martin Luther King Jr.:
  ``\href{https://www.cbsnews.com/news/mlk-a-riot-is-the-language-of-the-unheard/}{A
  riot is the language of the unheard}''
\item
  Amy Hempel's short story
  ``\href{http://fictionaut.com/stories/amy-hempel/in-the-cemetery-where-al-jolson-is-buried.pdf}{In
  the Cemetery Where Al Jolson Is Buried}''
\end{itemize}

\begin{center}\rule{0.5\linewidth}{\linethickness}\end{center}

\textbf{How to listen to ``The Argument'':}

\emph{Press play or read the transcript (available by midday Thursday
above the center teal eye) at the top of this page, or tune in on}
\href{https://itunes.apple.com/us/podcast/the-argument/id1438024613?mt=2}{\emph{iTunes}}\emph{,}
\href{https://play.google.com/music/listen?u=0\#/ps/Idxib4hsg3yviao4gtym76knjjy}{\emph{Google
Play}}\emph{,}
\href{https://open.spotify.com/episode/5fIsHqqunLBwoxPSUUSGre?si=Rz5D9VnlRFKdGMu8ixzBOw}{\emph{Spotify}}\emph{,}
\href{https://www.stitcher.com/podcast/the-new-york-times/the-argument}{\emph{Stitcher}}
\emph{or your preferred podcast listening app. Tell us what you think
at} \href{mailto:argument@NYTimes.com}{\emph{argument@NYTimes.com.}}

\begin{center}\rule{0.5\linewidth}{\linethickness}\end{center}

\hypertarget{meet-the-hosts}{%
\section{Meet the Hosts}\label{meet-the-hosts}}

\hypertarget{frank-bruni}{%
\subsection{Frank Bruni}\label{frank-bruni}}

Image

I've been an Op-Ed columnist for The Times since 2011, but my career
with the newspaper stretches back to 1995 and includes many twists and
turns that reflect my embarrassingly scattered interests. I covered
Congress, the White House and several political campaigns; I also spent
five years in the role of chief restaurant critic. As the Rome bureau
chief, I reported on the Vatican; as a staff writer for The Times's
Sunday magazine, I wrote many celebrity profiles. That jumble has
informed my various books, which focus on the Roman Catholic Church,
George W. Bush, my strange eating life, the college admissions process
and meatloaf. Politically, I'm grief-stricken over the way President
Trump has governed and I'm left of center, but I don't think that the
center is a bad place or ``compromise'' a dirty word. I'm
Italian-American, I'm gay and I write a
\href{https://www.nytimes3xbfgragh.onion/newsletters/frank-bruni}{weekly
Times newsletter} in which you'll occasionally encounter my dog, Regan,
who has the run of our Manhattan apartment.
\href{https://twitter.com/FrankBruni}{\emph{@FrankBruni}}

\hypertarget{ross-douthat}{%
\subsection{Ross Douthat}\label{ross-douthat}}

Image

I've been an Op-Ed columnist since 2009, and I write about politics,
religion, pop culture, sociology and the places where they intersect.
I'm a Catholic and a conservative, in that order, which means that I'm
against abortion and critical of the sexual revolution, but I tend to
agree with liberals that the Republican Party is too friendly to the
rich. I was against Donald Trump in 2016 for reasons specific to Donald
Trump, but in general I think the populist movements in Europe and
America have legitimate grievances and I often prefer the populists to
the ``reasonable'' elites. I've written books about Harvard, the G.O.P.,
American Christianity and Pope Francis, and decadence. Benedict XVI was
my favorite pope. I review movies for National Review and have strong
opinions about many prestige television shows. I have four small
children, three girls and a boy, and live in New Haven with my wife.
\href{https://twitter.com/DouthatNYT}{\emph{@DouthatNYT}}

\hypertarget{michelle-goldberg}{%
\subsection{Michelle Goldberg}\label{michelle-goldberg}}

Image

I've been an Op-Ed columnist at The New York Times since 2017, writing
mainly about politics, ideology and gender. These days people on the
right and the left both use ``liberal'' as an epithet, but that's
basically what I am, though the nightmare of Donald Trump's presidency
has radicalized me and pushed me leftward. I've written three books,
including one, in 2006, about the danger of right-wing populism in its
religious fundamentalist guise. (My other two were about the global
battle over reproductive rights and, in a brief detour from politics,
about an adventurous Russian émigré who helped bring yoga to the West.)
I love to travel; a long time ago, after my husband and I eloped, we
spent a year backpacking through Asia. Now we live in Brooklyn with our
son and daughter.
\href{https://twitter.com/michelleinbklyn}{\emph{@michelleinbklyn}}

``The Argument'' is a production of The New York Times Opinion section.
The team includes Phoebe Lett, Vishakha Darbha, Kathy Tu, Kristin Lin,
Paula Szuchman and Isaac Jones. Theme by Allison Leyton-Brown.

Advertisement

\protect\hyperlink{after-bottom}{Continue reading the main story}

\hypertarget{site-index}{%
\subsection{Site Index}\label{site-index}}

\hypertarget{site-information-navigation}{%
\subsection{Site Information
Navigation}\label{site-information-navigation}}

\begin{itemize}
\tightlist
\item
  \href{https://help.nytimes3xbfgragh.onion/hc/en-us/articles/115014792127-Copyright-notice}{©~2020~The
  New York Times Company}
\end{itemize}

\begin{itemize}
\tightlist
\item
  \href{https://www.nytco.com/}{NYTCo}
\item
  \href{https://help.nytimes3xbfgragh.onion/hc/en-us/articles/115015385887-Contact-Us}{Contact
  Us}
\item
  \href{https://www.nytco.com/careers/}{Work with us}
\item
  \href{https://nytmediakit.com/}{Advertise}
\item
  \href{http://www.tbrandstudio.com/}{T Brand Studio}
\item
  \href{https://www.nytimes3xbfgragh.onion/privacy/cookie-policy\#how-do-i-manage-trackers}{Your
  Ad Choices}
\item
  \href{https://www.nytimes3xbfgragh.onion/privacy}{Privacy}
\item
  \href{https://help.nytimes3xbfgragh.onion/hc/en-us/articles/115014893428-Terms-of-service}{Terms
  of Service}
\item
  \href{https://help.nytimes3xbfgragh.onion/hc/en-us/articles/115014893968-Terms-of-sale}{Terms
  of Sale}
\item
  \href{https://spiderbites.nytimes3xbfgragh.onion}{Site Map}
\item
  \href{https://help.nytimes3xbfgragh.onion/hc/en-us}{Help}
\item
  \href{https://www.nytimes3xbfgragh.onion/subscription?campaignId=37WXW}{Subscriptions}
\end{itemize}
