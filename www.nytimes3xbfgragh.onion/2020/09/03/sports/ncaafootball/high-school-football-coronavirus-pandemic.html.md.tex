Sections

SEARCH

\protect\hyperlink{site-content}{Skip to
content}\protect\hyperlink{site-index}{Skip to site index}

\href{/section/sports/ncaafootball}{College
Football}\textbar{}Scrambling Across State Lines to Play During the
Pandemic

\url{https://nyti.ms/3jKemlr}

\begin{itemize}
\item
\item
\item
\item
\item
\item
\end{itemize}

\includegraphics{https://static01.graylady3jvrrxbe.onion/images/2020/09/04/sports/04virus-hsfootball-12/merlin_176293866_159fe97d-2fdd-4d4d-a485-4730c8299585-articleLarge.jpg?quality=75\&auto=webp\&disable=upscale}

\hypertarget{scrambling-across-state-lines-to-play-during-the-pandemic}{%
\section{Scrambling Across State Lines to Play During the
Pandemic}\label{scrambling-across-state-lines-to-play-during-the-pandemic}}

Some high schools have postponed football seasons. Some are forging
ahead. And others are in limbo, with young athletes desperate to find a
way to avoid a lost season.

Credit...Christopher Smith for The New York Times

Supported by

\protect\hyperlink{after-sponsor}{Continue reading the main story}

\href{https://www.nytimes3xbfgragh.onion/by/david-waldstein}{\includegraphics{https://static01.graylady3jvrrxbe.onion/images/2018/02/20/multimedia/author-david-waldstein/author-david-waldstein-thumbLarge.jpg}}

By \href{https://www.nytimes3xbfgragh.onion/by/david-waldstein}{David
Waldstein}

\begin{itemize}
\item
  Sept. 3, 2020
\item
  \begin{itemize}
  \item
  \item
  \item
  \item
  \item
  \item
  \end{itemize}
\end{itemize}

Mario Sanchez thought his only option was to leave.

Covid-19 had claimed the life of his grandfather. It led his father, who
tested positive for the coronavirus last month, to isolate for 14 days
in the basement of the family home in Olathe, Kan. And it ended
Sanchez's dream of winning a state championship with his lifelong
friends at Olathe North High School, after its football program was
suspended amid high rates of coronavirus infections in the area.

So on Sunday, just two days after attending his grandfather's funeral,
Sanchez and his mother, Noemi Jurado, packed up their Honda Crosstour
and drove from their home in a suburb of Kansas City, Mo., to Norman,
Okla.

A slot receiver and defensive back, Sanchez plans to play his senior
year at Norman High School, about 350 miles from where he grew up. He is
among a handful of players from Olathe North to cross state lines to
enroll at a different high school, an interstate football migration
fueled by young athletes attempting to outrun the coronavirus and
preserve their athletic dreams.

``It really wasn't a hard process,'' Sanchez said in a telephone
interview before leaving home. ``I want to play football in college. I'm
looking around, and we are in the red zone in Kansas. I don't want to
risk my future by staying here without playing.''

The coronavirus pandemic has had an uneven impact on high school
football across the United States, causing havoc in some regions while
schools in other areas have made modifications and forged ahead, almost
as normal.

\includegraphics{https://static01.graylady3jvrrxbe.onion/images/2020/09/04/sports/04virus-hsfootball-1/merlin_176293722_971ae2ea-beba-49e7-9df4-b3862e000d78-articleLarge.jpg?quality=75\&auto=webp\&disable=upscale}

Teams in Utah, Alabama, Texas and other states have already played their
first games of the season. In Minnesota, six players on the
Lewiston-Altura High School varsity tested positive for the virus before
the state shifted football to the spring. In DeKalb County, Ind., an
entire team waits in quarantine after one player tested positive, and in
Kings Mills, Ohio, Kings High School had its first game canceled after a
player tested positive.

No state has canceled its entire football season --- or any sport ---
for the 2020-21 academic year, according to the National Federation of
State High School Associations, but 16 states, plus the District of
Columbia, have rescheduled football for the spring or winter instead of
its traditional schedule in the fall.

Other states, like Kansas, where high school football is embedded in the
cultural fabric of the region, are caught in the middle, where
uncertainty reigns.

Several Kansas school districts have ordered their football programs off
the field, while teams from neighboring towns play on. Last week, the
Kansas State High Schools Activities Association voted to allow schools
to move their seasons to the spring, but that brought another set of
complications, especially for multisport athletes or schools with
limited fields and facilities.

``It's all over the map,'' Karissa Niehoff, the executive director of
the national federation, which offers nonbinding guidelines to state
athletic associations, said recently in an interview. ``For many
programs that have re-engaged, there has been more of a successful
experience than an unsuccessful experience. But we are paying close
attention to the feedback over the next week or two.''

In some Kansas school districts, teams are allowed to practice as long
as the rate of virus infections in their county stays below a certain
level. Olathe North, considered a top contender to win the state title
before the pandemic hit, sits in Johnson County, Kan., a largely
affluent collection of suburbs across the state border from Kansas City,
Mo. About 11 percent of coronavirus tests in the county have been
positive over the last two weeks.

When the positivity rate creeps above 10 percent, sports considered high
risk, including football, are halted in Olathe, even after teams have
been practicing for weeks.

If the positivity rate is from 5 to 10 percent, teams can practice but
not play games. And since teams are required to complete 14 days of
practice before competition, some games have already been canceled.

Image

Sanchez with his parents in Olathe last week. His father isolated in the
family basement for 14 days after testing positive for the
coronavirus.~Credit...Christopher Smith for The New York Times

That was exactly what Sanchez feared. About a month ago, he began
pleading with his mother to move to Norman, where her father lives.
Initially, she was skeptical of the idea, in part because the past year
had been a challenging transition period for their family.

Sanchez's father was released from prison in May 2019 after serving 13
years on a drug trafficking charge. Mario was 3 when his father went
away, and Jurado, a real estate broker, handled the bulk of the
parenting. When Mario's father first came back home, there was friction
and disagreement between a teenage son and his newly returned father.
Now, Mario reports, they are ``tighter than a knot.''

But then Mario's grandfather, Richard Sanchez, whom Jurado called the
rock of the family, died of complications from Covid-19 on Aug. 13. By
then, with his high school's season in doubt, Mario was looking south to
Norman, where he could play football, basketball and baseball (he is a
shortstop), and his mother could spend time with her father.

Image

Sanchez with his nephews at a family gathering last week after the
funeral for his grandfather, who died of complications from
Covid-19.~Credit...Christopher Smith for The New York Times

``I told him, `If you're willing to make that sacrifice, I'll make it
with you,''' she said. ``Covid-19 has really impacted everyone in a
devastating way. It's been a tough run for us.''

Making the decision easier was the fact that Sanchez's best friend since
grade school, Arland Bruce IV, was doing the same thing. A star
quarterback at Olathe North and a cousin of the former N.F.L. receiver
Isaac Bruce, Arland Bruce IV enrolled at Ankeny High School in Iowa last
month.

\hypertarget{sports-and-the-virus}{%
\subsubsection{Sports and the Virus}\label{sports-and-the-virus}}

\paragraph{}

Updated Sept. 4, 2020

Here's what's happening as the world of sports slowly comes back to
life:

\begin{itemize}
\item
  \begin{itemize}
  \tightlist
  \item
    The 146th running of the Kentucky Derby, which was moved to Saturday
    from May 2, will have
    \href{https://www.nytimes3xbfgragh.onion/2020/09/04/sports/horse-racing/kentucky-derby-odds-picks.html?action=click\&pgtype=Article\&state=default\&region=MAIN_CONTENT_2\&context=storylines_keepup}{no
    spectators present because of the coronavirus pandemic}.
  \item
    The coronavirus pandemic has had an
    \href{https://www.nytimes3xbfgragh.onion/2020/09/03/sports/ncaafootball/high-school-football-coronavirus-pandemic.html?action=click\&pgtype=Article\&state=default\&region=MAIN_CONTENT_2\&context=storylines_keepup}{uneven
    impact on high school football}~across the United States.
  \item
    The
    \href{https://www.nytimes3xbfgragh.onion/2020/09/02/sports/ncaafootball/coronavirus-cal-athletics-season.html?action=click\&pgtype=Article\&state=default\&region=MAIN_CONTENT_2\&context=storylines_keepup}{most
    complicated puzzle in sports is the return of college
    athletics}~during a pandemic. The University of California, Berkeley
    is allowing The Times an inside look at their journey's ups and
    downs.
  \end{itemize}
\end{itemize}

But in a development that underscored the confusion surrounding the
whole season,
\href{https://www.hawkcentral.com/story/sports/high-school/2020/09/03/iowa-hawkeyes-recruit-arland-bruce-iv-appeal-court-recommends-ihsaa-rule-him-eligible-iahsfb/3452537001/}{Bruce
was ruled ineligible} by the Iowa High School Athletic Association on
the day of Ankeny's first game, and his family has hired a lawyer to
appeal the ruling.

Sanchez is optimistic he will avoid such a mess. He arrived in Norman on
Sunday evening and began practicing with the team the next day.

Back in Kansas, Chris McCartney, Bruce and Sanchez's former coach at
Olathe North, is left behind, clicking the refresh button on the Johnson
County health department's website ``several times a day,'' he said,
hoping to see the positive test rates in the county drop low enough so
the team can play.

Image

Chris McCartney, the football coach at Olathe North, doesn't know if or
when his team will be able to play games this
season.Credit...Christopher Smith for The New York Times

Olathe North was shut down for a week, but on Monday, the day after
Sanchez left for Oklahoma, the team was given permission to practice.
Whether it can play games is still unknown.

``This has been really tough on the kids,'' McCartney said. ``At this
point, we're just looking for a yes or no answer. Are we playing or
not?''

While each school district in Johnson County can determine its own
course, the health department there has established guidelines that most
of them follow. That makes Sanmi Areola, the Johnson County director of
health, an unpopular figure among those who want high school football at
all costs.

Areola said he also wanted children to play sports for the exercise,
socialization and lessons in teamwork and structure --- as long as it
was safe.

In addition to worrying about the spread of the virus, he is also
concerned about myocarditis, a heart inflammation that can lead to
cardiac arrest with exertion, which in one survey
\href{https://www.nytimes3xbfgragh.onion/2020/08/23/sports/ncaafootball/college-football-myocarditis-coronavirus.html}{was
found in 15 percent of college athletes} who had the virus. The science
around myocarditis is still evolving, and there is scant data on high
school athletes.

Image

McCartney's team was hoping to avenge a loss in last year's state
championship game.~Credit...Christopher Smith for The New York Times

``There is some evidence that there can be some cardiac effects,''
Areola said. ``What that translates to when you have high intensity
activities, no one is quite sure yet. Hence the need to be very, very
careful.''

High school coaches insist they are being careful and have implemented
protective modifications. One of McCartney's players tested positive
after a family gathering, McCartney said, and a few more had to be
quarantined. But the contagion never spread to other team members, he
said.

``To me, it showed we can handle this,'' he said. ``We are outside,
we're social distancing, we're masking up. To see it taken away from
them is heartbreaking.''

What makes it even more painful is that Olathe North lost in last year's
state championship game. This year, the players were determined to win
it all, at least before practices were halted and some of the best ones
left for other states.

They have finally resumed practices, but they know at least two games
have been canceled, and perhaps more. And there is no guarantee they
won't be shut down again.

Image

Weston Moore has continued training with teammates while his high school
is not allowed to conduct official practices.~~Credit...Christopher
Smith for The New York Times

A dozen miles from Olathe, Weston Moore has been working out harder than
ever, in preparation for his senior year at Shawnee Mission West High
School, also in Johnson County. This was to be Moore's big year, as the
starting quarterback of the Vikings, something he had dreamed of since
he was in first grade. Still, he understands that safety comes first.

``But at the same time, it's your senior year, what you've been working
so hard for all these years, and it might not even happen,'' he said.
``It's so disappointing.''

Despite organized practices being shut down, Moore has been spinning
perfect spirals to teammates in shorts and T-shirts at the school field,
all while friends at other schools --- some just five miles away, across
the border in Missouri --- prepared for their first games without fans
in the stadiums. And 40 miles west, Free State High School in Douglas
County, Kan., is also allowed to play.

Image

``It's so disappointing,'' Moore said of the prospect of a canceled
season.Credit...Christopher Smith for The New York Times

But the pandemic has still threatened Free State's season. Kevin
Stewart, the Firebirds' head coach, said that six of the eight teams on
their schedule, including Olathe North, had canceled games. He scrambled
to find replacement opponents, securing two so far, but the season
remains murky.

``Team morale is less enthusiastic,'' he said. ``We've lost a little bit
of energy.''

Stewart said that if his program was shut down, too, he would worry what
a few of his players might do with their spare time. They need the
structured supervision of football, he said.

Sanchez needs football this year, too, he said. He needs it to earn a
scholarship and get into college. As a sophomore defensive back, he led
his league in interceptions, with seven, but last year he was injured
and could sense fans and opponents wondering if his previous season was
a fluke.

This season, he was aiming to prove them wrong. To do so, he has to
play. But even in Oklahoma, more than 100 school districts have reported
a positive test result. Still, Sanchez remains confident his team will
play. He has to.

``This is a big step for me,'' he said. ``I won't be able to see my
family and friends for a while. But I'm starting a new journey.''

Advertisement

\protect\hyperlink{after-bottom}{Continue reading the main story}

\hypertarget{site-index}{%
\subsection{Site Index}\label{site-index}}

\hypertarget{site-information-navigation}{%
\subsection{Site Information
Navigation}\label{site-information-navigation}}

\begin{itemize}
\tightlist
\item
  \href{https://help.nytimes3xbfgragh.onion/hc/en-us/articles/115014792127-Copyright-notice}{©~2020~The
  New York Times Company}
\end{itemize}

\begin{itemize}
\tightlist
\item
  \href{https://www.nytco.com/}{NYTCo}
\item
  \href{https://help.nytimes3xbfgragh.onion/hc/en-us/articles/115015385887-Contact-Us}{Contact
  Us}
\item
  \href{https://www.nytco.com/careers/}{Work with us}
\item
  \href{https://nytmediakit.com/}{Advertise}
\item
  \href{http://www.tbrandstudio.com/}{T Brand Studio}
\item
  \href{https://www.nytimes3xbfgragh.onion/privacy/cookie-policy\#how-do-i-manage-trackers}{Your
  Ad Choices}
\item
  \href{https://www.nytimes3xbfgragh.onion/privacy}{Privacy}
\item
  \href{https://help.nytimes3xbfgragh.onion/hc/en-us/articles/115014893428-Terms-of-service}{Terms
  of Service}
\item
  \href{https://help.nytimes3xbfgragh.onion/hc/en-us/articles/115014893968-Terms-of-sale}{Terms
  of Sale}
\item
  \href{https://spiderbites.nytimes3xbfgragh.onion}{Site Map}
\item
  \href{https://help.nytimes3xbfgragh.onion/hc/en-us}{Help}
\item
  \href{https://www.nytimes3xbfgragh.onion/subscription?campaignId=37WXW}{Subscriptions}
\end{itemize}
