Sections

SEARCH

\protect\hyperlink{site-content}{Skip to
content}\protect\hyperlink{site-index}{Skip to site index}

\href{https://www.nytimes3xbfgragh.onion/section/arts/design}{Art \&
Design}

\href{https://myaccount.nytimes3xbfgragh.onion/auth/login?response_type=cookie\&client_id=vi}{}

\href{https://www.nytimes3xbfgragh.onion/section/todayspaper}{Today's
Paper}

\href{/section/arts/design}{Art \& Design}\textbar{}The Whitney Reopens
With 3 Powerhouse Shows

\url{https://nyti.ms/34ZUBSJ}

\begin{itemize}
\item
\item
\item
\item
\item
\item
\end{itemize}

Advertisement

\protect\hyperlink{after-top}{Continue reading the main story}

Supported by

\protect\hyperlink{after-sponsor}{Continue reading the main story}

Critics' Notebook

\hypertarget{the-whitney-reopens-with-3-powerhouse-shows}{%
\section{The Whitney Reopens With 3 Powerhouse
Shows}\label{the-whitney-reopens-with-3-powerhouse-shows}}

When the museum closed in March, it was showing Agnes Pelton's paintings
and Mexico's great muralists. Thankfully, these works are still up on
the walls.

\includegraphics{https://static01.graylady3jvrrxbe.onion/images/2020/09/03/arts/03whitney-reopen1/merlin_170076321_08cf9aec-13f1-483b-bad9-9b8df1089f04-articleLarge.jpg?quality=75\&auto=webp\&disable=upscale}

By \href{https://www.nytimes3xbfgragh.onion/by/roberta-smith}{Roberta
Smith},
\href{https://www.nytimes3xbfgragh.onion/by/holland-cotter}{Holland
Cotter} and Siddhartha Mitter

\begin{itemize}
\item
  Sept. 3, 2020
\item
  \begin{itemize}
  \item
  \item
  \item
  \item
  \item
  \item
  \end{itemize}
\end{itemize}

\emph{The Whitney Museum of American Art is reopening on Thursday, with
new safety guidelines that will require visitors to purchase timed
tickets in advance. By the time the museum announced its closure in
March, our critics had reviewed two remarkable shows: the first New York
museum exhibition of the still-mysterious painter Agnes Pelton; and a
grand retrospective of the Mexican muralists Diego Rivera, José Clemente
Orozco and David Alfaro Siqueiros.}

\emph{Below is an overview of those reviews, plus insights into another
strong show at the Whitney, ``Cauleen Smith: Mutualities.''}

\hypertarget{agnes-pelton-desert-transcendentalist}{%
\subsection{`Agnes Pelton: Desert
Transcendentalist'}\label{agnes-pelton-desert-transcendentalist}}

\includegraphics{https://static01.graylady3jvrrxbe.onion/images/2020/09/04/arts/03whitney-reopen2/merlin_170306019_052c73d3-4b87-40dd-bca9-5e8323625001-articleLarge.jpg?quality=75\&auto=webp\&disable=upscale}

\href{https://www.nytimes3xbfgragh.onion/2020/03/12/arts/design/agnes-pelton-review-whitney-museum.html}{This
survey}, extended through Nov. 1, presents the underappreciated but
inimitable art of the American painter Agnes Pelton (1881-1961). It also
offers a reminder that the history of modernist abstraction, and women's
contribution to it, is still being written.

Image

A 1957 portrait of Pelton.Credit...Carolyn Tilton Cunningham Family; via
Nyna Dolby

Pelton's exquisitely finished, otherworldly abstractions are the stuff
of dreams, visions and mirages; they often came to the artist while she
slept or meditated, and they arrived remarkably whole, as indicated by
the sketches from her journal reproduced in the catalog, which
originated, with the show, at the Phoenix Art Museum. (It was organized
by Gilbert Vicario, chief curator there, and overseen at the Whitney by
Barbara Haskell, with Sarah Humphreville.)

There is nothing quite like Pelton's paintings in 20th-century American
art. It is not just their much-admired spirituality that distinguishes
them --- their blend of theosophy, Buddhism, astrology and the occult
was not unusual among artists of the moment. It is rather the insouciant
ease with which her images navigate between high and low, making that
spirituality widely available, if not irresistible.

Image

``Storm Oil'' (1932), oil on canvas, by Pelton.Credit...Crystal Bridges
Museum of American Art

Pelton belonged to the first generation of American Modernists --- which
included Georgia O'Keeffe, Marsden Hartley and Arthur Dove --- but not
to their circle, which revolved around the advocacy and galleries of the
impresario Alfred Stieglitz. Her mature style arrived after a series of
efforts from the mid-1920s that read as mildly visionary Cubo-Futurist
motifs: frazzled flowers and an incandescent fountain.

In the 1929 work ``Star Gazer,'' a multicolored bud stands like a
pilgrim, offering itself to an azure vase, behind which brilliant red
hills soften into the distance. A single star reinforces the symmetry of
the scene. And then she does it again and again in deliriously perfect
paintings like ``Sand Storm'' and ``Messengers'' (both from 1932) and
``Even Song,'' from 1934, in which an immense vase aglow with inner fire
releases tendrils of smoke, flanked by two white shapes reminiscent of
O'Keeffe cattle skulls.

After the last Pelton retrospective, 25 years ago, her achievement
receded from view. That seems unlikely this time. The Whitney show
underscores too tellingly the lesson of
\href{https://www.nytimes3xbfgragh.onion/2018/10/11/arts/design/hilma-af-klint-review-guggenheim.html}{the
Guggenheim's Hilma af Klint exhibition}, that the largely all-male
narrative of modernist abstraction needs reworking, with much more
credit to female artists and their implicitly feminist embrace of
spirituality. Let's put it this way: Hilma af Klint and Agnes Pelton did
not act alone. \emph{ROBERTA SMITH}

\hypertarget{vida-americana-mexican-muralists-remake-american-art-1925-1945}{%
\subsection{`Vida Americana: Mexican Muralists Remake American Art,
1925-1945'}\label{vida-americana-mexican-muralists-remake-american-art-1925-1945}}

Image

An installation view of ``Vida Americana,'' from left: a reproduction of
Diego Rivera's ``Man, Controller of the Universe''; Hugo Gellert's ``Us
Fellas Gotta Stick Together, or the Last Defenses of Capitalism''; and
Ben Shahn's ``The Passion of Sacco and Vanzetti.''Credit...Emiliano
Granado for The New York Times

This \href{https://whitney.org/exhibitions/vida-americana}{exhibition},
on view through Jan. 31, represents a decade of hard thought and labor,
and that effort has paid off. The show is stupendous and complicated,
and lands right on time. Just by existing, it does three vital things:
It reshapes a stretch of art history to give credit where credit is due;
it suggests that the Whitney is, at last, on the way to fully embracing
American art; and it offers yet another argument for why this country's
build-the-wall mania has to go. Judging by the story told here, we
should be actively inviting our southern neighbor to enrich our cultural
soil.

That story begins in Mexico in the 1920s. After 10 years of civil war
and revolution, the country's new government turned to art to invent and
broadcast a unifying national self-image, one that emphasized both its
deep roots in Indigenous, pre-Hispanic culture and the heroism of its
recent revolutionary struggles.

The chosen medium for the message was mural painting, and three very
differently gifted practitioners quickly came to dominate the field:
Diego Rivera, José Clemente Orozco and David Alfaro Siqueiros: ``Los
Tres Grandes'' --- ``the three great ones'' --- as they came to be known
among admirers.

Image

``The Malinche (Young Girl of Yalala, Oaxaca),'' by Alfredo Ramos
Martínez.Credit...Emiliano Granado for The New York Times

The exhibition's opening gallery suggest a fiesta atmosphere, as do the
paintings gathered there: Alfredo Ramos Martínez's 1929 image of an
itinerant flower vendor; a 1928 painting by Rivera of Oaxacan dancers in
orchidaceous gowns; and, from the same year, a scene, in Rivera's
smooth-brushed, Paris-trained style, of women harvesting cactus by the
American artist Everett Gee Jackson. (Barbara Haskell is the show's
originating curator, joined by Marcela Guerrero, Sarah Humphreville and
Alana Hernandez.)

It was important for a nation that identified itself with populist
struggle to keep the memory of that struggle burning. You see this in a
large charcoal painting study by Rivera of the revolutionary Emiliano
Zapata trampling an enemy underfoot. And in an inky Siqueiros portrait
of the same leader, looking as blank-eyed as a corpse. And in a spiky,
depressed Orozco painting of the peasant guerrillas known as Zapatistas,
their figures as stiff as the machetes they carry, locked in a grim
forced march.

Image

A reproduction of José Clemente Orozco's ``Prometheus.''Credit...Artists
Rights Society (ARS), New York/SOMAAP, Mexico City; Emiliano Granado for
The New York Times

By the time these pictures were made in 1931, two of the artists were
working primarily in the United States; Siqueiros would arrive the next
year. Orozco came first, to New York in 1927. There he taught easel
painting and printmaking to a rapt cohort of local artists before moving
on to California to execute a mural commission for Pomona College in
Claremont --- a 1930 fresco called ``Prometheus'' that the teenage
Jackson Pollock, then living in Los Angeles, saw and never forgot.

The exhibition's final gallery is basically a Siqueiros-Pollock
showcase. It's set in New York, where, beginning in 1936, the two
artists worked together as teacher and student. We see examples of the
anti-conventional techniques the muralist developed: spraying,
splattered, dripping paint --- anything to make the results look
unpolished and unsettling. And we see Pollock beginning to test out
these unorthodoxies. It's clear that even in the 1930s, he was on fire.
And the evidence is that Siqueiros held the igniting match.

Did influence run both ways? Student to teacher? South to north and
back? Undoubtedly. The result at the Whitney is a study in
multidirectional flow, tides meeting and mingling, which is the basic
dynamic of art history, as it is, or should be, of American life. It's a
dynamic of generosity. It gives the show warmth and grandeur. Why on
earth would we want to stop the flow now? \emph{HOLLAND COTTER}

\hypertarget{cauleen-smith-mutualities}{%
\subsection{`Cauleen Smith:
Mutualities'}\label{cauleen-smith-mutualities}}

Image

A still from ``Sojourner'' (2018), by Cauleen Smith.Credit...Cauleen
Smith, Corbett vs. Dempsey, Chicago, and Kate Werble Gallery, New York

The sumptuous 22-minute film ``Sojourner'' anchors this presentation of
recent work by the Los Angeles-based artist Cauleen Smith, on view
through Jan. 31. Though made in 2018, it's a perfect piece to provide
solace, perspective and inspiration amid the fraught situation of
America today.

The film opens in North Philadelphia, with grainy footage of horses on
an empty lot and a rowhouse where John Coltrane lived in the 1950s. Soon
it takes in a Shaker cemetery in upstate New York, a community arts
center and an activist rally on Chicago's South Side and multiple
California spots --- a beach, poppy fields, the Watts Towers, the ashram
founded by Alice Coltrane as Swamini Turiyasangitananda.

The camera settles into an extended sequence filmed in intense desert
light at the
\href{http://www.noahpurifoy.com/joshua-tree-outdoor-museum}{found-object
sculpture garden} built by the artist Noah Purifoy in Joshua Tree,
Calif.There, a dozen women wearing exuberant Afro-Bohemian styles tune
into astral signals on an old radio, clasp hands to a reading of the
\href{https://www.newyorker.com/news/our-columnists/until-black-women-are-free-none-of-us-will-be-free}{Combahee
River Collective Statement} and process, bearing banners, toward a final
exalted tableau.

Image

The 22-minute film is on view at the Whitney through Jan.
31.Credit...Cauleen Smith, Corbett vs. Dempsey, Chicago, and Kate Werble
Gallery, New York

The film's coherence owes to its underpinning theme: how visionary
practice overflows the boundaries of art, spirituality and politics, and
gathers all these together when they're exercised with generosity.
Narrations of texts by Rebecca Cox Jackson, a 19th-century Black Shaker
eldress, and words and music by Alice Coltrane are crucial to the weave.
But this cumulative tour de force belongs to Ms. Smith, an experimental
filmmaker at the pinnacle of her craft who has brilliantly paced many
elements into a resonant journey in which the trace of fellow seekers,
past and present, ultimately leads to freedom.

The Whitney has installed ``Sojourner'' appropriately, with its own room
and a large screen. Another Smith film, ``Pilgrim,'' which revisits the
ashram and the Shaker site in a melancholy register, suffers from its
placement in a corridor that leads to the terrace. Also on view are Ms.
Smith's drawings of book covers; in this series, ``Firespitters,'' she
celebrates books by some of her favorite poets and others that they, in
turn, have recommended. A big show developing Ms. Smith's vision across
mediums --- film, installation, performance, textile --- is desperately
overdue in New York. But ``Sojourner,'' a masterpiece, is essential balm
and ballast for now. \emph{SIDDHARTHA MITTER}

\begin{center}\rule{0.5\linewidth}{\linethickness}\end{center}

\textbf{Whitney Museum of American Art}

99 Gansevoort Street, Manhattan; 212-570-3600,
\href{https://whitney.org/}{whitney.org}. Purchase of timed tickets in
advance is required. (Admission is pay what you wish through Sept. 28.)

Advertisement

\protect\hyperlink{after-bottom}{Continue reading the main story}

\hypertarget{site-index}{%
\subsection{Site Index}\label{site-index}}

\hypertarget{site-information-navigation}{%
\subsection{Site Information
Navigation}\label{site-information-navigation}}

\begin{itemize}
\tightlist
\item
  \href{https://help.nytimes3xbfgragh.onion/hc/en-us/articles/115014792127-Copyright-notice}{©~2020~The
  New York Times Company}
\end{itemize}

\begin{itemize}
\tightlist
\item
  \href{https://www.nytco.com/}{NYTCo}
\item
  \href{https://help.nytimes3xbfgragh.onion/hc/en-us/articles/115015385887-Contact-Us}{Contact
  Us}
\item
  \href{https://www.nytco.com/careers/}{Work with us}
\item
  \href{https://nytmediakit.com/}{Advertise}
\item
  \href{http://www.tbrandstudio.com/}{T Brand Studio}
\item
  \href{https://www.nytimes3xbfgragh.onion/privacy/cookie-policy\#how-do-i-manage-trackers}{Your
  Ad Choices}
\item
  \href{https://www.nytimes3xbfgragh.onion/privacy}{Privacy}
\item
  \href{https://help.nytimes3xbfgragh.onion/hc/en-us/articles/115014893428-Terms-of-service}{Terms
  of Service}
\item
  \href{https://help.nytimes3xbfgragh.onion/hc/en-us/articles/115014893968-Terms-of-sale}{Terms
  of Sale}
\item
  \href{https://spiderbites.nytimes3xbfgragh.onion}{Site Map}
\item
  \href{https://help.nytimes3xbfgragh.onion/hc/en-us}{Help}
\item
  \href{https://www.nytimes3xbfgragh.onion/subscription?campaignId=37WXW}{Subscriptions}
\end{itemize}
