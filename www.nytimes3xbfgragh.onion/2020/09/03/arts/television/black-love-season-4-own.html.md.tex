Sections

SEARCH

\protect\hyperlink{site-content}{Skip to
content}\protect\hyperlink{site-index}{Skip to site index}

\href{https://www.nytimes3xbfgragh.onion/section/arts/television}{Television}

\href{https://myaccount.nytimes3xbfgragh.onion/auth/login?response_type=cookie\&client_id=vi}{}

\href{https://www.nytimes3xbfgragh.onion/section/todayspaper}{Today's
Paper}

\href{/section/arts/television}{Television}\textbar{}`Black Love' Keeps
It Simple: Honesty, Not Antics

\url{https://nyti.ms/2QVnqHy}

\begin{itemize}
\item
\item
\item
\item
\item
\end{itemize}

Advertisement

\protect\hyperlink{after-top}{Continue reading the main story}

Supported by

\protect\hyperlink{after-sponsor}{Continue reading the main story}

\hypertarget{black-love-keeps-it-simple-honesty-not-antics}{%
\section{`Black Love' Keeps It Simple: Honesty, Not
Antics}\label{black-love-keeps-it-simple-honesty-not-antics}}

The documentary-style reality series maintains its focus on Black lives
and relationships while avoiding typical reality TV stunts.

\includegraphics{https://static01.graylady3jvrrxbe.onion/images/2020/09/04/arts/04Black-Love-PRINT2/04Black-Love-PRINT2-articleLarge-v2.jpg?quality=75\&auto=webp\&disable=upscale}

By Leigh-Ann Jackson

\begin{itemize}
\item
  Published Sept. 3, 2020Updated Sept. 8, 2020
\item
  \begin{itemize}
  \item
  \item
  \item
  \item
  \item
  \end{itemize}
\end{itemize}

In an episode from 2018, the actor Glynn Turman sat on a sofa with his
arm draped affectionately around his wife, Jo-An, as the two recounted
their experiences together for the documentary series
``\href{http://www.oprah.com/app/black-love.html}{Black Love}.'' Amid
talk of marriage, children and a meet-cute at Roscoe's House of Chicken
and Waffles, Turman paused to broaden the conversation with
\href{http://www.oprah.com/own-blacklove/glynn-turman-on-the-importance-of-black-love}{an
impassioned plea}.

``We're not angels, we're not saints --- we're human beings,'' he said,
speaking about the perception of Black Americans and their
relationships. ``Let's not leave out any of the wonderful, wonderful
love and the bonds that we, as a people,'' have shared, ``having gone
through an extremely, extremely unique experience in this country.''

``To have us come through it,'' he added, ``with our loved ones, and
what that all entails, is not only important, it's \emph{biblical}.''

On Sept. 5, ``Black Love'' returned to the Oprah Winfrey Network for a
fourth season, at a time when Turman's words and the show's honest
portrayal of Black lives seem even more urgent. But in a year marked by
pandemic and protests over racial injustice, the series also offers
respite and nuance --- an alternative to the relentless imagery of a
Black American experience bounded by anguish and rage.

``We know that Black people are happy and married and have been making
it work for a long time,'' said Tommy Oliver, who created the series
with his wife, Codie Elaine Oliver. The two spoke in a Zoom interview
last month from their home in Los Angeles.

``We need to see it, and we have not seen it,'' Tommy continued. ``It's
been relegated to \ldots{} \emph{nowhere} on TV for the longest.''

Across three seasons, the ``Black Love'' formula has remained as simple
as it has effective. Each episode features clips of various couples,
some famous and some not, at least one of whom (but usually both) is
Black. Couples are filmed side-by-side in their own homes, having frank
conversations about their relationships and delivering tender moments in
which they reminisce, cry, belly-laugh and comfort each other.

Their stories range from goofy to gut-wrenching. Some couples are still
in the honeymoon phase. Others have toasted to their Golden anniversary.
A few interviews focus on the loving bond between a parent and child.

For viewers feeling the pangs of sheltering in place, isolated from
friends and family, there's a sense of familiarity and comfort in
watching these couples discuss sex, parenthood, financial decisions,
divorce scares, infidelities and illnesses. But if the show feels
familiar, it is also unique.

\includegraphics{https://static01.graylady3jvrrxbe.onion/images/2020/09/03/arts/03black-lovea/merlin_176530185_ab911484-e8f1-463c-8c23-c960532f9f29-articleLarge.jpg?quality=75\&auto=webp\&disable=upscale}

``Has there been something exactly like this previously?'' asked Beretta
E. Smith-Shomade, an associate professor at Emory University who studies
race and representation in television. ``For Black folks, most certainly
not,'' she said, adding, ``I think it's tapping into a need that we all
have for connection, particularly, now.''

OWN pointed to the ratings, noting that the series ranked No. 1 in its
Friday time slot last season among African-American women ages 25 to 54.
The network president, Tina Perry, called the show ``a unicorn in the TV
universe,'' and said the Olivers capture stories that are typically
found only in scripted fare.

Season 4 includes the married TV actors Dulé Hill and Jazmyn Simon; the
sports journalist Jemele Hill and her husband, Ian Wallace; and the
comedic
\href{https://www.youtube.com/channel/UC6qABjcDkIUgYFkTKrCEXyg}{YouTubers}
Marcus and Angel Tanksley. A stand-alone special will be devoted to
Karega Bailey and Felicia Gangloff-Bailey, two San Francisco Bay Area
\href{https://soldevelopment.bandcamp.com/}{recording artists} who have
had to navigate the loss of their newborn daughter. Their story and
several others align with this season's focus on mental health.

``There was a concern for us about whether or not this conversation
would be able to hold our story,'' Karega said. ``It is incredibly
difficult to articulate all the nuances of grief. We hope viewers will
be able to gather that grief is love after loss.''

Though unscripted, the series sidesteps the explosive antics that typify
many reality TV franchises. Viewers won't see back-stabbing confessional
interludes. There's no expert aiming to ``fix'' the couples.

``The way we started this, it was meant to be a conversation,'' Codie
said of the series, which she and Tommy began shooting as an independent
feature documentary in 2014, shortly after getting engaged. In part,
they sought advice for themselves. They interviewed friends, colleagues
and acquaintances, soon amassing dozens of interviews --- including with
Viola Davis, Sterling K. Brown and their spouses.

``We came to them saying, `You're our example,''' Codie recalled. ```I
want all of the worst, scariest things that can happen in a marriage,
but I want to know how you got \emph{through} them.'''

The concept had originated in Codie's mind several years before she met
Tommy, when she was single and a graduate student in 2008 at the
University of Southern California. Bleak headlines at the time, noting
that
\href{http://www.nbcnews.com/id/32379727/ns/health-sexual_health/t/marriage-eludes-high-achieving-black-women/\#.X0hXfPhKhTZ}{high-achieving
Black women were less likely to marry} and that
\href{https://www.pewsocialtrends.org/2010/01/19/women-men-and-the-new-economics-of-marriage/}{marriage
among Black people was in decline}, left her fearful of her prospects
for a lasting relationship.

But as she watched then-Senator Barack Obama and his wife, Michelle,
ascend into the national spotlight, she regained hope.

``That was the thing that allowed me to understand how important it was
that Black love be visible,'' she said. ``That's when I decided that I
wanted to create a space where Black love lives.''

When Codie met Tommy in 2013, he was working as a film producer, and the
two soon started working on ``Black Love'' together. Ultimately, they
decided to pitch it as a series and partnered with OWN, which debuted
the show in 2017. (The Olivers own and license the content independently
through their entertainment production company,
\href{https://confluentialcontent.com/}{Confluential Content}.)

The actress Vanessa Bell Calloway and her husband, Tony Calloway,
appeared in the first episode. Without any idea of where the footage
might end up, they contributed to their friends' nascent project, Bell
Calloway said, because ``I think Black love often gets overlooked.''

``Sometimes,'' she continued, ``just seeing Black folks being together
and loving each other, it gives people inspiration.''

The production is as simple as the formula. During interviews, Codie
sits off-camera delivering conversation prompts, and Tommy operates the
camera. The two-person setup, taking place in the subjects' homes, is
the key to teasing out stories that feel genuine, they said.

``It's not about salaciousness,'' Tommy added. ``It's not about
manipulation.''

Image

The concept for ``Black Love'' originated in Codie Elaine Oliver's mind
several years before she met Tommy. She was inspired by Barack and
Michelle Obama.Credit...Philip Cheung for The New York Times

Part of what makes the show special is an ``aura of authenticity'' that
sets it apart from other televised fare, said Ann duCille, a professor
emerita of English at Wesleyan University. ``It's real people --- even
though many of them are actors and entertainers --- talking candidly
about their real lives and loves,'' she said.

``I want to believe that the subjects are indeed telling it as it is,''
she added, ``but here I find I don't care if I'm being snookered.''

Every groundbreaking effort, however, comes with its own challenges. In
her 2018 book, ``Technicolored: Reflections on Race in the Time of TV,''
duCille wrote about the ``burden of representation,'' referring to early
Black television stars who were not allowed to simply act. They were
expected to ``carry the whole history of the race on their backs,'' she
said.

The ``Black Love'' creators are familiar with that pressure. Some
viewers have taken to social media to criticize the show's relatively
low number of interracial and same-sex couples. Others have criticized
their inclusion at all.

Last month, many Twitter and YouTube users condemned a
\href{https://www.youtube.com/watch?v=KLkUpZmVHPE}{minute-long Season 4
teaser} that featured mostly fair-skinned Black women paired with darker
men. Critics said the video reinforced a painful, centuries-old
prejudice that treats darker-skinned women as less desirable. (Tommy
acknowledged that they had ``screwed up'' with the teaser, explaining
that a wider range of skin tones would be evident throughout the season,
apparent in \href{https://www.youtube.com/watch?v=O7iivDmUI4w}{a longer
trailer} released several days later.)

The Olivers said they would continue to find and feature Black stories
using their many platforms, which, aside from the docu-series, include
editorial and video content on their companion
\href{https://blacklove.com/}{website}.

But they won't feel compelled to do it just because their affirmation of
Black love dovetails with the current Hollywood trend, in which
expressions of support for Black lives can often ring hollow. They'll
continue, Tommy said, because it's what he and Codie have always done as
filmmakers.

``I think the world has become a bit more aligned with where we've
always been trying to go,'' he said. ``Now that more people are paying
attention to it? Cool. We've always known it's important. The world is
just now catching up.''

Advertisement

\protect\hyperlink{after-bottom}{Continue reading the main story}

\hypertarget{site-index}{%
\subsection{Site Index}\label{site-index}}

\hypertarget{site-information-navigation}{%
\subsection{Site Information
Navigation}\label{site-information-navigation}}

\begin{itemize}
\tightlist
\item
  \href{https://help.nytimes3xbfgragh.onion/hc/en-us/articles/115014792127-Copyright-notice}{©~2020~The
  New York Times Company}
\end{itemize}

\begin{itemize}
\tightlist
\item
  \href{https://www.nytco.com/}{NYTCo}
\item
  \href{https://help.nytimes3xbfgragh.onion/hc/en-us/articles/115015385887-Contact-Us}{Contact
  Us}
\item
  \href{https://www.nytco.com/careers/}{Work with us}
\item
  \href{https://nytmediakit.com/}{Advertise}
\item
  \href{http://www.tbrandstudio.com/}{T Brand Studio}
\item
  \href{https://www.nytimes3xbfgragh.onion/privacy/cookie-policy\#how-do-i-manage-trackers}{Your
  Ad Choices}
\item
  \href{https://www.nytimes3xbfgragh.onion/privacy}{Privacy}
\item
  \href{https://help.nytimes3xbfgragh.onion/hc/en-us/articles/115014893428-Terms-of-service}{Terms
  of Service}
\item
  \href{https://help.nytimes3xbfgragh.onion/hc/en-us/articles/115014893968-Terms-of-sale}{Terms
  of Sale}
\item
  \href{https://spiderbites.nytimes3xbfgragh.onion}{Site Map}
\item
  \href{https://help.nytimes3xbfgragh.onion/hc/en-us}{Help}
\item
  \href{https://www.nytimes3xbfgragh.onion/subscription?campaignId=37WXW}{Subscriptions}
\end{itemize}
