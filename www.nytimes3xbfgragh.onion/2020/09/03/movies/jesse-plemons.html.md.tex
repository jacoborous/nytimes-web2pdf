\href{/section/movies}{Movies}\textbar{}What Does Everyone See in Jesse
Plemons?

\url{https://nyti.ms/2QTn04J}

\begin{itemize}
\item
\item
\item
\item
\item
\item
\end{itemize}

\includegraphics{https://static01.graylady3jvrrxbe.onion/images/2020/09/06/arts/06jesse-plemons1/merlin_175543329_13988428-b6ef-4437-a449-ce691faf79f0-articleLarge.jpg?quality=75\&auto=webp\&disable=upscale}

Sections

\protect\hyperlink{site-content}{Skip to
content}\protect\hyperlink{site-index}{Skip to site index}

\hypertarget{what-does-everyone-see-in-jesse-plemons}{%
\section{What Does Everyone See in Jesse
Plemons?}\label{what-does-everyone-see-in-jesse-plemons}}

After a slew of supporting roles in Oscar contenders by major directors,
he is taking the lead in a new Charlie Kaufman film. His goal is to
never deliver a false moment.

``I love actors where you don't see them acting,'' Plemons
said.Credit...Jake Michaels for The New York Times

Supported by

\protect\hyperlink{after-sponsor}{Continue reading the main story}

\href{https://www.nytimes3xbfgragh.onion/by/kyle-buchanan}{\includegraphics{https://static01.graylady3jvrrxbe.onion/images/2019/06/20/reader-center/kyle-buchanan-now/kyle-buchanan-now-thumbLarge-v2.png}}

By \href{https://www.nytimes3xbfgragh.onion/by/kyle-buchanan}{Kyle
Buchanan}

\begin{itemize}
\item
  Sept. 3, 2020
\item
  \begin{itemize}
  \item
  \item
  \item
  \item
  \item
  \item
  \end{itemize}
\end{itemize}

LOS ANGELES --- Jesse Plemons had never felt so lost. It was early March
2019, just a few days before he was supposed to shoot Charlie Kaufman's
``\href{https://www.netflix.com/title/80211559}{I'm Thinking of Ending
Things},'' and Plemons still had two questions that he dared not ask of
his director.

The first was kind of a biggie. \emph{What exactly was the movie about?}
Adapted from the
\href{https://www.nytimes3xbfgragh.onion/2016/09/04/books/review/im-thinking-of-ending-things-iain-reid.html}{novel
by Iain Reid} and out on Netflix Sept. 4,
``\href{https://www.nytimes3xbfgragh.onion/2020/09/01/movies/im-thinking-of-ending-things-review.html}{I'm
Thinking of Ending Things}'' appears deceptively simple: A man named
Jake and his girlfriend embark on a snowy drive to meet his parents.
Afterward, they drive back.

Or do they? The story's true nature remains tantalizingly out of reach.
As their circumstances grow more and more strange, the characters'
shared sense of reality begins to smear, and the film unfolds like a
Rorschach blot: What you ultimately make of this lonesome little tale
may depend on what you bring to it.

Plemons knew that with a storyteller like Kaufman, a bit of
disorientation was to be expected --- this was the man who had written
meta mind-benders like ``Being John Malkovich'' and ``Eternal Sunshine
of the Spotless Mind,'' after all. Still, a good grip on the material
kept proving elusive. Plemons had hoped things would get better after
the first rehearsal; after that first rehearsal, he was convinced they
wouldn't.

\includegraphics{https://static01.graylady3jvrrxbe.onion/images/2020/09/06/arts/06jesse-plemons-netflix/06jesse-plemons-netflix-articleLarge.jpg?quality=75\&auto=webp\&disable=upscale}

\emph{Why had he been cast?} That was the second question Plemons
couldn't bring himself to ask, even as he grew certain he was the wrong
man to play Jake. He had just come off a string of supporting roles in
``The Irishman,'' ``Vice'' and other movies, and Kaufman had offered him
the male lead in ``I'm Thinking of Ending Things'' without so much as an
audition.

``I had no clue that Charlie had any idea who I was,'' Plemons said.
``There was a part of me that was like, `Are you sure, Charlie? You want
to see me do something first?'''

With only two days left before the shoot, Plemons went to dinner with
Kaufman and his castmates still feeling unmoored. To his surprise, the
other actors said they felt the same way. Even David Thewlis, who had
worked with Kaufman on the
\href{https://www.nytimes3xbfgragh.onion/2015/12/30/movies/review-anomalisa-pairs-charlie-kaufman-and-lonely-puppets.html}{animated
``Anomalisa,''} admitted to some confusion.

``David finally asked Charlie, ``So can you tell us what this is
about?'' Plemons recalled. It was the first of his two unasked
questions, and Plemons hung on Kaufman's answer. ``And Charlie was like,
`You know, I don't know.'''

Some actors might have been alarmed by such a confession, but to
Plemons, the material finally made sense. He had been trying to figure
out something that was meant to be experienced rather than completely
understood. ``Charlie kind of arrived at saying, `I think we just have
accept that we don't know, and just accept that we're going to fail
sometimes. We have to embrace that.'''

The answer to his first unasked question also suggested the answer to
his second. There was nothing that could be done on this film but live
in the moment, and if that's what you want from an actor --- well,
that's why you cast Jesse Plemons.

\textbf{WHEN KAUFMAN FIRST} laid eyes on his eventual lead, he wasn't
thinking ``movie star.'' He was thinking ``background extra.''

Kaufman was introduced to Plemons through ``Breaking Bad,'' the hit TV
drama that Plemons joined in its fifth and final season. At first, you
barely notice him: While Bryan Cranston and Aaron Paul rip through their
scenes with galvanizing grandeur, Plemons putters around as the mild
\href{https://breakingbad.fandom.com/wiki/Todd_Alquist}{pest-control
flunky Todd}, pitching a fumigation tent and mumbling a handful of
lines.

As Todd goes on to become a major player, embroiled in high crimes like
meth-making and child murder, Plemons barely lets on that the stakes
have been raised. So pronounced is his lack of affect that you'd be
forgiven for thinking this is a real person who's been pushed in front
of the camera and forced to wing it.

``I never saw Todd coming, and I think that's the thing about Jesse,''
Kaufman said. ``It's very interesting to watch him work because
everything is just so small and underplayed, which is very valuable in
film.''

Image

Plemons opposite Aaron Paul, left, and Bryan Cranston in ``Breaking
Bad.'' His performance in the show impressed Charlie Kaufman:
``Everything is just so small and underplayed, which is very valuable in
film.''Credit...Ursula Coyote/AMC

That verisimilitude has found him fans in major directors like Steven
Spielberg and Martin Scorsese, and the 32-year-old Plemons has recently
become a mainstay of prestige dramas, appearing in best picture nominees
four of the last five years. Utter naturalism is his goal: Plemons can
toggle easily between eggheads and dimwits, good guys and bad guys, and
it's almost impossible to describe what he's doing differently because
he doesn't appear to be doing anything at all.

``I love actors where you don't see them acting,'' Plemons told me in
early August, when we met outside his home in the Toluca Lake
neighborhood of Los Angeles. ``You don't see a false moment. You don't
catch them.''

Though the circumstances of our interview were dictated by the pandemic,
it felt fitting that we should hang out in his backyard: With his mussed
golden hair, slight paunch, and feet nestled in flip-flops, Plemons
looked for all the world like a good-natured dad at a barbecue. He was
amiable but not particularly expressive, and the gentle volume of his
Texas twang was frequently drowned out by garbage trucks.

And yet he was still able, somehow, to command attention. Was he a bit
of a Rorschach blot himself? ``It's that `still waters run deep' kind of
thing,'' his wife, Kirsten Dunst, would tell me later. ``I think there's
just some people that you're drawn to watching.''

Image

Plemons has had a role in best-picture nominees for four of the past
five years.Credit...Jake Michaels for The New York Times

Plemons and Dunst met while shooting the second season of
\href{https://variety.com/video/kirsten-dunst-and-jesse-plemons-on-the-dramatic-twists-of-fargo-season-two/}{FX's
``Fargo,''} in which they played a married couple engaged in a criminal
cover-up. ``I knew that she would be in my life for a long time,'' he
said. Though they didn't begin dating until a year and a half after the
season had wrapped (and both had netted Emmy nominations), the
connection was instant: They often stayed up late running lines with
each other, a level of dedication that had been drummed into them from a
lifetime spent in the entertainment industry.

``We laugh about the fact that we were two child actors,'' Dunst said,
``and we both made it out OK.''

As a young performer, Plemons would fly from his hometown Mart, Tex., to
Los Angeles for auditions. He lacked the over-the-top, pixie-stick
enthusiasm of his child-actor brethren, and he remembers a low-key
reading for the Disney Channel that left the casting director
``genuinely confused and almost worried,'' he said. But at 18, that
Everyman earnestness landed Plemons a breakout role as Landry, the
bookish friend of the star quarterback, in the NBC football drama
``\href{https://www.youtube.com/watch?v=pZQJ5CQyycY}{Friday Night
Lights}.''

The show's cinéma-vérité aesthetic played to Plemons's strengths: He
could imbue any plotline with a documentarylike rawness, and the series
leaned on him more and more as it went on. Scenes weren't rehearsed, and
he was allowed to improvise at will, a process that granted him total
immersion in his role. It was the perfect training ground, and it
spoiled him, too: ``I feel like I'm trying desperately to circle back to
what it was like during `Friday Night Lights,''' he said.

The low-rated show led to much higher-profile opportunities --- Plemons
would go on to play Philip Seymour Hoffman's son in the Paul Thomas
Anderson drama
``\href{https://www.nytimes3xbfgragh.onion/2012/09/14/movies/review-the-master-from-paul-thomas-anderson.html}{The
Master,}'' and appear in Spielberg's ``Bridge of Spies'' and ``The
Post'' --- but as he bore down on them, that youthful sense of freedom
was hard to recapture. ``I went through a period of time where I was
pretty hard on myself, where it was not as much fun as it should've
been,'' he said. ``I care so much and want to give everything that I
have, that it just starts eating you up and becomes less enjoyable.''

In other words, it's a lot of work to make it look like no work at all.
``He works \emph{so} hard at what he does,'' Dunst said. ``He takes
everything very seriously and embeds himself very deeply.''

Image

Kirsten Dunst and Plemons met when they were shooting
``Fargo.''~Credit...Chris Large/FX

Shooting Kaufman's film helped him shed some of that anxiety, once he
stopped overthinking his character. Some of the scenes between Plemons
and Jessie Buckley, who plays his girlfriend, were so lengthy that the
actors made a pact: If either of them forgot the next line, they'd just
sit in silence, as their characters might, until something finally came
to them. ``It was really intimidating at first,'' he said, ``and then
really exhilarating once you sink into it and give it all up.''

Life in quarantine with Dunst and their 2-year-old son, Ennis --- ``he's
the MVP of lockdown'' --- has helped, too. ``It forces you to look at
what's in front of you,'' Plemons said, and has reminded him that in
work and in life, it pays to stay in the moment.

``I've spent years of constantly learning the same lesson over and over
again, that you can work and work and work on something, and bang your
head against the wall and know it inside and out --- but then, in that
moment, if you're not relaxed in your mind and body, that's all for
nothing,'' he said. ``A lot of that work won't be seen unless you're
grounded and present. I just don't think there's ever anything wrong
with attempting to be present.''

There was a brief, inscrutable flicker on his face. And then an actual
smile.

``I guess I didn't know I had a philosophy on life,'' he said.

Advertisement

\protect\hyperlink{after-bottom}{Continue reading the main story}

\hypertarget{site-index}{%
\subsection{Site Index}\label{site-index}}

\hypertarget{site-information-navigation}{%
\subsection{Site Information
Navigation}\label{site-information-navigation}}

\begin{itemize}
\tightlist
\item
  \href{https://help.nytimes3xbfgragh.onion/hc/en-us/articles/115014792127-Copyright-notice}{©~2020~The
  New York Times Company}
\end{itemize}

\begin{itemize}
\tightlist
\item
  \href{https://www.nytco.com/}{NYTCo}
\item
  \href{https://help.nytimes3xbfgragh.onion/hc/en-us/articles/115015385887-Contact-Us}{Contact
  Us}
\item
  \href{https://www.nytco.com/careers/}{Work with us}
\item
  \href{https://nytmediakit.com/}{Advertise}
\item
  \href{http://www.tbrandstudio.com/}{T Brand Studio}
\item
  \href{https://www.nytimes3xbfgragh.onion/privacy/cookie-policy\#how-do-i-manage-trackers}{Your
  Ad Choices}
\item
  \href{https://www.nytimes3xbfgragh.onion/privacy}{Privacy}
\item
  \href{https://help.nytimes3xbfgragh.onion/hc/en-us/articles/115014893428-Terms-of-service}{Terms
  of Service}
\item
  \href{https://help.nytimes3xbfgragh.onion/hc/en-us/articles/115014893968-Terms-of-sale}{Terms
  of Sale}
\item
  \href{https://spiderbites.nytimes3xbfgragh.onion}{Site Map}
\item
  \href{https://help.nytimes3xbfgragh.onion/hc/en-us}{Help}
\item
  \href{https://www.nytimes3xbfgragh.onion/subscription?campaignId=37WXW}{Subscriptions}
\end{itemize}
