Sections

SEARCH

\protect\hyperlink{site-content}{Skip to
content}\protect\hyperlink{site-index}{Skip to site index}

\href{https://www.nytimes3xbfgragh.onion/section/travel}{Travel}

\href{https://myaccount.nytimes3xbfgragh.onion/auth/login?response_type=cookie\&client_id=vi}{}

\href{https://www.nytimes3xbfgragh.onion/section/todayspaper}{Today's
Paper}

\href{/section/travel}{Travel}\textbar{}Glamping for First-Timers

\url{https://nyti.ms/2EWhFqX}

\begin{itemize}
\item
\item
\item
\item
\item
\item
\end{itemize}

\href{https://www.nytimes3xbfgragh.onion/spotlight/at-home?action=click\&pgtype=Article\&state=default\&region=TOP_BANNER\&context=at_home_menu}{At
Home}

\begin{itemize}
\tightlist
\item
  \href{https://www.nytimes3xbfgragh.onion/2020/09/07/travel/route-66.html?action=click\&pgtype=Article\&state=default\&region=TOP_BANNER\&context=at_home_menu}{Cruise
  Along: Route 66}
\item
  \href{https://www.nytimes3xbfgragh.onion/2020/09/04/dining/sheet-pan-chicken.html?action=click\&pgtype=Article\&state=default\&region=TOP_BANNER\&context=at_home_menu}{Roast:
  Chicken With Plums}
\item
  \href{https://www.nytimes3xbfgragh.onion/2020/09/04/arts/television/dark-shadows-stream.html?action=click\&pgtype=Article\&state=default\&region=TOP_BANNER\&context=at_home_menu}{Watch:
  Dark Shadows}
\item
  \href{https://www.nytimes3xbfgragh.onion/interactive/2020/at-home/even-more-reporters-editors-diaries-lists-recommendations.html?action=click\&pgtype=Article\&state=default\&region=TOP_BANNER\&context=at_home_menu}{Explore:
  Reporters' Google Docs}
\end{itemize}

Advertisement

\protect\hyperlink{after-top}{Continue reading the main story}

Supported by

\protect\hyperlink{after-sponsor}{Continue reading the main story}

\hypertarget{glamping-for-first-timers}{%
\section{Glamping for First-Timers}\label{glamping-for-first-timers}}

A novice camper needed a vacation. What are the options during the
summer of Covid-19?

\includegraphics{https://static01.graylady3jvrrxbe.onion/images/2020/08/27/travel/27Glamping1/merlin_176095209_0dfd8b12-c7f7-4280-ac7d-00402e812a26-articleLarge.jpg?quality=75\&auto=webp\&disable=upscale}

\href{https://www.nytimes3xbfgragh.onion/by/nikita-stewart}{\includegraphics{https://static01.graylady3jvrrxbe.onion/images/2018/09/25/multimedia/author-nikita-stewart/author-nikita-stewart-thumbLarge-v2.png}}

By \href{https://www.nytimes3xbfgragh.onion/by/nikita-stewart}{Nikita
Stewart}

\begin{itemize}
\item
  Sept. 3, 2020
\item
  \begin{itemize}
  \item
  \item
  \item
  \item
  \item
  \item
  \end{itemize}
\end{itemize}

I've never been an outdoorswoman. Though I'm from Texas, spent pivotal
teenage years in Kentucky and grew up around hunters and farmers,
there's a difference between spending time outside and living, cooking
and sleeping with nature.

Giselle Burgess, a mother of five and a Girl Scout troop leader whom I
met while researching
\href{https://www.nytimes3xbfgragh.onion/2020/05/19/books/review/troop-6000-nikita-stewart.html}{a
book about a troop that started in a homeless shelter}, helped me with
my first camping trip in 2017. She loves camping, loves the lingering
scent of campfire smoke in her clothes and even prides herself on
locating, pinching and plucking ticks. Staying at Camp Kaufmann, the
sprawling campground owned by the Girl Scouts of Greater New York, I had
the right gear thanks to Giselle, but I slept in a bed and was allowed
to shower.

I'm not above sleeping outside, and because of my reporting, I never
forget that thousands of New Yorkers experiencing homelessness call the
city's streets home. But camping in wooded areas, in a tent? I was
scared. All I could imagine was a bear tearing into my tent because a
graham cracker crumb from a s'more had followed me.

What's scarier, more dangerous and more likely than bears this year? The
coronavirus.

\includegraphics{https://static01.graylady3jvrrxbe.onion/images/2020/08/27/travel/27Glamping2/merlin_176095455_3ad4a669-2bc3-49b6-bb02-51a486818445-articleLarge.jpg?quality=75\&auto=webp\&disable=upscale}

The thought of the virus creeping through the H.V.A.C. systems of hotels
and restaurants has paralyzed me. I refuse to even enter grocery stores,
opting instead for deliveries, open-air farmers' markets and a co-op in
my neighborhood that only allows one customer at a time every 15
minutes. And while cheap flights keep calling my name, enticing me to
travel to a faraway locale, I would spend the entire time worried about
other passengers.

All my fears meant the only vacation possible this summer would have to
involve the outdoors and camping. But stretching a blue tarp between
trees the way Giselle expertly shields a campsite from rain was not
happening. First, everyone else in the United States seems to be
camping. The best equipment was sold out or appeared to be back-ordered
for months. So I opted for a little more comfort and plunged into the
no-muss, no-fuss world of glamping.

\hypertarget{hitting-the-road-then-the-beach-and-the-woods}{%
\subsection{Hitting the road, then the beach and the
woods}\label{hitting-the-road-then-the-beach-and-the-woods}}

My partner and I planned three glamping trips: to Maine, the Finger
Lakes and the Hudson Valley. All were within driving distance of my
Manhattan apartment and all the sites had new virus-related health and
safety measures in place. Altogether, we spent nine nights sleeping
under the stars. Sort of.

While the options for ``glamorous camping'' have
\href{https://www.nytimes3xbfgragh.onion/2018/06/15/travel/luxury-camping.html}{expanded
at a rapid pace} in recent years, glamping can mean staying in anything
from a cabin or a tiny house or a yurt. The only consistency we
discovered is that some type of shelter has been pitched for you. We
encountered tents. At
\href{https://sandypinescamping.com/glamping-at-sandy-pines-campground-in-maine/\#tents}{Sandy
Pines campground} in Kennebunkport, Maine, it was a tented hotel room
with a chandelier made of oyster shells, complete with air conditioning
and a mini-fridge. At \href{https://firelightcamps.com/}{Firelight Camps
in Ithaca}, N.Y., the tent had fans and a private balcony. (Out of my
three destinations, Firelight had the most comfortable bed, though all
of the beds were better than a sleeping bag on the ground.)

Image

The inside of a tent at the Firelight Camps in Ithaca,
N.Y.Credit...Heather Ainsworth for The New York Times

At \href{http://www.gatherwild.com/}{Gatherwild Ranch in Germantown,
N.Y.}, we had a small, round tent that was chic, with premium sheets and
beautiful rugs. The view was scenic as the single tent sat in the middle
of an old apple orchard. It had no electricity, forcing me to rely on a
solar-powered lantern and solar twinkle lights that made it feel like we
were sleeping with the stars inside the tent.

As for camping gear, in Maine I swapped out my poncho and backpack for a
beach umbrella and bag. But in all three places I always carried bug
spray and the Go-Sun solar panel, to power my cellphone and computer. We
also took a projector to the Finger Lakes that allowed us to turn a wall
inside the tent into a movie screen for late-night horror movies. (That
wasn't so easy in the Hudson Valley, because the glamping site requires
guests to unplug by cutting off Wi-Fi.)

I saw complaints online that the grounds at Firelight did not feel
remote enough, but a rabbit greeted me when we arrived and on my last
day, no fewer than 10 rabbits had surrounded the perimeter of our tent.
It would have been worrisome had they not all had white cottontails.

\hypertarget{be-prepared-or-find-somewhere-to-eat}{%
\subsection{Be prepared or find somewhere to
eat}\label{be-prepared-or-find-somewhere-to-eat}}

Since glamping can mean remote or semi-remote, I packed food and snacks.
Though we planned to eat out occasionally, I also wanted to grill and
brought chicken, ribs and sausages, which I marinated and kept in a
small cooler.

In Maine, friends told me to be prepared to eat lots of lobster.
Kennebunkport also boasts some great restaurants with outdoor seating,
like \href{https://www.earthathiddenpond.com/}{Earth at Hidden Pond} and
\href{https://www.pearlkennebunk.com/}{Pearl Kennebunk}. I also picked
up haddock to grill at \href{http://freerangefish.com/}{Free Range Fish
\& Lobster,} and an assortment of local cheeses at
\href{https://www.thecheeseshopofportland.com/}{The Cheese Shop of
Portland}. The food from home included pizza dough, which we cooked on a
small grill, in a flat cast-iron pan that we placed atop the firepit. It
made for great, wood-fired pizza.

At Firelight, before the pandemic, a breakfast buffet would have been
offered to guests. Now we were offered a choice of continental breakfast
in a personal cooler each morning. I chose a boiled egg, berries,
granola and yogurt.

Image

Each glamping site at Gatherwild has its own private
outhouse.Credit...Piotr Redlinski for The New York Times

\hypertarget{hi-potty-friends}{%
\subsection{`Hi Potty Friends'}\label{hi-potty-friends}}

Another big concern for me, even before the virus, was bathroom
facilities. The campgrounds in Ithaca and Kennebunkport had shared
bathrooms with running water, so I had to have faith that other campers
were wearing their masks as directed by staffers and posted signs. While
Sandy Pines had small, all-in-one bathrooms with sinks, toilets and
showers, Firelight had separate showers and a shared area with toilets
and sinks. Every other sink at Firelight was covered in red tape that
formed X's to encourage social distancing. There were signs telling
guests to wear masks, but they weren't always followed. A father showed
up with his young daughter who had on a mask. He did not --- though he
was telling her how to best wash her hands, singing the alphabet song (a
little too fast for my taste).

But at Gatherwild, I had my very own bathroom a short walk away: an
outhouse with a compost toilet. A chalkboard sign read, ``Hi Potty
Friends, All paper products go INTO the potty. Generous scoop of wood
chips when done. Seat down \& Thanks!'' I shared an outdoor shower with
two other tents. Well-placed bushes made me feel more comfortable, but I
could see Pickles and Mama Goaty Sophia, the resident goats, staring at
me.

\hypertarget{no-vacancy}{%
\subsection{No vacancy}\label{no-vacancy}}

I'm not the only one who decided to let go of things we think we can't
live without, like privacy and indoor plumbing, to get a change of
scenery.

Image

An ``open-air'' barn was the site of a vintage store in
Germantown.~Credit...Piotr Redlinski for The New York Times

Gatherwild's seven tents and tiny houses were at 99 percent capacity,
Laura Sink, the owner, told me as we sat in a barn, six feet apart,
which also housed a vintage store. (I bought four lovely scarves there
and picked flowers from an adjacent garden before I left.)

``This year it's off the charts,'' said Robert Frisch, the owner of
Firelight and its 19 safari tents. ``We're full every night. Every
tent.''

At Firelight, the guests looked very different from the campground's
Instagram page, which shows millennial couples kissing in front of their
stylish tents and beneath waterfalls. During my stay, babies cried and
children screeched in delight or in disappointment. The camp had become
more about the fam' than the `gram. But I welcomed the chatter of
children who were obviously thirsting for fresh air. They touched the
leaves of trees and pushed each other in a swing. They climbed into
hammocks. They played red light, green light in an area that once
offered cornhole --- gone now to encourage social distancing and
discourage group play.

Image

One of the 10 firepits at Firelight.~Credit...Heather Ainsworth for The
New York Times

\hypertarget{wash-your-hands-wear-a-mask}{%
\subsection{Wash your hands, wear a
mask}\label{wash-your-hands-wear-a-mask}}

Firelight used to be much more adult, and much more communal, with
people gathering each night around a single firepit. Mandated social
distancing required the camp to increase the number of firepits: When I
arrived in August, there were 10, and some nights I could see the
disappointment in the faces of families when they failed to score a
firepit. The camp also has grills that took some timing and maneuvering
to use.

At Gatherwild, Ms. Sink will take orders for groceries and deliver them
to your tent, where a large cooler can keep them from spoiling.

The camp had to invest in these coolers, as well as picnic tables,
umbrellas and firepits for every tent, yurt and tiny house at the camp.
Because of increased demand and the necessary deep cleaning of tents,
pricing looks very different than in the past. I booked Gatherwild for
\$130 a night, but Ms. Sink said she is now charging no less than \$175
a night and guests must book two nights. At Sandy Pines, cleanings of
the tents and shared areas were also increased. Rules, such as no-touch
check-in and check-out, and capacity restrictions at the pool and
general store, were implemented to adhere to social distancing.

\hypertarget{repeat-guest}{%
\subsection{Repeat guest?}\label{repeat-guest}}

From my adventures this summer, I must admit that I'd prefer a flushing
toilet, a hot shower, Wi-Fi and a refrigerator. So maybe a cabin and a
cottage with lots of windows would be my best bet.

But if I go glamping again, I would pack less. I would leave behind
those just-in-case-I-go-to-a-nice-restaurant dresses and pack another
sweatsuit. There's a lot of sitting around at camp once night falls, and
it gets cold. You want to stay up to enjoy the fire and night sky.

Relaxing in my own space without worrying whether I was six feet away
from someone, however, was rejuvenating. At Sandy Pines, I went to a
beach and fell asleep on a blanket one afternoon. At Firelight, I took a
walk around the property and then followed a trail into nearby
Buttermilk Falls State Park. I rode one of the bikes that Gatherwild has
available to meander around its old apple orchard. And those starry
nights with no face mask in sight were worth a cold shower in front of
some goats.

Image

At Gatherwild, the Luxury Belle tent is a popular pick for
guests.Credit...Piotr Redlinski for The New York Times

\emph{\textbf{For more travel coverage}}*, follow us on*
\href{https://twitter.com/nytimestravel}{\emph{Twitter}} \emph{and}
\href{https://www.facebookcorewwwi.onion/nytimestravel/}{\emph{Facebook}}\emph{.
And don't forget to}
\href{https://www.nytimes3xbfgragh.onion/newsletters/traveldispatch?action=click\&module=inline\&pgtype=Article}{\emph{sign
up for our}} **
\href{https://www.nytimes3xbfgragh.onion/newsletters/traveldispatch}{\emph{Travel
Dispatch newsletter}}\emph{: Each week you'll receive tips on traveling
smarter, stories on hot destinations and access to photos from all over
the world.}

Advertisement

\protect\hyperlink{after-bottom}{Continue reading the main story}

\hypertarget{site-index}{%
\subsection{Site Index}\label{site-index}}

\hypertarget{site-information-navigation}{%
\subsection{Site Information
Navigation}\label{site-information-navigation}}

\begin{itemize}
\tightlist
\item
  \href{https://help.nytimes3xbfgragh.onion/hc/en-us/articles/115014792127-Copyright-notice}{©~2020~The
  New York Times Company}
\end{itemize}

\begin{itemize}
\tightlist
\item
  \href{https://www.nytco.com/}{NYTCo}
\item
  \href{https://help.nytimes3xbfgragh.onion/hc/en-us/articles/115015385887-Contact-Us}{Contact
  Us}
\item
  \href{https://www.nytco.com/careers/}{Work with us}
\item
  \href{https://nytmediakit.com/}{Advertise}
\item
  \href{http://www.tbrandstudio.com/}{T Brand Studio}
\item
  \href{https://www.nytimes3xbfgragh.onion/privacy/cookie-policy\#how-do-i-manage-trackers}{Your
  Ad Choices}
\item
  \href{https://www.nytimes3xbfgragh.onion/privacy}{Privacy}
\item
  \href{https://help.nytimes3xbfgragh.onion/hc/en-us/articles/115014893428-Terms-of-service}{Terms
  of Service}
\item
  \href{https://help.nytimes3xbfgragh.onion/hc/en-us/articles/115014893968-Terms-of-sale}{Terms
  of Sale}
\item
  \href{https://spiderbites.nytimes3xbfgragh.onion}{Site Map}
\item
  \href{https://help.nytimes3xbfgragh.onion/hc/en-us}{Help}
\item
  \href{https://www.nytimes3xbfgragh.onion/subscription?campaignId=37WXW}{Subscriptions}
\end{itemize}
