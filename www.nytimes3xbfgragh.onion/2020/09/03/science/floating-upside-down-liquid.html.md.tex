Sections

SEARCH

\protect\hyperlink{site-content}{Skip to
content}\protect\hyperlink{site-index}{Skip to site index}

\href{https://www.nytimes3xbfgragh.onion/section/science}{Science}

\href{https://myaccount.nytimes3xbfgragh.onion/auth/login?response_type=cookie\&client_id=vi}{}

\href{https://www.nytimes3xbfgragh.onion/section/todayspaper}{Today's
Paper}

\href{/section/science}{Science}\textbar{}Up Is Down in This Fun Physics
Experiment

\url{https://nyti.ms/3i9aH0h}

\begin{itemize}
\item
\item
\item
\item
\item
\end{itemize}

Advertisement

\protect\hyperlink{after-top}{Continue reading the main story}

Supported by

\protect\hyperlink{after-sponsor}{Continue reading the main story}

Trilobites

\hypertarget{up-is-down-in-this-fun-physics-experiment}{%
\section{Up Is Down in This Fun Physics
Experiment}\label{up-is-down-in-this-fun-physics-experiment}}

The liquid levitates, and a boat floats along its bottom side.

\includegraphics{https://static01.graylady3jvrrxbe.onion/images/2020/09/08/science/03TB-FLOAT-promo/03TB-FLOAT-promo-threeByTwoMediumAt2X.jpg}

\href{https://www.nytimes3xbfgragh.onion/by/kenneth-chang}{\includegraphics{https://static01.graylady3jvrrxbe.onion/images/2018/02/16/multimedia/author-kenneth-chang/author-kenneth-chang-thumbLarge.jpg}}

By \href{https://www.nytimes3xbfgragh.onion/by/kenneth-chang}{Kenneth
Chang}

\begin{itemize}
\item
  Sept. 3, 2020
\item
  \begin{itemize}
  \item
  \item
  \item
  \item
  \item
  \end{itemize}
\end{itemize}

Sail beneath a levitating sea --- upside down?

Through a couple of sleights of science, a team of French scientists
showed that not only could they make a layer of viscous liquid hover in
midair but that a little toy boat would also bob on the bottom side of
the liquid layer in the same way that one would normally float on top.

``That was a fun experiment,'' said Emmanuel Fort, a professor at
France's ESPCI Paris and an author of
\href{https://www.nature.com/articles/s41586-020-2643-8}{a paper
published this week in the journal Nature} that describes this seemingly
impossible feat. ``Everything worked well. And I'm still amazed by the
results.''

Usually, a denser liquid sinks to the bottom. That's why oil floats on
water. Even if you first pour a layer of oil into a container and then
carefully add water on top, the heavier water will start dripping
through the oil, forming tentacles that reach the bottom. Soon the water
will settle at the bottom beneath the oil.

That is similar to how the stable position of a rigid pendulum is to
hang straight downward. The inverted position, with the pendulum
pointing straight upward, is also a position of equilibrium as well,
with the forces perfectly balanced. But with the slightest disturbance,
that equilibrium is lost, and the pendulum swings downward.

Dr. Fort's levitating liquid research started when he heard a talk about
Kapitza's pendulum, named after Pyotr Kapitsa, a Russian physicist who
in 1951 described how, if the pendulum were vibrated up and down at the
correct frequency, it would remain in the upright configuration
indefinitely.

A spark of inspiration came to Dr. Fort: ``Instead of having some
pendulum upside down, we can maybe have some liquid layer upside down.''

In other words, they wanted to create a layer of liquid on top of air.

That does not work with a layer of water, which easily ripples and
becomes unstable. But it does work with glycerol and silicon oil, which
are thicker than water. The higher viscosity suppresses ripples.

\includegraphics{https://static01.graylady3jvrrxbe.onion/images/2020/09/03/autossell/03tb-float-liquid-still/03tb-float-liquid-still-superJumbo.png}

The vibrations, about 100 cycles a second, caused bubbles injected into
the liquid to be pushed downward, forming an air cushion below the
levitating liquid. The vibrations also generated a steady rhythm of
compressions that kept the levitating liquid intact. When a drip started
forming, the upward force of the air nudged the drip back into the
layer, keeping it intact.

And quite of bit of liquid can be levitated this way. The researchers
demonstrated they could lift about half a quart, and the liquid could
spread about eight inches wide.

In principle, they could have done much more. ``There's no limit; you
just have to shake more,'' Dr. Fort said.

But the bigger shaking platforms cost a lot more, and this was
peripheral to Dr. Fort's usual work: biomedical imaging. He has also
looked at how droplets and waves in water can serve as models for
certain aspects of quantum mechanics.

A search through the scientific literature revealed levitating liquids
with vibrations was not new knowledge; other scientists had discovered
the phenomenon decades ago.

But Dr. Fort's team identified something unusual: that objects could
float along that bottom layer of a levitating liquid.

Because of the weight of the liquid, the air underneath the levitating
layer is denser, and that denser air is pushing the boat up into the
liquid, counteracting the downward force of gravity.

The net effect is that it floats upside down.

``The global vibration helps you to stabilize this equilibrium
position,'' Dr. Fort said. ``It's not intuitive.''

Indeed, the scientists were surprised, too. ``We were thinking that it
would simply fall,'' Dr. Fort said.

The scientists initially used small round beads for their research, but
they then started using their 3-D printer for other shapes of plastic to
float upside down. That included ducks and frogs. Those all floated
upside down on the bottom side of the levitating liquid.

``But I think the boat was awesome,'' Dr. Fort said.

\includegraphics{https://static01.graylady3jvrrxbe.onion/images/2020/03/09/science/03tb-float-boat-again-image/03tb-float-boat-again-image-mediumSquareAt3X.jpg}

In an accompanying commentary, Vladislav Sorokin of the University of
Auckland in New Zealand and Iliya I. Blekhman of the Russian Academy of
Science wrote that the research ``suggests that many remarkable
phenomena arising in vibrating mechanical systems are yet to be revealed
and explained, particularly at interfaces between gases and fluids.''

Dr. Fort said that the research could have practical applications in the
mixing of liquids and solids and possibly unmixing them back into
separate components.

People who came to the laboratory and saw the experiment generally had
two reactions, Dr. Fort said. One was to not believe it, that it was
some sort of trick.

But others, with a more artistic point of view, compared it to poetry.

``Indeed when you see these boats, it's a bit like fantasy,'' Dr. Fort
said. ``That was also a very nice part outside of the narrow scope of
science.''

Advertisement

\protect\hyperlink{after-bottom}{Continue reading the main story}

\hypertarget{site-index}{%
\subsection{Site Index}\label{site-index}}

\hypertarget{site-information-navigation}{%
\subsection{Site Information
Navigation}\label{site-information-navigation}}

\begin{itemize}
\tightlist
\item
  \href{https://help.nytimes3xbfgragh.onion/hc/en-us/articles/115014792127-Copyright-notice}{©~2020~The
  New York Times Company}
\end{itemize}

\begin{itemize}
\tightlist
\item
  \href{https://www.nytco.com/}{NYTCo}
\item
  \href{https://help.nytimes3xbfgragh.onion/hc/en-us/articles/115015385887-Contact-Us}{Contact
  Us}
\item
  \href{https://www.nytco.com/careers/}{Work with us}
\item
  \href{https://nytmediakit.com/}{Advertise}
\item
  \href{http://www.tbrandstudio.com/}{T Brand Studio}
\item
  \href{https://www.nytimes3xbfgragh.onion/privacy/cookie-policy\#how-do-i-manage-trackers}{Your
  Ad Choices}
\item
  \href{https://www.nytimes3xbfgragh.onion/privacy}{Privacy}
\item
  \href{https://help.nytimes3xbfgragh.onion/hc/en-us/articles/115014893428-Terms-of-service}{Terms
  of Service}
\item
  \href{https://help.nytimes3xbfgragh.onion/hc/en-us/articles/115014893968-Terms-of-sale}{Terms
  of Sale}
\item
  \href{https://spiderbites.nytimes3xbfgragh.onion}{Site Map}
\item
  \href{https://help.nytimes3xbfgragh.onion/hc/en-us}{Help}
\item
  \href{https://www.nytimes3xbfgragh.onion/subscription?campaignId=37WXW}{Subscriptions}
\end{itemize}
