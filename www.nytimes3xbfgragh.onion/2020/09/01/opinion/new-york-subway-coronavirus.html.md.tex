Sections

SEARCH

\protect\hyperlink{site-content}{Skip to
content}\protect\hyperlink{site-index}{Skip to site index}

\href{https://myaccount.nytimes3xbfgragh.onion/auth/login?response_type=cookie\&client_id=vi}{}

\href{https://www.nytimes3xbfgragh.onion/section/todayspaper}{Today's
Paper}

\href{/section/opinion}{Opinion}\textbar{}The Subways Are Facing a
Five-Alarm Fire

\url{https://nyti.ms/3bjBGDn}

\begin{itemize}
\item
\item
\item
\item
\item
\end{itemize}

Advertisement

\protect\hyperlink{after-top}{Continue reading the main story}

\href{/section/opinion}{Opinion}

Supported by

\protect\hyperlink{after-sponsor}{Continue reading the main story}

\hypertarget{the-subways-are-facing-a-five-alarm-fire}{%
\section{The Subways Are Facing a Five-Alarm
Fire}\label{the-subways-are-facing-a-five-alarm-fire}}

New York City cannot recover without a robust transit system, and the
country cannot rebound without New York.

By Patrick J. Foye and John Samuelsen

Mr. Foye is the chairman and chief executive of the New York
Metropolitan Transportation Authority. Mr. Samuelsen is the
International President of the Transport Workers Union.

\begin{itemize}
\item
  Sept. 1, 2020
\item
  \begin{itemize}
  \item
  \item
  \item
  \item
  \item
  \end{itemize}
\end{itemize}

\includegraphics{https://static01.graylady3jvrrxbe.onion/images/2020/09/01/opinion/01Foye1/merlin_172115616_53f34c85-0460-4fb6-88a5-d70f6bebab78-articleLarge.jpg?quality=75\&auto=webp\&disable=upscale}

As chief executive of North America's largest transportation system, and
as president of the country's biggest transit workers' union, we have
had our fair share of disputes. But we agree on this: The Metropolitan
Transportation Authority is facing a five-alarm-fire --- and the
Republican majority in the U.S. Senate seems content to sit back and do
nothing while it burns.

This isn't hyperbole. The numbers are staggering: a
\href{http://www.mta.info/press-release/mta-headquarters/us-senate-considers-coronavirus-relief-bill-mta-details-16-billion}{\$16
billion deficit} through 2024, with \$200 million in revenue losses
every week. The pandemic has caused precipitous declines in ridership,
fare revenues, tolls and subsidies. Pandemic-related expenses, like the
cleaning and disinfecting of train cars, have soared.

Through it all, the M.T.A.'s employees have continued to operate and
maintain the public transit system. They have ensured that doctors,
nurses, supermarket workers and so many other essential employees get to
their jobs on the front lines.

But this courageous dedication came at the highest price. By late July,
131 transit workers who worked and fought for New York during this
crisis perished. We mourn their loss every single day.

Now the transit system that carried us through the pandemic is set to
run out of money. The workers who kept New York moving and put their own
health and families at risk are worried about layoffs. The M.T.A. has
\href{https://www.nytimes3xbfgragh.onion/2020/08/26/nyregion/nyc-subway-bus-service-cuts.html}{drafted
plans} for the worst service cuts in agency history: up to 40 percent
across the subway and bus network, and up to 50 percent across the
commuter rails. This is compounded by a looming fare hike and possibly
deep cuts to the \$51.5 billion capital construction plan necessary to
modernize the 116-year-old system.

We have been sounding the warning bell for months. The M.T.A. is in a
free fall. The only way to stop it is congressional action, and time is
running out.

The issue of layoffs, in particular, is putting the M.T.A. and its
unionized employees on an ugly collision course when the city needs
unity to get back on its feet. We are pleading with the federal
government to deliver, quickly, to ward off disaster.

Before Covid-19 hit, the M.T.A. was making the best progress it had seen
in decades, with an expected \$81 million operating surplus, six
consecutive months of on-time performance above 80 percent and growing
ridership, serving approximately 8.6 million customers per day.

Congress needs to step up and deliver for mass transit, not only in New
York but across the country. Republican Senate leadership is ignoring
the magnitude of this fiscal crisis. The M.T.A. needs \$12 billion to
get through the rest of this year and 2021. We're facing an existential
crisis; even the Great Depression had a less severe impact on the
revenue of New York City's transit system than what we are seeing now.

After the 1929 stock market crash, ridership on New York City buses and
street cars declined by 16 percent. Compare that with today's losses on
buses, which are roughly 45 percent. The subway comparison is even more
striking --- by 1933, ridership had dropped by only a modest 12 percent.
At the peak of the Covid-19 pandemic, it was down by 93 percent. And
even now, with New York's curve flattened, transit ridership is still
down 75 percent.

Punishing the M.T.A. and transit systems across the country over an
ideological political agenda is not only wrong; it is bad economics. The
downstate New York region --- New York City and the surrounding area ---
accounts for about 8 percent of the nation's gross domestic product. New
York City cannot recover without a robust M.T.A., and the country cannot
rebound economically without a healthy New York.

Even if federal funding does come through, the M.T.A. will still face
some extremely difficult choices. But New York deserves better. Over the
past four years, New York taxpayers have given \$116 billion more to the
federal government than they received back in federal spending for an
average annual negative balance of \$29 billion --- far exceeding every
other state. Our riders and transit workers make this possible.

Our fate rests squarely in the hands of the federal government. Every
day Congress fails to pass another Covid-19 relief bill is another day
closer to the end of mass transit as we know it.

Patrick J. Foye is chairman and chief executive of the New York
Metropolitan Transportation Authority. John Samuelsen is the
International President of the Transport Workers Union.

\emph{The Times is committed to publishing}
\href{https://www.nytimes3xbfgragh.onion/2019/01/31/opinion/letters/letters-to-editor-new-york-times-women.html}{\emph{a
diversity of letters}} \emph{to the editor. We'd like to hear what you
think about this or any of our articles. Here are some}
\href{https://help.nytimes3xbfgragh.onion/hc/en-us/articles/115014925288-How-to-submit-a-letter-to-the-editor}{\emph{tips}}\emph{.
And here's our email:}
\href{mailto:letters@NYTimes.com}{\emph{letters@NYTimes.com}}\emph{.}

\emph{Follow The New York Times Opinion section on}
\href{https://www.facebookcorewwwi.onion/nytopinion}{\emph{Facebook}}\emph{,}
\href{http://twitter.com/NYTOpinion}{\emph{Twitter (@NYTopinion)}}
\emph{and}
\href{https://www.instagram.com/nytopinion/}{\emph{Instagram}}\emph{.}

Advertisement

\protect\hyperlink{after-bottom}{Continue reading the main story}

\hypertarget{site-index}{%
\subsection{Site Index}\label{site-index}}

\hypertarget{site-information-navigation}{%
\subsection{Site Information
Navigation}\label{site-information-navigation}}

\begin{itemize}
\tightlist
\item
  \href{https://help.nytimes3xbfgragh.onion/hc/en-us/articles/115014792127-Copyright-notice}{©~2020~The
  New York Times Company}
\end{itemize}

\begin{itemize}
\tightlist
\item
  \href{https://www.nytco.com/}{NYTCo}
\item
  \href{https://help.nytimes3xbfgragh.onion/hc/en-us/articles/115015385887-Contact-Us}{Contact
  Us}
\item
  \href{https://www.nytco.com/careers/}{Work with us}
\item
  \href{https://nytmediakit.com/}{Advertise}
\item
  \href{http://www.tbrandstudio.com/}{T Brand Studio}
\item
  \href{https://www.nytimes3xbfgragh.onion/privacy/cookie-policy\#how-do-i-manage-trackers}{Your
  Ad Choices}
\item
  \href{https://www.nytimes3xbfgragh.onion/privacy}{Privacy}
\item
  \href{https://help.nytimes3xbfgragh.onion/hc/en-us/articles/115014893428-Terms-of-service}{Terms
  of Service}
\item
  \href{https://help.nytimes3xbfgragh.onion/hc/en-us/articles/115014893968-Terms-of-sale}{Terms
  of Sale}
\item
  \href{https://spiderbites.nytimes3xbfgragh.onion}{Site Map}
\item
  \href{https://help.nytimes3xbfgragh.onion/hc/en-us}{Help}
\item
  \href{https://www.nytimes3xbfgragh.onion/subscription?campaignId=37WXW}{Subscriptions}
\end{itemize}
