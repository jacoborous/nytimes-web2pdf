Sections

SEARCH

\protect\hyperlink{site-content}{Skip to
content}\protect\hyperlink{site-index}{Skip to site index}

\href{https://myaccount.nytimes3xbfgragh.onion/auth/login?response_type=cookie\&client_id=vi}{}

\href{https://www.nytimes3xbfgragh.onion/section/todayspaper}{Today's
Paper}

The Forgotten Colonial Forces of World War II

\url{https://nyti.ms/2GbYGc1}

\begin{itemize}
\item
\item
\item
\item
\item
\item
\end{itemize}

Advertisement

\protect\hyperlink{after-top}{Continue reading the main story}

Supported by

\protect\hyperlink{after-sponsor}{Continue reading the main story}

Beyond the World War II We Know

\hypertarget{the-forgotten-colonial-forces-of-world-war-ii}{%
\section{The Forgotten Colonial Forces of World War
II}\label{the-forgotten-colonial-forces-of-world-war-ii}}

``There's a scattered memory of their sacrifice all over Europe.'' The
Allied powers relied on colonial troops to defeat the Axis, but their
contributions are not often recognized.

\includegraphics{https://static01.graylady3jvrrxbe.onion/images/2020/09/06/multimedia/01ww2-Colonial-forces-01/01ww2-Colonial-forces-01-articleLarge.jpg?quality=75\&auto=webp\&disable=upscale}

\href{https://www.nytimes3xbfgragh.onion/by/maria-abi-habib}{\includegraphics{https://static01.graylady3jvrrxbe.onion/images/2018/10/08/multimedia/author-maria-abi-habib/author-maria-abi-habib-thumbLarge.png}}

By \href{https://www.nytimes3xbfgragh.onion/by/maria-abi-habib}{Maria
Abi-Habib}

\begin{itemize}
\item
  Published Sept. 1, 2020Updated Sept. 3, 2020
\item
  \begin{itemize}
  \item
  \item
  \item
  \item
  \item
  \item
  \end{itemize}
\end{itemize}

\emph{\emph{\emph{The latest article from
``}\href{https://www.nytimes3xbfgragh.onion/spotlight/beyond-wwii}{\emph{Beyond
the World War II We Know}}},'' a series from The Times that documents
lesser-known stories from the war, recounts the sacrifices of colonial
forces, particularly British-backed Indian troops who fought not only
the Axis powers, but also their compatriots.}**

They fought in every theater of World War II, from North Africa to
Europe and as far east as Hong Kong. They died and went missing in the
tens of thousands. And they formed
the\href{https://web.archive.org/web/20100618081321/http://www.cwgc.org/admin/files/cwgc_india.pdf}{largest
volunteer force} in history. But their contributions are often an
afterthought in history books.

The colonial forces that dotted the battle maps of World War II were
crucial for the Allies to fill out their ranks and keep up their
momentum. While India contributed the largest number of volunteers, at
some 2.5 million troops, Africans, Arabs and others fought and died for
the freedom of the Allied powers, although they were under the yoke of
colonial rule. ``I always say, Britain didn't fight the Second World
War, the British Empire did,'' said Yasmin Khan, a historian at Oxford
University and author of ``The Raj at War.''

About 15 percent of all the Victoria Crosses --- Britain's highest
decoration for valor --- awarded during the Second World War went to
Indian and Nepalese troops. The honor was bestowed upon service members
from other colonies as well. ``If you look at Commonwealth graves, you
can find burial spots of Indians everywhere,'' Khan said. ``There's a
scattered memory of their sacrifice all over Europe.''

Image

King George VI pinning the Victoria Cross on Sepoy Kamal Ram in July
1944.Credit...Imperial War Museums

Image

A scout car crew of Indian soldiers chat with young civilians in San
Felice, Italy, 1943.Credit...Imperial War Museums

While these colonial forces are often forgotten or overshadowed, they
not only helped the Allied powers win their war, they also set in motion
events that would eventually lead to some of the colonies' independence.

Despite their sacrifices, these troops were never treated as equals.
They were largely under the command of European or American officers,
although they were skilled fighters and even helped patrol the streets
of London. It was difficult for them to rise up the ranks and become
officers. Their compensation was far less than that of their white
peers, and it worsened the darker their skin was. As poorly as Indian
soldiers were treated, their African peers fared far worse.

Their skill on the battlefield helped stoke nationalism at home;
however, the colonial forces were in many ways helping Britain maintain
its crumbling empire, as it came under onslaught by Japanese, Italian
and German forces.

\includegraphics{https://static01.graylady3jvrrxbe.onion/images/2020/09/06/multimedia/01ww2-Colonial-forces-02/merlin_176205108_dfd0c146-5f23-4d9a-a106-5c79191a1a11-articleLarge.jpg?quality=75\&auto=webp\&disable=upscale}

Although the battlefronts of Europe were romanticized in novels, history
books and films, much of the war was fought in and over British (and to
a lesser extent, French) colonies, with front lines springing up from
North Africa to East Asia as both sides vied for control of the regions'
vast resources and wealth to sustain their militaries. In June 1940, the
Axis powers launched the North Africa campaign and fighting broke out
across Algeria, Morocco, Egypt and Tunisia as they tried to wrest those
colonies from British and French rule. Japan snatched up British
colonies like Singapore and Burma (now Myanmar) and tried to invade
India.

It would be the entry of the world's most vocal supporter of liberty and
self-determination, the United States, that would help the Allies
restore their momentum and shift the tide against the Axis.

Image

Nepalese soldiers, or Gurkhas, manning a mortar during training
exercises in what was then British-occupied Malaya, January
1942.Credit...Associated Press

Image

British, South African and East African troops assemble on the beach of
Majunga, Madagascar in November 1942.Credit...Associated Press

Image

Indian Air Force officers in their mess hall at Imphal in Manipur,
India.Credit...Imperial War Museums

But the alliance between the United States and Britain was forged in
tension over their clashing stances on colonialism. While the United
States remained on the sidelines for nearly half of the war, its calls
to end colonialism irked Britain, which needed its colonies more than
ever, as its financial reserves were nearly exhausted.

Indians were angry when Britain, which ruled them, declared war on Nazi
Germany in 1939 and exploited their resources to support the conflict.
Some Indians, such as upper-caste urbanites, were loyal to the raj ---
British rule over India --- and fought enthusiastically for the Allies,
but the vast majority volunteered because they were offered land, a
stable salary and steady meals. Others joined to refine their technical
or engineering skills as the military modernized over the course of the
war, allowing them to gain experience with more complicated machinery as
it was introduced.

In August 1941, Prime Minister Winston Churchill and President Franklin
D. Roosevelt signed what became known as the
\href{https://www.un.org/en/sections/history-united-nations-charter/1941-atlantic-charter/index.html}{Atlantic
Charter}, a new vision for the postwar world, highlighting the right of
all people to self-determination. Though the United States had not yet
entered the war as a combatant, it was supplying military hardware to
Britain and created the document as a justification for its support to
the Allies, laying out its anti-fascist hopes for the world. Britain was
desperate to bind itself to the United States and persuade it the join
the war, and Churchill begrudgingly signed the statement, although it
challenged the very foundation of the empire.

The Atlantic Charter spurred hopes of independence among the British
colonies. But a month after the charter was signed, Churchill clarified
that the right to self-determination outlined in the document applied
only to countries under German occupation. The damage, however, was
already done.

In 1942, Mohandas K. Gandhi began his Quit India movement, demanding the
end of British rule, galvanizing Indians against British colonial forces
and threatening the economic and natural resources London needed to
continue fighting.

A star of the Indian independence movement, Subhas Chandra Bose, split
with Gandhi's nonviolent campaign and aligned himself with the Axis
powers, who he believed would empower him to raise an army and win
India's autonomy. Bose toured the prison camps of Europe and Asia,
building a force by recruiting Indian expatriates and Indian prisoners
of war.

Image

Indian nationalist leader Subash Chandra Bose, right, with Japanese
Prime Minister Hideki Tojo in 1944 at a parade in support of Indian
national independence in Shonan,
Japan.Credit...Keystone-France/Gamma-Rapho, via Getty Images

Bose's military, the Indian National Army, was a roughly 40,000-strong
force. By 1943, he established the Azad Hind, or the provisional
government of India in exile, in Japanese-occupied Singapore and
declared war on the Allied powers. Bose's ultimate goal was to invade
India and liberate it from the British. Once the I.N.A. and the Axis
invaded, Bose bet, Indians would rise up en masse. The British forbade
their media from reporting on the rogue force, worried it would spur
Indian troop defections.

In March 1944, Bose had his chance to shatter British rule. The Japanese
military, with the support of the I.N.A., launched Operation U-Go, a
campaign to invade northeast India from Burma and smash a buildup of
Allied forces in the area. If the Japanese and the I.N.A. prevailed,
they could extract India's resources to revitalize their war effort,
perhaps prolonging the war, and use India's strategic ports to cut off
Allied supply lines spanning from East to West.

But they faced stiff resistance from Allied forces, which were
overwhelmingly nonwhite --- about 70 percent of the fighting force was
from India and to a lesser extent, African colonies. (British forces
were reluctant to serve in India, preferring the glamour of the European
front lines.) The fight, known as the Battle of Kohima and Imphal,
produced some of the worst bloodshed of the war in Asia.

As Britain-backed Indian troops killed their own compatriots, those
under Bose's command, they also killed thousands of Japanese, considered
some of the best fighters in World War II. The Japanese 15th Army,
\href{https://www.nytimes3xbfgragh.onion/2014/06/22/world/asia/a-largely-indian-victory-in-world-war-ii-mostly-forgotten-in-india.html}{85,000
strong at the start} of the invasion, saw 53,000 troops dead or missing
by the battle's end.

The defeat, one of the most devastating of the war for Japanese ground
forces, helped the Indian military come into its own, historians
believe, and helped spur nationalist movements in India and parts of
Africa.

``They demanded their liberation,'' said the historian Kaushik Roy, a
professor at Jadavpur University in Kolkata, India. ``There was this
feeling, `why should we fight to preserve colonialism?'''

It took a few years after the war ended, but the nationalists prevailed.
Britain dismantled its empire, and the colonial troops it used to prop
up its rule across the world were rolled into the national armies of the
independent states that formed out of the wreckage. India was granted
independence in 1947.

``Once that lifeblood of colonialism was broken,'' Roy added, ``they
gained confidence in their demands to rule themselves.''

\begin{center}\rule{0.5\linewidth}{\linethickness}\end{center}

Advertisement

\protect\hyperlink{after-bottom}{Continue reading the main story}

\hypertarget{site-index}{%
\subsection{Site Index}\label{site-index}}

\hypertarget{site-information-navigation}{%
\subsection{Site Information
Navigation}\label{site-information-navigation}}

\begin{itemize}
\tightlist
\item
  \href{https://help.nytimes3xbfgragh.onion/hc/en-us/articles/115014792127-Copyright-notice}{©~2020~The
  New York Times Company}
\end{itemize}

\begin{itemize}
\tightlist
\item
  \href{https://www.nytco.com/}{NYTCo}
\item
  \href{https://help.nytimes3xbfgragh.onion/hc/en-us/articles/115015385887-Contact-Us}{Contact
  Us}
\item
  \href{https://www.nytco.com/careers/}{Work with us}
\item
  \href{https://nytmediakit.com/}{Advertise}
\item
  \href{http://www.tbrandstudio.com/}{T Brand Studio}
\item
  \href{https://www.nytimes3xbfgragh.onion/privacy/cookie-policy\#how-do-i-manage-trackers}{Your
  Ad Choices}
\item
  \href{https://www.nytimes3xbfgragh.onion/privacy}{Privacy}
\item
  \href{https://help.nytimes3xbfgragh.onion/hc/en-us/articles/115014893428-Terms-of-service}{Terms
  of Service}
\item
  \href{https://help.nytimes3xbfgragh.onion/hc/en-us/articles/115014893968-Terms-of-sale}{Terms
  of Sale}
\item
  \href{https://spiderbites.nytimes3xbfgragh.onion}{Site Map}
\item
  \href{https://help.nytimes3xbfgragh.onion/hc/en-us}{Help}
\item
  \href{https://www.nytimes3xbfgragh.onion/subscription?campaignId=37WXW}{Subscriptions}
\end{itemize}
