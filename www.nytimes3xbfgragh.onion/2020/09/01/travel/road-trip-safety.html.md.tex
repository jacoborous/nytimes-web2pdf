Sections

SEARCH

\protect\hyperlink{site-content}{Skip to
content}\protect\hyperlink{site-index}{Skip to site index}

\href{https://www.nytimes3xbfgragh.onion/section/travel}{Travel}

\href{https://myaccount.nytimes3xbfgragh.onion/auth/login?response_type=cookie\&client_id=vi}{}

\href{https://www.nytimes3xbfgragh.onion/section/todayspaper}{Today's
Paper}

\href{/section/travel}{Travel}\textbar{}Help! What Are the Best
Precautions When Traveling by Car?

\url{https://nyti.ms/3lE0HhC}

\begin{itemize}
\item
\item
\item
\item
\item
\item
\end{itemize}

Advertisement

\protect\hyperlink{after-top}{Continue reading the main story}

Supported by

\protect\hyperlink{after-sponsor}{Continue reading the main story}

Tripped Up

\hypertarget{help-what-are-the-best-precautions-when-traveling-by-car}{%
\section{Help! What Are the Best Precautions When Traveling by
Car?}\label{help-what-are-the-best-precautions-when-traveling-by-car}}

Our columnist answers your coronavirus-related questions about health
and safety on road trips.

\includegraphics{https://static01.graylady3jvrrxbe.onion/images/2020/09/05/travel/05trippedup-roadtrip/01trippedup-roadtrip-articleLarge.jpg?quality=75\&auto=webp\&disable=upscale}

By Sarah Firshein

\begin{itemize}
\item
  Sept. 1, 2020
\item
  \begin{itemize}
  \item
  \item
  \item
  \item
  \item
  \item
  \end{itemize}
\end{itemize}

\hypertarget{dear-tripped-up}{%
\subsubsection{Dear Tripped Up,}\label{dear-tripped-up}}

I am 78 and my wife is 76; we're both in good health. We are planning to
drive from Chicago to Santa Monica before Thanksgiving. We are concerned
about how to handle hotels, meals, bathrooms and gas stops during the
pandemic. How can we stay safe? Paul

\hypertarget{dear-paul}{%
\subsubsection{\texorpdfstring{\textbf{Dear
Paul,}}{Dear Paul,}}\label{dear-paul}}

Travel is
\href{https://www.nytimes3xbfgragh.onion/interactive/2020/07/31/travel/coronavirus-travel-risk.html}{complicated}
right now, and tasks that seemed simple a year ago --- like checking
into a hotel or gassing up --- suddenly feel like a huge lift. Americans
are
\href{https://newsroom.aaa.com/2020/06/aaa-forecasts-americans-will-take-700-million-trips-this-summer/}{expected
to take nearly 700 million trips by car} this summer, and I have no
doubt that many of them share some of your uncertainty.

To help answer your road trip questions, I spoke to two public health
experts:
\href{https://www.publichealth.columbia.edu/people/our-faculty/ssa2018}{Sandra
Albrecht}, an assistant professor of epidemiology at Columbia's Mailman
School of Public Health, and
\href{https://www.hsph.harvard.edu/sarah-fortune/}{Sarah Fortune}, the
chair of the Department of Immunology and Infectious Diseases at
Harvard's T.H. Chan School of Public Health. Dr. Fortune just finished a
round-trip drive from Boston to New Orleans.

The first tip both of these experts offered? Accepting the fact that
there is some risk in everything right now.

``If you're going to travel outside your house, you're never going to
get the risk to zero,'' said Dr. Albrecht, who is also the chief
epidemiologist behind
``\href{https://www.instagram.com/dear_pandemic/?hl=en}{Dear
Pandemic,}'' a scientific communication effort on social media. ``That
said, you can travel --- you can enjoy your life. But you should also
engage in smart behaviors and strategies.''

STAYING AT A HOTEL

\hypertarget{do-ask-about-the-occupancy-buffer}{%
\subsubsection{\texorpdfstring{\textbf{Do ask about the occupancy
``buffer.''}}{Do ask about the occupancy ``buffer.''}}\label{do-ask-about-the-occupancy-buffer}}

\href{https://www.nytimes3xbfgragh.onion/2020/06/03/travel/the-most-important-word-in-the-hospitality-industry-clean.html}{Robot
cleaners and U.V. lights} are snazzy-sounding talking points, but
there's a better question to ask about a hotel's cleaning protocols: How
long has the room been unoccupied?

``We're still trying to understand how much of the virus lingers in the
air, but three days is now generally accepted as a good buffer,'' Dr.
Albrecht said. ``Even if you don't clean every nook and cranny of a
particular hotel room, that's a good amount of time to reasonably assume
that the virus has died off.''

If a hotel agent (or vacation-rental owner) can't answer that question,
``it would raise a red flag,'' she said.

\hypertarget{dont-get-lax-about-crowded-spaces}{%
\subsubsection{\texorpdfstring{\textbf{Don't get lax about crowded
spaces.}}{Don't get lax about crowded spaces.}}\label{dont-get-lax-about-crowded-spaces}}

Common spaces like pools and restaurants are closed in many hotels. To
further minimize interaction with strangers, Dr. Albrecht suggested
checking in and out at off-hours --- an industry trend that had
\href{https://www.nytimes3xbfgragh.onion/2020/02/18/travel/hotels-flexible-check-in.html}{already
been on the rise} pre-pandemic.

GRABBING A MEAL

\hypertarget{do-consider-a-dining-contingency-plan}{%
\subsubsection{Do consider a dining contingency
plan.}\label{do-consider-a-dining-contingency-plan}}

The health experts I spoke with agreed that outdoor dining is preferred
to indoor dining.

``We're still learning about indoor transmission, but regardless, most
of us are not going to research the air quality or air circulation
specifics of a particular restaurant,'' Dr. Albrecht said.

Dr. Fortune's experience this summer --- where she intended to only eat
outside but sometimes encountered no outdoor option or a patio already
at capacity --- underscored another road trip rule: When plans don't go
as expected, travelers should consider their own risk tolerance. ``When
you're on the road you've got to eat,'' she said.

And in November, outdoor dining may not be possible anyway.

``If you can actually get your food but eat it wherever it is you're
lodging, that's what would be ideal,'' Dr. Albrecht said. ``That way,
you're still contributing to the local economy.''

\hypertarget{dont-snub-the-chains}{%
\subsubsection{\texorpdfstring{\textbf{Don't snub the
chains.}}{Don't snub the chains.}}\label{dont-snub-the-chains}}

Love `em or hate `em, many big restaurant chains (like the ones Dr.
Fortune saw up and down the Eastern Seaboard) have enacted overarching
standards about masks and social distancing. That uniformity can be a
boon for risk-averse travelers navigating a country where pandemic laws
(and culture) vary so widely.

``Corporate America has really taken best practices to heart, and
they're pretty homogenized by now,'' Dr. Fortune said. ``They just make
it very easy right now to drive through 10 states --- you know there's
always going to be somewhere safe to get food.''

TAKING A BATHROOM BREAK

\hypertarget{do-wash-your-hands-then-wash-them-again}{%
\subsubsection{\texorpdfstring{\textbf{Do wash your hands. (Then, wash
them
again.)}}{Do wash your hands. (Then, wash them again.)}}\label{do-wash-your-hands-then-wash-them-again}}

``If you're somewhere and you need to use the bathroom, use the
bathroom,'' Dr. Albrecht said. ``I wouldn't be paranoid about that.''

She said that restrooms in restaurants, gas stations and the like are
generally fine. Wash your hands with soap and water twice: after
entering and before leaving.

``It's about your own internal risk barometer,'' said Dr. Fortune. ``I'm
pretty risk tolerant --- I'm not crazy, but I'm definitely not carrying
my own personal toilet around. And the bathrooms I saw this summer had
all been scrubbed within an inch of their life.''

\hypertarget{dont-rely-on-old-stalwarts}{%
\subsubsection{Don't rely on old
stalwarts.}\label{dont-rely-on-old-stalwarts}}

In ``before times,'' a large hotel or department store might have been
an obvious place to sneak in a bathroom break. Dr. Fortune said that if
her road trip is any indication, those options can no longer be counted
upon.

``One issue with traveling right now is that more things are closed ---
or if they're not closed, they're closed to people from the street,''
she said. ``And because there aren't many museums and things like that
open, it's important to pay attention to how you're structuring your
day.''

GASSING UP YOUR CAR

\hypertarget{do-keep-hand-sanitizer-in-the-car}{%
\subsubsection{\texorpdfstring{\textbf{Do keep hand sanitizer in the
car.}}{Do keep hand sanitizer in the car.}}\label{do-keep-hand-sanitizer-in-the-car}}

There's no avoiding gas stations on a road trip, but there's also little
reason to be concerned about them, said Dr. Albrecht.

``We do know that fomite transmission --- or transmission of Covid from
surfaces --- is relatively small,'' she said. ``Worst case scenario: You
come up to a gas pump and you don't have anything to clean the surface.
Pump the gas, then use hand sanitizer.''

\hypertarget{do-wear-a-mask}{%
\subsubsection{\texorpdfstring{\textbf{Do wear a
mask.}}{Do wear a mask.}}\label{do-wear-a-mask}}

Gas stations are outdoors, but given how pumps are designed --- often
with two sides, with drivers only separated by a few feet --- health
experts still recommend wearing masks.

``Sometimes it can take a little while to fill your tank, meaning you're
near other people who are not in your household for a not negligible
amount of time,'' Dr. Albrecht said. ``As far as I see, it's just an
easy strategy and I think it will go a long way to help minimize the
risk.''

\begin{center}\rule{0.5\linewidth}{\linethickness}\end{center}

\href{https://twitter.com/sfirshein?lang=en}{Sarah Firshein} is a
Brooklyn-based writer. If you need advice about a best-laid travel plan
that went awry, \textbf{\href{mailto:travel@NYTimes.com}{send an email
to travel@NYTimes.com}.}

\begin{center}\rule{0.5\linewidth}{\linethickness}\end{center}

\emph{\textbf{For more travel coverage}}*, follow us on*
\href{https://twitter.com/nytimestravel}{\emph{Twitter}} \emph{and}
\href{https://www.facebookcorewwwi.onion/nytimestravel/}{\emph{Facebook}}\emph{.
And don't forget to}
\href{https://www.nytimes3xbfgragh.onion/newsletters/traveldispatch?action=click\&module=inline\&pgtype=Article}{\emph{sign
up for our}} **
\href{https://www.nytimes3xbfgragh.onion/newsletters/traveldispatch}{\emph{Travel
Dispatch newsletter}}\emph{: Each week you'll receive tips on traveling
smarter, stories on hot destinations and access to photos from all over
the world.}

Advertisement

\protect\hyperlink{after-bottom}{Continue reading the main story}

\hypertarget{site-index}{%
\subsection{Site Index}\label{site-index}}

\hypertarget{site-information-navigation}{%
\subsection{Site Information
Navigation}\label{site-information-navigation}}

\begin{itemize}
\tightlist
\item
  \href{https://help.nytimes3xbfgragh.onion/hc/en-us/articles/115014792127-Copyright-notice}{©~2020~The
  New York Times Company}
\end{itemize}

\begin{itemize}
\tightlist
\item
  \href{https://www.nytco.com/}{NYTCo}
\item
  \href{https://help.nytimes3xbfgragh.onion/hc/en-us/articles/115015385887-Contact-Us}{Contact
  Us}
\item
  \href{https://www.nytco.com/careers/}{Work with us}
\item
  \href{https://nytmediakit.com/}{Advertise}
\item
  \href{http://www.tbrandstudio.com/}{T Brand Studio}
\item
  \href{https://www.nytimes3xbfgragh.onion/privacy/cookie-policy\#how-do-i-manage-trackers}{Your
  Ad Choices}
\item
  \href{https://www.nytimes3xbfgragh.onion/privacy}{Privacy}
\item
  \href{https://help.nytimes3xbfgragh.onion/hc/en-us/articles/115014893428-Terms-of-service}{Terms
  of Service}
\item
  \href{https://help.nytimes3xbfgragh.onion/hc/en-us/articles/115014893968-Terms-of-sale}{Terms
  of Sale}
\item
  \href{https://spiderbites.nytimes3xbfgragh.onion}{Site Map}
\item
  \href{https://help.nytimes3xbfgragh.onion/hc/en-us}{Help}
\item
  \href{https://www.nytimes3xbfgragh.onion/subscription?campaignId=37WXW}{Subscriptions}
\end{itemize}
