Sections

SEARCH

\protect\hyperlink{site-content}{Skip to
content}\protect\hyperlink{site-index}{Skip to site index}

\href{https://www.nytimes3xbfgragh.onion/section/sports/soccer}{Soccer}

\href{https://myaccount.nytimes3xbfgragh.onion/auth/login?response_type=cookie\&client_id=vi}{}

\href{https://www.nytimes3xbfgragh.onion/section/todayspaper}{Today's
Paper}

\href{/section/sports/soccer}{Soccer}\textbar{}FIFA Is Warned That
Banning Top Games in U.S. May Breach Antitrust Laws

\url{https://nyti.ms/31Nu1KA}

\begin{itemize}
\item
\item
\item
\item
\item
\end{itemize}

Advertisement

\protect\hyperlink{after-top}{Continue reading the main story}

Supported by

\protect\hyperlink{after-sponsor}{Continue reading the main story}

\hypertarget{fifa-is-warned-that-banning-top-games-in-us-may-breach-antitrust-laws}{%
\section{FIFA Is Warned That Banning Top Games in U.S. May Breach
Antitrust
Laws}\label{fifa-is-warned-that-banning-top-games-in-us-may-breach-antitrust-laws}}

The Justice Department told FIFA of its concerns as a lawsuit brought by
an American sports promoter challenges regulations on where
regular-season games involving overseas soccer clubs can be played.

\includegraphics{https://static01.graylady3jvrrxbe.onion/images/2020/09/01/sports/01fifa-doj/merlin_173919507_dd73e7c8-32d5-49fa-9ba9-0a5304b9295c-articleLarge.jpg?quality=75\&auto=webp\&disable=upscale}

By \href{https://www.nytimes3xbfgragh.onion/by/tariq-panja}{Tariq Panja}

\begin{itemize}
\item
  Sept. 1, 2020
\item
  \begin{itemize}
  \item
  \item
  \item
  \item
  \item
  \end{itemize}
\end{itemize}

For more than five years, the United States Department of Justice has
\href{https://www.nytimes3xbfgragh.onion/2015/12/03/sports/fifa-scandal-arrests-in-switzerland.html}{pursued
a corruption case} against the highest levels of international soccer,
charging dozens of the sport's senior officials with crimes like money
laundering and fraud. Now its antitrust division has joined the fight.

The antitrust division is focused on a yearslong dispute over where
matches can take place. FIFA, soccer's global governing body, is
considering strengthening rules that keep teams from playing competitive
regular-season games outside their home countries.

The antitrust division's interest comes as a sports promotion company
based in New York, Relevent Sports, is suing FIFA and the United States
Soccer Federation, accusing the organizations of conspiring to block
Relevent from bringing regular-season games from overseas leagues to
North America.

Relevent, which is owned by
\href{https://www.nytimes3xbfgragh.onion/2020/01/21/sports/soccer/icc-stephen-ross-summer-tour.html}{Stephen
M. Ross}, the real estate developer and principal owner of the Miami
Dolphins, filed its antitrust suit last September after its efforts to
stage a Liga game featuring Barcelona in Miami in 2018 met opposition
from FIFA and the Spanish federation, which blocked the teams from
playing overseas. A later effort to bring two teams from Ecuador to the
United States failed after the U.S. Soccer Federation refused to give
its permission for the game to be played.

Makan Delrahim, the assistant attorney general for the antitrust
division, wrote in March to the FIFA president,
\href{https://www.nytimes3xbfgragh.onion/2018/06/10/sports/gianni-infantino-fifa.html}{Gianni
Infantino}, and Cindy Parlow Cone, who heads U.S. Soccer, to express his
concerns after learning an influential FIFA advisory committee had
recommended to FIFA's governing council that ``official domestic matches
should take place on the territory of the member association
concerned.''

The FIFA Council has yet to ratify the recommendation.

``We specifically are concerned that FIFA could violate U.S. antitrust
laws by restricting the territory in which teams can play league
games,'' Delrahim wrote in the letter, which has not been previously
disclosed. Relevent included the letter as part of its amended complaint
when it filed a new lawsuit in the Southern District of New York on
Tuesday.

FIFA did not immediately reply to a request for comment.

The Justice Department's involvement adds a new dimension to the case
brought by Relevent. FIFA has tried to cultivate a close relationship
with U.S. authorities since the corruption indictments were handed down
in 2015, dismantling the top tier of international soccer's leadership.
FIFA's top lawyers have been in regular contact with American justice
officials in an attempt to prove the organization has changed
significantly and should be repaid some of the hundreds of millions of
dollars in restitution seized from defendants who admitted to
participating in bribery and kickback schemes.

Despite the changes, FIFA is still mired in legal troubles. The most
serious of them concerns its president, Infantino, who was informed in
July that
\href{https://www.nytimes3xbfgragh.onion/2020/07/30/sports/soccer/fifa-gianni-infantino-investigation.html}{he
is the subject of a criminal investigation} over his meetings with the
Swiss official overseeing the FIFA corruption inquiry. Infantino has
said any claims of improper behavior are ``absurd.'' The official,
Michael Lauber, has resigned as attorney general and last week was
stripped of immunity from prosecution.

FIFA has also invested significant time and effort in strengthening its
relationships in the United States after awarding the right to host
\href{https://www.nytimes3xbfgragh.onion/2018/06/04/sports/2026-world-cup-bid.html}{the
2026 World Cup} to a U.S.-led bid that included Canada and Mexico.
Infantino has visited the White House for meetings with President Trump,
and was also invited in January to address guests at a dinner that Trump
hosted at the World Economic Forum in Davos, Switzerland.

Relevent's announcement of a deal with La Liga for regular-season games
to be played in the United States, starting with the 2018 game between
Barcelona and Girona, has revived a contentious issue that had largely
been dormant since the English Premier League was forced to withdraw
plans for an international round of games, proposed in 2008, after an
outcry at home and overseas.

While efforts like Relevent's continue to face opposition from national
federations and traditionalists, the growing popularity of European
teams beyond their traditional markets has increased a demand for
competitive matches played far from teams' home arenas. Such games would
be worth millions of dollars in additional revenue for participants. In
recent years, Spain and Italy have signed contracts to play cup games in
Saudi Arabia.

Staging such games could pose a threat to interest in local
competitions, like Major League Soccer, the professional U.S. league.
Don Garber, the league's commissioner, who sits on the FIFA stakeholders
committee that recommends prohibiting overseas games, expressed his
opposition this year. ``I feel very strongly that local fans should have
the opportunity to see local games, and not for other purposes have
those games played outside the home market,'' Garber told ESPN in
February.

FIFA rules currently allow games to be played on overseas territory only
under ``exceptional circumstances.'' After the outcry over the plan to
play the Liga game in Miami, the FIFA Council doubled down, issuing a
declaration after a quarterly meeting in Kigali, Rwanda, in 2018.

``The council emphasized the sporting principle that official league
matches must be played within the territory of the respective member
association,'' FIFA said at the time.

Top U.S. sports leagues like the N.F.L., the N.B.A. and Major League
Baseball have in recent years embraced taking their teams abroad for
official games in hopes of reaching new audiences, with
\href{https://www.nytimes3xbfgragh.onion/2015/10/25/sports/football/nfl-pushes-deeper-into-overseas-market.html}{regular-season
football becoming a feature of London's sports scene}. Relevent is
arguing that American audiences are, by contrast, being deprived of the
opportunity to see the best soccer.

``This geographic market division unreasonably restrains competition in
the U.S., reduces output below competitive levels in the relevant
market, and directly inflicts antitrust injury on promoters, like
Relevent (who directly participate in the relevant market) and on U.S.
fans of top-tier men's professional soccer leagues,'' Relevent said,
according to a copy of its filing reviewed by The New York Times.

Advertisement

\protect\hyperlink{after-bottom}{Continue reading the main story}

\hypertarget{site-index}{%
\subsection{Site Index}\label{site-index}}

\hypertarget{site-information-navigation}{%
\subsection{Site Information
Navigation}\label{site-information-navigation}}

\begin{itemize}
\tightlist
\item
  \href{https://help.nytimes3xbfgragh.onion/hc/en-us/articles/115014792127-Copyright-notice}{©~2020~The
  New York Times Company}
\end{itemize}

\begin{itemize}
\tightlist
\item
  \href{https://www.nytco.com/}{NYTCo}
\item
  \href{https://help.nytimes3xbfgragh.onion/hc/en-us/articles/115015385887-Contact-Us}{Contact
  Us}
\item
  \href{https://www.nytco.com/careers/}{Work with us}
\item
  \href{https://nytmediakit.com/}{Advertise}
\item
  \href{http://www.tbrandstudio.com/}{T Brand Studio}
\item
  \href{https://www.nytimes3xbfgragh.onion/privacy/cookie-policy\#how-do-i-manage-trackers}{Your
  Ad Choices}
\item
  \href{https://www.nytimes3xbfgragh.onion/privacy}{Privacy}
\item
  \href{https://help.nytimes3xbfgragh.onion/hc/en-us/articles/115014893428-Terms-of-service}{Terms
  of Service}
\item
  \href{https://help.nytimes3xbfgragh.onion/hc/en-us/articles/115014893968-Terms-of-sale}{Terms
  of Sale}
\item
  \href{https://spiderbites.nytimes3xbfgragh.onion}{Site Map}
\item
  \href{https://help.nytimes3xbfgragh.onion/hc/en-us}{Help}
\item
  \href{https://www.nytimes3xbfgragh.onion/subscription?campaignId=37WXW}{Subscriptions}
\end{itemize}
