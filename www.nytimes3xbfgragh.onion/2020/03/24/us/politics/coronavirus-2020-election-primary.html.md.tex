Sections

SEARCH

\protect\hyperlink{site-content}{Skip to
content}\protect\hyperlink{site-index}{Skip to site index}

\href{https://www.nytimes3xbfgragh.onion/section/politics}{Politics}

\href{https://myaccount.nytimes3xbfgragh.onion/auth/login?response_type=cookie\&client_id=vi}{}

\href{https://www.nytimes3xbfgragh.onion/section/todayspaper}{Today's
Paper}

\href{/section/politics}{Politics}\textbar{}Is All of 2020 Postponed?

\url{https://nyti.ms/2JcPe6D}

\begin{itemize}
\item
\item
\item
\item
\item
\end{itemize}

\begin{itemize}
\item
  \href{https://www.nytimes3xbfgragh.onion/interactive/2020/09/08/us/elections/results-new-hampshire-primary-elections.html?action=click\&pgtype=Article\&state=default\&region=TOP_BANNER\&context=storylines_menu}{New
  Hampshire Results}
\item
  \href{https://www.nytimes3xbfgragh.onion/live/2020/09/08/us/trump-vs-biden?action=click\&pgtype=Article\&state=default\&region=TOP_BANNER\&context=storylines_menu}{Election
  Updates}
\item
  \href{https://www.nytimes3xbfgragh.onion/interactive/2020/us/elections/election-states-biden-trump.html?action=click\&pgtype=Article\&state=default\&region=TOP_BANNER\&context=storylines_menu}{Paths
  to 270}
\item
  \href{https://www.nytimes3xbfgragh.onion/interactive/2020/08/31/us/politics/vote-by-mail-deadlines.html?action=click\&pgtype=Article\&state=default\&region=TOP_BANNER\&context=storylines_menu}{Voting
  by Mail}
\item
  \href{https://www.nytimes3xbfgragh.onion/interactive/2019/us/elections/2020-presidential-election-calendar.html?action=click\&pgtype=Article\&state=default\&region=TOP_BANNER\&context=storylines_menu}{Key
  Dates}
\item
  \href{https://www.nytimes3xbfgragh.onion/newsletters/politics?action=click\&pgtype=Article\&state=default\&region=TOP_BANNER\&context=storylines_menu}{Politics
  Newsletter}
\end{itemize}

Advertisement

\protect\hyperlink{after-top}{Continue reading the main story}

Supported by

\protect\hyperlink{after-sponsor}{Continue reading the main story}

Campaign Memo

\hypertarget{is-all-of-2020-postponed}{%
\section{Is All of 2020 Postponed?}\label{is-all-of-2020-postponed}}

Primary elections are being pushed back. The Olympics are officially in
2021. The election-year calendar seemed set in stone and now it is up in
the air.

\includegraphics{https://static01.graylady3jvrrxbe.onion/images/2020/03/24/us/politics/24campaign-memo/merlin_170592744_98b0a11d-aec6-4070-a79f-e7d1cd91fafc-articleLarge.jpg?quality=75\&auto=webp\&disable=upscale}

\href{https://www.nytimes3xbfgragh.onion/by/matt-flegenheimer}{\includegraphics{https://static01.graylady3jvrrxbe.onion/images/2018/10/02/multimedia/author-matt-flegenheimer/author-matt-flegenheimer-thumbLarge.png}}

By \href{https://www.nytimes3xbfgragh.onion/by/matt-flegenheimer}{Matt
Flegenheimer}

\begin{itemize}
\item
  March 24, 2020
\item
  \begin{itemize}
  \item
  \item
  \item
  \item
  \item
  \end{itemize}
\end{itemize}

There is a rhythm to it all, in happier times, a procession of dates and
checkpoints to spread the politics and patriotism neatly across the
calendar.

The presidential primaries tick past, Tuesday by Tuesday, through the
spring. Former rivals hug it out at their party conventions in the
summer. The Olympics bring the sides together, briefly, in merry
distraction and shared cause before the fall. And then comes November,
when half of the country is disappointed again.

So, what happens when immovable dates become negotiable --- when
everything does --- in the throes of a pandemic? What must hold firm
when nothing seems to?

Primaries are postponed. Hugs are postponed. The Olympics are postponed.

The November election, everyone appears to agree, cannot be.

``We voted in the middle of a Civil War,''
\href{https://www.nytimes3xbfgragh.onion/interactive/2020/us/elections/joe-biden.html}{Joseph
R. Biden Jr.}, the heavy favorite for the Democratic nomination, told
supporters at a tele-fund-raiser on Sunday. ``We voted in the middle of
World War I and II. And so, the idea of postponing the electoral process
is just --- seems to me, out of the question.''

And yet for weeks, the moment has delivered one reminder after the next
of all that the crisis has made wobbly, all that seemed certain before
certainty went into quarantine.

Dates exist in stone until they don't. Younger people are spared the
worst of the virus's ravages until
\href{https://www.nytimes3xbfgragh.onion/2020/03/18/health/coronavirus-young-people.html}{they
aren't}. American ideals, in 2020, leave no room for the concession of
mass death, surely. But then here is a White House now
\href{https://www.nytimes3xbfgragh.onion/2020/03/23/us/politics/trump-coronavirus-restrictions.html}{suggesting}
that the cure (stifling business for a period to encourage social
distancing) cannot become worse than the disease (the disease).

The result is as disorienting as the messaging. What does progress look
like when the healthiest course is collective stasis? What is there to
drive toward, to look forward to, when the common benchmarks of the year
--- weddings, athletic seasons, state elections --- seem to be punted
further from reach?

``We're not that familiar with a sense of tribulation as a part of daily
life,'' said Jill Lepore, a professor of history at Harvard University
and author of ``These Truths: A History of the United States.'' ``The
basic American middle-class is not used to the idea that time could be
moving backwards. And at the moment, what it feels like for most people
is almost that time is standing still.''

Already, meaningful electoral proceedings have been delayed or recast.
Many states have pushed their primaries into June in the hopes of
waiting out the worst of the virus. Pennsylvania, slated for April 28,
is likely to be next. Senator
\href{https://www.nytimes3xbfgragh.onion/interactive/2020/us/elections/bernie-sanders.html}{Bernie
Sanders}, Mr. Biden's chief rival, has effectively converted his
campaign into a pandemic policy shop and vessel for progressive activism
as he confronts a significant delegate deficit. Planners for the
Democratic National Convention
\href{https://www.nytimes3xbfgragh.onion/2020/03/23/us/politics/democratic-convention-milwaukee-coronavirus.html}{said
this week} that they were assessing ``contingency options'' in case the
July gathering cannot go on as planned.

But the general election is another matter. No one in a position of
relevant authority has proposed moving it, though the subject has
instantly become a cause of angst for President Trump's critics. These
fears seem to owe more to Mr. Trump's attacks on democratic norms and
institutions during his time in office than anything the president has
said of late.

So far, Mr. Trump has appeared inclined to defy the guidance of public
health experts by suggesting Americans return to their workplaces and
public engagements well before the coronavirus has been tamed.

Even if he wished to delay the November election, the decision would
appear to be out of his hands. Any change to the date
\href{https://www.nytimes3xbfgragh.onion/2020/03/14/us/politics/election-postponed-canceled.html}{would
require} federal legislation, passed by Congress, to say nothing of
challenges in the court system.

The prospect of that kind of bipartisan collaboration on an issue of
this magnitude is exceedingly slim.

Still, some academics have processed the official coronavirus response
with alarm, warning that the present blend of institutional distrust and
health-minded limits on personal liberties could prove dangerous.

``In the next months, defenders of democracy need to sustain this very
careful balancing act between overreacting to justified emergency
measures on the one side and not easily going along with real attacks on
our democratic institutions on the other,'' said Yascha Mounk, an
associate professor at Johns Hopkins University who has written
extensively about threats to liberal democracy. ``That's going to be
incredibly difficult.''

Of heightened concern to some Democrats is the specter of state leaders
citing the virus to
\href{https://www.thenation.com/article/politics/trump-stealing-election/}{impose
voting constraints} that might disproportionately affect left-leaning
cities. ``They will find a way to make it hard,'' said Amanda Litman,
the executive director of Run for Something, an organization that helps
Democrats run for local office. ``And we have to find a way to mitigate
that as much as possible --- at home, without a lot of funds.''

Even absent any cynical application of voting restrictions, some
lawmakers worry that lingering civic confusion and uncertainty might
hamper turnout with key constituencies.

On Tuesday, a group of House Democrats addressed a
\href{https://khanna.house.gov/sites/khanna.house.gov/files/Shelter\%20in\%20Place\%20Letter.pdf}{letter}
to the White House requesting a shelter-in-place order of at least two
weeks for the entire country, arguing that such an approach made sense
on both public safety and long-term economic grounds.

The note's lead author, Representative Ro Khanna of California, said
that part of the urgency flowed from a need to head off as many
disruptions and complications as possible before November.

``Getting a shelter-in-place now for two to three weeks would mitigate
the risk of having a catastrophe or re-emergence closer to November,''
Mr. Khanna, a national campaign co-chair for Mr. Sanders, said in an
interview.

He added that officials should be exploring alternatives to traditional
voting like
\href{https://www.nytimes3xbfgragh.onion/2020/03/19/us/politics/voting-by-mail-coronavirus.html}{vote-by-mail}
wherever possible.

In the interim, Ms. Lepore, the historian, said it was dislocating
enough merely fumbling through the unknown. ``We keep saying around here
in my household: `If we knew it was two weeks, we could pace
ourselves,''' she said. ```If we knew it was a month, we could pace
ourselves.'''

But then, what about six months?

What about November?

\hypertarget{our-2020-election-guide}{%
\section{Our 2020 Election Guide}\label{our-2020-election-guide}}

Updated ~Sept. 8, 2020

\begin{center}\rule{0.5\linewidth}{\linethickness}\end{center}

\begin{itemize}
\item ~
  \hypertarget{the-latest}{%
  \subsection{The Latest}\label{the-latest}}

  \begin{itemize}
  \item
    President Trump and his party are using a playbook that aims to
    alarm people about crime in their backyards. It didn't work in 2018,
    but
    \href{https://www.nytimes3xbfgragh.onion/2020/09/08/us/politics/trump-republicans-fear-strategy.html?action=click\&pgtype=Article\&state=default\&region=BELOW_MAIN_CONTENT\&context=storylines_guide}{both
    parties think it could resonate more this year}.
  \end{itemize}
\item ~
  \hypertarget{how-to-win-270}{%
  \subsection{How to Win 270}\label{how-to-win-270}}

  \begin{itemize}
  \item
    Joe Biden and Donald Trump need 270 electoral votes to reach the
    White House. Try building
    \href{https://www.nytimes3xbfgragh.onion/interactive/2020/us/elections/election-states-biden-trump.html?action=click\&pgtype=Article\&state=default\&region=BELOW_MAIN_CONTENT\&context=storylines_guide}{your
    own coalition of battleground states}~to see potential outcomes.
  \end{itemize}
\item ~
  \hypertarget{voting-by-mail}{%
  \subsection{Voting by Mail}\label{voting-by-mail}}

  \begin{itemize}
  \item
    Will you have enough time to vote by mail in your state? Yes, but
    it's risky to procrastinate.
    \href{https://www.nytimes3xbfgragh.onion/interactive/2020/08/31/us/politics/vote-by-mail-deadlines.html?action=click\&pgtype=Article\&state=default\&region=BELOW_MAIN_CONTENT\&context=storylines_guide}{Check
    your state's deadline.}
  \item
    \href{https://www.nytimes3xbfgragh.onion/interactive/2020/us/elections/joe-biden.html?action=click\&pgtype=Article\&state=default\&region=BELOW_MAIN_CONTENT\&context=storylines_guide}{}

    \hypertarget{joe-biden}{%
    \section{Joe Biden}\label{joe-biden}}

    \hypertarget{democrat}{%
    \subsection{Democrat}\label{democrat}}

    \href{https://www.nytimes3xbfgragh.onion/interactive/2020/us/elections/donald-trump.html?action=click\&pgtype=Article\&state=default\&region=BELOW_MAIN_CONTENT\&context=storylines_guide}{}

    \hypertarget{donald-trump}{%
    \section{Donald Trump}\label{donald-trump}}

    \hypertarget{republican}{%
    \subsection{Republican}\label{republican}}
  \end{itemize}
\item
  \hypertarget{keep-up-with-our-coverage}{%
  \subsection{Keep Up With Our
  Coverage}\label{keep-up-with-our-coverage}}

  \begin{itemize}
  \item
    Get an
    \href{https://www.nytimes3xbfgragh.onion/newsletters/politics?action=click\&pgtype=Article\&state=default\&region=BELOW_MAIN_CONTENT\&context=storylines_guide}{email}~recapping
    the day's news
  \item
    Download our mobile app on
    \href{https://apps.apple.com/us/app/nytimes/id284862083?ls=1\&mat_click_id=5c79ae7455014fd1bd66b5610c05b8f2-20191112-16948\&referrer=mat_click_id\%3D5c79ae7455014fd1bd66b5610c05b8f2-20191112-16948\%26link_click_id\%3D722930677036718082}{iOS}~and
    \href{http://a.localytics.com/android?id=com.nytimes.android\&referrer=utm_source\%3Dother_nyt_mobile_web\%26utm_medium\%3DWeb\%2520page\%26utm_term\%3DGeneral\%2520Mobile\%2520Page\%26utm_campaign\%3DNYT\%2520Mobile\%2520General\%2520Page}{Android}~and
    turn on Breaking News and Politics alerts
  \end{itemize}
\end{itemize}

Advertisement

\protect\hyperlink{after-bottom}{Continue reading the main story}

\hypertarget{site-index}{%
\subsection{Site Index}\label{site-index}}

\hypertarget{site-information-navigation}{%
\subsection{Site Information
Navigation}\label{site-information-navigation}}

\begin{itemize}
\tightlist
\item
  \href{https://help.nytimes3xbfgragh.onion/hc/en-us/articles/115014792127-Copyright-notice}{©~2020~The
  New York Times Company}
\end{itemize}

\begin{itemize}
\tightlist
\item
  \href{https://www.nytco.com/}{NYTCo}
\item
  \href{https://help.nytimes3xbfgragh.onion/hc/en-us/articles/115015385887-Contact-Us}{Contact
  Us}
\item
  \href{https://www.nytco.com/careers/}{Work with us}
\item
  \href{https://nytmediakit.com/}{Advertise}
\item
  \href{http://www.tbrandstudio.com/}{T Brand Studio}
\item
  \href{https://www.nytimes3xbfgragh.onion/privacy/cookie-policy\#how-do-i-manage-trackers}{Your
  Ad Choices}
\item
  \href{https://www.nytimes3xbfgragh.onion/privacy}{Privacy}
\item
  \href{https://help.nytimes3xbfgragh.onion/hc/en-us/articles/115014893428-Terms-of-service}{Terms
  of Service}
\item
  \href{https://help.nytimes3xbfgragh.onion/hc/en-us/articles/115014893968-Terms-of-sale}{Terms
  of Sale}
\item
  \href{https://spiderbites.nytimes3xbfgragh.onion}{Site Map}
\item
  \href{https://help.nytimes3xbfgragh.onion/hc/en-us}{Help}
\item
  \href{https://www.nytimes3xbfgragh.onion/subscription?campaignId=37WXW}{Subscriptions}
\end{itemize}
