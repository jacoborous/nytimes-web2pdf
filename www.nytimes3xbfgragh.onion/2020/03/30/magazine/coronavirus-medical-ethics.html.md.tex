Sections

SEARCH

\protect\hyperlink{site-content}{Skip to
content}\protect\hyperlink{site-index}{Skip to site index}

\href{https://myaccount.nytimes3xbfgragh.onion/auth/login?response_type=cookie\&client_id=vi}{}

\href{https://www.nytimes3xbfgragh.onion/section/todayspaper}{Today's
Paper}

What Are My Obligations if I'm a Doctor Who Is High Risk?

\url{https://nyti.ms/2X6de3F}

\begin{itemize}
\item
\item
\item
\item
\item
\item
\end{itemize}

\hypertarget{the-coronavirus-outbreak}{%
\subsubsection{\texorpdfstring{\href{https://www.nytimes3xbfgragh.onion/news-event/coronavirus?name=styln-coronavirus-national\&region=TOP_BANNER\&block=storyline_menu_recirc\&action=click\&pgtype=Article\&impression_id=695e0320-f1cd-11ea-a3e0-dfa223dac892\&variant=undefined}{The
Coronavirus
Outbreak}}{The Coronavirus Outbreak}}\label{the-coronavirus-outbreak}}

\begin{itemize}
\tightlist
\item
  live\href{https://www.nytimes3xbfgragh.onion/2020/09/08/world/covid-19-coronavirus.html?name=styln-coronavirus-national\&region=TOP_BANNER\&block=storyline_menu_recirc\&action=click\&pgtype=Article\&impression_id=695e2a30-f1cd-11ea-a3e0-dfa223dac892\&variant=undefined}{Latest
  Updates}
\item
  \href{https://www.nytimes3xbfgragh.onion/interactive/2020/us/coronavirus-us-cases.html?name=styln-coronavirus-national\&region=TOP_BANNER\&block=storyline_menu_recirc\&action=click\&pgtype=Article\&impression_id=695e2a31-f1cd-11ea-a3e0-dfa223dac892\&variant=undefined}{Maps
  and Cases}
\item
  \href{https://www.nytimes3xbfgragh.onion/interactive/2020/science/coronavirus-vaccine-tracker.html?name=styln-coronavirus-national\&region=TOP_BANNER\&block=storyline_menu_recirc\&action=click\&pgtype=Article\&impression_id=695e2a32-f1cd-11ea-a3e0-dfa223dac892\&variant=undefined}{Vaccine
  Tracker}
\item
  \href{https://www.nytimes3xbfgragh.onion/2020/09/02/your-money/eviction-moratorium-covid.html?name=styln-coronavirus-national\&region=TOP_BANNER\&block=storyline_menu_recirc\&action=click\&pgtype=Article\&impression_id=695e2a33-f1cd-11ea-a3e0-dfa223dac892\&variant=undefined}{Eviction
  Moratorium}
\item
  \href{https://www.nytimes3xbfgragh.onion/interactive/2020/09/02/magazine/food-insecurity-hunger-us.html?name=styln-coronavirus-national\&region=TOP_BANNER\&block=storyline_menu_recirc\&action=click\&pgtype=Article\&impression_id=695e2a34-f1cd-11ea-a3e0-dfa223dac892\&variant=undefined}{American
  Hunger}
\end{itemize}

Advertisement

\protect\hyperlink{after-top}{Continue reading the main story}

Supported by

\protect\hyperlink{after-sponsor}{Continue reading the main story}

\href{/column/the-ethicist}{The Ethicist}

\hypertarget{what-are-my-obligations-if-im-a-doctor-who-is-high-risk}{%
\section{What Are My Obligations if I'm a Doctor Who Is High
Risk?}\label{what-are-my-obligations-if-im-a-doctor-who-is-high-risk}}

\includegraphics{https://static01.graylady3jvrrxbe.onion/images/2020/04/05/magazine/05Ethicist/05Ethicist-articleLarge.jpg?quality=75\&auto=webp\&disable=upscale}

By Kwame Anthony Appiah

\begin{itemize}
\item
  March 30, 2020
\item
  \begin{itemize}
  \item
  \item
  \item
  \item
  \item
  \item
  \end{itemize}
\end{itemize}

\emph{I am a physician in an urgent-care setting in the Pacific
Northwest, and thus I am at some risk for exposure to coronavirus even
when following appropriate precautions. I am in a category that is
considered high risk for complications were I to become ill with the
coronavirus. The C.D.C. has recommended as of March 8 that individuals
at high risk try to limit their public exposure.}

\emph{What is my ethical obligation to my community, to my colleagues,
to my spouse and to my employer in this situation? What is my employer's
ethical obligation to me and to others who may have similar risk
factors?} Name Withheld

\textbf{Health care workers} around the world are making heroic
contributions to fighting the pandemic, which has taken an especially
high toll on them. Many people have now heard of
\href{https://www.nytimes3xbfgragh.onion/2020/02/07/world/asia/china-coronavirus-doctor-death.html}{Dr.
Li Wenliang}, the ophthalmologist who tried to warn the medical
community in Wuhan as the new coronavirus first started to make its way
through the population and was taken into custody by the police for
``spreading false rumors.'' His death, on Feb. 7, caused an outpouring
of anger and grief in China and elsewhere. The disease claimed the lives
of at least three other physicians at Li's hospital alone.

On March 18, Dr. Marcello Natali, who helped lead the response in
Codogno, the center of the Covid-19 pandemic in Northern Italy, died as
well, at a time when health care workers represented a significant
proportion of Italians stricken with the disease. Health care workers
are a society's first line of defense, and many have been working long
hours with inadequate supplies, while dealing with the stressful
possibility of falling ill themselves, of infecting their families or of
spreading the virus to their patients. The rest of us owe you and your
co-workers a huge debt of gratitude --- and a good way for us to show
that gratitude is by following the recommendations for social distancing
and staying at home whenever possible.

But heroes are not people who take unnecessary risks. They are people
who respond intelligently to a challenge, assessing the likelihoods of
hazards and benefits, and making a judgment about which chances are
worth taking. As Aristotle put it, courage means that you ``endure or
fear the right things and for the right purpose and in the right manner
and at the right time.''

\hypertarget{latest-updates-the-coronavirus-outbreak}{%
\section{\texorpdfstring{\href{https://www.nytimes3xbfgragh.onion/2020/09/08/world/covid-19-coronavirus.html?action=click\&pgtype=Article\&state=default\&region=MAIN_CONTENT_1\&context=storylines_live_updates}{Latest
Updates: The Coronavirus
Outbreak}}{Latest Updates: The Coronavirus Outbreak}}\label{latest-updates-the-coronavirus-outbreak}}

Updated 2020-09-08T12:11:35.149Z

\begin{itemize}
\tightlist
\item
  \href{https://www.nytimes3xbfgragh.onion/2020/09/08/world/covid-19-coronavirus.html?action=click\&pgtype=Article\&state=default\&region=MAIN_CONTENT_1\&context=storylines_live_updates\#link-46162376}{Trillions
  of dollars separate lawmakers' proposals for virus relief.}
\item
  \href{https://www.nytimes3xbfgragh.onion/2020/09/08/world/covid-19-coronavirus.html?action=click\&pgtype=Article\&state=default\&region=MAIN_CONTENT_1\&context=storylines_live_updates\#link-679303d7}{Nine
  drugmakers pledge to thoroughly vet any coronavirus vaccine.}
\item
  \href{https://www.nytimes3xbfgragh.onion/2020/09/08/world/covid-19-coronavirus.html?action=click\&pgtype=Article\&state=default\&region=MAIN_CONTENT_1\&context=storylines_live_updates\#link-1c973131}{`The
  lockdown killed my father': Farmer suicides add to India's virus
  misery.}
\end{itemize}

\href{https://www.nytimes3xbfgragh.onion/2020/09/08/world/covid-19-coronavirus.html?action=click\&pgtype=Article\&state=default\&region=MAIN_CONTENT_1\&context=storylines_live_updates}{See
more updates}

More live coverage:
\href{https://www.nytimes3xbfgragh.onion/live/2020/09/08/business/stock-market-today-coronavirus?action=click\&pgtype=Article\&state=default\&region=MAIN_CONTENT_1\&context=storylines_live_updates}{Markets}

So what's the right purpose and manner for you at this time? Begin with
the fact that, as a doctor who has taken the Hippocratic oath, you have
assumed an ethically distinctive commitment: You've agreed to bear
certain risks necessary to the performance of your vocation. But you
won't be of use to anyone if you get seriously ill. And as happened in
the early stages of the Wuhan outbreak, health care workers who aren't
properly protected can themselves become significant sources of
infection.

Hospital systems, working with federal and state health officials, have
increasingly taken measures to try to keep staff members safe, including
the routine use of basic protection (masks, gloves), hand hygiene,
careful distancing and the like; using higher levels of protection when
dealing with certain patients or with respiratory procedures; asking
health care personnel with symptoms to stay home. Not a few hospitals
have been hampered by woefully inadequate supplies. But, I'm assured by
a professor at Harvard's school of public health, these protocols, when
implemented, really do seem to work. As always in ethical life, facts
matter, and it's incumbent on responsible people --- and organizations
--- to get the relevant facts right.

So your employers have an obligation to do their best to ensure, first,
that you and your colleagues have the resources needed to practice the
proper forms of hygiene; and second, that everyone in your system
rigorously adheres to the appropriate rules. That won't bring the chance
of your getting ill down to zero, but concerted action and the right
equipment can keep it low. In urgent care (which involves cases that
need prompt attention but not E.R.-level interventions), you'll
typically have time to approach newly arrived patients after a proper
assessment of the risks --- a precaution important to their well-being
and to yours.

You also have obligations, as you recognize, that are shaped not just by
your workplace responsibilities to colleagues and patients but also by
your ties to family and friends. That means (to repeat something that
bears repeating) observing social distancing and proper hand hygiene
away from work too. The real challenge for health care workers, as for
the rest of us, may be keeping your guard up over the weeks and months
ahead. But you may be able to help your colleagues here --- perhaps by
letting them know that the stakes for you are higher than for most.

\emph{I was supposed to have a friend over, but she and her husband were
ill, so we rescheduled. Since then she has become very ill with
coronavirus-like symptoms. She thinks it's just the flu but will not see
a doctor or explore testing options. Can I move the date again because I
do not want to risk infection? The dinner, though casual, is to
celebrate my friend's birthday.} Name Withheld

\textbf{A crisis like} this one brings out the ways in which we are all
united through a web of connections and thus of mutual responsibilities.
In the current circumstances, an untested person with a fever, dry cough
or unusual fatigue should assume she's carrying and shedding the
coronavirus and practice self-isolation, as every responsible body of
experts has recommended. Even if your friend has the flu, she ought to
be concerned to limit its spread. Especially these days, we need to
avoid adding to the burden of an already overburdened health care
system.

Sadly, a majority of Americans failed to get vaccinated by February, a
date well into the flu season. If more had done so, we would have had
fewer hospitalizations and more resources at our disposal for this new
threat. People who don't suffer much themselves from such infections can
spread them to those who do.

\href{https://www.nytimes3xbfgragh.onion/news-event/coronavirus?action=click\&pgtype=Article\&state=default\&region=MAIN_CONTENT_3\&context=storylines_faq}{}

\hypertarget{the-coronavirus-outbreak-}{%
\subsubsection{The Coronavirus Outbreak
›}\label{the-coronavirus-outbreak-}}

\hypertarget{frequently-asked-questions}{%
\paragraph{Frequently Asked
Questions}\label{frequently-asked-questions}}

Updated September 4, 2020

\begin{itemize}
\item ~
  \hypertarget{what-are-the-symptoms-of-coronavirus}{%
  \paragraph{What are the symptoms of
  coronavirus?}\label{what-are-the-symptoms-of-coronavirus}}

  \begin{itemize}
  \tightlist
  \item
    In the beginning, the coronavirus
    \href{https://www.nytimes3xbfgragh.onion/article/coronavirus-facts-history.html?action=click\&pgtype=Article\&state=default\&region=MAIN_CONTENT_3\&context=storylines_faq\#link-6817bab5}{seemed
    like it was primarily a respiratory illness}~--- many patients had
    fever and chills, were weak and tired, and coughed a lot, though
    some people don't show many symptoms at all. Those who seemed
    sickest had pneumonia or acute respiratory distress syndrome and
    received supplemental oxygen. By now, doctors have identified many
    more symptoms and syndromes. In April,
    \href{https://www.nytimes3xbfgragh.onion/2020/04/27/health/coronavirus-symptoms-cdc.html?action=click\&pgtype=Article\&state=default\&region=MAIN_CONTENT_3\&context=storylines_faq}{the
    C.D.C. added to the list of early signs}~sore throat, fever, chills
    and muscle aches. Gastrointestinal upset, such as diarrhea and
    nausea, has also been observed. Another telltale sign of infection
    may be a sudden, profound diminution of one's
    \href{https://www.nytimes3xbfgragh.onion/2020/03/22/health/coronavirus-symptoms-smell-taste.html?action=click\&pgtype=Article\&state=default\&region=MAIN_CONTENT_3\&context=storylines_faq}{sense
    of smell and taste.}~Teenagers and young adults in some cases have
    developed painful red and purple lesions on their fingers and toes
    --- nicknamed ``Covid toe'' --- but few other serious symptoms.
  \end{itemize}
\item ~
  \hypertarget{why-is-it-safer-to-spend-time-together-outside}{%
  \paragraph{Why is it safer to spend time together
  outside?}\label{why-is-it-safer-to-spend-time-together-outside}}

  \begin{itemize}
  \tightlist
  \item
    \href{https://www.nytimes3xbfgragh.onion/2020/05/15/us/coronavirus-what-to-do-outside.html?action=click\&pgtype=Article\&state=default\&region=MAIN_CONTENT_3\&context=storylines_faq}{Outdoor
    gatherings}~lower risk because wind disperses viral droplets, and
    sunlight can kill some of the virus. Open spaces prevent the virus
    from building up in concentrated amounts and being inhaled, which
    can happen when infected people exhale in a confined space for long
    stretches of time, said Dr. Julian W. Tang, a virologist at the
    University of Leicester.
  \end{itemize}
\item ~
  \hypertarget{why-does-standing-six-feet-away-from-others-help}{%
  \paragraph{Why does standing six feet away from others
  help?}\label{why-does-standing-six-feet-away-from-others-help}}

  \begin{itemize}
  \tightlist
  \item
    The coronavirus spreads primarily through droplets from your mouth
    and nose, especially when you cough or sneeze. The C.D.C., one of
    the organizations using that measure,
    \href{https://www.nytimes3xbfgragh.onion/2020/04/14/health/coronavirus-six-feet.html?action=click\&pgtype=Article\&state=default\&region=MAIN_CONTENT_3\&context=storylines_faq}{bases
    its recommendation of six feet}~on the idea that most large droplets
    that people expel when they cough or sneeze will fall to the ground
    within six feet. But six feet has never been a magic number that
    guarantees complete protection. Sneezes, for instance, can launch
    droplets a lot farther than six feet,
    \href{https://jamanetwork.com/journals/jama/fullarticle/2763852}{according
    to a recent study}. It's a rule of thumb: You should be safest
    standing six feet apart outside, especially when it's windy. But
    keep a mask on at all times, even when you think you're far enough
    apart.
  \end{itemize}
\item ~
  \hypertarget{i-have-antibodies-am-i-now-immune}{%
  \paragraph{I have antibodies. Am I now
  immune?}\label{i-have-antibodies-am-i-now-immune}}

  \begin{itemize}
  \tightlist
  \item
    As of right
    now,\href{https://www.nytimes3xbfgragh.onion/2020/07/22/health/covid-antibodies-herd-immunity.html?action=click\&pgtype=Article\&state=default\&region=MAIN_CONTENT_3\&context=storylines_faq}{~that
    seems likely, for at least several months.}~There have been
    frightening accounts of people suffering what seems to be a second
    bout of Covid-19. But experts say these patients may have a
    drawn-out course of infection, with the virus taking a slow toll
    weeks to months after initial exposure.~People infected with the
    coronavirus typically
    \href{https://www.nature.com/articles/s41586-020-2456-9}{produce}~immune
    molecules called antibodies, which are
    \href{https://www.nytimes3xbfgragh.onion/2020/05/07/health/coronavirus-antibody-prevalence.html?action=click\&pgtype=Article\&state=default\&region=MAIN_CONTENT_3\&context=storylines_faq}{protective
    proteins made in response to an
    infection}\href{https://www.nytimes3xbfgragh.onion/2020/05/07/health/coronavirus-antibody-prevalence.html?action=click\&pgtype=Article\&state=default\&region=MAIN_CONTENT_3\&context=storylines_faq}{.
    These antibodies may}~last in the body
    \href{https://www.nature.com/articles/s41591-020-0965-6}{only two to
    three months}, which may seem worrisome, but that's~perfectly normal
    after an acute infection subsides, said Dr. Michael Mina, an
    immunologist at Harvard University. It may be possible to get the
    coronavirus again, but it's highly unlikely that it would be
    possible in a short window of time from initial infection or make
    people sicker the second time.
  \end{itemize}
\item ~
  \hypertarget{what-are-my-rights-if-i-am-worried-about-going-back-to-work}{%
  \paragraph{What are my rights if I am worried about going back to
  work?}\label{what-are-my-rights-if-i-am-worried-about-going-back-to-work}}

  \begin{itemize}
  \tightlist
  \item
    Employers have to provide
    \href{https://www.osha.gov/SLTC/covid-19/standards.html}{a safe
    workplace}~with policies that protect everyone equally.
    \href{https://www.nytimes3xbfgragh.onion/article/coronavirus-money-unemployment.html?action=click\&pgtype=Article\&state=default\&region=MAIN_CONTENT_3\&context=storylines_faq}{And
    if one of your co-workers tests positive for the coronavirus, the
    C.D.C.}~has said that
    \href{https://www.cdc.gov/coronavirus/2019-ncov/community/guidance-business-response.html}{employers
    should tell their employees}~-\/- without giving you the sick
    employee's name -\/- that they may have been exposed to the virus.
  \end{itemize}
\end{itemize}

Tell your friend that, as much as you love her, you think she should be
keeping herself away from others and that you're going to delay the
dinner until you can both be sure she's not contagious. She shouldn't be
socializing with anyone, not just not with you. Help her to understand
that.

\emph{After reading about how there's a blood shortage as a result of
coronavirus, I went to a blood drive. It was the first time I've donated
in a few years, and I immediately remembered why. I'm a healthy
25-year-old with no medical or lifestyle factors that would prevent me
from donating, but the donation process takes about twice as long for me
as it does for everyone else. The nurse said my blood flow is weaker
than average and that my veins are uniquely hard to find. A nurse had to
stay by my side, continually readjusting the needle, while a backlog of
donors built up in the waiting area. My guess is that at least two other
people could have donated in the time I was there. I know that donating
blood is critical, especially during national emergencies, and for all I
know, the others could ultimately have been disqualified from donating.
Do you think the inconvenience of dealing with me outweighs the good I'm
doing?} Name Withheld

\textbf{If the time} it takes to collect your pint would really have
reduced the amount of blood collected at that site --- maybe because the
station closed with people still waiting in line --- you'd have reason
to take a pass. You might have asked the nurse or another blood-drive
official about this. But if you only created a minor hassle, well,
someone has to have the trickiest veins on a given day. Your donation
expressed your desire to contribute to the well-being of others. That's
valuable in itself and something to give at least a little weight to. In
these individualist times, we should take the opportunity to remember
that, as St. Paul put it, we are ``members one of another.'' So thank
you for responding as you did.

\emph{I live in a shared house with three other people. We are all
employees or graduate students at the university we graduated from last
spring. One of my housemates, with whom I also work, is planning to
self-quarantine in our apartment after a vacation to Spain, which she
took despite increased warnings about Covid-19 in the media and messages
from the university discouraging international travel.}

\emph{I am in my 20s but have struggled with some moderate health
issues; my housemates and I also work with professors who are in their
70s. Can I tell my housemate to quarantine elsewhere? Should the
university provide housing for one of us? Should I notify our boss?}
Name Withheld

\textbf{Your housemate acted} irresponsibly. It's her obligation to find
a place to quarantine herself without imposing risks on you. Ask her to
do so. If she can't or won't, ask your boss to let you stay out of town,
if you can, until she's out of quarantine, or as you suggest, ask the
university to provide accommodations for one of you elsewhere. Either
way, you should practice social distancing with those septuagenarian
professors. Given their risk profile, they should certainly be grateful
--- and grateful too that you took the initiative to do so.

Advertisement

\protect\hyperlink{after-bottom}{Continue reading the main story}

\hypertarget{site-index}{%
\subsection{Site Index}\label{site-index}}

\hypertarget{site-information-navigation}{%
\subsection{Site Information
Navigation}\label{site-information-navigation}}

\begin{itemize}
\tightlist
\item
  \href{https://help.nytimes3xbfgragh.onion/hc/en-us/articles/115014792127-Copyright-notice}{©~2020~The
  New York Times Company}
\end{itemize}

\begin{itemize}
\tightlist
\item
  \href{https://www.nytco.com/}{NYTCo}
\item
  \href{https://help.nytimes3xbfgragh.onion/hc/en-us/articles/115015385887-Contact-Us}{Contact
  Us}
\item
  \href{https://www.nytco.com/careers/}{Work with us}
\item
  \href{https://nytmediakit.com/}{Advertise}
\item
  \href{http://www.tbrandstudio.com/}{T Brand Studio}
\item
  \href{https://www.nytimes3xbfgragh.onion/privacy/cookie-policy\#how-do-i-manage-trackers}{Your
  Ad Choices}
\item
  \href{https://www.nytimes3xbfgragh.onion/privacy}{Privacy}
\item
  \href{https://help.nytimes3xbfgragh.onion/hc/en-us/articles/115014893428-Terms-of-service}{Terms
  of Service}
\item
  \href{https://help.nytimes3xbfgragh.onion/hc/en-us/articles/115014893968-Terms-of-sale}{Terms
  of Sale}
\item
  \href{https://spiderbites.nytimes3xbfgragh.onion}{Site Map}
\item
  \href{https://help.nytimes3xbfgragh.onion/hc/en-us}{Help}
\item
  \href{https://www.nytimes3xbfgragh.onion/subscription?campaignId=37WXW}{Subscriptions}
\end{itemize}
