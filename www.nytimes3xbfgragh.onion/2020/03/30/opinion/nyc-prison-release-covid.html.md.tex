Sections

SEARCH

\protect\hyperlink{site-content}{Skip to
content}\protect\hyperlink{site-index}{Skip to site index}

\href{https://myaccount.nytimes3xbfgragh.onion/auth/login?response_type=cookie\&client_id=vi}{}

\href{https://www.nytimes3xbfgragh.onion/section/todayspaper}{Today's
Paper}

\href{/section/opinion}{Opinion}\textbar{}Andrew Cuomo, Stop a
Coronavirus Disaster: Release People From Prison

\url{https://nyti.ms/3dBapxj}

\begin{itemize}
\item
\item
\item
\item
\item
\end{itemize}

Advertisement

\protect\hyperlink{after-top}{Continue reading the main story}

\href{/section/opinion}{Opinion}

Supported by

\protect\hyperlink{after-sponsor}{Continue reading the main story}

\hypertarget{andrew-cuomo-stop-a-coronavirus-disaster-release-people-from-prison}{%
\section{Andrew Cuomo, Stop a Coronavirus Disaster: Release People From
Prison}\label{andrew-cuomo-stop-a-coronavirus-disaster-release-people-from-prison}}

This is a public health crisis that threatens to become a humanitarian
disaster.

By Mary Bassett, Eric Gonzalez and Darren Walker

Dr. Bassett was the New York City health commissioner. Mr. Gonzalez is
the district attorney of Brooklyn. Mr. Walker is the president of the
Ford Foundation.

\begin{itemize}
\item
  March 30, 2020
\item
  \begin{itemize}
  \item
  \item
  \item
  \item
  \item
  \end{itemize}
\end{itemize}

\includegraphics{https://static01.graylady3jvrrxbe.onion/images/2020/03/30/opinion/30stevenson-walker/30stevenson-walker-articleLarge.jpg?quality=75\&auto=webp\&disable=upscale}

Gov. Andrew Cuomo has been an exemplar of leadership in this time of
crisis, someone to whom other state and local officials are looking for
guidance. On Friday, the governor took the crucial step of ordering the
release of 1,100 people from New York's jails and prisons. But he must
do more. If we don't act fast, we jeopardize the lives of many. Worse
yet, we risk creating a uniquely deadly incubator for the virus.

America incarcerates more people than any other nation on earth --- some
2.3 million people,
\href{https://www.prisonpolicy.org/profiles/NY.html}{nearly 80,000} of
whom are locked up in various New York correctional facilities for young
people and adults. No other country in the world faces the kind of
threat gathering in our jails and prisons.

Given the conditions in which incarcerated people live --- limited
access to soap and water; shared bathrooms, mess halls and living
quarters --- this population is especially vulnerable to the virus, and
largely unable to prevent its spread. In New York, we've already begun
to see the effects. Dozens of residents and correctional staff members
have tested positive. More will follow.

The consequences will be devastating, for people both inside and outside
prison walls. When officers and staff members who work in prisons get
infected, they will bring the virus home to their families. As happened
in Italy, mounting tensions inside prisons, where anxious residents no
longer have access to family visits or the proper supplies to protect
themselves, create security risks that will be compounded by short
staffing when correctional officers fall sick.

And because of inadequate medical care in most of these facilities,
people in prison who become infected will die unnecessarily, while
others will be transferred to local hospitals in rural communities
upstate, where most of New York's prisons are. These hospitals are
woefully unprepared for the influx of patients, adding to growing strain
on our state's health care system. Dr. Ross MacDonald, the chief
physician for the Rikers Island jail complex, was not being hyperbolic
when he
\href{https://twitter.com/RossMacDonaldMD/status/1240455800591732737}{wrote
on Twitter} recently: ``A storm is coming.''

We have little time to curb the spread of the virus within our state's
jails and prisons, and, by extension, across our state. This is not only
an issue about the health of people in prisons, but also a public health
crisis that threatens to become a humanitarian disaster.

For this reason, we and a number of
\href{https://thehill.com/opinion/criminal-justice/488802-coronavirus-behind-bars-4-priorities-to-save-the-lives-of-prisoners}{public
health experts} call on Mr. Cuomo to release as many people as possible
from New York's correctional facilities, and to ensure they have the
medical and re-entry help they need when they return home.

Here are several actions Mr. Cuomo must take,
\href{https://www.nytimes3xbfgragh.onion/2020/01/24/opinion/sunday/bail-reform-new-york.html}{in
addition to}ending money bail for most people charged with crimes:

First, the governor should grant compassionate release to elderly
inmates, as well as those with health conditions that put them at higher
risk, including people with chronic and co-morbid conditions such as
asthma and chronic obstructive pulmonary disease, and those who are
pregnant or have immune deficiencies. We know that older incarcerated
people are both more vulnerable to the disease and present the lowest
risk of reoffending.

Second, Mr. Cuomo should release the thousands of people currently
incarcerated on noncriminal technical violations of their parole, like
missing an appointment with a parole officer --- except in the very few
cases in which a technical violation involves a risk to public safety.
There are some 4,000 people incarcerated in New York prisons for such
non-crimes. Only the governor has the power to release these people ---
and 50 current and former parole commissioners around the country
already have recommended he do so.

Third, he should grant early parole to people who are up to 180 days
away from completing their sentences. After all, they're coming home in
weeks or months.

And, finally, the governor should direct prison and jail administrators
to furlough low-risk inmates and use his own emergency powers to extend
those furloughs through the end of the crisis.

Mr. Cuomo
\href{https://www.washingtonpost.com/opinions/andrew-cuomo-35-questions-about-our-current-political-climate/2019/04/22/e27b752a-6520-11e9-82ba-fcfeff232e8f_story.html}{has
often noted} that our state does not need public policies that only
``sound good,'' but rather ones that are ``good and sound.''

There is no good reason to keep putting residents, as well as police and
correctional officers, at risk --- or to turn a parole violation into a
death sentence. Why needlessly strain our state's systems or incubate
the virus inside correctional facilities for future spread? There is no
sound reason that elderly and vulnerable people, who pose no threat to
the public, should be sentenced to die in prison.

We must protect public safety. But, today, there is no greater threat to
public safety than the coronavirus. There are hundreds, if not
thousands, of incarcerated people in New York who can safely be released
to their families and communities.

Mr. Cuomo alone holds the power to save the lives of people trapped in
New York's prisons and jails. We call on him to build on the important
step he has already taken and do the good and sound thing for human
dignity, public health and a more humane justice system.

Mary Bassett
(\href{https://twitter.com/DrMaryTBassett}{@DrMaryTBassett}) is the
director of the François-Xavier Bagnoud Center for Health and Human
Rights at Harvard. Eric Gonzalez
(\href{https://twitter.com/BrooklynDA}{@BrooklynDA}) is the district
attorney of Brooklyn. Darren Walker
(\href{https://twitter.com/darrenwalker?ref_src=twsrc\%5Egoogle\%7Ctwcamp\%5Eserp\%7Ctwgr\%5Eauthor}{@DarrenWalker})
is the president of the Ford Foundation.

\emph{The Times is committed to publishing}
\href{https://www.nytimes3xbfgragh.onion/2019/01/31/opinion/letters/letters-to-editor-new-york-times-women.html}{\emph{a
diversity of letters}} \emph{to the editor. We'd like to hear what you
think about this or any of our articles. Here are some}
\href{https://help.nytimes3xbfgragh.onion/hc/en-us/articles/115014925288-How-to-submit-a-letter-to-the-editor}{\emph{tips}}\emph{.
And here's our email:}
\href{mailto:letters@NYTimes.com}{\emph{letters@NYTimes.com}}\emph{.}

\emph{Follow The New York Times Opinion section on}
\href{https://www.facebookcorewwwi.onion/nytopinion}{\emph{Facebook}}\emph{,}
\href{http://twitter.com/NYTOpinion}{\emph{Twitter (@NYTopinion)}}
\emph{and}
\href{https://www.instagram.com/nytopinion/}{\emph{Instagram}}\emph{.}

Advertisement

\protect\hyperlink{after-bottom}{Continue reading the main story}

\hypertarget{site-index}{%
\subsection{Site Index}\label{site-index}}

\hypertarget{site-information-navigation}{%
\subsection{Site Information
Navigation}\label{site-information-navigation}}

\begin{itemize}
\tightlist
\item
  \href{https://help.nytimes3xbfgragh.onion/hc/en-us/articles/115014792127-Copyright-notice}{©~2020~The
  New York Times Company}
\end{itemize}

\begin{itemize}
\tightlist
\item
  \href{https://www.nytco.com/}{NYTCo}
\item
  \href{https://help.nytimes3xbfgragh.onion/hc/en-us/articles/115015385887-Contact-Us}{Contact
  Us}
\item
  \href{https://www.nytco.com/careers/}{Work with us}
\item
  \href{https://nytmediakit.com/}{Advertise}
\item
  \href{http://www.tbrandstudio.com/}{T Brand Studio}
\item
  \href{https://www.nytimes3xbfgragh.onion/privacy/cookie-policy\#how-do-i-manage-trackers}{Your
  Ad Choices}
\item
  \href{https://www.nytimes3xbfgragh.onion/privacy}{Privacy}
\item
  \href{https://help.nytimes3xbfgragh.onion/hc/en-us/articles/115014893428-Terms-of-service}{Terms
  of Service}
\item
  \href{https://help.nytimes3xbfgragh.onion/hc/en-us/articles/115014893968-Terms-of-sale}{Terms
  of Sale}
\item
  \href{https://spiderbites.nytimes3xbfgragh.onion}{Site Map}
\item
  \href{https://help.nytimes3xbfgragh.onion/hc/en-us}{Help}
\item
  \href{https://www.nytimes3xbfgragh.onion/subscription?campaignId=37WXW}{Subscriptions}
\end{itemize}
