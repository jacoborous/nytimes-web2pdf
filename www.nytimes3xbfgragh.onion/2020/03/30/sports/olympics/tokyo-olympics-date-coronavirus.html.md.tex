Sections

SEARCH

\protect\hyperlink{site-content}{Skip to
content}\protect\hyperlink{site-index}{Skip to site index}

\href{https://www.nytimes3xbfgragh.onion/section/sports/olympics}{Olympics}

\href{https://myaccount.nytimes3xbfgragh.onion/auth/login?response_type=cookie\&client_id=vi}{}

\href{https://www.nytimes3xbfgragh.onion/section/todayspaper}{Today's
Paper}

\href{/section/sports/olympics}{Olympics}\textbar{}Summer Olympics in
Tokyo to Start on July 23, 2021

\url{https://nyti.ms/2QWNzq3}

\begin{itemize}
\item
\item
\item
\item
\item
\end{itemize}

Advertisement

\protect\hyperlink{after-top}{Continue reading the main story}

Supported by

\protect\hyperlink{after-sponsor}{Continue reading the main story}

\hypertarget{summer-olympics-in-tokyo-to-start-on-july-23-2021}{%
\section{Summer Olympics in Tokyo to Start on July 23,
2021}\label{summer-olympics-in-tokyo-to-start-on-july-23-2021}}

The new date for the Games, postponed for a year in response to the
coronavirus pandemic, gives athletes time to recalibrate their training
schedules.

\includegraphics{https://static01.graylady3jvrrxbe.onion/images/2020/03/30/world/30virus-olympics1/merlin_170950011_96307da4-33a8-4faa-a900-eae41ada4acc-articleLarge.jpg?quality=75\&auto=webp\&disable=upscale}

By \href{https://www.nytimes3xbfgragh.onion/by/tariq-panja}{Tariq Panja}
and \href{https://www.nytimes3xbfgragh.onion/by/motoko-rich}{Motoko
Rich}

\begin{itemize}
\item
  Published March 30, 2020Updated March 31, 2020
\item
  \begin{itemize}
  \item
  \item
  \item
  \item
  \item
  \end{itemize}
\end{itemize}

After weeks of uncertainty around having the Summer Olympics and when,
organizers of the event on Monday provided some clarity: The Games in
Tokyo will start July 23, 2021, almost exactly a year later than
originally scheduled.

For the 11,000 athletes and multitudes of others who have built lives,
careers and businesses around the Games, the new date sets off more than
a year of upheaval and complex planning, unprecedented for an event that
has been canceled only three times during war and never previously moved
to a new date.

Nearly a week after Olympic officials and Japanese organizers
\href{https://www.nytimes3xbfgragh.onion/2020/03/24/sports/olympics/coronavirus-summer-olympics-postponed.html}{bowed
to widespread pressure} and announced they would postpone,
\href{https://www.nytimes3xbfgragh.onion/2020/07/19/sports/2021-tokyo-olympics-protocols.html}{Tokyo
2020} organizers gave the new time frame, with an opening ceremony July
23 and the closing one Aug. 8. The Paralympic Games, which were supposed
to start Aug. 25, will now take place in 2021 from Aug. 24 to Sept. 5.

Thomas Bach, the president of the International Olympic Committee, told
sports federations on a conference call before the announcement that the
date was picked to give organizers the maximum time to deal with the
fallout from the coronavirus. When he called for a show of support on
the call, it was unanimous.

``I am confident that, working together with the Tokyo 2020 organizing
committee, the Tokyo metropolitan government, the Japanese government
and all our stakeholders, we can master this unprecedented challenge,''
Bach said in a statement Monday.

Bach made official what The New York Times and other news organizations
had
\href{https://www.nytimes3xbfgragh.onion/2020/03/28/sports/olympics/coronavirus-olympics-postponed-2021.html}{reported}
over the weekend.

\includegraphics{https://static01.graylady3jvrrxbe.onion/images/2020/03/30/world/30virus-olympics2/30virus-olympics2-articleLarge.jpg?quality=75\&auto=webp\&disable=upscale}

The decision quickly reverberated across oceans, with athletes, coaches,
national Olympic committees, sponsors and television partners scrambling
to figure out how to grapple with the delay.

For Kate Nye, 21, an American weight lifter, the delay means graduating
from college and deferring graduate school for another year in order to
continue punishing sets with the barbells and weight blocks in the
garage of her Michigan home. Still, that was preferable to having no
Games, which many athletes feared might happen until Bach promised four
days before the postponement that the Games would take place..

``It's a little bit longer of a delay than I had hoped, but I'm just
thankful they were able to successfully postpone the event,'' Nye said.

The Tokyo 2020 organizing committee, which for years had planned for an
opening ceremony July 24 this summer, on Monday announced measures aimed
at reducing some of the disruption. It said that the thousands of
volunteers who had been offered positions would be retained for next
year and that tickets remained valid, but full refunds would be given to
those who could no longer attend.

Organizers considered alternative dates, but having the event as close
to the original time quickly emerged as the preference among officials
in Japan and elsewhere.

Michael Lynch, a former director of global sponsorships for Visa who
works with a number of Olympic partners, said a one-year delay that
avoids conflict with most of Europe's and North America's major
professional leagues was the best alternative.

\hypertarget{sports-and-the-virus}{%
\subsubsection{Sports and the Virus}\label{sports-and-the-virus}}

\paragraph{}

Updated Sept. 8, 2020

Here's what's happening as the world of sports slowly comes back to
life:

\begin{itemize}
\item
  \begin{itemize}
  \tightlist
  \item
    As the United States Open enters its second week without fans, an
    Italian restaurateur stands outside the gates and
    \href{https://www.nytimes3xbfgragh.onion/2020/09/06/sports/tennis/US-Open-Matteo-Berrettini-fan.html?action=click\&pgtype=Article\&state=default\&region=MAIN_CONTENT_2\&context=storylines_keepup}{bellows
    his support}~for his favorite player.
  \item
    The coronavirus pandemic has had an
    \href{https://www.nytimes3xbfgragh.onion/2020/09/03/sports/ncaafootball/high-school-football-coronavirus-pandemic.html?action=click\&pgtype=Article\&state=default\&region=MAIN_CONTENT_2\&context=storylines_keepup}{uneven
    impact on high school football}~across the United States.
  \item
    The
    \href{https://www.nytimes3xbfgragh.onion/2020/09/02/sports/ncaafootball/coronavirus-cal-athletics-season.html?action=click\&pgtype=Article\&state=default\&region=MAIN_CONTENT_2\&context=storylines_keepup}{most
    complicated puzzle in sports is the return of college
    athletics}~during a pandemic. The University of California, Berkeley
    is allowing The Times an inside look at their journey's ups and
    downs.
  \end{itemize}
\end{itemize}

It also allows major Olympic sponsors, such as the Coca-Cola Company and
Procter \& Gamble, to reduce their expenses this year and delay them to
when the economy presumably would have recovered from any broad
downturn.

The date is also likely to suit NBCUniversal, the Olympics broadcaster
in the United States whose rights fees make up more of the I.O.C.'s
income than any other single entity.

Image

Runners passing the Zojoji Buddhist temple during the Tokyo Marathon in
February 2014. The Olympic~ marathon will be held in the cooler,
northern city of Sapporo.Credit...Eugene Hoshiko/Associated Press

The decision also means the Games will take place during the hottest
time of the year in Tokyo, an issue that was causing worry and
complications before the coronavirus outbreak became an issue.
\href{https://www.nytimes3xbfgragh.onion/2019/10/16/sports/olympic-marathon-tokyo-heat.html}{The
Olympic marathon was moved to Sapporo}, a cooler city in the north, and
organizers had planned several measures
\href{https://www.nytimes3xbfgragh.onion/2019/10/10/sports/tokyo-braces-for-the-hottest-olympics-ever.html}{to
try to keep fans and competitors cool}.

Some federations --- including those representing swimming, table
tennis, equestrian and triathlon --- expressed a preference to the
I.O.C. for having the Games in the spring.

But most remained neutral or called for keeping the Games in the summer.

Perhaps the most notable backer for maintaining a summer schedule was
track and field, even though the new date for the Olympics means its
world championships, set to open in August 2021, will have to be
rearranged. The I.O.C. especially wanted to steer clear of soccer
championships in Europe and South America, which usually take place in
June and are being moved to 2021.

The I.O.C.'s initial indecision and slow action in the face of the
spreading outbreak led to criticism from scientists in the United States
and Europe as well as athletes, whose training schedules were
obliterated by government-imposed restrictions on movement. Most major
sports had paused their competitions and many facilities had closed
before Olympic organizers acted on the inevitable.

Even those who could continue to train had expressed concerns that the
absence of a firm start date meant they could not calibrate their
schedules, to ensure they were at their peak during the Games.

David Marsh, an elite swim coach based in San Diego who had lobbied for
a one-year delay as the coronavirus crisis hit California, said he had
athletes rounding into top form who needed to reset and others who would
benefit from an extra year of development, during which a marginal
improvement could land them on the medal podium. He also has swimmers
who now will have to contemplate retirement because they have put off
careers outside the pool for so long.

``Will sponsors be ready to support Olympic athletes?'' Marsh asked.
``Will parents of swimmers have the ability to support their athlete
with Olympic ambitions?'' Marsh now has to find a pool to train his team
for another year, which will cost roughly \$100,000.

Image

Advertising promoting the Tokyo Games at Haneda Airport.Credit...Hiroko
Masuike/The New York Times

The cost of the delay will be felt most sharply in Japan. Local
organizers have described the cost of rescheduling what had already been
called the Recovery Olympics --- a reference to the rebuilding effort
after a deadly earthquake and tsunami in 2011 that damaged a nuclear
reactor in Fukushima --- as ``enormous.'' Estimates vary, but none put
the extra cost to Japan at less than \$2 billion.

Like previous Olympics, the actual cost of organizing and executing the
Tokyo Games will be much higher than the budget Tokyo first presented.
The 2020 organizers say they are spending \$12.6 billion to stage the
Games, while a government audit report published last year said the
actual cost was several billions more because of the construction of
secondary infrastructure.

For Japan, the Olympics has been a national project, one which domestic
companies have supported in amounts unmatched by any previous Olympics.
The national sponsorship program had brought in more than \$3 billion in
sponsorship revenue, three times more than at any previous Olympics.

Masa Takaya, a spokesman for the organizers, said they had not yet
started discussing the costs of the delay with sponsors, but hoped they
could count on their continued backing. ``We appreciate if sponsors
extend the contract terms and keep supporting us,'' Takaya said.

The traditional July-August date means the I.O.C. will probably be able
to call on the presence of top players from soccer, tennis and golf,
some of the biggest names in global sports and a big attraction for
television audiences.

Now that the date has been set, the next challenge will be to reorganize
the qualifying competitions. Bach said athletes who have already
qualified would be guaranteed a place for 2021.

Advertisement

\protect\hyperlink{after-bottom}{Continue reading the main story}

\hypertarget{site-index}{%
\subsection{Site Index}\label{site-index}}

\hypertarget{site-information-navigation}{%
\subsection{Site Information
Navigation}\label{site-information-navigation}}

\begin{itemize}
\tightlist
\item
  \href{https://help.nytimes3xbfgragh.onion/hc/en-us/articles/115014792127-Copyright-notice}{©~2020~The
  New York Times Company}
\end{itemize}

\begin{itemize}
\tightlist
\item
  \href{https://www.nytco.com/}{NYTCo}
\item
  \href{https://help.nytimes3xbfgragh.onion/hc/en-us/articles/115015385887-Contact-Us}{Contact
  Us}
\item
  \href{https://www.nytco.com/careers/}{Work with us}
\item
  \href{https://nytmediakit.com/}{Advertise}
\item
  \href{http://www.tbrandstudio.com/}{T Brand Studio}
\item
  \href{https://www.nytimes3xbfgragh.onion/privacy/cookie-policy\#how-do-i-manage-trackers}{Your
  Ad Choices}
\item
  \href{https://www.nytimes3xbfgragh.onion/privacy}{Privacy}
\item
  \href{https://help.nytimes3xbfgragh.onion/hc/en-us/articles/115014893428-Terms-of-service}{Terms
  of Service}
\item
  \href{https://help.nytimes3xbfgragh.onion/hc/en-us/articles/115014893968-Terms-of-sale}{Terms
  of Sale}
\item
  \href{https://spiderbites.nytimes3xbfgragh.onion}{Site Map}
\item
  \href{https://help.nytimes3xbfgragh.onion/hc/en-us}{Help}
\item
  \href{https://www.nytimes3xbfgragh.onion/subscription?campaignId=37WXW}{Subscriptions}
\end{itemize}
