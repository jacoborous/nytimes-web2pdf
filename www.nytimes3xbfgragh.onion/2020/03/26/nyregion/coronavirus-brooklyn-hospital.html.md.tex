Sections

SEARCH

\protect\hyperlink{site-content}{Skip to
content}\protect\hyperlink{site-index}{Skip to site index}

\href{/section/nyregion}{New York}\textbar{}`We're in Disaster Mode':
Courage Inside a Brooklyn Hospital Confronting Coronavirus

\url{https://nyti.ms/39dHj3y}

\begin{itemize}
\item
\item
\item
\item
\item
\item
\end{itemize}

\hypertarget{the-coronavirus-outbreak}{%
\subsubsection{\texorpdfstring{\href{https://www.nytimes3xbfgragh.onion/news-event/coronavirus?name=styln-coronavirus-national\&region=TOP_BANNER\&block=storyline_menu_recirc\&action=click\&pgtype=Article\&impression_id=0c771520-f2b3-11ea-8935-6965980a26b7\&variant=undefined}{The
Coronavirus
Outbreak}}{The Coronavirus Outbreak}}\label{the-coronavirus-outbreak}}

\begin{itemize}
\tightlist
\item
  live\href{https://www.nytimes3xbfgragh.onion/2020/09/09/world/covid-19-coronavirus.html?name=styln-coronavirus-national\&region=TOP_BANNER\&block=storyline_menu_recirc\&action=click\&pgtype=Article\&impression_id=0c773c30-f2b3-11ea-8935-6965980a26b7\&variant=undefined}{Latest
  Updates}
\item
  \href{https://www.nytimes3xbfgragh.onion/interactive/2020/us/coronavirus-us-cases.html?name=styln-coronavirus-national\&region=TOP_BANNER\&block=storyline_menu_recirc\&action=click\&pgtype=Article\&impression_id=0c773c31-f2b3-11ea-8935-6965980a26b7\&variant=undefined}{Maps
  and Cases}
\item
  \href{https://www.nytimes3xbfgragh.onion/interactive/2020/science/coronavirus-vaccine-tracker.html?name=styln-coronavirus-national\&region=TOP_BANNER\&block=storyline_menu_recirc\&action=click\&pgtype=Article\&impression_id=0c773c32-f2b3-11ea-8935-6965980a26b7\&variant=undefined}{Vaccine
  Tracker}
\item
  \href{https://www.nytimes3xbfgragh.onion/2020/09/02/your-money/eviction-moratorium-covid.html?name=styln-coronavirus-national\&region=TOP_BANNER\&block=storyline_menu_recirc\&action=click\&pgtype=Article\&impression_id=0c773c33-f2b3-11ea-8935-6965980a26b7\&variant=undefined}{Eviction
  Moratorium}
\item
  \href{https://www.nytimes3xbfgragh.onion/2020/09/09/upshot/coronavirus-surprise-test-fees.html?name=styln-coronavirus-national\&region=TOP_BANNER\&block=storyline_menu_recirc\&action=click\&pgtype=Article\&impression_id=0c773c34-f2b3-11ea-8935-6965980a26b7\&variant=undefined}{Surprise
  Test Fees}
\end{itemize}

\includegraphics{https://static01.graylady3jvrrxbe.onion/images/2020/03/27/us/politics/27virus-hospital-p1/merlin_170947869_743543ae-92a4-4505-9ec6-2cf95e8ec8a1-articleLarge.jpg?quality=75\&auto=webp\&disable=upscale}

\hypertarget{were-in-disaster-mode-courage-inside-a-brooklyn-hospital-confronting-coronavirus}{%
\section{`We're in Disaster Mode': Courage Inside a Brooklyn Hospital
Confronting
Coronavirus}\label{were-in-disaster-mode-courage-inside-a-brooklyn-hospital-confronting-coronavirus}}

Test kits and protective gear have been in short supply, doctors are
falling sick, and every day gets more difficult. But the staff keeps
showing up.

Credit...Victor J. Blue for The New York Times

Supported by

\protect\hyperlink{after-sponsor}{Continue reading the main story}

\href{https://www.nytimes3xbfgragh.onion/by/sheri-fink}{\includegraphics{https://static01.graylady3jvrrxbe.onion/images/2018/08/24/multimedia/author-sheri-fink/author-sheri-fink-thumbLarge.png}}

By \href{https://www.nytimes3xbfgragh.onion/by/sheri-fink}{Sheri Fink}

\begin{itemize}
\item
  Published March 26, 2020Updated April 15, 2020
\item
  \begin{itemize}
  \item
  \item
  \item
  \item
  \item
  \item
  \end{itemize}
\end{itemize}

\href{https://www.nytimes3xbfgragh.onion/es/2020/03/30/espanol/mundo/coronavirus-hospitales.html}{Leer
en español}

It was not even 9 in the morning and Dr. Sylvie de Souza's green N95
mask, which was supposed to form a seal against her face, was already
askew.

In freezing rain on Monday, she trudged in clogs between the emergency
department she chairs at the Brooklyn Hospital Center and a tent
outside, keeping a sharp eye on the trainee doctors, nurses and other
staff members who would screen nearly 100 walk-in patients for the
\href{https://www.nytimes3xbfgragh.onion/2020/04/15/podcasts/the-daily/coronavirus-brooklyn-hospital.html}{coronavirus}
that day.

Inside her E.R., more than a dozen people showing signs of infection
waited for evaluation in an area used just a few weeks ago for stitches
and casts. Another dozen lay on gurneys arranged one in front of the
next, like a New York City car park. One man on a ventilator was waiting
for space in the intensive care unit.

Minutes before paramedics wheeled in a heart attack patient, Dr. de
Souza pointed to beds reserved for serious emergencies, separated by a
newly constructed wall from the suspected virus cases. ``This is our
safe area,'' she told a reporter. Then she corrected herself: ``This is
thought to be safe.'' There was really no way to know.

The virus descended on the hospital three weeks ago. Dr. de Souza began
writing down details of each potential case on a sheet of paper, a list
that has grown to more than 800 patients, most of them seen in the
walk-in tent.

\includegraphics{https://static01.graylady3jvrrxbe.onion/images/2020/03/26/multimedia/26virus-hospital-02/merlin_170947839_0cfda671-8ed6-42bb-8156-17838d399cc7-articleLarge.jpg?quality=75\&auto=webp\&disable=upscale}

She and others at the hospital had prepared for the growing onslaught:
canceling most surgeries to bring down the census, designating an X-ray
room just for patients suspected of having the virus, searching for
supplies, barring most visitors, redeploying nurses to new roles,
opening a hotline for the community.

The 175-year-old hospital --- where Walt Whitman brought peaches and
poems to comfort the Civil War wounded and where Anthony Fauci, the
White House adviser who is now America's most famous doctor, was born
--- is scaling up, as Gov. Andrew M. Cuomo has required all New York
hospitals to do. The city, now the epicenter of the outbreak in the
United States, had reported more than 20,000 confirmed infections and
280 deaths as of late Wednesday.

Licensed to treat 464 patients, the Brooklyn medical center typically
has only enough staff and beds to handle 250 to 300. It is planning to
increase that number by half if needed, but it may have to double it.

``I have so many different fears,'' Dr. de Souza said on Wednesday. If
the patient volume increases at the current pace, she is concerned the
emergency room will be out of space by next week. If many patients are
desperately ill and need life support, she worries about having to
choose between them.

That morning for the first time, **** the health workers in the tent
lifted their arms at a safe distance, as if they were holding hands, and
said a prayer --- to make the right decisions; to be protected, along
with their patients, from the disease. Dr. de Souza plans to make it a
tradition. ``That's all we can do: just pray, stick together, encourage
each other, not get paralyzed by fear,'' she said.

Image

The hospital has seen more than 800 potential coronavirus cases in the
past few weeks.Credit...Victor J. Blue for The New York Times

\includegraphics{https://static01.graylady3jvrrxbe.onion/images/2017/01/29/podcasts/the-daily-album-art/the-daily-album-art-articleInline-v2.jpg?quality=75\&auto=webp\&disable=upscale}

\hypertarget{listen-to-the-daily-24-hours-inside-a-brooklyn-hospital}{%
\subsubsection{Listen to `The Daily': 24 Hours Inside a Brooklyn
Hospital}\label{listen-to-the-daily-24-hours-inside-a-brooklyn-hospital}}

What ``disaster mode'' looks like on the front lines of the crisis.

transcript

Back to The Daily

bars

0:00/26:39

-26:39

transcript

\hypertarget{listen-to-the-daily-24-hours-inside-a-brooklyn-hospital-1}{%
\subsection{Listen to `The Daily': 24 Hours Inside a Brooklyn
Hospital}\label{listen-to-the-daily-24-hours-inside-a-brooklyn-hospital-1}}

\hypertarget{hosted-by-michael-barbaro-produced-by-annie-brown-daniel-guillemette-and-clare-toeniskoetter-with-help-from-alex-young-and-sydney-harper-and-edited-by-lisa-chow-liz-o-baylen-and-lisa-tobin}{%
\subsubsection{Hosted by Michael Barbaro; produced by Annie Brown,
Daniel Guillemette and Clare Toeniskoetter; with help from Alex Young
and Sydney Harper; and edited by Lisa Chow, Liz O. Baylen, and Lisa
Tobin}\label{hosted-by-michael-barbaro-produced-by-annie-brown-daniel-guillemette-and-clare-toeniskoetter-with-help-from-alex-young-and-sydney-harper-and-edited-by-lisa-chow-liz-o-baylen-and-lisa-tobin}}

\hypertarget{what-disaster-mode-looks-like-on-the-front-lines-of-the-crisis}{%
\paragraph{What ``disaster mode'' looks like on the front lines of the
crisis.}\label{what-disaster-mode-looks-like-on-the-front-lines-of-the-crisis}}

\begin{itemize}
\tightlist
\item
  interposing voices\\
  Good morning, everyone. Hi.
\end{itemize}

sheri fink

So every morning in the Intensive Care Unit at the Brooklyn Hospital
Center, the doctors gather for something called morning report.

\begin{itemize}
\tightlist
\item
  doctor 1\\
  So now, I want you all to present in a straight, true way ---
\end{itemize}

sheri fink

The people who were on overnight, they stand around and the head doctor
is there, and they kind of give a report of what happened. And then, the
new doctors who are coming on, they get that information.

\begin{itemize}
\tightlist
\item
  doctor 2\\
  Yeah. When she was at rest this morning, she was breathing 23. She's
  very comfortable, thumbs up.
\end{itemize}

sheri fink

They talk about, you know, who was admitted, who got critically ill.

\begin{itemize}
\tightlist
\item
  doctor 3\\
  The overnight patient is not doing well. He had to be re-intubated
  almost immediately.
\end{itemize}

sheri fink

And one recent morning report was particularly intense.

{[}music{]}

\begin{itemize}
\tightlist
\item
  doctor\\
  OK. All right. OK. Next patient.
\end{itemize}

sheri fink

There were patients in their 80s and patients in their 30s.

\begin{itemize}
\item
  doctor 1\\
  31-year-old female, 30 weeks pregnant, asthma, obesity, admitted to
  the I.C.U. She was intubated yesterday evening.
\item
  doctor 2\\
  Jesus.
\item
  doctor
\item
  doctor 3\\
  All right. Good. Next.
\end{itemize}

sheri fink

There were patients from nursing homes and patients who were homeless.

\begin{itemize}
\item
  doctor 4\\
  She was intubated overnight. She's on azithromycin, klonopin,
  ceftriaxone.
\item
  doctor\\
  OK. Next.
\end{itemize}

sheri fink

Patients with asthma and diabetes, and patients with no underlying
conditions at all.

\begin{itemize}
\tightlist
\item
  doctor\\
  --- male. We just past medical history here for acute hypoxic
  respiratory failure.
\end{itemize}

sheri fink

But as the doctors race to get through the cases ---

\begin{itemize}
\tightlist
\item
  doctor\\
  Next patient.
\end{itemize}

sheri fink

--- they all shared a nearly identical description.

\begin{itemize}
\item
  doctor 1\\
  He was upgraded from acute hypoxic respiratory failure.
\item
  doctor 2\\
  OK, next.
\item
  doctor 3\\
  Male, acute hypoxic respiratory failure secondary to confirmed Covid.
\item
  doctor 4\\
  All right. Next.
\item
  doctor 5\\
  Admitted for acute hypoxic respiratory failure with confirmed
  Covid-19.
\item
  doctor 6\\
  Next.
\item
  doctor 7\\
  Male, it looks like acute hypoxic respiratory failure.
\end{itemize}

sheri fink

Acute hypoxic respiratory failure secondary to Covid-19.

\begin{itemize}
\tightlist
\item
  doctor 8\\
  All right. Next.
\end{itemize}

michael barbaro

From The New York Times, I'm Michael Barbaro. This is ``The Daily.''

{[}music{]}

Today: It's been more than a month since the coronavirus descended on
New York City's hospitals and on Brooklyn Medical Center, where the vast
majority of patients in critical care have the virus. My colleague,
Sheri Fink, followed one doctor through a single day there.

It's Wednesday, April 15.

\begin{itemize}
\item
  doctor\\
  Morning, everybody. {[}AMBIENT CHATTER{]}
\item
  doctor 1\\
  Josh, do you want to spend the --- do you mind? This is Sheri.
\item
  doctor 2\\
  Sure.
\item
  doctor 1\\
  She's with The New York Times, and she's gonna spend some time here a
  little bit.
\item
  doctor 2\\
  Pleasure.
\item
  doctor 1\\
  It's up to you.
\item
  doctor 2\\
  I'm fine with ---
\item
  sheri fink\\
  I'm a physician.
\item
  doctor 1\\
  A physician and a writer.
\end{itemize}

sheri fink

So for the past few weeks, I've been embedded in the Brooklyn Hospital
Center.

\begin{itemize}
\item
  doctor 1\\
  I'm going to finish rounding here, and then I'm going to go downstairs
  and cover SI.
\item
  doctor 2\\
  OK.
\end{itemize}

sheri fink

And what I've been able to see there is incredibly unique --- what's
happening? What is it like inside a hospital during a pandemic?

\begin{itemize}
\item
  dr. josh rosenberg\\
  --- then we'll figure out the rest.
\item
  doctor\\
  OK. All right.
\item
  sheri fink\\
  Do you want to give him your mic, or are you willing to wear a mic?
\end{itemize}

sheri fink

And there was one doctor I met who really embodied that transparency.

\begin{itemize}
\tightlist
\item
  dr. josh rosenberg\\
  Does it beep every time I say a four-letter word like South Park?
\end{itemize}

sheri fink

Dr. Josh Rosenberg.

\begin{itemize}
\tightlist
\item
  dr. josh rosenberg\\
  I am mildly inappropriate. I'm just warning you.
\end{itemize}

sheri fink

An attending physician in the Intensive Care Unit.

\begin{itemize}
\item
  sheri fink\\
  How are you, Peter?
\item
  doctor\\
  Hi, how are you ---
\item
  dr. josh rosenberg\\
  I didn't see you hiding over there, my friend.
\end{itemize}

sheri fink

There are people from all over the hospital recruited to work in the
I.C.U., so it's not just, like, I.C.U. doctors and nurses who are used
to intensive care treatment, but in fact ---

\begin{itemize}
\tightlist
\item
  dr. josh rosenberg\\
  And she's one of the podiatry residents, so all people who are good
  with knives and big needles.
\end{itemize}

sheri fink

When I was there that day, there was a podiatry doctor and two of her
residents. Those are doctors who work on the feet.

\begin{itemize}
\tightlist
\item
  dr. josh rosenberg\\
  No, no, no. What I would like to do is that, as much as possible,
  we're going to try to get all of the Covids on one side, and then the
  whole area is a dirty area.
\end{itemize}

sheri fink

And the I.C.U. had actually effectively doubled in size, so it was
completely full. And they had to turn to other areas of the hospital to
turn them into Intensive Care Units. In fact, a big part of the I.C.U.
is now in a place that just a few weeks ago was where patients would
come for outpatient chemotherapy treatments. That's now in I.C.U..

\begin{itemize}
\tightlist
\item
  dr. josh rosenberg\\
  Frankie, watch out. Don't trip Don't trip Don't trip. Don't trip.
  Don't trip.
\end{itemize}

sheri fink

It was also a bit of an obstacle course.

\begin{itemize}
\tightlist
\item
  dr. josh rosenberg\\
  Don't trip.
\end{itemize}

sheri fink

There were cords everywhere.

\begin{itemize}
\tightlist
\item
  dr. josh rosenberg\\
  Please be careful, Do you have gloves?
\end{itemize}

sheri fink

They had pulled apart the ventilators. They had the control --- parts of
the ventilators that were helping people breathe, those were in the
hallways so that nurses and respiratory therapists didn't have to go in
and out as much and expose themselves to risk.

\begin{itemize}
\item
  dr. josh rosenberg\\
  What?
\item
  speaker\\
  This is a disaster waiting to happen.
\item
  dr. josh rosenberg\\
  Yes and no, though.
\end{itemize}

sheri fink

And the nurses were doing the same thing with IVs, with the tubing that
the medicine flows through. So they had pulled the IV pumps out of the
room so that they can not have to go in and out and use up the personal
protective equipment.

\begin{itemize}
\tightlist
\item
  dr. josh rosenberg\\
  It's great. And yeah, I mean, you can trip over it.
\end{itemize}

sheri fink

You all have to be very careful.

\begin{itemize}
\item
  dr. josh rosenberg\\
  You just have to be careful.
\item
  sheri fink\\
  Yeah.
\item
  dr. josh rosenberg\\
  Right. It's making the best of what you can do.
\item
  sheri fink\\
  Yeah.
\item
  dr. josh rosenberg\\
  OK, guys, can we start with number two? I appreciate everybody being
  here and everybody's support massively.
\end{itemize}

sheri fink

So now, Dr. Rosenberg is taking over for the doctors who were working
the night before, and he's beginning to make his rounds.

\begin{itemize}
\tightlist
\item
  dr. josh rosenberg\\
  Let's start with number two, and then just go around the unit please.
  All right, so lucky number two.
\end{itemize}

sheri fink

So nearly all the patients in the I.C.U. are on ventilators.

\begin{itemize}
\tightlist
\item
  dr. josh rosenberg\\
  So do we have any history of smoking, shisha use, anything like that?
\end{itemize}

sheri fink

Some have asthma. Some have diabetes.

\begin{itemize}
\tightlist
\item
  dr. josh rosenberg\\
  All right. What did he do for a living? Occupational exposure?
\end{itemize}

sheri fink

But a lot of these patients don't have any underlying conditions at all.

\begin{itemize}
\tightlist
\item
  dr. josh rosenberg\\
  I'll just write --- because I mean, listen, on some of these you have
  a real reason why. You know, they may have bad lungs, and that makes
  it worse. Sometimes it's just the disease, but if there's something we
  can do to ---
\end{itemize}

sheri fink

So Josh and the other doctors are kind of confounded by some of the
patients. They don't understand why, if they don't have a lot of
underlying health issues, why their lungs look so bad.

\begin{itemize}
\tightlist
\item
  dr. josh rosenberg\\
  Crap. Reported any asthma?
\end{itemize}

sheri fink

And they also just don't have that much to offer.

\begin{itemize}
\item
  dr. josh rosenberg\\
  OK. So what are we going to do with him?
\item
  doctor\\
  Right now, we are --- well, at this point, I'm not too sure what we
  can do with him. We have --- we tried to {[}VOICE FADES{]}.
\item
  dr. josh rosenberg\\
  So what is he on drug-wise?
\end{itemize}

sheri fink

So, I mean, for most patients, they're trying this thing called the
Covid cocktail, which is that hydroxychloroquine and azithromycin.
That's that combination the President talks about a lot.

\begin{itemize}
\tightlist
\item
  dr. josh rosenberg\\
  I don't think it's doing much.
\end{itemize}

sheri fink

But there's really very little evidence, and Dr. Rosenberg in particular
is very unsure that those drugs really help.

\begin{itemize}
\tightlist
\item
  dr. josh rosenberg\\
  We'll see about remdesivir, and we'll see if we get some Covid results
  and see what we can do.
\end{itemize}

sheri fink

So they start talking about other possibilities. There's this
experimental drug called remdesivir that you have to apply to the
manufacturer for each patient, and they have to meet certain criteria.
You have to have a test result. They can't have certain complications.

\begin{itemize}
\item
  dr. josh rosenberg\\
  How do you guys feel about Kaletra or our other PIs?
\item
  doctor\\
  They don't work at all.
\end{itemize}

sheri fink

There's another drug called Kaletra that doctors think might have some
effect.

\begin{itemize}
\item
  dr. josh rosenberg\\
  The data's very --- I mean, I think the data is very weak all over the
  place. That's the basic problem. So I always look at it as, where are
  you starting these drugs? It's near the end of a sporting event.
  You're down by a lot, and I don't care you throw out there, right?
  Even freaking Jordan couldn't recover that basketball game outside of
  Space Jam when you're down by 100 points and starting the fourth
  quarter.
\item
  doctor\\
  That's why I don't think we should be giving it to patients who are
  already near the end.
\end{itemize}

sheri fink

So they kind of toss this around.

\begin{itemize}
\item
  dr. josh rosenberg\\
  Yeah. And so we don't know. I mean, that's the point. We really just
  don't know our data, but like, so looking at this ---

  yeah. So we'll figure out. We'll see if we get the remdesivir, which I
  doubt we'll be able to. We'll try to get a positive test result.

  Next. Let's move on along.

  OK. I.C.U. six. Going for c-section?
\item
  doctor\\
  Supposedly today, yeah.
\end{itemize}

sheri fink

There was another Covid patient in the Intensive Care Unit on a
ventilator, and she was pregnant, which adds a whole layer of
complexity.

\begin{itemize}
\item
  doctor\\
  She needs another dose of decadron, and then ---
\item
  dr. josh rosenberg\\
  Decadron? No. Beclomethasone.
\item
  doctor\\
  Oh, sorry. Beclomethasone. Did I say decadron?
\item
  dr. josh rosenberg\\
  Yes.
\end{itemize}

sheri fink

And they actually decided to deliver the baby by c-section two months
before the due date. They had to give a couple of doses of steroid
medication to help mature the baby's lungs. The whole goal was to save
the mother's life, because I think part of it is that it gives more
space for the lungs to expand once the baby is taken out.

\begin{itemize}
\item
  doctor\\
  So if she's going for a c-section then she won't need remdesivir,
  right?
\item
  dr. josh rosenberg\\
  I have no clue.
\end{itemize}

sheri fink

So far what's known is it tends to be quite rare that a baby would be
born with Covid if the mom has it. At least that's what the early
studies say.

\begin{itemize}
\tightlist
\item
  dr. josh rosenberg\\
  All right. Number four. Number four. How are we doing here?
\end{itemize}

sheri fink

It might be surprising how enthusiastic Dr. Rosenberg sounds while
discussing these patients, but he's leading this team. He's trying to
keep morale up.

\begin{itemize}
\item
  dr. josh rosenberg\\
  All right. So I'm going to stop here and head downstairs. Again, he's
  going to take six, seven, nine. Thank you. I will circle in with you
  guys. Good job.
\item
  doctor\\
  Thank you. OK.
\item
  dr. josh rosenberg\\
  Good job.
\item
  doctor\\
  Oh, me?
\end{itemize}

sheri fink

But actually, when we were going from one part of the I.C.U. to another
---

\begin{itemize}
\item
  dr. josh rosenberg\\
  Let's go downstairs.

  {[}SIGHS{]} I don't like taking the elevators.
\end{itemize}

sheri fink

He runs into one of his medical students.

\begin{itemize}
\item
  sheri fink\\
  Hi, guys.
\item
  dr. josh rosenberg\\
  How are you doing, buddy?
\item
  doctor\\
  As best as I can.
\item
  dr. josh rosenberg\\
  One, shouldn't you be home?
\item
  doctor\\
  Yeah.
\item
  dr. josh rosenberg\\
  Shouldn't you be home?
\item
  doctor\\
  My mom's here.
\item
  dr. josh rosenberg\\
  Oh, fuck.
\item
  doctor\\
  I know.
\item
  dr. josh rosenberg\\
  Which bed is she in on that side?
\item
  doctor\\
  She's in 10.
\item
  dr. josh rosenberg\\
  OK. I'm rounding her now.
\item
  doctor\\
  OK. May I speak to you at some point today when you have a chance?
\item
  dr. josh rosenberg\\
  Call me at any point. All right?
\item
  doctor\\
  Thanks, Doctor. Appreciate it.
\item
  dr. josh rosenberg\\
  I'll see you later. Call me if you need anything, in all seriousness.
  You have my cell, right?
\item
  doctor\\
  Yeah.
\item
  dr. josh rosenberg\\
  Perfect. He's one of our medical students. He's been here forever. So
  we sent home all the medical students that rotate with us very early
  in this crisis, because I kind of looked at this and I said, one, we
  don't have enough PPE, you know, for all of the medical students that
  are coming through. And two --- you know, I hate to say it like this
  --- like, I don't want to expose them. They have enough time to get
  the living daylights scared out of them.
\item
  sheri fink\\
  Right.
\item
  dr. josh rosenberg\\
  {[}LAUGHS{]} Let them actually be students for a bit.
\end{itemize}

{[}music{]}

{[}AMBIENT VOICES{]}

\begin{itemize}
\item
  doctor 1\\
  I'm going to give myself the option, because it's my clinic.
\item
  doctor 2\\
  OK, because tonight we're going to publish the new schedule, OK?
\item
  dr. josh rosenberg\\
  Next patient. Santos.
\item
  doctor\\
  Yeah. So this is our --- she's our 54-year-old female, history of
  hypertension, came here with shortness of breath, fever, is admitted
  for acute hypoxic ---
\item
  dr. josh rosenberg\\
  She's the mom of our med student, right?
\item
  doctor\\
  Yes. She's confirmed positive Covid.
\end{itemize}

sheri fink

And when we get to this medical student's mom, things are not looking
good.

\begin{itemize}
\item
  doctor\\
  Her FI, too, has been hovering between 100 to 80. I just want to make
  sure you know that she's not doing OK.
\item
  dr. josh rosenberg\\
  She's not doing well. Um, yeah, I'll speak to the son. I know him
  pretty well.
\item
  doctor\\
  Yeah he's in here always.
\item
  dr. josh rosenberg\\
  Is he the next of kin? Is he the next of kin? He's the decision maker?
\item
  doctor\\
  Right now he has family ---
\end{itemize}

sheri fink

And Dr. Rosenberg wants to find out, is the son --- is the medical
student --- the one who will be making decisions about her further
treatment, about even possibly end of life care.

\begin{itemize}
\item
  dr. josh rosenberg\\
  But is he giving us consents?
\item
  doctor\\
  Yes.
\item
  dr. josh rosenberg\\
  Or does she have a husband?
\item
  doctor\\
  Yeah, yeah. He's been giving consent.
\item
  dr. josh rosenberg\\
  This is going to be hard, because he knows. He's a smart kid.
\end{itemize}

sheri fink

I mean, to me it sounded like he feels that this medical student, even
though he's still a student, is enough of a doctor to understand that
the prognosis isn't great --- that perhaps his mom has some risk factors
for this being more severe, and for her to possibly not make it.

\begin{itemize}
\tightlist
\item
  dr. josh rosenberg\\
  He's a good dude. He's a very sweet man, so we'll figure it out.
\end{itemize}

sheri fink

Of course, when it's your family member, it's not so simple.

\begin{itemize}
\tightlist
\item
  dr. josh rosenberg\\
  All right. Here.
\end{itemize}

sheri fink

There are many cases where the doctors and the patient's families have
very different views of how to proceed with treatment.

\begin{itemize}
\item
  dr. josh rosenberg\\
  Covid?
\item
  doctor\\
  Acute respiratory --- yeah. Well, it's pending, but most likely.
\item
  dr. josh rosenberg\\
  OK. So his pulmonary prognosis is horrible, right?
\item
  doctor\\
  Yes.
\item
  dr. josh rosenberg\\
  He's not getting better.
\item
  doctor\\
  No, he's not.
\item
  dr. josh rosenberg\\
  Blood gas is ---
\item
  doctor\\
  Not good enough.
\item
  dr. josh rosenberg\\
  Not good, and he's on 100\%.
\item
  doctor\\
  Yes.
\item
  dr. josh rosenberg\\
  So what does the fam want us to do?
\item
  doctor\\
  The family wanted us to continue treatment. They agreed to the NI.
\end{itemize}

sheri fink

Where the family still wants to press forward with all the intensive
care available.

\begin{itemize}
\tightlist
\item
  dr. josh rosenberg\\
  Right. So how many organ systems do we have down on him? We have our
  kidneys are down, our respiratory system's down, his cardiovascular is
  bad. He's on multi-organ system failure, right? So I have three out of
  my systems down already. His prognosis at that point, given his
  disease status, is just poor, unfortunately.
\end{itemize}

sheri fink

And where the doctors had come to a different conclusion and really felt
like there wasn't much hope, and that in fact the goals of care should
shift away from trying to extend life and much more toward comfort and
end of life --- accepting that the patient was likely going to die.

\begin{itemize}
\item
  dr. josh rosenberg\\
  And I hate to say it like this, but I don't know what I'm able really
  to offer in terms of getting him back to where he was before.

  {[}SIGHS{]} Next.
\end{itemize}

{[}music{]}

doctor

Check vitals from there too ---

\begin{itemize}
\tightlist
\item
  speaker (on intercom)\\
  Attention, please. Attention, please. Code blue, 6B.
\end{itemize}

sheri fink

Suddenly, we hear this announcement go out over the hospital loudspeaker
saying, code blue ---

\begin{itemize}
\item
  speaker (on intercom)\\
  Code blue, 6B. {[}KNOCKING{]}
\item
  doctor\\
  Josh?
\end{itemize}

sheri fink

--- which means that somebody needs to be resuscitated, that they are
basically dying.

\begin{itemize}
\item
  dr. josh rosenberg\\
  OK. It's code blue. you're on outreach or RESA?
\item
  doctor\\
  RESA.
\item
  sheri fink\\
  OK. All right. Can we follow you?
\item
  doctor\\
  Yeah.
\end{itemize}

{[}music{]}

{[}AMBIENT CHATTER{]}

\begin{itemize}
\item
  doctor 1\\
  Covid or non-Covid?
\item
  doctor 2\\
  No, it's not Covid.
\item
  doctor 3\\
  OK.
\end{itemize}

sheri fink

So the code blue, it turned out, wasn't for a Covid patient, but for a
patient who had other medical problems.

\begin{itemize}
\tightlist
\item
  doctor\\
  192. That's the code for 6A.
\end{itemize}

sheri fink

And they did CPR, and the patient survived.

\begin{itemize}
\tightlist
\item
  doctor\\
  I got it. You're good.
\end{itemize}

sheri fink

And for me, the moment was really just highlighting the fact that, in a
hospital, that that work goes on --- that there are all these other
patients, too, who have different medical problems, and people are still
having other emergencies. So hospitals can't just stop being hospitals
for everybody else. I But it's hard, because the number of patients with
Covid is increasing. Usually, if you have people with a scary,
infectious disease you would put them in specific rooms in the hospital,
but, of course now there's many more patients than there are isolation
rooms. So I think the doctors are very concerned about this possibility
that somebody could come into the hospital for something else ---

\begin{itemize}
\item
  dr. josh rosenberg\\
  She's Covid negative?
\item
  doctor\\
  --- and then, you know, catch Covid there. That's the real worst case
  scenario.
\item
  dr. josh rosenberg\\
  She's not a Covid issue?
\item
  doctor\\
  Not really, no.
\item
  dr. josh rosenberg\\
  Let's try to get her the heck out of this unit, please. OK? Get her
  out.
\end{itemize}

sheri fink

But of course, one of the big risks is to be a person who is walking
into that hospital every day to work there.

\begin{itemize}
\tightlist
\item
  dr. josh rosenberg\\
  Hello. Dr. Rosenberg speaking. I was paged.
\end{itemize}

sheri fink

And in fact, at one point ---

\begin{itemize}
\tightlist
\item
  dr. josh rosenberg\\
  Yes, yes, yes.
\end{itemize}

sheri fink

Dr. Rosenberg gets word that one of his residents ---

\begin{itemize}
\tightlist
\item
  dr. josh rosenberg\\
  He has Covid.
\end{itemize}

sheri fink

--- tested positive for Covid and is in the emergency room downstairs.

\begin{itemize}
\item
  dr. josh rosenberg\\
  Thanks. All right.

  What's up? You have his X-ray up? OK. I'll look at it in two seconds.
\end{itemize}

sheri fink

Someone pulls up an X-ray of the resident's lungs for him to look at,
and he peels off his Personal Protective Equipment, which in this case
includes his own ski goggles, and he looks at the X-ray. And
immediately, the tone shifts.

\begin{itemize}
\item
  dr. josh rosenberg\\
  That's shitty. I don't like that. I want him here. He is one to come
  up.
\item
  doctor\\
  Yeah. Is that a ---
\item
  dr. josh rosenberg\\
  He comes right up, because he's high risk for getting intubated.
\item
  doctor\\
  Yeah.
\end{itemize}

sheri fink

What he sees on the X-ray is something that looks bad to him.

\begin{itemize}
\item
  dr. josh rosenberg\\
  That's what I'm worried about, because his X-ray looks crappy.
\item
  doctor\\
  You know that he works here, right? Yeah.
\item
  dr. josh rosenberg\\
  No.
\item
  doctor\\
  It was just, like, let's just go back ---
\item
  dr. josh rosenberg\\
  He's one of our surgical residents. Bring him to the I.C.U.. Bring him
  here. Don't dilly. Don't ---
\item
  doctor\\
  No, no, I'm not saying that. I'm just saying ---
\end{itemize}

sheri fink

I think what was really striking to him, or what sort of, like, shocked
him was that this was another doctor.

\begin{itemize}
\tightlist
\item
  dr. josh rosenberg\\
  That is ours. That is one of us.
\end{itemize}

sheri fink

And close to his age, and somebody who's been doing the same kind of
work that he's doing every day. And I think that shatters that sense of
invulnerability.

\begin{itemize}
\tightlist
\item
  dr. josh rosenberg\\
  This is insanity. For my first day after being back from a week in
  this crap, holy shit.
\end{itemize}

sheri fink

I actually found out partway through that day that Dr. Rosenberg,
himself, had been out the previous week with symptoms of Covid. He
actually didn't get a test until his symptoms had resolved, and it
turned out to be negative, but he's pretty sure he had Covid.

\begin{itemize}
\tightlist
\item
  dr. josh rosenberg\\
  Well, one of the things we'll discuss that the nurse was telling you,
  but we need more nurses.
\end{itemize}

sheri fink

And this is a huge problem. A third of the doctors and nurses were out
sick. A number of them had tested positive for Covid and were critically
ill. And it's not just a problem for this hospital. It's a problem all
over New York City, that as the hospitals are overwhelmed with Covid
patients, you have high numbers of health staff out sick.

\begin{itemize}
\tightlist
\item
  andrew cuomo\\
  Good afternoon. Thank you all for taking the time for being here
  today. As Governor of New York, I am asking health care professionals
  across the country, if you don't have a health care crisis in your
  community, please come help us in New York now.
\end{itemize}

{[}music{]}

sheri fink

The day that I was at the hospital, New York Governor Cuomo pleaded for
doctors and nurses and health care staff from around the U.S. to come to
New York ---

\begin{itemize}
\tightlist
\item
  andrew cuomo\\
  We need relief. We need relief for doctors. We need relief for
  attendants.
\end{itemize}

sheri fink

--- in part to help fill in for the workers who are falling ill across
the state.

\begin{itemize}
\item
  andrew cuomo\\
  So if you're not busy, come help us please.
\item
  dr. josh rosenberg\\
  Hey.

  Hey, he's going to be in I.C.U. 12, OK? No, not yet. They're about to
  bring him up shortly, but we're getting everything done.

  I know.

  I know. Trust me, it's freaky. I mean, he's only five years younger
  than me, you know? I'm 45, like half of our patients upstairs. We have
  40-year-olds who are intubated.

  Jesus.

  Geez.

  {[}GROANS{]} Man, this is brutal.

  All right, good. I just wanted to let you know where it would be, all
  right? You got it. I'll speak to you later. Bye.

  {[}SIGHS{]} I am tired.

  {[}AMBIENT CHATTER{]}
\end{itemize}

michael barbaro

A few days ago, as the daily death toll in New York began to decline,
state officials said it appeared that the pandemic was approaching its
peak, and that the worst was over. But on Monday, New York's daily death
toll spiked again to 778. So far, nearly 11,000 people in the state have
died from the coronavirus. Among them was the mother of Dr. Rosenberg's
medical student, who died the day after Sheri visited the hospital.

We'll be right back.

{[}music{]}

Here's what else you need to know today.

\begin{itemize}
\item
  archived recording\\
  Well, yesterday, the president at his news conference --- and this is
  his quote --- he said, he has the power. He says, when someone is
  President of the United States, the authority is total, and he said
  the governors know that. Do you know that?
\item
  andrew cuomo\\
  No. I don't know what the president is talking about, frankly. We have
  a Constitution. The Constitution ---
\end{itemize}

michael barbaro

On Tuesday, governors on the East and West Coast, led by Andrew Cuomo of
New York, rejected President Trump's claim that he has the legal
authority to reopen the American economy by himself once he determines
that the pandemic is over.

\begin{itemize}
\tightlist
\item
  andrew cuomo\\
  The federal government does not have absolute power. It says the exact
  opposite that the president said. It says, that would be a king. We
  would have had King George Washington. We didn't have King George
  Washington, and we don't have King Trump. We have President Trump.
\end{itemize}

michael barbaro

Appearing on NBC and MSNBC, Cuomo said that if Trump prematurely
instructed states to end their lockdowns, many governors would disregard
the order.

\begin{itemize}
\tightlist
\item
  andrew cuomo\\
  If he says to me, I declare it open, and that is a public health risk
  or it's reckless with the welfare of the people of my state, I will
  oppose it. And then, we will have a constitutional crisis like you
  haven't seen in decades where states tell the federal government,
  we're not going to follow your order.
\end{itemize}

michael barbaro

And ---

\begin{itemize}
\tightlist
\item
  donald trump\\
  As the organization's leading sponsor, the United States has a duty to
  insist on full accountability.
\end{itemize}

michael barbaro

President Trump said that he planned to end U.S. funding funding of the
World Health Organization, the international group responsible for
monitoring the pandemic, over what he said were its failures to properly
manage the crisis. Trump singled out the WHO's opposition to banning
travel from China, a position that he said has proven disastrous for the
countries that followed it.

That's it for ``The Daily.'' I'm Michael Barbaro. See you tomorrow.

More than 40 percent of the hospital's inpatients --- scattered
throughout the building --- were confirmed or suspected coronavirus
cases, as were more than two-thirds of the critical care patients. By
Wednesday four had died, three of them since Monday.

\hypertarget{latest-updates-the-coronavirus-outbreak}{%
\section{\texorpdfstring{\href{https://www.nytimes3xbfgragh.onion/2020/09/09/world/covid-19-coronavirus.html?action=click\&pgtype=Article\&state=default\&region=MAIN_CONTENT_1\&context=storylines_live_updates}{Latest
Updates: The Coronavirus
Outbreak}}{Latest Updates: The Coronavirus Outbreak}}\label{latest-updates-the-coronavirus-outbreak}}

Updated 2020-09-09T15:31:10.766Z

\begin{itemize}
\tightlist
\item
  \href{https://www.nytimes3xbfgragh.onion/2020/09/09/world/covid-19-coronavirus.html?action=click\&pgtype=Article\&state=default\&region=MAIN_CONTENT_1\&context=storylines_live_updates\#link-5b0bf0d1}{As
  drugmakers pledge to thoroughly vet vaccines, one company pauses its
  trials for a safety review.}
\item
  \href{https://www.nytimes3xbfgragh.onion/2020/09/09/world/covid-19-coronavirus.html?action=click\&pgtype=Article\&state=default\&region=MAIN_CONTENT_1\&context=storylines_live_updates\#link-6e2052bd}{The
  director of the N.I.H. and the surgeon general answer senators'
  questions.}
\item
  \href{https://www.nytimes3xbfgragh.onion/2020/09/09/world/covid-19-coronavirus.html?action=click\&pgtype=Article\&state=default\&region=MAIN_CONTENT_1\&context=storylines_live_updates\#link-74c78736}{Britain
  bans most gatherings of more than six people amid a spike in cases.}
\end{itemize}

\href{https://www.nytimes3xbfgragh.onion/2020/09/09/world/covid-19-coronavirus.html?action=click\&pgtype=Article\&state=default\&region=MAIN_CONTENT_1\&context=storylines_live_updates}{See
more updates}

More live coverage:
\href{https://www.nytimes3xbfgragh.onion/live/2020/09/09/business/stock-market-today-coronavirus?action=click\&pgtype=Article\&state=default\&region=MAIN_CONTENT_1\&context=storylines_live_updates}{Markets}

More than a half-dozen hospital workers have contracted the virus and
close to 50 staff members were potentially exposed by just one patient
--- the hospital's first --- who developed symptoms after being in the
I.C.U. for a different medical problem, according to hospital leaders.
Some of them have been in quarantine. Most worrisome, at the start of
the week, two hospital staff members were receiving intensive care
themselves. It feels, one employee said, like an invisible war.

In the emergency room on Monday, Dr. de Souza thought she saw a familiar
face. A patient was coughing so hard he could barely speak. The young
man was one of their own, Dr. Yijiao Fan, 31, an oral surgery resident
with no prior medical issues who had tested positive for the virus. He
had been in isolation at home all week and thought he was getting
better, but began coughing blood that morning. He was awaiting a chest
scan. He had no known risk factors other, perhaps, than practicing his
profession.

Dr. Fan, as both surgeon and patient, had a message for a nation
debating how to fight the pandemic. It was short enough to whisper
between coughing fits: Just stay home.

Image

Dr. de Souza, right, said she was concerned that if the patient volume
continued to grow at the current pace, the emergency room would be out
of space by next week.Credit...Victor J. Blue for The New York Times

\hypertarget{this-is-where-my-heart-is}{%
\subsection{`This Is Where My Heart
Is'}\label{this-is-where-my-heart-is}}

The hospital keeps personal protective equipment tightly guarded,
because it is rapidly consuming donations of masks and other supplies;
this week it was low on gowns. **** In the emergency room, those in the
know approach the busy unit clerk, Donna Mosley, who is surrounded by
ringing phones. ``Hold on, I can do one thing at a time,'' she told one
employee.

Soon she bent down below her desk, fished in a box and handed over a
set: an N95 mask that filters viruses; a surgical mask to go over it,
with a plastic shield in crinkly packaging, donated by the relative of
an emergency room doctor; a thin blue gown that covers a person's front
and arms and is open in back; and a pair of blue booties. Employees have
to sign a form. One set per day.

The hospital has no parent company to request extra supplies from, no
network of other institutions to share resources during the pandemic for
the predominantly low-income and culturally diverse population it
serves. It has resisted the era of mergers. ``As an independent hospital
we can control our destiny, control our resources, and really do what we
think is right by the community,'' said Gary G. Terrinoni, its president
and chief executive.

Last week the hospital ran dangerously short of testing swabs, and its
appeals for more reached the federal government. ``We're in disaster
mode,'' Mr. Terrinoni said.

The emergency room phone rang again. It was a man who lived down the
street, offering handmade masks. ``Are you selling them or donating
them?'' Dr. de Souza asked. Donating. She took his number and thanked
him. The hospital has received gifts of gloves, food and a brown bottle
with a mysterious liquid concocted by a local artisanal deodorant maker,
which said it could be used to disinfect face shields. For now, that
would be put aside.

Image

After the hospital ran dangerously short of testing swabs last week, a
delivery of test kits from the federal government arrived Sunday
night.Credit...Victor J. Blue for The New York Times

An even bigger gift had arrived the previous night in a convoy of black
sport utility vehicles that approached with flashing lights: boxes of
coronavirus test kits reportedly from the federal strategic national
stockpile, 200 in all. On Monday morning, two officers with the U.S.
Public Health Service in crisp blue uniforms arrived to oversee their
use.

But there was a problem. Test results from the kits would be delivered
directly to the patient, not to the hospital. Dr. de Souza asked the
public health officers how that could possibly work. ``We can't predict
the patient's clinical course,'' she said. If someone was using a
breathing tube, ``they're not going to be able to come to the phone and
get their result.'' Hospital leaders tried to sort out the issue, and
the boxes of tests were not opened.

Under new restrictions from the local health department, communicated by
fax to the hospital's laboratory, doctors were supposed to test only the
people sick enough to be admitted as inpatients. Dr. de Souza printed
out the revised testing protocol, the eighth the hospital had received
in recent weeks. She walked through the emergency department ripping
down copies of the old one and stapling the new guidelines to the walls.

A few weeks ago, the hospital was able to send swabs to the city's
public health laboratory, which returned results in a day. Now, swabs
were picked up by courier twice a day and sent to a Quest laboratory in
California. At first the results took two days, then four days, and now
it was a week.

``That's really killing us,'' Mr. Terrinoni said. On Wednesday the
hospital had 65 patients awaiting results. They each had to be isolated
in a room that was typically used for two patients.

Image

A designated area for patients suspected of having the
virus.Credit...Victor J. Blue for The New York Times

The state had asked the hospital for a plan to increase bed capacity by
50 percent. Mr. Terrinoni found the space, but ``we don't have the beds,
literally the physical beds, we don't have the staffing.'' The hospital
put out a call to the city's volunteer Medical Reserve Corps for
doctors, nurses and respiratory therapists.

There were other important roles. Marilyn Hunt pushed a cart with a
garbage can and supplies, stopping to change paper towels in one of the
emergency department's bathrooms. ``We're here in the front lines trying
to do the best,'' she said. ``We're supporting each other,'' she added,
``praying to God that this doesn't do a lot of damage.''

After the virus hit, Dr. de Souza, 55, worked three weeks straight; her
deputy was one of those quarantined for a while. Born in Paris, the
daughter of a diplomat from Benin, and raised in several countries, Dr.
de Souza trained at the Brooklyn hospital, located in Fort Greene.
``This is where my heart is,'' she said.

She offered to stay away from her family during the pandemic, but they
insisted that she come home at night. When she arrives, she immediately
takes a shower and washes her clothes in hot water. She sleeps in a
separate room from her husband and maintains distance from him and their
adult son and his girlfriend, who have moved in with them.

``Just trying to keep them safe, that's my main concern,'' she said. ``I
think every health care worker has the same concern.''

Image

Most of the patients suspected of having the virus are screened at a
walk-up testing tent outside the emergency room.Credit...Victor J. Blue
for The New York Times

\hypertarget{we-all-probably-have-it}{%
\subsection{`We All Probably Have It'}\label{we-all-probably-have-it}}

In the outdoor testing tent, Luciano Mahecha, 50, peeled off his ski
jacket. A surgical intern placed a stethoscope on his back. ``Your lungs
are nice and clear. There's no need to test,'' Dr. Robert Jardine said.
He told Mr. Mahecha to go home and stay there as long as his symptoms
--- a cough and fatigue --- persisted.

Mr. Mahecha, whose first language is not English, agreed to keep away
from other people, but he seemed to misunderstand whether he had the
virus. ``I thought I have it, but thank God everything is fine,'' he
said. ``I don't have it.''

``He probably has it,'' Dr. Jardine told a reporter, and then gestured
toward his colleagues. ``We all probably have it. We're exposed every
day to people who we know'' are more likely than others to be infected.
Medical students were told to stop coming to the hospital last week, but
residents like Dr. Jardine, less than a year out of medical school,
accounted for a majority of the doctors evaluating people in the tent.

The rain picked up outside, and the floor began buckling. ``We need
help. Tent is getting flooded,'' Dr. de Souza messaged the hospital's
engineers on the Signal app.

An older man shuffled into the tent, using a walker. He waited, sitting
side by side with others coughing behind surgical masks they were given
at the tent door. When he told the registrar he had come for wound care
treatment, the staff member was alarmed. ``You gotta get out of here!''
he instructed.

Image

Diana Purnell described her symptoms to~Dr. Samridhi Sinha: fever, dry
cough, extreme tiredness. ``Yeah, it's the virus,'' the doctor
said.Credit...Victor J. Blue for The New York Times

Diana Purnell had a fever for a week, was short of breath and was sicker
than most in the tent. Her age, 62, and high blood pressure put her at
greater risk for complications from the coronavirus, which she suspected
she had contracted from her Zumba teacher. She had called the New York
State coronavirus hotline at 1 in the morning, waiting for two hours on
hold until she woke up to a nurse's voice.

Ms. Purnell said she was told a doctor would call her back about
testing, but one never did. She said she reached out to an urgent care
doctor in her neighborhood, but his clinic was closed.

In the emergency room, Ms. Purnell sat in a blue chair in the former
fast track area **** with a dozen other listless patients, one of whom
was missing a surgical mask and coughing. When she was taken for an
X-ray, she was put on the side that was supposed to be kept for patients
not suspected of having coronavirus infection: ``She's not Covid, so we
can put her in this room,'' a staff member said.

Image

Because Ms. Purnell's chest X-ray was clear and her vital signs were
stable, she was sent home to await a call from the health department if
her coronavirus test was positive. Credit...Victor J. Blue for The New
York Times

After the X-ray, Dr. Samridhi Sinha, a second-year resident, asked about
her symptoms --- fever, dry cough, extreme tiredness. ``Yeah, it's the
virus,'' the doctor said.

Aside from the test kits from the federal stockpile, still unopened in
their boxes, the hospital was down to its last four testing swabs. They
were being saved for critically ill inpatients. Dr. Sinha asked Ms.
Purnell to go to a sink in the corner and spit into a cup typically used
for urine samples. The health department did not recommend this method
of coronavirus testing, but Quest was willing to accept it.

Because Ms. Purnell's vital signs were stable and chest X-ray was clear,
she was sent home to await a call from the health department if her test
was positive. ``If it's not positive you're not going to call?'' she
asked. Dr. Sinha said she didn't think so. ``Because we are testing
thousands of people, right now only people who are testing positive are
getting calls.''

Image

Dr. Yijiao Fan, an oral surgery resident at the hospital and a confirmed
case, had been in isolation at home all week and thought he was getting
better, but began coughing blood that morning.Credit...Victor J. Blue
for The New York Times

\hypertarget{its-going-to-get-worse}{%
\subsection{`It's Going to Get Worse'}\label{its-going-to-get-worse}}

Walking through the emergency department, Dr. de Souza stopped to talk
with two intensive care doctors.

``You've got one down here,'' she told them. Amid the patients waiting
to be moved upstairs was the critically ill patient on a ventilator.

The unit was full, Dr. Jose Orsini told her, adding, ``And it's going to
get worse.''

Dr. de Souza dreads that possibility, haunted by accounts of Italian
doctors denying lifesaving resources to older adults or providing
inadequate care at overrun hospitals. ``I'm asking myself if that's
where we're going,'' she said on Wednesday night. Some patients who were
screened and went home have since returned with difficulty breathing,
needing to be put on ventilators. ``It's getting really, really more
difficult every day.''

The intensive care unit had 18 staffed beds, and it added six more
Wednesday night. All are full and about two-thirds of the patients are
confirmed or suspected to have coronavirus, according to Dr. James
Gasperino, director of critical care services at the hospital.

He said eight more could be made immediately available in the surgical
intensive care unit, and more still could be opened up, with additional
staffing, in operating rooms, the surgical recovery area and a former
intermediate care unit on a different floor.

Patients with the coronavirus who develop pneumonia can often require
two to three weeks on a ventilator. ``The intensity level is higher,''
said Dr. Gasperino, who is also chair of medicine. ``It's harder to
oxygenate than your typical flu patient who's sick.'' He added, ``The
staff is anxious.'' So far none of the coronavirus patients requiring
ventilators have recovered enough not to need one, although several
younger patients were rapidly improving, he said.

Another patient, Dr. Gasperino said, went into cardiac arrest Sunday
night, and he and his team were able to bring the person back to life.
Four coronavirus patients at the hospital have died, including some
whose families opted to withdraw life support.

This week the hospital counted up all the ventilators it had, including
anesthesia machines used during surgeries. It found 61 in total. ``We're
looking to purchase new ventilators,'' Dr. Gasperino said. **** ``We're
looking at one ventilator for two patients,'' which some experts believe
would be risky and difficult. He said they would need to simulate the
process to make sure it worked.

While he hoped to avoid the worst-case scenario, Dr. Gasperino said he
and the head of the ethics committee were planning to draft a guideline
on how the hospital might ration ventilators, based on published
recommendations.

On Tuesday --- after 120 swabs from Quest arrived --- Lenny Singletary,
the hospital's senior vice president for external affairs, returned the
federal test swabs to the city's department of emergency management,
asking half-jokingly if he could trade them for ventilators.

The next day, he said the hospital accepted 12 ventilators from the
emergency management office, St. George's University and a company,
Comprehensive Equipment Management Corporation.

Image

Lenny Singletary,~the hospital's senior vice president for external
affairs, said the hospital must continue to serve other patients as
well.Credit...Victor J. Blue for The New York Times

For now, staff members are still pushing to do everything possible.
``The hospital cannot close to other patients,'' said Mr. Singletary,
who grew up in the neighborhood. The medical center cares for children,
women having babies, and people having strokes, among others. ``You
can't shut down the hospital to treat coronavirus'' alone, he said.

And so the staff members continue their work.

Image

The 175-year-old hospital is independent, not part of a larger network
with which it might share resources.Credit...Victor J. Blue for The New
York Times

``They just take their courage in their hands,'' Dr. de Souza said of
her team. ``They put on their garb and they show up. That's what they
do. Of course they have anxiety, of course they have fear, they're
human. None of us knows where this is taking us. We don't even know if
we might get sick. But none of them so far has defaulted on their duty,
their calling.''

Advertisement

\protect\hyperlink{after-bottom}{Continue reading the main story}

\hypertarget{site-index}{%
\subsection{Site Index}\label{site-index}}

\hypertarget{site-information-navigation}{%
\subsection{Site Information
Navigation}\label{site-information-navigation}}

\begin{itemize}
\tightlist
\item
  \href{https://help.nytimes3xbfgragh.onion/hc/en-us/articles/115014792127-Copyright-notice}{©~2020~The
  New York Times Company}
\end{itemize}

\begin{itemize}
\tightlist
\item
  \href{https://www.nytco.com/}{NYTCo}
\item
  \href{https://help.nytimes3xbfgragh.onion/hc/en-us/articles/115015385887-Contact-Us}{Contact
  Us}
\item
  \href{https://www.nytco.com/careers/}{Work with us}
\item
  \href{https://nytmediakit.com/}{Advertise}
\item
  \href{http://www.tbrandstudio.com/}{T Brand Studio}
\item
  \href{https://www.nytimes3xbfgragh.onion/privacy/cookie-policy\#how-do-i-manage-trackers}{Your
  Ad Choices}
\item
  \href{https://www.nytimes3xbfgragh.onion/privacy}{Privacy}
\item
  \href{https://help.nytimes3xbfgragh.onion/hc/en-us/articles/115014893428-Terms-of-service}{Terms
  of Service}
\item
  \href{https://help.nytimes3xbfgragh.onion/hc/en-us/articles/115014893968-Terms-of-sale}{Terms
  of Sale}
\item
  \href{https://spiderbites.nytimes3xbfgragh.onion}{Site Map}
\item
  \href{https://help.nytimes3xbfgragh.onion/hc/en-us}{Help}
\item
  \href{https://www.nytimes3xbfgragh.onion/subscription?campaignId=37WXW}{Subscriptions}
\end{itemize}
