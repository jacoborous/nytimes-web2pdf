Sections

SEARCH

\protect\hyperlink{site-content}{Skip to
content}\protect\hyperlink{site-index}{Skip to site index}

\href{https://www.nytimes3xbfgragh.onion/section/health}{Health}

\href{https://myaccount.nytimes3xbfgragh.onion/auth/login?response_type=cookie\&client_id=vi}{}

\href{https://www.nytimes3xbfgragh.onion/section/todayspaper}{Today's
Paper}

\href{/section/health}{Health}\textbar{}Lost Sense of Smell May Be
Peculiar Clue to Coronavirus Infection

\url{https://nyti.ms/2WB21rx}

\begin{itemize}
\item
\item
\item
\item
\item
\item
\end{itemize}

\hypertarget{the-coronavirus-outbreak}{%
\subsubsection{\texorpdfstring{\href{https://www.nytimes3xbfgragh.onion/news-event/coronavirus?name=styln-coronavirus-national\&region=TOP_BANNER\&block=storyline_menu_recirc\&action=click\&pgtype=Article\&impression_id=2a5e83f0-f1c5-11ea-b5fb-29abc66233f3\&variant=undefined}{The
Coronavirus
Outbreak}}{The Coronavirus Outbreak}}\label{the-coronavirus-outbreak}}

\begin{itemize}
\tightlist
\item
  live\href{https://www.nytimes3xbfgragh.onion/2020/09/08/world/covid-19-coronavirus.html?name=styln-coronavirus-national\&region=TOP_BANNER\&block=storyline_menu_recirc\&action=click\&pgtype=Article\&impression_id=2a5e83f1-f1c5-11ea-b5fb-29abc66233f3\&variant=undefined}{Latest
  Updates}
\item
  \href{https://www.nytimes3xbfgragh.onion/interactive/2020/us/coronavirus-us-cases.html?name=styln-coronavirus-national\&region=TOP_BANNER\&block=storyline_menu_recirc\&action=click\&pgtype=Article\&impression_id=2a5eab00-f1c5-11ea-b5fb-29abc66233f3\&variant=undefined}{Maps
  and Cases}
\item
  \href{https://www.nytimes3xbfgragh.onion/interactive/2020/science/coronavirus-vaccine-tracker.html?name=styln-coronavirus-national\&region=TOP_BANNER\&block=storyline_menu_recirc\&action=click\&pgtype=Article\&impression_id=2a5eab01-f1c5-11ea-b5fb-29abc66233f3\&variant=undefined}{Vaccine
  Tracker}
\item
  \href{https://www.nytimes3xbfgragh.onion/2020/09/02/your-money/eviction-moratorium-covid.html?name=styln-coronavirus-national\&region=TOP_BANNER\&block=storyline_menu_recirc\&action=click\&pgtype=Article\&impression_id=2a5eab02-f1c5-11ea-b5fb-29abc66233f3\&variant=undefined}{Eviction
  Moratorium}
\item
  \href{https://www.nytimes3xbfgragh.onion/interactive/2020/09/02/magazine/food-insecurity-hunger-us.html?name=styln-coronavirus-national\&region=TOP_BANNER\&block=storyline_menu_recirc\&action=click\&pgtype=Article\&impression_id=2a5eab03-f1c5-11ea-b5fb-29abc66233f3\&variant=undefined}{American
  Hunger}
\end{itemize}

Advertisement

\protect\hyperlink{after-top}{Continue reading the main story}

Supported by

\protect\hyperlink{after-sponsor}{Continue reading the main story}

\hypertarget{lost-sense-of-smell-may-be-peculiar-clue-to-coronavirus-infection}{%
\section{Lost Sense of Smell May Be Peculiar Clue to Coronavirus
Infection}\label{lost-sense-of-smell-may-be-peculiar-clue-to-coronavirus-infection}}

Doctor groups are recommending testing and isolation for people who lose
their ability to smell and taste, even if they have no other symptoms.

\includegraphics{https://static01.graylady3jvrrxbe.onion/images/2020/03/22/science/22virus-smell1/merlin_170767317_c09b46b1-0467-47ef-84af-5e61e80bd4fb-articleLarge.jpg?quality=75\&auto=webp\&disable=upscale}

\href{https://www.nytimes3xbfgragh.onion/by/roni-caryn-rabin}{\includegraphics{https://static01.graylady3jvrrxbe.onion/images/2018/02/20/multimedia/author-roni-caryn-rabin/author-roni-caryn-rabin-thumbLarge-v3.png}}

By \href{https://www.nytimes3xbfgragh.onion/by/roni-caryn-rabin}{Roni
Caryn Rabin}

\begin{itemize}
\item
  Published March 22, 2020Updated March 26, 2020
\item
  \begin{itemize}
  \item
  \item
  \item
  \item
  \item
  \item
  \end{itemize}
\end{itemize}

\href{https://cn.nytimes3xbfgragh.onion/health/20200325/coronavirus-symptoms-smell-taste/}{阅读简体中文版}\href{https://cn.nytimes3xbfgragh.onion/health/20200325/coronavirus-symptoms-smell-taste/zh-hant/}{閱讀繁體中文版}

A mother who was infected with the
\href{https://www.nytimes3xbfgragh.onion/article/coronavirus-body-symptoms.html}{coronavirus}
couldn't smell her baby's full diaper. Cooks who can usually name every
spice in a restaurant dish can't smell curry or garlic, and food tastes
bland. Others say they can't pick up the sweet scent of shampoo or the
foul odor of kitty litter.

Anosmia, the loss of sense of smell, and ageusia, an accompanying
diminished sense of taste, have emerged as peculiar telltale signs of
Covid-19, the disease caused by the coronavirus, and possible markers of
infection.

On Friday, British ear, nose and throat doctors, citing reports from
colleagues around the world, called on adults who lose their senses of
smell to isolate themselves for seven days, even if they have no other
symptoms, to slow the disease's spread. The published data is limited,
but doctors are concerned enough to raise warnings.

``We really want to raise awareness that this is a sign of infection and
that anyone who develops loss of sense of smell should self-isolate,''
Prof. Claire Hopkins, president of the British Rhinological Society,
wrote in an email. ``It could contribute to slowing transmission and
save lives.''

She and Nirmal Kumar, president of ENT UK, a group representing ear,
nose and throat doctors in Britain, issued a joint statement urging
health care workers to use personal protective equipment when treating
any patients who have lost their senses of smell, and advised against
performing nonessential sinus endoscopy procedures on anyone, because
the virus replicates in the nose and the throat and an exam can prompt
coughs or sneezes that expose the doctor to a high level of virus.

Two ear, nose and throat specialists in Britain who have been infected
with the coronavirus are in critical condition, Dr. Hopkins said.
Earlier reports from Wuhan, China, where the coronavirus first emerged,
had warned that ear, nose and throat specialists as well as eye doctors
were infected and dying in large numbers, Dr. Hopkins said.

The British physicians cited reports from other countries indicating
that significant numbers of coronavirus patients experienced anosmia,
saying that in South Korea, where testing has been widespread, 30
percent of 2,000 patients who tested positive experienced anosmia as
their major presenting symptom (these were mild cases).

\hypertarget{latest-updates-the-coronavirus-outbreak}{%
\section{\texorpdfstring{\href{https://www.nytimes3xbfgragh.onion/2020/09/08/world/covid-19-coronavirus.html?action=click\&pgtype=Article\&state=default\&region=MAIN_CONTENT_1\&context=storylines_live_updates}{Latest
Updates: The Coronavirus
Outbreak}}{Latest Updates: The Coronavirus Outbreak}}\label{latest-updates-the-coronavirus-outbreak}}

Updated 2020-09-08T11:04:36.368Z

\begin{itemize}
\tightlist
\item
  \href{https://www.nytimes3xbfgragh.onion/2020/09/08/world/covid-19-coronavirus.html?action=click\&pgtype=Article\&state=default\&region=MAIN_CONTENT_1\&context=storylines_live_updates\#link-4a77847f}{As
  senators return to Washington, an impasse over a virus relief package
  looms.}
\item
  \href{https://www.nytimes3xbfgragh.onion/2020/09/08/world/covid-19-coronavirus.html?action=click\&pgtype=Article\&state=default\&region=MAIN_CONTENT_1\&context=storylines_live_updates\#link-679303d7}{Nine
  drugmakers pledge to thoroughly vet any coronavirus vaccine.}
\item
  \href{https://www.nytimes3xbfgragh.onion/2020/09/08/world/covid-19-coronavirus.html?action=click\&pgtype=Article\&state=default\&region=MAIN_CONTENT_1\&context=storylines_live_updates\#link-1c973131}{`The
  lockdown killed my father': Farmer suicides add to India's virus
  misery.}
\end{itemize}

\href{https://www.nytimes3xbfgragh.onion/2020/09/08/world/covid-19-coronavirus.html?action=click\&pgtype=Article\&state=default\&region=MAIN_CONTENT_1\&context=storylines_live_updates}{See
more updates}

More live coverage:
\href{https://www.nytimes3xbfgragh.onion/live/2020/09/08/business/stock-market-today-coronavirus?action=click\&pgtype=Article\&state=default\&region=MAIN_CONTENT_1\&context=storylines_live_updates}{Markets}

The
\href{https://www.entnet.org/content/coronavirus-disease-2019-resources}{American
Academy of Otolaryngology on Sunday posted information on its website}
saying that mounting anecdotal evidence indicates that lost or reduced
sense of smell and loss of taste are significant symptoms associated
with Covid-19, and that they have been seen in patients who ultimately
tested positive with no other symptoms.

The symptoms, in the absence of allergies or sinusitis, should alert
doctors to screen patients for the virus and ``warrant serious
consideration for self isolation and testing of these individuals,'' the
academy said. The organization has reminded its members that the Centers
for Disease Control and Prevention
\href{https://www.entnet.org/content/new-recommendations-regarding-urgent-and-nonurgent-patient-care}{has
urged all clinicians to prioritize urgent and emergency visits} for the
next several weeks and to reschedule elective and routine procedures.

``There is evolving evidence that otolaryngologists are among the
highest risk group when performing upper airway surgeries and
examinations,'' said a notice posted on the academy's website on Friday.
``A high rate of transmission of Covid-19 to otolaryngologists has been
reported from China, Italy and Iran, many resulting in death.''

Dr. Rachel Kaye, an assistant professor of otolaryngology at Rutgers,
said colleagues in New Rochelle, N.Y., which has been the center of an
outbreak, first alerted her to the smell loss associated with the
coronavirus, sharing that patients who had first complained of anosmia
later tested positive for the coronavirus. ``This raised a lot of alarms
for me personally,'' Dr. Kaye said, because those patients ``won't know
to self quarantine.''

``Most ENTs have on their own accord tried to scale down,'' she said,
adding that her department at Rutgers had already started using personal
protective equipment and stopped performing nonessential exams.

\includegraphics{https://static01.graylady3jvrrxbe.onion/images/2020/03/22/science/22virus-smell2/22virus-smell2-articleLarge.jpg?quality=75\&auto=webp\&disable=upscale}

In the areas of Italy most heavily affected by the virus, doctors say
they have concluded that loss of taste and smell is an indication that a
person who otherwise seems healthy is in fact carrying the virus and may
be spreading it to others.

``Almost everybody who is hospitalized has this same story,'' said Dr.
Marco Metra, chief of the cardiology department at the main hospital in
Brescia, where 700 of 1,200 inpatients have the coronavirus. ``You ask
about the patient's wife or husband. And the patient says, `My wife has
just lost her smell and taste but otherwise she is well.' So she is
likely infected, and she is spreading it with a very mild form.''

\href{https://www.nytimes3xbfgragh.onion/news-event/coronavirus?action=click\&pgtype=Article\&state=default\&region=MAIN_CONTENT_3\&context=storylines_faq}{}

\hypertarget{the-coronavirus-outbreak-}{%
\subsubsection{The Coronavirus Outbreak
›}\label{the-coronavirus-outbreak-}}

\hypertarget{frequently-asked-questions}{%
\paragraph{Frequently Asked
Questions}\label{frequently-asked-questions}}

Updated September 4, 2020

\begin{itemize}
\item ~
  \hypertarget{what-are-the-symptoms-of-coronavirus}{%
  \paragraph{What are the symptoms of
  coronavirus?}\label{what-are-the-symptoms-of-coronavirus}}

  \begin{itemize}
  \tightlist
  \item
    In the beginning, the coronavirus
    \href{https://www.nytimes3xbfgragh.onion/article/coronavirus-facts-history.html?action=click\&pgtype=Article\&state=default\&region=MAIN_CONTENT_3\&context=storylines_faq\#link-6817bab5}{seemed
    like it was primarily a respiratory illness}~--- many patients had
    fever and chills, were weak and tired, and coughed a lot, though
    some people don't show many symptoms at all. Those who seemed
    sickest had pneumonia or acute respiratory distress syndrome and
    received supplemental oxygen. By now, doctors have identified many
    more symptoms and syndromes. In April,
    \href{https://www.nytimes3xbfgragh.onion/2020/04/27/health/coronavirus-symptoms-cdc.html?action=click\&pgtype=Article\&state=default\&region=MAIN_CONTENT_3\&context=storylines_faq}{the
    C.D.C. added to the list of early signs}~sore throat, fever, chills
    and muscle aches. Gastrointestinal upset, such as diarrhea and
    nausea, has also been observed. Another telltale sign of infection
    may be a sudden, profound diminution of one's
    \href{https://www.nytimes3xbfgragh.onion/2020/03/22/health/coronavirus-symptoms-smell-taste.html?action=click\&pgtype=Article\&state=default\&region=MAIN_CONTENT_3\&context=storylines_faq}{sense
    of smell and taste.}~Teenagers and young adults in some cases have
    developed painful red and purple lesions on their fingers and toes
    --- nicknamed ``Covid toe'' --- but few other serious symptoms.
  \end{itemize}
\item ~
  \hypertarget{why-is-it-safer-to-spend-time-together-outside}{%
  \paragraph{Why is it safer to spend time together
  outside?}\label{why-is-it-safer-to-spend-time-together-outside}}

  \begin{itemize}
  \tightlist
  \item
    \href{https://www.nytimes3xbfgragh.onion/2020/05/15/us/coronavirus-what-to-do-outside.html?action=click\&pgtype=Article\&state=default\&region=MAIN_CONTENT_3\&context=storylines_faq}{Outdoor
    gatherings}~lower risk because wind disperses viral droplets, and
    sunlight can kill some of the virus. Open spaces prevent the virus
    from building up in concentrated amounts and being inhaled, which
    can happen when infected people exhale in a confined space for long
    stretches of time, said Dr. Julian W. Tang, a virologist at the
    University of Leicester.
  \end{itemize}
\item ~
  \hypertarget{why-does-standing-six-feet-away-from-others-help}{%
  \paragraph{Why does standing six feet away from others
  help?}\label{why-does-standing-six-feet-away-from-others-help}}

  \begin{itemize}
  \tightlist
  \item
    The coronavirus spreads primarily through droplets from your mouth
    and nose, especially when you cough or sneeze. The C.D.C., one of
    the organizations using that measure,
    \href{https://www.nytimes3xbfgragh.onion/2020/04/14/health/coronavirus-six-feet.html?action=click\&pgtype=Article\&state=default\&region=MAIN_CONTENT_3\&context=storylines_faq}{bases
    its recommendation of six feet}~on the idea that most large droplets
    that people expel when they cough or sneeze will fall to the ground
    within six feet. But six feet has never been a magic number that
    guarantees complete protection. Sneezes, for instance, can launch
    droplets a lot farther than six feet,
    \href{https://jamanetwork.com/journals/jama/fullarticle/2763852}{according
    to a recent study}. It's a rule of thumb: You should be safest
    standing six feet apart outside, especially when it's windy. But
    keep a mask on at all times, even when you think you're far enough
    apart.
  \end{itemize}
\item ~
  \hypertarget{i-have-antibodies-am-i-now-immune}{%
  \paragraph{I have antibodies. Am I now
  immune?}\label{i-have-antibodies-am-i-now-immune}}

  \begin{itemize}
  \tightlist
  \item
    As of right
    now,\href{https://www.nytimes3xbfgragh.onion/2020/07/22/health/covid-antibodies-herd-immunity.html?action=click\&pgtype=Article\&state=default\&region=MAIN_CONTENT_3\&context=storylines_faq}{~that
    seems likely, for at least several months.}~There have been
    frightening accounts of people suffering what seems to be a second
    bout of Covid-19. But experts say these patients may have a
    drawn-out course of infection, with the virus taking a slow toll
    weeks to months after initial exposure.~People infected with the
    coronavirus typically
    \href{https://www.nature.com/articles/s41586-020-2456-9}{produce}~immune
    molecules called antibodies, which are
    \href{https://www.nytimes3xbfgragh.onion/2020/05/07/health/coronavirus-antibody-prevalence.html?action=click\&pgtype=Article\&state=default\&region=MAIN_CONTENT_3\&context=storylines_faq}{protective
    proteins made in response to an
    infection}\href{https://www.nytimes3xbfgragh.onion/2020/05/07/health/coronavirus-antibody-prevalence.html?action=click\&pgtype=Article\&state=default\&region=MAIN_CONTENT_3\&context=storylines_faq}{.
    These antibodies may}~last in the body
    \href{https://www.nature.com/articles/s41591-020-0965-6}{only two to
    three months}, which may seem worrisome, but that's~perfectly normal
    after an acute infection subsides, said Dr. Michael Mina, an
    immunologist at Harvard University. It may be possible to get the
    coronavirus again, but it's highly unlikely that it would be
    possible in a short window of time from initial infection or make
    people sicker the second time.
  \end{itemize}
\item ~
  \hypertarget{what-are-my-rights-if-i-am-worried-about-going-back-to-work}{%
  \paragraph{What are my rights if I am worried about going back to
  work?}\label{what-are-my-rights-if-i-am-worried-about-going-back-to-work}}

  \begin{itemize}
  \tightlist
  \item
    Employers have to provide
    \href{https://www.osha.gov/SLTC/covid-19/standards.html}{a safe
    workplace}~with policies that protect everyone equally.
    \href{https://www.nytimes3xbfgragh.onion/article/coronavirus-money-unemployment.html?action=click\&pgtype=Article\&state=default\&region=MAIN_CONTENT_3\&context=storylines_faq}{And
    if one of your co-workers tests positive for the coronavirus, the
    C.D.C.}~has said that
    \href{https://www.cdc.gov/coronavirus/2019-ncov/community/guidance-business-response.html}{employers
    should tell their employees}~-\/- without giving you the sick
    employee's name -\/- that they may have been exposed to the virus.
  \end{itemize}
\end{itemize}

A study from South Korea, where widespread testing has been done, found
that 30 percent of some 2,000 patients who tested positive for the
coronavirus reported experiencing anosmia.

Hendrik Streeck, a German virologist from the University of Bonn who
went from house to house in the country's Heinsberg district to
interview coronavirus patients, has said in interviews that at least
two-thirds of the more than 100 he talked to with mild disease reported
experiencing loss of smell and taste lasting several days.

Another physician
\href{https://www.medrxiv.org/content/10.1101/2020.03.05.20030502v1.full.pdf}{who
studied a cluster of coronavirus patients in Germany} said in an email
that roughly half of the patients had experienced a smell or taste
disorder, and that the sensory loss usually presented after the first
symptoms of respiratory illness, but could be used to distinguish people
who should be tested.

Dr. Clemens Wendtner, a professor of medicine at the Academic Teaching
Hospital of Ludwig-Maximilians University of Munich, said that the
patients regained their ability to smell after a few days or weeks, and
that the loss occurred regardless of how sick they got or whether they
were congested. Using nasal drops or sprays did not help, Dr. Wendtner
said.

\textbf{\emph{{[}}\href{http://on.fb.me/1paTQ1h}{\emph{Like the Science
Times page on Facebook.}}} ****** \emph{\textbar{} Sign up for the}
\textbf{\href{http://nyti.ms/1MbHaRU}{\emph{Science Times
newsletter.}}\emph{{]}}}

Several American patients who have had symptoms consistent with the
coronavirus, but who have not been tested or are still awaiting test
results, described losing their senses of smell and taste, even though
their noses were clear and they were not congested.

Andrew Berry, 30, developed a fever and
\href{https://www.nytimes3xbfgragh.onion/article/coronavirus-body-symptoms.html}{body}
aches about 10 days ago, and then a sore throat and debilitating
headaches. He tested negative for influenza and has not gotten the
result of a coronavirus test taken four days ago, but his physician was
convinced that he had the virus, he said.

Now, Mr. Berry said, he literally cannot smell the coffee.

``Even with a clear nose, I just realized I couldn't smell the food that
I was cooking, and I couldn't taste the food that I was making,'' said
Mr. Berry, a tattoo artist based in Orlando, Fla. He was cooking a
plantain dish with onions and vinegar, yet he could not smell it.

Amy Plattmier, a woman from Brooklyn, was not tested for the coronavirus
during a recent illness, but her husband then became sick and had a
positive test. Ms. Plattmier said she usually had a very sensitive nose,
but now could barely smell anything --- not the bleach she was using to
clean the counters, which usually makes her feel nauseated, or the dog's
accident in the bathroom, which she cleaned up.

Mr. Berry has also lost some weight, because he has not had much of an
appetite. ``Hopefully it's not a prolonged effect,'' he said. ``I can
imagine it changes the quality of life.''

David Kirkpatrick contributed reporting from London.

Advertisement

\protect\hyperlink{after-bottom}{Continue reading the main story}

\hypertarget{site-index}{%
\subsection{Site Index}\label{site-index}}

\hypertarget{site-information-navigation}{%
\subsection{Site Information
Navigation}\label{site-information-navigation}}

\begin{itemize}
\tightlist
\item
  \href{https://help.nytimes3xbfgragh.onion/hc/en-us/articles/115014792127-Copyright-notice}{©~2020~The
  New York Times Company}
\end{itemize}

\begin{itemize}
\tightlist
\item
  \href{https://www.nytco.com/}{NYTCo}
\item
  \href{https://help.nytimes3xbfgragh.onion/hc/en-us/articles/115015385887-Contact-Us}{Contact
  Us}
\item
  \href{https://www.nytco.com/careers/}{Work with us}
\item
  \href{https://nytmediakit.com/}{Advertise}
\item
  \href{http://www.tbrandstudio.com/}{T Brand Studio}
\item
  \href{https://www.nytimes3xbfgragh.onion/privacy/cookie-policy\#how-do-i-manage-trackers}{Your
  Ad Choices}
\item
  \href{https://www.nytimes3xbfgragh.onion/privacy}{Privacy}
\item
  \href{https://help.nytimes3xbfgragh.onion/hc/en-us/articles/115014893428-Terms-of-service}{Terms
  of Service}
\item
  \href{https://help.nytimes3xbfgragh.onion/hc/en-us/articles/115014893968-Terms-of-sale}{Terms
  of Sale}
\item
  \href{https://spiderbites.nytimes3xbfgragh.onion}{Site Map}
\item
  \href{https://help.nytimes3xbfgragh.onion/hc/en-us}{Help}
\item
  \href{https://www.nytimes3xbfgragh.onion/subscription?campaignId=37WXW}{Subscriptions}
\end{itemize}
