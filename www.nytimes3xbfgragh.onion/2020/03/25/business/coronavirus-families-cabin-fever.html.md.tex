Sections

SEARCH

\protect\hyperlink{site-content}{Skip to
content}\protect\hyperlink{site-index}{Skip to site index}

\href{https://www.nytimes3xbfgragh.onion/section/business}{Business}

\href{https://myaccount.nytimes3xbfgragh.onion/auth/login?response_type=cookie\&client_id=vi}{}

\href{https://www.nytimes3xbfgragh.onion/section/todayspaper}{Today's
Paper}

\href{/section/business}{Business}\textbar{}`A Week of Snow Days'? Ha!
Families Deal With Cabin Fever

\url{https://nyti.ms/3bm8Jpp}

\begin{itemize}
\item
\item
\item
\item
\item
\end{itemize}

\href{https://www.nytimes3xbfgragh.onion/spotlight/at-home?action=click\&pgtype=Article\&state=default\&region=TOP_BANNER\&context=at_home_menu}{At
Home}

\begin{itemize}
\tightlist
\item
  \href{https://www.nytimes3xbfgragh.onion/2020/09/07/travel/route-66.html?action=click\&pgtype=Article\&state=default\&region=TOP_BANNER\&context=at_home_menu}{Cruise
  Along: Route 66}
\item
  \href{https://www.nytimes3xbfgragh.onion/2020/09/04/dining/sheet-pan-chicken.html?action=click\&pgtype=Article\&state=default\&region=TOP_BANNER\&context=at_home_menu}{Roast:
  Chicken With Plums}
\item
  \href{https://www.nytimes3xbfgragh.onion/2020/09/04/arts/television/dark-shadows-stream.html?action=click\&pgtype=Article\&state=default\&region=TOP_BANNER\&context=at_home_menu}{Watch:
  Dark Shadows}
\item
  \href{https://www.nytimes3xbfgragh.onion/interactive/2020/at-home/even-more-reporters-editors-diaries-lists-recommendations.html?action=click\&pgtype=Article\&state=default\&region=TOP_BANNER\&context=at_home_menu}{Explore:
  Reporters' Google Docs}
\end{itemize}

Advertisement

\protect\hyperlink{after-top}{Continue reading the main story}

Supported by

\protect\hyperlink{after-sponsor}{Continue reading the main story}

\hypertarget{a-week-of-snow-days-ha-families-deal-with-cabin-fever}{%
\section{`A Week of Snow Days'? Ha! Families Deal With Cabin
Fever}\label{a-week-of-snow-days-ha-families-deal-with-cabin-fever}}

As people become hostages in their own homes, hired clowns and costume
nights may not be enough to maintain sanity.

\includegraphics{https://static01.graylady3jvrrxbe.onion/images/2020/03/25/business/25cabinfever/24cabinfever-articleLarge.jpg?quality=75\&auto=webp\&disable=upscale}

By \href{https://www.nytimes3xbfgragh.onion/by/nellie-bowles}{Nellie
Bowles}

\begin{itemize}
\item
  Published March 25, 2020Updated March 27, 2020
\item
  \begin{itemize}
  \item
  \item
  \item
  \item
  \item
  \end{itemize}
\end{itemize}

Anita Tandon and Sujit Chakravarthy, parents of three young children,
ages 3 months to 7 years old, have taken extreme measures to keep order
in their home during quarantine.

``At 9 o'clock, school's in session and I stop being `Mommy,''' said Ms.
Tandon, who runs a marketing advisory firm in Burlingame, Calif. ``They
have to call me `Teacher Anita.' They can't just goof off like they can
with Mom and Dad.''

There are worksheets, activities, Khan Academy online courses and
writing games. Around 5 p.m., Teacher Anita retires to work. Mr.
Chakravarthy takes over, springing out of his home office ready for P.E.
He goes by Coach Chakravarthy.

``It's Day 3 of God knows how many,'' Ms. Tandon said wearily.

It has been just over a week since
\href{https://www.nytimes3xbfgragh.onion/2020/03/17/us/california-shelter-in-place-coronavirus.html}{Americans
started to be ordered to stay at home} and out of the way of
\href{https://www.nytimes3xbfgragh.onion/news-event/coronavirus}{the
coronavirus pandemic}. For many people, it already feels like an
eternity.

Kids are trying to escape. Careers are falling apart as parents working
from home become de facto kindergarten teachers. Marriages are being
strained. Couples who wanted to break up are stuck together; Craigslist
roommates are suddenly family. And everyone has to stay put with others
24 hours a day, seven days a week, because there is nowhere else,
really, to go.

For many people, it is hard to complain: If they can stay home as a unit
and their work allows them to make a kitchen counter into an office,
they are the lucky ones.

But cabin fever is setting in. Families are going slightly mad --- and
getting mad at one another.

On Twitter, some people cracked jokes about selling their children. Some
were even tired of seeing so much of their pets. Gov. Andrew M. Cuomo of
New York
\href{https://www.wfmz.com/news/cnn/us-national/covid-gov-cuomo-my-dog-is-starting-to-annoy-me/video_eb15e561-282d-5432-9d0d-ebbdd97295e0.html}{said
on Sunday}: ``I live alone. I'm even getting annoyed with the dog, being
in one place.''

The stir craziness is likely to be just beginning. By the end of last
week,
\href{https://www.nytimes3xbfgragh.onion/2020/03/20/us/ny-ca-stay-home-order.html}{at
least one in five Americans was under orders} to shelter at home, with
more states following this week. It's unclear how long these
restrictions will last. Schools might not open again until the fall.

``There's going to be increased misbehavior, defiance, tantrums and
blowing up,'' said Jennifer Johnston-Jones, a child psychologist in Los
Angeles. ``After a natural disaster, you go back to normal. With this,
there's not going to be a back to normal.''

Image

Frances Geller doing homework online in her bedroom in Riverdale,
N.Y.Credit...Andrew White for The New York Times

Image

Disinfecting wipes were given the royal treatment by the Geller
family.Credit...Andrew White for The New York Times

Sabrina Benassaya, a privacy specialist in Menlo Park, Calif., has four
children between the ages of 2 and 10, whose school and day care have
been canceled.

``It's hard. I cannot lie,'' she said. To survive, she had David
Magidson, a clown who performs under the name Boswick, give a birthday
show last week for the kids via FaceTime.

The Benassayas have a house and a backyard. To quarantine in a home like
that is a privilege that many American families do not have, Ms.
Benassaya acknowledged. ``We are so lucky,'' she said.

Family coaches are offering tips to help get through this.

``One of the messages I've been trying to push to parents is there's
only the two of you,'' said Maryellen P. Mullin, a family therapist in
San Francisco. ``There's nowhere to go out, and no one can come in.''

Her schedule has been so full that she is starting to offer a
\href{http://messyparenting.com}{new workshop} for \$20 called ``My Kids
Are Home, I Need Help.''

Escapism seems key. Katie Jacobs Stanton, a mother of three and the
founder of Moxxie Ventures, a start-up investment firm in San Francisco,
dressed as if for a prom one day. Another day, the whole family wore
onesies.

``Last night, we came to dinner and pretended we were someone else in
the family. It was really funny until my son did his impression of me,''
Ms. Stanton said. ``I'm no longer paying for his college education.''

Her friend Aileen Lee, who is also a venture capitalist, has been
posting photos of her husband in different costumes every day. One day
\href{https://twitter.com/aileenlee/status/1241144621297614848}{he
dressed as a mermaid}, with a red wig and shiny sequined skirt.

\hypertarget{working-moms-confront-battles-they-thought-were-over}{%
\subsection{Working moms confront battles they thought were
over}\label{working-moms-confront-battles-they-thought-were-over}}

\includegraphics{https://static01.graylady3jvrrxbe.onion/images/2020/03/25/business/25virus-cabinfeveramerica5/merlin_170906379_2f08782e-f1d5-4a8e-a18c-76430b9a452c-articleLarge.jpg?quality=75\&auto=webp\&disable=upscale}

The burden of handling coronavirus quarantine in many homes was falling
on moms, families said, with much of the new tension in couples caused
by fights over what women thought were battles that had already been
won.

When Lea Geller, a novelist,
\href{https://thisisthecornerwepeein.wordpress.com/}{blogger} and mother
of five in Riverdale, N.Y., first thought about a quarantine, it seemed
it could be fun.

``I thought it would be a week of snow days,'' Ms. Geller said. ``But
now it's lasting forever and ever.''

``To some degree, it feels like we're running a WeWork,'' she said. ``My
husband's running tech support, running round with cables, and I'm just
shoveling food into everyone's mouths and loading and reloading the
dishwasher a million times a day.''

Ms. Geller thought maybe she would have extra time with her husband,
Mike Geller. But the only private time they have had was when they
``literally hid'' from their children in a back room the other day, she
said.

``My new office mates are significantly more high maintenance,'' Mr.
Geller said, referring to the kids. He added that at least the tech
support was now largely sorted and ready for Week 2 and more.

The hardest part is that Ms. Geller's own work has gone on hold. When
her husband's office shut down, she gave him her home work space. Now
she is having trouble thinking creatively in the 30-minute increments
when she can sneak away from the family.

``There's the constant certainty that someone is about to interrupt me
and ask me for food or a stapler,'' Ms. Geller said.

Image

Mr. Geller, working from home now, has displaced his wife, a writer,
from her usual space.Credit...Andrew White for The New York Times

Image

Sidney, left, and Fiona Geller playing catch.~``To some degree, it feels
like we're running a WeWork,'' their mother said.Credit...Andrew White
for The New York Times

Maria Colacurcio, chief executive of \href{https://synd.io/}{Syndio}, a
human resources analytics company in Seattle and the mother of six
children, said the lopsidedness was not a surprise. Even without a
pandemic, domestic labor largely falls on women.

``So now where do you think the extra falls?'' Ms. Colacurcio said.

Leah Wagner-Edelstein, a director of an academic institute at University
of California, Berkeley, and the mother of a 5-year-old and a 3-year-old
who are now home all day, said she and her husband, Jason, had what she
considered an equal arrangement.

``I still manage more or less our whole household, the cooking, most of
the cleaning, the bulk of the home schooling,'' she said. ``Those gender
divisions they just come out immediately.''

Toddlers need constant entertainment and can focus on only one thing for
a few minutes. So Ms. Wagner-Edelstein got some younger cousins to sign
up on a spreadsheet to help with entertaining her children every day.
They can choose hand puppets, a dance party or name-that-color.

Teachers, she said, need to be paid much more.

Despite it all, Ms. Wagner-Edelstein said she was finding that her love
for her husband was deeper. She is being gentler with him, and vice
versa. They are focusing on small joys.

``We think of fun as big vacations,'' she said. ``But now maybe it's
just digging a hole in the front yard and finding what color the soil
is.''

\hypertarget{when-adult-children-come-home}{%
\subsection{When adult children come
home}\label{when-adult-children-come-home}}

Image

Lisa Lurie and her husband have their daughter, Gillian, right,
quarantined in a room of their Pittsburgh home.Credit...Sean Stewart for
The New York Times

Not all cabin fever families are dealing with toddlers. College students
have been sent home, repopulating their parents' empty nests. Other
adult children, sometimes with friends and fiancés in tow, are turning
their parents' kitchens into co-working spaces.

But the reunions, at least initially, are careful. Many young adults
said they were scared they could be taking the virus home to their
parents, who may be more susceptible to the outbreak because
\href{https://www.nytimes3xbfgragh.onion/2020/03/14/health/coronavirus-elderly-protection.html}{older
people are more at risk}.

``All my Stanford friends and I are self-quarantining in our own rooms
away from our families after getting booted off campus,'' said Netta
Wang, 22, a Stanford University senior who returned to her parents'
house in San Mateo, Calif. Her parents leave trays of food at her
bedroom door.

Gillian Lurie, 20, had a great time on a study-abroad semester in
Florence, Italy, but as
\href{https://www.nytimes3xbfgragh.onion/2020/03/21/world/europe/italy-coronavirus-center-lessons.html}{coronavirus
swept through that country}, the program was shut down. Already on
spring break, she traveled through Spain, Germany, Portugal and Ireland.
This month, she came home.

``She managed to have a great time, but she brought home a souvenir,''
her mother, Lisa Lurie, said. ``A little something called the
coronavirus.''

Now Lisa Lurie and her husband, Brian, who run
\href{http://www.cancerbeglammed.com/}{Cancer Be Glammed}, a lifestyle
company that supports women coping with cancer, are quarantining their
daughter in a back room of their Pittsburgh home. They communicate via
FaceTime and drop meals at the door.

``The only thing keeping me sane is online mahjong,'' Lisa Lurie said.

Image

Family conversations in the Lurie house mean FaceTime, since Gillian
returned from a semester in Europe with the coronavirus.Credit...Sean
Stewart for The New York Times

Other parents are setting up rules for their suddenly multigenerational
households.

Haley Walker, 24, lives in Manhattan and works as a senior analyst for a
commercial real estate company. For quarantine, she went back to her
parents' house in Williston, Vt., with her two sisters. Also in tow: one
boyfriend and one fiancé.

When they all got to the four-bedroom house, they were thrilled and
spread out. They set up mobile offices all around, commandeering the
kitchen table and the living room.

Ms. Walker's parents, Adele and Bob, did something that she said had
never happened before: They called an emergency family meeting.

No more co-working and taking calls all day in the kitchen and living
room, the young adults were told. Everyone was assigned a little work
nook in a different part of the house. Also, there would be chores
(vacuuming, dishes, trash, cooking dinner) and time slots for laundry.

``They love having us home and all together,'' Haley Walker said. ``But
I think for my parents it's both good and bad.''

Adele Walker said she was enjoying having everyone close. But ``ask me
how I'm feeling on Day 50 and my answer may be very different,'' she
said.

Advertisement

\protect\hyperlink{after-bottom}{Continue reading the main story}

\hypertarget{site-index}{%
\subsection{Site Index}\label{site-index}}

\hypertarget{site-information-navigation}{%
\subsection{Site Information
Navigation}\label{site-information-navigation}}

\begin{itemize}
\tightlist
\item
  \href{https://help.nytimes3xbfgragh.onion/hc/en-us/articles/115014792127-Copyright-notice}{©~2020~The
  New York Times Company}
\end{itemize}

\begin{itemize}
\tightlist
\item
  \href{https://www.nytco.com/}{NYTCo}
\item
  \href{https://help.nytimes3xbfgragh.onion/hc/en-us/articles/115015385887-Contact-Us}{Contact
  Us}
\item
  \href{https://www.nytco.com/careers/}{Work with us}
\item
  \href{https://nytmediakit.com/}{Advertise}
\item
  \href{http://www.tbrandstudio.com/}{T Brand Studio}
\item
  \href{https://www.nytimes3xbfgragh.onion/privacy/cookie-policy\#how-do-i-manage-trackers}{Your
  Ad Choices}
\item
  \href{https://www.nytimes3xbfgragh.onion/privacy}{Privacy}
\item
  \href{https://help.nytimes3xbfgragh.onion/hc/en-us/articles/115014893428-Terms-of-service}{Terms
  of Service}
\item
  \href{https://help.nytimes3xbfgragh.onion/hc/en-us/articles/115014893968-Terms-of-sale}{Terms
  of Sale}
\item
  \href{https://spiderbites.nytimes3xbfgragh.onion}{Site Map}
\item
  \href{https://help.nytimes3xbfgragh.onion/hc/en-us}{Help}
\item
  \href{https://www.nytimes3xbfgragh.onion/subscription?campaignId=37WXW}{Subscriptions}
\end{itemize}
