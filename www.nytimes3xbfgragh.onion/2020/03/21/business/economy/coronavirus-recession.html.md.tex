Sections

SEARCH

\protect\hyperlink{site-content}{Skip to
content}\protect\hyperlink{site-index}{Skip to site index}

\href{https://www.nytimes3xbfgragh.onion/section/business/economy}{Economy}

\href{https://myaccount.nytimes3xbfgragh.onion/auth/login?response_type=cookie\&client_id=vi}{}

\href{https://www.nytimes3xbfgragh.onion/section/todayspaper}{Today's
Paper}

\href{/section/business/economy}{Economy}\textbar{}Coronavirus Recession
Looms, Its Course `Unrecognizable'

\url{https://nyti.ms/2U7dKfw}

\begin{itemize}
\item
\item
\item
\item
\item
\item
\end{itemize}

\hypertarget{the-coronavirus-outbreak}{%
\subsubsection{\texorpdfstring{\href{https://www.nytimes3xbfgragh.onion/news-event/coronavirus?name=styln-coronavirus-markets\&region=TOP_BANNER\&block=storyline_menu_recirc\&action=click\&pgtype=Article\&impression_id=3e72a070-f4bc-11ea-b75e-3f4305a6ad65\&variant=undefined}{The
Coronavirus
Outbreak}}{The Coronavirus Outbreak}}\label{the-coronavirus-outbreak}}

\begin{itemize}
\tightlist
\item
  live\href{https://www.nytimes3xbfgragh.onion/2020/09/11/world/covid-19-coronavirus.html?name=styln-coronavirus-markets\&region=TOP_BANNER\&block=storyline_menu_recirc\&action=click\&pgtype=Article\&impression_id=3e72a071-f4bc-11ea-b75e-3f4305a6ad65\&variant=undefined}{Latest
  Updates}
\item
  \href{https://www.nytimes3xbfgragh.onion/interactive/2020/us/coronavirus-us-cases.html?name=styln-coronavirus-markets\&region=TOP_BANNER\&block=storyline_menu_recirc\&action=click\&pgtype=Article\&impression_id=3e72c780-f4bc-11ea-b75e-3f4305a6ad65\&variant=undefined}{Maps
  and Cases}
\item
  \href{https://www.nytimes3xbfgragh.onion/interactive/2020/science/coronavirus-vaccine-tracker.html?name=styln-coronavirus-markets\&region=TOP_BANNER\&block=storyline_menu_recirc\&action=click\&pgtype=Article\&impression_id=3e72c781-f4bc-11ea-b75e-3f4305a6ad65\&variant=undefined}{Vaccine
  Tracker}
\item
  \href{https://www.nytimes3xbfgragh.onion/2020/09/10/us/politics/fda-coronavirus-vaccine.html?name=styln-coronavirus-markets\&region=TOP_BANNER\&block=storyline_menu_recirc\&action=click\&pgtype=Article\&impression_id=3e72c782-f4bc-11ea-b75e-3f4305a6ad65\&variant=undefined}{F.D.A.
  Regulators' Self-Defense}
\item
  \href{https://www.nytimes3xbfgragh.onion/2020/09/09/upshot/coronavirus-surprise-test-fees.html?name=styln-coronavirus-markets\&region=TOP_BANNER\&block=storyline_menu_recirc\&action=click\&pgtype=Article\&impression_id=3e72c783-f4bc-11ea-b75e-3f4305a6ad65\&variant=undefined}{Surprise
  Test Fees}
\end{itemize}

Advertisement

\protect\hyperlink{after-top}{Continue reading the main story}

Supported by

\protect\hyperlink{after-sponsor}{Continue reading the main story}

\hypertarget{coronavirus-recession-looms-its-course-unrecognizable}{%
\section{Coronavirus Recession Looms, Its Course
`Unrecognizable'}\label{coronavirus-recession-looms-its-course-unrecognizable}}

The U.S. economic outlook darkens daily, with millions facing
unemployment and businesses in a steep decline.

\includegraphics{https://static01.graylady3jvrrxbe.onion/images/2020/03/19/business/19recession/19recession-articleLarge.jpg?quality=75\&auto=webp\&disable=upscale}

By \href{https://www.nytimes3xbfgragh.onion/by/nelson-d-schwartz}{Nelson
D. Schwartz}

\begin{itemize}
\item
  Published March 21, 2020Updated April 29, 2020
\item
  \begin{itemize}
  \item
  \item
  \item
  \item
  \item
  \item
  \end{itemize}
\end{itemize}

The American
\href{https://www.nytimes3xbfgragh.onion/2020/04/29/business/economy/us-gdp.html}{economy}
is facing a plunge into uncharted waters.

Economists say there is little doubt that the nation is headed into a
\href{https://www.nytimes3xbfgragh.onion/2020/04/01/business/economy/coronavirus-recession.html}{recession}
because of the
\href{https://www.nytimes3xbfgragh.onion/2020/04/01/business/economy/coronavirus-recession.html}{coronavirus}
pandemic, with businesses shutting down and Americans being shut in. But
it is harder to foresee the bottom and how long it will take to climb
back.

Greg Daco, chief U.S. economist at Oxford Economics, says the economy is
assured of a recession --- at least two consecutive quarters of economic
decline --- with output falling 0.4 percent in the first quarter and 12
percent in the second. That would be the biggest quarterly contraction
on record, but Goldman Sachs upped the ante on Friday, saying it
expected a 24 percent drop in the second quarter.

``This is not just a blip,'' Mr. Daco said of the outlook. ``We've never
experienced something like this.''

\includegraphics{https://static01.graylady3jvrrxbe.onion/images/2020/03/20/autossell/COVID19_006/COVID19_006-videoSixteenByNineJumbo1600.jpg}

The abruptness of the descent --- and the near-lockdown of major cities
--- is unheard-of in advanced economies, more akin to wartime privation
than to the downturn that accompanied the financial crisis more than a
decade ago, or even the Great Depression.

``Even during previous recessions,'' noted Ellen Zentner, chief U.S.
economist at Morgan Stanley, ``no one's been told you can't go outside
or you can't gather.''

Smaller companies will be hit harder than large ones because of their
limited access to credit and less cash in the bank. ``There will be a
swath of small businesses that simply won't be able to survive this,''
Ms. Zentner added.

\includegraphics{https://static01.graylady3jvrrxbe.onion/images/2020/03/22/business/22virus-recession/merlin_170753895_bb1f1b3d-36bb-45e4-9838-f55800b0b6dc-articleLarge.jpg?quality=75\&auto=webp\&disable=upscale}

The result is an economy that has gone from full-speed-ahead in January
to a full-on freeze. Economists have had to update their models daily as
the pandemic increasingly throttles work, commerce and travel.

``Economic data in the near future will be not just bad but
unrecognizable,'' Credit Suisse said in a note on Friday.

On Thursday, the Labor Department reported that initial jobless claims
\href{https://www.nytimes3xbfgragh.onion/2020/03/19/business/economy/coronavirus-employers-unemployment.html}{jumped
30 percent} the previous week, to 281,000, the highest level since the
aftermath of a hurricane in 2017. But even that number looks tiny next
to the number of new claims that Goldman Sachs foresees in the next
weekly report: 2.25 million.

\hypertarget{latest-updates-the-coronavirus-outbreak-and-the-economy}{%
\section{\texorpdfstring{\href{https://www.nytimes3xbfgragh.onion/live/2020/09/11/business/stock-market-today-coronavirus?action=click\&pgtype=Article\&state=default\&region=MAIN_CONTENT_1\&context=storylines_live_updates}{Latest
Updates: The Coronavirus Outbreak and the
Economy}}{Latest Updates: The Coronavirus Outbreak and the Economy}}\label{latest-updates-the-coronavirus-outbreak-and-the-economy}}

\href{https://www.nytimes3xbfgragh.onion/live/2020/09/11/business/stock-market-today-coronavirus?action=click\&pgtype=Article\&state=default\&region=MAIN_CONTENT_1\&context=storylines_live_updates\#the-nyse-may-move-its-data-center-out-of-new-jersey-in-response-to-a-proposed-tax}{9h
ago}

\href{https://www.nytimes3xbfgragh.onion/live/2020/09/11/business/stock-market-today-coronavirus?action=click\&pgtype=Article\&state=default\&region=MAIN_CONTENT_1\&context=storylines_live_updates\#the-nyse-may-move-its-data-center-out-of-new-jersey-in-response-to-a-proposed-tax}{The
N.Y.S.E. may move its data center out of New Jersey in response to a
proposed tax.}

\href{https://www.nytimes3xbfgragh.onion/live/2020/09/11/business/stock-market-today-coronavirus?action=click\&pgtype=Article\&state=default\&region=MAIN_CONTENT_1\&context=storylines_live_updates\#the-federal-budget-deficit-hit-3-trillion-as-of-august}{12h
ago}

\href{https://www.nytimes3xbfgragh.onion/live/2020/09/11/business/stock-market-today-coronavirus?action=click\&pgtype=Article\&state=default\&region=MAIN_CONTENT_1\&context=storylines_live_updates\#the-federal-budget-deficit-hit-3-trillion-as-of-august}{The
federal budget deficit hit \$3 trillion as of August.}

\href{https://www.nytimes3xbfgragh.onion/live/2020/09/11/business/stock-market-today-coronavirus?action=click\&pgtype=Article\&state=default\&region=MAIN_CONTENT_1\&context=storylines_live_updates\#warner-bros-pushes-the-release-of-wonder-woman-1984-to-christmas}{12h
ago}

\href{https://www.nytimes3xbfgragh.onion/live/2020/09/11/business/stock-market-today-coronavirus?action=click\&pgtype=Article\&state=default\&region=MAIN_CONTENT_1\&context=storylines_live_updates\#warner-bros-pushes-the-release-of-wonder-woman-1984-to-christmas}{Warner
Bros. pushes the release of `Wonder Woman 1984' to Christmas.}

\href{https://www.nytimes3xbfgragh.onion/live/2020/09/11/business/stock-market-today-coronavirus?action=click\&pgtype=Article\&state=default\&region=MAIN_CONTENT_1\&context=storylines_live_updates}{See
more updates}

More live coverage:
\href{https://www.nytimes3xbfgragh.onion/2020/09/11/world/covid-19-coronavirus.html?action=click\&pgtype=Article\&state=default\&region=MAIN_CONTENT_1\&context=storylines_live_updates}{Global}

Mr. Daco says the unemployment rate could hit 10 percent in April, a
level unseen since the nadir of the last recession, with the possibility
of even higher jobless rates in the following months. Treasury Secretary
Steven Mnuchin reportedly
\href{https://www.nytimes3xbfgragh.onion/2020/03/17/us/politics/stimulus-package.html}{pointed
to 20 percent unemployment} in the absence of effective intervention,
though an aide later said the number was not a forecast. And this after
the labor market touched
\href{https://www.nytimes3xbfgragh.onion/2020/03/06/business/economy/jobs-report.html}{record
low unemployment} for the last several months.

If Mr. Daco's 10 percent figure is borne out, 16.5 million people would
be out of work, compared with 5.8 million in February.

One reason that things could get so bad so quickly is that economic
weakness feeds on itself, with demand falling as more businesses shut
their doors and layoffs spread. To make matters worse, a
\href{https://www.nytimes3xbfgragh.onion/2020/03/09/business/energy-environment/oil-opec-saudi-russia.html}{dispute
between Russia and Saudi Arabia} has resulted in a flood of crude oil,
depressing prices and hurting the domestic energy industry.

The pain is so severe because the economy is dominated by services, with
consumers powering overall demand, a shift from previous generations,
when the production of goods counted for a greater share of output.
About three-quarters of economic activity derives from consumer
spending, and half of that is at risk, Mr. Daco said.

Image

Before Junior's Restaurant in Brooklyn closed, a worker picked up a
paycheck.Credit...Mark Lennihan/Associated Press

Image

A closed restaurant in Manhattan.Credit...Spencer Platt/Getty Images

The
\href{https://www.nytimes3xbfgragh.onion/2020/03/19/us/politics/1000-checks-coronavirus-stimulus.html}{proposed
federal stimulus} --- which in a Senate Republican version would include
checks of up to \$1,200 for taxpayers --- would be helpful, economists
say, but it would probably only blunt the pandemic's impact, not stave
it off.

``A \$1,000 check, or even a \$2,000 one, won't pay the rent in New York
City, and I suspect it would run out pretty quickly in most parts of the
country,'' said Beth Ann Bovino, chief U.S. economist at S\&P Global.
``It's nice and it's needed, but it's just a Band-Aid.''

The credit-rating agency Moody's found that lodging, restaurants and
airlines would be among the most affected industries, with sectors like
health care, pharmaceuticals, mining and chemicals taking more modest
hits. Telecommunications, software and the steel industry would be among
the least affected.

``This will probably be the world's first recession that starts in the
service sector,'' said Gabriel Mathy, an assistant professor at American
University whose specialty is economic history. ``We can see employment
falling much faster than G.D.P. The spike in unemployment claims could
be eye-popping.''

Historically, recessions began in goods-producing areas of the economy,
according to Mr. Mathy. Some manufacturers build up inventories that can
be sold when conditions improve. But at restaurants and barbershops,
things have ground to a halt without warning, and that business is lost
forever.

Even if the pessimists are correct in their estimates so far, the
\href{https://www.nytimes3xbfgragh.onion/2020/04/13/business/coronavirus-economy.html}{coronavirus}
recession would not approach the devastation of the Great Depression.
From 1929-33, the economy shrank by one-third, unemployment jumped to 25
percent and the stock market fell 80 percent.

Image

Goldman Sachs says the next weekly total of initial jobless claims could
be 2.25 million.Credit...Hiroko Masuike/The New York Times

In any case, the ranks of the jobless will multiply in the coming weeks,
as the rise in jobless claims indicates. And with officials in
California, New York and a growing number of other places telling people
to stay inside, the economic toll could become worse.

``My concern isn't United Airlines or even small and medium-sized
corporations that issue bonds,'' said Michael Gapen, chief U.S.
economist at Barclays. ``It's the restaurant down the street --- they
are the ones most at risk.''

In New York, restaurants like Mexicue are hanging on by serving food for
takeout or delivery, albeit with fewer workers.

During a typical lunch shift at Mexicue's location on Fifth Avenue in
Manhattan, eight employees would work in the kitchen with four or five
servers and bartenders in the front of the house. Now the restaurant
operates with half its kitchen staff and no one out front. The rest of
the employees were laid off.

``We're really not sure whether the model will work,'' said one of
Mexicue's owners, Thomas Kelly. ``This is the hardest thing I've been
through in 10 years I've been in this business.''

At one point, it seemed as if manufacturers might be shielded, but the
pandemic is now forcing factories to halt operations, too. The nation's
largest automakers --- General Motors, Ford Motor and Fiat Chrysler ---
have
\href{https://www.nytimes3xbfgragh.onion/2020/03/18/business/economy/gm-ford-fiatchrysler-factories-virus.html}{idled
their plants} as a health precaution, even ahead of an inevitable
decline in demand.

Smaller manufacturers face tough choices and bleak prospects.

At MaineSole, a shoe manufacturer in Dexter, Maine, business is down by
a third in the last couple of weeks, said Kevin Cain, the company's
owner.

``We usually have a backlog, but we don't have that for April,'' he
said. ``In March, we'll probably be break-even. It doesn't look good.''

Mr. Cain said he was nervous because nine of his 10 workers are 60 or
older. ``If one of these guys gets sick, we'll shut down,'' he said. If
the company were forced to close, he'd be unable to pay hourly salaries,
he added.

``I wish I could, but I don't have the money in the bank for that,'' Mr.
Cain said. ``If I close, they'd have to apply for unemployment.''

Over the long run, the economic damage will be determined not just by
what happens on the medical front but also by the level of consumer
confidence.

Image

A virtually deserted shopping area at the Hudson Yards complex in
Manhattan.Credit...Ashley Gilbertson for The New York Times

Image

O.D. Madison Jr. waited for customers at Pall Mall Barbers in
Rockefeller Plaza.Credit...Ashley Gilbertson for The New York Times

Warwick McKibbin, an economist at the Australian National University,
has been modeling the impact of pandemics since 2003, when the World
Health Organization asked for an assessment of the fallout from SARS, a
coronavirus-caused disease that emerged in China at the end of 2002 and
killed 774 people.

At the request of the Brookings Institution in Washington this month, he
and a colleague, Roshen Fernando,
\href{https://www.brookings.edu/research/the-global-macroeconomic-impacts-of-covid-19-seven-scenarios/}{outlined
scenarios} for the global macroeconomic effect of the new coronavirus,
depending on how widely it spread and how many people it killed.

Even under a relatively modest set of assumptions in Mr. McKibbin and
Mr. Fernando's macroeconomic model, a one-year epidemic would kill
236,000 people in the United States and reduce the country's
\href{https://www.nytimes3xbfgragh.onion/2020/04/29/business/economy/us-gdp.html}{gross
domestic product} by 2 percent, or \$420 billion.

But if the virus spreads more broadly and has a higher mortality rate,
the economic effect would be proportionately greater. A one-year
epidemic that took just over one million lives, which is consistent with
\href{https://www.nytimes3xbfgragh.onion/2020/03/13/us/coronavirus-deaths-estimate.html}{recent
projections} based on scenarios from the Centers for Disease Control and
Prevention, would reduce the nation's G.D.P. in 2020 by \$1.8 trillion
--- 8.4 percent.

Even if the disease does turn out to be a one-year crisis, Mr. McKibbin
has little faith in a robust economic rebound once it has passed. In his
view, the disease has provided an opportunity to reassess a vast
overvaluation of financial assets.

``People may now realize the extent of their overvaluation, and this
could lead to persistent economic consequences,'' he said. ``This is a
day of reckoning.''

And what if the outbreak doesn't run its course in 2020? Were it to
become a recurring threat in the United States, like a supercharged flu,
killing 236,000 a year, the economists estimated that the virus would
trim annual output by 1.5 percent into the future.

Ultimately, the severity of the economy's slowdown depends on the length
and seriousness of the pandemic. But Torsten Slok, chief economist at
Deutsche Bank Securities, says consumers will continue to be cautious
even after authorities signal the all clear.

A strong rebound --- what economists call a V-shaped recovery, as
opposed to a U-shaped one with an extended low --- would require a
profound resurgence in confidence. But few see that on the horizon.

Image

Sixth Avenue and 53rd Street in New York City.~New York has begun to
encourage residents to stay inside and close the state's nonessential
businesses.Credit...Ashley Gilbertson for The New York Times

``There is a risk that the psychology has changed,'' Mr. Slok said.
``People will be very reluctant to do a lot of travel and spending and
may want to save for another day. There will be more caution.''

Reporting was contributed by Eduardo Porter, Patricia Cohen, David
Yaffe-Bellany and Ben Casselman.

Advertisement

\protect\hyperlink{after-bottom}{Continue reading the main story}

\hypertarget{site-index}{%
\subsection{Site Index}\label{site-index}}

\hypertarget{site-information-navigation}{%
\subsection{Site Information
Navigation}\label{site-information-navigation}}

\begin{itemize}
\tightlist
\item
  \href{https://help.nytimes3xbfgragh.onion/hc/en-us/articles/115014792127-Copyright-notice}{©~2020~The
  New York Times Company}
\end{itemize}

\begin{itemize}
\tightlist
\item
  \href{https://www.nytco.com/}{NYTCo}
\item
  \href{https://help.nytimes3xbfgragh.onion/hc/en-us/articles/115015385887-Contact-Us}{Contact
  Us}
\item
  \href{https://www.nytco.com/careers/}{Work with us}
\item
  \href{https://nytmediakit.com/}{Advertise}
\item
  \href{http://www.tbrandstudio.com/}{T Brand Studio}
\item
  \href{https://www.nytimes3xbfgragh.onion/privacy/cookie-policy\#how-do-i-manage-trackers}{Your
  Ad Choices}
\item
  \href{https://www.nytimes3xbfgragh.onion/privacy}{Privacy}
\item
  \href{https://help.nytimes3xbfgragh.onion/hc/en-us/articles/115014893428-Terms-of-service}{Terms
  of Service}
\item
  \href{https://help.nytimes3xbfgragh.onion/hc/en-us/articles/115014893968-Terms-of-sale}{Terms
  of Sale}
\item
  \href{https://spiderbites.nytimes3xbfgragh.onion}{Site Map}
\item
  \href{https://help.nytimes3xbfgragh.onion/hc/en-us}{Help}
\item
  \href{https://www.nytimes3xbfgragh.onion/subscription?campaignId=37WXW}{Subscriptions}
\end{itemize}
