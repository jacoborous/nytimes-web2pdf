Sections

SEARCH

\protect\hyperlink{site-content}{Skip to
content}\protect\hyperlink{site-index}{Skip to site index}

\href{https://myaccount.nytimes3xbfgragh.onion/auth/login?response_type=cookie\&client_id=vi}{}

\href{https://www.nytimes3xbfgragh.onion/section/todayspaper}{Today's
Paper}

\href{/section/opinion}{Opinion}\textbar{}Trump, His Eye on the Border,
Overlooked the Coronavirus Threat

\url{https://nyti.ms/39xUA85}

\begin{itemize}
\item
\item
\item
\item
\item
\end{itemize}

Advertisement

\protect\hyperlink{after-top}{Continue reading the main story}

\href{/section/opinion}{Opinion}

Supported by

\protect\hyperlink{after-sponsor}{Continue reading the main story}

\hypertarget{trump-his-eye-on-the-border-overlooked-the-coronavirus-threat}{%
\section{Trump, His Eye on the Border, Overlooked the Coronavirus
Threat}\label{trump-his-eye-on-the-border-overlooked-the-coronavirus-threat}}

His insistence that the danger was overseas and could be kept out led
officials to downplay the disease's spread and the need for tests.

By \href{https://www.cgdev.org/expert/jeremy-konyndyk}{Jeremy Konyndyk}

Mr. Konyndyk is a senior policy fellow at the Center for Global
Development.

\begin{itemize}
\item
  March 7, 2020
\item
  \begin{itemize}
  \item
  \item
  \item
  \item
  \item
  \end{itemize}
\end{itemize}

\includegraphics{https://static01.graylady3jvrrxbe.onion/images/2020/03/09/opinion/09konyndykWeb/merlin_169965075_3b527d27-46f2-4c17-a08c-bc27ff4bb0f6-articleLarge.jpg?quality=75\&auto=webp\&disable=upscale}

On Jan. 20, the federal Centers for Disease Control and Prevention
\href{https://www.nytimes3xbfgragh.onion/2020/01/21/health/cdc-coronavirus.html}{found
for the first time} that the Covid-19 coronavirus had infected an
American, a man from Snohomish County, Wash., who had returned from
China five days earlier.

``We have it totally under control,'' President Trump
\href{https://www.politico.com/news/2020/01/22/trump-chinese-coronavirus-totally-under-control-102054}{said}
the next day. ``It's one person coming in from China. It's going to be
just fine.''

On Feb. 2, two days after the administration announced that it would
restrict travel from China in response to the viral threat,
\href{https://www.nytimes3xbfgragh.onion/2020/02/02/us/politics/trump-super-bowl-interview-coronavirus.html}{the
president said}, ``We pretty much shut it down coming in from China.''

Researchers
\href{https://www.nytimes3xbfgragh.onion/2020/03/01/health/coronavirus-washington-spread.html}{now
believe} the outbreak that has killed 10 people in Washington may link
back to the first patient in the state, suggesting the disease was
spreading that whole time. Meanwhile, cases of unclear origin are now
being identified all over the country.

Yet Mr. Trump has continued to claim that ``we have it so well under
control'' and that his approach is succeeding
``\href{https://twitter.com/realDonaldTrump/status/1235604572850343937}{because
of quick action on closing our borders},'' and told reporters on Friday
``\href{https://www.whitehouse.gov/briefings-statements/remarks-president-trump-tour-centers-disease-control-prevention-atlanta-ga/}{we're
doing a really good job in this country at keeping it down\ldots{}a
tremendous job}.''

It's no coincidence that the coronavirus has broken out across the
country as the president has continued to brag about keeping the disease
outside America's border. Rather than vigorously preparing for a
pandemic, federal
officials\href{https://www.nytimes3xbfgragh.onion/2020/03/07/us/politics/trump-coronavirus.html?action=click\&module=Top\%20Stories\&pgtype=Homepage}{responded
in a way that suited the narrative Mr. Trump preferred}, focusing most
of their attention on travel restrictions, passenger screening and
quarantine. Pretending we could wall out the virus not only gave the
public a false sense of security, it also left the United States unready
for the threat it now faces.

Robust overseas containment was a defensible first step --- travel
restrictions can delay the arrival and spread of an outbreak by a few
weeks. Paired with the draconian measures imposed by China, this might
have extended the window for strengthening domestic preparedness.

But buying time only matters if it is linked to a clear plan for
enhancing readiness and a reliable surveillance strategy to signal
whether containment is working. Neither of those things happened.

American hospital capacity is lean. The
\href{http://www.centerforhealthsecurity.org/cbn/2020/cbnreport-02272020.html}{46,500
beds in intensive care} in the United States are mostly occupied.
Covid-19, if uncontrolled, might lead to up to 1.9 million I.C.U.
admissions, according to
\href{https://www.businessinsider.com/presentation-us-hospitals-preparing-for-millions-of-hospitalizations-2020-3}{projections}
presented to the American Hospital Association. With sufficient notice,
hospitals could have begun reallocating space and resources to expand
intensive care, as well as establishing pandemic preparedness
committees, reinforcing infection prevention and determining how to
ethically allocate finite treatment resources if overwhelmed.

But hospital executives need a clear signal from federal authorities to
trigger these costly and burdensome measures. And there was no such
signal. Instead, federal authorities spoke of a low risk to the United
States. Rather than urging specific actions, they passively suggested
that hospitals ``review'' their crisis plans.

State and local health departments, underfunded and not set up to handle
pandemic threats on their own, became more involved with traveler
screening and quarantine than with preparing hospitals and high-risk
facilities like nursing homes.

In keeping with the narrative that the virus could be walled off, the
administration decreed that only those who had traveled from China, been
in contact with someone who had traveled to China or had been exposed to
a lab-confirmed Covid-19 case would be tested. This left officials blind
to the domestic spread that was already occurring. It was not until
Wednesday that the C.D.C. finally said that anyone showing symptoms of
the disease could be tested (but implementation of this
\href{https://thehill.com/changing-america/well-being/prevention-cures/486198-pence-currently-not-enough-coronavirus-tests-to}{remains
slow}).

Compounding the problem was a decision by the C.D.C. to develop its own
test standard, rather than rely on the kits approved by the World Health
Organization and distributed to over 50 countries. When the C.D.C. test
kits proved flawed, expansion of testing across the country was delayed
for weeks. Without C.D.C. test kits, and with no other testing option
approved by the Food and Drug Administration for domestic use, hospitals
were forced to mail samples to the C.D.C. and wait days for results.
Many hospitals had to discharge suspected cases back into the community
when they did not fit the narrow testing criteria.

Neither impediment should have been difficult to fix. But rather than
borrow approaches used elsewhere in the world --- like drive-through
testing in South Korea, or the W.H.O.-approved kits --- the
administration waited for the C.D.C. to resolve the flaws in its own
kits. It has been almost two months since the W.H.O. published testing
guidance, and the United States remains unable to test at scale.

This lack of urgency on testing is defensible only if one accepts the
administration's political narrative: The disease comes from China, so
if we cut off travel from China, surveil for cases linked to China, and
see none, we must be succeeding. This became the self-sustaining logic
at the heart of the narrative.

It is stunning that no one in the administration's leadership team
recognized the vulnerability of this approach.

Good crisis management must constantly re-examine its assumptions.
During the Obama administration, I coordinated the Agency for
International Development's West African operations to end the 2014
Ebola outbreak. We constantly thought about potential weaknesses in our
strategy. The White House's ``Ebola czar,'' Ron Klain, was notorious for
pressing agencies to review potential weaknesses in strategy,
demonstrate that they were thinking around corners and test underlying
presumptions. My team once spent a weekend gaming out what would happen
if commercial air travel to West Africa was shut down. We did not expect
this to happen, but if it had it would have torpedoed our entire
strategy by keeping us from moving Ebola responders in and out of the
region.

The Trump administration failed to apply another lesson of the Ebola
fight: Overseas containment and domestic readiness go hand in hand. The
two cases of Ebola transmission in Dallas in October 2014 revealed that
hospitals were not always ready to manage new infectious threats. After
those incidents, the Obama administration began investing in health
systems readiness and created a new team on the National Security
Council to coordinate readiness for outbreaks abroad and at home. Yet,
in an astonishingly shortsighted move, John Bolton
\href{https://www.washingtonpost.com/news/to-your-health/wp/2018/05/10/top-white-house-official-in-charge-of-pandemic-response-exits-abruptly/}{dissolved
that directorate} upon his arrival as national security adviser. The
White House followed this up by allowing post-Ebola investments in
\href{https://www.washingtonpost.com/opinions/a-program-protecting-us-from-deadly-pandemic-is-about-to-expire/2019/12/27/7c216c26-2280-11ea-bed5-880264cc91a9_story.html}{American}
and
\href{https://www.washingtonpost.com/news/to-your-health/wp/2018/02/01/cdc-to-cut-by-80-percent-efforts-to-prevent-global-disease-outbreak/}{global}
outbreak readiness to lapse after their Obama-era funds began expiring.

We're seeing the results of these cascading mistakes. In California,
\href{https://www.cnn.com/2020/02/29/health/uc-davis-health-care-workers-self-quarantine/index.html}{124
health care staff members} went into self-quarantine, unable to do their
jobs, after exposure to a single person who had gone untested because of
C.D.C. guidelines. A transmission cluster in a Kirkland, Wash., nursing
home exposed numerous staff members and residents and forced a quarter
of the town's firefighters into isolation. Nurses in California have
\href{https://www.nytimes3xbfgragh.onion/2020/03/05/us/coronavirus-nurses.html}{bemoaned
the lack of support}and advance planning, which has put them at risk.
Twitter has been full of complaints about people with Covid-19-like
symptoms unable to access testing.

What's most infuriating is that these risks were widely predicted by
experts outside government. To give but one example, two people who had
served in the Trump administration --- the former Food and Drug
Administration commissioner Scott Gottlieb and a former National
Security Council official, Luciana Borio ---
\href{https://www.wsj.com/articles/act-now-to-prevent-an-american-epidemic-11580255335}{wrote}
in The Wall Street Journal on Jan. 28 that the United States should
immediately scale up testing and get the hospital system ready.

The strategic miscues, the surveillance failures and the lack of
attention to domestic readiness all flow from the same source: a White
House strategy driven more by a political narrative than public health
expertise.

Many officials have a hand in this mess, but the president is the
crucial variable. Errors happen in any crisis. But when a president
insists on claiming success irrespective of reality, it becomes much
harder for those under him to acknowledge and correct mistakes. When he
shows more interest in calming markets than in protecting Americans, he
makes it very hard for senior public health officials to take aggressive
action. (Mr. Trump reportedly became furious after a senior C.D.C.
official's prescient comments on looming risks sent markets tumbling.)

As Dr. Anthony Fauci of the National Institutes of Health told Politico
in
\href{https://www.politico.com/news/2020/03/03/anthony-fauci-trump-coronavirus-crisis-118961}{an
interview} last week, there is always a strong pull ``to tell the
president what you think he wants to hear,'' even as officials have to
``walk the fine balance of making sure you continue to tell the truth.''
These things should not be mutually exclusive in a public health
emergency.

The leadership changes announced by the president last week --- shifting
control to Vice President Mike Pence and bringing in a seasoned health
official like Ambassador Deborah Birx of the AIDS preparedness program
Pepfar --- are a positive step toward rebuilding White House capacities
that were dissolved during Mr. Bolton's tenure.

But simply reshuffling the leadership team will not get this response
back on track.

It is time to reckon with reality. The disease is spreading in the
homeland. Vaccines will not be available any time soon. The health
system is nowhere near ready, and our federal response has lost valuable
time that it cannot get back.

Acknowledging and addressing these hard facts is a vital precondition to
rebuilding a strategy that can realistically succeed.

Jeremy Konyndyk is a senior policy fellow at the Center for Global
Development. From 2013 to 2017 he served as director of the Agency for
International Development's Office of Foreign Disaster Assistance.

\begin{center}\rule{0.5\linewidth}{\linethickness}\end{center}

\emph{The Times is committed to publishing}
\href{https://www.nytimes3xbfgragh.onion/2019/01/31/opinion/letters/letters-to-editor-new-york-times-women.html}{\emph{a
diversity of letters}} \emph{to the editor. We'd like to hear what you
think about this or any of our articles. Here are some}
\href{https://help.nytimes3xbfgragh.onion/hc/en-us/articles/115014925288-How-to-submit-a-letter-to-the-editor}{\emph{tips}}\emph{.
And here's our email:}
\href{mailto:letters@NYTimes.com}{\emph{letters@NYTimes.com}}\emph{.}

\emph{Follow The New York Times Opinion section on}
\href{https://www.facebookcorewwwi.onion/nytopinion}{\emph{Facebook}}\emph{,}
\href{http://twitter.com/NYTOpinion}{\emph{Twitter (@NYTopinion)}}
\emph{and}
\href{https://www.instagram.com/nytopinion/}{\emph{Instagram}}\emph{.}

Advertisement

\protect\hyperlink{after-bottom}{Continue reading the main story}

\hypertarget{site-index}{%
\subsection{Site Index}\label{site-index}}

\hypertarget{site-information-navigation}{%
\subsection{Site Information
Navigation}\label{site-information-navigation}}

\begin{itemize}
\tightlist
\item
  \href{https://help.nytimes3xbfgragh.onion/hc/en-us/articles/115014792127-Copyright-notice}{©~2020~The
  New York Times Company}
\end{itemize}

\begin{itemize}
\tightlist
\item
  \href{https://www.nytco.com/}{NYTCo}
\item
  \href{https://help.nytimes3xbfgragh.onion/hc/en-us/articles/115015385887-Contact-Us}{Contact
  Us}
\item
  \href{https://www.nytco.com/careers/}{Work with us}
\item
  \href{https://nytmediakit.com/}{Advertise}
\item
  \href{http://www.tbrandstudio.com/}{T Brand Studio}
\item
  \href{https://www.nytimes3xbfgragh.onion/privacy/cookie-policy\#how-do-i-manage-trackers}{Your
  Ad Choices}
\item
  \href{https://www.nytimes3xbfgragh.onion/privacy}{Privacy}
\item
  \href{https://help.nytimes3xbfgragh.onion/hc/en-us/articles/115014893428-Terms-of-service}{Terms
  of Service}
\item
  \href{https://help.nytimes3xbfgragh.onion/hc/en-us/articles/115014893968-Terms-of-sale}{Terms
  of Sale}
\item
  \href{https://spiderbites.nytimes3xbfgragh.onion}{Site Map}
\item
  \href{https://help.nytimes3xbfgragh.onion/hc/en-us}{Help}
\item
  \href{https://www.nytimes3xbfgragh.onion/subscription?campaignId=37WXW}{Subscriptions}
\end{itemize}
