Sections

SEARCH

\protect\hyperlink{site-content}{Skip to
content}\protect\hyperlink{site-index}{Skip to site index}

\href{https://myaccount.nytimes3xbfgragh.onion/auth/login?response_type=cookie\&client_id=vi}{}

\href{https://www.nytimes3xbfgragh.onion/section/todayspaper}{Today's
Paper}

\href{/section/opinion}{Opinion}\textbar{}We Don't Need to Close Schools
to Fight the Coronavirus

\url{https://nyti.ms/3aIXS8w}

\begin{itemize}
\item
\item
\item
\item
\item
\end{itemize}

Advertisement

\protect\hyperlink{after-top}{Continue reading the main story}

\href{/section/opinion}{Opinion}

Supported by

\protect\hyperlink{after-sponsor}{Continue reading the main story}

\hypertarget{we-dont-need-to-close-schools-to-fight-the-coronavirus}{%
\section{We Don't Need to Close Schools to Fight the
Coronavirus}\label{we-dont-need-to-close-schools-to-fight-the-coronavirus}}

Shutdowns could likely do more harm than good, since there's little
evidence that children are a major source of the spread.

By
\href{http://www.centerforhealthsecurity.org/our-people/nuzzo/}{Jennifer
Nuzzo}

Dr. Nuzzo is a senior scholar at the Johns Hopkins Center for Health
Security.

\begin{itemize}
\item
  March 10, 2020
\item
  \begin{itemize}
  \item
  \item
  \item
  \item
  \item
  \end{itemize}
\end{itemize}

\includegraphics{https://static01.graylady3jvrrxbe.onion/images/2020/03/10/opinion/10Nuzzo2/10Nuzzo2-articleLarge.jpg?quality=75\&auto=webp\&disable=upscale}

Facing an accelerating spread of Covid-19, Italy and Japan have closed
schools to impede the epidemic. Some communities in the United States
have done so too, agreeing to significantly disrupt people's lives on
the theory that it will prevent deaths and serious illness.

But there is no clear evidence that such measures will slow this
outbreak.

Most of what we know about the impact of school closings on disease
transmission relates to influenza, to which children can be particularly
vulnerable, sometimes dying or becoming seriously ill from it.

Children are important drivers of influenza infections because they have
more interactions with people than do most adults and also give off more
of the virus. Closing schools, it is assumed, reduces the number of
contacts and thus the rate of transmission.

During the 2009 H1N1 influenza pandemic, schools across the country were
closed. A C.D.C.
\href{https://www.cdc.gov/mmwr/preview/mmwrhtml/mm5935a2.htm?s_cid=mm5935a2_w}{study}
showed that parents largely supported these measures, but other studies
found that children frequently got together outside the home or visited
public sites, despite official recommendations not to do so.
Fortunately, schools reopened in less than three days in most cases
because data showed the flu strain wasn't as severe as had been feared.

Still, some evidence suggests that these measures didn't reduce the
number of infections and only slowed the spread --- although that could
help reduce burdens on health systems.

That's influenza, though. Covid-19 is different.

There have been very few reports of children contracting Covid-19. It's
not clear why. It's possible that children do get infected, but so
mildly that it is not noticed or tested.

If children don't experience severe illness from or contribute to the
spread of Covid-19 --- and so far we have found no clear evidence that
they do --- it's likely that school closings will have little effect on
its spread.

Not all affected countries have closed schools. Singapore, which has
been heralded for its response to Covid-19, decided that closing schools
would do more harm than good. Political leaders and health officials
there have addressed concerns about Covid-19 through clear, consistent
and transparent communications about their response to the virus.

If schools remain open, officials could enact measures to limit any
potential spread among children and staff. All students could be checked
daily for fever, a possible sign of Covid-19 infection. Even more
attention should be given to hand washing and reminding children not to
touch their faces. Children should be taught to sneeze into their
sleeves. Schools can consider changing seating arrangements to keep
children six feet apart. As the weather warms, lessons can be taken
outside, if possible.

Nonetheless, government officials may feel pressure to close schools.
For true effectiveness, schools need to close before even 1 percent of
the population is infected and they need to stay closed until the
epidemic is over, which could mean months. Children couldn't gather in
other settings, which would be very difficult to enforce.

If schools close, child care programs will likely close too and working
parents may have to stay home to watch their children. Health care and
critical infrastructure workers would not be able to do their jobs for
the same reason. Those parents may not be paid, which would be a
tremendous hardship. States would have to consider expanding
unemployment benefits and help employers to allow workers to stay home
if needed.

Communities would need to feed and educate children while they are out
of school. Closing schools can interrupt social services like programs
that provide lunches to more 30 million children and breakfast to 11
million. For some children, including homeless youth, schools can be the
safest place and denying these children access may deny them much needed
support, even something as basic as a place to wash their clothes.

Children will need to continue learning. Interruptions in education can
profoundly harm child development and make it harder to reduce the
achievement gap between high- and low-income families. Schools may
consider online education as an alternative but need to ensure that all
families have access to the technologies required for these approaches.

If schools close, knowing when to reopen them would be difficult. To
have any public health impact, school closings would have to be
maintained for the duration of the epidemic.

State and local governments will have to clearly explain the reasons for
closing schools and how they would decide to reopen them so parents and
employers can plan how to manage daily routines.

Above all, officials need to be honest about what is known and what
isn't about the impact of these measures.

Though there may be an inclination to present school closings as a
well-established tool to protect public health, their full impact is
simply unknown.

Downplaying the disruption these measures may cause or overstating their
benefits can erode public confidence in government at a time when it is
needed the most.

Jennifer Nuzzo is a senior scholar at the Johns Hopkins Center for
Health Security and an associate professor at the university's school of
public health.

\emph{The Times is committed to publishing}
\href{https://www.nytimes3xbfgragh.onion/2019/01/31/opinion/letters/letters-to-editor-new-york-times-women.html}{\emph{a
diversity of letters}} \emph{to the editor. We'd like to hear what you
think about this or any of our articles. Here are some}
\href{https://help.nytimes3xbfgragh.onion/hc/en-us/articles/115014925288-How-to-submit-a-letter-to-the-editor}{\emph{tips}}\emph{.
And here's our email:}
\href{mailto:letters@NYTimes.com}{\emph{letters@NYTimes.com}}\emph{.}

\emph{Follow The New York Times Opinion section on}
\href{https://www.facebookcorewwwi.onion/nytopinion}{\emph{Facebook}}\emph{,}
\href{http://twitter.com/NYTOpinion}{\emph{Twitter (@NYTopinion)}}
\emph{and}
\href{https://www.instagram.com/nytopinion/}{\emph{Instagram}}\emph{.}

Advertisement

\protect\hyperlink{after-bottom}{Continue reading the main story}

\hypertarget{site-index}{%
\subsection{Site Index}\label{site-index}}

\hypertarget{site-information-navigation}{%
\subsection{Site Information
Navigation}\label{site-information-navigation}}

\begin{itemize}
\tightlist
\item
  \href{https://help.nytimes3xbfgragh.onion/hc/en-us/articles/115014792127-Copyright-notice}{©~2020~The
  New York Times Company}
\end{itemize}

\begin{itemize}
\tightlist
\item
  \href{https://www.nytco.com/}{NYTCo}
\item
  \href{https://help.nytimes3xbfgragh.onion/hc/en-us/articles/115015385887-Contact-Us}{Contact
  Us}
\item
  \href{https://www.nytco.com/careers/}{Work with us}
\item
  \href{https://nytmediakit.com/}{Advertise}
\item
  \href{http://www.tbrandstudio.com/}{T Brand Studio}
\item
  \href{https://www.nytimes3xbfgragh.onion/privacy/cookie-policy\#how-do-i-manage-trackers}{Your
  Ad Choices}
\item
  \href{https://www.nytimes3xbfgragh.onion/privacy}{Privacy}
\item
  \href{https://help.nytimes3xbfgragh.onion/hc/en-us/articles/115014893428-Terms-of-service}{Terms
  of Service}
\item
  \href{https://help.nytimes3xbfgragh.onion/hc/en-us/articles/115014893968-Terms-of-sale}{Terms
  of Sale}
\item
  \href{https://spiderbites.nytimes3xbfgragh.onion}{Site Map}
\item
  \href{https://help.nytimes3xbfgragh.onion/hc/en-us}{Help}
\item
  \href{https://www.nytimes3xbfgragh.onion/subscription?campaignId=37WXW}{Subscriptions}
\end{itemize}
