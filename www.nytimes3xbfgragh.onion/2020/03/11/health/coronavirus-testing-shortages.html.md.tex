Sections

SEARCH

\protect\hyperlink{site-content}{Skip to
content}\protect\hyperlink{site-index}{Skip to site index}

\href{https://www.nytimes3xbfgragh.onion/section/health}{Health}

\href{https://myaccount.nytimes3xbfgragh.onion/auth/login?response_type=cookie\&client_id=vi}{}

\href{https://www.nytimes3xbfgragh.onion/section/todayspaper}{Today's
Paper}

\href{/section/health}{Health}\textbar{}As Coronavirus Testing
Increases, Some Labs Fear a Shortage of Other Supplies

\url{https://nyti.ms/3cRWK4w}

\begin{itemize}
\item
\item
\item
\item
\item
\end{itemize}

\hypertarget{the-coronavirus-outbreak}{%
\subsubsection{\texorpdfstring{\href{https://www.nytimes3xbfgragh.onion/news-event/coronavirus?name=styln-coronavirus-national\&region=TOP_BANNER\&block=storyline_menu_recirc\&action=click\&pgtype=Article\&impression_id=46380e20-f4b8-11ea-b140-7b2ff790d7ab\&variant=undefined}{The
Coronavirus
Outbreak}}{The Coronavirus Outbreak}}\label{the-coronavirus-outbreak}}

\begin{itemize}
\tightlist
\item
  live\href{https://www.nytimes3xbfgragh.onion/2020/09/11/world/covid-19-coronavirus.html?name=styln-coronavirus-national\&region=TOP_BANNER\&block=storyline_menu_recirc\&action=click\&pgtype=Article\&impression_id=46383530-f4b8-11ea-b140-7b2ff790d7ab\&variant=undefined}{Latest
  Updates}
\item
  \href{https://www.nytimes3xbfgragh.onion/interactive/2020/us/coronavirus-us-cases.html?name=styln-coronavirus-national\&region=TOP_BANNER\&block=storyline_menu_recirc\&action=click\&pgtype=Article\&impression_id=46383531-f4b8-11ea-b140-7b2ff790d7ab\&variant=undefined}{Maps
  and Cases}
\item
  \href{https://www.nytimes3xbfgragh.onion/interactive/2020/science/coronavirus-vaccine-tracker.html?name=styln-coronavirus-national\&region=TOP_BANNER\&block=storyline_menu_recirc\&action=click\&pgtype=Article\&impression_id=46383532-f4b8-11ea-b140-7b2ff790d7ab\&variant=undefined}{Vaccine
  Tracker}
\item
  \href{https://www.nytimes3xbfgragh.onion/2020/09/10/us/politics/fda-coronavirus-vaccine.html?name=styln-coronavirus-national\&region=TOP_BANNER\&block=storyline_menu_recirc\&action=click\&pgtype=Article\&impression_id=46383533-f4b8-11ea-b140-7b2ff790d7ab\&variant=undefined}{F.D.A.
  Regulators' Self-Defense}
\item
  \href{https://www.nytimes3xbfgragh.onion/2020/09/09/upshot/coronavirus-surprise-test-fees.html?name=styln-coronavirus-national\&region=TOP_BANNER\&block=storyline_menu_recirc\&action=click\&pgtype=Article\&impression_id=46383534-f4b8-11ea-b140-7b2ff790d7ab\&variant=undefined}{Surprise
  Test Fees}
\end{itemize}

Advertisement

\protect\hyperlink{after-top}{Continue reading the main story}

Supported by

\protect\hyperlink{after-sponsor}{Continue reading the main story}

\hypertarget{as-coronavirus-testing-increases-some-labs-fear-a-shortage-of-other-supplies}{%
\section{As Coronavirus Testing Increases, Some Labs Fear a Shortage of
Other
Supplies}\label{as-coronavirus-testing-increases-some-labs-fear-a-shortage-of-other-supplies}}

Lab directors and federal officials are keeping a close eye on the
supply of other materials needed to conduct the tests.

\includegraphics{https://static01.graylady3jvrrxbe.onion/images/2020/03/11/science/11VIRUS-TESTKITS1/11VIRUS-TESTKITS1-articleLarge-v2.jpg?quality=75\&auto=webp\&disable=upscale}

By \href{https://www.nytimes3xbfgragh.onion/by/katie-thomas}{Katie
Thomas}

\begin{itemize}
\item
  March 11, 2020
\item
  \begin{itemize}
  \item
  \item
  \item
  \item
  \item
  \end{itemize}
\end{itemize}

Laboratories around the country are now facing potential shortages of
key materials and chemicals needed to run tests for the novel
coronavirus, as cases spread to more than two-thirds of the states and
the global pandemic strains testing resources even further.

Some lab directors say they are already beginning to run low of the
supplies needed to extract RNA from nasal swabs, a crucial initial step
that is separate from the millions of test kits that the federal
government has promised to ship to every state. Others say they are
weighing whether to borrow some materials from other research labs that
aren't involved in creating or running
\href{https://www.nytimes3xbfgragh.onion/2020/07/23/health/coronavirus-testing-supply-shortage.html}{coronavirus
tests}.

And some lab directors are worried about the future availability of the
reagents, or chemical ingredients, used in the tests themselves. Several
labs have also said that they have had trouble getting virus samples
that are needed to validate the tests to make sure they are properly
identifying positive samples.

Public health officials and health care providers have clamored to get
enough tests following a botched rollout of testing kits by the Centers
for Disease Control and Prevention --- and a delay by the Food and Drug
Administration in allowing independent labs to develop their own test
--- that led to weeks of delays in detecting the spread of the virus in
the country.

``The lack of testing in the United States is a debacle,'' said Dr. Marc
Lipsitch, a professor of epidemiology at the Harvard T.H. Chan School of
Public Health. ``We're supposed to be the best biomedical powerhouse in
the world and we're unable to do something almost every other country is
doing on an orders-of-magnitude bigger scale.''

Today, public health labs in every state say they are running the tests,
and academic and commercial labs have been scrambling to increase their
capacities to check for the virus.

\includegraphics{https://static01.graylady3jvrrxbe.onion/images/2020/03/11/science/11VIRUS-TESTKITS3/11VIRUS-TESTKITS3-articleLarge.jpg?quality=75\&auto=webp\&disable=upscale}

But Washington, New York and California are leading states with hundreds
of cases, as officials warned again on Wednesday that the numbers will
continue to rise.

People are also reporting that they still can't get tested, in some
cases because doctors and hospitals are evaluating patients based on
their symptoms and whether those are indicative of the virus or regular
flu.

The RNA extraction kits ``are usually things we wouldn't ever even
wonder if they were running out, because they're always around,'' said
Michael Mina, an assistant professor of epidemiology at the Harvard T.H.
Chan School of Public Health. ``But in this case, because everyone in
the world is trying to extract RNA right now, they seem to be low.''

At the University of California, Los Angeles, the chief of the
microbiology section of the medical center's clinical lab was so
concerned about his supply of RNA extraction kits made by the company
Qiagen that he recently sent an email to colleagues at the university's
research labs asking if they had any. ``While our investigators were
eager to help, none were using the kit in their labs,'' said Elaine
Schmidt, a spokeswoman for the medical center.

Eric Blank, the chief program officer at the Association of Public
Health Laboratories, said his group has also been hearing about back
orders of the extraction kits and other supplies. Now that independent
labs are able to run their own tests, ``it is increasing at a very rapid
pace,'' Mr. Blank said. ``It just depends on how rapidly the
manufacturers of some of these other ancillary materials needed to run
the tests can ramp up their production.''

Qiagen, a major manufacturer of the RNA extraction kits, said in a
statement this week that because of the coronavirus outbreak, demand is
``challenging our capacity to supply certain products'' and that it was
increasing production in sites in Germany, Spain and Maryland.

Image

A worker at the German biotech company Qiagen demonstrated a testing
device for infectious diseases at a plant in Hilden, Germany. The
company said it was trying to meet the demand for tests and was ramping
up its production in Germany, Spain and Maryland.Credit...Sascha
Schuermann/Getty Images

Roche, another supplier of lab materials and equipment, said in a
statement: ``Our manufacturing network has robust business continuity
plans for dealing with the impact of a potential health crisis and is
actively assessing and monitoring this evolving health situation.''

The F.D.A. and C.D.C. have also said they are watching for potential
shortages. The F.D.A. said this week it was ``monitoring this issue and
has heard from some manufacturers with questions about alternative
reagents, extraction methods and platforms.'' It said it was offering
guidance to labs and updates on the issue
\href{https://www.fda.gov/medical-devices/emergency-situations-medical-devices/faqs-diagnostic-testing-sars-cov-2}{on
its website}.

But the extraction kits are not the only supply item with uncertain
availability. The American Society for Microbiologists
\href{https://asm.org/Articles/Policy/2020/March/ASM-Expresses-Concern-about-Test-Reagent-Shortages}{said
Tuesday} that it was ``deeply concerned'' about a potential shortage of
the reagents needed to conduct the tests as well as other materials.
``Increased demand for testing has the potential to exhaust supplies
needed to perform the testing itself,'' the society said.

On Monday, the C.D.C. revised its guidelines to allow for the collection
of one specimen swab instead of the previously required two, a move that
the society said would cut the required amount of testing reagents in
half.

Dr. Robert Redfield, the director of the C.D.C.,
\href{https://www.politico.com/news/2020/03/10/coronavirus-testing-lab-materials-shortage-125212}{told
Politico on Tuesday} that the agency was keeping an eye on the supply of
materials needed to do the tests. But, when asked how the agency would
deal with a shortage of RNA extraction kits, he said: ``I don't know the
answer to that question.''

Integrated DNA Technologies, which is manufacturing coronavirus test
kits for the C.D.C., said in a statement that beginning next week, it
expects to be able to provide enough shipments of C.D.C. kits that would
allow for five million tests a week. The company added that ``is
accustomed to scaling up to meet customer demand and does not anticipate
needing to hire additional staff.''

Labs have also said they have had a difficult time getting so-called
positive controls, or samples of the virus to ensure the tests are
working properly.

``We have requested these from a couple of vendors, but it has taken
some time to get registered to have the controls shipped,'' said Dr. Jim
Dunn, the director of medical microbiology and virology at Texas
Children's Hospital in Houston, which is now running its own test for
coronavirus for the hospital's patients.

Veronique Greenwood and Denise Grady contributed reporting.

Advertisement

\protect\hyperlink{after-bottom}{Continue reading the main story}

\hypertarget{site-index}{%
\subsection{Site Index}\label{site-index}}

\hypertarget{site-information-navigation}{%
\subsection{Site Information
Navigation}\label{site-information-navigation}}

\begin{itemize}
\tightlist
\item
  \href{https://help.nytimes3xbfgragh.onion/hc/en-us/articles/115014792127-Copyright-notice}{©~2020~The
  New York Times Company}
\end{itemize}

\begin{itemize}
\tightlist
\item
  \href{https://www.nytco.com/}{NYTCo}
\item
  \href{https://help.nytimes3xbfgragh.onion/hc/en-us/articles/115015385887-Contact-Us}{Contact
  Us}
\item
  \href{https://www.nytco.com/careers/}{Work with us}
\item
  \href{https://nytmediakit.com/}{Advertise}
\item
  \href{http://www.tbrandstudio.com/}{T Brand Studio}
\item
  \href{https://www.nytimes3xbfgragh.onion/privacy/cookie-policy\#how-do-i-manage-trackers}{Your
  Ad Choices}
\item
  \href{https://www.nytimes3xbfgragh.onion/privacy}{Privacy}
\item
  \href{https://help.nytimes3xbfgragh.onion/hc/en-us/articles/115014893428-Terms-of-service}{Terms
  of Service}
\item
  \href{https://help.nytimes3xbfgragh.onion/hc/en-us/articles/115014893968-Terms-of-sale}{Terms
  of Sale}
\item
  \href{https://spiderbites.nytimes3xbfgragh.onion}{Site Map}
\item
  \href{https://help.nytimes3xbfgragh.onion/hc/en-us}{Help}
\item
  \href{https://www.nytimes3xbfgragh.onion/subscription?campaignId=37WXW}{Subscriptions}
\end{itemize}
