Sections

SEARCH

\protect\hyperlink{site-content}{Skip to
content}\protect\hyperlink{site-index}{Skip to site index}

\href{https://www.nytimes3xbfgragh.onion/section/politics}{Politics}

\href{https://myaccount.nytimes3xbfgragh.onion/auth/login?response_type=cookie\&client_id=vi}{}

\href{https://www.nytimes3xbfgragh.onion/section/todayspaper}{Today's
Paper}

\href{/section/politics}{Politics}\textbar{}Joe Biden Wins Primaries in
Florida, Illinois and Arizona: Highlights

\url{https://nyti.ms/33wB6yw}

\begin{itemize}
\item
\item
\item
\item
\item
\item
\end{itemize}

\begin{itemize}
\item
  \href{https://www.nytimes3xbfgragh.onion/interactive/2020/09/08/us/elections/results-new-hampshire-primary-elections.html?action=click\&pgtype=Article\&state=default\&region=TOP_BANNER\&context=storylines_menu}{New
  Hampshire Results}
\item
  \href{https://www.nytimes3xbfgragh.onion/live/2020/09/08/us/trump-vs-biden?action=click\&pgtype=Article\&state=default\&region=TOP_BANNER\&context=storylines_menu}{Election
  Updates}
\item
  \href{https://www.nytimes3xbfgragh.onion/interactive/2020/us/elections/election-states-biden-trump.html?action=click\&pgtype=Article\&state=default\&region=TOP_BANNER\&context=storylines_menu}{Paths
  to 270}
\item
  \href{https://www.nytimes3xbfgragh.onion/interactive/2020/08/31/us/politics/vote-by-mail-deadlines.html?action=click\&pgtype=Article\&state=default\&region=TOP_BANNER\&context=storylines_menu}{Voting
  by Mail}
\item
  \href{https://www.nytimes3xbfgragh.onion/interactive/2019/us/elections/2020-presidential-election-calendar.html?action=click\&pgtype=Article\&state=default\&region=TOP_BANNER\&context=storylines_menu}{Key
  Dates}
\item
  \href{https://www.nytimes3xbfgragh.onion/newsletters/politics?action=click\&pgtype=Article\&state=default\&region=TOP_BANNER\&context=storylines_menu}{Politics
  Newsletter}
\end{itemize}

Advertisement

\protect\hyperlink{after-top}{Continue reading the main story}

Supported by

\protect\hyperlink{after-sponsor}{Continue reading the main story}

\hypertarget{joe-biden-wins-primaries-in-florida-illinois-and-arizona-highlights}{%
\section{Joe Biden Wins Primaries in Florida, Illinois and Arizona:
Highlights}\label{joe-biden-wins-primaries-in-florida-illinois-and-arizona-highlights}}

Mr. Biden captured easy victories in all three states that voted on
Tuesday. Bernie Sanders's chances of a comeback in the Democratic
presidential race have all but evaporated.

\href{https://www.nytimes3xbfgragh.onion/by/reid-j-epstein}{\includegraphics{https://static01.graylady3jvrrxbe.onion/images/2019/06/25/reader-center/author-reid-epstein/9e877853d8234217b58e5762253aa771-thumbLarge.png}}\href{https://www.nytimes3xbfgragh.onion/by/lisa-lerer}{\includegraphics{https://static01.graylady3jvrrxbe.onion/images/2018/09/11/us/politics/author-lisa-lerer/lisa-lerer-headshot-thumbLarge.png}}\href{https://www.nytimes3xbfgragh.onion/by/thomas-kaplan}{\includegraphics{https://static01.graylady3jvrrxbe.onion/images/2019/08/28/reader-center/author-thomas-kaplan/author-thomas-kaplan-thumbLarge-v2.png}}

By \href{https://www.nytimes3xbfgragh.onion/by/reid-j-epstein}{Reid J.
Epstein}, \href{https://www.nytimes3xbfgragh.onion/by/lisa-lerer}{Lisa
Lerer} and
\href{https://www.nytimes3xbfgragh.onion/by/thomas-kaplan}{Thomas
Kaplan}

\begin{itemize}
\item
  Published March 17, 2020Updated March 18, 2020
\item
  \begin{itemize}
  \item
  \item
  \item
  \item
  \item
  \item
  \end{itemize}
\end{itemize}

\includegraphics{https://static01.graylady3jvrrxbe.onion/images/2020/03/17/us/politics/17primary-briefing-ill/merlin_170214012_fc142910-8342-4a43-9cf1-9f54b9d24168-articleLarge.jpg?quality=75\&auto=webp\&disable=upscale}

Former Vice President
\href{https://www.nytimes3xbfgragh.onion/interactive/2020/us/elections/joe-biden.html}{Joseph
R. Biden Jr.} easily won the Democratic primary elections in Florida,
Illinois and Arizona on Tuesday. The three states award a total of 441
delegates for the party's presidential nomination.

Mr. Biden's victories over Senator
\href{https://www.nytimes3xbfgragh.onion/interactive/2020/us/elections/bernie-sanders.html}{Bernie
Sanders} effectively
\href{https://www.nytimes3xbfgragh.onion/2020/03/17/us/politics/biden-florida-illinois-primary.html}{ended
his rival's hopes of a comeback}. Mr. Biden currently has
\href{https://www.nytimes3xbfgragh.onion/interactive/2020/us/elections/delegate-count-primary-results.html}{1,147
delegates to Mr. Sanders's 861}.

These were the first primaries to be held amid the heightened fear and
restrictions triggered by the coronavirus. The Trump administration has
recommended
\href{https://www.nytimes3xbfgragh.onion/2020/03/16/world/live-coronavirus-news-updates.html?action=click\&module=Spotlight\&pgtype=Homepage}{avoiding
groups of more than 10 people}, and turnout was down in Illinois on
Tuesday. But extensive early voting helped lift turnout in Florida and
Arizona.

\hypertarget{the-highlights}{%
\subsubsection{The highlights:}\label{the-highlights}}

\begin{itemize}
\tightlist
\item
  \protect\hyperlink{link-58c71c58}{Biden comes out on top in Arizona.}
\item
  \protect\hyperlink{link-25373fb4}{`I hear you,' Biden tells Sanders
  supporters.}
\item
  \protect\hyperlink{link-2e84c23f}{Florida and Arizona will surpass
  2016 turnout.}
\item
  \protect\hyperlink{link-402054e}{Biden is the winner in Illinois.}
\item
  \protect\hyperlink{link-7e0a4a3e}{Biden cruises past Sanders in
  Florida.}
\end{itemize}

Stay up to date on primaries and caucuses. Subscribe to ``On Politics,''
and we'll send you a link to the live results.

\href{https://www.nytimes3xbfgragh.onion/newsletters/politics}{Sign up
for our politics newsletter}

\hypertarget{biden-comes-out-on-top-in-arizona}{%
\subsection{Biden comes out on top in
Arizona.}\label{biden-comes-out-on-top-in-arizona}}

Mr. Biden continued his string of victories by winning the Arizona
primary, dashing Mr. Sanders's hopes in a state with many Latino voters.

Mr. Sanders held a rally in Phoenix this month before the coronavirus
curtailed in-person campaigning, and he did well with Latino voters in
previous contests. Younger Latino voters, in particular, have been
receptive to Mr. Sanders's candidacy.

But Mr. Biden was still victorious in Tuesday's primary, completing his
sweep of the day's three contests. Arizona was the smallest delegate
prize of the day, and Mr. Sanders also lost there in 2016 when he faced
Hillary Clinton.

Turnout was on pace to surpass the 2016 primary, according to the
secretary of state's office. In Maricopa County, which includes Phoenix,
more than 40,000 people voted in person Tuesday, compared to 35,000
there in 2016, according to the county registrar.

Much of Mr. Sanders's support has come from young voters, but it is
unclear how that played out in Arizona. Voter surveys suggested that the
turnout was driven largely by people over the age of 45. And activists
supporting Mr. Sanders said over the weekend that they expected many of
his core supporters would be less likely to cast ballots, as the
pandemic shut down much of the service industry and concerns mounted
over their health and keeping their jobs.

\hypertarget{i-hear-you-biden-tells-sanders-supporters}{%
\subsection{`I hear you,' Biden tells Sanders
supporters.}\label{i-hear-you-biden-tells-sanders-supporters}}

\includegraphics{https://static01.graylady3jvrrxbe.onion/images/2020/03/17/us/politics/17vid-Biden-Live/17vid-Biden-Live-videoSixteenByNine3000.jpg}

Mr. Biden made an explicit appeal to supporters of Mr. Sanders in a
brief live-streamed address from his home in Wilmington, Del., in which
he also spoke of the need for Americans to do their part in fighting the
coronavirus.

``Senator Sanders and I may disagree on tactics, but we share a common
vision,'' Mr. Biden said, citing health care, income inequality and
climate change. He continued, ``Senator Sanders and his supporters have
brought a remarkable passion and tenacity to all of these issues, and
together, they have shifted the fundamental conversation in this
country.''

His remarks were evidence of the delicate balancing act in which Mr.
Biden is engaging: signaling to Sanders supporters that he respects them
and wants their backing, without pressuring Mr. Sanders himself to leave
the race. One of Mr. Biden's most significant political weaknesses is
with younger voters, and he spoke directly to them on Tuesday.

``So let me say, especially to the young voters who have been inspired
by Senator Sanders: I hear you,'' he said. ``I know what's at stake. I
know what we have to do. Our goal as a campaign, and my goal as a
candidate for president, is to unify this party and then to unify the
nation.''

Mr. Biden appeared after notching victories in Tuesday's two largest
delegate prizes, Florida and Illinois. After first speaking about the
coronavirus, striking a somber note, he addressed the day's primaries.
Citing those two states, he said his campaign had ``a very good night.''

``We've moved closer to securing the Democratic Party's nomination for
president,'' he said, ``and we're doing it by building a broad coalition
that we need to win in November.''

\hypertarget{florida-and-arizona-will-surpass-2016-turnout}{%
\subsection{Florida and Arizona will surpass 2016
turnout.}\label{florida-and-arizona-will-surpass-2016-turnout}}

Two of the three states voting on Tuesday have now exceeded turnout
levels seen in the 2016 Democratic primary, despite the coronavirus
outbreak.

With 98 percent of precincts reporting, Florida had surpassed its 2016
turnout by more than 10,000 votes. In Arizona, turnout was also expected
to easily top 2016 levels.

Both states had invested heavily in early voting systems, and they
encouraged early voting and voting by mail as the outbreak worsened, in
order to help reduce crowds at polling places on Primary Day.

In Florida, roughly 140,000 more Democrats voted by mail than in 2016,
and nearly 75,000 more voted early. While some counties, including Palm
Beach, had to relocate polling centers amid a poll worker shortage,
turnout in Florida was lifted by the roughly 1.1 million people who
voted early.

In Arizona, more than 380,000 people voted before polls opened on
Tuesday, just 29,000 fewer voters than the total turnout for 2016. The
state also offered curbside ballot drop-off for voters on Tuesday for
those who didn't want to come into a polling location.

\hypertarget{biden-is-the-winner-in-illinois}{%
\subsection{Biden is the winner in
Illinois.}\label{biden-is-the-winner-in-illinois}}

Image

Voters cast ballots during Illinois primary in Chicago.Credit...Kamil
Krzaczynski/Agence France-Presse --- Getty Images

Mr. Biden won the primary in Illinois, the second-largest delegate prize
among Tuesday's contests. With the victory, he continued to pad his
sizable delegate lead over Mr. Sanders.

Mr. Sanders narrowly lost the state in 2016 to Mrs. Clinton, but he
appeared to be at a significant disadvantage heading into this year's
contest.

As in Florida, black voters make up a significant portion of the
Democratic primary electorate in Illinois. Strong support from black
voters in South Carolina and a number of other Southern states was
crucial to Mr. Biden's resurgence in the primary. Mr. Biden has also
performed well among white voters in recent contests.

Mr. Biden had widespread support among Democratic elected officials in
the state. His backers included Gov. J.B. Pritzker, who endorsed him on
Monday; Senators Richard J. Durbin and Tammy Duckworth; and Mayor Lori
Lightfoot of Chicago.

Mr. Biden had planned to hold a campaign event in Chicago last week, but
it was called off because of the coronavirus. Instead, Mr. Biden held a
``virtual town hall'' for Illinois voters, but his first attempt at
virtual campaigning
\href{https://www.nytimes3xbfgragh.onion/2020/03/13/us/politics/joe-biden-digital-campaign.html}{was
marred by technical problems}.

\hypertarget{biden-cruises-past-sanders-in-florida}{%
\subsection{Biden cruises past Sanders in
Florida.}\label{biden-cruises-past-sanders-in-florida}}

Image

Former Vice President Joseph R. Biden Jr. in Miami in
September.Credit...Alicia Vera for The New York Times

Mr. Biden easily won the Florida primary, racking up an early --- and
expected --- victory in the Sunshine State.

In 2016, Mr. Sanders captured just nine counties in the state, largely
the kind of rural white areas he's been struggling to hold against Mr.
Biden this campaign. He has failed to win large numbers of black voters,
who made up more than a quarter of the Democratic primary electorate
four years ago.

In Florida, Mr. Sanders's refusal to retract his praise of Fidel Castro
and aspects of the Communist Cuban revolution drew ire not just from
Cubans but also from a far more diverse group of Latinos, including
Colombians, Nicaraguans and Venezuelans. And though Mr. Sanders would be
the first Jewish president, his comments about Israel turned off many
Jewish voters, according to polling.

Florida was always going to be a good state for Mr. Biden. A.P.
VoteCast, a voter survey conducted in the days leading up to the primary
for The Associated Press, found that 25 percent of Florida's Democratic
electorate is African-American and 70 percent is 45 or older, two
demographics that have voted overwhelmingly for Mr. Biden.

Much of Mr. Biden's win most likely stems from the perception among
Democratic voters in Florida that he was the stronger candidate to beat
Mr. Trump and would fare the best in a national emergency, like the
coronavirus. More than seven in 10 voters surveyed by A.P. VoteCast said
that they saw Mr. Biden as the most electable and that they trusted him
the most to handle a major crisis.

Neither candidate campaigned in the state: Rallies for Mr. Biden in
Tampa and Miami were canceled because of fears about the spread of the
coronavirus. Most of the candidates' political activity was left to
television ads, volunteers and campaign surrogates.

Supporters of the candidate who spent the most time in the state ---
former Mayor Michael R. Bloomberg of New York City --- most likely went
to Mr. Biden.

\hypertarget{sanders-lays-out-a-coronavirus-plan-including-2000-payments-to-americans}{%
\subsection{Sanders lays out a coronavirus plan, including \$2,000
payments to
Americans.}\label{sanders-lays-out-a-coronavirus-plan-including-2000-payments-to-americans}}

\includegraphics{https://static01.graylady3jvrrxbe.onion/images/2020/03/17/us/politics/17vid-Sanders-Live/17vid-Sanders-Live-videoSixteenByNine3000.jpg}

Mr. Sanders addressed the escalating coronavirus crisis on Tuesday
night, calling it an ``unprecedented moment'' and laying out an
extensive list of policy proposals to deal with the emergency that he
said he would work with Democratic leadership to carry out. He estimated
that combating the crisis would require at least \$2 trillion in
funding.

Among the plans he put forth were activating the armed forces to build
mobile hospitals and testing facilities, as well as having the
government provide a ``direct emergency \$2,000 cash payment to every
household in America.'' He also proposed a moratorium on evictions and
utility shut-offs, and providing emergency unemployment assistance to
anyone who loses their job.

And he again used this moment to call for his signature policy proposal,
``Medicare for all.'' In the meantime, he called for Medicare to cover
all medical bills during the coronavirus crisis.

``What I believe we must do is empower Medicare to cover all medical
bills during this emergency,'' he said. But he also stressed that ``this
is not Medicare for all. We can't pass that right now.''

Mr. Sanders has publicly addressed the coronavirus several times in the
last week, and his remarks on Tuesday largely reflected his previous
comments.

During his speech, which he delivered before many polls closed, he did
not address the election. Neither CNN nor MSNBC showed his remarks live.

\hypertarget{marie-newman-beats-dan-lipinski-in-illinois}{%
\subsection{Marie Newman beats Dan Lipinski in
Illinois.}\label{marie-newman-beats-dan-lipinski-in-illinois}}

A closely watched Democratic congressional primary ended in an upset on
Tuesday, when
\href{https://www.nytimes3xbfgragh.onion/2020/03/18/us/politics/marie-newman-dan-lipinski-illinois.html}{Marie
Newman beat Representative Daniel Lipinski}, a conservative Democrat and
an eight-term incumbent. It was a second try for Ms. Newman, who lost to
Mr. Lipinski
\href{https://www.nytimes3xbfgragh.onion/elections/results/illinois-house-district-3-primary-election}{by
just 2,000 votes} in 2018. She is a progressive who was backed by the
Justice Democrats and
\href{https://www.nytimes3xbfgragh.onion/2019/09/17/us/politics/aoc-endorsement-marie-newman.html}{Representative
Alexandria Ocasio-Cortez} of New York.

The primary was an important test of whether a Democrat like Mr.
Lipinski,
\href{https://www.nytimes3xbfgragh.onion/2019/05/22/us/politics/dan-lipinski-abortion-cheri-bustos.html}{who
opposes abortion} and voted against the Affordable Care Act, was still
welcome in the party --- and conversely, whether a candidate from the
party's progressive wing could win in a district that, while solidly
Democratic, leans more conservative on social issues.

Ms. Newman, a business consultant and founder of an anti-bullying
program, had drawn support even from some veteran rank-and-file
Democrats, in a sign of how much of an outlier Mr. Lipinsky was in the
party.

``This is a critical victory for the progressive movement in showing
that voters are ready for a new generation of progressive leadership in
the Democratic Party,'' said Alexandra Rojas, executive director of
Justice Democrats. ``This isn't just a loss for one incumbent. It's a
defeat for machine politics and big corporate donors who want to stop
our movement for `Medicare for all,' a Green New Deal and reproductive
rights.''

\hypertarget{and-the-presumptive-republican-nominee-is-}{%
\subsection{And the presumptive Republican nominee is
\ldots{}}\label{and-the-presumptive-republican-nominee-is-}}

Image

President Trump made an announcement from the Oval Office about the
widening coronavirus crisis on Wednesday.Credit...Doug Mills/The New
York Times

President Trump achieved the inevitable on Tuesday night: After winning
122 delegates in Florida, he officially racked up enough delegates to
become the presumptive Republican nominee for president. A candidate
needs 1,276 delegates to win the nomination, and Mr. Trump on Tuesday
night had 1,330 delegates.

Brad Parscale, the president's campaign manager, said in a statement
that Mr. Trump's victory showed a unified Republican Party. He credited
it to ``his response to the coronavirus'' and a ``broad and strong
economy,'' even as markets plunged and a global recession appeared
inevitable.

Mr. Trump has barely had a contest in the Republican primary. A onetime
field of three challengers had already winnowed down to one left
standing, William Weld, the former governor of Massachusetts, who has
failed to make a dent in Mr. Trump's support among Republican voters.

\hypertarget{tuesday-is-the-last-political-action-for-a-while-what-comes-next}{%
\subsection{Tuesday is the last political action for a while. What comes
next?}\label{tuesday-is-the-last-political-action-for-a-while-what-comes-next}}

Image

Ballots were counted at the Maricopa County recorder's office in Phoenix
on Tuesday.Credit...Adriana Zehbrauskas for The New York Times

Three primary elections were held on Tuesday --- and yet there were no
rallies to trumpet victories or spin losses.

Presidential politics in the coronavirus era has left Mr. Biden and Mr.
Sanders in a new reality. They're running for president, but without the
running.

There are no get-out-the-vote efforts, no rallies, no commercials, no
fund-raising events and, for the foreseeable future, nowhere for them to
go.

Ohio was supposed to have a primary, but the
\href{https://www.nytimes3xbfgragh.onion/2020/03/17/us/politics/march-17-democratic-primary.html}{governor
ordered precincts closed}. Louisiana, Kentucky and Maryland have moved
primaries planned for the coming weeks back to June in hopes the
pandemic subsides by then. Democratic National Committee officials
insist the party's convention will take place as planned in Milwaukee in
July, but the truth is nobody really knows what the world will look like
in four days, let alone in four months.

If Tuesday night's contests were unfolding under normal circumstances,
the campaigns and the political press would be decamping for Georgia,
which was supposed to be the only state with a primary next week.

But Georgia officials on Saturday moved their state's primary to May 19.
There will be no campaigning in the Atlanta suburbs, no tracking TV
spending by the campaigns. No more counting the delegates Mr. Biden and
Mr. Sanders need to accumulate to clinch the presidential nomination.

On Tuesday, the Democratic National Committee's chairman, Tom Perez,
urged the states remaining on the election calendar to conduct contests
solely through vote by mail, ``instead of moving primaries to later in
the cycle when timing around the virus remains unpredictable.''

We're all left waiting and wondering the same thing: What comes next?

\hypertarget{biden-picks-up-secret-service-protection}{%
\subsection{Biden picks up Secret Service
protection.}\label{biden-picks-up-secret-service-protection}}

Mr. Biden now has Secret Service protection, the organization said on
Tuesday, a development that comes as he has achieved front-runner status
in the Democratic primary, and after several security incidents occurred
at campaign events.

``The U.S. Secret Service can confirm that we have initiated full
protective coverage for Democratic Presidential Candidate and former
Vice President Joseph Biden,'' a representative for the Secret Service
said.

On a number of occasions, voters or activists have come physically close
to Mr. Biden or his family, including on Super Tuesday, when animal
rights activists moved toward him and his wife as he spoke, and several
young campaign staff members physically interceded.

Reporting was contributed by Maggie Astor, Nick Corasaniti, Sydney
Ember, Katie Glueck, Shane Goldmacher, Jennifer Medina, Matt Stevens and
Annie Karni.

\hypertarget{our-2020-election-guide}{%
\section{Our 2020 Election Guide}\label{our-2020-election-guide}}

Updated ~Sept. 8, 2020

\begin{center}\rule{0.5\linewidth}{\linethickness}\end{center}

\begin{itemize}
\item ~
  \hypertarget{the-latest}{%
  \subsection{The Latest}\label{the-latest}}

  \begin{itemize}
  \item
    President Trump and his party are using a playbook that aims to
    alarm people about crime in their backyards. It didn't work in 2018,
    but
    \href{https://www.nytimes3xbfgragh.onion/2020/09/08/us/politics/trump-republicans-fear-strategy.html?action=click\&pgtype=Article\&state=default\&region=BELOW_MAIN_CONTENT\&context=storylines_guide}{both
    parties think it could resonate more this year}.
  \end{itemize}
\item ~
  \hypertarget{how-to-win-270}{%
  \subsection{How to Win 270}\label{how-to-win-270}}

  \begin{itemize}
  \item
    Joe Biden and Donald Trump need 270 electoral votes to reach the
    White House. Try building
    \href{https://www.nytimes3xbfgragh.onion/interactive/2020/us/elections/election-states-biden-trump.html?action=click\&pgtype=Article\&state=default\&region=BELOW_MAIN_CONTENT\&context=storylines_guide}{your
    own coalition of battleground states}~to see potential outcomes.
  \end{itemize}
\item ~
  \hypertarget{voting-by-mail}{%
  \subsection{Voting by Mail}\label{voting-by-mail}}

  \begin{itemize}
  \item
    Will you have enough time to vote by mail in your state? Yes, but
    it's risky to procrastinate.
    \href{https://www.nytimes3xbfgragh.onion/interactive/2020/08/31/us/politics/vote-by-mail-deadlines.html?action=click\&pgtype=Article\&state=default\&region=BELOW_MAIN_CONTENT\&context=storylines_guide}{Check
    your state's deadline.}
  \item
    \href{https://www.nytimes3xbfgragh.onion/interactive/2020/us/elections/joe-biden.html?action=click\&pgtype=Article\&state=default\&region=BELOW_MAIN_CONTENT\&context=storylines_guide}{}

    \hypertarget{joe-biden}{%
    \section{Joe Biden}\label{joe-biden}}

    \hypertarget{democrat}{%
    \subsection{Democrat}\label{democrat}}

    \href{https://www.nytimes3xbfgragh.onion/interactive/2020/us/elections/donald-trump.html?action=click\&pgtype=Article\&state=default\&region=BELOW_MAIN_CONTENT\&context=storylines_guide}{}

    \hypertarget{donald-trump}{%
    \section{Donald Trump}\label{donald-trump}}

    \hypertarget{republican}{%
    \subsection{Republican}\label{republican}}
  \end{itemize}
\item
  \hypertarget{keep-up-with-our-coverage}{%
  \subsection{Keep Up With Our
  Coverage}\label{keep-up-with-our-coverage}}

  \begin{itemize}
  \item
    Get an
    \href{https://www.nytimes3xbfgragh.onion/newsletters/politics?action=click\&pgtype=Article\&state=default\&region=BELOW_MAIN_CONTENT\&context=storylines_guide}{email}~recapping
    the day's news
  \item
    Download our mobile app on
    \href{https://apps.apple.com/us/app/nytimes/id284862083?ls=1\&mat_click_id=5c79ae7455014fd1bd66b5610c05b8f2-20191112-16948\&referrer=mat_click_id\%3D5c79ae7455014fd1bd66b5610c05b8f2-20191112-16948\%26link_click_id\%3D722930677036718082}{iOS}~and
    \href{http://a.localytics.com/android?id=com.nytimes.android\&referrer=utm_source\%3Dother_nyt_mobile_web\%26utm_medium\%3DWeb\%2520page\%26utm_term\%3DGeneral\%2520Mobile\%2520Page\%26utm_campaign\%3DNYT\%2520Mobile\%2520General\%2520Page}{Android}~and
    turn on Breaking News and Politics alerts
  \end{itemize}
\end{itemize}

Advertisement

\protect\hyperlink{after-bottom}{Continue reading the main story}

\hypertarget{site-index}{%
\subsection{Site Index}\label{site-index}}

\hypertarget{site-information-navigation}{%
\subsection{Site Information
Navigation}\label{site-information-navigation}}

\begin{itemize}
\tightlist
\item
  \href{https://help.nytimes3xbfgragh.onion/hc/en-us/articles/115014792127-Copyright-notice}{©~2020~The
  New York Times Company}
\end{itemize}

\begin{itemize}
\tightlist
\item
  \href{https://www.nytco.com/}{NYTCo}
\item
  \href{https://help.nytimes3xbfgragh.onion/hc/en-us/articles/115015385887-Contact-Us}{Contact
  Us}
\item
  \href{https://www.nytco.com/careers/}{Work with us}
\item
  \href{https://nytmediakit.com/}{Advertise}
\item
  \href{http://www.tbrandstudio.com/}{T Brand Studio}
\item
  \href{https://www.nytimes3xbfgragh.onion/privacy/cookie-policy\#how-do-i-manage-trackers}{Your
  Ad Choices}
\item
  \href{https://www.nytimes3xbfgragh.onion/privacy}{Privacy}
\item
  \href{https://help.nytimes3xbfgragh.onion/hc/en-us/articles/115014893428-Terms-of-service}{Terms
  of Service}
\item
  \href{https://help.nytimes3xbfgragh.onion/hc/en-us/articles/115014893968-Terms-of-sale}{Terms
  of Sale}
\item
  \href{https://spiderbites.nytimes3xbfgragh.onion}{Site Map}
\item
  \href{https://help.nytimes3xbfgragh.onion/hc/en-us}{Help}
\item
  \href{https://www.nytimes3xbfgragh.onion/subscription?campaignId=37WXW}{Subscriptions}
\end{itemize}
