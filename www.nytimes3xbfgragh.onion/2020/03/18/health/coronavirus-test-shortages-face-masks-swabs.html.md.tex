Sections

SEARCH

\protect\hyperlink{site-content}{Skip to
content}\protect\hyperlink{site-index}{Skip to site index}

\href{https://www.nytimes3xbfgragh.onion/section/health}{Health}

\href{https://myaccount.nytimes3xbfgragh.onion/auth/login?response_type=cookie\&client_id=vi}{}

\href{https://www.nytimes3xbfgragh.onion/section/todayspaper}{Today's
Paper}

\href{/section/health}{Health}\textbar{}The Latest Obstacle to Getting
Tested? A Shortage of Swabs and Face Masks

\url{https://nyti.ms/2U1AJJe}

\begin{itemize}
\item
\item
\item
\item
\item
\end{itemize}

\hypertarget{the-coronavirus-outbreak}{%
\subsubsection{\texorpdfstring{\href{https://www.nytimes3xbfgragh.onion/news-event/coronavirus?name=styln-coronavirus-national\&region=TOP_BANNER\&block=storyline_menu_recirc\&action=click\&pgtype=Article\&impression_id=9cc9e110-f4b7-11ea-84c5-ff3c2ad4221c\&variant=undefined}{The
Coronavirus
Outbreak}}{The Coronavirus Outbreak}}\label{the-coronavirus-outbreak}}

\begin{itemize}
\tightlist
\item
  live\href{https://www.nytimes3xbfgragh.onion/2020/09/11/world/covid-19-coronavirus.html?name=styln-coronavirus-national\&region=TOP_BANNER\&block=storyline_menu_recirc\&action=click\&pgtype=Article\&impression_id=9cc9e111-f4b7-11ea-84c5-ff3c2ad4221c\&variant=undefined}{Latest
  Updates}
\item
  \href{https://www.nytimes3xbfgragh.onion/interactive/2020/us/coronavirus-us-cases.html?name=styln-coronavirus-national\&region=TOP_BANNER\&block=storyline_menu_recirc\&action=click\&pgtype=Article\&impression_id=9cc9e112-f4b7-11ea-84c5-ff3c2ad4221c\&variant=undefined}{Maps
  and Cases}
\item
  \href{https://www.nytimes3xbfgragh.onion/interactive/2020/science/coronavirus-vaccine-tracker.html?name=styln-coronavirus-national\&region=TOP_BANNER\&block=storyline_menu_recirc\&action=click\&pgtype=Article\&impression_id=9cc9e113-f4b7-11ea-84c5-ff3c2ad4221c\&variant=undefined}{Vaccine
  Tracker}
\item
  \href{https://www.nytimes3xbfgragh.onion/2020/09/10/us/politics/fda-coronavirus-vaccine.html?name=styln-coronavirus-national\&region=TOP_BANNER\&block=storyline_menu_recirc\&action=click\&pgtype=Article\&impression_id=9cc9e114-f4b7-11ea-84c5-ff3c2ad4221c\&variant=undefined}{F.D.A.
  Regulators' Self-Defense}
\item
  \href{https://www.nytimes3xbfgragh.onion/2020/09/09/upshot/coronavirus-surprise-test-fees.html?name=styln-coronavirus-national\&region=TOP_BANNER\&block=storyline_menu_recirc\&action=click\&pgtype=Article\&impression_id=9cc9e115-f4b7-11ea-84c5-ff3c2ad4221c\&variant=undefined}{Surprise
  Test Fees}
\end{itemize}

Advertisement

\protect\hyperlink{after-top}{Continue reading the main story}

Supported by

\protect\hyperlink{after-sponsor}{Continue reading the main story}

\hypertarget{the-latest-obstacle-to-getting-tested-a-shortage-of-swabs-and-face-masks}{%
\section{The Latest Obstacle to Getting Tested? A Shortage of Swabs and
Face
Masks}\label{the-latest-obstacle-to-getting-tested-a-shortage-of-swabs-and-face-masks}}

Hospitals and doctors say they are critically low on swabs needed to
test patients for the coronavirus, as well as face masks and other gear
to protect health care workers.

\includegraphics{https://static01.graylady3jvrrxbe.onion/images/2020/03/19/science/19VIRUS-TESTSHORTAGE1/merlin_170638635_c2a56eaa-2756-40f1-a309-ecd9d57df36b-articleLarge.jpg?quality=75\&auto=webp\&disable=upscale}

By \href{https://www.nytimes3xbfgragh.onion/by/katie-thomas}{Katie
Thomas}

\begin{itemize}
\item
  Published March 18, 2020Updated July 23, 2020
\item
  \begin{itemize}
  \item
  \item
  \item
  \item
  \item
  \end{itemize}
\end{itemize}

Just as the nation's ability to
\href{https://www.nytimes3xbfgragh.onion/2020/07/23/health/coronavirus-testing-supply-shortage.html}{test
for coronavirus} is expanding, hospitals and clinics say another
obstacle is looming: shortages of testing
\href{https://www.nytimes3xbfgragh.onion/2020/04/13/nyregion/coronavirus-testing.html}{swabs}
and protective gear for health care workers.

At the UCSF Health --- a San Francisco hospital system at the heart of
one of the nation's coronavirus outbreaks --- officials said they would
have to stop testing patients in about five days because they will run
out of nasopharyngeal swabs, which are inserted into patients' nasal
passages to get samples for testing. Other hospitals elsewhere in the
country were ending their practice of using a second swab to test for
the flu in an effort to preserve their supply.

The main manufacturer of the swabs, Copan, is an Italian company whose
manufacturing plant is in Northern Italy, a region that has itself been
hard hit by the coronavirus outbreak. It says it has ramped up
manufacturing to deal with the extraordinary demand for an otherwise
unassuming product to which many doctors gave little thought, until now.

``We weren't really thinking about, `Wow, what's our swab supply?'
because we haven't really faced anything that depleted our swabs
before,'' said Dr. Josh Adler, the chief clinical officer of U.C.S.F.
Health. Making matters more difficult, the swabs must have just the
right ingredients --- the shafts cannot be wood, for example --- or the
virus might not be properly detected. ``You can't go to your local store
and get Q-tips,'' he said.

On Wednesday, UCSF Health said it had secured another two-day supply of
swabs.

Dr. Ulrike Sujansky, a physician in solo private practice in San Mateo,
Calif., one of the areas with the most cases, said she has only been
able to test a few patients because of problems with supplies.

``We cannot test. We cannot test,'' she said. The swab kits she ordered
from the two biggest lab companies, LabCorp and Quest Diagnostics,
either arrived late or were not the right type, she said. She also does
not have adequate masks or protective supplies, despite weeks of efforts
to purchase them and recent appeals to state and local health
authorities and hospitals. ``I'm not given any tools to deal with this
complete crisis.''

After a botched rollout of coronavirus tests in February, the Trump
administration has tried to quickly expand the number of tests available
to Americans by lowering regulatory hurdles and approving commercial
tests made by companies --- like Roche and Hologic --- whose machines
can run thousands of samples a day. Major lab companies like LabCorp and
Quest Diagnostics have been ramping up their capacity, as have hospital
labs around the country.

But shortages like those with the supply of swabs now threaten that
broader testing effort. Premier, which buys medical supplies on behalf
of many U.S. hospitals, said its members were also encountering
problems.

\includegraphics{https://static01.graylady3jvrrxbe.onion/images/2020/03/19/science/19VIRUS-TESTSHORTAGE2/merlin_170684235_9d8817a0-7e8d-47ea-aac3-b39d62fd1d5d-articleLarge.jpg?quality=75\&auto=webp\&disable=upscale}

``Our hospitals are eager to do their part to help expand access to
testing, but are struggling to do so absent necessary testing
supplies,'' said Soumi Saha, the senior director of advocacy at Premier.
She said that the company requested additional information from the
F.D.A. about how to address the swab shortage days ago, but has not
heard back.

On Wednesday, the Food and Drug Administration said in a statement, ``We
have heard concerns from labs who have questions about the availability
of certain supplies. We are updating frequently asked questions for labs
and test developers, providing information on alternative sources of
reagents, extraction kits, swabs and more.'' The agency said it also set
up a toll-free line, 1-888-INFO-FDA, to help labs with questions about
approvals or supplies.

\hypertarget{latest-updates-the-coronavirus-outbreak}{%
\section{\texorpdfstring{\href{https://www.nytimes3xbfgragh.onion/2020/09/11/world/covid-19-coronavirus.html?action=click\&pgtype=Article\&state=default\&region=MAIN_CONTENT_1\&context=storylines_live_updates}{Latest
Updates: The Coronavirus
Outbreak}}{Latest Updates: The Coronavirus Outbreak}}\label{latest-updates-the-coronavirus-outbreak}}

Updated 2020-09-12T04:56:54.924Z

\begin{itemize}
\tightlist
\item
  \href{https://www.nytimes3xbfgragh.onion/2020/09/11/world/covid-19-coronavirus.html?action=click\&pgtype=Article\&state=default\&region=MAIN_CONTENT_1\&context=storylines_live_updates\#link-dfb8a16}{Fauci
  cautions the virus could disrupt life in the U.S. until `maybe even
  towards the end of 2021.'}
\item
  \href{https://www.nytimes3xbfgragh.onion/2020/09/11/world/covid-19-coronavirus.html?action=click\&pgtype=Article\&state=default\&region=MAIN_CONTENT_1\&context=storylines_live_updates\#link-7104d154}{From
  Asia to Africa, China promotes its vaccine candidates to win friends.}
\item
  \href{https://www.nytimes3xbfgragh.onion/2020/09/11/world/covid-19-coronavirus.html?action=click\&pgtype=Article\&state=default\&region=MAIN_CONTENT_1\&context=storylines_live_updates\#link-393ad215}{The
  other way the virus will kill: hunger.}
\end{itemize}

\href{https://www.nytimes3xbfgragh.onion/2020/09/11/world/covid-19-coronavirus.html?action=click\&pgtype=Article\&state=default\&region=MAIN_CONTENT_1\&context=storylines_live_updates}{See
more updates}

More live coverage:
\href{https://www.nytimes3xbfgragh.onion/live/2020/09/11/business/stock-market-today-coronavirus?action=click\&pgtype=Article\&state=default\&region=MAIN_CONTENT_1\&context=storylines_live_updates}{Markets}

Gabriela Franco, a spokeswoman for Copan, which makes the testing swabs,
said the company has increased production at its plant in Brescia,
Italy, to 24 hours a day, seven days a week. ``We are asking our
customers and distributors to rationalize their ordering so we can
maximize throughout,'' she said, adding that in the United States, a
busy flu season had already depleted supply. The company has about half
of the market for testing swabs in the United States.

She said the lockdown in Italy --- which has been hit particularly hard
with more than 2,500 deaths --- --- had not affected the business or the
export of goods.

She said the company was taking precautions to protect its workers and
to keep production running. ``We are working with the Italian regional
and national authorities to preserve the current manufacturing
conditions in order to serve the world with our products, even in case
heavier restrictions are applied,'' she said.

The swab shortfall is just one example of the strain on the supply chain
amid global demand for testing for coronavirus. Testing
\href{https://www.nytimes3xbfgragh.onion/2020/03/11/health/coronavirus-testing-shortages.html}{has
also been hampered} by dwindling supplies of RNA extraction kits, which
are needed to extract the RNA from samples in order to run the tests in
many cases. In response, the F.D.A. has widened the number of extraction
products that can be used in the tests.

A shortage of masks, gowns and other protective gear ---
\href{https://www.nytimes3xbfgragh.onion/2020/03/09/health/coronavirus-n95-face-masks.html}{which
has put a strain} on many other areas of care --- is also getting in the
way of testing. Patients often sneeze or cough when the swab is inserted
deep into their noses, which can expose the workers who are doing the
tests to the virus.

``That's been the bottleneck for what I call the crisis within the
crisis,'' said Dr. Christopher Crow, the president of Catalyst Health
Network, who is setting up drive-through testing in locations around
North Texas. His network of more than 800 physician offices serves one
million people.

Until Tuesday, Dr. Crow was experiencing such a shortage of masks and
other gear that he was facing the possibility of closing the sites
Wednesday, just as he was preparing to vastly expand his capacity from
about 30 people a day.

"You can't go scuba diving without oxygen, a regulator and a mask,'' he
said. ``If you don't have that equipment to test, then you can't test.''

On Tuesday afternoon, he learned that he had been granted permission
\href{https://ncttrac.org/}{by a regional authority} to use a shipment
of gear from the national stockpile, solving his problem, for now.

Image

Aliesha O'Raw, a Ph.D. student in Durham, N.C., was refused a
coronavirus test because of a swab shortage, even though she was showing
symptoms.Credit...Phyllis B. Dooney for The New York Times

Some patients say the shortages have prevented them from getting tested,
even when they are showing symptoms of the coronavirus. Aliesha O'Raw, a
graduate student in Durham, N.C., said her primary care doctor sent her
to an urgent care clinic after she had a fever and a dry cough for two
weeks.

Ms. O'Raw, who is 26, has asthma and Ehlers-Danlos syndrome, an
inherited condition that affects the body's connective tissues, like
skin and joints. Both are underlying conditions that could make the
disease more severe. She tested negative for the flu, and antibiotics
did not work.

When she visited the urgent care clinic on Monday, the doctor declined
to test her because there was a shortage of swabs and they were
rationing, she said. Because she did not have known contact with an
exposed person, she was denied a test.

She would like to be tested ``not for my own knowledge,'' she said.
``It's that I know I have people around me who are starting to get
sick.'' Her boyfriend, who ran errands for her while she was ill, is now
sick, she said, as are some classmates who were around her before she
developed symptoms.

Her doctor ordered another test, and she was able to use a drive-through
testing site that just opened. She was tested on Wednesday and is
awaiting her results. ``It's happening,'' she said. ``It's just taking
longer than either of us expected.''

Sheri Fink contributed reporting.

\textbf{\emph{{[}}\href{http://on.fb.me/1paTQ1h}{\emph{Like the Science
Times page on Facebook.}}} ****** \emph{\textbar{} Sign up for the}
\textbf{\href{http://nyti.ms/1MbHaRU}{\emph{Science Times
newsletter.}}\emph{{]}}}

Advertisement

\protect\hyperlink{after-bottom}{Continue reading the main story}

\hypertarget{site-index}{%
\subsection{Site Index}\label{site-index}}

\hypertarget{site-information-navigation}{%
\subsection{Site Information
Navigation}\label{site-information-navigation}}

\begin{itemize}
\tightlist
\item
  \href{https://help.nytimes3xbfgragh.onion/hc/en-us/articles/115014792127-Copyright-notice}{©~2020~The
  New York Times Company}
\end{itemize}

\begin{itemize}
\tightlist
\item
  \href{https://www.nytco.com/}{NYTCo}
\item
  \href{https://help.nytimes3xbfgragh.onion/hc/en-us/articles/115015385887-Contact-Us}{Contact
  Us}
\item
  \href{https://www.nytco.com/careers/}{Work with us}
\item
  \href{https://nytmediakit.com/}{Advertise}
\item
  \href{http://www.tbrandstudio.com/}{T Brand Studio}
\item
  \href{https://www.nytimes3xbfgragh.onion/privacy/cookie-policy\#how-do-i-manage-trackers}{Your
  Ad Choices}
\item
  \href{https://www.nytimes3xbfgragh.onion/privacy}{Privacy}
\item
  \href{https://help.nytimes3xbfgragh.onion/hc/en-us/articles/115014893428-Terms-of-service}{Terms
  of Service}
\item
  \href{https://help.nytimes3xbfgragh.onion/hc/en-us/articles/115014893968-Terms-of-sale}{Terms
  of Sale}
\item
  \href{https://spiderbites.nytimes3xbfgragh.onion}{Site Map}
\item
  \href{https://help.nytimes3xbfgragh.onion/hc/en-us}{Help}
\item
  \href{https://www.nytimes3xbfgragh.onion/subscription?campaignId=37WXW}{Subscriptions}
\end{itemize}
