Sections

SEARCH

\protect\hyperlink{site-content}{Skip to
content}\protect\hyperlink{site-index}{Skip to site index}

\href{https://www.nytimes3xbfgragh.onion/section/world/asia}{Asia
Pacific}

\href{https://myaccount.nytimes3xbfgragh.onion/auth/login?response_type=cookie\&client_id=vi}{}

\href{https://www.nytimes3xbfgragh.onion/section/todayspaper}{Today's
Paper}

\href{/section/world/asia}{Asia Pacific}\textbar{}`None of Us Have a
Fear of Corona': The Faithful at an Outbreak's Center

\url{https://nyti.ms/3bbHcqv}

\begin{itemize}
\item
\item
\item
\item
\item
\item
\end{itemize}

\hypertarget{the-coronavirus-outbreak}{%
\subsubsection{\texorpdfstring{\href{https://www.nytimes3xbfgragh.onion/news-event/coronavirus?name=styln-coronavirus-national\&region=TOP_BANNER\&block=storyline_menu_recirc\&action=click\&pgtype=Article\&impression_id=df21ec90-f52c-11ea-868d-8bbf3a530245\&variant=undefined}{The
Coronavirus
Outbreak}}{The Coronavirus Outbreak}}\label{the-coronavirus-outbreak}}

\begin{itemize}
\tightlist
\item
  live\href{https://www.nytimes3xbfgragh.onion/2020/09/12/world/covid-19-coronavirus.html?name=styln-coronavirus-national\&region=TOP_BANNER\&block=storyline_menu_recirc\&action=click\&pgtype=Article\&impression_id=df21ec91-f52c-11ea-868d-8bbf3a530245\&variant=undefined}{Latest
  Updates}
\item
  \href{https://www.nytimes3xbfgragh.onion/interactive/2020/us/coronavirus-us-cases.html?name=styln-coronavirus-national\&region=TOP_BANNER\&block=storyline_menu_recirc\&action=click\&pgtype=Article\&impression_id=df2213a0-f52c-11ea-868d-8bbf3a530245\&variant=undefined}{Maps
  and Cases}
\item
  \href{https://www.nytimes3xbfgragh.onion/interactive/2020/science/coronavirus-vaccine-tracker.html?name=styln-coronavirus-national\&region=TOP_BANNER\&block=storyline_menu_recirc\&action=click\&pgtype=Article\&impression_id=df2213a1-f52c-11ea-868d-8bbf3a530245\&variant=undefined}{Vaccine
  Tracker}
\item
  \href{https://www.nytimes3xbfgragh.onion/2020/09/10/us/politics/fda-coronavirus-vaccine.html?name=styln-coronavirus-national\&region=TOP_BANNER\&block=storyline_menu_recirc\&action=click\&pgtype=Article\&impression_id=df2213a2-f52c-11ea-868d-8bbf3a530245\&variant=undefined}{F.D.A.
  Regulators' Self-Defense}
\item
  \href{https://www.nytimes3xbfgragh.onion/2020/09/09/upshot/coronavirus-surprise-test-fees.html?name=styln-coronavirus-national\&region=TOP_BANNER\&block=storyline_menu_recirc\&action=click\&pgtype=Article\&impression_id=df2213a3-f52c-11ea-868d-8bbf3a530245\&variant=undefined}{Surprise
  Test Fees}
\end{itemize}

Advertisement

\protect\hyperlink{after-top}{Continue reading the main story}

Supported by

\protect\hyperlink{after-sponsor}{Continue reading the main story}

\hypertarget{none-of-us-have-a-fear-of-corona-the-faithful-at-an-outbreaks-center}{%
\section{`None of Us Have a Fear of Corona': The Faithful at an
Outbreak's
Center}\label{none-of-us-have-a-fear-of-corona-the-faithful-at-an-outbreaks-center}}

A gathering of 16,000 at a Malaysian mosque became the pandemic's
largest known vector in Southeast Asia, spreading the coronavirus to
half a dozen countries.

\includegraphics{https://static01.graylady3jvrrxbe.onion/images/2020/03/20/world/20virus-mosque-1/20virus-mosque-1-articleLarge-v2.jpg?quality=75\&auto=webp\&disable=upscale}

\href{https://www.nytimes3xbfgragh.onion/by/hannah-beech}{\includegraphics{https://static01.graylady3jvrrxbe.onion/images/2018/10/08/multimedia/author-hannah-beech/author-hannah-beech-thumbLarge.png}}

By \href{https://www.nytimes3xbfgragh.onion/by/hannah-beech}{Hannah
Beech}

\begin{itemize}
\item
  March 20, 2020
\item
  \begin{itemize}
  \item
  \item
  \item
  \item
  \item
  \item
  \end{itemize}
\end{itemize}

BANGKOK --- The faithful prayed by the thousands, hands and faces washed
at communal taps to signify their purity. They crowded around platters
on the floor, scooping up coconut rice with their right hands in the
traditional way. And they slept in the mosque or in tents set up in the
religious compound, rows of pilgrims from nearly 30 countries, gathered
in Malaysia for spiritual renewal.

Three weeks later, participants in the 16,000-strong gathering of the
world's biggest Islamic missionary movement had spread the coronavirus
to half a dozen nations, creating the largest known viral vector in
Southeast Asia.

More than 620 people connected to the four-day conclave have tested
positive in Malaysia, prompting the country to seal its borders until
the end of the month. Most of the 73 coronavirus cases in Brunei are
tied to the gathering, as are 10 cases in Thailand. At least three
coronavirus deaths have been linked to the event.

``We talked about religious concepts and our faith in God, not the
coronavirus,'' said El Matly, a Cambodian seller of used phones who
participated in the conclave.

After returning home, Mr. El Matly and 22 other Cambodian pilgrims
tested positive. Two of their wives are also sick.

The outbreak underscores how
\href{https://www.nytimes3xbfgragh.onion/2020/03/19/world/asia/coronavirus-china-united-states.html}{the
pandemic's momentum has moved beyond China}, where the virus emerged.
And it has thrown a spotlight on Tablighi Jamaat, a century-old
missionary movement whose wandering bands of preachers depend on the
charity of mosques to shelter them.

Known in parts of Southeast Asia as Jemaah Tabligh, Tablighi Jamaat is
one of the largest faith-based movements in the world, according to the
Pew Research Center, and it counts politically influential people among
its followers.

The group's message: Return to the way life was during the time of the
Prophet Muhammad. Dress the same way, pray the same way. Some even
advocate using wooden sticks as toothbrushes. At conclaves attended by
thousands of people, Tablighi Jamaat's adherents pray and eat together
in mosques.

\includegraphics{https://static01.graylady3jvrrxbe.onion/images/2020/03/20/world/20virus-mosque-2/merlin_170759613_edf83395-ec49-440b-ab0a-563a432099c6-articleLarge.jpg?quality=75\&auto=webp\&disable=upscale}

Faith provides solace in uncertain times, but mass religious experiences
are also proving to be dangerous multipliers of the coronavirus.

\hypertarget{latest-updates-the-coronavirus-outbreak}{%
\section{\texorpdfstring{\href{https://www.nytimes3xbfgragh.onion/2020/09/11/world/covid-19-coronavirus.html?action=click\&pgtype=Article\&state=default\&region=MAIN_CONTENT_1\&context=storylines_live_updates}{Latest
Updates: The Coronavirus
Outbreak}}{Latest Updates: The Coronavirus Outbreak}}\label{latest-updates-the-coronavirus-outbreak}}

Updated 2020-09-12T12:04:20.515Z

\begin{itemize}
\tightlist
\item
  \href{https://www.nytimes3xbfgragh.onion/2020/09/11/world/covid-19-coronavirus.html?action=click\&pgtype=Article\&state=default\&region=MAIN_CONTENT_1\&context=storylines_live_updates\#link-dfb8a16}{Fauci
  cautions the virus could disrupt life in the U.S. until `maybe even
  towards the end of 2021.'}
\item
  \href{https://www.nytimes3xbfgragh.onion/2020/09/11/world/covid-19-coronavirus.html?action=click\&pgtype=Article\&state=default\&region=MAIN_CONTENT_1\&context=storylines_live_updates\#link-7104d154}{From
  Asia to Africa, China promotes its vaccine candidates to win friends.}
\item
  \href{https://www.nytimes3xbfgragh.onion/2020/09/11/world/covid-19-coronavirus.html?action=click\&pgtype=Article\&state=default\&region=MAIN_CONTENT_1\&context=storylines_live_updates\#link-393ad215}{The
  other way the virus will kill: hunger.}
\end{itemize}

\href{https://www.nytimes3xbfgragh.onion/2020/09/11/world/covid-19-coronavirus.html?action=click\&pgtype=Article\&state=default\&region=MAIN_CONTENT_1\&context=storylines_live_updates}{See
more updates}

More live coverage:
\href{https://www.nytimes3xbfgragh.onion/live/2020/09/11/business/stock-market-today-coronavirus?action=click\&pgtype=Article\&state=default\&region=MAIN_CONTENT_1\&context=storylines_live_updates}{Markets}

South Korea's outbreak, which for weeks was the worst outside China,
originated from a
\href{https://www.nytimes3xbfgragh.onion/2020/03/10/world/asia/south-korea-coronavirus-shincheonji.html}{secretive
sect} called the Shincheonji Church of Jesus. Thousands of the church's
followers and people who came into contact with them have tested
positive.

Two of the biggest
\href{https://www.nytimes3xbfgragh.onion/2020/03/17/world/asia/coronavirus-singapore-hong-kong-taiwan.html}{viral
clusters in Singapore} have been connected to churches. Pilgrims
returning from the holy city of Qom in Iran have spread the virus
through Central and South Asia.

And despite government warnings about the danger of convening large
groups of people during a pandemic, mass religious assemblies are
continuing. India is
\href{https://www.nytimes3xbfgragh.onion/2020/03/19/world/asia/coronavirus-india-socia-distancing.html}{urging
Hindu pilgrims not to come to the state of Uttar Pradesh} next week for
a nine-day celebration that hundreds of thousands had planned to attend.

Even as coronavirus cases from its Malaysia gathering surged, Tablighi
Jamaat had been planning another multiday event on the island of
Sulawesi in Indonesia, which was set to begin on Thursday and run
through this weekend.

After a public outcry, the Indonesian presidential spokesman announced
that the Sulawesi meeting was off, on the very morning it was to begin.
But nearly 8,700 worshipers from 10 countries had already congregated in
the town of Gowa, crowding into tents and sharing food, just like in
Malaysia.

``None of us have a fear of corona,'' said one of them, Roni Arif, the
head of a community health center in Mamuju, Sulawesi. ``We are afraid
of God.''

On March 2, Indonesia, the world's fourth most populous nation, had only
two confirmed coronavirus cases. As of Friday, that figure had risen to
369, with 32 deaths. A cabinet minister is among those who have tested
positive.

``All sickness and all health is from God,'' said Mr. Roni, who is
employed on the local level by the ministry of health. ``Whatever
happens to us is God's will.''

Image

Tablighi Jamaat devotees in Pakistan in 2009. The group calls for
returning to the way life was lived in the Prophet Muhammad's time.
~Credit...Tyler Hicks/The New York Times

Founded by an Islamic scholar in India in the 1920s, Tablighi Jamaat is
tight-lipped about its membership, but high-ranking politicians and
their relatives have been linked to the group. The son of Indonesia's
defense minister was a member. Western governments have tied Tablighi
Jamaat to the
\href{https://www.nytimes3xbfgragh.onion/2003/07/14/us/a-muslim-missionary-group-draws-new-scrutiny-in-us.html}{recruitment
of militants}, accusations its followers deny.

Mr. El Matly, the Cambodian who contracted the coronavirus in Malaysia,
said he had attended Tablighi events across Southeast Asia.

``I can afford it, and I think it's good if I spend my money on
religion,'' he said.

At the Malaysian assembly, Mr. El Matly slept in the main prayer hall,
where rows of mats were placed about a foot and a half apart, he said.
Others sheltered in tents that could hold up to 200 people.

``There was no announcement on virus protection,'' he said. ``I would
not have gone if I had known there was a virus there.''

The Tablighi conclave in multiethnic Malaysia took place as the country
was
\href{https://www.nytimes3xbfgragh.onion/2020/02/29/world/asia/malaysia-mahathir-mohamad.html}{locked
in a political showdown} between a reformist, multiracial bloc and a
conservative faction looking to increase the power of Malay Muslims. By
the final day of the Tablighi gathering, a new prime minister had been
sworn in, backed by a coalition including an Islamic party that has
campaigned to turn Malaysia into an Islamic state.

As Malaysian health officials tried to track the spread of the
coronavirus through the region, others in the political establishment
were preoccupied with the unfolding leadership crisis. Tablighi
participants were getting sick, but members of the new Malaysian
government focused on damage control.

``The likelihood of dying from the coronavirus is only 1 percent, while
the possibility of dying at any moment is 100 percent,'' wrote Siti
Zailah Mohd Yusoff, the deputy minister for women and family
development, on Twitter.

The health ministry initially said that 5,000 Malaysians had attended
the Tablighi conference. Days later, the number was revised to about
14,500 Malaysians and 1,500 foreigners.

Among the devotees were hundreds of Rohingya Muslims from Myanmar, who
had escaped persecution back home for lives as undocumented workers in
Malaysia. Locating them is proving to be difficult.

With its viral caseload proliferating in the wake of the Tablighi
gathering, Malaysia on Wednesday closed its borders to nearly all
travelers until March 31. With rare exceptions, no Malaysians are
allowed to leave the country and no foreigners are allowed to enter.
Only essential businesses can stay open. Mosques are closed for Friday
prayers.

On Thursday, the Ministry of Foreign Affairs said the 83 Malaysians who
had gone to Indonesia for the Tablighi event there would be allowed to
come home and would be screened for the coronavirus.

Tablighi followers had been streaming into Gowa, the town where the
Indonesian gathering was to take place, for days. The abrupt
cancellation has left many of them stranded.

Nurdin Abdullah, the governor of South Sulawesi, said that all
foreigners who had gathered in Gowa would be isolated in a hotel and
escorted to the airport. In the meantime, Indonesians were still passing
the time in tents on Thursday, reading religious texts and discussing
theology with scholars.

``It's not reckless for us to have come here and gathered in big
groups,'' said Ilman Murgan, a farmer. ``It's important for us to learn
how to draw ourselves closer to God.''

On Thursday, another religious assembly, this one involving Catholics,
took place on the island of Flores, further east in Indonesia. About
2,000 people, including nuns in starched habits, squeezed into a church
to celebrate the ordination of a bishop.

The Indonesian minister of information had flown in to join the
celebration. But he left after the National Disaster Mitigation Agency
advised on Thursday morning that it, like other large social gatherings,
should be canceled. The cardinal went ahead with the event.

Hans Jeharut, a priest who attended the four-hour ordination, said that
at least 30 bishops were among the congregants. Participants'
temperatures were taken twice. There had been no bishop in the area for
more than two years, Mr. Jeharut said, and canceling the celebration
would have disappointed the diocese.

``The people's euphoria has to be understood,'' he said. ``Yes, this was
a celebration. But it was a celebration of faith.''

Muktita Suhartono contributed reporting from Bangkok and Sun Narin from
Phnom Penh, Cambodia.

Advertisement

\protect\hyperlink{after-bottom}{Continue reading the main story}

\hypertarget{site-index}{%
\subsection{Site Index}\label{site-index}}

\hypertarget{site-information-navigation}{%
\subsection{Site Information
Navigation}\label{site-information-navigation}}

\begin{itemize}
\tightlist
\item
  \href{https://help.nytimes3xbfgragh.onion/hc/en-us/articles/115014792127-Copyright-notice}{©~2020~The
  New York Times Company}
\end{itemize}

\begin{itemize}
\tightlist
\item
  \href{https://www.nytco.com/}{NYTCo}
\item
  \href{https://help.nytimes3xbfgragh.onion/hc/en-us/articles/115015385887-Contact-Us}{Contact
  Us}
\item
  \href{https://www.nytco.com/careers/}{Work with us}
\item
  \href{https://nytmediakit.com/}{Advertise}
\item
  \href{http://www.tbrandstudio.com/}{T Brand Studio}
\item
  \href{https://www.nytimes3xbfgragh.onion/privacy/cookie-policy\#how-do-i-manage-trackers}{Your
  Ad Choices}
\item
  \href{https://www.nytimes3xbfgragh.onion/privacy}{Privacy}
\item
  \href{https://help.nytimes3xbfgragh.onion/hc/en-us/articles/115014893428-Terms-of-service}{Terms
  of Service}
\item
  \href{https://help.nytimes3xbfgragh.onion/hc/en-us/articles/115014893968-Terms-of-sale}{Terms
  of Sale}
\item
  \href{https://spiderbites.nytimes3xbfgragh.onion}{Site Map}
\item
  \href{https://help.nytimes3xbfgragh.onion/hc/en-us}{Help}
\item
  \href{https://www.nytimes3xbfgragh.onion/subscription?campaignId=37WXW}{Subscriptions}
\end{itemize}
