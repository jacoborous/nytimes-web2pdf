Sections

SEARCH

\protect\hyperlink{site-content}{Skip to
content}\protect\hyperlink{site-index}{Skip to site index}

\href{https://www.nytimes3xbfgragh.onion/section/arts}{Arts}

\href{https://myaccount.nytimes3xbfgragh.onion/auth/login?response_type=cookie\&client_id=vi}{}

\href{https://www.nytimes3xbfgragh.onion/section/todayspaper}{Today's
Paper}

\href{/section/arts}{Arts}\textbar{}For Future Generations, It's Time to
Reflect on Black Art

\url{https://nyti.ms/38XcP5z}

\begin{itemize}
\item
\item
\item
\item
\item
\end{itemize}

Advertisement

\protect\hyperlink{after-top}{Continue reading the main story}

Supported by

\protect\hyperlink{after-sponsor}{Continue reading the main story}

\hypertarget{for-future-generations-its-time-to-reflect-on-black-art}{%
\section{For Future Generations, It's Time to Reflect on Black
Art}\label{for-future-generations-its-time-to-reflect-on-black-art}}

Shifts in politics, performance and protest have all altered our culture
in a way not seen in years.

\includegraphics{https://static01.graylady3jvrrxbe.onion/images/2020/03/29/arts/29black-creators-promo/29black-creators-promo-articleLarge-v3.jpg?quality=75\&auto=webp\&disable=upscale}

By Eric V. Copage

\begin{itemize}
\item
  March 19, 2020
\item
  \begin{itemize}
  \item
  \item
  \item
  \item
  \item
  \end{itemize}
\end{itemize}

I grew up in Southern California neighborhoods that were overwhelmingly
white --- white schools, white teachers, white curriculum. I remember
visiting the Los Angeles County Museum of Art where a cast of
\href{https://collections.lacma.org/node/251434}{Auguste Rodin's
monumental sculpture of Balzac} towered over me. My friends and I
immersed ourselves in the works of white authors, and I can still recite
vast sections of T.S. Eliot's ``The Hollow Men,'' William Butler Yeats's
``The Second Coming,'' and, if I squeeze my eyes real tight, the final
luxurious paragraphs of ``The Dead,'' James Joyce's masterful short
story.

But I am black --- African-American, if you will --- and unambiguously
of African heritage. Thanks to my father, the rhythms and melodies of
Charlie Parker, Dizzy Gillespie, Miles Davis and John Coltrane flowed
through our household during my childhood. Yet, as an adolescent, Led
Zeppelin, Cream and the Who were favored by my friends and me. (My
enthusiasm for bluesmen such as Sonny Terry, Brownie McGhee, Howlin'
Wolf and Muddy Waters was rooted in my rock 'n' roll fandom.) In short,
I felt only incidentally ``black,'' despite being harassed by the local
police, who frequently stopped me as I walked home from school or rode
my bike to the grocery store.

That feeling slowly changed, of course, as I became an adult. But I
didn't feel I became fully black until the birth of my first child. His
arrival brought into focus the usual parental responsibilities of
providing food, shelter and moral support. But I also felt a special
duty to imbue my son --- and later my daughter --- with a sense of what
it means to be of African heritage in America. Not ``the talk'' about
navigating encounters with the police. I felt the need to address
meta-questions pertaining to the frame of mind that would help my
children prosper in the world, not only survive.

So, I
\href{https://www.amazon.com/Fruits-Harvest-Recipes-Celebrate-Holidays-ebook/dp/B003YCOP98}{wrote
a cookbook} whose recipes from the African diaspora were anchored by
stories from the black cooks who contributed them, as well as the wit
and wisdom from black cultures around the world and across time. Then
came
\href{https://www.amazon.com/Black-Pearls-Affirmations-Inspirations-African-Americans/dp/0688122914}{a
series of what I call ``affirmative action'' books} --- collections of
pithy proverbs and quotes from prominent historical and contemporary
black people meant to inspire the desire to implement concrete positive
action.

More recently, my thoughts turned to internalizing black culture in a
more formal way, especially after seeing Jacob Lawrence's ``The
Migration Series'' at the Museum of Modern Art in 2015. The 60 small
panel paintings
\href{https://www.khanacademy.org/humanities/ap-art-history/later-europe-and-americas/modernity-ap/v/lawrence-migration-series}{illustrating
the movement of black Southerners to the North} had affected me
tremendously. The formal beauty of the work and the message of black
agency felt like an oasis of pride and inspiration in a world of stop
and frisks, police killings of black men and women, and the general
denigration of ``black culture.''

Then, in 2018, there was ``Black Panther,''
\href{https://www.nytimes3xbfgragh.onion/2018/02/12/movies/black-panther-marvel-chadwick-boseman-ryan-coogler-lupita-nyongo.html}{the
first major superhero movie with an African protagonist};
\href{https://www.nytimes3xbfgragh.onion/2018/04/15/arts/music/beyonce-coachella-review.html}{Beyoncé's
Coachella performance}, during which she presented over a century of
black musical traditions; and, performing as Childish Gambino, Donald
Glover's violent, darkly comic video,
\href{https://www.youtube.com/watch?v=VYOjWnS4cMY}{``This Is America.''}
These works seemed to amount to encyclopedias of black cultural
touchstones, resources of strength in racially fraught times.

I began to wonder: What particular pieces of music, film, sculpture,
dance or other artistic expression do some of today's best and brightest
black creators find important? What aspects of the kaleidoscopic
collective black soul as expressed in the arts would they want to
emphasize and remember 25 or 50 years from now? Should such a list
include Public Enemy's 1990 album ``Fear of a Black Planet''? The ``Wade
in the Water'' segment
of\href{http://pressroom.alvinailey.org/file/wade-in-the-water-from-alvin-ailey-s-revelations?action=}{Alvin
Ailey's ``Revelations''}?

That's a lot to bite off when talking about centuries of global black
artistic expression, but what about narrowing it to black arts created
over these first two decades of the 21st century? Shifts in politics,
performance and protest have all altered our culture in a way not seen
in years. The beauty of this exercise in reflection is not only to
celebrate black cultural contributions to art but also record a pivotal
time for our country --- indeed the world. It's a starting place for
further exploration for my 18-month-old granddaughter and many future
generations after hers.

To play on the words, but not the meaning, of John Milton's ``Paradise
Lost'': Such a survey would be the latest hopeful sally to make darkness
not just visible, but aspirational. Works that bear witness to what
black people went through and still endure; works that reveal past and
continuing struggles and successes.

Lead image: Photograph by Jessica Pettway for The New York Times; Damon
Winter/The New York Times (Toni Morrison); Frederic J. Brown/Agence
France-Presse --- Getty Images (Donald Glover); Rozette Rago for The New
York Times (Issa Rae)

Advertisement

\protect\hyperlink{after-bottom}{Continue reading the main story}

\hypertarget{site-index}{%
\subsection{Site Index}\label{site-index}}

\hypertarget{site-information-navigation}{%
\subsection{Site Information
Navigation}\label{site-information-navigation}}

\begin{itemize}
\tightlist
\item
  \href{https://help.nytimes3xbfgragh.onion/hc/en-us/articles/115014792127-Copyright-notice}{©~2020~The
  New York Times Company}
\end{itemize}

\begin{itemize}
\tightlist
\item
  \href{https://www.nytco.com/}{NYTCo}
\item
  \href{https://help.nytimes3xbfgragh.onion/hc/en-us/articles/115015385887-Contact-Us}{Contact
  Us}
\item
  \href{https://www.nytco.com/careers/}{Work with us}
\item
  \href{https://nytmediakit.com/}{Advertise}
\item
  \href{http://www.tbrandstudio.com/}{T Brand Studio}
\item
  \href{https://www.nytimes3xbfgragh.onion/privacy/cookie-policy\#how-do-i-manage-trackers}{Your
  Ad Choices}
\item
  \href{https://www.nytimes3xbfgragh.onion/privacy}{Privacy}
\item
  \href{https://help.nytimes3xbfgragh.onion/hc/en-us/articles/115014893428-Terms-of-service}{Terms
  of Service}
\item
  \href{https://help.nytimes3xbfgragh.onion/hc/en-us/articles/115014893968-Terms-of-sale}{Terms
  of Sale}
\item
  \href{https://spiderbites.nytimes3xbfgragh.onion}{Site Map}
\item
  \href{https://help.nytimes3xbfgragh.onion/hc/en-us}{Help}
\item
  \href{https://www.nytimes3xbfgragh.onion/subscription?campaignId=37WXW}{Subscriptions}
\end{itemize}
