Sections

SEARCH

\protect\hyperlink{site-content}{Skip to
content}\protect\hyperlink{site-index}{Skip to site index}

\href{https://www.nytimes3xbfgragh.onion/section/politics}{Politics}

\href{https://myaccount.nytimes3xbfgragh.onion/auth/login?response_type=cookie\&client_id=vi}{}

\href{https://www.nytimes3xbfgragh.onion/section/todayspaper}{Today's
Paper}

\href{/section/politics}{Politics}\textbar{}Senate Rescue Package
Includes Corporate Tax Cuts and \$1,200 Checks

\url{https://nyti.ms/2WE44v1}

\begin{itemize}
\item
\item
\item
\item
\item
\item
\end{itemize}

\hypertarget{the-coronavirus-outbreak}{%
\subsubsection{\texorpdfstring{\href{https://www.nytimes3xbfgragh.onion/news-event/coronavirus?name=styln-coronavirus-national\&region=TOP_BANNER\&block=storyline_menu_recirc\&action=click\&pgtype=Article\&impression_id=2ff65570-f280-11ea-bf3d-b99d4e75db11\&variant=undefined}{The
Coronavirus
Outbreak}}{The Coronavirus Outbreak}}\label{the-coronavirus-outbreak}}

\begin{itemize}
\tightlist
\item
  live\href{https://www.nytimes3xbfgragh.onion/2020/09/09/world/covid-19-coronavirus.html?name=styln-coronavirus-national\&region=TOP_BANNER\&block=storyline_menu_recirc\&action=click\&pgtype=Article\&impression_id=2ff67c80-f280-11ea-bf3d-b99d4e75db11\&variant=undefined}{Latest
  Updates}
\item
  \href{https://www.nytimes3xbfgragh.onion/interactive/2020/us/coronavirus-us-cases.html?name=styln-coronavirus-national\&region=TOP_BANNER\&block=storyline_menu_recirc\&action=click\&pgtype=Article\&impression_id=2ff67c81-f280-11ea-bf3d-b99d4e75db11\&variant=undefined}{Maps
  and Cases}
\item
  \href{https://www.nytimes3xbfgragh.onion/interactive/2020/science/coronavirus-vaccine-tracker.html?name=styln-coronavirus-national\&region=TOP_BANNER\&block=storyline_menu_recirc\&action=click\&pgtype=Article\&impression_id=2ff67c82-f280-11ea-bf3d-b99d4e75db11\&variant=undefined}{Vaccine
  Tracker}
\item
  \href{https://www.nytimes3xbfgragh.onion/2020/09/02/your-money/eviction-moratorium-covid.html?name=styln-coronavirus-national\&region=TOP_BANNER\&block=storyline_menu_recirc\&action=click\&pgtype=Article\&impression_id=2ff67c83-f280-11ea-bf3d-b99d4e75db11\&variant=undefined}{Eviction
  Moratorium}
\item
  \href{https://www.nytimes3xbfgragh.onion/interactive/2020/09/02/magazine/food-insecurity-hunger-us.html?name=styln-coronavirus-national\&region=TOP_BANNER\&block=storyline_menu_recirc\&action=click\&pgtype=Article\&impression_id=2ff67c84-f280-11ea-bf3d-b99d4e75db11\&variant=undefined}{American
  Hunger}
\end{itemize}

Advertisement

\protect\hyperlink{after-top}{Continue reading the main story}

Supported by

\protect\hyperlink{after-sponsor}{Continue reading the main story}

\hypertarget{senate-rescue-package-includes-corporate-tax-cuts-and-1200-checks}{%
\section{Senate Rescue Package Includes Corporate Tax Cuts and \$1,200
Checks}\label{senate-rescue-package-includes-corporate-tax-cuts-and-1200-checks}}

Republicans sought to put their imprint on an emerging economic
stabilization package of \$1 trillion, unveiling a bill that includes
tax cuts and loans to businesses and would curtail paid leave.

\includegraphics{https://static01.graylady3jvrrxbe.onion/images/2020/03/19/us/politics/19dc-virus-cong1/19dc-virus-cong1-articleLarge-v2.jpg?quality=75\&auto=webp\&disable=upscale}

\href{https://www.nytimes3xbfgragh.onion/by/emily-cochrane}{\includegraphics{https://static01.graylady3jvrrxbe.onion/images/2018/11/28/multimedia/author-emily-cochrane/author-emily-cochrane-thumbLarge-v3.png}}\href{https://www.nytimes3xbfgragh.onion/by/jim-tankersley}{\includegraphics{https://static01.graylady3jvrrxbe.onion/images/2018/10/19/multimedia/author-jim-tankersley/author-jim-tankersley-thumbLarge.png}}\href{https://www.nytimes3xbfgragh.onion/by/alan-rappeport}{\includegraphics{https://static01.graylady3jvrrxbe.onion/images/2018/06/12/multimedia/author-alan-rappeport/author-alan-rappeport-thumbLarge-v2.png}}

By \href{https://www.nytimes3xbfgragh.onion/by/emily-cochrane}{Emily
Cochrane},
\href{https://www.nytimes3xbfgragh.onion/by/jim-tankersley}{Jim
Tankersley} and
\href{https://www.nytimes3xbfgragh.onion/by/alan-rappeport}{Alan
Rappeport}

\begin{itemize}
\item
  March 19, 2020
\item
  \begin{itemize}
  \item
  \item
  \item
  \item
  \item
  \item
  \end{itemize}
\end{itemize}

WASHINGTON --- The White House and lawmakers scrambled on Thursday to
flesh out details of a \$1 trillion economic stabilization plan to help
workers and businesses weather a potentially deep recession, negotiating
over the size and scope of direct payments to millions of people and aid
for companies facing devastation in the coronavirus pandemic.

Senate Republicans, racing to put their imprint on the crisis response,
unveiled a package that would provide hundreds of billions of dollars in
loans to big corporations and small businesses, large corporate tax cuts
and checks of up to \$1,200 for taxpayers. The plan would also place
limits on a paid-leave program enacted this week to respond to the
crisis.

But the 247-page measure, the product of a feverish round of
negotiations among Republicans, was all but certain to face opposition
from Democrats who have pressed for more generous paid-leave benefits
and targeting help to workers and families rather than large
corporations.

The details emerged as Washington grappled with the dimensions of an
extraordinary government rescue effort that is likely to last for many
months. At the White House, President Trump said he would be open to
having the government take equity stakes in companies that require
federal help, a move that would be unpopular with shareholders and would
give the government more oversight over businesses.

But he also injected new uncertainty into the government's response,
suggesting it was not his responsibility to meet the needs of health
care workers on the front lines of combating the disease. A day after he
said he would use the
\href{https://www.nytimes3xbfgragh.onion/2020/02/28/us/politics/trump-coronavirus.html}{Defense
Production Act} --- a Korean War-era law that allows presidents to force
American industry to ramp up production of critical equipment and
supplies --- Mr. Trump told reporters that he would rather rely on
states to deliver equipment to health care workers.

``The federal government's not supposed to be out there buying vast
amounts of items and then shipping,'' Mr. Trump said. ``You know, we're
not a shipping clerk,'' he said, adding that governors ``are supposed to
be doing it.''

On Capitol Hill, Republicans presented a bill that would offer bridge
loans of up to \$10 million each to small businesses, extend hundreds of
billions of dollars in loans to large corporations in distressed
industries and send checks as large as \$1,200 per adult to individuals
earning less than \$99,000 per year. The payments would phase in for
earners up to \$75,000 --- meaning lower earners would get smaller
checks --- and then phase out again at \$99,000. Those who did not earn
enough to pay income tax would receive much less: \$600.

The proposal is different from one pitched on Thursday by Steven
Mnuchin, the Treasury secretary, who said the administration wanted to
send two waves of \$1,000 checks to every American, one in April and one
in May should the crisis persist.

The Senate bill also includes a raft of temporary changes to the tax
code that would reduce the tax liability of large corporations, many of
them overriding provisions in the 2017 tax overhaul that were meant to
raise revenue to offset corporate rate cuts.

It would place new limits on a paid-leave program that Congress passed
and Mr. Trump signed into law this week, shielding small business owners
from any costs of paid leave for workers affected by the virus --- and
limiting how much pay those workers could receive if they are forced to
stay home.

\hypertarget{latest-updates-the-coronavirus-outbreak}{%
\section{\texorpdfstring{\href{https://www.nytimes3xbfgragh.onion/2020/09/09/world/covid-19-coronavirus.html?action=click\&pgtype=Article\&state=default\&region=MAIN_CONTENT_1\&context=storylines_live_updates}{Latest
Updates: The Coronavirus
Outbreak}}{Latest Updates: The Coronavirus Outbreak}}\label{latest-updates-the-coronavirus-outbreak}}

Updated 2020-09-09T09:25:52.418Z

\begin{itemize}
\tightlist
\item
  \href{https://www.nytimes3xbfgragh.onion/2020/09/09/world/covid-19-coronavirus.html?action=click\&pgtype=Article\&state=default\&region=MAIN_CONTENT_1\&context=storylines_live_updates\#link-70cea8bb}{As
  drugmakers pledge to thoroughly vet a vaccine, one company pauses its
  trials for a safety review.}
\item
  \href{https://www.nytimes3xbfgragh.onion/2020/09/09/world/covid-19-coronavirus.html?action=click\&pgtype=Article\&state=default\&region=MAIN_CONTENT_1\&context=storylines_live_updates\#link-4438dd7}{Facing
  a surge in cases, Britain plans to limit most gatherings to six
  people.}
\item
  \href{https://www.nytimes3xbfgragh.onion/2020/09/09/world/covid-19-coronavirus.html?action=click\&pgtype=Article\&state=default\&region=MAIN_CONTENT_1\&context=storylines_live_updates\#link-11cec4c0}{Quarantine
  breakdowns at colleges in the U.S. are leaving some at risk.}
\end{itemize}

\href{https://www.nytimes3xbfgragh.onion/2020/09/09/world/covid-19-coronavirus.html?action=click\&pgtype=Article\&state=default\&region=MAIN_CONTENT_1\&context=storylines_live_updates}{See
more updates}

More live coverage:
\href{https://www.nytimes3xbfgragh.onion/live/2020/09/08/business/stock-market-today-coronavirus?action=click\&pgtype=Article\&state=default\&region=MAIN_CONTENT_1\&context=storylines_live_updates}{Markets}

A preliminary estimate by the Tax Foundation in Washington shows that
the bill would cost at least \$1 trillion. The small-business loans
alone would be \$300 billion, according to documents circulated on
Thursday by Senate Republicans.

Mr. McConnell called for negotiations to begin with Senate Democrats on
Friday, adding that Mr. Mnuchin, Larry Kudlow, the director of the
National Economic Council, and Eric M. Ueland, the administration's
director of legislative affairs, would join them on behalf of the
administration.

\includegraphics{https://static01.graylady3jvrrxbe.onion/images/2020/03/19/us/politics/19dc-virus-cong2/merlin_170722803_7c57fe84-f807-4369-bb89-771684ec2b1e-articleLarge.jpg?quality=75\&auto=webp\&disable=upscale}

But little more than an hour after its release, the top House and Senate
Democrats indicated that the legislation did not meet their standards.

``We are beginning to review Senator McConnell's proposal, and on first
reading, it is not at all pro-worker and instead puts corporations way
ahead of workers,'' Speaker Nancy Pelosi and Senator Chuck Schumer of
New York, the minority leader, said in a joint statement.

Calling the legislation a ``significant next step,'' Mr. McConnell vowed
that the Senate would not leave until the rescue package had been
approved, and said there could be a possible fourth relief package to
follow as Congress seeks to address an extraordinary series of events.

But several of the Republican proposals are likely to be nonstarters for
Democrats in both chambers, whose support is needed in order for the
package to become law.

``Hopefully each side will give,'' said Mr. Schumer of New York, the
minority leader, speaking minutes after Mr. McConnell introduced the
proposal. ``We'll come up with a good plan, we'll send it to the
president, and we will help to begin the long path to eradicate this
awful virus.''

House Democrats, scattered across the country during an indefinite
recess, have also been exchanging their own proposals over conference
calls from their districts. Ms. Pelosi, who has spoken repeatedly with
Mr. Mnuchin in recent days, has said her committee leaders were
discussing the possibility of expanding unemployment insurance
eligibility, refundable tax credits and funds for small businesses to
ensure that workers continue to be paid.

One of the sticking points for members of both parties was the scope of
the direct payment program, with some Republican senators pushing to
strengthen unemployment insurance and loans for small businesses.

``Direct payments make sense when the economy is beginning to restart
--- it makes no sense now because it's just money,'' Senator Lindsey
Graham, Republican of South Carolina, told reporters. ``What I want is
income, not one check. I want you to get a check you can count on every
week, not one week.''

Others said it should be structured so that the lowest earners got the
most help --- not the other way around.

``Relief to families in this emergency shouldn't be regressive,''
Senator Josh Hawley, Republican of Missouri, said in a
\href{https://twitter.com/HawleyMO/status/1240769223267356673}{tweet}.
``Lower-income families shouldn't be penalized.''

While there is general agreement about the need to speed economic help
to millions of Americans, House Democrats are also debating the contours
of their own proposal, including how to target the direct payments and
the level of government intervention. During a private conference call
on Thursday, they debated where to set the income cap on individuals who
could receive government checks, according to three people familiar with
the discussion who insisted on anonymity to describe the private call.

Representative Richard E. Neal of Massachusetts, the Democratic chairman
of the Ways and Means Committee, which would have some jurisdiction over
the issue, suggested capping the direct payments to individuals with
incomes between \$50,000 and \$75,000, according to two people familiar
with the discussion. Other lawmakers advocated raising the limit to
individual incomes of \$130,000, while others suggested universal
payments.

Image

The Union Square green market in Manhattan. Some Democrats said the
government should intervene more directly and take on payroll or other
expenses for small businesses.Credit...Hiroko Masuike/The New York Times

And some Democrats believe the government should intervene more directly
and take on payroll or other expenses for small businesses, arguing that
could be a more targeted and effective way to keep them afloat and
people employed.

\href{https://www.nytimes3xbfgragh.onion/news-event/coronavirus?action=click\&pgtype=Article\&state=default\&region=MAIN_CONTENT_3\&context=storylines_faq}{}

\hypertarget{the-coronavirus-outbreak-}{%
\subsubsection{The Coronavirus Outbreak
›}\label{the-coronavirus-outbreak-}}

\hypertarget{frequently-asked-questions}{%
\paragraph{Frequently Asked
Questions}\label{frequently-asked-questions}}

Updated September 4, 2020

\begin{itemize}
\item ~
  \hypertarget{what-are-the-symptoms-of-coronavirus}{%
  \paragraph{What are the symptoms of
  coronavirus?}\label{what-are-the-symptoms-of-coronavirus}}

  \begin{itemize}
  \tightlist
  \item
    In the beginning, the coronavirus
    \href{https://www.nytimes3xbfgragh.onion/article/coronavirus-facts-history.html?action=click\&pgtype=Article\&state=default\&region=MAIN_CONTENT_3\&context=storylines_faq\#link-6817bab5}{seemed
    like it was primarily a respiratory illness}~--- many patients had
    fever and chills, were weak and tired, and coughed a lot, though
    some people don't show many symptoms at all. Those who seemed
    sickest had pneumonia or acute respiratory distress syndrome and
    received supplemental oxygen. By now, doctors have identified many
    more symptoms and syndromes. In April,
    \href{https://www.nytimes3xbfgragh.onion/2020/04/27/health/coronavirus-symptoms-cdc.html?action=click\&pgtype=Article\&state=default\&region=MAIN_CONTENT_3\&context=storylines_faq}{the
    C.D.C. added to the list of early signs}~sore throat, fever, chills
    and muscle aches. Gastrointestinal upset, such as diarrhea and
    nausea, has also been observed. Another telltale sign of infection
    may be a sudden, profound diminution of one's
    \href{https://www.nytimes3xbfgragh.onion/2020/03/22/health/coronavirus-symptoms-smell-taste.html?action=click\&pgtype=Article\&state=default\&region=MAIN_CONTENT_3\&context=storylines_faq}{sense
    of smell and taste.}~Teenagers and young adults in some cases have
    developed painful red and purple lesions on their fingers and toes
    --- nicknamed ``Covid toe'' --- but few other serious symptoms.
  \end{itemize}
\item ~
  \hypertarget{why-is-it-safer-to-spend-time-together-outside}{%
  \paragraph{Why is it safer to spend time together
  outside?}\label{why-is-it-safer-to-spend-time-together-outside}}

  \begin{itemize}
  \tightlist
  \item
    \href{https://www.nytimes3xbfgragh.onion/2020/05/15/us/coronavirus-what-to-do-outside.html?action=click\&pgtype=Article\&state=default\&region=MAIN_CONTENT_3\&context=storylines_faq}{Outdoor
    gatherings}~lower risk because wind disperses viral droplets, and
    sunlight can kill some of the virus. Open spaces prevent the virus
    from building up in concentrated amounts and being inhaled, which
    can happen when infected people exhale in a confined space for long
    stretches of time, said Dr. Julian W. Tang, a virologist at the
    University of Leicester.
  \end{itemize}
\item ~
  \hypertarget{why-does-standing-six-feet-away-from-others-help}{%
  \paragraph{Why does standing six feet away from others
  help?}\label{why-does-standing-six-feet-away-from-others-help}}

  \begin{itemize}
  \tightlist
  \item
    The coronavirus spreads primarily through droplets from your mouth
    and nose, especially when you cough or sneeze. The C.D.C., one of
    the organizations using that measure,
    \href{https://www.nytimes3xbfgragh.onion/2020/04/14/health/coronavirus-six-feet.html?action=click\&pgtype=Article\&state=default\&region=MAIN_CONTENT_3\&context=storylines_faq}{bases
    its recommendation of six feet}~on the idea that most large droplets
    that people expel when they cough or sneeze will fall to the ground
    within six feet. But six feet has never been a magic number that
    guarantees complete protection. Sneezes, for instance, can launch
    droplets a lot farther than six feet,
    \href{https://jamanetwork.com/journals/jama/fullarticle/2763852}{according
    to a recent study}. It's a rule of thumb: You should be safest
    standing six feet apart outside, especially when it's windy. But
    keep a mask on at all times, even when you think you're far enough
    apart.
  \end{itemize}
\item ~
  \hypertarget{i-have-antibodies-am-i-now-immune}{%
  \paragraph{I have antibodies. Am I now
  immune?}\label{i-have-antibodies-am-i-now-immune}}

  \begin{itemize}
  \tightlist
  \item
    As of right
    now,\href{https://www.nytimes3xbfgragh.onion/2020/07/22/health/covid-antibodies-herd-immunity.html?action=click\&pgtype=Article\&state=default\&region=MAIN_CONTENT_3\&context=storylines_faq}{~that
    seems likely, for at least several months.}~There have been
    frightening accounts of people suffering what seems to be a second
    bout of Covid-19. But experts say these patients may have a
    drawn-out course of infection, with the virus taking a slow toll
    weeks to months after initial exposure.~People infected with the
    coronavirus typically
    \href{https://www.nature.com/articles/s41586-020-2456-9}{produce}~immune
    molecules called antibodies, which are
    \href{https://www.nytimes3xbfgragh.onion/2020/05/07/health/coronavirus-antibody-prevalence.html?action=click\&pgtype=Article\&state=default\&region=MAIN_CONTENT_3\&context=storylines_faq}{protective
    proteins made in response to an
    infection}\href{https://www.nytimes3xbfgragh.onion/2020/05/07/health/coronavirus-antibody-prevalence.html?action=click\&pgtype=Article\&state=default\&region=MAIN_CONTENT_3\&context=storylines_faq}{.
    These antibodies may}~last in the body
    \href{https://www.nature.com/articles/s41591-020-0965-6}{only two to
    three months}, which may seem worrisome, but that's~perfectly normal
    after an acute infection subsides, said Dr. Michael Mina, an
    immunologist at Harvard University. It may be possible to get the
    coronavirus again, but it's highly unlikely that it would be
    possible in a short window of time from initial infection or make
    people sicker the second time.
  \end{itemize}
\item ~
  \hypertarget{what-are-my-rights-if-i-am-worried-about-going-back-to-work}{%
  \paragraph{What are my rights if I am worried about going back to
  work?}\label{what-are-my-rights-if-i-am-worried-about-going-back-to-work}}

  \begin{itemize}
  \tightlist
  \item
    Employers have to provide
    \href{https://www.osha.gov/SLTC/covid-19/standards.html}{a safe
    workplace}~with policies that protect everyone equally.
    \href{https://www.nytimes3xbfgragh.onion/article/coronavirus-money-unemployment.html?action=click\&pgtype=Article\&state=default\&region=MAIN_CONTENT_3\&context=storylines_faq}{And
    if one of your co-workers tests positive for the coronavirus, the
    C.D.C.}~has said that
    \href{https://www.cdc.gov/coronavirus/2019-ncov/community/guidance-business-response.html}{employers
    should tell their employees}~-\/- without giving you the sick
    employee's name -\/- that they may have been exposed to the virus.
  \end{itemize}
\end{itemize}

Ms. Pelosi has said publicly and privately that she will consider
including provisions that would expand eligibility for unemployment
insurance, as well as using refundable tax credits to expedite funds
directly toward people affected by the outbreak.

It is also likely that the House will address a request from the
administration to distribute emergency aid to agencies on the front
lines. Representative Nita M. Lowey of New York, the Democratic
chairwoman of the House Appropriations Committee, told her colleagues on
Thursday that her committee is expected to allocate between \$100
billion and \$150 billion, more than doubling the request, according to
a person familiar with the remarks, and include it in the emerging
package.

``Supplemental appropriations are an essential part of a
whole-of-government strategy to address this pandemic, and it is
irresponsible for Senate Republican leadership to omit these needed
resources from its proposal,'' said Evan Hollander, a spokesman for the
House committee.

Earlier this month, Congress
\href{https://www.nytimes3xbfgragh.onion/2020/03/04/us/politics/coronavirus-emergency-aid-congress.html}{approved
a first, \$8.3 billion round} of emergency money for federal health
agencies, and this week it finalized a second measure --- whose cost has
yet to be tallied --- to provide paid leave, jobless aid and food and
health care assistance, as well as free coronavirus testing. Mr. Trump
has signed both.

Time is of the essence in the talks. The news Wednesday night that two
House lawmakers had tested positive for coronavirus after voting early
Saturday has added pressure for senators to cut a swift deal on the
package and depart Washington indefinitely.

The fiscal relief package unveiled Thursday is only one part of the
administration's plan, which some analysts now anticipate topping \$1.5
trillion before the negotiations are completed. Mr. Mnuchin said the
Treasury Department and the Federal Reserve were working in lock step
and were prepared to do whatever was possible to provide liquidity to
American companies so they could weather the crisis without laying off
workers. The Federal Reserve said late Wednesday night that it would
\href{https://www.nytimes3xbfgragh.onion/2020/03/18/business/federal-reserve-mutual-funds-coronavirus-aid.html}{offer
emergency loans} to money market mutual funds, its latest in a series of
steps to keep the financial system functioning and prop up the economy.

``What we're really focused on is providing liquidity to American
businesses and American workers,'' Mr. Mnuchin said on the Fox Business
Network on Thursday. ``This is an unprecedented situation.''

He said he had advised the president to purchase oil, which is at
historically low prices, and fill up America's strategic reserve.

Economists are bracing for a deep recession. Analysts at J.P. Morgan
said this week that the United States economy could contract by 14
percent in the second quarter of this year.

The Treasury Department has not released updated economic projections,
but Mr. Mnuchin said that he expected the beginnings of growth again in
the third quarter and a ``gigantic'' rebound in the final three months
of the year.

Economic data is beginning to trickle out, offering a grim preview of
the damage that lies ahead. Official figures released on Thursday showed
claims for unemployment insurance reaching their highest level in more
than two years.

``The coronavirus outbreak is already starting to have a significant
impact on the economy,'' said Andrew Hunter, a senior U.S. economist at
Capital Economics. ``Timelier reports of state-level data point to an
unprecedented surge in layoffs over the next couple of weeks.''

The Treasury secretary indicated that he and the Federal Reserve chair,
Jerome H. Powell, would use all the tools at their disposal to allow
that workers and businesses to subsist for the next few months.

Reporting was contributed by Nicholas Fandos, Sheryl Gay Stolberg, Catie
Edmondson and Katie Rogers.

Advertisement

\protect\hyperlink{after-bottom}{Continue reading the main story}

\hypertarget{site-index}{%
\subsection{Site Index}\label{site-index}}

\hypertarget{site-information-navigation}{%
\subsection{Site Information
Navigation}\label{site-information-navigation}}

\begin{itemize}
\tightlist
\item
  \href{https://help.nytimes3xbfgragh.onion/hc/en-us/articles/115014792127-Copyright-notice}{©~2020~The
  New York Times Company}
\end{itemize}

\begin{itemize}
\tightlist
\item
  \href{https://www.nytco.com/}{NYTCo}
\item
  \href{https://help.nytimes3xbfgragh.onion/hc/en-us/articles/115015385887-Contact-Us}{Contact
  Us}
\item
  \href{https://www.nytco.com/careers/}{Work with us}
\item
  \href{https://nytmediakit.com/}{Advertise}
\item
  \href{http://www.tbrandstudio.com/}{T Brand Studio}
\item
  \href{https://www.nytimes3xbfgragh.onion/privacy/cookie-policy\#how-do-i-manage-trackers}{Your
  Ad Choices}
\item
  \href{https://www.nytimes3xbfgragh.onion/privacy}{Privacy}
\item
  \href{https://help.nytimes3xbfgragh.onion/hc/en-us/articles/115014893428-Terms-of-service}{Terms
  of Service}
\item
  \href{https://help.nytimes3xbfgragh.onion/hc/en-us/articles/115014893968-Terms-of-sale}{Terms
  of Sale}
\item
  \href{https://spiderbites.nytimes3xbfgragh.onion}{Site Map}
\item
  \href{https://help.nytimes3xbfgragh.onion/hc/en-us}{Help}
\item
  \href{https://www.nytimes3xbfgragh.onion/subscription?campaignId=37WXW}{Subscriptions}
\end{itemize}
