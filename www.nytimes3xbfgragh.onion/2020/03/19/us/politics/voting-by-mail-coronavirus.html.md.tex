Sections

SEARCH

\protect\hyperlink{site-content}{Skip to
content}\protect\hyperlink{site-index}{Skip to site index}

\href{https://www.nytimes3xbfgragh.onion/section/politics}{Politics}

\href{https://myaccount.nytimes3xbfgragh.onion/auth/login?response_type=cookie\&client_id=vi}{}

\href{https://www.nytimes3xbfgragh.onion/section/todayspaper}{Today's
Paper}

\href{/section/politics}{Politics}\textbar{}Voting by Mail Is the Hot
New Idea. Is There Time to Make It Work?

\url{https://nyti.ms/2wp5E90}

\begin{itemize}
\item
\item
\item
\item
\item
\item
\end{itemize}

\begin{itemize}
\item
  \href{https://www.nytimes3xbfgragh.onion/interactive/2020/09/08/us/elections/results-new-hampshire-primary-elections.html?action=click\&pgtype=Article\&state=default\&region=TOP_BANNER\&context=storylines_menu}{New
  Hampshire Results}
\item
  \href{https://www.nytimes3xbfgragh.onion/live/2020/09/08/us/trump-vs-biden?action=click\&pgtype=Article\&state=default\&region=TOP_BANNER\&context=storylines_menu}{Election
  Updates}
\item
  \href{https://www.nytimes3xbfgragh.onion/interactive/2020/us/elections/election-states-biden-trump.html?action=click\&pgtype=Article\&state=default\&region=TOP_BANNER\&context=storylines_menu}{Paths
  to 270}
\item
  \href{https://www.nytimes3xbfgragh.onion/interactive/2020/08/31/us/politics/vote-by-mail-deadlines.html?action=click\&pgtype=Article\&state=default\&region=TOP_BANNER\&context=storylines_menu}{Voting
  by Mail}
\item
  \href{https://www.nytimes3xbfgragh.onion/interactive/2019/us/elections/2020-presidential-election-calendar.html?action=click\&pgtype=Article\&state=default\&region=TOP_BANNER\&context=storylines_menu}{Key
  Dates}
\item
  \href{https://www.nytimes3xbfgragh.onion/newsletters/politics?action=click\&pgtype=Article\&state=default\&region=TOP_BANNER\&context=storylines_menu}{Politics
  Newsletter}
\end{itemize}

Advertisement

\protect\hyperlink{after-top}{Continue reading the main story}

Supported by

\protect\hyperlink{after-sponsor}{Continue reading the main story}

\hypertarget{voting-by-mail-is-the-hot-new-idea-is-there-time-to-make-it-work}{%
\section{Voting by Mail Is the Hot New Idea. Is There Time to Make It
Work?}\label{voting-by-mail-is-the-hot-new-idea-is-there-time-to-make-it-work}}

The coronavirus has left states rapidly searching for ways to protect
democracy's most sacred institution.

\includegraphics{https://static01.graylady3jvrrxbe.onion/images/2020/03/19/us/politics/19votebymail-01/19votebymail-01-articleLarge-v2.jpg?quality=75\&auto=webp\&disable=upscale}

\href{https://www.nytimes3xbfgragh.onion/by/nick-corasaniti}{\includegraphics{https://static01.graylady3jvrrxbe.onion/images/2018/06/13/multimedia/author-nick-corasaniti/author-nick-corasaniti-thumbLarge-v2.png}}\href{https://www.nytimes3xbfgragh.onion/by/stephanie-saul}{\includegraphics{https://static01.graylady3jvrrxbe.onion/images/2020/02/06/reader-center/author-stephanie-saul/author-stephanie-saul-thumbLarge.png}}

By \href{https://www.nytimes3xbfgragh.onion/by/nick-corasaniti}{Nick
Corasaniti} and
\href{https://www.nytimes3xbfgragh.onion/by/stephanie-saul}{Stephanie
Saul}

\begin{itemize}
\item
  Published March 19, 2020Updated June 19, 2020
\item
  \begin{itemize}
  \item
  \item
  \item
  \item
  \item
  \item
  \end{itemize}
\end{itemize}

In Wisconsin, Democrats sued elections officials to extend voting
deadlines.

In Rhode Island, the secretary of state wants all 788,000 registered
voters to receive absentee ballot applications.

In Maryland, a special election to replace the late Representative
\href{https://www.nytimes3xbfgragh.onion/2019/10/17/us/politics/elijah-cummings-death-illness.html}{Elijah
E. Cummings} will be conducted entirely by
\href{https://www.nytimes3xbfgragh.onion/2020/04/10/us/politics/vote-by-mail.html}{mail}.

As the coronavirus outbreak upends daily life and tears at the social
fabric of the country, states are rapidly searching for ways to protect
the most sacred institution in a democracy:
\href{https://www.nytimes3xbfgragh.onion/2020/04/10/us/politics/vote-by-mail.html}{voting}.

With gatherings of people suddenly presenting an imminent health threat,
state officials and voting rights activists have begun calling for an
enormous expansion of voting by mail --- for both the remaining
Democratic presidential primary race and, planning for the worst-case
scenario, the general election in November.

``The D.N.C. is urging the remaining primary states to use a variety of
other critical mechanisms that will make voting easier and safer for
voters and election officials alike,'' Tom Perez, the chairman of the
Democratic National Committee, said in a statement late Tuesday. ``The
simplest tool is vote by mail, which is already in use in a number of
states and should be made available to all registered voters.''

While rules vary somewhat state to state, 33 states and the District of
Columbia currently collect ballots by mail or permit ``no excuse''
absentee voting, in which people can vote absentee for any reason.
Colorado, Washington State and Oregon have all-mail elections.

Historically, going to the polls has been an American ritual --- so much
so that some communities shut down their schools on Election Day. Yet an
increasing number of people have opted to skip polling sites altogether
in recent years, choosing to vote from the comfort of their own homes.
More than 23 percent of voters had cast their ballots by mail
\href{https://www.eac.gov/documents/2017/10/17/eavs-deep-dive-early-absentee-and-mail-voting-data-statutory-overview}{in
the 2016 general election}, twice as many as voters did in 2004.

The next three nominating contests in the Democratic primary race ---
Hawaii, Wyoming and Alaska --- are all run by the state Democratic
Party, not the state government. All three have had extensive
\href{https://www.nytimes3xbfgragh.onion/2020/06/19/us/politics/nyc-vote-by-mail.html}{vote-by-mail}
operations in place for months; Wyoming even canceled its in-person
caucuses and went to a full vote-by-mail system.

But given the decentralized structure of American elections, which are
governed by states, counties and even municipalities, shifting to a
federally mandated, completely vote-by-mail system for the general
election could be impossible both logistically and legislatively.

Charles Stewart III, a professor of political science at the
Massachusetts Institute of Technology, expressed optimism that states
could gear up to expand mail and absentee voting for the coming primary
elections, which tend to have relatively low turnout.

The November general election will be another matter, he said.

``I think that once people take a deep breath and consider what's going
to be done in November, they're going to realize that the big lift
necessary to expand the amount of mail voting by a factor of four, five
or six in some states is going to be disruptive,'' said Mr. Stewart, who
studies both voting technology and election administration.

Under normal circumstances, states gradually transition to mail voting.

Mr. Stewart said he worried that states' lack of experience holding big
elections without in-person voting could have negative consequences.

``You can go step by step through the process and realize that there are
a lot of details that can cause the mail ballot pipeline to spring
leaks,'' he said. ``The one that's gotten the most attention in recent
years has been the issue of verifying signatures.''

In Maryland, the state's plan to run its special congressional race by
mail --- the first time the state has done so for a congressional
election --- will serve as a practice run in case the state is forced to
move to statewide mail-in voting, said former Representative Kweisi
Mfume, the Democratic candidate in that race.

``The one good thing that comes out of this is that, for the first time,
without having to conduct a statewide mail-in election, the state will
have a real opportunity, in a congressional district, to put in place a
procedure that is jointly agreed upon and to see to what extent it
works,'' said Mr. Mfume, who added that he supported the decision this
week by Gov. Larry Hogan, a Republican, to move to an all-mail
congressional election.

Gaining access to the ballot box has increasingly become a partisan
issue, with some Republicans, citing reports of voter fraud, adding
hurdles that include purging voter rolls and instituting voter ID
requirements, while Democrats promote ideas like same-day registration
and early and mail voting options.

Democrats in Congress have pushed to expand voting by mail amid the
coronavirus pandemic, with the possibility that it can still be wreaking
havoc in November.

A bill by Senator Ron Wyden of Oregon and Senator Amy Klobuchar of
Minnesota, the former presidential candidate, would require that all
states offered a mail-in or drop-off paper ballot option if 25 percent
of states declared a state of emergency related to the coronavirus or
another infectious disease or natural disaster, and that requests for a
ballot could be made electronically.

\includegraphics{https://static01.graylady3jvrrxbe.onion/images/2020/03/19/us/politics/19votebymail-02/merlin_170616024_25c22573-7a36-4316-9c60-15f6c786de65-articleLarge.jpg?quality=75\&auto=webp\&disable=upscale}

``There's a public health crisis --- that was central to my introducing
this bill,'' said Mr. Wyden, who in 1996 became the first senator
elected in an all-mail election.

Efforts to introduce all-mail voting have had plenty of detractors.
``I've known about the opposition in the past,'' Mr. Wyden said. ``But
if they're faced with the question of how do you actually hold an
election this fall, I think they're going to have trouble defying common
sense and making arguments for why you shouldn't do it.''

The bill would call for all states to send prepaid, self-sealing
envelopes, as well as ballot-tracking markers, to make sure that voters
incurred no personal cost that would act as a barrier to voting, and to
minimize any spread of the coronavirus through licking envelopes.

Though the bill does not specify a price tag, a previous iteration under
Mr. Wyden estimated a cost of roughly \$500 million. But the bill would
provide only the opportunity of vote-by-mail expansion, not fully
transition the country to voting by mail.

``We're not saying to get rid of all polling places by any means,'' Ms.
Klobuchar said. ``It's just that the more people we can get to vote this
way, the better off this is.''

Ms. Klobuchar, who is also the ranking member on the Senate Rules
Committee, said she had been speaking with Senator Roy Blunt, Republican
of Missouri and the committee's chairman, about getting bipartisan
support for the bill. She has also spoken to Senator Chuck Schumer of
New York, the minority leader, about funding.

``We're being practical about it,'' Ms. Klobuchar said. ``We can't
change every state into those three states in seven months.''

Individual states have also begun to marshal efforts to increase
vote-by-mail significantly while still preserving polling stations, in
part to help serve voters with disabilities.

In Rhode Island, Nellie M. Gorbea, the secretary of state, has been
working with the state's Board of Elections to try to make the April 28
primary ``predominantly a vote-by-mail election'' by automatically
sending applications to all of the state's 788,000 registered voters.

``The minute I became aware that the coronavirus has a particularly
strong impact on our elderly population, my thoughts went to polling
locations, because the demographics of polling locations tends to be
seniors and retired people,'' Ms. Gorbea said.

She said that while the plan would carry additional costs for the state,
expanding vote by mail through applications should be part of the
state's overall emergency spending. And, she added, it should be tried
now, with a general election looming.

``It's a good time to test these systems in case we are still in this
situation come later on in the fall,'' she said.

Other states with late primary elections, like New York and Connecticut,
are evaluating whether to expand mail voting options.

In New York, officials are considering both delaying the April 28
primary and expanding mail-in options, according to Douglas A. Kellner,
co-chairman of the State Board of Elections.

``I am not in favor of mail-in elections in general because of the
potential for fraud and the lack of verifiability,'' Mr. Kellner, a
Democrat, said. ``But it's certainly being discussed. So I wouldn't rule
that out.''

In New Jersey, multiple municipal elections affecting roughly 680,000
eligible voters were postponed until May 12, and all will be conducted
exclusively by mail.

In Connecticut, Secretary of State Denise Merrill asked Gov. Ned Lamont
to issue an executive order permitting anyone to obtain an absentee
ballot in the state's primary on April 28. Those ballots are currently
available only if you are ill.

The governor, a Democrat, has not yet ruled on that request, but
officials are also looking ahead to the fall. Mr. Lamont did announce on
Thursday that the Connecticut primary would be postponed until June 2.

``Our office is already talking about the August primary and the
November general election internally,'' said Gabe Rosenberg, a spokesman
for Ms. Merrill.

In Wisconsin, Democrats
\href{https://www.nytimes3xbfgragh.onion/2020/03/18/us/politics/wisconsin-primary-voting-coronavirus.html}{sued
elections officials on Wednesday} to drop photo ID requirements for
absentee ballots and voting by mail, and to extend the deadline for
requesting an absentee ballot and voting by mail for the state's April 7
election.

Aside from the lawsuit, the state Democratic Party said it had shifted
its entire organizing apparatus to a digital campaign during the
coronavirus outbreak. Part of that includes a huge digital canvassing
effort through emails, text messages and social media posts to persuade
voters to register online and to request a vote-by-mail or absentee
ballot.

``Our whole organizing operation has switched to working on getting
people to request absentee ballots in Wisconsin,'' said Ben Wikler, the
chair of the state Democratic Party.

But as states mount their own efforts to deal with primary elections ---
including postponing them, as
\href{https://www.nytimes3xbfgragh.onion/2020/03/16/us/politics/virus-primary-2020-ohio.html}{Ohio
and several other states have done} --- Ms. Klobuchar and other
Democrats are growing anxious about November.

``We really have got to get on this,'' Ms. Klobuchar said. ``Even though
there's all kinds of reasons they canceled in Ohio, this is going to be
jarring to people. They have to be able to vote. We're going to have to
figure this out.''

\hypertarget{our-2020-election-guide}{%
\section{Our 2020 Election Guide}\label{our-2020-election-guide}}

Updated ~Sept. 8, 2020

\begin{center}\rule{0.5\linewidth}{\linethickness}\end{center}

\begin{itemize}
\item ~
  \hypertarget{the-latest}{%
  \subsection{The Latest}\label{the-latest}}

  \begin{itemize}
  \item
    President Trump and his party are using a playbook that aims to
    alarm people about crime in their backyards. It didn't work in 2018,
    but
    \href{https://www.nytimes3xbfgragh.onion/2020/09/08/us/politics/trump-republicans-fear-strategy.html?action=click\&pgtype=Article\&state=default\&region=BELOW_MAIN_CONTENT\&context=storylines_guide}{both
    parties think it could resonate more this year}.
  \end{itemize}
\item ~
  \hypertarget{how-to-win-270}{%
  \subsection{How to Win 270}\label{how-to-win-270}}

  \begin{itemize}
  \item
    Joe Biden and Donald Trump need 270 electoral votes to reach the
    White House. Try building
    \href{https://www.nytimes3xbfgragh.onion/interactive/2020/us/elections/election-states-biden-trump.html?action=click\&pgtype=Article\&state=default\&region=BELOW_MAIN_CONTENT\&context=storylines_guide}{your
    own coalition of battleground states}~to see potential outcomes.
  \end{itemize}
\item ~
  \hypertarget{voting-by-mail}{%
  \subsection{Voting by Mail}\label{voting-by-mail}}

  \begin{itemize}
  \item
    Will you have enough time to vote by mail in your state? Yes, but
    it's risky to procrastinate.
    \href{https://www.nytimes3xbfgragh.onion/interactive/2020/08/31/us/politics/vote-by-mail-deadlines.html?action=click\&pgtype=Article\&state=default\&region=BELOW_MAIN_CONTENT\&context=storylines_guide}{Check
    your state's deadline.}
  \item
    \href{https://www.nytimes3xbfgragh.onion/interactive/2020/us/elections/joe-biden.html?action=click\&pgtype=Article\&state=default\&region=BELOW_MAIN_CONTENT\&context=storylines_guide}{}

    \hypertarget{joe-biden}{%
    \section{Joe Biden}\label{joe-biden}}

    \hypertarget{democrat}{%
    \subsection{Democrat}\label{democrat}}

    \href{https://www.nytimes3xbfgragh.onion/interactive/2020/us/elections/donald-trump.html?action=click\&pgtype=Article\&state=default\&region=BELOW_MAIN_CONTENT\&context=storylines_guide}{}

    \hypertarget{donald-trump}{%
    \section{Donald Trump}\label{donald-trump}}

    \hypertarget{republican}{%
    \subsection{Republican}\label{republican}}
  \end{itemize}
\item
  \hypertarget{keep-up-with-our-coverage}{%
  \subsection{Keep Up With Our
  Coverage}\label{keep-up-with-our-coverage}}

  \begin{itemize}
  \item
    Get an
    \href{https://www.nytimes3xbfgragh.onion/newsletters/politics?action=click\&pgtype=Article\&state=default\&region=BELOW_MAIN_CONTENT\&context=storylines_guide}{email}~recapping
    the day's news
  \item
    Download our mobile app on
    \href{https://apps.apple.com/us/app/nytimes/id284862083?ls=1\&mat_click_id=5c79ae7455014fd1bd66b5610c05b8f2-20191112-16948\&referrer=mat_click_id\%3D5c79ae7455014fd1bd66b5610c05b8f2-20191112-16948\%26link_click_id\%3D722930677036718082}{iOS}~and
    \href{http://a.localytics.com/android?id=com.nytimes.android\&referrer=utm_source\%3Dother_nyt_mobile_web\%26utm_medium\%3DWeb\%2520page\%26utm_term\%3DGeneral\%2520Mobile\%2520Page\%26utm_campaign\%3DNYT\%2520Mobile\%2520General\%2520Page}{Android}~and
    turn on Breaking News and Politics alerts
  \end{itemize}
\end{itemize}

Advertisement

\protect\hyperlink{after-bottom}{Continue reading the main story}

\hypertarget{site-index}{%
\subsection{Site Index}\label{site-index}}

\hypertarget{site-information-navigation}{%
\subsection{Site Information
Navigation}\label{site-information-navigation}}

\begin{itemize}
\tightlist
\item
  \href{https://help.nytimes3xbfgragh.onion/hc/en-us/articles/115014792127-Copyright-notice}{©~2020~The
  New York Times Company}
\end{itemize}

\begin{itemize}
\tightlist
\item
  \href{https://www.nytco.com/}{NYTCo}
\item
  \href{https://help.nytimes3xbfgragh.onion/hc/en-us/articles/115015385887-Contact-Us}{Contact
  Us}
\item
  \href{https://www.nytco.com/careers/}{Work with us}
\item
  \href{https://nytmediakit.com/}{Advertise}
\item
  \href{http://www.tbrandstudio.com/}{T Brand Studio}
\item
  \href{https://www.nytimes3xbfgragh.onion/privacy/cookie-policy\#how-do-i-manage-trackers}{Your
  Ad Choices}
\item
  \href{https://www.nytimes3xbfgragh.onion/privacy}{Privacy}
\item
  \href{https://help.nytimes3xbfgragh.onion/hc/en-us/articles/115014893428-Terms-of-service}{Terms
  of Service}
\item
  \href{https://help.nytimes3xbfgragh.onion/hc/en-us/articles/115014893968-Terms-of-sale}{Terms
  of Sale}
\item
  \href{https://spiderbites.nytimes3xbfgragh.onion}{Site Map}
\item
  \href{https://help.nytimes3xbfgragh.onion/hc/en-us}{Help}
\item
  \href{https://www.nytimes3xbfgragh.onion/subscription?campaignId=37WXW}{Subscriptions}
\end{itemize}
