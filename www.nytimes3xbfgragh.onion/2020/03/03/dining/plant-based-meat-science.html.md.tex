Sections

SEARCH

\protect\hyperlink{site-content}{Skip to
content}\protect\hyperlink{site-index}{Skip to site index}

\href{https://www.nytimes3xbfgragh.onion/section/food}{Food}

\href{https://myaccount.nytimes3xbfgragh.onion/auth/login?response_type=cookie\&client_id=vi}{}

\href{https://www.nytimes3xbfgragh.onion/section/todayspaper}{Today's
Paper}

\href{/section/food}{Food}\textbar{}How Do They Make Plant-Based Meat
Behave Like Beef?

\url{https://nyti.ms/3apVYcK}

\begin{itemize}
\item
\item
\item
\item
\item
\item
\end{itemize}

Advertisement

\protect\hyperlink{after-top}{Continue reading the main story}

Supported by

\protect\hyperlink{after-sponsor}{Continue reading the main story}

\hypertarget{how-do-they-make-plant-based-meat-behave-like-beef}{%
\section{How Do They Make Plant-Based Meat Behave Like
Beef?}\label{how-do-they-make-plant-based-meat-behave-like-beef}}

J. Kenji López-Alt explains the science behind the new vegan products.

\includegraphics{https://static01.graylady3jvrrxbe.onion/images/2020/03/04/dining/03fakescience1/merlin_169263873_da46c68e-daa3-4c6e-a3b0-f565909acd45-articleLarge.jpg?quality=75\&auto=webp\&disable=upscale}

By \href{https://www.nytimes3xbfgragh.onion/by/j-kenji-lopez-alt}{J.
Kenji López-Alt}

\begin{itemize}
\item
  March 3, 2020
\item
  \begin{itemize}
  \item
  \item
  \item
  \item
  \item
  \item
  \end{itemize}
\end{itemize}

Texture, appearance and flavor: These are the elements of meat that the
new vegan alternatives from Impossible Foods and Beyond Meat are trying
to capture, with varying degrees of success. Here's how they do it:

\hypertarget{texture}{%
\subsubsection{Texture}\label{texture}}

In ground beef, animal protein provides springy texture and allows the
meat to bind to itself. (Hamburgers would simply crumble if it didn't.)

But mimicking the texture of animal protein using plant-based
ingredients has always been difficult because of a fundamental
difference between animals and plants: muscles, which are by necessity
elastic and springy. To move their bodies, animals must be able to
easily change the shape and tension of their flesh without damaging it.
Plant cells, on the other hand, are relatively rigid and unflexing.

To put it simply, plants are crunchy, and meat is chewy. This is why
veggie burgers can often feel crumbly or mushy in texture, without the
bite and springiness of animal protein. To solve this problem,
researchers have spent years isolating and cataloging a wide variety of
plant-based protein sources. As a result, the texture of modern vegan
meat --- provided by wheat or pea proteins, among others --- can be
fantastic.

\emph{{[}}\href{https://www.nytimes3xbfgragh.onion/2019/10/22/dining/veggie-burger-taste-test.html}{\emph{Read
the results of our taste test of plant-based meats.}}\emph{{]}}

The other major factor in beef's texture is animal fat, which provides
mouth-coating richness and juiciness. Beef fat also tends to melt
slowly, over a wide temperature range. This slow release of fat results
in juiciness that lingers as you chew.

That's very hard to capture with plant-based fats, because of a crucial
difference between them and animal fats. The melting point of a fat is
linked to its level of saturation --- the number of single bonds versus
double bonds in its fatty-acid chain. Animal fats tend to be more highly
saturated than vegetable fats (usually referred to as oils in culinary
circles), which is why beef and pork fat are solid at room temperature
while olive and corn oils are liquid.

There are a few exceptions, notably palm and coconut oils, which are
highly saturated and thus solid at room temperature. Both Impossible
Foods and Beyond Meat use coconut oil as their primary fat, producing a
mouth-coating texture similar to animal fat.

But coconut oil melts at a much lower temperature than beef fat, and
much faster. In the mouth, this translates to bites that start off rich
and juicy; but that juiciness wears off much quicker. In this
department, plant-based meats still have a way to go.

\includegraphics{https://static01.graylady3jvrrxbe.onion/images/2020/03/04/dining/03Fakescience2/03Fakescience2-articleLarge.jpg?quality=75\&auto=webp\&disable=upscale}

\hypertarget{appearance}{%
\subsubsection{Appearance}\label{appearance}}

The new vegan meats have also made great advances in replicating the red
color we associate with beef.

In beef, that color comes from myoglobin, a compound that transmits
oxygen from the bloodstream to muscle cells. Beyond Meat uses beet
extracts to color its product, while Impossible Foods relies on another
iron-containing compound called
\href{https://www.nytimes3xbfgragh.onion/2017/08/08/business/impossible-burger-food-meat.html}{leghemoglobin},
an oxygen transport molecule found in the roots of legumes, such as soy.
Like myoglobin, it has a red color and --- according to Impossible --- a
meaty flavor. (The company produces its leghemoglobin with the help of
genetically modified yeast.)

\emph{{[}}\href{https://www.nytimes3xbfgragh.onion/2020/03/03/dining/impossible-beyond-meat.html}{\emph{Learn
how to cook with plant-based meats.}}\emph{{]}}

In both products, the coconut oil is incorporated in small, solid chunks
that mimic the appearance of animal fat. When you bite into a
medium-rare Impossible or Beyond burger, the resemblance to ground beef
in color and texture is uncanny.

\hypertarget{flavor}{%
\subsubsection{Flavor}\label{flavor}}

The precise makeup of the flavorings used in Beyond and Impossible meats
are harder to decipher. Food and Drug Administration labeling rules
don't require companies to disclose exact flavoring agents, only whether
they use ``natural flavors'' or ``artificial flavors.'' And like most
packaged products, Impossible and Beyond meats don't disclose the
sources of those flavors.

Even those terms can be misleading. Natural and artificial flavors can
be chemically identical to each other, but only those chemicals derived
from a natural source can be labeled ``natural,'' regardless of how
refined or processed it is.

As it does with juiciness, the propensity of plant-based fats to melt
quickly makes fat-soluble flavor compounds dissipate in the mouth faster
than with beef.

\hypertarget{at-a-glance}{%
\subsubsection{At a Glance}\label{at-a-glance}}

Here's a quick look at the primary ingredients used by the two
companies:

\textbf{Texture}

Impossible Foods: Soy and potato protein

Beyond Meat: Pea, rice and mung bean protein

\textbf{Fat source}

Impossible: Coconut and sunflower oil

Beyond: Coconut and canola oil

\textbf{Coloring}

Impossible: Leghemoglobin

Beyond: Beet extract

\textbf{Flavoring}

Impossible: Natural flavors and soy leghemoglobin

Beyond: Natural flavors

\emph{{[}}\href{https://www.nytimes3xbfgragh.onion/2020/03/03/dining/impossible-beyond-meat.html}{\emph{Read
more about cooking with vegan ground meats.}}\emph{{]}}

Recipes:
\textbf{\href{https://cooking.nytimes3xbfgragh.onion/recipes/1020865-vegan-turkish-kebabs-with-sumac-onions-and-garlic-dill-mayonnaise}{Vegan
Turkish Kebabs With Sumac Onions and Garlic-Dill Mayonnaise}} \textbar{}
\textbf{\href{https://cooking.nytimes3xbfgragh.onion/recipes/1020866-vegan-chili}{Vegan
Chili}} \textbar{}
\textbf{\href{https://cooking.nytimes3xbfgragh.onion/recipes/1020867-vegan-cheeseburgers}{Vegan
Cheeseburgers}}

\emph{Follow} \href{https://twitter.com/nytfood}{\emph{NYT Food on
Twitter}} \emph{and}
\href{https://www.instagram.com/nytcooking/}{\emph{NYT Cooking on
Instagram}}\emph{,}
\href{https://www.facebookcorewwwi.onion/nytcooking/}{\emph{Facebook}}\emph{,}
\href{https://www.youtube.com/nytcooking}{\emph{YouTube}} \emph{and}
\href{https://www.pinterest.com/nytcooking/}{\emph{Pinterest}}\emph{.}
\href{https://www.nytimes3xbfgragh.onion/newsletters/cooking}{\emph{Get
regular updates from NYT Cooking, with recipe suggestions, cooking tips
and shopping advice}}\emph{.}

Advertisement

\protect\hyperlink{after-bottom}{Continue reading the main story}

\hypertarget{site-index}{%
\subsection{Site Index}\label{site-index}}

\hypertarget{site-information-navigation}{%
\subsection{Site Information
Navigation}\label{site-information-navigation}}

\begin{itemize}
\tightlist
\item
  \href{https://help.nytimes3xbfgragh.onion/hc/en-us/articles/115014792127-Copyright-notice}{©~2020~The
  New York Times Company}
\end{itemize}

\begin{itemize}
\tightlist
\item
  \href{https://www.nytco.com/}{NYTCo}
\item
  \href{https://help.nytimes3xbfgragh.onion/hc/en-us/articles/115015385887-Contact-Us}{Contact
  Us}
\item
  \href{https://www.nytco.com/careers/}{Work with us}
\item
  \href{https://nytmediakit.com/}{Advertise}
\item
  \href{http://www.tbrandstudio.com/}{T Brand Studio}
\item
  \href{https://www.nytimes3xbfgragh.onion/privacy/cookie-policy\#how-do-i-manage-trackers}{Your
  Ad Choices}
\item
  \href{https://www.nytimes3xbfgragh.onion/privacy}{Privacy}
\item
  \href{https://help.nytimes3xbfgragh.onion/hc/en-us/articles/115014893428-Terms-of-service}{Terms
  of Service}
\item
  \href{https://help.nytimes3xbfgragh.onion/hc/en-us/articles/115014893968-Terms-of-sale}{Terms
  of Sale}
\item
  \href{https://spiderbites.nytimes3xbfgragh.onion}{Site Map}
\item
  \href{https://help.nytimes3xbfgragh.onion/hc/en-us}{Help}
\item
  \href{https://www.nytimes3xbfgragh.onion/subscription?campaignId=37WXW}{Subscriptions}
\end{itemize}
