Sections

SEARCH

\protect\hyperlink{site-content}{Skip to
content}\protect\hyperlink{site-index}{Skip to site index}

\href{https://www.nytimes3xbfgragh.onion/section/sports}{Sports}

\href{https://myaccount.nytimes3xbfgragh.onion/auth/login?response_type=cookie\&client_id=vi}{}

\href{https://www.nytimes3xbfgragh.onion/section/todayspaper}{Today's
Paper}

\href{/section/sports}{Sports}\textbar{}The Tokyo Olympics Will Open a
Year From Now. Maybe.

\url{https://nyti.ms/2WBXVhM}

\begin{itemize}
\item
\item
\item
\item
\item
\end{itemize}

Advertisement

\protect\hyperlink{after-top}{Continue reading the main story}

Supported by

\protect\hyperlink{after-sponsor}{Continue reading the main story}

\hypertarget{the-tokyo-olympics-will-open-a-year-from-now-maybe}{%
\section{The Tokyo Olympics Will Open a Year From Now.
Maybe.}\label{the-tokyo-olympics-will-open-a-year-from-now-maybe}}

Japan has largely controlled the coronavirus. Large parts of the rest of
the world, especially the United States, have not. A year before the
rescheduled Games, that is still a major problem.

\includegraphics{https://static01.graylady3jvrrxbe.onion/images/2020/07/19/sports/19Olympics2/merlin_174485964_6cfbfd27-65d7-4b3e-8cd6-6c2dd24c0903-articleLarge.jpg?quality=75\&auto=webp\&disable=upscale}

By
\href{https://www.nytimes3xbfgragh.onion/by/matthew-futterman}{Matthew
Futterman},
\href{https://www.nytimes3xbfgragh.onion/by/motoko-rich}{Motoko Rich}
and \href{https://www.nytimes3xbfgragh.onion/by/andrew-keh}{Andrew Keh}

\begin{itemize}
\item
  July 19, 2020
\item
  \begin{itemize}
  \item
  \item
  \item
  \item
  \item
  \end{itemize}
\end{itemize}

A year from now, the world will begin to gather in Japan to celebrate
the opening of the Tokyo Olympics, which were originally supposed to
begin this week.

\href{https://www.nytimes3xbfgragh.onion/2020/05/22/sports/olympics/coronavirus-tokyo-olympics.html}{Or
maybe they won't.}

Four months after the International Olympic Committee and officials in
Japan
\href{https://www.nytimes3xbfgragh.onion/2020/03/30/sports/olympics/tokyo-olympics-date-coronavirus.html}{postponed
the Games} amid soaring coronavirus infection rates and lockdowns across
the world, uncertainty prevails. The unpredictable nature of the virus
is making it impossible for officials to say definitively that the Games
will happen or, if they do, what they might look like.

Maybe there won't be spectators. Maybe only people living in Japan will
be able to attend. Or maybe only those from countries where the virus is
under control. Will there be an Olympic village, the traditional home
for the roughly 10,000 competitors? Will athletes from the United
States, where the
\href{https://www.nytimes3xbfgragh.onion/news-event/coronavirus?action=click\&pgtype=Article\&state=default\&module=styln-coronavirus\&variant=show\&region=TOP_BANNER\&context=storylines_menu}{pandemic
shows no signs of abating}, be allowed to attend?

In a news conference last week,
\href{https://www.nytimes3xbfgragh.onion/2020/03/19/sports/olympics/olympics-coronavirus-bach-ioc.html}{Thomas
Bach}, the president of the I.O.C., said that planning for the Games now
involves multiple options. All of them, he said, prioritize the health
of the athletes.

``It includes all different countermeasures,'' Bach said of the
planning. ``An Olympic Games behind closed doors is clearly something we
do not want. We are working for a solution that safeguards the health of
all the participants and is also reflecting of the Olympic spirit.''

Bach has said a further postponement is not an option at the moment; if
the Games cannot be held next summer, they will not be held at all.

\href{https://okcthunderwire.usatoday.com/2020/07/14/nba-players-in-orlando-bubble-positive-coronavirus-tests/}{As
sports leagues everywhere struggle} to return to some semblance of
normalcy while balancing virus outbreaks and safety concerns, the
challenges of planning a global event that is still a year away have
only grown --- or merely been exacerbated as hot spots for infections
continue to shift.

``People right now are focused on the health of the citizens of their
countries,'' said Harvey Schiller, the former chief executive of the
United States Olympic and Paralympic Committee.

\includegraphics{https://static01.graylady3jvrrxbe.onion/images/2020/07/19/sports/19olympics1/merlin_174252000_882a8b1d-0b39-4805-bb7a-66a8cb6045e3-articleLarge.jpg?quality=75\&auto=webp\&disable=upscale}

No one doubts the resolve of the I.O.C. and its local organizers in
Japan, who desperately want to hold the Games, given the resources
they've already committed and the money at stake. Japan has already
spent roughly \$12 billion to prepare for the Games. The I.O.C. stands
to lose the billions of dollars in
\href{https://www.nytimes3xbfgragh.onion/2014/05/08/sports/olympics/nbc-extends-olympic-tv-deal-through-2032.html}{revenues
from media rights}, tickets and sponsorships if the Games do not happen.

Despite a recent spike in coronavirus cases and
a\href{https://www.mofa.go.jp/ca/fna/page4e_001053.html}{ban on travel
from 129 countries}, the official line in Japan remains that the
postponed 2020 Games will open on July 23, 2021, in Tokyo.

Shortly after Yuriko Koike, the governor of
Tokyo,\href{https://www.nytimes3xbfgragh.onion/2020/07/05/world/asia/tokyo-governor-election.html?searchResultPosition=2}{won
a second term} earlier this month, she met with Japan's prime minister,
Shinzo Abe, to discuss measures to contain the virus. ``I would like to
lead the Olympics and Paralympics next year as proof that we have
overcome the coronavirus,'' she said.

On Wednesday, Tokyo raised its pandemic alert level to red, its highest
classification, in response to a recent spike in cases concentrated in
the metropolis's sprawling nightlife district. In the last two weeks,
Tokyo has recorded several consecutive daily records, hitting a peak of
293 new infections last Friday.

Compared with other international cities, Tokyo has been relatively
successful in containing the virus. A city of 14 million people, it has
reported less than 9,000 cases and 326 deaths since February, compared
with more than
\href{https://www.nytimes3xbfgragh.onion/2020/07/16/world/coronavirus-updates.html?action=click\&module=Top\%20Stories\&pgtype=Homepage}{3.5
million cases and nearly 140,000 deaths in the United States}.
Traditionally the financial engine of an Olympics, the United States
currently poses perhaps the biggest threat to the Games.

\hypertarget{sports-and-the-virus}{%
\subsubsection{Sports and the Virus}\label{sports-and-the-virus}}

\paragraph{}

Updated Sept. 8, 2020

Here's what's happening as the world of sports slowly comes back to
life:

\begin{itemize}
\item
  \begin{itemize}
  \tightlist
  \item
    As the United States Open enters its second week without fans, an
    Italian restaurateur stands outside the gates and
    \href{https://www.nytimes3xbfgragh.onion/2020/09/06/sports/tennis/US-Open-Matteo-Berrettini-fan.html?action=click\&pgtype=Article\&state=default\&region=MAIN_CONTENT_2\&context=storylines_keepup}{bellows
    his support}~for his favorite player.
  \item
    The coronavirus pandemic has had an
    \href{https://www.nytimes3xbfgragh.onion/2020/09/03/sports/ncaafootball/high-school-football-coronavirus-pandemic.html?action=click\&pgtype=Article\&state=default\&region=MAIN_CONTENT_2\&context=storylines_keepup}{uneven
    impact on high school football}~across the United States.
  \item
    The
    \href{https://www.nytimes3xbfgragh.onion/2020/09/02/sports/ncaafootball/coronavirus-cal-athletics-season.html?action=click\&pgtype=Article\&state=default\&region=MAIN_CONTENT_2\&context=storylines_keepup}{most
    complicated puzzle in sports is the return of college
    athletics}~during a pandemic. The University of California, Berkeley
    is allowing The Times an inside look at their journey's ups and
    downs.
  \end{itemize}
\end{itemize}

Part of Japan's strategy has been to close its borders to citizens
traveling from 129 countries, including the United States and large
portions of Europe, Africa, Latin America and the rest of Asia. Japan
has recently announced plans to negotiate some reciprocal travel between
Japan and Australia, New Zealand, Thailand and Vietnam, but it has not
indicated when it plans to reopen its borders to travelers from the rest
of the world.

Even inside Japan, citizens remain wary about traveling: a plan to
encourage domestic travel was met with resistance as people worried that
Tokyo residents could spread the virus to other parts of the country. On
Friday, the country's tourism minister discouraged Tokyo residents from
visiting other prefectures and said that government travel discounts
would not apply to travelers to or from Tokyo.

Polls suggest the public is also wary of the Olympics. In a survey late
last month by The Asahi Shimbun, one of Japan's largest daily
newspapers, 59 percent of those polled said they wanted the Olympics to
be postponed again or canceled. Koike, though, was recently re-elected
governor of Tokyo in a landslide, even as she adhered to the official
position of holding the Games in 2021.

In a sign of the continuing havoc that the coronavirus is inflicting on
the global sports calendar, the I.O.C. on Wednesday postponed the 2022
Summer Youth Olympic Games, planned for Senegal, until 2026. Bach said
holding three Olympics Games --- the Tokyo Games, the 2022 Winter
Olympics in Beijing and the 2024 Summer Games in Paris --- plus youth
Games in 2022 and 2024 was too overwhelming. The same day, Richard W.
Pound, a longtime member of the I.O.C. from Canada, floated the
possibility of a cancellation of the Beijing Games
\href{https://www.reuters.com/article/us-olympics-beijing-pound-interview/olympics-no-tokyo-games-likely-means-no-beijing-either-says-pound-idUSKCN24G30L}{in
an interview with Reuters}. On Friday, Bach said specific plans for how
the Tokyo Games would unfold were still far from complete.

``How can you know already in detail maybe the most complex event to
organize in the world?'' Bach said in a news conference in which he
announced his intention to seek another term as I.O.C. president. ``You
can put potential developments together, but you cannot have a solution
today.''

Image

I.O.C. President Thomas Bach said it was impossible to offer assurances
the Games would take place. ``You can put potential developments
together, but you cannot have a solution today.''Credit...IOC/Greg
Martin, via Reuters

The coronavirus has forced NBC, which has committed about \$8 billion
for the United States
\href{https://www.nytimes3xbfgragh.onion/2014/05/08/sports/olympics/nbc-extends-olympic-tv-deal-through-2032.html}{media
rights to the Games through 2032} and is the I.O.C.'s leading source of
revenue, to consider reducing its contingent of roughly 2,000 workers
and hundreds of guests. NBC also has had to rethink how it will present
the story of the Olympics, since that story has changed significantly.

``It's impossible to predict what the circumstances will be a year from
now,'' said Molly Solomon, the executive producer of NBC's Olympics
production. With social distancing limiting her team's access to
athletes, they have asked competitors to document their training
regimens so the network has footage of this transformative experience.
``I do think this has a chance to be the most memorable Games in
history,'' Solomon said.

During a recent conference call with athletes, though, leaders of the
U.S.O.P.C. had few concrete answers. No one could say if athletes would
still have to share rooms in the Olympic Village, if the common dining
hall would be a potentially germ-spreading buffet, or if the American
team --- traditionally the biggest contingent at any Games --- might
have to be housed separately from people representing other nations.

``Athletes are yearning for more concrete communication directly from
the I.O.C. and other organizations,'' said Han Xiao, chairman of the
U.S.O.P.C.'s Athletes' Advisory Council.

The United States team of more than 500 athletes might have to be
smaller, though so far the I.O.C. has maintained that it does not plan
to reduce the number of events or participants.

``There is a lot of speculation and proposals, but not one specific plan
that anyone is able to focus on,'' said Christian Taylor, a two-time
gold medalist in the triple jump.

Rick Adams, the chief of sport performance at the U.S.O.P.C., said the
organization remained focused on Plan A --- a typical Olympic Games with
most athletes living and eating in the Olympic Village and using a
training center the U.S.O.P.C. will set up in Tokyo's Setagaya City
neighborhood. But the organization also has considered how it would
adjust if it has to come up with an alternative plan for housing and
feeding its team, and for shrinking its support staff.

``We understand what a pivot might look like,'' Adams said. ``We know
how to adjust quickly and would be able to do that.''

Xiao, the athletes' council chairman, said thinking about travel
restrictions is keeping athletes awake at night. Many need to compete to
qualify for the Games, and also to hone their skills for an event that
for many is the zenith of their athletic lives. Doing that properly
requires the intensity of competition, Xiao said. But athletes also need
assurances that they will be allowed to take part.

Governments generally can't interfere with a qualified athlete's right
to participate in the Games, but those debates are usually related to
political issues. The coronavirus has changed the equation.

``Our athletes are used to, `You can't infringe upon my right to
compete,''' Xiao said. ``But this is going to be a real challenge to try
to figure out where that line is.''

Advertisement

\protect\hyperlink{after-bottom}{Continue reading the main story}

\hypertarget{site-index}{%
\subsection{Site Index}\label{site-index}}

\hypertarget{site-information-navigation}{%
\subsection{Site Information
Navigation}\label{site-information-navigation}}

\begin{itemize}
\tightlist
\item
  \href{https://help.nytimes3xbfgragh.onion/hc/en-us/articles/115014792127-Copyright-notice}{©~2020~The
  New York Times Company}
\end{itemize}

\begin{itemize}
\tightlist
\item
  \href{https://www.nytco.com/}{NYTCo}
\item
  \href{https://help.nytimes3xbfgragh.onion/hc/en-us/articles/115015385887-Contact-Us}{Contact
  Us}
\item
  \href{https://www.nytco.com/careers/}{Work with us}
\item
  \href{https://nytmediakit.com/}{Advertise}
\item
  \href{http://www.tbrandstudio.com/}{T Brand Studio}
\item
  \href{https://www.nytimes3xbfgragh.onion/privacy/cookie-policy\#how-do-i-manage-trackers}{Your
  Ad Choices}
\item
  \href{https://www.nytimes3xbfgragh.onion/privacy}{Privacy}
\item
  \href{https://help.nytimes3xbfgragh.onion/hc/en-us/articles/115014893428-Terms-of-service}{Terms
  of Service}
\item
  \href{https://help.nytimes3xbfgragh.onion/hc/en-us/articles/115014893968-Terms-of-sale}{Terms
  of Sale}
\item
  \href{https://spiderbites.nytimes3xbfgragh.onion}{Site Map}
\item
  \href{https://help.nytimes3xbfgragh.onion/hc/en-us}{Help}
\item
  \href{https://www.nytimes3xbfgragh.onion/subscription?campaignId=37WXW}{Subscriptions}
\end{itemize}
