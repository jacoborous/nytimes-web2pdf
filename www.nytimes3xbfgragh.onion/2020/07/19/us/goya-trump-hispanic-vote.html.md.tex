Sections

SEARCH

\protect\hyperlink{site-content}{Skip to
content}\protect\hyperlink{site-index}{Skip to site index}

\href{https://www.nytimes3xbfgragh.onion/section/us}{U.S.}

\href{https://myaccount.nytimes3xbfgragh.onion/auth/login?response_type=cookie\&client_id=vi}{}

\href{https://www.nytimes3xbfgragh.onion/section/todayspaper}{Today's
Paper}

\href{/section/us}{U.S.}\textbar{}How Buying Beans Became a Political
Statement

\url{https://nyti.ms/3jkTCBe}

\begin{itemize}
\item
\item
\item
\item
\item
\end{itemize}

\begin{itemize}
\item
  \href{https://www.nytimes3xbfgragh.onion/interactive/2020/09/08/us/elections/results-new-hampshire-primary-elections.html?action=click\&pgtype=Article\&state=default\&region=TOP_BANNER\&context=storylines_menu}{New
  Hampshire Results}
\item
  \href{https://www.nytimes3xbfgragh.onion/live/2020/09/08/us/trump-vs-biden?action=click\&pgtype=Article\&state=default\&region=TOP_BANNER\&context=storylines_menu}{Election
  Updates}
\item
  \href{https://www.nytimes3xbfgragh.onion/interactive/2020/us/elections/election-states-biden-trump.html?action=click\&pgtype=Article\&state=default\&region=TOP_BANNER\&context=storylines_menu}{Paths
  to 270}
\item
  \href{https://www.nytimes3xbfgragh.onion/interactive/2020/08/31/us/politics/vote-by-mail-deadlines.html?action=click\&pgtype=Article\&state=default\&region=TOP_BANNER\&context=storylines_menu}{Voting
  by Mail}
\item
  \href{https://www.nytimes3xbfgragh.onion/interactive/2019/us/elections/2020-presidential-election-calendar.html?action=click\&pgtype=Article\&state=default\&region=TOP_BANNER\&context=storylines_menu}{Key
  Dates}
\item
  \href{https://www.nytimes3xbfgragh.onion/newsletters/politics?action=click\&pgtype=Article\&state=default\&region=TOP_BANNER\&context=storylines_menu}{Politics
  Newsletter}
\end{itemize}

Advertisement

\protect\hyperlink{after-top}{Continue reading the main story}

Supported by

\protect\hyperlink{after-sponsor}{Continue reading the main story}

\hypertarget{how-buying-beans-became-a-political-statement}{%
\section{How Buying Beans Became a Political
Statement}\label{how-buying-beans-became-a-political-statement}}

The boycott and counter-boycott of Goya come as the major political
parties seek to energize Hispanic support ahead of the 2020 election.

\includegraphics{https://static01.graylady3jvrrxbe.onion/images/2020/07/19/us/19goya-print1-sub/merlin_174687735_4788dd33-f0cf-4042-9e87-ba40c9252ff4-articleLarge.jpg?quality=75\&auto=webp\&disable=upscale}

By \href{https://www.nytimes3xbfgragh.onion/by/farah-stockman}{Farah
Stockman}, \href{https://www.nytimes3xbfgragh.onion/by/kate-kelly}{Kate
Kelly} and
\href{https://www.nytimes3xbfgragh.onion/by/jennifer-medina}{Jennifer
Medina}

\begin{itemize}
\item
  Published July 19, 2020Updated July 21, 2020
\item
  \begin{itemize}
  \item
  \item
  \item
  \item
  \item
  \end{itemize}
\end{itemize}

\href{https://www.nytimes3xbfgragh.onion/es/2020/07/19/espanol/goya-boicot-trump.html}{Leer
en español}

For years, the Goya brand was synonymous with the Latino-American dream.
The sheer number of products that lined the grocery store aisles ---
from refried pinto beans to sazón con azafran seasoning --- spoke to the
growing number of Hispanic immigrants who bought them. Goya, the
nation's largest Hispanic food company, has sponsored Dominican art
shows, mariachi contests and soccer programs.

Advisers to President Trump considered it a victory when Goya's chief
executive, Robert Unanue, agreed to appear at the White House rollout of
what it called the
\href{https://www.whitehouse.gov/presidential-actions/executive-order-white-house-hispanic-prosperity-initiative/}{Hispanic
Prosperity Initiative}, an executive order that promised better access
to education and employment for Hispanics.

In the Rose Garden on July 9, Mr. Unanue praised Mr. Trump and compared
him to his grandfather, who founded Goya.

``We're all truly blessed at the same time to have a leader like
President Trump, who is a builder,'' said Mr. Unanue, a registered
Republican. ``And that's what my grandfather did.''

And just like that, a once-beloved brand became anathema in many Latino
homes across the United States. People posted videos and photos of
themselves clearing out their pantries and tossing cans of Goya beans
into the trash. It became a symbol of political resistance to share
recipes for Goya product substitutes. ``Oh look, it's the sound of me
Googling `how to make your own Adobo,''' Representative Alexandria
Ocasio-Cortez, Democrat of New York
\href{https://twitter.com/AOC/status/1281383352315125762}{wrote on
Twitter}, referring to a popular seasoning that Goya sells.

\includegraphics{https://static01.graylady3jvrrxbe.onion/images/2020/07/18/us/18goya02/merlin_174428367_d3132deb-aeb7-4d7f-a32b-0dba769a5de2-articleLarge.jpg?quality=75\&auto=webp\&disable=upscale}

Almost immediately, Trump loyalists pushed back --- filling shopping
carts full of Goya products and posting videos of themselves dutifully
swallowing Goya beans.

By the time
\href{https://twitter.com/IvankaTrump/status/1283221019684110337}{Ivanka
Trump tweeted} an endorsement of Goya, one thing had become clear: In a
polarized country, at a polarized time, the buying of beans had become a
political act.

Even as Mr. Trump's support has cratered among many demographics, he has
held on to a small but durable slice of Hispanic voters, many of them in
Florida, a state full of Cuban Republicans that is known for razor-thin
electoral margins.

Polls consistently show Mr. Trump with an approval rating among Hispanic
voters hovering around 25 percent, within the lower end of the range
that Republican presidents have attracted for decades. Before the
coronavirus pandemic tanked the economy, the Trump campaign repeatedly
pointed to the low unemployment rate among Hispanics as an indication
that the administration was delivering for the community, a group he has
also offended with inflammatory remarks about immigration.

Now Goya has fallen into this boiling pot of politics and anger, a
strange turn of events for a company that has prided itself on knowing
its customers intimately.

With each wave of Hispanic immigrants from Latin America and the
Caribbean, Goya has added new products to suit their cuisine, and over
the years it has distributed millions of pounds of food to pantries
after hurricanes and during the pandemic.

The company was founded in 1936 by Mr. Unanue's grandparents, who moved
from the Basque region of Spain to Puerto Rico, and then New York City,
where they sold sardines and olive oil from a storefront on Duane Street
in Lower Manhattan.

As the company expanded, it changed its name from Unanue \& Sons to Goya
Foods ---
\href{https://abcnews.go.com/Business/story?id=4507435\&page=1}{reportedly
buying rights to its new name for \$1} because it was easier to
pronounce than ``oo-NA-new-way'' --- and branched into manufacturing.
During the mid-1970s, Joseph Unanue, one of the founders' four sons,
took over as chief executive, and the company relocated to New Jersey.
By the time he stepped down, the company had established relationships
with Walmart and other big grocers and its annual revenue had grown to
\$1 billion from \$20 million.

Some noted that Robert Unanue's remarks at the White House showcased the
glaring disconnect between the wealthy executive whose family hailed
from Spain and the largely working-class Latinos who make up his
customer base. The harshest critics questioned whether he considered
himself Latino.

Image

Shoppers looked at Goya canned foods at a grocery store in Austin,
Texas, on Friday.Credit...Sergio Flores for The New York Times

The speed and size of the boycott speak to ``how raw people in the
community feel about the president,'' said Clarissa Martinez de Castro,
the deputy vice president for policy and advocacy for UnidosUS, a Latino
civic engagement organization. She said many Latinos blamed Mr. Trump's
attacks on undocumented immigrants for inciting discrimination and
violence against Latinos, particularly the massacre last summer in El
Paso.

For the first time, she said, anxieties about racial discrimination have
ranked in the top concerns
\href{https://www.unidosus.org/Assets/uploads/files/voting/UnidosUS_2020latinoelectoratesurvey_june2019.pdf}{among
Latino voters} in surveys. But Mr. Trump's supporters are betting that
this is a winning issue for them and that Americans won't understand or
empathize with the boycott.

The day after the Rose Garden ceremony, Senator Ted Cruz, Republican of
Texas,
\href{https://twitter.com/tedcruz/status/1281679899607142400}{tweeted}:
``Goya is a staple of Cuban food. My grandparents ate Goya black beans
twice a day for nearly 90 years. And now the Left is trying to cancel
Hispanic culture and silence free speech. \#BuyGoya.''

And suddenly, the once-beloved Hispanic brand became a cause célèbre on
the right.

Mr. Cruz said in an interview that he saw the boycott as an example of
``spirit of intolerance.''

``The offense is he dared to say he supported the president,'' Mr. Cruz
said, adding that ``anytime anyone dares disagree from their rigid
orthodoxy, they seek to punish, cancel or destroy to the dissenter.''

Mr. Unanue, who has contributed to the campaigns of both Democrats and
Republicans and worked with Michelle Obama on an anti-obesity
initiative, appeared unprepared for the firestorm. Neither he nor Goya
officials responded to requests for comment. But Mr. Unanue defended his
remarks at the White House, ****
\href{https://www.wsj.com/articles/goyas-ceo-softly-stands-his-ground-11595008616}{telling
The Wall Street Journal} that he went there out of respect. ``I remain
strong in my convictions that I feel blessed with the leadership of our
president,'' he told the newspaper.

Trump supporters filmed themselves filling shopping carts full of Goya
products, relishing in the opportunity to defend a Hispanic businessman
and accuse Democrats of being anti-Latino. Dinesh D'Souza, a
conservative political commentator,
\href{https://twitter.com/DineshDSouza/status/1282755880010817536}{shared
a video} of himself swallowing beans, which he admitted he rarely ate.

A few days later, Mr. Trump circulated a photo of himself sitting in the
Oval Office, smiling widely and with his thumbs up, in front of several
Goya products, including a package of chocolate wafers and coconut milk.

Responding to questions about whether Ms. Trump's tweet violated federal
law forbidding government employees from using their positions to
endorse products, Carolina Hurley, a White House spokeswoman, said the
president's daughter ``has every right to express her personal support''
for the company.

Image

Volunteers listening to a Hispanic engagement coordinator at the
Republican National Committee in Virginia in 2019.Credit...Samuel Corum
for The New York Times

``Only the media and the cancel culture movement would criticize Ivanka
for showing her personal support for a company that has been unfairly
mocked, boycotted and ridiculed for supporting this administration ---
one that has consistently fought for and delivered for the Hispanic
community,'' Ms. Hurley said.

Some political scientists said Mr. Trump appeared eager for the free
publicity that came by associating himself with a beloved Hispanic
brand.

``It's the Republican version of `Hispandering,''' said Geraldo Cadava,
a history professor at Northwestern University and author of ``The
Hispanic Republican.'' ``He's pandering to Hispanics the same way that
politicians have peppered their stump speeches with a few words in
Spanish. It's the same kind of signal.''

Mr. Trump has occasionally made visible efforts to reach Hispanic
voters. The Hispanic Prosperity Initiative, which included few details,
came during a week in which he also met with Venezuelans who had fled
socialism and held an interview with Telemundo, a Spanish-language
television station. Mr. Trump spoke in the interview about a ``road to
citizenship'' for undocumented immigrants brought to the United States
as children, even as his administration has pledged to fight a Supreme
Court decision upholding the Obama-era program that protected them.

It remains to be seen whether Hispanics who do not already support Trump
will be swayed by his sudden association with Goya or his attempt to
bring Hispanics onto the conservative side of the nation's
long-simmering culture war.

But for a few Latinos, the message resonated.

Alexander Otaola, a Cuban-American in Florida with
\href{https://slack-redir.net/link?url=https\%3A\%2F\%2Fwww.instagram.com\%2Falexanderotaola\%2F}{105,000
followers on Instagram}, issued a video in Spanish that likened the Goya
boycott to the destruction of statues and other cultural icons.

``What is Goya in the Latino community? It's an icon, a statue,'' he
said in the YouTube video. ``The left wants to destroy all icons.''

It is not clear how deeply the boycott has cut into Goya's bottom line,
or whether the impact of the ``buycott'' has canceled it out. Goya is a
privately held company, so its records are not public.

In Jerry's Supermarket in the predominantly Latino Oak Cliff community
in Dallas, Goya products lined the shelves, as usual, and were bought by
a steady stream of customers last weekend. In San Antonio's Alamo
Heights community, one cashier said managers of La Michoacana
Supermarket have not said they would quit carrying Goya products. Guava
paste and Salvadoran pickled salad, among other items, remained on the
shelves.

But in Tucson, Ariz., Patrick Robles, a 19-year-old student at the
University of Arizona, said his whole family was boycotting Goya
products even though the company's chickpeas had always been perfect for
cocido, or Mexican stew.

``It was a punch in the stomach for us,'' Mr. Robles said of Mr.
Unanue's comments praising a president who Mr. Robles felt has routinely
devalued Latinos. Now, they are going to turn to brands like La Costeña
or Rosarita.

But Pamela Ramirez, a 48-year-old Mexican-American small-business
consultant in East Los Angeles, said she strongly opposed the Goya
boycott. Since there is a large number of Latinos employed by the
company, she thinks that boycotting the product could harm her own
community. For every one of her Facebook friends who has posted about
boycotting the product, Ms. Ramirez bought \$10 worth of Goya products
and donated them to a food bank, she said.

``You've got to put your money where your mouth is,'' she said. ``If you
don't, then you're just part of the problem.''

Contributing reporting were Elda Lizzia Cantú, Giulia McDonnell Nieto
del Rio, Marina Trahan Martinez, Erin Coulehan and David Montgomery.
Sheelagh McNeill contributed research.

\hypertarget{our-2020-election-guide}{%
\section{Our 2020 Election Guide}\label{our-2020-election-guide}}

Updated ~Sept. 8, 2020

\begin{center}\rule{0.5\linewidth}{\linethickness}\end{center}

\begin{itemize}
\item ~
  \hypertarget{the-latest}{%
  \subsection{The Latest}\label{the-latest}}

  \begin{itemize}
  \item
    President Trump and his party are using a playbook that aims to
    alarm people about crime in their backyards. It didn't work in 2018,
    but
    \href{https://www.nytimes3xbfgragh.onion/2020/09/08/us/politics/trump-republicans-fear-strategy.html?action=click\&pgtype=Article\&state=default\&region=BELOW_MAIN_CONTENT\&context=storylines_guide}{both
    parties think it could resonate more this year}.
  \end{itemize}
\item ~
  \hypertarget{how-to-win-270}{%
  \subsection{How to Win 270}\label{how-to-win-270}}

  \begin{itemize}
  \item
    Joe Biden and Donald Trump need 270 electoral votes to reach the
    White House. Try building
    \href{https://www.nytimes3xbfgragh.onion/interactive/2020/us/elections/election-states-biden-trump.html?action=click\&pgtype=Article\&state=default\&region=BELOW_MAIN_CONTENT\&context=storylines_guide}{your
    own coalition of battleground states}~to see potential outcomes.
  \end{itemize}
\item ~
  \hypertarget{voting-by-mail}{%
  \subsection{Voting by Mail}\label{voting-by-mail}}

  \begin{itemize}
  \item
    Will you have enough time to vote by mail in your state? Yes, but
    it's risky to procrastinate.
    \href{https://www.nytimes3xbfgragh.onion/interactive/2020/08/31/us/politics/vote-by-mail-deadlines.html?action=click\&pgtype=Article\&state=default\&region=BELOW_MAIN_CONTENT\&context=storylines_guide}{Check
    your state's deadline.}
  \item
    \href{https://www.nytimes3xbfgragh.onion/interactive/2020/us/elections/joe-biden.html?action=click\&pgtype=Article\&state=default\&region=BELOW_MAIN_CONTENT\&context=storylines_guide}{}

    \hypertarget{joe-biden}{%
    \section{Joe Biden}\label{joe-biden}}

    \hypertarget{democrat}{%
    \subsection{Democrat}\label{democrat}}

    \href{https://www.nytimes3xbfgragh.onion/interactive/2020/us/elections/donald-trump.html?action=click\&pgtype=Article\&state=default\&region=BELOW_MAIN_CONTENT\&context=storylines_guide}{}

    \hypertarget{donald-trump}{%
    \section{Donald Trump}\label{donald-trump}}

    \hypertarget{republican}{%
    \subsection{Republican}\label{republican}}
  \end{itemize}
\item
  \hypertarget{keep-up-with-our-coverage}{%
  \subsection{Keep Up With Our
  Coverage}\label{keep-up-with-our-coverage}}

  \begin{itemize}
  \item
    Get an
    \href{https://www.nytimes3xbfgragh.onion/newsletters/politics?action=click\&pgtype=Article\&state=default\&region=BELOW_MAIN_CONTENT\&context=storylines_guide}{email}~recapping
    the day's news
  \item
    Download our mobile app on
    \href{https://apps.apple.com/us/app/nytimes/id284862083?ls=1\&mat_click_id=5c79ae7455014fd1bd66b5610c05b8f2-20191112-16948\&referrer=mat_click_id\%3D5c79ae7455014fd1bd66b5610c05b8f2-20191112-16948\%26link_click_id\%3D722930677036718082}{iOS}~and
    \href{http://a.localytics.com/android?id=com.nytimes.android\&referrer=utm_source\%3Dother_nyt_mobile_web\%26utm_medium\%3DWeb\%2520page\%26utm_term\%3DGeneral\%2520Mobile\%2520Page\%26utm_campaign\%3DNYT\%2520Mobile\%2520General\%2520Page}{Android}~and
    turn on Breaking News and Politics alerts
  \end{itemize}
\end{itemize}

Advertisement

\protect\hyperlink{after-bottom}{Continue reading the main story}

\hypertarget{site-index}{%
\subsection{Site Index}\label{site-index}}

\hypertarget{site-information-navigation}{%
\subsection{Site Information
Navigation}\label{site-information-navigation}}

\begin{itemize}
\tightlist
\item
  \href{https://help.nytimes3xbfgragh.onion/hc/en-us/articles/115014792127-Copyright-notice}{©~2020~The
  New York Times Company}
\end{itemize}

\begin{itemize}
\tightlist
\item
  \href{https://www.nytco.com/}{NYTCo}
\item
  \href{https://help.nytimes3xbfgragh.onion/hc/en-us/articles/115015385887-Contact-Us}{Contact
  Us}
\item
  \href{https://www.nytco.com/careers/}{Work with us}
\item
  \href{https://nytmediakit.com/}{Advertise}
\item
  \href{http://www.tbrandstudio.com/}{T Brand Studio}
\item
  \href{https://www.nytimes3xbfgragh.onion/privacy/cookie-policy\#how-do-i-manage-trackers}{Your
  Ad Choices}
\item
  \href{https://www.nytimes3xbfgragh.onion/privacy}{Privacy}
\item
  \href{https://help.nytimes3xbfgragh.onion/hc/en-us/articles/115014893428-Terms-of-service}{Terms
  of Service}
\item
  \href{https://help.nytimes3xbfgragh.onion/hc/en-us/articles/115014893968-Terms-of-sale}{Terms
  of Sale}
\item
  \href{https://spiderbites.nytimes3xbfgragh.onion}{Site Map}
\item
  \href{https://help.nytimes3xbfgragh.onion/hc/en-us}{Help}
\item
  \href{https://www.nytimes3xbfgragh.onion/subscription?campaignId=37WXW}{Subscriptions}
\end{itemize}
