Sections

SEARCH

\protect\hyperlink{site-content}{Skip to
content}\protect\hyperlink{site-index}{Skip to site index}

\href{https://www.nytimes3xbfgragh.onion/section/sports/soccer}{Soccer}

\href{https://myaccount.nytimes3xbfgragh.onion/auth/login?response_type=cookie\&client_id=vi}{}

\href{https://www.nytimes3xbfgragh.onion/section/todayspaper}{Today's
Paper}

\href{/section/sports/soccer}{Soccer}\textbar{}Federal Loan Saved a
Soccer Season Nearly Lost to the Pandemic

\url{https://nyti.ms/3eW2o6i}

\begin{itemize}
\item
\item
\item
\item
\item
\end{itemize}

Advertisement

\protect\hyperlink{after-top}{Continue reading the main story}

Supported by

\protect\hyperlink{after-sponsor}{Continue reading the main story}

\hypertarget{federal-loan-saved-a-soccer-season-nearly-lost-to-the-pandemic}{%
\section{Federal Loan Saved a Soccer Season Nearly Lost to the
Pandemic}\label{federal-loan-saved-a-soccer-season-nearly-lost-to-the-pandemic}}

The N.W.S.L. was one of millions of businesses that took Paycheck
Protection Program loans this year. The money got the league and its
players through eight uncertain weeks.

\includegraphics{https://static01.graylady3jvrrxbe.onion/images/2020/07/02/sports/02soccer-ppp/merlin_174137172_54438fbd-a49a-4f4c-9a24-8883ac868e41-articleLarge.jpg?quality=75\&auto=webp\&disable=upscale}

By \href{https://www.nytimes3xbfgragh.onion/by/andrew-keh}{Andrew Keh}
and Andrew Das

\begin{itemize}
\item
  July 2, 2020
\item
  \begin{itemize}
  \item
  \item
  \item
  \item
  \item
  \end{itemize}
\end{itemize}

As the coronavirus pandemic spread early this year, shutting down large
portions of the United States economy, the federal government handed out
more than \$520 billion in forgivable loans to more than five million
small businesses. Called the
\href{https://www.nytimes3xbfgragh.onion/2020/07/06/us/ppp-small-business-loans.html}{Paycheck
Protection Program}, the money was to be used to replace the incomes of
employees suddenly unable to work as cities and states entered lockdowns
to reduce infection rates.

The recipients ranged from clothing stores to car dealerships to coffee
shops.

They also included a women's professional soccer league.

The National Women's Soccer League confirmed Thursday afternoon that it
had received an unspecified amount from the program, and had used the
money as a bridge to pay players for two months. It expects the loan to
be forgiven.

Other sports entities were publicly involved in the rocky beginnings of
the Paycheck Protection Program. The Los Angeles Lakers --- the
eighth-richest sports franchise in the world, by one measurement ---
\href{https://www.nytimes3xbfgragh.onion/2020/04/28/us/politics/coronavirus-treasury-payment-protection-program.html}{secured
a \$4.6 million loan} before quickly returning it amid news reports that
smaller companies in desperate need of the funds were finding it hard to
access the program. The United States Soccer Federation
\href{https://www.espn.com/soccer/united-states-usa/story/4095017/us-soccer-returns-paycheck-protection-loan-to-us-treasury}{returned
the money it had received, too}, once it became clear that the layoffs
and furloughs it had implemented would disqualify it for loan
forgiveness.

But the N.W.S.L. desperately needed the money as it sought to salvage
its eighth season --- not to mention several hundred jobs. One hundred
percent of the funds were used to pay player salaries for more than two
months, the league said.

``Our sole intent in applying for the P.P.P. loan was to continue player
compensation,'' said Lisa Baird, the league's first-year commissioner.
``With ours, the calculus was pretty simple. Either you're going to pay
your players or you're going to furlough them. What could we do?''

\includegraphics{https://static01.graylady3jvrrxbe.onion/images/2020/07/02/sports/02soccer-ppp3/02soccer-ppp3-articleLarge.jpg?quality=75\&auto=webp\&disable=upscale}

The decision to apply for the loan came amid Baird's tumultuous start as
commissioner. She officially took over the position on March 10. Two
days later, she shut down the league in response to the pandemic. She
called this period ``the fog.'' The league, like every other business,
was scrambling.

``Nobody knew what was going on,'' she said.

Baird said the N.W.S.L., as a small operation with a brief history and
only a few hundred employees, was almost entirely reliant on stadium
ticket sales and local sponsorships for revenue. Those disappeared when
this spring's games were called off. When Baird took the job, the league
had only two national sponsors.

After consulting with league officials and owners, Baird said she
gathered a thousand backup documents --- ``literally a thousand'' ---
and sat at her computer for six hours to personally submit the
application.

``Over 85 percent of our total monthly expenses is player
compensation,'' she said of the league's finances. ``I didn't even want
to stare at the alternatives. For me, I had to have this happen.''

In important ways, the N.W.S.L. fulfilled the strict qualifications set
out by the federal government relief program: A relatively young
business, it had fewer than 500 employees; the pandemic had effectively
barred its employees from working; and its workers' incomes, and perhaps
careers, could have disappeared altogether if the league had been unable
to play matches.

Baird, who has only 16 people on staff to go along with more than 200
players, described the P.P.P. funding as the bridge that allowed them to
escape the early danger of the pandemic. (The league's top players and
biggest earners --- more than two dozen members of the women's national
team --- are paid by U.S. Soccer.)

The loan gave the league the wiggle room to plan an improvised
summertime tournament
\href{https://www.nytimes3xbfgragh.onion/2020/06/26/sports/soccer/nwsl-anthem-protest-kaiya-mccullough.html}{that
began last week in Utah} and to secure commitments from
\href{https://twitter.com/NWSL/status/1265612731186634755}{three new
national sponsors}. Those new deals, Baird said, will help ensure that
players will be paid and get all of their health benefits for the rest
of the year.

Image

Unable to play in its home markets, the N.W.S.L. is playing its 2020
season in empty stadiums in Utah this month.Credit...Alex Goodlett/Getty
Images

The Lakers episode, though, showed how touchy people could be about
learning that sports teams and leagues --- not to mention prominent
brands like Shake Shack and the Fortune 500 car retailer AutoNation ---
had
\href{https://www.nytimes3xbfgragh.onion/2020/04/28/us/politics/coronavirus-treasury-payment-protection-program.html}{applied
for and received relief funds}. In England, soccer teams like Tottenham
and Newcastle United were criticized by their own fans for seeking large
government loans amid the coronavirus crisis.

Baird said that she and the league's owners, who include wealthy
businessmen,
\href{https://www.reuters.com/article/us-soccer-women-nwsl/lyon-parent-company-acquires-nwsls-reign-fc-idUSKBN1YO002}{a
prominent French soccer club} and even
\href{https://skybluefc.com/ownership/}{the governor of New Jersey}, had
weighed the pros and cons and discussed the possibility of receiving
negative attention for applying for the loan. But they quickly realized
that it was too important for them to be able to pay the players, she
said. In the N.W.S.L., rank-and-file players are employees of the
league, not of the individual teams.

``That benefit outweighed anything that we could think of in terms of
optics,'' Baird said.

Image

The salaries of many of the N.W.S.L.'s best players, like the national
team midfielder Rose Lavelle, right, are paid by U.S. Soccer, not the
league.Credit...Alex Goodlett/Getty Images

Baird said she expected the league's P.P.P. loan would qualify for
forgiveness because 100 percent of the money was used for payroll (the
government requires that at least 60 percent of the funds be used for
payroll for a loan to be forgiven) and because the N.W.S.L. did not lay
off any employees, another condition.

``I pray for that in my nightly prayers,'' said Baird, who said she and
some other highly compensated league officials had taken pay cuts in a
separate attempt at belt-tightening.

Having used the loan as a lifeboat, Baird on Thursday cautiously deemed
the summer and the nascent tournament a success.

An entire team, the Orlando Pride, dropped out before play even began
after six players and four staff members tested positive for the virus.
But the first game --- in which players made headlines for kneeling
during the national anthem --- drew the highest television rating in the
league's history.

``It gives us confidence that we can be a small business that can make
it through this time,'' Baird said. ``That's a key thing: We're not a
large business. We're a small business.''

Advertisement

\protect\hyperlink{after-bottom}{Continue reading the main story}

\hypertarget{site-index}{%
\subsection{Site Index}\label{site-index}}

\hypertarget{site-information-navigation}{%
\subsection{Site Information
Navigation}\label{site-information-navigation}}

\begin{itemize}
\tightlist
\item
  \href{https://help.nytimes3xbfgragh.onion/hc/en-us/articles/115014792127-Copyright-notice}{©~2020~The
  New York Times Company}
\end{itemize}

\begin{itemize}
\tightlist
\item
  \href{https://www.nytco.com/}{NYTCo}
\item
  \href{https://help.nytimes3xbfgragh.onion/hc/en-us/articles/115015385887-Contact-Us}{Contact
  Us}
\item
  \href{https://www.nytco.com/careers/}{Work with us}
\item
  \href{https://nytmediakit.com/}{Advertise}
\item
  \href{http://www.tbrandstudio.com/}{T Brand Studio}
\item
  \href{https://www.nytimes3xbfgragh.onion/privacy/cookie-policy\#how-do-i-manage-trackers}{Your
  Ad Choices}
\item
  \href{https://www.nytimes3xbfgragh.onion/privacy}{Privacy}
\item
  \href{https://help.nytimes3xbfgragh.onion/hc/en-us/articles/115014893428-Terms-of-service}{Terms
  of Service}
\item
  \href{https://help.nytimes3xbfgragh.onion/hc/en-us/articles/115014893968-Terms-of-sale}{Terms
  of Sale}
\item
  \href{https://spiderbites.nytimes3xbfgragh.onion}{Site Map}
\item
  \href{https://help.nytimes3xbfgragh.onion/hc/en-us}{Help}
\item
  \href{https://www.nytimes3xbfgragh.onion/subscription?campaignId=37WXW}{Subscriptions}
\end{itemize}
