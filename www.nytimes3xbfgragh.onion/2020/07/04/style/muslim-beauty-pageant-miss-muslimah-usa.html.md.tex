Sections

SEARCH

\protect\hyperlink{site-content}{Skip to
content}\protect\hyperlink{site-index}{Skip to site index}

\href{https://www.nytimes3xbfgragh.onion/section/style}{Style}

\href{https://myaccount.nytimes3xbfgragh.onion/auth/login?response_type=cookie\&client_id=vi}{}

\href{https://www.nytimes3xbfgragh.onion/section/todayspaper}{Today's
Paper}

\href{/section/style}{Style}\textbar{}`I Am Here to Prove You Wrong'

\url{https://nyti.ms/2ZujX6I}

\begin{itemize}
\item
\item
\item
\item
\item
\end{itemize}

Advertisement

\protect\hyperlink{after-top}{Continue reading the main story}

Supported by

\protect\hyperlink{after-sponsor}{Continue reading the main story}

The Look

\hypertarget{i-am-here-to-prove-you-wrong}{%
\section{`I Am Here to Prove You
Wrong'}\label{i-am-here-to-prove-you-wrong}}

At Miss Muslimah USA, a pageant for young Muslim women, the complexity
of modesty is on full display.

Photographs by Farah Al Qasimi

Text by Liana Aghajanian

\includegraphics{https://static01.graylady3jvrrxbe.onion/images/2020/07/16/fashion/16THELOOK-MISSMUSLIMAH2/16THELOOK-MISSMUSLIMAH2-articleLarge-v2.jpg?quality=75\&auto=webp\&disable=upscale}

Last year, on a Thursday in June, long before live events and large
gatherings bore the threat of contagion, the ballroom of the Ford
Community and Performing Arts Center in Dearborn, Mich., was in full
pageant form.

Pink mini cupcakes filled the dessert table. A disco ball hung from the
ceiling, spinning subtly as the D.J. set the mood with music. Seats for
guests were draped in shiny gold fabric.

Wine however, was swapped for Welch's sparkling red grape juice. The
talent portion of the evening was made up entirely of readings from the
Quran. A magician performed what he jokingly called ``halal magic.'' The
musical act performed Muslim hip-hop.

For a century, the beauty pageant has embedded itself in the cultural
identity of America. Miss Muslimah USA offers a fresh take on the
well-worn event format, one that lies at the intersection of American
cultural identity and religious freedom at a time when both seem to be
in flux.

Image

Unlike traditional American beauty pageants, Miss Muslimah is focused on
modesty. Here, Zeytuna Mohamed, representing Iowa, got ready for the
special-occasion-wear segment.Credit...Farah Al Qasimi for The New York
Times

The pageant has given Muslim women, particularly those who wear the
hijab, the chance to participate in an American rite on their own terms,
without having to compromise their faith. (Its motto: ``promoting
modesty and inner beauty.'') It was created by Maghrib Shahid, a
39-year-old Black Muslim mother and modest clothing designer from
Columbus, Ohio.

As a hijabi, a Muslim woman who wears a head scarf, Ms. Shahid felt that
she and other women like her bore the brunt of discrimination against
Muslims, a diverse population
\href{https://www.pewresearch.org/fact-tank/2018/01/03/new-estimates-show-u-s-muslim-population-continues-to-grow/}{estimated
to number more than three million in the United States}.

President Trump --- a former pageant-world figure himself --- has
inflamed Islamophobia in the nation, through his rhetoric and by banning
migration from several majority-Muslim countries.

``We're visibly Muslim, it's us who will be attacked first,'' Ms. Shahid
said. ``I wanted to give Muslim women the opportunity to change
misconceptions about themselves.''

Image

Dearborn, Mich., is home to one of the largest and most concentrated
Arab-American populations in the country.Credit...Farah Al Qasimi for
The New York Times

Image

Islamic specialty stores, such as Mekkah Islamic Superstore, can be
found all around the city.Credit...Farah Al Qasimi for The New York
Times

Image

Hijab pins for sale at~Mekkah Islamic Superstore.Credit...Farah Al
Qasimi for The New York Times

Halima Yasin Abdullahi, 23, who was crowned in the first Miss Muslimah
pageant in 2017, said that two years on, she still feels its impact.

``I've gained a really strong and consistent confidence in myself, and
learned to appreciate my flaws,'' she said. ``This is me. This is how I
was born.''

To enter Miss Muslimah USA, contestants must be practicing Muslims aged
17 to 30, a range established after the first pageant, which accepted
contestants up to 40 years of age. There's a \$250 registration fee and
a screening process. Once they are enrolled, they can prepare to compete
in five categories: abayah (a loose, robelike dress), burkini (a
swimsuit that covers the whole body), modest special occasion dress
(dresses that are too tight could lead to disqualification) and talent,
which may be a spoken word poem or a Quran recitation.

Contestants must also answer this question: ``If you were crowned Miss
Muslimah USA, how would you use that title to change misconceptions
about Muslim women in the world?''

Image

Maghrib Shahid, the founder of Miss Muslimah USA.Credit...Farah Al
Qasimi for The New York Times

The winner holds the Miss Muslimah USA title for a year, signs a
contract to abide by certain codes of conduct, is managed by the
organization and walks in a show at an annual fashion convention hosted
by Perfect for Her, a modest wear brand. Ms. Shahid helps the winner
navigate sponsorships and fashion bookings.

The first pageant was advertised to include a \$5,000 prize for the
winner. Subsequent pageants have not offered monetary rewards, though
Ms. Shahid's hope is to offer scholarships in the future.

Running the pageant on a shoestring budget by herself, Ms. Shahid dipped
into her savings to bring Halima Aden, a Somali-American model, to
Columbus for the first Miss Muslimah USA. Ms. Aden was the first
contestant to wear a hijab in the Miss Minnesota pageant in 2016, and
went on to become the first woman to wear a hijab and burkini in Sports
Illustrated, in 2019.

Image

Andrea Rahal at home with her son. She convinced Ms. Shahid to move the
pageant, originally held in Columbus, Ohio, to her hometown,
Dearborn.Credit...Farah Al Qasimi for The New York Times

``It's not about becoming rich or wealthy. It's about making a true
difference, a real impact,'' Ms. Shahid said. ``I want people to really
benefit from this. I want to change your life. I want to change your
soul.''

Her passion for pageants began in childhood; she told herself that
someday she would enter a competition. ``As I got older, I realized, I
don't see anybody like me --- who looks like me and the way I dress,''
she said. ``It became a distant dream.''

Now that she has Miss Muslimah, she said, ``I'm living my dream through
these women.''

Image

Ms. Rahal's keys.Credit...Farah Al Qasimi for The New York Times

Backstage last July, the contestants strapped on heels, adjusted the
gowns they had modified with sleeves and high necklines, and helped one
another tuck in their scarves before being called onstage.

Andrea Rahal, 30, whose sister Amanda and cousin Amal were helping her
into a silver sequined gown and white hijab, was one of them. Born to
Lebanese parents and raised in Dearborn, home to one of the largest
Arab-American populations in the country, Ms. Rahal has worn a hijab
since she was 8., She now works as a phlebotomist and medical assistant,
and is a single mother of two.

Ms. Rahal rallied her community around last year's pageant. She found 30
sponsors for the event and convinced Ms. Shahid to move the event from
Columbus to Dearborn.

Image

Mariam Hussein, representing Michigan, backstage at the
pageant.Credit...Farah Al Qasimi for The New York Times

Image

Ms. Rahal prepared for the formal wear segment backstage.Credit...Farah
Al Qasimi for The New York Times

Image

Watching the show.Credit...Farah Al Qasimi for The New York Times

``When I found Miss Muslimah, I never thought an opportunity like that
would pop up,'' Ms. Rahal said. ``It was always a dream for me to be
part of a pageant, so when something comes your way, always take the
risk and take the chance.''

The contestants strutted down the catwalk in their gowns one by one.
Karter Zaher, a former member of Deen Squad, a popular Muslim hip-hop
group, sang the hit song ``Cover Girl'' (which includes lines such as
``she represents peace and got her own voice, she's not forced to wear
it cos' she made her own choice'' and ``she rocks the head scarf like
the mother of Jesus'').

Wearing their gowns, the women moved on to recite their speeches, which
touched on Islamophobia, feminism, self-care and the desire to be seen
as multidimensional people in American society.

Image

The burkini segment.Credit...Farah Al Qasimi for The New York Times

Image

The dessert table.Credit...Farah Al Qasimi for The New York Times

Image

Amina Abdikadir, representing New York.Credit...Farah Al Qasimi for The
New York Times

``I am a Muslim feminist,'' Zeytuna Mohamed, a 22-year-old nursing
student from Des Moines, said onstage. ``Many people think that those
two words are incompatible, but I am here to prove you wrong. I am not
oppressed. I am not passive, and I am certainly not caged.''

Umuhani Abdullahi, 20 and representing Ohio, said in her speech: ``This
is my home, America. This is the only home that I know right now. I
passionately dream of seeing girls like me in fashion books, on
billboards, in Coca-Cola advertisements and obviously in movies.
Hopefully Netflix.''

Just like several American beauty pageants, Miss Muslimah has had its
share of shake-ups while attempting to establish itself as a legitimate
organization.

In 2017, Dr. Khadijah Ismael, 42, won the first pageant, in which she
ran on a platform of knocking down stereotypes about Muslim women. After
winning, she traveled on a speaking tour which she paid for. But
disagreements between Dr. Ishmael and the Miss Muslimah organization
arose, and a month before her reign was over she was informed that she
was disqualified.

Contractual issues caused Rahma Mohamed, who was crowned the winner in
2019, and Miss Muslimah USA to part ways. Ms. Mohamed, a 17-year-old
from Wisconsin who is studying mechanical engineering, was a
semifinalist in the Miss Wisconsin Teen USA pageant and later went on to
represent her state in Miss Teen World America. She was the first Muslim
to place in the competition.

Image

Ms. Rahal on the pageant stage.Credit...Farah Al Qasimi for The New York
Times

Dr. Ismael, a dentist, went on to create Women of Wellness of New
Jersey, an organization that produces the Miss Glitz, Glamour, and
Brains USA in S.T.E.M pageant, which ``showcases the beauty of the
mind.'' She and Ms. Shahid are now on good terms. ``I thank the
organization for being the catalyst for me and many other women to do
many productive things in the community and beyond,'' Dr. Ishmael said.

Ms. Shahid said she has received backlash from fellow Muslims who
thought the premise of the pageant defied the very definition of modesty
by putting women in the spotlight. She remained undeterred.

``We're living in the real world. We have to make noise. If we want to
change we have to make change,'' she said. ``I found myself trying to
show Muslims it's OK to come out of your comfort zone, it's OK to be
part of a pageant. I understand that this opportunity was never provided
to you, but it's OK now.''

Image

A window display at Mekkah Islamic Superstore.Credit...Farah Al Qasimi
for The New York Times

The pageant itself is adapting, defying traditions it established early
on to embrace the complexity of the very community it hopes to uplift.
In 2018, as a way to welcome new converts and young women who couldn't
speak Arabic, contestants were given the choice between reciting from
the Quran or reading a poem.

This year, non-hijabi Muslims will be allowed to enter and compete
alongside hijab-wearing contestants, Two international contestants ---
from Kazakhstan and Britain --- will also be competing.

Ms. Shahid thinks there's still so much work to do to reach the
pageant's full potential. She pointed to the rise of the Miss USA
pageant, which grew out of the Miss America pageant after the winner
Yolande Betbeze Fox refused to pose for publicity shots while wearing a
swimsuit in 1950.

``It took time for them to build,'' Ms. Shahid said. ``If you support
Miss Muslimah, in the next 10 years we'll also have that great
momentum.''

\begin{center}\rule{0.5\linewidth}{\linethickness}\end{center}

\href{https://www.nytimes3xbfgragh.onion/column/the-look}{The Look} is a
column that examines identity through a visual-first lens. This year,
the column is focused on the relationship between American culture and
politics in the run-up to the 2020 presidential election, produced by
Eve Lyons and Tanner Curtis.

Advertisement

\protect\hyperlink{after-bottom}{Continue reading the main story}

\hypertarget{site-index}{%
\subsection{Site Index}\label{site-index}}

\hypertarget{site-information-navigation}{%
\subsection{Site Information
Navigation}\label{site-information-navigation}}

\begin{itemize}
\tightlist
\item
  \href{https://help.nytimes3xbfgragh.onion/hc/en-us/articles/115014792127-Copyright-notice}{©~2020~The
  New York Times Company}
\end{itemize}

\begin{itemize}
\tightlist
\item
  \href{https://www.nytco.com/}{NYTCo}
\item
  \href{https://help.nytimes3xbfgragh.onion/hc/en-us/articles/115015385887-Contact-Us}{Contact
  Us}
\item
  \href{https://www.nytco.com/careers/}{Work with us}
\item
  \href{https://nytmediakit.com/}{Advertise}
\item
  \href{http://www.tbrandstudio.com/}{T Brand Studio}
\item
  \href{https://www.nytimes3xbfgragh.onion/privacy/cookie-policy\#how-do-i-manage-trackers}{Your
  Ad Choices}
\item
  \href{https://www.nytimes3xbfgragh.onion/privacy}{Privacy}
\item
  \href{https://help.nytimes3xbfgragh.onion/hc/en-us/articles/115014893428-Terms-of-service}{Terms
  of Service}
\item
  \href{https://help.nytimes3xbfgragh.onion/hc/en-us/articles/115014893968-Terms-of-sale}{Terms
  of Sale}
\item
  \href{https://spiderbites.nytimes3xbfgragh.onion}{Site Map}
\item
  \href{https://help.nytimes3xbfgragh.onion/hc/en-us}{Help}
\item
  \href{https://www.nytimes3xbfgragh.onion/subscription?campaignId=37WXW}{Subscriptions}
\end{itemize}
