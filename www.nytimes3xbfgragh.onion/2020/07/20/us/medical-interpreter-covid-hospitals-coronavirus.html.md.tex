Sections

SEARCH

\protect\hyperlink{site-content}{Skip to
content}\protect\hyperlink{site-index}{Skip to site index}

\href{https://www.nytimes3xbfgragh.onion/section/us}{U.S.}

\href{https://myaccount.nytimes3xbfgragh.onion/auth/login?response_type=cookie\&client_id=vi}{}

\href{https://www.nytimes3xbfgragh.onion/section/todayspaper}{Today's
Paper}

\href{/section/us}{U.S.}\textbar{}Why a Medical Interpreter Felt
`Disposable' Amid Covid-19

\url{https://nyti.ms/3fMYxca}

\begin{itemize}
\item
\item
\item
\item
\item
\end{itemize}

\hypertarget{the-coronavirus-outbreak}{%
\subsubsection{\texorpdfstring{\href{https://www.nytimes3xbfgragh.onion/news-event/coronavirus?name=styln-coronavirus-national\&region=TOP_BANNER\&block=storyline_menu_recirc\&action=click\&pgtype=Article\&impression_id=6b8ad080-f2c5-11ea-bc20-931b35b979be\&variant=undefined}{The
Coronavirus
Outbreak}}{The Coronavirus Outbreak}}\label{the-coronavirus-outbreak}}

\begin{itemize}
\tightlist
\item
  live\href{https://www.nytimes3xbfgragh.onion/2020/09/09/world/covid-19-coronavirus.html?name=styln-coronavirus-national\&region=TOP_BANNER\&block=storyline_menu_recirc\&action=click\&pgtype=Article\&impression_id=6b8ad081-f2c5-11ea-bc20-931b35b979be\&variant=undefined}{Latest
  Updates}
\item
  \href{https://www.nytimes3xbfgragh.onion/interactive/2020/us/coronavirus-us-cases.html?name=styln-coronavirus-national\&region=TOP_BANNER\&block=storyline_menu_recirc\&action=click\&pgtype=Article\&impression_id=6b8af790-f2c5-11ea-bc20-931b35b979be\&variant=undefined}{Maps
  and Cases}
\item
  \href{https://www.nytimes3xbfgragh.onion/interactive/2020/science/coronavirus-vaccine-tracker.html?name=styln-coronavirus-national\&region=TOP_BANNER\&block=storyline_menu_recirc\&action=click\&pgtype=Article\&impression_id=6b8af791-f2c5-11ea-bc20-931b35b979be\&variant=undefined}{Vaccine
  Tracker}
\item
  \href{https://www.nytimes3xbfgragh.onion/2020/09/02/your-money/eviction-moratorium-covid.html?name=styln-coronavirus-national\&region=TOP_BANNER\&block=storyline_menu_recirc\&action=click\&pgtype=Article\&impression_id=6b8af792-f2c5-11ea-bc20-931b35b979be\&variant=undefined}{Eviction
  Moratorium}
\item
  \href{https://www.nytimes3xbfgragh.onion/2020/09/09/upshot/coronavirus-surprise-test-fees.html?name=styln-coronavirus-national\&region=TOP_BANNER\&block=storyline_menu_recirc\&action=click\&pgtype=Article\&impression_id=6b8af793-f2c5-11ea-bc20-931b35b979be\&variant=undefined}{Surprise
  Test Fees}
\end{itemize}

Advertisement

\protect\hyperlink{after-top}{Continue reading the main story}

Supported by

\protect\hyperlink{after-sponsor}{Continue reading the main story}

In Her Words

\hypertarget{why-a-medical-interpreter-felt-disposable-amid-covid-19}{%
\section{Why a Medical Interpreter Felt `Disposable' Amid
Covid-19}\label{why-a-medical-interpreter-felt-disposable-amid-covid-19}}

For Marta Rodriguez, helping very sick patients understand their
prognosis has become routine --- but it hasn't become any easier.

\includegraphics{https://static01.graylady3jvrrxbe.onion/images/2020/07/17/us/IHW-NEXTSTEPS-MARTATRANSLATOR/IHW-NEXTSTEPS-MARTATRANSLATOR-articleLarge.jpg?quality=75\&auto=webp\&disable=upscale}

By \href{https://www.nytimes3xbfgragh.onion/by/emma-goldberg}{Emma
Goldberg}

\begin{itemize}
\item
  July 20, 2020
\item
  \begin{itemize}
  \item
  \item
  \item
  \item
  \item
  \end{itemize}
\end{itemize}

\begin{center}\rule{0.5\linewidth}{\linethickness}\end{center}

\hypertarget{i-felt-like-we-were-disposable}{%
\subsection{``I felt like we were
disposable.''}\label{i-felt-like-we-were-disposable}}

\emph{--- Marta Rodriguez, a medical interpreter}

\begin{center}\rule{0.5\linewidth}{\linethickness}\end{center}

\emph{{[}In Her Words is available as a newsletter.}
\href{https://www.nytimes3xbfgragh.onion/newsletters/in-her-words}{\emph{Sign
up here to get it delivered to your inbox}}\emph{.{]}}

For Marta Rodriguez, a hospital interpreter, helping very sick patients
understand their prognosis has become routine --- but it hasn't become
any easier. She has a trick when she thinks she is about to cry: ``I'll
dig my nails into my hand,'' she said. ``I'll do something to stop the
tears, because if I fall apart then I won't do a good job.''

Ms. Rodriguez has worked in the role for more than 30 years, caring for
20 to 30 patients each week by interpreting what their doctors and
nurses say. On any given day she might help deliver a diabetes
diagnosis, or respond to a trauma code in the emergency room. On the
toughest days, she calls a patient's non-English speaking relatives on
behalf of the doctor to tell them that their mother, son or husband has
died of Covid-19.

The coronavirus pandemic has shown Ms. Rodriguez how urgently her work
is needed. The death rate for Hispanic people because of the coronavirus
is at least
\href{https://www.brookings.edu/blog/up-front/2020/06/16/race-gaps-in-covid-19-deaths-are-even-bigger-than-they-appear/}{six
times} as high as the death rate among white Americans, for adults ages
45 to 54. And
\href{https://www.jointcommission.org/assets/1/23/Quick_Safety_Issue_13_May_2015_EMBARGOED_5_27_15.pdf}{studies}
have shown that patients with limited English proficiency experience
adverse health outcomes at rates markedly higher than English speakers,
so they rely on interpreters like Ms. Rodriguez to help close that gap.

Ms. Rodriguez's family moved from Costa Rica to the Jamaica Plain
neighborhood in Boston when she was 10. Her father washed dishes at a
restaurant and her mother worked as a nanny. Ms. Rodriguez learned
English the first summer she arrived in America, because a nun at her
Catholic school, Sister Louise, told her to make sure she was fluent by
the start of the school year.

Although her parents came to America seeking economic opportunities for
their children, Ms. Rodriguez believes that if she had stayed in Costa
Rica, she could have become a physician, which was initially her dream
job. In America, a medical education was too expensive; even decades
ago, when she considered it, the
\href{https://www.nytimes3xbfgragh.onion/1981/11/08/magazine/a-nation-of-doctors-in-debt.html}{mean
debt} of medical school graduates was \$18,652, with at least three more
years of training to complete after that.

\hypertarget{latest-updates-the-coronavirus-outbreak}{%
\section{\texorpdfstring{\href{https://www.nytimes3xbfgragh.onion/2020/09/09/world/covid-19-coronavirus.html?action=click\&pgtype=Article\&state=default\&region=MAIN_CONTENT_1\&context=storylines_live_updates}{Latest
Updates: The Coronavirus
Outbreak}}{Latest Updates: The Coronavirus Outbreak}}\label{latest-updates-the-coronavirus-outbreak}}

Updated 2020-09-09T17:47:34.302Z

\begin{itemize}
\tightlist
\item
  \href{https://www.nytimes3xbfgragh.onion/2020/09/09/world/covid-19-coronavirus.html?action=click\&pgtype=Article\&state=default\&region=MAIN_CONTENT_1\&context=storylines_live_updates\#link-279e24e2}{Top
  U.S. health officials update Congress on vaccine development and
  distribution plans.}
\item
  \href{https://www.nytimes3xbfgragh.onion/2020/09/09/world/covid-19-coronavirus.html?action=click\&pgtype=Article\&state=default\&region=MAIN_CONTENT_1\&context=storylines_live_updates\#link-792ae257}{Indoor
  dining in N.Y.C. will return with limits on Sept. 30, Cuomo says.}
\item
  \href{https://www.nytimes3xbfgragh.onion/2020/09/09/world/covid-19-coronavirus.html?action=click\&pgtype=Article\&state=default\&region=MAIN_CONTENT_1\&context=storylines_live_updates\#link-5b0bf0d1}{As
  drugmakers pledge to thoroughly vet vaccines, one company pauses its
  trials for a safety review.}
\end{itemize}

\href{https://www.nytimes3xbfgragh.onion/2020/09/09/world/covid-19-coronavirus.html?action=click\&pgtype=Article\&state=default\&region=MAIN_CONTENT_1\&context=storylines_live_updates}{See
more updates}

More live coverage:
\href{https://www.nytimes3xbfgragh.onion/live/2020/09/09/business/stock-market-today-coronavirus?action=click\&pgtype=Article\&state=default\&region=MAIN_CONTENT_1\&context=storylines_live_updates}{Markets}

When she first started at the hospital, Ms. Rodriguez served as a lab
technician, working with patients in gynecology and obstetrics. Many of
them were navigating the complexities of childbirth on their own and
without much income. These were conditions that Ms. Rodriguez could
relate to; she had her first child at 19, and divorced her husband at 25
because of domestic violence.

``I didn't know anything about the world,'' Ms. Rodriguez said. ``I
wanted my patients to have a better chance than I had.''

She recalled caring for a Spanish-speaking woman who had been raped and
was trying to decide whether to carry her child to term. Ms. Rodriguez
helped the patient to understand her medical options, but took it a step
beyond, drawing on her understanding of the young woman's cultural and
religious values.

``She was struggling with the language barrier, but also the simple fact
that she was brought up thinking abortion was sinful,'' Ms. Rodriguez
said.

Ms. Rodriguez became known by her patients and relatives as ``the birth
control queen'' because she spoke openly about contraception and birth
control, breaking social taboos at a time when abortion was just
becoming legal.

While she spent many years tending to all kinds of diseases and
injuries, working across hospital specialties, her recent months have
focused mainly on Covid-19 care. And of course, the coronavirus outbreak
has made her and her colleagues worry about their own health, as well as
that of their patients. In early March, Ms. Rodriguez learned that there
were not enough masks for all members of the hospital staff; like
\href{https://www.nytimes3xbfgragh.onion/2020/07/08/health/coronavirus-masks-ppe-doc.html}{so
many institutions} across the country, the hospital struggled to secure
adequate levels of personal protective equipment, known as P.P.E.

``I felt like we were disposable,'' Ms. Rodriguez said. Two of her
colleagues got sick with Covid-19. While both survived, she said the
sense of fear was palpable among her co-workers, especially as they
watched members of their own families and communities get sick.

``I'm doing everything I can to not get Covid, but if it's God's will
then it's God's will,'' Ms. Rodriguez said. ``But I don't want to take
it home. I have too many people depending on me.''

The experience of working without P.P.E. has reminded Ms. Rodriguez that
medical practitioners often devalue the work of interpreters. But this
is a reality she has faced many times in the hospital. She teaches a
course for medical interpreters at the
\href{https://www.harvardpilgrim.org/public/our-foundation}{Harvard
Pilgrim Health Care Foundation}, and she remembered once grading the
students' exams at the hospital next to a nurse, who turned to her and
asked, ``You guys go to school for this?''

``I told her we need to learn medical terminology and how the body
works,'' Ms. Rodriguez said. ``If we don't understand it ourselves, how
are we going to break it down for patients?''

Recently, Ms. Rodriguez's hospital purchased iPads, so she has been
using the device to interpret for patients remotely, joining their
doctor appointments by video. The virtual work has brought its own
challenges. It is hard to hear the patients above the noise of the
hospital machinery, and when Covid-19 patients try to raise their
voices, they cough because of the disease's symptoms.

But the pandemic has also strengthened bonds in Ms. Rodriguez's
community of 35 interpreters, as they provide one another with emotional
support. ``There's a liquor store on the corner, and on the worst days
we say, `Let's go to the corner,''' Ms. Rodriguez said. ``We're scared,
not knowing if we are going to get the virus, but we're holding those
feelings in.''

\emph{{[}In Her Words is available as a newsletter.}
\href{https://www.nytimes3xbfgragh.onion/newsletters/in-her-words}{\emph{Sign
up here to get it delivered to your inbox}}\emph{. Write to us at}
\href{mailto:inherwords@NYTimes.com}{\emph{inherwords@NYTimes.com}}\emph{.{]}}

\begin{center}\rule{0.5\linewidth}{\linethickness}\end{center}

\begin{center}\rule{0.5\linewidth}{\linethickness}\end{center}

Advertisement

\protect\hyperlink{after-bottom}{Continue reading the main story}

\hypertarget{site-index}{%
\subsection{Site Index}\label{site-index}}

\hypertarget{site-information-navigation}{%
\subsection{Site Information
Navigation}\label{site-information-navigation}}

\begin{itemize}
\tightlist
\item
  \href{https://help.nytimes3xbfgragh.onion/hc/en-us/articles/115014792127-Copyright-notice}{©~2020~The
  New York Times Company}
\end{itemize}

\begin{itemize}
\tightlist
\item
  \href{https://www.nytco.com/}{NYTCo}
\item
  \href{https://help.nytimes3xbfgragh.onion/hc/en-us/articles/115015385887-Contact-Us}{Contact
  Us}
\item
  \href{https://www.nytco.com/careers/}{Work with us}
\item
  \href{https://nytmediakit.com/}{Advertise}
\item
  \href{http://www.tbrandstudio.com/}{T Brand Studio}
\item
  \href{https://www.nytimes3xbfgragh.onion/privacy/cookie-policy\#how-do-i-manage-trackers}{Your
  Ad Choices}
\item
  \href{https://www.nytimes3xbfgragh.onion/privacy}{Privacy}
\item
  \href{https://help.nytimes3xbfgragh.onion/hc/en-us/articles/115014893428-Terms-of-service}{Terms
  of Service}
\item
  \href{https://help.nytimes3xbfgragh.onion/hc/en-us/articles/115014893968-Terms-of-sale}{Terms
  of Sale}
\item
  \href{https://spiderbites.nytimes3xbfgragh.onion}{Site Map}
\item
  \href{https://help.nytimes3xbfgragh.onion/hc/en-us}{Help}
\item
  \href{https://www.nytimes3xbfgragh.onion/subscription?campaignId=37WXW}{Subscriptions}
\end{itemize}
