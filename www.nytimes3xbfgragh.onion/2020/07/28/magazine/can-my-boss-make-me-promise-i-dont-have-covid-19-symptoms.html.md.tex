Sections

SEARCH

\protect\hyperlink{site-content}{Skip to
content}\protect\hyperlink{site-index}{Skip to site index}

\href{https://myaccount.nytimes3xbfgragh.onion/auth/login?response_type=cookie\&client_id=vi}{}

\href{https://www.nytimes3xbfgragh.onion/section/todayspaper}{Today's
Paper}

Can My Boss Make Me Promise I Don't Have Covid-19 Symptoms?

\begin{itemize}
\item
\item
\item
\item
\item
\item
\end{itemize}

\hypertarget{the-coronavirus-outbreak}{%
\subsubsection{\texorpdfstring{\href{https://www.nytimes3xbfgragh.onion/news-event/coronavirus?name=styln-coronavirus-national\&region=TOP_BANNER\&block=storyline_menu_recirc\&action=click\&pgtype=Article\&impression_id=d14a3f50-f1da-11ea-bc3d-b7a508416139\&variant=undefined}{The
Coronavirus
Outbreak}}{The Coronavirus Outbreak}}\label{the-coronavirus-outbreak}}

\begin{itemize}
\tightlist
\item
  live\href{https://www.nytimes3xbfgragh.onion/2020/09/08/world/covid-19-coronavirus.html?name=styln-coronavirus-national\&region=TOP_BANNER\&block=storyline_menu_recirc\&action=click\&pgtype=Article\&impression_id=d14a6660-f1da-11ea-bc3d-b7a508416139\&variant=undefined}{Latest
  Updates}
\item
  \href{https://www.nytimes3xbfgragh.onion/interactive/2020/us/coronavirus-us-cases.html?name=styln-coronavirus-national\&region=TOP_BANNER\&block=storyline_menu_recirc\&action=click\&pgtype=Article\&impression_id=d14a6661-f1da-11ea-bc3d-b7a508416139\&variant=undefined}{Maps
  and Cases}
\item
  \href{https://www.nytimes3xbfgragh.onion/interactive/2020/science/coronavirus-vaccine-tracker.html?name=styln-coronavirus-national\&region=TOP_BANNER\&block=storyline_menu_recirc\&action=click\&pgtype=Article\&impression_id=d14a6662-f1da-11ea-bc3d-b7a508416139\&variant=undefined}{Vaccine
  Tracker}
\item
  \href{https://www.nytimes3xbfgragh.onion/2020/09/02/your-money/eviction-moratorium-covid.html?name=styln-coronavirus-national\&region=TOP_BANNER\&block=storyline_menu_recirc\&action=click\&pgtype=Article\&impression_id=d14a6663-f1da-11ea-bc3d-b7a508416139\&variant=undefined}{Eviction
  Moratorium}
\item
  \href{https://www.nytimes3xbfgragh.onion/interactive/2020/09/02/magazine/food-insecurity-hunger-us.html?name=styln-coronavirus-national\&region=TOP_BANNER\&block=storyline_menu_recirc\&action=click\&pgtype=Article\&impression_id=d14a6664-f1da-11ea-bc3d-b7a508416139\&variant=undefined}{American
  Hunger}
\end{itemize}

Advertisement

\protect\hyperlink{after-top}{Continue reading the main story}

Supported by

\protect\hyperlink{after-sponsor}{Continue reading the main story}

\href{/column/the-ethicist}{The Ethicist}

\hypertarget{can-my-boss-make-me-promise-i-dont-have-covid-19-symptoms}{%
\section{Can My Boss Make Me Promise I Don't Have Covid-19
Symptoms?}\label{can-my-boss-make-me-promise-i-dont-have-covid-19-symptoms}}

\includegraphics{https://static01.graylady3jvrrxbe.onion/images/2020/08/02/magazine/02Ethicist/02Ethicist-articleLarge.jpg?quality=75\&auto=webp\&disable=upscale}

By Kwame Anthony Appiah

\begin{itemize}
\item
  July 28, 2020
\item
  \begin{itemize}
  \item
  \item
  \item
  \item
  \item
  \item
  \end{itemize}
\end{itemize}

\emph{I relocated to a new state for a job this past December. In March,
my workplace closed because of the coronavirus pandemic, and I've been
working from home since. We're now preparing to go back to work, and my
employers are requiring every person coming to the office to sign a
daily affidavit affirming that they don't have any Covid-19 symptoms.
Anyone refusing to sign will be barred from the office and forced to use
paid time off or potentially lose a day's wages. Requiring me to sign
this statement feels like an invasion of my privacy --- they've never
required anyone to sign such an affidavit during a flu outbreak. Is it
ethical for a business to ask employees to sign something like this? I
should also note that my state hasn't met the benchmarks for reopening
set by the White House. Is it ethical for my employer to require us to
come back to work in the first place?} J. T.

\textbf{In ethics, there} are many different notions of privacy. Two are
relevant here. First, there's informational privacy, which concerns what
others don't have a right to know about. In this respect, your
(consensual) behavior in the bedroom and in the voting booth are
private, though you have a right to reveal them if you choose, at least
to people who are willing to listen. Informational privacy can also
impose restrictions on you: There are things you ought not to reveal,
like personal confidences, professionally privileged communications or
state and business secrets. They are not other people's business, and
putting extraordinary circumstances aside, you don't have the right to
make them other people's business.

Second, there's an arena that's private in the sense that authorities
--- your boss, the government --- have no right to decide them for you.
The issue here isn't what others can expect to know but what they can
legitimately require you to do. We could call this autonomy privacy: It
has to do with what areas of your life should be under your own control.
It's an aspect of liberty.

Does requiring people with the symptoms of a potentially fatal
communicable disease to stay away from work violate informational
privacy? It does not. For one thing, given the policy, your presence in
the office is already an implicit assertion that you are symptom-free.
What's more, the point of the affidavit isn't to gather information at
all; it's to get you to take your undertaking seriously and, perhaps, to
have some legal leverage if you don't.

Nor does the policy violate a proper understanding of autonomy privacy.
You are not entitled to go to work when doing so puts others at
significant risk. Indeed, your bosses have not just the right but the
duty to demand that you don't. And they can do what is reasonably
required to make sure you comply.

\hypertarget{latest-updates-the-coronavirus-outbreak}{%
\section{\texorpdfstring{\href{https://www.nytimes3xbfgragh.onion/2020/09/08/world/covid-19-coronavirus.html?action=click\&pgtype=Article\&state=default\&region=MAIN_CONTENT_1\&context=storylines_live_updates}{Latest
Updates: The Coronavirus
Outbreak}}{Latest Updates: The Coronavirus Outbreak}}\label{latest-updates-the-coronavirus-outbreak}}

Updated 2020-09-08T13:48:00.364Z

\begin{itemize}
\tightlist
\item
  \href{https://www.nytimes3xbfgragh.onion/2020/09/08/world/covid-19-coronavirus.html?action=click\&pgtype=Article\&state=default\&region=MAIN_CONTENT_1\&context=storylines_live_updates\#link-547feae1}{Senate
  Republicans plan to move forward with a scaled-back stimulus package.}
\item
  \href{https://www.nytimes3xbfgragh.onion/2020/09/08/world/covid-19-coronavirus.html?action=click\&pgtype=Article\&state=default\&region=MAIN_CONTENT_1\&context=storylines_live_updates\#link-679303d7}{Nine
  drugmakers pledge to thoroughly vet any coronavirus vaccine.}
\item
  \href{https://www.nytimes3xbfgragh.onion/2020/09/08/world/covid-19-coronavirus.html?action=click\&pgtype=Article\&state=default\&region=MAIN_CONTENT_1\&context=storylines_live_updates\#link-1c973131}{`The
  lockdown killed my father': Farmer suicides add to India's virus
  misery.}
\end{itemize}

\href{https://www.nytimes3xbfgragh.onion/2020/09/08/world/covid-19-coronavirus.html?action=click\&pgtype=Article\&state=default\&region=MAIN_CONTENT_1\&context=storylines_live_updates}{See
more updates}

More live coverage:
\href{https://www.nytimes3xbfgragh.onion/live/2020/09/08/business/stock-market-today-coronavirus?action=click\&pgtype=Article\&state=default\&region=MAIN_CONTENT_1\&context=storylines_live_updates}{Markets}

You contrast what's being asked of you because of the coronavirus with
the lack of similar demands in flu season. There are, of course, many
pertinent differences. We have vaccines for the influenza strains we
face, but we have none for this coronavirus, and many fewer serious
illnesses and deaths occur in a typical flu season. Still, one lesson of
the current pandemic may be that we should consider having clearer
guidelines for employees with flu symptoms and should consider too what
hand-hygiene and mask-wearing policies might continue to make sense
during the regular flu season. Even when Covid-19 is behind us, we may
not want to discard our pandemic practices entirely.

The real trouble with your office's policy is that it appears to be
badly designed to achieve its effect. It can penalize people who
responsibly stay home to protect others. A better policy would allow
such employees extra paid time off in addition to the regular allocation
of vacation days and medical leave. The C.D.C. recommends isolation for
those with symptoms, until they've had no fever for 24 hours, their
symptoms have improved and at least 10 days have elapsed since the
symptoms appeared. Effectively punishing compliance with such
recommendations by withholding pay gives people a reason to disobey the
strictures they're trying to enforce, and so endanger their colleagues.

As for whether you can rightly be asked to go back to the office, much
depends on the details. If your state or municipality hasn't yet met the
benchmarks for reopening, then your employer shouldn't require you to
change course. Even when those benchmarks are met, it's important that
your office requires masks and practices social distancing, encourages
regular hand washing, pays to have spaces appropriately cleaned and has
thought about making sure that the airflow through the space is managed
in a way that minimizes your exposure to droplets and aerosols.

Bear in mind that people infected with the coronavirus appear to be most
contagious shortly before the onset of symptoms. (Those who never become
symptomatic may also be able to spread infection.) Your company should,
accordingly, remind employees that they ought to follow the recommended
practices all the time, not just at work. No privacy concerns are raised
by an affidavit that requires them to attest to this too.

\href{https://www.nytimes3xbfgragh.onion/news-event/coronavirus?action=click\&pgtype=Article\&state=default\&region=MAIN_CONTENT_3\&context=storylines_faq}{}

\hypertarget{the-coronavirus-outbreak-}{%
\subsubsection{The Coronavirus Outbreak
›}\label{the-coronavirus-outbreak-}}

\hypertarget{frequently-asked-questions}{%
\paragraph{Frequently Asked
Questions}\label{frequently-asked-questions}}

Updated September 4, 2020

\begin{itemize}
\item ~
  \hypertarget{what-are-the-symptoms-of-coronavirus}{%
  \paragraph{What are the symptoms of
  coronavirus?}\label{what-are-the-symptoms-of-coronavirus}}

  \begin{itemize}
  \tightlist
  \item
    In the beginning, the coronavirus
    \href{https://www.nytimes3xbfgragh.onion/article/coronavirus-facts-history.html?action=click\&pgtype=Article\&state=default\&region=MAIN_CONTENT_3\&context=storylines_faq\#link-6817bab5}{seemed
    like it was primarily a respiratory illness}~--- many patients had
    fever and chills, were weak and tired, and coughed a lot, though
    some people don't show many symptoms at all. Those who seemed
    sickest had pneumonia or acute respiratory distress syndrome and
    received supplemental oxygen. By now, doctors have identified many
    more symptoms and syndromes. In April,
    \href{https://www.nytimes3xbfgragh.onion/2020/04/27/health/coronavirus-symptoms-cdc.html?action=click\&pgtype=Article\&state=default\&region=MAIN_CONTENT_3\&context=storylines_faq}{the
    C.D.C. added to the list of early signs}~sore throat, fever, chills
    and muscle aches. Gastrointestinal upset, such as diarrhea and
    nausea, has also been observed. Another telltale sign of infection
    may be a sudden, profound diminution of one's
    \href{https://www.nytimes3xbfgragh.onion/2020/03/22/health/coronavirus-symptoms-smell-taste.html?action=click\&pgtype=Article\&state=default\&region=MAIN_CONTENT_3\&context=storylines_faq}{sense
    of smell and taste.}~Teenagers and young adults in some cases have
    developed painful red and purple lesions on their fingers and toes
    --- nicknamed ``Covid toe'' --- but few other serious symptoms.
  \end{itemize}
\item ~
  \hypertarget{why-is-it-safer-to-spend-time-together-outside}{%
  \paragraph{Why is it safer to spend time together
  outside?}\label{why-is-it-safer-to-spend-time-together-outside}}

  \begin{itemize}
  \tightlist
  \item
    \href{https://www.nytimes3xbfgragh.onion/2020/05/15/us/coronavirus-what-to-do-outside.html?action=click\&pgtype=Article\&state=default\&region=MAIN_CONTENT_3\&context=storylines_faq}{Outdoor
    gatherings}~lower risk because wind disperses viral droplets, and
    sunlight can kill some of the virus. Open spaces prevent the virus
    from building up in concentrated amounts and being inhaled, which
    can happen when infected people exhale in a confined space for long
    stretches of time, said Dr. Julian W. Tang, a virologist at the
    University of Leicester.
  \end{itemize}
\item ~
  \hypertarget{why-does-standing-six-feet-away-from-others-help}{%
  \paragraph{Why does standing six feet away from others
  help?}\label{why-does-standing-six-feet-away-from-others-help}}

  \begin{itemize}
  \tightlist
  \item
    The coronavirus spreads primarily through droplets from your mouth
    and nose, especially when you cough or sneeze. The C.D.C., one of
    the organizations using that measure,
    \href{https://www.nytimes3xbfgragh.onion/2020/04/14/health/coronavirus-six-feet.html?action=click\&pgtype=Article\&state=default\&region=MAIN_CONTENT_3\&context=storylines_faq}{bases
    its recommendation of six feet}~on the idea that most large droplets
    that people expel when they cough or sneeze will fall to the ground
    within six feet. But six feet has never been a magic number that
    guarantees complete protection. Sneezes, for instance, can launch
    droplets a lot farther than six feet,
    \href{https://jamanetwork.com/journals/jama/fullarticle/2763852}{according
    to a recent study}. It's a rule of thumb: You should be safest
    standing six feet apart outside, especially when it's windy. But
    keep a mask on at all times, even when you think you're far enough
    apart.
  \end{itemize}
\item ~
  \hypertarget{i-have-antibodies-am-i-now-immune}{%
  \paragraph{I have antibodies. Am I now
  immune?}\label{i-have-antibodies-am-i-now-immune}}

  \begin{itemize}
  \tightlist
  \item
    As of right
    now,\href{https://www.nytimes3xbfgragh.onion/2020/07/22/health/covid-antibodies-herd-immunity.html?action=click\&pgtype=Article\&state=default\&region=MAIN_CONTENT_3\&context=storylines_faq}{~that
    seems likely, for at least several months.}~There have been
    frightening accounts of people suffering what seems to be a second
    bout of Covid-19. But experts say these patients may have a
    drawn-out course of infection, with the virus taking a slow toll
    weeks to months after initial exposure.~People infected with the
    coronavirus typically
    \href{https://www.nature.com/articles/s41586-020-2456-9}{produce}~immune
    molecules called antibodies, which are
    \href{https://www.nytimes3xbfgragh.onion/2020/05/07/health/coronavirus-antibody-prevalence.html?action=click\&pgtype=Article\&state=default\&region=MAIN_CONTENT_3\&context=storylines_faq}{protective
    proteins made in response to an
    infection}\href{https://www.nytimes3xbfgragh.onion/2020/05/07/health/coronavirus-antibody-prevalence.html?action=click\&pgtype=Article\&state=default\&region=MAIN_CONTENT_3\&context=storylines_faq}{.
    These antibodies may}~last in the body
    \href{https://www.nature.com/articles/s41591-020-0965-6}{only two to
    three months}, which may seem worrisome, but that's~perfectly normal
    after an acute infection subsides, said Dr. Michael Mina, an
    immunologist at Harvard University. It may be possible to get the
    coronavirus again, but it's highly unlikely that it would be
    possible in a short window of time from initial infection or make
    people sicker the second time.
  \end{itemize}
\item ~
  \hypertarget{what-are-my-rights-if-i-am-worried-about-going-back-to-work}{%
  \paragraph{What are my rights if I am worried about going back to
  work?}\label{what-are-my-rights-if-i-am-worried-about-going-back-to-work}}

  \begin{itemize}
  \tightlist
  \item
    Employers have to provide
    \href{https://www.osha.gov/SLTC/covid-19/standards.html}{a safe
    workplace}~with policies that protect everyone equally.
    \href{https://www.nytimes3xbfgragh.onion/article/coronavirus-money-unemployment.html?action=click\&pgtype=Article\&state=default\&region=MAIN_CONTENT_3\&context=storylines_faq}{And
    if one of your co-workers tests positive for the coronavirus, the
    C.D.C.}~has said that
    \href{https://www.cdc.gov/coronavirus/2019-ncov/community/guidance-business-response.html}{employers
    should tell their employees}~-\/- without giving you the sick
    employee's name -\/- that they may have been exposed to the virus.
  \end{itemize}
\end{itemize}

\emph{My friend works at a government agency. A portion of the staff
there is required to go out into the field to do their work, which puts
them at risk of coronavirus exposure. My friend's job at the agency can
be performed entirely remotely. As the agency plans for reopening,
senior management is contemplating a requirement that all employees work
at least one day in the office (there will be an exception for those who
must care for dependents at home). This requirement, the managers said,
would promote equity among agency workers and across departments. If
front-line staff are risking exposure to the coronavirus in the field,
they say, non-front-line staff should share a similar burden. Is this
requirement ethical? Does it matter if the policy is applied to all
staff, or only to managers and other senior employees? Are there other
ways to address equity issues related to different levels of coronavirus
risk among workers at the agency?} Name Withheld

\textbf{The rationale you've} described is profoundly misguided. It's
necessary for the work of the armed forces that certain troops sometimes
be put in harm's way. What would we think of a policy that required all
service members to play a version of Russian roulette during wartime, in
order to maintain risk equity? Historically speaking, there have been
times, in battle, when commanders exposed themselves to risk in order to
rally the troops. Doing so had a rational purpose. But exposing every
service member to mortal risk out of a desire for fairness would rightly
be regarded as bizarre.

Those who must face health risks should, of course, be protected to the
greatest extent possible, and they should be honored for their work.
There is also a case, in a society like ours, for recognizing their
sacrifice with extra pay. But increasing the overall disease burden of
the staff honors no one.

\emph{I live with my daughter and her two teenage sons. I am very close
to the 15-year-old, who confides in me even more than his mother does.
The other day he told me that he is given rides to various places by a
16-year-old friend who does not have his parents' permission to have any
friends in his car. He told me this in confidence. I responded that I
would not tell his mother but that I felt uncomfortable with that
information. If I told her, I can't imagine he would ever trust me
again, but if I don't tell her, am I being irresponsible? If the
unimaginable happened and a car accident ensued, how could I live with
myself?} Nancy K., Irvine, Calif.

\textbf{Putting Covid-19 issues} aside, your grandson is in danger not
because his friend doesn't have his parents' permission but because the
driver is 16. But your grandson is also abetting his friend in deceiving
and disobeying his parents about something serious. I'd tell him that he
shouldn't be accepting these rides and that you're keeping his
confidence on the assumption that he'll stop.

Advertisement

\protect\hyperlink{after-bottom}{Continue reading the main story}

\hypertarget{site-index}{%
\subsection{Site Index}\label{site-index}}

\hypertarget{site-information-navigation}{%
\subsection{Site Information
Navigation}\label{site-information-navigation}}

\begin{itemize}
\tightlist
\item
  \href{https://help.nytimes3xbfgragh.onion/hc/en-us/articles/115014792127-Copyright-notice}{©~2020~The
  New York Times Company}
\end{itemize}

\begin{itemize}
\tightlist
\item
  \href{https://www.nytco.com/}{NYTCo}
\item
  \href{https://help.nytimes3xbfgragh.onion/hc/en-us/articles/115015385887-Contact-Us}{Contact
  Us}
\item
  \href{https://www.nytco.com/careers/}{Work with us}
\item
  \href{https://nytmediakit.com/}{Advertise}
\item
  \href{http://www.tbrandstudio.com/}{T Brand Studio}
\item
  \href{https://www.nytimes3xbfgragh.onion/privacy/cookie-policy\#how-do-i-manage-trackers}{Your
  Ad Choices}
\item
  \href{https://www.nytimes3xbfgragh.onion/privacy}{Privacy}
\item
  \href{https://help.nytimes3xbfgragh.onion/hc/en-us/articles/115014893428-Terms-of-service}{Terms
  of Service}
\item
  \href{https://help.nytimes3xbfgragh.onion/hc/en-us/articles/115014893968-Terms-of-sale}{Terms
  of Sale}
\item
  \href{https://spiderbites.nytimes3xbfgragh.onion}{Site Map}
\item
  \href{https://help.nytimes3xbfgragh.onion/hc/en-us}{Help}
\item
  \href{https://www.nytimes3xbfgragh.onion/subscription?campaignId=37WXW}{Subscriptions}
\end{itemize}
