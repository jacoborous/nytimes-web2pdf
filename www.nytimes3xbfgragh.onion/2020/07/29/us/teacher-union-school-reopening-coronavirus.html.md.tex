Sections

SEARCH

\protect\hyperlink{site-content}{Skip to
content}\protect\hyperlink{site-index}{Skip to site index}

\href{https://www.nytimes3xbfgragh.onion/section/us}{U.S.}

\href{https://myaccount.nytimes3xbfgragh.onion/auth/login?response_type=cookie\&client_id=vi}{}

\href{https://www.nytimes3xbfgragh.onion/section/todayspaper}{Today's
Paper}

\href{/section/us}{U.S.}\textbar{}Teachers Are Wary of Returning to
Class, and Online Instruction Too

\url{https://nyti.ms/2CZkGpr}

\begin{itemize}
\item
\item
\item
\item
\item
\end{itemize}

\hypertarget{school-reopenings}{%
\subsubsection{\texorpdfstring{\href{https://www.nytimes3xbfgragh.onion/spotlight/schools-reopening?name=styln-coronavirus-schools-reopening\&region=TOP_BANNER\&block=storyline_menu_recirc\&action=click\&pgtype=Article\&impression_id=38515390-f4c5-11ea-96c6-737460281ea1\&variant=undefined}{School
Reopenings}}{School Reopenings}}\label{school-reopenings}}

\begin{itemize}
\tightlist
\item
  \href{https://www.nytimes3xbfgragh.onion/2020/09/08/us/school-districts-cyberattacks-glitches.html?name=styln-coronavirus-schools-reopening\&region=TOP_BANNER\&block=storyline_menu_recirc\&action=click\&pgtype=Article\&impression_id=38515391-f4c5-11ea-96c6-737460281ea1\&variant=undefined}{Remote
  Learning Glitches}
\item
  \href{https://www.nytimes3xbfgragh.onion/2020/09/08/upshot/children-testing-shortfalls-virus.html?name=styln-coronavirus-schools-reopening\&region=TOP_BANNER\&block=storyline_menu_recirc\&action=click\&pgtype=Article\&impression_id=38517aa0-f4c5-11ea-96c6-737460281ea1\&variant=undefined}{Limited
  Testing for Children}
\item
  \href{https://www.nytimes3xbfgragh.onion/2020/09/10/us/des-moines-school-opening-coronavirus.html?name=styln-coronavirus-schools-reopening\&region=TOP_BANNER\&block=storyline_menu_recirc\&action=click\&pgtype=Article\&impression_id=38517aa1-f4c5-11ea-96c6-737460281ea1\&variant=undefined}{District
  Defies Reopening Order}
\item
  \href{https://www.nytimes3xbfgragh.onion/interactive/2020/us/covid-college-cases-tracker.html?name=styln-coronavirus-schools-reopening\&region=TOP_BANNER\&block=storyline_menu_recirc\&action=click\&pgtype=Article\&impression_id=38517aa2-f4c5-11ea-96c6-737460281ea1\&variant=undefined}{Tracking
  College Cases}
\end{itemize}

Advertisement

\protect\hyperlink{after-top}{Continue reading the main story}

Supported by

\protect\hyperlink{after-sponsor}{Continue reading the main story}

\hypertarget{teachers-are-wary-of-returning-to-class-and-online-instruction-too}{%
\section{Teachers Are Wary of Returning to Class, and Online Instruction
Too}\label{teachers-are-wary-of-returning-to-class-and-online-instruction-too}}

Unions are threatening to strike if classrooms reopen, but are also
pushing to limit live remote teaching. Their demands will shape pandemic
education.

\includegraphics{https://static01.graylady3jvrrxbe.onion/images/2020/07/29/us/29virus-teacherunions01/29virus-teacherunions01-articleLarge.jpg?quality=75\&auto=webp\&disable=upscale}

\href{https://www.nytimes3xbfgragh.onion/by/dana-goldstein}{\includegraphics{https://static01.graylady3jvrrxbe.onion/images/2018/06/12/multimedia/author-dana-goldstein/author-dana-goldstein-thumbLarge.png}}\href{https://www.nytimes3xbfgragh.onion/by/eliza-shapiro}{\includegraphics{https://static01.graylady3jvrrxbe.onion/images/2018/12/28/multimedia/author-eliza-shapiro/author-eliza-shapiro-thumbLarge.png}}

By \href{https://www.nytimes3xbfgragh.onion/by/dana-goldstein}{Dana
Goldstein} and
\href{https://www.nytimes3xbfgragh.onion/by/eliza-shapiro}{Eliza
Shapiro}

\begin{itemize}
\item
  Published July 29, 2020Updated Aug. 13, 2020
\item
  \begin{itemize}
  \item
  \item
  \item
  \item
  \item
  \end{itemize}
\end{itemize}

As the nation heads toward a chaotic back-to-school season, with
officials struggling over when to reopen classrooms and how to engage
children online, teachers' unions are playing a powerful role in
determining the shape of public education as the coronavirus pandemic
continues to rage.

Teachers in many districts are fighting for longer school closures,
stronger safety requirements and limits on what they are required to do
in virtual classrooms, while flooding social media and state capitols
with their concerns and threatening to walk off their jobs if key
demands are not met.

On Tuesday, the nation's second-largest teachers' union raised the
stakes dramatically by
\href{https://www.nytimes3xbfgragh.onion/2020/07/28/world/coronavirus-covid-19.html\#link-541bdc40}{authorizing
its local and state chapters} to strike if their districts do not take
sufficient precautions --- such as requiring masks and updating
ventilation systems --- before reopening classrooms. Already, teachers'
unions have
\href{https://www.nytimes3xbfgragh.onion/2020/07/20/world/coronavirus-cases.html\#link-2e69dc1c}{sued
Florida's governor} over that state's efforts to require schools to
offer in-person instruction.

But even as unions exert their influence, they face enormous public and
political pressure because of widespread acknowledgment that getting
parents back to work requires functioning school systems, and that
remote learning
\href{https://www.nytimes3xbfgragh.onion/2020/06/05/us/coronavirus-education-lost-learning.html}{failed
many children} this spring, deepening
\href{https://www.nytimes3xbfgragh.onion/2020/06/05/us/coronavirus-education-lost-learning.html}{achievement
gaps} by race and income.

With the academic year set to begin next month in much of the country,
parents are desperate for teachers to provide more interactive,
face-to-face instruction this fall, both online and, where safe, in
person. But many unions, while concerned about the safety of classrooms,
are also fighting to limit the amount of time that teachers are required
to be on video over the course of a day.

The unions are ``really on the backs of their heels on this,'' said
Robin Lake, director of the Center on Reinventing Public Education, a
research and advocacy group that sometimes takes positions contrary to
unions. She is concerned that the urgent needs of children who have not
physically attended school for many months are getting lost. ``I feel
like we are treating kids as pawns in this game.''

Pressure from President Trump and Education Secretary Betsy DeVos, who
are distrusted by many educators, has hardened the opposition of many
teachers to returning to classrooms, even in places where the virus is
under control. They contend that political leaders are putting the needs
of the economy above their safety and pushing schools to reopen without
adequate guidance or financial support.

``It's been a terrible disservice to parents, to kids, to educators, who
basically are left holding the bag and trying to figure this out,'' said
Randi Weingarten, president of the American Federation of Teachers,
which voted to support its members who choose to strike while stressing
that such actions should be a ``last resort.''

\href{https://nces.ed.gov/surveys/ntps/tables/Table_TeachersUnion.asp}{About
70 percent} of American teachers were union members in 2016. Educators
have enjoyed significant parental support in recent years during
\href{https://www.nytimes3xbfgragh.onion/2019/01/14/us/lausd-teachers-strike.html}{a
series of walkouts}, including in Republican-led states, in favor of
higher wages and more school funding.

But now, with the economy sputtering and many parents struggling to
balance work and child care while overseeing remote learning, teachers
who resist demands to appear over video or to work in classrooms where
it is considered safe risk fraying those
\href{https://www.nytimes3xbfgragh.onion/2020/07/11/us/virus-teachers-classrooms.html}{hard-won
bonds}.

Some critics see teachers' unions as trying to have it both ways:
reluctant to return to classrooms, but also resistant in some districts
to providing a full day of remote school via tools like live video ---
the kind of interactive, online instruction that many parents say their
children need after watching them flounder in the spring.

Union leaders point out that many teachers went above and beyond the
work hours laid out in emergency labor agreements that were quickly
pulled together after schools closed in March. Their members provided
technical support to families and answered emails and text messages from
students and parents late into the night, leaders say.

Now, those representatives must balance the concerns of an often-feisty
membership against the urgent needs of vulnerable children and the
often-competing demands of local and federal officials. Complicating
matters, parents disagree sharply on what they want from schools during
the pandemic.

\href{https://www.kff.org/coronavirus-covid-19/report/kff-health-tracking-poll-july-2020/}{A
July poll} found that 60 percent of parents supported delaying school
reopenings until the virus is under control. Polls show that Black and
Latino families, who have suffered disproportionately from the pandemic,
\href{https://www.chalkbeat.org/2020/7/14/21324873/school-closure-reopening-parents-surveys}{have
expressed more concern about returning to school than white parents
have}, but are also more worried about the academic and social impacts
of online learning.

In New York City, where the coronavirus caseload is now relatively low,
a June parents' survey found that most respondents
\href{https://www.nytimes3xbfgragh.onion/2020/07/06/nyregion/nyc-school-reopening-plan.html}{were
at least somewhat willing} to send their children back to physical
classrooms, despite teachers' fears.

\includegraphics{https://static01.graylady3jvrrxbe.onion/images/2020/07/29/us/29virus-teacherunions03/merlin_171583887_7fbcf538-c28c-423c-a58e-50f82f5a783a-articleLarge.jpg?quality=75\&auto=webp\&disable=upscale}

Dionn Hurley, who lives in the South Bronx, said her 18-year-old son,
who has autism, ``regressed by a year in a month'' after schools
shuttered. ``Our kids need in-person learning,'' she said.

She and her husband are both essential workers who have been commuting
across the city to their jobs since the pandemic began. She contends
teachers should do the same, with adequate safety precautions.

``We all know there's a pandemic. It's affecting everyone,'' Ms. Hurley
said. ``You can't just keep saying you're scared. We're all scared.''

Union representatives said they were aware of those sentiments, and the
very real needs behind them. That makes the job of negotiating for their
members especially difficult.

``I would not say that being a teachers' union leader is a job most
people would want to have at this moment,'' said Michael Mulgrew,
president of the United Federation of Teachers in New York City, the
largest local teachers' union in the country.

Pressure from California's politically powerful teachers' unions helped
push Gov. Gavin Newsom to
\href{https://www.nytimes3xbfgragh.onion/2020/07/17/us/california-schools-reopening-newsom.html}{announce
guidelines this month} that will require many of the state's districts,
covering more than 80 percent of its population, to start school
remotely, opening classrooms only once new infections and
hospitalizations decline sufficiently in a region.

Los Angeles, the nation's second-largest school district, had
\href{https://www.nytimes3xbfgragh.onion/2020/07/13/us/lausd-san-diego-school-reopening.html}{already
made the decision} to start the year online because of soaring
infections. Now the union and administrators are engaged in long
negotiating sessions via Zoom, with one of the stickiest points of
contention being how many hours per day teachers should be required to
teach live via video.

Cecily Myart-Cruz, president of the United Teachers Los Angeles union,
said she understood the benefits --- she watched her own son engage with
teachers online during the spring shutdown --- but she argued that a
full school day over video would not be feasible for either students or
teachers (although
\href{https://www.nytimes3xbfgragh.onion/2020/05/09/us/coronavirus-public-private-school.html}{some
private schools have embraced it}).

``You're not going to see people engaged,'' she said. ``Kids will turn
off to that.''

The union's priorities, Ms. Myart-Cruz said, include ensuring that
remote mental health counseling is available to students, and that
teachers are reimbursed for work-from-home expenses such as upgrading
their internet connections.

In the Sacramento City Unified School District, a history of mistrust
between the union and administration has led to a series of repeated
breakdowns in talks during the pandemic.

The district will open in a remote-only mode on Sept. 3, and has
proposed that lessons delivered live over video or audio should be
recorded for families to access at times that are convenient for them.
But the union has objected, arguing that recording lessons could be a
violation of privacy for educators, students and families because their
likenesses could be posted and viewed without their explicit permission.

In the spring, the union argued in favor of providing more paper
materials to students, making the case that it was unfair to lean into
high-tech learning when some students lacked laptops and internet
access.

\href{https://www.nytimes3xbfgragh.onion/spotlight/schools-reopening?action=click\&pgtype=Article\&state=default\&region=MAIN_CONTENT_3\&context=storylines_keepup}{}

\hypertarget{school-reopenings-}{%
\subsubsection{School Reopenings ›}\label{school-reopenings-}}

\hypertarget{back-to-school}{%
\paragraph{Back to School}\label{back-to-school}}

Updated Sept. 11, 2020

The latest on how schools are reopening amid the pandemic.

\begin{itemize}
\item
  \begin{itemize}
  \tightlist
  \item
    School officials in Des Moines are refusing to hold in-person
    classes,
    \href{https://www.nytimes3xbfgragh.onion/2020/09/10/us/des-moines-school-opening-coronavirus.html?action=click\&pgtype=Article\&state=default\&region=MAIN_CONTENT_3\&context=storylines_keepup}{despite
    an order from Iowa's governor and a judge's ruling}, risking school
    funding and their jobs because they think it's unsafe.
  \item
    The University of Illinois at Urbana-Champaign had one of the most
    comprehensive plans by a major college to keep the virus under
    control. But it
    \href{https://www.nytimes3xbfgragh.onion/2020/09/10/health/university-illinois-covid.html?action=click\&pgtype=Article\&state=default\&region=MAIN_CONTENT_3\&context=storylines_keepup}{failed
    to account for students partying}.
  \item
    College students are
    \href{https://www.nytimes3xbfgragh.onion/2020/09/10/technology/coronavirus-quarantines-college.html?action=click\&pgtype=Article\&state=default\&region=MAIN_CONTENT_3\&context=storylines_keepup}{using
    apps to shame their schools}~into better coronavirus plans.
  \item
    For some families, the pandemic
    \href{https://www.nytimes3xbfgragh.onion/2020/09/10/parenting/family-second-language-coronavirus.html?action=click\&pgtype=Article\&state=default\&region=MAIN_CONTENT_3\&context=storylines_keepup}{has
    meant a return to their native languages}.
  \end{itemize}
\end{itemize}

Across the country, it is likely that most students will experience a
mix of online and in-person education this academic year, sometimes
during the same week. That means teachers will need to do two very
different jobs: teach in classrooms and online.

Districts without collective bargaining, like Marietta, Ga., have more
flexibility over assigning teachers' roles, and plan to staff their
remote learning programs with educators who have demonstrated skill in
engaging students online.

But unions elsewhere, including in Miami-Dade County, the nation's
fourth-largest district, are resisting that model, saying teachers with
their own health concerns should be the first to get the opportunity to
work online from home.

On Wednesday, the district announced that it would delay the start of
the academic year by a week, to Aug. 31, and that schools would open
online. It hopes to begin bringing students back to classrooms by early
October.

Image

Story Collins, 9, and her mother, Heather Correia, at a protest in
Jacksonville, Fla., earlier this month.~ Teachers in the Duval County
Public Schools say it's not safe for them to return to classrooms next
month as planned.Credit...Bob Self/The Florida Times-Union, via
Associated Press

This spring, when classrooms closed because of the coronavirus, an
emergency agreement between the district and union required Miami
teachers to interact with their students for a minimum of three hours
per day, which could include making phone calls or responding to emails.

Parents have since made it clear that was not enough, according to
Alberto Carvalho, the Miami superintendent. ``One of the biggest
concerns was how much of a difficult time they had in terms of time
management with their children,'' he said, adding that the district
expects teachers to provide something closer to a regular school day
this fall, with live instruction over video.

The local union president, Karla Hernandez-Mats, said her members were
willing to follow a more traditional schedule, but many teachers have
expressed anxiety about how they and their homes would look on camera
during live teaching.

``If a teacher does not feel comfortable, and the teacher is not secure
in the modality, they are not going to flourish and give the best of
themselves,'' she said.

In Orange County, another large Florida district that includes Orlando,
a major concern for the union is that teachers working in schools might
be expected to simultaneously broadcast their lessons live to students
at home and respond to children both in-person and virtually.

``You can't keep track of people remotely and in front of you at the
same time,'' said Wendy L. Doromal, president of the Orange County
Classroom Teachers Association.

New York City, the nation's largest district, is one of the few big
systems in the country
\href{https://www.nytimes3xbfgragh.onion/2020/07/08/nyregion/nyc-schools-reopening-plan.html}{planning
to reopen schools even part time this fall}. Mr. Mulgrew, the local
teachers' union leader, helped officials settle on an approach that
would allow children to report to classrooms one to three times per
week.

But in the weeks since the plan was unveiled, Mr. Trump's push to reopen
classrooms has magnified growing alarm among educators about returning
to work, and some have
\href{https://twitter.com/Liat_RO/status/1281288365472649216}{threatened
to stage a sick-out}.

In a town hall with members last week, Mr. Mulgrew threw the plan he
helped create into disarray, telling teachers he does not currently
believe it is safe for schools to reopen physically in September, absent
a major funding influx to pay for more nurses and upgraded air
filtration systems.

``I am preparing to do whatever we need to do if we think the schools
are not safe and the city disagrees with us,'' Mr. Mulgrew said on the
call.

City officials said they were caught off guard when Mr. Mulgrew backed
away from reopening, in part because the city had already agreed to a
number of safety measures, including requiring masks and social
distancing in the classroom, and to allow teachers over 65 and those
with pre-existing conditions to work remotely.

Some New York City teachers are encouraging their colleagues to apply
for medical exemptions that would allow them to teach at home, even if
they are not eligible, and asking parents not to send their children
back.

Still, educators hardly present a monolithic view.

``I'm a public servant, and I'm ready to serve wherever I'm needed,''
said Carlotta Pope, a high school English teacher in Brooklyn. Ms. Pope
said she had some lingering questions about safety but was hopeful they
would be resolved before September.

``I'm excited to go back, if that's what's decided,'' she said. ``I miss
my students.''

Advertisement

\protect\hyperlink{after-bottom}{Continue reading the main story}

\hypertarget{site-index}{%
\subsection{Site Index}\label{site-index}}

\hypertarget{site-information-navigation}{%
\subsection{Site Information
Navigation}\label{site-information-navigation}}

\begin{itemize}
\tightlist
\item
  \href{https://help.nytimes3xbfgragh.onion/hc/en-us/articles/115014792127-Copyright-notice}{©~2020~The
  New York Times Company}
\end{itemize}

\begin{itemize}
\tightlist
\item
  \href{https://www.nytco.com/}{NYTCo}
\item
  \href{https://help.nytimes3xbfgragh.onion/hc/en-us/articles/115015385887-Contact-Us}{Contact
  Us}
\item
  \href{https://www.nytco.com/careers/}{Work with us}
\item
  \href{https://nytmediakit.com/}{Advertise}
\item
  \href{http://www.tbrandstudio.com/}{T Brand Studio}
\item
  \href{https://www.nytimes3xbfgragh.onion/privacy/cookie-policy\#how-do-i-manage-trackers}{Your
  Ad Choices}
\item
  \href{https://www.nytimes3xbfgragh.onion/privacy}{Privacy}
\item
  \href{https://help.nytimes3xbfgragh.onion/hc/en-us/articles/115014893428-Terms-of-service}{Terms
  of Service}
\item
  \href{https://help.nytimes3xbfgragh.onion/hc/en-us/articles/115014893968-Terms-of-sale}{Terms
  of Sale}
\item
  \href{https://spiderbites.nytimes3xbfgragh.onion}{Site Map}
\item
  \href{https://help.nytimes3xbfgragh.onion/hc/en-us}{Help}
\item
  \href{https://www.nytimes3xbfgragh.onion/subscription?campaignId=37WXW}{Subscriptions}
\end{itemize}
