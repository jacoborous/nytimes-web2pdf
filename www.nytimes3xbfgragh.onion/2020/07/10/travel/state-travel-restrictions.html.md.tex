Sections

SEARCH

\protect\hyperlink{site-content}{Skip to
content}\protect\hyperlink{site-index}{Skip to site index}

\href{https://www.nytimes3xbfgragh.onion/section/travel}{Travel}

\href{https://myaccount.nytimes3xbfgragh.onion/auth/login?response_type=cookie\&client_id=vi}{}

\href{https://www.nytimes3xbfgragh.onion/section/todayspaper}{Today's
Paper}

\href{/section/travel}{Travel}\textbar{}Thinking of Traveling in the
U.S.? These States Have Travel Restrictions.

\url{https://nyti.ms/2W8L3Q5}

\begin{itemize}
\item
\item
\item
\item
\item
\end{itemize}

\hypertarget{the-coronavirus-outbreak}{%
\subsubsection{\texorpdfstring{\href{https://www.nytimes3xbfgragh.onion/news-event/coronavirus?name=styln-coronavirus-national\&region=TOP_BANNER\&block=storyline_menu_recirc\&action=click\&pgtype=Article\&impression_id=43485f20-f1c1-11ea-b492-3b10163695e2\&variant=undefined}{The
Coronavirus
Outbreak}}{The Coronavirus Outbreak}}\label{the-coronavirus-outbreak}}

\begin{itemize}
\tightlist
\item
  live\href{https://www.nytimes3xbfgragh.onion/2020/09/08/world/covid-19-coronavirus.html?name=styln-coronavirus-national\&region=TOP_BANNER\&block=storyline_menu_recirc\&action=click\&pgtype=Article\&impression_id=43485f21-f1c1-11ea-b492-3b10163695e2\&variant=undefined}{Latest
  Updates}
\item
  \href{https://www.nytimes3xbfgragh.onion/interactive/2020/us/coronavirus-us-cases.html?name=styln-coronavirus-national\&region=TOP_BANNER\&block=storyline_menu_recirc\&action=click\&pgtype=Article\&impression_id=43492270-f1c1-11ea-b492-3b10163695e2\&variant=undefined}{Maps
  and Cases}
\item
  \href{https://www.nytimes3xbfgragh.onion/interactive/2020/science/coronavirus-vaccine-tracker.html?name=styln-coronavirus-national\&region=TOP_BANNER\&block=storyline_menu_recirc\&action=click\&pgtype=Article\&impression_id=43492271-f1c1-11ea-b492-3b10163695e2\&variant=undefined}{Vaccine
  Tracker}
\item
  \href{https://www.nytimes3xbfgragh.onion/2020/09/02/your-money/eviction-moratorium-covid.html?name=styln-coronavirus-national\&region=TOP_BANNER\&block=storyline_menu_recirc\&action=click\&pgtype=Article\&impression_id=43492272-f1c1-11ea-b492-3b10163695e2\&variant=undefined}{Eviction
  Moratorium}
\item
  \href{https://www.nytimes3xbfgragh.onion/interactive/2020/09/02/magazine/food-insecurity-hunger-us.html?name=styln-coronavirus-national\&region=TOP_BANNER\&block=storyline_menu_recirc\&action=click\&pgtype=Article\&impression_id=43492273-f1c1-11ea-b492-3b10163695e2\&variant=undefined}{American
  Hunger}
\end{itemize}

Advertisement

\protect\hyperlink{after-top}{Continue reading the main story}

Supported by

\protect\hyperlink{after-sponsor}{Continue reading the main story}

\hypertarget{thinking-of-traveling-in-the-us-these-states-have-travel-restrictions}{%
\section{Thinking of Traveling in the U.S.? These States Have Travel
Restrictions.}\label{thinking-of-traveling-in-the-us-these-states-have-travel-restrictions}}

Nearly half of the states have measures in place for visitors, from
mandatory testing to quarantine requirements.

\includegraphics{https://static01.graylady3jvrrxbe.onion/images/2020/07/10/travel/10quarentine/merlin_174085854_8ae7efae-415f-409e-94f7-71b554aa1fb4-articleLarge.jpg?quality=75\&auto=webp\&disable=upscale}

By Karen Schwartz

\begin{itemize}
\item
  Published July 10, 2020Updated Sept. 3, 2020
\item
  \begin{itemize}
  \item
  \item
  \item
  \item
  \item
  \end{itemize}
\end{itemize}

\emph{This list will be updated as states continue to announce changes
to their travel advisories. Are we missing an update? Email us at}
\href{mailto:travelrestrictions@NYTimes.com}{\emph{travelrestrictions@NYTimes.com}}\emph{.}

While dog days of summer are winding down in the United States, the
pandemic is still ongoing. Airlines are working to make it easier for
fliers to make and adjust travel plans domestically by eliminating
\href{https://www.nytimes3xbfgragh.onion/aponline/2020/08/31/business/bc-us-airlines-change-fees.html}{most
change fees}, but travelers still face a lot of logistics, with
constantly evolving state restrictions from
\href{https://www.cdc.gov/quarantine/index.html}{mandatory testing to
quarantine requirements}. amid the ongoing pandemic

The Centers for Disease Control and Prevention notes that
\href{https://www.cdc.gov/coronavirus/2019-ncov/travelers/travel-during-covid19.html?CDC_AA_refVal=https\%3A\%2F\%2Fwww.cdc.gov\%2Fcoronavirus\%2F2019-ncov\%2Ftravelers\%2Ftravel-in-the-us.html}{travel
increases a person's chance of getting and spreading the virus}.
``Staying home is the best way to protect yourself and others from
Covid-19,'' the federal agency cautions.

For those who do take a trip, the C.D.C. says that each mode of
transportation has its own risks, and offers a series of recommendations
for safety: that people wear a face mask in public, wash hands
frequently, avoid touching their face, keep six feet from others, cover
coughs and sneezes, and use drive-through service and curbside pickup at
restaurants and stores.

Here is a summary of current restrictions in the United States for
leisure travelers, although some requirements do not apply to those
spending less than a day in the state. Many states also have exemptions
for essential workers who are on the job, including health care workers,
members of the military and others, but even
\href{https://www.ncsl.org/research/labor-and-employment/covid-19-essential-workers-in-the-states.aspx}{they
are subject to some restrictions}.

With the number of coronavirus cases surging across the country, check
the areas you plan to visit before you travel. Some municipalities or
counties may have more stringent regulations than issued by their state.

\hypertarget{alabama}{%
\subsubsection{\texorpdfstring{\href{https://alabama.travel/my-trip/staying-safe}{Alabama}}{Alabama}}\label{alabama}}

As of Sept. 1, there were no statewide restrictions in Alabama.

\hypertarget{alaska}{%
\subsubsection{\texorpdfstring{\href{https://covid19.alaska.gov/travelers/}{Alaska}}{Alaska}}\label{alaska}}

All nonresidents must upload proof of a negative or pending virus test
taken within 72 hours before departure to
\href{https://www.alaska.covidsecureapp.com/}{an online travel portal},
where they can also submit a travel declaration and self-isolation plan.
The state requests a second test be done seven to 14 days after arriving
in Alaska.

Visitors arriving without a previously taken test can get one for \$250,
and must quarantine while awaiting results at their own cost. Testing is
free for Alaska residents, who also have the option of a two-week
quarantine instead of a test.

\hypertarget{arizona}{%
\subsubsection{\texorpdfstring{\href{https://tourism.az.gov/covid-19-updates-2/}{Arizona}}{Arizona}}\label{arizona}}

As of Sept. 1, there were no statewide restrictions in Arizona.

\hypertarget{arkansas}{%
\subsubsection{\texorpdfstring{\href{https://www.arkansas.com/travel-advisory/covid-19}{Arkansas}}{Arkansas}}\label{arkansas}}

As of Sept. 1, there were no statewide restrictions in Arkansas.

\hypertarget{california}{%
\subsubsection{\texorpdfstring{\href{https://www.visitcalifornia.com/latest-covid-19-coronavirus}{California}}{California}}\label{california}}

As of Sept. 1, there were no statewide restrictions in California.

\hypertarget{colorado}{%
\subsubsection{\texorpdfstring{\href{https://covid19.colorado.gov/prepare-protect-yourself/prevent-the-spread/travel}{Colorado}}{Colorado}}\label{colorado}}

As of Sept. 1, there were no statewide restrictions in Colorado.

\hypertarget{connecticut}{%
\subsubsection{\texorpdfstring{\href{https://portal.ct.gov/Coronavirus/Covid-19-Knowledge-Base/Travel-In-or-Out-of-CT}{Connecticut}}{Connecticut}}\label{connecticut}}

With rare exceptions, those coming into Connecticut after more than 24
hours in a state or area with a high rate of confirmed infections must
self-quarantine for 14 days from their last contact with the affected
state.

The states currently affected by the order are Alabama, Alaska,
Arkansas, California, Florida, Georgia, Hawaii, Idaho, Illinois,
Indiana, Iowa, Kansas, Kentucky, Louisiana, Minnesota, Mississippi,
Missouri, Montana, Nebraska, Nevada, North Carolina, North Dakota,
Oklahoma, South Carolina, South Dakota, Tennessee, Texas, Utah, Virginia
and Wisconsin. Visitors or residents returning from Guam, Puerto Rico or
the U.S. Virgin Islands face the same restrictions.

Someone whose home state is added to the list after they are already
vacationing in Connecticut is asked to quarantine, but isn't required
to. People under self-quarantine may leave for medical visits, to obtain
medication or to shop for groceries. A person who cannot quarantine
because they are coming in for a funeral, for instance, may show proof
of negative results for a coronavirus test taken in the previous 72
hours. Those who have been tested but have not received the results are
required to quarantine until negative results are received and submitted
to the state.

\hypertarget{latest-updates-the-coronavirus-outbreak}{%
\section{\texorpdfstring{\href{https://www.nytimes3xbfgragh.onion/2020/09/08/world/covid-19-coronavirus.html?action=click\&pgtype=Article\&state=default\&region=MAIN_CONTENT_1\&context=storylines_live_updates}{Latest
Updates: The Coronavirus
Outbreak}}{Latest Updates: The Coronavirus Outbreak}}\label{latest-updates-the-coronavirus-outbreak}}

Updated 2020-09-08T10:37:22.362Z

\begin{itemize}
\tightlist
\item
  \href{https://www.nytimes3xbfgragh.onion/2020/09/08/world/covid-19-coronavirus.html?action=click\&pgtype=Article\&state=default\&region=MAIN_CONTENT_1\&context=storylines_live_updates\#link-4a77847f}{As
  senators return to Washington, an impasse over a virus relief package
  looms.}
\item
  \href{https://www.nytimes3xbfgragh.onion/2020/09/08/world/covid-19-coronavirus.html?action=click\&pgtype=Article\&state=default\&region=MAIN_CONTENT_1\&context=storylines_live_updates\#link-679303d7}{Nine
  drugmakers pledge to thoroughly vet any coronavirus vaccine.}
\item
  \href{https://www.nytimes3xbfgragh.onion/2020/09/08/world/covid-19-coronavirus.html?action=click\&pgtype=Article\&state=default\&region=MAIN_CONTENT_1\&context=storylines_live_updates\#link-1c973131}{`The
  lockdown killed my father': Farmer suicides add to India's virus
  misery.}
\end{itemize}

\href{https://www.nytimes3xbfgragh.onion/2020/09/08/world/covid-19-coronavirus.html?action=click\&pgtype=Article\&state=default\&region=MAIN_CONTENT_1\&context=storylines_live_updates}{See
more updates}

More live coverage:

Anyone arriving from one of the higher rate areas must fill out a
\href{https://appengine.egov.com/apps/ct/DPH/Connecticut-Travel-Health-Form}{mandatory
health form}. Failure to do that, or to quarantine as required, carries
a fine of up to \$1,000 per violation.

\hypertarget{delaware}{%
\subsubsection{\texorpdfstring{\href{https://www.visitdelaware.com/industry/covid-19-in-delaware/}{Delaware}}{Delaware}}\label{delaware}}

As of Sept. 1, there were no statewide restrictions in Delaware.

\hypertarget{district-of-columbia}{%
\subsubsection{\texorpdfstring{\href{https://washington.org/dc-information/coronavirus-travel-update-washington-dc}{District
of Columbia}}{District of Columbia}}\label{district-of-columbia}}

People who have been to a high-risk state for nonessential travel ** in
the previous two weeks then come into Washington for 24 hours or more
must self-quarantine for 14 days. The order excludes travelers from
Maryland and Virginia.

The states affected by the order are Alaska, Arkansas, Arizona, Alabama,
California, Delaware, Florida, Georgia, Hawaii, Idaho, Illinois,
Indiana, Iowa, Kansas, Kentucky, Louisiana, Minnesota, Mississippi,
Missouri, Nebraska, Nevada, North Carolina, North Dakota, Oklahoma,
South Carolina, South Dakota, Tennessee, Texas, Utah and Wisconsin. An
updated list is expected to be posted on Sept. 7 at
\href{http://coronavirus.dc.gov/}{coronavirus.dc.gov}.

\hypertarget{florida}{%
\subsubsection{\texorpdfstring{\href{https://floridahealthcovid19.gov/travelers/}{Florida}}{Florida}}\label{florida}}

As of Sept. 1, there were no statewide restrictions in Florida. County
and city restrictions may be in place.

\hypertarget{georgia}{%
\subsubsection{\texorpdfstring{\href{https://dph.georgia.gov/covid-19-travel}{Georgia}}{Georgia}}\label{georgia}}

As of Sept. 1, there were no statewide restrictions in Georgia.

\hypertarget{hawaii}{%
\subsubsection{\texorpdfstring{\href{https://www.hawaiitourismauthority.org/covid-19-updates/}{Hawaii}}{Hawaii}}\label{hawaii}}

Visitors and residents arriving in Hawaii from out of state must
quarantine for 14 days. A mandatory 14-day inter-island self-quarantine
is also required for those arriving to Kauai, Hawaii Island or Maui
County (Maui, Molokai, Lanai), and traveling between these islands. The
order does not include inter-island travelers arriving on Oahu.

Travelers should check the state and island websites to see which forms
need to be filled out, as different islands have different requirements.

Those violating state quarantine requirements face up to a \$5,000 fine
and up to a year in prison. The state had considered a pre-travel
testing program, but that has been delayed until Oct. 1, at the
earliest.

\hypertarget{idaho}{%
\subsubsection{\texorpdfstring{\href{https://visitidaho.org/covid-19-travel-alert/}{Idaho}}{Idaho}}\label{idaho}}

As of Sept. 1, there were no statewide restrictions in Idaho. County
restrictions may be in place: For instance, travelers to Boise and other
cities in Ada County are encouraged to self-quarantine for 14 days.

\hypertarget{illinois}{%
\subsubsection{\texorpdfstring{\href{https://www.dph.illinois.gov/topics-services/diseases-and-conditions/diseases-a-z-list/coronavirus/travel-guidance}{Illinois}}{Illinois}}\label{illinois}}

There are no statewide restrictions. However, under an
\href{https://www.chicago.gov/city/en/sites/covid-19/home/emergency-travel-order.html}{emergency
travel order}, a 14-day quarantine is required for those entering or
returning to Chicago from Alabama, Arkansas, California, Florida,
Georgia, Hawaii (as of Sept. 4), Idaho, Iowa, Kansas, Louisiana,
Mississippi, Missouri, Nebraska (as of Sept. 4), Nevada, North Carolina
(as of Sept. 4), North Dakota, Oklahoma, Puerto Rico, South Carolina,
South Dakota, Tennessee and Texas.

\hypertarget{indiana}{%
\subsubsection{\texorpdfstring{\href{https://www.coronavirus.in.gov}{Indiana}}{Indiana}}\label{indiana}}

As of Sept. 1, there were no statewide restrictions in Indiana.

\hypertarget{iowa}{%
\subsubsection{\texorpdfstring{\href{https://www.traveliowa.com/aspx/general/dynamicpage.aspx?id=204}{Iowa}}{Iowa}}\label{iowa}}

As of Sept. 1, there were no statewide restrictions in Iowa.

\hypertarget{kansas}{%
\subsubsection{\texorpdfstring{\href{https://www.coronavirus.kdheks.gov/175/Travel-Exposure-Related-Isolation-Quaran}{Kansas}}{Kansas}}\label{kansas}}

Those who attended any out-of-state gathering that included 500 people
or more on or after Aug. 11 must quarantine for 14 days when entering
Kansas.

\hypertarget{kentucky}{%
\subsubsection{\texorpdfstring{\href{https://govstatus.egov.com/kycovid19}{Kentucky}}{Kentucky}}\label{kentucky}}

Travelers who visited states with an infection rate approaching 15
percent or higher are asked to self-quarantine for 14 days.

The recommendation applies to travelers from Alabama, Iowa, Nevada,
North Dakota and South Dakota.

\hypertarget{louisiana}{%
\subsubsection{\texorpdfstring{\href{https://louisianatravelassociation.org/covid-19-resources}{Louisiana}}{Louisiana}}\label{louisiana}}

As of Sept. 1, there were no statewide restrictions in Louisiana.

\hypertarget{maine}{%
\subsubsection{\texorpdfstring{\href{https://www.maine.gov/covid19/restartingmaine/keepmainehealthy/faqs}{Maine}}{Maine}}\label{maine}}

Only residents of Connecticut, New Hampshire, New Jersey, New York and
Vermont can enter the state without restriction. Everyone else must
either self-quarantine for 14 days, or sign a document stating that they
had a negative result to the coronavirus test within the previous 72
hours. Those in quarantine may leave their hotel or campsite only for
limited outdoor activities, such as hiking, when no other people are
around.

Maine residents who travel to a state not on the exempted list must also
quarantine when they return or alternatively, test negative for the
virus.

\hypertarget{maryland}{%
\subsubsection{\texorpdfstring{\href{https://www.visitmaryland.org/article/travel-alerts}{Maryland}}{Maryland}}\label{maryland}}

As of Sept. 1, there were no statewide restrictions in Maryland.

\hypertarget{massachusetts}{%
\subsubsection{\texorpdfstring{\href{https://www.mass.gov/info-details/covid-19-updates-and-information}{Massachusetts}}{Massachusetts}}\label{massachusetts}}

Except for commuters, those passing through and people coming from
states with a lower coronavirus transmission rate, anyone over age 18
(or a minor traveling alone) who enters Massachusetts
\href{https://www.mass.gov/info-details/covid-19-travel-order}{must fill
out a travel form and either quarantine for 14 days or provide proof} of
a negative test for the coronavirus taken within the previous 72 hours.
Those awaiting test results must quarantine until a negative result is
received.

The exemption applies to those who were in
\href{https://www.mass.gov/info-details/covid-19-travel-order\#lower-risk-states-}{one
of the following states} for the two weeks before their visit to
Massachusetts: Colorado, Connecticut, Delaware, Maine, New Hampshire,
New Jersey, New York, Pennsylvania, Vermont and West Virginia.

Those who fail to comply with the rules face fines of up to \$500 per
day.

\hypertarget{michigan}{%
\subsubsection{\texorpdfstring{\href{https://www.michigan.gov/coronavirus/}{Michigan}}{Michigan}}\label{michigan}}

As of Sept. 1, there were no statewide restrictions in Michigan.

\hypertarget{minnesota}{%
\subsubsection{\texorpdfstring{\href{https://www.exploreminnesota.com/info/coronavirus-covid-19-information}{Minnesota}}{Minnesota}}\label{minnesota}}

As of Sept. 1, there were no statewide restrictions in Minnesota.

\hypertarget{mississippi}{%
\subsubsection{\texorpdfstring{\href{https://visitmississippi.org/covid-19-travel-alert/}{Mississippi}}{Mississippi}}\label{mississippi}}

As of Sept. 1, there were no statewide restrictions in Mississippi.

\hypertarget{missouri}{%
\subsubsection{\texorpdfstring{\href{https://www.visitmo.com/travel-updates}{Missouri}}{Missouri}}\label{missouri}}

As of Sept. 1, there were no statewide restrictions in Missouri.

\hypertarget{montana}{%
\subsubsection{\texorpdfstring{\href{https://www.visitmt.com/travel-alerts.html}{Montana}}{Montana}}\label{montana}}

As of Sept. 1, there were no statewide restrictions in Montana.

\hypertarget{nebraska}{%
\subsubsection{\texorpdfstring{\href{http://dhhs.ne.gov/Pages/COVID-19-Traveler-Recommendations.aspx}{Nebraska}}{Nebraska}}\label{nebraska}}

As of Sept. 1, there were no statewide restrictions in Nebraska.

\hypertarget{nevada}{%
\subsubsection{\texorpdfstring{\href{https://nvhealthresponse.nv.gov/info/travelers-visitors/}{Nevada}}{Nevada}}\label{nevada}}

As of Sept. 1, there were no statewide restrictions in Nevada.

\hypertarget{new-hampshire}{%
\subsubsection{\texorpdfstring{\href{https://www.covidguidance.nh.gov/out-state-visitors}{New
Hampshire}}{New Hampshire}}\label{new-hampshire}}

Those traveling to New Hampshire from non-New England states ``for an
extended period of time'' are asked to self-quarantine for two weeks.

\hypertarget{new-jersey}{%
\subsubsection{\texorpdfstring{\href{https://covid19.nj.gov/faqs/nj-information/general-public/which-states-are-on-the-travel-advisory-list-are-there-travel-restrictions-to-or-from-new-jersey}{New
Jersey}}{New Jersey}}\label{new-jersey}}

Those coming into New Jersey for more than 24 hours from a state or area
with a high rate of confirmed infections are asked to voluntarily
self-quarantine for 14 days, even if they had a recent negative virus
test.

The request applies to those who spent more than 24 hours in Alabama,
Alaska, Arkansas, California, Florida, Georgia, Guam, Hawaii, Idaho,
Illinois, Indiana, Iowa, Kansas, Kentucky, Louisiana, Minnesota,
Mississippi, Missouri, Montana, Nebraska, Nevada, North Carolina, North
Dakota, Oklahoma, Puerto Rico, South Carolina, South Dakota, Tennessee,
Texas, Utah, Virginia, the U.S. Virgin Islands and Wisconsin.

Travelers from those areas are also asked to complete an online survey
providing details about where they have been and where they plan to
stay.

\href{https://www.nytimes3xbfgragh.onion/news-event/coronavirus?action=click\&pgtype=Article\&state=default\&region=MAIN_CONTENT_3\&context=storylines_faq}{}

\hypertarget{the-coronavirus-outbreak-}{%
\subsubsection{The Coronavirus Outbreak
›}\label{the-coronavirus-outbreak-}}

\hypertarget{frequently-asked-questions}{%
\paragraph{Frequently Asked
Questions}\label{frequently-asked-questions}}

Updated September 4, 2020

\begin{itemize}
\item ~
  \hypertarget{what-are-the-symptoms-of-coronavirus}{%
  \paragraph{What are the symptoms of
  coronavirus?}\label{what-are-the-symptoms-of-coronavirus}}

  \begin{itemize}
  \tightlist
  \item
    In the beginning, the coronavirus
    \href{https://www.nytimes3xbfgragh.onion/article/coronavirus-facts-history.html?action=click\&pgtype=Article\&state=default\&region=MAIN_CONTENT_3\&context=storylines_faq\#link-6817bab5}{seemed
    like it was primarily a respiratory illness}~--- many patients had
    fever and chills, were weak and tired, and coughed a lot, though
    some people don't show many symptoms at all. Those who seemed
    sickest had pneumonia or acute respiratory distress syndrome and
    received supplemental oxygen. By now, doctors have identified many
    more symptoms and syndromes. In April,
    \href{https://www.nytimes3xbfgragh.onion/2020/04/27/health/coronavirus-symptoms-cdc.html?action=click\&pgtype=Article\&state=default\&region=MAIN_CONTENT_3\&context=storylines_faq}{the
    C.D.C. added to the list of early signs}~sore throat, fever, chills
    and muscle aches. Gastrointestinal upset, such as diarrhea and
    nausea, has also been observed. Another telltale sign of infection
    may be a sudden, profound diminution of one's
    \href{https://www.nytimes3xbfgragh.onion/2020/03/22/health/coronavirus-symptoms-smell-taste.html?action=click\&pgtype=Article\&state=default\&region=MAIN_CONTENT_3\&context=storylines_faq}{sense
    of smell and taste.}~Teenagers and young adults in some cases have
    developed painful red and purple lesions on their fingers and toes
    --- nicknamed ``Covid toe'' --- but few other serious symptoms.
  \end{itemize}
\item ~
  \hypertarget{why-is-it-safer-to-spend-time-together-outside}{%
  \paragraph{Why is it safer to spend time together
  outside?}\label{why-is-it-safer-to-spend-time-together-outside}}

  \begin{itemize}
  \tightlist
  \item
    \href{https://www.nytimes3xbfgragh.onion/2020/05/15/us/coronavirus-what-to-do-outside.html?action=click\&pgtype=Article\&state=default\&region=MAIN_CONTENT_3\&context=storylines_faq}{Outdoor
    gatherings}~lower risk because wind disperses viral droplets, and
    sunlight can kill some of the virus. Open spaces prevent the virus
    from building up in concentrated amounts and being inhaled, which
    can happen when infected people exhale in a confined space for long
    stretches of time, said Dr. Julian W. Tang, a virologist at the
    University of Leicester.
  \end{itemize}
\item ~
  \hypertarget{why-does-standing-six-feet-away-from-others-help}{%
  \paragraph{Why does standing six feet away from others
  help?}\label{why-does-standing-six-feet-away-from-others-help}}

  \begin{itemize}
  \tightlist
  \item
    The coronavirus spreads primarily through droplets from your mouth
    and nose, especially when you cough or sneeze. The C.D.C., one of
    the organizations using that measure,
    \href{https://www.nytimes3xbfgragh.onion/2020/04/14/health/coronavirus-six-feet.html?action=click\&pgtype=Article\&state=default\&region=MAIN_CONTENT_3\&context=storylines_faq}{bases
    its recommendation of six feet}~on the idea that most large droplets
    that people expel when they cough or sneeze will fall to the ground
    within six feet. But six feet has never been a magic number that
    guarantees complete protection. Sneezes, for instance, can launch
    droplets a lot farther than six feet,
    \href{https://jamanetwork.com/journals/jama/fullarticle/2763852}{according
    to a recent study}. It's a rule of thumb: You should be safest
    standing six feet apart outside, especially when it's windy. But
    keep a mask on at all times, even when you think you're far enough
    apart.
  \end{itemize}
\item ~
  \hypertarget{i-have-antibodies-am-i-now-immune}{%
  \paragraph{I have antibodies. Am I now
  immune?}\label{i-have-antibodies-am-i-now-immune}}

  \begin{itemize}
  \tightlist
  \item
    As of right
    now,\href{https://www.nytimes3xbfgragh.onion/2020/07/22/health/covid-antibodies-herd-immunity.html?action=click\&pgtype=Article\&state=default\&region=MAIN_CONTENT_3\&context=storylines_faq}{~that
    seems likely, for at least several months.}~There have been
    frightening accounts of people suffering what seems to be a second
    bout of Covid-19. But experts say these patients may have a
    drawn-out course of infection, with the virus taking a slow toll
    weeks to months after initial exposure.~People infected with the
    coronavirus typically
    \href{https://www.nature.com/articles/s41586-020-2456-9}{produce}~immune
    molecules called antibodies, which are
    \href{https://www.nytimes3xbfgragh.onion/2020/05/07/health/coronavirus-antibody-prevalence.html?action=click\&pgtype=Article\&state=default\&region=MAIN_CONTENT_3\&context=storylines_faq}{protective
    proteins made in response to an
    infection}\href{https://www.nytimes3xbfgragh.onion/2020/05/07/health/coronavirus-antibody-prevalence.html?action=click\&pgtype=Article\&state=default\&region=MAIN_CONTENT_3\&context=storylines_faq}{.
    These antibodies may}~last in the body
    \href{https://www.nature.com/articles/s41591-020-0965-6}{only two to
    three months}, which may seem worrisome, but that's~perfectly normal
    after an acute infection subsides, said Dr. Michael Mina, an
    immunologist at Harvard University. It may be possible to get the
    coronavirus again, but it's highly unlikely that it would be
    possible in a short window of time from initial infection or make
    people sicker the second time.
  \end{itemize}
\item ~
  \hypertarget{what-are-my-rights-if-i-am-worried-about-going-back-to-work}{%
  \paragraph{What are my rights if I am worried about going back to
  work?}\label{what-are-my-rights-if-i-am-worried-about-going-back-to-work}}

  \begin{itemize}
  \tightlist
  \item
    Employers have to provide
    \href{https://www.osha.gov/SLTC/covid-19/standards.html}{a safe
    workplace}~with policies that protect everyone equally.
    \href{https://www.nytimes3xbfgragh.onion/article/coronavirus-money-unemployment.html?action=click\&pgtype=Article\&state=default\&region=MAIN_CONTENT_3\&context=storylines_faq}{And
    if one of your co-workers tests positive for the coronavirus, the
    C.D.C.}~has said that
    \href{https://www.cdc.gov/coronavirus/2019-ncov/community/guidance-business-response.html}{employers
    should tell their employees}~-\/- without giving you the sick
    employee's name -\/- that they may have been exposed to the virus.
  \end{itemize}
\end{itemize}

\hypertarget{new-mexico}{%
\subsubsection{\texorpdfstring{\href{https://www.newmexico.org/covid-19-traveler-information/}{New
Mexico}}{New Mexico}}\label{new-mexico}}

Upon entering the state, most people, including residents who have
traveled, are required to self-quarantine for 14 days or the duration of
their stay, whichever is shorter.

\hypertarget{new-york}{%
\subsubsection{\texorpdfstring{\href{https://coronavirus.health.ny.gov/covid-19-travel-advisory}{New
York}}{New York}}\label{new-york}}

New York requires individuals who have spent more than 24 hours in a
state or area with significant community spread of the
\href{https://www.nytimes3xbfgragh.onion/2020/08/16/nyregion/coronavirus-quarantine-nyc.html}{coronavirus
to self-quarantine} for 14 days.

The states and territories affected by the quarantine order are Alabama,
Alaska, Arkansas, California, Florida, Georgia, Guam, Hawaii, Idaho,
Illinois, Indiana, Iowa, Kansas, Kentucky, Louisiana, Minnesota,
Mississippi, Missouri, Montana, Nebraska, Nevada, North Carolina, North
Dakota, Oklahoma, Puerto Rico, South Carolina, South Dakota, Tennessee,
Texas, Utah, Virginia, the U.S. Virgin Islands and Wisconsin.

Those arriving at airports in New York must fill out a
\href{https://forms.ny.gov/s3/Welcome-to-New-York-State-Traveler-Health-Form}{Health
Department traveler form}, or face a possible \$2,000 fine and a
mandatory quarantine order. Travelers arriving by air must fill out the
form before leaving the airport, while those arriving by car, train or
other modes of transportation must fill it out online. To ensure
compliance, travelers to New York City
\href{https://www.nytimes3xbfgragh.onion/2020/08/05/nyregion/nyc-coronavirus-quarantine-checkpoints.html}{may
be stopped at random at bridge and tunnel crossings}, in Penn Station
and at the Port Authority Bus Terminal.

\hypertarget{north-carolina}{%
\subsubsection{\texorpdfstring{\href{https://www.nc.gov/covid-19/covid-19-travel-resources}{North
Carolina}}{North Carolina}}\label{north-carolina}}

As of Sept. 1, there were no statewide restrictions in North Carolina.

\hypertarget{north-dakota}{%
\subsubsection{\texorpdfstring{\href{https://www.health.nd.gov/diseases-conditions/coronavirus/travel}{North
Dakota}}{North Dakota}}\label{north-dakota}}

As of Sept. 1, there were no statewide restrictions in North Dakota.

\hypertarget{ohio}{%
\subsubsection{\texorpdfstring{\href{https://coronavirus.ohio.gov/wps/portal/gov/covid-19/families-and-individuals/COVID-19-Travel-Advisory/}{Ohio}}{Ohio}}\label{ohio}}

Traveling Ohioans and out-of-state tourists who have visited an area of
high risk, or who have had possible exposure to the coronavirus, are
asked to voluntarily quarantine for 14 days.

As of Sept. 2, Ohio has identified the following states as high risk:
Alabama, Iowa, Kansas, Nevada, North Dakota and South Dakota.

\hypertarget{oklahoma}{%
\subsubsection{\texorpdfstring{\href{https://coronavirus.health.ok.gov/travel}{Oklahoma}}{Oklahoma}}\label{oklahoma}}

As of Sept. 1, there were no statewide restrictions in Oklahoma.

\hypertarget{oregon}{%
\subsubsection{\texorpdfstring{\href{https://traveloregon.com/travel-alerts/}{Oregon}}{Oregon}}\label{oregon}}

As of Sept. 1, there were no statewide restrictions in Oregon.

\hypertarget{pennsylvania}{%
\subsubsection{\texorpdfstring{\href{https://www.health.pa.gov/topics/disease/coronavirus/Pages/Travelers.aspx}{Pennsylvania}}{Pennsylvania}}\label{pennsylvania}}

The state asks travelers who have visited an area with a Covid-19 surge
to self-quarantine for 14 days. The states are Alabama, Arkansas,
California, Florida, Georgia, Idaho, Illinois, Kansas, Louisiana,
Mississippi, Missouri, Nevada, North Dakota, Oklahoma, South Carolina,
South Dakota, Tennessee and Texas.

\hypertarget{rhode-island}{%
\subsubsection{\texorpdfstring{\href{https://health.ri.gov/covid/}{Rhode
Island}}{Rhode Island}}\label{rhode-island}}

Those coming to Rhode Island from a state that has a positivity rate for
tests of greater than 5 percent are required to self-quarantine for two
weeks. Alternatively, visitors can provide a negative test for the virus
that was taken within the previous 72 hours. A person who receives a
negative test during their quarantine can stop isolating, although the
state recommends the full two-week quarantine.

\href{https://docs.google.com/spreadsheets/d/e/2PACX-1vSUCk9FlHBoJt5ZO0U6PKTTY7jHH8V4MovED0WiqpTTixdgMSCnUWI25xX5DCmQmtLknzu7Bo0jwY02/pubhtml?gid=0\&single=true}{The
states identified} are Alabama, Arizona, Arkansas, California, Florida,
Georgia, Hawaii, Idaho, Indiana, Iowa, Kansas, Kentucky, Minnesota,
Mississippi, Missouri, Nebraska, Nevada, North Carolina, North Dakota,
Oklahoma, South Carolina, South Dakota, Tennessee, Texas, Utah, Virginia
and Wisconsin. Visitors from Puerto Rico must also quarantine.

\hypertarget{south-carolina}{%
\subsubsection{\texorpdfstring{\href{https://scdhec.gov/infectious-diseases/viruses/coronavirus-disease-2019-covid-19/travelers-covid-19}{South
Carolina}}{South Carolina}}\label{south-carolina}}

The state recommends that people who have visited an area with
widespread or ongoing community transmission of the virus stay home as
much as possible for 14 days from the time they left that region.

\hypertarget{south-dakota}{%
\subsubsection{\texorpdfstring{\href{https://www.travelsouthdakota.com/coronavirus-covid-19}{South
Dakota}}{South Dakota}}\label{south-dakota}}

As of Sept. 1, there were no statewide restrictions in South Dakota.

\hypertarget{tennessee}{%
\subsubsection{\texorpdfstring{\href{https://www.tnvacation.com/articles/tennessee-travel-amid-coronavirus}{Tennessee}}{Tennessee}}\label{tennessee}}

As of Sept. 1, there were no statewide restrictions in Tennessee.

\hypertarget{texas}{%
\subsubsection{\texorpdfstring{\href{https://gov.texas.gov/travel-texas/page/covid19}{Texas}}{Texas}}\label{texas}}

As of Sept. 1, there were no statewide restrictions in Texas.

\hypertarget{utah}{%
\subsubsection{\texorpdfstring{\href{https://www.visitutah.com/plan-your-trip/covid-19/}{Utah}}{Utah}}\label{utah}}

As of Sept. 1, there were no statewide restrictions in Utah.

\hypertarget{vermont}{%
\subsubsection{\texorpdfstring{\href{https://www.healthvermont.gov/response/coronavirus-covid-19/traveling-vermont}{Vermont}}{Vermont}}\label{vermont}}

Visitors from counties in select states that have similar active
coronavirus rates to Vermont and who travel in a private vehicle do not
have to quarantine. The same is true for Vermont residents who visit
those regions when they return home.

\href{https://accd.vermont.gov/covid-19/restart/cross-state-travel}{These
counties} are in Connecticut, Maine, Massachusetts, New Hampshire, Rhode
Island, New York, Pennsylvania, Ohio, New Jersey, Delaware, Maryland,
Virginia, West Virginia and Washington, D.C.

Most other travelers need to self-quarantine upon arrival in Vermont,
but the state gives them a few options. People may self-quarantine out
of state before traveling to Vermont as long as their trip is in a
private vehicle and they make only necessary stops, while wearing a face
mask, social distancing and washing their hands frequently. Those opting
to self-quarantine before their visit to Vermont can either do it for 14
days, or they can shorten it to seven days if they then get a negative
test result.

Those arriving by public transportation or a longer car ride must
self-quarantine for 14 days, or for seven days followed by a negative
test.

\hypertarget{virginia}{%
\subsubsection{\texorpdfstring{\href{https://www.vdh.virginia.gov/coronavirus/frequently-asked-questions/u-s-travelers/}{Virginia}}{Virginia}}\label{virginia}}

As of Sept. 1, there were no statewide restrictions in Virginia.

\hypertarget{washington}{%
\subsubsection{\texorpdfstring{\href{https://www.experiencewa.com/articles/date-coronavirus-travel-advisory}{Washington}}{Washington}}\label{washington}}

As of Sept. 1, there were no statewide restrictions in Washington.

\hypertarget{west-virginia}{%
\subsubsection{\texorpdfstring{\href{https://wvtourism.com/travel-alert/}{West
Virginia}}{West Virginia}}\label{west-virginia}}

As of Sept. 1, there were no statewide restrictions in West Virginia.

\hypertarget{wisconsin}{%
\subsubsection{\texorpdfstring{\href{https://www.dhs.wisconsin.gov/covid-19/travel.htm}{Wisconsin}}{Wisconsin}}\label{wisconsin}}

As of Sept. 1, there were no statewide restrictions in Wisconsin. Local
quarantine restrictions may be in place at the county level.

\hypertarget{wyoming}{%
\subsubsection{\texorpdfstring{\href{https://health.wyo.gov/publichealth/infectious-disease-epidemiology-unit/disease/novel-coronavirus/covid-19-orders-and-guidance/}{Wyoming}}{Wyoming}}\label{wyoming}}

As of Sept. 1, there were no statewide restrictions in Wyoming.

Follow Karen Schwartz on Twitter:
\href{https://twitter.com/wanderwomanisme?lang=en}{@WanderWomanIsMe}

\begin{center}\rule{0.5\linewidth}{\linethickness}\end{center}

\emph{\textbf{Follow New York Times Travel}}
\emph{on}\href{https://www.instagram.com/nytimestravel/}{\emph{Instagram}}\emph{,}\href{https://twitter.com/nytimestravel}{\emph{Twitter}}
\emph{and}\href{https://www.facebookcorewwwi.onion/nytimestravel/}{\emph{Facebook}}\emph{.
And}\href{https://www.nytimes3xbfgragh.onion/newsletters/traveldispatch}{\emph{sign
up for our weekly Travel Dispatch newsletter}} \emph{to receive expert
tips on traveling smarter and inspiration for your next vacation.}

Advertisement

\protect\hyperlink{after-bottom}{Continue reading the main story}

\hypertarget{site-index}{%
\subsection{Site Index}\label{site-index}}

\hypertarget{site-information-navigation}{%
\subsection{Site Information
Navigation}\label{site-information-navigation}}

\begin{itemize}
\tightlist
\item
  \href{https://help.nytimes3xbfgragh.onion/hc/en-us/articles/115014792127-Copyright-notice}{©~2020~The
  New York Times Company}
\end{itemize}

\begin{itemize}
\tightlist
\item
  \href{https://www.nytco.com/}{NYTCo}
\item
  \href{https://help.nytimes3xbfgragh.onion/hc/en-us/articles/115015385887-Contact-Us}{Contact
  Us}
\item
  \href{https://www.nytco.com/careers/}{Work with us}
\item
  \href{https://nytmediakit.com/}{Advertise}
\item
  \href{http://www.tbrandstudio.com/}{T Brand Studio}
\item
  \href{https://www.nytimes3xbfgragh.onion/privacy/cookie-policy\#how-do-i-manage-trackers}{Your
  Ad Choices}
\item
  \href{https://www.nytimes3xbfgragh.onion/privacy}{Privacy}
\item
  \href{https://help.nytimes3xbfgragh.onion/hc/en-us/articles/115014893428-Terms-of-service}{Terms
  of Service}
\item
  \href{https://help.nytimes3xbfgragh.onion/hc/en-us/articles/115014893968-Terms-of-sale}{Terms
  of Sale}
\item
  \href{https://spiderbites.nytimes3xbfgragh.onion}{Site Map}
\item
  \href{https://help.nytimes3xbfgragh.onion/hc/en-us}{Help}
\item
  \href{https://www.nytimes3xbfgragh.onion/subscription?campaignId=37WXW}{Subscriptions}
\end{itemize}
