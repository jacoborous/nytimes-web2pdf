Sections

SEARCH

\protect\hyperlink{site-content}{Skip to
content}\protect\hyperlink{site-index}{Skip to site index}

\href{https://www.nytimes3xbfgragh.onion/section/science}{Science}

\href{https://myaccount.nytimes3xbfgragh.onion/auth/login?response_type=cookie\&client_id=vi}{}

\href{https://www.nytimes3xbfgragh.onion/section/todayspaper}{Today's
Paper}

\href{/section/science}{Science}\textbar{}Beyond the Milky Way, a
Galactic Wall

\url{https://nyti.ms/3037asy}

\begin{itemize}
\item
\item
\item
\item
\item
\end{itemize}

Advertisement

\protect\hyperlink{after-top}{Continue reading the main story}

Supported by

\protect\hyperlink{after-sponsor}{Continue reading the main story}

Out There

\hypertarget{beyond-the-milky-way-a-galactic-wall}{%
\section{Beyond the Milky Way, a Galactic
Wall}\label{beyond-the-milky-way-a-galactic-wall}}

Astronomers have discovered a vast assemblage of galaxies hidden behind
our own, in the ``zone of avoidance.''

\includegraphics{https://static01.graylady3jvrrxbe.onion/images/2020/07/21/science/10cosmicwall-mw/10cosmicwall-mw-articleLarge-v3.jpg?quality=75\&auto=webp\&disable=upscale}

\href{https://www.nytimes3xbfgragh.onion/by/dennis-overbye}{\includegraphics{https://static01.graylady3jvrrxbe.onion/images/2018/07/30/multimedia/author-dennis-overbye/author-dennis-overbye-thumbLarge.png}}

By \href{https://www.nytimes3xbfgragh.onion/by/dennis-overbye}{Dennis
Overbye}

\begin{itemize}
\item
  July 10, 2020
\item
  \begin{itemize}
  \item
  \item
  \item
  \item
  \item
  \end{itemize}
\end{itemize}

Astronomers have discovered that there is a vast wall across the
southern border of the local cosmos.

The South Pole Wall, as it is known, consists of thousands of galaxies
--- beehives of trillions of stars and dark worlds, as well as dust and
gas --- aligned in a curtain arcing across at least 700 million
light-years of space. It winds behind the dust, gas and stars of our own
galaxy, the Milky Way, from the constellation Perseus in the Northern
Hemisphere to the constellation Apus in the far south. It is so massive
that it perturbs the local expansion of the universe.

But don't bother trying to see it. The entire conglomeration is behind
the Milky Way, in what astronomers quaintly call the zone of avoidance.

An international team of astronomers led by Daniel Pomarède of
Paris-Saclay University and R. Brent Tully of the University of Hawaii
announced this new addition to the local universe on Friday in a paper
in \href{https://doi.org/10.3847/1538-4357/ab9952}{Astrophysical
Journal}. The paper is festooned with maps and diagrams of blobby and
stringy features of our local universe as well as a video tour of the
South Pole Wall.

It is the latest installment of an ongoing mission to determine where we
are in the universe --- to fix our neighborhood among the galaxies and
the endless voids --- and where we are going.

``The surprise for us is that this structure is as big as the Sloan
Great Wall and twice as close, and remained unnoticed, being hidden in
an obscured sector of the southern sky,'' Dr. Pomarède said in an email.

``The discovery is a wonderful poster child for the power of
visualizations in research,'' Dr. Tully said.

The new wall joins a host of other cosmographic features: arrangements
of galaxies, or a lack of them, that astronomers have come to know and
love over the last few decades, with names like the Great Wall, the
Sloan Great Wall, the
\href{https://en.wikipedia.org/wiki/Hercules\%E2\%80\%93Corona_Borealis_Great_Wall}{Hercules-Corona
Borealis Great Wall} and the Bootes Void.

The new paper was based on measurements, performed by Dr. Tully and his
colleagues, of the distances of 18,000 galaxies as far away as 600
million light-years. By comparison, the most distant objects we can see
--- quasars and galaxies that formed shortly after the Big Bang --- are
about 13 billion light years away.

\includegraphics{https://static01.graylady3jvrrxbe.onion/images/2020/07/10/science/10SCI-OUTTHERE-COSMICWALL/10SCI-OUTTHERE-COSMICWALL-articleLarge.jpg?quality=75\&auto=webp\&disable=upscale}

The galaxies in the wall cannot be seen, but Dr. Pomarède and his
colleagues were able to observe ** their gravitational effects by
assembling data from telescopes around the world.

In the expanding universe, as described in 1929 by the astronomer Edwin
Hubble and confirmed for almost a century, distant galaxies are flying
away from us as if they were dots on an inflating balloon; the farther
they are, the faster they recede from us, according to a relation called
the Hubble law.

That motion away from Earth causes their light to be shifted to longer,
redder wavelengths and lower frequencies, like retreating ambulance
sirens. Astronomers use this ``redshift,'' which is easily measured, as
a proxy for relative distance in the universe. By measuring the galaxy
distances independently, the ``Cosmicflows'' team, as Dr. Pomarède and
his colleagues call themselves, was able to distinguish the motion
caused by the cosmic expansion from motions caused by gravitational
irregularities.

As a result, they found that the galaxies between Earth and the South
Pole Wall are sailing away from us slightly faster than they otherwise
should be, by about 30 miles per second, drawn outward by the enormous
blob of matter in the wall. And galaxies beyond the wall are moving
outward more slowly than they otherwise should be, reined in by the
gravitational drag of the wall.

One astonishing aspect of the wall is how big it is compared to the
volume that the team was surveying: a contiguous filament of light 1.4
billion light-years long, packed into a cloud maybe 600 million in
radius. ``There is hardly room in the volume for anything bigger!'' Dr.
Tully said in an email. ``We'd have to anticipate that our view of the
filament is clipped; that it extends beyond our survey horizon.''

And yet the South Pole Wall is nearby in cosmological terms. ``One might
wonder how such a large and not-so-distant structure remained
unnoticed,'' Dr. Pomarède mused in a statement issued by his university.

But in the expanding universe, there is always something more to see.

On the largest scales, cosmologists attest, the universe should be
expanding smoothly, and the galaxies should be evenly distributed. But
on smaller, more local scales, the universe appears lumpy and gnarled.
Astronomers have found that galaxies are gathered, often by the
thousands, in giant clouds called clusters and that these are connected
to one another in lacy, luminous chains and filaments to form
superclusters extending across billions of light-years. In between are
vast deserts of darkness called voids.

From all of this has emerged what some astronomers call our ``long
address'': We live on Earth, which is in the solar system, which is in
the Milky Way galaxy. The Milky Way is part of a small cluster of
galaxies called the Local Group, which is on the edge of the Virgo
cluster, a conglomeration of several thousand galaxies.

In 2014, Dr. Tully suggested that these features were all connected, as
part of a giant conglomeration he called Laniakea --- Hawaiian for
``open skies'' or ``immense heaven.'' It consists of 100,000 galaxies
spread across 500 million light-years.

All this lumpiness has distorted the expansion of the universe. In 1986,
a group of astronomers who called themselves the Seven Samurai announced
that the galaxies in a huge swath of the sky in the direction of the
constellation Centaurus were flying away much faster than the Hubble law
predicted, as if being pulled toward something --- something the
astronomers called the Great Attractor. It was the beginning of
something big.

``We now see the Great Attractor as the downtown region of the
supercluster that we live in --- an overall entity that our team has
called the Laniakea Supercluster,'' Dr. Tully said. All the different
parts of this supercluster are tugging on us, he added.

As a result, the Great Attractor and its relatives are shedding light on
another enduring cosmic mystery --- namely, where we are headed.

Astronomers discovered in 1965 that space is suffused with microwave
radiation, a bath of heat --- with a temperature of 2.7 degrees Kelvin,
or minus 455 degrees Fahrenheit --- left over from the birth of the
universe 14 billion years ago. Subsequent observations revealed that
this bath is not uniform: It is slightly warmer in one direction,
suggesting that we --- Earth, our galaxy and the Local Group --- are
moving through the microwaves, like a goldfish in a fishbowl, at about
400 miles per second in the approximate direction of Centaurus, but
aiming far beyond.

Image

A projection of the South Pole Wall in celestial coordinates. The plane
of the Milky Way is shown by a dust map in shades of grey; what lies
behind it is obscured from direct observation.Credit...Daniel Pomarède

Why? What is over there, on the other side of the fishbowl, compelling
us? That is the kind of question that would come up in an Arthur C.
Clarke novel, where humanity is always gearing up for some definitive
expedition around the curve of the universe.

``A major goal in cosmology is to explain this motion,'' Dr. Tully said
in a series of emails. In theory, the motion arises from the lumpy
distribution of matter that grew out of tiny ripples in the density of
the early universe.

``The Great Attractor is certainly an important part of the cause of our
motion,'' Dr. Tully said. ``The South Pole Wall also contributes but,
again, only in part,'' he added, listing more local galaxy clusters and
voids. ``Every hill and valley in the density distribution makes itself
felt.''

\href{https://www.youtube.com/watch?v=UmW5TAULdAs}{Most of that is
stuff}that we cannot see directly. According to the prevailing theory of
a confoundingly preposterous universe, the cosmos contains about five
times as much invisible dark matter as luminous atomic matter.

\textbf{\emph{{[}}\href{http://on.fb.me/1paTQ1h}{\emph{Like the Science
Times page on Facebook.}}} ****** \emph{\textbar{} Sign up for the}
\textbf{\href{http://nyti.ms/1MbHaRU}{\emph{Science Times
newsletter.}}\emph{{]}}}

Nobody knows exactly what dark matter is made of, but according to
cosmologists it provides the gravitational scaffolding for the luminous
structures in the universe --- galaxies, galaxy clusters, superclusters,
voids and chains like the South Pole Wall, all connected by spidery
filaments in what's known as the cosmic web. The visible universe of
stars and galaxies, cosmologists like to say, is like snow on
mountaintops or lights on dark, distant Christmas trees.

But by following the lights and how they are moving, astronomers like
Dr. Tully and his cosmographers have now been able to probe the shadows
on which they sit: galumphing clouds of mass whose gravity shapes the
destiny of the visible cosmos, arranging it into shapes and
neighborhoods, walls, valleys and voids.

``It's just dark matter having its way,'' Dr. Tully said.

``We are like swimmers attempting to swim upstream but being carried
downstream faster.''

Advertisement

\protect\hyperlink{after-bottom}{Continue reading the main story}

\hypertarget{site-index}{%
\subsection{Site Index}\label{site-index}}

\hypertarget{site-information-navigation}{%
\subsection{Site Information
Navigation}\label{site-information-navigation}}

\begin{itemize}
\tightlist
\item
  \href{https://help.nytimes3xbfgragh.onion/hc/en-us/articles/115014792127-Copyright-notice}{©~2020~The
  New York Times Company}
\end{itemize}

\begin{itemize}
\tightlist
\item
  \href{https://www.nytco.com/}{NYTCo}
\item
  \href{https://help.nytimes3xbfgragh.onion/hc/en-us/articles/115015385887-Contact-Us}{Contact
  Us}
\item
  \href{https://www.nytco.com/careers/}{Work with us}
\item
  \href{https://nytmediakit.com/}{Advertise}
\item
  \href{http://www.tbrandstudio.com/}{T Brand Studio}
\item
  \href{https://www.nytimes3xbfgragh.onion/privacy/cookie-policy\#how-do-i-manage-trackers}{Your
  Ad Choices}
\item
  \href{https://www.nytimes3xbfgragh.onion/privacy}{Privacy}
\item
  \href{https://help.nytimes3xbfgragh.onion/hc/en-us/articles/115014893428-Terms-of-service}{Terms
  of Service}
\item
  \href{https://help.nytimes3xbfgragh.onion/hc/en-us/articles/115014893968-Terms-of-sale}{Terms
  of Sale}
\item
  \href{https://spiderbites.nytimes3xbfgragh.onion}{Site Map}
\item
  \href{https://help.nytimes3xbfgragh.onion/hc/en-us}{Help}
\item
  \href{https://www.nytimes3xbfgragh.onion/subscription?campaignId=37WXW}{Subscriptions}
\end{itemize}
