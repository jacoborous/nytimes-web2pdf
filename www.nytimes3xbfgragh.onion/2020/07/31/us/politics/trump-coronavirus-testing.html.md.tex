Sections

SEARCH

\protect\hyperlink{site-content}{Skip to
content}\protect\hyperlink{site-index}{Skip to site index}

\href{https://www.nytimes3xbfgragh.onion/section/politics}{Politics}

\href{https://myaccount.nytimes3xbfgragh.onion/auth/login?response_type=cookie\&client_id=vi}{}

\href{https://www.nytimes3xbfgragh.onion/section/todayspaper}{Today's
Paper}

\href{/section/politics}{Politics}\textbar{}Trump's Coronavirus Testing
Chief Concedes a Lag in Test Results

\url{https://nyti.ms/2XhAWZL}

\begin{itemize}
\item
\item
\item
\item
\item
\end{itemize}

\begin{itemize}
\item
  \href{https://www.nytimes3xbfgragh.onion/2020/07/31/us/elections/biden-vs-trump.html?action=click\&pgtype=Article\&state=default\&region=TOP_BANNER\&context=storylines_menu}{Election
  Updates}
\item
  \href{https://www.nytimes3xbfgragh.onion/article/biden-vice-president-2020.html?action=click\&pgtype=Article\&state=default\&region=TOP_BANNER\&context=storylines_menu}{Biden's
  V.P. Search}
\item
  \href{https://www.nytimes3xbfgragh.onion/interactive/2020/07/24/us/politics/trump-biden-campaign-donors.html?action=click\&pgtype=Article\&state=default\&region=TOP_BANNER\&context=storylines_menu}{Map
  of Donations}
\item
  \href{https://www.nytimes3xbfgragh.onion/interactive/2020/us/elections/delegate-count-primary-results.html?action=click\&pgtype=Article\&state=default\&region=TOP_BANNER\&context=storylines_menu}{Delegate
  Count}
\item
  \href{https://www.nytimes3xbfgragh.onion/interactive/2019/us/politics/2020-presidential-candidates.html?action=click\&pgtype=Article\&state=default\&region=TOP_BANNER\&context=storylines_menu}{The
  Candidates}
\item
  \href{https://www.nytimes3xbfgragh.onion/newsletters/politics?action=click\&pgtype=Article\&state=default\&region=TOP_BANNER\&context=storylines_menu}{Politics
  Newsletter}
\end{itemize}

Advertisement

\protect\hyperlink{after-top}{Continue reading the main story}

Supported by

\protect\hyperlink{after-sponsor}{Continue reading the main story}

\hypertarget{trumps-coronavirus-testing-chief-concedes-a-lag-in-test-results}{%
\section{Trump's Coronavirus Testing Chief Concedes a Lag in Test
Results}\label{trumps-coronavirus-testing-chief-concedes-a-lag-in-test-results}}

With the reopening plans of schools and businesses hinging on rapid test
results, the Trump administration's testing czar says a two- to
three-day turnaround ``is not possible.''

\includegraphics{https://static01.graylady3jvrrxbe.onion/images/2020/07/31/us/politics/31dc-virus-hearing/merlin_175159080_b6d984cc-ecd9-412a-9c63-9fe3fb62b4cf-articleLarge.jpg?quality=75\&auto=webp\&disable=upscale}

By
\href{https://www.nytimes3xbfgragh.onion/by/sheryl-gay-stolberg}{Sheryl
Gay Stolberg} and
\href{https://www.nytimes3xbfgragh.onion/by/katherine-j--wu}{Katherine
J. Wu}

\begin{itemize}
\item
  July 31, 2020
\item
  \begin{itemize}
  \item
  \item
  \item
  \item
  \item
  \end{itemize}
\end{itemize}

WASHINGTON --- With schools, universities and businesses pinning their
hopes for reopening on rapid coronavirus testing, the Trump
administration's testing czar told Congress on Friday that getting test
results within two to three days ``is not a possible benchmark we can
achieve today.''

But even that sober assessment from Adm. Brett P. Giroir, the assistant
secretary for health, most likely did not fully reflect the mounting
frustration among patients and health professionals just as the school
year struggles to get started.

During a lengthy House hearing with top government health officials, Dr.
Giroir told lawmakers that the nation was averaging about 820,000 tests
daily, up from 550,000 earlier this month. But the raw numbers belie the
testing crunch that officials around the country are facing amid soaring
caseloads, particularly in the South and West.

``Turnaround times are definitely improving,'' Dr. Giroir insisted,
adding that it was ``very atypical'' to wait more than 12 days for
results.

But many researchers are still grappling with severe shortages of the
testing supplies needed to collect samples from patients and process
them in laboratories. That leaves state and local officials without
information they need to make critical health decisions, and it creates
lags in contact tracing --- a necessary tool for controlling the
pandemic's spread.

``We're doing so many tests, sometimes it takes seven to 10 days to get
the results back,'' Gov. Ron DeSantis of Florida noted on Friday evening
in a briefing there with President Trump.

Coronavirus testing is essential to opening the economy and getting
people back to work and school, but it is almost useless if long lag
times keep people unnecessarily quarantined for days or allow them to
spread the virus while they await their results.

Dr. Giroir insisted that over all, 59 percent of tests report results
within three days, and 76 percent within five.

``I'm sure there's an outlier at 12 to 16 days because that happens,''
he added, ``but that's very atypical.''

\hypertarget{latest-updates-2020-election}{%
\section{\texorpdfstring{\href{https://www.nytimes3xbfgragh.onion/2020/07/31/us/elections/biden-vs-trump.html?action=click\&pgtype=Article\&state=default\&region=MAIN_CONTENT_1\&context=storylines_live_updates}{Latest
Updates: 2020
Election}}{Latest Updates: 2020 Election}}\label{latest-updates-2020-election}}

Updated 2020-08-01T01:26:45.732Z

\begin{itemize}
\tightlist
\item
  \href{https://www.nytimes3xbfgragh.onion/2020/07/31/us/elections/biden-vs-trump.html?action=click\&pgtype=Article\&state=default\&region=MAIN_CONTENT_1\&context=storylines_live_updates\#link-29fdff45}{Kamala
  Harris, a top vice-presidential contender, confronts double
  standards.}
\item
  \href{https://www.nytimes3xbfgragh.onion/2020/07/31/us/elections/biden-vs-trump.html?action=click\&pgtype=Article\&state=default\&region=MAIN_CONTENT_1\&context=storylines_live_updates\#link-13ec3d9c}{Karen
  Bass and Susan Rice are rising on Biden's vice-presidential
  shortlist.}
\item
  \href{https://www.nytimes3xbfgragh.onion/2020/07/31/us/elections/biden-vs-trump.html?action=click\&pgtype=Article\&state=default\&region=MAIN_CONTENT_1\&context=storylines_live_updates\#link-49e9a016}{Trump
  says Russian bounties to kill U.S. troops `never took place.'}
\end{itemize}

\href{https://www.nytimes3xbfgragh.onion/2020/07/31/us/elections/biden-vs-trump.html?action=click\&pgtype=Article\&state=default\&region=MAIN_CONTENT_1\&context=storylines_live_updates}{See
more updates}

Dr. Giroir's comments, during a hearing of the House Select Subcommittee
on the Coronavirus Crisis, were met with puzzlement by public health
experts, who say testing shortages persist. In some places, tests cannot
be processed at all because of a lack of reagents --- the chemicals
needed to detect whether the virus is present --- or lab capacity.

And anxious patients around the country paint a far bleaker picture.
Shawn Jain, who was tested along with several family members on July 7
in Nashville, waited 16 days for his results after they were processed
by Quest Diagnostics. Some of his family members still have not heard
back.

``I honestly thought they had lost the test,'' said Mr. Jain, who tested
negative for the virus. He added, ``It made me feel like, well if in the
future I do worry I have it, I can't even rely on something as basic as
testing.''

On Friday, the
\href{https://www.nih.gov/news-events/news-releases/nih-delivering-new-covid-19-testing-technologies-meet-us-demand}{National
Institutes of Health announced} awards totaling \$248.7 million for
seven companies to ramp up test production and deliver millions more
weekly tests as early as September. The N.I.H. director, Dr. Francis
Collins, described the announcement as the ``first of more awards to
come.'' Three of the tests are simple enough to deliver results in 30
minutes or less.

At the hearing, Dr. Anthony S. Fauci, the director of the National
Institute of Allergy and Infectious Diseases, said again that a safe and
effective coronavirus vaccine would most likely be ready by the end of
this year or early next, and cast doubt on efforts by Russia and China.

``I do hope that the Chinese and the Russians are actually testing the
vaccine before they're administering the vaccine to anyone,'' Dr. Fauci
said.

Until a vaccine is available, testing remains critical, but
\href{https://www.nytimes3xbfgragh.onion/2020/07/06/health/fast-coronavirus-tests.html}{new
diagnostic tools} will not come soon enough for the fall semester at
universities and colleges around the country. Many have decided to
transition to online classes in part because administrators cannot be
assured that enough testing will be available to keep students, faculty
and staff safe.

\includegraphics{https://static01.graylady3jvrrxbe.onion/images/2020/07/31/us/politics/31dc-virus-hearing2/merlin_175126200_5134a356-bdd7-4f84-bc77-73b307c92c0d-articleLarge.jpg?quality=75\&auto=webp\&disable=upscale}

``Covid-19 testing capacity and delays in reporting results remain a
challenge,'' Sylvia M. Burwell, a health secretary in the Obama
administration who is now the president of American University,
\href{https://www.american.edu/president/announcements/july-30-2020.cfm}{wrote
this week}, in announcing that there would be ``no residential
experience'' for students this fall.

``The ability to test and support contact tracing is critical to
reducing community spread of Covid-19,'' she added, ``and the ambiguity
in this area presents a significant hurdle for all.''

Other colleges were in the same position.

``The increased spread of the virus nationwide, the impact that this
resurgence has had on the availability of testing supplies needed to
satisfy our testing protocols, and the strong national trend of rising
rates of infection in younger populations lead us to conclude that our
community is best served by maintaining social distancing in miles
rather than feet,'' Alison R. Byerly, the president of Lafayette College
in Pennsylvania, wrote last week.

Democrats seized on reports of testing delays to demand a national
testing strategy.

``We once again call upon the president to get serious about this ---
no, testing is not overrated,'' Speaker Nancy Pelosi told reporters as
the hearing was underway.

Testing delays hurt efforts to contain the spread of the virus.
Diagnostic tests reflect only a person's health status on the day a
sample is collected. While those who visit testing sites are typically
told to quarantine at home while they await their results, that advice
becomes harder to take the longer people are forced to wait ---
especially for those who work essential jobs that cannot be done
remotely.

``It is an issue if you can't get it within a 24-to-48-hour period,''
Dr. Fauci said.

And diagnostic tests alone are not enough. Experts say that in order to
stop the pandemic, the country will have to expand testing in the
broader community --- not just to identify sick people, but to assess
the prevalence of disease in the general population and to catch
asymptomatic people who might be unknowingly carrying the virus.

About half of all coronavirus tests, Dr. Giroir said, are conducted in
so-called point-of-care settings --- like doctor's offices or urgent
care clinics, without the need to route samples through laboratories ---
or hospitals. Point-of-care tests, intended to be fast and simple enough
to obviate the need for specialized equipment or highly trained
personnel, can yield results in 15 minutes, he said, while hospital
tests take around a day.

The remainder of coronavirus tests are conducted by large-scale
commercial laboratory companies, like LabCorp and Quest Diagnostics,
which are strained near their limits.

Image

From left, Robert Redfield, Dr. Anthony Fauci, and Mr. Giroir,
testifying during a House Select Subcommittee hearing on Friday. ``It is
an issue if you cant get it within a 24 to 48 hour period,'' Dr. Fauci
said.Credit...Pool photo by Erin Scott

This month, Daniel Larremore, a mathematician and infectious disease
modeler at the University of Colorado, Boulder, collected data about
testing delays via an
\href{https://larremorelab.github.io/covid19testgroup}{informal survey
on Twitter}, showing that residents in multiple states were experiencing
prolonged test result turnaround times.

What is more, turnaround times do not reflect ``how long it takes for
people to get a test in the first place, and how convenient it is to get
the testing done,'' Dr. Larremore said. He also pointed out that many
places were still prioritizing patients with symptoms for testing.

But mounting evidence suggests that about 40 percent of coronavirus
infections could present without symptoms entirely. Tests are also still
failing to reach many of those who need it the most, including
communities marginalized by race and ethnicity, who have been
\href{https://www.nytimes3xbfgragh.onion/interactive/2020/07/05/us/coronavirus-latinos-african-americans-cdc-data.html}{disproportionately
affected by the coronavirus}.

Michael T. Osterholm, the director of the Center for Infectious Disease
Research and Policy at the University of Minnesota, said the
administration needed a ``national dashboard for testing'' where data
was collected and made publicly available.

``We need to know how many people are tested, by which test, how long
does it take to get a result back and where there is testing capacity
available, but they can't be done because there is an absence of reagent
or other critical components,'' he said.

While the country's capacity for testing has certainly increased, Dr.
Larremore said, ``That doesn't mean that we are where we need to be ---
just that we're continuing to accelerate.''

Sheryl Gay Stolberg reported from Washington, and Katherine J. Wu from
Boston.

\hypertarget{our-2020-election-guide}{%
\section{Our 2020 Election Guide}\label{our-2020-election-guide}}

Updated July 31, 2020

\begin{itemize}
\item
  \begin{center}\rule{0.5\linewidth}{\linethickness}\end{center}

  \hypertarget{the-latest}{%
  \subsection{The Latest}\label{the-latest}}

  \begin{itemize}
  \tightlist
  \item
    President Trump's assault on the Postal Service is intersecting with
    his attacks on mail-in voting.
    \href{https://www.nytimes3xbfgragh.onion/2020/07/31/us/politics/trump-usps-mail-delays.html?action=click\&pgtype=Article\&state=default\&region=BELOW_MAIN_CONTENT\&context=storylines_guide}{Voting
    rights groups say it is a recipe for disaster.}
  \end{itemize}
\item
  \begin{center}\rule{0.5\linewidth}{\linethickness}\end{center}

  \hypertarget{bidens-vp-search}{%
  \subsection{Biden's V.P. Search}\label{bidens-vp-search}}

  \begin{itemize}
  \tightlist
  \item
    \href{https://www.nytimes3xbfgragh.onion/article/biden-vice-president-2020.html?action=click\&pgtype=Article\&state=default\&region=BELOW_MAIN_CONTENT\&context=storylines_guide}{Here
    are 13 women} who have been under consideration to be Joe Biden's
    running mate, and why each might be chosen --- and might not be.
  \end{itemize}
\item
  \begin{center}\rule{0.5\linewidth}{\linethickness}\end{center}

  \hypertarget{keep-up-with-our-coverage}{%
  \subsection{Keep Up With Our
  Coverage}\label{keep-up-with-our-coverage}}

  \begin{itemize}
  \tightlist
  \item
    Get an
    \href{https://www.nytimes3xbfgragh.onion/newsletters/politics?action=click\&pgtype=Article\&state=default\&region=BELOW_MAIN_CONTENT\&context=storylines_guide}{email}
    recapping the day's news
  \end{itemize}

  \begin{itemize}
  \tightlist
  \item
    Download our mobile app on
    \href{https://apps.apple.com/us/app/nytimes/id284862083?ls=1\&mat_click_id=5c79ae7455014fd1bd66b5610c05b8f2-20191112-16948\&referrer=mat_click_id\%3D5c79ae7455014fd1bd66b5610c05b8f2-20191112-16948\%26link_click_id\%3D722930677036718082}{iOS}
    and
    \href{http://a.localytics.com/android?id=com.nytimes.android\&referrer=utm_source\%3Dother_nyt_mobile_web\%26utm_medium\%3DWeb\%2520page\%26utm_term\%3DGeneral\%2520Mobile\%2520Page\%26utm_campaign\%3DNYT\%2520Mobile\%2520General\%2520Page}{Android}
    and turn on Breaking News and Politics alerts
  \end{itemize}
\end{itemize}

Advertisement

\protect\hyperlink{after-bottom}{Continue reading the main story}

\hypertarget{site-index}{%
\subsection{Site Index}\label{site-index}}

\hypertarget{site-information-navigation}{%
\subsection{Site Information
Navigation}\label{site-information-navigation}}

\begin{itemize}
\tightlist
\item
  \href{https://help.nytimes3xbfgragh.onion/hc/en-us/articles/115014792127-Copyright-notice}{©~2020~The
  New York Times Company}
\end{itemize}

\begin{itemize}
\tightlist
\item
  \href{https://www.nytco.com/}{NYTCo}
\item
  \href{https://help.nytimes3xbfgragh.onion/hc/en-us/articles/115015385887-Contact-Us}{Contact
  Us}
\item
  \href{https://www.nytco.com/careers/}{Work with us}
\item
  \href{https://nytmediakit.com/}{Advertise}
\item
  \href{http://www.tbrandstudio.com/}{T Brand Studio}
\item
  \href{https://www.nytimes3xbfgragh.onion/privacy/cookie-policy\#how-do-i-manage-trackers}{Your
  Ad Choices}
\item
  \href{https://www.nytimes3xbfgragh.onion/privacy}{Privacy}
\item
  \href{https://help.nytimes3xbfgragh.onion/hc/en-us/articles/115014893428-Terms-of-service}{Terms
  of Service}
\item
  \href{https://help.nytimes3xbfgragh.onion/hc/en-us/articles/115014893968-Terms-of-sale}{Terms
  of Sale}
\item
  \href{https://spiderbites.nytimes3xbfgragh.onion}{Site Map}
\item
  \href{https://help.nytimes3xbfgragh.onion/hc/en-us}{Help}
\item
  \href{https://www.nytimes3xbfgragh.onion/subscription?campaignId=37WXW}{Subscriptions}
\end{itemize}
