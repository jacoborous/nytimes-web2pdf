Sections

SEARCH

\protect\hyperlink{site-content}{Skip to
content}\protect\hyperlink{site-index}{Skip to site index}

\href{https://www.nytimes3xbfgragh.onion/section/style}{Style}

\href{https://myaccount.nytimes3xbfgragh.onion/auth/login?response_type=cookie\&client_id=vi}{}

\href{https://www.nytimes3xbfgragh.onion/section/todayspaper}{Today's
Paper}

\href{/section/style}{Style}\textbar{}What Will Happen to All of the
Faces?

\url{https://nyti.ms/39nfS99}

\begin{itemize}
\item
\item
\item
\item
\item
\end{itemize}

\hypertarget{the-coronavirus-outbreak}{%
\subsubsection{\texorpdfstring{\href{https://www.nytimes3xbfgragh.onion/news-event/coronavirus?name=styln-coronavirus-national\&region=TOP_BANNER\&block=storyline_menu_recirc\&action=click\&pgtype=Article\&impression_id=e0e1ede0-f280-11ea-b468-27530939d74e\&variant=undefined}{The
Coronavirus
Outbreak}}{The Coronavirus Outbreak}}\label{the-coronavirus-outbreak}}

\begin{itemize}
\tightlist
\item
  live\href{https://www.nytimes3xbfgragh.onion/2020/09/09/world/covid-19-coronavirus.html?name=styln-coronavirus-national\&region=TOP_BANNER\&block=storyline_menu_recirc\&action=click\&pgtype=Article\&impression_id=e0e1ede1-f280-11ea-b468-27530939d74e\&variant=undefined}{Latest
  Updates}
\item
  \href{https://www.nytimes3xbfgragh.onion/interactive/2020/us/coronavirus-us-cases.html?name=styln-coronavirus-national\&region=TOP_BANNER\&block=storyline_menu_recirc\&action=click\&pgtype=Article\&impression_id=e0e1ede2-f280-11ea-b468-27530939d74e\&variant=undefined}{Maps
  and Cases}
\item
  \href{https://www.nytimes3xbfgragh.onion/interactive/2020/science/coronavirus-vaccine-tracker.html?name=styln-coronavirus-national\&region=TOP_BANNER\&block=storyline_menu_recirc\&action=click\&pgtype=Article\&impression_id=e0e1ede3-f280-11ea-b468-27530939d74e\&variant=undefined}{Vaccine
  Tracker}
\item
  \href{https://www.nytimes3xbfgragh.onion/2020/09/02/your-money/eviction-moratorium-covid.html?name=styln-coronavirus-national\&region=TOP_BANNER\&block=storyline_menu_recirc\&action=click\&pgtype=Article\&impression_id=e0e1ede4-f280-11ea-b468-27530939d74e\&variant=undefined}{Eviction
  Moratorium}
\item
  \href{https://www.nytimes3xbfgragh.onion/interactive/2020/09/02/magazine/food-insecurity-hunger-us.html?name=styln-coronavirus-national\&region=TOP_BANNER\&block=storyline_menu_recirc\&action=click\&pgtype=Article\&impression_id=e0e1ede5-f280-11ea-b468-27530939d74e\&variant=undefined}{American
  Hunger}
\end{itemize}

Advertisement

\protect\hyperlink{after-top}{Continue reading the main story}

Supported by

\protect\hyperlink{after-sponsor}{Continue reading the main story}

skin deep

\hypertarget{what-will-happen-to-all-of-the-faces}{%
\section{What Will Happen to All of the
Faces?}\label{what-will-happen-to-all-of-the-faces}}

If you thought the coronavirus would bring a slowdown in cosmetic
procedures, you'd have been wrong. Here's what to expect in a Covid-19
world.

\includegraphics{https://static01.graylady3jvrrxbe.onion/images/2020/07/23/fashion/22SKIN-PANDEMIC-ART-02/22SKIN-PANDEMIC-ART-02-articleLarge.jpg?quality=75\&auto=webp\&disable=upscale}

\href{https://www.nytimes3xbfgragh.onion/by/crystal-martin}{\includegraphics{https://static01.graylady3jvrrxbe.onion/images/2019/03/01/multimedia/author-crystal-martin/author-crystal-martin-thumbLarge.png}}

By \href{https://www.nytimes3xbfgragh.onion/by/crystal-martin}{Crystal
Martin}

\begin{itemize}
\item
  July 23, 2020
\item
  \begin{itemize}
  \item
  \item
  \item
  \item
  \item
  \end{itemize}
\end{itemize}

In late June, Holly Aubry, who owns a public relations firm in New York,
got the first aesthetic treatments she'd ever had in her life: Botox,
filler and a brow suture lift.

She had never spent much money or time on beauty treatments; she had had
few facials, and shopping at Sephora was rare. But when Dr. Lara Devgan,
a plastic surgeon in New York, reopened her office last month, Ms. Aubry
went in for a consultation and got treated the same day. That office
visit was one of the few outings she had taken since March.

Ms. Aubry, 40, could pinpoint her unease: ``Hearing the sirens from my
apartment. Being constantly terrified of getting sick. Having my kids
home, compromising my ability to run my company. Seeing what the
pandemic has done to the economy. All of it made me stressed, and I
started noticing that I was aging rapidly.''

Throughout the lockdown, wait lists for nonessential, noninvasive
skin-care appointments --- those laser procedures, fillers and Botox
injections --- grew. Dr. Ben Talei, a plastic surgeon in Los Angeles,
reported that he and his colleagues are seeing people who are clamoring
for care now, especially for anything that has healing time.

``They want to do it now while it's not interfering with their work and
social lives,'' Dr. Talei said.

A surge of catch-up appointments was probably predictable. But what will
the aesthetic world look like after an initial surge? Will more of us
have a list of things we'd like to fix after becoming better acquainted
with our features over innumerable virtual meetings?

Or, after a long break from a doctor's office and a reliance on at-home
skin care, will we realize that we don't need medical intervention after
all?

\hypertarget{latest-updates-the-coronavirus-outbreak}{%
\section{\texorpdfstring{\href{https://www.nytimes3xbfgragh.onion/2020/09/09/world/covid-19-coronavirus.html?action=click\&pgtype=Article\&state=default\&region=MAIN_CONTENT_1\&context=storylines_live_updates}{Latest
Updates: The Coronavirus
Outbreak}}{Latest Updates: The Coronavirus Outbreak}}\label{latest-updates-the-coronavirus-outbreak}}

Updated 2020-09-09T09:25:52.418Z

\begin{itemize}
\tightlist
\item
  \href{https://www.nytimes3xbfgragh.onion/2020/09/09/world/covid-19-coronavirus.html?action=click\&pgtype=Article\&state=default\&region=MAIN_CONTENT_1\&context=storylines_live_updates\#link-70cea8bb}{As
  drugmakers pledge to thoroughly vet a vaccine, one company pauses its
  trials for a safety review.}
\item
  \href{https://www.nytimes3xbfgragh.onion/2020/09/09/world/covid-19-coronavirus.html?action=click\&pgtype=Article\&state=default\&region=MAIN_CONTENT_1\&context=storylines_live_updates\#link-4438dd7}{Facing
  a surge in cases, Britain plans to limit most gatherings to six
  people.}
\item
  \href{https://www.nytimes3xbfgragh.onion/2020/09/09/world/covid-19-coronavirus.html?action=click\&pgtype=Article\&state=default\&region=MAIN_CONTENT_1\&context=storylines_live_updates\#link-11cec4c0}{Quarantine
  breakdowns at colleges in the U.S. are leaving some at risk.}
\end{itemize}

\href{https://www.nytimes3xbfgragh.onion/2020/09/09/world/covid-19-coronavirus.html?action=click\&pgtype=Article\&state=default\&region=MAIN_CONTENT_1\&context=storylines_live_updates}{See
more updates}

More live coverage:
\href{https://www.nytimes3xbfgragh.onion/live/2020/09/08/business/stock-market-today-coronavirus?action=click\&pgtype=Article\&state=default\&region=MAIN_CONTENT_1\&context=storylines_live_updates}{Markets}

Then there's the most important consideration of all: How do we do any
of this safely?

\hypertarget{the-pandemic-will-change-how-we-look}{%
\subsection{The pandemic will change how we
look.}\label{the-pandemic-will-change-how-we-look}}

With masks covering most of our faces, we'll likely turn our attention
to our eyes, doctors say. Dr. Devgan expects more requests for under-eye
filler, Botox brow lifts and eyelid surgery.

``I also think that as we cover our faces, we'll reveal more of our
bodies,'' she said. ``That will create an emphasis on the aesthetics of
the torso, buttocks and legs.''

As practices reopen, doctors are indeed seeing an increase in requests
for body treatments. Typically, summer would be a slow time for
surgeries as people plan for beach vacations spent in revealing
clothing. But these days, said Dr. Sachin Shridharani, a plastic surgeon
in New York, ``because the pandemic has limited travel, they're doing
these procedures now.''

According to the Aesthetic Society, a professional organization and
advocacy group for board-certified plastic surgeons that gathers data
from plastic surgery practices nationwide, liposuction and tummy tucks
made up 31 percent of total procedures in June, up from 26 percent in
June 2019. Breast procedures were up 4.3 percent over last June.

``In my own practice, if you take into account the time that we were
closed, breast augmentations and breast lifts are up significantly when
compared to last year,'' said Dr. Herluf Lund, a plastic surgeon in St.
Louis and the president of the Aesthetic Society.

Nonsurgical body treatments, particularly injectables, are in demand as
well.

``We're seeing a lot of interest in what can be done with a syringe
instead of a scalpel,'' Dr. Shridharani said. Injectables, typically
approved by the Food and Drug Administration for use in the face, can be
effective at tackling body concerns.

For instance, Dr. Shridharani treats patients with Kybella, an acid that
is injected to dissolve a double chin, to melt fat in the abdomen, arms
and thighs. He also has been injecting small amounts of diluted
Sculptra, a product that stimulates the body to produce more collagen,
into arms and thighs to help smooth crepey skin. (Dr. Shridharani is
compensated financially for work with the companies that manufacture
Kybella and Sculptra.) What may be on the wane are the excessive fillers
and Botox that we've grown used to seeing on celebrities and
influencers. Steven Pearlman, a plastic surgeon in New York, said that
he expects the baby-smooth, motionless foreheads and overfilled lips and
cheeks --- already diminishing in popularity --- to retreat even more
rapidly now.

``People have seen their faces relax into something more natural during
the lockdowns,'' Dr. Pearlman said. ``And because of all that's going on
in society, too, they are going to realize it's not important to have
that extreme look.''

\includegraphics{https://static01.graylady3jvrrxbe.onion/images/2020/07/22/fashion/22SKIN-PANDEMIC-ART-03/22SKIN-PANDEMIC-ART-03-articleLarge.jpg?quality=75\&auto=webp\&disable=upscale}

\hypertarget{social-media-habits-will-change-too}{%
\subsection{Social media habits will change,
too.}\label{social-media-habits-will-change-too}}

It's tough to say whether or not we'll be sharing (or oversharing)
scenes from our Botox appointments on Instagram. In a climate of
coronavirus concerns, economic suffering and mounting national unrest,
posting one's very expensive cosmetic procedures on social media could,
and arguably should, invite criticism.

At the end of May and the beginning of the Black Lives Matter protests,
aesthetic doctors noticeably paused their streams of striking
before-and-after shots. ``We wanted to be sensitive, of course,'' Dr.
Pearlman said. ``Everyone was considering, `What is the right thing to
post at this moment, and should we be posting at all?'''

\href{https://www.nytimes3xbfgragh.onion/news-event/coronavirus?action=click\&pgtype=Article\&state=default\&region=MAIN_CONTENT_3\&context=storylines_faq}{}

\hypertarget{the-coronavirus-outbreak-}{%
\subsubsection{The Coronavirus Outbreak
›}\label{the-coronavirus-outbreak-}}

\hypertarget{frequently-asked-questions}{%
\paragraph{Frequently Asked
Questions}\label{frequently-asked-questions}}

Updated September 4, 2020

\begin{itemize}
\item ~
  \hypertarget{what-are-the-symptoms-of-coronavirus}{%
  \paragraph{What are the symptoms of
  coronavirus?}\label{what-are-the-symptoms-of-coronavirus}}

  \begin{itemize}
  \tightlist
  \item
    In the beginning, the coronavirus
    \href{https://www.nytimes3xbfgragh.onion/article/coronavirus-facts-history.html?action=click\&pgtype=Article\&state=default\&region=MAIN_CONTENT_3\&context=storylines_faq\#link-6817bab5}{seemed
    like it was primarily a respiratory illness}~--- many patients had
    fever and chills, were weak and tired, and coughed a lot, though
    some people don't show many symptoms at all. Those who seemed
    sickest had pneumonia or acute respiratory distress syndrome and
    received supplemental oxygen. By now, doctors have identified many
    more symptoms and syndromes. In April,
    \href{https://www.nytimes3xbfgragh.onion/2020/04/27/health/coronavirus-symptoms-cdc.html?action=click\&pgtype=Article\&state=default\&region=MAIN_CONTENT_3\&context=storylines_faq}{the
    C.D.C. added to the list of early signs}~sore throat, fever, chills
    and muscle aches. Gastrointestinal upset, such as diarrhea and
    nausea, has also been observed. Another telltale sign of infection
    may be a sudden, profound diminution of one's
    \href{https://www.nytimes3xbfgragh.onion/2020/03/22/health/coronavirus-symptoms-smell-taste.html?action=click\&pgtype=Article\&state=default\&region=MAIN_CONTENT_3\&context=storylines_faq}{sense
    of smell and taste.}~Teenagers and young adults in some cases have
    developed painful red and purple lesions on their fingers and toes
    --- nicknamed ``Covid toe'' --- but few other serious symptoms.
  \end{itemize}
\item ~
  \hypertarget{why-is-it-safer-to-spend-time-together-outside}{%
  \paragraph{Why is it safer to spend time together
  outside?}\label{why-is-it-safer-to-spend-time-together-outside}}

  \begin{itemize}
  \tightlist
  \item
    \href{https://www.nytimes3xbfgragh.onion/2020/05/15/us/coronavirus-what-to-do-outside.html?action=click\&pgtype=Article\&state=default\&region=MAIN_CONTENT_3\&context=storylines_faq}{Outdoor
    gatherings}~lower risk because wind disperses viral droplets, and
    sunlight can kill some of the virus. Open spaces prevent the virus
    from building up in concentrated amounts and being inhaled, which
    can happen when infected people exhale in a confined space for long
    stretches of time, said Dr. Julian W. Tang, a virologist at the
    University of Leicester.
  \end{itemize}
\item ~
  \hypertarget{why-does-standing-six-feet-away-from-others-help}{%
  \paragraph{Why does standing six feet away from others
  help?}\label{why-does-standing-six-feet-away-from-others-help}}

  \begin{itemize}
  \tightlist
  \item
    The coronavirus spreads primarily through droplets from your mouth
    and nose, especially when you cough or sneeze. The C.D.C., one of
    the organizations using that measure,
    \href{https://www.nytimes3xbfgragh.onion/2020/04/14/health/coronavirus-six-feet.html?action=click\&pgtype=Article\&state=default\&region=MAIN_CONTENT_3\&context=storylines_faq}{bases
    its recommendation of six feet}~on the idea that most large droplets
    that people expel when they cough or sneeze will fall to the ground
    within six feet. But six feet has never been a magic number that
    guarantees complete protection. Sneezes, for instance, can launch
    droplets a lot farther than six feet,
    \href{https://jamanetwork.com/journals/jama/fullarticle/2763852}{according
    to a recent study}. It's a rule of thumb: You should be safest
    standing six feet apart outside, especially when it's windy. But
    keep a mask on at all times, even when you think you're far enough
    apart.
  \end{itemize}
\item ~
  \hypertarget{i-have-antibodies-am-i-now-immune}{%
  \paragraph{I have antibodies. Am I now
  immune?}\label{i-have-antibodies-am-i-now-immune}}

  \begin{itemize}
  \tightlist
  \item
    As of right
    now,\href{https://www.nytimes3xbfgragh.onion/2020/07/22/health/covid-antibodies-herd-immunity.html?action=click\&pgtype=Article\&state=default\&region=MAIN_CONTENT_3\&context=storylines_faq}{~that
    seems likely, for at least several months.}~There have been
    frightening accounts of people suffering what seems to be a second
    bout of Covid-19. But experts say these patients may have a
    drawn-out course of infection, with the virus taking a slow toll
    weeks to months after initial exposure.~People infected with the
    coronavirus typically
    \href{https://www.nature.com/articles/s41586-020-2456-9}{produce}~immune
    molecules called antibodies, which are
    \href{https://www.nytimes3xbfgragh.onion/2020/05/07/health/coronavirus-antibody-prevalence.html?action=click\&pgtype=Article\&state=default\&region=MAIN_CONTENT_3\&context=storylines_faq}{protective
    proteins made in response to an
    infection}\href{https://www.nytimes3xbfgragh.onion/2020/05/07/health/coronavirus-antibody-prevalence.html?action=click\&pgtype=Article\&state=default\&region=MAIN_CONTENT_3\&context=storylines_faq}{.
    These antibodies may}~last in the body
    \href{https://www.nature.com/articles/s41591-020-0965-6}{only two to
    three months}, which may seem worrisome, but that's~perfectly normal
    after an acute infection subsides, said Dr. Michael Mina, an
    immunologist at Harvard University. It may be possible to get the
    coronavirus again, but it's highly unlikely that it would be
    possible in a short window of time from initial infection or make
    people sicker the second time.
  \end{itemize}
\item ~
  \hypertarget{what-are-my-rights-if-i-am-worried-about-going-back-to-work}{%
  \paragraph{What are my rights if I am worried about going back to
  work?}\label{what-are-my-rights-if-i-am-worried-about-going-back-to-work}}

  \begin{itemize}
  \tightlist
  \item
    Employers have to provide
    \href{https://www.osha.gov/SLTC/covid-19/standards.html}{a safe
    workplace}~with policies that protect everyone equally.
    \href{https://www.nytimes3xbfgragh.onion/article/coronavirus-money-unemployment.html?action=click\&pgtype=Article\&state=default\&region=MAIN_CONTENT_3\&context=storylines_faq}{And
    if one of your co-workers tests positive for the coronavirus, the
    C.D.C.}~has said that
    \href{https://www.cdc.gov/coronavirus/2019-ncov/community/guidance-business-response.html}{employers
    should tell their employees}~-\/- without giving you the sick
    employee's name -\/- that they may have been exposed to the virus.
  \end{itemize}
\end{itemize}

On the other hand, social media has been an essential mode of connection
during monthslong lockdowns. People have grown comfortable sharing
life's details with their followers. Perhaps we will land in a middle
ground where instead of posting about procedures on their own social
media feeds, more patients will allow their doctors to share their
procedure photos.

``I wouldn't have given my permission to post before going through this
experience,'' Ms. Aubry said. ``But I wanted other women who were
feeling as I did to know that there are options, and they shouldn't have
any shame in pursuing them.''

Image

Credit...Monica Ahanonu

\hypertarget{but-is-it-safe-to-have-aesthetic-procedures-right-now}{%
\subsection{But is it safe to have aesthetic procedures right
now?}\label{but-is-it-safe-to-have-aesthetic-procedures-right-now}}

In medicine, everything is about risk-benefit,'' said Dr. Adolf
Karchmer, an infectious disease expert and professor of medicine at
Harvard Medical School. ``Some people feel they need these procedures
for psychological or even professional benefit.'' The risk of getting
the coronavirus when out in the world will never be zero, but offices
should enact safety protocols to reduce the risk to a negligible level,
he said.

Dr. Karchmer served on a task force that developed Project AesCert,
safety guidelines for reopening. For patients, there are a few main
lessons.

First, safety begins before you arrive at the office. Practices will be
screening patients based on presence of symptoms, potential exposures
and pre-existing conditions. Some practices may turn away individuals at
high risk for Covid-19.

``The first thing the patient should be asking is, `What is this
practice saying about their safety protocols,''' Dr. Lund said. ``When
you call, can they describe without hesitation what they're doing? Is it
on their website?''

When you arrive, you'll have your temperature taken. Paperwork will have
been handled online before your appointment. Everyone should be in
masks. The doctor will have on personal protective equipment, likely an
N95 mask, face shield, gown and gloves.

Doctors are in agreement that many consultations and follow-ups will by
default be done virtually to keep traffic in office to a minimum.

``Because of the risk of coronavirus exposure, the stakes are higher
than ever for aesthetics,'' Dr. Devgan said. Medical resources are still
not optimal, she said, making it a bad time to have a complication from
a procedure.

Advertisement

\protect\hyperlink{after-bottom}{Continue reading the main story}

\hypertarget{site-index}{%
\subsection{Site Index}\label{site-index}}

\hypertarget{site-information-navigation}{%
\subsection{Site Information
Navigation}\label{site-information-navigation}}

\begin{itemize}
\tightlist
\item
  \href{https://help.nytimes3xbfgragh.onion/hc/en-us/articles/115014792127-Copyright-notice}{©~2020~The
  New York Times Company}
\end{itemize}

\begin{itemize}
\tightlist
\item
  \href{https://www.nytco.com/}{NYTCo}
\item
  \href{https://help.nytimes3xbfgragh.onion/hc/en-us/articles/115015385887-Contact-Us}{Contact
  Us}
\item
  \href{https://www.nytco.com/careers/}{Work with us}
\item
  \href{https://nytmediakit.com/}{Advertise}
\item
  \href{http://www.tbrandstudio.com/}{T Brand Studio}
\item
  \href{https://www.nytimes3xbfgragh.onion/privacy/cookie-policy\#how-do-i-manage-trackers}{Your
  Ad Choices}
\item
  \href{https://www.nytimes3xbfgragh.onion/privacy}{Privacy}
\item
  \href{https://help.nytimes3xbfgragh.onion/hc/en-us/articles/115014893428-Terms-of-service}{Terms
  of Service}
\item
  \href{https://help.nytimes3xbfgragh.onion/hc/en-us/articles/115014893968-Terms-of-sale}{Terms
  of Sale}
\item
  \href{https://spiderbites.nytimes3xbfgragh.onion}{Site Map}
\item
  \href{https://help.nytimes3xbfgragh.onion/hc/en-us}{Help}
\item
  \href{https://www.nytimes3xbfgragh.onion/subscription?campaignId=37WXW}{Subscriptions}
\end{itemize}
