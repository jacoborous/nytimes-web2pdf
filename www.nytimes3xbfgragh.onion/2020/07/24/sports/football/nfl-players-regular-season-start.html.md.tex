Sections

SEARCH

\protect\hyperlink{site-content}{Skip to
content}\protect\hyperlink{site-index}{Skip to site index}

\href{https://www.nytimes3xbfgragh.onion/section/sports/football}{Pro
Football}

\href{https://myaccount.nytimes3xbfgragh.onion/auth/login?response_type=cookie\&client_id=vi}{}

\href{https://www.nytimes3xbfgragh.onion/section/todayspaper}{Today's
Paper}

\href{/section/sports/football}{Pro Football}\textbar{}Wary of Other
Leagues' Battles, N.F.L. and Players Agree on Terms to Return

\url{https://nyti.ms/3hDwD2z}

\begin{itemize}
\item
\item
\item
\item
\item
\end{itemize}

Advertisement

\protect\hyperlink{after-top}{Continue reading the main story}

Supported by

\protect\hyperlink{after-sponsor}{Continue reading the main story}

\hypertarget{wary-of-other-leagues-battles-nfl-and-players-agree-on-terms-to-return}{%
\section{Wary of Other Leagues' Battles, N.F.L. and Players Agree on
Terms to
Return}\label{wary-of-other-leagues-battles-nfl-and-players-agree-on-terms-to-return}}

As training camps get underway, the two sides finalized plans to reduce
the salary cap and scrap preseason games, an agreement that paves the
way for the regular season to begin as scheduled on Sept. 10.

\includegraphics{https://static01.graylady3jvrrxbe.onion/images/2020/07/24/sports/24nfl-hfo-cowboys/merlin_174709488_16f8e4bb-9c8c-4804-9083-46b86e8960c3-articleLarge.jpg?quality=75\&auto=webp\&disable=upscale}

\href{https://www.nytimes3xbfgragh.onion/by/ken-belson}{\includegraphics{https://static01.graylady3jvrrxbe.onion/images/2018/02/16/multimedia/author-ken-belson/author-ken-belson-thumbLarge.jpg}}

By \href{https://www.nytimes3xbfgragh.onion/by/ken-belson}{Ken Belson}

\begin{itemize}
\item
  July 24, 2020
\item
  \begin{itemize}
  \item
  \item
  \item
  \item
  \item
  \end{itemize}
\end{itemize}

The N.F.L. and the N.F.L. Players Association have finalized the last
key financial issues related to this season, paving the way for an
on-time start to the regular season on Sept. 10.

In the deal reached late Friday after a vote by the union's 32 team
representatives, the salary cap --- or the maximum amount teams can
spend on their rosters --- will remain at just under \$200 million per
team this season. But the cap will have a minimum of \$175 million next
season. Any shortfalls in revenue next year will be made up by reducing
the salary cap through the 2023 season.

The owners also agreed to a player proposal to scrap all preseason games
to reduce the risk of infection.

The sides had already agreed on several measures to reduce the risk of
infection from the coronavirus as teams return to camps, meetings and
practices, including outlining who can be inside team facilities and
daily player testing for the virus.

But the owners and the players' union had remained deadlocked on
thornier questions, even as players began reporting to team facilities
this week, leading some star players to
\href{https://www.nytimes3xbfgragh.onion/2020/07/20/sports/football/nfl-training-camp-players.html}{start
a public relations offensive on social media} pushing for their
concerns. Those included how much players will be paid if the season is
shortened or canceled, and how to reduce the players' share of the loss
of revenue if teams do not allow fans at games this season.

All players have to report to training camp by July 28. But with each
team required to test players and staff members at least twice before
allowing them to enter their facilities to take physicals, it is more
likely that practices will begin in early August.

Though all sides hope to open the regular season on Sept. 10, it remains
unclear whether teams will allow any fans to attend. Earlier this week,
the league said that fans would be
\href{https://twitter.com/NFLprguy/status/1285995479596228615}{required
to wear masks} at games and both the
\href{https://www.nytimes3xbfgragh.onion/2020/07/20/sports/football/jets-giants-rutgers-fans-metlife-stadium.html}{Giants
and Jets became the first N.F.L. franchises} to say that they would play
regular-season games at MetLife Stadium without spectators, heeding New
Jersey's prohibitions on mass gatherings. Some teams like the Miami
Dolphins and the New England Patriots have announced plans to limit
their stadium's capacity to allow for fans.

``The season will undoubtedly present new and additional challenges, but
we are committed to playing a safe and complete 2020 season, culminating
with the Super Bowl,'' Commissioner Roger Goodell said in a statement.

The league earns about one-quarter of its \$15 billion in annual revenue
from local sources, including ticket sales, parking, food and beverage
sales, luxury boxes and sponsorships. The loss of income from these
fan-less games could cost the owners and players several billion
dollars, though the precise amount will not be determined until the end
of the season.

The players and the league agreed on the wide-ranging parameters, having
had the benefit of observing other professional leagues negotiate
returns with their unions.

\hypertarget{sports-and-the-virus}{%
\subsubsection{Sports and the Virus}\label{sports-and-the-virus}}

\paragraph{}

Updated Sept. 8, 2020

Here's what's happening as the world of sports slowly comes back to
life:

\begin{itemize}
\item
  \begin{itemize}
  \tightlist
  \item
    As the United States Open enters its second week without fans, an
    Italian restaurateur stands outside the gates and
    \href{https://www.nytimes3xbfgragh.onion/2020/09/06/sports/tennis/US-Open-Matteo-Berrettini-fan.html?action=click\&pgtype=Article\&state=default\&region=MAIN_CONTENT_2\&context=storylines_keepup}{bellows
    his support}~for his favorite player.
  \item
    The coronavirus pandemic has had an
    \href{https://www.nytimes3xbfgragh.onion/2020/09/03/sports/ncaafootball/high-school-football-coronavirus-pandemic.html?action=click\&pgtype=Article\&state=default\&region=MAIN_CONTENT_2\&context=storylines_keepup}{uneven
    impact on high school football}~across the United States.
  \item
    The
    \href{https://www.nytimes3xbfgragh.onion/2020/09/02/sports/ncaafootball/coronavirus-cal-athletics-season.html?action=click\&pgtype=Article\&state=default\&region=MAIN_CONTENT_2\&context=storylines_keepup}{most
    complicated puzzle in sports is the return of college
    athletics}~during a pandemic. The University of California, Berkeley
    is allowing The Times an inside look at their journey's ups and
    downs.
  \end{itemize}
\end{itemize}

Although the N.F.L. and union locked horns the talks were considered to
be
\href{https://www.nytimes3xbfgragh.onion/2020/06/15/sports/baseball/rob-manfred-mlb-season.html}{less
acrimonious than those between Major League Baseball} and its players'
union.

While the owners are taking some steps, like selling additional
sponsorships, to reduce their revenue losses, the players and owners had
to
decide\href{https://www.nytimes3xbfgragh.onion/2020/07/02/sports/football/nfl-salary-cap-no-fans.html}{how
to offset the players' share of the losses}. The owners wanted to put 35
percent of player salaries this season in escrow, determine what the
losses were at the end of the year and return any difference.

The players preferred to spread out the losses over as many as 10 years
by reducing the salary cap. They ultimately agreed to recoup the losses
over three seasons.

In the new
\href{https://www.nytimes3xbfgragh.onion/2020/03/15/sports/football/nfl-cba-approved.html}{collective
bargaining agreement signed in March}, the players are owed 48 percent
of league revenue, which means they are on the hook for that percentage
of any losses. That labor deal does not include a clause that would
allow the owners to forgo paying the players following certain
extraordinary events, like natural disasters or terror attacks.

Still, the start of training camp was never in doubt because the owners
have the right to open camps at their discretion, and players who do not
report could face penalties. But the union had pushed for a host of
steps and the negotiations over player health and safety largely
concluded just before the first rookies had to report to training camp
on July 20.

The league also agreed to test the players every day for the first two
weeks they are in camp and if the rate of positive tests is below 5
percent, tests will be provided every other day.

According to the players' union, 95 players and staff members tested
positive for the coronavirus during the off-season. The league expects
potentially hundreds of players to test positive when the nearly 2,900
rookies, veterans and free agents travel from across the country to get
tested by their teams before they begin training camp.

Advertisement

\protect\hyperlink{after-bottom}{Continue reading the main story}

\hypertarget{site-index}{%
\subsection{Site Index}\label{site-index}}

\hypertarget{site-information-navigation}{%
\subsection{Site Information
Navigation}\label{site-information-navigation}}

\begin{itemize}
\tightlist
\item
  \href{https://help.nytimes3xbfgragh.onion/hc/en-us/articles/115014792127-Copyright-notice}{©~2020~The
  New York Times Company}
\end{itemize}

\begin{itemize}
\tightlist
\item
  \href{https://www.nytco.com/}{NYTCo}
\item
  \href{https://help.nytimes3xbfgragh.onion/hc/en-us/articles/115015385887-Contact-Us}{Contact
  Us}
\item
  \href{https://www.nytco.com/careers/}{Work with us}
\item
  \href{https://nytmediakit.com/}{Advertise}
\item
  \href{http://www.tbrandstudio.com/}{T Brand Studio}
\item
  \href{https://www.nytimes3xbfgragh.onion/privacy/cookie-policy\#how-do-i-manage-trackers}{Your
  Ad Choices}
\item
  \href{https://www.nytimes3xbfgragh.onion/privacy}{Privacy}
\item
  \href{https://help.nytimes3xbfgragh.onion/hc/en-us/articles/115014893428-Terms-of-service}{Terms
  of Service}
\item
  \href{https://help.nytimes3xbfgragh.onion/hc/en-us/articles/115014893968-Terms-of-sale}{Terms
  of Sale}
\item
  \href{https://spiderbites.nytimes3xbfgragh.onion}{Site Map}
\item
  \href{https://help.nytimes3xbfgragh.onion/hc/en-us}{Help}
\item
  \href{https://www.nytimes3xbfgragh.onion/subscription?campaignId=37WXW}{Subscriptions}
\end{itemize}
