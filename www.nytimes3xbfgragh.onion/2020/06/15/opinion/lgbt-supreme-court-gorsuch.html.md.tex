Sections

SEARCH

\protect\hyperlink{site-content}{Skip to
content}\protect\hyperlink{site-index}{Skip to site index}

\href{https://myaccount.nytimes3xbfgragh.onion/auth/login?response_type=cookie\&client_id=vi}{}

\href{https://www.nytimes3xbfgragh.onion/section/todayspaper}{Today's
Paper}

\href{/section/opinion}{Opinion}\textbar{}Surprise! Justice on L.G.B.T.
Rights From a Trump Judge

\url{https://nyti.ms/37xoMQt}

\begin{itemize}
\item
\item
\item
\item
\item
\item
\end{itemize}

Advertisement

\protect\hyperlink{after-top}{Continue reading the main story}

\href{/section/opinion}{Opinion}

Supported by

\protect\hyperlink{after-sponsor}{Continue reading the main story}

\hypertarget{surprise-justice-on-lgbt-rights-from-a-trump-judge}{%
\section{Surprise! Justice on L.G.B.T. Rights From a Trump
Judge}\label{surprise-justice-on-lgbt-rights-from-a-trump-judge}}

So much for ``But Gorsuch.''

\href{https://www.nytimes3xbfgragh.onion/by/michelle-goldberg}{\includegraphics{https://static01.graylady3jvrrxbe.onion/images/2018/04/02/opinion/michelle-goldberg/michelle-goldberg-thumbLarge.png}}

By
\href{https://www.nytimes3xbfgragh.onion/by/michelle-goldberg}{Michelle
Goldberg}

Opinion Columnist

\begin{itemize}
\item
  June 15, 2020
\item
  \begin{itemize}
  \item
  \item
  \item
  \item
  \item
  \item
  \end{itemize}
\end{itemize}

\includegraphics{https://static01.graylady3jvrrxbe.onion/images/2020/06/15/opinion/15goldberg1/merlin_173564142_0da3cb0c-5631-4937-b100-49154f16661f-articleLarge.jpg?quality=75\&auto=webp\&disable=upscale}

\hypertarget{listen-to-this-op-ed}{%
\subsubsection{Listen to This Op-Ed}\label{listen-to-this-op-ed}}

Audio Recording by Audm

\emph{To hear more audio stories from publishers like The New York
Times,
download}\href{https://www.audm.com/?utm_source=nytmag\&utm_medium=embed\&utm_campaign=left_behind_draper}{**}\href{https://www.audm.com/?utm_source=nytopinion\&utm_medium=embed\&utm_campaign=surprise_lgbt_trump}{\emph{Audm
for iPhone or Android}}\emph{.}

The new season of my
\href{https://www.nytimes3xbfgragh.onion/2019/05/03/opinion/the-good-fight-trump.html}{favorite
television show}, ``The Good Fight,'' begins with the heroine, the
feminist lawyer Diane Lockhart, awakening in what seems at first like a
giddy alternative reality in which Hillary Clinton won the 2016
election. She remembers the horrors of the last three and a half years,
but no one else seems to. A crushing weight lifts as she convinces
herself it was all an awful dream.

Then she is sent to a meeting with her firm's new client, Harvey
Weinstein. There's been no \#MeToo movement. Instead, corporate ``lean
in'' feminism is at its apogee. Diane realizes there have been gains
made since Donald Trump took office that are unbearable to give up.

Obviously, a world in which Clinton beat Trump would be better in a
million ways. Still, right now we have two big examples of how Trump's
perverse presidency has inadvertently led to progress.

The sudden, rapid embrace of the Black Lives Matter movement by white
people is a function of the undeniable brutality of George Floyd's
videotaped killing. But public opinion has also moved left on racial
issues
\href{https://www.vox.com/2019/3/22/18259865/great-awokening-white-liberals-race-polling-trump-2020}{in
reaction} to an unpopular president who behaves like a cross between
Bull Connor and Andrew Dice Clay.

And the thrilling
\href{https://www.nytimes3xbfgragh.onion/2020/06/15/us/politics/gorsuch-supreme-court-gay-transgender-rights.html}{6-3
decision} the Supreme Court just issued upholding L.G.B.T. equality
wouldn't be as devastating to the religious right if it had happened
under a President Clinton.

Before Monday, you could legally be fired for being gay, bisexual or
transgender in 26 states. Now the court has ruled that gay and
transgender people are protected by Title VII of the 1964 Civil Rights
Act, which prohibits employment discrimination on the basis of sex. The
decision has extra cultural force because it was written by Justice Neil
Gorsuch, a Trump appointee, and joined by the conservative chief justice
John Roberts.

``The whole point of the Federalist Society judicial project, the whole
point of electing Trump to implement it, was to deliver Supreme Court
victories to social conservatives,'' tweeted the
\href{https://twitter.com/varadmehta/status/1272532015292862464?s=20}{conservative
writer Varad Mehta}. ``If they can't deliver anything that basic,
there's no point for either. The damage is incalculable.''

The phrase ``But Gorsuch'' is shorthand for how conservatives justify
all the moral compromises they've made in supporting Trump; controlling
the Supreme Court makes it all worth it. So there's a special sweetness
in Gorsuch spearheading the most important L.G.B.T. rights decision
since the 2015 ruling in Obergefell v. Hodges, which established a
constitutional right to same-sex marriage.

This isn't simply Schadenfreude. The fact that this momentous ruling was
written by a right-wing judge sends a message that progress on L.G.B.T.
rights will be very hard to reverse.

Had Clinton, like Trump, been able to put two justices on the court, the
ultimate decision in this case would likely have been much the same,
perhaps with a different legal rationale. But social conservatives would
have been animated by outraged opposition, sure that winning the next
election was key to re-establishing power. Now they're demoralized.

The Trump administration will continue to try to roll back gay and
transgender rights. Just last Friday, it finalized a regulation saying
that the Affordable Care Act's ban on sex discrimination in medical care
\href{https://www.nytimes3xbfgragh.onion/2020/06/12/us/politics/trump-transgender-rights.html}{doesn't
apply to trans people}, using an argument similar to the one the Supreme
Court rejected on Monday. Trump judges on lower courts can be expected
to rule in favor of religious conservatives.

But these will be rear-guard actions. ``The Roe v. Wade of religious
liberty is here, and it was delivered by golden boy Neil Gorsuch,''
\href{https://twitter.com/josh_hammer/status/1272532875204853761?s=20}{lamented
conservative lawyer} Josh Hammer.

Legal experts who watched the arguments unfold weren't entirely shocked
that Gorsuch ruled as he did. The justice is well known as a textualist,
someone who holds that the meaning of a law turns on the text alone, not
the intentions of its drafters.

``What I saw in the argument was Gorsuch really struggling with the fact
that the textual argument seemed really powerful to him,'' Samuel
Bagenstos, a University of Michigan law professor, told me. ``There's no
way to think about sexual orientation discrimination without sex being
part of it.''

Bagenstos was more surprised that Roberts --- who, after all, wrote a
dissent in Obergefell --- joined the majority. Roberts may have simply
been persuaded by the merits of the case, but Bagenstos suspects he was
responsive to the political climate as well.

``This is going to be a very popular decision,'' Bagenstos said. ``It is
something that the American people will largely agree with. And you
never go wrong predicting that the Supreme Court is going to follow the
election returns.''

None of this means that progressives can rest easy about this court.
We're awaiting important decisions on DACA, which could put hundreds of
thousands of Dreamers in danger of deportation, and on
\href{https://www.nytimes3xbfgragh.onion/2020/06/29/us/supreme-court-abortion-louisiana.html}{June
Medical Services v. Russo}, which could end up eliminating abortion
access in many states.

But on Monday, Gorsuch delivered a blow to the religious right that a
court full of Clinton appointees could never have managed. Even the
darkest timeline has its consolations.

\emph{The Times is committed to publishing}
\href{https://www.nytimes3xbfgragh.onion/2019/01/31/opinion/letters/letters-to-editor-new-york-times-women.html}{\emph{a
diversity of letters}} \emph{to the editor. We'd like to hear what you
think about this or any of our articles. Here are some}
\href{https://help.nytimes3xbfgragh.onion/hc/en-us/articles/115014925288-How-to-submit-a-letter-to-the-editor}{\emph{tips}}\emph{.
And here's our email:}
\href{mailto:letters@NYTimes.com}{\emph{letters@NYTimes.com}}\emph{.}

\emph{Follow The New York Times Opinion section on}
\href{https://www.facebookcorewwwi.onion/nytopinion}{\emph{Facebook}}\emph{,}
\href{http://twitter.com/NYTOpinion}{\emph{Twitter (@NYTopinion)}}
\emph{and}
\href{https://www.instagram.com/nytopinion/}{\emph{Instagram}}\emph{.}

Advertisement

\protect\hyperlink{after-bottom}{Continue reading the main story}

\hypertarget{site-index}{%
\subsection{Site Index}\label{site-index}}

\hypertarget{site-information-navigation}{%
\subsection{Site Information
Navigation}\label{site-information-navigation}}

\begin{itemize}
\tightlist
\item
  \href{https://help.nytimes3xbfgragh.onion/hc/en-us/articles/115014792127-Copyright-notice}{©~2020~The
  New York Times Company}
\end{itemize}

\begin{itemize}
\tightlist
\item
  \href{https://www.nytco.com/}{NYTCo}
\item
  \href{https://help.nytimes3xbfgragh.onion/hc/en-us/articles/115015385887-Contact-Us}{Contact
  Us}
\item
  \href{https://www.nytco.com/careers/}{Work with us}
\item
  \href{https://nytmediakit.com/}{Advertise}
\item
  \href{http://www.tbrandstudio.com/}{T Brand Studio}
\item
  \href{https://www.nytimes3xbfgragh.onion/privacy/cookie-policy\#how-do-i-manage-trackers}{Your
  Ad Choices}
\item
  \href{https://www.nytimes3xbfgragh.onion/privacy}{Privacy}
\item
  \href{https://help.nytimes3xbfgragh.onion/hc/en-us/articles/115014893428-Terms-of-service}{Terms
  of Service}
\item
  \href{https://help.nytimes3xbfgragh.onion/hc/en-us/articles/115014893968-Terms-of-sale}{Terms
  of Sale}
\item
  \href{https://spiderbites.nytimes3xbfgragh.onion}{Site Map}
\item
  \href{https://help.nytimes3xbfgragh.onion/hc/en-us}{Help}
\item
  \href{https://www.nytimes3xbfgragh.onion/subscription?campaignId=37WXW}{Subscriptions}
\end{itemize}
