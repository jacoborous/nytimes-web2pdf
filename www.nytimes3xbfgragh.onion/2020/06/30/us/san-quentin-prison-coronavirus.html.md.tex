Sections

SEARCH

\protect\hyperlink{site-content}{Skip to
content}\protect\hyperlink{site-index}{Skip to site index}

\href{https://www.nytimes3xbfgragh.onion/section/us}{U.S.}

\href{https://myaccount.nytimes3xbfgragh.onion/auth/login?response_type=cookie\&client_id=vi}{}

\href{https://www.nytimes3xbfgragh.onion/section/todayspaper}{Today's
Paper}

\href{/section/us}{U.S.}\textbar{}San Quentin Prison Was Free of the
Virus. One Decision Fueled an Outbreak.

\url{https://nyti.ms/3dKVI9B}

\begin{itemize}
\item
\item
\item
\item
\item
\item
\end{itemize}

\hypertarget{the-coronavirus-outbreak}{%
\subsubsection{\texorpdfstring{\href{https://www.nytimes3xbfgragh.onion/news-event/coronavirus?name=styln-coronavirus-national\&region=TOP_BANNER\&block=storyline_menu_recirc\&action=click\&pgtype=Article\&impression_id=4b17b140-f4c5-11ea-ad1e-89255628ebac\&variant=undefined}{The
Coronavirus
Outbreak}}{The Coronavirus Outbreak}}\label{the-coronavirus-outbreak}}

\begin{itemize}
\tightlist
\item
  live\href{https://www.nytimes3xbfgragh.onion/2020/09/11/world/covid-19-coronavirus.html?name=styln-coronavirus-national\&region=TOP_BANNER\&block=storyline_menu_recirc\&action=click\&pgtype=Article\&impression_id=4b17b141-f4c5-11ea-ad1e-89255628ebac\&variant=undefined}{Latest
  Updates}
\item
  \href{https://www.nytimes3xbfgragh.onion/interactive/2020/us/coronavirus-us-cases.html?name=styln-coronavirus-national\&region=TOP_BANNER\&block=storyline_menu_recirc\&action=click\&pgtype=Article\&impression_id=4b17b142-f4c5-11ea-ad1e-89255628ebac\&variant=undefined}{Maps
  and Cases}
\item
  \href{https://www.nytimes3xbfgragh.onion/interactive/2020/science/coronavirus-vaccine-tracker.html?name=styln-coronavirus-national\&region=TOP_BANNER\&block=storyline_menu_recirc\&action=click\&pgtype=Article\&impression_id=4b17b143-f4c5-11ea-ad1e-89255628ebac\&variant=undefined}{Vaccine
  Tracker}
\item
  \href{https://www.nytimes3xbfgragh.onion/2020/09/10/us/politics/fda-coronavirus-vaccine.html?name=styln-coronavirus-national\&region=TOP_BANNER\&block=storyline_menu_recirc\&action=click\&pgtype=Article\&impression_id=4b17b144-f4c5-11ea-ad1e-89255628ebac\&variant=undefined}{F.D.A.
  Regulators' Self-Defense}
\item
  \href{https://www.nytimes3xbfgragh.onion/2020/09/09/upshot/coronavirus-surprise-test-fees.html?name=styln-coronavirus-national\&region=TOP_BANNER\&block=storyline_menu_recirc\&action=click\&pgtype=Article\&impression_id=4b17b145-f4c5-11ea-ad1e-89255628ebac\&variant=undefined}{Surprise
  Test Fees}
\end{itemize}

Advertisement

\protect\hyperlink{after-top}{Continue reading the main story}

Supported by

\protect\hyperlink{after-sponsor}{Continue reading the main story}

\hypertarget{san-quentin-prison-was-free-of-the-virus-one-decision-fueled-an-outbreak}{%
\section{San Quentin Prison Was Free of the Virus. One Decision Fueled
an
Outbreak.}\label{san-quentin-prison-was-free-of-the-virus-one-decision-fueled-an-outbreak}}

The virus arrived in San Quentin after busloads of prisoners were
transferred from another facility where infections were rising. What
happened is a warning for the nation's prisons, experts say.

\includegraphics{https://static01.graylady3jvrrxbe.onion/images/2020/06/29/us/00virus-sanquentin-01/merlin_174049632_42633f0e-8129-4df5-b7f5-7795ad410ea7-articleLarge.jpg?quality=75\&auto=webp\&disable=upscale}

By \href{https://www.nytimes3xbfgragh.onion/by/timothy-williams}{Timothy
Williams} and Rebecca Griesbach

\begin{itemize}
\item
  June 30, 2020
\item
  \begin{itemize}
  \item
  \item
  \item
  \item
  \item
  \item
  \end{itemize}
\end{itemize}

The coughing and complaints of sickness began as a procession of
busloads of prisoners made its way late last month from a Southern
California prison to San Quentin, California's oldest and most widely
known prison, perched on a bluff overlooking San Francisco Bay, not far
from the Golden Gate Bridge.

The inmates were being moved to San Quentin as part of a plan to halt
the spread of
\href{https://www.nytimes3xbfgragh.onion/news-event/coronavirus}{the
coronavirus} by reducing the number of inmates at the California
Institution for Men in Chino, where nine inmates had died and nearly 700
had been infected.

At the time, there were no inmates known to have had the virus at San
Quentin.

Within days, some of the 121 prisoners from the buses introduced the
virus at San Quentin, public health officials say. More than 1,100 of
the 3,700
\href{https://www.nytimes3xbfgragh.onion/2020/07/30/nyregion/New-jersey-inmate-release-Covid.html}{prisoners
have since been infected} at San Quentin, the foreboding structure
surrounded by barbed wire fences and dotted with guard towers that was
once famously home to inmates including Charles Manson; Sirhan Sirhan,
who assassinated Robert F. Kennedy; and
\href{https://www.nytimes3xbfgragh.onion/1971/08/22/archives/-soledad-brother-and-5-are-killed-in-prison-battle-george-jackson.html}{George
Jackson}, an inmate who wrote ``Soledad Brother,'' a series of letters
from prison.

The transfer of inmates --- an effort intended to slow the virus, which
instead apparently created a new outbreak --- has been denounced by
health officials, a federal judge and a growing number of state
lawmakers as a public health failure. How San Quentin went from being a
prison that had held off the virus for months to a place inundated with
sick inmates represents a cautionary tale for the nation's prison system
amid the pandemic.

``What happened --- what's happening --- it can really happen anywhere,
particularly in an overcrowded prison, which unfortunately is the
norm,'' said Dr. David Sears, a physician and professor of medicine at
the University of California, San Francisco, who toured San Quentin on
June 13 and
\href{https://www.documentcloud.org/documents/6956448-06-15-20-San-Quentin-Urgent-Memo.html}{warned
state officials}about the emerging crisis. ``San Quentin's not the first
prison to have a large outbreak, and unfortunately it won't be the
last.''

Days into the outbreak, the prison has grown increasingly chaotic,
inmates and others say. More than 160 of San Quentin's 725 death row
inmates have been infected, prison officials said, including one who has
died. A number of older prisoners have hung handwritten signs outside
their cells that read ``Immune Compromised'' so that guards will wear
masks around them. Other inmates refuse to leave their cells out of fear
of catching the virus, according to an inmate, and in recent days,
guards have been heard screaming over their radios, ``Man down!'' after
sickened inmates were unable to stand up.

The conversation has been dominated by talk of death.

``I don't want to see them die,'' Rahsaan Thomas, a 49-year-old inmate
said of some of the older prisoners in a telephone interview. ``I don't
know if I'm tough enough to survive Covid.''

The California Department of Corrections and Rehabilitation
\href{https://www.cdcr.ca.gov/covid19/}{said in a statement} that it was
very concerned about the surge in infections in San Quentin, adding that
prison workers had increased testing among inmates and had limited the
number of transfers between prisons.

At a hearing before the State Senate on Wednesday, prison officials took
some responsibility for the outbreak. ``We care about the inmates, we
care about the staff,'' said Ralph Diaz, secretary of the state's
Department of Corrections and Rehabilitation told the committee. ``Could
we have done better in many instances? Of course we can.''

Broadly, Dana Simas, a spokeswoman for the agency, said that California
officials were confident that they could halt the spread of the virus
given that the prison system had longstanding plans for managing other
outbreaks of influenza,
\href{https://www.cdc.gov/norovirus/trends-outbreaks/burden-US.html}{norovirus},
measles and mumps.

\hypertarget{latest-updates-the-coronavirus-outbreak}{%
\section{\texorpdfstring{\href{https://www.nytimes3xbfgragh.onion/2020/09/11/world/covid-19-coronavirus.html?action=click\&pgtype=Article\&state=default\&region=MAIN_CONTENT_1\&context=storylines_live_updates}{Latest
Updates: The Coronavirus
Outbreak}}{Latest Updates: The Coronavirus Outbreak}}\label{latest-updates-the-coronavirus-outbreak}}

Updated 2020-09-12T06:16:33.399Z

\begin{itemize}
\tightlist
\item
  \href{https://www.nytimes3xbfgragh.onion/2020/09/11/world/covid-19-coronavirus.html?action=click\&pgtype=Article\&state=default\&region=MAIN_CONTENT_1\&context=storylines_live_updates\#link-dfb8a16}{Fauci
  cautions the virus could disrupt life in the U.S. until `maybe even
  towards the end of 2021.'}
\item
  \href{https://www.nytimes3xbfgragh.onion/2020/09/11/world/covid-19-coronavirus.html?action=click\&pgtype=Article\&state=default\&region=MAIN_CONTENT_1\&context=storylines_live_updates\#link-7104d154}{From
  Asia to Africa, China promotes its vaccine candidates to win friends.}
\item
  \href{https://www.nytimes3xbfgragh.onion/2020/09/11/world/covid-19-coronavirus.html?action=click\&pgtype=Article\&state=default\&region=MAIN_CONTENT_1\&context=storylines_live_updates\#link-393ad215}{The
  other way the virus will kill: hunger.}
\end{itemize}

\href{https://www.nytimes3xbfgragh.onion/2020/09/11/world/covid-19-coronavirus.html?action=click\&pgtype=Article\&state=default\&region=MAIN_CONTENT_1\&context=storylines_live_updates}{See
more updates}

More live coverage:
\href{https://www.nytimes3xbfgragh.onion/live/2020/09/11/business/stock-market-today-coronavirus?action=click\&pgtype=Article\&state=default\&region=MAIN_CONTENT_1\&context=storylines_live_updates}{Markets}

Across the United States, the number of prison and jail inmates known to
be infected has doubled during the past month to more than 80,000,
according to a New York Times database. Prison deaths tied to the
coronavirus have also risen significantly, by nearly 30 percent since
mid-May.
\href{https://www.nytimes3xbfgragh.onion/interactive/2020/us/coronavirus-us-cases.html\#clusters}{Nine
of the 10 largest known clusters of the virus} in the United States are
inside correctional institutions, The Times's data shows.

In California prisons, the number of cases has risen by nearly 200
percent and deaths by 144 percent during the past month.

\includegraphics{https://static01.graylady3jvrrxbe.onion/images/2020/06/29/us/00virus-sanquentin-02/merlin_174049668_e5330161-1e8e-4d5d-b443-2393a09e82ec-articleLarge.jpg?quality=75\&auto=webp\&disable=upscale}

Public health officials in California and elsewhere have been bracing
for months for what they say was inevitable --- the spreading of the
coronavirus in correctional facilities, which possess unique
vulnerabilities.

Most jails and prisons were designed to focus on security. Ventilation
is often poor and access to health care is inconsistent. Prison health
care in California has historically been so substandard that health
services are administered by a
\href{https://cchcs.ca.gov/wp-content/uploads/sites/60/2018/09/Receiver-FactSheet.pdf}{federal
receiver}.

California prisons have required everyone to wear masks, but social
distancing policies and mask-wearing rules among prison guards are
nearly impossible to enforce. Longstanding prohibitions on cleaning
supplies that contain bleach or alcohol have made it difficult for
crowded facilities like San Quentin to meet even basic sanitary
standards given that hundreds of inmates share a limited number of
toilets, telephones and shower stalls.

Since the pandemic, California has
\href{https://www.cdcr.ca.gov/news/2020/03/31/cdcr-announces-plan-to-further-protect-staff-and-inmates-from-the-spread-of-covid-19-in-state-prisons/}{agreed
to release} as many as 3,500 inmates up to six months early and is
\href{https://www.cdcr.ca.gov/news/2020/06/16/cdcr-announces-community-supervision-program-to-further-protect-inmates-and-staff-from-the-spread-of-covid-19/}{considering
more early releases}, but the prison system remains at 124 percent of
capacity, according to
\href{https://www.cdcr.ca.gov/research/population-reports-2/}{state
records}.

Public health experts said deficiencies were made worse at San Quentin.
The prison is dominated by row after row of barred cells. Paint peels
from walls, state work orders show, and puddles form after rain showers
because the ceilings leak.

The prison opened in 1852, and is at 117 percent of its capacity,
according to state data. As many as
\href{https://cchcs.ca.gov/wp-content/uploads/sites/60/QM/Public-Dashboard-2019-10.pdf}{half
of all inmates} suffer from health conditions that make them especially
vulnerable to the virus.

``There's no way to address a public health problem when you need to
isolate people but your system is bursting at the seams,'' said Adamu
Chan, a San Quentin inmate.

Dr. Brie Williams, a physician and professor of medicine at the
University of California, San Francisco, and director of the
university's Criminal Justice \& Health Program, said absent a
coronavirus vaccine, prisons were outmatched, despite their plans for
managing other sorts of outbreaks.

``The difference with this infection is that with all of those other
conditions we were able to essentially, eventually throw money at them
in the way of fancy medications,'' she said.

Dr. Matt Willis, the top public health official in Marin County, where
San Quentin is, said state prison officials had told him they were
capable of handling the virus on their own.

The county's health department was told by state prison leaders ``very
clearly that this is not part of our jurisdiction,'' Dr. Willis said.
The corrections system, he said, has a ``lot of control over every
aspect of their processes'' and has not been transparent about their
handling of the virus.

``It may work in certain settings,'' he added, ``but when you have a
complex disaster that's moving quickly, I think we're finding that the
process is just not matching our needs.''

San Quentin's crisis began with a handful of decisions that were made as
officials were trying to quell the outbreak in Chino, interviews with
inmates, correctional officers, elected officials and health experts
show.

On May 30, the inmates from Chino boarded buses for San Quentin after
being told they were being transferred to reduce overcrowding, which
would protect vulnerable inmates at the prison they were leaving, the
California Institution for Men.

Each of the 121 inmates who boarded the buses had been tested at various
points over the previous several months, but few --- if any --- had been
tested during the previous three weeks, prison officials have
acknowledged.

Arriving at San Quentin, prisoners' temperatures were taken and they
were placed in a holding area, but no Covid-19 tests were given.

For days, the men used the same showers and ate in the same dining hall
as other San Quentin inmates.

It took only days, data from the prison system shows, for the virus to
make its way through the prison, where hundreds of inmates sleep in bunk
beds within a few inches of one another in a crowded dormitory that was
once a gymnasium. In other parts of the prison, men are paired inside
4-by-9 foot cells.

Over the past week, the prison has conducted mass testing. So far, more
than half the inmates tested have seen positive results, state data
shows.

The virus has spread so rapidly and there is so little unoccupied space
left at the sprawling prison that some infected inmates have been placed
in small isolation cells where, in normal times, death row inmates are
sent for punishment.

Marion Wickerd got a call last week from her husband Tommy Wickerd, 53,
an inmate in San Quentin.

``He said, `People are dropping right and left in front of me, but I'm
OK,'' Ms. Wickerd said.

A few hours later, though, he called back. He had tested positive. She
said she had not spoken to him in several days.

``All I know is that my husband is not dead because no one has called to
tell me that,'' she said. ``Worried? You bet. Scared? You bet.''

Reporting was contributed by Brendon Derr, Danya Issawi*,* Ann Hinga
Klein, Savannah Redl and Maura Turcotte.

Advertisement

\protect\hyperlink{after-bottom}{Continue reading the main story}

\hypertarget{site-index}{%
\subsection{Site Index}\label{site-index}}

\hypertarget{site-information-navigation}{%
\subsection{Site Information
Navigation}\label{site-information-navigation}}

\begin{itemize}
\tightlist
\item
  \href{https://help.nytimes3xbfgragh.onion/hc/en-us/articles/115014792127-Copyright-notice}{©~2020~The
  New York Times Company}
\end{itemize}

\begin{itemize}
\tightlist
\item
  \href{https://www.nytco.com/}{NYTCo}
\item
  \href{https://help.nytimes3xbfgragh.onion/hc/en-us/articles/115015385887-Contact-Us}{Contact
  Us}
\item
  \href{https://www.nytco.com/careers/}{Work with us}
\item
  \href{https://nytmediakit.com/}{Advertise}
\item
  \href{http://www.tbrandstudio.com/}{T Brand Studio}
\item
  \href{https://www.nytimes3xbfgragh.onion/privacy/cookie-policy\#how-do-i-manage-trackers}{Your
  Ad Choices}
\item
  \href{https://www.nytimes3xbfgragh.onion/privacy}{Privacy}
\item
  \href{https://help.nytimes3xbfgragh.onion/hc/en-us/articles/115014893428-Terms-of-service}{Terms
  of Service}
\item
  \href{https://help.nytimes3xbfgragh.onion/hc/en-us/articles/115014893968-Terms-of-sale}{Terms
  of Sale}
\item
  \href{https://spiderbites.nytimes3xbfgragh.onion}{Site Map}
\item
  \href{https://help.nytimes3xbfgragh.onion/hc/en-us}{Help}
\item
  \href{https://www.nytimes3xbfgragh.onion/subscription?campaignId=37WXW}{Subscriptions}
\end{itemize}
