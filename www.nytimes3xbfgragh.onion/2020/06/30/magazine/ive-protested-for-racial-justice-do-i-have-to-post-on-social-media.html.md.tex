Sections

SEARCH

\protect\hyperlink{site-content}{Skip to
content}\protect\hyperlink{site-index}{Skip to site index}

\href{https://myaccount.nytimes3xbfgragh.onion/auth/login?response_type=cookie\&client_id=vi}{}

\href{https://www.nytimes3xbfgragh.onion/section/todayspaper}{Today's
Paper}

I've Protested for Racial Justice. Do I Have to Post on Social Media?

\url{https://nyti.ms/38fy2Jp}

\begin{itemize}
\item
\item
\item
\item
\item
\item
\end{itemize}

\hypertarget{race-and-america}{%
\subsubsection{\texorpdfstring{\href{https://www.nytimes3xbfgragh.onion/news-event/george-floyd-protests-minneapolis-new-york-los-angeles?name=styln-george-floyd\&region=TOP_BANNER\&block=storyline_menu_recirc\&action=click\&pgtype=Article\&impression_id=e578cfa0-f1c6-11ea-a94a-39b63fbebb75\&variant=undefined}{Race
and America}}{Race and America}}\label{race-and-america}}

\begin{itemize}
\tightlist
\item
  \href{https://www.nytimes3xbfgragh.onion/2020/09/04/nyregion/rochester-police-daniel-prude.html?name=styln-george-floyd\&region=TOP_BANNER\&block=storyline_menu_recirc\&action=click\&pgtype=Article\&impression_id=e578cfa1-f1c6-11ea-a94a-39b63fbebb75\&variant=undefined}{How
  Police Handled Death of Daniel Prude}
\item
  \href{https://www.nytimes3xbfgragh.onion/2020/09/01/us/politics/trump-fact-check-protests.html?name=styln-george-floyd\&region=TOP_BANNER\&block=storyline_menu_recirc\&action=click\&pgtype=Article\&impression_id=e578cfa2-f1c6-11ea-a94a-39b63fbebb75\&variant=undefined}{Trump
  Fact Check}
\item
  \href{https://www.nytimes3xbfgragh.onion/2020/08/30/us/portland-shooting-explained.html?name=styln-george-floyd\&region=TOP_BANNER\&block=storyline_menu_recirc\&action=click\&pgtype=Article\&impression_id=e578f6b0-f1c6-11ea-a94a-39b63fbebb75\&variant=undefined}{Portland
  Shooting}
\item
  \href{https://www.nytimes3xbfgragh.onion/2020/08/30/us/breonna-taylor-police-killing.html?name=styln-george-floyd\&region=TOP_BANNER\&block=storyline_menu_recirc\&action=click\&pgtype=Article\&impression_id=e578f6b1-f1c6-11ea-a94a-39b63fbebb75\&variant=undefined}{Breonna
  Taylor's Life and Death}
\end{itemize}

Advertisement

\protect\hyperlink{after-top}{Continue reading the main story}

Supported by

\protect\hyperlink{after-sponsor}{Continue reading the main story}

\href{/column/the-ethicist}{The Ethicist}

\hypertarget{ive-protested-for-racial-justice-do-i-have-to-post-on-social-media}{%
\section{I've Protested for Racial Justice. Do I Have to Post on Social
Media?}\label{ive-protested-for-racial-justice-do-i-have-to-post-on-social-media}}

\includegraphics{https://static01.graylady3jvrrxbe.onion/images/2020/07/05/magazine/05Ethicist/05Ethicist-articleLarge.jpg?quality=75\&auto=webp\&disable=upscale}

By Kwame Anthony Appiah

\begin{itemize}
\item
  June 30, 2020
\item
  \begin{itemize}
  \item
  \item
  \item
  \item
  \item
  \item
  \end{itemize}
\end{itemize}

\emph{I am a white college student trying to be a better racial-justice
advocate in the wake of George Floyd's death. While I have not been very
politically active in the past (which I regret), recently I have
attended a protest, donated a lot of my savings to bail funds and
related nonprofits, called my representatives about these issues and
tried to educate myself about black history and black voices. However, I
have not posted on my personal social media regarding Black Lives
Matter. I don't think I have a unique perspective to add to this
conversation, and I don't like the scrutiny that comes with a social
media post. I also worry that if I do post about Black Lives Matter, it
would be motivated mainly by virtue signaling.}

\emph{I believe my time would be more effectively spent supporting the
B.L.M. movement in other ways. But I've also heard the argument that
everyone has an obligation to post and that staying silent on these
platforms is morally wrong. What do you think: Does being an ally
require engaging with social media?} Name Withheld

\textbf{Citizens in a} democracy share a collective responsibility for
guiding the ship of state. That's why progress toward racial justice in
our country is something that we should all do our fair share to
advance. Like so many Americans, you've been moved by the wrenching
spectacle of a man's life slowly snuffed out by an officer of the state,
and troubled that black people continue to suffer from police violence
far out of proportion to their numbers. Unlike most, you've set out to
make a difference. If everyone else had done as much as you've done,
we'd certainly be way ahead. So I see no cause to complain about how
much you're doing. Something is part of everyone's fair share only if
everyone's doing it is necessary to achieve justice. People are not
morally obliged to have a social media account in the first place; when
they do, they should be free to post as little or as much as they like,
to treat it as a narrowly personal zone or as a forum for conveying
their broader concerns.

The issue you raise about ``virtue signaling,'' however, merits closer
attention. Some philosophers have argued that public discourse is
debased when people seek to raise their status within an in-group by
displaying their moral qualities. ``Grandstanding,'' the philosophers
Justin Tosi and Brandon Warmke argue in a new book of that title,
involves forms of moral argument that are motivated by the vanity of
self-presentation, by a desire to show that one is on the side of the
angels. People engaged in moral grandstanding, they believe, will tend
to ``pile on,'' repeating a widely shared criticism; ``trump up,''
depicting an innocent act as a major offense; and ``ramp up,'' making
ever stronger and more polarizing claims in order to outdo the moral
claims of others. There's a Kantian aspect to their perspective: For
Kant, it was of great importance that we not only do the right thing but
do it for the right reason. The notion of virtue signaling, which was
coined as a term of reproach, has precisely this Kantian valence. (This
may come as a surprise to many who wield the term.)

But virtue signaling isn't necessarily a vice. Let's grant that it can
have unfortunate aspects. Doesn't it have substantial positive ones too?
When it comes to uncontested moral values, we can prize the
unadvertised, anonymous good deed. (You rescue the drowning child, say,
and keep quiet about your heroism.) Yet the moral revolutions I've
researched involve what political scientists have called ``norm
cascades,'' and the social dimension of position-taking plays a critical
role here. In the late 18th century, the English manufacturer Josiah
Wedgwood started producing ceramic medallions with the words ``Am I Not
a Man and a Brother?'' above a black figure in chains. They became an
immensely popular icon among abolitionists --- a form of virtue
signaling that helped fortify and spread a vital moral idea. In the
words of another philosopher, Neil Levy: ``Signaling is a central
function of public moral discourse, with an important role to play in
enabling cooperation.'' That's why bumper stickers and slogans posted on
walls, whether digital or physical, can be meaningful.

The malign effects of grandstanding are real, but typically happen when
an important instrument for moral progress is put in service of bad
goals rather than good ones. ``Piling on'' can mean that people have
collectively decided to renounce a previously tolerated evil: It
mattered that a great many law-enforcement officials and even some
police-union representatives piled on against the callous killing of
George Floyd. The charge of ``trumping up'' can arise from a genuine
reassessment of once-taken-for-granted conduct --- it's the charge that
the old-school sexually harassing boss makes when habits he considered
harmless (``What's wrong with telling a woman how sexy she looks?'') are
properly seen in a new light. ``Ramping up'' can take us from the notion
that homosexual conduct should be decriminalized to the notion that gays
and lesbians should be allowed to marry.

In moments of moral change, people shift from merely recognizing a wrong
to wanting to do something about it. And what drives that shift is, in
part, a sense that those who don't contribute to change aren't just not
doing something good; they're forfeiting their entitlement to the
respect of those around them. Not participating becomes dishonorable. In
many of the moral revolutions I've written about --- against slavery,
dueling and foot-binding, for example --- winning the moral argument is
only the first step. To move a majority of people to live and act in new
ways, you have to get them to feel that doing the right thing is now
required for social respect. This is one area where social media can
help. And if you believe that you can make a contribution here, you
should be undeterred by the fear that you would be --- or would be
thought to be --- virtue signaling. In order for new and better norms to
arise, there has to be talk. But, as you recognize, there has to be more
than talk. People have to take action. You're to be commended for having
done so already.

\emph{I am the co-author, along with three other scientists, of a paper
we plan to submit to a prominent physics journal. Before I knew which
journal we would choose, I invited a professor who is also a member of
the editorial board of the journal in question to be a keynote speaker
for a conference I helped organize. The professor accepted, and we've
been in touch during the past few months about the conference.}

\emph{I am concerned about a potential conflict of interest here. The
professor will likely end up being the editor for the paper, which
entails deciding whether to accept it for review. The eventual
publication of the manuscript is based on the opinions of two reviewers
and the editor. While I am certain that the conference invitation will
not bias the professor, I do not want even an appearance of bias, nor do
I want the professor to think I expect special treatment. One option is
to ask that the professor be excluded as the editor for this submission.
I have discussed this issue with the senior co-author of the paper, and
he does not think we should exclude the professor, who, in his view, is
the most competent person to handle the manuscript. Still, there are
other editors who are qualified. Should I insist on excluding this
professor as an editor?} Name Withheld

\textbf{Wait --- have you} done this professor a favor, or has the
professor done you a favor? No matter. Being assessed by a person with
whom you've had friendly dealings risks not just the appearance of bias
but the reality of it. Even when we're doing our level best to be
scrupulously fair, we can be biased (sometimes negatively, out of an
excess of caution). Still, in a small subfield it can be hard to avoid
such ties. And your professor has two presumably independent reviewers
to rely on. We can't rule out such bias entirely. The best we can do is
have systems that curtail it.

The journal's processes will work well if your paper is clearly
excellent --- unconscious bias on anyone's part won't determine the
outcome --- or clearly subpar, because the editors won't publish
something that would damage their reputations. Only if the paper is in
the middle zone is the pull of propinquity going to matter. So you can
probably let things be; any unfairness will be pretty marginal. The
universe can handle one article that's in a somewhat better journal than
it deserves.

Advertisement

\protect\hyperlink{after-bottom}{Continue reading the main story}

\hypertarget{site-index}{%
\subsection{Site Index}\label{site-index}}

\hypertarget{site-information-navigation}{%
\subsection{Site Information
Navigation}\label{site-information-navigation}}

\begin{itemize}
\tightlist
\item
  \href{https://help.nytimes3xbfgragh.onion/hc/en-us/articles/115014792127-Copyright-notice}{©~2020~The
  New York Times Company}
\end{itemize}

\begin{itemize}
\tightlist
\item
  \href{https://www.nytco.com/}{NYTCo}
\item
  \href{https://help.nytimes3xbfgragh.onion/hc/en-us/articles/115015385887-Contact-Us}{Contact
  Us}
\item
  \href{https://www.nytco.com/careers/}{Work with us}
\item
  \href{https://nytmediakit.com/}{Advertise}
\item
  \href{http://www.tbrandstudio.com/}{T Brand Studio}
\item
  \href{https://www.nytimes3xbfgragh.onion/privacy/cookie-policy\#how-do-i-manage-trackers}{Your
  Ad Choices}
\item
  \href{https://www.nytimes3xbfgragh.onion/privacy}{Privacy}
\item
  \href{https://help.nytimes3xbfgragh.onion/hc/en-us/articles/115014893428-Terms-of-service}{Terms
  of Service}
\item
  \href{https://help.nytimes3xbfgragh.onion/hc/en-us/articles/115014893968-Terms-of-sale}{Terms
  of Sale}
\item
  \href{https://spiderbites.nytimes3xbfgragh.onion}{Site Map}
\item
  \href{https://help.nytimes3xbfgragh.onion/hc/en-us}{Help}
\item
  \href{https://www.nytimes3xbfgragh.onion/subscription?campaignId=37WXW}{Subscriptions}
\end{itemize}
