Sections

SEARCH

\protect\hyperlink{site-content}{Skip to
content}\protect\hyperlink{site-index}{Skip to site index}

\href{https://www.nytimes3xbfgragh.onion/section/nyregion}{New York}

\href{https://myaccount.nytimes3xbfgragh.onion/auth/login?response_type=cookie\&client_id=vi}{}

\href{https://www.nytimes3xbfgragh.onion/section/todayspaper}{Today's
Paper}

\href{/section/nyregion}{New York}\textbar{}New Yorkers Can Now Go Back
to Offices, but Many Won't

\url{https://nyti.ms/3fKGmTV}

\begin{itemize}
\item
\item
\item
\item
\item
\item
\end{itemize}

\hypertarget{the-coronavirus-outbreak}{%
\subsubsection{\texorpdfstring{\href{https://www.nytimes3xbfgragh.onion/news-event/coronavirus?name=styln-coronavirus-national\&region=TOP_BANNER\&block=storyline_menu_recirc\&action=click\&pgtype=Article\&impression_id=3caed8f0-f4d3-11ea-b483-2d6765c32891\&variant=undefined}{The
Coronavirus
Outbreak}}{The Coronavirus Outbreak}}\label{the-coronavirus-outbreak}}

\begin{itemize}
\tightlist
\item
  live\href{https://www.nytimes3xbfgragh.onion/2020/09/11/world/covid-19-coronavirus.html?name=styln-coronavirus-national\&region=TOP_BANNER\&block=storyline_menu_recirc\&action=click\&pgtype=Article\&impression_id=3caed8f1-f4d3-11ea-b483-2d6765c32891\&variant=undefined}{Latest
  Updates}
\item
  \href{https://www.nytimes3xbfgragh.onion/interactive/2020/us/coronavirus-us-cases.html?name=styln-coronavirus-national\&region=TOP_BANNER\&block=storyline_menu_recirc\&action=click\&pgtype=Article\&impression_id=3caed8f2-f4d3-11ea-b483-2d6765c32891\&variant=undefined}{Maps
  and Cases}
\item
  \href{https://www.nytimes3xbfgragh.onion/interactive/2020/science/coronavirus-vaccine-tracker.html?name=styln-coronavirus-national\&region=TOP_BANNER\&block=storyline_menu_recirc\&action=click\&pgtype=Article\&impression_id=3caf0000-f4d3-11ea-b483-2d6765c32891\&variant=undefined}{Vaccine
  Tracker}
\item
  \href{https://www.nytimes3xbfgragh.onion/2020/09/10/us/politics/fda-coronavirus-vaccine.html?name=styln-coronavirus-national\&region=TOP_BANNER\&block=storyline_menu_recirc\&action=click\&pgtype=Article\&impression_id=3caf0001-f4d3-11ea-b483-2d6765c32891\&variant=undefined}{F.D.A.
  Regulators' Self-Defense}
\item
  \href{https://www.nytimes3xbfgragh.onion/2020/09/09/upshot/coronavirus-surprise-test-fees.html?name=styln-coronavirus-national\&region=TOP_BANNER\&block=storyline_menu_recirc\&action=click\&pgtype=Article\&impression_id=3caf0002-f4d3-11ea-b483-2d6765c32891\&variant=undefined}{Surprise
  Test Fees}
\end{itemize}

Advertisement

\protect\hyperlink{after-top}{Continue reading the main story}

Supported by

\protect\hyperlink{after-sponsor}{Continue reading the main story}

\hypertarget{new-yorkers-can-now-go-back-to-offices-but-many-wont}{%
\section{New Yorkers Can Now Go Back to Offices, but Many
Won't}\label{new-yorkers-can-now-go-back-to-offices-but-many-wont}}

As the city entered Phase 2 of reopening, subway cars and sidewalks were
relatively empty as many companies kept workers home.

\includegraphics{https://static01.graylady3jvrrxbe.onion/images/2020/06/23/nyregion/23nyvirus-reopen3/22nyvirus-reopen-top21-articleLarge.jpg?quality=75\&auto=webp\&disable=upscale}

By \href{https://www.nytimes3xbfgragh.onion/by/michael-gold}{Michael
Gold} and \href{https://www.nytimes3xbfgragh.onion/by/troy-closson}{Troy
Closson}

\begin{itemize}
\item
  Published June 22, 2020Updated July 16, 2020
\item
  \begin{itemize}
  \item
  \item
  \item
  \item
  \item
  \item
  \end{itemize}
\end{itemize}

Even as offices across New York City were allowed to welcome back
employees on Monday for the first time in months, the number of those
returning to work was far lower than the swarms that once jostled elbows
on public transit and packed into high-rise elevators.

With the coronavirus still a threat and businesses required to limit
their capacity and ensure distance between workers, sidewalks that would
typically be crammed were fairly empty.

Subway cars also had relatively few riders for the start of the
workweek, and parks in business districts were sparsely populated during
the usual lunch rush.

``I'm really surprised this is still this empty,'' Jason Blankenship, an
optometrist, said as he looked around a quiet Bryant Park. ``I thought
it would be more people than this for sure. I wonder if all these people
from these offices will ever come back.''

At the same time, many of those who returned to offices and stores were
eager to make any step, however symbolic, toward the pre-pandemic status
quo.

``It's nice to get back to kind of normal, even though it's not 100
percent normal,'' said Kiki Boyzuick, 45, who works in human resources
in
\href{https://www.nytimes3xbfgragh.onion/2020/07/26/nyregion/nyc-coronavirus-time-life-building.html}{Midtown
Manhattan}.

More than a hundred days ago, buildings across New York shut their doors
and companies sent workers home. As the pandemic swept across the city,
lockdown orders left offices dormant, stores shuttered and streets and
sidewalks all but abandoned.

On Monday in New York City, two weeks after officials first
\href{https://www.nytimes3xbfgragh.onion/2020/06/07/nyregion/new-york-reopening-coronavirus.html?module=inline}{began
easing restrictions}, a much larger reopening phase began --- one that
permits outdoor dining and some in-store shopping, and also allows hair
salons, barbershops and
\href{https://www.nytimes3xbfgragh.onion/2020/06/24/realestate/phase-2-showings.html}{real
estate firms} to restart their work.

``Phase 1 was a big deal,'' Mayor Bill de Blasio said at a news
briefing. ``But Phase 2 is really a giant step for this city. This is
where most of our economy is.''

This move toward normalcy, the city's broadest yet, will pose a major
test for efforts to keep the coronavirus at bay, with as many as 300,000
people projected to return this week to jobs that keep them in enclosed
spaces for hours at a time.

\includegraphics{https://static01.graylady3jvrrxbe.onion/images/2020/06/23/nyregion/23nyvirus-reopen1/merlin_173799381_af1732f5-2915-46b6-8e25-ebc198499d0d-articleLarge.jpg?quality=75\&auto=webp\&disable=upscale}

New York City was the last region in the state to enter the second of
the four stages in the state's reopening plan. On Monday, Gov. Andrew M.
Cuomo announced that the city's suburbs were set to enter Phase 3 this
week, which allows for indoor dining and personal-care services such as
nail salons.

Even as New York City has made significant progress fighting the
coronavirus ---~its positive test rate now hovers
\href{https://www1.nyc.gov/site/doh/covid/covid-19-data.page}{around 1
percent}, down significantly from about 60 percent in early April ---
many companies still see the virus as enough of a threat that they have
decided to not bring workers back for months, if not longer.

\hypertarget{latest-updates-the-coronavirus-outbreak}{%
\section{\texorpdfstring{\href{https://www.nytimes3xbfgragh.onion/2020/09/11/world/covid-19-coronavirus.html?action=click\&pgtype=Article\&state=default\&region=MAIN_CONTENT_1\&context=storylines_live_updates}{Latest
Updates: The Coronavirus
Outbreak}}{Latest Updates: The Coronavirus Outbreak}}\label{latest-updates-the-coronavirus-outbreak}}

Updated 2020-09-12T07:09:04.082Z

\begin{itemize}
\tightlist
\item
  \href{https://www.nytimes3xbfgragh.onion/2020/09/11/world/covid-19-coronavirus.html?action=click\&pgtype=Article\&state=default\&region=MAIN_CONTENT_1\&context=storylines_live_updates\#link-dfb8a16}{Fauci
  cautions the virus could disrupt life in the U.S. until `maybe even
  towards the end of 2021.'}
\item
  \href{https://www.nytimes3xbfgragh.onion/2020/09/11/world/covid-19-coronavirus.html?action=click\&pgtype=Article\&state=default\&region=MAIN_CONTENT_1\&context=storylines_live_updates\#link-7104d154}{From
  Asia to Africa, China promotes its vaccine candidates to win friends.}
\item
  \href{https://www.nytimes3xbfgragh.onion/2020/09/11/world/covid-19-coronavirus.html?action=click\&pgtype=Article\&state=default\&region=MAIN_CONTENT_1\&context=storylines_live_updates\#link-393ad215}{The
  other way the virus will kill: hunger.}
\end{itemize}

\href{https://www.nytimes3xbfgragh.onion/2020/09/11/world/covid-19-coronavirus.html?action=click\&pgtype=Article\&state=default\&region=MAIN_CONTENT_1\&context=storylines_live_updates}{See
more updates}

More live coverage:
\href{https://www.nytimes3xbfgragh.onion/live/2020/09/11/business/stock-market-today-coronavirus?action=click\&pgtype=Article\&state=default\&region=MAIN_CONTENT_1\&context=storylines_live_updates}{Markets}

In a survey conducted this month by the Partnership for New York City,
an influential business group, respondents from 60 companies with
Manhattan offices predicted that only 10 percent of their employees
would return by Aug. 15.

Several big media and technology companies with Manhattan offices had
already extended their work-from-home policies through the summer.
Others have said employees can work remotely through the end of the
year.

JPMorgan Chase, one of the city's largest commercial tenants, said it
would not send employees back this week and had not set a return date.
Other financial services firms, like Goldman Sachs, anticipated that a
small number of employees would return but said that most would not come
back until well into next year.

Image

People waiting to get inside a Foot Locker store in Manhattan. Retail
businesses reopened on Monday, but with limited capacity.Credit...Amr
Alfiky/The New York Times

The real estate company Rudin Management Company said that it reached a
collective 5.2 percent of capacity across its 14 office buildings in New
York that reopened on Monday.

Workers across the city returning found significantly different spaces
awaiting them.

Mike Chapman, 54, a technology consultant, said he was happy to return
to his office after three months of working in a small apartment with
his fiancée. But he was the only one of seven employees to go back.

``It's not going to feel normal to be in the office,'' said Ciara
Lakhani, the chief people officer of Dashlane, a software company with
about 100 employees in New York. ``You can't socialize the same way. You
can't really attend meetings in person.''

More than 100 cases of Covid-19 are still being reported each day in New
York, according to city data. A contact-tracing program that is supposed
to help track the spread of the virus as the city reopens has
\href{https://www.nytimes3xbfgragh.onion/2020/06/21/nyregion/nyc-contact-tracing.html}{gotten
off to a slow start}.

Worried about a surge of cases in states that moved more quickly to
reopen, New York officials are requiring that strict social-distancing
guidelines remain. Landlords of commercial buildings said they had been
preparing to reopen by implementing new safety and cleaning protocols.

Husein Sonara, the chief operating officer at the Sapir Organization,
which manages two properties in Midtown, said his company had put
markers on sidewalks outside its buildings, in the hallways inside and
in elevators so workers can maintain social distancing.

Ken Fisher, a partner at Fisher Brothers, which owns five office towers
in Manhattan, said his buildings would use thermal scanners to check the
temperatures of everyone who entered. Hand sanitizer would be placed in
all communal areas, and only four people would be allowed in each
elevator at a time.

Image

Patrons enjoyed an outdoor meal on Monday in the Financial District in
Lower Manhattan.Credit...Amr Alfiky/The New York Times

The public health concerns left some people worried that the reopening
was progressing too quickly.

Raj Banik, 37, who owns the bar and record store RPM Underground, was
concerned that a potential second wave of cases could mean
\href{https://www.nytimes3xbfgragh.onion/2020/07/16/business/economy/company-reopening-coronavirus.html}{shutting
down again}.

``I'd rather reopen when it's safe for everybody,'' Mr. Banik said.
``I'd rather everyone just be safe once and for all.''

Iveth Otero, 29, a typist for a textile company, was unhappy about
having to return to work on Monday.

``It was only Latinos and black people on the train,'' Ms. Otero said of
traveling on the No. 2 subway line from Upper Manhattan to Midtown. ``No
white people came to work.''

Work was not busy, she said, adding that she could have been more
productive from home.

``It doesn't feel nice,'' Mr. Otero said, ``but it's what you have to do
for your money.''

Still, others said they were eager to return to their desks.

Charles de Montebello, who runs an
\href{https://cdmstudios.com/}{audio-recording studio} in Manhattan's
Hell's Kitchen neighborhood, said that only two of his five recording
rooms would operate this week. That way, he said, he could ensure that
people did not come in close contact with one another.

Both studios were booked for full-day sessions on Monday.

``It's been hard to be away and hard to shut my doors and have literally
90 percent of my income evaporate,'' Mr. de Montebello said.

In addition to offices, retail stores --- which for two weeks had been
limited to curbside or in-store pickup --- were allowing customers
inside, though at a reduced capacity.

``It feels like it's the light at the end of a very long tunnel,'' said
Nancy Bass Wyden, the owner of the well-known
\href{https://www.nytimes3xbfgragh.onion/2018/12/03/nyregion/strand-bookstore-landmark.html}{Strand
bookstore} in Manhattan. ``Almost like we're Rip Van Winkle.''

Restaurants can also begin serving meals outdoors, real-estate brokers
can show listings, and salons and barbershops can welcome customers who
have been without grooming services for months.

At Fancy Wave Salon in Flushing, Queens, hairstylists wore face shields,
gloves and masks as they attended to their clients' hair. Derrick Chan,
the owner, said he was thrilled to reopen.

``We had to pretty much stay home, no income,'' Mr. Chan said. ``That's
why you have to save up for the rainy days.''

The second stage of reopening is also posing another test of how
effectively the city's transportation system can safely carry daily
commuters.

The Metropolitan Transportation Authority, which runs the city's subway
and buses, could not immediately provide ridership figures for Monday.
But more people had returned to public transportation during the first
phase of reopening than officials had anticipated.

In May, transit officials predicted that daily ridership on buses would
reach 40 percent of pre-pandemic levels --- 880,000 people --- during
the first phase. But bus ridership has already reached 56 percent of the
usual passenger load.

On the subway, daily ridership has climbed to 17 percent of pre-pandemic
levels --- two percentage points higher than the M.T.A.'s initial
projections. The transit agency expects that number to double, reaching
as many as two million people, during the second phase. Before the
pandemic, ridership exceeded five million.

Riders are required to
\href{https://www.nytimes3xbfgragh.onion/article/coronavirus-N95-mask-DIY-face-mask-health.html}{wear
face masks} under an executive order from Mr. Cuomo. While most people
appeared to follow suit on Monday, not everyone complied.

Nargiz Aziz, a Staten Island resident who commuted to her job near the
World Trade Center on Monday, said people stared at her on the train
because she was not wearing a face covering.

``It doesn't let me breathe well, and it's not comfortable,'' Ms. Aziz,
20, said.

Reporting was contributed by Jo Corona, Christina Goldbaum, Nate
Schweber and Daniel E. Slotnik.

Advertisement

\protect\hyperlink{after-bottom}{Continue reading the main story}

\hypertarget{site-index}{%
\subsection{Site Index}\label{site-index}}

\hypertarget{site-information-navigation}{%
\subsection{Site Information
Navigation}\label{site-information-navigation}}

\begin{itemize}
\tightlist
\item
  \href{https://help.nytimes3xbfgragh.onion/hc/en-us/articles/115014792127-Copyright-notice}{©~2020~The
  New York Times Company}
\end{itemize}

\begin{itemize}
\tightlist
\item
  \href{https://www.nytco.com/}{NYTCo}
\item
  \href{https://help.nytimes3xbfgragh.onion/hc/en-us/articles/115015385887-Contact-Us}{Contact
  Us}
\item
  \href{https://www.nytco.com/careers/}{Work with us}
\item
  \href{https://nytmediakit.com/}{Advertise}
\item
  \href{http://www.tbrandstudio.com/}{T Brand Studio}
\item
  \href{https://www.nytimes3xbfgragh.onion/privacy/cookie-policy\#how-do-i-manage-trackers}{Your
  Ad Choices}
\item
  \href{https://www.nytimes3xbfgragh.onion/privacy}{Privacy}
\item
  \href{https://help.nytimes3xbfgragh.onion/hc/en-us/articles/115014893428-Terms-of-service}{Terms
  of Service}
\item
  \href{https://help.nytimes3xbfgragh.onion/hc/en-us/articles/115014893968-Terms-of-sale}{Terms
  of Sale}
\item
  \href{https://spiderbites.nytimes3xbfgragh.onion}{Site Map}
\item
  \href{https://help.nytimes3xbfgragh.onion/hc/en-us}{Help}
\item
  \href{https://www.nytimes3xbfgragh.onion/subscription?campaignId=37WXW}{Subscriptions}
\end{itemize}
