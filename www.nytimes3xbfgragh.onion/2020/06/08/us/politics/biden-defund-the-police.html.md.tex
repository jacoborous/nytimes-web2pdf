Sections

SEARCH

\protect\hyperlink{site-content}{Skip to
content}\protect\hyperlink{site-index}{Skip to site index}

\href{https://www.nytimes3xbfgragh.onion/section/politics}{Politics}

\href{https://myaccount.nytimes3xbfgragh.onion/auth/login?response_type=cookie\&client_id=vi}{}

\href{https://www.nytimes3xbfgragh.onion/section/todayspaper}{Today's
Paper}

\href{/section/politics}{Politics}\textbar{}Biden Walks a Cautious Line
as He Opposes Defunding the Police

\url{https://nyti.ms/2Uic9U3}

\begin{itemize}
\item
\item
\item
\item
\item
\item
\end{itemize}

\hypertarget{race-and-america}{%
\subsubsection{\texorpdfstring{\href{https://www.nytimes3xbfgragh.onion/news-event/george-floyd-protests-minneapolis-new-york-los-angeles?name=styln-george-floyd\&region=TOP_BANNER\&block=storyline_menu_recirc\&action=click\&pgtype=Article\&impression_id=93af4c90-f2ab-11ea-aa8b-576dc9f1a157\&variant=undefined}{Race
and America}}{Race and America}}\label{race-and-america}}

\begin{itemize}
\tightlist
\item
  \href{https://www.nytimes3xbfgragh.onion/2020/09/04/nyregion/rochester-police-daniel-prude.html?name=styln-george-floyd\&region=TOP_BANNER\&block=storyline_menu_recirc\&action=click\&pgtype=Article\&impression_id=93af4c91-f2ab-11ea-aa8b-576dc9f1a157\&variant=undefined}{What
  Happened in Rochester, N.Y.}
\item
  \href{https://www.nytimes3xbfgragh.onion/2020/09/01/us/politics/trump-fact-check-protests.html?name=styln-george-floyd\&region=TOP_BANNER\&block=storyline_menu_recirc\&action=click\&pgtype=Article\&impression_id=93af73a0-f2ab-11ea-aa8b-576dc9f1a157\&variant=undefined}{Trump
  Fact Check}
\item
  \href{https://www.nytimes3xbfgragh.onion/2020/08/30/us/portland-shooting-explained.html?name=styln-george-floyd\&region=TOP_BANNER\&block=storyline_menu_recirc\&action=click\&pgtype=Article\&impression_id=93af73a1-f2ab-11ea-aa8b-576dc9f1a157\&variant=undefined}{Portland
  Shooting}
\item
  \href{https://www.nytimes3xbfgragh.onion/2020/08/30/us/breonna-taylor-police-killing.html?name=styln-george-floyd\&region=TOP_BANNER\&block=storyline_menu_recirc\&action=click\&pgtype=Article\&impression_id=93af73a2-f2ab-11ea-aa8b-576dc9f1a157\&variant=undefined}{Breonna
  Taylor's Life and Death}
\end{itemize}

Advertisement

\protect\hyperlink{after-top}{Continue reading the main story}

Supported by

\protect\hyperlink{after-sponsor}{Continue reading the main story}

\hypertarget{biden-walks-a-cautious-line-as-he-opposes-defunding-the-police}{%
\section{Biden Walks a Cautious Line as He Opposes Defunding the
Police}\label{biden-walks-a-cautious-line-as-he-opposes-defunding-the-police}}

As gruesome videos and energetic protests reshape public opinion about
racial discrimination, Joe Biden tried to balance protesters' calls for
a law enforcement overhaul while not alienating moderate voters.

\includegraphics{https://static01.graylady3jvrrxbe.onion/images/2020/06/08/us/politics/08biden-police/merlin_173110818_82ad57cf-15f4-4324-8f9e-605b850fb106-articleLarge.jpg?quality=75\&auto=webp\&disable=upscale}

\href{https://www.nytimes3xbfgragh.onion/by/jonathan-martin}{\includegraphics{https://static01.graylady3jvrrxbe.onion/images/2018/11/06/multimedia/author-jonathan-martin/author-jonathan-martin-thumbLarge.png}}\href{https://www.nytimes3xbfgragh.onion/by/alexander-burns}{\includegraphics{https://static01.graylady3jvrrxbe.onion/images/2018/09/25/multimedia/author-alexander-burns/author-alexander-burns-thumbLarge-v2.png}}\href{https://www.nytimes3xbfgragh.onion/by/thomas-kaplan}{\includegraphics{https://static01.graylady3jvrrxbe.onion/images/2019/08/28/reader-center/author-thomas-kaplan/author-thomas-kaplan-thumbLarge-v2.png}}

By \href{https://www.nytimes3xbfgragh.onion/by/jonathan-martin}{Jonathan
Martin},
\href{https://www.nytimes3xbfgragh.onion/by/alexander-burns}{Alexander
Burns} and
\href{https://www.nytimes3xbfgragh.onion/by/thomas-kaplan}{Thomas
Kaplan}

\begin{itemize}
\item
  Published June 8, 2020Updated Aug. 19, 2020
\item
  \begin{itemize}
  \item
  \item
  \item
  \item
  \item
  \item
  \end{itemize}
\end{itemize}

Former Vice President
\href{https://www.nytimes3xbfgragh.onion/interactive/2020/us/elections/joe-biden.html}{Joseph
R. Biden Jr.} staked out a careful position on Monday in support of a
law enforcement overhaul but not
\href{https://www.nytimes3xbfgragh.onion/2020/08/19/us/politics/democrats-biden-defund-police.html}{defunding
police departments}, rebutting a new Republican attack line as he tries
to harness growing activism against systemic racism while not alienating
protesters or more moderate voters.

In the face of continuing protest marches calling to ``defund the
police'' nationwide in the aftermath of George Floyd's killing,
\href{https://www.nytimes3xbfgragh.onion/2020/06/10/us/politics/joe-biden-black-vice-president.html}{Mr.
Biden's} campaign said in a statement that he ``hears and shares the
deep grief and frustration of those calling out for change'' and that he
``supports the urgent need for reform.'' But a campaign spokesman,
Andrew Bates, said flatly that Mr. Biden was opposed to cutting police
funding and believed more spending was necessary to help improve law
enforcement and community policing.

Mr. Biden's effort to address the calls of protesters while supporting
law enforcement comes after gruesome videos and energetic protests have
quickly reshaped public opinion about racial discrimination, seemingly
opening a substantial window for new policies that could bring
far-reaching change to law-enforcement agencies long accused of racially
discriminatory practices. But there are already signs of division
between activists who are eager to dismantle police departments and
congressional Democrats who favor a less drastic overhaul of law
enforcement.

\href{https://www.nytimes3xbfgragh.onion/interactive/2020/us/elections/donald-trump.html}{President
Trump}'s campaign and leading Republicans have sought to drive a wedge
between the immediate-but-incremental calls for change among elected
Democrats and the more sweeping demands that protesters are calling for
in places like Minneapolis, where the death of Mr. Floyd after police
officers pinned him down has prompted worldwide calls for racial
justice.

Mr. Trump, for his part, has not endorsed any new changes to policing
procedures or funding. On Monday, he met with law enforcement officials
at the White House and praised them, saying virtually all police
officers were ``great, great people'' and boasting on Twitter that crime
was low nationwide.

The debate within the Democratic Party was on plain display on Monday,
as congressional leaders
\href{https://www.nytimes3xbfgragh.onion/2020/06/08/us/politics/democrats-police-misconduct-bill-protests.html}{unveiled
a broad legislative program} on policing, including new limits on the
use of lethal force and on the legal protections afforded to officers
accused of misconduct. Only hours before, progressives at the municipal
level in Minneapolis pledged on Sunday to
\href{https://www.nytimes3xbfgragh.onion/2020/06/07/us/minneapolis-police-abolish.html}{take
apart the city's long-troubled Police Department} and rebuild it
altogether.

Asked by the CBS host Norah O'Donnell on Monday if he supported
defunding the police, Mr. Biden answered: ``No, I don't support
defunding the police. I support conditioning federal aid to police based
on whether or not they meet certain basic standards of decency and
honorableness.''

Mr. Biden's position --- a stance on policing that another prominent
Democrat, Bill Clinton, might have summed up as ``mend it, don't end
it'' --- aligned him far more closely with lawmakers in Washington than
with activists and left-wing lawmakers at the municipal level. His
approach drew wide support from Democratic Party officials and a number
of civil rights leaders, as well as politicians in the swing states
likeliest to decide the general election.

That's in part because few Democrats have embraced activist demands to
make deep cuts to police budgets or to shutter local law-enforcement
agencies altogether. Answering those calls, most party leaders have
tried to mingle sympathy for the underlying grievances of police critics
with Biden-style rejection of ideas like police abolition.

``We need police officers, and we need law enforcement,'' said Bob
Buckhorn, the former mayor of Tampa, Fla., and a supporter of Mr.
Biden's. ``They are oftentimes the only thing between good and evil, and
between chaos and calm.''

Mr. Buckhorn said policymakers needed to address legitimate ``built-up
anger,'' but he also cautioned, ``Anybody who is suggesting that they
defund the police, whatever that means, I think would be making a
terrible mistake.''

Even a number of progressive Democrats do not think Mr. Biden should
suddenly veer left.

Several young black leaders said Mr. Biden should pursue a criminal
justice overhaul, and seek out and elevate activists who are organizing
peaceful demonstrations.

``That puts him in front of people he's not usually in front of and
demonstrates what kind of leader he is,'' Lt. Gov. Garlin Gilchrist II
of Michigan said.

Quentin James, who runs an organization dedicated to electing
African-American officials, said, ``I don't think we need him to say
`defund police,' but he can help lift up groups like Collective PAC that
are trying to elect reform-minded prosecutors.''

Mr. Trump's attempts to mount a soft-on-crime campaign against Mr. Biden
may prove difficult. The 2020 presidential election, after all, is not
Mr. Biden's first attempt to balance the public's appetite for change
with deeply rooted conservatism in much of the country.

He channeled the rage of young voters who were furious about Vietnam and
Watergate to claim a Senate seat in Delaware in the same year that
Richard M. Nixon won a 49-state re-election landslide thanks to the
perceived excesses of the Democratic left and its presidential nominee,
George McGovern.

Ever since, Mr. Biden, 77, has carefully balanced the passions of
activists with the sensibilities of the political center, portraying
himself as both a champion for racial equality and a reliable ally of
law enforcement.

His statement on Monday and a speech he gave last week in Philadelphia
reflected the impulses of someone who, for much of his career,
represented a state that was politically competitive and split between a
heavily urban northern tier and its so-called ``slower lower'' rural
south.

Mr. Biden's positioning also owes to his grounding in a Senate where
forging consensus was a necessity even if it meant,
\href{https://www.nytimes3xbfgragh.onion/2019/06/19/us/politics/biden-segregationists.html}{as
he memorably noted last year}, working with segregationists.

Such comments infuriated many Democrats during the party's presidential
primary race, especially younger progressives who, at their most
charitable, viewed him as a well-meaning if out-of-touch relic from
yesteryear.

Yet then as now Mr. Biden has been greatly aided by Mr. Trump, whose
divisive message and conduct have only alienated up-for-grabs voters who
are craving stability. At this stage in the campaign, voters are
comparing Mr. Biden --- to borrow a Bidenism --- not to the Almighty but
to the alternative, and so far Mr. Trump has offered little to Americans
seeking a sympathetic and textured response to the crisis.

Mr. Biden is also finding it easier to stake out ground in support of
overhauling the police because the
\href{https://www.nytimes3xbfgragh.onion/2020/06/05/us/politics/polling-george-floyd-protests-racism.html}{polling
has shifted considerably on the issue}, making it harder for Mr. Trump
to gain much leverage from his repeated calls for ``law and order.''

A new NBC News/Wall Street Journal poll indicated that just 27 percent
of voters said they were more worried about the protests, some of which
have turned violent, than about the death of Mr. Floyd and police
conduct. Among white voters, that figure was just 30 percent.

``With the video footage we are now seeing, even the typical voter who's
all for law and order is not going to be able to ignore the fact that in
some police departments, in the department themselves, they lack law and
order,'' said Representative Anthony G. Brown, Democrat of Maryland.

Mr. Brown likened this moment to the violence that law enforcement
inflicted on civil rights protesters in the 1960s, which he said was
when ``decent Americans woke up and said, `Hey, that's wrong, we have to
do something.'''

Just as there were divisions in the civil rights movement between older
leaders and younger, more radical activists, however, some of the same
generational differences exist today.

``I've seen this in neighborhood meetings where a young person will come
in and say, `We have too many police officers here,' and I'll see a
senior saying, `Oh, no, no, that's not true,''' said Mayor Nan Whaley of
Dayton, Ohio, a Democrat.

This split has also found its way into the ranks of Democratic
lawmakers. On Monday, Representative James E. Clyburn of South Carolina
--- who is now the highest-ranking black member of Congress but, in his
youth, was jailed during protests --- complained on a private conference
call with other lawmakers about those trying to ``hijack'' the swelling
movement with calls to defund the police.

Later on the call, however, Representative Alexandria Ocasio-Cortez of
New York, 30, warned against actions that would ``demoralize or
undercut'' protest leaders, according to a House Democratic official
familiar with the conversation.

Mr. Biden has sought to avoid inflaming either constituency. In his
speech last week, he was blunt that the story of America ``isn't a fairy
tale.'' Yet he invoked the Civil War, the Great Depression and often
bloody civil rights protests to argue that ``in some of our darkest
moments of despair, we've made some of our greatest progress.''

Representative Colin Allred of Texas said that recalling past crises and
the progress that emerged from them was a way to assure voters that
better days are ahead while also recalling pre-Trump presidential
leadership --- ``a call for change and a call for restoration,'' as Mr.
Allred, a Democrat, put it.

In the swing state of Nevada, Chris Giunchigliani, a Democratic former
member of the Clark County Commission, warned her party that incremental
improvements to policing would not be enough. Making a few changes at
the federal level and then declaring victory, she said, might ``give
people false hope that things are fixed,'' when an overhaul would
require meticulous work at the state and local level.

``To folks who want to defund: You need a police department, I think
that's reasonable,'' said Ms. Giunchigliani, a former candidate for
governor. ``But I think it's time to take a step back and work on: What
should policing in the community be?''

As encouraging as the polling on police issues is for Democrats right
now, many in the party are wary of putting too much confidence in the
expressions of support from white voters who have historically recoiled
from seeing public officials at odds with law enforcement.

Partly for that reason, black leaders are eager to see Democrats press
hard in the coming weeks for the moderate slate of policy changes
proposed on Monday, in the hope that they could be passed swiftly in an
environment that is favorable but could turn out to be fragile.

Marc Morial, the president of the National Urban League, said he
believed the country was at a ``transformative moment,'' in which the
killing of Mr. Floyd had shaken white voters from passive sympathy to
active support for overhauling law enforcement.

``I think there's always been support there, but I think it's been
muted,'' said Mr. Morial, who served two terms as mayor of New Orleans.
Now, he said, there's a new majority and it's not at all silent.

Katie Glueck, Astead W. Herndon and Annie Karni contributed reporting.

Advertisement

\protect\hyperlink{after-bottom}{Continue reading the main story}

\hypertarget{site-index}{%
\subsection{Site Index}\label{site-index}}

\hypertarget{site-information-navigation}{%
\subsection{Site Information
Navigation}\label{site-information-navigation}}

\begin{itemize}
\tightlist
\item
  \href{https://help.nytimes3xbfgragh.onion/hc/en-us/articles/115014792127-Copyright-notice}{©~2020~The
  New York Times Company}
\end{itemize}

\begin{itemize}
\tightlist
\item
  \href{https://www.nytco.com/}{NYTCo}
\item
  \href{https://help.nytimes3xbfgragh.onion/hc/en-us/articles/115015385887-Contact-Us}{Contact
  Us}
\item
  \href{https://www.nytco.com/careers/}{Work with us}
\item
  \href{https://nytmediakit.com/}{Advertise}
\item
  \href{http://www.tbrandstudio.com/}{T Brand Studio}
\item
  \href{https://www.nytimes3xbfgragh.onion/privacy/cookie-policy\#how-do-i-manage-trackers}{Your
  Ad Choices}
\item
  \href{https://www.nytimes3xbfgragh.onion/privacy}{Privacy}
\item
  \href{https://help.nytimes3xbfgragh.onion/hc/en-us/articles/115014893428-Terms-of-service}{Terms
  of Service}
\item
  \href{https://help.nytimes3xbfgragh.onion/hc/en-us/articles/115014893968-Terms-of-sale}{Terms
  of Sale}
\item
  \href{https://spiderbites.nytimes3xbfgragh.onion}{Site Map}
\item
  \href{https://help.nytimes3xbfgragh.onion/hc/en-us}{Help}
\item
  \href{https://www.nytimes3xbfgragh.onion/subscription?campaignId=37WXW}{Subscriptions}
\end{itemize}
