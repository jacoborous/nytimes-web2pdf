Sections

SEARCH

\protect\hyperlink{site-content}{Skip to
content}\protect\hyperlink{site-index}{Skip to site index}

\href{https://www.nytimes3xbfgragh.onion/section/business}{Business}

\href{https://myaccount.nytimes3xbfgragh.onion/auth/login?response_type=cookie\&client_id=vi}{}

\href{https://www.nytimes3xbfgragh.onion/section/todayspaper}{Today's
Paper}

\href{/section/business}{Business}\textbar{}Forgetting Contagion, Stocks
Recover Their Losses

\url{https://nyti.ms/2YiJmQy}

\begin{itemize}
\item
\item
\item
\item
\item
\end{itemize}

\hypertarget{the-coronavirus-outbreak}{%
\subsubsection{\texorpdfstring{\href{https://www.nytimes3xbfgragh.onion/news-event/coronavirus?name=styln-coronavirus-markets\&region=TOP_BANNER\&block=storyline_menu_recirc\&action=click\&pgtype=Article\&impression_id=90442ff0-f52e-11ea-a263-739fdfaf6370\&variant=undefined}{The
Coronavirus
Outbreak}}{The Coronavirus Outbreak}}\label{the-coronavirus-outbreak}}

\begin{itemize}
\tightlist
\item
  live\href{https://www.nytimes3xbfgragh.onion/2020/09/12/world/covid-19-coronavirus.html?name=styln-coronavirus-markets\&region=TOP_BANNER\&block=storyline_menu_recirc\&action=click\&pgtype=Article\&impression_id=90445700-f52e-11ea-a263-739fdfaf6370\&variant=undefined}{Latest
  Updates}
\item
  \href{https://www.nytimes3xbfgragh.onion/interactive/2020/us/coronavirus-us-cases.html?name=styln-coronavirus-markets\&region=TOP_BANNER\&block=storyline_menu_recirc\&action=click\&pgtype=Article\&impression_id=90445701-f52e-11ea-a263-739fdfaf6370\&variant=undefined}{Maps
  and Cases}
\item
  \href{https://www.nytimes3xbfgragh.onion/interactive/2020/science/coronavirus-vaccine-tracker.html?name=styln-coronavirus-markets\&region=TOP_BANNER\&block=storyline_menu_recirc\&action=click\&pgtype=Article\&impression_id=90445702-f52e-11ea-a263-739fdfaf6370\&variant=undefined}{Vaccine
  Tracker}
\item
  \href{https://www.nytimes3xbfgragh.onion/2020/09/10/us/politics/fda-coronavirus-vaccine.html?name=styln-coronavirus-markets\&region=TOP_BANNER\&block=storyline_menu_recirc\&action=click\&pgtype=Article\&impression_id=90445703-f52e-11ea-a263-739fdfaf6370\&variant=undefined}{F.D.A.
  Regulators' Self-Defense}
\item
  \href{https://www.nytimes3xbfgragh.onion/2020/09/09/upshot/coronavirus-surprise-test-fees.html?name=styln-coronavirus-markets\&region=TOP_BANNER\&block=storyline_menu_recirc\&action=click\&pgtype=Article\&impression_id=90447e10-f52e-11ea-a263-739fdfaf6370\&variant=undefined}{Surprise
  Test Fees}
\end{itemize}

Advertisement

\protect\hyperlink{after-top}{Continue reading the main story}

Supported by

\protect\hyperlink{after-sponsor}{Continue reading the main story}

\hypertarget{forgetting-contagion-stocks-recover-their-losses}{%
\section{Forgetting Contagion, Stocks Recover Their
Losses}\label{forgetting-contagion-stocks-recover-their-losses}}

Published June 8, 2020Updated June 26, 2020

\begin{itemize}
\item
\item
\item
\item
\item
\end{itemize}

\hypertarget{heres-what-you-need-to-know}{%
\subsubsection{Here's what you need to
know:}\label{heres-what-you-need-to-know}}

\begin{itemize}
\tightlist
\item
  \protect\hyperlink{link-7bd27140}{The S\&P 500 erases its losses for
  the year as Wall Street's rally continues.}
\item
  \protect\hyperlink{link-2fe46873}{The U.S. economy entered a recession
  in February.}
\item
  \protect\hyperlink{link-75b50c7b}{Roller-coaster trading for shares of
  Chesapeake, a troubled oil and gas company.}
\item
  \protect\hyperlink{link-726b3581}{3M sues a third-party sellers on
  Amazon over masks.}
\item
  \protect\hyperlink{link-2b1e48ba}{The numbers are ambiguous, but the
  jobs report wasn't rigged.}
\end{itemize}

\hypertarget{the-sp-500-erases-its-losses-for-the-year-as-wall-streets-rally-continues}{%
\subsection{The S\&P 500 erases its losses for the year as Wall Street's
rally
continues.}\label{the-sp-500-erases-its-losses-for-the-year-as-wall-streets-rally-continues}}

Stocks on Wall Street erased their losses for the year, a remarkable
milestone for a market that was reeling just a few months ago as
investors feared the damage caused by the coronavirus pandemic.

The S\&P 500 rose more than 1 percent on Monday, adding to a weekslong
rebound that has been fueled by hopes for a quick economic recovery,
significant intervention by the Federal Reserve and a disregard for the
serious risks that businesses and consumers still face.

A familiar list of companies has been leading the recent gains. Airlines
have been lifted by signs that domestic travel is starting to pick up,
and they rallied again on Monday along with \textbf{Boeing}. With oil
prices rebounding, crude briefly crossed above \$40 a barrel on Monday
for the first time in months. Energy companies have also surged.

\emph{{[}Follow our}
\href{https://www.nytimes3xbfgragh.onion/2020/06/26/business/stock-market-today-coronavirus.html}{\emph{live
stock market tracker}} \emph{and business news coverage.{]}}

Stocks have been on an upward trajectory for weeks as investors have
responded to signs around the world that the virus was abating. New
York, the center of the coronavirus outbreak in the United States, on
Monday began to lift some restrictions on construction, manufacturing
and retail operations.

Progress like that, and early evidence that it means people are
returning to work, helped fuel a nearly 5 percent gain in the S\&P 500
last week --- its biggest weekly run since early April. But the market's
rebound really began in March, after the Federal Reserve signaled its
willingness to funnel unlimited amounts of liquidity into financial
markets. Since then, stocks have risen more than 44 percent.

Investors have plenty of reasons to be wary, of course: A second wave of
the coronavirus outbreak that forces governments to clamp down on public
activity again, a premature end to government spending or a
slower-than-expected return of business could all dampen enthusiasm for
stocks.

According to
\href{https://www.nytimes3xbfgragh.onion/interactive/2020/us/coronavirus-us-cases.html}{data}
compiled by The New York Times, new infections are still increasing in
more than a third of states. Public officials are also wary of a surge
in new cases as
\href{https://www.nytimes3xbfgragh.onion/2020/06/07/us/Protest-coronavirus-george-floyd.html}{thousands
of protesters} across the country demonstrate against police brutality
after the death of George Floyd.

\hypertarget{the-us-economy-entered-a-recession-in-february}{%
\subsection{The U.S. economy entered a recession in
February.}\label{the-us-economy-entered-a-recession-in-february}}

The United States
\href{https://www.nytimes3xbfgragh.onion/2020/06/08/business/economy/us-economy-recession-2020.html}{officially
entered a recession} in February 2020, the committee that calls
downturns announced on Monday, marking the beginning of the first
economic downturn since the 2007 to 2009 slump.

The \href{https://www.nber.org/cycles.html}{National Bureau of Economic
Research} said that the economy hit its peak in February and had since
fallen into a downturn, as pandemic-related shutdowns tanked activity
and brought an end to a record-long expansion --- one that had lasted
128 months.

\hypertarget{longest-expansion-comes-to-an-end}{%
\subsection{Longest Expansion Comes to an
End}\label{longest-expansion-comes-to-an-end}}

The sharp decline in economic activity in February marked the end of the
longest expansion in the U.S. since at least 1854, according to the
National Bureau of Economic Research. Here are expansions compared with
G.D.P since the end of WWII.

Economic

expansions

+

15

\%

G.D.P.

45

24

36

12

120

128

months

39

106

58

92

73

+

10

Recessions

+

5

0

--

5

--

10

1950

'60

'70

'80

'90

2000

'10

'20

G.D.P.

Economic expansions

+

15

\%

45

39

24

106

36

58

12

92

120

73

128

months

+

10

Recessions

+

5

0

--

5

--

10

1950

'60

'70

'80

'90

2000

'10

'20

G.D.P.

Economic expansions

+

15

\%

45

39

24

106

36

58

12

92

120

months

73

128

+

10

Recessions

+

5

0

--

5

--

10

1950

1960

1970

1980

1990

2000

2010

2020

Notes: Data are quarterly changes in gross domestic product, seasonally
adjusted at annual rates, and the duration of business cycle expansion
in months.

Sources: Bureau of Economic Analysis; National Bureau of Economic
Research

By Karl Russell

Analysts often refer to recessions as two consecutive quarters of
contraction. The National Bureau of Economic Research, a nonprofit group
that tracks economic cycles in the United States, formally determines
when recessions begin and end based on a range of factors, most
importantly gross domestic product and employment. Most economists
expect that this recession will be both deep and short, with growth
rebounding as state economies reopen and the world figures out how to
function amid the coronavirus pandemic.

``The unprecedented magnitude of the decline in employment and
production, and its broad reach across the entire economy, warrants the
designation of this episode as a recession, even if it turns out to be
briefer than earlier contractions,'' the bureau said in a statement.

Globally, ``this is almost certainly the deepest recession'' since at
least the Second World War, Jan Hatzius, Goldman Sachs chief economist,
wrote in a note on Monday. But it is also probably the shortest: He
noted that the bureau's database showed no other recession that had
lasted less than six months in records dating back to the mid-1800s.

Most economists believe the recovery has already begun. On Friday, after
weeks of data depicting enormous economic destruction, the Labor
Department reported that the unemployment rate fell and
\href{https://www.nytimes3xbfgragh.onion/2020/06/05/business/economy/jobs-report.html}{employers
added 2.5 million jobs in May}. But tens of millions are still out of
work, and the unemployment rate, which fell to 13.3 percent from 14.7
percent in April, remains worse than in any previous postwar recession.

\hypertarget{roller-coaster-trading-for-shares-of-chesapeake-a-troubled-oil-and-gas-company}{%
\subsection{Roller-coaster trading for shares of Chesapeake, a troubled
oil and gas
company.}\label{roller-coaster-trading-for-shares-of-chesapeake-a-troubled-oil-and-gas-company}}

\includegraphics{https://static01.graylady3jvrrxbe.onion/images/2020/06/08/business/08virus-markets-briefing-ce/merlin_172039116_8120c64f-ebbc-402f-87ce-4556bb218b00-articleLarge.jpg?quality=75\&auto=webp\&disable=upscale}

Stock in Chesapeake Energy, the troubled oil and gas company, made moves
on Monday that were astonishing even in a period in which the stock
market has been rocked with volatility.

The company's shares soared 182 percent during regular trading but then
plunged more than 30 percent in after-hours trading. The free fall was
most likely caused in part by a
\href{https://www.bloomberg.com/news/articles/2020-06-08/chesapeake-energy-plans-bankruptcy-that-may-give-lenders-control?sref=tnuvvlQG}{Bloomberg
News report} that Chesapeake was preparing to file for bankruptcy.

\hypertarget{latest-updates-the-coronavirus-outbreak-and-the-economy}{%
\section{\texorpdfstring{\href{https://www.nytimes3xbfgragh.onion/live/2020/09/11/business/stock-market-today-coronavirus?action=click\&pgtype=Article\&state=default\&region=MAIN_CONTENT_1\&context=storylines_live_updates}{Latest
Updates: The Coronavirus Outbreak and the
Economy}}{Latest Updates: The Coronavirus Outbreak and the Economy}}\label{latest-updates-the-coronavirus-outbreak-and-the-economy}}

\href{https://www.nytimes3xbfgragh.onion/live/2020/09/11/business/stock-market-today-coronavirus?action=click\&pgtype=Article\&state=default\&region=MAIN_CONTENT_1\&context=storylines_live_updates\#the-nyse-may-move-its-data-center-out-of-new-jersey-in-response-to-a-proposed-tax}{23h
ago}

\href{https://www.nytimes3xbfgragh.onion/live/2020/09/11/business/stock-market-today-coronavirus?action=click\&pgtype=Article\&state=default\&region=MAIN_CONTENT_1\&context=storylines_live_updates\#the-nyse-may-move-its-data-center-out-of-new-jersey-in-response-to-a-proposed-tax}{The
N.Y.S.E. may move its data center out of New Jersey in response to a
proposed tax.}

\href{https://www.nytimes3xbfgragh.onion/live/2020/09/11/business/stock-market-today-coronavirus?action=click\&pgtype=Article\&state=default\&region=MAIN_CONTENT_1\&context=storylines_live_updates\#the-federal-budget-deficit-hit-3-trillion-as-of-august}{26h
ago}

\href{https://www.nytimes3xbfgragh.onion/live/2020/09/11/business/stock-market-today-coronavirus?action=click\&pgtype=Article\&state=default\&region=MAIN_CONTENT_1\&context=storylines_live_updates\#the-federal-budget-deficit-hit-3-trillion-as-of-august}{The
federal budget deficit hit \$3 trillion as of August.}

\href{https://www.nytimes3xbfgragh.onion/live/2020/09/11/business/stock-market-today-coronavirus?action=click\&pgtype=Article\&state=default\&region=MAIN_CONTENT_1\&context=storylines_live_updates\#warner-bros-pushes-the-release-of-wonder-woman-1984-to-christmas}{26h
ago}

\href{https://www.nytimes3xbfgragh.onion/live/2020/09/11/business/stock-market-today-coronavirus?action=click\&pgtype=Article\&state=default\&region=MAIN_CONTENT_1\&context=storylines_live_updates\#warner-bros-pushes-the-release-of-wonder-woman-1984-to-christmas}{Warner
Bros. pushes the release of `Wonder Woman 1984' to Christmas.}

\href{https://www.nytimes3xbfgragh.onion/live/2020/09/11/business/stock-market-today-coronavirus?action=click\&pgtype=Article\&state=default\&region=MAIN_CONTENT_1\&context=storylines_live_updates}{See
more updates}

More live coverage:
\href{https://www.nytimes3xbfgragh.onion/2020/09/11/world/covid-19-coronavirus.html?action=click\&pgtype=Article\&state=default\&region=MAIN_CONTENT_1\&context=storylines_live_updates}{Global}

The company has a heavy debt load that it will struggle to repay at a
time when oil prices, even after a recent rally, are well below levels
reached in recent years. Chesapeake warned in a
\href{https://www.sec.gov/Archives/edgar/data/895126/000089512620000115/chk-2020033110q.htm\#s8319C3A53B5D5EA88220B634C1A28003}{securities
filing last month} that it may reorganize under bankruptcy protection.
At its after-hours trading price, Chesapeake has a stock market value
just above \$400 million.

Typically, shareholders get wiped out in bankruptcy, but in some cases,
like that of Pacific Gas \& Electric, the California utility, the shares
retain much of their value. But this is unlikely to be the outcome for
Chesapeake, judging by the price of its bonds, which are trading below
10 percent of their full value.

Bondholders come before shareholders when claiming assets of a bankrupt
company, so the fact that the bonds are trading at very low prices is a
strong signal that shareholders will get nothing.

Chesapeake was a pioneer of hydraulic fracturing, the drilling method
that enabled drillers to get at vast reserves of oil and gas in the
United States. Its former chief executive, Aubrey McClendon,
\href{https://www.nytimes3xbfgragh.onion/2016/03/04/business/energy-environment/aubrey-mcclendon-restless-and-reckless-wildcatter-was-deal-making-to-the-end.html}{died
in a car crash in 2016}.

\hypertarget{3m-sues-a-third-party-sellers-on-amazon-over-masks}{%
\subsection{3M sues a third-party sellers on Amazon over
masks.}\label{3m-sues-a-third-party-sellers-on-amazon-over-masks}}

Image

3M sued three third-party sellers on Amazon for trademark infringement
over the sales of N95 respirators.Credit...Nicholas Pfosi/Reuters

The industrial conglomerate 3M filed a trademark infringement lawsuit in
federal court in California on Monday, alleging price-gouging and
bait-and-switch sales of 3M respirators from third-party Amazon sellers.

The complaint claims that three third-party sellers --- all believed to
be owned and operated by a California resident named Mao Yu --- began in
late February to sell what were advertised to be 3M-branded N95 masks on
Amazon. The sellers charged for roughly 18 times 3M's \$1.27 list price
for the respirators. Buyers spent more than \$350,000 for such masks,
and sometimes received fewer masks than promised or masks that were
damaged or tampered with, according to the suit, which was filed in the
United States District Court for the Central District of California.

``By selling and delivering to customers counterfeit, damaged,
deficient, or otherwise altered respirators and engaging in
price-gouging, Defendants caused irreparable damage to 3M's
reputation,'' the suit states.

The defendants in the case could not be reached for comment.

3M, based in a suburb of Saint Paul, Minn., has filed 12 other such
suits as part of an effort to combat fraud, price-gouging and
counterfeiting tied its respirators and other high-demand health
products as a result of the coronavirus outbreak.

\hypertarget{the-numbers-are-ambiguous-but-the-jobs-report-wasnt-rigged}{%
\subsection{The numbers are ambiguous, but the jobs report wasn't
rigged.}\label{the-numbers-are-ambiguous-but-the-jobs-report-wasnt-rigged}}

When the Labor Department reported on Friday that
\href{https://www.nytimes3xbfgragh.onion/2020/06/06/business/economy/jobs-report-minorities.html}{employers
had added jobs in May} and that the unemployment rate had unexpectedly
fallen, economists were surprised.

Others had a different reaction: suspicion.

Social media sites over the weekend lit up with posts,
\href{https://twitter.com/GovHowardDean/status/1269325065310613505?s=20}{some}
from
\href{https://twitter.com/tedlieu/status/1269140901558800385}{Democratic
politicians}, saying the jobs numbers were misleading at best and
possibly manipulated.

For many, those suspicions seemed confirmed by a note,
\href{https://www.bls.gov/news.release/empsit.nr0.htm}{deep within the
report}, saying some workers had been improperly counted as employed
rather than unemployed. If those workers had been classified correctly,
the unemployment rate would have been about 16.4 percent in May, rather
than the official rate of 13.3 percent (although it still would have
been lower than in April).

But economists across the political spectrum say it would be all but
impossible to manipulate the jobs numbers undetected. And while there is
no question that the speed and severity of the economic collapse has
made gathering and interpreting economic data unusually difficult, they
say the Bureau of Labor Statistics --- the Labor Department office that
produces the jobs report --- has done an admirable job both ensuring
that the numbers are reliable and publicly identifying potential issues.

\hypertarget{the-travel-business-is-picking-up-as-americans-look-for-escape}{%
\subsection{The travel business is picking up as Americans look for
escape.}\label{the-travel-business-is-picking-up-as-americans-look-for-escape}}

Image

The Fontainebleau Miami Beach, a resort that had closed because of the
coronavirus pandemic, is enjoying brisk business since it reopened on
June 1.Credit...Josh Ritchie for The New York Times

The nation's largest airlines are preparing for a limited rebound next
month as more Americans book vacations in places like Florida and the
mountains and national parks in the West.

That resurgence would offer some hope to the travel industry, which
racked up billions of dollars in losses as tourists and businesspeople
canceled trips in the last three months because of the coronavirus
epidemic.

After cratering in April, the number of travelers and airline and
airport employees
\href{https://www.tsa.gov/coronavirus/passenger-throughput}{filtering
through the Transportation Security Administration's airport
checkpoints} has steadily climbed in recent weeks. The low point was
April 14, when the agency screened fewer than 90,000 people, just 4
percent of those screened the same date last year. On Sunday, the agency
screened more than 440,000 people, about 17 percent of last year's
number and the best day since March.

Investors appear to have noticed those numbers, and airline stock prices
have surged. \textbf{American Airlines} is up nearly 90 percent since
Monday morning last week, \textbf{United Airlines} is more than 70
percent higher, and \textbf{Delta Air Lines} is up more than 45 percent.

\hypertarget{scientists-are-looking-to-curb-the-virus-spread-at-a-less-severe-economic-cost}{%
\subsection{Scientists are looking to curb the virus spread at a less
severe economic
cost.}\label{scientists-are-looking-to-curb-the-virus-spread-at-a-less-severe-economic-cost}}

Employment in each ZIP code

209,000

BRONX

100,000

45,000

20,000

MANHATTAN

QUEENS

10,000

0

BROOKLYN

STATEN ISLAND

Covid-19 infections

4,400

Number of cases in each ZIP code as of May 28

BRONX

2,500

1,700

1,00

MANHATTAN

QUEENS

400

0

BROOKLYN

STATEN ISLAND

Employment in

each ZIP code

Covid-19

infections

BRONX

BRONX

QUEENS

QUEENS

MANHATTAN

MANHATTAN

BROOKLYN

BROOKLYN

STATEN ISLAND

STATEN ISLAND

Employment in thousands

Number of cases in each ZIP code as of May 28

0

10

20

45

100

209

0

400

1,000

1,700

2,500

4,400

Employment in

each ZIP code

Covid-19

infections

209,000

4,400

Number of cases in each ZIP code as of May 28

100,000

2,500

BRONX

BRONX

45,000

1,700

20,000

1,000

MANHATTAN

MANHATTAN

QUEENS

QUEENS

10,000

400

0

0

BROOKLYN

BROOKLYN

STATEN ISLAND

STATEN ISLAND

Sources: John R. Birge, Ozan Candogan (Univ. of Chicago) and Yiding Feng
(Northwestern Univ.); New York City Department of Health and Mental
Hygiene

By Karl Russell

As coronavirus cases took off in New York in March, Gov. Andrew M. Cuomo
imposed a lockdown of nonessential businesses to slow the spread of the
coronavirus, calling it
``\href{https://www.nytimes3xbfgragh.onion/2020/03/20/us/ny-ca-stay-home-order.html}{the
most drastic action we can take}.''

Now researchers say more targeted approaches might have protected public
health with less economic pain.

A study of New York City found that the economic cost could have been
reduced by a third or more by strategically choosing neighborhoods to
close, calibrating the risk of infection for local residents and workers
with the impact on local jobs.

``The blunt instrument of a uniform policy causes more economic and
related health harm than is necessary to accomplish the same goal of not
increasing infections,'' said John Birge, a mathematician at the
University of Chicago who was an author of the study.

Other researchers are taking on the problem by assessing the relative
level of risk posed by different businesses.

Businesses in New York City, where an initial phase of reopening began
on Monday, have been mostly shut down for 11 weeks.

With daily deaths abating in New York and many other parts of the
country, and cities and states easing lockdowns, researchers are
beginning to assess alternative strategies to manage the spread of the
virus.

\hypertarget{us-hospitals-received-billions-in-bailout-grants-as-ceos-got-millions}{%
\subsection{U.S. hospitals received billions in bailout grants as
C.E.O.s got
millions.}\label{us-hospitals-received-billions-in-bailout-grants-as-ceos-got-millions}}

Image

HCA Healthcare, based in Nashville, received about \$1 billion in
federal bailout funds, part of an effort to stabilize hospitals during
the pandemic.Credit...William DeShazer for The New York Times

As some of the wealthiest health care companies in the United States
received billions of dollars in taxpayer funds to help them cope with
lost revenue from the pandemic, they laid off or cut the pay of tens of
thousands of doctors, nurses and lower-paid workers, while continuing to
pay their top executives millions.

\href{https://www.nytimes3xbfgragh.onion/2020/06/08/business/hospitals-bailouts-ceo-pay.html}{The
New York Times analyzed tax and securities filings} by 60 of the
country's largest hospital chains, which together have received more
than \$15 billion in emergency funds through the economic stimulus
package in the federal CARES Act.

The hospitals --- including publicly traded juggernauts like
\textbf{HCA} and \textbf{Tenet Healthcare}, elite nonprofits like the
\textbf{Mayo Clinic}, and regional chains with thousands of beds --- are
collectively
\href{https://www.nytimes3xbfgragh.onion/2020/05/25/business/coronavirus-hospitals-bailout.html}{sitting
on tens of billions of dollars} of cash reserves that are supposed to
help them weather an unanticipated storm. And together, they awarded the
five highest-paid officials at each chain about \$874 million in the
most recent year for which they have disclosed their finances.

At least 36 of those hospital chains have laid off, furloughed or
reduced the pay of employees as they try to save money during the
pandemic.

More than a dozen workers at the wealthy hospitals said in interviews
that their employers had put the heaviest financial burdens on
front-line staff, including **** low-paid cafeteria workers, janitors
and nursing assistants. They said pay cuts and furloughs made it even
harder for medical workers to do their jobs, forcing them to treat more
patients in less time.

\hypertarget{corporate-america-has-failed-black-america}{%
\subsection{`Corporate America has failed black
America.'}\label{corporate-america-has-failed-black-america}}

Image

``Corporate America can no longer get away with token responses to
systemic problems,'' said Darren Walker, the president of the Ford
Foundation.Credit...Guerin Blask for The New York Times

In the past week, it has seemed like every major company has publicly
condemned racism. All-black squares cover corporate Instagram.
Executives have made multimillion-dollar pledges to anti-discrimination
efforts and programs to support black businesses.

Yet many of the same companies expressing solidarity have contributed to
systemic inequality, targeted the black community with unhealthy
products and services, and failed to hire, promote and fairly compensate
black men and women,
\href{https://www.nytimes3xbfgragh.onion/2020/06/06/business/corporate-america-has-failed-black-america.html}{David
Gelles writes}.

``Corporate America has failed black America,'' said Darren Walker, the
president of the \textbf{Ford Foundation} and a member of the board of
\textbf{Pepsi}, and who is black. ``Even after a generation of Ivy
League educations and extraordinary talented African-Americans going
into corporate America, we seem to have hit a wall.''

With dozens of cities protesting the violent deaths of George Floyd,
Ahmaud Arbery, Breonna Taylor and others, a national conversation about
racism is underway. For black executives, who have spent their lives
excelling at business while overcoming structural discrimination, the
killings and ensuing protests have unleashed an outpouring of emotion.
Many are speaking candidly about their private fears, as well as their
disappointment with the corporate apparatus that made them stars.

Robert F. Smith, a private equity billionaire and the
\href{https://www.nytimes3xbfgragh.onion/2019/05/19/business/robert-f-smith-morehouse-vista-equity.html}{richest
black man in America}, said he had been overwhelmed by conflicting
feelings. ``I am saddened, I am angry, I am upset and I am determined,''
he said. ``I run through that wave of emotions every minute.''

\hypertarget{catch-up-heres-what-else-is-happening}{%
\subsection{Catch up: Here's what else is
happening.}\label{catch-up-heres-what-else-is-happening}}

\begin{itemize}
\item
  \textbf{Dunkin' Donuts} said on Monday that it planned to hire up to
  25,000 new workers at its franchises to deal with an influx of
  customers as states start to reopen. Dunkin', which has 8,500
  restaurants in the United States, said about 90 percent of its
  locations were now open.
\item
  \textbf{BP} said Monday that it planned to eliminate 10,000 jobs ---
  nearly 15 percent of the company's total work force --- with most cuts
  coming by the end of the year. The company's chief executive, Bernard
  Looney, said in a companywide email that the cuts were needed to stem
  losses arising from the coronavirus pandemic as well as to create a
  leaner company to achieve his ambitions to sharply reduce BP's carbon
  dioxide emissions.
\end{itemize}

Reporting was contributed by Niraj Chokshi, Eduardo Porter, Peter Eavis,
Jeanna Smialek, Jessica Silver-Greenberg, Jesse Drucker, David Enrich,
Patricia Cohen, Stanley Reed, Ben Casselman, Jason Karaian, Jack Ewing,
David Gelles, Ann Carrns, Matt Phillips, Paul Sullivan, Carlos Tejada,
Katie Robertson and Kevin Granville.

Advertisement

\protect\hyperlink{after-bottom}{Continue reading the main story}

\hypertarget{site-index}{%
\subsection{Site Index}\label{site-index}}

\hypertarget{site-information-navigation}{%
\subsection{Site Information
Navigation}\label{site-information-navigation}}

\begin{itemize}
\tightlist
\item
  \href{https://help.nytimes3xbfgragh.onion/hc/en-us/articles/115014792127-Copyright-notice}{©~2020~The
  New York Times Company}
\end{itemize}

\begin{itemize}
\tightlist
\item
  \href{https://www.nytco.com/}{NYTCo}
\item
  \href{https://help.nytimes3xbfgragh.onion/hc/en-us/articles/115015385887-Contact-Us}{Contact
  Us}
\item
  \href{https://www.nytco.com/careers/}{Work with us}
\item
  \href{https://nytmediakit.com/}{Advertise}
\item
  \href{http://www.tbrandstudio.com/}{T Brand Studio}
\item
  \href{https://www.nytimes3xbfgragh.onion/privacy/cookie-policy\#how-do-i-manage-trackers}{Your
  Ad Choices}
\item
  \href{https://www.nytimes3xbfgragh.onion/privacy}{Privacy}
\item
  \href{https://help.nytimes3xbfgragh.onion/hc/en-us/articles/115014893428-Terms-of-service}{Terms
  of Service}
\item
  \href{https://help.nytimes3xbfgragh.onion/hc/en-us/articles/115014893968-Terms-of-sale}{Terms
  of Sale}
\item
  \href{https://spiderbites.nytimes3xbfgragh.onion}{Site Map}
\item
  \href{https://help.nytimes3xbfgragh.onion/hc/en-us}{Help}
\item
  \href{https://www.nytimes3xbfgragh.onion/subscription?campaignId=37WXW}{Subscriptions}
\end{itemize}
