Sections

SEARCH

\protect\hyperlink{site-content}{Skip to
content}\protect\hyperlink{site-index}{Skip to site index}

\href{https://www.nytimes3xbfgragh.onion/section/business}{Business}

\href{https://myaccount.nytimes3xbfgragh.onion/auth/login?response_type=cookie\&client_id=vi}{}

\href{https://www.nytimes3xbfgragh.onion/section/todayspaper}{Today's
Paper}

\href{/section/business}{Business}\textbar{}Despite Recession, Stock
Markets Turn Positive for the Year

\url{https://nyti.ms/3heuBqD}

\begin{itemize}
\item
\item
\item
\item
\item
\end{itemize}

\hypertarget{the-coronavirus-outbreak}{%
\subsubsection{\texorpdfstring{\href{https://www.nytimes3xbfgragh.onion/news-event/coronavirus?name=styln-coronavirus-markets\&region=TOP_BANNER\&block=storyline_menu_recirc\&action=click\&pgtype=Article\&impression_id=ed062330-f52c-11ea-a3d9-4be88c42bbf1\&variant=undefined}{The
Coronavirus
Outbreak}}{The Coronavirus Outbreak}}\label{the-coronavirus-outbreak}}

\begin{itemize}
\tightlist
\item
  live\href{https://www.nytimes3xbfgragh.onion/2020/09/12/world/covid-19-coronavirus.html?name=styln-coronavirus-markets\&region=TOP_BANNER\&block=storyline_menu_recirc\&action=click\&pgtype=Article\&impression_id=ed062331-f52c-11ea-a3d9-4be88c42bbf1\&variant=undefined}{Latest
  Updates}
\item
  \href{https://www.nytimes3xbfgragh.onion/interactive/2020/us/coronavirus-us-cases.html?name=styln-coronavirus-markets\&region=TOP_BANNER\&block=storyline_menu_recirc\&action=click\&pgtype=Article\&impression_id=ed062332-f52c-11ea-a3d9-4be88c42bbf1\&variant=undefined}{Maps
  and Cases}
\item
  \href{https://www.nytimes3xbfgragh.onion/interactive/2020/science/coronavirus-vaccine-tracker.html?name=styln-coronavirus-markets\&region=TOP_BANNER\&block=storyline_menu_recirc\&action=click\&pgtype=Article\&impression_id=ed062333-f52c-11ea-a3d9-4be88c42bbf1\&variant=undefined}{Vaccine
  Tracker}
\item
  \href{https://www.nytimes3xbfgragh.onion/2020/09/10/us/politics/fda-coronavirus-vaccine.html?name=styln-coronavirus-markets\&region=TOP_BANNER\&block=storyline_menu_recirc\&action=click\&pgtype=Article\&impression_id=ed064a40-f52c-11ea-a3d9-4be88c42bbf1\&variant=undefined}{F.D.A.
  Regulators' Self-Defense}
\item
  \href{https://www.nytimes3xbfgragh.onion/2020/09/09/upshot/coronavirus-surprise-test-fees.html?name=styln-coronavirus-markets\&region=TOP_BANNER\&block=storyline_menu_recirc\&action=click\&pgtype=Article\&impression_id=ed064a41-f52c-11ea-a3d9-4be88c42bbf1\&variant=undefined}{Surprise
  Test Fees}
\end{itemize}

Advertisement

\protect\hyperlink{after-top}{Continue reading the main story}

Supported by

\protect\hyperlink{after-sponsor}{Continue reading the main story}

\hypertarget{despite-recession-stock-markets-turn-positive-for-the-year}{%
\section{Despite Recession, Stock Markets Turn Positive for the
Year}\label{despite-recession-stock-markets-turn-positive-for-the-year}}

The S\&P 500 climbed back above where it began the year on the same day
that economists said the United States fell into a recession in
February.

By \href{https://www.nytimes3xbfgragh.onion/by/matt-phillips}{Matt
Phillips}

\begin{itemize}
\item
  June 8, 2020
\item
  \begin{itemize}
  \item
  \item
  \item
  \item
  \item
  \end{itemize}
\end{itemize}

And just like that, we're back to where we started.

The S\&P 500, a leading stock market index, on Monday climbed back above
where it began the year --- before the pandemic brought the United
States economy to a juddering halt, before more than 110,000 Americans
died from the coronavirus, and before the police killing of George Floyd
in Minneapolis set off nationwide protests.

A
\href{https://www.nytimes3xbfgragh.onion/2020/06/08/business/stock-market-today-coronavirus.html}{late-day
rally} pushed the index into positive territory for 2020, effectively
erasing one of the most tumultuous periods in recent American history
from the financial record. Stocks rose 1.2 percent --- on the same day
that economists said the
\href{https://www.nytimes3xbfgragh.onion/2020/06/08/business/economy/us-economy-recession-2020.html}{United
States fell into a recession in February}.

Even though the economy has begun to reopen, it is hard to overstate how
disastrous the past three months have been and what the long-term
consequences are for everything, from the nature of work to the future
of certain industries. Tens of millions of people are unemployed,
corporate earnings have plummeted, and industries such as tourism,
retail and entertainment might never fully recover from the blow dealt
to their businesses.

But in the stock market, it's like the pandemic never happened.

``Investors seem to have decided that the past three months were just a
bad dream that we're waking up from,'' said Scott Clemons, chief
investment strategist for private banking at Brown Brothers Harriman, an
investment bank.

After an initial few weeks of volatility, when the market dropped 34
percent, it has become inured to the near-daily drumbeat of bad news.
When the Commerce Department announced on April 29 that the economy
shrank at a nearly 5 percent annual rate, its
\href{https://www.nytimes3xbfgragh.onion/2020/04/29/business/economy/us-gdp.html}{fastest
drop since the 2008 recession}, stocks rose 2.7 percent. A month ago,
when the Bureau of Labor Statistics published what was essentially the
\href{https://www.nytimes3xbfgragh.onion/interactive/2020/05/08/business/economy/april-jobs-report.html}{worst
employment report on record} --- showing that more than 20 million jobs
disappeared in April as unemployment surged to 14.7 percent, the highest
since the Great Depression --- stocks rose 1.7 percent.

\hypertarget{latest-updates-the-coronavirus-outbreak-and-the-economy}{%
\section{\texorpdfstring{\href{https://www.nytimes3xbfgragh.onion/live/2020/09/11/business/stock-market-today-coronavirus?action=click\&pgtype=Article\&state=default\&region=MAIN_CONTENT_1\&context=storylines_live_updates}{Latest
Updates: The Coronavirus Outbreak and the
Economy}}{Latest Updates: The Coronavirus Outbreak and the Economy}}\label{latest-updates-the-coronavirus-outbreak-and-the-economy}}

\href{https://www.nytimes3xbfgragh.onion/live/2020/09/11/business/stock-market-today-coronavirus?action=click\&pgtype=Article\&state=default\&region=MAIN_CONTENT_1\&context=storylines_live_updates\#the-nyse-may-move-its-data-center-out-of-new-jersey-in-response-to-a-proposed-tax}{23h
ago}

\href{https://www.nytimes3xbfgragh.onion/live/2020/09/11/business/stock-market-today-coronavirus?action=click\&pgtype=Article\&state=default\&region=MAIN_CONTENT_1\&context=storylines_live_updates\#the-nyse-may-move-its-data-center-out-of-new-jersey-in-response-to-a-proposed-tax}{The
N.Y.S.E. may move its data center out of New Jersey in response to a
proposed tax.}

\href{https://www.nytimes3xbfgragh.onion/live/2020/09/11/business/stock-market-today-coronavirus?action=click\&pgtype=Article\&state=default\&region=MAIN_CONTENT_1\&context=storylines_live_updates\#the-federal-budget-deficit-hit-3-trillion-as-of-august}{25h
ago}

\href{https://www.nytimes3xbfgragh.onion/live/2020/09/11/business/stock-market-today-coronavirus?action=click\&pgtype=Article\&state=default\&region=MAIN_CONTENT_1\&context=storylines_live_updates\#the-federal-budget-deficit-hit-3-trillion-as-of-august}{The
federal budget deficit hit \$3 trillion as of August.}

\href{https://www.nytimes3xbfgragh.onion/live/2020/09/11/business/stock-market-today-coronavirus?action=click\&pgtype=Article\&state=default\&region=MAIN_CONTENT_1\&context=storylines_live_updates\#warner-bros-pushes-the-release-of-wonder-woman-1984-to-christmas}{26h
ago}

\href{https://www.nytimes3xbfgragh.onion/live/2020/09/11/business/stock-market-today-coronavirus?action=click\&pgtype=Article\&state=default\&region=MAIN_CONTENT_1\&context=storylines_live_updates\#warner-bros-pushes-the-release-of-wonder-woman-1984-to-christmas}{Warner
Bros. pushes the release of `Wonder Woman 1984' to Christmas.}

\href{https://www.nytimes3xbfgragh.onion/live/2020/09/11/business/stock-market-today-coronavirus?action=click\&pgtype=Article\&state=default\&region=MAIN_CONTENT_1\&context=storylines_live_updates}{See
more updates}

More live coverage:
\href{https://www.nytimes3xbfgragh.onion/2020/09/11/world/covid-19-coronavirus.html?action=click\&pgtype=Article\&state=default\&region=MAIN_CONTENT_1\&context=storylines_live_updates}{Global}

So, why is the market behaving this way?

In large part, it was the actions of the federal government. Early on,
the Federal Reserve stretched its financial safety net wide,
\href{https://www.nytimes3xbfgragh.onion/2020/05/19/business/too-big-to-fail-wall-street-businesses.html}{announcing
it would provide a backstop} by using its emergency lending powers to
buy assets --- from municipal to corporate debt --- with newly printed
money. Also, it began snapping up government-backed bonds through a
newly unlimited buying campaign. That had the effect of keeping bond
prices up and yields, which move in the opposite direction of prices,
low. And so investors, looking for better returns, began putting their
money into the stock market instead, creating upward pressure on prices.

``It's the only way that you can kind of explain what's going on, is
that people really do believe that there is no downside in equity
ownership,'' said James Montier, a member of the asset allocation team
at Grantham, Mayo, Van Otterloo \& Company, a Boston-based asset
management company.

Since March 23, the Dow Jones industrial average has soared 48 percent.
The Nasdaq composite index, which is heavily weighted toward technology,
is up 45 percent and closed at a record high on Monday, as investors bet
that tech behemoths like Amazon and Microsoft were well positioned to
benefit from stay-at-home orders around the country. The S\&P 500 is
also up nearly 45 percent.

``I understand fully the recovery in the market, I just think it's ahead
of schedule,'' said Leon Cooperman, the founder of the hedge fund Omega
Advisors, which
\href{https://www.nytimes3xbfgragh.onion/2018/07/23/business/dealbook/leon-cooperman-hedge-fund.html}{in
2018 announced it would convert to a family office to mainly manage the
billionaire's personal fortune}. ``It's ahead of schedule because of the
government's policy of giving out free money.''

Mr. Cooperman was referring to the flood of money --- from both the Fed
and the government itself --- that has been pumped into the economy and
markets.

Since the Fed first took steps to stabilize the markets in March, it has
created roughly \$2.9 trillion, the vast majority of which has gone into
financial markets. Separately, the federal government has said it would
borrow a record-breaking \$3 trillion from April to June, much of which
will be channeled to businesses and consumers to keep them afloat during
the shutdown.

The biggest winners in the stock market rally have been companies whose
very existence earlier appeared imperiled by the crisis, with investors
now swooping in to buy the most battered shares in hopes of generating
the biggest gains.

As oil prices stabilized since the worst of the sell-off this year, the
stock price of the Apache Corporation, an oil driller, quadrupled. Stock
in the oil field services giant Halliburton has tripled. (Both still
remain down for the year.) The cruise operators Norwegian and Royal
Caribbean are both up more than 150 percent. The S\&P's energy sector
stocks have risen more than 90 percent, while consumer discretionary
stocks and financials are both up roughly 50 percent.

Although it's well known that market activity is a leading indicator of
economic recovery --- and investors right now are excited about the
reopening of states and
\href{https://www.nytimes3xbfgragh.onion/2020/06/05/business/economy/jobs-report.html}{the
recent May jobs report} --- skeptics caution that investors may have
become overly bullish.

``Right now the market thinks we'll have a V-shaped recovery and a
vaccine by the end of the year and I think both of those views are too
optimistic,'' said Byron Wien, a longtime market observer and the vice
chairman of the private wealth group at Blackstone.

While Mr. Wien believes the economy is recovering, he thinks it will be
a slow return to normal.

Wall Street analysts don't expect that corporate profits for S\&P 500
companies will return to 2019 levels until 2021. But the market rally
has effectively already priced in all those gains, in part, some say,
because of the government actions. Even after accounting for the
government's support, there are significant risks facing investors that
could stop S\&P 500 companies from generating the profits they did
before the virus reached American shores.

In recent years, large companies have bought back hundreds of billions
of dollars' worth of stock through share repurchase programs. Such
programs helped prop up stock prices. But as companies become more
cautious, analysts expect them to instead conserve their cash, removing
a key support for share prices.

Economic and political tensions between the United States and China ---
the world's two largest economies --- also remain high, after the two
superpowers engaged in a disruptive on-again-off-again trade war over
the last two years. A renewed flurry of tariffs could further complicate
the recovery for large American corporations as well as the global
economy. Such tensions may also be more likely to re-emerge ahead of
what could be a contentious presidential election in November.

Then, there is the prospect of a second wave of the pandemic, which
could set back the economy once more.

But despite these potential challenges, market investors appear almost
wholly unconcerned.

``I think the market is best described as the way people think about
second marriages,'' said Mr. Wien, 87 years old, who has been watching
stocks on Wall Street since 1965. ``It's a triumph of hope over
experience.''

Advertisement

\protect\hyperlink{after-bottom}{Continue reading the main story}

\hypertarget{site-index}{%
\subsection{Site Index}\label{site-index}}

\hypertarget{site-information-navigation}{%
\subsection{Site Information
Navigation}\label{site-information-navigation}}

\begin{itemize}
\tightlist
\item
  \href{https://help.nytimes3xbfgragh.onion/hc/en-us/articles/115014792127-Copyright-notice}{©~2020~The
  New York Times Company}
\end{itemize}

\begin{itemize}
\tightlist
\item
  \href{https://www.nytco.com/}{NYTCo}
\item
  \href{https://help.nytimes3xbfgragh.onion/hc/en-us/articles/115015385887-Contact-Us}{Contact
  Us}
\item
  \href{https://www.nytco.com/careers/}{Work with us}
\item
  \href{https://nytmediakit.com/}{Advertise}
\item
  \href{http://www.tbrandstudio.com/}{T Brand Studio}
\item
  \href{https://www.nytimes3xbfgragh.onion/privacy/cookie-policy\#how-do-i-manage-trackers}{Your
  Ad Choices}
\item
  \href{https://www.nytimes3xbfgragh.onion/privacy}{Privacy}
\item
  \href{https://help.nytimes3xbfgragh.onion/hc/en-us/articles/115014893428-Terms-of-service}{Terms
  of Service}
\item
  \href{https://help.nytimes3xbfgragh.onion/hc/en-us/articles/115014893968-Terms-of-sale}{Terms
  of Sale}
\item
  \href{https://spiderbites.nytimes3xbfgragh.onion}{Site Map}
\item
  \href{https://help.nytimes3xbfgragh.onion/hc/en-us}{Help}
\item
  \href{https://www.nytimes3xbfgragh.onion/subscription?campaignId=37WXW}{Subscriptions}
\end{itemize}
