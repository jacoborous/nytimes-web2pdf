Sections

SEARCH

\protect\hyperlink{site-content}{Skip to
content}\protect\hyperlink{site-index}{Skip to site index}

\href{https://www.nytimes3xbfgragh.onion/section/style}{Style}

\href{https://myaccount.nytimes3xbfgragh.onion/auth/login?response_type=cookie\&client_id=vi}{}

\href{https://www.nytimes3xbfgragh.onion/section/todayspaper}{Today's
Paper}

\href{/section/style}{Style}\textbar{}So You're Thinking About Riding a
Bike

\url{https://nyti.ms/2UX4oDk}

\begin{itemize}
\item
\item
\item
\item
\item
\end{itemize}

\href{https://www.nytimes3xbfgragh.onion/spotlight/at-home?action=click\&pgtype=Article\&state=default\&region=TOP_BANNER\&context=at_home_menu}{At
Home}

\begin{itemize}
\tightlist
\item
  \href{https://www.nytimes3xbfgragh.onion/2020/09/07/travel/route-66.html?action=click\&pgtype=Article\&state=default\&region=TOP_BANNER\&context=at_home_menu}{Cruise
  Along: Route 66}
\item
  \href{https://www.nytimes3xbfgragh.onion/2020/09/04/dining/sheet-pan-chicken.html?action=click\&pgtype=Article\&state=default\&region=TOP_BANNER\&context=at_home_menu}{Roast:
  Chicken With Plums}
\item
  \href{https://www.nytimes3xbfgragh.onion/2020/09/04/arts/television/dark-shadows-stream.html?action=click\&pgtype=Article\&state=default\&region=TOP_BANNER\&context=at_home_menu}{Watch:
  Dark Shadows}
\item
  \href{https://www.nytimes3xbfgragh.onion/interactive/2020/at-home/even-more-reporters-editors-diaries-lists-recommendations.html?action=click\&pgtype=Article\&state=default\&region=TOP_BANNER\&context=at_home_menu}{Explore:
  Reporters' Google Docs}
\end{itemize}

Advertisement

\protect\hyperlink{after-top}{Continue reading the main story}

Supported by

\protect\hyperlink{after-sponsor}{Continue reading the main story}

\hypertarget{so-youre-thinking-about-riding-a-bike}{%
\section{So You're Thinking About Riding a
Bike}\label{so-youre-thinking-about-riding-a-bike}}

These days, with limits on public transportation and daily protests,
cyclists dominate cities around the country. Here's how to become one.

\includegraphics{https://static01.graylady3jvrrxbe.onion/images/2020/05/29/fashion/BIKEFAQ1/merlin_172547034_edfc2863-0b62-4a9a-afb2-4be84cba8ac7-articleLarge.jpg?quality=75\&auto=webp\&disable=upscale}

By \href{https://www.nytimes3xbfgragh.onion/by/john-herrman}{John
Herrman}

\begin{itemize}
\item
  Published June 18, 2020Updated June 19, 2020
\item
  \begin{itemize}
  \item
  \item
  \item
  \item
  \item
  \end{itemize}
\end{itemize}

It's the summer of Covid, and bikes are everywhere --- in parks, at
protests, speeding across bridges and locked up outside. Every day, more
people seem to be having the same thought: I should be riding a bike.

I can't blame them. Maintaining forward movement on two wheels so as not
to fall over is one of life's great joys. It's a way to stay healthy and
to rediscover your community. Maybe it's nostalgic, or aspirational.
It's a good way to get around when some of the alternatives don't feel
so great.

In 2020, however, riding a bike isn't quite, as they say, \emph{like
riding a bike.} It's not like when you were a kid; it's barely like it
was four months ago. ** There's an international
\href{https://www.nytimes3xbfgragh.onion/2020/05/18/nyregion/bike-shortage-coronavirus.html}{bike
shortage}. Shops are operating under strange and challenging retail
circumstances, and some recreational areas are closed.

It's enough to smother a fresh spark of interest, but it doesn't have to
be. It \emph{is} a great time to ride a bike, and far from impossible to
get rolling. Whether you're new to cycling or simply returning to the
saddle, here are some questions you may have, many of which I sourced
from would-be bikers like you, with answers.

\textbf{I know how to ride a bike, but I don't have one. Where do I get
one?}

The current bike shortage is real. Many large bike companies rely on
overseas manufacturing, mainly in China. Coronavirus-related factory
closures led to a supply crunch. Covid-19 arrived in the U.S. around the
time shops would have been planning to stock up for spring. On top of
that, demand for bikes during lockdown --- mostly for recreation, but
also for commuting --- has been enormous.

\textbf{Well that's sort of letting the air out of my proverbial tires.}

Yeah. It's not ideal. Estimates for when the biggest companies will
replenish stock range from July to August or beyond. In some areas, and
for some brands, available bikes have been narrowed down to fairly
specialized and expensive machines. ``The manufacturers are being very
vague,'' said Annie Byrne, who owns BFF Bikes in Chicago. She's telling
customers looking for basic bikes that the wait could stretch until
August.

The shortage, however, is not uniform, and depending on a number of
factors --- not just the sort of riding you want to do, or your
location, but your height, for example --- you \emph{might} still be
able to find what you need off the rack. 2021 bike models, which under
normal circumstances would start arriving in shops by June, have also
been delayed by a month or more, but will eventually alleviate the
crunch.

\textbf{So, wait, what should I do here? Where do I start?}

Start by reaching out to a local bike shop; depending on what you want,
they might still have something for you.

Generally speaking, while the upfront cost may be higher than buying a
bike from an online retailer, it usually comes with some limited free
maintenance. Unless you're extremely handy and planning to invest in
lots of tools, you're going to end up at a shop someday anyway.

Typically, a bike shop will carry a limited set of brands; most major
bike brands --- think Trek or Giant --- have something for almost
everyone. Shops are a valuable resource for cyclists under any
circumstances, and that includes during a bike shortage. Ms. Byrne, for
example, has been able to refer customers to used bike retailers in the
area.

\textbf{So I shouldn't buy online?}

The first thing to know about buying a bike online is that some assembly
will be required. In most cases this will involve following Ikea-level
instructions, though it's not the worst idea to have a new bike
assembled or inspected by a shop anyway, especially if you don't feel
confident spotting potential safety issues.

Brands like Priority Bicycles try to make the assembly process as easy
as possible. Some older brands that now sell online, including Raleigh,
will ship \href{https://youtu.be/hmu03iLcVSM}{mostly assembled bikes}
either directly to you or to a participating local shop. (Some of these
companies are experiencing shortages as well.)

You can buy a bike from a major online retailer like Jenson USA or Chain
Reaction Cycles, or from the discount retailer BikesDirect, which,
despite its \href{http://www.bikesdirect.com/}{extremely old-fashioned
website}, is a real high-volume retailer with many budget bikes still in
stock. You can buy a bike from Amazon or another online general store,
but what you get --- and in what state of assembly it arrives at your
door --- will depend on the brand and the seller.

Don't sleep on REI, which carries a fairly wide range of bikes online
and has service departments in its stores.

\textbf{What about other big box stores?}

A large majority of bikes sold in the U.S. are sold by stores like
Walmart, where the most expensive model in stock might be cheaper than
the least expensive one at your local bike shop.

Big box bikes get a bad rap, and not without reason. They're often
poorly assembled and sold with little support, and an inevitable tuneup
can cost a good portion of the bike's original price. They nominally
come in lots of varieties --- mountain, road, cruiser --- but it's best
to think of them all as casual commuter or leisure bikes.

That said, a big box bike will get you through a short work commute.
It'll get you to the beach, or around a lake path. Your kids will have a
blast hopping them off curbs. You will still feel the wind on your face.
If you really take to cycling, you'll want something better pretty
quickly, and like many cheap, borderline disposable products, their
eventual cost of ownership, or replacement, can be high. But the cycling
community can also be dismissive and a little bit classist on this
issue. Not everyone can spend hundreds (or thousands) of dollars to see
if they like cycling.

If you need to take the big box path, there are a few good online
resources to know about:
\href{http://bigboxbikes.com/}{BigBoxBikes.com}, a large and active
forum, and
\href{https://www.youtube.com/channel/UC_mxl2kDiu-Ls_m7dR5qjtw}{KevCentral},
a YouTube channel that is the closest thing this sector has to a trade
publication.

\textbf{What about used bikes?}

Buying a used bike is sort of like buying a bike online, except you're
even more on your own. Given the shortages, though, it is absolutely
worth browsing Craigslist, Facebook Marketplace and even eBay to see
what you can find.

Aside from the regular caveats about buying anything online from a
stranger, you'll have to narrow things down by type, size, condition and
price. I would recommend contacting a bike commuter or recreational
cyclist in your life and asking for a second set of eyes. Plenty of us
would enjoy buying another bike, even vicariously.

To any enthusiasts reading: Offer this sort of help to everyone you
know, and be patient. Between the various online marketplaces, there
\emph{are} currently bikes available for nearly everyone who wants one.
Help make the connection! Be proactive! (Be generous with your tools and
time, too.) This is, for the next couple months, the actual solution to
the bike crunch, but it will require some work from people who already
have bikes.

\textbf{OK. So what kind do I need?}

The good news is that nearly any bike that fits you and rolls will do.
This is the most important thing to keep in mind, in the context of
shortages: \emph{A bike is a bike}. Ever seen a bike-share bike, such as
a Citi Bike in New York or a Divvy bike in Chicago? They're heavy,
clumsy and come in one size. But they work well enough for most riders
with no more than a seat adjustment.

In cities around the world where cycling is most common, the most
popular bikes are often old-fashioned, simple and have very little
relationship to cycling as a cutting-edge gear sport. This is good to
keep in mind before you feel too constrained in your options, browsing
through various product lines or the Craigslist bike section. A bike is
a bike --- until you want something more from it.

If your goal is to get to work, or get some exercise in a reasonably
strenuous but not regimented way, a simple ``fitness'' bike or
``hybrid'' bike is a good place to start. Bikes like this are often
repurposed as rental bikes for sightseeing. They're affordable, they're
easy to step over in regular clothes, they have cheap but serviceable
parts, and they feel familiar and welcoming to casual riders. They steer
somewhat slowly and predictably. They won't run out of easier gears as
you're pedaling over a bridge. They'll come with seats designed to be
comfortable for most people over the short periods of time they'll be in
use.

You'll also find bikes marketed as ``city'' or ``commuter'' bikes. These
will usually be great beginner options as well. ``Comfort'' or
``cruiser'' bikes are also common in lower price ranges, and they're
wonderful for genuinely leisurely riding, but won't support more
ambitious exercising or commuting, and in some cases lack gears.

Your most important requirements for a commuter or get-some-air type
bike don't have much to do with performance, but rather more practical
questions. Do you want to carry stuff on your bike? Ask if it has mounts
for racks. Do you need to carry it on a train, or store it in a small
apartment? Maybe consider a folding bike.

Unless your needs are highly specific --- racing, serious mountain
biking, towing a trailer --- your options are probably much wider than
the cycling industry would have you think. Eben Weiss at Insider put
together a \href{https://www.insider.com/best-bikes}{great ``best
bikes'' list by category}, which doubles as an explainer about what each
type is for. If your summer of riding \emph{literally any available
bike} goes well, maybe you'll catch the bug and we'll line up next to
each other at a race next year, or run into each other in the woods. For
now, hopefully, I'll see you in the bike lane.

\textbf{What size do I need?}

Bike sizing often comes down to the specific brand and category of bike,
and for simplicity's sake many manufacturers have switched away from
numerical sizes for a small/medium/large schema that corresponds to a
rider's overall height. For casual riding in particular, getting a
close-enough bike size is usually fine --- you will be able to adjust
your seat and perhaps handlebars for a finer fit.

If you do see numbers, however, here's what they mean. A centimeter or
inch measurement refers to the length of the central, vertical-ish part
of the bike frame --- the seat tube. On road-style bikes this is often
described in centimeters. On mountain-style bikes, it's often inches. A
54-centimeter frame might be about right for someone 5-foot-10; a
16-inch bike might be suited to someone around 5-foot-6.

Unless you're dealing with a vintage bike, you should be able to find
the recommended height range for a given size from the bike's
manufacturer. This is the most vital info you can get from a used bike
seller, too. Bicycling put together
\href{https://www.bicycling.com/bikes-gear/a20047780/find-right-bike-size/)}{a
good guide for sizes by bike type}. As for setting the height of your
seat,
\href{https://roadcyclinguk.com/how-to/technique/beginners-guide-how-to-set-your-saddle-height-on-a-road-bike.html}{here
are a few good methods to get you close}. The ``heel-to-pedal'' method
has served me well for decades.

\textbf{What about gender-specific bikes?}

Don't worry too much about this --- that a bike fits is many times more
important than any gender-specific features or adjustments. If you'd
like to read more, here is a
\href{https://www.femmecyclist.com/mens-vs-womens-bikes/}{thorough link}
from Femmecyclist. For our purposes, again: A bike is a bike!

\textbf{How much do I need to spend?}

New bikes from major brands that will be easy to take care of and last a
long time start at around \$300 dollars, although many cost more.
Getting started with a purpose-built road bike or mountain bike will
push you past \$500; enthusiasts in either discipline will tell you not
to bother spending less than twice that. Ignore them unless you want to
be them. Remember, a bike is a bike.

There is no rule of thumb for used bikes, but with the help of a savvy
friend you should be able to find something safe, durable ready to roll
for under \$200, even accounting for crisis-time price gouging.

\textbf{Is there a sort of bike that will make me look like a jerk? How
do I look sensible, but also cool, but also like I don't care too much?}

Buy whatever suits your needs and you think looks cool and ride it with
confidence. Don't spend a huge unnecessary amount of money, I guess, if
that's something you might do? But aspirational purchases are fine, too!
We're trying to improve mobility here!

\textbf{OK, sorry. What do I need to take care of the reasonable bike
that I'm buying for sensible reasons?}

If you're interested, bike maintenance is a fulfilling hobby of its own.
For most riders, of course, it's a chore. A very short list of things
you'll need includes:

\begin{itemize}
\item
  A pump that works with both types of air valves (Presta is the thin
  one and Schrader is the old-fashioned one identical to those on a car
  tire)
\item
  A folding multi-tool with a range of hex keys
\item
  Some inner tubes in the size marked on your tires
\item
  A bottle of all-purpose bike lubricant
\item
  A bottle of bike chain lubricant
\item
  A sacrificial rag or two
\end{itemize}

\textbf{Right, but what will I actually need to do?}

Truthfully not much, until something goes wrong. New bikes will usually
require adjustments to their gears after a couple months of riding ---
something shops will often throw in for free. Later tuneups might be
reduced to twice-yearly or less. For casual riding, tough tires can last
years. Chains and brake pads too. A yearly checkup is often more than
enough, but it's nice to have a friendly relationship with a shop in
case anything comes up.

You'll want to lubricate your chain every once in a while, wiping off
excess lubricant and grime. Tires may seep air, so you'll want to make
sure they're firm. Even cheap pumps will come with a gauge, and your
tires will be marked with recommended pressures.

If you're bringing an old or used bike to a shop, be mindful that a
handful of repairs ---~a set of tires, a new chain, a new wheel --- can
add up quickly, and require a lot of labor. ``We always give people a
couple options,'' said Ms. Byrne, of BFF Bikes. It helps to be clear
with the mechanic. If you just want your rickety old bike safe enough to
ride through the summer, say so.

\textbf{What if I get a flat?}

Just pop off your wheel, then lever your tire from the wheel, figure out
what caused the flat, remove any debris from the tire, put a new tube
inside the wheel, pop the tire back on and inflate it. It's as easy as
that, which is to say \ldots{} not very easy at all! Unless you've
practiced.

For years, the best way to learn how to fix bike, aside from working in
a shop or hours of trial and error, was reading --- classics like Zinn
\& the Art of Road Bike Maintenance, or the Big Blue Book of Bike
Repair. Now, the best place to learn how to perform basic maintenance
tasks is YouTube. Park Tool, whose mechanics wrote the Big Blue Book,
has a comprehensive and accessible
\href{https://www.youtube.com/channel/UCzaZ1sPWEuZN-I8_XT6AH8g}{YouTube
channel} for most common maintenance issues, which always teach you
enough and usually teach you a bit too much. Here's
\href{https://www.youtube.com/watch?v=eqR6nlZNeU8}{the one} on changing
tires.

\textbf{Where do I keep my bike?}

Bike storage is a common post-purchase complaint. Bikes are huge ---
they don't fit as closely against a wall as you might imagine and
they're longer than you think. If you're in an apartment, your bike will
be decorative. Plan this out before you commit to buying anything. The
Wirecutter has a good
\href{https://www.nytimes3xbfgragh.onion/wirecutter/reviews/best-bike-racks-for-small-homes-and-apartments/}{roundup}
of bike storage solutions, of which there are plenty, but as you're
considering your options, know that there are affordable ways to hang
your bike in lots of different ways: from the ceiling; flat against a
wall; hanging out from the wall; propped against the wall, stacked two
high. Don't forget your stairs. Most bikes are pretty heavy. While
lightweight road bikes come in under 20 pounds, most casual bikes will
weigh 30 or more. Many e-bikes exceed 40 or even 50 pounds. A twice
daily three-flight journey with any bike gets old fast (trust me).

\textbf{Can I keep it outside? I live in an apartment.}

The risk of theft depends on where you live and what kind of bike you're
leaving outside, as well as what kind of lock you buy. (There is no such
thing as a theft-proof lock, but some are certainly
\href{https://www.nytimes3xbfgragh.onion/wirecutter/reviews/best-bike-lock/}{better
than others}.) Plenty of people leave affordable bikes out in New York
City, for example, because the risk is worth the convenience and cost.
More than theft, usually, the issue with storing a bike outside is that
it will age it fast. Parts will corrode, lubricant will wash off,
bearings and cables will degrade faster than usual. Your maintenance
schedule will be doubled at least. Depending on your living situation,
though, this could be as much a case for a very cheap used bike as it is
for keeping your bike inside.

\textbf{OK, well, whatever I ride, I need a helmet, right?}

Short answer: You should probably buy a helmet. Specifically a
\emph{new} helmet that fits snugly. If you live somewhere hot, make sure
it has good ventilation. All new helmets pass basic safety testing so
don't spend more than you want to. After a crash, or even dropping your
helmet from a height, get a new one. Think of them as single-use.

Long answer: It's actually sort of complicated! I wear a helmet on every
ride. They are an additional barrier between the asphalt and your brain.
There are good arguments, however, against making wearing a helmet at
the center of bike safety discussions, as helmet and visibility laws
shift responsibility from drivers to the people they're injuring and
killing. (Universal helmet usage, the thinking goes, is no substitute
for safe cycling infrastructure.)
\href{https://www.bicycling.com/news/a24110027/bike-helmet-safety/}{This
piece from Bicycling} gets into the arguments, and the data.

\textbf{So \ldots{} where should I ride?}

Google Maps now includes an option for cycling directions, which is
decent, or at least much better than it used to be. BikeMap, a free app
for iOS and Android, as well as a website, is a good resource both for
planning local bike routes and locating bike paths, greenways and lanes
where you live. Once again, a good local bike shop will have
recommendations for different types of riding.

\textbf{OK, but what if there's not a nice path or empty roads. It's not
really clear to me where bikes go, like, in the hierarchy of human
pathways. Can I ride on the sidewalk? The road? All roads?}

You're right, it's often not clear, and the answer, in many places in
America, is ``nowhere, really.'' Even in places with relatively
comprehensive networks of bike paths and lanes, to use a bike for
transportation will require learning how to ride as safely as possible
in traffic. This means knowing the rules of the road and signaling and
so forth, but more than that it means maintaining a vigilance and
awareness about what the cars around you tend to do. CityLab put
together an
\href{https://www.citylab.com/life/2017/05/urban-cycling-how-to/526500/}{excellent
guide} a few years ago. This will require a bit of additional equipment,
too ---~a bell and blinking visibility lights for the front and back of
your bike.

Getting on a bike in your hometown helps you see it differently. Maybe
it will feel bigger, or suddenly smaller. Maybe you'll notice hills you
never had to think about before. You'll almost certainly notice the ways
in which your community could improve cycling infrastructure, and, even
if this is not typical for you, you might become rather zealous about
it.

If you ride widely in your community, you will also notice how cycling
infrastructure is unevenly distributed, and rarely reaches into
neighborhoods without the political or social influence to demand it. As
tempting as it is for some cycling advocates to highlight how liberating
or equalizing cycling is, or to tweet about how \#outsideisfree, cycling
is dependent on the physical, social and racial realities of the
communities in which you try to do it. So getting on a bike might make
you more aware of this, too, if you aren't already. You might also find
that biking can be a useful mode of transportation to and from, say, a
protest.

\textbf{Should I be wearing a mask?}

You might as well bring one, if mask wearing is recommended where you
live. The evidence about Covid-19 transmission during cycling in
particular is unclear, and maybe your commute or recreational rides will
be naturally distanced or alone, but the reality of bike commuting in
any densely populated area is that you might find yourself suddenly
stopped at a light with five other people, or in a tightly spaced line
downwind from someone with a nasty cough, or vice versa. It is not
especially comfortable to wear a mask while cycling, I'll admit. I've
had the best luck with
\href{https://www.nytimes3xbfgragh.onion/article/how-to-make-face-mask-coronavirus.html}{pleated
cloth masks}.

Advertisement

\protect\hyperlink{after-bottom}{Continue reading the main story}

\hypertarget{site-index}{%
\subsection{Site Index}\label{site-index}}

\hypertarget{site-information-navigation}{%
\subsection{Site Information
Navigation}\label{site-information-navigation}}

\begin{itemize}
\tightlist
\item
  \href{https://help.nytimes3xbfgragh.onion/hc/en-us/articles/115014792127-Copyright-notice}{©~2020~The
  New York Times Company}
\end{itemize}

\begin{itemize}
\tightlist
\item
  \href{https://www.nytco.com/}{NYTCo}
\item
  \href{https://help.nytimes3xbfgragh.onion/hc/en-us/articles/115015385887-Contact-Us}{Contact
  Us}
\item
  \href{https://www.nytco.com/careers/}{Work with us}
\item
  \href{https://nytmediakit.com/}{Advertise}
\item
  \href{http://www.tbrandstudio.com/}{T Brand Studio}
\item
  \href{https://www.nytimes3xbfgragh.onion/privacy/cookie-policy\#how-do-i-manage-trackers}{Your
  Ad Choices}
\item
  \href{https://www.nytimes3xbfgragh.onion/privacy}{Privacy}
\item
  \href{https://help.nytimes3xbfgragh.onion/hc/en-us/articles/115014893428-Terms-of-service}{Terms
  of Service}
\item
  \href{https://help.nytimes3xbfgragh.onion/hc/en-us/articles/115014893968-Terms-of-sale}{Terms
  of Sale}
\item
  \href{https://spiderbites.nytimes3xbfgragh.onion}{Site Map}
\item
  \href{https://help.nytimes3xbfgragh.onion/hc/en-us}{Help}
\item
  \href{https://www.nytimes3xbfgragh.onion/subscription?campaignId=37WXW}{Subscriptions}
\end{itemize}
