Sections

SEARCH

\protect\hyperlink{site-content}{Skip to
content}\protect\hyperlink{site-index}{Skip to site index}

\href{https://www.nytimes3xbfgragh.onion/section/parenting}{Parenting}

\href{https://myaccount.nytimes3xbfgragh.onion/auth/login?response_type=cookie\&client_id=vi}{}

\href{https://www.nytimes3xbfgragh.onion/section/todayspaper}{Today's
Paper}

\href{/section/parenting}{Parenting}\textbar{}Worried About Your Kids'
Social Skills Post-Lockdown?

\url{https://nyti.ms/3fyhGOq}

\begin{itemize}
\item
\item
\item
\item
\item
\end{itemize}

\hypertarget{the-coronavirus-outbreak}{%
\subsubsection{\texorpdfstring{\href{https://www.nytimes3xbfgragh.onion/news-event/coronavirus?name=styln-coronavirus-national\&region=TOP_BANNER\&block=storyline_menu_recirc\&action=click\&pgtype=Article\&impression_id=a3ff7030-f4c6-11ea-a7aa-59ff5ebfc7d3\&variant=undefined}{The
Coronavirus
Outbreak}}{The Coronavirus Outbreak}}\label{the-coronavirus-outbreak}}

\begin{itemize}
\tightlist
\item
  live\href{https://www.nytimes3xbfgragh.onion/2020/09/11/world/covid-19-coronavirus.html?name=styln-coronavirus-national\&region=TOP_BANNER\&block=storyline_menu_recirc\&action=click\&pgtype=Article\&impression_id=a3ff7031-f4c6-11ea-a7aa-59ff5ebfc7d3\&variant=undefined}{Latest
  Updates}
\item
  \href{https://www.nytimes3xbfgragh.onion/interactive/2020/us/coronavirus-us-cases.html?name=styln-coronavirus-national\&region=TOP_BANNER\&block=storyline_menu_recirc\&action=click\&pgtype=Article\&impression_id=a3ff9740-f4c6-11ea-a7aa-59ff5ebfc7d3\&variant=undefined}{Maps
  and Cases}
\item
  \href{https://www.nytimes3xbfgragh.onion/interactive/2020/science/coronavirus-vaccine-tracker.html?name=styln-coronavirus-national\&region=TOP_BANNER\&block=storyline_menu_recirc\&action=click\&pgtype=Article\&impression_id=a3ff9741-f4c6-11ea-a7aa-59ff5ebfc7d3\&variant=undefined}{Vaccine
  Tracker}
\item
  \href{https://www.nytimes3xbfgragh.onion/2020/09/10/us/politics/fda-coronavirus-vaccine.html?name=styln-coronavirus-national\&region=TOP_BANNER\&block=storyline_menu_recirc\&action=click\&pgtype=Article\&impression_id=a3ff9742-f4c6-11ea-a7aa-59ff5ebfc7d3\&variant=undefined}{F.D.A.
  Regulators' Self-Defense}
\item
  \href{https://www.nytimes3xbfgragh.onion/2020/09/09/upshot/coronavirus-surprise-test-fees.html?name=styln-coronavirus-national\&region=TOP_BANNER\&block=storyline_menu_recirc\&action=click\&pgtype=Article\&impression_id=a3ff9743-f4c6-11ea-a7aa-59ff5ebfc7d3\&variant=undefined}{Surprise
  Test Fees}
\end{itemize}

Advertisement

\protect\hyperlink{after-top}{Continue reading the main story}

Supported by

\protect\hyperlink{after-sponsor}{Continue reading the main story}

\hypertarget{worried-about-your-kids-social-skills-post-lockdown}{%
\section{Worried About Your Kids' Social Skills
Post-Lockdown?}\label{worried-about-your-kids-social-skills-post-lockdown}}

Try not to fret. There's much to be gained from time with family, or
even on their own.

\includegraphics{https://static01.graylady3jvrrxbe.onion/images/2020/06/17/multimedia/17parenting-social-needs/17parenting-social-needs-articleLarge.jpg?quality=75\&auto=webp\&disable=upscale}

By Emily Sohn

\begin{itemize}
\item
  June 18, 2020
\item
  \begin{itemize}
  \item
  \item
  \item
  \item
  \item
  \end{itemize}
\end{itemize}

Before the coronavirus pandemic began, Michael Munson's 3-year-old son
saw a group of close friends at his preschool at least a few times a
week. When he wasn't in school, he and his 1-year-old sister often
played with other kids at the park.

But ever since much of the world shut down to prevent the spread of the
coronavirus, the kids have been home with Munson and his wife, both
lawyers, who take turns watching them while the other works. They have
tried to connect their preschooler to friends through video chats hosted
by his teacher, but his response was usually to withdraw, throw tantrums
or run away from the screen.

Like many other parents enduring months of stay-at-home orders and
school closures, Munson has added a new worry to his list: social
deprivation for his kids.

``I'm not a child psychologist or anything, but I understand that ages
2, 3 and 4 are really important for learning how to share, learning how
to play outside, learning how to ride bikes, and all these things that
are sort of rites of passage in American childhood,'' said Munson, who
lives in Chapel Hill, N.C. ``What isn't he getting by not being able to
run around the neighborhood with his friends?''

Social interactions are an important part of development throughout
childhood, and spending time with peers is typically part of that
process. But try not to fret too much about what they're missing right
now. Several pediatricians and psychologists offered reassurance about
the isolation many children have experienced because of Covid-19.

Children tend to be resilient and adaptable, they said. There is much to
be gained from interactions with parents, siblings and even pets. Time
alone is valuable, too. And connection through technology, like hanging
out or playing games through video chats, can fill in some of the
blanks. Even without peer interaction for a while, kids can still
develop socially and emotionally in ways that will prepare them to
pursue real-world friendships when those can resume.

``Even though this is unusual, most kids will come out of this fine
because we're biologically wired to adapt,'' said Dr. Jack Shonkoff,
M.D., a pediatrician and early-childhood development expert at Harvard's
\href{https://developingchild.harvard.edu/}{Center on the Developing
Child}. ``If we weren't, we would have gone extinct like the dinosaurs.
We wouldn't be able to survive because the environment is always
changing.''

\hypertarget{latest-updates-the-coronavirus-outbreak}{%
\section{\texorpdfstring{\href{https://www.nytimes3xbfgragh.onion/2020/09/11/world/covid-19-coronavirus.html?action=click\&pgtype=Article\&state=default\&region=MAIN_CONTENT_1\&context=storylines_live_updates}{Latest
Updates: The Coronavirus
Outbreak}}{Latest Updates: The Coronavirus Outbreak}}\label{latest-updates-the-coronavirus-outbreak}}

Updated 2020-09-12T07:06:44.310Z

\begin{itemize}
\tightlist
\item
  \href{https://www.nytimes3xbfgragh.onion/2020/09/11/world/covid-19-coronavirus.html?action=click\&pgtype=Article\&state=default\&region=MAIN_CONTENT_1\&context=storylines_live_updates\#link-dfb8a16}{Fauci
  cautions the virus could disrupt life in the U.S. until `maybe even
  towards the end of 2021.'}
\item
  \href{https://www.nytimes3xbfgragh.onion/2020/09/11/world/covid-19-coronavirus.html?action=click\&pgtype=Article\&state=default\&region=MAIN_CONTENT_1\&context=storylines_live_updates\#link-7104d154}{From
  Asia to Africa, China promotes its vaccine candidates to win friends.}
\item
  \href{https://www.nytimes3xbfgragh.onion/2020/09/11/world/covid-19-coronavirus.html?action=click\&pgtype=Article\&state=default\&region=MAIN_CONTENT_1\&context=storylines_live_updates\#link-393ad215}{The
  other way the virus will kill: hunger.}
\end{itemize}

\href{https://www.nytimes3xbfgragh.onion/2020/09/11/world/covid-19-coronavirus.html?action=click\&pgtype=Article\&state=default\&region=MAIN_CONTENT_1\&context=storylines_live_updates}{See
more updates}

More live coverage:
\href{https://www.nytimes3xbfgragh.onion/live/2020/09/11/business/stock-market-today-coronavirus?action=click\&pgtype=Article\&state=default\&region=MAIN_CONTENT_1\&context=storylines_live_updates}{Markets}

Social and emotional learning begins in infancy, and social skills form
the foundation for other types of learning, said Dr. Deborah Phillips,
Ph.D., a developmental psychologist at Georgetown University. Among the
skills that matter are the ability to understand your own emotions,
empathize with others, make decisions, cope with challenges, develop
relationships and take responsibility for mistakes. Spending time with
peers is one way that kids can develop those skills, which affect
physical and mental health throughout life,
\href{https://www.tandfonline.com/doi/abs/10.1080/10410236.2017.1384434?journalCode=hhth20}{research
suggests}.

While
\href{https://journals.sagepub.com/doi/10.1177/1745691614568352}{plenty
of studies} have documented the link between loneliness and long-term
health problems, quarantine itself is not necessarily causing harm or
depriving kids of what they need, experts said, especially in cases
where a child's needs are otherwise being met. In many cases, Dr.
Phillips said, precedent suggests that kids can handle big changes,
including spending long periods of time in the hospital, moving
frequently or being separated from a parent for stretches of time.

From a cultural perspective, kids grow up in all kinds of situations,
from nuclear families to communes, Dr. Shonkoff added. And historically,
the level of technology-aided, always-accessible communication we are
accustomed to with family and friends is relatively new.

In fact, having parents who worry excessively about what their kids are
missing out on is likely more damaging than missing out on experiences,
said Dr. Seth Pollak, Ph.D, a psychologist at the University of
Wisconsin-Madison. Stress is already widespread among parents who have
been balancing work and distance learning, or who have been unemployed,
for months. In a
\href{https://www.apa.org/news/press/releases/2020/05/stress-america-covid-19}{May
survey of more than 3,000 people,} the American Psychological
Association reported that 46 percent of parents rated their stress level
at eight or higher on a 10-point scale, compared with 28 percent of
adults without children. Neglect and abuse can have serious
consequences, and
\href{https://www.nytimes3xbfgragh.onion/2020/06/09/nyregion/coronavirus-nyc-child-abuse.html}{concerns
about those risks}have escalated during the pandemic.

To reduce some of the strain, Dr. Pollak said, parents can help their
children by trying to let go of their anxiety about temporary social
deprivation. ``I think it's really important for parents not to
catastrophize and panic,'' he said. ``There's no evidence that even a
few months of social distancing is going to have a long-term effect on
children's development.''

Instead, parents can take comfort in the value of relationships within
their own homes, Dr. Phillips said. Studies suggest that secure
attachments with parents set children up to have stronger friendships.
In households with more than one child, siblings help each other learn
to negotiate, deal with conflict and resolve differences --- something
parents can help facilitate by having conversations about the important
roles each sibling plays in the family.

Pets, too, can help teach empathy, responsibility, and how to see the
perspectives of others. Dr. Phillips suggested using animal companions
as a conversation starter about emotions, asking kids questions like,
``What do you think they're feeling? Do you think this is weird for
them?''

When it comes to practical tips for facilitating social development,
experts said that parents shouldn't feel pressured to replace peer
interaction by acting as if they are the same age as their children or
playing games they hate. Social interactions are like training wheels
that teach children how to negotiate social situations for the rest of
their lives, Dr. Pollak said, and it's valuable for children to see that
people --- even parents --- have moods, opinions, unique styles of play
and a need to take breaks.

To teach lessons about empathy and language that young children might
otherwise learn in school settings, Dr. Phillips suggested naming
feelings and reading out loud, particularly stories about other places
and circumstances. She also recommended giving children developmentally
appropriate responsibilities at home --- like asking a 5-year-old to
plan dinner or having an older child do laundry --- which can help them
develop a sense of competence and responsibility. ``All of that is
transferable and will help them be ready when they return to their
classrooms and their friends,'' she said.

Now is not necessarily the time for lectures about social skills, Dr.
Pollak added. Instead, he advised keeping learning moments playful. For
example, if a child tries taking two turns in a row, a parent could
laugh and say, ``Well, if you're going to take an extra turn, I'm going
to take another turn. Let's do that,''' he said. Finding opportunities
for fun is important because happiness breeds curiosity and imaginary
play, and play is an essential form of learning in young children.

Time at home also offers the chance for children to develop
independence. For young kids, that might mean
\href{https://www.nytimes3xbfgragh.onion/2020/04/03/parenting/kids-independent-play-coronavirus-quarantine.html}{picking
a toy to play with alone for a while}. For school-age kids, Dr. Shonkoff
said, figuring things out might mean finding new ways to keep in touch
with friends virtually.

Every kid is different, experts said, and some might need more social
time than others. But they don't need tons of friends or a certain kind
of interaction to thrive. Even one good friend can help children develop
an appreciation for relationships, Dr. Phillips said. And connection can
come in many forms, like drawing for each other, writing letters, having
video chats, communicating by text message through parents or talking
with neighbors through a fence.

Being apart from friends has been hard for many kids and some may be
having a particularly tough time. There are some behavioral signs to
look out for that suggest a child is struggling, Dr. Shonkoff said.
\href{https://www.nytimes3xbfgragh.onion/2020/06/02/parenting/virus-children-depression-signs.html}{If
parents notice children are withdrawn, sad,} harder to control, having
trouble sleeping or struggling with basic daily interactions, he
suggested seeking help from a health care provider.
\href{https://www.nytimes3xbfgragh.onion/2020/04/08/parenting/coronavirus-self-care.html}{The
same goes for parents}, whose mental health matters too. But a change in
circumstances, even one caused by a global pandemic, does not mean that
parents need to scramble to figure out all the best ways to promote
healthy development.

Instead, parents can foster healthy growth by trying to spend some time
each day socially engaged with their children, Dr. Shonkoff said.
Ideally, that time will not only be at the end of the day, when everyone
is tired. And it will fit into a relatively predictable routine of
mealtimes and bedtime. It's not important to track how many minutes the
social time lasts, whether the interaction is perfect, or if every
moment is optimally stimulating.

It is normal to worry, Dr. Shonkoff added. But parents can help their
children by agonizing less about whether their children will be damaged
by this challenging time. ``There's as much a danger of overthinking
this and trying to kind of create an overcontrolled environment for
children,'' he said. ``Kids just need a sense that the environment is
safe.''

\begin{center}\rule{0.5\linewidth}{\linethickness}\end{center}

Emily Sohn is a freelance health and science journalist and parent of
two boys.

Advertisement

\protect\hyperlink{after-bottom}{Continue reading the main story}

\hypertarget{site-index}{%
\subsection{Site Index}\label{site-index}}

\hypertarget{site-information-navigation}{%
\subsection{Site Information
Navigation}\label{site-information-navigation}}

\begin{itemize}
\tightlist
\item
  \href{https://help.nytimes3xbfgragh.onion/hc/en-us/articles/115014792127-Copyright-notice}{©~2020~The
  New York Times Company}
\end{itemize}

\begin{itemize}
\tightlist
\item
  \href{https://www.nytco.com/}{NYTCo}
\item
  \href{https://help.nytimes3xbfgragh.onion/hc/en-us/articles/115015385887-Contact-Us}{Contact
  Us}
\item
  \href{https://www.nytco.com/careers/}{Work with us}
\item
  \href{https://nytmediakit.com/}{Advertise}
\item
  \href{http://www.tbrandstudio.com/}{T Brand Studio}
\item
  \href{https://www.nytimes3xbfgragh.onion/privacy/cookie-policy\#how-do-i-manage-trackers}{Your
  Ad Choices}
\item
  \href{https://www.nytimes3xbfgragh.onion/privacy}{Privacy}
\item
  \href{https://help.nytimes3xbfgragh.onion/hc/en-us/articles/115014893428-Terms-of-service}{Terms
  of Service}
\item
  \href{https://help.nytimes3xbfgragh.onion/hc/en-us/articles/115014893968-Terms-of-sale}{Terms
  of Sale}
\item
  \href{https://spiderbites.nytimes3xbfgragh.onion}{Site Map}
\item
  \href{https://help.nytimes3xbfgragh.onion/hc/en-us}{Help}
\item
  \href{https://www.nytimes3xbfgragh.onion/subscription?campaignId=37WXW}{Subscriptions}
\end{itemize}
