Sections

SEARCH

\protect\hyperlink{site-content}{Skip to
content}\protect\hyperlink{site-index}{Skip to site index}

\href{https://www.nytimes3xbfgragh.onion/section/reader-center}{Times
Insider}

\href{https://myaccount.nytimes3xbfgragh.onion/auth/login?response_type=cookie\&client_id=vi}{}

\href{https://www.nytimes3xbfgragh.onion/section/todayspaper}{Today's
Paper}

\href{/section/reader-center}{Times Insider}\textbar{}How Times
Reporters Handle Scientific Studies

\url{https://nyti.ms/2YldKtu}

\begin{itemize}
\item
\item
\item
\item
\item
\item
\end{itemize}

\hypertarget{the-coronavirus-outbreak}{%
\subsubsection{\texorpdfstring{\href{https://www.nytimes3xbfgragh.onion/news-event/coronavirus?name=styln-coronavirus-national\&region=TOP_BANNER\&block=storyline_menu_recirc\&action=click\&pgtype=Article\&impression_id=3e70ded0-f4d4-11ea-850c-d13f9cab80f0\&variant=undefined}{The
Coronavirus
Outbreak}}{The Coronavirus Outbreak}}\label{the-coronavirus-outbreak}}

\begin{itemize}
\tightlist
\item
  live\href{https://www.nytimes3xbfgragh.onion/2020/09/11/world/covid-19-coronavirus.html?name=styln-coronavirus-national\&region=TOP_BANNER\&block=storyline_menu_recirc\&action=click\&pgtype=Article\&impression_id=3e70ded1-f4d4-11ea-850c-d13f9cab80f0\&variant=undefined}{Latest
  Updates}
\item
  \href{https://www.nytimes3xbfgragh.onion/interactive/2020/us/coronavirus-us-cases.html?name=styln-coronavirus-national\&region=TOP_BANNER\&block=storyline_menu_recirc\&action=click\&pgtype=Article\&impression_id=3e70ded2-f4d4-11ea-850c-d13f9cab80f0\&variant=undefined}{Maps
  and Cases}
\item
  \href{https://www.nytimes3xbfgragh.onion/interactive/2020/science/coronavirus-vaccine-tracker.html?name=styln-coronavirus-national\&region=TOP_BANNER\&block=storyline_menu_recirc\&action=click\&pgtype=Article\&impression_id=3e7105e0-f4d4-11ea-850c-d13f9cab80f0\&variant=undefined}{Vaccine
  Tracker}
\item
  \href{https://www.nytimes3xbfgragh.onion/2020/09/10/us/politics/fda-coronavirus-vaccine.html?name=styln-coronavirus-national\&region=TOP_BANNER\&block=storyline_menu_recirc\&action=click\&pgtype=Article\&impression_id=3e7105e1-f4d4-11ea-850c-d13f9cab80f0\&variant=undefined}{F.D.A.
  Regulators' Self-Defense}
\item
  \href{https://www.nytimes3xbfgragh.onion/2020/09/09/upshot/coronavirus-surprise-test-fees.html?name=styln-coronavirus-national\&region=TOP_BANNER\&block=storyline_menu_recirc\&action=click\&pgtype=Article\&impression_id=3e7105e2-f4d4-11ea-850c-d13f9cab80f0\&variant=undefined}{Surprise
  Test Fees}
\end{itemize}

Advertisement

\protect\hyperlink{after-top}{Continue reading the main story}

Supported by

\protect\hyperlink{after-sponsor}{Continue reading the main story}

Times Insider

\hypertarget{how-times-reporters-handle-scientific-studies}{%
\section{How Times Reporters Handle Scientific
Studies}\label{how-times-reporters-handle-scientific-studies}}

When is research considered reliable? The answer isn't always fully
known. Here's the approach our journalists take in evaluating studies
and their results.

\includegraphics{https://static01.graylady3jvrrxbe.onion/images/2020/06/12/insider/11-insider-science/11-insider-science-articleLarge.jpg?quality=75\&auto=webp\&disable=upscale}

By \href{https://www.nytimes3xbfgragh.onion/by/emily-palmer}{Emily
Palmer}

\begin{itemize}
\item
  June 9, 2020
\item
  \begin{itemize}
  \item
  \item
  \item
  \item
  \item
  \item
  \end{itemize}
\end{itemize}

\href{https://www.nytimes3xbfgragh.onion/series/times-insider}{\emph{Times
Insider}} \emph{explains who we are and what we do, and delivers
behind-the-scenes insights into how our journalism comes together.}

After early studies showed promising results of the anti-malarial drug
\href{https://www.nytimes3xbfgragh.onion/article/hydroxychloroquine-coronavirus.html}{hydroxychloroquine}
in coronavirus patients, President Trump quickly
\href{https://www.nytimes3xbfgragh.onion/2020/04/05/us/politics/trump-hydroxychloroquine-coronavirus.html}{promoted
it as a possible treatment} and later announced that he was
\href{https://www.nytimes3xbfgragh.onion/2020/05/18/us/politics/trump-hydroxychloroquine-covid-coronavirus.html?smid=tw-nytimes\&smtyp=cur}{taking
the drug} as a preventive measure.

But publishers of the
\href{https://www.sciencedirect.com/science/article/pii/S0924857920300996}{study
from France that Mr. Trump had referenced} said that it fell short of
their
\href{https://retractionwatch.com/2020/04/06/hydroxychlorine-covid-19-study-did-not-meet-publishing-societys-expected-standard/}{standards},
while
\href{https://www.medrxiv.org/content/10.1101/2020.04.07.20056424v2}{researchers}
in Brazil examining the related drug chloroquine
\href{https://www.nytimes3xbfgragh.onion/2020/04/12/health/chloroquine-coronavirus-trump.html?smid=em-share}{halted
their research} after patients given high doses developed potentially
fatal heart arrhythmias. Throughout its coverage, The Times has cited
scientists' reservations about the drug's effectiveness and
\href{https://www.nytimes3xbfgragh.onion/2020/06/03/health/hydroxychloroquine-coronavirus-trump.html}{reported
last week} that the
\href{https://www.nejm.org/doi/full/10.1056/NEJMoa2016638}{first
controlled study} of the drug found it did not prevent infections in
people exposed to the virus.

The
\href{https://www.nytimes3xbfgragh.onion/2020/05/21/us/politics/trump-fact-check-hydroxychloroquine-coronavirus-.html}{misjudgments
about hydroxychloroquine} underscore the importance of how reporters at
The Times cover scientific research: They study the study, and tell
readers what's known and what's not. Even as scientists work feverishly
to answer questions about the
\href{https://www.nytimes3xbfgragh.onion/news-event/coronavirus?action=click\&pgtype=Article\&state=default\&module=styln-coronavirus\&variant=show\&region=TOP_BANNER\&context=storylines_menu}{pandemic},
science reporters carefully parse fact from conjecture, truth from
folly.

That's necessary now especially because ``there's a flood of new
science, much of it seat-of-the-pants,'' said
\href{https://www.nytimes3xbfgragh.onion/by/celia-w-dugger}{Celia W.
Dugger}, the health and science editor.

\hypertarget{latest-updates-the-coronavirus-outbreak}{%
\section{\texorpdfstring{\href{https://www.nytimes3xbfgragh.onion/2020/09/11/world/covid-19-coronavirus.html?action=click\&pgtype=Article\&state=default\&region=MAIN_CONTENT_1\&context=storylines_live_updates}{Latest
Updates: The Coronavirus
Outbreak}}{Latest Updates: The Coronavirus Outbreak}}\label{latest-updates-the-coronavirus-outbreak}}

Updated 2020-09-12T07:09:04.082Z

\begin{itemize}
\tightlist
\item
  \href{https://www.nytimes3xbfgragh.onion/2020/09/11/world/covid-19-coronavirus.html?action=click\&pgtype=Article\&state=default\&region=MAIN_CONTENT_1\&context=storylines_live_updates\#link-dfb8a16}{Fauci
  cautions the virus could disrupt life in the U.S. until `maybe even
  towards the end of 2021.'}
\item
  \href{https://www.nytimes3xbfgragh.onion/2020/09/11/world/covid-19-coronavirus.html?action=click\&pgtype=Article\&state=default\&region=MAIN_CONTENT_1\&context=storylines_live_updates\#link-7104d154}{From
  Asia to Africa, China promotes its vaccine candidates to win friends.}
\item
  \href{https://www.nytimes3xbfgragh.onion/2020/09/11/world/covid-19-coronavirus.html?action=click\&pgtype=Article\&state=default\&region=MAIN_CONTENT_1\&context=storylines_live_updates\#link-393ad215}{The
  other way the virus will kill: hunger.}
\end{itemize}

\href{https://www.nytimes3xbfgragh.onion/2020/09/11/world/covid-19-coronavirus.html?action=click\&pgtype=Article\&state=default\&region=MAIN_CONTENT_1\&context=storylines_live_updates}{See
more updates}

More live coverage:
\href{https://www.nytimes3xbfgragh.onion/live/2020/09/11/business/stock-market-today-coronavirus?action=click\&pgtype=Article\&state=default\&region=MAIN_CONTENT_1\&context=storylines_live_updates}{Markets}

Before you read about a study in The Times, reporters will have looked
into the researchers' backgrounds and often consulted three to five
outside experts to determine the quality of the work. Reporters also ask
questions like: What are the margins of error? Did the study include
enough patients to get meaningful results? And what are the shortcomings
of the research?

Historically, reporters have considered studies published by major
science journals --- like Nature, The Lancet and The New England Journal
of Medicine --- to be the most reliable, because those publications use
expert editors and rigorously vet research methods and conclusions by
sending them to other scientists to evaluate.

But last week, both The Lancet and The New England Journal of Medicine
\href{https://www.nytimes3xbfgragh.onion/2020/06/04/health/coronavirus-hydroxychloroquine.html?searchResultPosition=4}{retracted}
big coronavirus studies because they were based on data that could not
be verified. The Times had reported on
\href{https://www.nytimes3xbfgragh.onion/2020/05/22/health/malaria-drug-trump-coronavirus.html?searchResultPosition=2}{one
of those studies}.

Cases like this highlight the need for vigilance. Before citing such
research, Times reporters will generally seek out independent experts to
comment. That's critical now because during the pandemic, the length of
time from experiment to published study has been ``short-circuited,''
said
\href{https://www.nytimes3xbfgragh.onion/2017/10/06/insider/cuba-illness-sonic-weapons.html}{Mike
Mason}, a deputy science editor, adding that what once took as long as a
year and a half has been shortened in some cases to weeks.

Now many researchers post their work online in what is known as a
\href{https://www.nytimes3xbfgragh.onion/2020/04/14/science/coronavirus-disinformation.html?smid=em-share}{preprint},
or a study that gets released **** without the standard practice of peer
review. Some preprints contain observational or anecdotal work that
doesn't meet more exacting scientific standards like randomly selected
patients or control groups enabling more definitive conclusions. Times
reporters who write about preprints try to make clear to readers what
the research shows and what it leaves unanswered.

Those outside the field have rarely paid attention to these preprint
studies --- until now. Politicians and everyday citizens sometimes look
to studies in their infancy.

Reporters are not looking for the perfect study --- science is not
definitive and even the highest quality research has limitations. But
``right now, with all these reports pouring out, there's more
uncertainty than usual,'' said
\href{https://www.nytimes3xbfgragh.onion/by/denise-grady}{Denise Grady},
a science reporter. ``That's hard for everybody to accept at a time when
we all wish there were answers and a clear way forward.''

The Times's guidelines recognize that reporters must balance the weight
that they give to such studies with the need to provide information
about those that wind up in the global conversation.

``We will do everything we can to be straight with readers about what we
know and what we don't know,'' said Ms. Dugger. ``Science isn't the
discovery of a final truth or the be-all-end-all. It's a process. Our
knowledge will get better as we go along, but it's still worth sharing
with people the evolution of what we know.''

\begin{center}\rule{0.5\linewidth}{\linethickness}\end{center}

Advertisement

\protect\hyperlink{after-bottom}{Continue reading the main story}

\hypertarget{site-index}{%
\subsection{Site Index}\label{site-index}}

\hypertarget{site-information-navigation}{%
\subsection{Site Information
Navigation}\label{site-information-navigation}}

\begin{itemize}
\tightlist
\item
  \href{https://help.nytimes3xbfgragh.onion/hc/en-us/articles/115014792127-Copyright-notice}{©~2020~The
  New York Times Company}
\end{itemize}

\begin{itemize}
\tightlist
\item
  \href{https://www.nytco.com/}{NYTCo}
\item
  \href{https://help.nytimes3xbfgragh.onion/hc/en-us/articles/115015385887-Contact-Us}{Contact
  Us}
\item
  \href{https://www.nytco.com/careers/}{Work with us}
\item
  \href{https://nytmediakit.com/}{Advertise}
\item
  \href{http://www.tbrandstudio.com/}{T Brand Studio}
\item
  \href{https://www.nytimes3xbfgragh.onion/privacy/cookie-policy\#how-do-i-manage-trackers}{Your
  Ad Choices}
\item
  \href{https://www.nytimes3xbfgragh.onion/privacy}{Privacy}
\item
  \href{https://help.nytimes3xbfgragh.onion/hc/en-us/articles/115014893428-Terms-of-service}{Terms
  of Service}
\item
  \href{https://help.nytimes3xbfgragh.onion/hc/en-us/articles/115014893968-Terms-of-sale}{Terms
  of Sale}
\item
  \href{https://spiderbites.nytimes3xbfgragh.onion}{Site Map}
\item
  \href{https://help.nytimes3xbfgragh.onion/hc/en-us}{Help}
\item
  \href{https://www.nytimes3xbfgragh.onion/subscription?campaignId=37WXW}{Subscriptions}
\end{itemize}
