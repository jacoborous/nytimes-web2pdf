Sections

SEARCH

\protect\hyperlink{site-content}{Skip to
content}\protect\hyperlink{site-index}{Skip to site index}

\href{https://myaccount.nytimes3xbfgragh.onion/auth/login?response_type=cookie\&client_id=vi}{}

\href{https://www.nytimes3xbfgragh.onion/section/todayspaper}{Today's
Paper}

The Valet and Drifter Helping Me Get Through Quarantine

\url{https://nyti.ms/2UqUObr}

\begin{itemize}
\item
\item
\item
\item
\item
\item
\end{itemize}

\href{https://www.nytimes3xbfgragh.onion/spotlight/at-home?action=click\&pgtype=Article\&state=default\&region=TOP_BANNER\&context=at_home_menu}{At
Home}

\begin{itemize}
\tightlist
\item
  \href{https://www.nytimes3xbfgragh.onion/2020/09/07/travel/route-66.html?action=click\&pgtype=Article\&state=default\&region=TOP_BANNER\&context=at_home_menu}{Cruise
  Along: Route 66}
\item
  \href{https://www.nytimes3xbfgragh.onion/2020/09/04/dining/sheet-pan-chicken.html?action=click\&pgtype=Article\&state=default\&region=TOP_BANNER\&context=at_home_menu}{Roast:
  Chicken With Plums}
\item
  \href{https://www.nytimes3xbfgragh.onion/2020/09/04/arts/television/dark-shadows-stream.html?action=click\&pgtype=Article\&state=default\&region=TOP_BANNER\&context=at_home_menu}{Watch:
  Dark Shadows}
\item
  \href{https://www.nytimes3xbfgragh.onion/interactive/2020/at-home/even-more-reporters-editors-diaries-lists-recommendations.html?action=click\&pgtype=Article\&state=default\&region=TOP_BANNER\&context=at_home_menu}{Explore:
  Reporters' Google Docs}
\end{itemize}

Advertisement

\protect\hyperlink{after-top}{Continue reading the main story}

Supported by

\protect\hyperlink{after-sponsor}{Continue reading the main story}

\href{/column/letter-of-recommendation}{Letter of Recommendation}

\hypertarget{the-valet-and-drifter-helping-me-get-through-quarantine}{%
\section{The Valet and Drifter Helping Me Get Through
Quarantine}\label{the-valet-and-drifter-helping-me-get-through-quarantine}}

\includegraphics{https://static01.graylady3jvrrxbe.onion/images/2020/06/14/magazine/14Mag-LOR-01/14Mag-LOR-01-articleLarge.jpg?quality=75\&auto=webp\&disable=upscale}

By David Rees

\begin{itemize}
\item
  June 9, 2020
\item
  \begin{itemize}
  \item
  \item
  \item
  \item
  \item
  \item
  \end{itemize}
\end{itemize}

The last few months have clarified some things for me. One is that I'm
never going to finish reading Schreber's ``Memoirs of My Nervous
Illness'' or Wittgenstein's ``Culture and Value,'' so there's no reason
to keep those mind-bending clunkers on my bedside table. This awful era
asks many things of us; the maintenance of private pretensions is low on
the list. It took me about five minutes of self-quarantine to reach this
understanding. My second realization followed a couple of weeks later,
but it had been brewing since February. My second realization was that
my two favorite heroes in all of literature are Jeeves and Jack Reacher.

One of these heroes, created by the immortal humorist P.G. Wodehouse, is
a charming and brilliant British valet employed by a member of the idle
rich named Bertie Wooster, to Wooster's perpetual relief; the other is a
6-foot-5 former military police investigator and Marine sniper-contest
champion who wanders America punching his way into and out of trouble
thanks to the wonderfully artless imagination of Lee Child. These are
the characters whose stories I've been craving since this nightmare
began. I kicked off my pandemic reading binge by pulling a Jeeves
anthology from the bookshelf and a Reacher novel from the public
library. Once the libraries closed, I maintained my fix by downloading
Reacher novels to my phone. I end each day in bed mentally spooning with
one of these two goofy, gallant men.

For all the hours I've spent enjoying Wodehouse's stories over the
years, I couldn't tell you if I've read 20 Jeeves books or the same two
books 10 different times. They all flow together in an effervescent
stream of recurring characters with names like Gussie Fink-Nottle;
farcical mix-ups during lazy weekends at country estates; pratfalls and
plot twists and poached eggs, with language as carefree as skipped
stones across water. And above it all, the problem-solving Jeeves,
shimmering in and out of view like a river god's reflection. The
experience is as delightful as a dry martini sneaked from a
millionaire's bar cart. Jeeves books leave you lighter than they found
you.

As with Jeeves, I struggle when I try to draw apart the strands of the
enormous, tangled Jack Reacher knot in my memory. Is the book where Al
Qaeda operatives are working out of a suite at the Four Seasons Hotel
the same one in which Reacher somehow shoots a double agent in the
shoulder from outside a windowless punishment hut? Is the spunky police
officer Reacher beds before she detonates a dirty bomb the same woman he
makes love to after they're forced to dig the grave of an F.B.I. agent
who's been crucified by a cult leader? Regardless, the experience is as
wistfully carefree as a pile of cement shooting a flaming machine gun
made out of meat. Reacher books leave you kind of worn out.

Jeeves works within systems. His knowledge of the propriety and vanity
of the upper classes means he can anticipate their behavior and leverage
their own stupidity to serve Wooster's needs as well as his own. His
specialty is subterfuge as delicate as lacework, as when he convinced
Sir Roderick Glossop that the reason Wooster pushed his son into the
lake at Ditteredge Hall was that Wooster was insane, as evidenced by the
fish and cats stored in his bedroom, a theory that made its way to Lord
Bittlesham, undercutting the latter's planned confrontation with Wooster
for his (Wooster's) impersonation of Rosie M. Banks, author of ``Only a
Factory Girl,'' at the request of his old friend Bingo Little, who was
trying to acclimate his uncle to the idea of marrying beneath his
station.

Unlike Jeeves, Reacher has no fixed address; he has spent the last 20
years or so wandering the country, mostly by bus. (Sample travelogue:
``All cities are the same, and all cities are different. They all have
colors. Some are gray. This one was brown.'') He helps whoever stumbles
into him asking for a hand. While Jeeves seems to have all known facts
at his fingertips, when Reacher needs answers, he trudges to the library
or consults a phone book, or just wanders around town asking questions.
He doesn't own a phone or a car and spends more time walking than any
other fictional character I know. I read one Reacher book in which he
was so intrigued by a suspicious recycling plant that he spent something
like eight hours trudging to it and back, stopping only because he
tripped over a corpse. Jeeves makes doing good look easy; Reacher can't
hide that it's hard.

The vast stylistic differences between the two make it easier to
identify what they have in common: a superpower of unyielding
competence, the ability to do what they've been called to do using the
gifts they have, which is the most precious superpower of all. It always
hits me like a radio signal cutting through static, a broadcast from a
parallel dimension of possibility. More and more, I feel as if my
leverage on the world is slipping, that any talents I enjoy are inferior
to the darkness of the moment, that I'm failing as a man and a citizen.
Jeeves and Reacher suggest there's another way --- or maybe two, or
maybe more. For me, their complementary approaches of demure, catlike
problem-solving and doggish, walnut-knuckled obstinacy are as
mind-expanding and revelatory as anything Schreber and Wittgenstein ever
wrote. They are models for my better self, animated by the delight of a
better world and the fury of having to fight for it.

Advertisement

\protect\hyperlink{after-bottom}{Continue reading the main story}

\hypertarget{site-index}{%
\subsection{Site Index}\label{site-index}}

\hypertarget{site-information-navigation}{%
\subsection{Site Information
Navigation}\label{site-information-navigation}}

\begin{itemize}
\tightlist
\item
  \href{https://help.nytimes3xbfgragh.onion/hc/en-us/articles/115014792127-Copyright-notice}{©~2020~The
  New York Times Company}
\end{itemize}

\begin{itemize}
\tightlist
\item
  \href{https://www.nytco.com/}{NYTCo}
\item
  \href{https://help.nytimes3xbfgragh.onion/hc/en-us/articles/115015385887-Contact-Us}{Contact
  Us}
\item
  \href{https://www.nytco.com/careers/}{Work with us}
\item
  \href{https://nytmediakit.com/}{Advertise}
\item
  \href{http://www.tbrandstudio.com/}{T Brand Studio}
\item
  \href{https://www.nytimes3xbfgragh.onion/privacy/cookie-policy\#how-do-i-manage-trackers}{Your
  Ad Choices}
\item
  \href{https://www.nytimes3xbfgragh.onion/privacy}{Privacy}
\item
  \href{https://help.nytimes3xbfgragh.onion/hc/en-us/articles/115014893428-Terms-of-service}{Terms
  of Service}
\item
  \href{https://help.nytimes3xbfgragh.onion/hc/en-us/articles/115014893968-Terms-of-sale}{Terms
  of Sale}
\item
  \href{https://spiderbites.nytimes3xbfgragh.onion}{Site Map}
\item
  \href{https://help.nytimes3xbfgragh.onion/hc/en-us}{Help}
\item
  \href{https://www.nytimes3xbfgragh.onion/subscription?campaignId=37WXW}{Subscriptions}
\end{itemize}
