Sections

SEARCH

\protect\hyperlink{site-content}{Skip to
content}\protect\hyperlink{site-index}{Skip to site index}

\href{https://www.nytimes3xbfgragh.onion/section/food/drinks}{Wine, Beer
\& Cocktails}

\href{https://myaccount.nytimes3xbfgragh.onion/auth/login?response_type=cookie\&client_id=vi}{}

\href{https://www.nytimes3xbfgragh.onion/section/todayspaper}{Today's
Paper}

\href{/section/food/drinks}{Wine, Beer \& Cocktails}\textbar{}Why Wine?
Why Burgundy? Why Now?

\url{https://nyti.ms/2YEx4Cf}

\begin{itemize}
\item
\item
\item
\item
\item
\item
\end{itemize}

Advertisement

\protect\hyperlink{after-top}{Continue reading the main story}

Supported by

\protect\hyperlink{after-sponsor}{Continue reading the main story}

\href{/column/wines-of-the-times}{Wines of The Times}

\hypertarget{why-wine-why-burgundy-why-now}{%
\section{Why Wine? Why Burgundy? Why
Now?}\label{why-wine-why-burgundy-why-now}}

\includegraphics{https://static01.graylady3jvrrxbe.onion/images/2020/06/17/dining/11wine3/merlin_171788607_6f3454cf-1539-4ee8-9ea4-e5f9c06d0ffe-articleLarge.jpg?quality=75\&auto=webp\&disable=upscale}

By \href{https://www.nytimes3xbfgragh.onion/by/eric-asimov}{Eric Asimov}

\begin{itemize}
\item
  Published June 11, 2020Updated June 18, 2020
\item
  \begin{itemize}
  \item
  \item
  \item
  \item
  \item
  \item
  \end{itemize}
\end{itemize}

I had a lot of reasons not to write this column.

For one, it stretches back to the other side of the divide, before the
pandemic, when the wine panel was able to convene and taste together
without trepidation.

It seems unreal to recall sitting down in unmasked fellowship on a
chilly day at the end of February. We had the pleasure of tasting
village reds from the Côte de Beaune, the southern section of the heart
of Burgundy, all from the deliciously drinkable 2017 vintage.

As usual, Florence Fabricant and I were joined by two guests, Katja
Scharnagl, the chef sommelier at \href{https://www.le-bernardin.com/}{Le
Bernardin}, and Matthew Conway, the wine director and a partner at
\href{http://www.marcforgione.com/}{Marc Forgione} and
\href{https://www.peasantnyc.com/}{Peasant}.

So much has changed. Now, the future of these and many restaurants is
uncertain, as are the fate of millions of jobs and livelihoods.

The cost of the wine also gave me pause. Although these bottles are good
values given the current high cost of Burgundy, they are nonetheless
relatively expensive at \$30 to \$70. With so many people hurting,
should we really be considering them?

I had doubts as well beyond the pandemic. Given the strife and animosity
in the wake of the George Floyd killing, and the existential questions
Americans are now debating, how do we even begin to talk about wine?

It's an internal dialogue that I imagine occurs among many
serious-minded individuals who occasionally feel a sense of futility, as
if their fields have overnight become irrelevant frivolities given the
world's problems.

It is true that wine is often no more than a pleasant triviality,
something to take the edge off, to ease pain. Opening a bottle and
pouring a glass has always served as a popular mode of self-medication,
no more so than in these fearful, lonely times.

But good wine can also inspire thoughtful contemplation and
introspection, which perhaps now more than ever is in short supply. And
it can lead to caring conversations as well, to listening as well as
talking, to shared bonds, to new memories and more humane ways of
thinking.

I'm not saying good wine is a panacea. It's up to people to find
solutions. But wine has the power to bring people together as surely as
a great meal. Few wines are better equipped than Burgundy to inspire
this sort of reflection on values, joy, sorrow and shared humanity.

It's no accident that in the last 20 years, Burgundy has become the
\href{https://www.nytimes3xbfgragh.onion/2019/09/12/dining/drinks/value-burgundy-wine.html}{most
coveted wine} in the world. That is due in part to status-seeking and
trophy-chasing, undeniably.

But Burgundy's arrival as a luxury good followed its rise in popularity.
Part of its appeal is its soulfully complex, subtle and joyful aromas
and flavors. The pleasure of Burgundy is amplified many times over by an
intellectual attraction to the idea of terroir.

This notion that wine can express the culture of the place in which it
was produced is at its most powerful in Burgundy. Nowhere else are the
characteristics of land and community envisioned with such intricate
detail.

The basic idea is that where the grapes grow will dictate a wine's
personality. A Gevrey-Chambertin, in this way of thinking, will taste
different from a Volnay. It ought to, and if it doesn't, something is
wrong.

But that's just the beginning. The culture and upbringing of the
vigneron, the person who grows the grapes and makes the wine, is also
paramount. A Gevrey-Chambertin made by a Volnay vigneron will differ
from a Gevrey vigneron's Gevrey.

In Burgundy, that's to be expected. Not only do the soil, drainage,
bedrock and climate of Gevrey differ from those of Volnay, the entire
way of thinking and feeling, and therefore of making wine, differs as
well.

This is what makes Burgundy so fascinating to so many people. That's not
to suggest that the exercise of identifying one terroir rather than
another is easy. The differences are subtle and nuanced, and it takes
long-term immersive experience to achieve that level of expertise.

But experts can do it. For the rest of us, it's great and delicious fun
to try to decipher terroir, though just as often it's fine to ignore the
question altogether in favor of a good meal, good company and a
conversation that has nothing to do with wine.

I mentioned that our subject was reds from the Côte de Beaune. The heart
of Burgundy, the Côte d'Or, is a long, narrow swath of land that
encompasses limestone-and-clay soils, flatlands and a series of east-
and south-facing slopes south of the city of Dijon.

The northern half, the Côte de Nuits, contains the most esteemed red
wine appellations, including Gevrey-Chambertin, Morey-St.-Denis,
Chambolle-Musigny and Vosne/-Romanée.

The southern half, the Côte de Beaune, begins just north of the city of
Beaune, and includes wonderful white wine appellations like
Corton-Charlemagne, Meursault, Puligny-Montrachet and
Chassagne-Montrachet.

It also includes a number of red wine areas like Volnay, Pommard,
Monthelie and Santenay; numerous appellations around Beaune, including
Savigny-lès-Beaune and Chorey-lès-Beaune, and around Corton, like
Aloxe-Corton and Pernand-Vergelesses.

Within the hierarchy of Burgundy, in which vineyards are judged on their
potential to yield great and distinctive wines, the most exalted
vineyards are the grand crus. Just underneath are the premier cru
vineyards, which are thought to be exceptional enough to warrant
singling out, but not so great as to achieve the peak ranking.

These days, grand cru Burgundies are priced well beyond the reach of
most consumers. Increasingly, premier crus are as well, with many over
\$100 a bottle.

But village wines --- those that are distinctive enough to reflect the
characteristics of a village, but from vineyards not judged to have
further singular features --- are still sometimes within reach. Those
from the Côte de Beaune are especially good values, relatively speaking,
because the region has generally been more exalted for its whites than
for its reds.

We tasted 20 bottles from the 2017 vintage, which I called deliciously
drinkable because they won't require years of aging and because, yes,
the wines are easygoing in the best sense.

We all loved the wines. Our favorites were elegant, beautiful
expressions of what Burgundy has to offer. Matthew felt that climate
change had benefited village wines in that fully ripening the grapes was
far less of a problem than it might have been in some past decades.

Florence said 2017 was a great vintage for drinkers if not collectors,
while Katja pointed out that, as good as the vintage was, some of the
wines seemed too oaky. Those did not make our top 10.

Our top bottle was the graceful, energetic Savigny-lès-Beaune from
Chandon de Briailles, a serious wine that was nonetheless joyful, a neat
trick. But then, Chandon de Briailles is an excellent producer, and its
wines are often great values. Even in a year like 2017, they will reward
a bit of aging.

The three producers in the next tier were likewise superb, including our
runner-up, the opulent yet focused Pommard from Benjamin Leroux; the
balanced, energetic, slightly exotic Monthelie from Pierre Morey at No.
3; and the savory, complex Volnay from Henri Boillot.

These were followed by the juicy, graceful Chorey-lès-Beaune Les
Beaumonts from Louis Chenu Père \& Filles. This was my first encounter
with Chenu, run by two sisters who took over from their father. I very
much appreciated the easygoing grace of the wine.

At No. 6 was the subtle, complex Savigny-lès-Beaune Vieilles Vignes from
Pierre Guillemot, while No. 7 was the spicy, floral Aloxe-Corton from
Tollot-Beaut \& Fils, at \$65 the most expensive bottle in our top 10.

The next two bottles were cheaper, both \$30, excellent values even if
we liked them a cut less than the top seven. They included the rich yet
slightly rustic Chorey-lès-Beaune Clos Margot from Bernard Dubois \&
Fils and the lightly tannic, alluring Chorey-lès-Beaune Le Grand Saussy
from Camus-Bruchon.

The rich and resonant Santenay Vieilles Vignes from Paul Pillot rounded
out our top 10.

As I mentioned earlier, this was the result of a February pre-pandemic
tasting. It may be that you will find different wines in the marketplace
now, though 2017s should still be available.

I understand these wines are not cheap. But it's the cost of doing
business if you want to understand or enjoy Burgundy.

Some less expensive options might include wines from the Maranges area,
bottles labeled Hautes Côtes de Nuits or Hautes Côtes de Beaune and
those from the
\href{https://www.nytimes3xbfgragh.onion/2019/05/09/dining/drinks/wine-school-mercurey-burgundy-red.html}{Côte
Chalonnaise}, an area to the south of the Côte d'Or. Or consider a
splurge if the news should turn good.

It's only wine, I know. It's not a solution, but I'm glad that we have
it.

\hypertarget{tasting-notes-the-village-reds}{%
\subsection{Tasting Notes: The Village
Reds}\label{tasting-notes-the-village-reds}}

★★★½ \href{http://www.chandondebriailles.com/en/}{\textbf{Chandon de
Briailles}} \textbf{Savigny-lès-Beaune 2017 \$47}

Lovely, graceful and energetic, with delicate floral aromas and
lingering flavors of earthy red fruits. (Bowler Wine, New York)

★★★
\href{https://www.beckywasserman.com/domaines/benjamin-leroux/\#.Xtv-s55KiMI}{\textbf{Benjamin
Leroux}} \textbf{Pommard 2017 \$63}

Full, rich and opulent, yet tightly focused, with aromas and flavors of
flowers, red fruits and minerals. (Becky Wasserman \& Co./Verity Wine
Partners, New York)

★★★ \href{http://www.morey-meursault.fr/en/}{Pierre Morey} Monthelie
\$50 2017

Rich, balanced and energetic, with exotic aromas and flavors of earthy
red fruits; needs some time. (Becky Wasserman \& Co./Grand Cru
Selections, New York)

★★★ \href{http://www.henri-boillot.com/\#!en/}{\textbf{Henri Boillot}}
\textbf{Volnay 2017 \$53}

Rich, substantial and balanced, with savory, complex, earthy flavors.
(Wine Cellars, Briarcliff Manor, N.Y.)

★★★ \href{http://www.louischenu.com/indexuk.php}{\textbf{Louis Chenu
Père \& Filles}} \textbf{Chorey-lès-Beaune Les Beaumonts 2017 \$40}

Rich and juicy, yet graceful, with easygoing aromas and flavors of
flowers and red fruits. (Wilson Daniels Wholesale, New York)

★★★
\href{https://www.kermitlynch.com/our-wines/domaine-pierre-guillemot/}{\textbf{Pierre
Guillemot}} \textbf{Savigny-lès-Beaune Vieilles Vignes 2017 \$42}

Subtle, complex and focused, with floral, mineral and red-fruit flavors.
(Kermit Lynch Wine Merchant, Berkeley, Calif.)

★★★
\href{https://www.winebow.com/our-brands/domaine-tollot-beaut}{\textbf{Tollot-Beaut
\& Fils}} \textbf{Aloxe-Corton 2017 \$65}

Lightly tannic, yet pretty, with floral aromas and spicy, herbal
accents. (Craft \& Estate/Winebow, New York)

\textbf{Best Value}

★★½
\href{https://www.skurnik.com/producer/domaine-dubois-bernard-et-fils/}{\textbf{Bernard
Dubois \& Fils}} \textbf{Chorey-lès-Beaune Clos Margot 2017 \$30}

Earthy, lightly tannic and rustic, with rich flavors of dark fruits.
(Skurnik Wines, New York)

★★½
\href{https://polanerselections.com/producer/camus-bruchon}{\textbf{Camus-Bruchon}}
\textbf{Chorey-lès-Beaune Le Grand Saussy 2017 \$30}

Lightly tannic, with floral and red-fruit flavors, and a touch of
citrus. (Polaner Selections, Mount Kisco, N.Y.)

★★½
\href{https://www.skurnik.com/producer/domaine-paul-pillot/}{\textbf{Paul
Pillot}} \textbf{Santenay Vieilles Vignes 2017 \$50}

Rich, robust and resonant, with fresh, red berry fruit. (Skurnik Wines)

Pairings:
\href{https://cooking.nytimes3xbfgragh.onion/recipes/1021120-swordfish-au-poivre}{\textbf{Swordfish
au Poivre}}

Au poivre, the peppery French finish for a steak, is simpler and more
versatile than its fancy-sounding name suggests. A quick pan sauce of
cream and Cognac enrobes a seared piece of meat fueled with crushed
black or green peppercorns. But it doesn't have to be used only for
meat. For example, at Veronika, a gilded restaurant that landed in
Fotografiska New York in Gramercy before the pandemic, the chef, Robert
Aikens, applied the au poivre formula to a thick fist of tender
celeriac, with excellent results. Boneless chicken breasts are another
possibility. Here, to accompany the youngish pinot noirs in our glasses,
I opted for something less robust than beef: swordfish steaks, though
you could use another densely textured slab of fish like halibut or
grouper. But producing au poivre is strictly à la minute: Have your
ingredients ready to apply so the wait time for serving is minimal. The
recipe is easily reduced to serve two for that date-night dinner while
sequestered at home with a good bottle of Burgundy to share. Light the
candles.

\textbf{FLORENCE FABRICANT}

Advertisement

\protect\hyperlink{after-bottom}{Continue reading the main story}

\hypertarget{site-index}{%
\subsection{Site Index}\label{site-index}}

\hypertarget{site-information-navigation}{%
\subsection{Site Information
Navigation}\label{site-information-navigation}}

\begin{itemize}
\tightlist
\item
  \href{https://help.nytimes3xbfgragh.onion/hc/en-us/articles/115014792127-Copyright-notice}{©~2020~The
  New York Times Company}
\end{itemize}

\begin{itemize}
\tightlist
\item
  \href{https://www.nytco.com/}{NYTCo}
\item
  \href{https://help.nytimes3xbfgragh.onion/hc/en-us/articles/115015385887-Contact-Us}{Contact
  Us}
\item
  \href{https://www.nytco.com/careers/}{Work with us}
\item
  \href{https://nytmediakit.com/}{Advertise}
\item
  \href{http://www.tbrandstudio.com/}{T Brand Studio}
\item
  \href{https://www.nytimes3xbfgragh.onion/privacy/cookie-policy\#how-do-i-manage-trackers}{Your
  Ad Choices}
\item
  \href{https://www.nytimes3xbfgragh.onion/privacy}{Privacy}
\item
  \href{https://help.nytimes3xbfgragh.onion/hc/en-us/articles/115014893428-Terms-of-service}{Terms
  of Service}
\item
  \href{https://help.nytimes3xbfgragh.onion/hc/en-us/articles/115014893968-Terms-of-sale}{Terms
  of Sale}
\item
  \href{https://spiderbites.nytimes3xbfgragh.onion}{Site Map}
\item
  \href{https://help.nytimes3xbfgragh.onion/hc/en-us}{Help}
\item
  \href{https://www.nytimes3xbfgragh.onion/subscription?campaignId=37WXW}{Subscriptions}
\end{itemize}
