Sections

SEARCH

\protect\hyperlink{site-content}{Skip to
content}\protect\hyperlink{site-index}{Skip to site index}

\href{https://myaccount.nytimes3xbfgragh.onion/auth/login?response_type=cookie\&client_id=vi}{}

\href{https://www.nytimes3xbfgragh.onion/section/todayspaper}{Today's
Paper}

\href{/section/opinion}{Opinion}\textbar{}Your Climate Disaster Tax Bill
Is Growing

\url{https://nyti.ms/37Xkq5w}

\begin{itemize}
\item
\item
\item
\item
\item
\end{itemize}

Advertisement

\protect\hyperlink{after-top}{Continue reading the main story}

\href{/section/opinion}{Opinion}

Supported by

\protect\hyperlink{after-sponsor}{Continue reading the main story}

\hypertarget{your-climate-disaster-tax-bill-is-growing}{%
\section{Your Climate Disaster Tax Bill Is
Growing}\label{your-climate-disaster-tax-bill-is-growing}}

The federal government's spending on calamities related to global
warming is a rapidly rising fiscal threat.

By Paul Bodnar and Tamara Grbusic

The authors focus on global climate finance at Rocky Mountain Institute.

\begin{itemize}
\item
  June 23, 2020
\item
  \begin{itemize}
  \item
  \item
  \item
  \item
  \item
  \end{itemize}
\end{itemize}

\includegraphics{https://static01.graylady3jvrrxbe.onion/images/2020/06/23/opinion/23bodhar1/merlin_160189707_fa15f423-f0bc-4b35-b55b-72e5c47a44d2-articleLarge.jpg?quality=75\&auto=webp\&disable=upscale}

The coronavirus crisis has forced the federal government to step up
suddenly with fiscal stimulus to sustain the U.S. economy and help avoid
a global depression. This necessary intervention comes at a price --- a
spike in federal debt that will need to be repaid over the long term.
The resulting pressure on the government, U.S. taxpayers and the broader
economy will intersect with another major fiscal challenge, one that we
have yet to reckon with:
\href{https://www.nytimes3xbfgragh.onion/2020/07/14/us/politics/biden-climate-plan.html}{climate
change}.

Even before the coronavirus pandemic struck, the federal government's
spending on climate-related disaster recovery was a rapidly rising
fiscal threat. In response to climate-related disasters in 2017, for
example, Congress
\href{https://fas.org/sgp/crs/homesec/R45084.pdf}{appropriated \$136
billion} in additional funding for recovery --- amounting to about
\$1,000 for every American taxpayer.

The government faces wide exposure, including repairing damage to
federal property and lands, federal insurance for property and crops,
the cost of making public infrastructure resilient to
\href{https://www.nytimes3xbfgragh.onion/2020/07/14/us/politics/biden-climate-plan.html}{climate}
impacts, and disaster aid (including
\href{https://www.npr.org/2017/01/10/509176361/alaskan-village-citing-climate-change-seeks-disaster-relief-in-order-to-relocate\#:~:text=Alaska\%20Climate\%20Change\%3A\%20Village\%20Makes\%20Historic\%20Request\%20To\%20Be\%20Declared,needs\%20to\%20move\%20the\%20community.}{relocation}
of entire populations in harm's way of persistent climate repercussions
like sea level rise).

Fourteen\href{https://www.climate.gov/news-features/blogs/beyond-data/2010-2019-landmark-decade-us-billion-dollar-weather-and-climate}{billion-dollar}
weather and climate calamities struck last year, the fifth year in a row
with 10 or more. And projections don't look good.

The National Oceanic and Atmospheric Administration
\href{https://www.ncdc.noaa.gov/billions/}{estimates} that the average
annual number of disasters causing over \$1 billion in damages over the
past five years was double the average over the past four decades. The
agency warned last year, ``The number and cost of disasters are
increasing over time due to a combination of increased exposure,
vulnerability, and the fact that climate change is increasing the
frequency of some types of extremes that lead to billion-dollar
disasters.''

Overall, according to the government's
\href{https://nca2018.globalchange.gov/}{national climate assessment} in
2018, continued warming ``is expected to cause substantial net damage to
the U.S. economy throughout this century, especially in the absence of
increased adaptation efforts.''

BlackRock, the global investment management firm, calculates a
\href{https://www.blackrock.com/us/individual/literature/whitepaper/bii-physical-climate-risks-april-2019.pdf}{275
percent increase} in major hurricane risk by 2050 under a ``no climate
action'' scenario that assumes the continued use of fossil fuels. In
California, devastating wildfires forced the utility PG\&E to declare
bankruptcy last year when its insurance policy of \$1.4 billion
\href{https://www.forbes.com/sites/christopherhelman/2018/11/15/californias-electric-giant-faces-possible-30-billion-in-damages-for-fires-that-have-killed-more-than-100/\#5079293238c2}{paled
in comparison} with damages of \$30 billion.

The federal government will bear an increasing share of these losses as
private insurance declines to provide coverage for
\href{https://www.nytimes3xbfgragh.onion/interactive/2020/06/29/climate/hidden-flood-risk-maps.html}{flood}-
and
\href{https://www.wsj.com/articles/no-one-can-agree-on-how-to-price-california-home-insurance-for-wildfires-11568649298}{wildfire-prone}
property. U.S. coastal real estate properties have been likened to
\href{https://nam11.safelinks.protection.outlook.com/?url=https\%3A\%2F\%2Fwww.bloomberg.com\%2Fopinion\%2Farticles\%2F2018-05-03\%2Fflood-risk-makes-coastal-real-estate-look-like-a-junk-bond\%3Fsref\%3D8nb4B7zL\&data=02\%7C01\%7Cnsteel\%40rmi.org\%7Cadec35ab733049a0b57008d816fd9823\%7C8ed8a585d8e64b00b9ccd370783559f6\%7C1\%7C0\%7C637284625728410773\&sdata=mCgVuOl8Z2sg76JVje\%2B100Q5ANUcp2gtGuEpCqupcok\%3D\&reserved=0}{junk
bonds} --- ``something that will probably go up in value, but has a
small to moderate chance of going to zero,'' as the Bloomberg opinion
columnist Noah Smith put it. It's unsurprising that private insurers
have not filled the gap left by
\href{http://www.ouazad.com/resources/paper_kahn_ouazad.pdf}{declining}
federal flood insurance.

\begin{center}\rule{0.5\linewidth}{\linethickness}\end{center}

Additionally, the ability of state and local governments to absorb
disaster costs is limited because they cannot borrow as the federal
government can. The only recourse for states overburdened with disaster
costs is to turn to the federal government. And the pandemic has
\href{https://www.nytimes3xbfgragh.onion/2020/05/22/climate/fema-volunteer-disaster-response.html}{underscored
the shortcomings} in the federal government's capacity to respond to
multiple disasters.

Furthermore, banks are
\href{https://www.nytimes3xbfgragh.onion/2019/09/27/climate/mortgage-climate-risk.html?campaignId=7JFJX}{actively
offloading risky mortgages} onto the government-backed mortgage lenders
Fannie Mae and Freddie Mac. As defaults rise with worsening climate, so
will the direct liability of the federal government. The Federal
Emergency Management Agency's National Flood Insurance Program is about
\href{https://www.insurancejournal.com/blogs/right-street/2018/09/13/500996.htm}{\$20.5
billion in debt} --- meaning the government has so far chosen to absorb
losses while also increasing premiums. As the Federal Reserve Bank of
San Francisco put it, because of low ``risk awareness and insurance
affordability,'' many government agencies ``have found themselves being
expected to act as
\href{https://www.frbsf.org/community-development/publications/community-development-investment-review/2019/october/insurance-innovation-and-community-based-adaptation-finance/\#_ftn2}{insurers
of}\href{https://www.frbsf.org/community-development/publications/community-development-investment-review/2019/october/insurance-innovation-and-community-based-adaptation-finance/\#_ftn2}{\emph{first}}\href{https://www.frbsf.org/community-development/publications/community-development-investment-review/2019/october/insurance-innovation-and-community-based-adaptation-finance/\#_ftn2}{resort}.''

A substantial increase in disaster spending could threaten the factor
that has pushed the ballooning federal debt in the first place: the U.S.
government's low cost of borrowing. U.S. Treasury notes are considered a
safe haven for investors, but there are signs this may be changing.
While the credit rating agencies
\href{https://www.reuters.com/article/usa-ratings-sp/sp-affirms-us-amid-coronavirus-outbreak-says-debt-and-deficit-will-worsen-idUSFWN2BQ1CV}{Fitch
and S \& P} maintained their U.S. ratings at the beginning of the
coronavirus outbreak,
\href{https://www.fitchratings.com/research/us-public-finance/fitch-affirms-united-states-at-aaa-outlook-stable-26-03-2020}{Fitch
noted} that the short-term risk of downgrades increased in light of the
economic shock. At the same time, debt projections continue to increase:
The federal deficit, which exceeded
\href{https://www.nytimes3xbfgragh.onion/2020/01/13/business/budget-deficit-1-trillion-trump.html}{\$1
trillion in 2019}, is expected to reach
\href{https://www.washingtonpost.com/us-policy/2020/04/18/record-government-corporate-debt-risk-tipping-point-after-pandemic-passes/}{\$4
trillion} this year.

To fund climate disaster expenses of the magnitude described above, the
federal government would have to significantly escalate its borrowing.
Rating agencies are increasingly focused on climate-specific fiscal
pressures. As BlackRock stated, Moody's ``warned that climate change
would have a growing negative impact on the creditworthiness of U.S.
state and local insurers.'' A 2018
\href{http://wedocs.unep.org/bitstream/handle/20.500.11822/26007/Climate_Change_Costs.pdf?sequence=1\&isAllowed=y}{study}
by the United Nations Environment Program found that ``countries with
higher degrees of climate vulnerability face higher sovereign borrowing
costs.''

What can the government do to reduce its exposure to climate-related
disasters? Cut greenhouse gas emissions and ramp up spending to reduce
property exposure to climate-fueled storms and droughts.

We have a choice between a carbon tax and a spiraling climate disaster
tax. In a fast-approaching future where higher public spending and
escalating debt will require higher levels of taxation, a carbon tax is
a prudent choice. It can provide an important source of revenue,
encourage industries to decarbonize and lower the danger of further
credit rating downgrades --- all while decreasing future disaster risk
by reducing emissions.

While doing this now
\href{https://earthobservatory.nasa.gov/features/HeatBucket/heatbucket4.php}{will
not appreciably affect} climate disasters for some time, there is no
doubt that ambitious action to reduce emissions worldwide under American
leadership can reduce the long-term financial exposure of today's young
Americans. It will leave the country better prepared to pay for other
crises that arise --- like the one we are currently facing with
Covid-19.

\href{https://rmi.org/people/paul-bodnar/}{Paul Bodnar} is a managing
director at \href{https://rmi.org/}{Rocky Mountain Institute}, where
\href{https://rmi.org/people/tamara-grbusic/}{Tamara Grbusic} is an
associate, both focused on global climate finance.

\emph{The Times is committed to publishing}
\href{https://www.nytimes3xbfgragh.onion/2019/01/31/opinion/letters/letters-to-editor-new-york-times-women.html}{\emph{a
diversity of letters}} \emph{to the editor. We'd like to hear what you
think about this or any of our articles. Here are some}
\href{https://help.nytimes3xbfgragh.onion/hc/en-us/articles/115014925288-How-to-submit-a-letter-to-the-editor}{\emph{tips}}\emph{.
And here's our email:}
\href{mailto:letters@NYTimes.com}{\emph{letters@NYTimes.com}}\emph{.}

\emph{Follow The New York Times Opinion section on}
\href{https://www.facebookcorewwwi.onion/nytopinion}{\emph{Facebook}}\emph{,}
\href{http://twitter.com/NYTOpinion}{\emph{Twitter (@NYTopinion)}}
\emph{and}
\href{https://www.instagram.com/nytopinion/}{\emph{Instagram}}\emph{.}

Advertisement

\protect\hyperlink{after-bottom}{Continue reading the main story}

\hypertarget{site-index}{%
\subsection{Site Index}\label{site-index}}

\hypertarget{site-information-navigation}{%
\subsection{Site Information
Navigation}\label{site-information-navigation}}

\begin{itemize}
\tightlist
\item
  \href{https://help.nytimes3xbfgragh.onion/hc/en-us/articles/115014792127-Copyright-notice}{©~2020~The
  New York Times Company}
\end{itemize}

\begin{itemize}
\tightlist
\item
  \href{https://www.nytco.com/}{NYTCo}
\item
  \href{https://help.nytimes3xbfgragh.onion/hc/en-us/articles/115015385887-Contact-Us}{Contact
  Us}
\item
  \href{https://www.nytco.com/careers/}{Work with us}
\item
  \href{https://nytmediakit.com/}{Advertise}
\item
  \href{http://www.tbrandstudio.com/}{T Brand Studio}
\item
  \href{https://www.nytimes3xbfgragh.onion/privacy/cookie-policy\#how-do-i-manage-trackers}{Your
  Ad Choices}
\item
  \href{https://www.nytimes3xbfgragh.onion/privacy}{Privacy}
\item
  \href{https://help.nytimes3xbfgragh.onion/hc/en-us/articles/115014893428-Terms-of-service}{Terms
  of Service}
\item
  \href{https://help.nytimes3xbfgragh.onion/hc/en-us/articles/115014893968-Terms-of-sale}{Terms
  of Sale}
\item
  \href{https://spiderbites.nytimes3xbfgragh.onion}{Site Map}
\item
  \href{https://help.nytimes3xbfgragh.onion/hc/en-us}{Help}
\item
  \href{https://www.nytimes3xbfgragh.onion/subscription?campaignId=37WXW}{Subscriptions}
\end{itemize}
