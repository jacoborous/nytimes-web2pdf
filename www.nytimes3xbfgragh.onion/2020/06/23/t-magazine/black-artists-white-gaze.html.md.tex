Nine Black Artists and Cultural Leaders on Seeing and Being Seen

\url{https://nyti.ms/2Z0n06H}

\begin{itemize}
\item
\item
\item
\item
\item
\end{itemize}

\includegraphics{https://static01.graylady3jvrrxbe.onion/images/2020/06/28/t-magazine/18tmag-blackwriters-slide-1V55/18tmag-blackwriters-slide-1V55-articleLarge.jpg?quality=75\&auto=webp\&disable=upscale}

Sections

\protect\hyperlink{site-content}{Skip to
content}\protect\hyperlink{site-index}{Skip to site index}

\hypertarget{nine-black-artists-and-cultural-leaders-on-seeing-and-being-seen}{%
\section{Nine Black Artists and Cultural Leaders on Seeing and Being
Seen}\label{nine-black-artists-and-cultural-leaders-on-seeing-and-being-seen}}

Amy Sherald, Michael R. Jackson and others discuss the challenges and
opportunities of cultivating black audiences and dismantling
historically white institutions.

Amy Sherald's ``Precious Jewels by the Sea'' (2019).Credit...© Amy
Sherald. Courtesy of Hauser \& Wirth

Supported by

\protect\hyperlink{after-sponsor}{Continue reading the main story}

``If you're silent about your pain, they'll kill you and say you enjoyed
it,'' wrote Zora Neale Hurston in her 1937 novel
``\href{https://www.harpercollins.com/9780060838676/their-eyes-were-watching-god/}{Their
Eyes Were Watching God}.'' Throughout this country's history, black
Americans have been reminded near daily that this remains true --- both
literally and more obliquely. In creative fields, for instance, from the
visual arts to theater, the white gaze has long determined whose stories
are told --- what gets to be seen, what's given value and what's deemed
worthy enough to be recorded and remembered --- enforcing a seemingly
immovable standard by which black artists and other artists of color are
nearly always cast in supporting roles to the mostly white stars of the
Western canon.

Today, though, many black artists are actively resisting that idea,
creating work that speaks directly to a black audience, a black gaze, in
order to reform the often whitewashed realms in which they practice. We
talked with nine of them --- each a voice of this moment, as the nation
reckons with the deaths of
\href{https://www.nytimes3xbfgragh.onion/news-event/george-floyd-protests-minneapolis-new-york-los-angeles}{George
Floyd},
\href{https://www.nytimes3xbfgragh.onion/article/breonna-taylor-police.html}{Breonna
Taylor} and others, and beyond --- about making work that captures ****
the richness and variety of black life. Whether it's the artist
\href{https://www.nytimes3xbfgragh.onion/2019/05/31/t-magazine/tschabalala-self-artist-studio.html}{Tschabalala
Self} discussing the fraught experience of seeing her paintings be sold,
like her ancestors, at auction or the Pulitzer Prize-winning playwright
\href{https://www.nytimes3xbfgragh.onion/interactive/2019/04/10/t-magazine/virgil-abloh-michael-jackson.html}{Michael
R. Jackson} searching for his characters' interiority, their
perspectives distill what it means (and what it has meant) to be black
in America. --- NOOR BRARA

\emph{These interviews have been edited and condensed.}

Image

Amy Sherald's ``When I Let Go of What I Am, I Become What I Might Be
(Self-Imagined Atlas)'' (2018).Credit...© Amy Sherald. Courtesy of
Hauser \& Wirth

Image

Sherald's ``The Girl Next Door'' (2019).Credit...© Amy Sherald. Courtesy
of Hauser \& Wirth

\hypertarget{i-always-want-the-work-to-be-a-resting-place-for-black-people}{%
\subsection{`I always want the work to be a resting place for black
people.'}\label{i-always-want-the-work-to-be-a-resting-place-for-black-people}}

\hypertarget{by-amy-sherald-46-a-baltimore-based-painter}{%
\subsubsection{\texorpdfstring{\textbf{By}
\textbf{\href{http://www.amysherald.com/}{Amy Sherald}, 46, a
Baltimore-based
painter}}{By Amy Sherald, 46, a Baltimore-based painter}}\label{by-amy-sherald-46-a-baltimore-based-painter}}

I realized very quickly, once I crossed into painting the black figure,
that we are a political statement in and of ourselves, especially when
we're hanging on the walls of museums and institutions. Because of that,
I knew I didn't want the work to be marginalized any further, and I
didn't want the conversation to be solely about identity or politics ---
our images deserve more than that. And that accounts, I think, for why I
paint in grayscale.

For a long time, I felt the work wasn't good enough. But then I started
asking the right questions: If I hadn't been born in Columbus, Ga.,
where I had to perform my identity based on how the lines were drawn
down in the South, who would I be? If I wasn't so aware of my blackness
because it had been placed against the stark white background of my
private school, how would I see myself?

I was excited by American realism in the early 2000s and began thinking
about how I hadn't seen any work about just black people being black,
captured in moments that were nothing special. For years, I've been
trying to find the language for what draws me to my subjects, most of
whom I've cast by just running into them while out living my life, then
photographing and painting them. I still can't explain it, but I always
use this example of walking into a room and catching the eye of someone
warm and familiar-looking and thinking, ``Huh.'' If these people were
furniture, they'd be like antique furniture, like midcentury modern, you
know what I mean? They seem as though, in their spirit, they've been
around for a while. I always want the work to be a resting place for
black people, one where you can let your guard down among figures you
**** understand.

Yet white collectors continue to ask me if I'm ever going to paint white
people. It's interesting to me because it shows me they recognize the
absence of themselves in a room full of my paintings but don't recognize
the absence of us in the greater narrative. I always tell them, ``You
should go look at a history book and get back to me. Thumb through and
take note of how many times you see something that looks like this, and
then let's have another conversation.'' --- \emph{As told to N.B.}

\begin{center}\rule{0.5\linewidth}{\linethickness}\end{center}

\includegraphics{https://static01.graylady3jvrrxbe.onion/images/2020/06/23/t-magazine/18tmag-blackwriters-slide-6MXX/18tmag-blackwriters-slide-6MXX-articleLarge.jpg?quality=75\&auto=webp\&disable=upscale}

\hypertarget{appropriating-my-favorite-white-gaze-of-all}{%
\subsection{`Appropriating my favorite white gaze of
all.'}\label{appropriating-my-favorite-white-gaze-of-all}}

\hypertarget{by-michael-r-jackson-39-a-new-york-based-playwright}{%
\subsubsection{\texorpdfstring{\textbf{By}
\textbf{\href{https://www.thelivingmichaeljackson.com/}{Michael R.
Jackson}, 39, a New York-based
playwright}}{By Michael R. Jackson, 39, a New York-based playwright}}\label{by-michael-r-jackson-39-a-new-york-based-playwright}}

When I think about the now popular idea of ``confronting the white
gaze'' in theater, I think about the fact that I was born into a black
family in a predominantly black city (Detroit), where I attended a black
church and predominantly black schools taught by predominantly black
teachers alongside predominantly black students. The first boys I kissed
were black. The first boys I did anything more than kissing with were
black. When my father would sit me down to tell me about the evils of
the white man, I would roll my eyes because, at the time, the man I felt
most spooked by was not some racist white man --- it was my father, who
was black. I grew up in such a black context that eventually I had to
rebel against it. So I moved to New York City at 18 to study
playwriting.

In my college plays at New York University, as in most of the short
stories and poems I had written in high school, the central characters
were black. I remember taking a master class with the playwright Kenneth
Lonergan, who brought in two white actors to read scenes from all of our
plays aloud. Because this was a pre-woke world, I had to listen to them
read my very black, Southern-born characters' dialogue in a ``This Is
Our Youth'' dialect. As cringe-worthy as that experience was, it was a
seminal moment: the first time I recognized white consciousness as the
default in theater. But while I recognized that, I was not intimidated
by it, because my default consciousness had always been black. I saw the
world through artistically, culturally and sociopolitically black eyes.
Because of how race is constructed, I understood how whiteness shaped
the world my blackness lived in. But I did not cater to it.

In my musical ``A Strange Loop'' (which last year completed a run at
Playwrights Horizons in association with Page 73 Productions and
recently won the 2020 Pulitzer Prize in drama), the protagonist is a
black queer man named Usher who is writing a musical about a black queer
man who is writing a musical about a black queer man who is writing a
musical about a black queer man ad infinitum. I constructed the play in
this way in order to explore the interiority of a black man without
having to sacrifice him to the trauma of slavery or police violence. I
wanted to capture the everyday misery of being a self. For some, this
structure is about ``confronting the white gaze.'' For me, it's about
what it's been since I first began writing stories: being myself. If
being myself is confronting the white gaze, then I suppose the only way
I can explain my supposedly confrontational strategy is by appropriating
my favorite white gaze of all: that of the character Joanne in George
Furth and Stephen Sondheim's 1970 musical ``Company,'' in which she
says, ``Sometimes I catch him looking and looking. And I just look right
back.'' \emph{--- Michael R. Jackson}

\begin{center}\rule{0.5\linewidth}{\linethickness}\end{center}

Image

Tschabalala Self's ``Loner'' (2016).Credit...Courtesy of Pilar Corrias
and Eva Presenhuber Gallery

Image

Self's ``No'' (2019).Credit...Courtesy of Pilar Corrias and Eva
Presenhuber Gallery

\hypertarget{i-do-not-feel-comfortable-for-example-with-my-works-being-up-for-auction}{%
\subsection{`I do not feel comfortable, for example, with my works being
up for
auction.'}\label{i-do-not-feel-comfortable-for-example-with-my-works-being-up-for-auction}}

\hypertarget{by-tschabalala-self-30-a-new-haven-conn-based-painter}{%
\subsubsection{\texorpdfstring{By
\href{https://tschabalalaself.com/}{Tschabalala Self}, 30, a New Haven,
Conn.-based
painter}{By Tschabalala Self, 30, a New Haven, Conn.-based painter}}\label{by-tschabalala-self-30-a-new-haven-conn-based-painter}}

I focus on the fantasies placed upon the female body because I can speak
more earnestly to that experience (having lived it), but I think it's
obvious from what everyone has seen --- everyone who cares to see,
anyway --- that there are lots of falsehoods associated with black
people.

Race is understood primarily on a physiological level: through one's
color, features and build --- one's literal physical form. And so racism
is therefore a preoccupation with control over the body and, in turn,
disdain and desire are projected onto that body. There is one kind of
blackness in America that's publicly praised --- one that seems to
support the general consensus of black worth --- while the larger, more
commonplace reality of American blackness is often ignored. I attempt to
explore that duality in my practice.

When I'm making a work, I primarily think about the subject in the
painting. They're always imagined individuals; I don't work with real
people. My main objective, first and foremost, is to create a
charismatic, interesting and complex character so that they can function
as a true subject instead of an object. When I'm thinking about them in
relation to gaze, I think about the community I'm from, which is the
black community --- I grew up in Harlem --- so that's who I'm imagining
experiencing the work. That's the cultural framework in which I feel the
work is best understood.

Still, I'm very skeptical of the fetishization of black artists that's
consumed the current moment. I do not feel comfortable, for example,
with my works being up for auction. It's entirely inappropriate and
unnecessary to auction work, especially mine: I'm a black American
artist, and I paint black bodies. I'm a descendant of slaves in this
country, so it's unfathomable that people could come to me, with glee,
to ask if I'm excited about seeing my work, which shows black figures
and bodies, being auctioned. That shows me that people have no real
understanding of black American history, and they don't understand
anything about me and the specificity of my ethnicity as a black person
in America. It's over their heads. --- \emph{As told to N.B.}

\begin{center}\rule{0.5\linewidth}{\linethickness}\end{center}

Image

Wardell Milan's ``Amerika: Klansman, Theophilus''
(2019).Credit...Courtesy of the artist and David Nolan Gallery

Image

Milan's ``Amerika: Klansman, Pulaski'' (2019).Credit...Courtesy of the
artist and David Nolan Gallery

\hypertarget{i-am-extending-an-invitation-to-the-viewer-to-discuss-issues-that-are-troubling-prescient-and-fraught}{%
\subsection{`I am extending an invitation to the viewer to discuss
issues that are troubling, prescient and
fraught.'}\label{i-am-extending-an-invitation-to-the-viewer-to-discuss-issues-that-are-troubling-prescient-and-fraught}}

\hypertarget{by-wardell-milan-42-a-new-york-based-visual-artist}{%
\subsubsection{\texorpdfstring{\textbf{By}
\textbf{\href{https://wardellmilan.com/}{Wardell Milan}, 42, a New
York-based visual
artist}}{By Wardell Milan, 42, a New York-based visual artist}}\label{by-wardell-milan-42-a-new-york-based-visual-artist}}

Some of my most recent collages deal with Klansmen, in the hopes of
producing conversations about race relations, both contemporary and
historical, here in America, especially given the rise of white
nationalism from 2016 on. I am captivated by the people behind these
masks. I think about their level of humanity. I think about how they
exist in the world as people with souls, morals, jobs and families. We
don't share the same beliefs in those ethics, but people have these
roles within the Klan as individuals.

I am interested in having straightforward conversations, and I am
extending an invitation to the viewer to discuss issues that are
troubling, prescient and fraught --- issues that some may deem
inconsequential. I'm trying to communicate these conceptual narratives
in a way that allows audiences from a number of different backgrounds to
engage: I want to shift the focus of the conversation around
predominantly white institutions so that the institutions that have
grown around these hegemonic ideals can be restructured. I am not
considering one specific audience when making the work. I am just
focusing on the work itself, and how it relates to a white viewer, a
black **** viewer and a transgender viewer depends on the viewer
themselves. I cease to have a sense of ownership over my work once it
leaves the studio, but I want the work to have life outside of \emph{me}
--- to have agency --- and for the audience to consider what I'm trying
to say. The goal is to create pieces that will be relevant long after I
am here on this earth. They are my own personal pyramids. --- \emph{As
told to Tiana Reid}

Image

Renée Cox's ``Raje for President'' (1998), part of her ``Raje, a
Superhero'' series.Credit...Courtesy of the artist

Image

Cox's ``A Covid Wavelength'' (2020).Credit...Courtesy of the artist

\hypertarget{im-trying-to-create-my-own-propaganda-for-the-enhancement-of-black-folks}{%
\subsection{`I'm trying to create my own propaganda for the enhancement
of black
folks.'}\label{im-trying-to-create-my-own-propaganda-for-the-enhancement-of-black-folks}}

\hypertarget{by-renuxe9e-cox-59-a-new-york-based-visual-artist}{%
\subsubsection{\texorpdfstring{\textbf{By}
\textbf{\href{https://www.reneecox.org/}{Renée Cox}, 59, a New
York-based visual
artist}}{By Renée Cox, 59, a New York-based visual artist}}\label{by-renuxe9e-cox-59-a-new-york-based-visual-artist}}

In my work, I return the power of the gaze to the subject --- and
usually my subjects are black. My subjects come off very strong and
empowered. They don't fall into the stereotypes of black people that
white people have created. That's something that has been interesting to
present in the art world since 1993, when I was the first artist of
color to blow up the black body to over seven feet tall and
unapologetically return the gaze to the viewer. If you're presenting
black people as victims, that goes a longer way to the bank, but that
doesn't change the status quo of the power structure of racism (because
racism is about power and economics). I have been more interested in
upsetting that paradigm, in at least having the fantasy of having ****
the power, if not the reality.

A lot of my subject matter is me. I was born in Jamaica in 1960, and my
family never made me feel like I was a victim. They always made me feel
like I was on the same level as anybody else --- intellectually and,
though perhaps not penny for penny, economically. I certainly wasn't
struggling. Coming from that background, I've never had to walk with my
eyes cast down to the floor because I couldn't make eye contact with
somebody. I've always felt that I was either on the same level or above
people {[}\emph{laughs}{]}. I never felt like I had to pretend to be
lesser than in order to make Caucasians feel comfortable.

I'm trying to create my own propaganda for the enhancement of black
folks. I'm focused on people who look like me. That's why I've returned
the gaze: to let my people know that they don't need to have this
subservient slave mentality. Returning the gaze is natural for me but
there's a radical nature to it, too. A lot of artists don't want to talk
about it because they're afraid. Some are comfortable with the monies
and the accolades. As far as I'm concerned, they're not doing anything
for the race. And they'll say, ``Well, I don't have to. I'm just an
artist. I'm not a black artist, I'm just an artist. I'm not a black
woman artist, I'm just an artist.'' What are you talking about? When I
walk into the room, they see I'm a black woman artist. When they look at
the work, they assume it wasn't Muffy in New England who made the work.
Why are we running away from who we are? For whose benefit? I'm black
and I'm proud and, as a woman, I'm all-powerful. I am the giver of life.
Put my ass on a pedestal. --- \emph{As told to T.R.}

\begin{center}\rule{0.5\linewidth}{\linethickness}\end{center}

Image

Calida Rawles's ``Radiating My Sovereignty'' (2019).Credit...Courtesy of
the artist and Various Small Fires, Los Angeles/Seoul

Image

Rawles's ``Reflecting My Grace'' (2019).Credit...Courtesy of the artist
and Various Small Fires, Los Angeles/Seoul

\hypertarget{i-didnt-think-that-making-work-would-become-about-teaching-my-culture}{%
\subsection{`I didn't think that making work would become about teaching
my
culture.'}\label{i-didnt-think-that-making-work-would-become-about-teaching-my-culture}}

\hypertarget{by-calida-rawles-43-a-los-angeles-based-visual-artist}{%
\subsubsection{\texorpdfstring{\textbf{By}
\textbf{\href{https://www.instagram.com/calidagarciarawles/?hl=en}{Calida
Rawles}, 43, a Los Angeles-based visual
artist}}{By Calida Rawles, 43, a Los Angeles-based visual artist}}\label{by-calida-rawles-43-a-los-angeles-based-visual-artist}}

I learned how to swim much later in life --- just seven years ago ---
and through the quiet laps and the breathing, it became very therapeutic
for me. Whatever I was dealing with before I got into the pool, I didn't
feel the weight of it after I got out. A few years ago, I started to
think about how I could explore that in my art. I learned about
water-memory theory: this idea that water retains the substance of
things that run through it. I thought about that in regards to the
Middle Passage and how many memories must be in that water.

My parents didn't learn how to swim, and neither did their parents.
That's a direct result of segregation. We didn't have access to pools
growing up in Wilmington, Del., and my father would tell us stories of
his own childhood spent on the Maryland Eastern Shore. Though his family
lived just 10 miles away from the beach, they weren't allowed to go
there except once a week. And even then they were only allowed on the
boardwalk. The beach itself was a symbol of rejection.

I was thinking about the residual effects of that, which manifest in the
numbers: We in the black community have the highest rate of drowning.
That fear still lives with us. So, ever since I was a little girl, I've
said, ``My kids are going to learn to swim.''

In my last series, ``A Dream for My Lilith'' (2020), I tried to work
through this layered experience by putting black people and black bodies
in water --- there's a lot of emotion in that. I photographed and then
painted girls swimming in and around the rippling waves and liquid blue,
surrounded by flickering stars that form when light hits the water in
the right way.

With this work, I wanted to discuss the intersectionality of the black
female experience, **** as well as the theory of triple consciousness,
which stipulates that black women in this country view themselves
through three lenses: the American experience, largely defined by white
men; the female experience, generally written by white women; and the
black experience, usually associated with black men. To make work, for
me, is to seek a kind of spiritual healing from all of that. While I
can't say gaze doesn't affect me, I try not to think too hard about what
people want from me.

It's funny, though, because I didn't think that making work would become
about teaching my culture as much as it has. There's a whole black
swimmer community that's reached out to tell me they're so happy I'm
depicting these images. And there are people who can't swim who love
them, too, and they like to get lost in the beauty of just seeing us in
water. Sometimes you want to see yourself in places you've never been
before. \emph{--- As told to N.B.}

\begin{center}\rule{0.5\linewidth}{\linethickness}\end{center}

Image

Meleko Mokgosi's ``Acts of Resistance I'' (2018).Credit...© Meleko
Mokgosi. Courtesy of the artist and Jack Shainman Gallery, New York

Image

A still from Christie Neptune's film ``Two Miles Deep in La La Land''
(2007-12), featured in We Buy Gold's online video group show
``FIVE.''Credit...Courtesy of the artist

\hypertarget{its-crucial-for-art-workers-to-help-develop-future-audiences}{%
\subsection{`It's crucial for art workers to help develop future
audiences.'}\label{its-crucial-for-art-workers-to-help-develop-future-audiences}}

\hypertarget{by-joeonna-bellorado-samuels-40-a-new-york-based-gallerist}{%
\subsubsection{\texorpdfstring{\textbf{By}
\textbf{\href{https://www.instagram.com/joeonna/?hl=en}{Joeonna
Bellorado-Samuels}, 40, a New York-based
gallerist}}{By Joeonna Bellorado-Samuels, 40, a New York-based gallerist}}\label{by-joeonna-bellorado-samuels-40-a-new-york-based-gallerist}}

I work in an art world that has historically been and continues to be a
very white space, in terms of collectors, curators, the entire
ecosystem. So I'm always concerned with how I present the artists I work
with to the world, particularly because I have positioned my work life
in such a way that centers black artists. To me, that means paying
attention to how we articulate the mission of the artists and their
work, and to how they and their works will live beyond the gallery. We
are challenging the historical narrative by amplifying the artists'
intention and/or providing a context that is generally not examined or
considered with depth.

I have been with Jack Shainman Gallery for 12 years now, which has given
me the opportunity to collaborate with so many artists who are
disrupting and growing the canon. Last fall, we opened three exhibitions
across our New York spaces with the 38-year-old Botswana-born artist
Meleko Mokgosi, titled ``Democratic Intuition.'' In his practice ---
which also includes figurative, cinematic painting --- he takes existing
texts and then annotates and injects them onto canvas, quite literally
presenting a history that is a departure from Western hegemonic
academia. Mokgosi's work is undeniably **** a challenge to the white
gaze.

But it's hard for me to only talk about Shainman or only talk about We
Buy Gold, a roving gallery I started in 2017, which happens to center
artists of color. They're both continuations of how I think, but the
latter had a lot to do with circumventing the white gaze, or perhaps
refusing it: I wanted to create exhibitions that had a different
audience, that resisted the traditional elite white space that Chelsea
represents. Many people don't feel like Chelsea is a space for them,
even though it's an area in New York where there are hundreds of
exhibitions a year --- all free and open to the public. It's crucial for
art workers to help develop future audiences, whether they're patrons or
people who show up and participate. We Buy Gold interrogates the
exclusionary ethos of Chelsea, and is a way to bring art closer to my
own position, to where I live.

Earlier this month, We Buy Gold opened ``FIVE,'' an online video group
show curated by the New York artist Nina Chanel Abney, and released a
publication, We Buy Gold's deconstructed manifesto, which not only
includes contributions from the artists who have shown in each iteration
of the gallery but is also another means of questioning how we present
our cultural production. Even as we resist the traditional catalog,
archives and publications are still important in terms of people looking
back and knowing what we did, where we were at and that we were here.
\emph{--- As told to T.R.}

\begin{center}\rule{0.5\linewidth}{\linethickness}\end{center}

Image

Rashid Johnson's ``Two Standing Broken Men'' (2019).Credit...Photo:
Martin Parsekian. Courtesy of David Kordansky Gallery, Los Angeles, and
Hauser \& Wirth

Image

Johnson's ``Untitled Escape Collage'' (2020).Credit...Photo: Martin
Parsekian. Courtesy of David Kordansky Gallery, Los Angeles, and Hauser
\& Wirth

\hypertarget{we-would-never-ask-picasso-why-he-painted-white-people}{%
\subsection{`We would never ask Picasso why he painted white
people.'}\label{we-would-never-ask-picasso-why-he-painted-white-people}}

\hypertarget{by-rashid-johnson-42-a-new-york-based-visual-artist}{%
\subsubsection{\texorpdfstring{\textbf{By}
\textbf{\href{https://www.instagram.com/rashidjohnson/}{Rashid Johnson},
42, a New York-based visual
artist}}{By Rashid Johnson, 42, a New York-based visual artist}}\label{by-rashid-johnson-42-a-new-york-based-visual-artist}}

What \emph{is} the white gaze? \emph{Which} white gaze? Most of my work
has challenged the idea that blackness is monolithic. The fact that I
and artists like me have so aggressively challenged that position calls
into question why we might suggest whiteness is something so simple.

I have people in my community who are white --- friends, family, people
who influence and participate in my work. If it's their gaze that we're
discussing, then it's quite an informed one. If it's a bigoted white
gaze, then it's different. But I don't imagine the latter having much
access to my work. All of that is to suggest that I don't believe there
is a white gaze that we can speak about without delving into the
complexity of whiteness.

We need to have these conversations: What whiteness are we talking
about? Is it the white liberal? The white New Yorker? Is it European
whiteness? Is there a privilege that is also qualified by a real
financial agency as opposed to poverty? This produces different kinds of
perspectives. Although we like to imagine that white privilege is
inherently linked to white wealth, it's not. That's clumsy at best. I'm
as guilty as anyone of referencing whiteness with a tremendous
implicitness.

I was quite lucky because of how I was raised in Chicago. My mother and
father took it upon themselves to introduce me to a black literary and
intellectual tradition at an early age. I never had to search. There was
never a suggestion that they didn't exist. There are other artists and
black thinkers who have had to more or less discover what they felt was
an underground world of black intellectualism, having gone to schools
that put more of an emphasis on white and Western traditions. When they
discover black thinkers, it's a revelation to them. For me, it was never
a revelation --- it was the way things were --- so I don't conjure black
literary figures in my work as an opposition to the white underlying
concepts and traditions that someone would probably think I'm reacting
against. I'm not.

We would never ask Picasso why he painted white people. We wouldn't
position him as an outsider, and yet we consistently find new ways to
position the work of black artists as inherently being in response to
the obstacles presented by a white world. I'm just speaking from how I
understand the world, how I see it. And at the center of my world is not
whiteness. --- \emph{As told to T.R.}

\begin{center}\rule{0.5\linewidth}{\linethickness}\end{center}

Image

Mary Lovelace O'Neal's ``Running With My Black Panthers and White Doves
aka Running With My Daemons'' (circa 1989-90).Credit...Courtesy of the
artist and Mnuchin Gallery

\hypertarget{i-didnt-try-to-make-work-for-black-people-or-brown-people-or-white-people-or-red-people-or-yellow-people-or-crazy-people}{%
\subsection{`I didn't try to make work for black people or brown people
or white people or red people or yellow people or crazy
people.'}\label{i-didnt-try-to-make-work-for-black-people-or-brown-people-or-white-people-or-red-people-or-yellow-people-or-crazy-people}}

\hypertarget{by-mary-lovelace-oneal-78-an-oakland-based-painter}{%
\subsubsection{\texorpdfstring{\textbf{By}
\textbf{\href{http://www.mnuchingallery.com/exhibitions/mary-lovelace-oand39neal?j=42038704\&sfid=003i000004Bs34GAAR\&sfmc_sub=21111890\&l=1368034_HTML\&u=459507965\&mid=10902205\&jb=1}{Mary
Lovelace O'Neal}, 78, an Oakland-based
painter}}{By Mary Lovelace O'Neal, 78, an Oakland-based painter}}\label{by-mary-lovelace-oneal-78-an-oakland-based-painter}}

I ran to the studio so that I didn't have to consider anybody or
anything, and I've always said that my work was my way of freeing myself
--- from students, from my own teachers, from concepts that I had
accepted. There's this painting I made in the early '80s called
``Meaningless Ritual, Senseless Superstition'' that has to do with
coming into your studio and maybe jumping around three times, lighting
your candles, emptying all of your cigarette butts and sweeping. There
are these things I had to do every time to get all of that ``outside
stuff'' out of my head.

Whenever I was teaching --- at U.C. Berkeley or the San Francisco Art
Institute or the California College of the Arts in Oakland --- I would
come home and maybe have dinner or take a shower, and then I would get
dressed. I'd put on earrings and makeup and my work clothes --- a blue
work shirt and corduroy trousers or a wool or cotton dress, my favorite
lab coat, my clogs --- and go to the studio. I didn't know it at the
time, but what I was doing through those little rituals was cleansing
myself so that I could get rid of all the awful work my students were
doing and all the horrible stuff I was seeing in the museums where I was
taking those kids, and in the galleries where I'd go for openings. In my
studio, I would try to come into myself. I didn't try to make work for
black people or brown people or white people or red people or yellow
people or crazy people. That wasn't it. I was there to deal with my
stuff, to deal with \emph{me}.

I knew early on in the days of the civil rights movement, when I was at
Howard University as a young student and artist, that I didn't have the
ability to do what lots of artists could do, like those Cuban artists
and those artists in Chile in the '60s who were making these incredible
heartbreaking prints that talked about the pain of oppression and taught
you how to resist it. I didn't think I could make work that was strong
enough to voice what needed to be said in that call-to-action kind of
way.

But I am so thrilled when I see people who do really know what they're
talking about, like the painter Joe Overstreet, who was one of my best
friends back then, or Kerry James Marshall now. I love Kerry's work, and
I love, even more than that, the way he talks about it: how he digested
what he learned, and what school meant to him, and how he makes things
legible for people. For me, though, I knew it wasn't going to move
anybody to do anything like that. All I could do was put myself on the
line. --- \emph{As told to N.B.}

Advertisement

\protect\hyperlink{after-bottom}{Continue reading the main story}

\hypertarget{site-index}{%
\subsection{Site Index}\label{site-index}}

\hypertarget{site-information-navigation}{%
\subsection{Site Information
Navigation}\label{site-information-navigation}}

\begin{itemize}
\tightlist
\item
  \href{https://help.nytimes3xbfgragh.onion/hc/en-us/articles/115014792127-Copyright-notice}{©~2020~The
  New York Times Company}
\end{itemize}

\begin{itemize}
\tightlist
\item
  \href{https://www.nytco.com/}{NYTCo}
\item
  \href{https://help.nytimes3xbfgragh.onion/hc/en-us/articles/115015385887-Contact-Us}{Contact
  Us}
\item
  \href{https://www.nytco.com/careers/}{Work with us}
\item
  \href{https://nytmediakit.com/}{Advertise}
\item
  \href{http://www.tbrandstudio.com/}{T Brand Studio}
\item
  \href{https://www.nytimes3xbfgragh.onion/privacy/cookie-policy\#how-do-i-manage-trackers}{Your
  Ad Choices}
\item
  \href{https://www.nytimes3xbfgragh.onion/privacy}{Privacy}
\item
  \href{https://help.nytimes3xbfgragh.onion/hc/en-us/articles/115014893428-Terms-of-service}{Terms
  of Service}
\item
  \href{https://help.nytimes3xbfgragh.onion/hc/en-us/articles/115014893968-Terms-of-sale}{Terms
  of Sale}
\item
  \href{https://spiderbites.nytimes3xbfgragh.onion}{Site Map}
\item
  \href{https://help.nytimes3xbfgragh.onion/hc/en-us}{Help}
\item
  \href{https://www.nytimes3xbfgragh.onion/subscription?campaignId=37WXW}{Subscriptions}
\end{itemize}
