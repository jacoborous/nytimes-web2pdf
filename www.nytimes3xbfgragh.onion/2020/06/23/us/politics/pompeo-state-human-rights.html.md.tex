Sections

SEARCH

\protect\hyperlink{site-content}{Skip to
content}\protect\hyperlink{site-index}{Skip to site index}

\href{https://www.nytimes3xbfgragh.onion/section/politics}{Politics}

\href{https://myaccount.nytimes3xbfgragh.onion/auth/login?response_type=cookie\&client_id=vi}{}

\href{https://www.nytimes3xbfgragh.onion/section/todayspaper}{Today's
Paper}

\href{/section/politics}{Politics}\textbar{}Pompeo's Human Rights Panel
Could Hurt L.G.B.T. and Women's Rights, Critics Say

\url{https://nyti.ms/2ASECJK}

\begin{itemize}
\item
\item
\item
\item
\item
\item
\end{itemize}

Advertisement

\protect\hyperlink{after-top}{Continue reading the main story}

Supported by

\protect\hyperlink{after-sponsor}{Continue reading the main story}

\hypertarget{pompeos-human-rights-panel-could-hurt-lgbt-and-womens-rights-critics-say}{%
\section{Pompeo's Human Rights Panel Could Hurt L.G.B.T. and Women's
Rights, Critics
Say}\label{pompeos-human-rights-panel-could-hurt-lgbt-and-womens-rights-critics-say}}

Human rights observers warn the document could be a tool to advance Mr.
Pompeo's religious beliefs and political aspirations, and not women's
and L.G.B.T.Q. rights.

\includegraphics{https://static01.graylady3jvrrxbe.onion/images/2020/06/22/us/politics/22dc-pompeo-rights1/merlin_169703184_62cf0900-2faf-4c8c-95c2-1815c55567f0-articleLarge.jpg?quality=75\&auto=webp\&disable=upscale}

\href{https://www.nytimes3xbfgragh.onion/by/pranshu-verma}{\includegraphics{https://static01.graylady3jvrrxbe.onion/images/2020/07/07/reader-center/author-pranshu-verma/author-pranshu-verma-thumbLarge.png}}

By \href{https://www.nytimes3xbfgragh.onion/by/pranshu-verma}{Pranshu
Verma}

\begin{itemize}
\item
  Published June 23, 2020Updated July 21, 2020
\item
  \begin{itemize}
  \item
  \item
  \item
  \item
  \item
  \item
  \end{itemize}
\end{itemize}

WASHINGTON --- Inside the State Department, the definition of human
rights is up for debate.

Secretary of State
\href{https://www.nytimes3xbfgragh.onion/2020/07/21/world/europe/mike-pompeo-boris-johnson-china.html}{Mike
Pompeo}, an
\href{https://www.nytimes3xbfgragh.onion/2019/03/30/us/politics/pompeo-christian-policy.html}{evangelical
Christian}, created a commission last July to
\href{https://www.nytimes3xbfgragh.onion/2019/07/08/us/politics/state-human-rights.html}{provide
a new vision for human rights policy} that would more closely align with
the ``nation's founding principles'' and uphold religious freedom as
America's most fundamental value.

Human rights scholars have criticized the panel, saying it is filled
with conservatives intent on promoting views against abortion and
marriage equality. Critics also warn the commission sidesteps the State
Department's internal bureau responsible for promoting human rights
abroad.

And former agency officials caution that elevating the importance of
religion could reverse the country's longstanding belief that ``all
rights are created equal'' --- and embolden countries that persecute
same-sex couples or deny women access to reproductive health services
for religious reasons.

``There are those who would have preferred I didn't do it, and are
concerned about the answers that our foundational documents will
provide,''
\href{https://www.state.gov/secretary-michael-r-pompeo-at-the-concerned-women-for-america-40th-anniversary-luncheon/}{Mr.
Pompeo said of the commission last fall} to a conservative women's group
at the Trump International Hotel in Washington. ``I know where those
rights came from. They came from our Lord.''

He added: ``Indeed, for years under the last administration, fighting
for religious freedom was just an afterthought. But President Trump, our
administration, recognizes it as our country's first freedom, and it's
found at the very top of the Bill of Rights, so we kind of got it
right.''

The commission's report is expected to be released in early July, and is
tightly held among Mr. Pompeo's top aides. Diplomats note the report
could be a tool to advance Mr. Pompeo's religious beliefs and political
aspirations, while proving detrimental to preserving the rights of women
and gay, lesbian, bisexual and transgender people abroad.

``This is about the only human right they seem to care about,'' David
Kramer, who was assistant secretary of state for the Bureau of
Democracy, Human Rights and Labor in the George W. Bush administration,
said of the commission's focus on religion. ``It seems to be a play for
political support domestically, that could rebound to our detriment in
foreign policy.''

The panel's recommendations come as America's commitment to human rights
faces skepticism from organizations like the United Nations. The
peacekeeping body issued a resolution on Friday condemning police
brutality and ``systemic racism'' against people of African descent.
Diplomats had to drop specific references to the United States to gain
passage.

In response to the resolution, Mr. Pompeo said on Saturday that bodies
like the United Nation's Human Rights Council should ``recognize the
strengths of American democracy and urge authoritarian regimes around
the world to model'' America's values. (The United States
\href{https://www.nytimes3xbfgragh.onion/2018/06/19/us/politics/trump-israel-palestinians-human-rights.html}{quit
the council two years ago} after accusing it of bias against Israel.)

Experts warn this type of criticism from Mr. Pompeo will hold less sway
if the secretary's Commission on Unalienable Rights produces a document
prioritizing religion above all else. Such a document could also play
into the hands of repressive governments like Iran and Saudi Arabia that
seek to narrowly define human rights.

The State Department declined to comment on the questions regarding the
commission.

Mr. Pompeo's advisory panel has met five times. The meetings were public
and have been minimally attended. Human rights advocates, former State
Department officials and academics say they have been alarmed at what
has taken place.

``The bottom line: The commission is poised to adversely shape U.S.
foreign policy,'' Jayne Huckerby and Sarah Knuckey, human rights
professors at Duke University and Columbia Law School, wrote in a
\href{https://www.justsecurity.org/69150/pompeos-rights-commission-is-worse-than-feared-part-i/}{recent
blog post} detailing the panel's work. In their analysis of the panel's
meetings, they noted that a ``a general skepticism'' toward
international human rights pervaded committee discussions.

Many commission members, they note, believe there are too many human
rights, including Mary Ann Glendon, the head of the commission, who has
said ``if everything is a right, then nothing is.''

If the commission's report to Mr. Pompeo reflects the panel's
discussions to date, and makes a case to prioritize one human right over
another, observers say it could upend diplomatic efforts to stop the
Chinese persecution of members of the Uighur minority and promote
women's rights in places like Iran and Saudi Arabia.

\includegraphics{https://static01.graylady3jvrrxbe.onion/images/2020/06/22/us/politics/22dc-pompeo-rights2/merlin_168380358_bcbf7ba6-26a8-4c3b-a06b-c7b4ad44e9d5-articleLarge.jpg?quality=75\&auto=webp\&disable=upscale}

``My hope is that this document doesn't come close to establishing
something that looks like a hierarchy of rights,'' said Rob Berschinski,
a deputy assistant secretary of state for the Bureau of Democracy, Human
Rights and Labor in the Obama administration. ``But if it does,
repressive governments are going to point to that fact and use it
against this, and future administrations, to basically say `we are no
different than you. You have your priorities, we have ours, now butt
out.'''

Committee members were handpicked by Mr. Pompeo's staff, and most of
them are conservatives with strong academic credentials.

In the months after its creation, Mr. Pompeo expressed confidence the
panel would create a document that enshrines religious freedom as a
central tenet of American human rights policy, which diplomats could
refer to for ``decades to come.''

The panel is grounded in the vision of Robert George, a Princeton
professor and leading proponent of
\href{https://www.britannica.com/topic/natural-law}{``natural law''}
theory, a term human rights scholars say is code for ``God-given
rights'' and is commonly deployed in fights to roll back rights for
women and L.G.B.T.Q. persons.

``The commission's charge is not to `discover' new principles,'' Mr.
George wrote in a document outlining the commission's vision, ``but
rather to point the way towards that more perfect fidelity to our
nation's founding principles of natural law and natural rights.''

Early language defining the commission in federal documents echoed Mr.
George's notion, saying the panel would provide ``fresh thinking'' on
human rights discussions, since conversations have ``departed from our
nation's founding principles of natural law and natural rights.''

This drew significant criticism from human rights advocates, and since
then, the mission has altered to say members will ``furnish advice to
the secretary for the promotion of individual liberty, human equality,
and democracy through U.S. foreign policy.''

The commission is led by Ms. Glendon, a Harvard professor and former
ambassador to the Vatican, who has garnered controversy for statements
like The Boston Globe's receiving the Pulitzer Prize for its
investigation into child abuse by priests ``would be like giving the
Nobel Peace Prize to Osama bin Laden.''

This ``is a group of individuals who want to redefine how this country
balances human rights interests and to tip the scales in favor of
religious freedom, '' said Mark Bromley, the chairman of the Council for
Global Equality, a coalition of 30 human rights groups advocating
lesbian, gay, bisexual and transgender rights in American foreign
policy.

Two Democratic representatives, Jamie Raskin of Maryland and Joaquin
Castro of Texas, warned the commission's report could ``undermine our
nation's ability to lead on critical issues of universal human rights,
including reproductive freedom and protections for millions of people
globally in the L.G.B.T.Q. community.''

Several human rights organizations have sued the State Department,
saying it is violating a federal law that requires advisory panels like
the Commission on Unalienable Rights to be ``fairly balanced'' and
transparent with meeting documents at the time of hearings.

The lawsuit is pending, and lawyers representing the State Department
said last week the committee would invite public comment on the report
before the commission's work concluded.

Human rights observers warned that any public comment might not change
what they predicted to be a preordained outcome to prioritize religious
freedom as America's most valued human right based on Mr. Pompeo's
beliefs and personal interest in the panel.

``Through sheer force of political will and personality,'' Mr. Bromley
said, ``he's been pushing it forward and has a very clear idea, if you
look at his writings and speakings, of where he wants it to end up.''

Advertisement

\protect\hyperlink{after-bottom}{Continue reading the main story}

\hypertarget{site-index}{%
\subsection{Site Index}\label{site-index}}

\hypertarget{site-information-navigation}{%
\subsection{Site Information
Navigation}\label{site-information-navigation}}

\begin{itemize}
\tightlist
\item
  \href{https://help.nytimes3xbfgragh.onion/hc/en-us/articles/115014792127-Copyright-notice}{©~2020~The
  New York Times Company}
\end{itemize}

\begin{itemize}
\tightlist
\item
  \href{https://www.nytco.com/}{NYTCo}
\item
  \href{https://help.nytimes3xbfgragh.onion/hc/en-us/articles/115015385887-Contact-Us}{Contact
  Us}
\item
  \href{https://www.nytco.com/careers/}{Work with us}
\item
  \href{https://nytmediakit.com/}{Advertise}
\item
  \href{http://www.tbrandstudio.com/}{T Brand Studio}
\item
  \href{https://www.nytimes3xbfgragh.onion/privacy/cookie-policy\#how-do-i-manage-trackers}{Your
  Ad Choices}
\item
  \href{https://www.nytimes3xbfgragh.onion/privacy}{Privacy}
\item
  \href{https://help.nytimes3xbfgragh.onion/hc/en-us/articles/115014893428-Terms-of-service}{Terms
  of Service}
\item
  \href{https://help.nytimes3xbfgragh.onion/hc/en-us/articles/115014893968-Terms-of-sale}{Terms
  of Sale}
\item
  \href{https://spiderbites.nytimes3xbfgragh.onion}{Site Map}
\item
  \href{https://help.nytimes3xbfgragh.onion/hc/en-us}{Help}
\item
  \href{https://www.nytimes3xbfgragh.onion/subscription?campaignId=37WXW}{Subscriptions}
\end{itemize}
