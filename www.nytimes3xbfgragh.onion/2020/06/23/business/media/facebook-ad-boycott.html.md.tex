Sections

SEARCH

\protect\hyperlink{site-content}{Skip to
content}\protect\hyperlink{site-index}{Skip to site index}

\href{https://www.nytimes3xbfgragh.onion/section/business/media}{Media}

\href{https://myaccount.nytimes3xbfgragh.onion/auth/login?response_type=cookie\&client_id=vi}{}

\href{https://www.nytimes3xbfgragh.onion/section/todayspaper}{Today's
Paper}

\href{/section/business/media}{Media}\textbar{}Ad Boycott of Facebook
Keeps Growing

\url{https://nyti.ms/2zWnxhh}

\begin{itemize}
\item
\item
\item
\item
\item
\end{itemize}

Advertisement

\protect\hyperlink{after-top}{Continue reading the main story}

Supported by

\protect\hyperlink{after-sponsor}{Continue reading the main story}

\hypertarget{ad-boycott-of-facebook-keeps-growing}{%
\section{Ad Boycott of Facebook Keeps
Growing}\label{ad-boycott-of-facebook-keeps-growing}}

Eddie Bauer, Magnolia Pictures, Ben \& Jerry's and others have suspended
campaigns over the platform's content moderation practices.

\includegraphics{https://static01.graylady3jvrrxbe.onion/images/2020/06/23/business/23facebook-ads/23facebook-ads-articleLarge.jpg?quality=75\&auto=webp\&disable=upscale}

\href{https://www.nytimes3xbfgragh.onion/by/tiffany-hsu}{\includegraphics{https://static01.graylady3jvrrxbe.onion/images/2018/12/06/multimedia/author-tiffany-hsu/author-tiffany-hsu-thumbLarge.png}}

By \href{https://www.nytimes3xbfgragh.onion/by/tiffany-hsu}{Tiffany Hsu}

\begin{itemize}
\item
  Published June 23, 2020Updated July 7, 2020
\item
  \begin{itemize}
  \item
  \item
  \item
  \item
  \item
  \end{itemize}
\end{itemize}

Facebook put on an upbeat presentation to advertisers on Tuesday, the
same day the clothing chain Eddie Bauer, the film distributor Magnolia
Pictures and the Ben \& Jerry's ice cream brand announced that they
would stop advertising on the platform through July.

Those companies joined Patagonia, the North Face, REI and others in a
growing
\href{https://www.nytimes3xbfgragh.onion/2020/07/07/technology/facebook-ad-boycott-civil-rights.html}{boycott
that has targeted Facebook's content moderation practices}.

In a short video, part of a weeklong showcase for digital companies
hoping to attract advertising dollars, Facebook displayed posts that
companies like Delta Air Lines and Calvin Klein ran during the
coronavirus pandemic. The prerecorded presentation did not specifically
address the boycott.

While several large companies have pulled away from Facebook, smaller
businesses that make up the bulk of its eight million advertisers have
been considering their options.

Jason Dille, who oversees media planning for 20 clients at the ad agency
Chemistry, said many of them had considered putting a halt to buying ads
on Facebook but that the pandemic had complicated their plans.

``Some of my clients are just starting to come back,'' Mr. Dille said.
``If they don't create sales and get business to turn around, they're
going to go under.''

He added: ``Facebook is a double-edged sword. You don't want to support
it, but you have to use it in order to reach a large audience.''

The backlash intensified late last month, as
\href{https://www.nytimes3xbfgragh.onion/2020/06/22/technology/antifa-local-disinformation.html}{a
flurry of misinformation} appeared on Facebook amid worldwide protests
against racism and police brutality. The company
\href{https://www.nytimes3xbfgragh.onion/2020/05/29/technology/twitter-facebook-zuckerberg-trump.html}{declined
to take action} against posts from President Trump --- the same ones
that Twitter flagged as misleading or glorifying violence.

In recent days, Facebook
\href{https://www.nytimes3xbfgragh.onion/2020/06/18/us/politics/facebook-trump-ads-antifa-red-triangle.html}{removed
ads} from Mr. Trump's re-election campaign that featured a red
triangular symbol used by the Nazis during World War II. The company
also announced that it would gradually allow users to opt out of
\href{https://www.nytimes3xbfgragh.onion/2020/06/16/technology/opt-out-political-ads-facebook.html}{seeing
political ads}. On Sunday, it acknowledged
\href{https://www.facebookcorewwwi.onion/business/news/where-facebook-stands-racial-equality-justice}{in
a blog post} that its enforcement of content rules ``isn't perfect.''

Facebook sidestepped sensitive issues during its presentation at the
so-called NewFronts, an annual event for digital media companies to
promote themselves as advertising venues. Before the Facebook section on
Tuesday, Snap pledged to shield advertisers from harmful content, and
Condé Nast said it had been forced to ``hold a mirror up to ourselves''
after
\href{https://www.nytimes3xbfgragh.onion/2020/06/13/business/media/conde-nast-racial.html}{an
internal uproar} over how the company has dealt with race. Immediately
after Facebook's presentation, the Ad Council, a nonprofit group,
presented \href{https://www.youtube.com/watch?v=Y8_598SQAts}{a video}
about the Black Lives Matter movement.

In explaining why it would stop advertising on Facebook,
\href{https://twitter.com/MagnoliaPics/status/1275420847151300615?s=20}{Magnolia
Pictures} said on Tuesday that it was ``seeking meaningful change at
Facebook and the end to their amplification of hate speech.''

Ben \& Jerry's
\href{https://www.benjerry.com/about-us/media-center/stop-hate-for-profit}{pushed
Facebook} on Tuesday ``to take stronger action to stop its platforms
from being used to divide our nation, suppress voters, foment and fan
the flames of racism and violence, and undermine our democracy.''

The freelancing platform
\href{https://twitter.com/Upwork/status/1274110997695873024?s=20}{Upwork}
and the password manager
\href{https://blog.dashlane.com/make-facebook-stop-hate/}{Dashlane} are
also participating in the boycott, which advocacy groups such as the
National Association for the Advancement of Colored People and the
Anti-Defamation League have promoted with the hashtag
\#StopHateForProfit.

The effort, which began taking shape
\href{https://www.nytimes3xbfgragh.onion/2020/06/09/business/media/facebook-advertisers-trump-zuckerberg.html}{this
month}, gained traction Friday and through the weekend as several
\href{https://twitter.com/Arcteryx/status/1275482126599639040?s=20}{outdoor-gear
retailers}, including
\href{https://twitter.com/REI/status/1274110350703554560?s=20}{REI} and
\href{https://twitter.com/patagonia/status/1274832569398292480}{Patagonia},
joined in.

\href{https://twitter.com/thenorthface/status/1273985578564870145?s=20}{The
North Face} has stopped posting content and buying ads on Facebook
through July, but will continue putting free posts on Instagram, which
Facebook owns, the company's global vice president of marketing, Steve
Lesnard, said in an interview. The North Face spends more on Facebook
than it does on any other platform besides Google, Mr. Lesnard said.

``The stakes are too high,'' he said. ``The platform needs to evolve.''

The efforts against Facebook have gotten support from ad agencies. In an
email to more than 50 clients last week, the digital ad agency 360i said
it supported the boycott,
\href{https://www.wsj.com/articles/ad-agency-encourages-clients-to-join-facebook-ad-boycott-11592517885?mod=tech_lead_pos2}{The
Wall Street Journal} reported. Some companies have quietly joined the
effort, and several ad agencies have developed guidelines for major
companies interested in participating, said three people with knowledge
of recent discussions, who requested anonymity because the talks are
confidential.

``It feels like we've come to an inflection point,'' said Stephan
Loerke, the chief executive of the World Federation of Advertisers, a
trade group. ``There's a growing awareness that this isn't a brand
safety issue anymore --- it's a societal safety issue.''

Facebook executives have tried to limit the damage. In an email sent to
some of its largest advertising clients last week, obtained by The New
York Times, the company said it had taken steps to mitigate the effects
of potentially harmful speech on the site.

``There are competing pressures every day when managing a platform,''
the memo said. ``Our focus is to act on what is most important: removing
hate speech and content that harms communities while using our platform
for efforts like providing authoritative voting information and
registering people to vote.''

Carolyn Everson, Facebook's vice president for global marketing
solutions, said in a statement that the company was in discussions with
advertisers and civil rights groups ``about how, together, we can be a
force for good.''

``We deeply respect any brand's decision, and remain focused on the
important work of removing hate speech and providing critical voting
information,'' she said in the statement.

Most companies that have turned away from Facebook are not shutting down
their Facebook accounts. They expect to return to buying ads on the
platform after July.

``It almost feels a little hypocritical to me,'' said Barry Lowenthal,
the chief executive of the Media Kitchen agency. ``How do you justify
going back?''

He suggested that companies make a gradual separation from Facebook as
they experiment with alternatives like Amazon, Snap or TikTok.

``They're not doing a good job, and advertisers want better
protections,'' Mr. Lowenthal said of Facebook.

Facebook generates nearly all its revenue from advertisements. The
research firm eMarketer expects that the platform's ad revenue will
increase nearly 5 percent this year.

Around the same time of Facebook's presentation on Tuesday,
representatives of some of the biggest companies that regularly
advertise on the platform gathered online for a previously scheduled
meeting to discuss its handling of misinformation and hate speech.

Mike Isaac contributed reporting.

Advertisement

\protect\hyperlink{after-bottom}{Continue reading the main story}

\hypertarget{site-index}{%
\subsection{Site Index}\label{site-index}}

\hypertarget{site-information-navigation}{%
\subsection{Site Information
Navigation}\label{site-information-navigation}}

\begin{itemize}
\tightlist
\item
  \href{https://help.nytimes3xbfgragh.onion/hc/en-us/articles/115014792127-Copyright-notice}{©~2020~The
  New York Times Company}
\end{itemize}

\begin{itemize}
\tightlist
\item
  \href{https://www.nytco.com/}{NYTCo}
\item
  \href{https://help.nytimes3xbfgragh.onion/hc/en-us/articles/115015385887-Contact-Us}{Contact
  Us}
\item
  \href{https://www.nytco.com/careers/}{Work with us}
\item
  \href{https://nytmediakit.com/}{Advertise}
\item
  \href{http://www.tbrandstudio.com/}{T Brand Studio}
\item
  \href{https://www.nytimes3xbfgragh.onion/privacy/cookie-policy\#how-do-i-manage-trackers}{Your
  Ad Choices}
\item
  \href{https://www.nytimes3xbfgragh.onion/privacy}{Privacy}
\item
  \href{https://help.nytimes3xbfgragh.onion/hc/en-us/articles/115014893428-Terms-of-service}{Terms
  of Service}
\item
  \href{https://help.nytimes3xbfgragh.onion/hc/en-us/articles/115014893968-Terms-of-sale}{Terms
  of Sale}
\item
  \href{https://spiderbites.nytimes3xbfgragh.onion}{Site Map}
\item
  \href{https://help.nytimes3xbfgragh.onion/hc/en-us}{Help}
\item
  \href{https://www.nytimes3xbfgragh.onion/subscription?campaignId=37WXW}{Subscriptions}
\end{itemize}
