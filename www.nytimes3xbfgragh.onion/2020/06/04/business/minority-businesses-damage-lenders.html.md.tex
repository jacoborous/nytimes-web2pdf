Sections

SEARCH

\protect\hyperlink{site-content}{Skip to
content}\protect\hyperlink{site-index}{Skip to site index}

\href{https://www.nytimes3xbfgragh.onion/section/business}{Business}

\href{https://myaccount.nytimes3xbfgragh.onion/auth/login?response_type=cookie\&client_id=vi}{}

\href{https://www.nytimes3xbfgragh.onion/section/todayspaper}{Today's
Paper}

\href{/section/business}{Business}\textbar{}For Some Minority-Owned
Businesses, Their Lenders Are Now Their Defenders

\url{https://nyti.ms/309Ixf5}

\begin{itemize}
\item
\item
\item
\item
\item
\end{itemize}

\hypertarget{race-and-america}{%
\subsubsection{\texorpdfstring{\href{https://www.nytimes3xbfgragh.onion/news-event/george-floyd-protests-minneapolis-new-york-los-angeles?name=styln-george-floyd\&region=TOP_BANNER\&block=storyline_menu_recirc\&action=click\&pgtype=Article\&impression_id=53f8e600-f28e-11ea-906f-39f9ccaeecdc\&variant=undefined}{Race
and America}}{Race and America}}\label{race-and-america}}

\begin{itemize}
\tightlist
\item
  \href{https://www.nytimes3xbfgragh.onion/2020/09/04/nyregion/rochester-police-daniel-prude.html?name=styln-george-floyd\&region=TOP_BANNER\&block=storyline_menu_recirc\&action=click\&pgtype=Article\&impression_id=53f8e601-f28e-11ea-906f-39f9ccaeecdc\&variant=undefined}{What
  Happened in Rochester, N.Y.}
\item
  \href{https://www.nytimes3xbfgragh.onion/2020/09/01/us/politics/trump-fact-check-protests.html?name=styln-george-floyd\&region=TOP_BANNER\&block=storyline_menu_recirc\&action=click\&pgtype=Article\&impression_id=53f8e602-f28e-11ea-906f-39f9ccaeecdc\&variant=undefined}{Trump
  Fact Check}
\item
  \href{https://www.nytimes3xbfgragh.onion/2020/08/30/us/portland-shooting-explained.html?name=styln-george-floyd\&region=TOP_BANNER\&block=storyline_menu_recirc\&action=click\&pgtype=Article\&impression_id=53f90d10-f28e-11ea-906f-39f9ccaeecdc\&variant=undefined}{Portland
  Shooting}
\item
  \href{https://www.nytimes3xbfgragh.onion/2020/08/30/us/breonna-taylor-police-killing.html?name=styln-george-floyd\&region=TOP_BANNER\&block=storyline_menu_recirc\&action=click\&pgtype=Article\&impression_id=53f90d11-f28e-11ea-906f-39f9ccaeecdc\&variant=undefined}{Breonna
  Taylor's Life and Death}
\end{itemize}

Advertisement

\protect\hyperlink{after-top}{Continue reading the main story}

Supported by

\protect\hyperlink{after-sponsor}{Continue reading the main story}

\hypertarget{for-some-minority-owned-businesses-their-lenders-are-now-their-defenders}{%
\section{For Some Minority-Owned Businesses, Their Lenders Are Now Their
Defenders}\label{for-some-minority-owned-businesses-their-lenders-are-now-their-defenders}}

Black- and Latino-owned businesses have suffered damage from vandals and
arsonists on the fringes of the protests over police brutality. Non-bank
community lenders are out to protect them.

\includegraphics{https://static01.graylady3jvrrxbe.onion/images/2020/06/05/business/04JPunrest-minoritylenders3-print/merlin_173172672_263e36eb-9e1c-457c-9600-10936f87a8a8-articleLarge.jpg?quality=75\&auto=webp\&disable=upscale}

\href{https://www.nytimes3xbfgragh.onion/by/emily-flitter}{\includegraphics{https://static01.graylady3jvrrxbe.onion/images/2019/06/19/reader-center/author-emily-flitter/author-emily-flitter-thumbLarge.png}}

By \href{https://www.nytimes3xbfgragh.onion/by/emily-flitter}{Emily
Flitter}

\begin{itemize}
\item
  June 4, 2020
\item
  \begin{itemize}
  \item
  \item
  \item
  \item
  \item
  \end{itemize}
\end{itemize}

On Tuesday evening in Ferguson, Mo., a dozen people formed a human chain
around Reds The One and Only BBQ, a restaurant. Vandals had smashed
other storefronts in the area amid protests over the killing of George
Floyd in police custody, but Reds was still standing.

By forming a protective line around the restaurant, the group was hoping
to discourage any further violence. For two hours, members of the chain
kept vigil. But they were neither hired guards, nor friends or relatives
of the restaurant's owner, Red Harris. They were employees of Mr.
Harris's lender, a community organization called Justine Petersen.

Galen Gondolfi, a senior loan counselor at Justine Petersen, said the
gesture was largely symbolic because his group was not set up to provide
physical protection. But nonetheless, he said, it was a way to show
clients its commitment ``literally and figuratively.''

\includegraphics{https://static01.graylady3jvrrxbe.onion/images/2020/06/05/business/04Punrest0minoritylenders1-print/merlin_173172774_ce288433-8670-42df-9ef0-5f6606b0643f-articleLarge.jpg?quality=75\&auto=webp\&disable=upscale}

Image

Mr. Harris, left, behind his counter. Outside, members of Justine
Petersen formed a chain to protect the restaurant as protesters made
their way through Ferguson.Credit...Vanessa Charlot for The New York
Times

Groups such as Justine Petersen, which mostly lend to minority-owned
businesses across the United States, are not regular banks. They are
called Community Development Financial Institutions, and they use a
combination of government funds and private donations to seed businesses
that banks won't deal with because they view their owners as too poor
and too disconnected from the financial system to qualify for standard
loans.

Many C.D.F.I.s, which first came into existence in the early 1970s,
evolved out of groups that were formed to help minorities recover from
attacks ---
\href{https://www.nytimes3xbfgragh.onion/2019/12/17/us/tulsa-graves-black-wall-street-massacre.html}{like
the 1921 massacre of black Americans in Tulsa, Okla.} --- that have
occurred regularly throughout America's history. More recently, during
the coronavirus pandemic, such groups have been the go-to lenders for
minority business owners who
\href{https://www.nytimes3xbfgragh.onion/2020/04/10/business/minority-business-coronavirus-loans.html}{could
not find a bank to help them} tap a federal government aid program.

C.D.F.I.s, which are often nonprofits, offer their borrowers far more
than just cash. They also walk them through the myriad paperwork
required to get their businesses up and running, offer them management
training and sometimes even provide spaces from which to launch.

But the looting and damage that have marred protests in the past week
have added a new set of tasks for many of these organizations, akin to
those of a security guard or emergency workers. In places like Ferguson,
Minneapolis and Wilmington, Del., where violent groups have destroyed
property by smashing windows and setting fires, representatives from
these lenders have been the first to make contact with devastated
business owners and help organize their defense.

``We're like the National Guard for small businesses,'' Mr. Gondolfi of
Justine Petersen said. ``I love the idea of us being dispatched.''

His organization has long had a presence in Ferguson. When vandals
destroyed businesses there during protests after the killing of an
unarmed local teenager, Michael Brown, by a police officer in 2014, the
group made small loans to around two dozen businesses to help them get
up and running again.

``Historically, events like these have a 10- to 20-year impact,'' said
Paul Calistro, the founder of Cornerstone West, a community development
organization in Wilmington. Mr. Calistro is working with other groups to
contact small businesses that were damaged last weekend and provide the
funds they need to rebuild.

But, he said, ``it's not just in money --- it's in time.''

C.D.F.I.s have helped revive poor neighborhoods, replacing empty
storefronts with active commercial spaces, increasing local economic
activity, building residents' wealth and reducing crime. Because they
make a wide variety of loans, including housing loans, they amass deep
knowledge of their neighborhoods and can tailor their activities to the
area's needs.

Over the past 35 years, they have made loans that helped start more than
400,000 small businesses around the country, according to the
Opportunity Finance Network, the trade group that represents them.
Around 58 percent of their borrowers are minorities, according to the
trade group's data. Their lending, which is a mix of small-business
loans and loans to housing and community facility projects, has totaled
more than \$74 billion over that time.

In Minneapolis, three organizations that focus on minority businesses
have helped transform the Midtown neighborhood from a depressed area
with few active businesses to a trendy spot where small businesses
flourish and city residents flock. The charitable aspect of the groups'
missions has helped to keep the ills of gentrification at bay.

But the current violence is threatening that progress.

Minneapolis is where the bulk of the destruction has occurred so far,
and local officials said it was the result of premeditated attacks on
black- and Hispanic-owned businesses.

Jeff Hayden, a state senator whose district includes the Midtown
neighborhood, said Minnesota officials found evidence that fire-starting
materials had been stashed in the neighborhood ahead of recent planned
protests and that businesses had been marked for attack.

``Based on what the governor is able to share with us and based on what
we see on the ground, there was definitely a coordinated attack,'' Mr.
Hayden said in an interview on Monday. ``It was clear that they were
going after ethnic businesses.''

Rolando Borja, whose firm, Integrated Staffing Solutions, helps connect
Minneapolis-area companies with workers who have often just arrived from
Puerto Rico or from other countries, was relieved when, at first, his
storefront in Midtown was spared. After seeing footage on the nightly
news of nearby windows breaking and buildings burning, Mr. Borja
ventured out on Wednesday last week to check on his space and found it
covered in graffiti but otherwise intact. At 5 the next morning,
however, an employee called him in tears: His building had been burned
to the ground.

Meda, the C.D.F.I. that lent Mr. Borja money when he started out 10
years ago and helped him get aid under the federal government's Paycheck
Protection Program when the pandemic forced him to shut down this year,
is where he will turn to again to rebuild.

``Meda, for me, has been a partner through the whole life of my
business, helping me out to have access to capital, access to legal
advice, business consulting, everything,'' Mr. Borja said.

Image

Alfredo Martel, the chief executive of Meda, is appealing to damaged
businesses in the Midtown neighborhood of Minneapolis to report
experiences and needs.Credit...Russell Frederick for The New York Times

Alfredo Martel, chief executive of Meda, said his staff had been walking
the streets assessing the damage. ``We are at this point like economic
first responders.'' But, he added, ``We're not cops, we're not firemen
or E.M.T.s.''

He is appealing to anyone in the neighborhood who has sustained damage
to call Meda and report experiences and needs, and says Meda is prepared
to help them start again from the ground up. After all, Mr. Martel
pointed out, his group was formed to help another Minneapolis
neighborhood rebuild itself after riots over police brutality destroyed
it 53 years ago, in the summer of 1967.

While Meda's staff is looking to help businesses rebuild, another
Midtown development organization is trying to prevent the ones still
standing from being attacked again. Its founder, Mihailo Temali, was so
worried about the premeditated aspect of the attacks that he asked The
New York Times not to name his organization.

Mr. Temali has been helping organize neighborhood watches, inspecting
newly painted graffiti for signs that it was meant to designate a
particular location for an attack and encouraging the business owners
his group has funded to defend their spaces by making sure they have
someone stationed there each night.

Advertisement

\protect\hyperlink{after-bottom}{Continue reading the main story}

\hypertarget{site-index}{%
\subsection{Site Index}\label{site-index}}

\hypertarget{site-information-navigation}{%
\subsection{Site Information
Navigation}\label{site-information-navigation}}

\begin{itemize}
\tightlist
\item
  \href{https://help.nytimes3xbfgragh.onion/hc/en-us/articles/115014792127-Copyright-notice}{©~2020~The
  New York Times Company}
\end{itemize}

\begin{itemize}
\tightlist
\item
  \href{https://www.nytco.com/}{NYTCo}
\item
  \href{https://help.nytimes3xbfgragh.onion/hc/en-us/articles/115015385887-Contact-Us}{Contact
  Us}
\item
  \href{https://www.nytco.com/careers/}{Work with us}
\item
  \href{https://nytmediakit.com/}{Advertise}
\item
  \href{http://www.tbrandstudio.com/}{T Brand Studio}
\item
  \href{https://www.nytimes3xbfgragh.onion/privacy/cookie-policy\#how-do-i-manage-trackers}{Your
  Ad Choices}
\item
  \href{https://www.nytimes3xbfgragh.onion/privacy}{Privacy}
\item
  \href{https://help.nytimes3xbfgragh.onion/hc/en-us/articles/115014893428-Terms-of-service}{Terms
  of Service}
\item
  \href{https://help.nytimes3xbfgragh.onion/hc/en-us/articles/115014893968-Terms-of-sale}{Terms
  of Sale}
\item
  \href{https://spiderbites.nytimes3xbfgragh.onion}{Site Map}
\item
  \href{https://help.nytimes3xbfgragh.onion/hc/en-us}{Help}
\item
  \href{https://www.nytimes3xbfgragh.onion/subscription?campaignId=37WXW}{Subscriptions}
\end{itemize}
