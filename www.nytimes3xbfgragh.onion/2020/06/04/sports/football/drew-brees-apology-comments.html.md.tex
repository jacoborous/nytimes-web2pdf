Sections

SEARCH

\protect\hyperlink{site-content}{Skip to
content}\protect\hyperlink{site-index}{Skip to site index}

\href{https://www.nytimes3xbfgragh.onion/section/sports/football}{Pro
Football}

\href{https://myaccount.nytimes3xbfgragh.onion/auth/login?response_type=cookie\&client_id=vi}{}

\href{https://www.nytimes3xbfgragh.onion/section/todayspaper}{Today's
Paper}

\href{/section/sports/football}{Pro Football}\textbar{}Drew Brees's
Unchanged Stance on Kneeling Is Suddenly Out of Step

\url{https://nyti.ms/2UbrCoZ}

\begin{itemize}
\item
\item
\item
\item
\item
\end{itemize}

\hypertarget{race-and-america}{%
\subsubsection{\texorpdfstring{\href{https://www.nytimes3xbfgragh.onion/news-event/george-floyd-protests-minneapolis-new-york-los-angeles?name=styln-george-floyd\&region=TOP_BANNER\&block=storyline_menu_recirc\&action=click\&pgtype=Article\&impression_id=b4da16d0-f1ce-11ea-bdd2-7fcbbccad7d5\&variant=undefined}{Race
and America}}{Race and America}}\label{race-and-america}}

\begin{itemize}
\tightlist
\item
  \href{https://www.nytimes3xbfgragh.onion/2020/09/04/nyregion/rochester-police-daniel-prude.html?name=styln-george-floyd\&region=TOP_BANNER\&block=storyline_menu_recirc\&action=click\&pgtype=Article\&impression_id=b4da3de0-f1ce-11ea-bdd2-7fcbbccad7d5\&variant=undefined}{How
  Police Handled Death of Daniel Prude}
\item
  \href{https://www.nytimes3xbfgragh.onion/2020/09/01/us/politics/trump-fact-check-protests.html?name=styln-george-floyd\&region=TOP_BANNER\&block=storyline_menu_recirc\&action=click\&pgtype=Article\&impression_id=b4da3de1-f1ce-11ea-bdd2-7fcbbccad7d5\&variant=undefined}{Trump
  Fact Check}
\item
  \href{https://www.nytimes3xbfgragh.onion/2020/08/30/us/portland-shooting-explained.html?name=styln-george-floyd\&region=TOP_BANNER\&block=storyline_menu_recirc\&action=click\&pgtype=Article\&impression_id=b4da3de2-f1ce-11ea-bdd2-7fcbbccad7d5\&variant=undefined}{Portland
  Shooting}
\item
  \href{https://www.nytimes3xbfgragh.onion/2020/08/30/us/breonna-taylor-police-killing.html?name=styln-george-floyd\&region=TOP_BANNER\&block=storyline_menu_recirc\&action=click\&pgtype=Article\&impression_id=b4da3de3-f1ce-11ea-bdd2-7fcbbccad7d5\&variant=undefined}{Breonna
  Taylor's Life and Death}
\end{itemize}

Advertisement

\protect\hyperlink{after-top}{Continue reading the main story}

Supported by

\protect\hyperlink{after-sponsor}{Continue reading the main story}

\hypertarget{drew-breess-unchanged-stance-on-kneeling-is-suddenly-out-of-step}{%
\section{Drew Brees's Unchanged Stance on Kneeling Is Suddenly Out of
Step}\label{drew-breess-unchanged-stance-on-kneeling-is-suddenly-out-of-step}}

The New Orleans Saints quarterback again called national anthem protests
``disrespectful.'' The response from his own locker room was more
ferocious than when he first said so in 2016.

\includegraphics{https://static01.graylady3jvrrxbe.onion/images/2020/06/05/sports/05nfl-brees-1/05nfl-brees-1-articleLarge.jpg?quality=75\&auto=webp\&disable=upscale}

\href{https://www.nytimes3xbfgragh.onion/by/ken-belson}{\includegraphics{https://static01.graylady3jvrrxbe.onion/images/2018/02/16/multimedia/author-ken-belson/author-ken-belson-thumbLarge.jpg}}

By \href{https://www.nytimes3xbfgragh.onion/by/ken-belson}{Ken Belson}

\begin{itemize}
\item
  June 4, 2020
\item
  \begin{itemize}
  \item
  \item
  \item
  \item
  \item
  \end{itemize}
\end{itemize}

The fury unleashed on Drew Brees came fast. The New Orleans Saints
quarterback
\href{https://twitter.com/YahooFinance/status/1268206174073126915?s=20}{said
in an interview Wednesday} that he would never agree with N.F.L. players
who knelt during the national anthem to protest police brutality, and he
was immediately condemned.

A host of players, including some of his teammates, responded with
statements of their own, calling Brees's comments hurtful and
criticizing him for ignorance of or callousness to the struggles of
African-Americans.

``Drew Brees, you don't understand how hurtful, how insensitive your
comments are,'' Malcolm Jenkins, Brees's teammate, said in a
\href{https://twitter.com/MalcolmJenkins/status/1268315207299981312}{video
posted to Twitter}. ``I'm disappointed, I'm hurt, because while the
world tells you, `You are not worthy,' that your life doesn't matter,
the last place you want to hear it from are the guys you go to war with
and that you consider to be your allies and your friends.

``Even though we are teammates, I can't let this slide.''

The backlash to Brees comes during a week in which a range of players
have been speaking out on the killing of George Floyd in police custody
in Minneapolis and racial injustice.

On Thursday night, players posted a montage of some of the league's
biggest stars --- including Chiefs quarterback Patrick Mahomes and
Cowboys running back Ezekiel Elliott --- calling on the N.F.L. to
condemn racism and admit that silencing players from peacefully
protesting is wrong, an apparent reference to the Colin Kaepernick
affair. (The N.F.L.
\href{https://www.instagram.com/p/CBB6ewBA-aO/?igshid=1w2oif5h7qxgi}{said
on Instagram on Thursday} that it stands ``with the black community
because Black Lives Matter.'')

But the backlash Brees received was notable not just because it targeted
one of the league's premier players, but because it broke an unwritten
rule that teammates vent their personal differences in private,
extolling the primacy of the team in public.

Brees had made similar remarks in 2016, when he
\href{https://www.espn.com/blog/new-orleans-saints/post/_/id/23063/drew-brees-wholeheartedly-disagrees-with-colin-kaepernicks-method-of-protest}{criticized
Kaepernick} for initially protesting social and racial injustice by
sitting during the national anthem.

Four years ago, a small group of players openly supported Kaepernick and
joined his protest, after much deliberation over his method and message.
Now, the ferocity of the reaction to Brees's comments is a sign of how
what was once a third rail issue in the league --- acknowledging
racially driven violence and systemic racism --- has become a necessary
conversation to be supported, if not outright embraced, not just by
African-American players, but also by white players, coaches and owners.

Unlike in 2016, when the league struggled to address the wave of police
shootings headlined that year by
\href{https://www.nytimes3xbfgragh.onion/2017/06/20/us/police-shooting-castile-trial-video.html}{that
of Philando Castile}, the video of George Floyd's killing while in
police custody has occurred at a time when the N.F.L. is more proactive
about addressing the issue.

On Saturday, the league and many team owners, including those who
previously said little or opposed Kaepernick's protest, issued
statements. For their part, players, like much of the country, have been
outspoken in condemning the status quo, and have done so without public
retribution --- a rarity in a league where the norm is to avoid any
commentary that might draw uncomfortable headlines.

``The existence of the protests worldwide and in all 50 states has
provided cover for the players to say things, and they have taken
advantage of that cover,'' said Jay Coakley, a sociologist and author of
the textbook, ``Sports in Society: Issues and Controversies.''

The act of kneeling in protest during the anthem is still subject to
laissez-faire oversight, as it is still not technically allowed by
league policy. In May 2018,
\href{https://www.nytimes3xbfgragh.onion/2018/05/23/sports/nfl-anthem-kneeling.html}{the
N.F.L. owners said that players could no longer kneel} during the
national anthem without being subject to punishment or their teams
facing possible financial penalties. Players could, however, stay in the
locker room during the pregame ceremony, when the anthem is played.

But after the N.F.L.
\href{https://www.nytimes3xbfgragh.onion/2018/07/10/sports/nfl-anthem.html}{Players
Association filed a grievance} challenging the decision, the league
never enforced the policy. Last year, several players, including Eric
Reid and
\href{https://www.nytimes3xbfgragh.onion/2018/09/07/sports/kenny-stills-national-anthem-protest.html}{Kenny
Stills}, continued to kneel during the playing of the anthem and were
not penalized by the league.

On Thursday, Brees
\href{https://www.instagram.com/p/CBA1P3gHpT_/?igshid=1qstwzxn87p2n}{walked
back his position in a post on Instagram}, saying his earlier comments
were ``insensitive and completely missed the mark.'' Brees also asked
for forgiveness and said that he took full responsibility for his words.

``I recognize that I should do less talking and more listening \ldots{}
and when the black community is talking about their pain, we all need to
listen,'' he wrote.

A few players praised Brees for his apology, including Demario Davis, a
Saints linebacker. Brees's apology ``is a form of true leadership and I
would say it because that's taking ownership,''
\href{https://twitter.com/AllisonLHedges/status/1268526582496444418}{Davis
said on CNN}. ``It's not easy to come out and admit when you're wrong.''

Brees's offensive teammate, Alvin Kamara, a Saints running back, also
seemed to welcome the apology. ``We talked and I explained to him where
he dropped the ball and he understood," Kamara
\href{https://twitter.com/A_kamara6/status/1268623227103911943}{wrote on
Twitter}. ``But now it's time for us to be part of the solution, not the
problem.''

Many others may not be so quick to forgive or forget.

Trayvon Mullen Jr., a cornerback for the Las Vegas Raiders,
\href{https://twitter.com/MullenIsland1}{wrote on Twitter}, ``September
21,'' the date his team is scheduled to play Brees's Saints.

Some players have pointed back to 2016 as a means of calling out players
who have been sanctimonious in their criticism of Brees. After Aaron
Rodgers, the Green Bay Packers quarterback, seemed to take aim at
Brees's \href{https://www.instagram.com/p/CA_Mq6ip8J6/}{in an Instagram
post} on Wednesday, Martellus Bennett, Rodgers's former teammate, said
that many white players were silent a few years ago when Kaepernick
opened the discussion on police brutality. Bennett was one of the
players to take a knee, protesting during the 2017 season he spent in
Green Bay. Rodgers linked arms with his teammates, but did not kneel.

``I don't want to see y'all paint these dudes as white saviors that were
always speaking up,''
\href{https://twitter.com/MartysaurusRex/status/1268399925982670848}{he
said on Twitter}. ``It's just not true. I was there.''

Still, the debate in all corners of the N.F.L. shows a change in culture
over the past four years. It is unclear how many players and coaches
will take concrete steps to address social injustice, but they appear
more willing to acknowledge it or express their support of change in
public
statements\href{https://www.espn.com/nfl/story/_/id/29263075/broncos-coach-vic-fangio-apologizes-comments-see-racism-all-nfl}{.}
That could change once they return to training camp and have to face
television cameras and potentially fans, some of whom are bound to
disagree with them.

But for now, the dialogue appears to be subsuming the league. The
Detroit
\href{https://www.si.com/nfl/2020/06/03/nfl-mailbag-matt-patricia-detroit-lions-george-floyd}{Lions
on Monday decided not to talk about football}, instead holding a
team-wide discussion between players and coaches about the killing of
Floyd in Minneapolis.

On Thursday, the
\href{https://twitter.com/packers/status/1268539283591356416}{Packers
released a two-minute video} that begins with the title, ``Enough is
Enough.'' It is followed by players calling for steps to address social
injustice against African-Americans and players saying ``It is time for
change.''

Troy Vincent, the N.F.L.'s executive vice president of football
operations, wrote an op-ed for
\href{https://theathletic.com/1850491/2020/06/03/op-ed-enough-troy-vincent-sr-on-harnessing-this-moment-to-change-the-future/?source=rss}{The
Athletic} that he reposted in a series of
\href{https://twitter.com/TroyVincentSr/status/1268269339364347905}{tweets
about speaking to his son} Taron. Both Vincent and his son are
African-American, and the writing details a conversation about how Taron
should respond if he were to be stopped by police as he drove back to
the Ohio State campus.

But even in his lofty position, Vincent said that he had been challenged
by his daughter Desiré to speak publicly.

``As an institution, the N.F.L. has the ability to bring people to the
table,'' he said. ``We can be bridge-builders and facilitate discussion.
We can drive awareness around issues. It doesn't matter what the issue
is, our platform allows us to do that.''

Advertisement

\protect\hyperlink{after-bottom}{Continue reading the main story}

\hypertarget{site-index}{%
\subsection{Site Index}\label{site-index}}

\hypertarget{site-information-navigation}{%
\subsection{Site Information
Navigation}\label{site-information-navigation}}

\begin{itemize}
\tightlist
\item
  \href{https://help.nytimes3xbfgragh.onion/hc/en-us/articles/115014792127-Copyright-notice}{©~2020~The
  New York Times Company}
\end{itemize}

\begin{itemize}
\tightlist
\item
  \href{https://www.nytco.com/}{NYTCo}
\item
  \href{https://help.nytimes3xbfgragh.onion/hc/en-us/articles/115015385887-Contact-Us}{Contact
  Us}
\item
  \href{https://www.nytco.com/careers/}{Work with us}
\item
  \href{https://nytmediakit.com/}{Advertise}
\item
  \href{http://www.tbrandstudio.com/}{T Brand Studio}
\item
  \href{https://www.nytimes3xbfgragh.onion/privacy/cookie-policy\#how-do-i-manage-trackers}{Your
  Ad Choices}
\item
  \href{https://www.nytimes3xbfgragh.onion/privacy}{Privacy}
\item
  \href{https://help.nytimes3xbfgragh.onion/hc/en-us/articles/115014893428-Terms-of-service}{Terms
  of Service}
\item
  \href{https://help.nytimes3xbfgragh.onion/hc/en-us/articles/115014893968-Terms-of-sale}{Terms
  of Sale}
\item
  \href{https://spiderbites.nytimes3xbfgragh.onion}{Site Map}
\item
  \href{https://help.nytimes3xbfgragh.onion/hc/en-us}{Help}
\item
  \href{https://www.nytimes3xbfgragh.onion/subscription?campaignId=37WXW}{Subscriptions}
\end{itemize}
