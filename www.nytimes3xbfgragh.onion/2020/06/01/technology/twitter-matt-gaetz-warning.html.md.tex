Sections

SEARCH

\protect\hyperlink{site-content}{Skip to
content}\protect\hyperlink{site-index}{Skip to site index}

\href{https://www.nytimes3xbfgragh.onion/section/technology}{Technology}

\href{https://myaccount.nytimes3xbfgragh.onion/auth/login?response_type=cookie\&client_id=vi}{}

\href{https://www.nytimes3xbfgragh.onion/section/todayspaper}{Today's
Paper}

\href{/section/technology}{Technology}\textbar{}Twitter Places Warning
on Congressman's Tweet for Glorifying Violence

\url{https://nyti.ms/2AxHjQ1}

\begin{itemize}
\item
\item
\item
\item
\item
\end{itemize}

\hypertarget{race-and-america}{%
\subsubsection{\texorpdfstring{\href{https://www.nytimes3xbfgragh.onion/news-event/george-floyd-protests-minneapolis-new-york-los-angeles?name=styln-george-floyd\&region=TOP_BANNER\&block=storyline_menu_recirc\&action=click\&pgtype=Article\&impression_id=cc3ccf30-f2ae-11ea-8292-9134a9f2d48e\&variant=undefined}{Race
and America}}{Race and America}}\label{race-and-america}}

\begin{itemize}
\tightlist
\item
  \href{https://www.nytimes3xbfgragh.onion/2020/09/04/nyregion/rochester-police-daniel-prude.html?name=styln-george-floyd\&region=TOP_BANNER\&block=storyline_menu_recirc\&action=click\&pgtype=Article\&impression_id=cc3ccf31-f2ae-11ea-8292-9134a9f2d48e\&variant=undefined}{What
  Happened in Rochester, N.Y.}
\item
  \href{https://www.nytimes3xbfgragh.onion/2020/09/01/us/politics/trump-fact-check-protests.html?name=styln-george-floyd\&region=TOP_BANNER\&block=storyline_menu_recirc\&action=click\&pgtype=Article\&impression_id=cc3ccf32-f2ae-11ea-8292-9134a9f2d48e\&variant=undefined}{Trump
  Fact Check}
\item
  \href{https://www.nytimes3xbfgragh.onion/2020/08/30/us/portland-shooting-explained.html?name=styln-george-floyd\&region=TOP_BANNER\&block=storyline_menu_recirc\&action=click\&pgtype=Article\&impression_id=cc3cf640-f2ae-11ea-8292-9134a9f2d48e\&variant=undefined}{Portland
  Shooting}
\item
  \href{https://www.nytimes3xbfgragh.onion/2020/08/30/us/breonna-taylor-police-killing.html?name=styln-george-floyd\&region=TOP_BANNER\&block=storyline_menu_recirc\&action=click\&pgtype=Article\&impression_id=cc3cf641-f2ae-11ea-8292-9134a9f2d48e\&variant=undefined}{Breonna
  Taylor's Life and Death}
\end{itemize}

Advertisement

\protect\hyperlink{after-top}{Continue reading the main story}

Supported by

\protect\hyperlink{after-sponsor}{Continue reading the main story}

\hypertarget{twitter-places-warning-on-congressmans-tweet-for-glorifying-violence}{%
\section{Twitter Places Warning on Congressman's Tweet for Glorifying
Violence}\label{twitter-places-warning-on-congressmans-tweet-for-glorifying-violence}}

The post, from Representative Matt Gaetz, a Florida Republican, had
likened protesters to terrorists and called for them to be hunted down.

\includegraphics{https://static01.graylady3jvrrxbe.onion/images/2020/06/01/business/01twitter/merlin_171795111_8d201492-f42f-4907-ac4c-0e0ebaee3351-articleLarge.jpg?quality=75\&auto=webp\&disable=upscale}

By \href{https://www.nytimes3xbfgragh.onion/by/kate-conger}{Kate Conger}

\begin{itemize}
\item
  June 1, 2020
\item
  \begin{itemize}
  \item
  \item
  \item
  \item
  \item
  \end{itemize}
\end{itemize}

OAKLAND, Calif. --- Days after
\href{https://www.nytimes3xbfgragh.onion/2020/05/29/technology/trump-twitter-minneapolis-george-floyd.html}{restricting
one of President Trump's posts} from view for glorifying violence,
Twitter went at it again.

On Monday, the social media service used the same label to hide a
message by Representative Matt Gaetz, Republican of Florida --- which
likened those who were protesting police violence to terrorists and
called for them to be hunted down. The move also meant that the tweet
could not be retweeted or liked, to prevent it from being amplified.

``Now that we clearly see Antifa as terrorists, can we hunt them down
like we do those in the Middle East?'' Mr. Gaetz had tweeted on Monday,
referring to the
\href{https://www.nytimes3xbfgragh.onion/article/what-antifa-trump.html}{far-left
anti-fascist activist movement}. Shortly after his tweet was hidden, he
\href{https://twitter.com/mattgaetz/status/1267605537790853133}{reposted
a message from the president} that called for a law that gives
technology companies some legal immunities to be revoked. ``Their
warning is my badge of honor,''
\href{https://twitter.com/mattgaetz/status/1267615035767107586}{he wrote
later on Monday}.

Twitter last week engaged in a face-off with the president after adding
fact-check labels to two of his tweets and then restricting a post in
which Mr. Trump said that looting during the protests would lead to
shooting. While the San Francisco company was applauded by some for
taking more responsibility for the kinds of posts that appear on its
platform, others said it was biased against conservatives like Mr.
Trump.

Still, Twitter acted again, hiding Mr. Gaetz's post behind a warning
label --- though it stopped short of taking down his message altogether.

``The Tweet is in violation of our glorification of violence policy,'' a
Twitter spokesman said.

\includegraphics{https://static01.graylady3jvrrxbe.onion/images/2020/06/01/business/01twitter2/01twitter2-articleLarge.jpg?quality=75\&auto=webp\&disable=upscale}

Twitter last year announced a labeling system that marks tweets from
public figures that violate its policies while allowing the messages to
remain because they are the subject of significant public interest.

But it did not use the system to flag any messages from U.S. politicians
until Friday, when Mr. Trump weighed in on the clashes between the
police and protesters in Minneapolis over the killing of George Floyd,
an African-American man who died in police custody. Mr. Trump tweeted,
``\href{https://www.nytimes3xbfgragh.onion/2020/05/29/us/looting-starts-shooting-starts.html}{when
the looting starts, the shooting starts.}''

After Twitter added the label to the message that it had violated its
policy about glorifying violence, the official Twitter account for the
White House republished Mr. Trump's tweet. Twitter hid that tweet as
well. The phrase was used in the 1960s by a Miami police chief widely
condemned by civil rights groups.

Earlier last week, the company had invoked a separate policy against
election interference to place a fact-check label on two tweets from Mr.
Trump in which he falsely asserted that mail-in ballots for the November
election were being illegally distributed and would lead to widespread
election fraud.

In response to Twitter's actions, Mr. Trump last Thursday
\href{https://www.nytimes3xbfgragh.onion/2020/05/28/us/politics/trump-order-social-media.html}{signed
an executive order} that was meant to chip away at liability protections
that social media companies have for the content that is posted on their
sites. The executive order specifically targets a statute known as
\href{https://www.nytimes3xbfgragh.onion/2020/05/28/business/section-230-internet-speech.html}{Section
230 of the Communications Decency Act}.

Technology companies have argued that the law is essential to their
operations. But some lawmakers have said that the companies enjoy
unchecked power and have proposed modifications to the law. Mr. Trump's
order is likely to face significant legal challenges, experts have said.

Twitter has drawn criticism for acting inconsistently with the labels.
On Monday, Senator Tom Cotton, a Republican from Arkansas, also posted
on Twitter calling for a military crackdown on protests, adding, ``No
quarter for insurrectionists, anarchists, rioters, and looters.''

Some people argued that the tweet also glorified violence and called on
Twitter to remove it, but the company said Mr. Cotton's post did not
violate its rules.

``It's never possible to be completely consistent with any policy about
speech. You have to draw lines somewhere and the lines are always going
to be a little bit arbitrary,'' said James Grimmelmann, a law professor
at Cornell University. ``There is no apolitical position.''

Cecilia Kang contributed reporting from Washington.

Advertisement

\protect\hyperlink{after-bottom}{Continue reading the main story}

\hypertarget{site-index}{%
\subsection{Site Index}\label{site-index}}

\hypertarget{site-information-navigation}{%
\subsection{Site Information
Navigation}\label{site-information-navigation}}

\begin{itemize}
\tightlist
\item
  \href{https://help.nytimes3xbfgragh.onion/hc/en-us/articles/115014792127-Copyright-notice}{©~2020~The
  New York Times Company}
\end{itemize}

\begin{itemize}
\tightlist
\item
  \href{https://www.nytco.com/}{NYTCo}
\item
  \href{https://help.nytimes3xbfgragh.onion/hc/en-us/articles/115015385887-Contact-Us}{Contact
  Us}
\item
  \href{https://www.nytco.com/careers/}{Work with us}
\item
  \href{https://nytmediakit.com/}{Advertise}
\item
  \href{http://www.tbrandstudio.com/}{T Brand Studio}
\item
  \href{https://www.nytimes3xbfgragh.onion/privacy/cookie-policy\#how-do-i-manage-trackers}{Your
  Ad Choices}
\item
  \href{https://www.nytimes3xbfgragh.onion/privacy}{Privacy}
\item
  \href{https://help.nytimes3xbfgragh.onion/hc/en-us/articles/115014893428-Terms-of-service}{Terms
  of Service}
\item
  \href{https://help.nytimes3xbfgragh.onion/hc/en-us/articles/115014893968-Terms-of-sale}{Terms
  of Sale}
\item
  \href{https://spiderbites.nytimes3xbfgragh.onion}{Site Map}
\item
  \href{https://help.nytimes3xbfgragh.onion/hc/en-us}{Help}
\item
  \href{https://www.nytimes3xbfgragh.onion/subscription?campaignId=37WXW}{Subscriptions}
\end{itemize}
