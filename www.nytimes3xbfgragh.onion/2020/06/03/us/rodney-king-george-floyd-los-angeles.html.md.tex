Sections

SEARCH

\protect\hyperlink{site-content}{Skip to
content}\protect\hyperlink{site-index}{Skip to site index}

\href{https://www.nytimes3xbfgragh.onion/section/us}{U.S.}

\href{https://myaccount.nytimes3xbfgragh.onion/auth/login?response_type=cookie\&client_id=vi}{}

\href{https://www.nytimes3xbfgragh.onion/section/todayspaper}{Today's
Paper}

\href{/section/us}{U.S.}\textbar{}In Los Angeles, the Ghosts of Rodney
King and Watts Rise Again

\url{https://nyti.ms/2XWu0Rp}

\begin{itemize}
\item
\item
\item
\item
\item
\item
\end{itemize}

\hypertarget{race-and-america}{%
\subsubsection{\texorpdfstring{\href{https://www.nytimes3xbfgragh.onion/news-event/george-floyd-protests-minneapolis-new-york-los-angeles?name=styln-george-floyd\&region=TOP_BANNER\&block=storyline_menu_recirc\&action=click\&pgtype=Article\&impression_id=f8a7dc80-f2be-11ea-8fbf-5db5f4b653c8\&variant=undefined}{Race
and America}}{Race and America}}\label{race-and-america}}

\begin{itemize}
\tightlist
\item
  \href{https://www.nytimes3xbfgragh.onion/2020/09/04/nyregion/rochester-police-daniel-prude.html?name=styln-george-floyd\&region=TOP_BANNER\&block=storyline_menu_recirc\&action=click\&pgtype=Article\&impression_id=f8a7dc81-f2be-11ea-8fbf-5db5f4b653c8\&variant=undefined}{What
  Happened in Rochester, N.Y.}
\item
  \href{https://www.nytimes3xbfgragh.onion/2020/09/01/us/politics/trump-fact-check-protests.html?name=styln-george-floyd\&region=TOP_BANNER\&block=storyline_menu_recirc\&action=click\&pgtype=Article\&impression_id=f8a7dc82-f2be-11ea-8fbf-5db5f4b653c8\&variant=undefined}{Trump
  Fact Check}
\item
  \href{https://www.nytimes3xbfgragh.onion/2020/08/30/us/portland-shooting-explained.html?name=styln-george-floyd\&region=TOP_BANNER\&block=storyline_menu_recirc\&action=click\&pgtype=Article\&impression_id=f8a80390-f2be-11ea-8fbf-5db5f4b653c8\&variant=undefined}{Portland
  Shooting}
\item
  \href{https://www.nytimes3xbfgragh.onion/2020/08/30/us/breonna-taylor-police-killing.html?name=styln-george-floyd\&region=TOP_BANNER\&block=storyline_menu_recirc\&action=click\&pgtype=Article\&impression_id=f8a80391-f2be-11ea-8fbf-5db5f4b653c8\&variant=undefined}{Breonna
  Taylor's Life and Death}
\end{itemize}

Advertisement

\protect\hyperlink{after-top}{Continue reading the main story}

Supported by

\protect\hyperlink{after-sponsor}{Continue reading the main story}

\hypertarget{in-los-angeles-the-ghosts-of-rodney-king-and-watts-rise-again}{%
\section{In Los Angeles, the Ghosts of Rodney King and Watts Rise
Again}\label{in-los-angeles-the-ghosts-of-rodney-king-and-watts-rise-again}}

Los Angeles has been one of America's reference points for racial
unrest. This time protesters are bringing their anger to the people they
say need to hear it most: the white and wealthy.

\includegraphics{https://static01.graylady3jvrrxbe.onion/images/2020/05/30/us/01UNREST-LARIOTS-bevfire/merlin_173006694_b1a7d9bc-a163-42aa-86be-6ec05725fa6b-articleLarge.jpg?quality=75\&auto=webp\&disable=upscale}

By \href{https://www.nytimes3xbfgragh.onion/by/tim-arango}{Tim Arango}

\begin{itemize}
\item
  June 3, 2020
\item
  \begin{itemize}
  \item
  \item
  \item
  \item
  \item
  \item
  \end{itemize}
\end{itemize}

LOS ANGELES --- Patrisse Cullors was 8 in 1992, when Los Angeles
\href{https://www.nytimes3xbfgragh.onion/1992/05/05/us/riots-los-angeles-overview-rioting-mounted-gates-remained-political-event.html}{erupted
in riots} after four police officers were
\href{https://archive.nytimes3xbfgragh.onion/www.nytimes3xbfgragh.onion/books/98/02/08/home/rodney-verdict.html}{acquitted
of assault} for the beating of Rodney King, which occurred outside a San
Fernando Valley apartment building not far from where Ms. Cullors grew
up.

``I was scared as hell,'' she recalled. ``As children, when we would see
the police, our parents would tell us, `Behave, be quiet, don't say
anything.' There was such fear of law enforcement in this city.''

With America seized by racial unrest, as
\href{https://www.nytimes3xbfgragh.onion/2020/06/02/us/protester-profiles-floyd-minneapolis.html}{protests
convulse cities from coast to coast} after the death of
\href{https://www.nytimes3xbfgragh.onion/article/george-floyd-autopsy-michael-baden.html}{George
Floyd}, Los Angeles is on fire again. As peaceful protests in the city
turned violent over the past few days, with images of looting and
burning buildings captured by news helicopters shown late into the
night, Ms. Cullors, like many Angelenos, was pulled back to the trauma
of 1992.

The parallels are easy to see: looting and destruction, fueled by anger
over police abuses; shopkeepers, with long guns, protecting their
businesses. The differences, though, between 1992 and now, are stark.
This time, the faces of the protesters are more diverse --- black,
white, Latino, Asian; there has been little if any racially motivated
violence among Angelenos; and the geography of the chaos is very
different, with protesters bringing their message to Los Angeles'
largely white and rich Westside.

``South Central has been completely quiet and peaceful,'' said Ms.
Cullors, now a prominent activist and co-founder of Black Lives Matter
who organized a protest on Saturday in the Fairfax District, west of
downtown. ``That's an important distinction, that these current
situations are not happening in black communities.''

\emph{{[}Sign up}
\href{https://www.nytimes3xbfgragh.onion/newsletters/california-today}{\emph{for
California Today}}\emph{, our daily newsletter from the Golden
State.{]}}

Los Angeles, in many ways, is America's reference point for urban racial
unrest, including the
\href{https://www.nytimes3xbfgragh.onion/2015/08/11/us/50-years-after-watts-riots-a-recovery-is-in-progress.html}{Watts
riots in 1965} and the uprising in 1992. The Rodney King beating in
1991, captured on film, was one of the first viral videos of a black man
being abused by the police, before cellphones even existed. In those
uprisings, dozens of people were killed --- 34 in 1965, and more than 40
in 1992.

Some of the most searing images from 1992 were of racially motivated
violence on the streets --- the beating of Reginald Denny, a white truck
driver; gun battles between Korean shop owners and black looters. But
the mayhem largely stayed in the historically black community of South
Los Angeles and in Koreatown.

\includegraphics{https://static01.graylady3jvrrxbe.onion/images/2020/06/02/us/02UNREST-LARIOTS-socentral/merlin_121329809_94e2fd72-2835-46ae-8618-42bd19032bae-articleLarge.jpg?quality=75\&auto=webp\&disable=upscale}

Now, organizers here say, they have very deliberately brought their
anger to those they believe need to hear it the most: the white and the
wealthy.

In 2013, when Black Lives Matter held its first demonstration in Los
Angeles, it was in Beverly Hills.

``We launched it there because we said, `Hey, our community knows about
this issue,''' Ms. Cullors said. ```Let's go into the heart of what is
symbolically white in Los Angeles, which is Beverly Hills. These people
need to hear our pain and our grief.''' Ms. Cullors added, ``We wanted
to bring this to communities who often aren't dealing with police
violence.''

The protest on Saturday in Fairfax stayed peaceful for hours before
descending into chaos after confrontations with the police. Looters
ransacked hip boutiques, running off with expensive sneakers. They
looted expensive purses from Alexander McQueen and tagged graffiti on
the walls and windows of Rodeo Drive in Beverly Hills, a symbol of
privilege and luxury.

The people turning out this time are different as well. The first rocks
and bottles hurled in the 1992 riots were in working-class black
neighborhoods, where white and Hispanic bystanders were attacked. This
time, the participants are mostly young and from diverse backgrounds and
races.

In these protests, rage and anger over racism and police abuses have
been compounded by outrage at another of America's most profound issues
--- growing income inequality. In an annual
\href{https://ucla.app.box.com/s/zy3dxv91oywq70iht1giyvlhknafjs8d}{countywide
survey} by the Luskin School of Public Affairs at University of
California, Los Angeles, nearly two-thirds of residents under 40 said
this year that the Los Angeles area was not a place where people who
worked hard could succeed, but rather ``a place where the rich keep
getting richer, and the average person cannot get ahead.''

Erwin Chemerinsky, who in 1992 was a law professor in Los Angeles living
in the Fairfax District and is now dean of the law school at University
of California, Berkeley, said he remembered explaining to his children
what was happening in 1992.

Now, he said, they are calling him from L.A. about new images on CNN of
burned-out cars and broken storefronts, stretching far more deeply into
their neighborhood than last time.

``It's just sad,'' Mr. Chemerinsky said. ``It's sad that the police
violence against African-Americans continues in the same way that
precipitated riots in the 1960s. It's sad that we have this enormous
economic disparity that has made people so desperate. It's sad that
there is so much anger and that we are so divided."

Image

Police officers were deployed on Saturday near the Rodeo Drive area of
Beverly Hills, where high-end boutiques were boarded up after protesters
entered the area and isolated looting incidents occurred.Credit...Bryan
Denton for The New York Times

For city and county officials in Los Angeles struggling to contain the
violence, the trauma of 1992 never fully healed and has rarely been far
from mind in recent days.

Hours before the city erupted in violence over the weekend, Mayor Eric
M. Garcetti tried to assure his anxious city, saying he would not need
to call on the National Guard.

``This is not 1992,'' he said.

A few hours later, with chaos growing, Mr. Garcetti was on the phone
with Gov. Gavin Newsom, asking him to send in the Guard. Mr. Garcetti
made a point to say that they would not be patrolling South Los Angeles
--- now predominantly Latino --- an acknowledgment of the painful
history, but also largely a moot point because the streets there have
been calm.

``I think it's very different from 1992 because this is a collective
national pain,'' Mr. Garcetti said in an interview. ``It happened in
Minneapolis on top of an incident in Louisville on top of an incident in
Georgia.''

In 1965, Mark Ridley-Thomas was an 11-year-old boy, standing on the
corner of Vernon and Hooper in South Los Angeles, watching the National
Guard roll through his neighborhood.

In 1992, as a first-term City Council member, he was at the First A.M.E.
Church awaiting the verdict in the Rodney King case. On Saturday night,
Mr. Ridley-Thomas, who is African-American and a member of the powerful
county board of supervisors, was at home because of the pandemic
watching the footage on television, and thinking about history.

``The locus of the crisis was not pinpointed in Los Angeles,'' he said,
of Mr. Floyd's death in Minneapolis. ``So there's some real differences,
but the pain, the hurt, the disgust, the frustration, the anger is real
and is cumulative. So it's not hard for people to reach back to '92 and
many events since that time.''

\href{https://www.nytimes3xbfgragh.onion/interactive/2020/06/03/us/minneapolis-police-use-of-force.html}{}

\includegraphics{https://static01.graylady3jvrrxbe.onion/images/2020/06/02/us/minneapolis-police-violence-promo-1591152401883/minneapolis-police-violence-promo-1591152401883-articleLarge.png}

\hypertarget{minneapolis-police-use-force-against-black-people-at-7-times-the-rate-of-whites}{%
\subsection{Minneapolis Police Use Force Against Black People at 7 Times
the Rate of
Whites}\label{minneapolis-police-use-force-against-black-people-at-7-times-the-rate-of-whites}}

When the officers use kicks, chokeholds, punches, takedowns, Mace spray,
Tasers and the like, the person subject to that force is black about 60
percent of the time.

George Gascon, who was a 38-year-old L.A.P.D. sergeant on the ground in
South Central in 1992, when the riots erupted at the corner of Florence
and Normandie, said he was, ``brokenhearted'' over the weekend watching
the television footage.

After the 1992 unrest and the Rampart corruption scandal, the L.A.P.D.
underwent a number of reforms and improved relationships with black and
brown communities. But the city still faces accusations of abuse, and to
this day, police officers are rarely prosecuted for shootings --- the
last time an officer faced charges was in 2000.

``At the end of the day, as soon as the dust settles, as soon as the
fires are put out and the broken glass is fixed,'' Mr. Gascon said, ``we
go back to business as usual. And we go back to giving a pass over and
over and over again to a broken policing system.''

He added, ``Floyd was definitely the spark, but I can tell you that
there are many Floyds in L.A. County happening all the time, just as
there are in other parts of the country.''

On Sunday, the protests moved further west, reaching Santa Monica, just
on the ocean.

Jaaye Person-Lynn, a lawyer who was out protesting there on Sunday, was
mindful of the history of Watts and Rodney King and said in an interview
with Spectrum News 1, a local news channel, that this time was
different.

Image

Protesters chanted and raised their hands as~some~faced~the~police in
Santa Monica, Calif., on Sunday.Credit...Bryan Denton for The New York
Times

``Now we are right here on the water,'' he told Spectrum. ``We can't get
any further west.''

And he vowed that protesters would continue to bring their voice to
enclaves of white privilege.

``We're going to start hitting these farmers' markets right where people
are most comfortable,'' he said. ``While they are buying their
gluten-free bread and their organic tomatoes, they're going to have to
feel it the same way we do.''

Shawn Hubler contributed reporting from Sacramento, and Jennifer Medina
from Los Angeles.

Advertisement

\protect\hyperlink{after-bottom}{Continue reading the main story}

\hypertarget{site-index}{%
\subsection{Site Index}\label{site-index}}

\hypertarget{site-information-navigation}{%
\subsection{Site Information
Navigation}\label{site-information-navigation}}

\begin{itemize}
\tightlist
\item
  \href{https://help.nytimes3xbfgragh.onion/hc/en-us/articles/115014792127-Copyright-notice}{©~2020~The
  New York Times Company}
\end{itemize}

\begin{itemize}
\tightlist
\item
  \href{https://www.nytco.com/}{NYTCo}
\item
  \href{https://help.nytimes3xbfgragh.onion/hc/en-us/articles/115015385887-Contact-Us}{Contact
  Us}
\item
  \href{https://www.nytco.com/careers/}{Work with us}
\item
  \href{https://nytmediakit.com/}{Advertise}
\item
  \href{http://www.tbrandstudio.com/}{T Brand Studio}
\item
  \href{https://www.nytimes3xbfgragh.onion/privacy/cookie-policy\#how-do-i-manage-trackers}{Your
  Ad Choices}
\item
  \href{https://www.nytimes3xbfgragh.onion/privacy}{Privacy}
\item
  \href{https://help.nytimes3xbfgragh.onion/hc/en-us/articles/115014893428-Terms-of-service}{Terms
  of Service}
\item
  \href{https://help.nytimes3xbfgragh.onion/hc/en-us/articles/115014893968-Terms-of-sale}{Terms
  of Sale}
\item
  \href{https://spiderbites.nytimes3xbfgragh.onion}{Site Map}
\item
  \href{https://help.nytimes3xbfgragh.onion/hc/en-us}{Help}
\item
  \href{https://www.nytimes3xbfgragh.onion/subscription?campaignId=37WXW}{Subscriptions}
\end{itemize}
