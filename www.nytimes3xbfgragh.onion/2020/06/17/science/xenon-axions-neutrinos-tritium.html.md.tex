Sections

SEARCH

\protect\hyperlink{site-content}{Skip to
content}\protect\hyperlink{site-index}{Skip to site index}

\href{https://www.nytimes3xbfgragh.onion/section/science}{Science}

\href{https://myaccount.nytimes3xbfgragh.onion/auth/login?response_type=cookie\&client_id=vi}{}

\href{https://www.nytimes3xbfgragh.onion/section/todayspaper}{Today's
Paper}

\href{/section/science}{Science}\textbar{}Seeking Dark Matter, They
Detected Another Mystery

\url{https://nyti.ms/37CnALH}

\begin{itemize}
\item
\item
\item
\item
\item
\item
\end{itemize}

Advertisement

\protect\hyperlink{after-top}{Continue reading the main story}

Supported by

\protect\hyperlink{after-sponsor}{Continue reading the main story}

Out There

\hypertarget{seeking-dark-matter-they-detected-another-mystery}{%
\section{Seeking Dark Matter, They Detected Another
Mystery}\label{seeking-dark-matter-they-detected-another-mystery}}

Do signals from beneath an Italian mountain herald a revolution in
physics?

\includegraphics{https://static01.graylady3jvrrxbe.onion/images/2020/06/23/science/17SCI-OUTTHERE-XENON1/merlin_173596674_5dd10fac-7a46-4b88-8083-d766973058ca-articleLarge.jpg?quality=75\&auto=webp\&disable=upscale}

\href{https://www.nytimes3xbfgragh.onion/by/dennis-overbye}{\includegraphics{https://static01.graylady3jvrrxbe.onion/images/2018/07/30/multimedia/author-dennis-overbye/author-dennis-overbye-thumbLarge.png}}

By \href{https://www.nytimes3xbfgragh.onion/by/dennis-overbye}{Dennis
Overbye}

\begin{itemize}
\item
  June 17, 2020
\item
  \begin{itemize}
  \item
  \item
  \item
  \item
  \item
  \item
  \end{itemize}
\end{itemize}

It could be a key to the secret of the universe. Or just annoying
background noise, another item to be calibrated in future experiments.

A team of scientists hunting dark matter has recorded suspicious pings
coming from a vat of liquid xenon underneath a mountain in Italy. They
are not claiming to have discovered dark matter --- or anything, for
that matter --- yet. But these pings, they say, could be tapping out a
new view of the universe.

If the signal is real and persists, the scientists say, it may be
evidence of a species of subatomic particles called axions --- long
theorized to play a crucial role in keeping nature symmetrical but never
seen --- streaming from the sun.

``It's not dark matter but discovering a new particle would be
phenomenal,'' said Elena Aprile of Columbia University, who leads the
Xenon Collaboration, the project that made the detection.

In a statement, the collaboration said that detecting the axions would
have ``a large impact on our understanding of fundamental physics, but
also on astrophysical phenomena.''

But there are other explanations for the finding. Instead of axions, the
scientists may have detected a new, unexpected property of the slippery
ghostly particles called neutrinos. Yet another equally likely
explanation is that their detector has been contaminated by vanishingly
tiny amounts of tritium, a rare radioactive form of hydrogen.

The collaboration posted a paper
\href{http://science.purdue.edu/xenon1t/wp-content/uploads/2020/06/xenon1tlowersearches.pdf}{describing
the results to its website on Wednesday}.

Or it could all just be a statistical fluctuation that will go away with
more data. Members of Dr. Aprile's team conceded that the best
explanation they had right now --- that axions were to blame --- has two
chances in 10,000 of being a fluke, a far cry from the ``5-sigma''
criterion of less than one chance in a million needed in particle
physics to certify a ``discovery.''

``We want to be very clear that all we are reporting is observation of
an excess (a fairly significant one) and not a discovery of any kind,''
said Evan Shockley of the University of Chicago in an email.

Frank Wilczek, a Nobel laureate at the Massachusetts Institute of
Technology who was one of the first physicists to propose the axion,
noted the collaboration's own caveats in the paper. But he said it was
``certainly intriguing, and the physics community will be eagerly
awaiting further developments.''

\includegraphics{https://static01.graylady3jvrrxbe.onion/images/2020/06/23/science/17SCI-OUTTHERE-XENON2/merlin_171264138_772f3191-5a9c-4127-a07b-71da4de2e125-articleLarge.jpg?quality=75\&auto=webp\&disable=upscale}

Other scientists responded with cautious excitement, or excited caution.

``I'm trying to be calm here, but it's hard not to be hyperbolic,'' said
Neal Weiner, a particle theorist at New York University. ``If this is
real, calling it a game changer would be an understatement.''

Michael Turner, a cosmologist with the Kavli Foundation in Los Angeles,
called the Xenon collaboration ``a beautiful experiment.''

``I really want to believe it, but I think it will probably break my
heart,'' he said. ``But for now, I am excited that it could be something
new and important that cheers us all up.''

\hypertarget{wishing-on-a-wimp}{%
\subsection{Wishing on a WIMP}\label{wishing-on-a-wimp}}

Dr. Aprile's Xenon experiment is currently the largest and most
sensitive in an alphabet soup of efforts aimed at detecting and
identifying dark matter, the mysterious substance that astronomers have
concluded swamps the universe, outweighing ordinary atomic matter by a
factor of five to one.

In modern cosmology, dark matter is the secret sauce of the universe. It
collects in invisible clouds, attracting ordinary atomic matter into
lumps that eventually light up as stars and galaxies.

The best guess is that this dark matter consists of clouds of exotic
subatomic particles left over from the Big Bang and known generically as
WIMPs, for weakly interacting massive particles, hundreds or thousands
of times more massive than a hydrogen atom.

The \href{http://www.xenon1t.org/}{Xenon Collaboration} is a
multinational team of 163 scientists from 28 institutions and 11
countries. In a tunnel a mile under the rock at the Gran Sasso National
Laboratory in Italy, Dr. Aprile and her colleagues have wired a
succession of vats containing liquid xenon with photomultipliers and
other sensors. The hope is that her team's device --- far underground to
shield it from cosmic rays and other worldly forms of interference ---
would spot the rare collision between a WIMP and a xenon atom. The
collision should result in a flash of light and a cloud of electrical
charge.

So far,
\href{https://www.nytimes3xbfgragh.onion/2013/10/31/science/space/dark-matter-experiment-has-found-nothing-scientists-say-proudly.html}{it
hasn't happened}.

The latest version, called Xenon1T, ran from 2016 to 2018 with two tons
of xenon as the target.

Luca Grandi of the University of Chicago explained that in its most
recent analysis of that experiment, the team had looked for electrons,
rather than the heavier xenon nuclei, recoiling from collisions. Among
other things, that could be the signature of particles much lighter than
the putative WIMPs striking the xenon.

Simulations and calculations suggested that random events should have
produced about 232 such recoils over the course of a year.

But from February 2017 to February 2018, the detector recorded 285, an
excess of 53 recoils.

Dr. Grandi said, ``We have seen the excess more than a year ago, and we
have tried in any way to destroy it,'' referring to the measurements.

The collaboration is in the final stages of preparing a bigger, more
sensitive version of its experiment. It was
\href{https://www.nytimes3xbfgragh.onion/2020/04/07/science/dark-matter-elena-aprile-coronavirus.html}{delayed
by the coronavirus}lockdown in Italy but could now start up by the end
of this year.

If the excess is real, it should show up within a month or two after it
starts running, Dr. Grandi said.

So for now, all three possibilities --- axions, neutrinos or tritium ---
are still alive, he said.

\hypertarget{subatomic-laundry}{%
\subsection{Subatomic Laundry}\label{subatomic-laundry}}

And so axions could be about to enter onto the main stage of cosmology.

The story of axions begins in 1977, when
\href{https://en.wikipedia.org/wiki/Roberto_Peccei}{Roberto Peccei}, a
professor at the University of California, Los Angeles, who died on June
1, and \href{https://en.wikipedia.org/wiki/Helen_Quinn}{Helen Quinn},
emerita professor at Stanford, suggested a slight modification to the
theory that governs strong nuclear forces, making sure that it is
invariant to the direction of time, a feature that physicists consider a
necessity for the universe.

Both Dr. Wilczek and Steven Weinberg of the University of Texas at
Austin independently realized that this modification implied the
existence of a new subatomic particle. Dr. Wilczek called it the axion,
and the name stuck.

Image

The Gran Sasso Laboratory is the largest underground research center in
the world, buried more than 4,500 feet underground.Credit...Tommaso
Guicciardini/Science Source

``A few years before, a supermarket display of brightly colored boxes of
a laundry detergent named Axion had caught my eye,'' he related in a
\href{https://www.quantamagazine.org/how-axions-may-explain-times-arrow-20160107/}{recent
essay in Quanta}. ``It occurred to me that `axion' sounded like the name
of a particle and really ought to be one.''

When he realized that the Peccei-Quinn theory implied a particle, he saw
his chance.

Axions have never been detected either directly or indirectly. And the
theory does not predict their mass, which makes it hard to look for
them. It only predicts that they would be weird and would barely
interact with regular matter. Theorists have imagined many versions of
axions that could play different roles in the universe, including being
the dark matter that, rather than WIMPS, fills the universe and binds
galaxies. And although they are not WIMPS, they share some of those
particles' imagined weird abilities, such as being able to float through
Earth and our bodies like smoke through a screen door.

In order to fulfill the requirements of cosmologists, however, such
dark-matter axions would need to have a mass of less than a thousandth
of an electron volt in the units of mass and energy preferred by
physicists, according to Dr. Turner. (By comparison, the electrons that
dance around in your smartphone weigh in at half a million electron
volts each.) What they lack in heft they would more than make up for in
numbers.

That would make individual cosmic dark-matter axions too slow and
ethereal to be detected by the Xenon experiment.

But axions could also be produced by nuclear reactions in the sun, and
those ``solar axions'' would have enough energy to ping the Xenon
detector right where it is most sensitive, Dr. Grandi said.

Solar axions would not be dark matter, but verifying that they actually
exist would be a major step toward opening up the possibility that
another kind of axion could be dark matter, according to Dr. Wilczek.

Other experiments are underway to try to detect cosmic dark matter
axions directly. Among them are the
\href{https://depts.washington.edu/admx/}{Axion Dark Matter Experiment}
at the University of Washington, which uses a strong magnetic field to
detect the axions by watching them turn into
\href{https://en.wikipedia.org/wiki/Microwave}{microwaves}. And an
experiment at CERN in Switzerland, CAST for
\href{https://home.cern/science/experiments/cast}{CERN Axion Solar
Telescope,} has also looked for axions from the sun.

\hypertarget{flies-in-the-ointment}{%
\subsection{Flies in the Ointment}\label{flies-in-the-ointment}}

Image

Scientists at the Gran Sasso Laboratory working on the
photomultiplier~tube array.Credit...Xenon Experiment

The other exciting, though slightly less likely, possibility is that the
Xenon collaboration's excess signals come from
\href{https://www.nytimes3xbfgragh.onion/2020/04/15/science/physics-neutrino-antimatter-ichikawa-t2k.html}{the
wispy particles known as neutrinos}, which are real, and weird, and
zipping through our bodies by the trillions every second.

Ordinarily, these neutrinos would not contribute much to the excess of
events the detector read. But they would do so if they had an intrinsic
magnetism that physicists call a magnetic moment. That would give them a
higher probability of interacting with the xenon and tripping the
detector. According to the standard lore, neutrinos, which are
electrically neutral, do not carry magnetism. The discovery that they
did would require rewriting the rules as they apply to neutrinos.

That, said Dr. Weiner, would be ``a very very big deal,'' because it
would imply that there are new fundamental particles out there to look
for --- new physics.

However, Dr. Weiner and others, including the Xenon authors themselves,
cautioned that both the axion and the magnetic neutrino hypotheses
conflict with astronomical observations.

Dead stars, like white dwarfs, that have used up their nuclear fuel fade
and cool off over time as they radiate their energy away. If they were
emitting axions or these magnetic neutrinos like the sun, Dr. Weiner
pointed out, they would be losing energy and fading faster than what
astronomers see. He called this problem ``a big tension'' that he and
other theorists will be brainstorming.

Tritium remains another fly in the ointment.

Hydrogen is the lightest and most abundant element in the universe.
Tritium is one of its isotopes, radioactive with a half-life of 12.3
years. It is mostly produced by cosmic rays interacting with the
atmosphere and is used in hydrogen bombs to help increase their
explosive power.

If the isotope is the cause of the excess, the amount that would cause
the readings is about ``3 tritium atoms per kilogram of xenon, really an
insanely low number,'' Dr. Grandi said. That is almost impossible to
measure except with an instrument as sensitive as the Xenon detector, he
said.

It may turn out, he admitted, that tritium explains the excess and that
tritium contamination will just be one more detail that has to be
considered or calibrated in future detectors.

``But of course we are really excited about the possibility that these
are actual signals,'' Dr. Grandi said.

He added: ``It's pointing towards physics beyond the standard model, so
it's a big deal. So I think it would be an important discovery.''

Dr. Grandi is now in Northern Italy and is anxious to get back to Gran
Sasso and start the work of getting the next phase of the Xenon
experiment online.

Dr. Aprile, who is leaving New York for Italy in a few weeks, said, ``I
am mostly excited, but the really the excitement is it makes you feel so
good that you have a new detector coming up.''

The universe is waiting for an answer.

``We need to push,'' Dr. Grandi said, ``And now I think that, you know,
we might be sitting on something that might be really exciting.''

Advertisement

\protect\hyperlink{after-bottom}{Continue reading the main story}

\hypertarget{site-index}{%
\subsection{Site Index}\label{site-index}}

\hypertarget{site-information-navigation}{%
\subsection{Site Information
Navigation}\label{site-information-navigation}}

\begin{itemize}
\tightlist
\item
  \href{https://help.nytimes3xbfgragh.onion/hc/en-us/articles/115014792127-Copyright-notice}{©~2020~The
  New York Times Company}
\end{itemize}

\begin{itemize}
\tightlist
\item
  \href{https://www.nytco.com/}{NYTCo}
\item
  \href{https://help.nytimes3xbfgragh.onion/hc/en-us/articles/115015385887-Contact-Us}{Contact
  Us}
\item
  \href{https://www.nytco.com/careers/}{Work with us}
\item
  \href{https://nytmediakit.com/}{Advertise}
\item
  \href{http://www.tbrandstudio.com/}{T Brand Studio}
\item
  \href{https://www.nytimes3xbfgragh.onion/privacy/cookie-policy\#how-do-i-manage-trackers}{Your
  Ad Choices}
\item
  \href{https://www.nytimes3xbfgragh.onion/privacy}{Privacy}
\item
  \href{https://help.nytimes3xbfgragh.onion/hc/en-us/articles/115014893428-Terms-of-service}{Terms
  of Service}
\item
  \href{https://help.nytimes3xbfgragh.onion/hc/en-us/articles/115014893968-Terms-of-sale}{Terms
  of Sale}
\item
  \href{https://spiderbites.nytimes3xbfgragh.onion}{Site Map}
\item
  \href{https://help.nytimes3xbfgragh.onion/hc/en-us}{Help}
\item
  \href{https://www.nytimes3xbfgragh.onion/subscription?campaignId=37WXW}{Subscriptions}
\end{itemize}
