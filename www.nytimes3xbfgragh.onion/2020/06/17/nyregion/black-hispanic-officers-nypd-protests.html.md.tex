\href{/section/nyregion}{New York}\textbar{}How Black N.Y.P.D. Officers
Really Feel About the Floyd Protesters

\url{https://nyti.ms/2Y6NuUM}

\begin{itemize}
\item
\item
\item
\item
\item
\item
\end{itemize}

\hypertarget{race-and-america}{%
\subsubsection{\texorpdfstring{\href{https://www.nytimes3xbfgragh.onion/news-event/george-floyd-protests-minneapolis-new-york-los-angeles?name=styln-george-floyd\&region=TOP_BANNER\&block=storyline_menu_recirc\&action=click\&pgtype=Article\&impression_id=cc9f6e50-f52f-11ea-bb13-2130c1c2f70c\&variant=undefined}{Race
and America}}{Race and America}}\label{race-and-america}}

\begin{itemize}
\tightlist
\item
  \href{https://www.nytimes3xbfgragh.onion/2020/09/11/us/black-police-chiefs-reform.html?name=styln-george-floyd\&region=TOP_BANNER\&block=storyline_menu_recirc\&action=click\&pgtype=Article\&impression_id=cc9f6e51-f52f-11ea-bb13-2130c1c2f70c\&variant=undefined}{Black
  Police Chiefs}
\item
  \href{https://www.nytimes3xbfgragh.onion/2020/09/04/nyregion/rochester-police-daniel-prude.html?name=styln-george-floyd\&region=TOP_BANNER\&block=storyline_menu_recirc\&action=click\&pgtype=Article\&impression_id=cc9f6e52-f52f-11ea-bb13-2130c1c2f70c\&variant=undefined}{What
  Happened in Rochester, N.Y.}
\item
  \href{https://www.nytimes3xbfgragh.onion/2020/08/30/us/portland-shooting-explained.html?name=styln-george-floyd\&region=TOP_BANNER\&block=storyline_menu_recirc\&action=click\&pgtype=Article\&impression_id=cc9f6e53-f52f-11ea-bb13-2130c1c2f70c\&variant=undefined}{Portland
  Shooting}
\item
  \href{https://www.nytimes3xbfgragh.onion/2020/08/30/us/breonna-taylor-police-killing.html?name=styln-george-floyd\&region=TOP_BANNER\&block=storyline_menu_recirc\&action=click\&pgtype=Article\&impression_id=cc9f9560-f52f-11ea-bb13-2130c1c2f70c\&variant=undefined}{Breonna
  Taylor's Life and Death}
\end{itemize}

\includegraphics{https://static01.graylady3jvrrxbe.onion/images/2020/06/16/nyregion/00nyunrest-copvoices/00nyunrest-copvoices-articleLarge.jpg?quality=75\&auto=webp\&disable=upscale}

Sections

\protect\hyperlink{site-content}{Skip to
content}\protect\hyperlink{site-index}{Skip to site index}

\hypertarget{how-black-nypd-officers-really-feel-about-the-floyd-protesters}{%
\section{How Black N.Y.P.D. Officers Really Feel About the Floyd
Protesters}\label{how-black-nypd-officers-really-feel-about-the-floyd-protesters}}

Most officers of color share the protesters' mission to defeat racism,
but the unrest has reminded the officers that they are still often seen
as the enemy.

Detective Dmaine Freeland posted a video denouncing the role of police
officers in the death of George Floyd.Credit...Elias Williams for The
New York Times

Supported by

\protect\hyperlink{after-sponsor}{Continue reading the main story}

By \href{https://www.nytimes3xbfgragh.onion/by/ashley-southall}{Ashley
Southall} and
\href{https://www.nytimes3xbfgragh.onion/by/edgar-sandoval}{Edgar
Sandoval}

\begin{itemize}
\item
  June 17, 2020
\item
  \begin{itemize}
  \item
  \item
  \item
  \item
  \item
  \item
  \end{itemize}
\end{itemize}

\href{https://www.nytimes3xbfgragh.onion/2016/02/21/magazine/a-black-police-officers-fight-against-the-nypd.html}{Edwin
Raymond}, a black lieutenant in the Police Department, heard racial
insults --- ``Sellout!'' and ``Uncle Tom!'' --- rising above protesters'
chants as he helped to control the crowds at recent demonstrations in
Brooklyn against police brutality and racism.

He said he understood the words were aimed at black officers like him.
He tried not to take them personally, but the shouts were particularly
painful, he said, because he has long been an outspoken critic of what
he sees as racial discrimination within the department.

``I'm not blind to the issues, but I'm torn,'' Lieutenant Raymond said.
``As I'm standing there with my riot helmet and being called a `coon,'
people have no idea that I identify with them. I understand them. I'm
here for them. I've been trying to be here as a change agent.''

Lieutenant Raymond, 34, is one of hundreds of black and Hispanic
officers in New York City who have found themselves caught between
competing loyalties. Many said they sympathized with protesters
\href{https://apnews.com/c33525fc78ad95b39098f720688d2991}{across the
city and the country} who have turned out en masse to demonstrate
against police brutality in the wake of George Floyd's death at the
hands of a white officer in Minneapolis.

The officers said they had experienced racism and share the protesters'
mission to combat it. Still, the unrest offers painful reminders that
many black and Hispanic New Yorkers see them as enemies in uniform,
worsening the
\href{https://www.nytimes3xbfgragh.onion/2016/07/19/nyregion/black-police-officers-feel-the-inner-tug-of-a-dual-role.html}{internal
tug-of-war}between their identity and their badges.

Since Mayor Bill de Blasio took office in 2014, the Police Department
has become ``majority-minority'': White officers now make up less than
half of the 36,000 uniformed members of the force. The number of
Hispanic officers has grown to make up 29 percent of the force, while
the percentage of Asian officers in the force doubled to 9 percent,
according to the department's data. (In the 2010 census, about 29
percent of city residents were Hispanic and 14 percent Asian.)

But the department has struggled to boost the ranks of black officers.
Black people make up about 24 percent of the city but only 15 percent of
the force, a number that has remained flat since 2014. And even though
more black and Hispanic chiefs have been elevated to leadership roles
under Mr. de Blasio, two-thirds of the officers in the department's top
ranks, from lieutenant to chief, are still white, the data show.

\includegraphics{https://static01.graylady3jvrrxbe.onion/images/2020/06/18/nyregion/00nyvirus-copvoices2/00nyvirus-copvoices2-articleLarge.jpg?quality=75\&auto=webp\&disable=upscale}

In the wake of Mr. Floyd's death on May 25, some black officers felt a
duty to speak out. Two days later, Dmaine Freeland, a black detective in
Brooklyn, put on his uniform, sat at his kitchen table, clasped his
hands and recorded a video on his cellphone.

The detective denounced the officer who had knelt on Mr. Floyd's neck
for nearly nine minutes while three other officers watched. He called
him an ``enemy'' and asked ``every good cop to speak up.'' Then he
posted
\href{https://www.facebookcorewwwi.onion/dromarr.freeland/videos/10216336652991409}{the
two-minute video} on Facebook.

``I just spiritually felt the need to speak for good cops out there,''
Detective Freeland, 44, said in an interview, so ``that we don't get
bunched in with the actions of one or four bad cops.''

Sgt. Khadijah Faison, a black officer in Jamaica, Queens, took a public
step of another kind: She knelt with protesters in a gesture of
solidarity.

Sergeant Faison had been working at a midday protest on May 31 near the
103rd Precinct station, where she is part of the community affairs unit.
The demonstrators formed a circle and beckoned her and other officers to
join them for prayer. She said she felt moved to do so.

``If you are asking to pray, you kneel. So I kneeled too,'' she said.
``I think we were all looking for a sign.''

Two other officers also decided to kneel next to her, including her
commanding officer, who is white.

The department frowns on officers' making political statements in
uniform, but the Floyd protests have created a different dynamic, as top
police officials and union leaders have condemned the officers in
Minneapolis.

The police commissioner, Dermot F. Shea,
\href{https://twitter.com/NYPDShea/status/1267177766769889282}{shared a
NY1 News reporter's photos} of the moment Sergeant Faison knelt,
writing: ``We need more of this, to see and hear each other, to work
together, to recognize that our differences are our strength.''

The next day, Chief Terence A. Monahan, the department's highest-ranking
uniformed officer, also knelt with protesters in Manhattan, and other
officers have followed suit.

Black and Hispanic officers said the show of support from white officers
and commanders like Chief Monahan was one of the ways the current
protests have been different from past demonstrations over police
killings.

Several of the officers said they were disappointed by the looting and
violence that has occurred during and after some protest marches,
\href{https://www.nytimes3xbfgragh.onion/2020/06/05/nyregion/police-kettling-protests-nyc.html}{including
instances of unnecessary force used by the police against
demonstrators}.

The violence has made it difficult to have reasonable exchanges with
protesters, these officers said. The police have been videotaped
shoving, beating and pepper-spraying demonstrators. One officer has been
arrested, at least two others have been suspended, and dozens are under
investigation over attacks on protesters.

At the same time, the black and Hispanic officers say they feel unnerved
by violence aimed at the police. Protesters have hit officers with rocks
and bricks and have surrounded occupied police cars, throwing heavy
objects at them. Some have even hurled
\href{https://www.nytimes3xbfgragh.onion/2020/06/07/nyregion/molotov-cocktail-lawyers-nyc.html}{Molotov
cocktails}.

``You've got to worry that someone's going to hurt you,'' Officer Pedro
Serrano, who works in the South Bronx, said. ``But on the other hand,
you understand the fight. You've seen the racism.''

But they have also been heartened that the crowds marching after Mr.
Floyd's death are bigger and more racially diverse than those that
turned out after the death of Eric Garner, a Staten Island father who
died after he was put in an illegal chokehold by a New York police
officer in 2014.

``You'd think that would be the big one,'' Officer Serrano said,
referring to the Garner case. ``But police departments across the world
are showing time and time again that people of color, they don't matter.
So I'm glad to see that more and more people are speaking up and seeing
what's really happening.''

Lieutenant Raymond said the protesters were calling attention to some of
the same policing practices that he and 11 other New York officers were
seeking to change when they sued
\href{https://www.nytimes3xbfgragh.onion/2016/02/21/magazine/a-black-police-officers-fight-against-the-nypd.html}{the
city and the department in 2015} over racial discrimination in
enforcement and in promotions. Officer Serrano is also one of the
plaintiffs.

The officers said in their lawsuit that the Police Department uses a
\href{https://www.nytimes3xbfgragh.onion/2019/12/06/nyregion/nyc-police-subway-racial-profiling.html}{racist
and illegal quota system} to target black and Hispanic people for
arrests and summonses. Their careers were stalled, the officers claimed,
because they objected to the quotas as unfair.

Lieutenant Raymond said the emphasis on numerical targets has led to the
overly aggressive policing of black and Hispanic neighborhoods, which in
turn has led to more fatal encounters between residents of those areas
and the police. ``It just becomes an oppressive organization,'' he said.

The Police Department has denied the existence of quotas and disputes
the accusation that its strategies are racist. The city is fighting the
lawsuit.

But similar charges of racism in police departments across the country
lie at the heart of the protesters' complaints.

Image

Across the city, protesters have turned out en masse to demonstrate
against police brutality.Credit...Demetrius Freeman for The New York
Times

Officer Serrano said recent measures passed by the State Legislature
aimed at addressing some problems raised by Mr. Floyd's death --- a
\href{https://www.nytimes3xbfgragh.onion/2020/06/12/nyregion/50a-repeal-police-floyd.html?searchResultPosition=20}{statewide
ban on police chokeholds} and the repeal of a statute that kept officer
misconduct secret --- were small improvements.

But he said they fail to address the main issue, which, in his view, is
the racial biases of the Police Department's leaders. ``If you have a
racist leadership who is never held responsible, nothing's going to
happen,'' he said. ``You're putting Band-Aids instead of fixing what the
problem is.''

Commissioner Shea has defended his department's record on race and
diversity. He has pointed out that the department in the last six years
has moved away from flawed strategies like ``stop and frisk,'' which a
judge found disproportionately affected black and Hispanic residents and
ruled unconstitutional. They have also steadily reduced arrests and
summonses.

Detective Yuseff Hamm, the former president of the N.Y.P.D. Guardians
Association, a group of about 1,000 black police officers, said the
killing of Mr. Floyd had eroded the progress that black leaders in the
department, like Chief Jeffrey Maddrey in Brooklyn, have made in
creating a positive image of policing in black and brown neighborhoods.

``We have to make up ground that we previously had and lost, to get
people interested again in becoming police officers,'' he said.

Detective Felicia Richards, the current president of the N.Y.P.D.
Guardians, said the visibility of black officers had been important in
keeping the peace at the protests. ``Kids who look like me need to know
they are secured,'' she said. ``This protest is as much about us as it
is about them.''

But sometimes visibility makes officers targets for abuse. During a
protest earlier this month outside Trump International Hotel in Columbus
Circle, a young black woman saw a black police commander wearing a white
shirt and raised her voice to get his attention.

``Hey, you, Uncle Tom! When are you leaving your master's house!'' she
shouted over and over again, using an epithet for black people accused
of appeasing whites.

The police official glared at the heckler before shifting his gaze, his
face a rigid mask showing no emotion.

Image

Officers forming a shield wall during a protest after curfew in Brooklyn
on June 3.Credit...Amr Alfiky for The New York Times

Other protesters have seemed to ignore the possibility black officers
might be sympathetic to their cause. On another night, two black
officers tried to get protesters to move from the street onto the
sidewalk in front of the Barclays Center. A white woman gave one of them
a hard time, but he remained polite.

Then a white man walked up behind him and yelled, ``He wants to stomp on
your neck and kill you!'' The officer flashed a look of exasperation
before turning around and asking the woman, again, to move toward the
sidewalk.

Other protesters have tried to make black officers feel guilty,
suggesting they are insufficiently upset about Mr. Floyd. ``They'll say,
`How do you feel if it was your child? If that was your husband or that
was your father?''' Sergeant Faison, in Queens, said. ``It's not about a
side. I experience the same pain that you experience.''

Officer Oriade Harbor, 38, a transgender black man assigned to Police
Headquarters, said that even though he often speaks out against what he
sees as social injustice, when he does police work he is still seen as
``part of a system that is oppressive to black people.''

``People treat me different in uniform, because they only see the
uniform,'' he said. He added, ``At the end of the day I am a black
person who dons a blue uniform. I am a trans male. I walk in all of
these worlds.''

Ali Watkins and Alan Feuer contributed reporting. Alain Delaquérière
contributed research.

Advertisement

\protect\hyperlink{after-bottom}{Continue reading the main story}

\hypertarget{site-index}{%
\subsection{Site Index}\label{site-index}}

\hypertarget{site-information-navigation}{%
\subsection{Site Information
Navigation}\label{site-information-navigation}}

\begin{itemize}
\tightlist
\item
  \href{https://help.nytimes3xbfgragh.onion/hc/en-us/articles/115014792127-Copyright-notice}{©~2020~The
  New York Times Company}
\end{itemize}

\begin{itemize}
\tightlist
\item
  \href{https://www.nytco.com/}{NYTCo}
\item
  \href{https://help.nytimes3xbfgragh.onion/hc/en-us/articles/115015385887-Contact-Us}{Contact
  Us}
\item
  \href{https://www.nytco.com/careers/}{Work with us}
\item
  \href{https://nytmediakit.com/}{Advertise}
\item
  \href{http://www.tbrandstudio.com/}{T Brand Studio}
\item
  \href{https://www.nytimes3xbfgragh.onion/privacy/cookie-policy\#how-do-i-manage-trackers}{Your
  Ad Choices}
\item
  \href{https://www.nytimes3xbfgragh.onion/privacy}{Privacy}
\item
  \href{https://help.nytimes3xbfgragh.onion/hc/en-us/articles/115014893428-Terms-of-service}{Terms
  of Service}
\item
  \href{https://help.nytimes3xbfgragh.onion/hc/en-us/articles/115014893968-Terms-of-sale}{Terms
  of Sale}
\item
  \href{https://spiderbites.nytimes3xbfgragh.onion}{Site Map}
\item
  \href{https://help.nytimes3xbfgragh.onion/hc/en-us}{Help}
\item
  \href{https://www.nytimes3xbfgragh.onion/subscription?campaignId=37WXW}{Subscriptions}
\end{itemize}
