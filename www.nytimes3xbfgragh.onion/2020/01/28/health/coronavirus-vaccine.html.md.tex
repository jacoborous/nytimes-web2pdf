Sections

SEARCH

\protect\hyperlink{site-content}{Skip to
content}\protect\hyperlink{site-index}{Skip to site index}

\href{https://www.nytimes3xbfgragh.onion/section/health}{Health}

\href{https://myaccount.nytimes3xbfgragh.onion/auth/login?response_type=cookie\&client_id=vi}{}

\href{https://www.nytimes3xbfgragh.onion/section/todayspaper}{Today's
Paper}

\href{/section/health}{Health}\textbar{}Researchers Are Racing to Make a
Coronavirus Vaccine. Will It Help?

\url{https://nyti.ms/2RAOsp9}

\begin{itemize}
\item
\item
\item
\item
\item
\item
\end{itemize}

\hypertarget{the-coronavirus-outbreak}{%
\subsubsection{\texorpdfstring{\href{https://www.nytimes3xbfgragh.onion/news-event/coronavirus?name=styln-coronavirus-national\&region=TOP_BANNER\&block=storyline_menu_recirc\&action=click\&pgtype=Article\&impression_id=92efbf00-f4cd-11ea-a9ee-334f178d1f9e\&variant=undefined}{The
Coronavirus
Outbreak}}{The Coronavirus Outbreak}}\label{the-coronavirus-outbreak}}

\begin{itemize}
\tightlist
\item
  live\href{https://www.nytimes3xbfgragh.onion/2020/09/11/world/covid-19-coronavirus.html?name=styln-coronavirus-national\&region=TOP_BANNER\&block=storyline_menu_recirc\&action=click\&pgtype=Article\&impression_id=92efbf01-f4cd-11ea-a9ee-334f178d1f9e\&variant=undefined}{Latest
  Updates}
\item
  \href{https://www.nytimes3xbfgragh.onion/interactive/2020/us/coronavirus-us-cases.html?name=styln-coronavirus-national\&region=TOP_BANNER\&block=storyline_menu_recirc\&action=click\&pgtype=Article\&impression_id=92efbf02-f4cd-11ea-a9ee-334f178d1f9e\&variant=undefined}{Maps
  and Cases}
\item
  \href{https://www.nytimes3xbfgragh.onion/interactive/2020/science/coronavirus-vaccine-tracker.html?name=styln-coronavirus-national\&region=TOP_BANNER\&block=storyline_menu_recirc\&action=click\&pgtype=Article\&impression_id=92efbf03-f4cd-11ea-a9ee-334f178d1f9e\&variant=undefined}{Vaccine
  Tracker}
\item
  \href{https://www.nytimes3xbfgragh.onion/2020/09/10/us/politics/fda-coronavirus-vaccine.html?name=styln-coronavirus-national\&region=TOP_BANNER\&block=storyline_menu_recirc\&action=click\&pgtype=Article\&impression_id=92efbf04-f4cd-11ea-a9ee-334f178d1f9e\&variant=undefined}{F.D.A.
  Regulators' Self-Defense}
\item
  \href{https://www.nytimes3xbfgragh.onion/2020/09/09/upshot/coronavirus-surprise-test-fees.html?name=styln-coronavirus-national\&region=TOP_BANNER\&block=storyline_menu_recirc\&action=click\&pgtype=Article\&impression_id=92efbf05-f4cd-11ea-a9ee-334f178d1f9e\&variant=undefined}{Surprise
  Test Fees}
\end{itemize}

Advertisement

\protect\hyperlink{after-top}{Continue reading the main story}

Supported by

\protect\hyperlink{after-sponsor}{Continue reading the main story}

\hypertarget{researchers-are-racing-to-make-a-coronavirus-vaccine-will-it-help}{%
\section{Researchers Are Racing to Make a Coronavirus Vaccine. Will It
Help?}\label{researchers-are-racing-to-make-a-coronavirus-vaccine-will-it-help}}

New technology and better coordination have sped up development. But a
coronavirus vaccine is still months --- and most likely years --- away.

\includegraphics{https://static01.graylady3jvrrxbe.onion/images/2020/01/28/science/28VIRUS-VACCINE1/28VIRUS-VACCINE1-articleLarge.jpg?quality=75\&auto=webp\&disable=upscale}

By \href{https://www.nytimes3xbfgragh.onion/by/knvul-sheikh}{Knvul
Sheikh} and
\href{https://www.nytimes3xbfgragh.onion/by/katie-thomas}{Katie Thomas}

\begin{itemize}
\item
  Published Jan. 28, 2020Updated June 10, 2020
\item
  \begin{itemize}
  \item
  \item
  \item
  \item
  \item
  \item
  \end{itemize}
\end{itemize}

In the early days of January, as cases of a strange, pneumonia-like
illness were reported in China, researchers at the National Institutes
of Health in Maryland readied themselves to hunt for a
\href{https://www.nytimes3xbfgragh.onion/2020/04/27/world/europe/coronavirus-vaccine-update-oxford.html}{vaccine}
to prevent the new disease.

They had clues that a
\href{https://www.nytimes3xbfgragh.onion/2020/04/27/world/europe/coronavirus-vaccine-update-oxford.html}{coronavirus},
similar to ones that caused the SARS outbreak in 2003 and MERS in 2012,
was the culprit. Dr. Barney Graham, deputy director of the Vaccine
Research Center at the N.I.H, urged government scientists in China to
share the genetic makeup of the virus so his team could begin its race
to develop a vaccine.

On Friday, Jan. 10, the Chinese scientists posted the information on
\href{https://www.ncbi.nlm.nih.gov/nuccore/MN908947}{a public database}.
The next morning, Dr. Graham's team was in the lab. And within hours,
they had pinpointed the letters of the genetic code that could be used
to make a vaccine.

Scientists in Australia and at least three companies --- Johnson \&
Johnson,
\href{https://www.nytimes3xbfgragh.onion/2020/05/18/health/coronavirus-vaccine-moderna.html}{Moderna
Therapeutics} and Inovio Pharmaceuticals --- are also working on vaccine
candidates to stop the spread of the disease, which has infected about
6,000 people and killed more than 130.

``Everybody is trying to move as quickly as possible,'' said Jacqueline
Shea, the chief operating officer at Inovio.

Inovio received a grant of up to \$9 million to develop a coronavirus
vaccine from the Coalition for Epidemic Preparedness Innovations, a
group whose aim is to speed vaccines to market. Moderna, which is
working with Dr. Graham's team at the N.I.H.,
\href{https://cepi.net/news_cepi/cepi-to-fund-three-programmes-to-develop-vaccines-against-the-novel-coronavirus-ncov-2019/}{received
a similar grant}, as did researchers at the University of Queensland in
Australia.

Historically, vaccines have been one of the greatest public health tools
to prevent disease. But even as new technology, advancements in genomics
and improved global coordination have allowed researchers to move at
unprecedented speed, vaccine development remains an expensive and risky
process. It takes months and even years because the vaccines must
undergo extensive testing in animals and humans. In the best case, it
takes at least a year --- and most likely longer --- for any vaccine to
become available to the public.

``They may not help in the very early stages of an outbreak, but if
we're able to develop vaccines in time, they will be an asset later,''
said Richard Hatchett, the chief executive of the epidemic preparedness
coalition.

\emph{{[}}\href{https://www.nytimes3xbfgragh.onion/interactive/2020/science/coronavirus-vaccine-tracker.html}{\emph{Follow
our Live Coronavirus Vaccine Tracker}}\emph{.{]}}

\includegraphics{https://static01.graylady3jvrrxbe.onion/images/2020/01/28/science/28VIRUS-VACCINE2/28VIRUS-VACCINE2-articleLarge.jpg?quality=75\&auto=webp\&disable=upscale}

With each new outbreak, scientists typically have to start from scratch.
After the SARS outbreak in 2003, it took researchers about 20 months
from the release of the viral genome to get a vaccine ready for human
trials. By the time an epidemic caused by the Zika virus occurred in
2015, researchers had brought the timeline down to six months. Now, they
hope the joint efforts will cut that time in half.

\hypertarget{latest-updates-the-coronavirus-outbreak}{%
\section{\texorpdfstring{\href{https://www.nytimes3xbfgragh.onion/2020/09/11/world/covid-19-coronavirus.html?action=click\&pgtype=Article\&state=default\&region=MAIN_CONTENT_1\&context=storylines_live_updates}{Latest
Updates: The Coronavirus
Outbreak}}{Latest Updates: The Coronavirus Outbreak}}\label{latest-updates-the-coronavirus-outbreak}}

Updated 2020-09-12T07:09:04.082Z

\begin{itemize}
\tightlist
\item
  \href{https://www.nytimes3xbfgragh.onion/2020/09/11/world/covid-19-coronavirus.html?action=click\&pgtype=Article\&state=default\&region=MAIN_CONTENT_1\&context=storylines_live_updates\#link-dfb8a16}{Fauci
  cautions the virus could disrupt life in the U.S. until `maybe even
  towards the end of 2021.'}
\item
  \href{https://www.nytimes3xbfgragh.onion/2020/09/11/world/covid-19-coronavirus.html?action=click\&pgtype=Article\&state=default\&region=MAIN_CONTENT_1\&context=storylines_live_updates\#link-7104d154}{From
  Asia to Africa, China promotes its vaccine candidates to win friends.}
\item
  \href{https://www.nytimes3xbfgragh.onion/2020/09/11/world/covid-19-coronavirus.html?action=click\&pgtype=Article\&state=default\&region=MAIN_CONTENT_1\&context=storylines_live_updates\#link-393ad215}{The
  other way the virus will kill: hunger.}
\end{itemize}

\href{https://www.nytimes3xbfgragh.onion/2020/09/11/world/covid-19-coronavirus.html?action=click\&pgtype=Article\&state=default\&region=MAIN_CONTENT_1\&context=storylines_live_updates}{See
more updates}

More live coverage:
\href{https://www.nytimes3xbfgragh.onion/live/2020/09/11/business/stock-market-today-coronavirus?action=click\&pgtype=Article\&state=default\&region=MAIN_CONTENT_1\&context=storylines_live_updates}{Markets}

The morning after the Chinese scientists published their data earlier
this month, Dr. Graham's team got to work checking the sequence and
comparing it with what they already had for SARS and MERS. They wanted
to focus on the spike protein, which forms the crown of the coronavirus
and recognizes receptors, or entry points, on a host cell.

``If you can block the spike protein from binding to a cell, then you've
effectively prevented an infection,'' said Kizzmekia Corbett, the
scientific lead for Dr. Graham's coronavirus team.

Dr. Corbett and others had studied the spike proteins on SARS and MERS
viruses in detail, using them to develop experimental vaccines. The
vaccines never made it to market because SARS was successfully contained
with public health measures before the vaccine was ready and
preliminary\href{https://www.thelancet.com/journals/laninf/article/PIIS1473-3099(19)30266-X/fulltext}{human
trials for the MERS vaccine showed success} last year.

But the scientists had a method for developing vaccines that could help
them fast-track production for the new coronavirus. They used the
template for the SARS vaccine and swapped in just enough genetic code
that would make it work for the new virus. ``I call it plug and play,''
Dr. Corbett said.

Within a few hours, Dr. Corbett was able to prepare the modified
sequence that the researchers needed. On Tuesday, Jan. 14, the team held
a conference call to discuss the next steps with collaborators in labs
across the country, and sent off the sequence to Moderna.

Scientists at the company plan to use the genetic information to create
synthetic messenger RNA, which carries instructions for cells'
protein-making machinery. The technology will help induce high levels of
antibodies that can identify the spike protein and fight off an
infection.

Once Moderna manufactures the messenger RNA in a few weeks, the N.I.H.
will run more tests, Dr. Corbett said. Collaborators in academic labs
will then test the vaccine in mice infected with the virus and check
blood samples from the animals to see how well the experimental vaccine
worked.

Image

Dr. Anthony Fauci, director of the National Institute of Allergy and
Infectious Diseases, during a coronavirus update by U.S. public health
officials on Tuesday.Credit...Amanda Voisard/Reuters

Dr. Anthony Fauci, director of the National Institute of Allergy and
Infectious Diseases at the N.I.H., who oversees Dr. Graham's team, said
he expected the vaccine research to move quickly.

``If we don't run into any unforeseen obstacles, we'll be able to get a
Phase 1 trial going within the next three months, which will be record
speed,'' he said, referring to early human trials that test for safety.

Other researchers are using different methods to develop their vaccines.

Inovio, which is also developing a vaccine for MERS, uses a DNA-based
technology. Johnson \& Johnson delivers vaccines through adenoviruses
--- which can cause coldlike symptoms but have been made harmless. And
researchers at the University of Queensland are testing particles that
mimic the structure of a virus.

``We don't know which vaccine approach will be successful at this stage,
so we have to try everything in our arsenal,'' said Dr. Gregory Poland,
a vaccine expert at the Mayo Clinic in Rochester, Minn.

In interviews, company executives said that partnerships with
governments and philanthropic foundations were essential to developing
vaccines for outbreaks because there
\href{https://www.nytimes3xbfgragh.onion/2014/10/24/health/without-lucrative-market-potential-ebola-vaccine-was-shelved-for-years.html}{are
so many uncertainties}.

Dr. Paul Stoffels, Johnson \& Johnson's chief scientific officer,
estimated it could take eight to 12 months before his company's vaccines
reach human clinical trials. By then, it is possible the coronavirus
outbreak will have been contained. Testing of Johnson \& Johnson's Zika
vaccine is currently halted, he said, because new outbreaks of the
disease have slowed.

``You have to be brave and you have to be a solid company to do this,
because there is no real incentive to do this, no financial incentive,''
he said.

Stéphane Bancel, the chief executive of Moderna, said vaccines were
necessary, even if an outbreak wanes, because it could always return.
``I think it's important to be prepared,'' he said.

Experts believe that the frequency of outbreaks will only increase
because of
\href{https://www.nytimes3xbfgragh.onion/interactive/2019/06/10/climate/dengue-mosquito-spread-map.html}{climate
change}, urbanization and global travel, among other factors.

``We probably need to start thinking about putting in a special
infrastructure for coronavirus infections the same way we have for the
flu,'' said Dr. Peter Hotez, who is co-director of the Texas Children's
Hospital Center for Vaccine Development and was involved in the
production of a SARS vaccine that may be repurposed for the new
coronavirus. The detection and monitoring of infections, as well as the
development of vaccines, will put an insurance policy in place for
future outbreaks, he said.

``We're just starting to realize that the power of vaccines goes way
beyond public health,'' he said. ``They are also critical to the global
economy and global security.''

\textbf{\emph{{[}}\href{http://on.fb.me/1paTQ1h}{\emph{Like the Science
Times page on Facebook.}}} ****** \emph{\textbar{} Sign up for the}
\textbf{\href{http://nyti.ms/1MbHaRU}{\emph{Science Times
newsletter.}}\emph{{]}}}

Advertisement

\protect\hyperlink{after-bottom}{Continue reading the main story}

\hypertarget{site-index}{%
\subsection{Site Index}\label{site-index}}

\hypertarget{site-information-navigation}{%
\subsection{Site Information
Navigation}\label{site-information-navigation}}

\begin{itemize}
\tightlist
\item
  \href{https://help.nytimes3xbfgragh.onion/hc/en-us/articles/115014792127-Copyright-notice}{©~2020~The
  New York Times Company}
\end{itemize}

\begin{itemize}
\tightlist
\item
  \href{https://www.nytco.com/}{NYTCo}
\item
  \href{https://help.nytimes3xbfgragh.onion/hc/en-us/articles/115015385887-Contact-Us}{Contact
  Us}
\item
  \href{https://www.nytco.com/careers/}{Work with us}
\item
  \href{https://nytmediakit.com/}{Advertise}
\item
  \href{http://www.tbrandstudio.com/}{T Brand Studio}
\item
  \href{https://www.nytimes3xbfgragh.onion/privacy/cookie-policy\#how-do-i-manage-trackers}{Your
  Ad Choices}
\item
  \href{https://www.nytimes3xbfgragh.onion/privacy}{Privacy}
\item
  \href{https://help.nytimes3xbfgragh.onion/hc/en-us/articles/115014893428-Terms-of-service}{Terms
  of Service}
\item
  \href{https://help.nytimes3xbfgragh.onion/hc/en-us/articles/115014893968-Terms-of-sale}{Terms
  of Sale}
\item
  \href{https://spiderbites.nytimes3xbfgragh.onion}{Site Map}
\item
  \href{https://help.nytimes3xbfgragh.onion/hc/en-us}{Help}
\item
  \href{https://www.nytimes3xbfgragh.onion/subscription?campaignId=37WXW}{Subscriptions}
\end{itemize}
