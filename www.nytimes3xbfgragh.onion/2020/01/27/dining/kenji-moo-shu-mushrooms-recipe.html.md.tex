Sections

SEARCH

\protect\hyperlink{site-content}{Skip to
content}\protect\hyperlink{site-index}{Skip to site index}

\href{https://www.nytimes3xbfgragh.onion/section/food}{Food}

\href{https://myaccount.nytimes3xbfgragh.onion/auth/login?response_type=cookie\&client_id=vi}{}

\href{https://www.nytimes3xbfgragh.onion/section/todayspaper}{Today's
Paper}

\href{/section/food}{Food}\textbar{}Moo Shu, Now With Less Meat

\url{https://nyti.ms/2Gt0kTz}

\begin{itemize}
\item
\item
\item
\item
\item
\item
\end{itemize}

Advertisement

\protect\hyperlink{after-top}{Continue reading the main story}

Supported by

\protect\hyperlink{after-sponsor}{Continue reading the main story}

\hypertarget{moo-shu-now-with-less-meat}{%
\section{Moo Shu, Now With Less Meat}\label{moo-shu-now-with-less-meat}}

Faced with a glut of mushrooms, J. Kenji López-Alt added them to a
classic Joyce Chen recipe. The result? Tasty, and alliterative.

\includegraphics{https://static01.graylady3jvrrxbe.onion/images/2020/01/29/dining/28Kenji1/merlin_167526669_1dfa8068-9dca-47db-8fcf-991b7e858aed-articleLarge.jpg?quality=75\&auto=webp\&disable=upscale}

By \href{https://www.nytimes3xbfgragh.onion/by/j-kenji-lopez-alt}{J.
Kenji López-Alt}

\begin{itemize}
\item
  Jan. 27, 2020
\item
  \begin{itemize}
  \item
  \item
  \item
  \item
  \item
  \item
  \end{itemize}
\end{itemize}

Moo shu pork is a Northern Chinese dish from Shandong, popularized in
the United States in the mid-20th century. My parents were introduced to
it at Joyce Chen's restaurant in Cambridge, Mass., where it had been
served since 1958, alongside now-ubiquitous dishes like Peking duck,
pan-fried dumplings (which she called ``Peking ravioli'' --- a term
you'll still find on Boston-area menus), double-cooked pork and won-ton
soup, among others.

The recipe for moo shu (or ``moo shi,'' as she calls it) pork in my
parents' copy of her now-long-out-of-print cookbook tracks closely with
the traditional Shandong version. Eggs are fried until puffy (the name
of the dish derives from muxi, the Chinese name for sweet osmanthus
flowers, which the puffy eggs are said to resemble), then lightly
marinated pork is stir-fried with wood ear mushrooms and day lily buds.

In Beijing, you may find the dish made with sliced cucumbers in place of
day lily buds, whereas in the United States, variants of the dish with
mung bean sprouts, carrots or green cabbage are not uncommon. The
stir-fry itself is relatively mild, flavored only with soy sauce and
wine, toasted sesame oil, a little white pepper and a hint of ginger. A
paper-thin Mandarin pancake, brushed with sweet hoisin sauce and folded
around the filling like a little burrito, gives it an adjustable punch
of flavor.

\includegraphics{https://static01.graylady3jvrrxbe.onion/images/2020/01/29/dining/28Kenji2/merlin_167526639_f846aef2-3460-497a-8b5d-a0e2c0048c5a-articleLarge.jpg?quality=75\&auto=webp\&disable=upscale}

I've been eating and making my own variant of the Chen recipe for years.
It's easy to make and, aside from the pork, uses ingredients I always
have on hand. (I, like my parents, keep a stash of dried wood ears and
day lilies in my pantry at all times. They last for years in a cool,
dark pantry and are also a staple ingredient in hot and sour soup,
another Chen dish I keep in regular rotation.) Over the years, aside
from modifying a few ingredients, I've made a couple changes to her
process. First, I've found that vigorously washing and draining sliced
meat in water, followed by very roughly stirring it with marinade
ingredients can significantly improve tenderness and marinade
absorption. (This is true for any stir-fry!)

The other central difference is that I break down the stir-fry into
several small batches, cooking one ingredient at a time, removing it to
a bowl and allowing the wok to reheat before adding the next ingredient.
This ensures that the wok maintains enough heat to properly sear and
fry, as opposed to overloading it, which can lead to steamed meats and
watery vegetables. This is a particularly useful technique if you, like
me, are cursed with weak burners. I return all of the food to the wok
for one final toss with the sauce just before serving.

Recently I found myself with a glut of mushrooms (which sometimes
happens when I'm left unsupervised at the farmers' market) and thought
to myself that in addition to the fantastic alliteration, moo shu
mushrooms would make a tasty variant. So I cut back on the amount of
pork that a typical moo shu recipe calls for and replaced it with a big
pile of mixed mushrooms that I stir-fried until browned and slightly
crispy around the edges. (We're after moo shu mushrooms after all, not
mushy mushrooms.)

As for the pancakes, they are surprisingly simple, incredibly satisfying
and a little bit magical to make at home. There's a clever trick:
stacking two disks of dough with a thin layer of sesame oil in between
as you roll out and cook them on a hot griddle or dry skillet. While
still hot, the disks are peeled apart, giving you two soft, paper-thin
pancakes that smell faintly of sesame oil.

Or you can do what many strip-mall Chinese restaurants do: Serve the
stir-fry with warm flour tortillas. Even the best-quality flour tortilla
is going to be thicker than a Mandarin pancake, but it'll still work
just fine.

Recipes:
\textbf{\href{https://cooking.nytimes3xbfgragh.onion/recipes/1020817-moo-shu-mushrooms}{Moo
Shu Mushrooms}} \textbar{}
\textbf{\href{https://cooking.nytimes3xbfgragh.onion/recipes/1020819-mandarin-pancakes}{Mandarin
Pancakes}}

\emph{Follow} \href{https://twitter.com/nytfood}{\emph{NYT Food on
Twitter}} \emph{and}
\href{https://www.instagram.com/nytcooking/}{\emph{NYT Cooking on
Instagram}}\emph{,}
\href{https://www.facebookcorewwwi.onion/nytcooking/}{\emph{Facebook}}\emph{,}
\href{https://www.youtube.com/nytcooking}{\emph{YouTube}} \emph{and}
\href{https://www.pinterest.com/nytcooking/}{\emph{Pinterest}}\emph{.}
\href{https://www.nytimes3xbfgragh.onion/newsletters/cooking}{\emph{Get
regular updates from NYT Cooking, with recipe suggestions, cooking tips
and shopping advice}}\emph{.}

Advertisement

\protect\hyperlink{after-bottom}{Continue reading the main story}

\hypertarget{site-index}{%
\subsection{Site Index}\label{site-index}}

\hypertarget{site-information-navigation}{%
\subsection{Site Information
Navigation}\label{site-information-navigation}}

\begin{itemize}
\tightlist
\item
  \href{https://help.nytimes3xbfgragh.onion/hc/en-us/articles/115014792127-Copyright-notice}{©~2020~The
  New York Times Company}
\end{itemize}

\begin{itemize}
\tightlist
\item
  \href{https://www.nytco.com/}{NYTCo}
\item
  \href{https://help.nytimes3xbfgragh.onion/hc/en-us/articles/115015385887-Contact-Us}{Contact
  Us}
\item
  \href{https://www.nytco.com/careers/}{Work with us}
\item
  \href{https://nytmediakit.com/}{Advertise}
\item
  \href{http://www.tbrandstudio.com/}{T Brand Studio}
\item
  \href{https://www.nytimes3xbfgragh.onion/privacy/cookie-policy\#how-do-i-manage-trackers}{Your
  Ad Choices}
\item
  \href{https://www.nytimes3xbfgragh.onion/privacy}{Privacy}
\item
  \href{https://help.nytimes3xbfgragh.onion/hc/en-us/articles/115014893428-Terms-of-service}{Terms
  of Service}
\item
  \href{https://help.nytimes3xbfgragh.onion/hc/en-us/articles/115014893968-Terms-of-sale}{Terms
  of Sale}
\item
  \href{https://spiderbites.nytimes3xbfgragh.onion}{Site Map}
\item
  \href{https://help.nytimes3xbfgragh.onion/hc/en-us}{Help}
\item
  \href{https://www.nytimes3xbfgragh.onion/subscription?campaignId=37WXW}{Subscriptions}
\end{itemize}
