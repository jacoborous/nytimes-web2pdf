Sections

SEARCH

\protect\hyperlink{site-content}{Skip to
content}\protect\hyperlink{site-index}{Skip to site index}

\href{https://www.nytimes3xbfgragh.onion/section/health}{Health}

\href{https://myaccount.nytimes3xbfgragh.onion/auth/login?response_type=cookie\&client_id=vi}{}

\href{https://www.nytimes3xbfgragh.onion/section/todayspaper}{Today's
Paper}

\href{/section/health}{Health}\textbar{}China Identifies New Virus
Causing Pneumonialike Illness

\url{https://nyti.ms/301Kecs}

\begin{itemize}
\item
\item
\item
\item
\item
\end{itemize}

\hypertarget{the-coronavirus-outbreak}{%
\subsubsection{\texorpdfstring{\href{https://www.nytimes3xbfgragh.onion/news-event/coronavirus?name=styln-coronavirus-national\&region=TOP_BANNER\&block=storyline_menu_recirc\&action=click\&pgtype=Article\&impression_id=e61df750-f2d9-11ea-b16d-01616827df08\&variant=undefined}{The
Coronavirus
Outbreak}}{The Coronavirus Outbreak}}\label{the-coronavirus-outbreak}}

\begin{itemize}
\tightlist
\item
  live\href{https://www.nytimes3xbfgragh.onion/2020/09/09/world/covid-coronavirus.html?name=styln-coronavirus-national\&region=TOP_BANNER\&block=storyline_menu_recirc\&action=click\&pgtype=Article\&impression_id=e61df751-f2d9-11ea-b16d-01616827df08\&variant=undefined}{Latest
  Updates}
\item
  \href{https://www.nytimes3xbfgragh.onion/interactive/2020/us/coronavirus-us-cases.html?name=styln-coronavirus-national\&region=TOP_BANNER\&block=storyline_menu_recirc\&action=click\&pgtype=Article\&impression_id=e61df752-f2d9-11ea-b16d-01616827df08\&variant=undefined}{Maps
  and Cases}
\item
  \href{https://www.nytimes3xbfgragh.onion/interactive/2020/science/coronavirus-vaccine-tracker.html?name=styln-coronavirus-national\&region=TOP_BANNER\&block=storyline_menu_recirc\&action=click\&pgtype=Article\&impression_id=e61df753-f2d9-11ea-b16d-01616827df08\&variant=undefined}{Vaccine
  Tracker}
\item
  \href{https://www.nytimes3xbfgragh.onion/2020/09/02/your-money/eviction-moratorium-covid.html?name=styln-coronavirus-national\&region=TOP_BANNER\&block=storyline_menu_recirc\&action=click\&pgtype=Article\&impression_id=e61df754-f2d9-11ea-b16d-01616827df08\&variant=undefined}{Eviction
  Moratorium}
\item
  \href{https://www.nytimes3xbfgragh.onion/2020/09/09/upshot/coronavirus-surprise-test-fees.html?name=styln-coronavirus-national\&region=TOP_BANNER\&block=storyline_menu_recirc\&action=click\&pgtype=Article\&impression_id=e61df755-f2d9-11ea-b16d-01616827df08\&variant=undefined}{Surprise
  Test Fees}
\end{itemize}

Advertisement

\protect\hyperlink{after-top}{Continue reading the main story}

Supported by

\protect\hyperlink{after-sponsor}{Continue reading the main story}

Global health

\hypertarget{china-identifies-new-virus-causing-pneumonialike-illness}{%
\section{China Identifies New Virus Causing Pneumonialike
Illness}\label{china-identifies-new-virus-causing-pneumonialike-illness}}

The new coronavirus doesn't appear to be readily spread by humans, but
researchers caution that more study is needed.

\includegraphics{https://static01.graylady3jvrrxbe.onion/images/2020/01/09/science/09CHINA-VIRUS1/09CHINA-VIRUS1-articleLarge.jpg?quality=75\&auto=webp\&disable=upscale}

\href{https://www.nytimes3xbfgragh.onion/by/sui-lee-wee}{\includegraphics{https://static01.graylady3jvrrxbe.onion/images/2018/07/13/multimedia/author-sui-lee-wee/author-sui-lee-wee-thumbLarge.png}}\href{https://www.nytimes3xbfgragh.onion/by/donald-g-mcneil-jr}{\includegraphics{https://static01.graylady3jvrrxbe.onion/images/2018/06/13/multimedia/author-donald-g-mcneil-jr/author-donald-g-mcneil-jr-thumbLarge-v4.png}}

By \href{https://www.nytimes3xbfgragh.onion/by/sui-lee-wee}{Sui-Lee Wee}
and
\href{https://www.nytimes3xbfgragh.onion/by/donald-g-mcneil-jr}{Donald
G. McNeil Jr.}

\begin{itemize}
\item
  Published Jan. 8, 2020Updated Jan. 21, 2020
\item
  \begin{itemize}
  \item
  \item
  \item
  \item
  \item
  \end{itemize}
\end{itemize}

\href{https://cn.nytimes3xbfgragh.onion/china/20200109/china-pneumonia-outbreak-virus/}{阅读简体中文版}\href{https://cn.nytimes3xbfgragh.onion/china/20200109/china-pneumonia-outbreak-virus/zh-hn}{閱讀繁體中文版}

HONG KONG --- Chinese researchers say they have identified a new virus
behind an illness that has infected dozens of people across Asia,
setting off fears in a region that was struck by a deadly epidemic 17
years ago.

There is no evidence that the new virus is readily spread by humans,
which would make it particularly dangerous, and it has not been tied to
any deaths. But health officials in China and elsewhere are watching it
carefully to ensure that the outbreak does not develop into something
more severe.

Researchers in China have ``initially identified'' the new virus, a
coronavirus, as the pathogen behind a
\href{https://www.nytimes3xbfgragh.onion/2020/01/06/world/asia/china-SARS-pneumonialike.html}{mysterious,
pneumonialike illness that has sickened 59 people in the city of Wuhan}
and caused a panic in the central Chinese region, the state broadcaster,
China Central Television, said on Thursday. They detected this virus in
15 of the people who fell ill, the report said.

The new coronavirus ``is different from previous human coronaviruses
that were previously discovered, and more scientific research is needed
for further understanding,'' the report said.

Coronaviruses are a large family of viruses that infect animals and
people. Some cause only the symptoms known as the common cold, although
many other viruses also do that.

The announcement signals that researchers are making progress in
containing the outbreak, but Asian officials are not likely to relax
their vigilance until they learn more. The disease has evoked memories
of
\href{https://www.nytimes3xbfgragh.onion/2003/04/27/world/the-sars-epidemic-the-path-from-china-s-provinces-a-crafty-germ-breaks-out.html}{the
outbreak of SARS, or severe acute respiratory syndrome}, in 2003. SARS,
a respiratory disease and also a coronavirus, spread from southern China
in 2003 and infected more than 3,000 people, killing 774.

\hypertarget{latest-updates-the-coronavirus-outbreak}{%
\section{\texorpdfstring{\href{https://www.nytimes3xbfgragh.onion/2020/09/09/world/covid-coronavirus.html?action=click\&pgtype=Article\&state=default\&region=MAIN_CONTENT_1\&context=storylines_live_updates}{Latest
Updates: The Coronavirus
Outbreak}}{Latest Updates: The Coronavirus Outbreak}}\label{latest-updates-the-coronavirus-outbreak}}

Updated 2020-09-09T20:18:29.428Z

\begin{itemize}
\tightlist
\item
  \href{https://www.nytimes3xbfgragh.onion/2020/09/09/world/covid-coronavirus.html?action=click\&pgtype=Article\&state=default\&region=MAIN_CONTENT_1\&context=storylines_live_updates\#link-279e24e2}{Top
  U.S. health officials update Congress on vaccine development and
  distribution plans.}
\item
  \href{https://www.nytimes3xbfgragh.onion/2020/09/09/world/covid-coronavirus.html?action=click\&pgtype=Article\&state=default\&region=MAIN_CONTENT_1\&context=storylines_live_updates\#link-38cb0bfc}{The
  AstraZeneca vaccine trial was halted after a person enrolled in it
  developed a rare inflammatory condition.}
\item
  \href{https://www.nytimes3xbfgragh.onion/2020/09/09/world/covid-coronavirus.html?action=click\&pgtype=Article\&state=default\&region=MAIN_CONTENT_1\&context=storylines_live_updates\#link-792ae257}{Indoor
  dining in N.Y.C. will return with limits on Sept. 30, Cuomo says.}
\end{itemize}

\href{https://www.nytimes3xbfgragh.onion/2020/09/09/world/covid-coronavirus.html?action=click\&pgtype=Article\&state=default\&region=MAIN_CONTENT_1\&context=storylines_live_updates}{See
more updates}

More live coverage:
\href{https://www.nytimes3xbfgragh.onion/live/2020/09/09/business/stock-market-today-coronavirus?action=click\&pgtype=Article\&state=default\&region=MAIN_CONTENT_1\&context=storylines_live_updates}{Markets}

\emph{{[}Read:}
\href{https://www.nytimes3xbfgragh.onion/2020/01/21/world/asia/china-coronavirus-australia-wuhan-travel.html}{\emph{China's
coronavirus outbreak tests the transparency of China's Communist
Party}}\emph{.{]}}

China initially covered up the extent of the SARS outbreak and was
criticized by health officials around the world for doing so. On
Thursday, the World Health Organization praised the Chinese response to
the new outbreak and said it did not recommend any restrictions on trade
or travel to China because of the virus.

``Preliminary identification of a novel virus in a short period of time
is a notable achievement and demonstrates China's increased capacity to
manage new outbreaks,'' Dr. Gauden Galea, the W.H.O.'s representative to
China, said in a statement.

Many questions about the new virus remain. While it appears to be
transmitted to humans via animals, the Chinese government has not said
which animals, nor has it disclosed other details about the outbreak,
like the transmission route, the incubation period or the ages and
genders of the patients.

``So, there are still a lot of question marks,'' said David Hui, an
expert in emerging infections at the Chinese University of Hong Kong.

The authorities in Wuhan are still closely monitoring 163 people who
were in close contact with the patients, Dr. Hui said. He added that 15
days, the minimum incubation period for some viral infections, had not
yet passed since the last reported instance of the disease, on Dec. 29.

``It's premature to say that there's no human-to-human transmission,''
Dr. Hui said.

The Wuhan government
\href{http://wjw.wuhan.gov.cn/front/web/showDetail/2019123108989}{confirmed
on Dec. 31} that the health authorities were treating dozens of cases of
pneumonia of unknown cause. Symptoms of the new illness include high
fever, difficulty breathing and lung lesions, according to the Wuhan
health commission. Seven people have become critically ill, the Wuhan
authorities have said. On Wednesday, the local health commission said
eight people had been discharged.

Researchers have been encouraged by the fact that patients' relatives
and hospital workers have not been reported to have gotten sick,
signaling that the virus may not spread easily among humans.

``We can assume that this virus transmissibility is not that high,''
said Guan Yi, a professor of infectious diseases at the University of
Hong Kong, who was part of a team that successfully identified the
coronavirus that caused SARS.

Dr. Guan said he was not surprised that the virus identified was a
coronavirus, because coronaviruses can pass from animals to humans
easily. But he said it would probably be a while before researchers came
up with treatments for the illness.

So far, the cases in China have circulated only in Wuhan. The initial
cases were linked to workers at a market that sold live fish, animals
and birds. Workers disinfected and shut down the market after the city
health department said many of the cases had been traced to it.

The new illness appeared just weeks before the Spring Festival, China's
biggest holiday, when hundreds of millions of people travel. The
authorities have urged the public to be on alert for pneumonialike
symptoms like fevers, body aches and breathing difficulties.

Until Thursday's announcement, it was not clear what was causing the
illnesses in Wuhan. The World Health Organization said Wednesday that it
had concluded that it was most likely a coronavirus. ``More
comprehensive information is required to confirm the pathogen,'' the
W.H.O. said in a statement.

Early reports on ProMED, a disease-alert service, said there was no
evidence of human-to-human transmission.

Last weekend, laboratory tests in China ruled out SARS; the deadly
Middle East Respiratory Syndrome, or MERS; the flu; bird flu;
adenoviruses; and other common pathogens that cause pneumonia.

Health officials in Asia have stepped up screenings and isolated
patients with flulike symptoms who had traveled to Wuhan. **** In Hong
Kong, eight people with fever and respiratory symptoms who had recently
been in Wuhan were hospitalized on Wednesday.

In South Korea, the authorities said on Wednesday that they had
\href{https://www.scmp.com/news/asia/east-asia/article/3045249/chinese-worker-who-visited-wuhan-quarantined-south-korea}{put
a Chinese woman under isolated treatment} after she was found to have
pneumonia after trips to China, including Wuhan.

In Singapore, the authorities placed a Chinese girl with pneumonia in
isolation because she had traveled to Wuhan. On Sunday, they said
doctors had determined that the child
\href{https://www.moh.gov.sg/news-highlights/details/update-on-local-situation-regarding-severe-pneumonia-cluster-in-wuhan-5-jan}{had
a common childhood viral illness}.

Officials in Hong Kong have installed additional thermal imaging systems
at its airport to monitor passengers coming from Wuhan, scanning for
people with fevers.

SARS is believed to have jumped to humans from live-animal markets. It
was eventually traced to civet cats, raccoon dogs and some other species
that were raised and slaughtered for the exotic food trade. The virus
normally circulates in bats, and the animals may have gotten it from
them, possibly by eating food contaminated by bat droppings.

Most outbreaks of MERS, which appeared in 2012, have been traced to
people who raise or sell camels, which in the Middle East are kept for
meat, milk, racing, hauling cargo and as pets. Like SARS, MERS can jump
from person to person, particularly in hospitals. Some MERS patients
infected many others after they were put on machines to help them
breathe --- the mechanisms helped spew viral particles into the air as
they exhaled.

Despite Thursday's announcement, many Chinese were still expressing fear
that there could be a repeat of SARS.

``In fact, what people are more concerned about is the transmission
route and whether it can be cured,'' wrote a user on Weibo, a popular
social media tool in China.

\textbf{\emph{{[}}\href{http://on.fb.me/1paTQ1h}{\emph{Like the Science
Times page on Facebook.}}} ****** \emph{\textbar{} Sign up for the}
\textbf{\href{http://nyti.ms/1MbHaRU}{\emph{Science Times
newsletter.}}\emph{{]}}}

Sui-Lee Wee reported from Hong Kong and Donald G. McNeil Jr. from New
York. Elsie Chen contributed research from Beijing.

Advertisement

\protect\hyperlink{after-bottom}{Continue reading the main story}

\hypertarget{site-index}{%
\subsection{Site Index}\label{site-index}}

\hypertarget{site-information-navigation}{%
\subsection{Site Information
Navigation}\label{site-information-navigation}}

\begin{itemize}
\tightlist
\item
  \href{https://help.nytimes3xbfgragh.onion/hc/en-us/articles/115014792127-Copyright-notice}{©~2020~The
  New York Times Company}
\end{itemize}

\begin{itemize}
\tightlist
\item
  \href{https://www.nytco.com/}{NYTCo}
\item
  \href{https://help.nytimes3xbfgragh.onion/hc/en-us/articles/115015385887-Contact-Us}{Contact
  Us}
\item
  \href{https://www.nytco.com/careers/}{Work with us}
\item
  \href{https://nytmediakit.com/}{Advertise}
\item
  \href{http://www.tbrandstudio.com/}{T Brand Studio}
\item
  \href{https://www.nytimes3xbfgragh.onion/privacy/cookie-policy\#how-do-i-manage-trackers}{Your
  Ad Choices}
\item
  \href{https://www.nytimes3xbfgragh.onion/privacy}{Privacy}
\item
  \href{https://help.nytimes3xbfgragh.onion/hc/en-us/articles/115014893428-Terms-of-service}{Terms
  of Service}
\item
  \href{https://help.nytimes3xbfgragh.onion/hc/en-us/articles/115014893968-Terms-of-sale}{Terms
  of Sale}
\item
  \href{https://spiderbites.nytimes3xbfgragh.onion}{Site Map}
\item
  \href{https://help.nytimes3xbfgragh.onion/hc/en-us}{Help}
\item
  \href{https://www.nytimes3xbfgragh.onion/subscription?campaignId=37WXW}{Subscriptions}
\end{itemize}
