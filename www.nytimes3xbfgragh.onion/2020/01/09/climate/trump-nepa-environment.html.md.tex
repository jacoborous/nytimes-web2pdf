Sections

SEARCH

\protect\hyperlink{site-content}{Skip to
content}\protect\hyperlink{site-index}{Skip to site index}

\href{https://www.nytimes3xbfgragh.onion/section/climate}{Climate}

\href{https://myaccount.nytimes3xbfgragh.onion/auth/login?response_type=cookie\&client_id=vi}{}

\href{https://www.nytimes3xbfgragh.onion/section/todayspaper}{Today's
Paper}

\href{/section/climate}{Climate}\textbar{}Trump's Move Against Landmark
Environmental Law Caps a Relentless Agenda

\url{https://nyti.ms/2tMhYOV}

\begin{itemize}
\item
\item
\item
\item
\item
\item
\end{itemize}

\hypertarget{climate-and-environment}{%
\subsubsection{\texorpdfstring{\href{https://www.nytimes3xbfgragh.onion/section/climate?name=styln-climate\&region=TOP_BANNER\&block=storyline_menu_recirc\&action=click\&pgtype=Article\&impression_id=b0e51df0-f2d6-11ea-a053-9b22bc94b6cf\&variant=undefined}{Climate
and
Environment}}{Climate and Environment}}\label{climate-and-environment}}

\begin{itemize}
\tightlist
\item
  \href{https://www.nytimes3xbfgragh.onion/interactive/2020/08/24/climate/racism-redlining-cities-global-warming.html?name=styln-climate\&region=TOP_BANNER\&block=storyline_menu_recirc\&action=click\&pgtype=Article\&impression_id=b0e51df1-f2d6-11ea-a053-9b22bc94b6cf\&variant=undefined}{Environmental
  Racism}
\item
  \href{https://www.nytimes3xbfgragh.onion/interactive/2020/climate/trump-environment-rollbacks.html?name=styln-climate\&region=TOP_BANNER\&block=storyline_menu_recirc\&action=click\&pgtype=Article\&impression_id=b0e51df2-f2d6-11ea-a053-9b22bc94b6cf\&variant=undefined}{Trump's
  Changes}
\item
  \href{https://www.nytimes3xbfgragh.onion/interactive/2020/04/19/climate/climate-crash-course-1.html?name=styln-climate\&region=TOP_BANNER\&block=storyline_menu_recirc\&action=click\&pgtype=Article\&impression_id=b0e51df3-f2d6-11ea-a053-9b22bc94b6cf\&variant=undefined}{Climate
  101}
\item
  \href{https://www.nytimes3xbfgragh.onion/interactive/2018/08/30/climate/how-much-hotter-is-your-hometown.html?name=styln-climate\&region=TOP_BANNER\&block=storyline_menu_recirc\&action=click\&pgtype=Article\&impression_id=b0e51df4-f2d6-11ea-a053-9b22bc94b6cf\&variant=undefined}{Is
  Your Hometown Hotter?}
\end{itemize}

Advertisement

\protect\hyperlink{after-top}{Continue reading the main story}

Supported by

\protect\hyperlink{after-sponsor}{Continue reading the main story}

News Analysis

\hypertarget{trumps-move-against-landmark-environmental-law-caps-a-relentless-agenda}{%
\section{Trump's Move Against Landmark Environmental Law Caps a
Relentless
Agenda}\label{trumps-move-against-landmark-environmental-law-caps-a-relentless-agenda}}

\includegraphics{https://static01.graylady3jvrrxbe.onion/images/2020/01/09/climate/09CLI-NEPA/merlin_162408951_7fec6d83-25c7-4c57-a348-43945e594808-articleLarge.jpg?quality=75\&auto=webp\&disable=upscale}

\href{https://www.nytimes3xbfgragh.onion/by/lisa-friedman}{\includegraphics{https://static01.graylady3jvrrxbe.onion/images/2018/07/18/multimedia/author-lisa-friedman/author-lisa-friedman-thumbLarge.png}}

By \href{https://www.nytimes3xbfgragh.onion/by/lisa-friedman}{Lisa
Friedman}

\begin{itemize}
\item
  Published Jan. 9, 2020Updated Jan. 13, 2020
\item
  \begin{itemize}
  \item
  \item
  \item
  \item
  \item
  \item
  \end{itemize}
\end{itemize}

WASHINGTON --- President Trump on Thursday capped a three-year drive to
roll back clean air and water protections by proposing stark changes to
the nation's oldest and most established environmental law that could
exempt major infrastructure projects from environmental review.

The revisions to the law ---
\href{https://www.nytimes3xbfgragh.onion/2020/01/03/climate/trump-nepa-climate-change.html}{the
50-year-old National Environmental Policy Act}, a landmark measure that
touches nearly every highway, bridge, pipeline and other major federal
construction in the country --- underscored Mr. Trump's focus on
stripping away regulations, to the consternation of conservationists. In
the middle of a foreign-policy crisis and on the cusp of an impeachment
trial in the Senate, Mr. Trump appeared in his element on Thursday,
flanked by men in hard hats and orange safety vests.

``America's most critical infrastructure projects have been tied up and
bogged down by an outrageously slow and burdensome federal approval
process, and I've been talking about it for a long time,'' he said.

Mr. Trump, who made his fortune as a real estate developer, spoke as if
personally aggrieved: ``The builders are not happy. Nobody's happy.''

Since taking office Mr. Trump has proposed
\href{https://www.nytimes3xbfgragh.onion/interactive/2019/climate/trump-environment-rollbacks.html}{nearly
100 environmental rollbacks}, including
\href{https://www.nytimes3xbfgragh.onion/2019/08/12/climate/endangered-species-act-changes.html}{weakening
protections for endangered species}, relaxing rules that
\href{https://www.nytimes3xbfgragh.onion/2019/06/19/climate/epa-coal-emissions.html}{limit
emissions from coal plants} and blocking the phaseout of
\href{https://www.nytimes3xbfgragh.onion/2019/09/04/climate/trump-light-bulb-rollback.html}{older
incandescent light bulbs}. Hundreds of thousands of public comments
against the president's moves have flowed in. Scientists have spoken out
in opposition. Democrats have vowed to stop him, all with little effect.

``He sees himself as the kingpin of an anti-federal-regulatory
movement,'' said Douglas Brinkley, a presidential historian at Rice
University who has written about environmental policy.

But
\href{https://www.nytimes3xbfgragh.onion/2018/04/07/climate/scott-pruitt-epa-rollbacks.html}{haste
and zeal may work against the administration}. Nearly 70 lawsuits have
been filed to challenge the administration's deregulatory moves,
asserting that officials have violated federal procedures in their
rollback efforts.
\href{https://policyintegrity.org/trump-court-roundup}{The Trump
administration has, so far, been successful just four times}, according
to New York University School of Law data.

Some of Mr. Trump's moves have been never been tried before, such as the
\href{https://www.nytimes3xbfgragh.onion/2017/12/04/us/trump-bears-ears.html}{reversal
of national monument designations} by his predecessors. Some have been
remarkably defiant, like Thursday's effort to alter a half-century-old
law by decree, carving out a new category of infrastructure projects not
subject to environmental review.

The interior secretary, David Bernhardt, who has overseen plans to
weaken limits on the release of methane, a potent greenhouse gas, and
loosen offshore drilling safety rules, called the proposed changes to
the National Environmental Policy Act the Trump administration's most
significant deregulatory proposal yet.

\href{\%3Ca\%20href=\%22https://www.nytimes3xbfgragh.onion/section/climate?action=click\&pgtype=Article\&state=default\&region=MAIN_CONTENT_1\&context=storylines_keepup\%22\%3Ehttps://www.nytimes3xbfgragh.onion/section/climate?action=click\&pgtype=Article\&state=default\&region=MAIN_CONTENT_1\&context=storylines_keepup\%3C/a\%3E}{}

\hypertarget{climate-and-environment-}{%
\subsubsection{Climate and Environment
›}\label{climate-and-environment-}}

\hypertarget{keep-up-on-the-latest-climate-news}{%
\paragraph{Keep Up on the Latest Climate
News}\label{keep-up-on-the-latest-climate-news}}

Updated Sept. 6, 2020

Here's what you need to know this week:

\begin{itemize}
\item
  \begin{itemize}
  \tightlist
  \item
    Americans back
    \href{https://www.nytimes3xbfgragh.onion/2020/09/04/climate/flood-fire-building-restrictions.html?action=click\&pgtype=Article\&state=default\&region=MAIN_CONTENT_1\&context=storylines_keepup}{tough
    limits on building in fire and flood zones}, new research shows.
  \item
    California's wildfires are driving another crisis: More and more
    \href{https://www.nytimes3xbfgragh.onion/2020/09/02/climate/wildfires-insurance.html?action=click\&pgtype=Article\&state=default\&region=MAIN_CONTENT_1\&context=storylines_keepup}{homeowners
    can't get insurance}.
  \item
    The Trump administration has relaxed Obama-era rules limiting the
    release of
    \href{https://www.nytimes3xbfgragh.onion/2020/08/31/climate/trump-coal-plants.html?action=click\&pgtype=Article\&state=default\&region=MAIN_CONTENT_1\&context=storylines_keepup}{toxic
    waste from coal plants}.
  \end{itemize}
\end{itemize}

Critics agreed. James A. Thurber, a political-science professor at
American University, described Mr. Trump's latest actions to ``altering
the Ten Commandments of environmental policy.''

All told, Mr. Trump has gone further than any other president, including
Ronald Reagan, in dismantling clean air and water protections. The
National Environmental Protection Act was signed into law by Richard M.
Nixon after calls for greater oversight when the heavily polluted
Cuyahoga River in Ohio caught fire and a tanker spilled three million
gallons of crude off the coast of Santa Barbara, Calif., in 1969.

``No other president has had the gall to try to back polluters and turn
back the clock to pre-Santa Barbara,'' Mr. Brinkley said. ``Nothing
compares to what Donald Trump is doing.''

But Mr. Trump's moves also have won wide praise --- not just from the
oil and gas industry but also from labor unions that Mr. Trump is eager
to win over in November. In 2016, union members, who had traditionally
voted for Democrats, helped Mr. Trump win the White House.

Under the National Environmental Policy Act, major federal projects like
bridges, highways, pipelines or power plants that will have a
significant impact on the environment require a review, or environmental
impact statement, outlining potential consequences.

The proposed new rules would change the regulations that guide the
implementation of the law in a number of ways, including by narrowing
the range of projects that require such an assessment and by imposing
strict new deadlines on completing the studies.

The changes would also eliminate the need for agencies to consider the
``cumulative impacts'' of projects. In recent years, courts have said
that includes studying the planet-warming consequences of emitting more
greenhouse gases. Mary B. Neumayr, the chairwoman of the White House
Council on Environmental Quality, said the change did not prevent or
exclude consideration of the impact of greenhouse gases; consideration
would no longer be required.

And the changes would set hard deadlines of one year to complete reviews
of smaller projects and two years to complete reviews of larger ones.

``Today it can take more than 10 years to build just a very simple
road,'' Mr. Trump said. ``And, usually, you're not able to even get the
permit.''

Mr. Bernhardt said he had seen environmental studies that prevented the
timely construction of schools on tribal lands and visitor centers at
national parks, and hindered the ability of farmers to secure water
supplies.

Nancy Pelosi, the House speaker, saw it differently. ``This means more
polluters will be right there next to the water supply of our
children,'' she said. ``That's a public health issue.''

The changes were expected to appear in the federal register on Friday.
There will be a 60-day window for public comment and two public hearings
before a final regulation is issued, most likely in the fall.

Richard L. Revesz, a professor of environmental law at New York
University, said he did not believe the changes would hold up in court.
The National Environmental Policy Act requires that all the
environmental consequences of a project be taken into account, he said,
and that core requirement cannot be changed by fiat.

``A regulation can't change the requirements of a statute as interpreted
by the courts,'' Mr. Revesz said. In fact, he argued, under the Trump
administration's guidance, federal agencies are more likely to be sued
for inadequate reviews, ``leading to far longer delays than if they had
done a proper analysis in the first place.''

The proposed regulation does not set a dollar threshold for what
constitutes a large federal footprint, a factor that one official said
could also allow major mining, drilling and other projects to avoid
environmental assessments.

Representative Rob Bishop of Utah, the ranking Republican on the House
Natural Resources Committee, said he believed the changes would bring
``rationality'' to federal bureaucracy.

``There has been nothing more detrimental to the development of
transportation, clean water, and energy infrastructure than America's
broken environmental review and permitting process,'' he said.

Environmental groups said the revisions to the act would threaten
species and lead to more greenhouse gases in the atmosphere. The
proposal does not mention the words ``climate change,'' but courts have
interpreted the requirement to consider ``cumulative consequences'' as a
mandate to study the effects of allowing more planet-warming greenhouse
gas emissions into the atmosphere. It also has meant understanding the
impacts of rising sea levels and other results of climate change on a
given project.

That means agencies will not have to examine whether a pipeline, mine or
other fossil fuel project would worsen climate change. It also means
there will not be any requirement to understand how or whether a road or
bridge in a coastal area would be threatened by sea-level rise.

William K. Reilly, the administrator of the Environmental Protection
Agency under President George Bush, said of the changes, ``This one hits
home for me.'' He wrote the first regulations for environmental impact
statements as a White House aide in 1970.

The National Environmental Policy Act, he said, has been ``very
important'' in preventing environmental harm on major infrastructure
projects. He particularly took issue with Mr. Trump's remarks Thursday
that other countries ``look at the United States and they can't
believe'' the restrictions on development.

``It has been a model,'' Mr. Reilly said of the law. ``It's one of those
things that other countries around the world have copied and admired.''

\emph{For more climate news sign up for}
\href{https://www.nytimes3xbfgragh.onion/newsletters/climate-change}{\emph{the
Climate Fwd: newsletter}} \emph{or follow}
\href{https://twitter.com/nytclimate}{\emph{@NYTClimate on
Twitter}}\emph{.}

Advertisement

\protect\hyperlink{after-bottom}{Continue reading the main story}

\hypertarget{site-index}{%
\subsection{Site Index}\label{site-index}}

\hypertarget{site-information-navigation}{%
\subsection{Site Information
Navigation}\label{site-information-navigation}}

\begin{itemize}
\tightlist
\item
  \href{https://help.nytimes3xbfgragh.onion/hc/en-us/articles/115014792127-Copyright-notice}{©~2020~The
  New York Times Company}
\end{itemize}

\begin{itemize}
\tightlist
\item
  \href{https://www.nytco.com/}{NYTCo}
\item
  \href{https://help.nytimes3xbfgragh.onion/hc/en-us/articles/115015385887-Contact-Us}{Contact
  Us}
\item
  \href{https://www.nytco.com/careers/}{Work with us}
\item
  \href{https://nytmediakit.com/}{Advertise}
\item
  \href{http://www.tbrandstudio.com/}{T Brand Studio}
\item
  \href{https://www.nytimes3xbfgragh.onion/privacy/cookie-policy\#how-do-i-manage-trackers}{Your
  Ad Choices}
\item
  \href{https://www.nytimes3xbfgragh.onion/privacy}{Privacy}
\item
  \href{https://help.nytimes3xbfgragh.onion/hc/en-us/articles/115014893428-Terms-of-service}{Terms
  of Service}
\item
  \href{https://help.nytimes3xbfgragh.onion/hc/en-us/articles/115014893968-Terms-of-sale}{Terms
  of Sale}
\item
  \href{https://spiderbites.nytimes3xbfgragh.onion}{Site Map}
\item
  \href{https://help.nytimes3xbfgragh.onion/hc/en-us}{Help}
\item
  \href{https://www.nytimes3xbfgragh.onion/subscription?campaignId=37WXW}{Subscriptions}
\end{itemize}
