Sections

SEARCH

\protect\hyperlink{site-content}{Skip to
content}\protect\hyperlink{site-index}{Skip to site index}

\href{https://www.nytimes3xbfgragh.onion/section/opinion/sunday}{Sunday
Review}

\href{https://myaccount.nytimes3xbfgragh.onion/auth/login?response_type=cookie\&client_id=vi}{}

\href{https://www.nytimes3xbfgragh.onion/section/todayspaper}{Today's
Paper}

\href{/section/opinion/sunday}{Sunday Review}\textbar{}There's a Strong
Case for Sticking With Bail Reform

\url{https://nyti.ms/2RpJaN9}

\begin{itemize}
\item
\item
\item
\item
\item
\item
\end{itemize}

Advertisement

\protect\hyperlink{after-top}{Continue reading the main story}

\href{/section/opinion}{Opinion}

Supported by

\protect\hyperlink{after-sponsor}{Continue reading the main story}

\hypertarget{theres-a-strong-case-for-sticking-with-bail-reform}{%
\section{There's a Strong Case for Sticking With Bail
Reform}\label{theres-a-strong-case-for-sticking-with-bail-reform}}

The New York law has been on the books for less than a month and already
the critics are on the attack.

By \href{https://www.nytimes3xbfgragh.onion/by/emily-bazelon}{Emily
Bazelon} and Insha Rahman

Ms. Bazelon is a staff writer for The New York Times Magazine. Ms.
Rahman is the director of strategy and new initiatives at the Vera
Institute of Justice.

\begin{itemize}
\item
  Jan. 24, 2020
\item
  \begin{itemize}
  \item
  \item
  \item
  \item
  \item
  \item
  \end{itemize}
\end{itemize}

\includegraphics{https://static01.graylady3jvrrxbe.onion/images/2020/01/26/opinion/24bazelon/merlin_162395442_d5de969e-82d3-40f3-9a83-c092bd7ae7b5-articleLarge.jpg?quality=75\&auto=webp\&disable=upscale}

Since bail reform began in New York on Jan. 1, opponents have been hard
at work to create a backlash against the new law, sometimes
\href{https://www.newsday.com/long-island/suffolk/jordan-randolph-bail-new-york-1.40686221}{wrongly
blaming it} for the commission of crimes by people they say should have
been jailed for earlier offenses, and
\href{https://www.nytimes3xbfgragh.onion/2020/01/23/opinion/shea-nypd-bail-reform.html}{warning
of broad danger} to the public.

Using this old script, prosecutors and police chiefs are demanding
rollbacks as a
public\href{https://nypost.com/2020/01/21/new-york-voters-have-turned-against-bail-reform-new-poll-says/}{opinion
survey shows} that support for bail reform is falling. Largely untold
are the stories of thousands of New Yorkers who have been released
\href{https://www.nydailynews.com/opinion/ny-oped-bail-reform-is-already-working-20200122-yyj35x7v4rc6te5djpprmnseye-story.html}{without
incident}, and allowed to return to their families, jobs and communities
while charges against them are pending. The law is unusual, compared
with those in some other states, and that's one reason it's politically
vulnerable. But it has the potential to end mass incarceration as we
know it in New York, with an anticipated drop in the jail population
statewide of
\href{https://www.vera.org/downloads/publications/new-york-new-york-2019-bail-reform-law-highlights.pdf}{40
percent}. It deserves a chance to prove itself.

Before the new law, 21,000 New Yorkers were in jail on any given night
around the state. More than
\href{https://www.vera.org/state-of-incarceration}{60 percent} were
being held before trial, primarily because they couldn't afford to pay
bail. The new law has already reduced the number to 15,000 by making
release before trial automatic for people accused of most misdemeanors
and nonviolent felonies, which make up
\href{https://www.criminaljustice.ny.gov/crimnet/ojsa/adult-arrest-demographics/2018/NYS.pdf}{90
percent} of the roughly 400,000 arrests each year.

Judges still have the authority in all cases to impose conditions like
oversight by a case manager, referrals to treatment or counseling, and
in some cases, electronic monitoring. For example, take the case of
Tiffany Harris, who was arrested and charged with slapping three
Orthodox Jewish women on the street in Brooklyn, released without bail,
and then rearrested for hitting someone else. A judge could have ordered
mental health counseling after the first incident that might have helped
her
\href{https://dailycaller.com/2019/12/30/new-york-anti-semitic-bail-reform-tiffany-harris/}{instead
of just letting her go.}

Judges can also still set bail on almost all violent felony offenses,
any case involving sexual abuse or misconduct, all felony and some
misdemeanor domestic violence offenses, and witness intimidation and
tampering cases. The most common charges for which release must now be
allowed are drug possession, theft and unlicensed driving.

The best evidence for the success of bail reform comes from Brooklyn.
Prosecutors there
\href{https://www.nycja.org/publications/cja-annual-report-2017}{stopped
demanding bail}for almost all misdemeanors in April 2017. Over the next
year, the number of people held on bail in Brooklyn declined by 43
percent, and Brooklyn has also experienced a decline in crime, with
fewer shootings and the lowest number of murders in the borough's
history in 2019,
\href{http://brooklynda.org/2020/01/03/brooklyn-continued-to-record-historic-decline-in-violent-crime-in-2019/}{according
to the district attorney's office}.

In 2019, the Bronx and Manhattan followed Brooklyn's lead with
\href{https://criminaljustice.cityofnewyork.us/wp-content/uploads/2019/06/Scorecard-Jan-to-Mar-2019-TJ06072019-2-pgs.pdf}{similar
results.} The real impact of bail reform is being felt in upstate New
York, where 60 percent of the people in jail are charged with
misdemeanors. Already, there have been major drops in the number of
people in jail in counties like Herkimer and Onondaga, with no upticks
in crime.

Despite those facts, bail reform is in political trouble mainly because
the state is the only one in the country that bars a judge who is
deciding if bail is warranted from considering whether a person poses a
threat to public safety. For decades, New York judges have been allowed
to consider only the likelihood that the defendant will appear for
future court dates (though it is likely that at least some judges factor
in public safety in their calculus).

When bail reform was debated last year, the New York State Legislature
considered adding a public-safety provision for a small subset of
serious cases. But many proponents of reform, especially defense
lawyers, feared such a provision would give judges too much leeway to
impose preventive detention. The Legislature and Gov. Andrew Cuomo
instead struck a compromise that left money bail as an option for almost
all violent crimes, the same small subset of serious cases for which a
public-safety provision was considered and rejected. Now the governor is
mostly defending bail reform, but he also called it an
\href{https://www.nydailynews.com/news/politics/ny-cuomo-bail-reform-20200121-bn5gjfj5hrbjtcnxfd4vntptwi-story.html}{``ongoing
process''} in his budget address Tuesday.

If New York decides to tweak its law, it should do so based on months,
not weeks, of data. Any amendments should safeguard the principle that
the state cannot have two systems of justice, one for the rich and one
for the poor. If the Legislature decides to allow judges to detain
people for the sake of public safety, it should also eliminate money
bail entirely, so that wealthy people aren't released while others
facing the same charges are jailed. And a public-safety provision should
apply only to serious offenses, and require a hearing with facts showing
that the person is likely to threaten someone else's physical safety.

The results of bail reform in New Jersey show the benefits of staying
the course. In 2017, New Jersey courts essentially stopped imposing cash
bail for all criminal charges. Early on,
\href{https://www.nbcnewyork.com/news/local/bail-reform-new-jersey-criminals-streets-law-jail-investigation/141250/}{opponents
said things like}, ``Nobody's afraid to commit crimes anymore,'' and one
mother
\href{https://www.courthousenews.com/murder-victims-mother-sues-chris-christie-nj-bail-reform/}{sued}
the state after her son was killed, allegedly by a man who had been
released on a gun possession charge. The New Jersey Supreme Court
\href{https://www.nj.com/politics/2017/05/nj_supreme_court_tightens_bail_reform_rules_for_gu.html}{made
it easier} to detain people accused of gun crimes or repeat offenses,
but otherwise, the state weathered the criticism. In 2017,
\href{https://njcourts.gov/courts/assets/criminal/2018cjrannual.pdf?c=ZZI}{according
to state data}, less than 3 percent of people released before trial were
rearrested for a serious violent felony or for gun possession. On any
given day in 2018, 6,000 fewer people were in jail than had been six
years earlier. At the same time, violent crime in New Jersey
\href{https://www.americashealthrankings.org/explore/annual/measure/Crime/state/NJ}{continued
to fall}.

The experience of other states also shows the value of another element
of New York's new law: reducing the time that prosecutors have to turn
over evidence in their possession to the defense. New York's so-called
blindfold law had allowed prosecutors to withhold key evidence until the
morning of trial, and had contributed to wrongful convictions,
\href{https://www.nysba.org/Journal/2019/May/New_York_Removes_the_Blindfold/}{according
to the state bar association}.

North Carolina and Texas enacted laws in the last decade requiring broad
disclosure of evidence, known as discovery. At first, prosecutors said
the requirements were unworkable and put witnesses and victims at risk
by potentially disclosing their identities. But judges in those states
--- and in New York --- can limit disclosure when necessary. Over time,
the laws in North Carolina and Texas proved their worth. A
\href{https://scholarlycommons.law.wlu.edu/wlulr-online/vol73/iss1/20/}{2016
study} of North Carolina's law found that 91 percent of prosecutors
reported the law was working well. Texas prosecutors
\href{https://citylimits.org/2019/02/18/opinion-texas-paves-the-way-for-ny-to-make-criminal-justice-more-fair/}{have
urged} New York to view their state as a model for discovery practices.

Studies
\href{https://craftmediabucket.s3.amazonaws.com/uploads/PDFs/LJAF_Report_hidden-costs_FNL.pdf}{show}
that it takes only two or three days behind bars to increase the risk
that someone charged with a minor, nonviolent crime will be arrested
again. It may seem counterintuitive, but
\href{https://papers.ssrn.com/sol3/papers.cfm?abstract_id=3335138}{jailing
people leads to less, not more, public safety}. Jail destabilizes people
who may already be struggling to pay the rent or get to work.
Fear-mongering makes for spicy headlines but terrible policy. New
Yorkers should not give in to it.

\href{https://www.vera.org/people/insha-rahman}{Insha Rahman} is the
director of strategy and new initiatives at the
\href{https://www.vera.org/}{Vera Institute of Justice}.

Advertisement

\protect\hyperlink{after-bottom}{Continue reading the main story}

\hypertarget{site-index}{%
\subsection{Site Index}\label{site-index}}

\hypertarget{site-information-navigation}{%
\subsection{Site Information
Navigation}\label{site-information-navigation}}

\begin{itemize}
\tightlist
\item
  \href{https://help.nytimes3xbfgragh.onion/hc/en-us/articles/115014792127-Copyright-notice}{©~2020~The
  New York Times Company}
\end{itemize}

\begin{itemize}
\tightlist
\item
  \href{https://www.nytco.com/}{NYTCo}
\item
  \href{https://help.nytimes3xbfgragh.onion/hc/en-us/articles/115015385887-Contact-Us}{Contact
  Us}
\item
  \href{https://www.nytco.com/careers/}{Work with us}
\item
  \href{https://nytmediakit.com/}{Advertise}
\item
  \href{http://www.tbrandstudio.com/}{T Brand Studio}
\item
  \href{https://www.nytimes3xbfgragh.onion/privacy/cookie-policy\#how-do-i-manage-trackers}{Your
  Ad Choices}
\item
  \href{https://www.nytimes3xbfgragh.onion/privacy}{Privacy}
\item
  \href{https://help.nytimes3xbfgragh.onion/hc/en-us/articles/115014893428-Terms-of-service}{Terms
  of Service}
\item
  \href{https://help.nytimes3xbfgragh.onion/hc/en-us/articles/115014893968-Terms-of-sale}{Terms
  of Sale}
\item
  \href{https://spiderbites.nytimes3xbfgragh.onion}{Site Map}
\item
  \href{https://help.nytimes3xbfgragh.onion/hc/en-us}{Help}
\item
  \href{https://www.nytimes3xbfgragh.onion/subscription?campaignId=37WXW}{Subscriptions}
\end{itemize}
