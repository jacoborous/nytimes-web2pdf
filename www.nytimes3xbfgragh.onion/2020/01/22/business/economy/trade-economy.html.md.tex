Sections

SEARCH

\protect\hyperlink{site-content}{Skip to
content}\protect\hyperlink{site-index}{Skip to site index}

\href{https://www.nytimes3xbfgragh.onion/section/business/economy}{Economy}

\href{https://myaccount.nytimes3xbfgragh.onion/auth/login?response_type=cookie\&client_id=vi}{}

\href{https://www.nytimes3xbfgragh.onion/section/todayspaper}{Today's
Paper}

\href{/section/business/economy}{Economy}\textbar{}The Trade War, Paused
for Now, Is Still Wreaking Damage

\url{https://nyti.ms/38uIsDP}

\begin{itemize}
\item
\item
\item
\item
\item
\end{itemize}

Advertisement

\protect\hyperlink{after-top}{Continue reading the main story}

Supported by

\protect\hyperlink{after-sponsor}{Continue reading the main story}

\hypertarget{the-trade-war-paused-for-now-is-still-wreaking-damage}{%
\section{The Trade War, Paused for Now, Is Still Wreaking
Damage}\label{the-trade-war-paused-for-now-is-still-wreaking-damage}}

Manufacturers and farmers struggled last year because of tariffs, and
there are signs that damage is spreading to other sectors of the
economy.

\includegraphics{https://static01.graylady3jvrrxbe.onion/images/2020/01/21/business/00trade-damage1/merlin_167484075_fa35383c-7cb5-49ed-a1bb-dfef216b229f-articleLarge.jpg?quality=75\&auto=webp\&disable=upscale}

\href{https://www.nytimes3xbfgragh.onion/by/ben-casselman}{\includegraphics{https://static01.graylady3jvrrxbe.onion/images/2018/11/09/multimedia/author-ben-casselman/author-ben-casselman-thumbLarge.png}}\href{https://www.nytimes3xbfgragh.onion/by/niraj-chokshi}{\includegraphics{https://static01.graylady3jvrrxbe.onion/images/2018/02/20/multimedia/author-niraj-chokshi/author-niraj-chokshi-thumbLarge.jpg}}\href{https://www.nytimes3xbfgragh.onion/by/jim-tankersley}{\includegraphics{https://static01.graylady3jvrrxbe.onion/images/2018/10/19/multimedia/author-jim-tankersley/author-jim-tankersley-thumbLarge.png}}

By \href{https://www.nytimes3xbfgragh.onion/by/ben-casselman}{Ben
Casselman},
\href{https://www.nytimes3xbfgragh.onion/by/niraj-chokshi}{Niraj
Chokshi} and
\href{https://www.nytimes3xbfgragh.onion/by/jim-tankersley}{Jim
Tankersley}

\begin{itemize}
\item
  Jan. 22, 2020
\item
  \begin{itemize}
  \item
  \item
  \item
  \item
  \item
  \end{itemize}
\end{itemize}

The trade war is de-escalating, at least for now. But the economic
damage it caused could be far from over.

Two years of tit-for-tat tariffs and on-again-off-again trade talks have
left American farmers reeling. The manufacturing sector is
\href{https://www.washingtonpost.com/business/2020/01/17/us-manufacturing-was-mild-recession-during-2019-sore-spot-economy/}{in
a recession}, albeit a relatively mild one, and factory employment
\href{https://www.nytimes3xbfgragh.onion/2020/01/10/business/economy/december-jobs-report.html}{declined
in December} after rising slowly for most of last year. And in recent
months, there have been signs that the damage is spreading: Railroads
and trucking companies have been cutting jobs, and consumers --- at
least in the parts of the country most affected by the trade disputes
--- may be pulling back as well.

``Even if manufacturing started to recover, there's still going to be
some continuing cutbacks in nonmanufacturing industries as they start to
respond,'' said Michael Hicks, an economist at Ball State University in
Indiana. ``The full effect of the layoffs hasn't really been transmitted
to the full economy yet.''

Events last week in Washington signaled a shift from confrontation to
conciliation. President Trump signed a
\href{https://www.nytimes3xbfgragh.onion/2020/01/15/business/economy/china-trade-deal.html}{preliminary
trade deal} with China that, if fully carried out, would increase
American exports and prevent new tariffs, though it will not remove most
duties already in place. And the Senate approved an
\href{https://www.nytimes3xbfgragh.onion/2020/01/16/us/politics/usmca-vote.html}{overhaul
of the North American Free Trade Agreement}, which now awaits Mr.
Trump's signature.

Experts said the agreements should help restore confidence among
business leaders after months of trade-related uncertainty. But even if
those deals hold, the ripple effects of the trade war could take time to
dissipate.

Other factors are also hurting manufacturing. A global economic slowdown
--- caused partly, but not entirely, by trade tensions --- has curbed
demand for American products abroad. Falling energy prices have led to a
pullback in oil drilling and reduced the need for oil field equipment.
Boeing's recent decision to halt production of its
\href{https://www.nytimes3xbfgragh.onion/2019/12/16/business/boeing-737-max.html}{troubled
737 Max} aircraft has sent shock waves through the company's vast supply
chain; economists at Moody's Analytics estimate that the shutdown could
\href{https://www.economy.com/dismal/analysis/todays-economy/377567/Boeings-737-Max-Issues-Will-Ding-Q1-GDP/}{shave
half a percentage point} off first-quarter economic growth.

But economists say there is little doubt that trade has been the driving
force of the industrial slowdown, with implications for the rest of the
economy. The spillover effects are clearest in the transportation
sector, where business slowed for railroads and trucking companies as
trade slumped last year.

Freight volumes fell 7.9 percent in December from a year earlier, the
greatest year-over-year decline since the recession a decade ago,
according to
\href{https://www.cassinfo.com/freight-audit-payment/cass-transportation-indexes/december-2019}{data}
from Cass Information Systems. More than 10,000 jobs were cut by
transportation and warehouse employers in December, the biggest drop in
nearly four years.

Job growth has slowed sharply --- from an annual rate of 2.6 percent at
the start of 2019 to 1.3 percent at the end --- in so-called middle wage
sectors that include mining, construction and transportation, according
to calculations by Nick Bunker, an economist at the Indeed Hiring Lab.
That slowdown is driving the deceleration of job growth across the
American economy.

Railroads have been hit particularly hard, analysts said. Ian Jefferies,
president of the Association of American Railroads, said the slowdown in
trade had led to lower rail volumes in grains and industrial equipment
in particular.

``Trade uncertainty has played a pretty large role'' in the rail
slowdown, Mr. Jefferies said. The recent deals, he added, should
``provide some much-needed certainty back into the system.''

To Kevin Luke, who transports goods from the Port of Long Beach near Los
Angeles to local distribution centers, the effect of the trade war has
been clear and pronounced.

\includegraphics{https://static01.graylady3jvrrxbe.onion/images/2020/01/21/business/21trade-damage3/merlin_167484078_58ab9483-6bf7-4d2c-8913-827c22b89d81-articleLarge.jpg?quality=75\&auto=webp\&disable=upscale}

Business was brisk in 2018, with Mr. Luke's company, KNL Luxury,
collecting about \$250,000 in revenue, prompting an investment in a
second truck. But the slowdown in imports meant the demand he had
expected never materialized. In the end, he collected only slightly more
revenue in 2019 --- just shy of \$290,000 --- despite having doubled his
capacity.

``I tried to invest in the future, I tried to be ready for
opportunity,'' said Mr. Luke, who has seven children ranging in age from
7 to 19.

If conditions don't improve, Mr. Luke may have to take up long-haul
trucking, which would keep him away from home for long stretches.

The trade war hit the trucking industry at a vulnerable moment, said
Aaron Terrazas, director of economic research at Convoy, a
\href{https://www.nytimes3xbfgragh.onion/2019/08/29/automobiles/trucking-apps-shipping.html}{shipping-focused
technology company}. Trucking companies expanded aggressively in recent
years, adding trucks and drivers more quickly than demand was growing.
The resulting glut pushed down prices, just as the slowdown in trade
began eating into demand.

``There was almost this one-two punch where we were having this normal
supply correction in the market and subsequently we got hit by the trade
war,'' Mr. Terrazas said.

Mr. Trump and his allies have said the trade deals will
\href{https://www.nytimes3xbfgragh.onion/2020/01/16/business/trade-deals-economy.html}{deliver
a jolt to the economy} and lead to faster growth this year. But
economists are skeptical. Wall Street analysts expect growth, which is
already cooling, to slow further in early 2020, and few have marked up
their estimates in response to the trade announcements.

``I think we're seeing the bottom, but we're going to bounce around for
a period of time before we really see any noticeable growth,'' said Eric
Starks, chief executive of FTR, a freight research firm. ``That is
assuming that there are no outside shocks and it seems like every other
day a new shock keeps happening.''

Economists said the agreements should help shore up the struggling
manufacturing sector and prevent further damage to the economy. But they
won't necessarily heal the damage that has already been done. Companies
that
\href{https://www.nytimes3xbfgragh.onion/2019/05/30/business/economy/trump-tariff-manufacturer.html}{shifted
supply chains} away from China in response to the trade war won't
necessarily move them back now that tensions have cooled, for example.
And companies may be hesitant to commit to long-term investments until
they see evidence that the trade deals will last.

``How much can we really go back to the way things were before this
tiff?'' Mr. Terrazas asked rhetorically. ``Are they going to go back
quickly to the way things were before, or are companies going to say
this new uncertainty is going to be a feature of the global trade
picture for the years ahead?''

Manufacturing is a relatively small part of the American economy, and
there is little risk that even a sustained slump in manufacturing could,
on its own, push the country into a recession. Consumer spending remains
robust, and the fears of a downturn that gripped financial markets over
the summer have eased.

Still, the factory sector remains the centerpiece of Mr. Trump's
economic appeal to voters, especially in the industrial states that
lifted him to the White House in 2016. ``This is a blue-collar boom,''
Mr. Trump said on Tuesday in a speech in Davos, Switzerland. ``We have
created 1.2 million manufacturing and construction jobs --- a number
also unthinkable.''

Only 197,000 of those jobs were created last year, however, a sharp
deceleration from the first two years of his administration. The United
States created 1.1 million manufacturing and construction jobs in the
three years before he took office.

There are signs that in the
\href{https://www.nytimes3xbfgragh.onion/interactive/2019/12/16/business/trump-midwest-swing-jobs.html}{places
most exposed to the trade war} --- particularly Wisconsin and other
Midwestern states --- those effects have spread beyond the industrial
sector and begun to affect consumers. In a
\href{http://www.waugheconomics.com/uploads/2/2/5/6/22563786/waugh_consumption.pdf}{recent
working paper}, Michael E. Waugh, an economist at New York University,
found that automobile sales were growing significantly more slowly in
the counties most affected by the tariffs than in the rest of the
country. Those places have also seen slower job growth in their retail
sectors.

``Things are spilling over in these communities that are relatively more
affected,'' Mr. Waugh said. ``New York is all fine. But there are places
in the U.S. that are really struggling.''

The Midwest went through a similar economic soft patch in 2015 and 2016,
when falling oil prices and other factors caused a
\href{https://www.nytimes3xbfgragh.onion/2018/09/29/upshot/mini-recession-2016-little-known-big-impact.html}{mini-recession
in the industrial sector}. Mr. Waugh said he saw parallels --- although
it isn't clear how they will play out given that this time Mr. Trump is
the incumbent.

``Those places slowed down'' in 2016, Mr. Waugh said. ``Those places
influenced the election. And now what do you have today? You have those
same places slowing down, and they're looking pivotal in the election
again.''

While some of the effects of the trade war could soon be reversed,
others may last longer.

``A lot of customers moved their production out of Asia to Europe and
some of them moved their production from Asia to Mexico, so there's a
migration,'' said Lidia Yan, the chief executive and co-founder of Next
Trucking, a start-up that matches shippers and truckers.

Even in cases where production has been shifted from China to other
parts of Asia, including Vietnam and India, West Coast ports may be
losers, as exports from the more southerly Asian countries tend to be
shipped to the East Coast.

``It's early days,'' said Gene Seroka, the executive director of the
Port of Los Angeles. ``But it's enough to notice at this point in time,
and we're watching it very closely.''

Advertisement

\protect\hyperlink{after-bottom}{Continue reading the main story}

\hypertarget{site-index}{%
\subsection{Site Index}\label{site-index}}

\hypertarget{site-information-navigation}{%
\subsection{Site Information
Navigation}\label{site-information-navigation}}

\begin{itemize}
\tightlist
\item
  \href{https://help.nytimes3xbfgragh.onion/hc/en-us/articles/115014792127-Copyright-notice}{©~2020~The
  New York Times Company}
\end{itemize}

\begin{itemize}
\tightlist
\item
  \href{https://www.nytco.com/}{NYTCo}
\item
  \href{https://help.nytimes3xbfgragh.onion/hc/en-us/articles/115015385887-Contact-Us}{Contact
  Us}
\item
  \href{https://www.nytco.com/careers/}{Work with us}
\item
  \href{https://nytmediakit.com/}{Advertise}
\item
  \href{http://www.tbrandstudio.com/}{T Brand Studio}
\item
  \href{https://www.nytimes3xbfgragh.onion/privacy/cookie-policy\#how-do-i-manage-trackers}{Your
  Ad Choices}
\item
  \href{https://www.nytimes3xbfgragh.onion/privacy}{Privacy}
\item
  \href{https://help.nytimes3xbfgragh.onion/hc/en-us/articles/115014893428-Terms-of-service}{Terms
  of Service}
\item
  \href{https://help.nytimes3xbfgragh.onion/hc/en-us/articles/115014893968-Terms-of-sale}{Terms
  of Sale}
\item
  \href{https://spiderbites.nytimes3xbfgragh.onion}{Site Map}
\item
  \href{https://help.nytimes3xbfgragh.onion/hc/en-us}{Help}
\item
  \href{https://www.nytimes3xbfgragh.onion/subscription?campaignId=37WXW}{Subscriptions}
\end{itemize}
