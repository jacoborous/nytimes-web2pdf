Sections

SEARCH

\protect\hyperlink{site-content}{Skip to
content}\protect\hyperlink{site-index}{Skip to site index}

\href{https://www.nytimes3xbfgragh.onion/section/sports/soccer}{Soccer}

\href{https://myaccount.nytimes3xbfgragh.onion/auth/login?response_type=cookie\&client_id=vi}{}

\href{https://www.nytimes3xbfgragh.onion/section/todayspaper}{Today's
Paper}

\href{/section/sports/soccer}{Soccer}\textbar{}Envisioning a Champions
League on Tour, a Soccer Investor Demands More for His Money

\url{https://nyti.ms/2sHGvof}

\begin{itemize}
\item
\item
\item
\item
\item
\item
\end{itemize}

Advertisement

\protect\hyperlink{after-top}{Continue reading the main story}

Supported by

\protect\hyperlink{after-sponsor}{Continue reading the main story}

\hypertarget{envisioning-a-champions-league-on-tour-a-soccer-investor-demands-more-for-his-money}{%
\section{Envisioning a Champions League on Tour, a Soccer Investor
Demands More for His
Money}\label{envisioning-a-champions-league-on-tour-a-soccer-investor-demands-more-for-his-money}}

The International Champions Cup, which brings Europe's top clubs to
North America, has been lucrative for the teams but not the organizer.
Now there is a threat to end the event unless teams agree to make the
competition more serious.

\includegraphics{https://static01.graylady3jvrrxbe.onion/images/2020/01/21/sports/21rosssoccer-1/merlin_141926622_09be1472-bb44-4cac-8be9-cfc44a82acf5-articleLarge.jpg?quality=75\&auto=webp\&disable=upscale}

By \href{https://www.nytimes3xbfgragh.onion/by/tariq-panja}{Tariq Panja}

\begin{itemize}
\item
  Jan. 21, 2020
\item
  \begin{itemize}
  \item
  \item
  \item
  \item
  \item
  \item
  \end{itemize}
\end{itemize}

Stephen M. Ross, the real estate developer and principal owner of the
Miami Dolphins, flew to Paris this month to speak with leaders of
Europe's most powerful soccer clubs, teams that have benefited from the
billionaire's largess in bankrolling a summer tournament in the United
States and beyond.

For seven years Ross has plowed millions into the International
Champions Cup, an annual showcase that has become very lucrative for the
superstar-laden clubs that receive a coveted spot in the competition.
But for Ross and RSE Ventures --- the sports investment company he
co-founded that controls the I.C.C. --- the price of running the event
has grown to more than \$100 million, with no signs of a profit.

Addressing management and owners from the likes of Manchester United,
Juventus, Paris St.-Germain and Liverpool, Ross and RSE executives said
something had to change --- organizers could not keep losing money.

According to multiple people familiar with the talks who requested
anonymity to discuss a private meeting, the clubs were told that either
the event --- which teams treat as little more than a preseason tuneup
--- had to become more meaningful or RSE would be forced to turn off the
spigot.

While the I.C.C.'s games have drawn some of the biggest crowds in soccer
history --- 109,318 squeezed into Michigan Stadium in 2014 to watch
Manchester United beat Real Madrid, 3-1 --- it has failed to attract the
type of investment from broadcasters and commercial partners that is
necessary for a profitable future.

Since the event coincides with Europe's preseason, most clubs use it as
an opportunity to build up players' fitness and test out new, unproven
talent. Top performers and big names often play limited minutes --- if
any --- to the annoyance of fans.

Ross's group wants the teams to commit to a tournament with legitimate
stakes, one with the kind of competitive tension seen in other events
--- something akin to taking the Champions League, Europe's wildly
popular club competition, on tour.

The teams were told that, if they agreed, they could make much more
money than they currently do from I.C.C. games, while Ross would finally
be able to see his investment pay off.

Ross's trip included a meeting with European soccer's governing body,
UEFA. The I.C.C. organizers want to secure UEFA's backing for any new
event by having it partner with them or allow them to use its branding
to give the tournament the sheen of excellence and importance they
crave.

Ben Sosenko, a spokesman for the tournament's partner Relevent Sports,
which is owned by RSE, declined to comment.

\includegraphics{https://static01.graylady3jvrrxbe.onion/images/2020/01/21/sports/21rosssoccer-2/merlin_164557041_baf25a56-6c4e-4dce-b0d7-4d070f733faa-articleLarge.jpg?quality=75\&auto=webp\&disable=upscale}

The clubs acknowledged that the tournament needed changes, but no
decisions were reached in Paris. Instead, the teams agreed to set up a
joint working group with Relevent to study whether another high-profile
event was desirable or feasible in
\href{https://www.nytimes3xbfgragh.onion/2019/06/14/sports/copa-america-brazil.html}{global
soccer's increasingly crowded calendar.} They are expected to report
back in the spring.

The talks came during a critical time for global soccer, as the game's
influencers are competing to shape the future of the sport and carve out
their own interests.

A week after the meeting in Paris, some of the same club officials
traveled to Zurich for talks with Gianni Infantino, the president of
FIFA. Under Infantino,
\href{https://www.nytimes3xbfgragh.onion/2019/03/15/sports/fifa-2022-world-cup.html}{FIFA
has created its own international club tournament}, a 24-team event to
be held every four years, starting in China in 2021.

The fortunes of FIFA's new event depend on the complete buy-in of its 12
European participants, and their meeting with FIFA concluded with the
clubs asking for a seat at the decision-making table and the ability to
have some control over the tournament, similar to an agreement they
already have with UEFA for European club tournaments.

A FIFA spokesman declined to comment.

Player unions may offer resistance to the I.C.C.'s plan, as they become
increasingly wary of stakeholders adding to players' workloads in
pursuit of new revenue streams. Two days after the meeting in Paris,
FIFPro, the main global player's union, announced it had founded a new
council that includes Vincent Kompany, the Belgian star who has spoken
out about the number of games players are expected to appear in.

Top leagues may also be an obstacle. They have been largely opposed to
suggestions for new competitions.

The Champions League itself is expected to add at least four more games
starting in the 2024 season, and may even overhaul the competition
entirely by implementing a so-called Swiss model. Under that proposal,
all participants would play 10 games in a league format, with the top
eight teams qualifying directly for the knockout stages, while the 16
teams below would compete in a playoff to join them.

The idea of Champions League on tour, similar to what the I.C.C. would
like to become, isn't a new one. In fact, UEFA was considering a
something along those lines when Infantino was its chief administrator,
before he was elected FIFA's president in 2016.

To earn UEFA's seal, the I.C.C. would most likely have to modify its
invitation-only model, according to a person familiar with the talks.
Clubs would have to be considered on merit, the person said, rather than
factors like marketability and the size of their fan bases.

If such an event came to fruition, it could compete with FIFA's
competition and create another front in the often bitter relationship
between soccer's two most powerful figures, Infantino and UEFA's
president, Aleksander Ceferin.

The two have spent much of the last two years clashing over various
issues, including FIFA's World Cup. Most recently, tempers flared after
details leaked from a meeting between Infantino and Florentino Pérez,
the Real Madrid president, over Pérez's desire to create a Super League
that would unmoor the biggest teams from the domestic competitions in
which they have played for decades.

Advertisement

\protect\hyperlink{after-bottom}{Continue reading the main story}

\hypertarget{site-index}{%
\subsection{Site Index}\label{site-index}}

\hypertarget{site-information-navigation}{%
\subsection{Site Information
Navigation}\label{site-information-navigation}}

\begin{itemize}
\tightlist
\item
  \href{https://help.nytimes3xbfgragh.onion/hc/en-us/articles/115014792127-Copyright-notice}{©~2020~The
  New York Times Company}
\end{itemize}

\begin{itemize}
\tightlist
\item
  \href{https://www.nytco.com/}{NYTCo}
\item
  \href{https://help.nytimes3xbfgragh.onion/hc/en-us/articles/115015385887-Contact-Us}{Contact
  Us}
\item
  \href{https://www.nytco.com/careers/}{Work with us}
\item
  \href{https://nytmediakit.com/}{Advertise}
\item
  \href{http://www.tbrandstudio.com/}{T Brand Studio}
\item
  \href{https://www.nytimes3xbfgragh.onion/privacy/cookie-policy\#how-do-i-manage-trackers}{Your
  Ad Choices}
\item
  \href{https://www.nytimes3xbfgragh.onion/privacy}{Privacy}
\item
  \href{https://help.nytimes3xbfgragh.onion/hc/en-us/articles/115014893428-Terms-of-service}{Terms
  of Service}
\item
  \href{https://help.nytimes3xbfgragh.onion/hc/en-us/articles/115014893968-Terms-of-sale}{Terms
  of Sale}
\item
  \href{https://spiderbites.nytimes3xbfgragh.onion}{Site Map}
\item
  \href{https://help.nytimes3xbfgragh.onion/hc/en-us}{Help}
\item
  \href{https://www.nytimes3xbfgragh.onion/subscription?campaignId=37WXW}{Subscriptions}
\end{itemize}
