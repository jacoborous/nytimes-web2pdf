Sections

SEARCH

\protect\hyperlink{site-content}{Skip to
content}\protect\hyperlink{site-index}{Skip to site index}

\href{https://www.nytimes3xbfgragh.onion/section/health/policy}{Money \&
Policy}

\href{https://myaccount.nytimes3xbfgragh.onion/auth/login?response_type=cookie\&client_id=vi}{}

\href{https://www.nytimes3xbfgragh.onion/section/todayspaper}{Today's
Paper}

\href{/section/health/policy}{Money \& Policy}\textbar{}Obama Reaffirms
Insurers Must Cover Contraception

\begin{itemize}
\item
\item
\item
\item
\item
\item
\end{itemize}

Advertisement

\protect\hyperlink{after-top}{Continue reading the main story}

Supported by

\protect\hyperlink{after-sponsor}{Continue reading the main story}

\hypertarget{obama-reaffirms-insurers-must-cover-contraception}{%
\section{Obama Reaffirms Insurers Must Cover
Contraception}\label{obama-reaffirms-insurers-must-cover-contraception}}

By \href{https://www.nytimes3xbfgragh.onion/by/robert-pear}{Robert Pear}

\begin{itemize}
\item
  Jan. 20, 2012
\item
  \begin{itemize}
  \item
  \item
  \item
  \item
  \item
  \item
  \end{itemize}
\end{itemize}

WASHINGTON --- The Obama administration said Friday that most health
insurance plans must cover contraceptives for women free of charge, and
it rejected a broad exemption sought by the Roman Catholic Church for
insurance provided to employees of Catholic hospitals, colleges and
charities.

Federal officials said they would give such church-affiliated
organizations one additional year --- until Aug. 1, 2013 --- to comply
with the requirement. Most other employers and insurers must comply by
this Aug. 1.

Leaders of the Roman Catholic Church had personally appealed to
President Obama to grant the broad exemption. He made the final decision
on the issue after hearing from them, as well as from family planning
advocates, scientific experts and members of Congress, administration
officials said.

The rule takes a big step to remove cost as a barrier to birth control,
a longtime goal of advocates for women's rights and experts on women's
health.

In announcing details of the final rule on Friday, Kathleen Sebelius,
the secretary of health and human services, said it ``strikes the
appropriate balance between respecting religious freedom and increasing
access to important preventive services.''

``Scientists have abundant evidence that birth control has significant
health benefits for women,'' Ms. Sebelius said, and ``it is documented
to significantly reduce health costs.''

Catholic bishops issued \href{http://usccb.org/news/2012/12-012.cfm}{a
statement saying they would fight} the ``edict'' from the government.

``In effect, the president is saying we have a year to figure out how to
violate our consciences,'' said Archbishop Timothy M. Dolan of New York,
the president of the United States Conference of Catholic Bishops.

In an interview, Archbishop Dolan, who is to become a cardinal next
month, said, ``We're unable to live with this.''

Other opponents of the rule said they would seek legislation to block it
and might challenge it in court as well.

The rule includes an exemption for certain ``religious employers,''
including houses of worship. But church groups said the exemption was so
narrow that it was almost meaningless. A religious employer cannot
qualify for the exemption if it employs or serves large numbers of
people of a different faith, as many Catholic hospitals, universities
and social service agencies do.

Ms. Sebelius said the one-year grace period would be available to
certain ``nonprofit employers who, based on religious beliefs, do not
currently provide contraceptive coverage in their insurance plan.'' The
extra time will allow them to ``adapt to the new rule,'' Ms. Sebelius
said.

Chris Jacobs, a health policy analyst for Senate Republicans, said,
``This decision looks suspiciously like yet another political stunt
designed to delay the controversy by a year, until after the president's
re-election campaign.''

Senator Orrin G. Hatch, Republican of Utah, said the transition period
was pointless.

``The problem is not that religious institutions do not have enough time
to comply,'' Mr. Hatch said. ``It's that they are forced to comply at
all. Unfortunately, the administration has shown a complete lack of
regard for our constitutional commitment to religious liberty.''

The
\href{http://www.nae.net/news/715-press-release-evangelicals-disappointed-with-white-house-decision-on-conscience-protection}{National
Association of Evangelicals} said that as a result of the White House
decision, ``Employers with religious objections to contraception will be
forced to pay for services and procedures they believe are morally
wrong.''

The
\href{http://www.becketfund.org/obama-administration-refuses-to-change-abortion-drug-mandate/}{Becket
Fund for Religious Liberty}, a nonprofit law firm, has filed lawsuits
challenging an earlier version of the rule in federal courts on behalf
of a Catholic college connected to a monastery in North Carolina and an
evangelical university in Colorado.

The 2010 health care law says insurers must cover ``preventive health
services'' and cannot charge for them.

The new rule interprets this mandate. It requires coverage of the full
range of contraceptive methods approved by the Food and Drug
Administration. Among the drugs and devices that must be covered are
emergency contraceptives including pills known as ella and Plan B. The
rule also requires coverage of sterilization procedures for women
without co-payments or deductibles.

The issue forced Mr. Obama to weigh competing claims of Catholic leaders
and advocates for women's rights.

The administration said in August that it intended to require coverage
of contraceptives for women, as recommended by an expert panel of the
National Academy of Sciences. But the White House reconsidered the issue
after hearing protests from the Catholic Church and many Republicans in
Congress.

The protests prompted debate within the administration. Ms. Sebelius and
the president's health policy team strongly supported the new rule. But
Democratic members of Congress who lobbied the White House said they
believed that Mr. Obama's chief of staff, William M. Daley, and his
special assistant for religious affairs, Joshua DuBois, favored a
broader exemption.

Senator Richard Blumenthal, Democrat of Connecticut, described the final
rule as a huge victory for women's health. It will, he said, ``ensure
that women have access to full health care services, regardless of their
employer, so they can make the best health choices for themselves and
their families.''

Representative Lois Capps, Democrat of California, said, ``The
administration deserves credit for standing its ground and following the
science.''

\href{http://www.plannedparenthood.org/about-us/newsroom/press-releases/planned-parenthood-applauds-hhs-ensuring-access-affordable-birth-control-38582.htm}{Cecile
Richards,} president of the Planned Parenthood Federation of America,
said the decision ``means that millions of women, who would otherwise
pay \$15 to \$50 a month, will have access to affordable birth control,
helping them save hundreds of dollars each year.''

Archbishop Dolan said he discussed the issue with Mr. Obama last
November and came away reassured that the president understood the
Catholic Church's position. Now, the archbishop said in the interview,
``The sentiments of hope that stemmed from reassurances that I thought I
received in November were apparently misplaced.''

The archbishop said he had heard from evangelical, Greek Orthodox and
Orthodox Jewish leaders who were also concerned about the rule.

Under the government's narrow criteria, the bishops said, ``even the
ministry of Jesus and the early Christian Church would not qualify as
`religious,' because they did not confine their ministry to their
co-religionists,'' but urged compassion for the sick and the poor,
regardless of faith or creed.

Advertisement

\protect\hyperlink{after-bottom}{Continue reading the main story}

\hypertarget{site-index}{%
\subsection{Site Index}\label{site-index}}

\hypertarget{site-information-navigation}{%
\subsection{Site Information
Navigation}\label{site-information-navigation}}

\begin{itemize}
\tightlist
\item
  \href{https://help.nytimes3xbfgragh.onion/hc/en-us/articles/115014792127-Copyright-notice}{©~2020~The
  New York Times Company}
\end{itemize}

\begin{itemize}
\tightlist
\item
  \href{https://www.nytco.com/}{NYTCo}
\item
  \href{https://help.nytimes3xbfgragh.onion/hc/en-us/articles/115015385887-Contact-Us}{Contact
  Us}
\item
  \href{https://www.nytco.com/careers/}{Work with us}
\item
  \href{https://nytmediakit.com/}{Advertise}
\item
  \href{http://www.tbrandstudio.com/}{T Brand Studio}
\item
  \href{https://www.nytimes3xbfgragh.onion/privacy/cookie-policy\#how-do-i-manage-trackers}{Your
  Ad Choices}
\item
  \href{https://www.nytimes3xbfgragh.onion/privacy}{Privacy}
\item
  \href{https://help.nytimes3xbfgragh.onion/hc/en-us/articles/115014893428-Terms-of-service}{Terms
  of Service}
\item
  \href{https://help.nytimes3xbfgragh.onion/hc/en-us/articles/115014893968-Terms-of-sale}{Terms
  of Sale}
\item
  \href{https://spiderbites.nytimes3xbfgragh.onion}{Site Map}
\item
  \href{https://help.nytimes3xbfgragh.onion/hc/en-us}{Help}
\item
  \href{https://www.nytimes3xbfgragh.onion/subscription?campaignId=37WXW}{Subscriptions}
\end{itemize}
