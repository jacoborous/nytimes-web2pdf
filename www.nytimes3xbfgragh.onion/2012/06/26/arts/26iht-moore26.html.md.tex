Sections

SEARCH

\protect\hyperlink{site-content}{Skip to
content}\protect\hyperlink{site-index}{Skip to site index}

\href{https://www.nytimes3xbfgragh.onion/section/arts}{Arts}

\href{https://myaccount.nytimes3xbfgragh.onion/auth/login?response_type=cookie\&client_id=vi}{}

\href{https://www.nytimes3xbfgragh.onion/section/todayspaper}{Today's
Paper}

\href{/section/arts}{Arts}\textbar{}Henry Moore Goes Indoors

\url{https://nyti.ms/KIcgQa}

\begin{itemize}
\item
\item
\item
\item
\item
\end{itemize}

Advertisement

\protect\hyperlink{after-top}{Continue reading the main story}

Supported by

\protect\hyperlink{after-sponsor}{Continue reading the main story}

Art

\hypertarget{henry-moore-goes-indoors}{%
\section{Henry Moore Goes Indoors}\label{henry-moore-goes-indoors}}

By \href{https://www.nytimes3xbfgragh.onion/by/roslyn-sulcas}{Roslyn
Sulcas}

\begin{itemize}
\item
  June 25, 2012
\item
  \begin{itemize}
  \item
  \item
  \item
  \item
  \item
  \end{itemize}
\end{itemize}

LONDON --- A Henry Moore sculpture is an oddly familiar sight. Its size
and scale, weight and heft; the undulating curves and geometric angles
that recall both the specificity of the human figure and the
abstractness of large-scale landscape --- these elements are present in
Moore's work on public plazas and piazzas, in front of city halls and
banks, public squares and parks, all over the world. ``More than any
other artist of our own time,'' John Russell wrote in The New York Times
in 1983, ``he has been brought out of the museum and into the open and
offered the gift of ubiquity.''

Along with that ubiquity has come a certain critical indifference, a
feeling that so much popularity surely meant artistic complacency. But a
new show that opened at the Gagosian Gallery in London on May 31, ``Late
Large Forms,'' tries to revise that jaded view of Moore with an
exhibition of some of the huge bronze pieces he created from 1960 to
1980 for sculpture parks and outdoor spaces.

The surprise of the show is too see these monumental works --- most of
which have never been shown indoors --- in the neutral white space of
the Gagosian, which occupies a warehouse-like space near King's Cross
Station.

``A lot of collectors and dealers have said, they are meant to go
outside, what would Moore have thought?'' said Anita Feldman, the
curator of ``Large Late Forms'' and the head of collections and
exhibitions for the Henry Moore Foundation at the sculptor's house in
Perry Green, Hertfordshire.

In an interview at the gallery before the show's opening, she continued:
``But one of the great challenges as a curator of Henry Moore's work is
getting beyond general perceptions. People think they know Moore because
his work is all over the world. I wanted to do something that would
change how you approach the work, how you read the forms. Here, you
don't have the distractions of nature. The pieces evoke the feeling of
the sublime in sculpture, nature as overwhelming and slightly
threatening, and I think that aspect is even more pronounced in an
interior space.''

\includegraphics{https://static01.graylady3jvrrxbe.onion/images/2012/06/26/arts/26iht-moore26-pic1/26iht-moore26-pic1-jumbo.jpg?quality=75\&auto=webp\&disable=upscale}

Moore did not achieve international fame until 1948, at 50, when he won
the grand prize for sculpture at that year's Venice Biennale. Until
then, he had been known in his native England (he grew up in Yorkshire)
for his drawings of World War II Londoners' waiting and sleeping in Tube
stations while bombs rained upon the city during the Blitz.

In the decade after Venice, however, and until his death in 1986, he was
to become the unquestioned voice of British sculpture, a byword for
public art, and immensely popular in the United States. (``Sometimes it
seems as if there are Americans who cannot get out of bed in the morning
until they have bought a Henry Moore,'' Mr. Russell wrote.)

``We are used to seeing his pieces in the context of nature,'' said
Stefan Ratibor, director of the London branch of Gagosian. ``I thought
it would be interesting to show them in a clean well-lit space. There is
a sense of their hugeness, which you get less of outside, and a tension,
because you get the feeling they can almost break down the building.''

That hugeness is indeed the overwhelming impression as you enter the
Gagosian gallery and are faced with ``Two Large Forms,'' a 1966 piece in
which two enormous curving, swelling shapes are positioned close
together, one arcing over the other without touching, like
almost-interlocking pieces of a puzzle. Close-up, the sea-green patina
and changing surface of the weathered bronze is remarkably beautiful,
showing layered colors that gleam against the whiteness of the gallery
walls.

``Bringing these pieces inside completely changes the sense of the
scale,'' said Will Gompertz, arts editor at the BBC. ``They seem much
bigger, and it forces you to engage with them from much more of a
material point of view. I think it changes them as works of art.''

Getting works like ``Two Large Forms'' and the dramatic ``Two Piece
Reclining Figure: Cut'' into the Gagosian was not a simple process.
Several walls had to be removed and a number of pieces had to be lowered
into the gallery by a crane. The logistics, however, were never an
issue, Mr. Ratibor said.

``What I really admire about this organization is that the first
response is never no,'' he said. ``We have had great experiences with
Richard Serra and David Smith and pushed the technology to get the
pieces in. It doesn't frighten us. Part of ongoing scholarship is to try
to analyze artworks in new ways --- they can go back outside, but
meanwhile we've been able to do something that has never been done
before by putting these outdoor works indoors.''

Image

Henry Moore working on~ ``Two Piece Reclining Figure,'' finished in
1960.Credit...John Hedgecoe/The Henry Moore Foundation Archive

Not everyone is convinced that Gagosian's motives are all about the art.

``It's a totally commercial idea,'' said Rachel Campbell-Johnston, chief
art critic for The Times of London. ``It's an immensely expensive thing
to achieve because these things weigh tons, but it allows Gagosian to
yoke itself to a big name. Everyone has heard of Henry Moore. They have
put on some fantastic shows like this before, but it's all part of
creating an image of themselves as linked to great artists.''

But though there is still some feeling in the art world that museums
shouldn't collaborate with commercial galleries, Gagosian's willingness
to do --- and pay --- whatever it took to stage the exhibition was a
compelling incentive for the Henry Moore Foundation to agree to the idea
when it was approached 18 months ago.

``The distinctions or boundaries between commercial galleries and
museums have become much more blurred in the last few years,'' said
Richard Calvocovessi, director of the Henry Moore Foundation.
``Galleries like Gagosian put on museum-quality shows which outweigh any
doubts. And Moore did show in commercial galleries. But in the end, it
was mostly the space that convinced us to do it --- what it could offer
in terms of space and light.''

The nine works at Gagosian exude an almost erotically tactile quality in
their enclosed space. From the giant knucklebones of ``Three Piece
Sculpture: Vertebrae'' (1968) to the more industrialized forms of
``Large Spindle Piece'' (1974) and the brutal cleavage that sharply
divides ``Two Piece Reclining Figure: Cut'' (1981), the sensory
qualities of surface (hard and shiny, rough and rock-like, often showing
the marks of the maker's tools), the rearing verticalities and soft,
organic swellings seem heightened and exposed.

A small shelf of maquettes offers another vision of Moore's work as it
becomes apparent that these gigantic pieces first took form as tiny
models that could fit into the sculptor's hand. Many are constructed
from natural elements --- pieces of driftwood, sea shells, animal bones
and sculls --- that Moore collected and stored at Perry Green.

``They are quite personal objects,'' Ms. Feldman said. ``He was really
working on a small scale, with small objects in a small studio. He
wanted to be able to hold them, turn them, and see them from many
angles. There is something a bit disconcerting about seeing fragments of
body parts, bones, mixed with these found objects, but he is also saying
that we are part of organic matter. The genesis of the forms all comes
back to nature. And he thought that there was a right size for every
idea. That's something we can feel even more acutely in this context.''

\textbf{Henry Moore: Late Large Forms.} Gagosian Gallery. \emph{Through
Aug. 18.}

Advertisement

\protect\hyperlink{after-bottom}{Continue reading the main story}

\hypertarget{site-index}{%
\subsection{Site Index}\label{site-index}}

\hypertarget{site-information-navigation}{%
\subsection{Site Information
Navigation}\label{site-information-navigation}}

\begin{itemize}
\tightlist
\item
  \href{https://help.nytimes3xbfgragh.onion/hc/en-us/articles/115014792127-Copyright-notice}{©~2020~The
  New York Times Company}
\end{itemize}

\begin{itemize}
\tightlist
\item
  \href{https://www.nytco.com/}{NYTCo}
\item
  \href{https://help.nytimes3xbfgragh.onion/hc/en-us/articles/115015385887-Contact-Us}{Contact
  Us}
\item
  \href{https://www.nytco.com/careers/}{Work with us}
\item
  \href{https://nytmediakit.com/}{Advertise}
\item
  \href{http://www.tbrandstudio.com/}{T Brand Studio}
\item
  \href{https://www.nytimes3xbfgragh.onion/privacy/cookie-policy\#how-do-i-manage-trackers}{Your
  Ad Choices}
\item
  \href{https://www.nytimes3xbfgragh.onion/privacy}{Privacy}
\item
  \href{https://help.nytimes3xbfgragh.onion/hc/en-us/articles/115014893428-Terms-of-service}{Terms
  of Service}
\item
  \href{https://help.nytimes3xbfgragh.onion/hc/en-us/articles/115014893968-Terms-of-sale}{Terms
  of Sale}
\item
  \href{https://spiderbites.nytimes3xbfgragh.onion}{Site Map}
\item
  \href{https://help.nytimes3xbfgragh.onion/hc/en-us}{Help}
\item
  \href{https://www.nytimes3xbfgragh.onion/subscription?campaignId=37WXW}{Subscriptions}
\end{itemize}
