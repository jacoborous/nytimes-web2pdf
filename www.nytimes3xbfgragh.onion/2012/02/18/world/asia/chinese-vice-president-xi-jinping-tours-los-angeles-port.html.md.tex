Sections

SEARCH

\protect\hyperlink{site-content}{Skip to
content}\protect\hyperlink{site-index}{Skip to site index}

\href{https://www.nytimes3xbfgragh.onion/section/world/asia}{Asia
Pacific}

\href{https://myaccount.nytimes3xbfgragh.onion/auth/login?response_type=cookie\&client_id=vi}{}

\href{https://www.nytimes3xbfgragh.onion/section/todayspaper}{Today's
Paper}

\href{/section/world/asia}{Asia Pacific}\textbar{}Chinese Vice President
Ends U.S. Tour on Friendly Note

\begin{itemize}
\item
\item
\item
\item
\item
\end{itemize}

Advertisement

\protect\hyperlink{after-top}{Continue reading the main story}

Supported by

\protect\hyperlink{after-sponsor}{Continue reading the main story}

\hypertarget{chinese-vice-president-ends-us-tour-on-friendly-note}{%
\section{Chinese Vice President Ends U.S. Tour on Friendly
Note}\label{chinese-vice-president-ends-us-tour-on-friendly-note}}

\includegraphics{https://static01.graylady3jvrrxbe.onion/images/2012/02/18/world/18xi_337/18xi_337-articleLarge.jpg?quality=75\&auto=webp\&disable=upscale}

By \href{https://www.nytimes3xbfgragh.onion/by/edward-wong}{Edward Wong}

\begin{itemize}
\item
  Feb. 17, 2012
\item
  \begin{itemize}
  \item
  \item
  \item
  \item
  \item
  \end{itemize}
\end{itemize}

LOS ANGELES --- Vice President Xi Jinping of China wrapped up a
carefully scripted cross-country tour of the United States on Friday
having tried to cast himself not just as a fitting leader to follow
President Hu Jintao, as he almost certainly will this fall, but as an
unusually approachable one as well.

It was a performance aimed both at an American audience whose skepticism
toward China has been compounded by election-year language, and at an
audience back home whose cynicism toward its Communist rulers can also
run deep.

Mr. Hu, China's wooden and distant leader over the last decade, has left
many Chinese more comfortable economically, but also more removed from a
party and leadership that seem to have little to do with their lives.
Mr. Xi --- the joke-cracking, farm-visiting, 58-year-old statesman who
told a classroom filled with students on Friday that he likes to hit the
pool for a swim --- presents a sharp contrast to Mr. Hu.

Whether a leader with a different image will translate into better
relations with the United States and to changes in China, though,
remains as much a mystery at the end of Mr. Xi's trip as it was before
he arrived.

Vice President Joseph R. Biden Jr., who said he had spent more than 20
hours in relatively private talks with Mr. Xi during this trip and a
similar one that Mr. Biden made to China in August, would not speculate
on how his visitor's openness might translate to policy-making. But he
said that in their personal conversations, Mr. Xi ``has been very frank
about the economic and political dilemma he faces in China.''

He said he found Mr. Xi ``very, very direct,'' adding: ``He's absolutely
responsive. When we disagree, it's a clear statement of disagreement.''

Mr. Xi buffed his regular-guy image on Friday with a visit to the
International Studies Learning Center in the town of South Gate, Calif.,
just southeast of Los Angeles. Responding to a high school student's
question during a classroom visit, he said that he likes to read, swim
(his favorite sport) and watch American basketball, baseball and
football.

``Of course we always want more time to ourselves,'' Mr. Xi said in
Mandarin, the language the class was studying. ``But to borrow a title
from an American film, it's like `Mission: Impossible.'~`` The room
broke up in laughter.

Many senior Chinese officials, especially Mr. Hu, are known for trying
to maintain a distance between themselves and ordinary citizens, which
often infuriates those they govern and leads to the Internet
vilification of politicians. But Mr. Xi appeared confident and at ease
in settings like the one at the South Gate school, at a Los Angeles
Lakers game on Friday night and during a midweek stop in Muscatine,
Iowa, where he met with Americans who had been his hosts there 27 years
earlier when he was part of a delegation looking at pig-farming
techniques.

Mr. Xi's trip, with its tone of camaraderie, also had a message for
Americans at a time of growing unease in Washington over China's
economic and military rise. The tensions have been fueled in part by
strident public speeches against Chinese policies given by Republican
presidential candidates and by President Obama, who have their eyes on
the 2012 election.

Mr. Biden told reporters after the school visit that he saw in Mr. Xi a
Chinese leader with a distinctive style.

``This is unusual for any foreign leader, in particular for a Chinese
foreign leader, to want to expose himself this much to the American
public,'' Mr. Biden said. ``His going back to Muscatine was not my idea.
It was our idea to do many other things. But this is a guy who wants to
feel it and taste it, and he's prepared to show another side of the
Chinese leadership that could be useful for Americans to see as well.''

Mr. Biden said that Mr. Xi also seemed to want to learn everything he
could about the American political system. On Tuesday night, after Mr.
Xi and Mr. Biden finished dinner at the Naval Observatory, Mr. Biden's
home in Washington, the two sat in the library and, as Mr. Biden
recounted it, talked about individual members of Congress, with Mr. Xi
asking about the motivations behind some lawmakers' actions.

Mr. Xi has long stressed aspects of his personal background that show
that he can connect with a range of people. He has written about his
years toiling in the village of Liangjiahe in Shaanxi Province during
the Cultural Revolution. He has also boasted of traveling to every
county in Zhejiang Province when he was its top party official.

On Friday, he seemed keen to tell the students about the connection he
had made with Americans in his 1985 visit to Iowa. ``That trip to the
United States was the first trip I made to this country,'' he said. ``If
anything, my trip back to Muscatine the other day reinforced my
impressions of 27 years ago.''

But Mr. Xi may be engaging in mythmaking. Nicholas Platt, a senior
American diplomat, said in a telephone interview on Friday that Mr. Xi
had visited Washington in 1980 as part of a delegation led by Geng Biao,
a top Chinese general. Mr. Xi was the general's aide. ``He wasn't on
protocol lists,'' said Mr. Platt, who organized the general's visit.
``To the best of my knowledge, based on a reliable source, that was his
first trip to the U.S.''

Jin Zhong, a magazine editor in Hong Kong who has researched Mr. Xi,
also said in an earlier interview that Mr. Xi had traveled with General
Geng to Washington in 1980.

Mr. Xi was on a different mission this time. Business deals were
announced on Friday, including one between Chinese companies and
DreamWorks Animation, the American film company, at an economic forum
that Mr. Xi attended before the school visit. And in the afternoon in
Los Angeles, Mr. Xi was led on a tour of the Walt Disney Concert Hall,
designed by Frank Gehry, by Mr. Gehry himself.

Mr. Xi is not the first Chinese leader to engage in image-polishing on
an American visit. Deng Xiaoping caused a sensation in 1979 when he
donned a 10-gallon cowboy hat during a visit to Texas. On a state visit
in 1997, Mr. Hu's predecessor, Jiang Zemin, rang the opening bell at the
New York Stock Exchange. Appearing on a ``60 Minutes'' segment taped in
China in 2000, he recited part of the Gettysburg Address in English.

Both men raised hopes for better cooperation with Washington. But behind
their images and push for economic reforms, both were also thoroughly
authoritarian. And while economic ties with the United States improved
steadily under their watches, the basic differences separating the two
powers remained.

Advertisement

\protect\hyperlink{after-bottom}{Continue reading the main story}

\hypertarget{site-index}{%
\subsection{Site Index}\label{site-index}}

\hypertarget{site-information-navigation}{%
\subsection{Site Information
Navigation}\label{site-information-navigation}}

\begin{itemize}
\tightlist
\item
  \href{https://help.nytimes3xbfgragh.onion/hc/en-us/articles/115014792127-Copyright-notice}{©~2020~The
  New York Times Company}
\end{itemize}

\begin{itemize}
\tightlist
\item
  \href{https://www.nytco.com/}{NYTCo}
\item
  \href{https://help.nytimes3xbfgragh.onion/hc/en-us/articles/115015385887-Contact-Us}{Contact
  Us}
\item
  \href{https://www.nytco.com/careers/}{Work with us}
\item
  \href{https://nytmediakit.com/}{Advertise}
\item
  \href{http://www.tbrandstudio.com/}{T Brand Studio}
\item
  \href{https://www.nytimes3xbfgragh.onion/privacy/cookie-policy\#how-do-i-manage-trackers}{Your
  Ad Choices}
\item
  \href{https://www.nytimes3xbfgragh.onion/privacy}{Privacy}
\item
  \href{https://help.nytimes3xbfgragh.onion/hc/en-us/articles/115014893428-Terms-of-service}{Terms
  of Service}
\item
  \href{https://help.nytimes3xbfgragh.onion/hc/en-us/articles/115014893968-Terms-of-sale}{Terms
  of Sale}
\item
  \href{https://spiderbites.nytimes3xbfgragh.onion}{Site Map}
\item
  \href{https://help.nytimes3xbfgragh.onion/hc/en-us}{Help}
\item
  \href{https://www.nytimes3xbfgragh.onion/subscription?campaignId=37WXW}{Subscriptions}
\end{itemize}
