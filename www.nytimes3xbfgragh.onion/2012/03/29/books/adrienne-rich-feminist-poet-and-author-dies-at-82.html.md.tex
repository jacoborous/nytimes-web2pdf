Sections

SEARCH

\protect\hyperlink{site-content}{Skip to
content}\protect\hyperlink{site-index}{Skip to site index}

\href{https://www.nytimes3xbfgragh.onion/section/books}{Books}

\href{https://myaccount.nytimes3xbfgragh.onion/auth/login?response_type=cookie\&client_id=vi}{}

\href{https://www.nytimes3xbfgragh.onion/section/todayspaper}{Today's
Paper}

\href{/section/books}{Books}\textbar{}Adrienne Rich, Influential
Feminist Poet, Dies at 82

\begin{itemize}
\item
\item
\item
\item
\item
\item
\end{itemize}

Advertisement

\protect\hyperlink{after-top}{Continue reading the main story}

Supported by

\protect\hyperlink{after-sponsor}{Continue reading the main story}

\hypertarget{adrienne-rich-influential-feminist-poet-dies-at-82}{%
\section{Adrienne Rich, Influential Feminist Poet, Dies at
82}\label{adrienne-rich-influential-feminist-poet-dies-at-82}}

\includegraphics{https://static01.graylady3jvrrxbe.onion/images/2012/03/29/obituaries/subRICH/subRICH-articleLarge.jpg?quality=75\&auto=webp\&disable=upscale}

By \href{https://www.nytimes3xbfgragh.onion/by/margalit-fox}{Margalit
Fox}

\begin{itemize}
\item
  March 28, 2012
\item
  \begin{itemize}
  \item
  \item
  \item
  \item
  \item
  \item
  \end{itemize}
\end{itemize}

Adrienne Rich, a poet of towering reputation and towering rage, whose
work --- distinguished by an unswerving progressive vision and a
dazzling, empathic ferocity --- brought the oppression of women and
lesbians to the forefront of poetic discourse and kept it there for
nearly a half-century, died on Tuesday at her home in Santa Cruz, Calif.
She was 82.

The cause was complications of rheumatoid arthritis, with which she had
lived for most of her adult life, her family said.

Widely read, widely anthologized, widely interviewed and widely taught,
Ms. Rich was for decades among the most influential writers of the
feminist movement and one of the best-known American public
intellectuals. She wrote two dozen volumes of poetry and more than a
half-dozen of prose; the poetry alone has sold nearly 800,000 copies,
according to W. W. Norton \& Company, her publisher since the mid-1960s.

Triply marginalized --- as a woman, a lesbian and a Jew --- Ms. Rich was
concerned in her poetry, and in her many essays, with identity politics
long before the term was coined.

She accomplished in verse what Betty Friedan, author of ``The Feminine
Mystique,'' did in prose. In describing the stifling minutiae that had
defined women's lives for generations, both argued persuasively that
women's disenfranchisement at the hands of men must end.

For Ms. Rich, the personal, the political and the poetical were
indissolubly linked; her body of work can be read as a series of urgent
dispatches from the front. While some critics called her poetry
polemical, she remained celebrated for the unflagging intensity of her
vision, and for the constant formal reinvention that kept her verse ---
often jagged and colloquial, sometimes purposefully shocking, always
controlled in tone, diction and pacing --- sounding like that of few
other poets.

All this helped ensure Ms. Rich's continued relevance long after she
burst genteelly onto the scene as a Radcliffe senior in the early 1950s.

Her constellation of honors includes a MacArthur Foundation ``genius''
grant in 1994 and a National Book Award for poetry in 1974 for ``Diving
Into the Wreck.'' That volume, published in 1973, is considered her
masterwork.

In the title poem, Ms. Rich uses the metaphor of a dive into dark,
unfathomable waters to plumb the depths of women's experience:

\emph{I am here, the mermaid whose dark hair}\\
\emph{streams black, the merman in his armored body}\\
\emph{We circle silently about the wreck}\\
\emph{we dive into the hold. ...}\\
\emph{We are, I am, you are}\\
\emph{by cowardice or courage}\\
\emph{the one who find our way}\\
\emph{back to the scene}\\
\emph{carrying a knife, a camera}\\
\emph{a book of myths}\\
\emph{in which}\\
\emph{our names do not appear.}

Ms. Rich was far too seasoned a campaigner to think that verse alone
could change entrenched social institutions. ``Poetry is not a healing
lotion, an emotional massage, a kind of linguistic aromatherapy,'' she
said in an acceptance speech to the National Book Foundation in 2006, on
receiving its medal for distinguished contribution to American letters.
``Neither is it a blueprint, nor an instruction manual, nor a
billboard.''

But at the same time, as she made resoundingly clear in interviews, in
public lectures and in her work, Ms. Rich saw poetry as a keen-edged
beacon by which women's lives --- and women's consciousness --- could be
illuminated.

She was never supposed to have turned out as she did.

Adrienne Cecile Rich was born in Baltimore on May 16, 1929. Her father,
Arnold Rice Rich, a doctor and assimilated Jew, was an authority on
tuberculosis who taught at Johns Hopkins University. Her mother, Helen
Gravely Jones Rich, a Christian, was a pianist and composer who,
cleaving to social norms of the day, forsook her career to marry and
have children. Adrienne was baptized and confirmed in the Episcopal
Church.

Theirs was a bookish household, and Adrienne, as she said afterward, was
groomed by her father to be a literary prodigy. He encouraged her to
write poetry when she was still a child, and she steeped herself in the
poets in his library --- all men, she later ruefully observed. But those
men gave her the formalist grounding that let her make her mark when she
was still very young.

When Ms. Rich was in her last year at Radcliffe (she received a
bachelor's degree in English there in 1951), W. H. Auden chose her first
collection, ``A Change of World,'' for publication in the Yale Younger
Poets series, a signal honor. Released in 1951, the book, with its sober
mien, dutiful meter and scrupulous rhymes, was praised by reviewers for
its impeccable command of form.

She had learned the lessons of her father's library well, or so it
seemed. For even in this volume Ms. Rich had begun, with subtle
subversion, to push against a time-honored thematic constraint --- the
proscription on making poetry out of the soul-numbing dailiness of
women's lives.

A poem in the collection, ``Aunt Jennifer's Tigers,'' depicting a woman
at her needlework and reprinted here in full, is concerned with
precisely this:

\emph{Aunt Jennifer's tigers prance across a screen,}\\
\emph{Bright topaz denizens of a world of green.}\\
\emph{They do not fear the men beneath the tree;}\\
\emph{They pace in sleek chivalric certainty.}\\
\emph{Aunt Jennifer's fingers fluttering through her wool}\\
\emph{Find even the ivory needle hard to pull.}\\
\emph{The massive weight of Uncle's wedding band}\\
\emph{Sits heavily upon Aunt Jennifer's hand.}\\
\emph{When Aunt is dead, her terrified hands will lie}\\
\emph{Still ringed with ordeals she was mastered by.}\\
\emph{The tigers in the panel that she made}\\
\emph{Will go on prancing, proud and unafraid.}

Once mastered, poetry's formalist rigors gave Ms. Rich something to
rebel against, and by her third collection, ``Snapshots of a
Daughter-in-Law,'' published by Harper \& Row, she had pretty well
exploded them. That volume appeared in 1963, a watershed moment in
women's letters: ``The Feminine Mystique'' was also published that year.

In the collection's title poem, Ms. Rich chronicles the pulverizing onus
of traditional married life. It opens this way:

\emph{You, once a belle in Shreveport,}\\
\emph{with henna-colored hair, skin like a peachbud,}\\
\emph{still have your dresses copied from that time. ...}\\
\emph{Your mind now, mouldering like wedding-cake,}\\
\emph{heavy with useless experience, rich}\\
\emph{with suspicion, rumor, fantasy,}\\
\emph{crumbling to pieces under the knife-edge}\\
\emph{of mere fact.}

Though the book horrified some critics, it sealed Ms. Rich's national
reputation.

She knew the strain of domestic duty firsthand. In 1953 Ms. Rich had
married a Harvard economist, Alfred Haskell Conrad, and by the time she
was 30 she was the mother of three small boys. When Professor Conrad
took a job at the City College of New York, the family moved to New York
City, where Ms. Rich became active in the civil rights and antiwar
movements.

By 1970, partly because she had begun, inwardly, to acknowledge her
erotic love of women, Ms. Rich and her husband had grown estranged. That
autumn, he died of a gunshot wound to the head; the death was ruled a
suicide. To the end of her life, Ms. Rich rarely spoke of it.

Ms. Rich effectively came out as a lesbian in 1976, with the publication
of ``Twenty-One Love Poems,'' whose subject matter --- sexual love
between women --- was still considered disarming and dangerous. In the
years that followed her poetry and prose ranged over her increasing
self-identification as a Jewish woman, the Holocaust and the struggles
of black women.

Ms. Rich's other volumes of poetry include ``The Dream of a Common
Language'' (1978), ``A Wild Patience Has Taken Me This Far'' (1981),
``The Fact of a Doorframe'' (1984), ``An Atlas of the Difficult World''
(1991) and, most recently, ``Tonight No Poetry Will Serve,'' published
last year.

Her prose includes the essay collections ``On Lies, Secrets, and
Silence'' (1979); ``Blood, Bread, and Poetry'' (1986); an influential
essay, ``Compulsory Heterosexuality and Lesbian Existence,'' published
as a slender volume in 1981; and the nonfiction book ``Of Woman Born''
(1976), which examines the institution of motherhood as a socio-historic
construct.

For Ms. Rich, the getting of literary awards was itself a political act
to be reckoned with. On sharing the National Book Award for poetry in
1974 (the other recipient that year was Allen Ginsberg), she declined to
accept it on her own behalf. Instead, she appeared onstage with two of
that year's finalists, the poets Audre Lorde and Alice Walker; the three
of them accepted the award on behalf of all women.

In 1997, in a widely reported act, Ms. Rich declined the National Medal
of Arts, the United States government's highest award bestowed upon
artists. In a letter to Jane Alexander, then chairwoman of the National
Endowment for the Arts, which administers the award, she expressed her
dismay, amid the ``increasingly brutal impact of racial and economic
injustice,'' that the government had chosen to honor ``a few token
artists while the people at large are so dishonored.''

Art, Ms. Rich added, ``means nothing if it simply decorates the dinner
table of power which holds it hostage.''

Ms. Rich's other laurels --- and these she did accept --- include the
Bollingen Prize for Poetry, the Academy of American Poets Fellowship and
the Ruth Lilly Poetry Prize.

She taught widely, including at Columbia, Brandeis, Rutgers, Cornell and
Stanford Universities.

Ms. Rich's survivors include her partner of more than 30 years, the
writer Michelle Cliff; three sons, David, Pablo and Jacob, from her
marriage to Professor Conrad; a sister, Cynthia Rich; and two
grandchildren.

For all her verbal prowess, for all her prolific output, Ms. Rich
retained a dexterous command of the plain, pithy utterance. In a 1984
speech she summed up her reason for writing --- and, by loud unspoken
implication, her reason for being --- in just seven words.

What she and her sisters-in-arms were fighting to achieve, she said, was
simply this: ``the creation of a society without domination.''

Advertisement

\protect\hyperlink{after-bottom}{Continue reading the main story}

\hypertarget{site-index}{%
\subsection{Site Index}\label{site-index}}

\hypertarget{site-information-navigation}{%
\subsection{Site Information
Navigation}\label{site-information-navigation}}

\begin{itemize}
\tightlist
\item
  \href{https://help.nytimes3xbfgragh.onion/hc/en-us/articles/115014792127-Copyright-notice}{©~2020~The
  New York Times Company}
\end{itemize}

\begin{itemize}
\tightlist
\item
  \href{https://www.nytco.com/}{NYTCo}
\item
  \href{https://help.nytimes3xbfgragh.onion/hc/en-us/articles/115015385887-Contact-Us}{Contact
  Us}
\item
  \href{https://www.nytco.com/careers/}{Work with us}
\item
  \href{https://nytmediakit.com/}{Advertise}
\item
  \href{http://www.tbrandstudio.com/}{T Brand Studio}
\item
  \href{https://www.nytimes3xbfgragh.onion/privacy/cookie-policy\#how-do-i-manage-trackers}{Your
  Ad Choices}
\item
  \href{https://www.nytimes3xbfgragh.onion/privacy}{Privacy}
\item
  \href{https://help.nytimes3xbfgragh.onion/hc/en-us/articles/115014893428-Terms-of-service}{Terms
  of Service}
\item
  \href{https://help.nytimes3xbfgragh.onion/hc/en-us/articles/115014893968-Terms-of-sale}{Terms
  of Sale}
\item
  \href{https://spiderbites.nytimes3xbfgragh.onion}{Site Map}
\item
  \href{https://help.nytimes3xbfgragh.onion/hc/en-us}{Help}
\item
  \href{https://www.nytimes3xbfgragh.onion/subscription?campaignId=37WXW}{Subscriptions}
\end{itemize}
