 **NYTimes.com no longer supports Internet Explorer 9 or earlier. Please
upgrade your browser.
\href{http://www.nytimes3xbfgragh.onion/content/help/site/ie9-support.html}{LEARN
MORE »}

**Sections

**Home

**Search

\hypertarget{the-new-york-times}{%
\subsection{\texorpdfstring{\href{http://www.nytimes3xbfgragh.onion/}{The
New York Times}}{The New York Times}}\label{the-new-york-times}}

\hypertarget{-magazine-}{%
\subsubsection{\texorpdfstring{
\href{https://www.nytimes3xbfgragh.onion/section/magazine}{Magazine}
}{ Magazine }}\label{-magazine-}}

 \href{https://www.nytimes3xbfgragh.onion/section/magazine}{Magazine}
\textbar{}Nicolas Cage on Acting, Philosophy and Searching for the Holy
Grail

**Close search

\hypertarget{site-search-navigation}{%
\subsection{Site Search Navigation}\label{site-search-navigation}}

Search NYTimes.com

**Clear this text input

Go

\url{https://nyti.ms/2TaLE0S}

\hypertarget{site-navigation}{%
\subsection{Site Navigation}\label{site-navigation}}

\hypertarget{site-mobile-navigation}{%
\subsection{Site Mobile Navigation}\label{site-mobile-navigation}}

\hypertarget{nicolas-cage-on-acting-philosophy-and-searching-for-the-holy-grail}{%
\section{Nicolas Cage on Acting, Philosophy and Searching for the Holy
Grail}\label{nicolas-cage-on-acting-philosophy-and-searching-for-the-holy-grail}}

``I wanted to have the mystery of the old stars, always preserved in an
enigmatic aura.''

\includegraphics{https://static01.graylady3jvrrxbe.onion/newsgraphics/2019/06/07/talk-template/36062325fcedca2f7ba51c94d7cc8b08b37b34e2/close.svg}

\textbf{Talk} Aug. 7, 2019

Nicolas Cage on his legacy, his philosophy of acting and his
metaphorical --- and literal --- search for the Holy Grail.

By David Marchese Photograph by Mamadi Doumbouya

There are questions I've wanted to ask Nicolas Cage for years. A lot of
questions. I wanted to know why this divisive, mercurial actor has waged
a career-long, one-man war against naturalism, refusing to let staid
ideas about how people might behave in ``real life'' dictate his
performances. I wanted to know why Cage, Hollywood's greatest
surrealist, whose personal and creative unpredictability has led him to
attain near-mythological status in certain corners of the internet, acts
in so many movies --- 20 in the last two years --- and why so few of
them make mainstream ripples. (His most recent release: the
straightforwardly titled thriller ``A Score to Settle.'') But mostly I
wanted to know the method behind the seeming madness that informs so
many of his performances.

Unlike most movie stars --- who are walking answers, machines who
reliably fill expectations rather than confound them --- Nicolas Cage
rarely does the obvious thing, whether in his choice of roles or how he
plays them. Which is what's so enthralling about the alien intensity and
oddball flourishes that Cage has brought to art-house fantasia (``Wild
at Heart''), whimsical romantic comedy (``Honeymoon in Vegas''), bleak
drama (``Leaving Las Vegas''), cerebral comedy (``Adaptation''),
sensational Hollywood blockbuster (``Con Air''), balletic, high-concept
action (``Face/Off''), quiet character studies (``Joe'') and psychedelic
horror (``Mandy''). He also, as if living according to lines from a
surreal folk song, has owned pet cobras and castles, was forced to
return a stolen dinosaur skull, has made and lost a fortune and is
keeping a pyramid waiting for him --- as a tomb --- down in New Orleans.

But why has he done all of these things? I sat wondering in a private
room at a small Italian restaurant not far from the Las Vegas Strip.
Cage is a walking why, a performer who sees possibilities in art --- and
maybe life --- that no one else does. And then the door to the room
swung open and in flew Cage, hopefully to provide some answers.

In person, he's tall, with a jangly energy complemented, on this day, by
oversize sunglasses, a dragon ring the size of a walnut and a black
velveteen jacket over a Bruce Lee T-shirt. He explained that he'd been
busy preparing for a trip to TIFF. Not the Toronto International Film
Festival, he clarified --- the one in Transylvania. ``I can't pretend to
know what people think or want to think about me,'' he said. ``I'm not
Stravinsky, I'm not Van Gogh, I'm not Monk, but these people were not
understood, and my favorite artists were misunderstood.'' Then he
scanned the menu and asked, in a bemused tone that suggested he was
simultaneously questioning the waitress, me and the universe: ``Could I
get into the branzino?'' Yes, and everything else too.

\textbf{With any movie star, there's the actor, and then there's the
persona. Earlier in your career, you had an obvious interest in
cultivating the latter. Do you still?} I once had brunch with Warren
Beatty, and I said, ``Do you have any idea how lucky you are that you
were Warren Beatty in the '70s, before everyone had a cellphone with a
video camera?'' He just smiled. It's so true. You go to a karaoke bar
with a male friend in the neighborhood, the bar says ``no videotaping''
and suddenly, there's two different videos of you doing karaoke. Who did
that? Who exposed the videotape? Who sold it?

\textbf{You're talking about the clips that went around of you singing
``Purple Rain.''} Yeah. It was around the anniversary of Prince's
passing. Everyone knows how much I admire him as an artist. But
honestly, I wasn't even doing that to sing. It was more like
primal-scream therapy. It was a holiday weekend, and I didn't want to go
anywhere, but my friend who was with me said: ``You can't sit here in
your apartment. You've got to go out.'' So I went to the one place in my
neighborhood that I knew had no video recording, just to have some fun,
and that became everybody's business.

\textbf{What were you primal-screaming about?} I have to be careful
about what I can divulge. There was a recent
\href{http://nytimes3xbfgragh.onion\#tooltip-1}{breakup.1} I don't
really want to talk about it. I was pretty upset about that and the way
things happened. To answer your question, earlier in my career I was
very specific in my concept of who I wanted to be. I saw myself as a
surrealist. This is going to sound pretentious, but I was, quote, trying
to invent my own mythology, unquote, around myself.

\textbf{Has that mythology shaped the perception of your work? I mean,
you went on ``Letterman'' and talked about your pet cobras wanting to
kill you and about getting high on mushrooms with your cat. You were
clearly trying to project a certain image.} I know what you're asking,
and it's a good question. But let me say one thing: I am completely
antidrug. I don't do drugs. I don't drink when I work. Sometimes in
between movies I'll have some drinks, but not always. I make so many
characters, and I go so internal with them, that sometimes, when I'm not
filming, wine or Champagne is like an eraser to a chalkboard. You can
erase the character and make a clean slate so you can start making a new
character. I hope that makes sense.

\textbf{Yeah, it does.} O.K. So those stories that you mentioned, those
are true stories. I did have two king cobras, and they were not happy.
They would try to hypnotize me by showing me their backs, and then
they'd lunge at me. After I told that story on ``Letterman,'' the
neighborhood wasn't too pleased that I had cobras, so I had them
re-homed in a zoo. The cat --- a friend of mine gave me this bag of
mushrooms, and my cat would go in my refrigerator and grab it, almost
like he knew what it was. He loved it. Then I started going, ``I guess
I'll do it.'' It was a peaceful and beautiful experience. But I
subsequently threw them out.

\textbf{Have animals ever influenced your acting?} The cobras,
definitely. They would try to hypnotize you by going side to side, and
when I did ``Ghost Rider: Spirit of Vengeance,'' that's something my
character does before he attacks. Animals are fun places to get
inspiration. Actually, I thought Heath Ledger was doing some reptilian
stuff as the Joker, with the tongue darting out all the time.

\textbf{I can't think of another actor whose performances so frequently
pay homage to other actors and movies. In ``Mandy,'' there's a scene in
which you snort angel dust or whatever it is and then give the camera a
look that's similar to one Bruce Lee used
\href{http://nytimes3xbfgragh.onion\#tooltip-2}{to give.2} In
``Moonstruck,'' there's a moment where you put on a glove that's
intended as a nod to ``Metropolis.'' There were Max
Schreck-in-``Nosferatu'' nods in
\href{http://nytimes3xbfgragh.onion\#tooltip-3}{``Vampire's
Kiss.''}\href{http://nytimes3xbfgragh.onion\#tooltip-3}{3} The list goes
on.} It speaks to my truth as a film enthusiast. It's also that these
are moments that I know work. Right before I snorted that stuff in
``Mandy,'' I asked the director to look at that Bruce Lee shot. I said,
``Is it going to work?'' And he said, ``It already has worked.'' That's
what I mean. I knew it would be satisfying. And when I saw that movie
with an audience, they erupted at that moment.

\textbf{What happens when your director isn't interested in your
experiments?} As a film actor, my job is to facilitate the director's
vision. If there's something I'm doing that they don't agree with, I
drop it.

\textbf{Always?} In the beginning, there were examples of locking horns.
``Raising Arizona'': Perhaps there was confusion about where I was
going, but the Coen brothers went along. They didn't mind that I was
channeling \href{http://nytimes3xbfgragh.onion\#tooltip-4}{Woody
Woodpecker.4} They, on some level, got it. With Francis, he didn't. I
didn't want to make that movie.

\textbf{You're talking about
\href{http://nytimes3xbfgragh.onion\#tooltip-5}{Francis Ford Coppola5}
and ``Peggy Sue Got Married.''} Yeah, I didn't want to make that movie.
I must have said no five or six times. I said, ``Uncle, why do you want
to make this movie at all?'' He said, ``It's like `Our Town'!'' By the
way, I couldn't stand ``Our Town.'' I had bad memories about ``Our
Town.'' In high school I was cast as Constable Warren, and Jon
Turteltaub, who later directed me in ``National Treasure,'' got the
lead. He never let me forget it. And I just don't like the play. It's a
Norman Rockwell borefest. So I said, ``Uncle, I don't want to be in `Our
Town.' '' He said, ``Just come to rehearsal.'' I said, ``Look, I'll do
it if you let me go really far out with the character.'' ``How far
out?'' ``I want to talk like Pokey from `The Gumby Show.' '' So I went
to rehearsal, and everybody was rolling their eyes because I was talking
like that, and my co-star Kathleen Turner was very upset, because she
wanted me to be Al, my character from ``Birdy,'' and instead she got
Jerry Lewis on psychedelia. It did not go over well. In fact, Ray Stark
from Tri-Star flew up to fire me, and thankfully Uncle went to bat and
said, ``Young Nicky's doing this.' '' But needless to say, I never
worked for them again after that.

\textbf{There are times, I think, when it can feel as if your
performances are vibrating at a different frequency than the movies in
which they appear.} Can you give me an example?

\textbf{A couple of recent ones come to mind. You did a movie called
``Rage.'' Or was it called ``Tokarev''?} It was originally ``Tokarev,''
and they switched it to ``Rage.''

\textbf{There's a confrontation in that movie between your character and
an informant when you scream, ``You talked!'' And you hold a scream on
the word ``talked'' for maybe 5 percent longer than feels normal within
the context of an otherwise down-the-middle movie.} I was sustaining a
vocal sound there, because I was trying to play with
\href{http://nytimes3xbfgragh.onion\#tooltip-6}{Stockhausen}\href{http://nytimes3xbfgragh.onion\#tooltip-6}{6}
and mess up the EQ of my vocal.

\textbf{You're really saying that in that moment you wanted to achieve a
Stockhausen effect? This is not a rationalization you've come up with
after?} Yeah, Stockhausen.

\textbf{O.K., here's another: In ``Army of One,'' you use this nasal
voice that doesn't at all sound like the real-life person your character
was based on. It's choices like that that almost make it seem as if
you're offering a meta-comment on acting somehow.} Well, there are times
when I'm intentionally being mischievous with a character.
\href{http://nytimes3xbfgragh.onion\#tooltip-7}{``The Wicker
Man''}\href{http://nytimes3xbfgragh.onion\#tooltip-7}{7} is me playing
with the situation because it's so absurd. I could have had a little
more help with that film. Initially I wanted them to leave me in the
bear suit to burn me. That would have made the whole farce of the film
more disturbing. Because of what I was trying to do there. Do you
remember an old movie by Roger Corman called ``The Masque of the Red
Death''?

\textbf{Yeah, with Vincent Price.} Vincent Price and Patrick Magee.
Patrick Magee gets tricked into wearing an ape suit, and a dwarf throws
brandy on it and lights him on fire. What began as absurd and comical
became horrifying because insult was added to injury. In ``The Wicker
Man,'' I was trying to get this whole trajectory to go along with the
absurdity by having them light me on fire in the bear suit. That really
would have been horrific. But ``Army of One,'' I have umbrage with that.
My friend Charlie and his dad, Martin, they watched that, and they think
it's the most hilarious thing.

\textbf{Are you talking about the Sheens?} Yeah. I'm very upset about
that movie, because I know what it could have been. I know what we shot,
and they took it all out.
\href{http://nytimes3xbfgragh.onion\#tooltip-8}{Bob Weinstein8} got hold
of it and cut the whole thing. Larry Charles, the director, was
disappointed, and I was disappointed. There were moments in that movie
that were shocking and irreverent. I wish I could see them.

\textbf{Let me ask you something unrelated: Is it true that in the early
'80s you met Johnny Depp playing Monopoly?} The true story is that we
were already friends. I was living in an old building in Hollywood
called the Fontenoy, and I think I ultimately rented the apartment to
Johnny, and he started living there. He was at the point in his career
where he was selling pens or something to get by. He would take my money
and buy cocktails but wouldn't tell me about it. He admitted it later.
But anyway, we were good friends, and we would play Monopoly, and he was
winning a game, and I was watching him and I said, ``Why don't you just
try acting?'' He wanted to be a musician at the time, and he told me,
``No, I can't act.'' I said, ``I think you can act.'' So I sent him to
meet with my agent. She sent him out on his first audition, which was
``A Nightmare on Elm Street.'' He got the part that day. Overnight
sensations don't happen. But it happened with him.

\textbf{In the past you've talked about your acting in terms of specific
styles you'd developed: nouveau shamanic and Western Kabuki. Do you
think about your acting in that way now?} Yeah, I do. Laurence Olivier
said, ``What is acting but lying, and what is good lying but convincing
lying?'' I don't want to look at acting that way. Why not experiment?
Western Kabuki to me was, let's go all the way out. Nouveau shamanic is
nothing other than trying to augment your imagination to get to the
performance without feeling like you're faking it. This author Brian
Bates wrote a book called ``The Way of Wyrd,'' and he put forth the
notion that actors hailed from the old shamans. So I was kind of making
a statement about that, and I added ``nouveau'' to be fancy.

\textbf{Could you teach nouveau shamanic acting?} I put this line in
``Mandy'': ``The psychotic drowns where the
\href{http://nytimes3xbfgragh.onion\#tooltip-9}{mystic
swims.''}\href{http://nytimes3xbfgragh.onion\#tooltip-9}{9} You either
have the proclivity to open up your imagination or you don't. If you
have that propensity and are on camera about to do a scene, what would
make you believe in what you're about to do? Say you're playing a demon
biker with an ancient spirit. What power objects could you find that
might trick your imagination? Would you find an antique from an ancient
pyramid? Maybe a little sarcophagus that's a greenish color and looks
like King Tut? Would you sew that into your jacket and know that it's
right next to you when the director says ``action''? Could you open
yourself to that power?

\textbf{Those aren't rhetorical questions, are they?} Right. I did that.

\textbf{I would hope there are ways of teaching nouveau shamanic acting
that don't involve acquiring ancient artifacts.} True. There are other
ways. What is a poem that you like? You could take that poem and write
it out by hand on paper, then fold it up and put it in your pocket. The
trigger doesn't have to be something that's extraordinarily expensive.

\textbf{You grew up middle-class among a lot of rich people, right?}
\href{http://nytimes3xbfgragh.onion\#tooltip-10}{Dad}\href{http://nytimes3xbfgragh.onion\#tooltip-10}{10}
was a professor. He was teaching at California State Long Beach, then he
was writing books. We lived modestly. We were on the outskirts of
Beverly Hills, right next to the Porsche dealer. I would take the bus to
school, and some of the older boys were going to school in Maseratis and
Ferraris. I felt that because of my name being Coppola, there was a
misunderstanding as to what I did and didn't have. So it was frustrating
for me because, like any other young man, I was interested in dating and
wanting to be impressive, and I didn't know how to do that taking a
young lady out on the bus while the other guys were taking her out in
Ferraris. But my uncle was very generous. I would visit him for summers,
and those summers --- I wanted to be him. I wanted to have the mansions.
That was driving me.

\textbf{Did being a young man who was insecure about money color your
attitude about buying things and what success looks like?} You have good
investments and bad investments. The good investments came from personal
interest and my honest enjoyment of the history. For example, Action
Comics No. 1: I bought that for
\href{http://nytimes3xbfgragh.onion\#tooltip-11}{\$150,000.}\href{http://nytimes3xbfgragh.onion\#tooltip-11}{11}
Then it was stolen. I got it back and sold it for \$2 million. But that
was a good thing to have, because I had an interest that was sincere.
The funny thing is, my real estate
\href{http://nytimes3xbfgragh.onion\#tooltip-12}{buying spree12} was
what the \href{http://nytimes3xbfgragh.onion\#tooltip-13}{real
problem13} was. It wasn't these other things like shrunken heads that
the media liked to talk about.

\textbf{Or that
\href{http://nytimes3xbfgragh.onion\#tooltip-14}{dinosaur
skull?}\href{http://nytimes3xbfgragh.onion\#tooltip-14}{14}} Or an
octopus. What is an octopus, \$80? You're not going to go into dire
straits buying an octopus. The dinosaur skull was an unfortunate thing,
because I did spend \$276,000 on that. I bought it at a legitimate
auction and found out it was abducted from Mongolia illegally, and then
I had to give it back. Of course it should be awarded to its country of
origin. But who knew? Plus, I never got my money back. So that stank.
But I went years where all I was doing was meditating three times a day
and reading books on philosophy, not drinking whatsoever. That was the
time when I almost went on --- you might call it a grail quest. I
started following mythology, and I was finding properties that aligned
with that. It was almost like
\href{http://nytimes3xbfgragh.onion\#tooltip-15}{``National
Treasure.''}\href{http://nytimes3xbfgragh.onion\#tooltip-15}{15} Of
course, that didn't sustain. On top of which, I said, ``I'm going to get
off philosophy,'' because I became like a kite with a string but no
anchor. No one could understand what I was talking about. And I thought
people would rather see me as an orangutan than as an eagle meditating
on the mountaintop anyway.

\textbf{Wait, what did you mean when you said you were on a grail quest
and finding properties that aligned with that?} One thing would lead to
another. It's like when you build a library. You read a book, and in it
there's a reference to another book, and then you buy that book, and
then you attach the references. For me it was all about where was the
grail? Was it here? Was it there? Is it at
\href{http://nytimes3xbfgragh.onion\#tooltip-16}{Glastonbury?}\href{http://nytimes3xbfgragh.onion\#tooltip-16}{16}
Does it exist?

\textbf{Oh, O.K. I thought you were being metaphorical about going on a
grail quest.} Yeah, if you go to Glastonbury and go to the Chalice Well,
there's a spring that does taste
\href{http://nytimes3xbfgragh.onion\#tooltip-17}{like
blood.}\href{http://nytimes3xbfgragh.onion\#tooltip-17}{17} I guess it's
really because there's a lot of iron in the water. But legend had it
that in that place was a grail chalice, or two cruets rather, one of
blood and one of sweat. But that led to there being talk that people had
come to Rhode Island, and they were looking for something
\href{http://nytimes3xbfgragh.onion\#tooltip-18}{as well.18}

\emph{\textbf{That's}} \textbf{why you bought property in Rhode Island?}
I don't know if I'm going to say that's why I bought the Rhode Island
property. But I will say that is why I went to Rhode Island, and I
happened to find the place beautiful. But yes, this had put me on a
search around different areas, mostly in England, but also some places
in the States. What I ultimately found is: What is the Grail but Earth
itself?

\textbf{I find that grail quests tend to be more fulfilling when they're
metaphorical.} Well, I knew that, and the metaphor for me is the earth.
The divine object is Earth.

\textbf{What's your grail quest now?} There's this old sci-fi movie
called ``Quatermass and the Pit.'' In the movie, someone says to
Professor Quatermass, ``Do you ever find your early career catching up
with you?'' And he says, ``I never had a career, only work.'' I feel
like that's where I'm at now. I never had a career, only work. I'm just
going to keep working.

\textbf{Why do you work
\href{http://nytimes3xbfgragh.onion\#tooltip-19}{so
much?}\href{http://nytimes3xbfgragh.onion\#tooltip-19}{19} You've said
you wanted to make 150 films.} That's me speaking to my golden-age
heroes. Those guys all did like 150. I also want to argue with the
concept of supply and demand. I grew up in the '70s watching Rock Hudson
on ``McMillan \& Wife,'' Dennis Weaver in ``McCloud,'' Charles Bronson
in the movie of the week, Peter Falk in ``Columbo.'' I began to develop
a relationship with these characters and these actors. The more I saw
them, the more I wanted to see them. And by design, with video on
demand, I felt that if I made more movies, not only was it good for me
financially, people would be able to tune in at home and go, ``What's
the next movie that Nick made?'' They'd have a large selection. So I'm
not worried about too much supply and not enough demand. I'm just trying
to get back to a feeling that I enjoyed as a child on my little Zenith
television in the '70s.

\textbf{I don't want to dance around this: How much has money driven
your work choices?} I can't go into specifics or percentages or ratios,
but yeah, money is a factor. I'm going to be completely direct about
that. There's no reason not to be. There are times when it's more of a
factor than not. I still have to feel that, whether or not the movie
around me entirely works, I'll be able to deliver something and be fun
to watch. But yes, it's no secret that mistakes have been made in my
past that I've had to try to correct. Financial mistakes happened with
the real estate implosion that occurred, in which the lion's share of
everything I had earned was pretty much eradicated. But one thing I
wasn't going to do was file for bankruptcy. I had this pride thing where
I wanted to work my way through anything, which was both good and bad.
Not all the movies have been blue chip, but I've kept getting closer to
my instrument. And maybe there's been more supply than demand, but on
the other hand, I'm a better man when I'm working. I have structure. I
have a place to go. I don't want to sit around and drink mai tais and
Dom Pérignon and have mistakes in my personal life. I want to be on set.
I want to be performing. In any other business, hard work is something
to behold. Why not in film performance?

\textbf{Are there things about you and your work that people don't get?}
For an actor to say, ``I want to try something else,'' is a challenging
road to take. I can't worry that people aren't going to get it. I think
the movies have matured well, ``Lord of War'' or ``Peggy Sue Got
Married.'' ``Raising Arizona,'' I knew that my cartoonish behaviorisms
would play well. ``Vampire's Kiss'' is still on the fence, but I'm happy
with those results. I've taken risks. But there has been a collision
between the acting experiments and the memeification extrapolated from
them. That has not been intentional. I have no social media presence.
I'm not on Instagram. I am not on Facebook. I have no Twitter account. I
genuinely am a private person who does not want his personal life
exposed. I wanted to have the mystery of the old stars, always preserved
in an enigmatic aura. It's hard to do that now.

\textbf{Your aura is plenty enigmatic to me.} At this point in my life,
David, I heavily prefer to not go out. I'd rather just stay at home. I
don't think I can decompress ever again, even at a karaoke bar. It's too
vulnerable. I'm not trying to complain. It's a fact of life that I have
to accept. I'd much rather let my work and not my personal life speak
for me. Rob Zombie once said to me, ``Be as normal in your own life as
you can be, so you can be as messed up as you want in your art.''

\textbf{I think Rob Zombie took that from
\href{http://nytimes3xbfgragh.onion\#tooltip-20}{Flaubert.20}} That's
what I want. I want to be on the
\href{http://nytimes3xbfgragh.onion\#tooltip-21}{Axl
Rose}\href{http://nytimes3xbfgragh.onion\#tooltip-21}{21} program. I
don't want to go anywhere. I just want to look at my aquarium, look at
my sea horse, read my Murakami, watch Bergman. I've been on a great
Bergman run. I just saw ``The Virgin Spring,'' ``Hour of the Wolf,''
``Persona.'' I also love Tarkovsky. I love ``The Sacrifice.'' I looked
at ``Stalker'' again. I have all the time in the world in between movies
to lose myself in these maestros' films.

\textbf{Your acting has gone through very distinct phases. Near the
beginning of your career you were trying radical things; then you had
your \href{http://nytimes3xbfgragh.onion\#tooltip-22}{``sunshine
trilogy''}\href{http://nytimes3xbfgragh.onion\#tooltip-22}{22} of
whimsical comedies; and after that you tried to bring weirdness to
big-budget Hollywood
\href{http://nytimes3xbfgragh.onion\#tooltip-23}{action films.23} What
phase are you in now?} Well, what you described is a true reflection on
everything I was experimenting with. My roots, though, were in
independently spirited cinema. Movies like ``Raising Arizona,''
``Vampire's Kiss,'' ``Birdy.'' That's my base, and my journey has been
about getting back to that with movies like ``Joe'' or ``Mandy.''

\textbf{How do you define good acting and bad acting?} It can be a very
blurry line. I've seen some horrible acting that I think is wonderful.

\textbf{Like what?} Well, it cracks me up, and I don't want to mention
names, but in film acting you can do things that seem erratic or out of
touch or not in sync, but it's a valid stylization as long as you anchor
it within the context of the character and situation. When you listen to
Stockhausen's ``Punkte,'' or ``Stimmung'' or ``Mantra'' --- it's all
these voices and quick, snappy chords that seem discordant to a point
and as if they don't make any sense. Yet it is of a piece and does
belong together. Similarly, you can read a script and go: ``Why would a
character do that? That doesn't make any sense.'' But people are like
weather vanes. We don't always blow in the same direction. Sometimes you
do things that there's no explanation for other than that we're human,
and that can apply within a performance. So can I get back to your
question?

\textbf{Yeah.} What is good acting? What is bad acting? Olivier had his
argument, but look at James Cagney in ``White Heat.'' ``Made it, Ma! Top
of the world!'' That's not ``real.'' But is it fun to watch? Is it
exciting? Is it truthful? Yeah, and to me, that is great acting. It's a
matter of which paintbrush you want to work with. I can look at TV
commercials and see cringeworthy acting, and it makes me laugh, and I'm
probably going to wind up putting it in one of my performances. I mean,
I've done it.

\textbf{When?} No offense to John Stamos, because he's a beautiful man
and a lot of fun to watch on camera, but a million years ago he did a
commercial for L'eggs pantyhose. In it he said, ``I
\href{http://nytimes3xbfgragh.onion\#tooltip-24}{\emph{love} L'eggs
pantyhose!''}\href{http://nytimes3xbfgragh.onion\#tooltip-24}{24} And
the way he went ``love'' --- he expressed it with almost a rock 'n' roll
screech. I saw that commercial, and I had to put it in ``Peggy Sue Got
Married.'' I was playing Charlie Bodell, and I'm with Kathleen Turner,
and I said: ``I'm in \emph{love} with you.'' I've told John about this.
He took the compliment.

\textbf{Do you know when you're good or bad?} I have a pretty good
barometer of when I'm on point. Which is why I can tell you right now
without fear of seeming like I'm boasting that I'm at the top of my game
by virtue of the fact that I've been practicing so much.

\textbf{What's the interplay of sincerity and irony in your work?
Sometimes it feels as if the almost operatic sincerity you're going for
--- in your death scene in ``Kick-Ass'' for example --- comes back
around to irony.} It's a great observation, and this is something that I
have put a lot of thought in to. I have gone out of my way not to be
ironic and --- with the risk of looking ridiculous --- to be genuinely
emotionally naked. And that gets uncomfortable. There were times where
people saw ``Mandy,'' and I was having to break down in a scene, and
people were laughing. They don't know how to handle it. But that's not
ironic, that's naked, which is embarrassing for people.

\textbf{Are the things that make you a great actor, like an inclination
toward risk or emotional abandon, ever a problem in your personal
relationships?} I think there has to be some unusualness to be able to
be in a \href{http://nytimes3xbfgragh.onion\#tooltip-25}{relationship
with me.}\href{http://nytimes3xbfgragh.onion\#tooltip-25}{25} I feel
things very deeply. I have had melancholia my whole life. I am sensitive
to my environment. I have to be in order to do what I do. And I can't
just go to a pharmacy and say, ``Hey, let me have some Prozac.'' I can't
do it, because that would put my instrument at risk. If I can't inform
the dialogue with genuine emotional content I will be a phony on camera,
and I don't want to be that.

\textbf{Have you ever done therapy?} I haven't been in any kind of
analysis for at least 20 years. The times that I've done it, there were
some benefits. It's kind of like writing in a diary. You get things out.
However, inevitably, there was a point where I'd look at the person and
I'd start to go: ``Why am I talking to you? I'm more interesting than
you.'' Then I'd get up and walk out. So I stopped going.

\textbf{You've obviously stayed introspective, though.} It's more from
when I studied philosophy. It was pretty esoteric stuff I was reading:
William Blake, Jakob Böhme, Dion Fortune, Paracelsus, Emerson's
``Self-Reliance.'' There was a time when I was at a crossroads, and I
think I would've been perfectly happy just meditating and reading and
thinking and contemplating what we're all doing here, but you can't do
that as a film star. It was a challenge. I didn't want to do ``The
Taking of Pelham 123'' with Denzel Washington simply because Sony
Pictures had a script where I had to shoot Denzel's character in the
back. I didn't want to do that, because it didn't jibe with where I was
at in terms of my life of contemplation. I became out of the Hollywood
element and started living a life of introspection, and that doesn't
work if you want to be a film actor.

\textbf{I know that when you were growing up, you felt a sense of being
different from other kids. Do you still feel alienated at 55?} Not as
much, but yes. As a child I was always shocked when my father would take
me to a doctor and they didn't tell me that my blood was green and I had
20 ribs, that I wasn't some anomaly from outer space. But I've become
extremely bored with myself. I spend my time watching movies or reading
books and being with \href{http://nytimes3xbfgragh.onion\#tooltip-26}{my
son.}\href{http://nytimes3xbfgragh.onion\#tooltip-26}{26} That's about
it, with the occasional hiccup.

\textbf{Do you think your talent ebbs and flows depending on the
material?} Elia Kazan said talent never dies. It can be discouraged, but
it never dies. I also like to use the words \emph{genius loci.} My
ability coalesces with the genius of a place. I've made very good movies
in Las Vegas; there's a \emph{genius loci} that is a good match for me.
New Orleans has a \emph{genius loci.}

\textbf{That city must have special meaning for you. It's where
\href{http://nytimes3xbfgragh.onion\#tooltip-27}{your tomb
is.}\href{http://nytimes3xbfgragh.onion\#tooltip-27}{27}} I became a man
in New Orleans, if you know what I mean. The city has a soft spot in my
heart, though there are things that can go horribly wrong there.

\textbf{At this point in your career, do you still have something like a
dream role?} Captain Nemo. My first love, even before my parents, was
the ocean. When I read Jules Verne's ``Twenty Thousand Leagues Under the
Sea,'' the depiction of Nemo was that he was also in love with the
ocean. He had freedom, and he lived in a palace that was also a
submarine, playing the organ. To me, that was a beautiful life.

\textbf{How do you think your life's work will be remembered?} I think
time is a friend. Many of my movies that were mocked are enjoying a
renaissance. So I'm hopeful that time will be on my side.

David Marchese is the magazine's Talk columnist.

*This interview has been edited and condensed from two conversations.

*Annotations: Lee: Corbis, via Getty Images. Cage: Hemdale/Everett
Collection. Woody Woodpecker: Walter Lantz Productions/Photofest. Action
Comics: Timothy A. Clary/Agence France-Presse --- Getty Images.
Tyrannosaurus Bataar: Byambasuren Byamba-Ochir/Agence France-Presse ---
Getty Images. Glastonbury: David Lefranc, via Getty Images. Rose: Kevin
Mazur/WireImage, via Getty Images. Cage and Travolta: Paramount
Pictures/Everett Collection. Stamos: Screen grab from YouTube. Tomb:
Beth J. Harpaz/Associated Press.

\hypertarget{more-on-nytimescom}{%
\subsection{More on NYTimes.com}\label{more-on-nytimescom}}

Advertisement

\hypertarget{site-information-navigation}{%
\subsection{Site Information
Navigation}\label{site-information-navigation}}

\begin{itemize}
\tightlist
\item
  \href{https://help.nytimes3xbfgragh.onion/hc/en-us/articles/115014792127-Copyright-notice}{©
  2020 The New York Times Company}
\item
  \href{https://www.nytimes3xbfgragh.onion}{Home}
\item
  \href{https://www.nytimes3xbfgragh.onion/search/}{Search}
\item
  Accessibility concerns? Email us at
  \href{mailto:accessibility@NYTimes.com}{\nolinkurl{accessibility@NYTimes.com}}.
  We would love to hear from you.
\item
  \href{https://help.nytimes3xbfgragh.onion/hc/en-us/articles/115015385887-Contact-Us}{Contact
  Us}
\item
  \href{https://www.nytco.com/careers/}{Work with us}
\item
  \href{https://nytmediakit.com/}{Advertise}
\item
  \href{https://help.nytimes3xbfgragh.onion/hc/en-us/articles/115014892108-Privacy-policy\#pp}{Your
  Ad Choices}
\item
  \href{https://help.nytimes3xbfgragh.onion/hc/en-us/articles/115014892108-Privacy-policy}{Privacy}
\item
  \href{https://help.nytimes3xbfgragh.onion/hc/en-us/articles/115014893428-Terms-of-service}{Terms
  of Service}
\item
  \href{https://help.nytimes3xbfgragh.onion/hc/en-us/articles/115014893968-Terms-of-sale}{Terms
  of Sale}
\end{itemize}

\hypertarget{site-information-navigation-1}{%
\subsection{Site Information
Navigation}\label{site-information-navigation-1}}

\begin{itemize}
\tightlist
\item
  \href{https://spiderbites.nytimes3xbfgragh.onion}{Site Map}
\item
  \href{https://help.nytimes3xbfgragh.onion/hc/en-us}{Help}
\item
  \href{https://help.nytimes3xbfgragh.onion/hc/en-us/articles/115015385887-Contact-Us?redir=myacc}{Site
  Feedback}
\item
  \href{https://www.nytimes3xbfgragh.onion/subscription?campaignId=37WXW}{Subscriptions}
\end{itemize}
