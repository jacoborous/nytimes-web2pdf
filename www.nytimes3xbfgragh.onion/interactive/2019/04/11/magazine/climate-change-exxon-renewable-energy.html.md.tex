 **NYTimes.com no longer supports Internet Explorer 9 or earlier. Please
upgrade your browser.
\href{http://www.nytimes3xbfgragh.onion/content/help/site/ie9-support.html}{LEARN
MORE »}

**Sections

**Home

**Search

\hypertarget{the-new-york-times}{%
\subsection{\texorpdfstring{\href{http://www.nytimes3xbfgragh.onion/}{The
New York Times}}{The New York Times}}\label{the-new-york-times}}

\hypertarget{-magazine-}{%
\subsubsection{\texorpdfstring{
\href{https://www.nytimes3xbfgragh.onion/section/magazine}{Magazine}
}{ Magazine }}\label{-magazine-}}

 \href{https://www.nytimes3xbfgragh.onion/section/magazine}{Magazine}
\textbar{}How Big Business Is Hedging Against the Apocalypse

**Close search

\hypertarget{site-search-navigation}{%
\subsection{Site Search Navigation}\label{site-search-navigation}}

Search NYTimes.com

**Clear this text input

Go

\url{https://nyti.ms/2X4aUa0}

\hypertarget{site-navigation}{%
\subsection{Site Navigation}\label{site-navigation}}

\hypertarget{site-mobile-navigation}{%
\subsection{Site Mobile Navigation}\label{site-mobile-navigation}}

\hypertarget{how-big-business-is-hedging-against-the-apocalypse}{%
\section{How Big Business Is Hedging Against the
Apocalypse}\label{how-big-business-is-hedging-against-the-apocalypse}}

Investors are finally paying attention to climate change --- though not
in the way you might hope.

The Climate Issue

\begin{itemize}
\tightlist
\item
  \href{https://www.nytimes3xbfgragh.onion/interactive/2019/04/11/magazine/climate-change-bangladesh-scavenging.html}{What
  Survival Looks Like After the Oceans Rise}
\item
  \href{https://www.nytimes3xbfgragh.onion/interactive/2019/04/10/magazine/climate-change-pinkertons.html}{Climate
  Chaos Is Coming --- and the Pinkertons Are Ready}
\item
  \href{https://www.nytimes3xbfgragh.onion/interactive/2019/04/11/magazine/climate-change-exxon-renewable-energy.html}{How
  Big Business Is Hedging Against the Apocalypse}
\item
  \href{https://www.nytimes3xbfgragh.onion/interactive/2019/04/09/magazine/climate-change-capitalism.html}{The
  Next Reckoning: Capitalism and Climate Change}
\item
  \href{https://www.nytimes3xbfgragh.onion/interactive/2019/04/09/magazine/climate-change-politics-economics.html}{The
  Problem With Putting a Price on the End of the World}
\item
  \href{https://www.nytimes3xbfgragh.onion/interactive/2019/04/09/magazine/climate-change-peru-law.html}{Climate
  Change Could Destroy His Home in Peru. So He Sued an Energy Company in
  Germany.}
\end{itemize}

\protect\hyperlink{}{} \protect\hyperlink{}{}

\includegraphics{https://static01.graylady3jvrrxbe.onion/newsgraphics/2019/04/09/climate2019/51f6a9588f2c17e118fdd8781dfa8806f5d62a95/caret.svg}

\hypertarget{how-big-business-is-hedging-against-the-apocalypse-1}{%
\section{How Big Business Is Hedging Against the
Apocalypse}\label{how-big-business-is-hedging-against-the-apocalypse-1}}

Investors are finally paying attention to climate change --- though not
in the way you might hope.

By JESSE BARRON APRIL 11, 2019

Rex Tillerson stood under a 32-foot pipe organ at the Morton H. Meyerson
Symphony Center in Dallas, explaining how the world worked. It was May
2015, in the middle of an oil-price crash, and Exxon Mobil's earnings
had fallen 46 percent compared with the same quarter the year before.
But Tillerson, then Exxon's chief executive, told his shareholders to be
confident in the future. Oil and gas furnished billions of people,
including the very poor, with cheap, reliable fuel --- a fact not easily
negated by a weak fiscal quarter. ``Our view reflects the reality,''
Tillerson said, ``that abundant energy enables modern life.''

Later that morning, a Capuchin Franciscan friar rose to speak. A
so-called faith-based investor, Michael Crosby belonged to a tight
circle of religious leaders who bought stock in public companies in the
hope of exerting a moral influence on them. While Tillerson, head of one
of the largest oil companies in the world and a power broker in
international geopolitics, was accustomed to ignoring protesters, Crosby
proved more tactical than most. He submitted a motion to appoint a
climate-change expert to Exxon's board, which gave him the floor for
several minutes. Then he laid into Tillerson for having uttered ``not
one word or syllable'' about climate change. He asked why Saudi Arabia
invested in solar panels while Exxon spent nothing. ``You're living out
of the past,'' he told Tillerson.

At Exxon's annual meetings --- as in most rooms where important business
happens --- people speak in the subdued patter of corporate jargon,
language that camouflages the reality it describes. So in the 2,000-seat
auditorium, it would have taken a moment to appreciate the gravity of
what Crosby was actually describing, which was not a few numbers on a
balance sheet but something closer to the fate of the species. Global
energy consumption is rocketing upward every year: The Energy
Information Administration expects it
\href{https://www.eia.gov/todayinenergy/detail.php?id=32912}{to climb
another 28 percent} within a generation. Hydropower, wind and solar
contribute about 22 percent of the total, and their share grows yearly.
But the net amount of energy generated by hydrocarbons is growing
yearly, too. It's all rising because demand is rising. Global
hydrocarbon producers, meanwhile, have so much product in reserve that
burning even half of it would leave us with slightly worse than
heads-or-tails odds of staying under the two-degree-Celsius threshold
that, according to climate models, could bring mass famine, drought,
flooding and fires.

\emph{{[}\href{https://www.nytimes3xbfgragh.onion/interactive/2019/04/10/magazine/climate-change-pinkertons.html}{One
storied American institution is ready for the coming climate chaos: the
Pinkertons}.{]}}

From his spot beneath the pipe organ, Tillerson regarded the friar.
``Like it or not,'' he said, the world would depend on fossil fuels
``for the next several decades'' --- well into the middle of the
century. This was Tillerson's line whenever people asked him about the
future of hydrocarbons: Remind them how dependent they are and paint
alternatives as childlike fantasies. Tillerson said the motion for a
climate-change expert would be defeated. Turning to renewables, he
dismissed them as a sucker's bet. ``Quite frankly, Father Crosby,'' he
said, ``we choose not to lose money on purpose.'' The crowd at the
Symphony Center showered him with applause.

Three years later, an Irishman named Declan Flanagan, chief executive of
the renewables company Lincoln Clean Energy, was addressing his own
shareholders in Copenhagen when he delivered a cryptic announcement.
Lincoln, he said, was going to build a solar farm in the Permian Basin
--- the heart of West Texas oil country --- with funding put up by a
``blue-chip counterparty.'' Flanagan let this hang for a moment in the
room while he breezed through a jargony update on regulatory matters.
Finally he returned to the story. ``I mentioned the blue-chip
counterparty,'' he reminded his listeners. ``\emph{That},'' he said in
his strong Irish accent, ``is Exxon Mobil.''

\subsection{}

Between Exxon's meeting in Dallas and Flanagan's announcement in
Copenhagen, the oil giant had installed a new chief executive ---
Tillerson having exited for a brief sojourn in Washington --- but had
not experienced a change of heart. No decision had been made to execute
a bootleg turn away from hydrocarbons. Exxon's executives, like everyone
in the energy business, had watched as the cost of renewable power
tumbled ever lower in Texas, where a lattice of high-tension power lines
carried electricity from the bright, windy plains of the far West and
the Panhandle to the thirsty cities below. Far from feeling worried,
Exxon saw an opportunity. Fracking is a very electricity-intensive
method of extracting hydrocarbons. By using solar energy for just a
portion of its operations in Texas, Exxon could save on electricity
costs and keep more cash. It could profit by turning renewable power
back into the hydrocarbon power it existed to replace.

Exxon's arrangement in Texas reflects, in miniature, our national state
of indecision about the best approach to climate change. Depending on
whom you ask, climate change doesn't exist, or is an engineering
problem, or requires global mobilization, or could be solved by simply
nudging the free market into action. Absent a coherent strategy,
opportunists can step in and benefit in wily ways from the shifting
landscape. Tax-supported renewables in Texas take coal plants offline,
but they also support oil extraction. Technology advances, but not the
system underneath. Faced with this volatile and chaotic situation, the
system does what it does best: It searches out profits in the short
term.

\textbf{Unlike almost every} other future event, climate change is 100
percent certain to happen. What we don't know is everything else: where,
or how, or when, or what the changes mean for Facebook or Pfizer or
notes of Chinese-government debt. Navigating these thickets of
complexity is theoretically what Wall Street excels at; the industry
prides itself on its ability to price risk for the whole economy, to
determine companies' values based on their likelihood of generating
earnings. But traders are compensated on their quarterly or yearly
performance, not on their distant foresight. It takes confidence to walk
into your boss's office talking about sea levels in Mozambique in 2030,
when your colleague has a reason to short-sell the Turkish lira this
week. Practically no one in the financial system is directly
incentivized in the near term to worry about the biggest risk
conceivable.

The simplest response is to keep investing in companies that, like
Exxon, conduct their business as usual while adapting where they can.
Another response is to forget about the immediate term and go long on
more sustainable bets. Al Gore, for instance, whiles away his hours
running a climate-focused fund called Generation Investment Management.
On a slightly higher plane sit the gigantic banks and mutual funds,
which continue to invest traditionally but use ``climate analytics'' to
see where their portfolios might contain problems, like public utilities
that could be bankrupted by wildfires.

\emph{{[}\href{https://www.nytimes3xbfgragh.onion/interactive/2019/04/09/magazine/climate-change-politics-economics.html}{Read
David Leonhardt on the economics of climate change}.{]}}

Other strategies display more cleverness. Electric vehicles and green
power grids require, for their batteries, valuable minerals and metals.
Spot prices for nickel and cobalt fluctuate by double-digit percentages
on commodities exchanges, while investors eye shares in lithium mines.
Anticipating future food crises, strategists at Merrill Lynch advise
clients to snap up vertical farms and ``smart hydroponics'';
anticipating water shortages, they also recommend investing in Chinese
wastewater-recycling businesses.

As the earth becomes hotter, the air becomes less dense. In June 2017 in
Phoenix, airlines grounded multiple jets because their wings couldn't
achieve lift in the 119-degree heat. Assuming more 119-degree days,
aerospace companies like MTU Aero Engines and Rolls-Royce are
``lightweighting'' some of their machines to adapt. In Australia, an
agribusiness conglomerate waits for family farms to fold for lack of
rainfall, then considers buying their land at a discount. With drought
conditions, the chief executive told The Australian Financial Review
last year, ``we are seeing more opportunities than would have been there
normally.'' A real estate manager in Dallas told a Bloomberg reporter
that he purchased hotels right before Hurricane Harvey to take advantage
of the need for short-term housing, and made a 25 to 30 percent return.
The Harvard endowment has bought up vineyards in California, acquiring
their water rights in the midst of a long drought.

By the middle of the century, the climate of the Southeastern United
States will most likely be tropical, no longer ideal for peach trees but
perfect for the Aedes aegypti species of disease-bearing mosquito. In
response, some investors are going long on firms conducting clinical
trials for dengue and Zika vaccines: One asset manager told me he knew
of multiple ``Zika strategies.'' Pharmaceutical companies foresee robust
demand for antimalarials, products typically confined to poor countries;
they can look forward to a market in the rich parts of the globe. In
Miami, where the expensive neighborhoods lie low near the water, there
may be a wave of ``widespread relocations,''
\href{https://iopscience.iop.org/article/10.1088/1748-9326/aabb32}{researchers
warn}, as the flight from the coast serves to ``gentrify
higher-elevation communities'' like Little Haiti. One study warns that
speculators may already be ``hedging on South Florida's gradual exodus''
to the central and northern parts of the state. In Greenland, mining
companies buy previously useless land rights in order to extract the
minerals that melting ice will shortly expose. In addition to uranium
and molybdenum --- a silvery metal used in steel alloys --- the miners
expect to find rich reserves of oil, which they fully expect to burn.

The Greenland play was best reported by McKenzie Funk, whose 2014 book,
``Windfall,'' profiles the first generation of climate profiteers.
Schemers prowl these pages. A London ``climate-change fund'' invests in
Russian farmland, whose value is expected to spike amid ``drought-fueled
global food crises.'' Betting on the same thing, a former partner of
A.I.G. flies to Sudan to strike a farmland-lease deal with a rebel
general. A former C.I.A. analyst buys ``billions of gallons of water''
in the American Southwest and Australia. An Israeli entrepreneur goes
long on desalination plants (some powered by coal, Funk notes).

What is odd about many of these climate plays, which rely on such
complex assumptions about the future, is how myopic they seem. They
assume that the world will change around a stable, fixed point. American
weather will curdle to such a degree that Tennessee will become an
incubator for malaria, yet Wall Street banks and patent lawyers will
saunter along as usual. Rising oceans will submerge coastal financial
centers beneath several feet of saltwater, yet commodities markets will
pay top dollar for Greenlandic uranium. Taken individually, these
assumptions sound dubious. But as a whole, they mirror what's happening
on Wall Street. Each successive year incinerates the temperature figures
of the previous one, yet the stock market continues to break records.

\textbf{An unsettling fact} of Wall Street today is that some of the
same people who accurately predicted the housing bubble are now
describing another bubble, whose collapse will make the financial crisis
of 2008 look mild. Perhaps the most famous is Jeremy Grantham, a founder
of the Boston-based asset-management firm G.M.O. and a commander of the
British Empire. In 2005, Grantham began to write letters to his
investors saying that the housing market appeared overleveraged; in
2007, he warned of ``the first truly global bubble.'' His latest
prediction overshadows the preceding one. We are, he says, in the midst
of a historic period of mispricing. Because the global economy depends
on hydrocarbons, practically every asset in the world relates in some
way to oil and gas. Grantham believes hydrocarbons will be priced, or
regulated, into submission. In light of that belief, not only oil
companies' stock but practically everything else on the market looks
falsely inflated.

\emph{{[}\href{https://www.nytimes3xbfgragh.onion/interactive/2019/04/09/magazine/climate-change-peru-law.html}{Read
about new legal strategies to make the world's biggest polluters pay for
climate change}.{]}}

In the last few years, Grantham has committed all but 2 percent of his
personal fortune to funding projects --- energy storage, pesticides,
lightweight cars --- that might help save us in the event of two degrees
of warming. In June 2018, he gave a keynote address at an investment
conference in Chicago. The two speeches before Grantham's were called
``Take a Balanced Approach to Sourcing Cash Flows'' and ``Making Sense
of the Multitude of Multifactor ETFs.'' Grantham called his speech
\href{https://www.morningstar.com/videos/870606/watch-jeremy-granthams-race-of-our-lives-speech.html}{``The
Race of Our Lives.''}

``You could call this presentation the story of carbon dioxide and Homo
sapiens,'' he began. Then he spoke for nearly an hour about glacial
runoff, food scarcity and lithium batteries. He explained how a
turbine's efficiency increased exponentially with the length of its
blades. He said that offshore wind farms in the churning North Sea could
soon provide the cheapest power on the planet. He rose to a
techno-utopian pitch, speaking about our obligations to our
grandchildren, decency over profit. Even as he spoke lucidly about
climate change, Grantham represented a tangled and confusing paradox.
Perched as he was at the pinnacle of the market, he was developing an
acute sense of the market's failure to address the problem that most
obsessed him. Yet he continued to help oversee a \$70 billion firm,
which was the main source of his wealth. If anyone was living inside the
tortured contradictions between the market and the climate --- between
our modern economy and its ultimate external cost --- Grantham, I
thought, was the person.

When I knocked on the door of his Beacon Hill townhouse at 8:45 in the
morning on a Friday, the door swung open, and Grantham appeared on the
staircase, 80 and impish, wearing a dark purple sweater over a
pink-and-green striped shirt. ``Were you waiting long?'' he said. I
followed him up into a second-story living room where winter light
flooded the bay windows, falling on a hand-painted Dutch children's
sleigh from the 17th century. Grantham took my coat, then seated me on
the couch. Every so often, a heat pipe hissed somewhere behind me.

When Grantham started his climate fund, the Grantham Foundation, in
1997, many asset managers on Wall Street viewed his work as a fringe
pursuit. ``There was an undercurrent of, `Oh, this is a load of
{[}expletive{]},' '' he said. During the past two years, however, the
cavalcade of hurricanes, droughts, floods and displacements has made it
impossible to maintain the same level of denial in the polite corporate
circles that ring Wall Street. This did not mean that climate-based
investment strategies had become popular. It was the opposite: No one
was willing to risk all of his or her ``career units,'' as Grantham
called them, on climate.

``The problem of doing it accurately is, of course, massive,'' he said,
referring to betting on climate change. ``It's a fast-moving area with
even more uncertainty than an uncertain world. It's the cutting edge of
uncertainty.''

Grantham's fund is going long on lithium and copper, which he believes
will form the vascular system of future renewable-powered supergrids.
His confidence derives from an odd and specific conviction that ``sooner
or later, there will be a carbon tax,'' and much of the market
capitalization of the leading oil and gas companies will be erased.
``You have a certainty,'' he said. ``It will happen. Or we'll be on our
way to a failed civilization.''

It took me a moment to process what this meant. Grantham was saying that
a bet on a future carbon tax was a sure thing because the absence of a
carbon tax meant civilizational catastrophe. If he were right, he could
make billions. If he were wrong, it wouldn't matter, because the world
would be on fire. ``Perfectly fine logic,'' Grantham said, as the old
radiators gurgled around him.

If Grantham's logic was so perfect, why didn't everyone see it? It's
often said, on Wall Street, that the stock market's prices reflect all
available information --- an idea known as ``efficient market theory.''
The idea has dominated the financial sector for half a century. If it
really were luminously obvious that a carbon bubble was about to
explode, the theory says that prices should reflect that --- in other
words, that Grantham had no edge and his thesis made no sense.

``They say everything's priced in,'' I ventured.

``Complete {[}expletive{]},'' Grantham said. Then he embarked on a
detailed explanation of why, summarizing John Maynard Keynes's theory of
career risk --- ``that it is better for reputation to fail
conventionally than to succeed unconventionally'' --- before returning
to present-day Wall Street, where asset managers, impelled by short-term
self-interest or outright denial, feared to stick their necks out on
climate-related bets. Faced with such irrational behavior, efficient
market theory seemed wobbly. ``It's {[}expletive{]} because people are
incompetent,'' Grantham continued, ``it's {[}expletive{]} because ---''

Around 10 a.m., his cellphone crackled to life, and a woman's voice
said: ``Jeremy, you have to get to your next meeting.''

For a second, Grantham appeared almost mournful. He held my coat up for
me to put my arms through. Then he dashed downstairs into the snow.

\textbf{The future site of} the solar farm that Lincoln is building for
Exxon, the Permian Basin, is hilly and semiarid, with white caliche
roads running toward the oil rigs and the nights punctured by the flames
of natural-gas flares. Sometimes, when the wind picks up, the highways
smell like sulfur. In recent years, the Permian became the most
productive oil-and-gas field in the United States, as advancements in
horizontal drilling and fracking technology made it possible to shatter
the tightly packed shale. Exxon, Chevron and their peers can now access
natural gas and oil that was previously unreachable, organic material
that was deposited by surging oceans and subsiding land some 200 million
years ago. If the Permian were a country, it would rank among the
largest oil states in the world.

Every cliché about oil booms applies right now in the Permian: the
18-year-olds earning six figures driving trucks, the petrochemical
Ph.D.s living in man-camps, the overtaxed public schools and doctors'
offices. Unemployment in Midland, where Permian energy companies have
their headquarters and where George W. Bush was raised, hovered this
winter around 2.2 percent, the fourth-lowest metropolitan rate in the
country. Yet the salient feature of the landscape is not the drilling
infrastructure. For one thing, fracking takes place underground, in
10,000-foot tunnels, no more than eight inches in diameter and marked by
only a single wellhead. For another, you can't keep your eyes off the
wind turbines, which in certain counties seem studded in every acre of
ranch land. Texas produces more wind power than every other state in the
country, four times as much as the runner-up, California.

The Texas power grid inspires awe. Every five minutes, 24 hours a day,
the Electric Reliability Council of Texas calculates the cheapest power
being produced from every kind of generator in the state, then sends
that electricity down the path of least resistance to its customers.
From Ercot's perspective, it doesn't matter whether the cheapest power
comes from a solar panel or coal, or whether the customer is a
greenhouse or an oil rig.

Last April, Texas consumed about 25.7 million megawatts of power. Of
that, 62 percent came from hydrocarbons and 27 percent from wind and
solar. Electricity from all power sources feeds into the Ercot grid like
tributaries into a river. Though Exxon's deal with Lincoln is one of the
most visible examples of a fossil-fuel company using renewable energy,
in reality all the Permian extraction outfits consume it, whether or not
they intend to, simply because the grid is designed to serve it to them.
In 2017, demand for electricity rose 8 percent in West Texas, compared
with 1 percent for the state's grid as a whole. Warren Lasher, the
senior director of system planning at Ercot, told me that most of that
change comes from oil and gas.

\textbf{Frosty Gilliam, an} independent oilman in the Permian, greeted
me at the reception desk of his office, on a sparse stretch of business
highway in Odessa, Tex., and beckoned me into a conference room
decorated in what he described as a ``Tuscan feel,'' with marble,
hand-troweled plaster and antique lamps. Sitting across from me at the
darkly polished table, Gilliam was small and reticent, with glinting
eyes and short white hair.

Gilliam grew up in West Texas and earned his degree at Texas A\&M in
petroleum engineering, graduating into the oil boom of the 1980s and
easily finding a job at Amoco, the conglomerate that descended from John
D. Rockefeller's Standard Oil. In the late '80s, as many small-time
oil-and-gas entrepreneurs used to do in Texas, Gilliam started to build
himself a business by scraping together mineral rights. His company,
Aghorn Energy, now controls some 1,100 wells in the Permian, making him
a relative lightweight.

\emph{{[}How is climate change affecting your area?
\href{https://www.nytimes3xbfgragh.onion/2019/03/11/reader-center/climate-change.html}{We
want to hear from you}.{]}}

The atmosphere in the office was convivial, and I hesitated to raise a
question that would poison it, but after half an hour I asked how he
thought about climate change. For 14 seconds, Gilliam stared at me
across the table, the hint of a smirk on his mouth.

``Now you're setting me up for a bunch of hate mail,'' he said.
Personally, he didn't believe climate change was an important issue. He
said that when he saw data about rising sea levels or scorching
temperatures, he suspected it was falsified or manipulated in order to
further a political agenda. Gilliam knew that many of the big energy
companies were investing in renewables, and he viewed their maneuvers
skeptically. ``The politically correct path is, `We're going to increase
our renewable energy production by 10 percent a year,' O.K.? But in
reality, they make their business selling oil and gas, right?''

For the most part Gilliam spoke in the first person, stressing to me
that he was delivering only his own opinions. Toward the end of our time
together, though, he switched into a collective voice, to explain the
reaction you'd arouse if you showed up in West Texas in a Tesla. ``We
wouldn't tar and feather you,'' he said. ``We would just think, Well, he
has his opinion, but our opinion is that there's not a problem.''

The tension between the individual and the species, between Gilliam and
the ``we,'' runs through the heart of capitalism and always has. In
economics, there is a theory called the Lauderdale Paradox, which the
Scottish politician James Maitland articulated in the early 19th
century. The theory says that capitalism undervalues public resources,
like air and water and soil, because they are so plentiful, and
overvalues whatever is private and scarce. A barrel of oil sells for
\$50 or \$60, yet the emissions from that oil appear on no one's balance
sheet.

When the paradox was first articulated, the mill industry of England was
transitioning from water to coal, a long and contentious process. Coal
allowed mill owners to site factories in cities, where wages were lower,
and to run their machines at all hours. But water was cheaper, cleaner
and more plentiful. In newspapers and private clubs, politicians and
journalists fervently debated the merits of each fuel source, and
England hovered for decades on the knife's edge between two possible
futures, until --- as the scholar Andreas Malm recounts in ``Fossil
Capital,'' a history of early industrialism --- the more aggressive
urban mill owners triumphed and the country switched to hydrocarbons.
Seeing the whole problem like that, as a result of an economic
arrangement rather than an unsparing fate or a flaw in human character,
is exceedingly grim but also kind of optimistic. One system can dominate
for a while, then another can sneak up and take its place.

At the far end of Gilliam's conference table, a surveyor's map lay open,
its corners secured by brown leather document weights. The map depicted
a mineral play in which Gilliam had an interest, containing perhaps a
few hundred thousand barrels' worth of oil. PROVISIONAL \& CONFIDENTIAL
was stamped in red along the bottom. The map depicted the typical
hydrocarbon infrastructure, like well locations and the large container
drums known as tank batteries, shown as blue and yellow rectangles. But
a checkerboard of gray lines had been drawn over these features,
dividing the 3,700 acres into clean, even squares. I asked Gilliam what
the squares were. ``Solar farm,'' he said, casually.

\hypertarget{correction-april-12-2019}{%
\subparagraph{\texorpdfstring{\textbf{Correction} April 12,
2019}{Correction April 12, 2019}}\label{correction-april-12-2019}}

An earlier version of this article mischaracterized, in one instance,
G.M.O., the asset-management company. It is a firm, not a fund. And the
article misstated Jeremy Grantham's role at the firm. He does not manage
the firm; he helps oversee it. The article also misstated the title of a
speech by Grantham. It was ``The Race of Our Lives,'' not ``Protect the
Planet, Profit From Green Causes.''

Jesse Barron is a writer in Los Angeles.
\href{https://www.nytimes3xbfgragh.onion/interactive/2018/12/27/magazine/lives-they-lived-david-buckel.html}{He
last wrote for the magazine about the environmentalist David Buckel.}

\hypertarget{more-climate}{%
\subsection{MORE CLIMATE}\label{more-climate}}

\begin{itemize}
\tightlist
\item
  \href{https://www.nytimes3xbfgragh.onion/interactive/2019/04/11/magazine/climate-change-bangladesh-scavenging.html}{}
\item
  \href{https://www.nytimes3xbfgragh.onion/interactive/2019/04/10/magazine/climate-change-pinkertons.html}{}
\item
  \href{https://www.nytimes3xbfgragh.onion/interactive/2019/04/09/magazine/climate-change-capitalism.html}{}
\item
  \href{https://www.nytimes3xbfgragh.onion/interactive/2019/04/09/magazine/climate-change-politics-economics.html}{}
\item
  \href{https://www.nytimes3xbfgragh.onion/interactive/2019/04/09/magazine/climate-change-peru-law.html}{}
\end{itemize}

JARVIS

\hypertarget{more-on-nytimescom}{%
\subsection{More on NYTimes.com}\label{more-on-nytimescom}}

Advertisement

\hypertarget{site-information-navigation}{%
\subsection{Site Information
Navigation}\label{site-information-navigation}}

\begin{itemize}
\tightlist
\item
  \href{https://help.nytimes3xbfgragh.onion/hc/en-us/articles/115014792127-Copyright-notice}{©
  2020 The New York Times Company}
\item
  \href{https://www.nytimes3xbfgragh.onion}{Home}
\item
  \href{https://www.nytimes3xbfgragh.onion/search/}{Search}
\item
  Accessibility concerns? Email us at
  \href{mailto:accessibility@NYTimes.com}{\nolinkurl{accessibility@NYTimes.com}}.
  We would love to hear from you.
\item
  \href{https://help.nytimes3xbfgragh.onion/hc/en-us/articles/115015385887-Contact-Us}{Contact
  Us}
\item
  \href{https://www.nytco.com/careers/}{Work with us}
\item
  \href{https://nytmediakit.com/}{Advertise}
\item
  \href{https://help.nytimes3xbfgragh.onion/hc/en-us/articles/115014892108-Privacy-policy\#pp}{Your
  Ad Choices}
\item
  \href{https://help.nytimes3xbfgragh.onion/hc/en-us/articles/115014892108-Privacy-policy}{Privacy}
\item
  \href{https://help.nytimes3xbfgragh.onion/hc/en-us/articles/115014893428-Terms-of-service}{Terms
  of Service}
\item
  \href{https://help.nytimes3xbfgragh.onion/hc/en-us/articles/115014893968-Terms-of-sale}{Terms
  of Sale}
\end{itemize}

\hypertarget{site-information-navigation-1}{%
\subsection{Site Information
Navigation}\label{site-information-navigation-1}}

\begin{itemize}
\tightlist
\item
  \href{https://spiderbites.nytimes3xbfgragh.onion}{Site Map}
\item
  \href{https://help.nytimes3xbfgragh.onion/hc/en-us}{Help}
\item
  \href{https://help.nytimes3xbfgragh.onion/hc/en-us/articles/115015385887-Contact-Us?redir=myacc}{Site
  Feedback}
\item
  \href{https://www.nytimes3xbfgragh.onion/subscription?campaignId=37WXW}{Subscriptions}
\end{itemize}
