 **NYTimes.com no longer supports Internet Explorer 9 or earlier. Please
upgrade your browser.
\href{http://www.nytimes3xbfgragh.onion/content/help/site/ie9-support.html}{LEARN
MORE »}

**Sections

**Home

**Search

\hypertarget{the-new-york-times}{%
\subsection{\texorpdfstring{\href{http://www.nytimes3xbfgragh.onion/}{The
New York Times}}{The New York Times}}\label{the-new-york-times}}

\hypertarget{-magazine-}{%
\subsubsection{\texorpdfstring{
\href{https://www.nytimes3xbfgragh.onion/section/magazine}{Magazine}
}{ Magazine }}\label{-magazine-}}

 \href{https://www.nytimes3xbfgragh.onion/section/magazine}{Magazine}
\textbar{}What Survival Looks Like After the Oceans Rise

**Close search

\hypertarget{site-search-navigation}{%
\subsection{Site Search Navigation}\label{site-search-navigation}}

Search NYTimes.com

**Clear this text input

Go

\url{https://nyti.ms/2GjpFAk}

\hypertarget{site-navigation}{%
\subsection{Site Navigation}\label{site-navigation}}

\hypertarget{site-mobile-navigation}{%
\subsection{Site Mobile Navigation}\label{site-mobile-navigation}}

\hypertarget{what-survival-looks-like-after-the-oceans-rise}{%
\section{What Survival Looks Like After the Oceans
Rise}\label{what-survival-looks-like-after-the-oceans-rise}}

At the site of a Bangladeshi town lost to devastating storms, locals
make do by scavenging what remains.

The Climate Issue

\begin{itemize}
\tightlist
\item
  \href{https://www.nytimes3xbfgragh.onion/interactive/2019/04/11/magazine/climate-change-bangladesh-scavenging.html}{What
  Survival Looks Like After the Oceans Rise}
\item
  \href{https://www.nytimes3xbfgragh.onion/interactive/2019/04/10/magazine/climate-change-pinkertons.html}{Climate
  Chaos Is Coming --- and the Pinkertons Are Ready}
\item
  \href{https://www.nytimes3xbfgragh.onion/interactive/2019/04/11/magazine/climate-change-exxon-renewable-energy.html}{How
  Big Business Is Hedging Against the Apocalypse}
\item
  \href{https://www.nytimes3xbfgragh.onion/interactive/2019/04/09/magazine/climate-change-capitalism.html}{The
  Next Reckoning: Capitalism and Climate Change}
\item
  \href{https://www.nytimes3xbfgragh.onion/interactive/2019/04/09/magazine/climate-change-politics-economics.html}{The
  Problem With Putting a Price on the End of the World}
\item
  \href{https://www.nytimes3xbfgragh.onion/interactive/2019/04/09/magazine/climate-change-peru-law.html}{Climate
  Change Could Destroy His Home in Peru. So He Sued an Energy Company in
  Germany.}
\end{itemize}

\protect\hyperlink{}{} \protect\hyperlink{}{}

\includegraphics{https://static01.graylady3jvrrxbe.onion/newsgraphics/2019/04/09/climate2019/51f6a9588f2c17e118fdd8781dfa8806f5d62a95/caret.svg}

\hypertarget{what-survival-looks-like-after-the-oceans-rise-1}{%
\section{What Survival Looks Like After the Oceans
Rise}\label{what-survival-looks-like-after-the-oceans-rise-1}}

At the site of a Bangladeshi town lost to devastating storms, locals
make do by scavenging what remains.

Photographs By ANDREA FRAZZETTA APRIL 11, 2019

Standing sometimes waist-deep in seawater on the shores of the Bay of
Bengal in Bangladesh, they work to find bricks, dig them out of the
sludge and cart them to the side of the road to sell. The job is new, a
result of devastating storm surges a little more than a decade ago. In
2007, and then again in 2009, cyclones battered the coastline just south
of Kuakata, destroying homes and structures and drowning entire
villages. The storms submerged forests of mangroves and left 99 local
residents dead.

The sisters Kulsum and Komola Begum survived. Now they scavenge, looking
for debris. They wait until low tide, when the receding waves reveal the
rubble. Once they've wheeled bricks to the embankment, they break them
into small, chestnut-size pieces. These shards are used in the
foundations for homes in the new village, a mile up the shore.

Despite being responsible for only 0.3 percent of the emissions that
cause global warming, Bangladesh is near the top of the Global Climate
Risk Index, a ranking of 183 countries and territories most vulnerable
to climate change. When scientists and researchers predict how global
warming will affect populations, they usually use 20- and 50-year
trajectories. For Bangladesh, the effects of climate change are
happening now. Cyclones are growing stronger as temperatures rise and
are occurring with more frequency.

\emph{{[}\href{https://www.nytimes3xbfgragh.onion/interactive/2019/04/09/magazine/climate-change-peru-law.html}{Read
about new legal strategies to make the world's biggest polluters pay for
climate change}.{]}}

Researchers warn that within a few decades, Bangladesh may lose more
than 10 percent of its land to sea-level rise, displacing as many as 18
million people. Decisions to leave coastal communities aren't really
decisions at all. Families leave because there are no other options.
There is no work. There are no homes. Over the past decade, an average
of 700,000 Bangladeshis a year migrated because of natural disasters,
moving to Dhaka to live in sprawling slums as climate refugees. Kulsum
and Komola have managed to forge opportunity from disaster; they will
stay, for now. They will continue to collect bricks to build the new
village, even if the new village will most likely meet the same fate as
the old one. \emph{--- Jaime Lowe}

\begin{center}\rule{0.5\linewidth}{\linethickness}\end{center}

\hypertarget{making-a-living-in-the-ruins}{%
\paragraph{Making a Living in the
Ruins}\label{making-a-living-in-the-ruins}}

The sisters Kulsum and Komola Begum make a living scavenging bricks,
which they sell to construction workers for roughly \$1.40 a sack.

During monsoon season, when currents are stronger and tides wash away
the sand, the family can bag 60 to 70 sacks. Over all, they earn enough
to send the children to school and buy uniforms and books.

Komola Begum's sons sometimes help their mother collect the bricks.
``When I can earn, my children can eat. If I don't, they will starve,''
Komola said. ``I do this for my kids.''

Her son Nur-un-Nabi plays outside his family's home, which is surrounded
by fields of rice and grasses. When he is not at school, and not helping
his mother on the shore, Nur-un-Nabi can often be found running on thin
slippery dams, occasionally chasing a water snake slithering out of the
flooded rice fields.

The dozen miles of beach crowns the tourist town of Kuakata, roughly two
hundred miles south of Dhaka. The beach is surrounded by forests of
mangroves and palm plantations, which are falling victim to increasingly
aggressive cyclones, tidal surges and rising seas. ``When we were young,
the old people used to say that the sea was very far from here,'' Komola
said. ``They packed up their meals and walked their way to the sea. But
now you can reach it in no time.''

Komola Begum loads bricks onto a cart that her son Bellal Nabi will
pedal a few hundred yards along a path of hard-beaten earth up to an
embankment where the bricks will be unloaded and broken into smaller
chunks.

Nur-un-Nabi breaks bricks, while his aunt Kulsum does the same a short
distance away. The piles of bricks rest on an embankment that was
recently raised to make it more resistant to cyclones. The Begum
families' homes are about a hundred yards from the embankment --- which
the more pessimistic local residents expect will withstand just a few
more cyclones before being washed away.

Komola and Kulsum Begum load a bag of brick for a client. A bag can be
as heavy as 40 kilos, and the two sisters often help each other with the
task. ``It is a good business so far,'' Komola said. ``Sometimes we get
pre-orders, and this is good money.''

At each low tide, new scraps of bricks are revealed in the mud. A few
decades ago, Komola Begum recalled, there were fishing villages here,
and roads, rice fields and plantations.

``Some bricks come from the fishing nets,'' where they are used as
weights, she said. ``We don't know where the others come from.'' She
assumes that many come from homes that have been swept away. ``Now
everything is under the sea,'' she said from the beach, pointing toward
the ocean.

\begin{center}\rule{0.5\linewidth}{\linethickness}\end{center}

Photo captions by Jacopo Pasotti.

\hypertarget{more-climate}{%
\subsection{MORE CLIMATE}\label{more-climate}}

\begin{itemize}
\tightlist
\item
  \href{https://www.nytimes3xbfgragh.onion/interactive/2019/04/10/magazine/climate-change-pinkertons.html}{}
\item
  \href{https://www.nytimes3xbfgragh.onion/interactive/2019/04/11/magazine/climate-change-exxon-renewable-energy.html}{}
\item
  \href{https://www.nytimes3xbfgragh.onion/interactive/2019/04/09/magazine/climate-change-capitalism.html}{}
\item
  \href{https://www.nytimes3xbfgragh.onion/interactive/2019/04/09/magazine/climate-change-politics-economics.html}{}
\item
  \href{https://www.nytimes3xbfgragh.onion/interactive/2019/04/09/magazine/climate-change-peru-law.html}{}
\end{itemize}

JARVIS

\hypertarget{more-on-nytimescom}{%
\subsection{More on NYTimes.com}\label{more-on-nytimescom}}

Advertisement

\hypertarget{site-information-navigation}{%
\subsection{Site Information
Navigation}\label{site-information-navigation}}

\begin{itemize}
\tightlist
\item
  \href{https://help.nytimes3xbfgragh.onion/hc/en-us/articles/115014792127-Copyright-notice}{©
  2020 The New York Times Company}
\item
  \href{https://www.nytimes3xbfgragh.onion}{Home}
\item
  \href{https://www.nytimes3xbfgragh.onion/search/}{Search}
\item
  Accessibility concerns? Email us at
  \href{mailto:accessibility@NYTimes.com}{\nolinkurl{accessibility@NYTimes.com}}.
  We would love to hear from you.
\item
  \href{https://help.nytimes3xbfgragh.onion/hc/en-us/articles/115015385887-Contact-Us}{Contact
  Us}
\item
  \href{https://www.nytco.com/careers/}{Work with us}
\item
  \href{https://nytmediakit.com/}{Advertise}
\item
  \href{https://help.nytimes3xbfgragh.onion/hc/en-us/articles/115014892108-Privacy-policy\#pp}{Your
  Ad Choices}
\item
  \href{https://help.nytimes3xbfgragh.onion/hc/en-us/articles/115014892108-Privacy-policy}{Privacy}
\item
  \href{https://help.nytimes3xbfgragh.onion/hc/en-us/articles/115014893428-Terms-of-service}{Terms
  of Service}
\item
  \href{https://help.nytimes3xbfgragh.onion/hc/en-us/articles/115014893968-Terms-of-sale}{Terms
  of Sale}
\end{itemize}

\hypertarget{site-information-navigation-1}{%
\subsection{Site Information
Navigation}\label{site-information-navigation-1}}

\begin{itemize}
\tightlist
\item
  \href{https://spiderbites.nytimes3xbfgragh.onion}{Site Map}
\item
  \href{https://help.nytimes3xbfgragh.onion/hc/en-us}{Help}
\item
  \href{https://help.nytimes3xbfgragh.onion/hc/en-us/articles/115015385887-Contact-Us?redir=myacc}{Site
  Feedback}
\item
  \href{https://www.nytimes3xbfgragh.onion/subscription?campaignId=37WXW}{Subscriptions}
\end{itemize}
