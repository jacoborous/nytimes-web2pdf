Sections

SEARCH

\protect\hyperlink{site-content}{Skip to
content}\protect\hyperlink{site-index}{Skip to site index}

\hypertarget{comments}{%
\subsection{\texorpdfstring{\protect\hyperlink{commentsContainer}{Comments}}{Comments}}\label{comments}}

\href{}{Robert De Niro Thinks Donald Trump Is Worse Than Any Gangster
He's Played}\href{}{Skip to Comments}

The comments section is closed. To submit a letter to the editor for
publication, write to
\href{mailto:letters@NYTimes.com}{\nolinkurl{letters@NYTimes.com}}.

Great Performers

\hypertarget{robert-de-niro-thinks-donald-trump-is-worse-than-any-gangster-hes-played}{%
\section{Robert De Niro Thinks Donald Trump Is Worse Than Any Gangster
He's
Played}\label{robert-de-niro-thinks-donald-trump-is-worse-than-any-gangster-hes-played}}

By David MarcheseDec. 9, 2019

\begin{itemize}
\item
\item
\item
\item
\item
  \emph{+}
\end{itemize}

``The rule in acting is you never make a judgment about your character.
The characters have their reasons, and you understand them.''

\href{https://www.nytimes3xbfgragh.onion/interactive/2019/12/09/magazine/best-actors.html}{Great
Performers} · Robert De Niro

\hypertarget{robert-de-niro-thinks-donald-trump-is-worse-than-any-gangster-hes-played-1}{%
\section{Robert De Niro Thinks Donald Trump Is Worse Than Any Gangster
He's
Played}\label{robert-de-niro-thinks-donald-trump-is-worse-than-any-gangster-hes-played-1}}

Robert De Niro is one of the
\href{https://www.nytimes3xbfgragh.onion/interactive/2019/12/09/magazine/best-actors.html}{10
best actors of the year}. He talks about Scorsese, `The Irishman' ---
and President Trump.

By David Marchese

Photographs by Jack Davison

Dec. 9, 2019

SHARE

Disturbingly stoic, violent and seeking absolution he's not sure he
needs, the mob killer Frank Sheeran allowed Robert De Niro to deliver a
majestic, subtle performance in
\href{https://www.nytimes3xbfgragh.onion/2019/09/27/movies/the-irishman-review.html}{``The
Irishman''} that has the feel of a crowning achievement --- and for
reasons that go beyond the screen. Based on Sheeran's memoir, ``I Heard
You Paint Houses,'' the film is haunted by the cinematic moments that De
Niro, the director Martin Scorsese and the co-stars Al Pacino and Joe
Pesci have made in so many movies about hard men with hollowed hearts.
``The fact that me, Joe and Al were doing this film is something in and
of itself,'' said the halting, taciturn De Niro, who also played a key
role in this fall's controversial, Scorsese-indebted
\href{https://www.nytimes3xbfgragh.onion/2019/10/03/movies/joker-review.html}{``Joker.''}
``Marty directing it says something. It all sets a tone. The audience's
perception of each character, us actors being together and what the
story is --- the film is all those things.'' It's also a reminder, as if
we needed one, of the brutal and beautifully unsentimental revelations
that only a peak De Niro performance can provide.

\textbf{In getting ready to play Frank Sheeran, you dug deep into the
source material, and you also spoke with people who knew the guy. But
I'm curious how your thinking about preparation has changed over the
years. You've said in the past that you don't kill yourself with it the
way you did when you were younger.} What I meant was that maybe it's not
as necessary to be so obsessed. It's better at times to be relaxed. Do
all the preparation before, and then just do the scene, and don't be
anxious about it or amped up about what it is. Getting so concerned
about an emotional scene --- you can kind of short-circuit whatever's
going to come.

\textbf{Was there a performance that led to that realization?} No. I
just felt that a real emotional situation in life comes due to the
circumstances around you. If you prepare too much --- you know the joke
about the actor who couldn't remember any lines?

\textbf{No, I don't know it.} This actor can't remember lines, so he
can't get a job. A director he knows runs into him at the gas station
where he's working. The director says: ``I have a play that in the third
act, what you do is go and say, `Hark, I hear the cannons roar.' Can I
count on you to do that?'' The actor says he'll do it. He goes and
rehearses, rehearses, rehearses. ``Hark, I hear the cannons roar. Hark,
I hear the cannons roar.'' On the day of the play, the third act comes,
and the actor runs out onstage. Boom! The cannon goes boom, and the
actor goes, ``What the {[}expletive{]} was that?!'' The point is, you
don't want to lose spontaneity.

\textbf{Earlier in your career, there was a lot of attention paid to how
you changed your body for your work in, to pick just the most famous
example, ``Raging Bull.'' In ``The Irishman,'' your body changed too,
but the changes were made digitally, to allow you to look younger. How
did it affect the performance not to be able to feel those changes
physically?} Well, it's harder to act younger than it is the other way
round. We had a guy named Gary Tacon who was a movement coach. He would
tap you and say, ``You're 39 in this scene.'' In one case, I was walking
down the stairs a little more carefully than my character would've, and
Gary showed me that you kind of fall down the stairs when you're
younger. So I did that. I did it well. Marty cut it out because he
didn't need it. But it was that kind of stuff. You have to be aware of
having a certain spryness.

\textbf{And you felt that you could credibly achieve that?} I felt that,
but even so, some people felt it was not --- they weren't criticizing
it. They were saying they could see my real age. O.K., fine, that's
interesting. I should've taken steroids or something. They'll youth-ify
you or de-age you or whatever, but you still can't look like you're
crotchety. It's a good thing. You know, Marty would see, and I saw it,
too, that there would be an expression in my eyes during a scene, but
after they youth-ified me, my eyes had a different emotional expression.
Marty was concerned about that. I had the right emotional intention, the
right attitude, but when that de-aging came, the expression in the eye
changed. So they had to figure out a way to make sure that after I was
youth-ified it would not alter the intention of the scene as we acted
it. It was an interesting problem.

\textbf{You could think about a character like Frank --- or a lot of
people you've played --- as fundamentally inhumane as written on the
page. But you have a way of infusing all these vicious characters with
something approaching soul. Are there keys to doing that?} The rule in
acting is you never make a judgment about your character. The characters
have their reasons, and you understand them. You're trying to look at
their point of view. I mean, in ``The Irishman,'' Frank has a problem
with his daughter. He has problems that anybody can relate to. I never
thought of him as being amoral or immoral. He lives in a world where the
penalties are harsh if you don't do what you're supposed to do. He says
he's going to do something, he does it. I don't like to go to Trump, but
he is a person who, to me, has no morals, no ethics, no sense of right
and wrong, is a dirty player.

\href{https://www.nytimes3xbfgragh.onion/interactive/2019/12/09/magazine/best-actors.html}{}

This article is part of The New York Times Magazine's annual Great
Performers issue, honoring the best actors of the year.

\textbf{Could you find your way into the character of President Trump?}
I wouldn't want to play him. He's such an awful person. There's nothing
redeemable about him, and I never say that about any character.

\textbf{You found redemptive qualities in Travis Bickle, and you're
saying you couldn't do the same if you were playing President Trump?} I
can't compare. There's not one moment that Trump said: ``I'm sorry. I
realize I've done something that I shouldn't have done.'' He has not one
speck of redeemability in him. He's not owed one speck of redeemability.

\textbf{People have argued that some of Trump's rhetoric has emboldened
others to make threats or enact violence. Those arguments are not a
world away from ones that people made about Travis Bickle or ``Joker.''
Do you think those arguments hold water?} They might, but Trump has
people who follow him who are crazy and want to do crazy things. What
we're doing in film, it's like a dream. We know it's not real. There are
people who will take anything to be real and that we have no control
over. The president is supposed to set an example of trying to do the
right thing. Not be a nasty little bitch. Because that's what he is.
He's a petulant little punk. There's not one thing that I see in him or
his family, not any redeeming qualities. They're out on the take. It's
like a gangster family.

\textbf{To shift subjects a bit, what about if somebody were looking to
play you? Would you be willing to talk with them and help out with their
preparation?} That's a good question. I don't know. I've always
experienced that people are open because they want you to get it right.
They want to give you information. With
\href{https://www.nytimes3xbfgragh.onion/1980/11/14/archives/robert-de-niro-in-raging-bull.html}{``Raging
Bull,''} Jake LaMotta was great with me and Marty. He was happy that we
were making a movie about him. Certain things, maybe it was our
interpretation. That's the same with Frank Sheeran and ``The Irishman.''
In acting they say: Make it your own. Personalize it. It's the same
thing with these stories. There has to be some --- I don't like to say
poetic license, because that has a negative connotation when it
shouldn't --- but it's a way of expressing how you see it. It doesn't
mean it's right. But it's how you see yourself.

\textbf{What did you see in yourself that you put into Frank Sheeran?}
Aha! \emph{That} is the question.

\textbf{What's the answer?} That is the question, but the answer is
personal. I mean, when I talked to Marty about certain things about the
film --- sometimes he's like a priest. We talk, and I have to be honest
with him in order to get stuff in the film that we need to say. But it's
personal stuff that I would express through the character. It's not
stuff I'd tell other people.

\textbf{I know you've thought about one day sitting down and watching
all your own movies. What would you hope to see?} I would probably be
apprehensive, because I'm critical about what I did. But the other thing
is what I could learn if I looked at all my stuff and got an idea of
what I've done, what the pattern is. Because I'd like to do something
that's really different from what I've done or been known to do.

\textbf{If you watched all your performances, do you think you'd feel
any pride?} I have reasons that I look at my stuff and I'm not happy.
Other people look at my stuff and say they don't even know what I'm
talking about. I don't know. It's not for me to say.

\textbf{David Marchese} is a staff writer for The New York Times
Magazine and the Talk columnist. Recently he interviewed
\href{https://www.nytimes3xbfgragh.onion/interactive/2019/11/25/magazine/pete-townshend-the-who-interview.html}{Pete
Townshend on rock's legacy},
\href{https://www.nytimes3xbfgragh.onion/interactive/2019/10/21/magazine/patti-lupone-broadway-company.html}{Patti
LuPone about being bullied on Broadway} and
\href{https://www.nytimes3xbfgragh.onion/interactive/2019/07/08/magazine/whoopi-goldberg-controversy.html}{Whoopi
Goldberg about creative fulfillment}. \textbf{Jack Davison} is a British
photographer. His work has been featured in British Vogue, Modern Weekly
China and recently in the magazine with
\href{https://www.nytimes3xbfgragh.onion/2019/03/27/magazine/glenda-jackson-king-lear.html}{a
cover photograph of Glenda Jackson}. His first book, ``Photographs,''
was published by Loose Joints earlier this year.

Stylist: Brian Molloy. Grooming: Lynda Eichner. Clothing: Charvet.

This interview has been edited and condensed from two conversations.

\hypertarget{more-great-performers}{%
\subsection{More Great Performers}\label{more-great-performers}}

\href{https://www.nytimes3xbfgragh.onion/interactive/2019/12/09/magazine/best-actors.html}{See
the Best Actors of 2019}

\begin{itemize}
\tightlist
\item
  \href{/interactive/2019/12/09/magazine/brad-pitt-interview.html}{}
\item
  \href{/interactive/2019/12/09/magazine/jennifer-lopez-hustlers.html}{}
\item
  \href{/interactive/2019/12/09/magazine/lupita-nyongo-us.html}{}
\end{itemize}

Write a comment

\begin{itemize}
\item
\item
\item
\item
\end{itemize}

Advertisement

\protect\hyperlink{after-bottom}{Continue reading the main story}

\hypertarget{site-index}{%
\subsection{Site Index}\label{site-index}}

\hypertarget{site-information-navigation}{%
\subsection{Site Information
Navigation}\label{site-information-navigation}}

\begin{itemize}
\tightlist
\item
  \href{https://help.nytimes3xbfgragh.onion/hc/en-us/articles/115014792127-Copyright-notice}{©~2020~The
  New York Times Company}
\end{itemize}

\begin{itemize}
\tightlist
\item
  \href{https://www.nytco.com/}{NYTCo}
\item
  \href{https://help.nytimes3xbfgragh.onion/hc/en-us/articles/115015385887-Contact-Us}{Contact
  Us}
\item
  \href{https://www.nytco.com/careers/}{Work with us}
\item
  \href{https://nytmediakit.com/}{Advertise}
\item
  \href{http://www.tbrandstudio.com/}{T Brand Studio}
\item
  \href{https://www.nytimes3xbfgragh.onion/privacy/cookie-policy\#how-do-i-manage-trackers}{Your
  Ad Choices}
\item
  \href{https://www.nytimes3xbfgragh.onion/privacy}{Privacy}
\item
  \href{https://help.nytimes3xbfgragh.onion/hc/en-us/articles/115014893428-Terms-of-service}{Terms
  of Service}
\item
  \href{https://help.nytimes3xbfgragh.onion/hc/en-us/articles/115014893968-Terms-of-sale}{Terms
  of Sale}
\item
  \href{https://spiderbites.nytimes3xbfgragh.onion}{Site Map}
\item
  \href{https://help.nytimes3xbfgragh.onion/hc/en-us}{Help}
\item
  \href{https://www.nytimes3xbfgragh.onion/subscription?campaignId=37WXW}{Subscriptions}
\end{itemize}
