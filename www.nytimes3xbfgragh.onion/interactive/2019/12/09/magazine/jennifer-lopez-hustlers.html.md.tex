Sections

SEARCH

\protect\hyperlink{site-content}{Skip to
content}\protect\hyperlink{site-index}{Skip to site index}

\hypertarget{comments}{%
\subsection{\texorpdfstring{\protect\hyperlink{commentsContainer}{Comments}}{Comments}}\label{comments}}

\href{}{Jennifer Lopez on Finally Getting the Part She Was Born to
Play}\href{}{Skip to Comments}

The comments section is closed. To submit a letter to the editor for
publication, write to
\href{mailto:letters@NYTimes.com}{\nolinkurl{letters@NYTimes.com}}.

Great Performers

\hypertarget{jennifer-lopez-on-finally-getting-the-part-she-was-born-to-play}{%
\section{Jennifer Lopez on Finally Getting the Part She Was Born to
Play}\label{jennifer-lopez-on-finally-getting-the-part-she-was-born-to-play}}

By \href{https://www.nytimes3xbfgragh.onion/by/a-o--scott}{A.O.
Scott}Dec. 9, 2019

\begin{itemize}
\item
\item
\item
\item
\item
  \emph{+}
\end{itemize}

For decades, Hollywood demanded she play ditzy or desperate. It took the
role an exuberant, amoral stripper to set her free.

\href{https://www.nytimes3xbfgragh.onion/interactive/2019/12/09/magazine/best-actors.html}{Great
Performers} · Jennifer Lopez

\hypertarget{jennifer-lopez-on-finally-getting-the-part-she-was-born-to-play-1}{%
\section{Jennifer Lopez on Finally Getting the Part She Was Born to
Play}\label{jennifer-lopez-on-finally-getting-the-part-she-was-born-to-play-1}}

Jennifer Lopez is one of the
\href{https://www.nytimes3xbfgragh.onion/interactive/2019/12/09/magazine/best-actors.html}{10
best actors of the year} --- playing an exuberant, amoral stripper who
reveals more than skin.

By A.O. Scott

Photographs by Jack Davison

Dec. 9, 2019

SHARE

We see Ramona do a final flourish and walk down the stage. As Jennifer
Lopez remembers it, that's how an early draft of the ``Hustlers'' script
described her character's first scene. The ``flourish'' turned into
something much more elaborate: an extended pole dance in which Lopez,
dressed in something close to nothing, spins, twists and kicks through a
display of erotic athleticism that ends with a strip club full of
patrons roaring and cheering, the stage carpeted with dollar bills and a
struggling young dancer named Destiny (played by Constance Wu) in a
state of slack-jawed adoration. ``Doesn't money make you horny?'' Ramona
asks Destiny as she heads for the roof, where she stretches out in her
fur coat and lights a cigarette.

The pole work, which required six weeks of training with a Cirque du
Soleil acrobat, was Lopez's idea. She explained the genesis of the scene
on a gray afternoon in November, almost exactly two months after
\href{https://www.nytimes3xbfgragh.onion/2019/09/11/movies/hustlers-review.html}{``Hustlers,''}
shot in 29 days on a relatively low budget the previous spring, opened,
becoming one of the few nonfranchise hits of the movie year. ``I said to
Lorene'' --- Scafaria, who wrote and directed ``Hustlers'' --- ``that we
have to see why Ramona is the star of the club. We can't say it. We have
to show it. I'm going to do this amazing dance. I don't know what it is,
but it's going to be good. And from there you will see that she has
total control of the club and the crowd, and Destiny is going to fall in
love. She can't help it.''

Lopez was drinking cappuccino in a cafe attached to Pier 59 Studios, a
photography studio on the West Side of Manhattan. She paused our
conversation for a moment of mutual-admiration pantomime with the
designer Vera Wang, who was in the building for a different shoot. As
tech crews, publicists and security details went about their business,
there was some pointing, head-swiveling and whispering. Even in this
celebrity-saturated space, Lopez commanded attention.

Like Ramona --- but in a radically different context and on a global
scale --- she wears her stardom on her skin. ``It's a performance, but
she's real,'' she said. ``She's always performing, and yet she's the
most authentic person you could meet.''

It's tempting to say the same about Lopez, but it would be a mistake ---
a misreading not only of ``Hustlers'' but also of the career path that
led to it --- to say that she is simply playing herself. Especially in
the early scenes, she is leveraging the image she has built up over the
years, but that's not the same thing. Lopez, who turned 50 in July, has
been a familiar presence in pop culture for at least a quarter century,
embodying a unique and durable combination of strength and grace, of
self-confidence and sexual allure. She has portrayed performers before,
most notably in
\href{https://www.nytimes3xbfgragh.onion/1997/03/21/movies/a-short-life-remembered-with-songs-and-sunshine.html}{``Selena,''}
the 1997 biopic of the Tejano singer that was her breakout role. But
it's the different versions of Jennifer Lopez that make her not so much
a star as a constellation. She's JLo and she's Jenny from the block; a
genre-crossing musician and a music-video icon; a pioneering pop diva
and a lifestyle entrepreneur; a fixture of gossip columns and
fashion-magazine covers alike.

In the midst of all that, she appeared in a lot of movies. After 1998
(when she and George Clooney met cute in the trunk of a sedan in the
wonderful
\href{https://archive.nytimes3xbfgragh.onion/www.nytimes3xbfgragh.onion/library/film/062698sight-film-review.html}{``Out
of Sight''}), a few of them were O.K. Then again, one of them was
``Gigli.'' In 2018, reviewing the mom-com ``Second Act'' for The New
York Times, Wesley Morris
\href{https://www.nytimes3xbfgragh.onion/2018/12/19/movies/second-act-review-jennifer-lopez.html}{gave
voice to a disappointment} that was surely not his alone. ``Basically,''
he wrote ``it's been 20 years, and I'm still waiting to see her let all
the way go, to be as natural, funny, sharp, overjoyed, surprising and
\emph{open} as she can be in interviews, on Instagram, as a competition
judge and as a recording artist.''

The thing is that, through it all and in addition to everything else,
Lopez has always been a more-than-capable actress. She has a knack for
balancing glamour and realness that recalls the great screen goddesses
of Old Hollywood: some of Claudette Colbert's smart, flirtatious energy;
some of Veronica Lake's tough-and-tender wit; and more than a little
Lauren Bacall in the poise she maintains in almost every circumstance.

\href{https://www.nytimes3xbfgragh.onion/interactive/2019/12/09/magazine/best-actors.html}{}

This article is part of The New York Times Magazine's annual Great
Performers issue, honoring the best actors of the year.

It's the circumstances that have been the problem. The conventions of
modern Hollywood have demanded that Lopez play ditzy or desperate,
generically female in ways that could make her seem like a stand-in
Sandra Bullock or a convenient Kate Hudson substitute. Anyone who saw
her in concert or in music videos, on ``American Idol'' or a red carpet
somewhere --- anyone with eyes, a pulse and a television --- knew that
she was more than the sum of those big-screen parts. Which led Morris to
wonder: ``When are the movies going to start working for her?''

We finally have an answer. ``Hustlers'' is
\href{https://www.thecut.com/2015/12/robin-hood-strippers-scores-c-v-r.html}{inspired
by a New York Magazine article} about a gang of strippers who ripped off
wealthy clients by spiking their drinks and maxing out their credit
cards. Though Ramona and Destiny are fictional characters, much of what
they do reflects what really happened, and even at its wildest, the
movie has a gritty, grounded credibility.

On first viewing, I took it as a caper, a fizzy New York crime story
with a shot of pop feminism and a twist of post-great-recession
class-consciousness mixed in. But that wasn't quite right. Following the
money, I discounted the love story between Ramona and Destiny. And I
almost missed the way Ramona and her crew, as they separated the marks
from their wallets, were staging a brazen raid on genre territory
occupied, for as long as anyone could remember, by men.

``A woman gangster movie'' is how Lopez described it to me, with
transgressive pleasures and an undercurrent of moral queasiness. ``It's
fun and funny, but it's sick. It's like we're eating spaghetti in the
front, and in the back we just blew someone's head off and sawed off his
arms.'' Nothing quite so gruesome happens to Ramona's victims, who
suffer wounds mainly to their egos and their bank balances. And the
exuberant amorality of the strippers contains an element of rough
justice, as they turn the tables on men who had been happy to exploit,
objectify and take advantage of them.

Like many gangster pictures, ``Hustlers'' is as much about the dynamics
of what Lopez calls its ``sexy, underground world'' --- about the
friendships, rivalries, alliances and inevitable betrayals among the
players --- as it is about specific illegal actions. The arc of the
narrative is a familiar one, in which an innocent (Destiny) falls under
the spell of an old-timer (Ramona) who educates her and the audience in
the codes and customs of a new way of life. Yet in spite of the
quasi-show-business setting, this is not yet another variation on the
theme of ``All About Eve.'' Ramona may be a maternal figure, but she
also incarnates what has usually been a male archetype: the problematic
mentor who both schools and corrupts the newcomer, setting up a drama of
education through disillusionment.

Lopez is Denzel Washington to Wu's Ethan Hawke (in ``Training Day'').
She's Al Pacino to --- take your pick --- Johnny Depp (``Donnie
Brasco''), Keanu Reeves (``Devil's Advocate''), Chris O'Donnell (``Scent
of a Woman''), Colin Farrell (``The Recruit'') or Hilary Swank
(``Insomnia''). There's an even more canonical example. ``I get to be
Joe Pesci,'' she said (and Wu gets to be Ray Liotta). ``That didn't
exist for women.''

Not that it's just a matter of retrofitting familiar pop-culture figures
with new costumes and pronouns. There's no shortage of movies that
traffic in boss-lady and badass-action-heroine clichés of female
empowerment without troubling an aggressively male status quo.

Ramona is much more complicated, a character who breaks existing molds
even as she feels immediately, intuitively recognizable, like a person
you might know. ``Ramona was independent,'' Lopez said. ``It wasn't
about men. It wasn't about: `I need these guys to like me. I need them
to be attracted to me. My goal is to get all of your money.' And her
having a daughter, being a mama bear --- those things made it fun to
play, because I could be a maternal Mother Earth, caring and loving, and
then be like, I'll stab you.

``I love that you never see Ramona with a man,'' she continued, ``except
when she's in the club or working. When you do certain roles, you
realize something about yourself. I've always been so much a romantic,
so much about having a relationship, and this woman is the total
opposite. And to play that, to live in those shoes, to walk in those
very high heels, in that skin, made me realize I'm out here on my own.
That's what I need to teach my daughter, that aspect of it, that you can
do it on your own. Women are not taught that all the time.''

The movie doesn't ignore the harsh side of this autonomy. Destiny was
abandoned by her parents as a young girl, and she and Ramona's other
protégées deal with possessive or unreliable partners, disapproving
families and (not to put too fine a point on it) a capitalist,
patriarchal society rigged against their interests. What they have is
one another, a sisterhood that isn't always harmonious or kind but
offers protection and comfort as well as money and fun.

There is a lot of that in ``Hustlers'' --- loose, spontaneous-feeling
scenes backstage at the club and in Ramona's immaculate high-rise
apartment. The women exchange advice and Christmas presents, pop
Champagne corks, snuggle on the couch. Their intimacy, their solidarity,
their love --- and the tensions and suspicions that inevitably emerge
--- are enough to give the story momentum and emotional gravity.

According to Lopez, the presence of so many women on the set had a lot
to do with how the movie turned out: ``Women producers, woman writer and
director, all-star woman cast, woman editor, woman production designer
--- all of it. It was different. It was new.'' What is on the screen
also seems new. Even as Ramona's professional and artistic identity
involves soliciting the attention of men, the male gaze is as marginal
to ``Hustlers'' as its male characters.

At the beginning, all eyes --- Destiny's, those of the audiences in the
club and the theater --- are on Ramona's almost-naked body. But
something deeper than skin is being revealed, and the rest of Lopez's
performance is a complex choreography of disclosure and concealment.
Ramona can be guarded and elusive as well as open and generous, and we
get to know her without the usual shortcuts. She has no back story, it
seems, and no particular interest in explaining herself.

``Every character I do,'' Lopez said, ``I feel like the physical parts
of her are the most important. Maybe that's the dancer in me. The gait,
the shoulders, the chin, how that person is, where her confidence lives,
her sexuality --- it's all physical. With Ramona, the pole dance is part
of the essence of who she is. She's an actress and a performer, so the
physical part of that made me know who she was and how she could be.''

Listening to Lopez talk about the ordeal of training for the pole dance
--- ``bruised everywhere, burned and chafed, pulled my shoulder out''
--- and her insistence that Scafaria shoot her first scene for maximal
realism --- ``when I'm upside-down, make sure you get a close-up of my
face; I don't care if I look like a bat hanging upside-down; I want
people to know it's me and not a double'' --- I realized that I had
missed another major genre that ``Hustlers'' belongs to, one that
accounts for its blend of grit and sentimentality. It's a boxing movie.

I know: It sounds as if I'm trying to assimilate a movie by and about
women into the canon of guy cinema. But what I'm really trying to do is
account for a film and a performance that open new territory by making
us understand what we had been missing for so long. There was always the
potential for Ramona, which is to say that there might have been, should
have been, a whole tradition of Ramonas. And there should have been more
roles like Ramona for Jennifer Lopez to play. I don't mean strippers,
necessarily --- there are a lot of strippers in movies --- but women
with ambition and complexity and the kind of magnetism that goes beyond
sex.

``I remember going to a club,'' Lopez said, recalling some of the
research that went into her performance, ``and one of the girls --- the
floor was just full of money. It wasn't that she was the most beautiful
or had the most bodacious or perfect body --- it was what she did up
there. It was how she made people feel. And that was Ramona's gift. She
made everybody feel exciting, special --- like something amazing can
happen.''

\textbf{A.O. Scott} is a chief film critic for The Times and the author
of ``Better Living Through Criticism: How to Think About Art, Pleasure,
Beauty and Truth.''
\href{https://www.nytimes3xbfgragh.onion/interactive/2019/10/08/magazine/susan-sontag.html}{He
last wrote an essay for the magazine about Susan Sontag's influence on
his life as a critic.} In January, he will be a resident at the American
Academy in Rome. \textbf{Jack Davison} is a British photographer. His
work has been featured in British Vogue, Modern Weekly China and
recently in the magazine with
\href{https://www.nytimes3xbfgragh.onion/2019/03/27/magazine/glenda-jackson-king-lear.html}{a
cover photograph of Glenda Jackson}. His first book, ``Photographs,''
was published by Loose Joints earlier this year.

Stylist: Brian Molloy. Hair: Chris Appleton. Makeup: Scott Barnes.
Manicure: Eri Ishizu. Clothing: Ann Demeulemeester and Hermès.

Additional design and development by Jacky Myint.

\hypertarget{more-great-performers}{%
\subsection{More Great Performers}\label{more-great-performers}}

\href{https://www.nytimes3xbfgragh.onion/interactive/2019/12/09/magazine/best-actors.html}{See
the Best Actors of 2019}

\begin{itemize}
\tightlist
\item
  \href{/interactive/2019/12/09/magazine/brad-pitt-interview.html}{}
\item
  \href{/interactive/2019/12/09/magazine/robert-deniro-interview.html}{}
\item
  \href{/interactive/2019/12/09/magazine/lupita-nyongo-us.html}{}
\end{itemize}

Write a comment

\begin{itemize}
\item
\item
\item
\item
\end{itemize}

Advertisement

\protect\hyperlink{after-bottom}{Continue reading the main story}

\hypertarget{site-index}{%
\subsection{Site Index}\label{site-index}}

\hypertarget{site-information-navigation}{%
\subsection{Site Information
Navigation}\label{site-information-navigation}}

\begin{itemize}
\tightlist
\item
  \href{https://help.nytimes3xbfgragh.onion/hc/en-us/articles/115014792127-Copyright-notice}{©~2020~The
  New York Times Company}
\end{itemize}

\begin{itemize}
\tightlist
\item
  \href{https://www.nytco.com/}{NYTCo}
\item
  \href{https://help.nytimes3xbfgragh.onion/hc/en-us/articles/115015385887-Contact-Us}{Contact
  Us}
\item
  \href{https://www.nytco.com/careers/}{Work with us}
\item
  \href{https://nytmediakit.com/}{Advertise}
\item
  \href{http://www.tbrandstudio.com/}{T Brand Studio}
\item
  \href{https://www.nytimes3xbfgragh.onion/privacy/cookie-policy\#how-do-i-manage-trackers}{Your
  Ad Choices}
\item
  \href{https://www.nytimes3xbfgragh.onion/privacy}{Privacy}
\item
  \href{https://help.nytimes3xbfgragh.onion/hc/en-us/articles/115014893428-Terms-of-service}{Terms
  of Service}
\item
  \href{https://help.nytimes3xbfgragh.onion/hc/en-us/articles/115014893968-Terms-of-sale}{Terms
  of Sale}
\item
  \href{https://spiderbites.nytimes3xbfgragh.onion}{Site Map}
\item
  \href{https://help.nytimes3xbfgragh.onion/hc/en-us}{Help}
\item
  \href{https://www.nytimes3xbfgragh.onion/subscription?campaignId=37WXW}{Subscriptions}
\end{itemize}
