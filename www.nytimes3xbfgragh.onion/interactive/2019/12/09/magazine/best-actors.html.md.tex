Sections

SEARCH

\protect\hyperlink{site-content}{Skip to
content}\protect\hyperlink{site-index}{Skip to site index}

\hypertarget{comments}{%
\subsection{\texorpdfstring{\protect\hyperlink{commentsContainer}{Comments}}{Comments}}\label{comments}}

\href{}{The Best Actors of 2019}\href{}{Skip to Comments}

The comments section is closed. To submit a letter to the editor for
publication, write to
\href{mailto:letters@NYTimes.com}{\nolinkurl{letters@NYTimes.com}}.

Great Performers

\hypertarget{the-best-actors-of-2019}{%
\section{The Best Actors of 2019}\label{the-best-actors-of-2019}}

Selected by \href{https://www.nytimes3xbfgragh.onion/by/a-o--scott}{A.O.
Scott} and
\href{https://www.nytimes3xbfgragh.onion/by/wesley-morris}{Wesley
Morris}Dec. 9, 2019

\begin{itemize}
\item
\item
\item
\item
\item
  \emph{+}
\end{itemize}

Great Performers

\includegraphics{https://static01.graylady3jvrrxbe.onion/packages/flash/multimedia/ICONS/transparent.png}

\includegraphics{https://static01.graylady3jvrrxbe.onion/images/2019/12/15/magazine/15mag-greatperformers-12/15mag-greatperformers-12-master1050.jpg}

\hypertarget{great-performers}{%
\subsection{Great Performers}\label{great-performers}}

\includegraphics{https://static01.graylady3jvrrxbe.onion/packages/flash/multimedia/ICONS/transparent.png}

\includegraphics{https://static01.graylady3jvrrxbe.onion/images/2019/12/15/magazine/15mag-greatperformers-21/15mag-greatperformers-21-master1050.jpg}

\hypertarget{the-10-best-actors}{%
\subsection{The 10 Best Actors}\label{the-10-best-actors}}

\hypertarget{of-the-year}{%
\subsection{of the Year}\label{of-the-year}}

\includegraphics{https://static01.graylady3jvrrxbe.onion/packages/flash/multimedia/ICONS/transparent.png}

\includegraphics{https://static01.graylady3jvrrxbe.onion/images/2019/12/15/magazine/15mag-greatperformers-10/15mag-greatperformers-10-master1050.jpg}

\includegraphics{https://static01.graylady3jvrrxbe.onion/packages/flash/multimedia/ICONS/transparent.png}

\includegraphics{https://static01.graylady3jvrrxbe.onion/images/2019/12/15/magazine/15mag-greatperformers-04/15mag-greatperformers-04-master1050.jpg}

\includegraphics{https://static01.graylady3jvrrxbe.onion/packages/flash/multimedia/ICONS/transparent.png}

\includegraphics{https://static01.graylady3jvrrxbe.onion/images/2019/12/15/magazine/15mag-greatperformers-09/15mag-greatperformers-09-master1050.jpg}

\hypertarget{chosen-by-ao-scott}{%
\subsection{Chosen by A.O. Scott}\label{chosen-by-ao-scott}}

\hypertarget{and-wesley-morris}{%
\subsection{and Wesley Morris}\label{and-wesley-morris}}

\includegraphics{https://static01.graylady3jvrrxbe.onion/packages/flash/multimedia/ICONS/transparent.png}

\includegraphics{https://static01.graylady3jvrrxbe.onion/images/2019/12/15/magazine/15mag-greatperformers-17/15mag-greatperformers-17-master1050.jpg}

\includegraphics{https://static01.graylady3jvrrxbe.onion/packages/flash/multimedia/ICONS/transparent.png}

\includegraphics{https://static01.graylady3jvrrxbe.onion/images/2019/12/15/magazine/15mag-greatperformers-19/15mag-greatperformers-19-master1050.jpg}

\hypertarget{photo-portfolio-by}{%
\subsection{Photo Portfolio by}\label{photo-portfolio-by}}

\hypertarget{jack-davison}{%
\subsection{Jack Davison}\label{jack-davison}}

\includegraphics{https://static01.graylady3jvrrxbe.onion/packages/flash/multimedia/ICONS/transparent.png}

\includegraphics{https://static01.graylady3jvrrxbe.onion/images/2019/12/15/magazine/15mag-greatperformers-08/15mag-greatperformers-08-master1050.jpg}

\includegraphics{https://static01.graylady3jvrrxbe.onion/packages/flash/multimedia/ICONS/transparent.png}

\includegraphics{https://static01.graylady3jvrrxbe.onion/images/2019/12/15/magazine/15mag-greatperformers-15/15mag-greatperformers-15-master1050.jpg}

\includegraphics{https://static01.graylady3jvrrxbe.onion/packages/flash/multimedia/ICONS/transparent.png}

\includegraphics{https://static01.graylady3jvrrxbe.onion/images/2019/12/15/magazine/15mag-greatperformers-06/15mag-greatperformers-06-master1050.jpg}

These are the 10 actors whose work in movies we found most captivating,
challenging, shocking and inspiring in 2019. The performances are wildly
varied **** and yet, this year, we stumbled up on a theme, or at least a
pattern. Call it a motif of meta-ness: Most of these actors --- not all,
but a clear majority --- have been chosen for their portrayals of other
performing artists, people who live on the stage or screen or some other
space where authenticity and artifice collide.

This isn't new. Movie actors have been impersonating actual and
fictional thespians and thrushes at least since ``The Jazz Singer,'' and
the musical or theatrical biopic may be the most reliable route to an
Academy Award. Just ask Barbra Streisand (Fanny Brice), Sissy Spacek
(Loretta Lynn), Jamie Foxx (Ray Charles) and Rami Malek (Freddie
Mercury). This year is no different, with Taron Egerton's Elton John,
Renée Zellweger's Judy Garland and Tom Hanks's Fred Rogers all vying for
Oscar biopic love. But the performers we chose did more than embody the
stars of the past. They created new ones.

In a culture saturated with celebrity and ruled by reality television,
that is no small feat. The roster of great performers in these pages
includes not only Jennifer Lopez but also Ramona Vega, the
gentlemen's-club dancer who brings dazzle and drama to ``Hustlers.'' Not
only Elisabeth Moss but also Becky Something, a rock 'n' roll diva who
spends nearly the entirety of ``Her Smell'' teetering on the edge of
abjection and transcendence. Not just Adam Driver and Scarlett Johansson
but also Charlie and Nicole Barber, the theater artists in ``Marriage
Story'' who started out as actors and find themselves competing to
direct the sad comedy of their divorce. Not just Antonio Banderas but
also Salvador Mallo, the cineaste facing the austere autumn of a career
defined by flamboyance in ``Pain and Glory.'' Not only Leonardo DiCaprio
and Brad Pitt but also Rick Dalton and Cliff Booth, a minor Hollywood
star and his stunt double, holding the line against the counterculture
in ``Once Upon a Time \ldots{}in Hollywood.''

Professional performers aren't the only ones putting on an act. The hit
man played by Robert De Niro in ``The Irishman'' pretends to be anyone
other than who he really is, even as he risks losing touch with his self
and his soul. Julianne Moore, as the utterly ordinary title character
Gloria Bell, enacts a dramatic critique of the roles that women are
compelled to play in everyday life. And Lupita Nyong'o, in ``Us,''
splits herself in two, brilliantly challenging our assumptions about
private and public behavior and the masks we wear to deceive our
families, our societies and ourselves.

We're in an era of high performance, of constantly questioning the
reality not simply of what we see but also of who people are. Is this
elected official, reality-show participant, Facebook friend to be
trusted? That wariness makes perfect sense. (Who wants to be duped?) But
it might miss the point of a proper performance, a proper great
performance. You need a sense of artistry for that, a talent for
magnifying the central elements of a character or an idea. We tend to
talk about performance as though it's the definition of falsehood when,
at its best, it's the height of truth. \emph{--- A.O. Scott and Wesley
Morris}

\protect\hyperlink{}{Read More}

\includegraphics{https://static01.graylady3jvrrxbe.onion/packages/flash/multimedia/ICONS/transparent.png}

\includegraphics{https://static01.graylady3jvrrxbe.onion/images/2019/12/15/magazine/15mag-greatperformers-20/15mag-greatperformers-20-master1050.jpg}

\hypertarget{adam-driver}{%
\subsection{Adam Driver}\label{adam-driver}}

\hypertarget{marriage-story}{%
\subsection{`Marriage Story'}\label{marriage-story}}

If you want to insult an actor, you call him ``interesting.'' On the
other hand, if you said that about someone you met in real life, it
would be a pretty unambiguous compliment. An interesting person is
someone you want to know better, someone worth thinking about, someone
who has shown you something of who he is but at the same time held a
little something back to encourage your ... interest.

Adam Driver has to thread this needle in
\href{https://www.nytimes3xbfgragh.onion/2019/11/05/movies/marriage-story-review.html}{``Marriage
Story,''} Noah Baumbach's tale of a divorce. There's no question that
Charlie Barber, Driver's character, is an interesting guy. He knows it,
too. A director who runs a theater company in Manhattan, he is used to
being listened to, admired and liked. His confidence in his own appeal
just makes him more appealing. His charm is unforced. His
soon-to-be-ex-mother-in-law adores him, maybe more than she likes her
own daughter, his soon-to-be-ex-wife.

But his wife --- Nicole, played by Scarlett Johansson --- doesn't love
Charlie the way she used to. That's largely why they are splitting up,
and a lot of the plot of ``Marriage Story'' involves Charlie's
struggling to understand what is happening. Like many intellectuals,
he's deficient in both self-knowledge and the full awareness of other
people's existence. Nicole was always there, part of the unified field
of his ego, which is now fracturing. In order to hold on to the people
he loves, Charlie needs to let go of some of his narcissism, which is to
say that he has to learn to be less interested in --- less interesting
to --- himself.

Driver, of course, has to move in the opposite direction, drawing us
closer to the character and deepening our willingness to care about him.
From the very beginning of his career --- ever since he was that other
Adam, on ``Girls'' --- he has displayed a complicated charisma. He's
like a jigsaw puzzle with too many pieces. The essential picture keeps
shifting from sweet to angry, sardonic to sincere. This is more than
just the technical mastery of emotional complexity. There is an element
of epistemological volatility in his acting: You never know for sure if
the hints of self-consciousness, of anti-realism, come from him or from
the characters.

That's always interesting. In ``Marriage Story,'' it's devastating.
\emph{--- A.O. Scott}

\includegraphics{https://static01.graylady3jvrrxbe.onion/packages/flash/multimedia/ICONS/transparent.png}

\includegraphics{https://static01.graylady3jvrrxbe.onion/images/2019/12/15/magazine/15mag-greatperformers-13/15mag-greatperformers-13-master1050.jpg}

\hypertarget{lupita-nyongo}{%
\subsection{Lupita Nyong'o}\label{lupita-nyongo}}

\hypertarget{us}{%
\subsection{`Us'}\label{us}}

I'm greedy about my stars. There is usually never enough of the good
ones. Take Lupita Nyong'o. This woman can make a perfume ad worthy of
the Louvre. She can turn a red carpet into the Yellow Brick Road. But I
also like my stars onscreen. And she's just not there very often. How
can the fashion world be treating an Oscar winner better than the
movies? I'm not the only person who wants to know. Jordan Peele appears
to have been so determined to intervene that he cast Nyong'o in his
horror-thriller
\href{https://www.nytimes3xbfgragh.onion/2019/03/20/movies/us-movie-review.html}{``Us''}
--- not once but two times, as a woman and her clone.

One is ``sane'' and the other is ``evil,'' meaning Nyong'o alternates,
terrifyingly, between two poles of psychological extremity. Sure, that
in itself is a feat. But it's merely the most obvious thing to applaud.
The rigor of her achievement is that it won't stop revealing itself. For
the movie's first third, what she's doing might seem rather
unremarkable. She plays Adelaide Wilson, who is bright,
upper-middle-class and on vacation at her California ranch house with
her goofy husband and their two children. Her biggest worry appears to
be her teenage daughter's decision to quit the track team. But you can
sense her gathering fear that some terrible event is on its way; it's
dimming her glow as it heightens our anticipation.

The event, of course, is the other Nyong'o. ...

\href{https://www.nytimes3xbfgragh.onion/interactive/2019/12/09/magazine/lupita-nyongo-us.html}{Read
Wesley Morris on Lupita Nyong'o}

\includegraphics{https://static01.graylady3jvrrxbe.onion/packages/flash/multimedia/ICONS/transparent.png}

\includegraphics{https://static01.graylady3jvrrxbe.onion/images/2019/12/15/magazine/15mag-greatperformersadd-02/15mag-greatperformersadd-02-master1050.jpg}

\hypertarget{julianne-moore}{%
\subsection{Julianne Moore}\label{julianne-moore}}

\hypertarget{gloria-bell}{%
\subsection{`Gloria Bell'}\label{gloria-bell}}

What a crapshoot actor Julianne Moore is! She'll go for broke even if
she breaks the movie, even if the movie might break her. (No movie has
yet managed that.) Her best mode is some combination of wisdom,
carnality and lunacy (the film's as much as the character's). Another
way to put this is: I love deep-end Julianne Moore. To start: ``Safe,''
``Boogie Nights,'' ``The Big Lebowski,'' ``Magnolia,'' ``Chloe,'' ``Don
Jon,'' ``Maps to the Stars.''

\href{https://www.nytimes3xbfgragh.onion/2019/03/07/movies/gloria-bell-review.html}{``Gloria
Bell''} is mid-deep-end Moore. She's an owlish divorcée who manhunts at
her favorite nightspot. The movie is less about romantic predation than
it is about nostalgia. Gloria believes in love in this club because once
upon a time we all did. Moore makes Gloria radiantly lonely. Then she
meets a prospect (John Turturro), and Moore gets to sensualize herself.

Gloria doesn't appear to be a deep-ender at all --- until you realize
she has already leapt into a void. Moore plays her in a way that seems
suspended between an afterlife and actual existence, as a viable
middle-aged woman resisting the retiring ghostliness expected of her.
Solitude has left her enticingly \emph{off}. That's what Moore makes the
most of: how Gloria's a little too eager to connect, to be known --- by
men in that nightclub, by her own adult children. ``It's your mother,''
she'll say at the end of a rambling voice mail message.

The entire performance (based on the Gloria that Paulina García so
archly played
\href{https://www.nytimes3xbfgragh.onion/2014/01/24/movies/in-gloria-a-chilean-in-her-late-50s-embraces-passion.html}{in
the original 2013 Chilean movie}) is suffused with subdued panic. Is
Gloria as sad, nervous and faux-casual as Moore makes her appear? If she
falls in love again, will she stay in it? The thrill of Moore in this
movie is the nature of that leap. It's physics. It's existential.
Where's she going to land? Will she land at all?

Moore thrives in the peculiar gray zones of adult feeling. She can give
a character life. But lots of good actors can do that. Moore can give
you something greater: a character you can see has actually lived.
\emph{--- Wesley Morris}

\includegraphics{https://static01.graylady3jvrrxbe.onion/packages/flash/multimedia/ICONS/transparent.png}

\includegraphics{https://static01.graylady3jvrrxbe.onion/images/2019/12/15/magazine/15mag-greatperformers-16/15mag-greatperformers-16-master1050.jpg}

\hypertarget{brad-pitt}{%
\subsection{Brad Pitt}\label{brad-pitt}}

\hypertarget{once-upon-a-time--in-hollywood}{%
\subsection{`Once Upon a Time \ldots{} in
Hollywood'}\label{once-upon-a-time--in-hollywood}}

\hypertarget{and-ad-astra}{%
\subsection{and `Ad Astra'}\label{and-ad-astra}}

As the stuntman Cliff Booth in Quentin Tarantino's
\href{https://www.nytimes3xbfgragh.onion/2019/07/24/movies/once-upon-a-time-in-hollywood-review.html}{``Once
Upon a Time ... in Hollywood,''} Brad Pitt laid down a performance of
vintage Hollywood dudeness. His character is equally at ease being a
human security blanket for his B-list-actor boss, played by Leonardo
DiCaprio, as he is subduing murderous Manson family members while
tripping on acid.

In James Gray's
\href{https://www.nytimes3xbfgragh.onion/2019/09/19/movies/ad-astra-review-brad-pitt.html}{``Ad
Astra,''} Pitt used the same tools he wielded so deftly in Tarantino's
film --- laconic cool; understated emotion --- to build an entirely
different version of masculinity. In it, he's Roy McBride, an astronaut
on an interplanetary mission to find his absentee (in multiple senses of
the word) father. But McBride's imperturbability is rooted in repression
and hurt, nothing like Booth's so-it-goes acceptance.

``The two characters could be connected,'' Pitt says, ``in the sense
that you have to go through an evolution to get to a place of comfort.
You have to go through profound internal hardships.'' \emph{--- David
Marchese}

\href{https://www.nytimes3xbfgragh.onion/interactive/2019/12/09/magazine/brad-pitt-interview.html}{Read
an interview with Brad Pitt}

\includegraphics{https://static01.graylady3jvrrxbe.onion/packages/flash/multimedia/ICONS/transparent.png}

\includegraphics{https://static01.graylady3jvrrxbe.onion/images/2019/12/15/magazine/15mag-greatperformersadd-06/15mag-greatperformersadd-06-master1050.jpg}

\hypertarget{antonio-banderas}{%
\subsection{Antonio Banderas}\label{antonio-banderas}}

\hypertarget{pain-and-glory}{%
\subsection{`Pain and Glory'}\label{pain-and-glory}}

Aging should be the easiest thing to perform. Strictly speaking --- and
leaving aside enhancements like prosthetic wrinkles and talcum-powdered
hair --- you don't have to play it at all. Sooner or later, age plays
you. An actor who reaches a certain stage of life, and has been a
familiar face for at least half of his earthly span, needs only to show
us that face. We'll do the rest, measuring the longevity of our
attraction in crags and furrows and whitened follicles as we muse on the
mercies and ravages of time.

In
\href{https://www.nytimes3xbfgragh.onion/2019/10/03/movies/pain-and-glory-review.html}{``Pain
and Glory,''} his eighth collaboration with Pedro Almodóvar (but only
their third since 1989), Antonio Banderas lets his grizzled, melancholy,
beautiful features carry their share of the burden. It's a little
shocking to see him looking not just gray but frail, as if his almost-60
body harbored a soul in deep senescence. That body is racked by a
painful medical condition, and also by memories. Banderas's character, a
Spanish filmmaker named Salvador Mallo, was once a rebellious and
celebrated cultural figure, very much like Almodóvar himself. Now, 30
years past his groundbreaking early prime and in a semiretirement that
looks a lot like creative paralysis, Salvador reconnects with his former
leading man, Alberto, a wayward, charismatic actor who might have a
resemblance to Antonio Banderas.

But ``Pain and Glory'' is more than a game of biographical peekaboo.
Banderas, trailed by memories of his studly-sensitive, sometimes brutish
personas in Almodóvar's
\href{https://www.nytimes3xbfgragh.onion/1987/03/27/movies/new-directors-new-films-spanish-law-of-desire.html}{``Law
of Desire,''}
\href{https://www.nytimes3xbfgragh.onion/1988/09/16/movies/reviews-film-almodovar-s-matador-surrealist-sex-comedy.html}{``Matador''}
and
\href{https://www.nytimes3xbfgragh.onion/1990/05/04/movies/review-film-when-love-s-ties-are-real-ropes.html}{``Tie
Me Up! Tie Me Down!''} embodies a faded, unapologetic flamboyance. He is
a maximalist in a shrunken time, a former thirst trap playing a man
whose throat has grown dry. He moves slowly and cautiously. His voice is
heavy. His face is gentle and impassive.

The performance is so quiet and specific you might wonder if Banderas is
even acting. He is, of course, but that isn't all he's doing. He is
paying tribute to a friend and mourning a friendship, being himself and
channeling the man who made him who he is, reclaiming his prime and
leaving us to wonder how it all went by so fast. \emph{--- A.O. Scott}

\includegraphics{https://static01.graylady3jvrrxbe.onion/packages/flash/multimedia/ICONS/transparent.png}

\includegraphics{https://static01.graylady3jvrrxbe.onion/images/2019/12/15/magazine/15mag-greatperformers-14/15mag-greatperformers-14-master1050.jpg}

\hypertarget{jennifer-lopez}{%
\subsection{Jennifer Lopez}\label{jennifer-lopez}}

\hypertarget{hustlers}{%
\subsection{`Hustlers'}\label{hustlers}}

\emph{We see Ramona do a final flourish and walk down the stage.} As
Jennifer Lopez remembers it, that's how an early draft of the
\href{https://www.nytimes3xbfgragh.onion/2019/09/11/movies/hustlers-review.html}{``Hustlers''}
script described her character's first scene. The ``flourish'' turned
into something much more elaborate: an extended pole dance in which
Lopez, dressed in something close to nothing, spins, twists and kicks
through a display of erotic athleticism that ends with 300 strip-club
patrons on their feet roaring, the stage carpeted with dollar bills and
a struggling young dancer named Destiny (played by Constance Wu) in a
state of slack-jawed adoration. ``Doesn't money make you horny?'' Ramona
asks Destiny as she heads for the roof, where she stretches out in her
fur coat and lights a cigarette.

The pole work, which required six weeks of training with a Cirque du
Soleil acrobat, was Lopez's idea. She explained the genesis of the scene
on a gray afternoon in November, almost exactly two months after
``Hustlers,'' shot in 29 days on a relatively low budget the previous
spring, opened, becoming one of the few nonfranchise hits of the movie
year.

``I said to Lorene'' --- Scafaria, who wrote and directed ``Hustlers''
--- ``that we have to see why Ramona is the star of the club. We can't
say it. We have to show it. I'm going to do this amazing dance. I don't
know what it is, but it's going to be good. And from there you will see
that she has total control of the club and the crowd, and Destiny is
going to fall in love. She can't help it.'' \emph{...}

\href{https://www.nytimes3xbfgragh.onion/interactive/2019/12/09/magazine/jennifer-lopez-hustlers.html}{Read
A.O. Scott on Jennifer Lopez}

\includegraphics{https://static01.graylady3jvrrxbe.onion/packages/flash/multimedia/ICONS/transparent.png}

\includegraphics{https://static01.graylady3jvrrxbe.onion/images/2019/12/15/magazine/15mag-greatperformers-02/15mag-greatperformers-02-master1050.jpg}

\hypertarget{scarlett-johansson}{%
\subsection{Scarlett Johansson}\label{scarlett-johansson}}

\hypertarget{marriage-story-1}{%
\subsection{`Marriage Story'}\label{marriage-story-1}}

Sides are the way of divorce. Each partner gets one, and then everyone
else has to choose. So whose side is
\href{https://www.nytimes3xbfgragh.onion/2019/11/05/movies/marriage-story-review.html}{``Marriage
Story''} on? Seems like Charlie's. Adam Driver plays him with so much
coiled-up charm that you might excuse his self-absorption (he's a
worshiped downtown director) and fail to notice Nicole, the actress
exiting his shadow.

She emerges, in the opening shot, from darkness into light, then floods
a montage with the attributes that Charlie finds most adorable. Minutes
later, she's slumped in a mediator's office, irate. Her eyes are wet and
concerningly tiny. She doesn't want to be the dream girl from that
montage. She wants to be who \emph{she} is. And the only way to figure
that out is to decamp.

Playing Nicole, Scarlett Johansson might have the hardest acting
assignment of the year. She has to observe and absorb while Driver
simmers, Laura Dern declaims, Ray Liotta leaks unction and Julie Hagerty
pilfers everything she gets her hands on. But Johansson's combination of
emotional steadiness and personal uncertainty is the core of this movie.

At some point, Nicole visits the cozy, skyscraping law office of Dern's
Los Angeles divorce warrior, who leans over and all but whispers, ``What
we're going to do together is tell your story.'' And so for about 10
straight minutes, Johansson wanders around the room, in rumination,
exclamation, exhalation, tears. \emph{Telling}. At last. Johansson
doesn't get enough credit for being a great talker in the movies; for
Woody Allen, in
\href{https://www.nytimes3xbfgragh.onion/2006/07/28/movies/28scoo.html}{``Scoop''}
and
\href{https://www.nytimes3xbfgragh.onion/2008/08/15/movies/15barc.html}{``Vicki
Cristina Barcelona,''} and in a film like
\href{https://www.nytimes3xbfgragh.onion/2013/12/18/movies/her-directed-by-spike-jonze.html}{Spike
Jonze's ``Her,''} where her honeyed alto is the voice of an entire
operating system. But giving language to years of unexpressed hope and
exasperation in that lawyer's office is the most wonderful, most human
speaking Johansson has done.

Johansson is playing a woman whose certitude has, for years, been
divided by marital second guesses, by Charlie's (and Driver's) emotional
bigness. What's left is rue, weariness, indignation and
self-rediscovery. Maybe it takes a second viewing to discover that
Nicole's rationality (and Johansson's) obviate such a concept as sides.
But this is a divorce film; and if it's taking us to the battlements,
I'm on hers. \emph{--- Wesley Morris}

\includegraphics{https://static01.graylady3jvrrxbe.onion/packages/flash/multimedia/ICONS/transparent.png}

\includegraphics{https://static01.graylady3jvrrxbe.onion/images/2019/12/15/magazine/15mag-greatperformersadd-05/15mag-greatperformersadd-05-master1050-v2.jpg}

\hypertarget{robert-de-niro}{%
\subsection{Robert De Niro}\label{robert-de-niro}}

\hypertarget{the-irishman}{%
\subsection{`The Irishman'}\label{the-irishman}}

Disturbingly stoic, violent and seeking absolution he's not sure he
needs, the mob killer Frank Sheeran allowed Robert De Niro to deliver a
majestic, subtle performance in
\href{https://www.nytimes3xbfgragh.onion/2019/09/27/movies/the-irishman-review.html}{``The
Irishman''} that has the feel of a crowning achievement --- and for
reasons that go beyond the screen. Based on Sheeran's memoir, ``I Heard
You Paint Houses,'' the film is haunted by the cinematic moments that De
Niro, the director Martin Scorsese and the co-stars Al Pacino and Joe
Pesci have made in so many movies about hard men with hollowed hearts.

``The fact that me, Joe and Al were doing this film is something in and
of itself,'' said the halting, taciturn De Niro, who also played a key
role in this fall's controversial, Scorsese-indebted
\href{https://www.nytimes3xbfgragh.onion/2019/10/03/movies/joker-review.html}{``Joker.''}
``Marty directing it says something. It all sets a tone. The audience's
perception of each character, us actors being together and what the
story is --- the film is all those things.''

It's also a reminder, as if we needed one, of the brutal and beautifully
unsentimental revelations that only a peak De Niro performance can
provide. \emph{--- David Marchese}

\href{https://www.nytimes3xbfgragh.onion/interactive/2019/12/09/magazine/robert-deniro-interview.html}{Read
an interview with Robert De Niro}

\includegraphics{https://static01.graylady3jvrrxbe.onion/packages/flash/multimedia/ICONS/transparent.png}

\includegraphics{https://static01.graylady3jvrrxbe.onion/images/2019/12/15/magazine/15mag-greatperformersadd-04/15mag-greatperformersadd-04-master1050.jpg}

\hypertarget{elisabeth-moss}{%
\subsection{Elisabeth Moss}\label{elisabeth-moss}}

\hypertarget{her-smell}{%
\subsection{`Her Smell'}\label{her-smell}}

There are difficult characters, antiheroes, supervillains, leading men
and women who test the limits of likability --- and then there is Becky
Something. She is the frontwoman and main creative force in Something
She, a fictional but unnervingly real-seeming '90s all-female power
trio. Also its main destructive force. Becky's greatest talent may be
alienating the people who care most about her. Those include an
ex-husband, her mother, two long-suffering bandmates, the head of her
record label and just about everyone in ``Her Smell'' not played by
Elisabeth Moss.

There are stories of addiction and recovery, portraits of artists on the
verge of breakdowns, tales of rock 'n' roll dysfunction --- and then
there is
\href{https://www.nytimes3xbfgragh.onion/2019/04/10/movies/her-smell-review.html}{``Her
Smell.''} Moss, obliterating the memory of Peggy Olson's ``Mad Men''
pluck and Offred's
\href{https://www.nytimes3xbfgragh.onion/2019/08/29/books/testaments-margaret-atwood-handmaids-tale.html}{``Handmaid's
Tale''} stoicism --- and also, of course, playing off those same
qualities --- gives a performance that is both violently verbal and
abrasively physical. No vanity, but a kind of abject bravura. She does
not fear sweat, spit, snot, vomit or smeared mascara, but she also
relishes Becky's flights of wit, invective, insight and inspired
nonsense. (``Big bad bossy Becky makes maudlin Mari mope'' is how she
summarizes a fight with a bandmate in the middle of the fight.) The
surprise of her language is the key: It's why everyone keeps coming
around, even when what they get is disdain, humiliation and abuse.

There are directors who test the audience's tolerance for discomfort,
rubbing our noses in ordinary human awfulness --- and then there is Alex
Ross Perry. ``Her Smell'' is his sixth feature. Moss has appeared in
half of them, often unhinged or in tears. This one, like a horror movie
in reverse, lets the monster out early. And then, in the middle, Becky
goes quiet. Sober and penitent, she's almost catatonic, whispering as if
afraid to wake up the internal demons. Her young daughter, visiting for
the first time in a while, asks for a song, and Becky plays one on the
piano: Bryan Adams's ``Heaven,'' a cheesy Gen X ballad that is the
opposite of punk rock. It's a clumsy, heartfelt cover of a bad classic
song, and it's proof of Becky's artistry, which is to say of Moss's too.
\emph{--- A.O. Scott}

\includegraphics{https://static01.graylady3jvrrxbe.onion/packages/flash/multimedia/ICONS/transparent.png}

\includegraphics{https://static01.graylady3jvrrxbe.onion/images/2019/12/15/magazine/15mag-greatperformers-03/15mag-greatperformers-03-master1050.jpg}

\hypertarget{leonardo-dicaprio}{%
\subsection{Leonardo DiCaprio}\label{leonardo-dicaprio}}

\hypertarget{once-upon-a-time--in-hollywood-1}{%
\subsection{`Once Upon a Time \ldots{} in
Hollywood'}\label{once-upon-a-time--in-hollywood-1}}

Do you consider Leonardo DiCaprio funny? Like, on purpose? Well, please
do! Some of his best moments are the riotous ones. Once, in
\href{https://www.nytimes3xbfgragh.onion/2013/12/25/movies/dicaprio-stars-in-scorseses-the-wolf-of-wall-street.html}{``The
Wolf of Wall Street,''} as the wolf, he downed some quaaludes and rolled
down the steps of a country club like a sack of apples in a stop-motion
dream. Another time, he was one of those genteel antebellum racists ---
Calvin Candie in ``Django Unchained'' --- whom he inflated with a lot of
``I do de-clahr!'' effrontery. (With all due respect to Django,
\emph{DiCaprio} was unchained.) Rick Dalton is the latest and most
embarrassed enrollee in DiCaprio's Comedy Club.

Rick is an actor whose star, in 1969, has grown dingy. And in
\href{https://www.nytimes3xbfgragh.onion/2019/07/24/movies/once-upon-a-time-in-hollywood-review.html}{``Once
Upon a Time ... in Hollywood,''} DiCaprio has a ball recreating Rick's
TV-western mulch and B-movie schlock. He gives the gunslinging every
ounce of deadpan machismo he can summon and becomes exactly the
flamethrowing maniac you need for an action pageant called ``Fourteen
Fists of McCluskey.'' DiCaprio has to hold on to the movie's satirical
showbiz insanity as well as Rick's alcoholism, square bravado,
insecurity, faded stardom and private misery.

None of that is funny, per se, except that DiCaprio wills it to be so,
not simply in the furious mock-Hollywood bits but in a long, gorgeous
passage right in the middle of the movie, on the set of a western
series. Rick has taken a gig as a villain (another one), and before the
cameras roll, he finds himself chatting with a young co-star who tells
him he's the best actor she's ever worked with. In between, Rick flubs a
line and, in costume and in his trailer, proceeds to berate himself for
being an undisciplined hack. It's as divine as any of DiCaprio's great
eruptions, at once a joke on acting and perhaps a window into the soul
of a star --- Jack Lord devastated that he'll never be Jack Lemmon. I'm
with the kid. Sort of. Rick is one of the most mediocre actors I've ever
seen. But it takes a real maestro to summon all that talentlessness and
keep knocking you out of your chair. --- \emph{Wesley Morris}

Stylist: Brian Molloy.

Set designer: Julia Wagner.

On-set producers: Nicole Tondre (New York); Michael Kachuba/3Star
Productions (Los Angeles); Ana Gámiz (Málaga, Spain).

Grooming: De Niro: Lynda Eichner; DiCaprio: Kara Yoshimoto Bua; Driver:
Amy Komorowski; Pitt: Stacey Panepinto.

Hair: Johansson: David Von Cannon; Lopez: Chris Appleton; Moore: DJ
Quintero; Moss: David Von Cannon; Nyong'o: Nai'vasha; Pitt: Sal Salcedo.

Makeup: Johansson: Frankie Boyd; Lopez: Scott Barnes; Moore: Hung
Vanngo; Moss: Daniel Martin; Nyong'o: Nick Barose.

Manicure: Johansson: Casey Herman; Lopez: Eri Ishizu; Moore: Gina
Eppolito; Moss: Casey Herman; Nyong'o: Sonya Belakhlef.

Assistant stylist: Sarah Lequimener.

Clothing: Banderas: Margaret Howell; De Niro: Charvet; Johansson: Molly
Goddard; Lopez: Ann Demeulemeester and Hermès; Moore: Yohji Yamamoto and
The Row; Moss: Issey Miyake from New York Vintage; Nyong'o: Loewe and
Valentino; Pitt: Lemaire

Additional design and development by Jacky Myint.

Write a comment

\begin{itemize}
\item
\item
\item
\item
\end{itemize}

Advertisement

\protect\hyperlink{after-bottom}{Continue reading the main story}

\hypertarget{site-index}{%
\subsection{Site Index}\label{site-index}}

\hypertarget{site-information-navigation}{%
\subsection{Site Information
Navigation}\label{site-information-navigation}}

\begin{itemize}
\tightlist
\item
  \href{https://help.nytimes3xbfgragh.onion/hc/en-us/articles/115014792127-Copyright-notice}{©~2020~The
  New York Times Company}
\end{itemize}

\begin{itemize}
\tightlist
\item
  \href{https://www.nytco.com/}{NYTCo}
\item
  \href{https://help.nytimes3xbfgragh.onion/hc/en-us/articles/115015385887-Contact-Us}{Contact
  Us}
\item
  \href{https://www.nytco.com/careers/}{Work with us}
\item
  \href{https://nytmediakit.com/}{Advertise}
\item
  \href{http://www.tbrandstudio.com/}{T Brand Studio}
\item
  \href{https://www.nytimes3xbfgragh.onion/privacy/cookie-policy\#how-do-i-manage-trackers}{Your
  Ad Choices}
\item
  \href{https://www.nytimes3xbfgragh.onion/privacy}{Privacy}
\item
  \href{https://help.nytimes3xbfgragh.onion/hc/en-us/articles/115014893428-Terms-of-service}{Terms
  of Service}
\item
  \href{https://help.nytimes3xbfgragh.onion/hc/en-us/articles/115014893968-Terms-of-sale}{Terms
  of Sale}
\item
  \href{https://spiderbites.nytimes3xbfgragh.onion}{Site Map}
\item
  \href{https://help.nytimes3xbfgragh.onion/hc/en-us}{Help}
\item
  \href{https://www.nytimes3xbfgragh.onion/subscription?campaignId=37WXW}{Subscriptions}
\end{itemize}
