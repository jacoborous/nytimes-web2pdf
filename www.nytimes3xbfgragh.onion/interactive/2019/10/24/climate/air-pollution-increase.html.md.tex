Sections

SEARCH

\protect\hyperlink{site-content}{Skip to
content}\protect\hyperlink{site-index}{Skip to site index}

\href{https://www.nytimes3xbfgragh.onion/section/climate}{Climate}

\href{https://myaccount.nytimes3xbfgragh.onion/auth/login?response_type=cookie\&client_id=vi}{}

\href{https://www.nytimes3xbfgragh.onion/section/todayspaper}{Today's
Paper}

\href{/section/climate}{Climate}\textbar{}America's Air Quality Worsens,
Ending Years of Gains, Study Says

\url{https://nyti.ms/2odhpLM}

\begin{itemize}
\item
\item
\item
\item
\item
\end{itemize}

\hypertarget{climate-and-environment}{%
\subsubsection{\texorpdfstring{\href{https://www.nytimes3xbfgragh.onion/section/climate?name=styln-climate\&region=TOP_BANNER\&block=storyline_menu_recirc\&action=click\&pgtype=Interactive\&impression_id=2b2092a0-f4be-11ea-9caa-f145e0172db9\&variant=undefined}{Climate
and
Environment}}{Climate and Environment}}\label{climate-and-environment}}

\begin{itemize}
\tightlist
\item
  \href{https://www.nytimes3xbfgragh.onion/2020/09/08/climate/california-wildfires-climate.html?name=styln-climate\&region=TOP_BANNER\&block=storyline_menu_recirc\&action=click\&pgtype=Interactive\&impression_id=2b20b9b0-f4be-11ea-9caa-f145e0172db9\&variant=undefined}{Wildfires}
\item
  \href{https://www.nytimes3xbfgragh.onion/interactive/2020/climate/trump-environment-rollbacks.html?name=styln-climate\&region=TOP_BANNER\&block=storyline_menu_recirc\&action=click\&pgtype=Interactive\&impression_id=2b20b9b1-f4be-11ea-9caa-f145e0172db9\&variant=undefined}{Trump's
  Changes}
\item
  \href{https://www.nytimes3xbfgragh.onion/interactive/2020/04/19/climate/climate-crash-course-1.html?name=styln-climate\&region=TOP_BANNER\&block=storyline_menu_recirc\&action=click\&pgtype=Interactive\&impression_id=2b20b9b2-f4be-11ea-9caa-f145e0172db9\&variant=undefined}{Climate
  101}
\item
  \href{https://www.nytimes3xbfgragh.onion/interactive/2020/08/24/climate/racism-redlining-cities-global-warming.html?name=styln-climate\&region=TOP_BANNER\&block=storyline_menu_recirc\&action=click\&pgtype=Interactive\&impression_id=2b20b9b3-f4be-11ea-9caa-f145e0172db9\&variant=undefined}{Environmental
  Racism}
\end{itemize}

Advertisement

\protect\hyperlink{after-top}{Continue reading the main story}

\hypertarget{americas-air-quality-worsens-ending-years-of-gains-study-says}{%
\section{America's Air Quality Worsens, Ending Years of Gains, Study
Says}\label{americas-air-quality-worsens-ending-years-of-gains-study-says}}

By \href{https://www.nytimes3xbfgragh.onion/by/nadja-popovich}{Nadja
Popovich}Oct. 24, 2019

\begin{itemize}
\item
\item
\item
\item
\end{itemize}

+25\% difference from 2016 levels

+20

+15

After years of national decline \ldots{}

fine particulate pollution (PM2.5) started to increase after 2016.

+10

+5.5\%

+5

0

2009

2010

2011

2012

2013

2014

2015

2016

2017

2018

+25\% difference from 2016 levels

+20

+15

After years of national decline \ldots{}

fine particulate pollution (PM2.5) started to increase after 2016.

+10

+5.5\%

+5

0

2010

2012

2014

2016

2018

+25\% difference from 2016 levels

+20

+15

After years of national decline \ldots{}

PM2.5 pollution started to increase after 2016.

+10

+5.5\%

+5

0

2010

2012

2014

2016

2018

+25\% difference from 2016 levels

+20

+15

After years of national decline \ldots{}

fine particulate pollution (PM2.5) started to increase after 2016.

+10

+5.5\%

+5

0

2009

2010

2011

2012

2013

2014

2015

2016

2017

2018

+25\% difference from 2016 levels

+20

+15

After years of national decline \ldots{}

PM2.5 pollution started to increase after 2016.

+10

+5.5\%

+5

0

2010

2012

2014

2016

2018

Source: National Bureau of Economic Research

New data reveals that damaging air pollution has increased nationally
since 2016, reversing a decades-long
\href{https://www.nytimes3xbfgragh.onion/interactive/2019/06/19/climate/us-air-pollution-trump.html}{trend
toward cleaner air}.

An analysis of Environmental Protection Agency data
\href{https://www.nber.org/papers/w26381}{published this week} by
researchers at Carnegie Mellon University found that fine particulate
pollution increased 5.5 percent on average across the country between
2016 and 2018, after decreasing nearly 25 percent over the previous
seven years.

``After a decade or so of reductions,'' said Nick Muller, a professor of
economics, engineering and public policy at Carnegie Mellon, and one of
the study's co-authors, ``this increase is a real about-face.''

The research identified
\href{https://www.nytimes3xbfgragh.onion/interactive/2019/10/10/climate/driving-emissions-map.html?rref=collection\%2Fbyline\%2Fnadja-popovich}{recent
increases
in}\href{https://www.nytimes3xbfgragh.onion/interactive/2019/10/10/climate/driving-emissions-map.html?rref=collection\%2Fbyline\%2Fnadja-popovich}{driving}
and the burning of natural gas as likely contributors to the uptick in
unhealthy air, even as coal use and related pollution have declined. In
the West, wildfires contributed to the rise in particulate matter.

+30 \% difference from 2016 levels

+30

West

Midwest

+20

+20

+11.5\%

+9.3\%

+10

+10

Fine particulate pollution (PM2.5)

0

0

­--5

­--5

2010

2012

2014

2016

2018

2010

2012

2014

2016

2018

+30 \% difference from 2016 levels

+30 \%

Northeast

South

+20

+20

+10

+10

0

0

--0.8\%

--1.3\%

­--5

­--5

2010

2012

2014

2016

2018

2010

2012

2014

2016

2018

+30 \% difference from 2016 levels

+30

West

Midwest

+20

+20

+11.5\%

+9.3\%

+10

+10

Fine particulate pollution (PM2.5)

0

0

­--5

­--5

2010

2012

2014

2016

2018

2010

2012

2014

2016

2018

+30 \% difference from 2016 levels

+30 \%

Northeast

South

+20

+20

+10

+10

0

0

--0.8\%

--1.3\%

­--5

­--5

2010

2012

2014

2016

2018

2010

2012

2014

2016

2018

+30 \% difference from 2016 levels

+30

West

Midwest

+20

+20

+11.5\%

+9.3\%

+10

+10

Fine particulate pollution (PM2.5)

0

0

­--5

­--5

2010

2012

2014

2016

2018

2010

2012

2014

2016

2018

+30 \% difference from 2016 levels

+30 \%

Northeast

South

+20

+20

+10

+10

0

0

--0.8\%

--1.3\%

­--5

­--5

2010

2012

2014

2016

2018

2010

2012

2014

2016

2018

+30 \% difference from 2016 levels

West

+20

+11.5\%

+10

Fine particulate pollution (PM2.5)

0

­--5

2010

2012

2014

2016

2018

+30

Midwest

+20

+9.3\%

+10

0

­--5

2010

2012

2014

2016

2018

+30 \% difference from 2016 levels

Northeast

+20

+10

0

--0.8\%

­--5

2010

2012

2014

2016

2018

+30 \%

South

+20

+10

0

--1.3\%

­--5

2010

2012

2014

2016

2018

Source: National Bureau of Economic Research

Researchers also suggested that a decrease in enforcement of the Clean
Air Act may have contributed to the recent rise in pollution. That law
and its subsequent updates put in place strict air pollution standards
for power plants, factories, vehicles and other sources, and has been
credited with dramatically improving air quality across the country and
\href{https://www.epa.gov/clean-air-act-overview/benefits-and-costs-clean-air-act-1990-2020-second-prospective-study}{saving
hundreds of thousands of lives}.

The new analysis estimated that the increase of slightly more than 5
percent in fine particulate pollution nationwide between 2016 and 2018
was associated with nearly 10,000 additional premature deaths during
that time.

Fine particulate pollution -- known as PM2.5 because the particles are
less than 2.5 micrometers in diameter, or one-thirtieth the size of a
human hair -- has been linked to a range of health problems including
asthma and respiratory inflammation, lung cancer, heart attack and
stroke. A recent study found a significant link between air pollution
and
\href{http://nytimes3xbfgragh.onion/2019/10/14/world/asia/china-air-pollution-miscarriages-study.html}{the
risk of miscarriage}.

\begin{itemize}
\item
\item
\item
\item
\end{itemize}

Advertisement

\protect\hyperlink{after-bottom}{Continue reading the main story}

\hypertarget{site-index}{%
\subsection{Site Index}\label{site-index}}

\hypertarget{site-information-navigation}{%
\subsection{Site Information
Navigation}\label{site-information-navigation}}

\begin{itemize}
\tightlist
\item
  \href{https://help.nytimes3xbfgragh.onion/hc/en-us/articles/115014792127-Copyright-notice}{©~2020~The
  New York Times Company}
\end{itemize}

\begin{itemize}
\tightlist
\item
  \href{https://www.nytco.com/}{NYTCo}
\item
  \href{https://help.nytimes3xbfgragh.onion/hc/en-us/articles/115015385887-Contact-Us}{Contact
  Us}
\item
  \href{https://www.nytco.com/careers/}{Work with us}
\item
  \href{https://nytmediakit.com/}{Advertise}
\item
  \href{http://www.tbrandstudio.com/}{T Brand Studio}
\item
  \href{https://www.nytimes3xbfgragh.onion/privacy/cookie-policy\#how-do-i-manage-trackers}{Your
  Ad Choices}
\item
  \href{https://www.nytimes3xbfgragh.onion/privacy}{Privacy}
\item
  \href{https://help.nytimes3xbfgragh.onion/hc/en-us/articles/115014893428-Terms-of-service}{Terms
  of Service}
\item
  \href{https://help.nytimes3xbfgragh.onion/hc/en-us/articles/115014893968-Terms-of-sale}{Terms
  of Sale}
\item
  \href{https://spiderbites.nytimes3xbfgragh.onion}{Site Map}
\item
  \href{https://help.nytimes3xbfgragh.onion/hc/en-us}{Help}
\item
  \href{https://www.nytimes3xbfgragh.onion/subscription?campaignId=37WXW}{Subscriptions}
\end{itemize}
