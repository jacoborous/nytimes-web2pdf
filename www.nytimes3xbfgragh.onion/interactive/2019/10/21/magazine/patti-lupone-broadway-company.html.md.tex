 **NYTimes.com no longer supports Internet Explorer 9 or earlier. Please
upgrade your browser.
\href{http://www.nytimes3xbfgragh.onion/content/help/site/ie9-support.html}{LEARN
MORE »}

**Sections

**Home

**Search

\hypertarget{the-new-york-times}{%
\subsection{\texorpdfstring{\href{http://www.nytimes3xbfgragh.onion/}{The
New York Times}}{The New York Times}}\label{the-new-york-times}}

\hypertarget{-magazine-}{%
\subsubsection{\texorpdfstring{
\href{https://www.nytimes3xbfgragh.onion/section/magazine}{Magazine}
}{ Magazine }}\label{-magazine-}}

 \href{https://www.nytimes3xbfgragh.onion/section/magazine}{Magazine}
\textbar{}Patti LuPone on Getting Bullied by Broadway. And Why She Keeps
Coming Back.

**Close search

\hypertarget{site-search-navigation}{%
\subsection{Site Search Navigation}\label{site-search-navigation}}

Search NYTimes.com

**Clear this text input

Go

\url{https://nyti.ms/2J8d2J3}

\hypertarget{site-navigation}{%
\subsection{Site Navigation}\label{site-navigation}}

\hypertarget{site-mobile-navigation}{%
\subsection{Site Mobile Navigation}\label{site-mobile-navigation}}

\hypertarget{patti-lupone-on-getting-bullied-by-broadway-and-why-she-keeps-coming-back}{%
\section{Patti LuPone on Getting Bullied by Broadway. And Why She Keeps
Coming
Back.}\label{patti-lupone-on-getting-bullied-by-broadway-and-why-she-keeps-coming-back}}

The actress on her hard-won tough-as-nails reputation, ``Evita'' and the
changing nature of celebrity.

\includegraphics{https://static01.graylady3jvrrxbe.onion/newsgraphics/2019/10/13/talk-norton/f7725356839dc87c70b39e7b3c83755965654deb/close.svg}

\textbf{Talk} Oct. 21, 2019

Patti LuPone on getting bullied by Broadway. And why she keeps coming
back.

By David Marchese Photograph by Mamadi Doumbouya

There are many pleasures to be had in watching Patti LuPone perform,
particularly on a Broadway stage. There's her powerful, flexible singing
voice, undiminished at age 70. There's her operatically expressive face,
her sharply detailed characterizations. There's also the plain fact that
to see LuPone at maximum, commanding intensity --- her default mode ---
is to see that most thrilling and increasingly rare of theatrical
sights: a true diva. (And one who has endured a diva's share of
backstage drama.) ``I knew at 4 years old where I was going and what I
was going to do,'' said LuPone, a two-time Tony Award winner who will be
starring in a revival of Stephen Sondheim's ``Company'' next year. She
added with a snap, ``And I didn't think I was going to be in the
chorus.''

\textbf{There's a bit of a paradox going on with Broadway musicals right
now. On one hand, they seem to be in good shape, because shows like
``Dear Evan Hansen'' and ``Frozen'' are making so much money. But on the
other hand, those shows are not really vehicles for traditional
musical-theater stars like yourself.
\href{http://nytimes3xbfgragh.onion\#tooltip-1}{``War
Paint''1}\emph{was}} \textbf{that kind of show and struggled to find an
audience. Does that at all make you wonder where you fit in the Broadway
ecosystem these days?} No. But some of those shows should be in Las
Vegas and not on a Broadway stage. The thing really bugging me now about
Broadway musicals is that they're making me deaf. They're all so damn
loud. But you don't know what's going to hit. You don't know what's
going to flop. I was disappointed that
\href{https://www.nytimes3xbfgragh.onion/2017/04/06/theater/review-war-paint-patti-lupone-christine-ebersole.html}{``War
Paint''} didn't catch on, because it was beautiful, and Christine
Ebersole and I played like gangbusters. How can you know why it didn't
hit? It could have been where the theater was located. It could have
been because other musicals attracted people. So when you ask me how do
I fit: I know that I have box-office draw, and I know that I'm relied on
for it. In a way, that's unfair. The pressure shouldn't be on me to draw
a crowd. The pressure should be on the producers.

\textbf{There was a time on Broadway when having a theatrical star like
Mary Martin or Ethel Merman in a show was a guarantee that it would play
for at least a full season. Is that kind of star power a thing of the
past?} No. They used to write for the stars, and they don't anymore.
Cole Porter and Richard Rodgers were writing for Ethel. So you know you
had a good combination. But I don't think it's over. Last night, there
was a line for the cast of ``Betrayal,'' especially for Charlie Cox and
Tom Hiddleston. Hugh Jackman is going to come onstage for ``The Music
Man.''

\textbf{But Hugh Jackman and Tom Hiddleston are movie stars who can draw
an audience to their stage work. Ethel Merman and Mary Martin were}
\emph{\textbf{Broadway}} \textbf{stars. There's a difference.} O.K.,
you're right. Unless you have some sort of broader visibility, it may be
harder to pull an audience. I think I'm a product of that old line of
musical-theater women, because I don't have that other \emph{thing} to
bring people in. Some people may know me from
\href{http://nytimes3xbfgragh.onion\#tooltip-2}{``Life Goes On''2} or
\href{http://nytimes3xbfgragh.onion\#tooltip-3}{``Steven Universe,''3}
but most know me mostly from musicals. Actually, they know me most from
the commercial for
\href{http://nytimes3xbfgragh.onion\#tooltip-4}{``Evita.''4}

\textbf{Which was a good commercial. Can I ask you a random ``Evita''
question? Why does Evita sing ``Don't Cry for Me, Argentina'' when she
does? Isn't everyone happy for her at that point in the show? I don't
get the narrative logic.} I thought the same thing. I was going, ``What
the hell is this song about?'' I understand exactly what you're saying.
I never wanted to do ``Evita,'' because it was the most bizarre music
I'd ever heard. You're raised on Rodgers and Hammerstein, Meredith
Willson, Lerner and Loewe, and then you hear that? I heard the ``Evita''
concept album, and I went, ``Ow, my ear.''

**Did you read \href{http://nytimes3xbfgragh.onion\#tooltip-5}{Andrew
Lloyd Webber's memoir?5}**No. Am I in it?

\textbf{Oh, yeah.} Oh, dear.

\textbf{He rehashed \href{http://nytimes3xbfgragh.onion\#tooltip-6}{the
expected stuff.6} He also made a point of criticizing your diction.}
\href{http://nytimes3xbfgragh.onion\#tooltip-7}{John Houseman7} used to
call me ``flannel mouth.'' You don't know, when you're in the moment,
that you're not enunciating. As an audience member, I can understand the
problem. I saw ``The Iceman Cometh.'' I did not understand a word those
guys were saying. I've seen a lot of theater where I don't understand
what the actors are saying, because they're forgetting that they need to
project. They need to enunciate. In some of my performances, I was
oblivious to that; I was busy emoting. Apparently, when I was doing
``Three Sisters,'' John Houseman wanted to yell at me about my diction.
They kept him away from me, until he literally strangled me.

\textbf{Literally?} Literally put his hands around my throat and said,
``I want to beat you black and blue until you're bloody all over and you
have bandages all over your face.'' And I'm going, Well, that's bizarre.
Then I just went to pieces. But I'm an emotional, organic actor, and
that gets in the way of me technically speaking clearly. So the fact
that criticism of my diction follows me around makes total sense. Was
Andrew Lloyd Webber talking about ``Sunset Boulevard''?

\textbf{He was talking about ``Evita.''} How could he talk about
``Evita''? The whole thing is sung. He's a jerk. He's a sad sack. He is
the definition of sad sack.

\textbf{Do you like any of his music?} I thought ``Evita'' was the best
thing he and Tim Rice did. But the rest of it is schmaltz.

\textbf{We talked about how what's popular on Broadway has changed. But
I'm wondering, too, if you've seen any differences between your
generation of performers and younger generations?} Yes. I am blown away
by the talent onstage in New York, but I see too many actors relying on
microphones. They do not know how to fill a house with their voice, and
therefore their presence. That's bad. And so is when somebody doesn't
know the history of theater, or who Marlon Brando or Shirley MacLaine or
Chita Rivera are. It's like: What are you doing this for? Are you doing
it because it's a time-honored profession? A necessary profession for
society? Or are you doing it because you want to be famous and rich?

\textbf{You're seeing more of that than you used to?} I think basically
everybody wants to be famous and rich, but I don't see the commitment,
maybe. It takes sacrifice. It's hard work to delve into a character.
It's hard work doing eight shows a week. It's hard work to protect your
instrument, which is your entire body. I say: ``Eight shows a week. No
life.'' That's exactly what it is if you're onstage.

\textbf{A couple of years ago, you were saying you thought you had done
your last Broadway musical. But you're going to be back on Broadway,
\href{https://www.nytimes3xbfgragh.onion/2019/08/30/theater/broadway-company-sondheim-elliott-katrina-lenk-patti-lupone.html}{playing
Joanne in ``Company.''} Did you feel as if you had to reckon with what
\href{http://nytimes3xbfgragh.onion\#tooltip-8}{Elaine Stritch8} did in
that part? I imagine that, even just because of
\href{https://phfilms.com/films/company-original-cast-album/}{the}\href{https://phfilms.com/films/company-original-cast-album/}{D.A.
Pennebaker documentary} that we've all seen, it would be hard not to
have her somewhere in your mind.} It would have been hard not to have
Ethel Merman in my mind for ``Gypsy.'' It would have been hard not to
have Zoe Caldwell in my mind for ``Master Class.'' It would have been
hard not to have Angela Lansbury in my mind for ``Sweeney Todd.'' These
are great actors. Well, Ethel was not such a great actor, but these are
icons. Elaine is Elaine, and I am me. Steve Sondheim actually said to
me, years before I did ``Company,'' that he was surprised that I
understood ``Ladies Who Lunch.''

\textbf{What was that supposed to mean?} Exactly! I think he thought
that I was of the lower classes and wouldn't understand the Upper East
Side. I was surprised that he would think that I wouldn't be able to do
it. Then I was thrilled that he thought I could. You go through all
these things in your mind. You're going, Really? Then you're going, Oh,
great, he loves me! It's crazy.

\textbf{Do you still have doubts about Sondheim's estimation of you?}
Always.

\textbf{Do you still care?} Of course. He's the master. Some actors
don't care. I do. I wanted validation. I think Stephen thinks I'm a
strong person and --- I don't know. I'm speculating on what he thinks. I
don't know what he thinks. Maybe he took a dislike to me early on in my
career. Maybe I'm making this whole thing up. Maybe he likes me. I don't
know. But I know that he's happy with this production of ``Company'' and
was happy with my performance. I think.

\textbf{You know, in
\href{http://nytimes3xbfgragh.onion\#tooltip-9}{Arthur Laurents's9
memoir}, he wrote about taking you out to lunch before your doing
``Gypsy'' and going over some performance ruts that he thought you'd
fallen into. What was he talking about?} ``Ruts''?

\textbf{He used ``ruts.''} He never told me that. Here's the deal. I was
raw meat having gone through that lovely experience of ``Sunset
Boulevard.'' I come home, and I get a telephone call. I'm offered a play
by Arthur Laurents: ``Jolson Sings Again.'' I read the play, and I
didn't like the play. It wasn't very good. I said, ``Are you bringing
this to Broadway?'' And I was told, ``No, we're doing it in Seattle.'' I
went: ``I just spent a year in London. I'm not going to pack a suitcase
and go to Seattle.'' So I said no. Then the producer David Stone got it,
and they \emph{were} going to do it on Broadway. But the deal was
completely unacceptable, so I turned it down. Oh! I'm missing a whole
part. First, I got a call from my manager at the time saying Arthur
Laurents wants you to go to his house and meet with him. I went. I
knocked on the door. Arthur answered, and then out from behind him came
David Saint, who would direct the play. I saw a weak chin on that one. I
was like, He's not going to direct me. Then Arthur was very convincing,
and I went, Yeah, sounds great.

\textbf{Then what happened?} The deal was terrible, and I passed! Then
I'm shooting
\href{https://www.nytimes3xbfgragh.onion/2001/11/09/movies/film-review-forget-the-girl-and-gold-look-for-the-chemistry.html}{``Heist''}
in Montreal, and I'm in my hotel room, and I pick up the phone, and it's
Arthur. He told me I sank his play. In my head I thought, No, actors
don't sink plays; playwrights sink plays. I did say to him that the deal
was terrible. Then he hung up on me. The next thing I heard was that I
was \href{http://nytimes3xbfgragh.onion\#tooltip-10}{banned from his
work.10} All of it. I thought: Why me? I just turned down a play. Why am
I getting beat up? Why do these things happen to me?

\textbf{What's your answer to that question?} I'm telling you the truth:
I don't understand it. I don't understand if it's because people think I
can take it because I'm tough as nails. If I am, I've been made tough by
this business in order to survive, in order to continue to perform,
which is what I was born to do. They're not going to stop me from
getting on a stage. Whatever they tried to do, they, whoever they are,
didn't succeed. But it did succeed because I felt it.

\textbf{Felt what?} I've been bullied in this business. There's
something that happened that I did not put in
\href{http://nytimes3xbfgragh.onion\#tooltip-11}{my book,11} and I wish
I had. \href{http://nytimes3xbfgragh.onion\#tooltip-12}{Hal Prince12}
did something to me. My persona in this business has not been: ``Let's
bed Patti. Get on the casting couch, Patti.'' It's either been: ``Nope,
out the door,'' or getting slammed on the head. The bullying with Hal
Prince had been in the book, and out of respect for the guy, I took it
out. I wish I had left it in, because when we talk about bullying, it
has to be better defined. I've been bullied all my life.

\textbf{By whom?} Starting when I was a kid.
\href{http://nytimes3xbfgragh.onion\#tooltip-13}{My dad13} was the
principal of my elementary school. I remember going to kindergarten, and
I got hit in the face with a snowball with a rock in it. I always
carried around that it happened because my dad was the principal. And my
dad bullied me in front of my class. It wasn't bullying --- he didn't
know what to do --- but in today's world, you would call it bullying. I
ran out of line at school and embraced my father, and my father didn't
even look me in the eye; he took me by the shoulders and put me back in
line without explanation. I was humiliated, not understanding what
happened. Do you treat a kid that way? Now, take all that bullying that
you get used to as a child --- because apparently that's what life is
--- and then you're in show business, and it's the same thing. In the
case with Hal Prince, what happened was so scarring that I said, ``I
will never work with this man again.'' And I never did.

\textbf{Can you tell me what happened with Hal Prince?} Well, it was a
rehearsal with the New York company of ``Evita'' after he had just
opened the L.A. company of the show. He started the rehearsal with a
bullhorn turned up to 10, saying, ``The L.A. company is better than you
are, and now rehearse!'' Then maybe 10 minutes into it, he accused me of
changing blocking. I went, ``No, you changed it in previews.'' An
argument --- this humiliation --- ensued for the entire rehearsal. I
ended up in a fetal position in my dressing room, crying my eyes out.
Stage management came in, and I said: ``Why didn't you defend me? The
changes were in the prompt book.'' They were Hal Prince's men, the stage
management, and one of them said, ``Oh, honey, he does that to all his
leading ladies.'' As if it were acceptable. That was a form of bullying,
but you just go, O.K. I never understood it.

\textbf{Is bullying still accepted in the theatrical world?} Maybe not
now. I don't know how I feel about bullying in show business, because it
has made me stronger. Sometimes you think: Is this a test from the gods?
Is it what you have to go through to get what you want? Or is it just
abuse? In a lot of cases, it is just abuse. But what do you do? There
was nobody I could talk to. That was my ignorance. I should have called
Equity. I should've walked out of rehearsal and called my agent. But I
would've been fired, and I knew that. What Hal Prince did has never left
me. It did many things besides humiliate me. It diminished my status in
the company as the leading lady. He treated me like a stupid chorus
girl. It was so demoralizing and defeating. He actually said, ``Now,
who's going to win this argument?'' I said, ``You, because you're the
director.'' He said: ``That's right. Now sing.'' ``Evita'' was the thing
that shot me to stardom, but when I say I didn't like the experience,
that's one of the reasons. It was hard as hell.

\textbf{You described yourself earlier as an organic, emotional actor.
That force-of-nature aspect of what you do is a huge part of what people
like about your performances. Is that also what} \emph{\textbf{you}}
\textbf{like about what you do?} I am a tragedian. I am a comedian. I am
fearless on a stage. I'm scared to death in my own life. Paranoid.
Terrified. But put me on the stage, and there's nothing I won't do to
the fullest. It hasn't been easy. But there was nothing that was going
to stop me from doing what I was supposed to do. This was my calling.
Does that answer your question?

\textbf{Sort of, but let me go a little deeper. You must know that when
people buy a ticket for a musical with Patti LuPone in it, they want to
see Patti LuPone be} \emph{\textbf{Patti LuPone}} \textbf{up there
onstage. They don't want a shrinking violet. Do those expectations
affect how you approach a performance?} I don't know what people are
coming to see. How much I commit depends on what is required from the
script. If you want to go the distance --- and I always want to go the
distance --- I \emph{will} let it out. The fact that there's a deep well
inside me is just how I was built. That's the Italian in me. There were
a lot of big emotions and big fights and big sobs growing up. When it
was required of me to express those things in a role, I found that the
well went deeper and deeper and deeper. That has to do with what I'm
made of.

\textbf{You said you're scared in your life. Of what?} Everything. A
boogeyman. I am terrified when I go home to Connecticut at dusk. I close
all the first-floor blinds, because I'm afraid somebody is going to be
looking in. If I hear a noise, I'm awake and scared. I don't know where
that came from. But the fearlessness onstage is because that's home to
me. There, I'm not afraid.

\emph{This interview has been edited and condensed from two
conversations.}

David Marchese is a staff writer and the Talk Columnist for the
magazine.

\emph{This interview has been edited and condensed from two
conversations.}

\hypertarget{related-coverage}{%
\subsection{Related Coverage}\label{related-coverage}}

\begin{itemize}
\tightlist
\item
  \href{https://www.nytimes3xbfgragh.onion/interactive/2019/10/07/magazine/edward-norton-interview.html}{}
\item
  \href{https://www.nytimes3xbfgragh.onion/interactive/2019/09/15/magazine/pam-grier-interview.html}{}
\item
  \href{https://www.nytimes3xbfgragh.onion/interactive/2019/09/30/magazine/bill-maher-interview.html}{}
\item
  \href{https://www.nytimes3xbfgragh.onion/interactive/2019/08/07/magazine/nicolas-cage-interview.html}{}
\item
  \href{https://www.nytimes3xbfgragh.onion/interactive/2019/09/02/magazine/ram-dass-interview.html}{}
\end{itemize}

\textbf{Correction: Oct. 21, 2019}

An earlier version of a headline with this article rendered incorrectly
part of the interview subject's surname. It is Patti LuPone, not Lupone.

\hypertarget{more-on-nytimescom}{%
\subsection{More on NYTimes.com}\label{more-on-nytimescom}}

Advertisement

\hypertarget{site-information-navigation}{%
\subsection{Site Information
Navigation}\label{site-information-navigation}}

\begin{itemize}
\tightlist
\item
  \href{https://help.nytimes3xbfgragh.onion/hc/en-us/articles/115014792127-Copyright-notice}{©
  2020 The New York Times Company}
\item
  \href{https://www.nytimes3xbfgragh.onion}{Home}
\item
  \href{https://www.nytimes3xbfgragh.onion/search/}{Search}
\item
  Accessibility concerns? Email us at
  \href{mailto:accessibility@NYTimes.com}{\nolinkurl{accessibility@NYTimes.com}}.
  We would love to hear from you.
\item
  \href{https://help.nytimes3xbfgragh.onion/hc/en-us/articles/115015385887-Contact-Us}{Contact
  Us}
\item
  \href{https://www.nytco.com/careers/}{Work with us}
\item
  \href{https://nytmediakit.com/}{Advertise}
\item
  \href{https://help.nytimes3xbfgragh.onion/hc/en-us/articles/115014892108-Privacy-policy\#pp}{Your
  Ad Choices}
\item
  \href{https://help.nytimes3xbfgragh.onion/hc/en-us/articles/115014892108-Privacy-policy}{Privacy}
\item
  \href{https://help.nytimes3xbfgragh.onion/hc/en-us/articles/115014893428-Terms-of-service}{Terms
  of Service}
\item
  \href{https://help.nytimes3xbfgragh.onion/hc/en-us/articles/115014893968-Terms-of-sale}{Terms
  of Sale}
\end{itemize}

\hypertarget{site-information-navigation-1}{%
\subsection{Site Information
Navigation}\label{site-information-navigation-1}}

\begin{itemize}
\tightlist
\item
  \href{https://spiderbites.nytimes3xbfgragh.onion}{Site Map}
\item
  \href{https://help.nytimes3xbfgragh.onion/hc/en-us}{Help}
\item
  \href{https://help.nytimes3xbfgragh.onion/hc/en-us/articles/115015385887-Contact-Us?redir=myacc}{Site
  Feedback}
\item
  \href{https://www.nytimes3xbfgragh.onion/subscription?campaignId=37WXW}{Subscriptions}
\end{itemize}
