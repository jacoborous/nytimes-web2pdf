 **NYTimes.com no longer supports Internet Explorer 9 or earlier. Please
upgrade your browser.
\href{http://www.nytimes3xbfgragh.onion/content/help/site/ie9-support.html}{LEARN
MORE »}

**Sections

**Home

**Search

\hypertarget{the-new-york-times}{%
\subsection{\texorpdfstring{\href{http://www.nytimes3xbfgragh.onion/}{The
New York Times}}{The New York Times}}\label{the-new-york-times}}

\hypertarget{-magazine-}{%
\subsubsection{\texorpdfstring{
\href{https://www.nytimes3xbfgragh.onion/section/magazine}{Magazine}
}{ Magazine }}\label{-magazine-}}

 \href{https://www.nytimes3xbfgragh.onion/section/magazine}{Magazine}
\textbar{}Rosalía's Incredible Journey From Flamenco to Megastardom

**Close search

\hypertarget{site-search-navigation}{%
\subsection{Site Search Navigation}\label{site-search-navigation}}

Search NYTimes.com

**Clear this text input

Go

\url{https://nyti.ms/33azEAc}

\hypertarget{site-navigation}{%
\subsection{Site Navigation}\label{site-navigation}}

\hypertarget{site-mobile-navigation}{%
\subsection{Site Mobile Navigation}\label{site-mobile-navigation}}

\hypertarget{rosaluxedas-incredible-journey-from-flamenco-to-megastardom}{%
\section{Rosalía's Incredible Journey From Flamenco to
Megastardom}\label{rosaluxedas-incredible-journey-from-flamenco-to-megastardom}}

Before her videos were racking up millions of views on YouTube, the
artist spent more than a decade training in one of the world's oldest
and most complex musical art forms.

\includegraphics{https://static01.graylady3jvrrxbe.onion/newsgraphics/2019/10/13/culture/1a9865dae9d1a0d21273eb2bff896ea8752f0273/close.svg}

\hypertarget{rosaluxedas-incredible-journey-from-flamenco-to-megastardom-1}{%
\section{Rosalía's Incredible Journey From Flamenco to
Megastardom}\label{rosaluxedas-incredible-journey-from-flamenco-to-megastardom-1}}

Before her videos were racking up millions of views on YouTube, the
artist spent more than a decade training in one of the world's oldest
and most complex musical art forms.

By MARCELA VALDES OCT. 8, 2019

\href{https://www.nytimes3xbfgragh.onion/2019/10/08/magazine/rosalia-espanol.html}{Leer
en español}

One Wednesday in July, 40,000 people gathered on a synthetic-grass field
on the outskirts of Madrid to watch Rosalía headline the Mad Cool
festival's ``welcome party.'' For the next three days, 103 acts ---
including Bon Iver, Iggy Pop and Prophets of Rage --- would appear at
the festival grounds, which were ornamented by a towering white Ferris
wheel. Wednesday, however, was for Rosalía, Spain's greatest pop export
since Julio Iglesias. For her, Mad Cool was a quick stop in the middle
of a nine-month tour through Latin America, Europe and North America.
But as she took the stage in white platform sneakers and an aqua top
with an enormous ruffle running over her arms and chest, she gushed in
Spanish, ``I am so, so, so thrilled to be here!''

Giant digital screens hung on either side of the stage, projecting her
face to the crowd. Less visible was the garter-belt tattoo peeking out
below her tiny, high-waisted shorts: a replica of
\href{http://www.artnet.com/artists/valie-export/tatoo-i-wa4xN3xGsJ3mOixnrc1O3A2}{the
one} that the Austrian feminist performance artist Valie Export gave
herself onstage in 1970. Counterpointing sexiness with a kind of
SoulCycle strength, Rosalía's troupe of female dancers wore white bike
shorts and hoop earrings. They finished ``Como Ali'' --- an unreleased
tribute to Muhammad Ali --- with a series of long, rapid-fire air
punches, then held their hands overhead in fists.

On one side of the stage stood two male percussionists --- experts in
performing palmas, traditional flamenco hand claps. Palmas formed the
backbone of Rosalía's next number, ``Pienso en Tu Mirá'' (``Thinking of
Your Gaze''), a deceptively sweet-sounding tune about jealousy that
mixes voices, palmas and electronic samples in a shifting 12/8, 10/8
beat pattern. The song,
\href{https://pitchfork.com/features/rising/get-to-know-rosalia-the-spanish-singer-giving-flamencos-age-old-sound-a-bracingly-modern-twist/}{a
Pitchfork critic wrote}, ``stands out from virtually everything else on
the global pop landscape.'' It appears on Rosalía's 2018 CD, ``El Mal
Querer'' (``Bad Love''), a.k.a. ``E.M.Q.,'' which earned raves all over
Europe and the United States as one of the best albums of last year. The
album's first track,
\href{https://www.nytimes3xbfgragh.onion/interactive/2019/03/07/magazine/top-songs.html\#/rosala}{``Malamente''
(``Badly'')}, streamed 15 million times in its first week in May 2018
and went platinum on the United States Latin charts earlier this year.
``Malamente'' also earned Rosalía two Latin Grammys and five total
nominations, making her the most-nominated female artist in 2018. Last
month, with the rest of ``E.M.Q.'' finally eligible, she repeated the
feat, picking up five more Latin Grammy nominations, including Album of
the Year. (This year's winners will be announced next month.)

``E.M.Q.'' transformed Rosalía's life, turning her into a sought-after
collaborator who has recorded songs with James Blake, A. Chal, Ozuna and
Pharrell Williams. In March, she posted
\href{https://www.youtube.com/watch?v=p7bfOZek9t4}{``Con Altura''}
(``High Class''), a reggaeton collaboration with J. Balvin, on YouTube.
It now has more than 951 million views, crushing Billie Ellish's
\href{https://www.youtube.com/watch?v=DyDfgMOUjCI}{``Bad Guy''} (567
million views), Ariana Grande's
\href{https://www.youtube.com/watch?v=gl1aHhXnN1k}{``thank u, next''}
(432 million) and Taylor Swift's
\href{https://www.youtube.com/watch?v=Dkk9gvTmCXY}{``You Need to Calm
Down''} (145 million). When ``Con Altura'' took the MTV Video Music
Award for Best Latin in August, J. Balvin bowed toward Rosalía several
times onstage, a gentleman's acknowledgment of her role in composing the
hit. ``I think she's the only artist who can be compared to Beyoncé in
the Latin world,'' Juanes, the Colombian rock star who has won two
Grammys and 23 Latin Grammys, told me. This month she appears in the
opening scene of Pedro Almodóvar's new film
\href{https://www.nytimes3xbfgragh.onion/2019/10/03/movies/pain-and-glory-review.html}{``Dolor
y Gloria''} (``Pain and Glory''), cast opposite Penélope Cruz.

Success like this inevitably provokes backlash. In Spain, rumors suggest
that Rosalía is a fake created by industry professionals to satisfy
market trends. Spanish Romani Gypsies have attacked her for using words
of \emph{caló} (Romani dialect) in her lyrics and for adopting
Andalusian pronunciations and street styles in her videos. Catalan
nationalists have complained that she should be using her platform to
win support for their independence movement. In the United States, she
has been accused of ``Latinx appropriation'' by critics on Twitter who
argue that as a European country, Spain should be excluded from winning
Best Latin awards. But if you love music, Rosalía's groundbreaking
compositions and otherworldly voice are themselves the best answers to
these sociocultural darts. Before she started topping YouTube and
Spotify ranks, Rosalía spent more than a decade training in flamenco,
one of the world's oldest, most heartfelt and most complex musical art
forms. It is as if a rising mezzo-soprano decided to leave opera and
bring coloratura to R\&B.

``A song with a cappella or without a cappella?'' Rosalía asked the
crowd at Mad Cool.

``A cappella!'' the field shouted

``I need lots, lots, lots, lots of silence, O.K.?''

The field hushed. Everyone onstage went silent. As Rosalía poured out
the flamenco classic
\href{https://www.youtube.com/watch?v=0OMwDZUWl5g}{``Catalina,''} people
wiped away tears. Her voice is both raw and liquid, vaulting smoothly
from aching tenderness to angry longing. When she performs pure
flamenco, Rosalía often sounds as if she's pulling her heart out through
her mouth. Duende --- the ability to transmit deep, authentic emotion
--- is a moment of nearly mystical self-effacement in flamenco, akin to
an actor's disappearance inside a role. Its emotional vulnerability
cannot be faked. Whenever she performs, Rosalía told me, she always
tries to find that moment when ``you're not there as an artist, not
there as a person with your first and last name,'' she said. ``You're
nothing more than a channel'' for the song's ``soul'' to pass through to
an audience. ``You're there more than ever, and super awake, but at the
same time you're gone.'' She smiled with embarrassment, as if she were
sharing a secret.

Two Spaniards, a Canadian, a Colombian and a clip from a Dominican
television show came together to make ``Con Altura,'' an unlikely
tribute to reggaeton, and one of the hottest international songs of the
summer. By Joe Coscarelli, Alexandra Eaton, Alicia DeSantis, Antonio de
Luca, Will Lloyd and Eden Weingart

\textbf{The day after} her Mad Cool performance, I found Rosalía Vila
Tobella lounging on a red velvet love seat in her hotel suite, wearing
no makeup, her hair twisted into a messy topknot. At 26, she could pass
for 19, and much of her style these days seems like an attempt to
balance her genetic predisposition to look adorable with expressions of
her ferocious drive. The print on her romper was psychedelic flames. She
gestured with nails that were long and sharp, polished pink and
ornamented with tiny doughnuts, avocados, caramels and kittens.

``I started from zero,'' she told me in Spanish. ``Nobody in my family
is connected to the industry. Not a single contact in the music industry
or in the entertainment industry.'' Rosalía didn't want to talk much
about her family, whose privacy she says she tries to protect. But she,
her parents and her younger sister, Pilar, are close. Her mother, who
was an executive at a local company, quit to become Rosalía's business
manager. Pilar, who studied art history, often works as Rosalía's
stylist and creative consultant. Touring for ``E.M.Q.,'' Rosalía has
traveled with them both.

The industry that, indirectly, made Rosalía's career was car
manufacturing. In 1989, S.E.A.T., Spain's largest auto company, opened a
facility on the outskirts of Rosalía's hometown, Sant Esteve Sesrovires,
in Catalonia, in northeast Spain. S.E.A.T.'s move turned a sleepy
collection of red-tiled roofs and tree-lined streets --- best known as
the place where Chupa Chups lollipops were invented --- into an
industrial hub. Corporations sowed factories and warehouses in the
rolling hills surrounding her town, and this crop lured workers from all
over Spain. In 30 years, the town's population quadrupled, to 7,800.
Many of these workers came from Andalusia, a region in southern Spain,
and many of Rosalía's friends were their children. At her house, the
stereo played Bob Marley, Bruce Springsteen, Bob Dylan and Queen, but
her friends listened to flamenco.

The origins of flamenco are not entirely known. The music emerged in
Andalusia, which has hosted Arab, Jewish, Romani and Christian cultures
for at least 600 years, and the Romani have always been associated with
its performance, though flamenco's guitar style owes much to Catalonia.
One day when Rosalía was 13, she was hanging with her friends in a park
with some souped-up cars --- doors open, songs blasting into the trees
--- when someone put on Camarón de la Isla, one of the greatest flamenco
singers of all time. Listening to him, she told the newspaper El Mundo,
``my head exploded.''

In the wake of that blast, Rosalía fixated on learning to sing flamenco.
Traditionally, cantaoras absorb flamenco's complexities through their
families, in the same way that other children learn language. For
Rosalía, that wasn't an option. So she took flamenco dance classes and
inhaled records by Camarón, who cut several well-regarded albums in the
1970s with the guitarist Paco de Lucía. These albums are pure flamenco:
nothing more than singing, guitar and palmas. But in 1979 Camarón broke
with de Lucía to make ``La Leyenda del Tiempo'' (``The Legend of
Time''), an album inspired by progressive rock that used an electric
bass. In flamenco circles, it was a scandal, like Bob Dylan's going
electric at the Newport Folk Festival. Some people in the press called
Camarón the ``Gypsy Mick Jagger.'' Camarón and de Lucía made up,
recording some stunning albums in the 1980s, but the friction between
pure flamenco and flamenco fusionists has never really disappeared. In
Sant Esteve Sesrovires the winners were Camarón's fusion offspring;
Rosalía saturated herself with their music. And she sang wherever she
could: at home, in the streets, at local flamenco performances.

\textbf{``I'm certain that} I will be an artist,'' Rosalía declared in
2007 on the Barcelona talent contest ``Tú Sí Que Vales'' (roughly,
``You're Worth It''), when she was only 15. She took the stage in
spike-heeled boots, tight jeans and a tank top. Playing an acoustic
guitar, with a medallion of the Virgin Mary dangling from her neck, she
sang ``Como en un Mar Eterno,'' a nasal ballad by the flamenco fusionist
Hanna. The judges looked bored. One of them demanded that Rosalía
display ``some character.'' In response, she belted out ``No One,'' by
Alicia Keys. That was enough to pass her on to the finals, where she
bombed. ``Rosalía, you were regularly off-key during the song,'' a judge
pointed out. ``I can't do everything,'' she responded defensively. ``I'm
trying to perform, sing and dance.''

Rosalía told me that she auditioned for ``Tú Sí Que Vales'' because it
was the only way she saw to break into show business. Failure revised
her ambitions. After that she wanted, above all, to become a great
musician. She gave up dancing and got serious about learning piano. She
started composing music, with melodies, harmonies, chords and lyrics,
trying to teach herself song structure. ``I wanted to have absolute
control over my music,'' she told me, ``from the chords and the voicings
of the songs to the arrangements and the production.''

\subsection{}

There was just one problem: her voice. All those years of faking a
\emph{cantaora's} power without proper training had damaged her vocal
cords. She could no longer project normally. Doctors recommended
surgery. After that, she needed a year of vocal rehabilitation.
``There's something I'm supposed to learn from this: What is it?''
Rosalía recalls thinking during rehab.

Rosalía believes that nothing happens by chance. ``I don't force
things,'' she told me. ``I can have a wish, and then I let God lead me
on the path, bringing me what I need --- and always trying to be alert
to receive it.'' As she recovered in silence, she learned to appreciate
the discipline of vocal technique. And she realized that there was
someone who could train her to become a cantaora. He taught at
Barcelona's premier music school, La Escola Superior de Música de
Catalunya (ESMUC), which usually accepts only one university student a
year to study flamenco singing. But it just so happened that when
Rosalía needed him most, that professor, José Miguel Vizcaya, was also
taking students at another Barcelona music school, the Taller de Músics,
where teenagers could enroll.

The day Rosalía first appeared before Vizcaya with a doctor's report in
her hand, he ran through his standard series of questions. What do you
like about flamenco? How much flamenco have you listened to? Does your
family sing flamenco? What types of music do you sing now? Then he
tested her. Sing me some flamenco. Sing me some jazz. Sing me some pop.
In this way, Vizcaya assembles a ``diagnostic profile'' of his students,
so he can decide how he'll train them. He soon figured out that Rosalía
was an absolute amateur: The only thing she knew about pure flamenco was
Camarón. ``I didn't know anything,'' she agreed. ``And he taught me
everything, everything, everything.''

Vizcaya, who performs under the name El Chiqui de la Línea and is called
Chiqui by everyone, has gray hair, kind eyes and a serious, patient
manner. He was born in 1951 in Cádiz, the same part of Andalusia where
Camarón was born in 1950. His father sang flamenco at bars and parties;
Chiqui himself began singing at flamenco clubs when he was 19. A
brilliant \emph{cantaor}, he signed a contract with Los Tarantos, one of
Barcelona's oldest flamenco venues, in the 1970s. There, at the height
of the tourist season, he sang four performances a day. After he
accepted an offer to teach at ESMUC in 2002, he pioneered a way of
translating flamenco's tradition of private mentorship to a university
setting.

Working with Chiqui, Rosalía stretched the range of her voice, learning
to project while maintaining its flexibility and clarity. ``I sang with
fear after my operation,'' Rosalía said. ``I didn't want to damage my
cords again. And I had to sort of relearn to sing.'' With Chiqui, she
also learned to improvise melismas, the singing of multiple notes over a
single syllable of lyric text. The song ``Nos Quedamos Solitos''
(``We're Left All Alone''), which Rosalía recorded for her first album,
shows how this works. The lyrics are just a few sentences, roughly: ``It
was 2 in the morning/my brother came to wake me/why don't you wake,
little brother?/Our mother has died/and we're left all alone.'' Rosalía
draws out these words over several minutes with melismas, making you
feel the suspense of the events: the waking, the older brother's voice,
the death, the abandonment. Melismas also transform her singing into an
embroidered sob that makes you feel the children's sorrow. Almost every
time I listen to it, it brings me to tears.

Melismas are not notated on sheet music. To study them, \emph{cantaoras}
must train their ears. Chiqui taught Rosalía to dissect them note by
note so she could replicate the recorded songs that he gave her to
practice. ``You have to learn what's happening, one by one, until you
know how to execute,'' she said. That's part of the reason becoming a
good \emph{cantaora} takes years. Flamenco melodies can switch from the
modern major scale, common in Western music, to the Phrygian scale,
common in Arabic and Klezmer music, in a phrase. Rosalía couldn't read
these shifts in the songs she studied. She needed to hear them.

Rosalía may have known nothing when Chiqui met her, but he told me that
she had a ``perfect ear,'' a prodigious memory and an exceptional work
ethic. Songs that would take an average student nearly a month to learn,
she would master in a week. ``She drives me crazy,'' Chiqui said. ``In
classes when she sang the things that I assigned her and she interpreted
them'' --- he reached forward and mimed pinching both her cheeks with
pleasure --- ``I couldn't stand how well she did it. She was
tremendous.'' The admiration was mutual. When Chiqui stopped teaching at
Taller de Músics, Rosalía applied to ESMUC so that she could follow him
there. For a while, Chiqui told me, Rosalía considered recording a pure
flamenco album in homage to one of her favorite \emph{cantaoras},
Pastora Pavón Cruz, a.k.a. La Niña de los Peines (the girl with the
combs). In the end, however, she dropped that idea in favor of recording
an experimental flamenco album with a guitarist who once performed in a
punk-rock band. Rosalía didn't have just discipline, taste and ambition;
like all great artists, she also had an appetite for risk.

\textbf{``In `Los Ángeles'} you can find traditional lyrics, traditional
melodies from flamenco,'' Rosalía said, gesturing with her
doughnut-avocado nails as she described her first album. What isn't
traditional, she said, ``is the way in which it's performed, both in the
singing and the guitar and above all in the production.'' The CD, which
Rosalía released when she was 24, sprouted from her friendship with Raül
Fernandez Miró, a classically trained pianist 16 years her senior who
performs under the name Raül Refree. In the 1990s, Refree ditched music
school to form a punk band, then went on to become an important producer
and musician in Barcelona. The first time they met, in 2015, they talked
not only flamenco but also James Blake, Kanye West and Frank Ocean.
After that, they met in his studio several times a month, sharing music
in long listening sessions. ``It was like a festival of eclecticism,''
Refree told me. She might play him a reggaeton video. He might respond
with a clip of American folk. Then they might listen to some pure
flamenco or some Glenn Gould. They never talked about recording
together. But one day that fall, Rosalía began singing Bonnie ``Prince''
Billy's ``I See a Darkness.'' Refree, who had introduced her to the
ballad, sat down at his piano to accompany her.

A few weeks later, she suggested they perform a flamenco show at a bar
in Barcelona in front of an audience of about 20 people. Though he had
produced flamenco albums, Refree had never played flamenco live. But
that night he improvised a flamenco-inspired response on guitar to
Rosalía's melismas. For both, the set felt astonishing. ``We could leap
into the void and catch each other,'' Refree recalled. Rosalía said:
``For me it was, like, cathartic. Every time I sang I connected very
viscerally, you know?'' Walking together afterward, they decided that
they had an obligation to make an album.

The songs Rosalía picked for ``Los Ángeles'' dated mostly to the 1940s
and 1950s, the era when La Niña de los Peines was at her height. Ever
since the 1970s, flamenco aficionados have favored rough, husky
\emph{afillá} voices, which many people associate with \emph{duende}.
But for this album, Rosalía and Refree both preferred laína voices:
clean, high voices, ideal for executing elaborate vocal arabesques. This
style suits Rosalía's own voice, and Chiqui, with his big data set of
flamenco memories, spent years helping her explore it. Many of the
recordings Rosalía took to Refree's studio were old and scratchy, time
capsules from another age. Together they improvised new versions of each
one. ``What we wanted was to find a personal reading of that song,''
Refree said.

Even then Rosalía knew that she wanted to combine flamenco with
electronic music. During the years she trained with Chiqui, she
performed nonstop in and around Barcelona. She sang in so many tablaos,
jazz bars and hip-hop jam sessions that she couldn't remember them all.
She sang at weddings, private parties and restaurants. She performed in
a film festival in Panama and an avant-garde theatrical production in
Singapore. She appeared on two dance hall tracks by her friend C.
Tangana and on hip-hop tracks by DJ Swet and Cálido Lehamo. She joined a
baroque chorus at ESMUC. And, to earn a living, she sometimes recorded
music for commercials. With Refree, she even tried out a few songs that
she had composed but decided to kept them off ``Los Ángeles.''

She deployed all this experience when she wrapped recording on ``Los
Ángeles'' and began devising the senior thesis that eventually turned
into ``El Mal Querer.'' Her first thought was simply to find a way to
draw melismas over danceable beats, so she could do more than sit in a
chair, the way \emph{cantaoras} traditionally perform. On the
recommendation of a friend, she read ``Flamenca,'' an anonymous lyric
poem from the 13th century that has nothing to do with flamenco. She
threw out most of the poem's love-triangle plot and all of its
aristocratic fawning. She kept its vision of a toxic marriage: a man
whose insecurity leads him to lock up his wife. She blurred the
gender-based roles and read up on the psychology of jealousy. She gave
the songs she composed secondary titles like ``Chapter 2: Wedding'' and
``Chapter 4: Fight'' to suggest a narrative. Her lyrics read like a
feminist version of ``Carmen,'' a bad romance told mostly from a female
point of view. When ``Los Ángeles'' earned her a Latin Grammy nomination
for Best New Artist in 2017, she had already performed early versions of
``El Mal Querer'' at ESMUC.

\textbf{Rolf Bäcker,} a musicologist and professor at ESMUC who has
studied flamenco, told me a funny story about a Portuguese jazz expert
who once decided to check out some flamenco in Seville. The expert came
out of the \emph{tablao} crying: He couldn't understand the rhythms.
Each flamenco song style, or palo, is associated not only with a
particular kind of melody and harmony but also with a specific metrical
pattern. Many \emph{palos} have shifting beat accents, known as
amalgamated rhythms, which can sound like fused meters: a 3/4, for
example, meshed with a 2/4. ``It's super hard for someone who doesn't
come from listening to amalgamated rhythms,'' Rosalía told me on the
velvet love seat in her hotel.

I asked her to show me how it was done.

She leaned forward on the couch, curving her body over her hands to
teach me a fast palo called \emph{bulería}. The moment she began
striking her palms together, something flicked on: an intensity and also
a playfulness. The beat was in her hands, her right foot, her shoulders,
her throat. She clicked syncopations with her tongue and growled Mae
West-like ``ums'' on each beat. ``Think in like 6/8,'' she said,
clapping quickly, ``and the \emph{bulería} is here: \emph{One}, two,
three. \emph{Four}, five, six.''

I tried to clap and count along, but soon messed up.

``Think in six claps.'' She got me clapping six fast monotonous beats.
``And now, we'll give it accent. \emph{One}, two, three. \emph{Four},
five, six.'' I fumbled along, like a toddler chasing after an adult.
Later, when I practiced alone, I saw that the rhythm feels like a
driving waltz, at once elegant and aggressive. But, really, Rosalía had
taken pity on me. In many of her songs, the pattern is harder. ``Pienso
en Tu Mirá,'' for example, uses an amalgamated 12/8. It took her years,
she said, to feel comfortable singing in this kind of rhythm. ``It's not
something immediate. It's something that each time you feel that the
more that you know, the more you realize that you need to keep studying
and learning, because there are details. The ones who know most are the
older people. Grandparents are the ones who sing \emph{bulería} best.''

Pablo Díaz-Reixa, known as El Guincho, knew nothing about amalgamated
rhythms or Phrygian scales or melismas before he collaborated with
Rosalía. An electronic musician and producer who once worked with Björk,
El Guincho disappeared from the music scene for three years to care for
his mother, who eventually died of cancer. But in early 2016, he
resurfaced with a dense, brilliant album called ``HiperAsia,'' which
showcased his skill with sampling and vocal production. A few months
after it appeared, Rosalía invited him to watch her perform a pure
flamenco concert. By then, she already knew that she wanted ``E.M.Q.''
to be what she called a ``voice-centric'' project that sampled different
voices and played with vocal harmonies.

She later invited him to help her record a song for ``E.M.Q.,'' ``A
Ningún Hombre.'' Their chemistry in the studio felt natural, Rosalía
told me. ``I've worked with other guys and the way Pablo is is very
special, because Pablo really listens a lot,'' Rosalía said. ``He
embraces the fact that I make a lot of decisions.'' They spent the next
year and half making ``E.M.Q.,'' meeting nearly every day to work at his
place on laptops, a sound card, a microphone and a keyboard. If she
taught him flamenco, El Guincho taught her about song structure and
helped refine her lyrics. By the time they cut their final two songs ---
``Malamente'' and ``Pienso en Tu Mirá'' --- they were working hand in
hand on every aspect of the composition and production.

To my ears, ``E.M.Q.'' sounds like an album about escaping domestic
abuse. There are sounds that suggest beating, weeping, knifing. When I
mentioned this to Rosalía, she smiled: ``That's an interpretation.'' She
kept the lyrics ambiguous, so that different people could hear different
plots as they listened to the songs. One of the hardest for her to write
was ``De Aquí No Sales'' (``You're Not Getting Out of Here''), which
required her to find a way to speak truthfully from a place full of
rage:

\begin{quote}
With the back of my hand\\
I make it clear to you.\\
I sell you bitter sorrows.\\
Caramels I can also do.
\end{quote}

These lines, even in translation, distill the seesaw of abuse. They also
allude to the \emph{cantaor} Macandé, a peddler whose melisma-filled
sales cries for caramels were once a sensation in Cadíz. In
``Malamente'' and elsewhere she inserted other Easter eggs, a few words
of \emph{caló} that would later enrage some Romani. Rosalía never
intended to appropriate Gypsy culture, she told me. For her, the words
were simply ``a wink honoring that flamenco tradition where \emph{caló}
is present and the Gypsy community has contributed so much.''

Rosalía loved working with El Guincho, but there were times while they
made ``E.M.Q.'' that she felt nervous, even anguished. She was in debt,
taking on serious artistic and financial risks. What if her experiments
mixing flamenco with urban music didn't pay off? What if nobody would
bet on her? She wasn't signed with a record label. Independence let her
make ``E.M.Q.'' exactly how she wanted. It also meant she had no
guarantee that her music would ever see proper distribution. Then she
met Juanes.

\textbf{In September 2017,} the Colombian superstar Juanes saw Rosalía
sing pure flamenco at a theater in Madrid. Looking every inch the
\emph{cantaora}, she appeared before an audience of about 100 people
wearing a long velvet dress and sitting in a chair next to the Romani
guitarist Joselito Acedo. ``And this woman started singing,'' Juanes
told me. ``I wanted to die. I mean, I had never felt something so strong
with someone singing in front of me like what I felt that day, and on
top of everything else she was such a young woman, you know? For me it
was like seeing Carlos Gardel or Edith Piaf or someone like that sing.''

The first thing Juanes did after leaving the theater was call his
business partner, Rebeca León, a former vice president of Latin talent
at AEG Live/Goldenvoice. ``Rebeca,'' he says he told her, ``please, you
have to take a train to Barcelona. You have to talk with Rosalía. You
have to talk with this girl.'' Earlier that year, Juanes and León had
started the Latin entertainment company Lionfish. The company's
strategy, León told Vanity Fair España last year, is to find artists who
``are bilingual, bicultural and who can easily turn into global
phenomena.'' Their clients have included the Colombian music producer
Sky Rompiendo, the Dominican-American trap artist Fuego and the
Colombian reggaeton superstar J. Balvin. In Barcelona, León found
Rosalía finishing ``El Mal Querer.'' In January 2018, León joined
Rosalía's team as the artist's manager.

But even León had trouble finding a home for ``E.M.Q.'' ``In the
beginning many people didn't understand,'' León told Variety. ``They
said it was too left of center, you're gonna have to spend too much
money, it's never gonna work.'' In cases like these, YouTube videos have
become a lifeline for Spanish-speaking artists. A decade ago, they
needed to sing in English to win corporate support and penetrate
international markets. ``What YouTube has done is opened up both genres
and different-language music,'' Vivien Lewit, YouTube's global head of
artist relations told me. ``The music itself is connecting with the
audience, and the language itself seems to be in a place where it's
almost secondary.'' Latin artists now make up more than 30 percent of
YouTube's billion-views club, and so far this year, five of YouTube's
Top 10 most-viewed artists are Latin:
\href{https://www.youtube.com/channel/UCt-k6JwNWHMXDBGm9IYHdsg}{J.
Balvin},
\href{https://www.youtube.com/channel/UCRI7hheejBbWS6etTNwMT0g}{Anuel
AA},
\href{https://www.youtube.com/channel/UCjIA3wwhi0QjSOXAZwOXbPA}{Ozuna},
\href{https://www.youtube.com/channel/UCmBA_wu8xGg1OfOkfW13Q0Q}{Bad
Bunny} and
\href{https://www.youtube.com/user/DaddyYankeeOFFTV/videos}{Daddy
Yankee}. Numbers like these can convince record-label executives that
international artists are worth signing.

For the videos that she needed to make ``E.M.Q.'' feel accessible,
Rosalía turned to her friends at Canada, a Barcelona company that shoots
multinational advertisements and music videos. In their productions for
``Malamente'' and ``Pienso en Tu Mirá,'' the director Nicolás Méndez
stirred together Spanish icons and urban signifiers: olives and big-rig
trucks, bullfighters and track pants. In one shot, a hooded Catholic
penitent on a skateboard does an ollie in front of a huge cross. Setting
Rosalía's music in industrial parks like the ones surrounding her
hometown, Méndez revealed its potential to express a new vision of
Spanish identity, one that is profoundly traditional and yet urban.

\href{https://www.youtube.com/watch?v=Rht7rBHuXW8}{``Malamente''} now
has more than 107 million views on YouTube.
\href{https://www.youtube.com/watch?v=p_4coiRG_BI}{``Pienso en Tu
Mirá''} has crossed 50 million. Weeks after they went live, Rosalía
signed a contract with Sony Spain. Suddenly ``E.M.Q.,'' an album made by
two friends holed up in an apartment, didn't look so left of center, and
Rosalía had secured the backing she needed to promote it all over the
world.

\textbf{This summer,} at the MTV music awards, Rosalía took the stage in
a crystal-trimmed bustier and black platform boots and performed the
closest thing that she has written to a straight love song: ``Yo x Ti,
Tu x Mí'' (``Me x You, You x Me''), a collaboration with El Guincho and
the Puerto Rican trap and reggaeton sensation Ozuna. Rosalía sang it
onstage shortly after Camila Cabello and Shawn Mendes did their love
song ``Señorita.'' The contrast was impressive. There was Cabello in
heels and a transparent dress rubbing against Mendes like a cat. And
then there was Rosalía mixing flamenco gestures with modern and African
dance and hip-hop. Onstage, she was fierce: Ozuna's equal, not his pet.

Dancing was the first thing that Rosalía gave up when she decided to
become a serious musician. And it was the last element she needed to
transform herself into a global pop sensation. She could have simply
twerked and stripped, turning herself into a moving centerfold as so
many female artists do. Instead she messaged Charm La'Donna, the
choreographer behind Kendrick Lamar's dramatic and militaristic Grammy
performance in 2018. Rosalía didn't want her dances to suggest
submission. ``There's always an intention to show the figure of the
woman with strength and power,'' she said. With La'Donna, who has
choreographed most of Rosalía's videos and live performances, she
hatched a dance style that blended flamenco's sinuous gestures with
hip-hop's punch.

It's still too early to know how Rosalía's career will evolve. ``When
you're in a relationship, there's all kinds of moments,'' she said,
imagining her future as a musician. ``The beginning tends to be one way,
the middle tends to be another and the ending tends to be another.'' She
smiled. ``So I know I'm going to pass through all that. I know that I
will age making music, and I want to see how my music changes with the
years.'' What she wants most is ``never to lose the desire to keep
making music.''

Last February, standing onstage in a floor-length red dress at the Goya
Awards, Spain's version of the Oscars, Rosalía sang, in Spanish, ``Me
Quedo Contigo'' (``I Stay With You''), by Los Chunguitos:

\begin{quote}
If you make me choose\\
Between you and glory\\
So that history speaks of me\\
For centuries, oh, love ...\\
I stay with you.
\end{quote}

In her arrangement, Rosalía replaced the electric guitar and the drums
with backup singing by the youth chorus Cor Jove de l'Orfeó Català. The
change brought forward the song's subtext, making it a declaration not
just of romantic love but also of spiritual devotion. Her voice danced
between strength and vulnerability, the knowing sneers and smiles of her
pop performances were wiped from her face. In the final bars, her eyes
glistened; her lower lip trembled. It was beauty. It was innovation. It
was \emph{duende}. Could it also have been a song about a life devoted
to music? ``You still haven't seen everything that Rosalía can do
singing pure flamenco,'' Chiqui told me. ``Everything that she is,
everything that she has inside, it still hasn't been seen.''

\textbf{Marcela Valdes} is a contributing writer for the magazine. She
last wrote about
\href{https://www.nytimes3xbfgragh.onion/2018/12/13/magazine/alfonso-cuaron-roma-mexico-netflix.html}{the
Mexican film director Alfonso Cuarón}. \textbf{Christopher Anderson} is
the author of five photographic books. His latest, ``COP,'' was released
in June. He lives in Paris.

Stylist: Samantha Burkhart. Hair and makeup: Javier Ceferino. Pink
jacket by Karizma. Red coat by Kirin from Neiman Marcus Beverly Hills

The Culture Issue

\begin{itemize}
\tightlist
\item
  \href{https://www.nytimes3xbfgragh.onion/interactive/2019/10/09/magazine/tyler-perry-black-theater.html}{Black
  Theater Is Having a Moment. Thank Tyler Perry. (Seriously.)}
\item
  \href{https://www.nytimes3xbfgragh.onion/interactive/2019/10/09/magazine/PewDiePie-interview.html}{What
  Does PewDiePie Really Believe?}
\item
  \href{https://www.nytimes3xbfgragh.onion/interactive/2019/10/09/magazine/kathryn-hahn-mrs-fletcher.html}{Kathryn
  Hahn's Funny, Sensual Portrayals of Female Desire}
\item
  \href{https://www.nytimes3xbfgragh.onion/interactive/2019/10/08/magazine/black-women-artists-conversation.html}{`I
  Want to Explore the Wonder of What It Is to Be a Black American'}
\item
  \href{https://www.nytimes3xbfgragh.onion/interactive/2019/10/08/magazine/rosalia-flamenco.html}{Rosalía's
  Incredible Journey From Flamenco to Megastardom}
\item
  \href{https://www.nytimes3xbfgragh.onion/interactive/2019/10/08/magazine/ben-lerner-topeka-school.html}{To
  Decode White Male Rage, First He Had to Write in His Mother's Voice}
\item
  \href{https://www.nytimes3xbfgragh.onion/interactive/2019/10/08/magazine/susan-sontag.html}{How
  Susan Sontag Taught Me to Think}
\item
  \href{https://www.nytimes3xbfgragh.onion/interactive/2019/10/09/magazine/moma-reopening.html}{Backstage
  at the Modern}
\end{itemize}

\protect\hyperlink{}{} \protect\hyperlink{}{}

\includegraphics{https://static01.graylady3jvrrxbe.onion/newsgraphics/2019/10/13/culture/1a9865dae9d1a0d21273eb2bff896ea8752f0273/caret.svg}

\hypertarget{more-culture}{%
\subsection{More Culture}\label{more-culture}}

\begin{itemize}
\tightlist
\item
  \href{https://www.nytimes3xbfgragh.onion/interactive/2019/10/09/magazine/tyler-perry-black-theater.html}{}
\item
  \href{https://www.nytimes3xbfgragh.onion/interactive/2019/10/09/magazine/PewDiePie-interview.html}{}
\item
  \href{https://www.nytimes3xbfgragh.onion/interactive/2019/10/09/magazine/kathryn-hahn-mrs-fletcher.html}{}
\item
  \href{https://www.nytimes3xbfgragh.onion/interactive/2019/10/08/magazine/black-women-artists-conversation.html}{}
\item
  \href{https://www.nytimes3xbfgragh.onion/interactive/2019/10/08/magazine/ben-lerner-topeka-school.html}{}
\item
  \href{https://www.nytimes3xbfgragh.onion/interactive/2019/10/08/magazine/susan-sontag.html}{}
\item
  \href{https://www.nytimes3xbfgragh.onion/interactive/2019/10/09/magazine/moma-reopening.html}{}
\end{itemize}

Stylist: Samantha Burkhart. Hair and makeup: Javier Ceferino. Pink
jacket by Karizma. Red coat by Kirin from Neiman Marcus Beverly Hills

\hypertarget{more-on-nytimescom}{%
\subsection{More on NYTimes.com}\label{more-on-nytimescom}}

Advertisement

\hypertarget{site-information-navigation}{%
\subsection{Site Information
Navigation}\label{site-information-navigation}}

\begin{itemize}
\tightlist
\item
  \href{https://help.nytimes3xbfgragh.onion/hc/en-us/articles/115014792127-Copyright-notice}{©
  2020 The New York Times Company}
\item
  \href{https://www.nytimes3xbfgragh.onion}{Home}
\item
  \href{https://www.nytimes3xbfgragh.onion/search/}{Search}
\item
  Accessibility concerns? Email us at
  \href{mailto:accessibility@NYTimes.com}{\nolinkurl{accessibility@NYTimes.com}}.
  We would love to hear from you.
\item
  \href{https://help.nytimes3xbfgragh.onion/hc/en-us/articles/115015385887-Contact-Us}{Contact
  Us}
\item
  \href{https://www.nytco.com/careers/}{Work with us}
\item
  \href{https://nytmediakit.com/}{Advertise}
\item
  \href{https://help.nytimes3xbfgragh.onion/hc/en-us/articles/115014892108-Privacy-policy\#pp}{Your
  Ad Choices}
\item
  \href{https://help.nytimes3xbfgragh.onion/hc/en-us/articles/115014892108-Privacy-policy}{Privacy}
\item
  \href{https://help.nytimes3xbfgragh.onion/hc/en-us/articles/115014893428-Terms-of-service}{Terms
  of Service}
\item
  \href{https://help.nytimes3xbfgragh.onion/hc/en-us/articles/115014893968-Terms-of-sale}{Terms
  of Sale}
\end{itemize}

\hypertarget{site-information-navigation-1}{%
\subsection{Site Information
Navigation}\label{site-information-navigation-1}}

\begin{itemize}
\tightlist
\item
  \href{https://spiderbites.nytimes3xbfgragh.onion}{Site Map}
\item
  \href{https://help.nytimes3xbfgragh.onion/hc/en-us}{Help}
\item
  \href{https://help.nytimes3xbfgragh.onion/hc/en-us/articles/115015385887-Contact-Us?redir=myacc}{Site
  Feedback}
\item
  \href{https://www.nytimes3xbfgragh.onion/subscription?campaignId=37WXW}{Subscriptions}
\end{itemize}
