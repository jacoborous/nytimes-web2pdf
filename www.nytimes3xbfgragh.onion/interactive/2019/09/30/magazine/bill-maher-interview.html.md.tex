 **NYTimes.com no longer supports Internet Explorer 9 or earlier. Please
upgrade your browser.
\href{http://www.nytimes3xbfgragh.onion/content/help/site/ie9-support.html}{LEARN
MORE »}

**Sections

**Home

**Search

\hypertarget{the-new-york-times}{%
\subsection{\texorpdfstring{\href{http://www.nytimes3xbfgragh.onion/}{The
New York Times}}{The New York Times}}\label{the-new-york-times}}

\hypertarget{-magazine-}{%
\subsubsection{\texorpdfstring{
\href{https://www.nytimes3xbfgragh.onion/section/magazine}{Magazine}
}{ Magazine }}\label{-magazine-}}

 \href{https://www.nytimes3xbfgragh.onion/section/magazine}{Magazine}
\textbar{}Bill Maher on the Perils of Political Correctness

**Close search

\hypertarget{site-search-navigation}{%
\subsection{Site Search Navigation}\label{site-search-navigation}}

Search NYTimes.com

**Clear this text input

Go

\url{https://nyti.ms/2mhZYsA}

\hypertarget{site-navigation}{%
\subsection{Site Navigation}\label{site-navigation}}

\hypertarget{site-mobile-navigation}{%
\subsection{Site Mobile Navigation}\label{site-mobile-navigation}}

\hypertarget{bill-maher-on-the-perils-of-political-correctness}{%
\section{Bill Maher on the Perils of Political
Correctness}\label{bill-maher-on-the-perils-of-political-correctness}}

``Religions always talk about the one true religion. Now on the left we
have the one true opinion.''

\includegraphics{https://static01.graylady3jvrrxbe.onion/newsgraphics/2019/06/07/talk-template/ab12322e42290f73d80027dd11c8f213be358338/close.svg}

\textbf{Talk} Sept. 30, 2019

Bill Maher on the perils of political correctness.

By David Marchese Photograph by Mamadi Doumbouya

Deeply caustic and supremely confident, Bill Maher is the kind of
satirist who causes even his many admirers --- his HBO talk show ``Real
Time'' draws more than four million viewers per episode --- to throw up
their hands now and again. Avoiding the public comfort of a party line,
Maher lights into the political excesses and orthodoxies of the left as
well as the right, on an anti-P.C., antihypocrisy crusade that skewers
Democrats and Republicans alike. ``My whole career,'' Maher says, ``has
been this battle: Why can't I talk on TV the way I talk at home or with
my friends? My goal was to take that gap, which on most shows you can
drive a truck through, and close it to nothing.''

\textbf{Most late-night hosts don't criticize both the right and the
left as much as you do. Why do you think that is?} It's hard to answer
that question without sounding self-serving. I will say this: Our studio
audience is not representative of liberals across the country. Your
paper and The Atlantic had
\href{http://nytimes3xbfgragh.onion\#tooltip-1}{long articles1} in the
last year saying that 80 percent of Americans think this politically
correct BS has gone too far. But the people on Twitter are the people
who control the media a lot. They're the millennials who probably grew
up with helicopter parents who afforded them a sense of entitlement.
They are certainly more fragile than previous generations. Trigger
warnings. Safe spaces. Crying rooms. Microaggressions. That crowd feels
like anything that upsets their tender sensibilities is completely out
of line.

\textbf{Isn't it important to distinguish between the fundamental
arguments being made in favor of those sensibilities and the people
being loudest on social media about them?} Yes. The most important thing
that the Democrats can do to win the next election is to broom this
element out of their party and stand up to the Twitter mob and the
ultrawoke. And I don't like the term ``woke,'' because it implies I am
asleep. I was woke before some of these people were born. I grew up in a
household with two liberal parents who were
\href{http://nytimes3xbfgragh.onion\#tooltip-2}{ahead of their
time}\href{http://nytimes3xbfgragh.onion\#tooltip-2}{.2} My father and
mother told me about civil rights. I knew what the right thing was. The
difference is that liberals protect people, and P.C. people protect
feelings. They don't \emph{do} anything. They're pointing at other
people who are somehow falling short of their standards, which could
have changed three weeks ago. They're constantly moving the goalposts so
they can go, ``Gotcha!'' For example, when I was growing up, the most
liberal thing you could do is not see color. Well, that's wrong now. You
see color, always, so you can register your white privilege. But I grew
up in the Martin Luther King era: Judge by the content of their
character, not the color of their skin. I still think that's the best
way to do it. Not see it.

\textbf{But we do see color, and no one is arguing that people shouldn't
be judged by their character. So what problem is being caused by the
shift you just described?} If someone walks in the room, after a minute,
I should not be thinking about color. And I am not. That's how I have
always been. I have actual black friends. I don't think they want me to
be always thinking: Black person. Black person. I'm talking to a black
person. Look, I tried to drive a stake through political correctness
\href{http://nytimes3xbfgragh.onion\#tooltip-3}{in the
'}\href{http://nytimes3xbfgragh.onion\#tooltip-3}{90s.3} I obviously
failed dismally. It's worse than ever.

\textbf{You've talked about the negative effects of the ``Twitter mob''
on your show, but you've also talked about how most people don't care
what's on Twitter. If people don't care about the Twitter conversation,
why bother railing against it?} Because the Twitter-mob mentality has an
effect on the rest of the world. Everyone fears the wrath of the Twitter
mob and the social justice warriors and the P.C. police. Religions
always talk about the one true religion. Now on the left we have the one
true \emph{opinion}. If you go against that, you do so at your peril.
That's why the air on the left is becoming stale. I railed for years
against the Fox News bubble, and that is as strong as ever, but I didn't
think it would get this bad on the left. Comedians are afraid to make
jokes in clubs, because somebody will tape it and send it out on Twitter
and get the mob after you.

\textbf{That's a concern we often hear from comedians these days. How
much of that fear is coming from comedians still adjusting to the
reality of there being possible consequences for their material? You can
still make whatever joke you want. The difference is that more people
are calling you out if they find it offensive.} That's naïve. You can
make the joke if you don't mind giving up your career or being fired.
Come on. The politically correct people are not concerned about social
justice. They care about putting scalps on the wall. Liam Neeson.
\href{http://nytimes3xbfgragh.onion\#tooltip-4}{Remember that?4} Are we
at this place where we can't admit that we've ever had bad thoughts and
gotten over them and become a better person? You can't judge today by
yesterday. We evolve.

\textbf{Let's take the Liam Neeson thing.} Who I don't even like, by the
way.

\textbf{What's your problem with Liam Neeson?} He's for horse-drawn
carriages in Central Park. And I'm a PETA board member.

\textbf{I didn't know that. But the controversy around him was a story
for a day, and then the world moved on. His career is fine, isn't it?}
The world doesn't move on for
\href{http://nytimes3xbfgragh.onion\#tooltip-5}{Megyn Kelly5} and
\href{http://nytimes3xbfgragh.onion\#tooltip-6}{Roseanne,6} and
\href{http://nytimes3xbfgragh.onion\#tooltip-7}{Aziz Ansari7} had to fly
below the radar for a year. I think you're downplaying how serious this
stuff is. We live in an age where people want to cancel other people and
disappear them. Who's going to be left?

\textbf{You've had two big controversies during your career. The first
was in 2001 when you said that the 9/11 hijackers
\href{http://nytimes3xbfgragh.onion\#tooltip-8}{were not
cowards.}\href{http://nytimes3xbfgragh.onion\#tooltip-8}{8} The second
was two years ago, when you made that joke
\href{http://nytimes3xbfgragh.onion\#tooltip-9}{using the N-word.9} Did
it feel different to be at the center of a controversy during the social
media era?} Controversies are never pleasant to go through. On the
second controversy, I'm saving an in-depth discussion for my memoirs. If
we were living in a country that could handle nuance, I'd be happy to
talk about it, but we're no longer in that country. There's no winning
there. You're going to have to read my memoirs. We live in an era where
I don't think people's main focus is the truth and/or sussing out
something valuable or teachable. We live in a time in which people are
more concerned with scalps and clicks.

\textbf{Did the discussion that happened after you made that joke reveal
anything new to you about our culture's or your own understanding of
that language?} I just think there's no way to have that conversation
with you, David. I'm sorry, I don't blame you for trying. It's a shame,
because there is lots of learning that can be happening. As I said at
the time, anytime someone is hurt by a word like that you have my
sincere apology. But that's the beginning of a discussion, and it's too
bad that we don't live in a place where you can have the end of it.

\textbf{Well, so my next question is related to the 9/11 controversy.
You've always been critical of all religions, but is there something
distinct about your criticism of Islam? Fairly or not, you've been
called an Islamophobe a few times over the years.} It's ridiculous to
label criticism of a religion as a phobia of a religion. I'm going to
criticize any person or group that violates liberal principles, and so
should you. Almost all religions, by their nature, are intolerant and
supremacist. At any time in history one religion will be the most
fundamentalist. At this moment I think it's pretty evident that religion
is Islam. Of course, intolerance exists everywhere, but the places
where, let's say, human rights workers have their work cut out for them
the most are probably traditional Islamic societies. To conflate
thinking that with Islamophobia is a facile and unconvincing trick.

\textbf{I do wonder if, at least in the past, you've done some
conflating of your own as far as, for example, treating theocracies or
dictators as exemplars of Islamic rank and file.} I think you have it
backwards. The government of Pakistan is more liberal than the people.
Their senate recently passed legislation to end child marriages and
local police forces have intervened. Yes, we have things in \emph{our}
country that are at odds with liberal values, but someone once said
that, at some point, a difference in degree becomes a difference in
kind. It's frustrating for me. I know that people who ask me these
questions actually agree with me, and yet they're like, ``Are you
\emph{crazy}?'' It's like, Can I just be real?

\textbf{It could be that there are complexities that your criticisms of
Islam don't address.} There are many factors, none of which I've ever
denied. Poverty has been shown to have little to do with terrorism. You
can always bring in a million things to make this look like a phobia.
``But what about white supremacy?'' Also a bad thing! Never said it
wasn't. It's interesting to me that even the people who criticize me
about this sometimes have used the word cancer. As in, ``Islamism is a
cancer upon Islam.'' And to those who say, when I mention instances of
Islamism, ``But it's not everywhere,'' I say, ``If a doctor tells you
you have cancer, do you go, Yeah, but it's not everywhere?''

\textbf{Do you see any way out of this cultural and political tailspin
we're in right now, in which everyone's default stance is ``If you don't
agree with me, then screw you?''} You have to find a way to begin with
what you share and then explore why you differ so vehemently on other
issues, and that's what we seem to have lost the ability to do. I don't
see a lot of desire for people to talk to each other, to accept that,
``O.K., this person doesn't agree with me on a lot of stuff, but I don't
have to think he's a monster.'' We want to beat our chests and vanquish
the other side. Compromise seems like a dead concept.

\textbf{On the ``Real Time'' anniversary special last year, the things
people were saying about why they like you --- especially your
fearlessness about saying what you really think --- reminded me of the
things people say about why they like President Trump, whom
\href{http://nytimes3xbfgragh.onion\#tooltip-10}{you're no fan of.10} Is
there any way to productively channel people's enthusiasm for those
qualities? So much of it seems like it's mostly about the pleasure we
get from seeing our opponents insulted.} During the second year of
``Politically Incorrect'' we had a contest: ``Politically incorrect or
just stupid?'' We were trying to make the point that saying something
that's contrary is not necessarily politically incorrect. It's sometimes
just stupid. I define political correctness as the elevation of
sensitivity over truth. That's my beef with it. We're not getting to the
truth, because we're too sensitive.

\textbf{Let me totally switch subjects. I went and read
\href{http://nytimes3xbfgragh.onion\#tooltip-11}{your novel.11}} I'm
\emph{verklempt}. That's something no interviewer has ever said to me.

\textbf{It has this lovingly detailed evocation of a very particular
time in the comedy world, back when the boom was starting to happen in
the late '70s, and how that was a real moment of change for comedians
and their work. Have you seen any similar sea changes since?} I'm
probably not the best one to ask, because it has been a long time since
I was in the comedy clubs. I do hear a lot of complaints that comedians
are frustrated that they can't freely try out new bits. When I was
coming up, the great thing about the comedy clubs was that they were
laboratories for our experimentation. That was the deal. They didn't pay
us, and we didn't have to be good --- and weren't --- but that's how we
honed our craft. Now people are afraid, and comedy does not function
well in that atmosphere of fear. We want to be saying whatever,
especially if it's funny, and it hurts us that the audience won't trust
us. Do you really think I'm on the side of the bad people? Chris Rock,
Larry the Cable Guy and Jerry Seinfeld a few years ago all were talking
about the fact that they don't work campuses anymore. \emph{Jerry
Seinfeld} is too out there? His act is so clean it whitens teeth. Comedy
is about saying those true things that everyone else isn't saying.
That's where the fun is.

\textbf{You mentioned colleges. Students are another group that you talk
a lot about on the show. There has been no time over the last 50 or so
years when people haven't been criticizing college kids' social and
political ideas. But isn't that a reaction to the fact that college is a
place where students are pushing hard and figuring out their ideas about
the world? Isn't that what these kids are supposed to be doing at that
age and in that setting?} I don't think someone who's at Harvard is a
child, and I do think they should know that everybody in America gets a
lawyer. Yet they \href{http://nytimes3xbfgragh.onion\#tooltip-12}{did
not understand that.12}

\textbf{The students at Harvard weren't saying Harvey Weinstein wasn't
entitled to a lawyer. They were objecting to a residential dean}
\emph{\textbf{being}} \textbf{his lawyer. That's different.} Well,
that's wrong, too. Everybody gets the lawyer that they want. Harvard
doesn't understand the very basis of the Sixth Amendment? I don't think
a lot of us who are criticizing that are criticizing the kids as much as
the administrators.

\textbf{Who you think are spineless.} Very spineless. The way parents
have been spineless in disciplining their kids. When I was growing up
you could never drive a wedge between your parents and the teacher. Now
the parents always back their precious darlings, and that's why you have
grade inflation and kids who leave school without knowing anything.
``It's not the kid's fault that he doesn't know anything. It's the
teacher's fault.'' That's not helping our country. Being brought up this
way is going to lead these kids to ruin. Of course, they're not
\emph{all} brought up the same way. I don't think in the middle of the
country they're raising their kids like that. I saw Mario Lopez got in
trouble, did you see that?

\textbf{I didn't.} I saw this headline: ``Mario Lopez Apologizes.'' It
was this groveling apology to the L.G.B.T.Q. community. You know what
the problem was? They asked him about this trend in Hollywood of letting
your 3-year-old decide their gender and Mario Lopez said maybe 3 is a
little young \href{http://nytimes3xbfgragh.onion\#tooltip-13}{for that
decision.13} Monster!

\textbf{This is making me think of when
\href{http://nytimes3xbfgragh.onion\#tooltip-14}{you had Dr. Debra
Soh14} on the show talking about gender dysphoria, and you were pointing
to what you see as the problem of parental permissiveness towards gender
identification and transitioning. You were saying that parents let their
kids gender reidentify because it's easier than telling them not to.
That seemed pretty glib.} It was.

\textbf{It's a bit hard to imagine that parents who support a child's
transitioning are doing it because they think it's the easier path.}
That's not true. I know people who've done it, and that is exactly what
it is. They never discipline their kids. They think they're making it
easier by giving the kid what they want. I mean, you're right, what I
said was glib, but I am serving many masters. ``Real Time'' is an
entertainment show on an entertainment network, and I'm a comedian. Not
everything I say can stand up to the scrutiny of the ultimate
fact-check. But I think that there is some truth to this. There are kids
--- and this is what Dr. Soh was saying and I wasn't disagreeing with
--- who have transitioned who were really just gay. I don't think it's
the worst thing in the world to wait a few years to find out what's
going on. I'm not a doctor. I'm not a scientist. But if I had a kid I
would tell them: ``As long as you're living under my roof you're not
cutting anything off. Until you're 18. Then you cut off whatever you
want.'' Here I am, being glib again.

\textbf{What's something encouraging to you about millennials? And
what's the most disappointing thing about your own generation? Aside
from ruining the world environmentally.} We've left a dark, stinking
husk of a planet, haven't we? My generation started this mess. The Baby
Boomers were the first ``Me'' generation. They were the first spoiled
kids. There definitely was more discipline, but there was also more
indulgence, and that seemed to continue on and on. So I think we have to
look in the mirror as to when that trend started. As for the most
encouraging thing about millennials, it's idealism. You need people to
look at anything with a fresh pair of eyes. That sort of idealism is
essential to temper the necessary cynicism.

\textbf{And you don't see any idealism in the identity politics of
younger people?} I don't know how that's connected to idealism. What I'm
complaining about is fragility. What I'm complaining about is people who
were overindulged as children and somehow believe that they should not
have to endure even the slightest measure of discomfort.

\textbf{I'm sure I'm overly Pollyanna-ish about all this, and obviously
not everyone is arguing these issues in good faith, but isn't the root
of what you're identifying just people's attempt to figure out how to
get through life with more dignity and less pain?} But there are
negative repercussions. People get disappeared. When I was a young
person the conservatives were the ones who --- I don't know what you'd
call it.

\textbf{Drew hard lines about what was or wasn't culturally acceptable?}
Thank you, yes. Now it's reversed, and I feel like that's backwards.
Young people should be the free ones pushing the boundaries and not the
ones inhibiting us. ``Well, I'm not a woman, so I could not possibly
know that experience.'' ``I'm not a person of color, so I can't speak
about that.'' Professors are afraid to speak, because what they say,
even if it's science, might go against the politically correct notion.
This is pernicious. I'm sorry, but I have to lay that at the doorstep of
the far left and the younger generation. It's not the worst thing in the
world to hear something you find somewhat offensive. You can turn the
channel. Look at something else. Go to a puppet show; you'll never be
offended.

\textbf{I'm curious about how your own comedy has evolved. Back when you
were doing ``Politically Incorrect'' you used to do a lot more}
\emph{\textbf{hubba hubba}} \textbf{jokes about women.} It's funny you
mention that. When I turned 50, I had a talk with my writers and I said,
``no more I'm-in-the-hot-tub-with-twins jokes.'' Back in the '90s it was
a different point of view to say, ``I'm single, and that is not a bad
choice.'' I stood up for that idea --- and it was not well accepted at
the time --- that you can have children, that's fine, but I do not want
them. I was a bit of a militant single person. But when I was 50, I
said, ``I'm too old to be doing these jokes.'' At a certain point it's
not funny anymore. It's creepy. I never did those kind of jokes again.

\textbf{Do you still have a stripper pole in your house?} It's not in my
house.

\textbf{Guest house?} Well, yes. I bought my house in 2001, and in 2004
my next-door neighbor was selling his bachelor pad. He had a small house
he lived in, and there was this other little bungalow on the property
that I use if I have a party. I don't know how you knew that I had a
stripper pole put in.

\textbf{Let me ask you a nonpolitics, noncomedy question. I know that
you're a big Beatles fan. In one of your books you said you could
probably do a better job interviewing them than anybody has yet.} I
definitely could.

\textbf{So if you could snap your fingers and have Paul McCartney and
Ringo Starr on your show, what would you ask them?} I would love to
present my theory as to why the Beatles really broke up. Which is that
John Lennon could not keep up in the battle for A-sides. Imagine writing
a song as great as ``Revolution'' and it loses out to ``Hey Jude.''
That's, I think, why John Lennon didn't want to continue going with the
Beatles. I don't think he liked losing. Paul McCartney would never admit
that, by the way.

\textbf{Well, there you go. O.K., back to your work! For more than 25
years you've been going on TV and making jokes about Republicans being
hypocritical and corrupt and Democrats being too PC and lacking
backbone. Does it ever feel like you're banging your head against a
wall? These people don't change.} Yes but I never thought that people
would hear my jokes and go: ``He's right! I've got to amend my behavior
right now.'' But I'm very fortunate as a standup comedian who still goes
on the road a lot, because I'm always given new material. I had John
Boehner jokes, and now I have Mitch McConnell jokes.

\textbf{I wonder if you could get away with Mad-Libbing your material.
Just swap new names into old jokes.} I have repurposed junk. I think I
had one about Newt Gingrich having the moral compass of an opportunistic
infection. Who doesn't that apply to? I have a plethora of material, but
if an old joke perfectly fits somewhere I'm not above repurposing. You
know, John Lennon wrote a song called ``Child of Nature,'' and it was a
great tune. He repurposed it with different lyrics a few years later as
``Jealous Guy.'' Artists are blue jays. We find little scraps here and
there and build a nest. We're shameless about it.

David Marchese is the magazine's Talk columnist.

\emph{This interview has been edited and condensed from two
conversations.}

\hypertarget{more-on-nytimescom}{%
\subsection{More on NYTimes.com}\label{more-on-nytimescom}}

Advertisement

\hypertarget{site-information-navigation}{%
\subsection{Site Information
Navigation}\label{site-information-navigation}}

\begin{itemize}
\tightlist
\item
  \href{https://help.nytimes3xbfgragh.onion/hc/en-us/articles/115014792127-Copyright-notice}{©
  2020 The New York Times Company}
\item
  \href{https://www.nytimes3xbfgragh.onion}{Home}
\item
  \href{https://www.nytimes3xbfgragh.onion/search/}{Search}
\item
  Accessibility concerns? Email us at
  \href{mailto:accessibility@NYTimes.com}{\nolinkurl{accessibility@NYTimes.com}}.
  We would love to hear from you.
\item
  \href{https://help.nytimes3xbfgragh.onion/hc/en-us/articles/115015385887-Contact-Us}{Contact
  Us}
\item
  \href{https://www.nytco.com/careers/}{Work with us}
\item
  \href{https://nytmediakit.com/}{Advertise}
\item
  \href{https://help.nytimes3xbfgragh.onion/hc/en-us/articles/115014892108-Privacy-policy\#pp}{Your
  Ad Choices}
\item
  \href{https://help.nytimes3xbfgragh.onion/hc/en-us/articles/115014892108-Privacy-policy}{Privacy}
\item
  \href{https://help.nytimes3xbfgragh.onion/hc/en-us/articles/115014893428-Terms-of-service}{Terms
  of Service}
\item
  \href{https://help.nytimes3xbfgragh.onion/hc/en-us/articles/115014893968-Terms-of-sale}{Terms
  of Sale}
\end{itemize}

\hypertarget{site-information-navigation-1}{%
\subsection{Site Information
Navigation}\label{site-information-navigation-1}}

\begin{itemize}
\tightlist
\item
  \href{https://spiderbites.nytimes3xbfgragh.onion}{Site Map}
\item
  \href{https://help.nytimes3xbfgragh.onion/hc/en-us}{Help}
\item
  \href{https://help.nytimes3xbfgragh.onion/hc/en-us/articles/115015385887-Contact-Us?redir=myacc}{Site
  Feedback}
\item
  \href{https://www.nytimes3xbfgragh.onion/subscription?campaignId=37WXW}{Subscriptions}
\end{itemize}
