 **NYTimes.com no longer supports Internet Explorer 9 or earlier. Please
upgrade your browser.
\href{http://www.nytimes3xbfgragh.onion/content/help/site/ie9-support.html}{LEARN
MORE »}

**Sections

**Home

**Search

\hypertarget{the-new-york-times}{%
\subsection{\texorpdfstring{\href{http://www.nytimes3xbfgragh.onion/}{The
New York Times}}{The New York Times}}\label{the-new-york-times}}

\hypertarget{-business-}{%
\subsubsection{\texorpdfstring{ \href{/section/business}{Business}
}{ Business }}\label{-business-}}

 \href{/section/business}{Business} \textbar{}Pregnancy Discrimination
Is Rampant Inside America's Biggest Companies

**Close search

\hypertarget{site-search-navigation}{%
\subsection{Site Search Navigation}\label{site-search-navigation}}

Search NYTimes.com

**Clear this text input

Go

\url{https://nyti.ms/2MvZVSi}

\hypertarget{site-navigation}{%
\subsection{Site Navigation}\label{site-navigation}}

\hypertarget{site-mobile-navigation}{%
\subsection{Site Mobile Navigation}\label{site-mobile-navigation}}

\hypertarget{pregnancy-discrimination-is-rampant-inside-americas-biggest-companies}{%
\section{Pregnancy Discrimination Is Rampant Inside America's Biggest
Companies}\label{pregnancy-discrimination-is-rampant-inside-americas-biggest-companies}}

Many pregnant women have been systematically sidelined in the workplace.
They're passed over for promotions and raises. They're fired when they
complain.

\hypertarget{pregnancy-discrimination-is-rampant-inside-americas-biggest-companies-1}{%
\section{Pregnancy Discrimination Is Rampant Inside America's Biggest
Companies}\label{pregnancy-discrimination-is-rampant-inside-americas-biggest-companies-1}}

Many pregnant women have been systematically sidelined in the workplace.
They're passed over for promotions and raises. They're fired when they
complain.

By \href{https://www.nytimes3xbfgragh.onion/by/natalie-kitroeff}{NATALIE
KITROEFF} and
\href{https://www.nytimes3xbfgragh.onion/by/jessica-silver-greenberg}{JESSICA
SILVER-GREENBERG} FEB. 8, 2019

When she got pregnant, Otisha Woolbright asked to stop lifting heavy
trays at Walmart. Her boss said she had seen Demi Moore do a flip on TV
when she was nearly full-term --- so being pregnant was ``no excuse.''
Ms. Woolbright kept lifting until she got hurt.

When she got pregnant, Rachel Mountis was winning awards for being a top
saleswoman at Merck. She was laid off three weeks before giving birth.

When she got pregnant, Erin Murphy, a senior employee at the financial
giant Glencore, was belittled on the trading floor. After returning from
maternity leave, she was told to pump milk in a supply closet cluttered
with recycling bins.

American companies have spent years trying to become more welcoming to
women. They have rolled out generous parental leave policies, designed
cushy lactation rooms and plowed millions of dollars into programs aimed
at retaining mothers.

But these advances haven't changed a simple fact: Whether women work at
Walmart or on Wall Street, getting pregnant is often the moment they are
knocked off the professional ladder.

Throughout the American workplace, pregnancy discrimination remains
widespread. It can start as soon as a woman is showing, and it often
lasts through her early years as a mother.

The New York Times reviewed thousands of pages of court and public
records and interviewed dozens of women, their lawyers and government
officials. A clear pattern emerged. Many of the country's largest and
most prestigious companies still systematically sideline pregnant women.
They pass them over for promotions and raises. They fire them when they
complain.

In physically demanding jobs --- where an increasing number of women
unload ships, patrol streets and hoist boxes --- the discrimination can
be blatant. Pregnant women risk losing their jobs when they ask to carry
water bottles or take rest breaks.

In corporate office towers, the discrimination tends to be more subtle.
Pregnant women and mothers are often perceived as less committed,
steered away from prestigious assignments, excluded from client meetings
and slighted at bonus season.

Each child chops 4 percent off a woman's hourly wages, according to a
2014 analysis by a sociologist at the University of Massachusetts,
Amherst. Men's earnings increase by 6 percent when they become fathers,
after controlling for experience, education, marital status and hours
worked.

``Some women hit the maternal wall long before the glass ceiling,'' said
Joan C. Williams, a professor at University of California Hastings
College of Law who has testified about pregnancy discrimination at
regulatory hearings. ``There are 20 years of lab studies that show the
bias exists and that, once triggered, it's very strong.''

Of course, plenty of women decide to step back from their careers after
becoming mothers. Some want to devote themselves to parenthood. Others
lack affordable child care.

But for those who want to keep working at the same level, getting
pregnant and having a child often deals them an involuntary setback.

The number of pregnancy discrimination claims filed annually with the
Equal Employment Opportunity Commission has been steadily rising for two
decades and is hovering near an all-time high.

It's not just the private sector. In September, a federal appeals court
ruled in favor of Stephanie Hicks, who sued the Tuscaloosa, Ala., police
department for pregnancy discrimination. Ms. Hicks was lactating, and
her doctor told her that her bulletproof vest was too tight and risked
causing a breast infection. Her superior's solution was a vest so baggy
that it left portions of her torso exposed.

Tens of thousands of women have taken legal action alleging pregnancy
discrimination at companies including Walmart, Merck, AT\&T, Whole
Foods, 21st Century Fox, KPMG, Novartis and the law firm Morrison \&
Foerster. All of those companies boast on their websites about
celebrating and empowering women.

\hypertarget{womens-brains}{%
\paragraph{Women's Brains}\label{womens-brains}}

As a senior woman at Glencore, the world's largest commodity trading
company, Erin Murphy is a rarity. She earns a six-figure salary plus a
bonus coordinating the movement of the oil that Glencore buys and sells.
Most of the traders whom she works with are men.

The few women at the company have endured a steady stream of sexist
comments, according to Ms. Murphy. Her account of Glencore's culture was
verified by two employees, one of whom recently left the company. They
requested anonymity because they feared retaliation.

On the company's trading floor, men bantered about groping the Queen of
England's genitals. As Glencore was preparing to relocate from
Connecticut to New York last February, the traders --- including Ms.
Murphy's boss, Guy Freshwater --- openly discussed how much ``hot ass''
there would be at the gym near the new office.

In 2013, a year after Ms. Murphy arrived, Mr. Freshwater described her
in a performance review as ``one of the hardest working'' colleagues. In
a performance review the next year, he called her a ``strong leader''
who is ``diligent, conscientious and determined.''

But when Ms. Murphy told Mr. Freshwater she was pregnant with her first
child, he told her it would ``definitely plateau'' her career, she said
in the affidavit. In 2016, she got pregnant with her second child. One
afternoon, Mr. Freshwater announced to the trading floor that
\href{http://www.bbc.com/news/health-38341901}{the most-read article} on
the BBC's website was about pregnancy altering women's brains. Ms.
Murphy, clearly showing, was the only pregnant woman there.

``It was like they assumed my brain had totally changed overnight,'' Ms.
Murphy, 41, said in an interview. ``I was seen as having no more
potential.''

When she was eight months pregnant, she discussed potential future
career moves with Mr. Freshwater. According to her, Mr. Freshwater
responded, ``You're old and having babies so there's nowhere for you to
go.''

A Glencore spokesman declined to comment on Mr. Freshwater's behalf.

After she came back from four months of maternity leave, she organized
her life so that having children wouldn't interfere with her career. She
arranged for child care starting at 7 a.m. so she would never be late.

But as her co-workers were promoted, her bosses passed her over and her
bonuses barely rose, Ms. Murphy said.

When there was an opening to be the head of her department, Ms. Murphy
said she never got a chance to apply. The job instead went to a less
experienced man. Ms. Murphy said an executive involved in the selection
process had previously asked repeatedly whether she had adequate child
care.

Ms. Murphy said that after she missed out on another job, the same
Glencore executive told her it was because of the timing of her
maternity leave. Ms. Murphy has retained a lawyer and is planning to
file a lawsuit against Glencore.

Glencore's spokesman, Charles Watenphul, defended the company's
practices. ``Glencore Ltd. is committed to supporting women going on and
returning from maternity leave,'' he said. He said Ms. Murphy was never
passed over for promotions or treated differently because of her
pregnancies. He said that she received bonuses and pay increases every
year. Her lawyer, Mark Carey, said that Ms. Murphy was only given
cost-of-living increases and was denied opportunities to advance.

Ms. Murphy's problems are not rare. Managers often regard women who are
visibly pregnant as less committed, less dependable, less authoritative
and more irrational than other women.

A study conducted by Shelley Correll, a Stanford sociologist, presented
hundreds of real-world hiring managers with two résumés from equally
qualified women. Half of them signaled that the candidate had a child.
The managers were twice as likely to call the apparently childless woman
for an interview. Ms. Correll called it a ``motherhood penalty.''

``There is a cultural perception that if you're a good mother, you're so
dedicated to your children that you couldn't possibly be that dedicated
to your career,'' Ms. Correll said.

A \href{https://www2.census.gov/ces/wp/2017/CES-WP-17-68.pdf}{paper}
published in November by researchers at the Census Bureau examined the
pay of spouses. Two years before they had their first child, the
husbands made only slightly more than their wives. By the time their
children turned 1, the size of that pay gap had doubled to more than
\$25,000. Women taking maternity leave, dropping out of the work force
or working fewer hours could contribute to that disparity, but it does
not explain all of it, the researchers said.

Ms. Murphy still works at Glencore. In January, she filed a complaint of
pregnancy discrimination with the Equal Employment Opportunity
Commission. Last year, the agency received 3,184 pregnancy
discrimination complaints, about twice as many as in 1992, when it began
keeping electronic records. Regulators say many women never file
complaints because they can't afford an attorney, don't recognize that
what happened to them is illegal or fear retaliation.

\begin{quote}
\textbf{\href{https://www.nytimes3xbfgragh.onion/2018/06/26/podcasts/the-daily/pregnancy-discrimination.html}{\emph{The
Rampant Problem of Pregnancy Discrimination}}}

\emph{Subscribe to our podcast from your mobile device}:
\href{https://itunes.apple.com/us/podcast/the-daily/id1200361736?mt=2}{\emph{Via
Apple Podcasts}} \textbar{}
\emph{\href{https://play.radiopublic.com/88f7d8c3-7289-4dc6-b300-5ba71b43f5e5}{Via
RadioPublic} \textbar{}
\href{http://www.stitcher.com/podcast/the-new-york-times/the-daily-10}{Via
Stitcher}}
\end{quote}

\hypertarget{lost-momentum}{%
\paragraph{Lost Momentum}\label{lost-momentum}}

Merck, the giant pharmaceutical company based in Kenilworth, N.J.,
presents itself as a champion of professional women. ``We celebrate the
women whose hard work and tenacity have helped us continue to invent for
life,''
\href{https://www.merck.com/about/featured-stories/women-in-science.html}{the
company's website boasts}.

That is part of the reason Rachel Mountis wanted to work there.

Within a year of joining in 2005, she was given a coveted job selling
vaccines. She was promoted four years later. She won a Vice President's
Club Award for sales and a Peer Award for ``outstanding leadership.''
Merck paid for her to get a master's degree in business at New York
University.

Ms. Mountis knew that when she got pregnant in 2010 she would need to
take several weeks off for maternity leave. That meant she wouldn't be
able to stay in constant contact with the doctors she had cultivated as
customers --- and that her absence could cost Merck business.

A few weeks before Ms. Mountis's due date, Merck told her and a handful
of colleagues that they were being laid off in a downsizing.

``On paper, I was the same professional that I was nine months
earlier,'' she said. Being pregnant ``was the only thing that was
different.''

Ms. Mountis eventually got another job at Merck, but it was a demotion
with lower bonus potential.

Merck was already facing a lawsuit accusing the company of paying women
less than men and denying them professional opportunities. That suit, in
New Jersey federal court, was brought by Kelli Smith, a Merck saleswoman
who said her career was derailed when she got pregnant. ``You're not
going anywhere'' at the company, a male colleague told Ms. Smith,
according to the suit.

The women involved in the litigation say they were harassed by male
superiors.

At a conference, a Merck executive referred to a female employee as the
``hottest one in here'' and asked what he could do to get her upstairs
to his hotel room, according to court documents.

At another company event, the same executive referred to a group of
women from a company that Merck had just acquired as ``whores'' and said
``they are much hotter than the Merck whores.''

In 2014, Ms. Mountis joined the lawsuit, which now covers roughly 3,900
women.

A trial date has not been set. A Merck spokeswoman said the company
``has a strong anti-discrimination policy.'' Ms. Mountis, the
spokeswoman said, ``was supported throughout her career to ensure she
had opportunities to advance and succeed.''

Ms. Mountis tried to make the best of her less prestigious job. Merck
demoted her again in 2012, while she was on maternity leave after giving
birth to her second child. The next year, Ms. Mountis resigned. She
eventually took a job at a pharmaceutical company that is a fraction of
Merck's size.

``I am still trying to get my momentum back,'' Ms. Mountis said.

Ms. Smith also moved to a much smaller drug company.

Other drug companies have faced similar complaints. Novartis in 2010
agreed to pay \$175 million to settle a class-action lawsuit in which
thousands of current and former sales representatives said the company
discriminated against women, including expecting mothers, in pay and
promotions.

One former Novartis saleswoman, Christine Macarelli, said that her boss
told her that ``women who find themselves in my position --- single,
unmarried --- should consider an abortion.'' When she returned from
maternity leave, she said she was told to stop trying to get a promotion
``because of my unfortunate circumstances at home --- being my son
Anthony.''

\hypertarget{did-your-employer-discriminate-against-you-because-you-were-pregnant}{%
\subsection{Did Your Employer Discriminate Against You Because You Were
Pregnant?}\label{did-your-employer-discriminate-against-you-because-you-were-pregnant}}

We would like to hear from you.

\includegraphics{https://static01.graylady3jvrrxbe.onion/images/2018/05/25/business/00PREGNANCY-4/merlin_138147915_e312aa58-bb33-4b63-9bd2-3677db8888a7-master1050.jpg}

In September, a federal appeals court ruled in favor of Stephanie Hicks,
who sued the Tuscaloosa, Ala., police department for pregnancy
discrimination. Ms. Hicks was lactating, and her doctor told her that
her bulletproof vest was too tight and risked causing a breast
infection. Her superior's solution was a vest so baggy that it left
portions of her torso exposed. Credit Melissa Golden for The New York
Times

We would like to hear from you. Were you discriminated at work because
of your pregnancy? Did that negatively impact the health of you and your
child, resulting in severe health consequences for you and your baby?

For more, read our investigation:
\emph{\href{https://www.nytimes3xbfgragh.onion/interactive/2018/06/15/business/pregnancy-discrimination.html}{Pregnancy
Discrimination Is Rampant Inside America's Biggest Companies}}

Your name and comments will not be published without your permission,
and your contact information will not be published. A reporter or editor
may follow up with you to hear more about your story.

What is your name? *

First and last preferred, please.

What is your email? *

What did you experience? *

You have \textbf{500} words left.

By clicking the submit button, you agree that you have read, understand
and accept the
\href{https://help.nytimes3xbfgragh.onion/hc/en-us/articles/360004901454-Reader-submission-terms}{Reader
Submission Terms} in relation to all of the content and other
information you send to us ('Your Content'). If you do not accept these
terms, do not submit any content. Of note:

\begin{itemize}
\tightlist
\item
  Your Content must not be false, defamatory, misleading or hateful or
  infringe any copyright or any other third party rights or otherwise be
  unlawful.
\item
  We will use the contact details that you provide to verify your
  identity and answers to the questionnaire, as well as to contact you
  for further information on this story. If we publish Your Content, we
  may include your name and location.
\end{itemize}

I have read, understood and accept the
\href{https://help.nytimes3xbfgragh.onion/hc/en-us/articles/360004901454-Reader-submission-terms}{Reader
Submission Terms}

Thank you for your submission.

\hypertarget{a-feminist-revolt}{%
\paragraph{A Feminist Revolt}\label{a-feminist-revolt}}

The nation's first law against pregnancy discrimination traces back to a
1970s case about how General Electric treated expectant mothers.

The company at the time gave paid time off to workers with disabilities,
but not to pregnant women. The Supreme Court ruled in 1976 that the
company's policy wasn't discriminatory.

Feminist leaders and unions campaigned to change the law to protect
pregnant women. In 1978, Congress passed the
\href{https://www.gpo.gov/fdsys/pkg/STATUTE-92/pdf/STATUTE-92-Pg2076.pdf}{Pregnancy
Discrimination Act}, which made it illegal to treat pregnant women
differently from other people ``similar in their ability or inability to
work.''

That didn't resolve the issue. Employers argued in court that pregnant
women were most ``similar'' to workers injured off the job and,
therefore, didn't deserve accommodations.

Then, Peggy Young sued U.P.S. for discrimination. She had been an
early-morning driver when she got pregnant in 2006. Her doctor
instructed her not to lift heavy boxes. U.P.S. told her it couldn't give
her a light-duty job. She ended up on unpaid leave without health
insurance.

At the time, U.P.S. gave reprieves from heavy lifting to drivers injured
on the job and those who were permanently disabled. Even employees who
had lost their licenses after driving drunk got different assignments.
Ms. Young argued that she should have gotten the same deal.

Two federal courts ruled in U.P.S.'s favor. Ms. Young appealed to the
Supreme Court. During oral arguments in 2014, Justice Ruth Bader
Ginsburg challenged U.P.S.'s lawyer to cite ``a single instance of
anyone who needed a lifting dispensation who didn't get it except for
pregnant people.'' The U.P.S. lawyer drew a blank.

In 2015, the court ruled 6 to 3 in Ms. Young's favor. But the justices
stopped short of establishing an outright protection for expectant
mothers. They just said that if employers are accommodating big groups
of other workers --- people with disabilities, for example --- but not
pregnant women, they are probably violating the Pregnancy Discrimination
Act.

\hypertarget{demi-moores-stunt}{%
\paragraph{Demi Moore's Stunt}\label{demi-moores-stunt}}

Otisha Woolbright heaved 50-pound trays of chickens into industrial
ovens every day at her job in the deli and bakery of a Walmart in
Jacksonville, Fla.

In 2013, when she was three months pregnant, she started bleeding and
went to the emergency room. She was told that she was at risk of
miscarrying. She returned to Walmart with a physician's note saying that
she should avoid heavy lifting. She asked for light duty.

That's when her boss, Teresa Blalock, said she had seen a pregnant Demi
Moore \href{https://www.youtube.com/watch?v=OQX50pPWDyA}{do acrobatics}
on TV.

In an email to The Times, Ms. Moore said that a stunt double actually
performed the routine.

``You would have to be extremely ignorant and inexperienced with
pregnancy or just completely uncaring and insensitive to use a moment of
comedic entertainment, like my appearance on David Letterman while I was
eight and a half months pregnant, to pressure a pregnant woman into
doing something that put her or her baby at risk,'' she said.

According to Ms. Woolbright, Ms. Blalock said that if she couldn't lift
chickens, she could ``walk out those doors.''

Ms. Woolbright couldn't afford to lose her paycheck, so she kept lifting
chickens.

``What choice did I have? There was no other job that was going to hire
me being pregnant,'' she said.

Later that month, Ms. Woolbright said, she was lifting a tray of
chickens when she felt a sharp pain. Scared she was having a
miscarriage, she went back to the hospital. Walmart then put her on
light duty.

``We disagree that a specific request for accommodations due to
pregnancy was made and that we denied that request,'' a Walmart
spokesman, Ragan Dickens, said. He said that ``Ms. Blalock, a mother and
a grandmother, was supportive of Ms. Woolbright.''

Ms. Woolbright asked about maternity leave. Three days later, she said
she was called into a cramped office. She stood there sweating, seven
months pregnant. ``Walmart will no longer be needing your services,'' a
supervisor said.

Ms. Woolbright sued Walmart, the nation's largest employer. Her suit,
which is seeking class-action status, is pending.

It took Ms. Woolbright a year to land another job. Her children outgrew
their clothes. She thought about swallowing enough antidepressants to
kill herself. After stints at a restaurant and a van rental company, she
stopped working, because she couldn't get shifts that allowed her to
take care of her children.

Walmart is the least expensive store in town, and Ms. Woolbright goes
there to buy baby formula and diapers. ``It's torture,'' she said.

\hypertarget{pausing-to-vomit}{%
\paragraph{Pausing to Vomit}\label{pausing-to-vomit}}

Seven hundred miles to the north, Candis Riggins was scrubbing toilets
at a Walmart in Laurel, Md., when she started to feel sick. She was five
months pregnant, and the smell of the cleaning fluids nauseated her. She
complained several times to a manager, who refused to permanently
reassign her to another position. So she kept cleaning bathrooms, often
pausing to vomit.

Doctors told her that chemicals in the cleaning products were
endangering her and her unborn child.

One chilly morning on her way to work, she fainted at the bus stop.

Ms. Riggins again asked a manager for a different job. This time,
Walmart let her clean the store's doors instead of the bathrooms. But
she said the chemicals still made her ill.

She was eight months pregnant when she started regularly missing shifts.
Walmart fired her, citing the absences. She now works at Target.

Mr. Dickens, the Walmart spokesman, said the company allowed her to stop
working with the chemicals she complained about and occasionally let her
work as a cashier or store greeter. Ms. Riggins's lawyer, Dina Bakst,
said that her client still had to spend most of her days cleaning.

In 2017, under pressure from Ms. Woolbright's class-action lawsuit and
E.E.O.C. complaints, Walmart updated its guidelines on how to
accommodate pregnant women. The nationwide policy now includes a
temporary, less taxing job as a ``possible'' solution. It doesn't
provide a guarantee.

Additional production by Whitney Richardson and Jessica White

\hypertarget{more-on-nytimescom}{%
\subsection{More on NYTimes.com}\label{more-on-nytimescom}}

Advertisement

\hypertarget{site-information-navigation}{%
\subsection{Site Information
Navigation}\label{site-information-navigation}}

\begin{itemize}
\tightlist
\item
  \href{https://help.nytimes3xbfgragh.onion/hc/en-us/articles/115014792127-Copyright-notice}{©
  2020 The New York Times Company}
\item
  \href{https://www.nytimes3xbfgragh.onion}{Home}
\item
  \href{https://www.nytimes3xbfgragh.onion/search/}{Search}
\item
  Accessibility concerns? Email us at
  \href{mailto:accessibility@NYTimes.com}{\nolinkurl{accessibility@NYTimes.com}}.
  We would love to hear from you.
\item
  \href{https://help.nytimes3xbfgragh.onion/hc/en-us/articles/115015385887-Contact-Us}{Contact
  Us}
\item
  \href{https://www.nytco.com/careers/}{Work with us}
\item
  \href{https://nytmediakit.com/}{Advertise}
\item
  \href{https://help.nytimes3xbfgragh.onion/hc/en-us/articles/115014892108-Privacy-policy\#pp}{Your
  Ad Choices}
\item
  \href{https://help.nytimes3xbfgragh.onion/hc/en-us/articles/115014892108-Privacy-policy}{Privacy}
\item
  \href{https://help.nytimes3xbfgragh.onion/hc/en-us/articles/115014893428-Terms-of-service}{Terms
  of Service}
\item
  \href{https://help.nytimes3xbfgragh.onion/hc/en-us/articles/115014893968-Terms-of-sale}{Terms
  of Sale}
\end{itemize}

\hypertarget{site-information-navigation-1}{%
\subsection{Site Information
Navigation}\label{site-information-navigation-1}}

\begin{itemize}
\tightlist
\item
  \href{https://spiderbites.nytimes3xbfgragh.onion}{Site Map}
\item
  \href{https://help.nytimes3xbfgragh.onion/hc/en-us}{Help}
\item
  \href{https://help.nytimes3xbfgragh.onion/hc/en-us/articles/115015385887-Contact-Us?redir=myacc}{Site
  Feedback}
\item
  \href{https://www.nytimes3xbfgragh.onion/subscription?campaignId=37WXW}{Subscriptions}
\end{itemize}
