 **NYTimes.com no longer supports Internet Explorer 9 or earlier. Please
upgrade your browser.
\href{http://www.nytimes3xbfgragh.onion/content/help/site/ie9-support.html}{LEARN
MORE »}

**Sections

**Home

**Search

\hypertarget{the-new-york-times}{%
\subsection{\texorpdfstring{\href{http://www.nytimes3xbfgragh.onion/}{The
New York Times}}{The New York Times}}\label{the-new-york-times}}

\hypertarget{-business-}{%
\subsubsection{\texorpdfstring{ \href{/section/business}{Business}
}{ Business }}\label{-business-}}

 \href{/section/business/energy-environment}{Energy \& Environment}
\textbar{}The \$3 Billion Plan to Turn Hoover Dam Into a Giant Battery

**Close search

\hypertarget{site-search-navigation}{%
\subsection{Site Search Navigation}\label{site-search-navigation}}

Search NYTimes.com

**Clear this text input

Go

\url{https://nyti.ms/2mDCEly}

\hypertarget{site-navigation}{%
\subsection{Site Navigation}\label{site-navigation}}

\hypertarget{site-mobile-navigation}{%
\subsection{Site Mobile Navigation}\label{site-mobile-navigation}}

\hypertarget{the-3-billion-plan-to-turn-hoover-dam-into-a-giant-battery}{%
\section{The \$3 Billion Plan to Turn Hoover Dam Into a Giant
Battery}\label{the-3-billion-plan-to-turn-hoover-dam-into-a-giant-battery}}

JULY 24, 2018

Hoover Dam was a public works project likened to the pyramids. Now,
after channeling a river, what if it could tap the power of the sun and
wind?

\hypertarget{the-3-billion-plan-to-turn-hoover-dam-into-a-giant-battery-1}{%
\section{The \$3 Billion Plan to Turn Hoover Dam Into a Giant
Battery}\label{the-3-billion-plan-to-turn-hoover-dam-into-a-giant-battery-1}}

By \href{https://www.nytimes3xbfgragh.onion/by/ivan-penn}{Ivan Penn}.
Graphic by
\href{http://www.nytimes3xbfgragh.onion/by/mika-grondahl}{Mika
Gr}\href{http://www.nytimes3xbfgragh.onion/by/mika-grondahl}{ö}\href{http://www.nytimes3xbfgragh.onion/by/mika-grondahl}{ndahl}.
Photo and video by David Walter Banks. Aerial video by
\href{https://www.nytimes3xbfgragh.onion/by/josh-haner}{Josh Haner} and
\href{https://www.nytimes3xbfgragh.onion/by/josh-williams}{Josh
Williams}.

Hoover Dam helped transform the American West, harnessing the force of
the Colorado River --- along with millions of cubic feet of concrete and
tens of millions of pounds of steel --- to power millions of homes and
businesses. It was one of the great engineering feats of the 20th
century.

Now it is the focus of a distinctly 21st-century challenge: turning the
dam into a vast reservoir of excess electricity, fed by the solar farms
and wind turbines that represent the power sources of the future.

The Los Angeles Department of Water and Power, an original operator of
the dam when it was erected in the 1930s, wants to equip it with a \$3
billion pipeline and a pump station powered by solar and wind energy.
The pump station, downstream, would help regulate the water flow through
the dam's generators, sending water back to the top to help manage
electricity at times of peak demand.

The net result would be a kind of energy storage --- performing much the
same function as the giant lithium-ion batteries being developed to
absorb and release power.

\textbf{The process begins when the dam converts water into energy.
Here's how:}

The Hoover Dam project may help answer a looming question for the energy
industry: how to come up with affordable and efficient power storage,
which is seen as the key to transforming the industry and helping curb
carbon emissions.

Because the sun does not always shine, and winds can be inconsistent,
power companies look for ways to bank the electricity generated from
those sources for use when their output slacks off. Otherwise, they have
to fire up fossil-fuel plants to meet periods of high demand.

And when solar and wind farms produce more electricity than consumers
need, California utilities have had to find ways to get rid of it ---
including giving it away to other states --- or risk overloading the
electric grid and causing blackouts.

``I think we have to look at this as a once-in-a-century moment,'' said
Mayor Eric M. Garcetti of Los Angeles. ``So far, it looks really
possible. It looks sustainable, and it looks clean.''

The target for completion is 2028, and some say the effort could inspire
similar innovations at other dams. Enhancing energy storage could also
affect plans for billions of dollars in wind projects being proposed by
the billionaires Warren E. Buffett and Philip F. Anschutz.

But the proposal will have to contend with political hurdles, including
environmental concerns and the interests of those who use the river for
drinking, recreation and services.

In Bullhead City, Ariz., and Laughlin, Nev. --- sister cities on
opposite sides of the Colorado, about 90 miles south of the dam ---
water levels along certain stretches depend on when dams open and close,
and some residents see a change in its flow as a disruption, if not a
threat.

``Any idea like this has to pass much more than engineering
feasibility,'' Peter Gleick, a co-founder of the Pacific Institute, a
think tank in Oakland, Calif., and a member of the National Academy of
Sciences, internationally known for his work on climate issues. ``It has
to be environmentally, politically and economically vetted, and that's
likely to prove to be the real problem.''

\includegraphics{https://static01.graylady3jvrrxbe.onion/packages/flash/multimedia/ICONS/transparent.png}

\includegraphics{https://static01.graylady3jvrrxbe.onion/packages/flash/multimedia/ICONS/transparent.png}

Housed inside Hoover Dam's 726-foot structure are massive
power-generating units. The proposed pump station would help regulate
the water flow through the dam's generators, sending water back to the
top to help manage electricity at times of peak demand.

\hypertarget{an-idea-comes-of-age}{%
\subsection{AN IDEA COMES OF AGE}\label{an-idea-comes-of-age}}

Using Hoover Dam to help manage the electricity grid has been mentioned
informally over the last 15 years. But no one pursued the idea seriously
until about a year ago, as California began grappling with the need to
better manage its soaring alternative-electricity production --- part of
weaning itself from coal-fired and nuclear power plants.

In California, by far the leading state in solar power production, that
has sometimes meant paying other states to take excess electricity.
Companies like Tesla have gotten into the picture, making lithium-ion
batteries that are deployed by some utilities, but that form of storage
generally remains pricey.

Lazard, the financial advisory and asset management firm, has estimated
that utility-scale lithium-ion batteries cost 26 cents a kilowatt-hour,
compared with 15 cents for a pumped-storage hydroelectric project. The
typical household pays about 12.5 cents a kilowatt-hour for electricity.

Some dams already provide a basis for the Hoover Dam proposal. Los
Angeles operates a hydroelectric plant at Pyramid Lake, about 50 miles
northwest of the city, that stores energy by using the electric grid to
spin a turbine backward and pump water back into the lake.

\emph{{[}Read more: It's tricky to store energy on an industrial scale,
but}
\href{https://www.nytimes3xbfgragh.onion/2017/06/03/business/energy-environment/biggest-batteries.html}{engineers
have devised clever workarounds}\emph{.{]}}

But the Hoover Dam proposal would operate differently. The dam, with its
towering 726-foot concrete wall and its 17 power generators that came
online in 1936, would not be touched. Instead, engineers propose
building a pump station about 20 miles downstream from the main
reservoir, Lake Mead, the nation's largest artificial lake. A pipeline
would run partly or fully underground, depending on the location
ultimately approved.

``Hoover Dam is ideal for this,'' said Kelly Sanders, an assistant
professor of civil and environmental engineering at the University of
Southern California. ``It's a gigantic plant. We don't have anything on
the horizon as far as batteries of that magnitude.''

Sri Narayan, a chemistry professor at the university, said his studies
of lithium-ion batteries showed that they simply weren't ready to store
the loads needed to manage all of the wind and solar power coming
online.

``With lithium-ion batteries, you have durability issues,'' Mr. Narayan
said. ``If they last five to 10 years, that would be a stretch,
especially because we expect to use these facilities at full capacity.
It has to be 10 times more durable than it is today.''

Mr. Narayan said he felt the Hoover Dam project should be given serious
consideration because pumped-storage projects had been tested and proven
for decades. In a comparison with lithium-ion batteries, he said, ``I
think the argument is very good.''

\includegraphics{https://static01.graylady3jvrrxbe.onion/packages/flash/multimedia/ICONS/transparent.png}

An aerial view of the Colorado River downstream from Lake Mead and
Hoover Dam. Harnessing the river's force, the dam helped transform the
American West.

\hypertarget{waiting-for-washington}{%
\subsection{WAITING FOR WASHINGTON}\label{waiting-for-washington}}

The Los Angeles Department of Water and Power, the nation's largest
municipal utility, says its proposal would increase the productivity of
the dam, which operates at just 20 percent of its potential, to avoid
releasing too much water at once and flooding towns downstream.

Engineers have conducted initial feasibility studies, including a review
of locations for the pump station that would have as little adverse
impact on the environment and nearby communities as possible.

But because Hoover Dam sits on federal land and operates under the
Bureau of Reclamation, part of the Interior Department, the bureau must
back the project before it can proceed.

``We're aware of the concept, but at this point our regional management
has not seen the concept in enough detail to know where we would stand
on the overall project,'' said Doug Hendrix, a bureau spokesman.

If the bureau agrees to consider the project, the National Park Service
will review the environmental, scientific and aesthetic impact on the
downstream recreation area. If the Los Angeles utility receives
approval, Park Service officials have told it, the agency wants the
pumping operation largely invisible to the public, which could require
another engineering feat.

Among the considerations is the effect on bighorn sheep that roam Black
Canyon, just below the dam, and on drinking water for places like
Bullhead City. Some environmentalists worry that adding a pump facility
would impair water flow farther downstream, in particular at the
Colorado River Delta, a mostly dry riverbed in Mexico that no longer
connects to the sea.

Another concern is that the pump station would draw water from or close
to Lake Mohave, where water enthusiasts boat, fish, ride Jet-Skis, kayak
and canoe.

Keri Simons, a manager of Watercraft Adventures, a 27-year-old rental
business in Laughlin, said water levels already fluctuated in stretches
of the Colorado close to the river towns. The smaller Davis Dam, just
north of Laughlin, shuts off the flow overnight.

One morning this year, the water level just outside town dropped so low
that you could walk across the riverbed, Ms. Simons said. ``We couldn't
put any boats out until noon,'' she said. ``Half the river was a
sandbar.''

Even if no water is lost because of the pumping project, the thought of
any additional stress on the system worries Toby Cotter, the city
manager of Bullhead City.

The town thrives on the summer tourism that draws some two million
visitors to the area for recreation on the greenish-blue waters, Mr.
Cotter said. ``That lake is the lifeblood of this community,'' he said.
``It's not uncommon to see 100 boats on that lake.''

\includegraphics{https://static01.graylady3jvrrxbe.onion/packages/flash/multimedia/ICONS/transparent.png}

\includegraphics{https://static01.graylady3jvrrxbe.onion/packages/flash/multimedia/ICONS/transparent.png}

Despite the possible benefits of the project, there are concerns among
community members and business owners in the area. Keri Simons, manager
of a watercraft-rental business in Laughlin, said water levels were
already inconsistent along stretches of the Colorado, sometimes leaving
a footpath across the river.

\hypertarget{a-troubled-relationship}{%
\subsection{A TROUBLED RELATIONSHIP}\label{a-troubled-relationship}}

Environmentalists have been pushing Los Angeles to stop using fossil
fuels and produce electricity from alternative sources like solar and
wind power. And Mayor Garcetti said he would like his city to be the
first in the nation to operate solely on clean energy, while maintaining
a reliable electric system.

``Our challenge is: How do we get to 100 percent green?'' he said.
``Storage helps. There's no bigger battery in our system than Hoover
Dam.''

But old wounds are still raw with some along the Colorado. A coal-fired
power plant in Laughlin that the Department of Water and Power and other
utilities operated was shut down in 2006, costing 500 jobs and causing
the local economies to buckle. And a decision long ago to allot Nevada a
small fraction of the water that California and Arizona can draw remains
a sore point.

``There's nothing going on in California with power that has given
people who are dealing with them any comfort,'' said Joseph Hardy, a
Nevada state senator. ``I think from a political standpoint, we would
have to allay the fears of California, Nevada and Arizona. There will be
a myriad of concerns.''

The decision to close the coal plant angered many residents. They wanted
the utility to simply add emission-control features known as scrubbers
to reduce carbon pollution. The community later hoped a natural-gas
plant would replace the coal facility, but Los Angeles could not agree
with the local communities on a site.

The 2,500-acre parcel where the coal plant stood remains largely vacant.
``There's still some sting here,'' said Mr. Cotter, the Bullhead City
official.

There have been local efforts to convert the site into a development of
housing and businesses --- or to build a solar farm on a plot of land,
if Los Angeles would buy the power.

Mr. Garcetti said other states and cities had worked with Los Angeles to
build economic development projects for their communities, so he would
like to consider similar ideas for the Hoover Dam project, as well as
ways to benefit the entire region. ``I'm all open ears to what their
needs are,'' he said.

Mr. Hardy is wary of big-city promises. The Department of Water and
Power has treated Nevada so cavalierly, he said, that a security guard
at the old coal plant site once refused to return a ball to children
after it bounced over the property's fence. He said the guard had told
the children's parents that they could file a claim to get it back --- a
process that would take two to three years.

``Not the kindest neighbor,'' Mr. Hardy said.

But he said he was willing to meet with Los Angeles officials to make
the project successful.

``The hurdles are minimal and the negotiations simple, as long as
everybody agrees with Nevada,'' Mr. Hardy said. ``It would be nice if
there was a table that they would come to. I'll provide the table.''

\includegraphics{https://static01.graylady3jvrrxbe.onion/packages/flash/multimedia/ICONS/transparent.png}

The Hoover Dam is visited by millions of tourists every year. If the
plan to pump water back to Lake Mead comes to fruition, federal
officials want the operation to be largely invisible to the public.

Sources: The Los Angeles Department of Water and Power; Bureau of
Reclamation

Produced by Keith Collins, Meg Felling, Whitney Richardson and Josh
Williams.

\hypertarget{more-on-nytimescom}{%
\subsection{More on NYTimes.com}\label{more-on-nytimescom}}

Advertisement

\hypertarget{site-information-navigation}{%
\subsection{Site Information
Navigation}\label{site-information-navigation}}

\begin{itemize}
\tightlist
\item
  \href{https://help.nytimes3xbfgragh.onion/hc/en-us/articles/115014792127-Copyright-notice}{©
  2020 The New York Times Company}
\item
  \href{https://www.nytimes3xbfgragh.onion}{Home}
\item
  \href{https://www.nytimes3xbfgragh.onion/search/}{Search}
\item
  Accessibility concerns? Email us at
  \href{mailto:accessibility@NYTimes.com}{\nolinkurl{accessibility@NYTimes.com}}.
  We would love to hear from you.
\item
  \href{https://help.nytimes3xbfgragh.onion/hc/en-us/articles/115015385887-Contact-Us}{Contact
  Us}
\item
  \href{https://www.nytco.com/careers/}{Work with us}
\item
  \href{https://nytmediakit.com/}{Advertise}
\item
  \href{https://help.nytimes3xbfgragh.onion/hc/en-us/articles/115014892108-Privacy-policy\#pp}{Your
  Ad Choices}
\item
  \href{https://help.nytimes3xbfgragh.onion/hc/en-us/articles/115014892108-Privacy-policy}{Privacy}
\item
  \href{https://help.nytimes3xbfgragh.onion/hc/en-us/articles/115014893428-Terms-of-service}{Terms
  of Service}
\item
  \href{https://help.nytimes3xbfgragh.onion/hc/en-us/articles/115014893968-Terms-of-sale}{Terms
  of Sale}
\end{itemize}

\hypertarget{site-information-navigation-1}{%
\subsection{Site Information
Navigation}\label{site-information-navigation-1}}

\begin{itemize}
\tightlist
\item
  \href{https://spiderbites.nytimes3xbfgragh.onion}{Site Map}
\item
  \href{https://help.nytimes3xbfgragh.onion/hc/en-us}{Help}
\item
  \href{https://help.nytimes3xbfgragh.onion/hc/en-us/articles/115015385887-Contact-Us?redir=myacc}{Site
  Feedback}
\item
  \href{https://www.nytimes3xbfgragh.onion/subscription?campaignId=37WXW}{Subscriptions}
\end{itemize}
