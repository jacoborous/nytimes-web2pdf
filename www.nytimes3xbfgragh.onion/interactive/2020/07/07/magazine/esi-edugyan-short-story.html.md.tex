Sections

SEARCH

\protect\hyperlink{site-content}{Skip to
content}\protect\hyperlink{site-index}{Skip to site index}

\hypertarget{comments}{%
\subsection{\texorpdfstring{\protect\hyperlink{commentsContainer}{Comments}}{Comments}}\label{comments}}

\href{}{Esi Edugyan: `To the Wall,' a Short Story}\href{}{Skip to
Comments}

The comments section is closed. To submit a letter to the editor for
publication, write to
\href{mailto:letters@NYTimes.com}{\nolinkurl{letters@NYTimes.com}}.

The Decameron Project

\hypertarget{esi-edugyan-to-the-wall-a-short-story}{%
\section{Esi Edugyan: `To the Wall,' a Short
Story}\label{esi-edugyan-to-the-wall-a-short-story}}

By Esi EdugyanJuly 8, 2020

\begin{itemize}
\item
\item
\item
\item
\item
  \emph{+}
\end{itemize}

The Decameron Project

\href{https://www.nytimes3xbfgragh.onion/section/magazine}{\includegraphics{https://static01.graylady3jvrrxbe.onion/packages/flash/multimedia/ICONS/transparent.png}\includegraphics{https://static01.graylady3jvrrxbe.onion/packages/flash/multimedia/ICONS/transparent.png}}

New Fiction

Four years before **** the outbreak, I traveled into the snowbound hills
west of Beijing with my first husband, Tomas.

He was an installation artist from Lima who was working at the time on a
replica of a 10th-century cloister. Years before, he became obsessed
with the story of a nun in medieval France who awoke screaming one
morning and couldn't stop. She was joined over the following days by
another sister, then another, until the whole convent echoed with their
cries. They only quieted when the local soldiers threatened to beat
them. What compelled Tomas, I think, was the lack of choice in these
women's lives, in their fates, placed as girls in convents by parents
who didn't want them, or couldn't support them. The screaming seemed
like a choice that they could make. In any case, he was struggling with
the project. At the time of our trip, he didn't think he'd finish it,
and neither did I. Already then, something was going out of him.

But that morning of our journey out to see the Great Wall, the hours
felt whole and unspoiled. We had been bickering for weeks, but the
novelty of the Chinese countryside, with its strange textures and
weather and food, had shifted things between us. Tomas grinned as we
arrived at the tourists' entrance, his teeth very straight and white in
his narrow face.

Vendors along the stone path called to us, their breath clouding on the
air. A woman hollered for us to buy polished jade paperweights and
shimmering cloth wallets, fake money tied with red string and
transparent pens in which small plastic boats floated through viscous
liquid as if journeying up the Yangtze. The wind was sharp and fresh,
with an almost grasslike scent you didn't get in the city.

We crawled into the glass cable car that would carry us to the upper
paths. As it began to lurch its way across the canyon, above trees black
as night water, we laughed nervously. Then we were up, finally, walking
the ancient stone corridor, the pale light cold on our foreheads. The
air tasted faintly of metal.

``Should we have bought something back there, from that woman?'' I said.
``For my mother?''

``Gabriel wants Chinese cigarettes,'' Tomas said, his dark eyes watering
in the strong wind. ``I don't know. Somehow it's more stylish to smoke
foreign ones.''

``You're hard on him,'' I said.

I shouldn't have said it. Tomas glanced at me, quiet. He didn't like to
talk about his brother much in those days. Between them lay a gentle
hatred whose childhood roots were still murky to me, despite a decade of
marriage. It could only be made worse, later, by the accident that
happened two years after we returned from China. Tomas would strike his
nephew with his car, killing the boy. The child just 3. By then Tomas
and I had entered the era of our disaffection. What I'd know I'd learn
through a mutual friend. The death would be a barrier through which
nothing could pass, and everyone connected with it would disappear on
the far side, lost.

But that day, over the coming hours, the twisting rock path stretched
out before us into the distant fog. We walked along a section that had
purple veining on the stones, as well as starker, whiter rock, and stone
of such muddy gray you felt intensely how ancient and elemental it was.
And though we spoke easily, laughing, I could feel --- we both could ---
the shadow of my earlier remark.

The fog grew heavier. Snow began to fall.

It seemed the right time to leave. We retraced our steps back to the
glass cable-car entrance, but it was nowhere to be found. We tried
another path, but it ended in a lookout. We stared at each other. The
snow got thicker.

Behind us, a sudden figure was striding away. Tomas called out to the
man, but as we rounded the corner, he was gone.

The afternoon was growing darker. A strong smell of soil filled the air.
We ascended a set of crooked steps that led onto a landing that stopped
abruptly at a barrier. Another set descended to a solid wall. One path
seemed to stretch into nowhere, and we gave up following it. My
fingertips began to burn with cold. I pictured Beijing at this hour, the
bright restaurants on the street near our hotel, the air smelling of
exhaust and fried meat and sun-warmed blossoms, their fallen petals like
drops of pale wax on the pavement.

``We are in an Escher drawing,'' Tomas cried, strangely elated.

I smiled, too, but shivering, the wind a high whistle in my ears. Snow
had clotted on my eyelashes, so that I blinked hard.

Two dark-haired women appeared then, a cluster of canisters at their
feet. I was surprised to see a mild disappointment in Tomas's face. I
began to gesture and explain we were lost. They listened without
expression, their wet wrinkles glistening. Then one turned to Tomas, and
speaking shyly in Mandarin, she lifted her ancient hands and brushed the
flakes of ice from his hair. He gave a boyish laugh, delighted.

The second woman drew from a canister by her feet two foam cups steaming
with tea. When she had poured these, or how she'd managed to keep the
water hot on so cold a day high up in those hills, I did not know. But
Tomas took his with great ceremony. I waved mine away.

The women gestured behind them, and there they were --- the cable cars.
The glass domes swayed over the open black valley as if newly restored.

Tomas made a noise of astonishment. As we went toward the cable cars, he
spoke in wonder at the feel of the woman's palms on his head, their
surprising weight, the roughness of her skin.

But on the drive back to Beijing, we said little. It felt strange not to
talk, after so long. Tomas was always garrulous in his moments of
happiness, but now he seemed emptied, as if something had been slowly
forced out of him. As we reached the hotel, I could tell by the tension
in his mouth that he was still troubled by a thing I couldn't quite
grasp. Gently, I took his hand. He gripped mine back, as if he knew
where our lives were going, as if the ravages had already happened. All
over the world there were lights going out, even then.

\hypertarget{to-the}{%
\subsection{To The}\label{to-the}}

+wAll+

\hypertarget{by}{%
\paragraph{By}\label{by}}

\hypertarget{esi-edugyan}{%
\paragraph{Esi Edugyan}\label{esi-edugyan}}

A short story from The New York Times Magazine's Decameron Project.

1

2

\includegraphics{https://static01.graylady3jvrrxbe.onion/packages/flash/multimedia/ICONS/transparent.png}

\includegraphics{https://static01.graylady3jvrrxbe.onion/images/2020/07/12/magazine/12mag-Edugyan/12mag-Edugyan-master1050.jpg}

3

4

5

Spot illustrations and lettering by \emph{Sophy Hollington}

---

\emph{Esi Edugyan} is the author of ``Washington Black,'' ``Half-Blood
Blues'' and ``Dreaming of Elsewhere: Observations on Home.'' She lives
in Victoria, British Columbia.

\emph{Sophy Hollington} is a British artist and illustrator. She is
known for her use of relief prints, created using the process of the
linocut and inspired by meteoric folklore as well as alchemical
symbolism.

\includegraphics{https://static01.graylady3jvrrxbe.onion/packages/flash/multimedia/ICONS/transparent.png}

\includegraphics{https://static01.graylady3jvrrxbe.onion/images/2020/07/12/magazine/12mag-Edugyan/12mag-Edugyan-master1050.jpg}

\hypertarget{read-more-from-the-decameron-project}{%
\subsubsection{Read More from the Decameron
Project}\label{read-more-from-the-decameron-project}}

\href{https://www.nytimes3xbfgragh.onion/interactive/2020/07/07/magazine/victor-lavalle-short-story.html}{}

\hypertarget{victor}{%
\section{VICtor}\label{victor}}

LavaLLe

\includegraphics{https://static01.graylady3jvrrxbe.onion/packages/flash/multimedia/ICONS/transparent.png}

\includegraphics{https://static01.graylady3jvrrxbe.onion/images/2020/07/12/magazine/12mag-LaValle-2/12mag-LaValle-2-master1050.jpg}

VICtor LavaLLe Read the story

\href{https://www.nytimes3xbfgragh.onion/interactive/2020/07/07/magazine/alejandro-zambra-short-story.html}{}

\hypertarget{alejandro}{%
\section{alejandro}\label{alejandro}}

ZaMBRa

\includegraphics{https://static01.graylady3jvrrxbe.onion/packages/flash/multimedia/ICONS/transparent.png}

\includegraphics{https://static01.graylady3jvrrxbe.onion/images/2020/07/12/magazine/12mag-zambra/12mag-zambra-master1050.jpg}

alejandro ZaMBRa Read the story

\href{https://www.nytimes3xbfgragh.onion/interactive/2020/07/07/magazine/david-mitchell-short-story.html}{}

\hypertarget{david}{%
\section{DAviD}\label{david}}

mitchell

My brain's a featherweight stuck in a cage with the Hulk. He just keeps
pummeling.

DAviD mitchell Read the story

\hypertarget{see-all}{%
\subparagraph{\texorpdfstring{\href{https://www.nytimes3xbfgragh.onion/interactive/2020/07/07/magazine/decameron-project-short-story-collection.html}{See
All}}{See All}}\label{see-all}}

\hypertarget{the-decameron-project-1}{%
\subparagraph{\texorpdfstring{\href{https://www.nytimes3xbfgragh.onion/interactive/2020/07/07/magazine/decameron-project-short-story-collection.html}{The
Decameron
Project}}{The Decameron Project}}\label{the-decameron-project-1}}

\hypertarget{read-the-next-story}{%
\subparagraph{\texorpdfstring{\href{https://www.nytimes3xbfgragh.onion/interactive/2020/07/07/magazine/laila-lalami-short-story.html}{Read
the Next Story}}{Read the Next Story}}\label{read-the-next-story}}

Credits

Additional design and development by \emph{Shannon Lin} and \emph{Jacky
Myint.}

The Decameron Project ·

Write a comment

\begin{itemize}
\item
\item
\item
\item
\end{itemize}

Advertisement

\protect\hyperlink{after-bottom}{Continue reading the main story}

\hypertarget{site-index}{%
\subsection{Site Index}\label{site-index}}

\hypertarget{site-information-navigation}{%
\subsection{Site Information
Navigation}\label{site-information-navigation}}

\begin{itemize}
\tightlist
\item
  \href{https://help.nytimes3xbfgragh.onion/hc/en-us/articles/115014792127-Copyright-notice}{©~2020~The
  New York Times Company}
\end{itemize}

\begin{itemize}
\tightlist
\item
  \href{https://www.nytco.com/}{NYTCo}
\item
  \href{https://help.nytimes3xbfgragh.onion/hc/en-us/articles/115015385887-Contact-Us}{Contact
  Us}
\item
  \href{https://www.nytco.com/careers/}{Work with us}
\item
  \href{https://nytmediakit.com/}{Advertise}
\item
  \href{http://www.tbrandstudio.com/}{T Brand Studio}
\item
  \href{https://www.nytimes3xbfgragh.onion/privacy/cookie-policy\#how-do-i-manage-trackers}{Your
  Ad Choices}
\item
  \href{https://www.nytimes3xbfgragh.onion/privacy}{Privacy}
\item
  \href{https://help.nytimes3xbfgragh.onion/hc/en-us/articles/115014893428-Terms-of-service}{Terms
  of Service}
\item
  \href{https://help.nytimes3xbfgragh.onion/hc/en-us/articles/115014893968-Terms-of-sale}{Terms
  of Sale}
\item
  \href{https://spiderbites.nytimes3xbfgragh.onion}{Site Map}
\item
  \href{https://help.nytimes3xbfgragh.onion/hc/en-us}{Help}
\item
  \href{https://www.nytimes3xbfgragh.onion/subscription?campaignId=37WXW}{Subscriptions}
\end{itemize}
