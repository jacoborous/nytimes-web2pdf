Sections

SEARCH

\protect\hyperlink{site-content}{Skip to
content}\protect\hyperlink{site-index}{Skip to site index}

\hypertarget{comments}{%
\subsection{\texorpdfstring{\protect\hyperlink{commentsContainer}{Comments}}{Comments}}\label{comments}}

\href{}{Victor LaValle: `Recognition,' a Short Story}\href{}{Skip to
Comments}

The comments section is closed. To submit a letter to the editor for
publication, write to
\href{mailto:letters@NYTimes.com}{\nolinkurl{letters@NYTimes.com}}.

The Decameron Project

\hypertarget{victor-lavalle-recognition-a-short-story}{%
\section{Victor LaValle: `Recognition,' a Short
Story}\label{victor-lavalle-recognition-a-short-story}}

By Victor LaValleJuly 8, 2020

\begin{itemize}
\item
\item
\item
\item
\item
  \emph{+}
\end{itemize}

The Decameron Project

\href{https://www.nytimes3xbfgragh.onion/section/magazine}{\includegraphics{https://static01.graylady3jvrrxbe.onion/packages/flash/multimedia/ICONS/transparent.png}\includegraphics{https://static01.graylady3jvrrxbe.onion/packages/flash/multimedia/ICONS/transparent.png}}

New Fiction

Not easy to find a good apartment in New York City, so imagine finding a
good building. No, this isn't a story about me buying a building. I'm
talking about the people, of course. I found a good apartment, and a
great building, in Washington Heights. Six-story tenement on the corner
of 180th and Fort Washington Avenue; a one-bedroom apartment, which was
plenty for me. Moved in December 2019. You might already see where this
is going. The virus hit, and within four months half the building had
emptied out. Some of my neighbors fled to second homes or to stay with
their parents outside the city; others, the older ones, the poorer ones,
disappeared into the hospital 12 blocks away. I'd moved into a crowded
building and suddenly I lived in an empty house.

And then I met Mirta.

``Do you believe in past lives?''

We were in the lobby, waiting for the elevator. This was right after the
lockdown started. She asked, but I didn't say anything. Which isn't the
same as saying I didn't respond. I gave my tight little smile while
looking down at my feet. I'm not rude, just fantastically shy. That
condition doesn't go away, not even during a pandemic. I'm a Black
woman, and people act surprised when they discover some of us can be
awkward, too.

``There's no one else here,'' Mirta continued. ``So I must be talking to
you.''

Her tone managed to be both direct and, somehow, still playful. As the
elevator arrived, I looked toward her, and that's when I saw her shoes.
Black-and-white pointed oxfords; the white portion had been painted to
look like piano keys. Despite the lockdown, Mirta had taken the trouble
to slip on a pair of shoes that nice. I was returning from the
supermarket wearing my raggedy old slides.

I pulled the elevator door open and finally looked at her face.

``There she is,'' Mirta said, the way you might compliment a shy bird
for settling on your finger.

Mirta might've been 20 years older than me. I turned 40 the same month I
moved into the building. My mom and dad called to sing ``Happy
Birthday'' from Pittsburgh. Despite the news, they didn't ask me to come
home. And I didn't make the request. When we're together, they ask
questions about my life, my plans, that turn me into a grouchy teenager
again. My father ordered me a bunch of basics though; he had them
shipped. It's how he has always loved me --- by making sure I'm well
supplied.

``I tried to get toilet paper,'' Mirta said in the elevator. ``But these
people are panicking, so I couldn't find any. They think a clean butt is
going to save them from the virus?''

Mirta watched me; the elevator reached the fourth floor. She stepped out
and held the door open.

``You don't laugh at my jokes, and you won't even tell me your name?''

Now I smiled because it had turned into a game.

``A challenge then,'' she said. ``I will see you again.'' She pointed
down the hall. ``I am in No. 41.''

She let the elevator door go, and I rode up to the sixth floor, unpacked
the things I'd bought. At that time I still thought it would all be over
by April. It's laughable now. I went into the bathroom. One of the
things my dad sent me was 32 rolls of toilet paper. I slipped back down
to the fourth floor and left three rolls in front of Mirta's door.

\textbf{A month later,} I was used to logging in to my ``remote
office,'' the grid of screens --- all our little heads --- looked like
the open office we once worked in; I probably spoke with my co-workers
about as much now as I did then. When the doorbell rang, I leapt at the
chance to get away from my laptop. \emph{Maybe it's Mirta.} I slipped on
a pair of buckled loafers; they were raggedy, too, but better than the
slippers I wore the last time she saw me.

But it wasn't her.

It was the super, Andrés. Nearly 60, born in Puerto Rico, he had a
tattoo of a leopard crawling up his neck.

``Still here,'' he said, sounding pleasant behind his blue mask.

``Nowhere else to go.''

He nodded and snorted, a mix between a laugh and a cough. ``The city
says I got to check every apartment now. Every day.''

He carried a bag that rattled like a sack of metal snakes. When I
looked, he pulled it open: silver spray-paint cans. ``I don't get a
answer, and I got to use this.''

Andrés stepped to the side. Down the hall; Apartment 66. The green door
had been defaced with a giant silver ``V.'' So fresh, the letter still
dripped.

`` `V.' For `virus'?''

Andrés's eyebrows rose and fell.

``Vacant,'' he said.

``That's a nicer way to put it, I guess.'' We stood quietly, him in the
hall and me in the apartment. I realized I hadn't put on my mask when I
answered, and I covered my mouth when I spoke.

``The city is making you do this?'' I asked.

``In some neighborhoods,'' Andrés said. ``Bronx, Queens, Harlem. And us.
Hot spots.'' He took out one of the cans and shook it. The ball bearing
clicked and clacked inside. ``I'll knock tomorrow,'' he said. ``If you
don't answer, I got the keys.''

I watched him go.

``How many people are left?'' I called out. ``In the building?''

He'd already reached the stairs, started down. If he answered, I didn't
hear it. I walked onto the landing. There were six apartments on my
floor. Five doors had been decorated with the letter ``V.'' No one here
but me.

You'd think I would run right down to Mirta's place, but I couldn't
afford to lose my job. The landlord hadn't said a word about rent
forgiveness. I went back to the computer until end of day. I felt such
relief when No. 41 hadn't been painted. I knocked until Mirta opened.
She wore her mask, just like me now, but I could tell she was smiling.
She looked from my face to my feet.

``Those shoes have seen better days,'' she said, and laughed so joyfully
that I hardly even felt embarrassed.

\textbf{Mirta and I} made trips to the supermarket together; two trips
to the store each week. We walked side by side, arm's length apart, and
when we crossed paths with others, we marched single file. Mirta talked
the whole time, whether I was next to her or behind her. I know some
people criticize chatty folks, but her chatter fell upon me like a
nourishing rain.

She came to New York from Cuba, with a short stay in Key West, Fla., in
between. She'd lived in Manhattan, from the bottom to the top, over the
span of 40 years. She played piano and idolized Peruchín; had performed
with Chucho Valdés. And now she gave lessons to children in her
apartment for \$35 an hour. Or at least she had done that, until the
virus made it unsafe to have them over. I miss them, she said, every
time we talked, as four weeks became six, and six became 12. She
wondered if she'd ever see her students and their parents again.

I offered to help set up remote piano lessons. I'd use my job's account
to set up free chat sessions for her. But this was three months in and
Mirta had lost her playful ways. She said: ``The screens give the
illusion that we're all still connected. But it's not true. The ones who
could leave, left. The rest of us? We were abandoned.''

She stepped off the elevator.

``Why pretend?''

\textbf{She scared me.} I can see that now. But I told myself I'd become
busier. As if I'd transformed. But I fled from her. We were all living
on the ledge of despair, so when she said it --- ``We were abandoned.
Why pretend?'' --- it was as if she spoke from down in that pit. A place
I found myself slipping into often enough already. So I went to the
store by myself, and I held my breath when the elevator passed the
fourth floor.

Meanwhile, Andrés continued to work. I didn't see him. He knocked on the
door each morning, and I knocked from the other side. But I saw evidence
of his work. Three apartments on the first floor marked with a ``V'' one
week. Next time I went to the store the other three were painted.

Four on the second floor.

Five on the third.

One afternoon I heard him kicking at a door on the fourth floor.
Shouting a name I hardly recognized through the muzzle that was his
mask. I left my place and walked down. Andrés looked shrunken at the
door of No. 41. He kicked at it desperately.

``Mirta!'' he shouted again.

He turned with surprise when I appeared. His eyes were red. The fingers
of his right hand were entirely silver now; it looked permanent. I
wondered if he'd ever be able to wash off the spray paint. But how could
he, if the job was never done?

``I left my keys,'' he said. ``I gotta get them.''

``I'll stay,'' I said.

He sprinted down the stairs. I stood by the door, didn't bother
knocking. If that kicking didn't wake her, what could I do?

``Is he gone?''

I almost collapsed.

``Mirta! Were you messing with him?''

``No,'' she said through the door. ``But I wasn't waiting on him. I was
waiting around for you.''

I sat so my head was at about the same level as her voice. I heard her
labored breathing through the door. ``It's been a while,'' she finally
said.

I rested the side of my head against the cool door. ``I'm sorry.''

She sniffed. ``Even women like us are scared of women like us.''

I lowered my mask, as if it were getting in the way of what I truly
needed to say. But I still couldn't find the words.

``Do you believe in past lives?'' she said.

``That's the first thing you ever asked me.''

``When I saw you by the elevator, I knew we met before. Recognition.
Like seeing a member of my family.''

The elevator arrived. Andrés stepped out. I raised my mask and got to my
feet. He unlocked the door.

``Be careful,'' I said. ``She's right there.''

But when he pushed the door open, the hall sat empty.

Andrés found her in bed. Dead. He came out carrying a bag, my name
written on it. Her black-and-white oxfords were inside. A note in the
left shoe. \emph{Give them back when you see me again.}

I have to slip on an extra pair of socks to make them fit, but I wear
them everywhere I go.

\includegraphics{https://static01.graylady3jvrrxbe.onion/packages/flash/multimedia/ICONS/transparent.png}

\includegraphics{https://static01.graylady3jvrrxbe.onion/images/2020/07/12/magazine/12mag-LaValle-2/12mag-LaValle-2-master1050.jpg}

\hypertarget{recognition}{%
\subsection{recognition}\label{recognition}}

\hypertarget{by}{%
\paragraph{By}\label{by}}

\hypertarget{victor-lavalle}{%
\paragraph{Victor LaValle}\label{victor-lavalle}}

A short story from The New York Times Magazine's Decameron Project.

1

2

One of the things my dad sent me was 32 rolls of toilet paper. I slipped
back down to the fourth floor and left three rolls in front of Mirta's
door.

3

4

5

Illustration by \emph{Kyutae Lee}

Spot illustrations and lettering by \emph{Sophy Hollington}

---

\emph{Victor LaValle} is the author of seven works of fiction. His most
recent novel is ``The Changeling.'' He teaches at Columbia University.

\emph{Kyutae Lee} is an illustrator and animator from Seoul, South
Korea, who recreates scenes using light-handed drawing.

\emph{Sophy Hollington} is a British artist and illustrator. She is
known for her use of relief prints, created using the process of the
linocut and inspired by meteoric folklore as well as alchemical
symbolism.

\includegraphics{https://static01.graylady3jvrrxbe.onion/packages/flash/multimedia/ICONS/transparent.png}

\includegraphics{https://static01.graylady3jvrrxbe.onion/newsgraphics/2020/06/29/mag-decameron/assets/images/star3.png}

\hypertarget{read-more-from-the-decameron-project}{%
\subsubsection{Read More from the Decameron
Project}\label{read-more-from-the-decameron-project}}

\href{https://www.nytimes3xbfgragh.onion/interactive/2020/07/07/magazine/Paolo-Giordano-short-story.html}{}

\hypertarget{paolo}{%
\section{PAoLO}\label{paolo}}

giordano

\includegraphics{https://static01.graylady3jvrrxbe.onion/packages/flash/multimedia/ICONS/transparent.png}

\includegraphics{https://static01.graylady3jvrrxbe.onion/images/2020/07/12/magazine/12mag-Giordano/12mag-Giordano-master1050-v3.jpg}

PAoLO giordano Read the story

\href{https://www.nytimes3xbfgragh.onion/interactive/2020/07/07/magazine/matthew-baker-short-story.html}{}

\hypertarget{matthew}{%
\section{matthew}\label{matthew}}

BaKEr

\includegraphics{https://static01.graylady3jvrrxbe.onion/packages/flash/multimedia/ICONS/transparent.png}

\includegraphics{https://static01.graylady3jvrrxbe.onion/images/2020/07/12/magazine/12mag-baker/12mag-baker-master1050-v3.jpg}

matthew BaKEr Read the story

\href{https://www.nytimes3xbfgragh.onion/interactive/2020/07/07/magazine/charles-yu-short-story.html}{}

\hypertarget{charles}{%
\section{Charles}\label{charles}}

YU

\includegraphics{https://static01.graylady3jvrrxbe.onion/packages/flash/multimedia/ICONS/transparent.png}

\includegraphics{https://static01.graylady3jvrrxbe.onion/images/2020/07/12/magazine/12mag-yu/12mag-yu-master1050-v2.jpg}

Charles YU Read the story

\hypertarget{see-all}{%
\subparagraph{\texorpdfstring{\href{https://www.nytimes3xbfgragh.onion/interactive/2020/07/07/magazine/decameron-project-short-story-collection.html}{See
All}}{See All}}\label{see-all}}

\hypertarget{the-decameron-project-1}{%
\subparagraph{\texorpdfstring{\href{https://www.nytimes3xbfgragh.onion/interactive/2020/07/07/magazine/decameron-project-short-story-collection.html}{The
Decameron
Project}}{The Decameron Project}}\label{the-decameron-project-1}}

\hypertarget{read-the-next-story}{%
\subparagraph{\texorpdfstring{\href{https://www.nytimes3xbfgragh.onion/interactive/2020/07/07/magazine/mia-cuoto-short-story.html}{Read
the Next Story}}{Read the Next Story}}\label{read-the-next-story}}

Credits

Additional design and development by \emph{Shannon Lin} and \emph{Jacky
Myint.}

The Decameron Project ·

Write a comment

\begin{itemize}
\item
\item
\item
\item
\end{itemize}

Advertisement

\protect\hyperlink{after-bottom}{Continue reading the main story}

\hypertarget{site-index}{%
\subsection{Site Index}\label{site-index}}

\hypertarget{site-information-navigation}{%
\subsection{Site Information
Navigation}\label{site-information-navigation}}

\begin{itemize}
\tightlist
\item
  \href{https://help.nytimes3xbfgragh.onion/hc/en-us/articles/115014792127-Copyright-notice}{©~2020~The
  New York Times Company}
\end{itemize}

\begin{itemize}
\tightlist
\item
  \href{https://www.nytco.com/}{NYTCo}
\item
  \href{https://help.nytimes3xbfgragh.onion/hc/en-us/articles/115015385887-Contact-Us}{Contact
  Us}
\item
  \href{https://www.nytco.com/careers/}{Work with us}
\item
  \href{https://nytmediakit.com/}{Advertise}
\item
  \href{http://www.tbrandstudio.com/}{T Brand Studio}
\item
  \href{https://www.nytimes3xbfgragh.onion/privacy/cookie-policy\#how-do-i-manage-trackers}{Your
  Ad Choices}
\item
  \href{https://www.nytimes3xbfgragh.onion/privacy}{Privacy}
\item
  \href{https://help.nytimes3xbfgragh.onion/hc/en-us/articles/115014893428-Terms-of-service}{Terms
  of Service}
\item
  \href{https://help.nytimes3xbfgragh.onion/hc/en-us/articles/115014893968-Terms-of-sale}{Terms
  of Sale}
\item
  \href{https://spiderbites.nytimes3xbfgragh.onion}{Site Map}
\item
  \href{https://help.nytimes3xbfgragh.onion/hc/en-us}{Help}
\item
  \href{https://www.nytimes3xbfgragh.onion/subscription?campaignId=37WXW}{Subscriptions}
\end{itemize}
