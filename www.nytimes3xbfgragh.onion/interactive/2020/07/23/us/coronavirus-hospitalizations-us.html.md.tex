Sections

SEARCH

\protect\hyperlink{site-content}{Skip to
content}\protect\hyperlink{site-index}{Skip to site index}

\href{https://www.nytimes3xbfgragh.onion/section/us}{U.S.}

\href{https://myaccount.nytimes3xbfgragh.onion/auth/login?response_type=cookie\&client_id=vi}{}

\href{https://www.nytimes3xbfgragh.onion/section/todayspaper}{Today's
Paper}

\href{/section/us}{U.S.}\textbar{}Where the Virus Is Sending People to
Hospitals

\url{https://nyti.ms/2ZPsjHF}

\begin{itemize}
\item
\item
\item
\item
\item
\item
\end{itemize}

\hypertarget{the-coronavirus-outbreak}{%
\subsubsection{\texorpdfstring{\href{https://www.nytimes3xbfgragh.onion/news-event/coronavirus?name=styln-coronavirus-national\&region=TOP_BANNER\&block=storyline_menu_recirc\&action=click\&pgtype=Interactive\&impression_id=7a017310-f1c8-11ea-8758-972576ee5dee\&variant=undefined}{The
Coronavirus
Outbreak}}{The Coronavirus Outbreak}}\label{the-coronavirus-outbreak}}

\begin{itemize}
\tightlist
\item
  live\href{https://www.nytimes3xbfgragh.onion/2020/09/08/world/covid-19-coronavirus.html?name=styln-coronavirus-national\&region=TOP_BANNER\&block=storyline_menu_recirc\&action=click\&pgtype=Interactive\&impression_id=7a017311-f1c8-11ea-8758-972576ee5dee\&variant=undefined}{Latest
  Updates}
\item
  \href{https://www.nytimes3xbfgragh.onion/interactive/2020/us/coronavirus-us-cases.html?name=styln-coronavirus-national\&region=TOP_BANNER\&block=storyline_menu_recirc\&action=click\&pgtype=Interactive\&impression_id=7a017312-f1c8-11ea-8758-972576ee5dee\&variant=undefined}{Maps
  and Cases}
\item
  \href{https://www.nytimes3xbfgragh.onion/interactive/2020/science/coronavirus-vaccine-tracker.html?name=styln-coronavirus-national\&region=TOP_BANNER\&block=storyline_menu_recirc\&action=click\&pgtype=Interactive\&impression_id=7a017313-f1c8-11ea-8758-972576ee5dee\&variant=undefined}{Vaccine
  Tracker}
\item
  \href{https://www.nytimes3xbfgragh.onion/2020/09/02/your-money/eviction-moratorium-covid.html?name=styln-coronavirus-national\&region=TOP_BANNER\&block=storyline_menu_recirc\&action=click\&pgtype=Interactive\&impression_id=7a019a20-f1c8-11ea-8758-972576ee5dee\&variant=undefined}{Eviction
  Moratorium}
\item
  \href{https://www.nytimes3xbfgragh.onion/interactive/2020/09/02/magazine/food-insecurity-hunger-us.html?name=styln-coronavirus-national\&region=TOP_BANNER\&block=storyline_menu_recirc\&action=click\&pgtype=Interactive\&impression_id=7a019a21-f1c8-11ea-8758-972576ee5dee\&variant=undefined}{American
  Hunger}
\end{itemize}

Advertisement

\protect\hyperlink{after-top}{Continue reading the main story}

\hypertarget{comments}{%
\subsection{\texorpdfstring{\protect\hyperlink{commentsContainer}{Comments}}{Comments}}\label{comments}}

\href{}{Where the Virus Is Sending People to Hospitals}\href{}{Skip to
Comments}

The comments section is closed. To submit a letter to the editor for
publication, write to
\href{mailto:letters@NYTimes.com}{\nolinkurl{letters@NYTimes.com}}.

\hypertarget{where-the-virus-is-sending-people-to-hospitals}{%
\section{Where the Virus Is Sending People to
Hospitals}\label{where-the-virus-is-sending-people-to-hospitals}}

By \href{https://www.nytimes3xbfgragh.onion/by/lazaro-gamio}{Lazaro
Gamio}, \href{https://www.nytimes3xbfgragh.onion/by/sarah-mervosh}{Sarah
Mervosh} and
\href{https://www.nytimes3xbfgragh.onion/by/keith-collins}{Keith
Collins}July 23, 2020

\begin{itemize}
\item
\item
\item
\item
\item
  \emph{161}
\end{itemize}

\hypertarget{covid-19-patients-per-100000}{%
\subsubsection{Covid-19 patients per
100,000}\label{covid-19-patients-per-100000}}

At overflowing hospitals in South Texas, patients wait hours in
sweltering ambulances and on recliner beds set up in hallways. The
number of patients intubated in hospital beds in Tampa, Fla., is growing
by the day. In Corpus Christi, Texas, a mobile morgue has arrived.

About as many people are now
\href{https://covidtracking.com/data/charts/us-currently-hospitalized}{known
to be hospitalized} with the coronavirus in the United States as during
any other time in the pandemic, matching the previous peak in April.

Public health experts say detailed local data on where people are
hospitalized --- a real-time measure that does not depend on levels of
testing --- is crucial to understanding the epidemic, but federal
officials have not made this data public. The New York Times gathered
data for nearly 50 metropolitan areas, including 15 of the 20 largest
cities in the country, from state and local health departments to
provide the first detailed national look at where people are falling
seriously ill.

The data, as well as interviews across the country, show a far-reaching
crisis. The worst-hit areas in Texas and Florida have approached the
peak rates of hospitalization that New York, New Orleans, Chicago and
other cities hit in the spring. A wide and growing expanse of hot spots
around the country --- including Las Vegas, Nashville and Tulsa, Okla.
--- have worsened over the past two weeks.

Not every hospital system is overwhelmed, and new treatments have
improved the chances of survival for seriously ill people. But experts
say a small but significant proportion of those currently hospitalized
will die, and those who survive may face serious long-term health
issues.

Months ago, the endless wail of ambulances in New York City conveyed the
urgency of the virus outbreak in a concentrated area. Now, the scale of
the crisis is dispersed and harder to grasp.

``There's this pandemic fatigue,'' said Thomas Tsai, an assistant
professor of health policy at Harvard University. ``All eyes were on New
York. Houston is New York now. Miami is New York now. Phoenix is New
York now. We need that shared collective urgency.''

No place comes close to matching New York City's sheer numbers: At its
hospitalization peak in mid-April, more than 12,000 New Yorkers were
hospitalized at one time.

Some places today look more like New Orleans, an early epicenter that at
one point had about 1,000 hospitalized coronavirus patients.

This region on the southern border of Texas most likely has the worst
rate of hospitalization in the country.
\href{https://www.nytimes3xbfgragh.onion/2020/07/19/us/coronavirus-texas-rio-grande-valley.html}{Hospitals
are full}. Moving from bed to bed, medical workers wrapped in protective
layers yell over blaring alarms. Nurses softly soothe dying patients.
There is little time to grieve. A new patient fills an emptied bed.

At least 87 people have died from the coronavirus in the past three
weeks; as recently as the end of June, only eight people had. ``The
movie that you never wanted to be living in --- that's what it's like,''
said Annette Rodriguez, the county's public health director.

``Your hospitals are drowning,'' Carlos Migoya, the chief executive of
Jackson Health System, the largest public hospital in Miami,
\href{https://www.miamiherald.com/opinion/op-ed/article244194817.html}{wrote
in an op-ed in The Miami Herald}. ``We are teetering on the edge of
disaster.''

Of the top 10 places we found with the most severe coronavirus
hospitalization rates, six were in Texas.

Imperial County, home to many food-processing workers and farmworkers on
the California-Mexico border, became the first county in California to
revert to a stay-at-home order this month as cases soared with positive
test rates four times the state average.

The number of coronavirus patients in Chicago is now less than a third
of what it was at its peak.

Several areas in Florida, including the Fort Lauderdale region, are
under nightly curfews to slow the spread of the virus.

The number of newly hospitalized patients in the Houston area is down
slightly in recent days, but the county is still home to the most
coronavirus patients in Texas. More than 2,200 people with the virus are
in hospitals.

The number of patients who need to be intubated at Tampa General
Hospital keeps growing. ``We've gone from single digits to double
digits, basically every day,'' said Dr. Andrew Myers, an internist at
the hospital.

``Things are worse than they've ever been,'' said Nelson Wolff, the
county judge for San Antonio's Bexar County, which had more Covid-19
patients die in 10 days this month than during the first three months of
the pandemic. ``I had two friends die within a week.''

President Trump is expected to formally accept his party's nomination at
the Republican National Convention in Jacksonville next month. Of the
top 20 metro regions we found with the most severe hospitalization
rates, seven were in Florida.

Arizona is one of several states with hot spots that did not provide
local data, making it difficult to know the precise hospital situation
in Phoenix, a center of the crisis. Statewide, Covid-19 hospitalizations
have leveled off in recent days, offering hope that measures like mask
mandates and the closure of gyms, bars and nightclubs are working.

Inside hospitals in Orlando's Orange County this week, the number of
Covid-19 patients was the highest it has been during the pandemic.
Outside, Disney World had reopened, and SeaWorld was splashing again. It
is one of the jarring contrasts in a crisis that is now spread widely
and sometimes hard to see.

At the beginning of the crisis, New Orleans was at the center of
Louisiana's coronavirus outbreak. Now, New Orleans is doing relatively
well, while cities like Lafayette are seeing sharp increases.

``We are really worried,'' said Sara Kalaoram, whose mother, a guest
room attendant at a Las Vegas hotel, is hospitalized with the virus and
on oxygen. Ms. Kalaoram, her father and her teenage brother have also
tested positive.

Covid-19 patients make up 33 percent of I.C.U. beds in Mississippi, up
from 18 percent in June.

Los Angeles County has surpassed its previous peak of coronavirus
patients from earlier this year, though fewer patients are in intensive
care.

``Every metric is heading in the wrong direction,'' said Dr. George
Monks, president of the Oklahoma State Medical Association.
Hospitalizations have been on the rise across Oklahoma, but the
situation is most pressing around Oklahoma City.

The Tulsa City Council adopted a mask order last week, in part due to
rising hospitalizations.

The number of coronavirus patients in Chicago is now less than a third
of what it was at its peak, though recent upticks led Mayor Lori
Lightfoot to close bars for indoor use once again.

``We are hoping we don't see a huge spike again,'' said Gregory Burrell,
the funeral director at Terry Funeral Home, Inc., a funeral home in
Philadelphia that handled dozens of funerals for people who died of the
virus in April and May. So far this month, he said he had handled about
five such funerals.

New admissions for the coronavirus in New York City have dropped to a
few dozen a day, down from more than 1,000 people a day in late March
and early April. Statewide, hospitalizations from the virus are at their
lowest since March 18.

Connecticut, like New York, has seen significant improvements since the
spring, but there are still seriously ill people. Statewide, 62 people
were hospitalized with the virus as of this week, down from a high of
nearly 2,000 in April.

\hypertarget{where-people-are-sick-from-the-coronavirus}{%
\subsubsection{Where people are sick from the
coronavirus}\label{where-people-are-sick-from-the-coronavirus}}

\hypertarget{rates-per-100000-people}{%
\paragraph{Rates per 100,000 people}\label{rates-per-100000-people}}

Major cities within each county or area are shown. Hospitalization and
intensive care figures are census counts of the number of Covid-19
patients at any one time, not hospital admissions or cumulative
hospitalizations. They include patients that have confirmed Covid-19
illness based on lab tests as well as probable or suspected patients who
exhibited certain symptoms. Not all states provide information on
probable patients. Hospitalization figures are from July 21, or the most
recent day that data was available.

Sources: State and local health departments;
\href{https://carlsonschool.umn.edu/mili-misrc-covid19-tracking-project}{Covid-19
Hospitalization Tracking Project} at the University of Minnesota's
Carlson School of Management; \href{https://covidtracking.com/}{The
Covid Tracking Project};
\href{https://public.tableau.com/profile/jason.salemi\#!/vizhome/COVID-19inFloridaanepidemiologiststake/StoryWeb}{Covid-19
in Florida}, Jason L. Salemi, University of South Florida; Natalie E.
Dean, University of Florida; Megan Murray, Harvard University; Jonathan
Skinner, Dartmouth College; Joe Gerald, The University of Arizona.

Caitlin Dickerson, Michael Gold, Kimiko de Freytas-Tamura, J. David
Goodman, Manny Fernandez, Shawn Hubler, Patricia Mazzei and Jin Wu
contributed reporting.

Read 161 Comments

\begin{itemize}
\item
\item
\item
\item
\end{itemize}

Advertisement

\protect\hyperlink{after-bottom}{Continue reading the main story}

\hypertarget{site-index}{%
\subsection{Site Index}\label{site-index}}

\hypertarget{site-information-navigation}{%
\subsection{Site Information
Navigation}\label{site-information-navigation}}

\begin{itemize}
\tightlist
\item
  \href{https://help.nytimes3xbfgragh.onion/hc/en-us/articles/115014792127-Copyright-notice}{©~2020~The
  New York Times Company}
\end{itemize}

\begin{itemize}
\tightlist
\item
  \href{https://www.nytco.com/}{NYTCo}
\item
  \href{https://help.nytimes3xbfgragh.onion/hc/en-us/articles/115015385887-Contact-Us}{Contact
  Us}
\item
  \href{https://www.nytco.com/careers/}{Work with us}
\item
  \href{https://nytmediakit.com/}{Advertise}
\item
  \href{http://www.tbrandstudio.com/}{T Brand Studio}
\item
  \href{https://www.nytimes3xbfgragh.onion/privacy/cookie-policy\#how-do-i-manage-trackers}{Your
  Ad Choices}
\item
  \href{https://www.nytimes3xbfgragh.onion/privacy}{Privacy}
\item
  \href{https://help.nytimes3xbfgragh.onion/hc/en-us/articles/115014893428-Terms-of-service}{Terms
  of Service}
\item
  \href{https://help.nytimes3xbfgragh.onion/hc/en-us/articles/115014893968-Terms-of-sale}{Terms
  of Sale}
\item
  \href{https://spiderbites.nytimes3xbfgragh.onion}{Site Map}
\item
  \href{https://help.nytimes3xbfgragh.onion/hc/en-us}{Help}
\item
  \href{https://www.nytimes3xbfgragh.onion/subscription?campaignId=37WXW}{Subscriptions}
\end{itemize}
