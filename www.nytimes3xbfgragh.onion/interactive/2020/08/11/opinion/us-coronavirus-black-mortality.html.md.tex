Sections

SEARCH

\protect\hyperlink{site-content}{Skip to
content}\protect\hyperlink{site-index}{Skip to site index}

\hypertarget{comments}{%
\subsection{\texorpdfstring{\protect\hyperlink{commentsContainer}{Comments}}{Comments}}\label{comments}}

\href{}{Racism's Hidden Toll}\href{}{Skip to Comments}

The comments section is closed. To submit a letter to the editor for
publication, write to
\href{mailto:letters@NYTimes.com}{\nolinkurl{letters@NYTimes.com}}.

\href{/section/opinion}{Opinion}

\hypertarget{racisms-hidden-toll}{%
\section{Racism's Hidden Toll}\label{racisms-hidden-toll}}

By Gus WezerekAug. 11, 2020

\begin{itemize}
\item
\item
\item
\item
\item
  \emph{274}
\end{itemize}

``Hands,'' 1996, by Glenn Ligon.Glenn Ligon, via Hauser \& Wirth, Regen
Projects, Thomas Dane Gallery and Chantel Crousel; Brian Forrest

In 2018, white people in the United States died at an age-adjusted rate
of 89 per 10,000 people.

The pandemic is likely to cause the biggest jump in white mortality
since World War II.Many Americans will lose a parent or grandparent.
Some will lose a brother, sister or child.

If Black people were immune to the coronavirus, their mortality rate in
2020 would still probably surpass white people's.

The gap in death rates has existed for as long as we have had data.

As hospital beds filled up this spring, health departments in cities
like Milwaukee and Charlotte, N.C., began to report an alarming trend: A
disproportionate number of their patients were Black.

Data eventually revealed that the pattern was nationwide. Black people
were three times more likely than white people to contract the
coronavirus, six times more likely to be hospitalized as a result and
twice as likely to die of Covid-19.

The gap in Black and white infections has become part of a conversation
this year about how deeply racism is embedded in the day-to-day lives of
Black people.

``An epidemic shows in a short period of time what's been going on for
hundreds of years,'' said David Ansell, who directs community health
equity at Rush University Medical Center in Chicago.

What's been going on for hundreds of years is the systematic neglect of
Black Americans' health. In 2018, Black people died at higher
age-adjusted rates than white people from nine of the top 15 causes of
death.

\hypertarget{gaps-between-black-and-white-mortality-rates-for-the-top-15-causes-of-death}{%
\subsubsection{Gaps between Black and white mortality rates for the top
15 causes of
death}\label{gaps-between-black-and-white-mortality-rates-for-the-top-15-causes-of-death}}

+15 white

deaths per 100k

0

+15

+30 Black

deaths

Higher Black death rate

Higher Black death rate

Heart disease

Heart disease

Seventh-largest

cause of all deaths

149 per 100k

Seventh-largest

cause of all deaths

21 per 100k

Diabetes

Diabetes

Second-largest

149 per 100k

Cancer

Cancer

Second-largest

164 per 100k

Homicide

Homicide

Stroke

Stroke

Kidney disease

Kidney disease

Hypertension

Hypertension

Liver disease

Liver disease

Septicemia

Septicemia

Flu and

pneumonia

Flu and

pneumonia

Parkinson's

Parkinson's

Alzheimer's

Alzheimer's

Accidents

Accidents

Suicide

Suicide

Lung

disease

Lung

disease

Higher white

death rate

Higher white

death rate

+10 white deaths

per 100k

0

+10

+20

+30

+40 Black

deaths

Higher Black death rate

Higher Black death rate

Heart disease

Heart disease

Seventh-largest

cause of all deaths

21 per 100k

Seventh-largest

cause of all deaths

149 per 100k

Diabetes

Diabetes

Black people are more likely

to die by homicide.

Black people are more likely

to die by homicide.

Second-largest

149 per 100k

Second-largest

164 per 100k

Cancer

Cancer

But homicides are so rare that the difference barely contributes to the
overall mortality gap.

But homicides are so rare that the difference barely contributes to the
overall mortality gap.

Homicide

Homicide

Stroke

Stroke

Kidney disease

Kidney disease

Hypertension

Hypertension

Liver disease

Septicemia

Septicemia

Flu and

pneumonia

Flu and

pneumonia

Parkinson's

Parkinson's

Alzheimer's

Alzheimer's

Accidents

Accidents

Suicide

Suicide

Lung disease

Lung disease

Higher white death rate

Higher white death rate

Note: For non-Hispanic Black and white people in 2018. Rates have been
adjusted for age and sex. Source: Centers for Disease Control and
Prevention

``Health inequities have literally killed thousands and thousands of
Black people who would not have died if they had the death rates of
white people,'' said Dorothy Roberts, a professor of Africana studies,
law and sociology at the University of Pennsylvania.

If Black people had died at the same age-adjusted rate as white people
in 2018, they would have avoided 65,000 premature, excess deaths --- the
equivalent of three coach buses filled with Black people crashing and
killing them all every day of the year.

Note: Estimates based on age- and sex-specific mortality rates for ages
0-79.·Source: Mary R. Jackman and Kimberlee A. Shauman,
``\href{https://www.cambridge.org/core/services/aop-cambridge-core/content/view/9F0994F128271F21B04749C1E8D905C1/S1742058X20000028a.pdf/toll_of_inequality.pdf}{The
Toll of Inequality: Excess African American Deaths in the United States
over the Twentieth Century}.''

Between 1900 and 2015, Black excess deaths totalled 8.8 million.

In recent decades, the excess deaths have been concentrated among Black
people in their prime earning years, depriving families of income and
stability.

Perceptions of health in America depend on who you are. A 2011 survey
found that
\href{https://www.healthaffairs.org/doi/full/10.1377/hlthaff.2010.0702}{only
55 percent} of white people knew about inequalities in Black and white
health, compared with 89 percent of Black people.

When confronted with disparities in Black health, Americans have been
slow to acknowledge that those inequalities are ``not a Negro affair,
but an index of social condition,'' as W.E.B. Du Bois, a pioneering
sociologist, wrote in 1906.

``Usually what people will say is, `Oh,
\href{https://www.healthaffairs.org/do/10.1377/hblog20200630.939347/full/}{clearly
it's genetics}, clearly it's socioeconomics, clearly it's individual
behavior,''' said Jay Pearson, an assistant professor of public policy
at Duke University.

``Well, it's \href{https://thenewpress.com/books/fatal-invention}{not
genetics at all},'' Dr. Pearson said. As for income, education and
behavior, ``those explain away some of the difference but not all of it.
What we're really talking about is
\href{https://pdfs.semanticscholar.org/14f1/6214e22d273fd9f9d8490a7c0c4171eebb41.pdf}{structural
racism}.''

Tracing the origins of Black health disparities, you can go all the way
back to the slave traders'
\href{http://slaveryimages.org/s/slaveryimages/item/1937}{barracoons}.
Historians estimate that at least half of the Africans who were captured
and brought to America died before they could be sold as slaves.

But the modern history of the segregation that is keeping the Black
death rate separate and unequal begins during Du Bois's era, at the turn
of the 20th century.

More than a million Black people migrated from the South to Northern
cities throughout the early decades of the 20th century. Short on money
and job opportunities, new arrivals were forced to rent third-rate
housing
\href{https://uncpress.org/book/9780807859346/infectious-fear/}{that
lacked clean water and sewer lines}.

The unsanitary living conditions caused Black city dwellers to die from
tuberculosis and the flu at about
\href{https://link.springer.com/article/10.1007/s13524-019-00789-z?shared-article-renderer}{twice
the rates} of white people between 1910 and 1935.

Stories today of
\href{https://www.washingtonpost.com/local/a-woman-her-brother-and-their-mother-all-lost-to-covid-19-the-viruss-devastating-toll-on-one-dc-area-family/2020/05/02/f0c3403a-8bef-11ea-9dfd-990f9dcc71fc_story.html}{entire}\href{https://www.washingtonpost.com/local/a-woman-her-brother-and-their-mother-all-lost-to-covid-19-the-viruss-devastating-toll-on-one-dc-area-family/2020/05/02/f0c3403a-8bef-11ea-9dfd-990f9dcc71fc_story.html}{familie}\href{https://www.washingtonpost.com/local/a-woman-her-brother-and-their-mother-all-lost-to-covid-19-the-viruss-devastating-toll-on-one-dc-area-family/2020/05/02/f0c3403a-8bef-11ea-9dfd-990f9dcc71fc_story.html}{s}
dying of Covid-19 correlate with
\href{https://uncpress.org/book/9780807859346/infectious-fear/}{similar
accounts} of Black ``house infection'' a hundred years before.

``Tuberculosis is the family skeleton, the ever haunting dread,''
Charles Frederick Weller wrote in 1909 about the Black families who
lived in the alleys behind rowhouses in Washington, D.C.

So bleak were Black peoples' death rates that Frederick Hoffman, a
statistician at the Prudential Life Insurance Company, published a
329-page report in 1896 arguing that if trends held, the ``gradual
extinction'' of Black people was ``only a question of time.''

\includegraphics{https://static01.graylady3jvrrxbe.onion/packages/flash/multimedia/ICONS/transparent.png}

\includegraphics{https://static01.graylady3jvrrxbe.onion/newsgraphics/2020/07/28/mortality/5475b961028b0d480766ebd7bb7329ff66885cad/backyards.jpg}

Urban tenements and alley houses often lacked ventilation and waste
disposal. Runoff from shared privies would stream down streets and into
basement apartments.From ``Housing Conditions in Baltimore,'' by Janet
Kemp, 1907

The state of Black health was cause for much hand-wringing by white
writers and medical professionals. Often, their concerns were founded in
a fear that the Black maids and cooks they employed would bring disease
into their homes.

``The fact is not pleasant to contemplate, but is nevertheless true,
that there are colored persons afflicted with gonorrhea, syphilis, and
tuberculosis employed as servants in many of the best homes in the South
today,'' a physician said in a speech in 1914 about ``the Negro health
problem'' to the American Public Health Association.

In pamphlets and editorials, white people repeatedly linked Black people
to disease and danger. ``Lung block'' maps, which overlaid city grids
with dots representing tuberculosis cases, were used by city health
departments to establish authority over Black neighborhoods and justify
the surveillance of residents and demolition of buildings.

\includegraphics{https://static01.graylady3jvrrxbe.onion/packages/flash/multimedia/ICONS/transparent.png}

\includegraphics{https://static01.graylady3jvrrxbe.onion/newsgraphics/2020/07/28/mortality/5475b961028b0d480766ebd7bb7329ff66885cad/map_inset.jpg}

Baltimore's Lower Druid Hill neighborhood on a tuberculosis map created
by the city's health department in 1902. Blue dots represent Black
deaths; red dots represent white deaths.The Sheridan Libraries, Johns
Hopkins University

Caricatures of Black people as physically and morally degenerate
contributed to white people's fear of integration. In large Northern
cities in the 1920s, for every Black person who arrived in a
neighborhood, three white people left.

Black people, meanwhile, were often prevented from leaving increasingly
overcrowded neighborhoods. A
\href{https://digitalcommons.law.umaryland.edu/cgi/viewcontent.cgi?article=2498\&context=mlr}{1911
law in Baltimore} was the first of many across the country to make it
illegal for Black people to relocate to blocks where more than half the
residents were white.

The Baltimore journalist H.L. Mencken described how the ordinance kept
Black people sick, writing that when a Black person ``tries to move out
of his sty and into human habitation a policeman now stops him. The law
practically insists that he keep on incubating typhoid and
tuberculosis.''

In cities where residential segregation wasn't perpetuated by the
government, white people found other ways to keep Black people at a
distance. In Chicago, a Black home was bombed every month between 1917
and 1921. White property owners also used tactics like restrictive
covenants, which prevented future deed holders from selling neighborhood
homes to Black people.

\includegraphics{https://static01.graylady3jvrrxbe.onion/packages/flash/multimedia/ICONS/transparent.png}

\includegraphics{https://static01.graylady3jvrrxbe.onion/newsgraphics/2020/07/28/mortality/5475b961028b0d480766ebd7bb7329ff66885cad/test_2.jpg}

From ``Neglected Neighbors,'' by Charles Frederick Weller and Eugenia
Winston Weller, photograph by Lewis Hine, 1909, via Widener Library,
Harvard University

Inextricable from the story of Black health in America, residential
segregation may have also been behind one of its more surprising
footnotes.

At the peak of the 1918 pandemic, Black people were less likely to catch
the flu than white people. Historians have speculated that a milder
version of the disease may have spread throughout Black neighborhoods in
the spring and given residents partial immunity to the more deadly flu
strain that arrived in the fall.

Nevertheless, the white death rate during 1918 was still lower than
Black mortality had ever been. The same is
\href{https://osf.io/9csa6/}{likely to be true} for the white death rate
during the current pandemic.

On the opening track of her recent visual album, "Black Is King,"
Beyoncé sings, "Life is your birthright / they hid that in the fine
print / take the pen and rewrite it."

Black people in the early 20th century embodied the spirit of that
charge, defying the predictions of people like Mr. Hoffman and
surviving.

Barred from white-only hospitals and medical schools, Black people
started their own. They fought for sanitation systems and organized
health education weeks. Antibiotics curbed tuberculosis infections,
though the drugs arrived in Black communities years after white patients
got them.

\includegraphics{https://static01.graylady3jvrrxbe.onion/packages/flash/multimedia/ICONS/transparent.png}

\includegraphics{https://static01.graylady3jvrrxbe.onion/newsgraphics/2020/07/28/mortality/5475b961028b0d480766ebd7bb7329ff66885cad/p1.jpg}

\includegraphics{https://static01.graylady3jvrrxbe.onion/packages/flash/multimedia/ICONS/transparent.png}

\includegraphics{https://static01.graylady3jvrrxbe.onion/newsgraphics/2020/07/28/mortality/5475b961028b0d480766ebd7bb7329ff66885cad/p2.jpg}

A nurse and a medical intern at Provident Hospital in Chicago, which was
founded in 1890 by Daniel Hale Williams, a renowned Black surgeon.Jack
Delano, Library of Congress

The Black-white death gap has narrowed to an all-time low in recent
years, thanks to legislation like the Social Security Act, the Civil
Rights Act and the Affordable Care Act.

Residential segregation and wealth inequality persist, however, and have
forced Black people to relive health inequities that their ancestors
faced a century ago. Today, Black people continue to:

\textbf{Work jobs that limit their ability to quarantine.} Black people
account for 12 percent of workers overall, but
\href{https://www.epi.org/publication/black-workers-covid/}{17 percent}
of front-line workers.

\textbf{Live in overcrowded housing.} Black renters are
\href{https://www.gao.gov/assets/710/707179.pdf}{twice as likely} as
white renters to live in a household with more than two people per
bedroom.

\textbf{Live closer to environmental hazards.} Black people are exposed
to almost
\href{https://ajph-aphapublications-org.ezp-prod1.hul.harvard.edu/doi/pdf/10.2105/AJPH.2017.304297}{twice
as much} air pollution as white people.

\textbf{Have limited access to health care.} Black people are
\href{https://www.census.gov/content/dam/Census/library/publications/2019/demo/p60-267.pdf}{twice
as likely} to be uninsured as white people.

\textbf{Interact with a largely white medical establishment.} Black
people make up 13 percent of the country's population but only
\href{https://www.aamc.org/data-reports/workforce/interactive-data/figure-18-percentage-all-active-physicians-race/ethnicity-2018}{5
percent} of physicians.

Marching alongside Black protesters this year, white people finally
started to hear what Black people have been saying for years:
``\href{https://www.nytimes3xbfgragh.onion/2015/05/10/magazine/our-demand-is-simple-stop-killing-us.html}{Stop
killing us}.''

It remains to be seen whether white people will continue to address the
racial prejudice inherent in America's law enforcement, politics and
health care. When the pandemic ends and the media moves on to the next
story, it will be tempting to remember this year as a tragic ``episode''
in America's history.

But if nothing changes, Black men will still lose their lives to AIDS at
six times the rate of white men. Black women will still be three times
as likely to die from childbirth. And Black infants will still be
expected to live three fewer years than white babies.

``We have to tell the truth. This country isn't fair and it never has
been,'' Dr. Pearson at Duke said. ``And unless we make major, wholesale
changes, it won't be in the future.''

Top image: Glenn Ligon, ``Hands,'' 1996, silkscreen ink and gesso on
unstretched canvas, 82 x 144 inches; Copyright Glenn Ligon; via the
artist, Hauser \& Wirth, New York, Regen Projects, Los Angeles, Thomas
Dane Gallery, London, and Chantal Crousel, Paris; Brian Forrest

Gus Wezerek is a graphics editor and writer in the Opinion section.

Read 274 Comments

\begin{itemize}
\item
\item
\item
\item
\end{itemize}

Advertisement

\protect\hyperlink{after-bottom}{Continue reading the main story}

\hypertarget{site-index}{%
\subsection{Site Index}\label{site-index}}

\hypertarget{site-information-navigation}{%
\subsection{Site Information
Navigation}\label{site-information-navigation}}

\begin{itemize}
\tightlist
\item
  \href{https://help.nytimes3xbfgragh.onion/hc/en-us/articles/115014792127-Copyright-notice}{©~2020~The
  New York Times Company}
\end{itemize}

\begin{itemize}
\tightlist
\item
  \href{https://www.nytco.com/}{NYTCo}
\item
  \href{https://help.nytimes3xbfgragh.onion/hc/en-us/articles/115015385887-Contact-Us}{Contact
  Us}
\item
  \href{https://www.nytco.com/careers/}{Work with us}
\item
  \href{https://nytmediakit.com/}{Advertise}
\item
  \href{http://www.tbrandstudio.com/}{T Brand Studio}
\item
  \href{https://www.nytimes3xbfgragh.onion/privacy/cookie-policy\#how-do-i-manage-trackers}{Your
  Ad Choices}
\item
  \href{https://www.nytimes3xbfgragh.onion/privacy}{Privacy}
\item
  \href{https://help.nytimes3xbfgragh.onion/hc/en-us/articles/115014893428-Terms-of-service}{Terms
  of Service}
\item
  \href{https://help.nytimes3xbfgragh.onion/hc/en-us/articles/115014893968-Terms-of-sale}{Terms
  of Sale}
\item
  \href{https://spiderbites.nytimes3xbfgragh.onion}{Site Map}
\item
  \href{https://help.nytimes3xbfgragh.onion/hc/en-us}{Help}
\item
  \href{https://www.nytimes3xbfgragh.onion/subscription?campaignId=37WXW}{Subscriptions}
\end{itemize}
