Sections

SEARCH

\protect\hyperlink{site-content}{Skip to
content}\protect\hyperlink{site-index}{Skip to site index}

\href{https://www.nytimes3xbfgragh.onion/section/climate}{Climate}

\href{https://myaccount.nytimes3xbfgragh.onion/auth/login?response_type=cookie\&client_id=vi}{}

\href{https://www.nytimes3xbfgragh.onion/section/todayspaper}{Today's
Paper}

\href{/section/climate}{Climate}\textbar{}How bad is climate change now?

\url{https://nyti.ms/2KhfywV}

\begin{itemize}
\item
\item
\item
\item
\item
\item
\end{itemize}

\hypertarget{climate-and-environment}{%
\subsubsection{\texorpdfstring{\href{https://www.nytimes3xbfgragh.onion/section/climate?name=styln-climate\&region=TOP_BANNER\&block=storyline_menu_recirc\&action=click\&pgtype=Interactive\&impression_id=5a6bf990-f2cb-11ea-a43f-b1079632598a\&variant=undefined}{Climate
and
Environment}}{Climate and Environment}}\label{climate-and-environment}}

\begin{itemize}
\tightlist
\item
  \href{https://www.nytimes3xbfgragh.onion/interactive/2020/08/24/climate/racism-redlining-cities-global-warming.html?name=styln-climate\&region=TOP_BANNER\&block=storyline_menu_recirc\&action=click\&pgtype=Interactive\&impression_id=5a6bf991-f2cb-11ea-a43f-b1079632598a\&variant=undefined}{Environmental
  Racism}
\item
  \href{https://www.nytimes3xbfgragh.onion/interactive/2020/climate/trump-environment-rollbacks.html?name=styln-climate\&region=TOP_BANNER\&block=storyline_menu_recirc\&action=click\&pgtype=Interactive\&impression_id=5a6c20a0-f2cb-11ea-a43f-b1079632598a\&variant=undefined}{Trump's
  Changes}
\item
  \href{https://www.nytimes3xbfgragh.onion/interactive/2020/04/19/climate/climate-crash-course-1.html?name=styln-climate\&region=TOP_BANNER\&block=storyline_menu_recirc\&action=click\&pgtype=Interactive\&impression_id=5a6c20a1-f2cb-11ea-a43f-b1079632598a\&variant=undefined}{Climate
  101}
\item
  \href{https://www.nytimes3xbfgragh.onion/interactive/2018/08/30/climate/how-much-hotter-is-your-hometown.html?name=styln-climate\&region=TOP_BANNER\&block=storyline_menu_recirc\&action=click\&pgtype=Interactive\&impression_id=5a6c20a2-f2cb-11ea-a43f-b1079632598a\&variant=undefined}{Is
  Your Hometown Hotter?}
\end{itemize}

\hypertarget{comments}{%
\subsection{\texorpdfstring{\protect\hyperlink{commentsContainer}{Comments}}{Comments}}\label{comments}}

\href{}{How bad is climate change now?}\href{}{Skip to Comments}

The comments section is closed. To submit a letter to the editor for
publication, write to
\href{mailto:letters@NYTimes.com}{\nolinkurl{letters@NYTimes.com}}.

\hypertarget{how-bad-is-climate-change-now}{%
\section{How bad is climate change
now?}\label{how-bad-is-climate-change-now}}

By \href{https://www.nytimes3xbfgragh.onion/by/henry-fountain}{Henry
Fountain}April 20, 2020

\begin{itemize}
\item
\item
\item
\item
\item
  \emph{57}
\end{itemize}

From the Times's Climate Desk's series that addresses big climate
questions: ``The only real debates are over how fast and how far the
climate will change, and what society should do.''

\hypertarget{a-crash-course-on-climate-change-50-years-after-the-first-earth-day}{%
\subsubsection{A crash course on climate change, 50 years after the
first Earth
Day}\label{a-crash-course-on-climate-change-50-years-after-the-first-earth-day}}

The science is clear: The world is warming dangerously, humans are the
cause of it, and a failure to act today will deeply affect the future of
the Earth.

This is a seven-day New York Times crash course on climate change, in
which reporters from the Times's Climate desk address the big questions:

\href{/interactive/2020/04/19/climate/climate-crash-course-1.html}{**1.**How
bad is climate change now?}
\href{/interactive/2020/04/19/climate/climate-crash-course-2.html}{**2.**How
do scientists know what they know?}
\href{/interactive/2020/04/19/climate/climate-crash-course-3.html}{**3.**Who
is influencing key decisions?}
\href{/interactive/2020/04/19/climate/climate-crash-course-4.html}{**4.**How
do we stop fossil fuel emissions?}
\href{/interactive/2020/04/19/climate/climate-crash-course-5.html}{**5.**Do
environmental rules matter?}
\href{/interactive/2020/04/19/climate/climate-crash-course-6.html}{**6.**Can
insurance protect us?}
\href{/interactive/2020/04/19/climate/climate-crash-course-7.html}{**7.**Is
what I do important?}

\includegraphics{https://static01.graylady3jvrrxbe.onion/newsgraphics/2020/04/07/climate-crash-course/c741d02a3e3bf8e0c8cab1a34d2aee8858e0ca3a/1.jpg}

\hypertarget{1-how-bad-is-climate-change-now}{%
\section{\texorpdfstring{\textbf{1.} How bad is climate change
now?}{1. How bad is climate change now?}}\label{1-how-bad-is-climate-change-now}}

By \href{https://www.nytimes3xbfgragh.onion/by/henry-fountain}{Henry
Fountain}, who has been a Times science writer for 20 years and has
traveled to both the Arctic and Antarctica.

Amid the horror and uncertainty of
\href{https://www.nytimes3xbfgragh.onion/interactive/2020/world/coronavirus-maps.html}{a
global health crisis} it can be easy to forget that another worldwide
disaster is unfolding, although much more slowly.

Global warming is happening, and its effects are being felt around the
world. The only real debates are over how fast and how far the climate
will change, and what society should do --- the global-warming
equivalents of lockdowns and social distancing --- to slow or stop it
and limit the damage.

As of now, the damage seems to be getting worse.
\href{https://www.nytimes3xbfgragh.onion/2019/12/04/climate/climate-change-acceleration.html}{As
I wrote in December,} impacts that scientists predicted years ago ---
including severe storms, heat waves and the melting of glaciers and ice
sheets --- are accelerating.

The coronavirus pandemic can seem overwhelming because of its sheer
scope; so can climate change. As a science writer at The Times for more
than 20 years, I've learned that, to avoid being overwhelmed, it helps
to start by understanding one part of the larger problem.

So let's take a closer look at one piece: what's happening at the top of
the world, the Arctic. It's a good place to understand the science of
climate change, and, it turns out, a critically important one to
understand its effects.

Since the mid-1990s, the Arctic has been warming faster than any other
region of the planet: currently, at least two and a half times as fast.
(Last year, average air temperatures were about 3.5 degrees Fahrenheit,
or 1.9 degrees Celsius, higher than the average from 1981-2010.)

In large part, the Arctic is warming the way the rest of the world
warms, only up north the process has run amok.

As the concentration of carbon dioxide and other greenhouse gases
increase in the atmosphere, so does the amount of heat they trap. But
the source of that heat is sunlight striking the Earth, and the amount
of heat radiated differs depending on the surface the sunlight hits.
Just as a black car gets much hotter than a white car on a sunny day,
darker parts of the planet absorb more sunlight, and in turn radiate
more heat, than lighter parts.

The Central Arctic is all ocean --- dark water that is covered, to a
varying extent, by light ice. The ice absorbs only about 30 to 40
percent of the sunlight hitting it; the rest is reflected. Ocean, on the
other hand, absorbs more than 90 percent.

As the Arctic warms more of the ice disappears, leaving more dark ocean
to absorb more sunlight and radiate even more heat, causing even more
loss of ice. It's a vicious cycle that contributes to rapid warming in
the region.

Is this happening at the South Pole as well? No, because while the
Arctic is mostly water surrounded by land, Antarctica is the opposite, a
huge land mass surrounded by ocean. Some of the ice that covers the
continent is melting, but no dark ocean is being exposed. (That's not to
say that the continent isn't losing ice: it is, mostly through
\href{https://www.nytimes3xbfgragh.onion/2019/10/01/climate/antarctica-iceberg-d28.html}{calving
of icebergs} and melting of the undersides of ice shelves.)

In the Arctic, currents and winds flow out of the region and affect
weather elsewhere.

Weakening of the high-altitude winds known as the polar jet stream can
bring extra-frigid winter weather to North America and Europe. Cold
snaps like these have occurred for a long time although, because of
global warming, studies have found that they are not as cold as they
used to be. But some scientists now say they think Arctic warming is
causing the jet stream to wobble in ways that lead to more extreme
weather year round, by creating zones of high-pressure air that can
cause weather systems --- the ones that bring extreme heat, for example
--- to stall.

Arctic warming may also be affecting climate over the longer term. As
Greenland's ice sheet melts, the fresh water it releases lowers the
saltiness of the nearby ocean. These salinity changes may eventually
have an effect on some of the large ocean currents that help determine
long-term climate trends in parts of the world.

As climate researchers are fond of saying, what happens in the Arctic
doesn't stay in the Arctic.

\hypertarget{a-crash-course-on-climate-change-50-years-after-the-first-earth-day-1}{%
\paragraph{A crash course on climate change, 50 years after the first
Earth
Day}\label{a-crash-course-on-climate-change-50-years-after-the-first-earth-day-1}}

\href{/interactive/2020/04/19/climate/climate-crash-course-1.html}{**1.**How
bad is climate change now?}
\href{/interactive/2020/04/19/climate/climate-crash-course-2.html}{**2.**How
do scientists know what they know?}
\href{/interactive/2020/04/19/climate/climate-crash-course-3.html}{**3.**Who
is influencing key decisions?}
\href{/interactive/2020/04/19/climate/climate-crash-course-4.html}{**4.**How
do we stop fossil fuel emissions?}
\href{/interactive/2020/04/19/climate/climate-crash-course-5.html}{**5.**Do
environmental rules matter?}
\href{/interactive/2020/04/19/climate/climate-crash-course-6.html}{**6.**Can
insurance protect us?}
\href{/interactive/2020/04/19/climate/climate-crash-course-7.html}{**7.**Is
what I do important?}

\includegraphics{https://static01.graylady3jvrrxbe.onion/newsgraphics/2020/04/07/climate-crash-course/c741d02a3e3bf8e0c8cab1a34d2aee8858e0ca3a/2.jpg}

\hypertarget{2-how-do-scientists-know-what-they-know}{%
\subsection{\texorpdfstring{\textbf{2.} How do scientists know what they
know?}{2. How do scientists know what they know?}}\label{2-how-do-scientists-know-what-they-know}}

By
\href{https://www.nytimes3xbfgragh.onion/by/kendra-pierre-louis}{Kendra
Pierre-Louis}, who writes about climate change and its social and
ecological consequences.

When it comes to climate, there's a lot that we know.
\href{https://www.nytimes3xbfgragh.onion/2020/01/08/climate/2019-temperatures.html}{The
second warmest year on record was 2019}, and it closed out the hottest
recorded decade.
\href{https://www.nytimes3xbfgragh.onion/2020/01/13/climate/ocean-temperatures-climate-change.html}{Ocean
temperatures are rising}, too, hitting a high in 2019 as well, and
\href{https://www.nytimes3xbfgragh.onion/2019/01/10/climate/ocean-warming-climate-change.html}{increasing
faster} than previously estimated.

The changes over just the last few decades are stark, making plain that
the planet's climate is warming and that it's human activity behind the
temperature rise. But scientists can also look back even further to
figure out temperatures on Earth before any humans were alive.

Understanding how scientists figure out what's going on with the climate
is an interesting part of being a climate reporter. My favorite piece of
equipment is arguably a bathythermograph, essentially an open water
thermometer, simply because it's a fun word to say. Instruments like it,
together with the GPS-connected devices in the global Argo floats
network, are how researchers monitor ocean temperatures.

For annual temperature reports, scientists rely on a historical
temperature record ---
\href{https://www.nytimes3xbfgragh.onion/2018/05/09/climate/white-roofs.html}{someone
or some machine taking daily temperatures}. This is how we know, for
example, that 2019 was hotter than 1942. But the temperature record only
stretches back to the 1800s for much of the world, and has some gaps. To
cover them, and to look back even further, researchers rely on proxy, or
indirect, measures.

In much the same way that data on the daily consumption of chicken wings
can help us
\href{https://www.nationalchickencouncil.org/americans-to-eat-more-than-1-3-billion-chicken-wings-for-super-bowl/}{suss
out the dates of Super Bowl Sundays}, things like ice core samples, tree
rings, corals, pollen and cave deposits can help us understand how the
climate behaved in the past, said Jacquelyn Gill, a paleoecologist and
associate professor at the University of Maine.

``I like to think of it as environmental forensics,'' Dr. Gill said.
``Rather than directly observe the past, we use some of the same tools
that forensic scientists use to reconstruct the environment through
time.''

For example, some tree species can live for thousands of years. When cut
into, their rings, which resemble a bull's-eye on a tree stump, can clue
researchers into not only past temperatures but also moisture levels
from year to year.

``We're not just guessing about how trees record climate in their rings
because we have a century or more of actual measurements that we can
then compare to tree rings,'' Dr. Gill said.

In northern regions like the Arctic, researchers rely on another life
form: tiny non-biting midges that spend years living in lakes as larvae
before turning into winged insects. As they grow they shed parts of
their exoskeletons, which are well preserved in lake sediments. If
sediment samples transition from layers that contain species that prefer
cooler temperatures into layers with species that prefer warmer ones,
it's a signal that temperatures increased.

Using multiple records means scientists can validate their findings, Dr.
Gill said. With tree rings, lake sediments and ice cores from the same
region, you can ``look across those different proxies and see where you
have good agreement and where you don't.''

But to measure the levels of human caused climate emissions, researchers
have other tools.

Since 1958, an observatory near the top of the Mauna Loa volcano in
Hawaii has been recording the amount of carbon dioxide in the air and,
more recently, observatories in Alaska, Samoa and the South Pole have
also been recording measurements. Data is also collected from eight tall
towers located across the United States, small aircraft, and volunteers
at some 50 locations worldwide. Because carbon dioxide that comes from
burning oil and coal is slightly different than the carbon that comes
from living animals and plants, researchers know burning fossil fuels is
behind the increase.

If you're noticing a lot of redundancy in how researchers make sense of
the climate, that's the point. They aren't using a single piece of data,
but lots of pieces to stitch together a comprehensive picture that
points in a single direction: the climate is warming and humans are
causing it.

\hypertarget{a-crash-course-on-climate-change-50-years-after-the-first-earth-day-2}{%
\paragraph{A crash course on climate change, 50 years after the first
Earth
Day}\label{a-crash-course-on-climate-change-50-years-after-the-first-earth-day-2}}

\href{/interactive/2020/04/19/climate/climate-crash-course-1.html}{**1.**How
bad is climate change now?}
\href{/interactive/2020/04/19/climate/climate-crash-course-2.html}{**2.**How
do scientists know what they know?}
\href{/interactive/2020/04/19/climate/climate-crash-course-3.html}{**3.**Who
is influencing key decisions?}
\href{/interactive/2020/04/19/climate/climate-crash-course-4.html}{**4.**How
do we stop fossil fuel emissions?}
\href{/interactive/2020/04/19/climate/climate-crash-course-5.html}{**5.**Do
environmental rules matter?}
\href{/interactive/2020/04/19/climate/climate-crash-course-6.html}{**6.**Can
insurance protect us?}
\href{/interactive/2020/04/19/climate/climate-crash-course-7.html}{**7.**Is
what I do important?}

\includegraphics{https://static01.graylady3jvrrxbe.onion/newsgraphics/2020/04/07/climate-crash-course/c741d02a3e3bf8e0c8cab1a34d2aee8858e0ca3a/3.jpg}

\hypertarget{3-who-is-influencing-key-decisions}{%
\subsection{\texorpdfstring{\textbf{3.} Who is influencing key
decisions?}{3. Who is influencing key decisions?}}\label{3-who-is-influencing-key-decisions}}

By \href{https://www.nytimes3xbfgragh.onion/by/hiroko-tabuchi}{Hiroko
Tabuchi}, an investigative reporter on the climate desk who focuses on
the fossil fuel industry.

When an administration, Republican or Democratic, proposes a change to a
federal rule, it can look like a cut-and-dried affair.

But behind the scenes, rule-making involves extensive lobbying. My job
as a journalist looking at the intersection of climate and industry has
been to follow the money trail to figure out who's asking for what, and
who's getting what they want.

That often involves scrutinizing the powerful fossil fuels industry,
which for years has lobbied against policies to tackle global warming,
and funded efforts to obscure the well-established science that global
warming is caused primarily by greenhouse gases generated by burning
fossil fuels and other human activities. These efforts are often
obscured from public view, but their influence becomes clear in
regulatory and lobbying records and by piecing together information from
insiders and other sources willing to talk to us.

The industry has gotten results. Since taking office, President Trump
has begun
\href{https://www.nytimes3xbfgragh.onion/2019/10/23/climate/trump-paris-climate-accord.html}{withdrawing
the United States from the landmark Paris climate accord}, signed five
years ago by almost 200 countries to help reduce global emissions. At
the urging of coal companies like Peabody Energy, the president
\href{https://www.nytimes3xbfgragh.onion/2017/10/09/climate/clean-power-plan.html}{halted
the Obama administration's Clean Power Plan}, designed to rein in
emissions from coal-fired power plants. (That hasn't halted the decline
of the coal industry, now on even more precarious footing as the
Covid-19 outbreak
\href{https://www.nytimes3xbfgragh.onion/2020/04/07/business/energy-environment/coronavirus-oil-wind-solar-energy.html}{triggers
a slump in coal use}.)

A powerful oil and gas group also backed weaker oversight for emissions
of methane, an invisible, particularly potent greenhouse gas; my video
colleague Jonah Kessel and I
\href{https://www.nytimes3xbfgragh.onion/interactive/2019/12/12/climate/texas-methane-super-emitters.html}{made
some of the gas leaks visible last year with the help of infrared
technology}.

Led by Marathon Petroleum, the country's largest refiner, a separate
group representing fuel and petrochemical
manufacturers\href{https://www.nytimes3xbfgragh.onion/2018/12/13/climate/cafe-emissions-rollback-oil-industry.html}{ran
a stealth campaign to roll back car tailpipe emissions standards}, the
biggest climate initiative ever adopted by the United States. The
rollback has gone so far that it has alarmed even some of the carmakers
the measure was supposed to help.

According to the nonpartisan \href{https://www.opensecrets.org/}{Center
for Responsive Politics}, the oil and gas industry spent more than \$125
million in lobbying at the federal level in 2019 alone. The coal mining
industry spent close to an additional \$7 million on lobbying. And
together, fossil fuel companies have already made at least \$50 million
in political contributions this year, the vast majority to Republican
politicians.

In recent years,
\href{https://www.nytimes3xbfgragh.onion/2019/09/20/climate/global-climate-strike.html}{as
climate activism has gathered steam}, oil and gas companies have made
commitments to help combat climate change. As world leaders gathered at
the United Nations climate summit last fall to discuss the urgency of
slashing carbon emissions, for example, 13 of the world's biggest fossil
fuel companies
\href{https://www.nytimes3xbfgragh.onion/2019/09/23/climate/oil-industry-climate-investment.html}{announced
a set of wide-ranging pledges}, from supporting a carbon tax, promising
to cut down on methane leaks and investing in technology to scrub carbon
dioxide from the air.

But there are concerns those efforts could fall by the wayside, as the
oil and gas industry, reeling from the global pandemic, reins in
spending. As the coronavirus has spread, industry groups have lobbied,
successfully, for drastic rollbacks of environmental rules governing
power plants and other industrial facilities.
\href{https://www.nytimes3xbfgragh.onion/2020/03/26/climate/epa-coronavirus-pollution-rules.html}{The
Environmental Protection Agency has said it will temporarily halt fines}
for violations of certain air, water and hazardous waste reporting
requirements.

As the historians
\href{https://www.nytimes3xbfgragh.onion/2015/06/16/science/naomi-oreskes-a-lightning-rod-in-a-changing-climate.html}{Naomi
Oreskes} and Erik Conway argue in their seminal book, ``Merchants of
Doubt,'' the methods used by industry to deny the harms of fossil fuel
use were in many cases the same as those used by the tobacco industry to
deny the harms of cigarettes.

At least in the United States, the tobacco industry is in a long
decline. It remains to be seen whether the fossil fuel industry will
tread a similar path.

\hypertarget{a-crash-course-on-climate-change-50-years-after-the-first-earth-day-3}{%
\paragraph{A crash course on climate change, 50 years after the first
Earth
Day}\label{a-crash-course-on-climate-change-50-years-after-the-first-earth-day-3}}

\href{/interactive/2020/04/19/climate/climate-crash-course-1.html}{**1.**How
bad is climate change now?}
\href{/interactive/2020/04/19/climate/climate-crash-course-2.html}{**2.**How
do scientists know what they know?}
\href{/interactive/2020/04/19/climate/climate-crash-course-3.html}{**3.**Who
is influencing key decisions?}
\href{/interactive/2020/04/19/climate/climate-crash-course-4.html}{**4.**How
do we stop fossil fuel emissions?}
\href{/interactive/2020/04/19/climate/climate-crash-course-5.html}{**5.**Do
environmental rules matter?}
\href{/interactive/2020/04/19/climate/climate-crash-course-6.html}{**6.**Can
insurance protect us?}
\href{/interactive/2020/04/19/climate/climate-crash-course-7.html}{**7.**Is
what I do important?}

\includegraphics{https://static01.graylady3jvrrxbe.onion/newsgraphics/2020/04/07/climate-crash-course/c741d02a3e3bf8e0c8cab1a34d2aee8858e0ca3a/4.jpg}

\hypertarget{4-how-do-we-stop-fossil-fuel-emissions}{%
\subsection{\texorpdfstring{\textbf{4.} How do we stop fossil fuel
emissions?}{4. How do we stop fossil fuel emissions?}}\label{4-how-do-we-stop-fossil-fuel-emissions}}

By \href{https://www.nytimes3xbfgragh.onion/by/brad-plumer}{Brad
Plumer}, a climate reporter specializing in policy and technology
efforts to cut carbon dioxide emissions.

To stop global warming, we'll need to zero out greenhouse gas emissions
from billions of different sources worldwide: every coal plant in China,
every steel mill in Europe, every car and truck on American highways.

It's such an enormous task that it can be tough to figure out where to
begin.

As a reporter covering climate policy, I've spoken to hundreds of
experts and read through
\href{http://deepdecarbonization.org/countries/}{countless}
\href{https://www.iea.org/reports/world-energy-model/sustainable-development-scenario}{dense}
\href{https://www.post2020hlp.org/wp-content/uploads/docs/Rockstroem-Sachs-Oehman-Schmidt-Traub_Sustainable-Development-and-Planetary-Boundaries.pdf}{reports}
about how countries can slash their emissions. There's often fierce
debate over the best path forward. But I've found it helpful to think
about all the different proposals out there as essentially boiling down
to four broad steps. Consider this a rough game plan for how the world
might solve climate change.

\hypertarget{clean-up-electric-power-plants}{%
\subparagraph{Clean up electric power
plants}\label{clean-up-electric-power-plants}}

Today,
\href{https://www.wri.org/blog/2020/02/greenhouse-gas-emissions-by-country-sector}{roughly
one-quarter} of humanity's emissions come from power plants that
generate the electricity we use for our lights, air-conditioners and
factories. Most power plants still burn coal, natural gas or oil,
producing carbon dioxide that heats the planet.

The good news is there are lots of available technologies that can
produce electricity without emissions. France cleaned up its grid with
nuclear power. California is aiming for zero-emissions electricity by
2045 by installing solar panels and wind turbines. Some companies plan
to capture carbon dioxide from existing coal plants and
\href{https://www.nytimes3xbfgragh.onion/2020/02/11/climate/carbon-capture-tax.html}{bury
it underground}.

Experts often disagree on which technologies are best, and technical
hurdles remain
\href{http://news.mit.edu/2018/adding-power-choices-reduces-cost-risk-carbon-free-electricity-0906}{in
cutting emissions all the way to zero};~better batteries to juggle wind
and solar power would help. But there's broad agreement that we could
greatly reduce power-plant emissions with the tools we have today.

\hypertarget{electrify-much-of-our-economy}{%
\subparagraph{\texorpdfstring{\textbf{Electrify much of our
economy}}{Electrify much of our economy}}\label{electrify-much-of-our-economy}}

As our power plants get greener, the next step is to rejigger big chunks
of our economy to run on clean electricity instead of burning fossil
fuels.

For example, we can replace cars that run on gasoline with electric
vehicles charged by low-carbon grids. We can replace gas-burning
furnaces with
\href{https://www.nytimes3xbfgragh.onion/2019/05/01/opinion/climate-change-gas-electricity.html}{electric
heat pumps}. Instead of steel mills that burn coal, shift to electric
furnaces that melt scrap. Roughly another one-quarter of global
emissions could conceivably be electrified in this fashion.

This daunting task of
\href{https://www.vox.com/2016/9/19/12938086/electrify-everything}{``electrifying
everything''} becomes easier if we're also curbing our energy use at the
same time. That could entail~making cities less dependent on cars,
upgrading home insulation and boosting energy-efficiency in factories.

\hypertarget{develop-new-technology-for-the-hard-to-electrify-bits}{%
\subparagraph{\texorpdfstring{\textbf{Develop new technology for the
hard-to-electrify
bits}}{Develop new technology for the hard-to-electrify bits}}\label{develop-new-technology-for-the-hard-to-electrify-bits}}

Parts of the modern economy, alas, can't easily be electrified.
Batteries are still too heavy for most airplanes or long-haul trucks.
Many key industries, like cement or glass, require
\href{https://energypolicy.columbia.edu/research/report/low-carbon-heat-solutions-heavy-industry-sources-options-and-costs-today}{e}\href{https://energypolicy.columbia.edu/research/report/low-carbon-heat-solutions-heavy-industry-sources-options-and-costs-today}{xtreme
heat} and currently burn coal or gas.

One
\href{https://science.sciencemag.org/content/360/6396/eaas9793}{recent
study concluded} that about one-quarter of emissions fall into this
``difficult to decarbonize'' category.

Governments and businesses will need to invest in new technologies. Some
possibilities: power airplanes with sustainable biofuels from crop
waste; use green hydrogen, created from renewable energy, to produce
industrial heat; or
\href{https://www.nytimes3xbfgragh.onion/2019/04/07/business/energy-environment/climate-change-carbon-engineering.html}{suck
carbon dioxide out of the air} to offset the emissions we can't
eliminate. We'll have to get creative.

\hypertarget{fix-farming}{%
\subparagraph{\texorpdfstring{\textbf{Fix
farming}}{Fix farming}}\label{fix-farming}}

A final one-fourth of global emissions comes from agriculture and
deforestation; think cows belching up methane or farmers clearing swaths
of the Amazon for cropland. Figuring out how to feed billions while
using less land and producing fewer emissions
\href{https://www.nytimes3xbfgragh.onion/2018/12/05/climate/agriculture-food-global-warming.html}{will
take an array of solutions}, from improving ranching practices to
reducing food waste, but it's crucial.

This list is simplified, of course, and figuring out how to actually
achieve these four steps is the hard part. A
\href{https://www.nytimes3xbfgragh.onion/2014/05/30/science/a-price-tag-on-carbon-as-a-climate-rescue-plan.html}{tax
on carbon emissions} could give businesses incentive to find fixes.
Governments could ramp up spending on clean technologies. International
cooperation and policies to help dislocated workers are vital. And
powerful industry interests who prefer the status quo will fight major
changes.

But it's a basic road map if we want to zero out emissions, which,
scientists agree, is what is ultimately needed to keep the world from
heating up endlessly.

\hypertarget{a-crash-course-on-climate-change-50-years-after-the-first-earth-day-4}{%
\paragraph{A crash course on climate change, 50 years after the first
Earth
Day}\label{a-crash-course-on-climate-change-50-years-after-the-first-earth-day-4}}

\href{/interactive/2020/04/19/climate/climate-crash-course-1.html}{**1.**How
bad is climate change now?}
\href{/interactive/2020/04/19/climate/climate-crash-course-2.html}{**2.**How
do scientists know what they know?}
\href{/interactive/2020/04/19/climate/climate-crash-course-3.html}{**3.**Who
is influencing key decisions?}
\href{/interactive/2020/04/19/climate/climate-crash-course-4.html}{**4.**How
do we stop fossil fuel emissions?}
\href{/interactive/2020/04/19/climate/climate-crash-course-5.html}{**5.**Do
environmental rules matter?}
\href{/interactive/2020/04/19/climate/climate-crash-course-6.html}{**6.**Can
insurance protect us?}
\href{/interactive/2020/04/19/climate/climate-crash-course-7.html}{**7.**Is
what I do important?}

\includegraphics{https://static01.graylady3jvrrxbe.onion/newsgraphics/2020/04/07/climate-crash-course/c741d02a3e3bf8e0c8cab1a34d2aee8858e0ca3a/5.jpg}

\hypertarget{5-do-environmental-rules-matter}{%
\subsection{\texorpdfstring{\textbf{5.} Do environmental rules
matter?}{5. Do environmental rules matter?}}\label{5-do-environmental-rules-matter}}

By \href{https://www.nytimes3xbfgragh.onion/by/lisa-friedman}{Lisa
Friedman}, who reports on federal climate and environmental policy.

As a reporter in Washington for more than 20 years, I've had a front-row
seat to the gridlock that has gripped Congress on climate change.

By 2009, partisanship over the issue was already deeply entrenched. The
House, then controlled by Democrats, passed a landmark bill that year
that would have created a market-based system to cap greenhouse gas
emissions. It died in the Senate. In 2010, amid a Tea Party wave that
swept the G.O.P. back into power and many of the House Republicans who
voted for the legislation either retired or were voted out of office.

In the words of one ousted Republican, it felt like
\href{https://www.pbs.org/wgbh/frontline/article/bob-inglis-climate-change-and-the-republican-party/}{even
acknowledging climate change was ``heresy.''}

That ushered in the era of climate policy by executive order.

Over the next several years, President Barack Obama's administration
enacted a series of regulations cutting emissions from
\href{https://www.nytimes3xbfgragh.onion/2009/01/26/world/americas/26iht-26calif.19665165.html?searchResultPosition=6}{automobiles},
\href{https://www.nytimes3xbfgragh.onion/2015/01/14/us/politics/obama-administration-to-unveil-plans-to-cut-methane-emissions.html?searchResultPosition=25}{oil
and gas wells} and
\href{https://www.nytimes3xbfgragh.onion/2009/10/01/science/earth/01epa.html?searchResultPosition=18}{power
plants}. He
\href{https://www.nytimes3xbfgragh.onion/2016/12/20/us/obama-drilling-ban-arctic-atlantic.html}{banned
offshore drilling} in parts of the Atlantic and the Arctic oceans,
established
\href{https://www.nytimes3xbfgragh.onion/2016/12/28/us/politics/obama-national-monument-bears-ears-utah-gold-butte.html}{national
monuments} across 1.7 million acres of federal land and linked
\href{https://www.nytimes3xbfgragh.onion/2015/05/21/us/obama-recasts-climate-change-as-a-more-far-reaching-peril.html}{climate
change to national security} policy.

In 2015, after covering more than seven years of negotiations toward a
global agreement many thought would never come, I pushed my way into a
crowded tent on the outskirts of Paris to watch world leaders ink
\href{https://www.nytimes3xbfgragh.onion/2015/12/13/world/europe/climate-change-accord-paris.html}{a
historic accord} that was fundamentally shaped by the Obama
administration.

``If Congress won't act, I will,'' Mr. Obama had
\href{https://twitter.com/BarackObama/status/366932015439810560?s=20}{declared}.
Unlike laws, however, regulations are highly vulnerable to political
winds. And back in Washington, the House and Senate, then
Republican-controlled, were fighting many of the Obama administration's
plans.

A few years later, voters elected President Trump. As a candidate Mr.
Trump mocked climate change, and as president he quickly made good on
promises to eliminate his predecessor's
\href{https://www.nytimes3xbfgragh.onion/2017/03/28/climate/trump-executive-order-climate-change.html}{``job-killing''
regulations}, increase fossil fuel production and withdraw from the
Paris Agreement. So far, the Trump administration has moved to eliminate
\href{https://www.nytimes3xbfgragh.onion/interactive/2019/climate/trump-environment-rollbacks.html}{nearly
100 environmental rules}.

It's too soon to tell what the impact of the rollbacks will be on the
climate. In 2017 the World Resources Institute
\href{https://www.wri.org/blog/2017/12/insider-climate-effect-trump-administration}{estimated}
that if all Mr. Trump's policies were enacted, emissions in the United
States by 2025 would range from the equivalent of 5.6 to 6.8 gigatons
--- compared with a range of about 5.0 to 6.6 gigatons if Mr. Obama's
regulations had remained in place. A single gigaton is about the annual
emissions of Italy, France and the United Kingdom combined.

Former Vice President Joseph R. Biden, the presumptive Democratic
presidential nominee, has pledged to use the ``full authority of the
executive branch'' to cut emissions and move the United States to clean
energy by 2050.

His \$1.7 trillion plan includes several major executive actions
including ``aggressive'' methane pollution limits; cutting
transportation emissions; enacting new efficiency standards for
buildings and appliances; and halting new oil and gas permits on public
lands and waters. Mr. Biden has not embraced a nationwide ban on
fracking, for which he has been heavily
\href{https://www.nytimes3xbfgragh.onion/2019/10/09/climate/climate-change-biden.html}{criticized
by climate activists}.

Congress, though, remains stuck. Republicans have embraced some plans
like
\href{https://www.nytimes3xbfgragh.onion/2020/02/12/climate/trump-trees-climate-change.html}{planting
trees} and technology to capture carbon dioxide emissions, but
agreements on broad solutions remain elusive.

Even Republicans who have opposed efforts to contain climate change
acknowledge that Congress ultimately holds the key.

In a recent House hearing, Interior Secretary David Bernhardt noted
that, among more than 600 laws mandating the agency ``shall'' do things,
none orders it to respond to climate change.

``You know what, there's not a shall for `I shall manage the land to
stop climate change,' or something similar to that,'' Mr. Bernhardt told
lawmakers. ``You guys come up with the shalls.''

\hypertarget{a-crash-course-on-climate-change-50-years-after-the-first-earth-day-5}{%
\paragraph{A crash course on climate change, 50 years after the first
Earth
Day}\label{a-crash-course-on-climate-change-50-years-after-the-first-earth-day-5}}

\href{/interactive/2020/04/19/climate/climate-crash-course-1.html}{**1.**How
bad is climate change now?}
\href{/interactive/2020/04/19/climate/climate-crash-course-2.html}{**2.**How
do scientists know what they know?}
\href{/interactive/2020/04/19/climate/climate-crash-course-3.html}{**3.**Who
is influencing key decisions?}
\href{/interactive/2020/04/19/climate/climate-crash-course-4.html}{**4.**How
do we stop fossil fuel emissions?}
\href{/interactive/2020/04/19/climate/climate-crash-course-5.html}{**5.**Do
environmental rules matter?}
\href{/interactive/2020/04/19/climate/climate-crash-course-6.html}{**6.**Can
insurance protect us?}
\href{/interactive/2020/04/19/climate/climate-crash-course-7.html}{**7.**Is
what I do important?}

\includegraphics{https://static01.graylady3jvrrxbe.onion/newsgraphics/2020/04/07/climate-crash-course/c741d02a3e3bf8e0c8cab1a34d2aee8858e0ca3a/6.jpg}

\hypertarget{6-can-insurance-protect-us}{%
\subsection{\texorpdfstring{\textbf{6.} Can insurance protect
us?}{6. Can insurance protect us?}}\label{6-can-insurance-protect-us}}

By
\href{https://www.nytimes3xbfgragh.onion/by/christopher-flavelle}{Christopher
Flavelle}, who focuses on efforts to cope with global warming's effects.

So you just achieved your dream of becoming a homeowner.
Congratulations! But climate change has added a new caveat to
homeownership: Whether it's near the water or the woods, in a city or
farther out, your home may be increasingly vulnerable to hurricanes,
flooding or wildfire.

At least you can always buy insurance, right? About that: There's good
news and
\href{https://www.nytimes3xbfgragh.onion/2019/08/20/climate/fire-insurance-renewal.html}{bad
news}. But mostly it's
\href{https://www.nytimes3xbfgragh.onion/2019/08/20/climate/fire-insurance-renewal.html}{bad}.

While most of the climate debate is focused on how to curb greenhouse
gas emissions, there's another fight going on over a seemingly simple
question: As climate change increases the risk to American homeowners,
should governments allow the cost of insurance to keep pace with that
risk?

This is where regulators, lawmakers and budget officials start to
cringe. During my years of reporting on global warming, I've watched the
question of insurance become one of the most intractable policy dilemmas
facing governments and homeowners --- and one with no obvious solution.

The obvious approach might be to let insurance work the way it's meant
to, with premiums that reflect the odds of getting hit by a disaster.
That would let insurance companies --- or, in the case of flood
insurance, the federal government --- collect enough money to pay out
claims. Higher premiums are also a warning to homeowners to avoid living
in risky areas.

But homeowners vote. Last year, the Trump administration proposed
changing the deeply indebted federal flood insurance program in a way
that would make premiums
\href{https://www.bloomberg.com/news/articles/2019-03-12/insurance-rates-seen-rising-in-flood-prone-areas-with-trump-plan}{reflect
actual risk}. Members of Congress from both parties expressed alarm and
the administration
\href{https://www.fema.gov/news-release/2019/11/07/fema-defers-implementation-risk-rating-20}{backed
down}, delaying the change until after this year's election --- if it
\href{https://www.politico.com/news/2019/11/07/fema-postpones-flood-insurance-rate-revamp-amid-backlash-067505}{happens
at all}.

In California, which was hit by huge wildfires in recent years,
regulators and lawmakers have made it
\href{https://www.bloomberg.com/news/articles/2018-08-15/why-two-years-of-historic-wildfires-haven-t-made-southern-california-safer}{harder}
for insurers to pass costs onto consumers and barred insurance companies
from canceling coverage for homeowners in or alongside ZIP codes hit by
fires.

The instinct to keep rates low reflects more than just political
self-preservation. If costs go up too much, whole neighborhoods could
\href{https://www.bloomberg.com/news/articles/2018-01-04/california-says-wildfires-are-making-home-insurance-unaffordable}{become
unaffordable} --- ruining home values, collapsing the local economy and
shattering the tax base.

That leaves a second option: As risks increase, governments can keep
subsidizing insurance either directly, through publicly funded programs
like flood insurance, or indirectly, by forcing private insurers to
spread the burden of high-risk coverage by raising prices elsewhere.
Both approaches seek to
\href{https://www.bloomberg.com/news/features/2018-03-01/why-is-california-rebuilding-in-fire-country-because-you-re-paying-for-it}{shield
people from the cost of their decisions}.

That, dear homeowner, is the good news: At this point in the climate
debate, officials have generally erred on the side of protecting at-risk
homeowners, financially
\href{https://www.nytimes3xbfgragh.onion/2019/10/26/climate/building-codes-secret-deal.html}{if
not physically}. A beach house or mountain home may put you in harm's
way, but at least you should be able to afford your insurance premiums
for a few more years.

But by keeping premiums low, governments encourage more homes to
\href{https://www.nytimes3xbfgragh.onion/2019/07/31/climate/climate-change-new-homes-flooding.html}{go
up in risky areas}, which means more homeowners exposed to storms or
fires. Call it the sympathy paradox: Actions intended to help people
today by making it easier for them to stay in their homes risk hurting
more people tomorrow.

This dilemma will only become harder to navigate. Growing risks will
make governments even more reluctant to expose voters to the true cost
of insurance. But voters far from flood zones will increasingly resent
\href{https://www.nytimes3xbfgragh.onion/2019/09/27/climate/mortgage-climate-risk.html}{footing
the bill for
risky}\href{https://www.nytimes3xbfgragh.onion/2019/09/27/climate/mortgage-climate-risk.html}{homes}.

What does this mean for you? For now, maybe nothing: Congress continues
to have little appetite for large increases to flood insurance costs,
and most state regulators will resist insurers' demands for big rate
hikes. And if they change their minds, armies of homeowners, home
builders, real estate agents and local officials are likely to push
back.

But the cost of the current approach
\href{https://www.nytimes3xbfgragh.onion/2019/12/04/climate/florida-keys-climate-change.html}{keeps
growing} with
\href{https://www.nytimes3xbfgragh.onion/2020/03/11/climate/government-land-eviction-floods.html}{every
disaster}. If you want to follow a truly searing debate about climate
change in the United States, watch this space.

\hypertarget{a-crash-course-on-climate-change-50-years-after-the-first-earth-day-6}{%
\paragraph{A crash course on climate change, 50 years after the first
Earth
Day}\label{a-crash-course-on-climate-change-50-years-after-the-first-earth-day-6}}

\href{/interactive/2020/04/19/climate/climate-crash-course-1.html}{**1.**How
bad is climate change now?}
\href{/interactive/2020/04/19/climate/climate-crash-course-2.html}{**2.**How
do scientists know what they know?}
\href{/interactive/2020/04/19/climate/climate-crash-course-3.html}{**3.**Who
is influencing key decisions?}
\href{/interactive/2020/04/19/climate/climate-crash-course-4.html}{**4.**How
do we stop fossil fuel emissions?}
\href{/interactive/2020/04/19/climate/climate-crash-course-5.html}{**5.**Do
environmental rules matter?}
\href{/interactive/2020/04/19/climate/climate-crash-course-6.html}{**6.**Can
insurance protect us?}
\href{/interactive/2020/04/19/climate/climate-crash-course-7.html}{**7.**Is
what I do important?}

\includegraphics{https://static01.graylady3jvrrxbe.onion/newsgraphics/2020/04/07/climate-crash-course/c741d02a3e3bf8e0c8cab1a34d2aee8858e0ca3a/7.jpg}

\hypertarget{7-is-what-i-do-important}{%
\subsection{\texorpdfstring{\textbf{7.} Is what I do
important?}{7. Is what I do important?}}\label{7-is-what-i-do-important}}

By \href{https://www.nytimes3xbfgragh.onion/by/somini-sengupta}{Somini
Sengupta}, an international climate correspondent who has reported on
warming around the world.

This is one of the most common and most vexing questions in the age of
climate change: Can I address a problem so big, or can the world solve
this only when powerful leaders in business and government make big
structural changes?

It's impossible to separate the two. Personal actions and international
cooperation are inextricably linked.

First, the answer depends on whose actions we're talking about. Those of
a middle-class American matter a lot more than the actions of say, a
farmer in Bangladesh. Why? Because we consume much more, and so our
choices matter much more to global emissions: Per capita emissions in
the United States are 30 times bigger than per capita emissions in
Bangladesh.

Many of my consumption choices have large implications. What car I buy,
or whether I buy one at all, matters hugely, because transportation is
the single
\href{https://www.nytimes3xbfgragh.onion/interactive/2019/10/10/climate/driving-emissions-map.html}{biggest
source of emissions} in most American cities. Same with how much I fly.
Most lipsticks I impulse-buy contain palm oil, the production of which
is linked to deforestation in Southeast Asia.

And what I eat has an
\href{https://www.nytimes3xbfgragh.onion/interactive/2019/04/30/dining/climate-change-food-eating-habits.html}{enormous
climate footprint}. The average person in North America eats more than
six times the recommended amount of red meat,
\href{https://www.nytimes3xbfgragh.onion/2019/01/16/climate/meat-environment-climate-change.html}{a
report published last year} found, while the average person in South
Asia eats half of what's recommended. Perhaps most important is what I
don't eat and toss into the garbage. From farm to plate, food waste
accounts for nearly 10 percent of global greenhouse gas emissions.

Is there one fix we can make to avert a climate catastrophe? No. It is
inevitable we will have to change much about how we live, for our own
survival and the survival of others we don't know. It's a bit like what
we're doing to stop the
\href{https://www.nytimes3xbfgragh.onion/2020/03/12/climate/climate-change-coronavirus-lessons.html}{coronavirus
pandemic}, except forever.

Second, individual behavior
\href{https://www.nature.com/articles/s41560-019-0541-9}{can influence
others}. One house with solar panels can lead to others in the
neighborhood
\href{https://pubsonline.informs.org/doi/10.1287/mksc.1120.0727}{installing
solar panels of their own}. Likewise, we tend to conserve our
electricity consumption when our utility bills tell us how our usage
compares with our neighbors.

Third, individual action is a prerequisite for collective action.
Without young individual activists, there would be no Sunrise Movement
to camp out in the halls of Congress, nor would
\href{https://www.nytimes3xbfgragh.onion/2019/09/20/climate/global-climate-strike.html}{millions
of children fill the streets} of major world capitals, demanding that
the adults in charge take swift climate action.

On the whole, though, humans tend to be really bad at changing their
behavior today to address risks tomorrow. This ``present bias,'' as
cognitive scientists call it, makes it hard for us, as individuals, to
make lifestyle changes now to prevent a catastrophe down the road. So we
need government policies to protect us from future risks.

Because the world has deferred climate action for so long, scientists
estimate global emissions must be cut by half in the next 10 years in
order to avoid the most catastrophic effects of global warming.

It's hard to imagine how such sharp emissions cuts can be made without
ambitious government policies, including carbon prices that make it
sufficiently costly to burn coal or oil, investments in public
transportation, and enforceable energy efficiency standards.

And this is where the Paris Agreement comes in. Every country is
supposed to set their own climate targets and figure out how to meet
them. What one country does is supposed to inspire other countries. Peer
pressure is built in.

Five years after that hard-won diplomatic pact, the world as a whole is
not yet close to reining in global temperatures.

And so that raises the fourth and final dilemma: Is it too late to make
a difference?

No. It's true that we have already warmed the planet by burning fossil
fuels for a century and a half, setting in motion
\href{https://www.nytimes3xbfgragh.onion/2019/07/01/climate/europe-heat-wave.html}{heat
waves},
\href{https://www.nytimes3xbfgragh.onion/2019/10/24/climate/california-wildfires-climate-change.html}{wildfires}
and
\href{https://www.nytimes3xbfgragh.onion/2020/04/06/world/australia/great-barrier-reefs-bleaching-dying.html}{mass
bleaching of coral reefs}. But the future isn't set in stone. There are
many futures possible, ranging from quite bad to really catastrophic.
Which one plays out is up to us to decide. Each and every one of us.

Read 57 Comments

\begin{itemize}
\item
\item
\item
\item
\end{itemize}

Advertisement

\protect\hyperlink{after-bottom}{Continue reading the main story}

\hypertarget{site-index}{%
\subsection{Site Index}\label{site-index}}

\hypertarget{site-information-navigation}{%
\subsection{Site Information
Navigation}\label{site-information-navigation}}

\begin{itemize}
\tightlist
\item
  \href{https://help.nytimes3xbfgragh.onion/hc/en-us/articles/115014792127-Copyright-notice}{©~2020~The
  New York Times Company}
\end{itemize}

\begin{itemize}
\tightlist
\item
  \href{https://www.nytco.com/}{NYTCo}
\item
  \href{https://help.nytimes3xbfgragh.onion/hc/en-us/articles/115015385887-Contact-Us}{Contact
  Us}
\item
  \href{https://www.nytco.com/careers/}{Work with us}
\item
  \href{https://nytmediakit.com/}{Advertise}
\item
  \href{http://www.tbrandstudio.com/}{T Brand Studio}
\item
  \href{https://www.nytimes3xbfgragh.onion/privacy/cookie-policy\#how-do-i-manage-trackers}{Your
  Ad Choices}
\item
  \href{https://www.nytimes3xbfgragh.onion/privacy}{Privacy}
\item
  \href{https://help.nytimes3xbfgragh.onion/hc/en-us/articles/115014893428-Terms-of-service}{Terms
  of Service}
\item
  \href{https://help.nytimes3xbfgragh.onion/hc/en-us/articles/115014893968-Terms-of-sale}{Terms
  of Sale}
\item
  \href{https://spiderbites.nytimes3xbfgragh.onion}{Site Map}
\item
  \href{https://help.nytimes3xbfgragh.onion/hc/en-us}{Help}
\item
  \href{https://www.nytimes3xbfgragh.onion/subscription?campaignId=37WXW}{Subscriptions}
\end{itemize}
