 **NYTimes.com no longer supports Internet Explorer 9 or earlier. Please
upgrade your browser.
\href{http://www.nytimes3xbfgragh.onion/content/help/site/ie9-support.html}{LEARN
MORE »}

**Sections

**Home

**Search

\hypertarget{the-new-york-times}{%
\subsection{\texorpdfstring{\href{http://www.nytimes3xbfgragh.onion/}{The
New York Times}}{The New York Times}}\label{the-new-york-times}}

 \href{https://www.nytimes3xbfgragh.onion/section/t-magazine}{T
Magazine} \textbar{}Jay-Z Discusses Rap, Marriage and Being a Black Man
in Trump's America

**Close search

\hypertarget{site-search-navigation}{%
\subsection{Site Search Navigation}\label{site-search-navigation}}

Search NYTimes.com

**Clear this text input

Go

\url{https://nyti.ms/2kaqfaJ}

\hypertarget{site-navigation}{%
\subsection{Site Navigation}\label{site-navigation}}

\hypertarget{site-mobile-navigation}{%
\subsection{Site Mobile Navigation}\label{site-mobile-navigation}}

\hypertarget{jay-z-discusses-rap-marriage-and-being-a-black-man-in-trumps-america}{%
\section{Jay-Z Discusses Rap, Marriage and Being a Black Man in Trump's
America}\label{jay-z-discusses-rap-marriage-and-being-a-black-man-in-trumps-america}}

The music mogul talks with The New York Times's executive editor at the
close of an eventful year.

On therapy, politics, marriage, the state of rap and being a black man
in Trump's America

\href{http://www.nytimes3xbfgragh.onion/video/t-magazine/100000005574909/jayz-interview.html}{Watch
the 35-minute interview}

My conversation with Jay-Z began with O.J.

When I was a kid growing up in black New Orleans in the 1960s, O.J.
Simpson was a god. We imitated his moves, his swagger. We didn't want to
just play like him. We wanted to be him, gorgeous and running in the
California sun. We practiced his juking moves in the mirror, our hands
too small to hold the ball loosely, the way he did. We even wanted to go
to U.S.C., where he led the nation in rushing two years in a row. We
were angry when he lost the Heisman Trophy to the white, All-American,
clean-cut U.C.L.A. quarterback Gary Beban, known as ``The Great One.''
We were triumphant when he won it the next year.

But O.J. was not a perfect hero for young black boys, even though he
launched himself from poverty in San Francisco to superstardom. He was
racially ambivalent. At a time when other athletes were starting to make
their blackness a cause, he was trying to make his a footnote.

So when I was invited to interview Jay-Z, I wanted to talk about his
song ``The Story of O.J.,'' from his most recent album, ``4:44,'' in
which he quotes the legendary, maybe apocryphal, Simpson line ``I'm not
black, I'm O.J.''

I was less engaged by the rapper's marital troubles or his infamous,
caught-on-video 2014 elevator dust-up with his sister-in-law. But I did
want to try to understand how, with an \$88 million Bel Air mansion a
freeway ride from neighborhoods where black people endure with so
little, Jay-Z holds onto his younger self --- a black man who grew up in
the '70s in the Marcy projects of Brooklyn. It seemed from his new body
of work that examining this high-wire act of straddling two places had
been stirring more deeply within him --- much the way it stirs in me, a
Southern black man who grew up revering O.J. and whose own success is
infinitely greater than anyone in my early life would have imagined for
me.

What is it about the story of O.J. Simpson that moved us both?

O.J. must have locked down part of himself when he presented himself as
the noncontroversial star who never talked about race, the perfect foil
for his fellow football player, Cleveland Browns running back Jim Brown,
who seemed more threatening, angry. I had to wonder if the pressure of
that denial caused him to explode decades later.

All of this was on my mind when I met with Jay-Z for two hours in an
executive office at The Times this past September. Besides O.J. and
racial identity, we talked about his mother's sexuality, and how he
could possibly raise socially aware children who shuttled between
mansions: After years of rapping about growing up in the `hood, he has
produced an album that sounds like a middle-aged black man's deeply
introspective therapy session put to music.

\emph{This interview has been edited and condensed. Annotations by
Wesley Morris, critic at large for The New York Times, and Reggie Ugwu,
pop culture reporter for The New York Times.}

DEAN BAQUETFirst, welcome.

JAY-ZThank you.

BAQUET The things I want to talk to you about: I want to talk a little
bit about race. Your music some, too. I thought the song
{[}"\href{https://www.youtube.com/watch?v=RM7lw0Ovzq0}{The Story of
O.J.}," from the album
"\href{https://listen.tidal.com/album/75413011}{4:44}," 2017{]} was
particularly powerful. I took the message as, "You can be rich, you can
be poor, you're still black." Who were you speaking to? Who did you want
to listen to that and be moved by it?

JAY-Z It's a nuanced song, you know. It's like, I'm specifically
speaking to us. And about who we are and how do you maintain the sense
of self while pushing it forward and holding us to have a responsibility
for our actions. Because in America, it is what it is. And there's a
solution for us: If we had a power base together, it would be a much
different conversation than me having a conversation by myself and
trying to change America by myself. If I come with 40 million people,
there's a different conversation, right? It's just how it works. I can
effect change and get whomever in office because this many people, we're
all on the same page. Right? So the conversation is, like, "I'm not
rich, I'm O.J." For us to get in that space and then disconnect from the
culture. That's how it starts. This is what happens. And then you know
what happens? You're on your own, and you see how that turned out.

BAQUET Was it a reminder, too, that the thing O.J. forgot, maybe, was
that as rich as he was, as entitled as his life was, he was reminded
very forcefully when he became a subject of racial debate that he was
also a black man, whether he accepted that or not?

\includegraphics{https://static01.graylady3jvrrxbe.onion/newsgraphics/2017/10/30/jayz/27fad94c678e831a0fb514ffbcdcda96925ebdcb/jayz-cover.jpg}

On the Cover

Jay-Z is featured in
\href{https://www.nytimes3xbfgragh.onion/issue/t-magazine/2017/11/17/ts-dec-3-holiday-issue}{T's
Dec. 3 Holiday issue}, with a portrait created exclusively for T by
Henry Taylor. Learn more about Taylor's process and work
\href{https://www.nytimes3xbfgragh.onion/2017/11/29/t-magazine/henry-taylor-jay-z-painting.html}{in
T's studio visit}.

``I Am a Man,'' 2017.\\
© Henry Taylor.\\
Photo: © Manuel Franquelo-Giner. Courtesy of the artist and Blum \& Poe,
Los Angeles/New York/Tokyo.

JAY-Z That's right. Absolutely. And for us, like I'm saying, to speak to
that the point is, "Don't forget that," because that's really not the
goal. The goal is not to be successful and famous. That's not the goal.
The goal is, if you have a specific God-given ability, is to live your
life out through that. One. And two, we have a responsibility to push
the conversation forward until we're all equal. Till we're all equal in
this place. Because until everyone's free, no one's free, and that's
just a fact.

BAQUET When you're as amazingly successful as you are, your kids will
live in a very different world from the world you grew up in. How do you
go about making sure that they understand the world you grew up in?

JAY-Z There's a delicate balance to that, right? Because you have to
educate your children on the world as it exists today and how it got to
that space, but my child doesn't need the same tools that I needed
growing up. I needed certain tools to survive my area that my child
doesn't need. They're growing up in a different environment.1 But also
they have to know their history. Have a sense of what it took to get to
this place. And have compassion for others. The most important thing I
think out of all this is to teach compassion and to identify with
everyone's struggle and to know these people made these sacrifices for
us to be where we are and to push that forward --- for us. I believe
that's the most important thing to show them, because they don't have to
know things that I knew growing up. Like being tough.

1Reggie Ugwu: Jay-Z anticipated this dilemma on the 2011 song
"\href{https://open.spotify.com/track/3SIgBSv9D8Nm3CvP0o0Q9Z}{New Day}"
from the album
"\href{https://open.spotify.com/album/2P2Xwvh2xWXIZ1OWY9S9o5}{Watch the
Throne}," in which he and Kanye West both address verses to their future
sons. In Jay's verse, he raps: "Took me 26 years just to find my path /
My only job is cut the time in half." (Jay-Z's son, Sir, and his twin
sister, Rumi,
\href{https://www.nytimes3xbfgragh.onion/2017/07/14/arts/music/beyonce-twins-photo.html}{were
born last June}. His eldest daughter, Blue Ivy, is 5.)

BAQUET Do you worry at all that as much as you will teach them history,
and as much as you yourself are seen as an important figure among black
people in America, that there's something they'll be missing? Or do you
think that's silly, {[}that{]} in fact they'll have so many advantages
that that's too negative of a way to approach it?

JAY-Z Exactly. Like, they'll be who they are, right? And it's just
certain tools that you would hope for your child to have. You know,
like, again, fairness and compassion and empathy and a loving heart. And
those things translate in any environment. Those are the main base
things that you want --- well, for me, I would want my child to have.
You know? Treat people as they are, no matter who they are, no matter
where they sit in the world, not to, like, be super nice to someone at a
high position or mean to someone who they've deemed to be below them. I
can't buy you love, I can't show it to you. I can show you affection and
I can, you know, I can express love, but I can't put it in your hand. I
can't put compassion in your hand. I can't show you that. So the most
beautiful things are things that are invisible. That's where the
important things lie.

BAQUET For me as a black man of a certain age, when I was a kid O.J.
Simpson was God. I'm 61, so I was a little kid when he was {[}around{]}.
Do you expect black people and white people and young people and old
people to hear different things in your music? I'm sure I heard some
things in that song that you may not even have thought of 'cause I'm a
different generation. What do you want a young white kid to hear in that
song that maybe a young black kid would not hear?

\includegraphics{https://static01.graylady3jvrrxbe.onion/packages/flash/multimedia/ICONS/transparent.png}

Created exclusively for T magazine, Henry Taylor's ``Go Next Door and
Ask Michelle's Momma Mrs Robinson if I Can Borrow 20 Dollars Til Next
Week?'' (2017), alludes to themes from Jay-Z's music: being raised in
the projects, racial protest and the
\href{https://www.nytimes3xbfgragh.onion/2017/11/17/opinion/jay-z-meek-mill-probation.html}{incarceration
rates} of black men. © Henry Taylor. Photo: Manuel Franquelo-Giner.
Courtesy of the artist and Blum \& Poe, Los Angeles/New York/Tokyo.

JAY-Z That's a great question. I think when you make music, you want
people to hear different things, and then you want it to start a
dialogue. Because that's how we get to understanding. "Oh, you felt that
way about it." "This is actually what I meant, because this happened,
and these things happened, that led to me saying this specific thing."

BAQUET How did you react when that one line in that song where you
referred to Jews and wealth2 {[}"You ever wonder why Jewish people own
all the property in America? This how they did it"{]} --- some people
got upset. How did you feel about that?

2Reggie Ugwu: A representative of the Anti-Defamation League
\href{http://www.rollingstone.com/music/news/adl-rebukes-jay-z-over-jewish-lyric-in-the-story-of-oj-w491217}{told
Rolling Stone}, "the lyric does seem to play into deep-seated
anti-Semitic stereotypes about Jews and money."

JAY-Z I felt it was really hypocritical. Only because it's obvious the
song is, like, "Do you want to be rich? Do what people got rich done."
Of course, it's a general statement, right? It's obviously a general
statement, like the video attached to it was a general statement. And if
you didn't have a problem with the general statement I made about black
people, and people eating watermelon and things like that
{[}\href{https://www.youtube.com/watch?v=RM7lw0Ovzq0}{the animated music
video} for the song, which references racist cartoons, includes a
caricature of a black man eating watermelon{]} --- if \emph{that} was
fine, {[}but{]} that line about wealth bothered you, then that's very
hypocritical, and, you know, that's something within yourself. 'Cause
basically, I was saying, you know, Michael Jordan, LeBron James, is a
great basketball player. He trains in the off-season. If you want to be
great, train in off-season like him. That's basically the statement. You
can't miss the context of the song. You have to be like 5 years old or
something.

BAQUET Some people think that the election of Donald Trump has revived
the debate about race in America. Some people think that, in fact,
there's always been racism in America; that it hasn't changed and that
the debate isn't any different. It's just people are paying attention to
it. What do you think?

JAY-Z Yeah, there was a great Kanye West line in one of {[}his{]} songs:
"Racism's still alive, they just be concealin' it."
{[}"\href{https://open.spotify.com/track/06TqBu4Vi2DEu2F2wNTEnb}{Never
Let Me Down}," from West's 2004 album,
"\href{https://open.spotify.com/album/3ff2p3LnR6V7m6BinwhNaQ}{The
College Dropout}."{]} Take a step back. I think when Donald Sterling3
got kicked out of the N.B.A., I thought it was a misstep, because when
you kick someone out, of course he's done wrong, right? But you also
send everyone else back in hiding. People talk like that. They talk like
that. Let's deal with that.

3Reggie Ugwu: In 2014, Sterling, then owner of the Los Angeles Clippers,
was
\href{https://www.nytimes3xbfgragh.onion/2014/04/30/sports/basketball/nba-donald-sterling-los-angeles-clippers.html}{banned
for life} from the N.B.A. after a recording emerged in which he made
racist comments about black people to a female friend.

I wouldn't just, like, leave him alone. It should have been some sort of
penalties. He could have lost some draft picks. But getting rid of him
just made everyone else go back into hiding, and now we can't have the
dialogue. The great thing about Donald Trump being president is now
we're forced to have the dialogue. Now we're having the conversation on
the large scale; he's provided the platform for us to have the
conversation.

BAQUET And you think that's better? That we should be having a
conversation?

JAY-Z Absolutely. That's why this is happening.

BAQUET Do you think the debate over race in America is happening in a
healthy way?

JAY-Z Well, an ideal way is to have a president that says, "I'm open to
dialogue and fixing this." That's ideal. But it's still happening in a
good way, because you can't have a solution until you start dealing with
the problem: What you reveal, you heal.

BAQUET Mm-hmm. Mm-hmm.

JAY-Z Right? If I have like a tumor, and I don't know it, it doesn't
mean it goes away. I have to diagnose it first. No matter how it
happens. If I get hit with a football, and, like, Oh, I feel something
there, and then I go to the doctor --- it still happened.

BAQUET Right.

JAY-Z You know what I'm saying? So however it happens, we're just
getting hit with a lot of footballs. To use {[}an{]} analogy that goes
next to the N.F.L.

BAQUET If you were an owner, you would sign
\href{https://www.nytimes3xbfgragh.onion/2017/09/07/sports/colin-kaepernick-nfl-protests.html}{Colin
Kaepernick}, right?

JAY-Z Yeah. I dedicated "The Story of O.J." to him at the Meadows
concert.4

4Wesley Morris: Ah, Colin Kaepernick, the former San Francisco 49ers
quarterback, whose protest against racism and racial inequality in
America might be the most incendiary act of political umbrage since
\href{https://www.youtube.com/watch?v=zIUzLpO1kxI}{Kanye West declared},
at a 2005 national telethon for victims of Hurricane Katrina, that
George W. Bush "doesn't care about black people." Kaepernick's decision
last year to take a knee during the performance of the national anthem
has since been adopted by dozens of other athletes. The president's
outrage at the protests and the ensuing debate have upstaged the
injustice Kaepernick knelt against. In October, after several months
without being signed to a team,
\href{https://www.nytimes3xbfgragh.onion/2017/10/15/sports/colin-kaepernick-nfl-collusion.html}{Kaepernick
filed} a grievance against the N.F.L. accusing the owners of collusion
against him. So there's an apt sting in dedicating "The Story of O.J."
to him, since he's now taking a hit for being agreeable, race-blind O.J.
Simpson's opposite.

BAQUET Have you met him?

JAY-Z No. We just had dialogue over the phone, but we supposed to get
together.

BAQUET Do you have any doubt that if this had not happened, he would be
signed by a team?

JAY-Z Yeah, yeah. Of course.

BAQUET Do you think basketball is more politically active than football?

JAY-Z Yeah.

BAQUET Why is that?

JAY-Z I think because, first of all, it's smaller numbers. It's 12
people on a team. In football you have 53 people. So it's harder to get
53 people thinking the same thing. It's easier to have a conversation to
get 12 people on the same page. For one. Two, {[}the N.B.A. has{]} a
great ... they have a great commissioner5 who's really open. And, you
know, supports them. And you feel that. You feel like, you know, when
you have someone behind you that really believe in what's right, it
motivates you to do the right thing. I think those two factors show why
they're much further along.

5Wesley Morris: Adam Silver is the current, generally beloved N.B.A.
commissioner. He has yet to drastically stifle expressions of political
empathy by teams or individual players. (Let's see what happens should a
player defy his expectation that all N.B.A. players
\href{https://www.nytimes3xbfgragh.onion/2017/09/28/sports/basketball/nba-adam-silver-anthem.html}{stand
during the anthem}.) Silver has merrily ridden for the last two years on
the N.B.A. float in New York City's gay pride parade. He oversaw the
expulsion of the Los Angeles Clippers owner Donald Sterling, who could
never have remained in the league --- his continued presence might have
torn it apart. Ridding basketball of Sterling started as many
conversations about the sort of open-carry racism we're currently living
with as keeping him around would have. One thing that makes Silver a
great commissioner is that he understands that.

BAQUET Are there incidents even at this stage in your life --- you're
famous, you're rich, you own stuff --- where you run into racism that's
evident to you, that's easy to recognize?

JAY-Z Yeah. Yes. Yeah. But it mostly comes when you try to challenge the
status quo.

If I'm being quiet and entertaining, everyone's cool. Ah man, it's
great. You don't feel racism. But when you try to challenge the club,
it's like, Oh, nah, we should have a seat at --- to use the
\href{https://open.spotify.com/album/3Yko2SxDk4hc6fncIBQlcM}{Solange
album title} --- we should have a seat at this table. And then it gets
into a space where it's like, wait, you guys are mad at me about the
same thing you guys are doing. It gets into a weird space.

BAQUET Are you in meetings now in your business life6 where you're the
only black man in the room?

6Reggie Ugwu: Jay-Z's numerous business ventures --- launched in
parallel to his prolific music career --- created a template for the
rapper/mogul. He is the founder of the entertainment company and record
label Roc Nation, owns the champagne brand Armand de Brignac and co-owns
the
\href{https://www.nytimes3xbfgragh.onion/2015/03/31/business/media/jay-z-reveals-plans-for-tidal-a-streaming-music-service.html}{streaming
service Tidal}, among other businesses.

JAY-Z Well, when I was doing the Nets7, I was definitely the only black
guy in the room.

7Reggie Ugwu: Jay-Z acquired a
\href{http://www.nytimes3xbfgragh.onion/2012/08/16/nyregion/with-the-nets-jay-z-rewrites-the-celebrity-investors-playbook.html}{small
minority stake} in the New Jersey Nets in 2003. (He sold it in 2013.) He
played an influential role in bringing the team to Brooklyn, including
helping design the new team logo and jerseys.

BAQUET And what was that like? Describe that.

JAY-Z It was um, it's strange, but at the same time I think that ... I
think that in that room, my celebrity allowed me a voice that probably
would have been awkward for someone {[}else{]} in my position being the
only black person in the room to break through.

BAQUET This album {[}"4:44."{]} sounds to me like a therapy session.

JAY-Z Yeah, yeah.

BAQUET Have you been in therapy?

JAY-Z Yeah, yeah.

\includegraphics{https://static01.graylady3jvrrxbe.onion/packages/flash/multimedia/ICONS/transparent.png}

In ``Glory'' (2017), created exclusively for T, the London-based artist
Chantal Joffe found herself able to connect with the wonder and weight
of parenthood that Jay-Z so clearly expresses in his 2012 song of the
same name. Courtesy of Victoria Miro, London/Venice and Cheim \& Read,
New York.

BAQUET First off, how does Jay-Z find a therapist? Not in the Phone
book, right?

JAY-Z No, through great friends of mine. You know. Friends of mine
who've been through a lot and, you know, come out on the other side as,
like, whole individuals.

BAQUET What was that like, being in therapy? What did you talk about
that you had never acknowledged to yourself or talked about?

JAY-Z I grew so much from the experience. But I think the most important
thing I got is that everything is connected. Every emotion is connected
and it comes from somewhere. And just being aware of it. Being aware of
it in everyday life puts you at such a ... you're at such an advantage.
You know, you realize that if someone's racist toward you, it ain't
about you. It's about their upbringing and what happened to them, and
how that led them to this point. You know, most bullies bully. It just
happen. Oh, you got bullied as a kid so you trying to bully me. I
understand.

And once I understand that, instead of reacting to that with anger, I
can provide a softer landing and maybe, ``Aw, man, is you O.K.?'' I was
just saying there was a lot of fights in our neighborhood that started
with ``What you looking at? Why you looking at me? You looking at me?''
And then you realize: ``Oh, you think I see you. You're in this space
where you're hurting, and you think I see you, so you don't want me to
look at you. And you don't want me to see you.''

BAQUET You think I see your pain.

JAY-Z You don't want me to see your pain. You don't ... So you put on
this shell of this tough person that's really willing to fight me and
possibly kill me 'cause I looked at you. You know what I'm saying, like,
so ... Knowing that and understanding that changes life completely.

BAQUET Was that a moment that came from therapy?

JAY-Z Yeah --- just realizing that, oh my goodness, these young men
coming from these ... they just in pain.

BAQUET Mm-hmm.

JAY-Z You have to survive. So you go into survival mode, and when you go
into survival mode what happen? You shut down all emotions. So even with
women, you gonna shut down emotionally, so you can't connect.

BAQUET You can't connect because of the way you feel about yourself, you
mean?

JAY-Z Yes. In my case, like it's, it's deep. And then all the things
happen from there: infidelity ...

BAQUET You've bared your soul so much. Not only in this album --- you
can sort of see the evolution of a person in your music. Part of me
would think, Oh my god, I gotta talk about my marriage, I gotta talk
about my mother, I gotta talk about my other ancestors. Part of me would
think that would make me nuts. Does it make you nuts, or do you feel
like the heart of your art is to tell the story of your life?

JAY-Z That's who I --- that's who I am. And I've done it from the
beginning of my career. Two things: one, no one knew the characters
{[}back then{]}. So it wasn't as impactful. And two, it wasn't coming
from a place where it was as evolved.8 And it's very difficult. It's
hard to hear songs back. It's hard to perform ... songs, but, um, I feel
it's the most important work that I've done and I'm very proud of it and
the effect that it's having on people. Even like the studio sessions,
you know, we were having four-hour conversations after playing one song.
I learned so much about people that was around me, just my friends, I
learned things about them that I didn't know, in a 20-year relationship,
just from this one song. So I knew it would have that sort of impact
beyond myself. It's my responsibility as an artist to go to these
places.

8Reggie Ugwu: Many of Jay-Z's early hits, including
"\href{https://www.youtube.com/watch?v=SoqZutsd1m0}{Can I Get
a\ldots{}}" and "\href{https://www.youtube.com/watch?v=Cgoqrgc_0cM}{Big
Pimpin'}," gloried in juvenile, and misogynistic, attitudes toward
women.

BAQUET But you probably couldn't have gotten away with, O.K., you do the
album, wife, that talks about our pain, I'm gonna go do an album that
talks about, you know, my love of art.

JAY-Z Yeah, you just, you never know. I think it turned out for the
best, but you just never know, because people like to be entertained.
Again, back to our president. You would think, Man, after the composed
manner in which Obama stood at that podium, the dignity he brought to
that place, that this couldn't exist. But it does.

\includegraphics{https://static01.graylady3jvrrxbe.onion/packages/flash/multimedia/ICONS/transparent.png}

``Jay-Z with Blue Ivy'' (2017), created exclusively for T by Chantal
Joffe, who
\href{https://www.nytimes3xbfgragh.onion/slideshow/2016/03/20/t-magazine/just-kids/s/03well-portfolio-slide-C2YH.html}{is
well known for} her tender, gestural portraits of women and children.
``I couldn't stop painting them,'' the artist says of the music mogul
and his eldest child. ``There was something about the juxtaposition of
their two heads that was so beautiful; I don't paint men very often, and
his face --- it was like a Picasso, full of planes.'' Courtesy of
Victoria Miro, London/Venice and Cheim \& Read, New York.

BAQUET Do you have any disappointments in Obama? There are people who
say the expectations of him as the first black president were so great:
He was supposed to get rid of racism and fix everything. Is that unfair?
Did he live up to all of your expectations?

JAY-Z Yes, because all he could do was the best he can do. He's not a
superhero. And it's unfair to place unfulfillable expectations on this
man just because of his color. You're actually doing the opposite. It's
like, what do you think is gonna happen? He's there for eight years. And
he has to undo what 43 presidents have done. In eight years. That's not
fair.

BAQUET What do you think of the state of --- I'm not gonna say just
black leadership, but leadership, period, on the things you care about
in the country? Who do you, like, look at and say, "This man or woman
speaks for the things I care about?"

JAY-Z {[}laughs{]} I find it funny, but ... I like
\href{https://www.nytimes3xbfgragh.onion/2017/04/19/t-magazine/dave-chappelle-profile.html}{Dave
Chappelle}'s {[}laughs{]}.

BAQUET {[}laughs{]} Go ahead.

JAY-Z You know what I'm saying?

BAQUET You gonna vote for Dave Chappelle for president?

JAY-Z Yeah. 'Cause he tells it in humor so you can deal with it, but
it's always a nice chunk of truth in there.

BAQUET Is there a part of you, because you have a certain amount of
money, that gets a little more conservative, or has having money9 not
changed your politics?

9Reggie Ugwu: I love the politeness in the phrase "a certain amount of
money." Jay-Z's net worth is an estimated \$810 million. He spent most
of 2013's
"\href{https://open.spotify.com/album/37uqAKt9dLsLob7yomDWY4}{Magna
Carta Holy Grail}" both admiring his wealth and wielding it. Being a
rich black rapper had put him on the defensive when his strongest
creative mode involves the moral costs and social perks of his being
self-made. One of the most poignant aspects of the new album is the way
he's trying to reconcile what affluence means for him, his family and
his race.

JAY-Z No. No, because I believe in people. I want what's best for
people. I love people. You know, so I don't have that sort of thing,
like, I want to vote Republican just to save more money.

BAQUET Right.

JAY-Z That's not the endgame. It's not about who got more money and who
got more houses. Yes, you know, you've earned it, buy what you want.

BAQUET Right.

JAY-Z You know? But don't forget what's important. Without people, being
rich would be very boring.

BAQUET Right {[}laughs{]}.

JAY-Z {[}laughs{]} No one to share with, no one to have ... You know
what I mean? You'd just be a rich person, one person on the planet ---
just, like, well then what do you do?

BAQUET When I heard this latest album, and then I thought about the
earlier albums, one theme was sort of reaching the promised land. You
know, you've acquired influence, and not just money, but your life is
good. And then when you listen to the newest album, you're thinking: He
must have been in a lot of pain when life was good.

JAY-Z Absolutely.

BAQUET Is that true?

JAY-Z Yeah. I did this song called
"\href{https://www.youtube.com/watch?v=w5srnNrICJo}{Song Cry}."10

10Reggie Ugwu: The song, from Jay-Z's Grammy-nominated 2001 album
``\href{https://open.spotify.com/album/69CmkikTHkGKdkrUZTtyWl}{The
Blueprint},'' describes the dissolution of three early relationships and
the rapper's inability to reconcile his emotions and grieve fully, or
openly. ``Pride won't let me show it / Pretend to be heroic.''

BAQUET Mm-hmm.

JAY-Z And the idea of the hook --- "never seen it comin' down my eyes,
but I gotta make the song cry." It tells you right there what I was, I
was hiding.

The strongest thing a man can do is cry. To expose your feelings, to be
vulnerable in front of the world. That's real strength. You know, you
feel like you gotta be this guarded person. That's not real. It's fake.

BAQUET Does that mean you were unhappy during that period and didn't
have a handle on it, or what?

JAY-Z Well, you compartmentalize, right? So you can be, you can be
inside your body and be happy, but at the core of it, something else is
going on.

BAQUET As a parent, I thought one of the most painful scenes in the
album was when you are talking about having almost lost your marriage,
and what it would have been like to watch another man play football with
your kid. Given that you have talked so much about your life in your
music, are there things that you put a wall around? You've talked about
the pain of growing up where you grew up, how you grew up, your father
leaving early, the pain of your marriage, being in therapy: Are there
things {[}about which{]} you say, "I'm not going there"?

JAY-Z Yeah. And it mostly involves other people 'cause when other people
are involved, you may be ready to expose these things, {[}but{]} it's
also other people truth as well.

A perfect example is my mom. I didn't have permission to
\href{https://www.nytimes3xbfgragh.onion/2017/07/02/arts/music/jay-z-4-44-review.html}{do
that song first}.11 It's just like we had a beautiful conversation.

11Wesley Morris: I'm a sucker for mommy music. The third track on "4:44"
--- "Smile" --- features Jay-Z's mother, Gloria Carter, and contains
this couplet about her: "Momma had four kids, but she's a lesbian / Had
to pretend so long that she's a thespian." His rapping here is buttery
and exuberant: "I just wanna see you smile through all the hate / Marie
Antoinette, baby, let 'em eat cake." So his approach to Gloria manages
to exult her and, for anybody annoyed by his "eat the cake, Anna Mae"
line on Beyoncé's 2013 track
"\href{https://open.spotify.com/track/6jG2YzhxptolDzLHTGLt7S}{Drunk in
Love}," arguably to exonerate him from having cast himself as Ike Turner
to Beyoncé's Tina. Anyway, "Smile" ends with an inspirational poem that
Gloria wrote and recites and that puts the concept of shadows to moving
metaphorical use: "The world is changing and they say it's time to be
free / But you live with the fear of just being me / Living in the
shadow feels like the safe place to be / No harm for them, no harm for
me / But life is short, and it's time to be free / Love who you love,
because life isn't guaranteed / Smile."

BAQUET When did you realize your mother was gay?

JAY-Z Uh, really early on when, when I was ...

BAQUET Like as a little kid?

JAY-Z Not, no, not --- let's call it teenage years.

BAQUET So you realized that and talked to her about it?

JAY-Z We never spoke about it. We --- it just exist. It was there.
Everyone knew.

BAQUET Gotcha.

JAY-Z But we never spoke about it. Until, like, recently, now we start
having these beautiful conversations, and just really getting to know
each other. We were always good friends but now we're really great
friends. You know. And we were just talking as friends. And then she was
sharing that she was in love. She can be herself {[}now{]}. She doesn't
have to hide for her kids or feel like she's embarrassing her kids. It
was a much different time then. {[}Now{]} she can just live her full
life, her whole life, and be her.

BAQUET Will this get harder over time? Like, you know, as a young man,
your music was the way a lot of young rappers are --- it's like, you
know, {[}your music was about{]} the violent life. If that's chapter one
of the autobiography, chapter two of the autobiography --- I'm
oversimplifying --- is like, "Now I'm really rich. I have a lot of
stuff. Let me tell you how cool that is." And then chapter three of the
autobiography is, "Oh my god, I've run myself into the ground." So
what's chapter four?

JAY-Z No, chapter three is: Oh my goodness, oh, the most beautiful
things are not these objects. The most beautiful things are inside. The
most beautiful things are the friendships I have. I have really golden
friendships. The compassion and the person I've become --- that's what
this chapter is. You know? And the conversation with my mom. Those are
the real enriching experiences.

BAQUET But will you have the same adventures in your life? Will you have
the same stuff to write about? Or maybe you don't know.

JAY-Z I think that rap in particular is
\href{https://www.nytimes3xbfgragh.onion/2017/07/19/arts/music/jay-z-and-the-politics-of-rapping-in-middle-age.html}{a
young man's sport}, that I'll move out of that white-hot space. Rap is
about the gift of discovery. The white-hot space is when it's fresh and
new, and it's like, this is the hottest song ever. I mean I pushed the
window, like ---

BAQUET You still --- you think you're still in that space?

JAY-Z I stretched it. Oh, I stood in that window a really long time. But
still, no, I don't think people are looking to me as like, The Thing.

BAQUET Is that hard to deal with, or did you feel like, I'm O.K. with
that, because I've moved on?

JAY-Z No.

BAQUET You don't want to be.

JAY-Z 'Cause I, at the end of the day we gonna find out it's not about
the white-hot space, but it's about finding the truth. That white-hot
space --- people think it's the biggest thing, but it's really small.
It's almost like a trend.

Would you rather be a trend, or you rather be Ralph Lauren? You know
what I mean; like, you rather be a trend, or you rather be forever?

I'm the person that looked at the Mona Lisa and be like, Man, that's
gonna be cool in 40 years. I play forever. And so my whole thing is to
identify with the truth. Not to be the youngest, hottest, new, trendy
thing.

BAQUET One of the things you rap about also is the pain you caused the
people you sold drugs to.12 Have you ever had conversations with people
like that you caused pain to as a young man and talked about it?

12Reggie Ugwu: A motif of Jay-Z's discography is the mutually assured
destruction of drug users and drug dealers alike. From
"\href{https://open.spotify.com/track/0qSfuaTaSacJfsNqXZPbkX}{Can I
Live}" on 1996's
"\href{https://open.spotify.com/album/5lheTytGmdGaGlxzXu4uDY}{Reasonable
Doubt}": "We hustle out of a sense of hopelessness ... We become
addicted / Sort of like the fiends we accustomed to servin'."

JAY-Z No, I haven't. No.

BAQUET What would you say to them? Or is that impossible to do at this
point?

JAY-Z Nothing's impossible. I guess that conversation would definitely
take ownership for my part in, um, you know, the part I played in
occupying that space. Because knowing what I know now, you know, you
can't sacrifice others for your life. There's a karmic debt that has to
be paid. Had I had the level of consciousness then that I have now,
things would have turned out differently. And just knowing that ... I
definitely want everyone to know that.

BAQUET Do black artists have a different obligation than white artists?
Do you feel you have a different kind of obligation to the people who
listen to you than if you were a white musician?

JAY-Z Yeah, 'cause I have an obligation,
\href{https://www.nytimes3xbfgragh.onion/interactive/2016/02/01/arts/television/oj-simpson-murder-trial-coverage.html}{going
back to} the story of O.J.,
\href{https://www.nytimes3xbfgragh.onion/2017/11/17/opinion/jay-z-meek-mill-probation.html}{to
further conversation} of an entire race of people. And to . . . Not me
--- all of us. But specifically me, since you're asking the question,
it's to open up dialogue. {[}...{]} It's O.K. to think. It's O.K. to be
smart. You know, there was a time when people was like, "you talkin'
white." It's like, what does that even mean? I know words? Intelligence
is not a tribute to color. And I'm sure you've heard it growing up many
times.

BAQUET Of course.

JAY-Z "You speaking white." Like, what?

BAQUET Yeah. Yeah.

JAY-Z I'm speaking like I know words. And it's O.K., it's fine. You
know, so I have an obligation to further the conversation and always,
you know, our stature in America. Our emotional maturity. And so on and
so forth. It's humbling; at the same time it's like, you know, it's what
you've been charged with in life. And I believe since the beginning of
time the poets have been charged with that. Like it was the poets that's
explaining the emotions and making these songs that people like, "Oh,
that's what I feel."

BAQUET Are there black artists, and I won't ask you to name them unless
you want to, who you think don't live up to that obligation to start a
conversation about race? Do you think there are people you wish did
more?

JAY-Z Well, I mean, and for one, O.J., right? 'Cause that's the one that
we can all identify. There are those who don't uphold their mantle, and
we know how that story plays out.

BAQUET What would you say to him if you could talk to him?

JAY-Z I don't know. I would probably say, "Man, I'm sorry that so much
happened to you, man." You know, people act out in this way based on
their life experiences and, you know, I'm sure he's been through a lot
of trauma in his life. I think that'll start the conversation.

BAQUET Did you watch the documentary about him?

JAY-Z I watched every one.

BAQUET I did too.

JAY-Z Yeah, there was like
\href{https://www.nytimes3xbfgragh.onion/2016/06/26/arts/television/oj-simpson-trial-made-in-america.html}{eight
of them on} at the same time.

BAQUET You could read the story of O.J. two ways. You could say it's a
reminder of people that they're black. I could read that as a negative
message or a positive message. The positive message being: You're black
and you should be more proud of it. The negative message is: Who are you
kidding? You can't escape this by joining a private country club and
playing golf.

JAY-Z Right.

BAQUET Which message feels like the right {[}one{]}?

JAY-Z They both, they both dual messages at the same time. It's like, be
proud of who you are and realize that we're gonna get further together.
Don't check out. You can't just turn your back on the place you come
from. You come from a community. Your job is to uplift it now.

BAQUET So now I gotta ask my one gossipy question. Talk about Kanye West
and your relationship with him, which you alluded to a little bit in the
album.13 When's the last time you talked to him?

13Wesley Morris: It's an allusion to a saga. Together, Jay-Z and Kanye
West made the 2011 album "Watch the Throne," which reached an apogee of
rap collaboration. The story goes that last year, Jay-Z, through his
partnership with Live Nation, gave West \$20 million as part of his
contract to cover expenses for his Saint Pablo tour. Then, from the
stage of that tour in Sacramento, West embarked on a tirade that
included shots at, among others, Beyoncé, Mark Zuckerberg and, yes,
Jay-Z: "I've been sent here to give y'all my truth even at the risk of
my own life, even at the risk of my own success, my own career. ...
Jay-Z --- call me, brah. You still ain't called me. ... Jay-Z, I know
you got killers. Please don't send 'em at my head. Just call me. Talk to
me like a man." He eventually ended the concert after four songs and,
two days later,
\href{https://www.nytimes3xbfgragh.onion/2016/11/21/arts/music/kanye-west-hospitalized-exhaustion.html}{checked
himself into a hospital} for an eight-day stay. So, in response, Jay-Z
opens "4:44" with
"\href{https://www.youtube.com/watch?v=rQLVU4j7Ejo}{Kill Jay-Z}," nearly
three minutes of second-person self-laceration. "You ain't a saint /
This ain't KumbaYe," Jay-Z raps. "But you got hurt because you did cool
by Ye / You gave him \$20 without blinking / He gave you 20 minutes on
stage / {[}What{]} was he thinkin'?" I've been interpreting this as
Jay-Z realizing the usefulness of the old business credo advising
against mixing business with friendship and blaming himself for ignoring
it.

JAY-Z I {[}talked to{]} Kanye the other day, just to tell him, like,
he's my brother. I love Kanye. I do. It's a complicated relationship
with us.

BAQUET Why is it complicated?

JAY-Z 'Cause, you know --- Kanye came into this business on my label. So
I've always been like his big brother. And we're both entertainers. It's
always been like a little underlying competition with your big brother.
And we both love and respect each other's art, too. So it's like, we
both --- everyone wants to be the greatest in the world. You know what
I'm saying? And then there's like a lot of other factors that play in
it. But it's gonna, we gonna always be good.

BAQUET But there's tension now, right?

JAY-Z Yeah, yeah, yeah. But that happens. In the long relationship, you
know, hopefully when we're 89 we look at this six months or whatever
time and we laugh at that. You know what I'm saying? There's gonna be
complications in the relationship that we have to get through. And the
only way to get through that is we sit down and have a dialogue and say,
"These are the things that I'm uncomfortable with. These are the things
that are unacceptable to me. This is what I feel." I'm sure he feels
that I've done things to him as well. You know what I'm saying? These
are --- I'm not a perfect human being by no stretch. You know.

BAQUET Is he as evolved as you?

JAY-Z He's highly evolved. No, he's ... I think he started out in a more
compassionate position than me. You know what I'm saying. I don't know
if he's had the level of --- I mean, I had to survive by my instincts.
I'm here because I grew up a different way. And I got out of that.

You know, my first album came out when I was 26. So I was already a
different artist. You know, a lot of people's album come out they're 17,
18. So their subject matter is that of a 17- or 18-year-old. Unless
you're \href{https://www.nytimes3xbfgragh.onion/topic/person/nas}{Nas},
and you like, well-read14... --- like, he was way more advanced with the
album that he wrote. So I just grew up a different way. But {[}West
is{]} a very compassionate person. And a lot of times he get in trouble
trying to help others. So I can identify with it. It's just that there's
certain things that happened that's not really acceptable to me.

14Reggie Ugwu: Recorded when he was still a teenager, Nas's landmark
1994 debut album, "Illmatic," was praised for its vivid, incisive lyrics
about coming of age among urban blight. A Queens native and autodidact,
he dropped out of school after the eighth grade.

BAQUET Right.

JAY-Z And we just need to speak about it. But there's genuine love
there.

BAQUET I'm trying to picture the scene when you and your wife both
talked about making these very confessional, open albums. Was it
difficult to say: "I'm gonna talk about the problems in our marriage.
I'm gonna talk about how we almost lost things." And for her to say:
"I'm gonna talk about my pain and anger at you." What were those
conversations like?

JAY-Z Again, it didn't --- it didn't happen in that way. It happened ---
we were using our art almost like a therapy session. And we started
making music together.

And then the
\href{https://www.nytimes3xbfgragh.onion/2016/04/25/arts/music/beyonce-unearths-pain-and-lets-it-flow-in-lemonade.html}{music
she was making} at that time was further along. So her album came out as
opposed to the joint album that we were working on. Um, we still have a
lot of that music. And this is what it became. There was never a point
where it was like, ``I'm making this album.'' I was right there the
entire time.

BAQUET And what was her reaction to your work and what was your reaction
to hers? They must have caused pain for each of you, right?

JAY-Z Of course. And both very, very uncomfortable, but {[}...{]} the
best place in the, you know, hurricane is like in the middle of it.

BAQUET Yeah.

JAY-Z We were sitting in the eye of that hurricane. Uh, maybe not use
hurricane because so many people are being affected right now. {[}This
interview took place nine days after
\href{https://www.nytimes3xbfgragh.onion/2017/09/21/us/hurricane-maria-puerto-rico.html}{Hurricane
Maria} made landfall in Puerto Rico, on the heels of two other
devastating hurricanes,
\href{https://www.nytimes3xbfgragh.onion/2017/09/07/us/irma-florida-coverage.html}{Irma}
and
\href{https://www.nytimes3xbfgragh.onion/2017/08/28/us/hurricane-harvey-texas.html}{Harvey},
that struck the U.S. mainland.{]}

BAQUET Yeah.

JAY-Z But the best place is right in the middle of the pain.

BAQUET Right.

JAY-Z And that's where we were sitting. And it was uncomfortable. And we
had a lot of conversations. You know. {[}I was{]} really proud of the
music she made, and she was really proud of the art I released. And, you
know, at the end of the day we really have a healthy respect for one
another's craft. I think she's amazing.

You know, most people walk away, and like divorce rate is like 50
percent or something 'cause most people can't see themselves. The
hardest thing is seeing pain on someone's face that you caused, and then
have to deal with yourself.

BAQUET Yeah.

JAY-Z So, you know, most people don't want to do that. You don't want to
look inside yourself.

BAQUET Yeah.

JAY-Z And so you walk away.

\href{http://www.nytimes3xbfgragh.onion/video/t-magazine/100000005574909/jayz-interview.html}{Watch
the 35-minute interview}

Produced by HILARY MOSS and ANDREW ROSSBACK. Video by NICK BENTGEN.

\hypertarget{more-on-nytimescom}{%
\subsection{More on NYTimes.com}\label{more-on-nytimescom}}

Advertisement

\hypertarget{site-information-navigation}{%
\subsection{Site Information
Navigation}\label{site-information-navigation}}

\begin{itemize}
\tightlist
\item
  \href{https://help.nytimes3xbfgragh.onion/hc/en-us/articles/115014792127-Copyright-notice}{©
  2020 The New York Times Company}
\item
  \href{https://www.nytimes3xbfgragh.onion}{Home}
\item
  \href{https://www.nytimes3xbfgragh.onion/search/}{Search}
\item
  Accessibility concerns? Email us at
  \href{mailto:accessibility@NYTimes.com}{\nolinkurl{accessibility@NYTimes.com}}.
  We would love to hear from you.
\item
  \href{https://help.nytimes3xbfgragh.onion/hc/en-us/articles/115015385887-Contact-Us}{Contact
  Us}
\item
  \href{https://www.nytco.com/careers/}{Work with us}
\item
  \href{https://nytmediakit.com/}{Advertise}
\item
  \href{https://help.nytimes3xbfgragh.onion/hc/en-us/articles/115014892108-Privacy-policy\#pp}{Your
  Ad Choices}
\item
  \href{https://help.nytimes3xbfgragh.onion/hc/en-us/articles/115014892108-Privacy-policy}{Privacy}
\item
  \href{https://help.nytimes3xbfgragh.onion/hc/en-us/articles/115014893428-Terms-of-service}{Terms
  of Service}
\item
  \href{https://help.nytimes3xbfgragh.onion/hc/en-us/articles/115014893968-Terms-of-sale}{Terms
  of Sale}
\end{itemize}

\hypertarget{site-information-navigation-1}{%
\subsection{Site Information
Navigation}\label{site-information-navigation-1}}

\begin{itemize}
\tightlist
\item
  \href{https://spiderbites.nytimes3xbfgragh.onion}{Site Map}
\item
  \href{https://help.nytimes3xbfgragh.onion/hc/en-us}{Help}
\item
  \href{https://help.nytimes3xbfgragh.onion/hc/en-us/articles/115015385887-Contact-Us?redir=myacc}{Site
  Feedback}
\item
  \href{https://www.nytimes3xbfgragh.onion/subscription?campaignId=37WXW}{Subscriptions}
\end{itemize}
