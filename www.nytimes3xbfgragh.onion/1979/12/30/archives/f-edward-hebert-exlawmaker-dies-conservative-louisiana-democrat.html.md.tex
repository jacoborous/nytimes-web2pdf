Sections

SEARCH

\protect\hyperlink{site-content}{Skip to
content}\protect\hyperlink{site-index}{Skip to site index}

\href{https://myaccount.nytimes3xbfgragh.onion/auth/login?response_type=cookie\&client_id=vi}{}

\href{https://www.nytimes3xbfgragh.onion/section/todayspaper}{Today's
Paper}

Archives\textbar{}F. EDWARD HEBERT, EX‐LAWMAKER, DIES

\url{https://nyti.ms/1PutwIa}

\begin{itemize}
\item
\item
\item
\item
\item
\end{itemize}

Advertisement

\protect\hyperlink{after-top}{Continue reading the main story}

Supported by

\protect\hyperlink{after-sponsor}{Continue reading the main story}

\hypertarget{f-edward-hebert-exlawmaker-dies}{%
\section{F. EDWARD HEBERT, EX‐LAWMAKER,
DIES}\label{f-edward-hebert-exlawmaker-dies}}

Dec. 30, 1979

\begin{itemize}
\item
\item
\item
\item
\item
\end{itemize}

\includegraphics{https://s1.graylady3jvrrxbe.onion/timesmachine/pages/1/1979/12/30/111213872_360W.png?quality=75\&auto=webp\&disable=upscale}

See the article in its original context from\\
December 30, 1979, Page
14\href{https://store.nytimes3xbfgragh.onion/collections/new-york-times-page-reprints?utm_source=nytimes\&utm_medium=article-page\&utm_campaign=reprints}{Buy
Reprints}

\href{http://timesmachine.nytimes3xbfgragh.onion/timesmachine/1979/12/30/111213872.html}{View
on timesmachine}

TimesMachine is an exclusive benefit for home delivery and digital
subscribers.

About the Archive

This is a digitized version of an article from The Times's print
archive, before the start of online publication in 1996. To preserve
these articles as they originally appeared, The Times does not alter,
edit or update them.

Occasionally the digitization process introduces transcription errors or
other problems; we are continuing to work to improve these archived
versions.

NEW ORLEANS, Dec. 29 (UPI) --- F. Edward Hebert, the Democratic
Congressman who served Louisiana's First District from 1941 until his
retirement in 1977, died at 5 P.M. today of ``massive heart failure,''
hospital officials said. He was 78 years old.

Mr. Hebert had been hospitalized at Hotel Dieu Hospital here for the
last two weeks. His condition was diagnosed as congestive heart failure.

\textbf{Staunch Pentagon Backer}

\textbf{By ROBERT McG. THOMAS Jr.}

In his 36 years in Congress, one of the longest tenures on record, Felix
Edward Hebert left few doubts about where he stood. A hard‐line
conservative, he established a virtually unblemished voting record in
support of a strong military and against civil rights legislation,
welfare programs and other issues he saw as evils of a permissive
society.

A longtime member of the House Armed Services Committee, he protested
against a ``no win'' military policy in the Korean War, and, taking the
same position in the Vietnam War, once told Secretary of Defense Robert
S. McNamara that the United States should send 4 million troops to
Vietnam if it would hasten victory.

He managed to win some plaudits from the antiwar movement in 1970,
however, when he conducted subcommittee hearings into the My Lai
incident --- in which American servicemen were accused of murdering
Vietnamese civilians --- and concluded that a massive and planned
``cover‐up'' had suppressed information about the killings.

After he succeeded L. Mendel Rivers as chairman of the Armed Services
Committee in 1971, he won respect for his fairness to political
opponents but earned such a reputation for enthusiastic support of
Pentagon policies and budget requests that he helped kindle a revolt by
young House liberals who ousted him from the post, along with three
other aging committee chairmen, in 1975.

\textbf{Early Criticism of Pentagon}

Long overshadowed by his friend Mr. Rivers (the two entered Congress the
same day in 1941), Mr. Hebert, pronounced ``A‐bear,'' was little known
outside Washington and his native Louisiana when he took over the
committee chairmanship after Mr. Rivers' death in 1970, but only because
the nation's political memory was shorter than the Hebert tenure.

In the criticism of his pro‐Pentagon posture, it was largely forgotten
that in the 1950's, as chairman of an Armed Services subcommittee, he
had been known as the scourge of the Pentagon, leading a well‐publicized
campaign against wasteful procurement practices. Conducting hearings
that he labeled ``my chamber of horrors,'' he gained headlines by
displaying identical items purchased by different Defense Department
units for widely varying prices and by his rhetorical pursuit of the
elusive ``Phantom of the Pentagon --- the little man in charge of
military purchases.''

``Whenever I try to get him before the committee,'' he complained,
``he's either just retired or left on a trip to the Far East.''

Mr. Hébert came by his flair for headlines honestly. A longtime
newspaperman, he parlayed a devastating exposé of the Huey Long
political machine into successful race for Congress as a reformer in
1940.

Mr. Hébert, whose father was a streetcar motorman, was born in New
Orleans on Oct. 12, 1901. He broke into the newspaper business by
covering sports while in high school and at Tulane University. In 1929
he became political editor of The New Orleans States and began a daily
front page column that regularly attacked Long, the Louisiana political
boss, and, after Mr. Long's assassination in 1935, his political heirs.

In 1939, after being promoted to city editor, Mr. Hebert broke the story
of political corruption that became known as the ``Louisiana scandals,''
leading to the jailing of many Long associates and triggering Mr.
Hebert's political career, reportedly at the urging of former Gov. James
A. Noe, who had broken with the Long machine and reportedly supplied the
tip that led to the expose.

Warned by friends against giving up safe job as city editor of The New
Orleans States, he reportedly told them that two years in Congress would
be valuable experience for a newsman, and that at any rate, he just
might win re‐election. He did --- 17 times, giving him a tenure matched
or exceeded by fewer than 20 of the more than 9,000 men and women who
have served in Congress since the nation's founding.

Once in Washington, Mr. Hebert soon established himself as a flamboyant
individualist whose life style and wardrobe all but obscured his
predictable conservative stand on issues. He acknowledged a; one point
that he owned 100 suits, 50 sports shirts, 30 pairs of shoes, two dozen
pairs of silk pajamas, 15 hats and six watches.

Despite a sharp tongue and an independent stance that kept him at odds
with the House Democratic leadership for much of his career, he combined
a cantankerous political posture with a jovial, courtly manner that
often charmed his enemies.

His candor, journalistic background and reputation as a phrase‐maker
made him a favorite of newspapermen, many I of whom were, nevertheless,
outraged in 1951 when he wrote an exclusive, graphic account of the
secret Eniwetok atom bomb test for his old newspaper after observing the
test --- so secret that reporters were barred --- as a member of the
Armed Services Committee. He joined the states rights Dixiecrat revolt
in 1948, the only member of the Louisiana delegation to the Democratic
National Convention to do so, kindling a feud with President Truman.

Associated Press

\textbf{F. Edward Hebert}

Advertisement

\protect\hyperlink{after-bottom}{Continue reading the main story}

\hypertarget{site-index}{%
\subsection{Site Index}\label{site-index}}

\hypertarget{site-information-navigation}{%
\subsection{Site Information
Navigation}\label{site-information-navigation}}

\begin{itemize}
\tightlist
\item
  \href{https://help.nytimes3xbfgragh.onion/hc/en-us/articles/115014792127-Copyright-notice}{©~2020~The
  New York Times Company}
\end{itemize}

\begin{itemize}
\tightlist
\item
  \href{https://www.nytco.com/}{NYTCo}
\item
  \href{https://help.nytimes3xbfgragh.onion/hc/en-us/articles/115015385887-Contact-Us}{Contact
  Us}
\item
  \href{https://www.nytco.com/careers/}{Work with us}
\item
  \href{https://nytmediakit.com/}{Advertise}
\item
  \href{http://www.tbrandstudio.com/}{T Brand Studio}
\item
  \href{https://www.nytimes3xbfgragh.onion/privacy/cookie-policy\#how-do-i-manage-trackers}{Your
  Ad Choices}
\item
  \href{https://www.nytimes3xbfgragh.onion/privacy}{Privacy}
\item
  \href{https://help.nytimes3xbfgragh.onion/hc/en-us/articles/115014893428-Terms-of-service}{Terms
  of Service}
\item
  \href{https://help.nytimes3xbfgragh.onion/hc/en-us/articles/115014893968-Terms-of-sale}{Terms
  of Sale}
\item
  \href{https://spiderbites.nytimes3xbfgragh.onion}{Site Map}
\item
  \href{https://help.nytimes3xbfgragh.onion/hc/en-us}{Help}
\item
  \href{https://www.nytimes3xbfgragh.onion/subscription?campaignId=37WXW}{Subscriptions}
\end{itemize}
