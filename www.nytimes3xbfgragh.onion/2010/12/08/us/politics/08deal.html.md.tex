Sections

SEARCH

\protect\hyperlink{site-content}{Skip to
content}\protect\hyperlink{site-index}{Skip to site index}

\href{https://www.nytimes3xbfgragh.onion/section/politics}{Politics}

\href{https://myaccount.nytimes3xbfgragh.onion/auth/login?response_type=cookie\&client_id=vi}{}

\href{https://www.nytimes3xbfgragh.onion/section/todayspaper}{Today's
Paper}

\href{/section/politics}{Politics}\textbar{}Biden and G.O.P. Leader
Helped Hammer Out Bipartisan Tax Accord

\begin{itemize}
\item
\item
\item
\item
\item
\end{itemize}

Advertisement

\protect\hyperlink{after-top}{Continue reading the main story}

Supported by

\protect\hyperlink{after-sponsor}{Continue reading the main story}

\hypertarget{biden-and-gop-leader-helped-hammer-out-bipartisan-tax-accord}{%
\section{Biden and G.O.P. Leader Helped Hammer Out Bipartisan Tax
Accord}\label{biden-and-gop-leader-helped-hammer-out-bipartisan-tax-accord}}

By \href{https://www.nytimes3xbfgragh.onion/by/carl-hulse}{Carl Hulse}
and \href{https://www.nytimes3xbfgragh.onion/by/jackie-calmes}{Jackie
Calmes}

\begin{itemize}
\item
  Dec. 7, 2010
\item
  \begin{itemize}
  \item
  \item
  \item
  \item
  \item
  \end{itemize}
\end{itemize}

WASHINGTON --- As a bipartisan group of Congressional tax writers opened
widely watched negotiations last Wednesday over what to do about
expiring tax cuts, a secret set of tax talks was taking place just steps
off the Senate floor.

Senator Mitch McConnell of Kentucky and a single staff member had
slipped into the ceremonial office of Vice President Joseph R. Biden Jr.
to try to hash out a compromise directly with the vice president, who
was accompanied by a top aide of his own, Ron Klain, his chief of staff.

The meeting was one of a number of direct conversations over the next
few days between Mr. Biden and Mr. McConnell, the Senate Republican
leader and a Senate colleague of Mr. Biden's for nearly 25 years, that
ultimately led to
\href{http://bucks.blogs.nytimes3xbfgragh.onion/2010/12/07/what-the-tax-deal-means-for-you/?ref=politics}{the
agreement} reached Monday. It was a bipartisan bargain that --- in a
startling departure from the past two years in the capital --- ended
with Republicans praising it and Democrats claiming they were blindsided
and undercut.

According to those knowledgeable about the events that played out over
less than a week, the agreement was the product of a fast-paced series
of telephone contacts, conference calls and consultations with
Congressional leaders. A critical negotiation on Sunday led to a
surprise cut in employee payroll taxes as the men sought to wrap up the
deal.

The final pieces came together Monday, when the vice president called
Mr. McConnell to inform him of the White House's price for accepting the
Republican plan to provide a generous exemption for taxing wealthy
estates. Mr. McConnell called Mr. Biden back late in the day to deliver
the Republican sign-off on the administration's proposed tax breaks for
low- and middle-income workers.

The intense back and forth suggests
\href{http://www.nytimes3xbfgragh.onion/2010/12/07/us/politics/07assess.html}{a
possible template} for how the two parties might interact next year when
Republicans control the House and have added to their numbers in the
Senate.

``There is still much left to be done for the American people in the
next two years, and I am hoping this won't be the last time we can do
something for the country on a bipartisan basis,'' Mr. McConnell said on
Tuesday, suggesting that divided government could provide an opening for
tackling such explosive subjects as entitlement spending.

With hefty Democratic majorities in Congress the past two years, the
White House has had little direct engagement with the Republican
leadership, trying instead to win over individual Republicans on a
bill-by-bill basis.

That situation seems to have shifted markedly since the midterm
elections, though the White House sought to portray Mr. Biden's role in
the tax negotiations as similar to his efforts to produce forward
movement on a nuclear arms reduction treaty and other issues that could
benefit from his nearly four decades in the Senate.

``He has from the beginning been an über liaison to his former
colleagues,'' one aide said.

But the depth of the personal negotiations between Mr. Biden and Mr.
McConnell was remarkable. It began with an unsolicited phone call from
the vice president to the Republican leader's office just hours after
the first post-election bipartisan meeting at the White House on Nov.
30.

By that session, according to administration officials, Mr. Obama had
decided not to side with those in his administration and among
Congressional Democrats who were spoiling to fight Republicans on the
Bush-era tax cuts for those with high incomes even though the Democrats
appeared to lack the votes in the Senate. Instead, he would test
Republicans' willingness to make concessions for economic stimulus
measures and ``the Obama tax cuts'' for low- and middle-income workers.
Then, if Republicans gave him the back of the hand, he would fight.

Mr. Obama was propelled to his decision in part by a Nov. 18 meeting
with Democratic Congressional leaders that persuaded him the Democrats
were not unified behind a realistic plan for moving forward.

So at the bipartisan leadership meeting last week, the White House and
Congressional leaders agreed to what became known as ``the six-pack''
talks, with two negotiators each to represent the White House,
Congressional Democrats and Congressional Republicans. At the same time,
Mr. Obama made an undisclosed move that proved more conclusive: He gave
a green light to Mr. Biden to pursue a parallel line of communication
with Mr. McConnell.

\includegraphics{https://static01.graylady3jvrrxbe.onion/images/2010/12/08/us/08deal_inline/DEAL-jumbo.jpg?quality=75\&auto=webp\&disable=upscale}

Still, administration officials insist the six-member negotiations were
pivotal as well, and early decisions on handling the alternative minimum
tax and a separate array of expired or expiring tax breaks, including
the credit for corporations' research and development costs, were worked
out in that group.

``There was never a time when that was a fake meeting and the real thing
was somewhere else,'' said an administration official. ``I know what
it's like to go to a meeting that's Kabuki theater; this wasn't like
that.''

The administration's original wish list included the 13 months of
extended federal unemployment compensation for people out of jobs for a
long time and the extended tax credits for the working poor, college
students and lower-income families with children. But the White House
was pessimistic that Republicans would go along with the tax credits for
lower-income people.

Indeed, the Republicans' resistance to Mr. Obama's signature tax cut ---
the ``Making Work Pay'' payroll tax reduction that was part of his
original 2009 economic stimulus package --- forced the administration
team to look for an alternative.

The late-hour substitute on Sunday was the proposal for a reduction of
two percentage points in employees' 6.2 percent Social Security payroll
tax for 2011. A payroll-tax holiday has been an idea on Mr. Obama's
table for months, but he and Congressional Democrats always pushed it
aside, given concerns that voters, especially older people, would see it
as taking revenues that are supposed to pay for Social Security
benefits.

But pushing the idea all along was Treasury Secretary Timothy F.
Geithner and his counselor Gene Sperling, a former top economic policy
adviser to President Bill Clinton. Last Friday, the report that the
unemployment rate had inched up to 9.8 percent gave new impetus to the
administration's push --- and to Mr. Biden's talks with Mr. McConnell.

By Sunday night, the two sides had agreed to a two-year extension of all
the Bush tax rates in return for the unemployment aid and a payroll-tax
holiday. The final negotiations came down to Republicans' demand for a
generous new estate-tax formulation --- and the White House's insistence
on extending the package of tax breaks for low- and middle-income
students, workers and families with children.

Those tax breaks came to be called ``the refundables'' because eligible
taxpayers would get a tax refund check for any amount that exceeded
their actual income-tax liability. Republicans generally oppose
refundable tax credits, considering it, in effect, welfare spending. But
they saw the talks as a golden opportunity to win an estate-tax
agreement that had eluded them even when they controlled Congress and
the White House.

On Monday morning, Mr. Biden met with Mr. Obama in the Oval Office
before the president left for a day trip to Winston-Salem, N.C., to
speak about education. Mr. Obama told him to give Mr. McConnell an
ultimatum.

``My strong instinct is that we make the deal if we can,'' the president
said. But, he added, to accept Republicans' estate tax break ``would be
too heavy a lift.''

Mr. Biden returned to his West Wing office and called Mr. McConnell.

``We will not do the estate tax without the other stuff,'' he told him,
according to officials. ``There's just no deal without the refundables.
Won't do it.''

Mr. McConnell did not call Mr. Biden back with an answer until 5 p.m.,
after consulting with other Republicans. ``We have the deal,'' Mr.
McConnell said.

By then Mr. Obama was already back from North Carolina, and Mr. Biden
went to the Oval Office to inform him. The president said, ``O.K., if
that's a deal we can live with, let's do it.'' Mr. Biden walked back to
his office to call Mr. McConnell.

``We're on,'' the vice president informed Mr. McConnell.

Advertisement

\protect\hyperlink{after-bottom}{Continue reading the main story}

\hypertarget{site-index}{%
\subsection{Site Index}\label{site-index}}

\hypertarget{site-information-navigation}{%
\subsection{Site Information
Navigation}\label{site-information-navigation}}

\begin{itemize}
\tightlist
\item
  \href{https://help.nytimes3xbfgragh.onion/hc/en-us/articles/115014792127-Copyright-notice}{©~2020~The
  New York Times Company}
\end{itemize}

\begin{itemize}
\tightlist
\item
  \href{https://www.nytco.com/}{NYTCo}
\item
  \href{https://help.nytimes3xbfgragh.onion/hc/en-us/articles/115015385887-Contact-Us}{Contact
  Us}
\item
  \href{https://www.nytco.com/careers/}{Work with us}
\item
  \href{https://nytmediakit.com/}{Advertise}
\item
  \href{http://www.tbrandstudio.com/}{T Brand Studio}
\item
  \href{https://www.nytimes3xbfgragh.onion/privacy/cookie-policy\#how-do-i-manage-trackers}{Your
  Ad Choices}
\item
  \href{https://www.nytimes3xbfgragh.onion/privacy}{Privacy}
\item
  \href{https://help.nytimes3xbfgragh.onion/hc/en-us/articles/115014893428-Terms-of-service}{Terms
  of Service}
\item
  \href{https://help.nytimes3xbfgragh.onion/hc/en-us/articles/115014893968-Terms-of-sale}{Terms
  of Sale}
\item
  \href{https://spiderbites.nytimes3xbfgragh.onion}{Site Map}
\item
  \href{https://help.nytimes3xbfgragh.onion/hc/en-us}{Help}
\item
  \href{https://www.nytimes3xbfgragh.onion/subscription?campaignId=37WXW}{Subscriptions}
\end{itemize}
