Sections

SEARCH

\protect\hyperlink{site-content}{Skip to
content}\protect\hyperlink{site-index}{Skip to site index}

\href{https://myaccount.nytimes3xbfgragh.onion/auth/login?response_type=cookie\&client_id=vi}{}

\href{https://www.nytimes3xbfgragh.onion/section/todayspaper}{Today's
Paper}

\href{/section/opinion}{Opinion}\textbar{}Ending the Slavery Blame-Game

\begin{itemize}
\item
\item
\item
\item
\item
\end{itemize}

Advertisement

\protect\hyperlink{after-top}{Continue reading the main story}

Supported by

\protect\hyperlink{after-sponsor}{Continue reading the main story}

\href{/section/opinion}{Opinion}

Op-Ed Contributor

\hypertarget{ending-the-slavery-blame-game}{%
\section{Ending the Slavery
Blame-Game}\label{ending-the-slavery-blame-game}}

By Henry Louis Gates Jr.

\begin{itemize}
\item
  April 22, 2010
\item
  \begin{itemize}
  \item
  \item
  \item
  \item
  \item
  \end{itemize}
\end{itemize}

Cambridge, Mass.

THANKS to an unlikely confluence of history and genetics --- the fact
that he is African-American and president --- Barack Obama has a unique
opportunity to reshape the debate over one of the most contentious
issues of America's racial legacy: reparations, the idea that the
descendants of American slaves should receive compensation for their
ancestors' unpaid labor and bondage.

There are many thorny issues to resolve before we can arrive at a
judicious (if symbolic) gesture to match such a sustained, heinous
crime. Perhaps the most vexing is how to parcel out blame to those
directly involved in the capture and sale of human beings for immense
economic gain.

While we are all familiar with the role played by the United States and
the European colonial powers like Britain, France, Holland, Portugal and
Spain, there is very little discussion of the role Africans themselves
played. And that role, it turns out, was a considerable one, especially
for the slave-trading kingdoms of western and central Africa. These
included the Akan of the kingdom of Asante in what is now Ghana, the Fon
of Dahomey (now Benin), the Mbundu of Ndongo in modern Angola and the
Kongo of today's Congo, among several others.

For centuries, Europeans in Africa kept close to their military and
trading posts on the coast. Exploration of the interior, home to the
bulk of Africans sold into bondage at the height of the slave trade,
came only during the colonial conquests, which is why Henry Morton
Stanley's pursuit of Dr. David Livingstone in 1871 made for such
compelling press: he was going where no (white) man had gone before.

How did slaves make it to these coastal forts? The historians John
Thornton and Linda Heywood of Boston University estimate that 90 percent
of those shipped to the New World were enslaved by Africans and then
sold to European traders. The sad truth is that without complex business
partnerships between African elites and European traders and commercial
agents, the slave trade to the New World would have been impossible, at
least on the scale it occurred.

Advocates of reparations for the descendants of those slaves generally
ignore this untidy problem of the significant role that Africans played
in the trade, choosing to believe the romanticized version that our
ancestors were all kidnapped unawares by evil white men, like Kunta
Kinte was in ``Roots.'' The truth, however, is much more complex:
slavery was a business, highly organized and lucrative for European
buyers and African sellers alike.

The African role in the slave trade was fully understood and openly
acknowledged by many African-Americans even before the Civil War. For
Frederick Douglass, it was an argument against repatriation schemes for
the freed slaves. ``The savage chiefs of the western coasts of Africa,
who for ages have been accustomed to selling their captives into bondage
and pocketing the ready cash for them, will not more readily accept our
moral and economical ideas than the slave traders of Maryland and
Virginia,''
\href{http://www.nytimes3xbfgragh.onion/books/first/g/gates-wonders.html?_r=1}{he
warned.} ``We are, therefore, less inclined to go to Africa to work
against the slave trade than to stay here to work against it.''

To be sure, the African role in the slave trade was greatly reduced
after 1807, when abolitionists, first in Britain and then, a year later,
in the United States, succeeded in banning the importation of slaves.
Meanwhile, slaves continued to be bought and sold within the United
States, and slavery as an institution would not be abolished until 1865.
But the culpability of American plantation owners neither erases nor
supplants that of the African slavers. In recent years, some African
leaders have become more comfortable discussing this complicated past
than African-Americans tend to be.

Image

Credit...Scott Bakal

In 1999, for instance, President Mathieu Kerekou of Benin astonished an
all-black congregation in Baltimore by falling to his knees and begging
African-Americans' forgiveness for the ``shameful'' and ``abominable''
role Africans played in the trade. Other African leaders, including
Jerry Rawlings of Ghana, followed Mr. Kerekou's bold example.

Our new understanding of the scope of African involvement in the slave
trade is not historical guesswork. Thanks to the
\href{http://www.slavevoyages.org/tast/index.faces}{Trans-Atlantic Slave
Trade Database}, directed by the historian David Eltis of Emory
University, we now know the ports from which more than 450,000 of our
African ancestors were shipped out to what is now the United States (the
database has records of 12.5 million people shipped to all parts of the
New World from 1514 to 1866). About 16 percent of United States slaves
came from eastern Nigeria, while 24 percent came from the Congo and
Angola.

Through the work of Professors Thornton and Heywood, we also know that
the victims of the slave trade were predominantly members of as few as
50 ethnic groups. This data, along with the tracing of blacks' ancestry
through DNA tests, is giving us a fuller understanding of the identities
of both the victims and the facilitators of the African slave trade.

For many African-Americans, these facts can be difficult to accept.
Excuses run the gamut, from ``Africans didn't know how harsh slavery in
America was'' and ``Slavery in Africa was, by comparison, humane'' or,
in a bizarre version of ``The devil made me do it,'' ``Africans were
driven to this only by the unprecedented profits offered by greedy
European countries.''

But the sad truth is that the conquest and capture of Africans and their
sale to Europeans was one of the main sources of foreign exchange for
several African kingdoms for a very long time. Slaves were the main
export of the kingdom of Kongo; the Asante Empire in Ghana exported
slaves and used the profits to import gold. Queen Njinga, the brilliant
17th-century monarch of the Mbundu, waged wars of resistance against the
Portuguese but also conquered polities as far as 500 miles inland and
sold her captives to the Portuguese. When Njinga converted to
Christianity, she sold African traditional religious leaders into
slavery, claiming they had violated her new Christian precepts.

Did these Africans know how harsh slavery was in the New World?
Actually, many elite Africans visited Europe in that era, and they did
so on slave ships following the prevailing winds through the New World.
For example, when Antonio Manuel, Kongo's ambassador to the Vatican,
went to Europe in 1604, he first stopped in Bahia, Brazil, where he
arranged to free a countryman who had been wrongfully enslaved.

African monarchs also sent their children along these same slave routes
to be educated in Europe. And there were thousands of former slaves who
returned to settle Liberia and Sierra Leone. The Middle Passage, in
other words, was sometimes a two-way street. Under these circumstances,
it is difficult to claim that Africans were ignorant or innocent.

Given this remarkably messy history, the problem with reparations may
not be so much whether they are a good idea or deciding who would get
them; the larger question just might be from whom they would be
extracted.

So how could President Obama untangle the knot? In David Remnick's new
book ``The Bridge: The Life and Rise of Barack Obama,'' one of the
president's former students at the University of Chicago comments on Mr.
Obama's mixed feelings about the reparations movement: ``He told us what
he thought about reparations. He agreed entirely with the \emph{theory}
of reparations. But in practice he didn't think it was really
workable.''

About the practicalities, Professor Obama may have been more right than
he knew. Fortunately, in President Obama, the child of an African and an
American, we finally have a leader who is uniquely positioned to bridge
the great reparations divide. He is uniquely placed to publicly
attribute responsibility and culpability where they truly belong, to
white people and black people, on both sides of the Atlantic, complicit
alike in one of the greatest evils in the history of civilization. And
reaching that understanding is a vital precursor to any just and lasting
agreement on the divisive issue of slavery reparations.

Advertisement

\protect\hyperlink{after-bottom}{Continue reading the main story}

\hypertarget{site-index}{%
\subsection{Site Index}\label{site-index}}

\hypertarget{site-information-navigation}{%
\subsection{Site Information
Navigation}\label{site-information-navigation}}

\begin{itemize}
\tightlist
\item
  \href{https://help.nytimes3xbfgragh.onion/hc/en-us/articles/115014792127-Copyright-notice}{©~2020~The
  New York Times Company}
\end{itemize}

\begin{itemize}
\tightlist
\item
  \href{https://www.nytco.com/}{NYTCo}
\item
  \href{https://help.nytimes3xbfgragh.onion/hc/en-us/articles/115015385887-Contact-Us}{Contact
  Us}
\item
  \href{https://www.nytco.com/careers/}{Work with us}
\item
  \href{https://nytmediakit.com/}{Advertise}
\item
  \href{http://www.tbrandstudio.com/}{T Brand Studio}
\item
  \href{https://www.nytimes3xbfgragh.onion/privacy/cookie-policy\#how-do-i-manage-trackers}{Your
  Ad Choices}
\item
  \href{https://www.nytimes3xbfgragh.onion/privacy}{Privacy}
\item
  \href{https://help.nytimes3xbfgragh.onion/hc/en-us/articles/115014893428-Terms-of-service}{Terms
  of Service}
\item
  \href{https://help.nytimes3xbfgragh.onion/hc/en-us/articles/115014893968-Terms-of-sale}{Terms
  of Sale}
\item
  \href{https://spiderbites.nytimes3xbfgragh.onion}{Site Map}
\item
  \href{https://help.nytimes3xbfgragh.onion/hc/en-us}{Help}
\item
  \href{https://www.nytimes3xbfgragh.onion/subscription?campaignId=37WXW}{Subscriptions}
\end{itemize}
