Sections

SEARCH

\protect\hyperlink{site-content}{Skip to
content}\protect\hyperlink{site-index}{Skip to site index}

\href{https://www.nytimes3xbfgragh.onion/section/us}{U.S.}

\href{https://myaccount.nytimes3xbfgragh.onion/auth/login?response_type=cookie\&client_id=vi}{}

\href{https://www.nytimes3xbfgragh.onion/section/todayspaper}{Today's
Paper}

\href{/section/us}{U.S.}\textbar{}For Most G.I.'s, Only Few Hints of
Hate Groups

\url{https://nyti.ms/29iQzHx}

\begin{itemize}
\item
\item
\item
\item
\item
\end{itemize}

Advertisement

\protect\hyperlink{after-top}{Continue reading the main story}

Supported by

\protect\hyperlink{after-sponsor}{Continue reading the main story}

\hypertarget{for-most-gis-only-few-hints-of-hate-groups}{%
\section{For Most G.I.'s, Only Few Hints of Hate
Groups}\label{for-most-gis-only-few-hints-of-hate-groups}}

By \href{https://www.nytimes3xbfgragh.onion/by/james-brooke}{James
Brooke}

\begin{itemize}
\item
  Dec. 21, 1995
\item
  \begin{itemize}
  \item
  \item
  \item
  \item
  \item
  \end{itemize}
\end{itemize}

\includegraphics{https://s1.graylady3jvrrxbe.onion/timesmachine/pages/1/1995/12/21/075175_360W.png?quality=75\&auto=webp\&disable=upscale}

See the article in its original context from\\
December 21, 1995, Section A, Page
14\href{https://store.nytimes3xbfgragh.onion/collections/new-york-times-page-reprints?utm_source=nytimes\&utm_medium=article-page\&utm_campaign=reprints}{Buy
Reprints}

\href{http://timesmachine.nytimes3xbfgragh.onion/timesmachine/1995/12/21/075175.html}{View
on timesmachine}

TimesMachine is an exclusive benefit for home delivery and digital
subscribers.

About the Archive

This is a digitized version of an article from The Times's print
archive, before the start of online publication in 1996. To preserve
these articles as they originally appeared, The Times does not alter,
edit or update them.

Occasionally the digitization process introduces transcription errors or
other problems; we are continuing to work to improve these archived
versions.

For Bob, a burly white Army sergeant, the underground presence of white
supremacists at this prairie post is as clear as the Ku Klux Klan and
Aryan Brotherhood graffiti he says he sees scrawled on portable toilets
on firing ranges.

"It's a natural for them to recruit among soldiers -\/- that's where the
guns are," the 12-year Army veteran said as music boomed and dancers
gyrated at an off-post strip joint. Motioning to his chest, he said,
"They might slip a few extra rounds in their pockets."

But across town, managers at Power Rentals South, the company that
maintains the post's portable toilets, said racist graffiti were rare.
Darrell Fox, the assistant manager, said, "Occasionally, you see a
swastika, but they are few and far between."

The presence of graffiti is only one clue the Army is looking for as it
embarks on a worldwide hunt for white supremacists in its 510,000-member
ranks. The inquiry, announced on Dec. 12 after two white soldiers were
charged in the slayings of a black man and woman in North Carolina, has
yet to yield results.

But in conversations with reporters at bases around the country, Army
officers and enlisted personnel mentioned only a few dark hints of
hatred like the elusive graffiti. For the most part, they supported the
Pentagon position that few soldiers actively participate in white
supremacist groups.

"People are not going to come up to you and say, 'Hey, we're neo-Nazis,
we're with the Klan' -\/- it's subtler than that," Dennis Morgan, an
Army specialist from New Jersey, said this week at Fort Benning, Ga.
"The Army does its best to screen out the paramilitary and the
right-wing types. But occasionally, somebody's going to hook up with the
wrong people."

Lieut. Gen. Anthony C. Zinni, commander of six Marine Corps bases in
Southern California and Arizona, said that the fight against white
supremacy recruitment should become part of military routine, similar to
the military's ongoing fight against racism and drug use, including
sensitivity training.

"We haven't seen it where we are, but I'm really concerned about it,"
General Zinni told reporters on Thursday. "We went through this in the
early 1970's with the racial efforts. It's never fixed. It's like drugs.
It's continuous. You have to stay on it."

Human rights groups welcomed the investigation, which the Pentagon said
would focus on violence-prone groups of long standing like the Klan,
skinheads and neo-Nazis. But they said the Army should expand its
inquiry to the far-right threat that gained attention after the Oklahoma
City bombing last April: anti-Government paramilitary groups.

Army troops are prime targets for recruitment by such groups, said Joe
T. Roy, director of the Militia Task Force for the Southern Poverty Law
Center.

"They preach survivalism, training, stockpiling weapons," Mr. Roy said.
"The paramilitary mindset is attractive to veterans and active-duty
personnel."

In June, Francisco Duran, an Army veteran who lived about 10 miles south
of Fort Carson, was sentenced to 40 years in Federal prison for raking
the White House with bullets from a Chinese assault rifle last year. Mr.
Duran reportedly had attended meetings of a local group, the Save
America Militia.

As officers question soldiers, check personnel records and search
barracks and off-post housing for neo-Nazi paraphernalia, they are
driven by the Army's failure to act on Nazi leanings exhibited in the
last 10 months by Pvt. James N. Burmeister 2d, a 20-year-old member of
the Army's elite Special Forces who was stationed at Fort Bragg, N.C.

In February, Private Burmeister was reprimanded for wearing a Nazi
medallion under his uniform. In August, he lost his security clearance
after fighting with a black soldier. If his commanding officer had
inspected a room he rented in an off-post trailer, he would have found
Nazi flags, skinhead magazines and books on Hitler.

Today, Private Burmeister and a fellow soldier stand charged with
murdering a black couple in Fayetteville, N.C., near Fort Bragg, on Dec.
7; another soldier faces related charges. Last week, Army investigators
seeking to identify other skinheads in uniform showed Fayetteville
nightclub managers photographs of Private Burmeister and friends giving
Nazi salutes.

In the early 1990's, Timothy J. McVeigh, then a soldier stationed at
Fort Riley, Kan., encouraged his Army comrades to read "The Turner
Diaries," a novel by William Pierce that advocates armed attacks on the
Government. Earlier this year Mr. McVeigh was charged with masterminding
the Oklahoma City bombing.

At the same time, the National Alliance, a white supremacist group led
by Mr. Pierce that is based in Hillsboro, W.Va., posted a recruiting
notice on a billboard outside Fort Bragg.

Despite such cases, military commanders say that white supremacist
recruiting is a small problem and maintain that the Army, whose
membership is nearly 40 percent minority, has a proud record of racial
integration.

Indeed, since 1990, only a handful of American military personnel have
been publicly linked to white supremacist acts -\/- largely stealing
weapons for groups or joining their clandestine ranks, according to the
Anti-Defamation League of New York and Klanwatch of Montgomery, Ala. In
1990, five Air Force policemen were discharged for their involvement
with the Ku Klux Klan at Carswell Air Force Base in Texas.

But Abraham H. Foxman, national director of the Anti-Defamation League,
said such statistics did not tell the whole story. "It's a closed
society, so, therefore, a lot of incidents of a racist nature don't
surface," he said. "It's only when there is such a grotesque incident as
these murders that the public focuses on it."

And a few soldiers say there has been a long history of looking away.

Last week a marine based at Camp Pendleton, Calif., recalled
accompanying a group of marines to Tijuana, Mexico. Tattooed and dressed
in skinhead attire, they drank and chanted racist slogans on a town
square, said the soldier, who would give his name only as Josh. The
incident apparently never came to the attention of the marines'
commanders.

Fort Carson, whose 17,000 soldiers make it Colorado's largest military
installation, seems to fit the national profile: a post with little
visible white supremacist activity.

Searching back to 1988, military officials found no complaint against a
soldier for white supremacist activity in the criminal files of Fort
Carson or of the El Paso County police.

The only entry in police files was a tip in 1990 that a soldier
stationed at Fort Carson was seen fraternizing with skinheads in
Boulder, 90 miles north of here. The tip was noted in a 1992 Pentagon
report on gang participation within the Army, but did not result in
disciplinary action.

Around Fort Carson's commercial fringe of gun shops, pawn shops, used
car lots and strip joints, random interviews generally supported the
perception that few soldiers at the post were involved in extremist
activity.

At Holey Rollers, a tattoo salon, Scott Toy said requests for white
power symbols like swastikas or lightning bolts were rare. "In three
years," he said, "I think I have only gotten three requests for racial
tattoos -\/- and they weren't military."

Military officials say that Army policy, heavy training schedules and
frequent reassignments help to curb soldiers from joining extremist
groups off base. For example, since July, the 3,500 soldiers of the
Third Brigade Combat Team, which is based here, followed this deployment
itinerary: Trinidad; Colorado; California; Guantanamo Bay, Cuba; Canada,
and Kuwait.

"Joe, the military guy, is going to be out of here in two years," said
Maj. Timothy P. Edinger, the Fort Carson spokesman. "It's much more
interesting for a group to recruit Joe, the hardware store guy -\/- he's
in the National Guard and part of the community."

The armed forces do not allow "active" participation in extremist groups
-\/- propagandizing, collecting money, attending rallies or leading a
cell. "Passive" participation like membership or receiving literature is
tolerated.

Today the armed forces' biggest gray area involves the new right-wing
phenomenon, the paramilitary groups. With 20 such groups, Colorado ranks
third in the nation, after Michigan with 30 and California with 22,
according to the Southern Poverty Law Center.

Officials at Fort Carson said they did not know whether the El Paso
County paramilitary group attracted any soldiers here.

Ten miles east of Fort Carson, down a dirt road past trailers and
junkyards, Mel Bernstein presides over Dragon Arms -\/- a gun lover's
paradise that includes a firing range, a paintball range, a military
vehicles museum and the county's only store licensed to sell automatic
weapons.

"The El Paso County Militia is my best customer," said Mr. Bernstein,
who still mourns the day last April when Federal agents raided his
complex and confiscated a Vietnam War-era armored personnel carrier. Mr.
Bernstein, who comes from Brooklyn and is a former tank commander, was
vague when asked if Fort Carson soldiers participated. "It's all kinds
of people -\/- firemen, family people," he said.

But the owner of a gun shop in Boulder said he was certain that some
military personnel took part in activities of local paramilitary groups.

"There is no doubt in my mind that you have active-duty people
involved," said Bob Glass, another New York native. "I've gone to a few
militia meetings. There are young men with very short haircuts, a
military stature to them. Are they active duty? You tell me."

Advertisement

\protect\hyperlink{after-bottom}{Continue reading the main story}

\hypertarget{site-index}{%
\subsection{Site Index}\label{site-index}}

\hypertarget{site-information-navigation}{%
\subsection{Site Information
Navigation}\label{site-information-navigation}}

\begin{itemize}
\tightlist
\item
  \href{https://help.nytimes3xbfgragh.onion/hc/en-us/articles/115014792127-Copyright-notice}{©~2020~The
  New York Times Company}
\end{itemize}

\begin{itemize}
\tightlist
\item
  \href{https://www.nytco.com/}{NYTCo}
\item
  \href{https://help.nytimes3xbfgragh.onion/hc/en-us/articles/115015385887-Contact-Us}{Contact
  Us}
\item
  \href{https://www.nytco.com/careers/}{Work with us}
\item
  \href{https://nytmediakit.com/}{Advertise}
\item
  \href{http://www.tbrandstudio.com/}{T Brand Studio}
\item
  \href{https://www.nytimes3xbfgragh.onion/privacy/cookie-policy\#how-do-i-manage-trackers}{Your
  Ad Choices}
\item
  \href{https://www.nytimes3xbfgragh.onion/privacy}{Privacy}
\item
  \href{https://help.nytimes3xbfgragh.onion/hc/en-us/articles/115014893428-Terms-of-service}{Terms
  of Service}
\item
  \href{https://help.nytimes3xbfgragh.onion/hc/en-us/articles/115014893968-Terms-of-sale}{Terms
  of Sale}
\item
  \href{https://spiderbites.nytimes3xbfgragh.onion}{Site Map}
\item
  \href{https://help.nytimes3xbfgragh.onion/hc/en-us}{Help}
\item
  \href{https://www.nytimes3xbfgragh.onion/subscription?campaignId=37WXW}{Subscriptions}
\end{itemize}
