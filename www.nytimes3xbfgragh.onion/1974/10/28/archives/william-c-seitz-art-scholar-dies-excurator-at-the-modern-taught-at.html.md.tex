Sections

SEARCH

\protect\hyperlink{site-content}{Skip to
content}\protect\hyperlink{site-index}{Skip to site index}

\href{https://myaccount.nytimes3xbfgragh.onion/auth/login?response_type=cookie\&client_id=vi}{}

\href{https://www.nytimes3xbfgragh.onion/section/todayspaper}{Today's
Paper}

Archives\textbar{}WILLIAM C. SEITZ, ART SCHOLAR, DIES

\url{https://nyti.ms/1XVwsAC}

\begin{itemize}
\item
\item
\item
\item
\item
\end{itemize}

Advertisement

\protect\hyperlink{after-top}{Continue reading the main story}

Supported by

\protect\hyperlink{after-sponsor}{Continue reading the main story}

\hypertarget{william-c-seitz-art-scholar-dies}{%
\section{WILLIAM C. SEITZ, ART SCHOLAR,
DIES}\label{william-c-seitz-art-scholar-dies}}

Oct. 28, 1974

\begin{itemize}
\item
\item
\item
\item
\item
\end{itemize}

\includegraphics{https://s1.graylady3jvrrxbe.onion/timesmachine/pages/1/1974/10/28/80432533_360W.png?quality=75\&auto=webp\&disable=upscale}

See the article in its original context from\\
October 28, 1974, Page
34\href{https://store.nytimes3xbfgragh.onion/collections/new-york-times-page-reprints?utm_source=nytimes\&utm_medium=article-page\&utm_campaign=reprints}{Buy
Reprints}

\href{http://timesmachine.nytimes3xbfgragh.onion/timesmachine/1974/10/28/80432533.html}{View
on timesmachine}

TimesMachine is an exclusive benefit for home delivery and digital
subscribers.

About the Archive

This is a digitized version of an article from The Times's print
archive, before the start of online publication in 1996. To preserve
these articles as they originally appeared, The Times does not alter,
edit or update them.

Occasionally the digitization process introduces transcription errors or
other problems; we are continuing to work to improve these archived
versions.

Dr. William Chapin Seitz, art scholar, professor and former curator at
the Museum of Modern Art here, died Saturday night of cancer at Martha
Jefferson Hospital in Charlottesville, Va. He was 60 years old.

An inspirational teacher and an imaginative, innovative exhibitor, Dr.
Seitz received perhaps his widest recognition for the trend‐setting
collections he arranged during his five years at the museum.

In one such display devoted to construction, he achieved something of a
blend, a unified whole, from the seemingly disparate mediums of
sculpture and collage. Equally influential, some critics say, was his
work with optical. (op) art.

Pop art and abtractionism were other major fields of concentration for
him. Yet, Dr. Seitz was also at ease in dealing with the realism of
Edward Hopper and the impressionism of Claude Monet.

In his early years of teaching at Princeton University, he found a
condescending attitude toward contemporary American art. Yet he resisted
it and became the first member of the faculty to create a firm interest
in the field among the better art students.

\textbf{Critic in Residence}

An amiable, enthusiastic man, Dr. Seitz enjoyed an easy rapport with his
students in the eight years he spent at Princeton from 1952 to 1960 as
critic in residence and assistant professor.

Much of Dr. Seitz's early life was spent in Buffalo. He was born there
on June 19, 1914, and studied at the Albright Art School and Art
Institute of Buffalo during his late teens and early 20's. For three
years in the mid‐nineteen‐thirties, he worked here for the Federal Art
Project and exhibited some one‐man shows. Then he returned to Buffalo
and worked as a draftsman and engineer for a rubber company while
pursuing studies for a bachelor's degree in fine arts, which he received
in 1946 from the University of Buffalo.

Afterward he taught at the university Afterward, at the State Teacher's
College there until 1949; took a year's study in Europe; and began
studies at Princeton in 1950 for a master's degree, which he received in
1952. Three years later, he won a doctorate.

\textbf{Became Associate Curator}

After, leaving Princeton in 1960, Dr. Seitz became Associate Curator of
Painting and Sculpture Exhibitions at the Museum of Modern Art. His
first major effort was an exhibition of more than 100 of Monet's
landscapes, called ``Claude Monet: Seasons and Moments.''

Other important collections Dr. Seitz arranged were by two abstract
artists, Mark Tobey (1962) and Hans Hofmann (1963); and 45 pop art
pieces and 39 works by Edward Hopper from 1913 to 1965 that made up the
United States exhibition at the ninth Bienal de São Paulo, in Brazil, in
1967.

By than year, Dr. Seitz had already been serving for two years as a fine
arts professor and director of the Rose Art Museum at Brandeis
University. In 1970, he became the William R. Kenan Jr. Professor of Art
History at the University of Virginia at Charlottesville. He was also a
research fellow in 1971 and 1972 at the National Gaiety of Art in
Washington.

Surviving is his widow, the former Irma J. Seigelman.

Advertisement

\protect\hyperlink{after-bottom}{Continue reading the main story}

\hypertarget{site-index}{%
\subsection{Site Index}\label{site-index}}

\hypertarget{site-information-navigation}{%
\subsection{Site Information
Navigation}\label{site-information-navigation}}

\begin{itemize}
\tightlist
\item
  \href{https://help.nytimes3xbfgragh.onion/hc/en-us/articles/115014792127-Copyright-notice}{©~2020~The
  New York Times Company}
\end{itemize}

\begin{itemize}
\tightlist
\item
  \href{https://www.nytco.com/}{NYTCo}
\item
  \href{https://help.nytimes3xbfgragh.onion/hc/en-us/articles/115015385887-Contact-Us}{Contact
  Us}
\item
  \href{https://www.nytco.com/careers/}{Work with us}
\item
  \href{https://nytmediakit.com/}{Advertise}
\item
  \href{http://www.tbrandstudio.com/}{T Brand Studio}
\item
  \href{https://www.nytimes3xbfgragh.onion/privacy/cookie-policy\#how-do-i-manage-trackers}{Your
  Ad Choices}
\item
  \href{https://www.nytimes3xbfgragh.onion/privacy}{Privacy}
\item
  \href{https://help.nytimes3xbfgragh.onion/hc/en-us/articles/115014893428-Terms-of-service}{Terms
  of Service}
\item
  \href{https://help.nytimes3xbfgragh.onion/hc/en-us/articles/115014893968-Terms-of-sale}{Terms
  of Sale}
\item
  \href{https://spiderbites.nytimes3xbfgragh.onion}{Site Map}
\item
  \href{https://help.nytimes3xbfgragh.onion/hc/en-us}{Help}
\item
  \href{https://www.nytimes3xbfgragh.onion/subscription?campaignId=37WXW}{Subscriptions}
\end{itemize}
