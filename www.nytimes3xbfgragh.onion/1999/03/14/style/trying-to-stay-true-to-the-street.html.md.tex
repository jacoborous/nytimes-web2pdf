Sections

SEARCH

\protect\hyperlink{site-content}{Skip to
content}\protect\hyperlink{site-index}{Skip to site index}

\href{https://www.nytimes3xbfgragh.onion/section/style}{Style}

\href{https://myaccount.nytimes3xbfgragh.onion/auth/login?response_type=cookie\&client_id=vi}{}

\href{https://www.nytimes3xbfgragh.onion/section/todayspaper}{Today's
Paper}

\href{/section/style}{Style}\textbar{}Trying To Stay True to The Street

\begin{itemize}
\item
\item
\item
\item
\item
\end{itemize}

Advertisement

\protect\hyperlink{after-top}{Continue reading the main story}

Supported by

\protect\hyperlink{after-sponsor}{Continue reading the main story}

\hypertarget{trying-to-stay-true-to-the-street}{%
\section{Trying To Stay True to The
Street}\label{trying-to-stay-true-to-the-street}}

By \href{https://www.nytimes3xbfgragh.onion/by/leslie-kaufman}{Leslie
Kaufman}

\begin{itemize}
\item
  March 14, 1999
\item
  \begin{itemize}
  \item
  \item
  \item
  \item
  \item
  \end{itemize}
\end{itemize}

See the article in its original context from\\
March 14, 1999, Section 9, Page
1\href{https://store.nytimes3xbfgragh.onion/collections/new-york-times-page-reprints?utm_source=nytimes\&utm_medium=article-page\&utm_campaign=reprints}{Buy
Reprints}

\href{http://timesmachine.nytimes3xbfgragh.onion/timesmachine/1999/03/14/731145.html}{View
on timesmachine}

TimesMachine is an exclusive benefit for home delivery and digital
subscribers.

WHEN Daymond John, the co-founder and chief executive of Fubu hip-hop
clothing, drives through his old neighborhood in Hollis, Queens, he
leaves no doubt that he has moved on -\/- and up.

His Bentley, royal blue with a matching pinstripe on the cream-color
upholstery and so new that plastic wrap still covers the wood paneling,
makes heads spin in this working-class enclave. Mr. John shook his head
when he saw two women walking by to inspect the car. ''I haven't brought
it around here before because it makes us look unattainable,'' he
explained.

Appearing attainable is no small concern. Fubu stands for ''For us, by
us.'' The company, which was founded in 1992, has prospered by promoting
itself as a black-owned label by and for young urban men. The founders,
who also include Carl Brown, Keith Perrin and J. Alexander Martin, all
from Hollis, insist that ''For us, by us'' is not a call to racial
exclusiveness, but clearly race is part of their appeal. ''Customers see
our pictures and can relate, whereas they couldn't relate to some
45-year-old man living in Italy,'' Mr. John, 30, said. ''We are just
like them. We are the consumer.''

In other words, Fubu's origins give it street credibility. And these
days, street credibility translates into sales. The company says its
revenues rocketed nearly tenfold, from \$40 million in 1997 to a
stunning \$350 million last year, which catapults it beyond a niche
market into a league with labels like Donna Karan (\$670 million) and
Tommy Hilfiger (\$847 million.)

But because Fubu's success is so dependent on its homeboy appeal, this
wow-wow growth carries risks. If the company is going to expand as much
as Mr. John and his partners would like -\/- and in the last year
they've added everything from backpacks to a men's suit line -\/- it
will have to do even more business with middle-class, mainstream
America. The problem is that nothing destroys the urban buzz of a label
quite so fast as having Westchester teen-agers clamor for it. Tony
Shellman, vice president of Enyce, a Fubu competitor, predicts, ''Fubu
is going to lose its street essence as little Johnny in suburbia starts
picking up its stuff.''

For now, the brand is having an it moment. Not only is the rapper LL
Cool J, a company spokesman, seen wearing Fubu designs (he sports Fubu's
signature oversize football and hockey shirts), but also stars like
Janet Jackson, Will Smith and Whitney Houston. At night, Mr. John and
his partners make the scene with style players like Sean (Puffy) Combs,
the rap impresario. They recently received an award for black economic
enterprise from Mayor Rudolph W. Giuliani. Mr. John, who wore a black,
sueded rubber Fubu track suit to Gracie Maison to accept it, took the
attention in stride. ''Just business,'' he explained.

Coming up fast, however, are copycat competitors who say they have a
better bead on the street's needs than Fubu. These days it seems that
every other rap and inner-city entrepreneur is hawking his own line of
baggy fashions.

At some of Manhattan's most fashion-conscious emporiums of urban gear,
Fubu is already losing ground to challengers like Phat Farm, Mecca USA
and Enyce (pronounced en-EE-chay). At Transit, today's must
stop-and-shop, on Lower Broadway, Nikki Abreu, a manager, said: ''We
dont even carry Fubu anymore. It wasn't selling that much. Enyce is the
hottest.''

Because Fubu is privately owned, it does not have to publish its sales
figures, and, in fact, next year it won't. Some analysts speculate that
it is a sign of trouble. But Mr. John and his partners don't appear to
be panicked by their rivals. Short, thick and supremely confident, Mr.
John is the de facto company spokesman and persona. He brags that Fubu
has a far closer relationship with customers than other clothing labels
have.

''How many companies do you know that get 5,000 E-mail messages a day?''
he asked. The E-mail often includes detailed design suggestions. Some of
those, Mr. John said, like making a soft leather jacket covered with
small embroidered patterns, have been followed up by Fubu.

Mr. John is too smooth to say so, but he clearly keeps close tabs on how
his company is faring on the streets. He does not want his Bentley
mentioned in this article, for example, because ''people might get the
wrong idea.''

The founders also keep a high profile in the old neighborhood, doing
good works like handing out Thanksgiving turkeys and filling the
Christmas wish lists of needy local children.

In the end, the Fubu men rely on one another to stay grounded. All of
their clothes, whether designed in the house by Mr. Martin or by outside
licensees, have to be approved by three of the four partners. They make
a wide range of products, from the core men's casual line, emphasizing
oversize cuts and bright colors, to footwear and lounge wear (think red
satin robes a la Hugh Hefner).

''If we all like it, it'll probably sell,'' Mr. John said confidently.
Whatever the changing public perception, he believes that he and his
partners are too rooted in their community to be cut off from it by a
little flash and money.

Their life story, as he tells it, is a Horatio Alger tale updated for
the VH-1 age. In 1992, Mr. John was waiting tables at a local Red
Lobster and working any get-rich scheme he could think of. These
included, as he told Vibe magazine this month, dealing crack cocaine.
''It wasn't until I started the company that I didn't have something
crooked going on,'' he was quoted as saying by Vibe.

Although being known as a former drug dealer may enhance a rap star's
''street cred,'' it is no asset for a businessman whose clothes are
manufactured and distributed by Samsung, the Korean electronics giant.
Mr. John declined to comment on the Vibe article. A company spokeswoman,
Leslie Short, said, ''Those remarks were taken out of context, and
frankly I am very angry.''

Fubu began modestly enough. The fashion rage at the time was wool hats
with their tops cut off and tied with string. When Mr. John saw the hats
being sold for \$20, he corraled a next-door neighbor, Carl Brown, to
help sew up a bunch. The two sold their homemade headwear for \$10 a pop
in front of the New York Coliseum. They made \$800 in a single day, and
a clothing label was born.

There were years of struggle as Mr. John and Mr. Brown (whose job now is
to handle licensees), joined by Mr. Martin and Mr. Perrin (who now
manages celebrity relations), moved into Mr. John's mother's house and
sewed logos on hockey jerseys and sweatshirts. Their first break came in
1993, when they convinced LL Cool J, a double-platinum rapper, who grew
up a just few blocks from Mr. John's home, to pose in a Fubu T-shirt. It
took months more of cajoling and sitting in on video shoots to persuade
him to wear the line while performing. When he did, Fubu started taking
off. The rapper, a paid spokesman for the company since 1996, managed to
pull off a perfect bit of insider subversion by wearing a Fubu hat in an
advertisement he did for the Gap a couple of years ago.

Mr. John says the company also found motivation in what the partners
perceived as efforts by white-owned outdoor apparel brands like
Timberland and North Face, popular with inner-city youths, to distance
themselves from the customers who were making them street chic.

Fubu's marketing played off a sense of shared resentment at being
ignored. Gritty pictures of Fubu's founders, posed with exposed tattoos,
reversed baseball caps and a hint of gang-style menace, started popping
up in magazine advertisements and even on the company's labels. The
message was clear enough to the initiated.

''I knew what they were talking about, and it made me happy,'' said
Tracii McGregor, executive editor of The Source, a magazine on hip-hop
culture. ''The way they are marketing is unmistakably black.''

But gangsta style has it limitations, said Irma Zandl, president of a
youth marketing consulting firm in New York. ''There is something about
their premise that is a little alienating for nonblacks,'' she
continued. ''I've had white kids say to me, 'They don't want me to wear
this.' ''

David Watkins, president and chief executive of Icon Lifestyle
Marketing, a marketing company that focuses on urban areas, said that
even if this were true, concessions of any type to the suburban market
would be the wrong way to go. So far, Fubu has done well, he said,
because ''they have let people outside the market come to them, instead
of the other way around.''

He offered Fubu a bit of Shakespearean advice: ''They just have to keep
their marketing message the same -\/- and be true to themselves.''

Advertisement

\protect\hyperlink{after-bottom}{Continue reading the main story}

\hypertarget{site-index}{%
\subsection{Site Index}\label{site-index}}

\hypertarget{site-information-navigation}{%
\subsection{Site Information
Navigation}\label{site-information-navigation}}

\begin{itemize}
\tightlist
\item
  \href{https://help.nytimes3xbfgragh.onion/hc/en-us/articles/115014792127-Copyright-notice}{©~2020~The
  New York Times Company}
\end{itemize}

\begin{itemize}
\tightlist
\item
  \href{https://www.nytco.com/}{NYTCo}
\item
  \href{https://help.nytimes3xbfgragh.onion/hc/en-us/articles/115015385887-Contact-Us}{Contact
  Us}
\item
  \href{https://www.nytco.com/careers/}{Work with us}
\item
  \href{https://nytmediakit.com/}{Advertise}
\item
  \href{http://www.tbrandstudio.com/}{T Brand Studio}
\item
  \href{https://www.nytimes3xbfgragh.onion/privacy/cookie-policy\#how-do-i-manage-trackers}{Your
  Ad Choices}
\item
  \href{https://www.nytimes3xbfgragh.onion/privacy}{Privacy}
\item
  \href{https://help.nytimes3xbfgragh.onion/hc/en-us/articles/115014893428-Terms-of-service}{Terms
  of Service}
\item
  \href{https://help.nytimes3xbfgragh.onion/hc/en-us/articles/115014893968-Terms-of-sale}{Terms
  of Sale}
\item
  \href{https://spiderbites.nytimes3xbfgragh.onion}{Site Map}
\item
  \href{https://help.nytimes3xbfgragh.onion/hc/en-us}{Help}
\item
  \href{https://www.nytimes3xbfgragh.onion/subscription?campaignId=37WXW}{Subscriptions}
\end{itemize}
