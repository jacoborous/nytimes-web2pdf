Sections

SEARCH

\protect\hyperlink{site-content}{Skip to
content}\protect\hyperlink{site-index}{Skip to site index}

\href{https://myaccount.nytimes3xbfgragh.onion/auth/login?response_type=cookie\&client_id=vi}{}

\href{https://www.nytimes3xbfgragh.onion/section/todayspaper}{Today's
Paper}

\href{/section/opinion}{Opinion}\textbar{}What if You Couldn't See This
Page?

\url{https://nyti.ms/2jTUIGZ}

\begin{itemize}
\item
\item
\item
\item
\item
\end{itemize}

Advertisement

\protect\hyperlink{after-top}{Continue reading the main story}

Supported by

\protect\hyperlink{after-sponsor}{Continue reading the main story}

\href{/section/opinion}{Opinion}

Op-Ed Contributor

\hypertarget{what-if-you-couldnt-see-this-page}{%
\section{What if You Couldn't See This
Page?}\label{what-if-you-couldnt-see-this-page}}

By Nick Frisch

\begin{itemize}
\item
  Dec. 14, 2017
\item
  \begin{itemize}
  \item
  \item
  \item
  \item
  \item
  \end{itemize}
\end{itemize}

\href{https://cn.nytimes3xbfgragh.onion/opinion/20171215/net-neutrality-china-internet/}{阅读简体中文版}\href{https://cn.nytimes3xbfgragh.onion/opinion/20171215/net-neutrality-china-internet/zh-hant/}{閱讀繁體中文版}

\includegraphics{https://static01.graylady3jvrrxbe.onion/images/2017/12/13/opinion/13Frisch/13Frisch-articleLarge.jpg?quality=75\&auto=webp\&disable=upscale}

\emph{\textbf{Update:}} \emph{The F.C.C.}
\href{https://www.nytimes3xbfgragh.onion/2017/12/14/technology/net-neutrality-repeal-vote.html?_r=0}{\emph{voted
on Thursday to repeal net neutrality rules}}\emph{.}

HONG KONG --- To taste a future without net neutrality, try browsing the
web in Beijing. China's internet, provided through telecom giants
aligned with the Communist Party, is a digital dystopia, filtered by the
vast censorship apparatus known as China's Great Firewall. Some sites
load with soul-withering slowness, or not at all. Others appear
instantly. Content vanishes without warning or explanation. The culprit
is rarely knowable. A faulty Wi-Fi router? A neighborhood power failure?
Commercial sabotage? A clampdown on political dissent? To most Chinese
netizens, the reason matters little. They simply gravitate to the few
sites that aren't slowed or blocked entirely: the Chinese counterparts
of Facebook, Google, and Twitter. But these Chinese platforms come with
heavy government surveillance and censorship by corporate and party
apparatchiks. For the Communist Party and its commercial allies, this is
win-win, cementing respective monopolies on political markets and
consumer power.

The Trump administration's plan to dismantle net neutrality regulations
has brought this nightmare scenario to America's digital doorstep. With
the Federal Communications Commission scheduled to vote on the issue
today, the threatened rollback not only imperils fair play and free
speech; it will also empower foreign entities with substantial
market-making power, like China's government, to meddle in American
public discourse on a scale dwarfing Russia's recent cyber-chicanery.
Worse, abolishing net neutrality gives American corporations the means,
motive and opportunity to become accomplices in selling out our freedom
of speech.

Net neutrality is called the First Amendment of the internet for a good
reason. Obama-era net neutrality rules classed telecom giants, such as
AT\&T, as ``common carriers,'' de facto public utilities like water and
electricity companies. This status prohibits corporate bosses from
abusing control over network infrastructure to stifle rivals or favor
subsidiaries. Under net neutrality rules, a company like Comcast, which
owns NBC, cannot throttle data flows carrying Netflix's competing TV
shows, any more than General Electric, once a majority stakeholder in
NBC and corporate parent to Jay Leno's ``Tonight Show,'' could have cut
the power to David Letterman's ``Late Show'' studios at CBS. These
content-neutral safeguards apply to political speech as much as to
``Orange Is the New Black*.''* They enshrine a basic American value:
that diverse opinions, from diverse sources, are a pillar of public
welfare. Eliminating net neutrality allows corporations to tamper with
data flows on their networks without public oversight or accountability.
If a connection is slow for MSNBC but not for Fox News, you may never
learn why.

Beijing, meanwhile, is not shy about using its political and economic
heft to blot out dissent beyond China's borders, employing tactics that
blend politics and commerce. The process starts at home: Foreign media
firms seeking access to China's enormous markets face intense pressure
from Communist Party gatekeepers to make odious concessions on content
control and privacy. Then, the chilling effects spread overseas. This
has already happened in Hollywood: If you can't distribute a film like
``Seven Years in Tibet'' in China's giant market, why bother making it
at all? (That 1997 project would never be greenlit today.) Facebook,
long locked out of the Chinese market, is
\href{https://www.nytimes3xbfgragh.onion/2016/11/22/technology/facebook-censorship-tool-china.html}{flirting
with censorship} in a likely bid to gain entry. After the Facebook page
of Guo Wengui, a prominent regime critic living in exile,
\href{https://www.nytimes3xbfgragh.onion/2017/04/21/technology/guo-wengui-china-facebook-account-suspended.html}{vanished}
last spring, Facebook blamed a systems error. As a private corporation,
it is under no obligation to provide more detail.

As the incentives and opportunities to self-censor accumulate, content
providers will gravitate to producing and promoting whatever reaches the
most customers worldwide, even if that means pleasing Chinese
cyber-commissars who control access to a billion of them. Content
providers --- companies like Facebook and LinkedIn --- are not, after
all, common carriers. They do not control the pipes, or carry a unique
public trust in the eyes of the government. As private businesses, they
are not bound by the First Amendment. Self-censorship is simply good
business.

It is much more dangerous to grant American telecom companies --- those
that \emph{do} control the pipes --- the right to tamper with data flows
and discriminate among content. American businesses' track record of
helping China export censorship and Beijing's aggressive and
platform-agnostic efforts to squelch unwanted speech overseas are a
dangerous combination.

For those telecom conglomerates that have current or potential business
before the Chinese government, the temptation to take behind-the-scenes
guidance on content ``management'' as a quid pro quo for market access
may prove irresistible. China already exports its cutting-edge internet
censorship technologies, honed in the world's largest natural laboratory
for ``opinion guidance,'' to authoritarian countries around the world.
These technologies are masterpieces of subtlety and sophistication,
offering a world-class censorship experience amenable to consumers and
commissars alike. They rarely resort to the ``hard'' censorship of crude
deletion, but rather give a suite of ``softer'' monitoring and
management tools to steer public discourse and discourage all but the
most dogged netizens from accessing undesirable content. For those who
stick to domestic and approved Chinese sites, the browsing experience is
speedy and seamless, the sutures where unwanted content has been excised
barely visible.

Browsing the web in China today, one rarely encounters the once
ubiquitous ``your connection has been reset'' or ``due to relevant laws
and regulations, this content cannot be shown.'' You're likelier to
endure a load time that's just a split-second too long, get bored and
move elsewhere. Under this system, even content creators who refuse
self-censorship, regardless of consequences --- such as The Times, which
has been blocked in China since reporting on the party leaders' family
wealth in 2012 --- may find their ability to reach consumers at the
mercy of the companies that run the pipes. Without net neutrality,
American firms will have no obligation to provide equal access for
content, and
\href{https://www.theverge.com/2017/2/23/14714142/fcc-lifts-net-neutrality-transparency-rules-smaller-isps}{minimal
statutory requirement to explain} why one piece of content might arrive
more slowly than another.

In the future, if the article you're reading loads slowly, or not at
all, you might not know the reason. But you can guess.

Advertisement

\protect\hyperlink{after-bottom}{Continue reading the main story}

\hypertarget{site-index}{%
\subsection{Site Index}\label{site-index}}

\hypertarget{site-information-navigation}{%
\subsection{Site Information
Navigation}\label{site-information-navigation}}

\begin{itemize}
\tightlist
\item
  \href{https://help.nytimes3xbfgragh.onion/hc/en-us/articles/115014792127-Copyright-notice}{©~2020~The
  New York Times Company}
\end{itemize}

\begin{itemize}
\tightlist
\item
  \href{https://www.nytco.com/}{NYTCo}
\item
  \href{https://help.nytimes3xbfgragh.onion/hc/en-us/articles/115015385887-Contact-Us}{Contact
  Us}
\item
  \href{https://www.nytco.com/careers/}{Work with us}
\item
  \href{https://nytmediakit.com/}{Advertise}
\item
  \href{http://www.tbrandstudio.com/}{T Brand Studio}
\item
  \href{https://www.nytimes3xbfgragh.onion/privacy/cookie-policy\#how-do-i-manage-trackers}{Your
  Ad Choices}
\item
  \href{https://www.nytimes3xbfgragh.onion/privacy}{Privacy}
\item
  \href{https://help.nytimes3xbfgragh.onion/hc/en-us/articles/115014893428-Terms-of-service}{Terms
  of Service}
\item
  \href{https://help.nytimes3xbfgragh.onion/hc/en-us/articles/115014893968-Terms-of-sale}{Terms
  of Sale}
\item
  \href{https://spiderbites.nytimes3xbfgragh.onion}{Site Map}
\item
  \href{https://help.nytimes3xbfgragh.onion/hc/en-us}{Help}
\item
  \href{https://www.nytimes3xbfgragh.onion/subscription?campaignId=37WXW}{Subscriptions}
\end{itemize}
