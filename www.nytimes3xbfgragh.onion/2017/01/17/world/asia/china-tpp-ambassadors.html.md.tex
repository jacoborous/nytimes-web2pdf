Sections

SEARCH

\protect\hyperlink{site-content}{Skip to
content}\protect\hyperlink{site-index}{Skip to site index}

\href{https://www.nytimes3xbfgragh.onion/section/world/asia}{Asia
Pacific}

\href{https://myaccount.nytimes3xbfgragh.onion/auth/login?response_type=cookie\&client_id=vi}{}

\href{https://www.nytimes3xbfgragh.onion/section/todayspaper}{Today's
Paper}

\href{/section/world/asia}{Asia Pacific}\textbar{}U.S. Ambassadors in
Asia Make Final Plea for Dead Trans-Pacific Trade Pact

\url{https://nyti.ms/2iH8I6O}

\begin{itemize}
\item
\item
\item
\item
\item
\item
\end{itemize}

Advertisement

\protect\hyperlink{after-top}{Continue reading the main story}

Supported by

\protect\hyperlink{after-sponsor}{Continue reading the main story}

\href{/column/sinosphere}{Sinosphere}

\hypertarget{us-ambassadors-in-asia-make-final-plea-for-dead-trans-pacific-trade-pact}{%
\section{U.S. Ambassadors in Asia Make Final Plea for Dead Trans-Pacific
Trade
Pact}\label{us-ambassadors-in-asia-make-final-plea-for-dead-trans-pacific-trade-pact}}

\includegraphics{https://static01.graylady3jvrrxbe.onion/images/2017/01/18/world/18CHINATRADE-1/18CHINATRADE-1-articleLarge.jpg?quality=75\&auto=webp\&disable=upscale}

By \href{http://www.nytimes3xbfgragh.onion/by/edward-wong}{Edward Wong}

\begin{itemize}
\item
  Jan. 17, 2017
\item
  \begin{itemize}
  \item
  \item
  \item
  \item
  \item
  \item
  \end{itemize}
\end{itemize}

When hope of enacting the ambitious
\href{http://www.nytimes3xbfgragh.onion/interactive/2016/business/tpp-explained-what-is-trans-pacific-partnership.html}{Trans-Pacific
Partnership} trade pact ended in November, Max Baucus, the United States
ambassador to China, was among the officials who had to grapple with
disappointment.

The partnership, called the TPP, was a hallmark of the Obama
administration. It would have been one of the largest trade agreements
in history, covering about 40 percent of the world's economy and setting
new terms and standards for trade for the United States and 11 other
Pacific Rim nations. China was not included but would have been able to
join.

When President Obama plucked Mr. Baucus in 2013 from the United States
Senate to be the ambassador to China, he chose a politician with a
record of
\href{https://sinosphere.blogs.nytimes3xbfgragh.onion/2013/12/18/obamas-choice-for-china-envoy-a-longtime-free-trade-advocate/}{promoting
free trade}. As ambassador, Mr. Baucus supported the Trans-Pacific
Partnership and tried to dampen
\href{https://www.nytimes3xbfgragh.onion/2015/05/13/us/politics/as-obama-plays-china-card-on-trade-chinese-pursue-their-own-deals.html}{alarm
in China} over the American-led effort.

Last week Mr. Baucus took the unusual step, with five other American
ambassadors in the Asia-Pacific region, of sending an open letter to
Congress asking its members to support the pact in an effort to cement a
leadership position for the United States in regional trade and not
\href{https://www.nytimes3xbfgragh.onion/2016/11/20/business/international/apec-trade-china-obama-trump-tpp-trans-pacific-partnership.html}{yield
that role to China}, which has the second-biggest economy in the world.

In their letter, the ambassadors warn that ``walking away from TPP may
be seen by future generations as the moment America chose to cede
leadership to others in this part of the world and accept a diminished
role.''

Image

Max Baucus, the United States ambassador to China.Credit...Lintao
Zhang/Getty Images

``Such an outcome would be cause for celebration among those who favor
`Asia for the Asians' and state capitalism,'' it added.

This passage critiques President Xi Jinping of China, who
\href{http://www.globaltimes.cn/content/861573.shtml}{has said that Asia
should be run by Asians} and is a champion of a Chinese economic system
that relies on industrial policy. (Mr. Xi was
\href{https://www.nytimes3xbfgragh.onion/2017/01/10/world/asia/davos-china-xi-jinping-trump.html}{scheduled
to appear on Tuesday} at the pro-free-trade World Economic Forum in
Davos, Switzerland, the first Chinese leader to do so.)

The 2016 presidential race was shaped by anti-globalization trends.
Donald J. Trump promised to destroy the pact if he became president.
Hillary Clinton also denounced it, even though she supported a form of
it as secretary of state.

Senator Mitch McConnell, Republican of Kentucky and the majority leader,
said after the election in November that Congress would not take it up.
That meant it was dead.

The letter by the six ambassadors, all of whom are political appointees
who leave their jobs on Friday, was symbolic. It put them on record
supporting Mr. Obama's plan more than a year after the trade chiefs of
the nations involved endorsed the pact. Following is the text of the
letter:

\begin{quote}
An Open Letter to Members of Congress:

Seventy-five years ago last month, an attack on the United States set us
on the path to becoming the Asia-Pacific power we are today. As U.S.
Ambassadors assigned to the region, we interact daily with governmental,
business, and civil society leaders who appreciate profoundly the role
the United States has played in underpinning the region's security and
prosperity ever since. These same leaders are now asking an alarming
question: Will we relinquish our mantle as the pre-eminent force for
good in the planet's most dynamic region? The cause for their concern
--- possible U.S. withdrawal from the Trans-Pacific Partnership (TPP).
We believe their fears are justified, that walking away from TPP may be
seen by future generations as the moment America chose to cede
leadership to others in this part of the world and accept a diminished
role. Such an outcome would be cause for celebration among those who
favor ``Asia for the Asians'' and state capitalism. It would be
disastrous for supporters of inclusive politics, rule of law, and market
economics --- and for U.S. national interests.

Let's be clear. The alternative to a TPP world is not the status quo.
Others are actively engaged in setting the rules of commerce in the
Asia-Pacific region without the United States. In addition to its
massive Eurasian infrastructure initiative, China is working on a trade
pact called the Regional Comprehensive Economic Partnership (RCEP) with
fifteen other countries, many of whom are TPP signatories. RCEP is a
much lower-standard agreement that, in the absence of TPP, would likely
serve as the template for economic integration in Asia and shift trade
away from America, which would face higher tariffs. That would mean less
U.S. exports and more jobs moving overseas.

TPP would not just cut tariffs for U.S. products. Unlike RCEP, it would
compel stronger intellectual property rights, limits on subsidies to
state-owned enterprises, and protection of worker rights, the
environment, and a free and open internet. These enforceable commitments
would give a leg up to U.S. companies already adhering to high standards
--- and the U.S. workers who make them the most productive in the world
--- and provide a powerful lever for change that we are unlikely to
replicate in any other form in the near future. Without them, our
companies will face even more competitive disadvantages in Asia's
booming markets.

The blow to our strategic position is even more worrisome. This is not
speculation. To turn our back on our allies and friends at this critical
juncture, when the tectonic plates of regional power are shifting faster
than ever, would undermine our credibility not only as a reliable trade
partner, but as a leader on both sides of the Pacific. It would also
create a potentially destabilizing void that might even lead to
conflict, an outcome which would hurt everyone in the region, including
China.

The bottom line is this: TPP is good for American workers, American
values, and American strategic interests. We urge the Congress to work
with the new administration to find a way to realize its many benefits
before the window for doing so closes. As we reflect on more than seven
decades of U.S. sacrifice and stewardship in the region that will define
our destiny in coming decades, we should understand that, if we fail to
answer today's call, history will pose a stern question --- why did
America forsake its best chance to shape the Pacific Century?

\emph{Signed by: Max Baucus, ambassador to China; Nina Hachigian,
ambassador to the Association of Southeast Asian Nations; Caroline
Kennedy, ambassador to Japan; Mark Lippert, ambassador to South Korea;
Mark Gilbert, ambassador to New Zealand and Samoa; and Kirk Wagar,
ambassador to Singapore.}
\end{quote}

Advertisement

\protect\hyperlink{after-bottom}{Continue reading the main story}

\hypertarget{site-index}{%
\subsection{Site Index}\label{site-index}}

\hypertarget{site-information-navigation}{%
\subsection{Site Information
Navigation}\label{site-information-navigation}}

\begin{itemize}
\tightlist
\item
  \href{https://help.nytimes3xbfgragh.onion/hc/en-us/articles/115014792127-Copyright-notice}{©~2020~The
  New York Times Company}
\end{itemize}

\begin{itemize}
\tightlist
\item
  \href{https://www.nytco.com/}{NYTCo}
\item
  \href{https://help.nytimes3xbfgragh.onion/hc/en-us/articles/115015385887-Contact-Us}{Contact
  Us}
\item
  \href{https://www.nytco.com/careers/}{Work with us}
\item
  \href{https://nytmediakit.com/}{Advertise}
\item
  \href{http://www.tbrandstudio.com/}{T Brand Studio}
\item
  \href{https://www.nytimes3xbfgragh.onion/privacy/cookie-policy\#how-do-i-manage-trackers}{Your
  Ad Choices}
\item
  \href{https://www.nytimes3xbfgragh.onion/privacy}{Privacy}
\item
  \href{https://help.nytimes3xbfgragh.onion/hc/en-us/articles/115014893428-Terms-of-service}{Terms
  of Service}
\item
  \href{https://help.nytimes3xbfgragh.onion/hc/en-us/articles/115014893968-Terms-of-sale}{Terms
  of Sale}
\item
  \href{https://spiderbites.nytimes3xbfgragh.onion}{Site Map}
\item
  \href{https://help.nytimes3xbfgragh.onion/hc/en-us}{Help}
\item
  \href{https://www.nytimes3xbfgragh.onion/subscription?campaignId=37WXW}{Subscriptions}
\end{itemize}
