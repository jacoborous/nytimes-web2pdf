Sections

SEARCH

\protect\hyperlink{site-content}{Skip to
content}\protect\hyperlink{site-index}{Skip to site index}

\href{https://myaccount.nytimes3xbfgragh.onion/auth/login?response_type=cookie\&client_id=vi}{}

\href{https://www.nytimes3xbfgragh.onion/section/todayspaper}{Today's
Paper}

The Last Days of Wicker

\url{https://nyti.ms/2uY0Bpi}

\begin{itemize}
\item
\item
\item
\item
\item
\end{itemize}

Advertisement

\protect\hyperlink{after-top}{Continue reading the main story}

Supported by

\protect\hyperlink{after-sponsor}{Continue reading the main story}

\hypertarget{the-last-days-of-wicker}{%
\section{The Last Days of Wicker}\label{the-last-days-of-wicker}}

Just a handful of Europe's rattan ateliers are still in operation. The
three finest have been weaving furniture according to ancient techniques
for the last hundred years.

\includegraphics{https://static01.graylady3jvrrxbe.onion/images/2017/08/11/t-magazine/wicker-slide-RYZK/wicker-slide-RYZK-articleLarge.jpg?quality=75\&auto=webp\&disable=upscale}

By Deborah Needleman

\begin{itemize}
\item
  Aug. 15, 2017
\item
  \begin{itemize}
  \item
  \item
  \item
  \item
  \item
  \end{itemize}
\end{itemize}

An air of fustiness hovers over the very words ``wicker furniture'': One
imagines sepia-toned photos of peacock-backed chairs on verandas in
colonial India, and ornately curved chaises in Victorian conservatories.
And indeed, the style is old --- while wicker's apex was in the 19th
century, when trade routes from Asia brought a steady supply of rattan
to Europe, the art of weaving objects from natural fibers actually goes
back at least as far as ancient Egypt.

Today, however, quality handwoven wicker is still being produced in a
few of those original century-old workshops, most of which are small
family-run operations. The designs have evolved with the times, but the
construction process has barely changed in thousands of years. (Wicker
refers to a technique of weaving fibers rather than to any particular
material, and so can be done with anything from plant-based elements to
synthetics.) These remaining ateliers continue to use natural rattan,
the stem of a sturdy yet flexible climbing palm, to form the base of the
structure of the piece, and thin spaghetti-like strands harvested from
the interior of the stem for the fine weaving work that gives wicker its
name. Competition has come mostly in the form of imported cheaper
pieces, which might be a fine thing, especially for the countries where
rattan grows, like Indonesia and Malaysia. But rarely is the
craftsmanship or design comparable to the work coming from these
traditional studios. Here, highly skilled artisans create furniture the
way they always have: by heating stiff poles of rattan into malleable
rods that are bent on a metal frame and willed into functional works of
art.

Three of the last ateliers left --- in Sweden, England and Italy --- are
by all accounts creating the most beautiful wicker furniture today.
While all use the same over-under weaving technique practiced by
basket-makers for millennia, the styles of each house are reflections of
their individual countries --- making wicker a small but lovely prism
through which to view the world.

\includegraphics{https://static01.graylady3jvrrxbe.onion/images/2017/08/11/t-magazine/wicker-slide-YJF1/wicker-slide-YJF1-articleLarge.jpg?quality=75\&auto=webp\&disable=upscale}

\hypertarget{bonacina}{%
\subsubsection{Bonacina}\label{bonacina}}

\emph{Lurago d'Erba, Italy, est. 1889}

Long a secret resource of Europe's most discerning decorators,
\href{http://www.bonacina1889.it/}{Bonacina} is a 128-year-old wicker
furniture company north of Milan run by the founder's grandson Mario,
along with his wife and children. Most of the workers are locals who
have been with the company for years, and who return home each afternoon
for lunch. Men do the hot, physical work of bending the thick rods of
rattan, and women do the weaving. The day I visited, Luigi Meroni, a
Bonacina employee of 50 years, was making a coffee table, while two
sisters, Cecilia and Rosa, were weaving chairs.

In addition to the thousands of designs created over three generations
--- everything from a classic armchair with a sunburst pattern designed
by decorator Renzo Mongiardino to, more recently, a thin chaise
resembling a bent sheet of metal created by Mario Bonacina --- the
company has collaborated with a number of other designers such as Gae
Aulenti, Gio Ponti, Joe Colombo and Franco Albini. Most famous, perhaps,
is the decades-­long collaboration with Mongiardino, who created for the
house, among other things, a collection of charming, old-fashioned
chairs and settees with Italian doyenne Marella Agnelli for all of her
properties, including those in St. Moritz, Rome, New York City and
Marrakesh. (There's an oft-told story that after Agnelli toured a
nouveau riche American's home, she remarked of her hostess, ``It will
take her another lifetime to understand wicker.'')

It also hasn't been difficult for contemporary designers and architects
--- including Peter Marino, Jacques Grange and Daniel Romualdez --- to
appreciate the company's humble but elegant design; Bonacina's grand
sloping armchairs and spare, streamlined loungers can be found in
Italian hotels like Le Sirenuse, Francis Ford Coppola's Palazzo
Margherita and Villa Feltrinelli.

Image

At Soane's workshop, a technically complex vanity and side table next to
an unfinished Venus chair, which takes three days to
weave.Credit...Danilo Scarpati

\hypertarget{soane}{%
\subsubsection{Soane}\label{soane}}

\emph{Leicestershire, U.K., originally est. as Angraves, 1912}

In 2010, Nigel Angrave was faced with shuttering the company his
grandfather had started nearly 100 years earlier. It meant not only the
loss of jobs and his family's business, but as the company was the last
rattan-weaving workshop in England, forever losing the art of
traditional wicker-making in Britain. Enter Lulu Lytle and Christopher
Hodsoll, owners of \href{http://www.soane.co.uk/}{Soane}, a London
enterprise committed to creating furniture and fabrics made by
craftspeople from across Britain. They purchased the Angraves workshop
in Leicestershire --- at one time the heart of the British rattan
industry, which reached its peak in the Victorian era --- and rehired
its two chief craftsmen, frame maker Mick Gregory and master weaver Phil
Ayres, both of whom began working in rattan over 40 years ago, when they
were still teenagers.

Lytle revitalized the company's old-fashioned designs with playful
updates of English country classics like tall barstools with
diamond-patterned backs, petite side tables crafted entirely of tightly
woven rattan and lampshades made in a traditional basket weave. Many of
Soane's pieces are also technically complex, including the Ripple
console, which resembles folds of gently undulating fabric, and the
Venus chair, the back of which crests into petal-like scallops,
requiring multiple applications of naked flame and water to achieve a
fluidity of form.

By creating a new market and appreciation for British wicker, Hodsoll
and Lytle were able to start an apprenticeship program and add another
seven artisans (six men and one woman) trained by Gregory and Ayres.
``Because this is the very last workshop,'' Lytle says, ``it was crucial
we train the next generation before the skills are lost forever.''

Image

Larsson Korgmakare started a partnership with Josef Frank in the 1930s
--- his designs are still woven here by hand.Credit...Danilo Scarpati

\hypertarget{larsson-korgmakare}{%
\subsubsection{Larsson Korgmakare}\label{larsson-korgmakare}}

\emph{Stockholm, est. 1903}

On a narrow street in the old city of Stockholm, an ancient building
holds the warehouse of the 114-year-old
\href{http://www.larssonkorgmakare.se/}{Larsson Korgmakare}, the only
artisanal rattan workshop left in Sweden. Next door, in the cellar,
you'll find 49-year-old Erica Larsson, carrying on the family tradition
begun by her great-grandfather. She is not just the owner, but the
company's sole artisan, doing the physically taxing work of firing and
shaping rattan with the help of just one assistant. The history and
continued existence of the business owes much to the great Swedish
designer Josef Frank. Larsson's grandfather, a craftsmen and designer,
began a partnership in the 1930s with Frank, whose rustic, minimal
wicker sofas and chairs she continues to produce for the famed Stockholm
design emporium Svenskt Tenn.

Image

A cane seat, mid-realization. Cane comes from fibers made from the bark
of rattan stems.Credit...Danilo Scarpati

When Larsson decided to join the business, her father sent her to
Malaysia for six months to apprentice in the harvest and processing of
rattan, an experience she may pass on to her 11-year-old daughter if she
decides to continue the legacy. A testament to a Swedish love of good
and simple design, Larsson Korgmakare's style of seating is crafted
almost exclusively from the larger rattan poles, incorporating almost no
intricate weaving. Most of the pieces Larsson creates --- skeletal,
honey-colored love seats and armchairs, woven in loose looping patterns
--- are from the original designs created by Frank or her grandfather.
``In {[}my grandfather's{]} last year, when he was 80, and more or less
blind, he continued to make furniture,'' she recalls. ``He had the
knowledge in his fingers.''

Advertisement

\protect\hyperlink{after-bottom}{Continue reading the main story}

\hypertarget{site-index}{%
\subsection{Site Index}\label{site-index}}

\hypertarget{site-information-navigation}{%
\subsection{Site Information
Navigation}\label{site-information-navigation}}

\begin{itemize}
\tightlist
\item
  \href{https://help.nytimes3xbfgragh.onion/hc/en-us/articles/115014792127-Copyright-notice}{©~2020~The
  New York Times Company}
\end{itemize}

\begin{itemize}
\tightlist
\item
  \href{https://www.nytco.com/}{NYTCo}
\item
  \href{https://help.nytimes3xbfgragh.onion/hc/en-us/articles/115015385887-Contact-Us}{Contact
  Us}
\item
  \href{https://www.nytco.com/careers/}{Work with us}
\item
  \href{https://nytmediakit.com/}{Advertise}
\item
  \href{http://www.tbrandstudio.com/}{T Brand Studio}
\item
  \href{https://www.nytimes3xbfgragh.onion/privacy/cookie-policy\#how-do-i-manage-trackers}{Your
  Ad Choices}
\item
  \href{https://www.nytimes3xbfgragh.onion/privacy}{Privacy}
\item
  \href{https://help.nytimes3xbfgragh.onion/hc/en-us/articles/115014893428-Terms-of-service}{Terms
  of Service}
\item
  \href{https://help.nytimes3xbfgragh.onion/hc/en-us/articles/115014893968-Terms-of-sale}{Terms
  of Sale}
\item
  \href{https://spiderbites.nytimes3xbfgragh.onion}{Site Map}
\item
  \href{https://help.nytimes3xbfgragh.onion/hc/en-us}{Help}
\item
  \href{https://www.nytimes3xbfgragh.onion/subscription?campaignId=37WXW}{Subscriptions}
\end{itemize}
