Sections

SEARCH

\protect\hyperlink{site-content}{Skip to
content}\protect\hyperlink{site-index}{Skip to site index}

\href{https://myaccount.nytimes3xbfgragh.onion/auth/login?response_type=cookie\&client_id=vi}{}

\href{https://www.nytimes3xbfgragh.onion/section/todayspaper}{Today's
Paper}

How to Have Your Health Food and Love It, Too

\url{https://nyti.ms/2uJbDyx}

\begin{itemize}
\item
\item
\item
\item
\item
\end{itemize}

Advertisement

\protect\hyperlink{after-top}{Continue reading the main story}

Supported by

\protect\hyperlink{after-sponsor}{Continue reading the main story}

\href{/column/magazine-eat}{Eat}

\hypertarget{how-to-have-your-health-food-and-love-it-too}{%
\section{How to Have Your Health Food and Love It,
Too}\label{how-to-have-your-health-food-and-love-it-too}}

\includegraphics{https://static01.graylady3jvrrxbe.onion/images/2017/08/13/magazine/13eat1/13eat1-articleInline-v2.png?quality=75\&auto=webp\&disable=upscale}

By Gabrielle Hamilton

\begin{itemize}
\item
  Aug. 10, 2017
\item
  \begin{itemize}
  \item
  \item
  \item
  \item
  \item
  \end{itemize}
\end{itemize}

At some point in the '70s my mom bought her first pair of bluejeans. She
didn't suddenly throw away all her tailored wool skirts and silk
scarves, or dump all the cashmere sweaters from the dresser drawers into
bags destined for the Salvation Army, but there they were, in rotation:
A pair of soft bluejeans, modestly flared at the ankle, with two flat
front pockets, that she wore, if I may say, with exceptional and
enviable style.

And then, around this same time, you opened the fridge one day and found
she had glass jars lying on their sides, cheesecloth held with rubber
bands over their mouths, alfalfa sprouts growing inside. And there on
the kitchen ­counter, nestled like a flock of broken fledglings fallen
too early from the nest, were eight little glass jars wrapped in kitchen
towels and set on an electric medical heating pad meant for sore back
muscles, incubating her homemade cultured yogurt. Which turned out tangy
and creamy and expert.

My mother was dispositionally unwilling to sacrifice pleasure for
politics, or style for trends, enough so that I did not mind the
dialed-down frequency of her customary brown-butter sauces, ripe, oozing
full-fat cheeses and visits to the butchers. And I welcomed the
open-faced avocado sandwiches on pumpernickel with cream cheese, red
onion and alfalfa sprouts (hers were clean and fresh and lively) and
fruit preserves stirred into yogurt for dessert and shopping trips to
the memorably dirty health-food store. There the bulk jugs of tamari and
tahini and separated almond butter under an inch of rancid oil had crud
on their spouts, and the bulk bins of oats and millet and whole-grain
flours were lively with meal moths. This was as fascinating to me as the
whole sides of bloody animals hanging from hooks in the refrigerated
walk-in at the Italian butcher we used.

I thought it was just as miraculous and cool to see her making her own
yogurt and granola, and sprouting her own sprouts, as I did watching her
make Irish soda bread or duck-leg confit or the annual birthday baked
alaska that she set under mesmerizing blue rivulets of fire with kirsch
flowing from half an empty eggshell set at the top of the meringue
Vesuvius. In a way, her French background, her impeccable kitchen skills
and her intractable devotion to pleasure in eating made her a kind of
perfect precursor and model for healthful whole-foods cooking, 45 years
ago.

\includegraphics{https://static01.graylady3jvrrxbe.onion/images/2017/08/13/magazine/13eat2/13eat2-articleLarge.png?quality=75\&auto=webp\&disable=upscale}

Tofu, however, I came to for the first time during my second attempt at
college in the early '80s, at a lefty, rigorously political liberal-arts
college in New England, under decidedly less stylish and markedly less
pleasure-principled circumstances. It seemed as if there were a
dog-eared ``Moosewood Cookbook'' in every kitchen on campus. They sold
it in the campus bookstore next to Wollstonecraft and Hume and John
Stuart Mill. The chore wheel --- that egalitarian method for
distributing household chores to make these experiments in communal
living harmonious and less fetid --- in our on-campus apartment had a
slot wedged right there between Clean Bathrooms and Vacuum Common Area:
Water the Tofu! Everybody cut the tofu into cubes and steamed it in a
wok with celery and onions and mushrooms and garlic and ginger. We meant
to stir-fry it, but you could never get a wok hot enough on those
electric coil burners, so tofu dinner was always wet and limp, then
drenched in tamari. It was not stylish, expert or enviable.

When I started making soft, silken tofu about five years ago, it was
like getting my own first pair of bluejeans. I did not toss out all my
marrow bones and suckling pig and the crème Chantilly to remake myself
in soy. I started my first batch in a stainless-steel pot on the kitchen
counter and finished in glass Ball jars. I took exceptional care of the
beans, the soak, the milk. The tofu was rich, almost nutty. It had, and
will always have, a faint, chalky mouth-feel, which the unctuousness of
salted French butter will completely smooth out in the finished dish. If
you went at this in the spirit of the chefs who have labs/test
kitchens/ateliers with interesting appliances and chemicals, I think you
could add lecithin or other mail-ordered emulsifiers to ``correct'' that
effect in the milk before you coagulate the tofu. But I really want you
to go at this the way my mom eventually went for the macramé bikini and
the addition of brewer's yeast on our popcorn: Enjoy yourself, but
remain yourself.

\textbf{Recipe:}
\href{https://cooking.nytimes3xbfgragh.onion/recipes/1018892-warm-tofu-and-fresh-soybeans-cooked-in-salted-french-butter-and-celery-seed-gastrique}{Warm
Tofu and Fresh Soybeans Cooked in Salted French Butter and Celery-Seed
Gastrique}

Advertisement

\protect\hyperlink{after-bottom}{Continue reading the main story}

\hypertarget{site-index}{%
\subsection{Site Index}\label{site-index}}

\hypertarget{site-information-navigation}{%
\subsection{Site Information
Navigation}\label{site-information-navigation}}

\begin{itemize}
\tightlist
\item
  \href{https://help.nytimes3xbfgragh.onion/hc/en-us/articles/115014792127-Copyright-notice}{©~2020~The
  New York Times Company}
\end{itemize}

\begin{itemize}
\tightlist
\item
  \href{https://www.nytco.com/}{NYTCo}
\item
  \href{https://help.nytimes3xbfgragh.onion/hc/en-us/articles/115015385887-Contact-Us}{Contact
  Us}
\item
  \href{https://www.nytco.com/careers/}{Work with us}
\item
  \href{https://nytmediakit.com/}{Advertise}
\item
  \href{http://www.tbrandstudio.com/}{T Brand Studio}
\item
  \href{https://www.nytimes3xbfgragh.onion/privacy/cookie-policy\#how-do-i-manage-trackers}{Your
  Ad Choices}
\item
  \href{https://www.nytimes3xbfgragh.onion/privacy}{Privacy}
\item
  \href{https://help.nytimes3xbfgragh.onion/hc/en-us/articles/115014893428-Terms-of-service}{Terms
  of Service}
\item
  \href{https://help.nytimes3xbfgragh.onion/hc/en-us/articles/115014893968-Terms-of-sale}{Terms
  of Sale}
\item
  \href{https://spiderbites.nytimes3xbfgragh.onion}{Site Map}
\item
  \href{https://help.nytimes3xbfgragh.onion/hc/en-us}{Help}
\item
  \href{https://www.nytimes3xbfgragh.onion/subscription?campaignId=37WXW}{Subscriptions}
\end{itemize}
