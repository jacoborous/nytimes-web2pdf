Sections

SEARCH

\protect\hyperlink{site-content}{Skip to
content}\protect\hyperlink{site-index}{Skip to site index}

\href{https://myaccount.nytimes3xbfgragh.onion/auth/login?response_type=cookie\&client_id=vi}{}

\href{https://www.nytimes3xbfgragh.onion/section/todayspaper}{Today's
Paper}

How to Handle a Sleepwalker

\url{https://nyti.ms/2uMi8Rs}

\begin{itemize}
\item
\item
\item
\item
\item
\end{itemize}

Advertisement

\protect\hyperlink{after-top}{Continue reading the main story}

Supported by

\protect\hyperlink{after-sponsor}{Continue reading the main story}

\href{/column/magazine-tip}{Tip}

\hypertarget{how-to-handle-a-sleepwalker}{%
\section{How to Handle a
Sleepwalker}\label{how-to-handle-a-sleepwalker}}

By Malia Wollan

\begin{itemize}
\item
  Aug. 11, 2017
\item
  \begin{itemize}
  \item
  \item
  \item
  \item
  \item
  \end{itemize}
\end{itemize}

``Don't startle the person,'' says Charlene Gamaldo, the medical
director at the Johns Hopkins Center for Sleep. Sleepwalkers exist in a
semiwakeful state and can become testy and disoriented when forced to
come to full consciousness. Instead, speak to them in a quiet voice and
lead them gently back to their bed. In most cases, they'll settle easily
and in the morning remember nothing of their nighttime ambulations.

\includegraphics{https://static01.graylady3jvrrxbe.onion/images/2017/08/13/magazine/13tip/13tip-articleInline-v6.png?quality=75\&auto=webp\&disable=upscale}

To determine whether you're dealing with a sleepwalker, as opposed to,
say, a night owl (or someone with another, more worrisome form of
parasomnia), watch for open eyes, a blank expression, physical
clumsiness and a lack of reactivity. ``They look zoned out,'' Gamaldo
says. Sleepwalkers tend to perform tasks from memory, including texting,
shopping online, cooking and even driving and having sex, all with a
noticeably odd flair. ``They may get up and eat a raw TV dinner,''
Gamaldo says.

Researchers attribute a surge in sleepwalking in the 21st century to a
rise in the use of hypnosedative sleeping medications. A popular hotel
chain in the United Kingdom even issued sleepwalker-care guidelines to
staff members after noting a sevenfold increase in sleepwalking patrons
over one year, 95 percent of whom were men wandering out of their rooms
naked. Other triggers include stress, genetics, fatigue, heat and what
Gamaldo calls ``poor sleep hygiene,'' or loud, overly bright bedrooms
filled with TVs and digital devices. To protect a sleepwalker in your
home, make it as safe and soporific as possible. Keep him or her away
from stairs and sharp objects. ``The bedroom should be uncluttered,''
Gamaldo says.

Sleepwalking tends to occur in the first third of the night, when sleep
cycles include more time spent in the phase characterized by slow, even
brain waves and nonrapid eye movement. For neurodevelopmental reasons,
children experience more of this so-called slow-wave sleep. The peak
sleepwalking age is between 4 and 6, when as many as 30 percent of
children experience it, compared with about 3 percent of the adult
population. Be prepared for the zombielike spookiness of a sleepwalker.
Gamaldo's eldest daughter walked in her sleep. ``I'd wake up and see
this glassy-eyed 6-year-old staring at me,'' she says. ``It totally
freaked me out.'' Don't fret. If someone you know sleepwalks, it doesn't
mean they're neurologically damaged. ``Usually,'' Gamaldo says,
``they'll grow out of it.''

Advertisement

\protect\hyperlink{after-bottom}{Continue reading the main story}

\hypertarget{site-index}{%
\subsection{Site Index}\label{site-index}}

\hypertarget{site-information-navigation}{%
\subsection{Site Information
Navigation}\label{site-information-navigation}}

\begin{itemize}
\tightlist
\item
  \href{https://help.nytimes3xbfgragh.onion/hc/en-us/articles/115014792127-Copyright-notice}{©~2020~The
  New York Times Company}
\end{itemize}

\begin{itemize}
\tightlist
\item
  \href{https://www.nytco.com/}{NYTCo}
\item
  \href{https://help.nytimes3xbfgragh.onion/hc/en-us/articles/115015385887-Contact-Us}{Contact
  Us}
\item
  \href{https://www.nytco.com/careers/}{Work with us}
\item
  \href{https://nytmediakit.com/}{Advertise}
\item
  \href{http://www.tbrandstudio.com/}{T Brand Studio}
\item
  \href{https://www.nytimes3xbfgragh.onion/privacy/cookie-policy\#how-do-i-manage-trackers}{Your
  Ad Choices}
\item
  \href{https://www.nytimes3xbfgragh.onion/privacy}{Privacy}
\item
  \href{https://help.nytimes3xbfgragh.onion/hc/en-us/articles/115014893428-Terms-of-service}{Terms
  of Service}
\item
  \href{https://help.nytimes3xbfgragh.onion/hc/en-us/articles/115014893968-Terms-of-sale}{Terms
  of Sale}
\item
  \href{https://spiderbites.nytimes3xbfgragh.onion}{Site Map}
\item
  \href{https://help.nytimes3xbfgragh.onion/hc/en-us}{Help}
\item
  \href{https://www.nytimes3xbfgragh.onion/subscription?campaignId=37WXW}{Subscriptions}
\end{itemize}
