Sections

SEARCH

\protect\hyperlink{site-content}{Skip to
content}\protect\hyperlink{site-index}{Skip to site index}

\href{https://myaccount.nytimes3xbfgragh.onion/auth/login?response_type=cookie\&client_id=vi}{}

\href{https://www.nytimes3xbfgragh.onion/section/todayspaper}{Today's
Paper}

When to Cook Your Vegetables Long Past `Done'

\url{https://nyti.ms/2vnFZab}

\begin{itemize}
\item
\item
\item
\item
\item
\end{itemize}

Advertisement

\protect\hyperlink{after-top}{Continue reading the main story}

Supported by

\protect\hyperlink{after-sponsor}{Continue reading the main story}

\href{/column/magazine-eat}{Eat}

\hypertarget{when-to-cook-your-vegetables-long-past-done}{%
\section{When to Cook Your Vegetables Long Past
`Done'}\label{when-to-cook-your-vegetables-long-past-done}}

\includegraphics{https://static01.graylady3jvrrxbe.onion/images/2017/08/27/magazine/27eat1/27eat1-articleInline.jpg?quality=75\&auto=webp\&disable=upscale}

By Samin Nosrat

\begin{itemize}
\item
  Aug. 23, 2017
\item
  \begin{itemize}
  \item
  \item
  \item
  \item
  \item
  \end{itemize}
\end{itemize}

Growing up, I was aware of the kids-don't-like-vegetables trope, but it
didn't make much sense to me. I never had any choice; all the
traditional Iranian dishes my mom cooked teemed with herbs and
vegetables. There was no eating around the fava beans, celery and
eggplant that made up the fragrant rice and stew ­dishes she served each
night, though my younger brothers certainly tried. I ate the food but
didn't think much of the vegetables one way or the other. Then I moved
to Berkeley for college, and for the first time, I understood how
someone could hate her vegetables: The pallid, overcooked steam-table
brussels sprouts and zucchini served in the dining hall were depressing
at best. So when I started busing tables at Chez Panisse a couple of
years later, I wasn't prepared for the daily sight of grown men and
women cooing over fruits and vegetables.

Soon after, I began working in the kitchen and quickly learned why each
produce delivery was met with such excitement: flawless, just-picked
vegetables are sweeter and more flavorful than anything you can get at
the store. I learned to cook vegetables with the aim of preserving that
perfection. That usually meant doing as little to them as possible. Much
of the time, we'd simply boil the haricots verts, marble-size turnips or
thick spears of asparagus in ample, salted water until they were barely
cooked through, then pull them out and let them cool on a baking sheet
in the fridge. We'd later quickly reheat them in boiling water or a
sauté pan, then drizzle them with immoderate amounts of fruity olive oil
before serving. No matter the vegetable, the only rule in the kitchen
was ``do not overcook.'' The memory of that dining-hall mush was enough
to scare me straight; my green beans were always perfectly crisp.

Then I went to Italy. I apprenticed myself to Benedetta Vitali, a
Florentine chef who ran a tiny trattoria on the outskirts of town. Eager
to please my new boss, I tried to work ahead on the prep list one
morning while she was upstairs in the office. I found the filet beans
among the vegetable delivery, set a huge pot of water on the stove and
trimmed away the stems while the water came to a boil. I cooked them
just as I'd learned to in California, careful not to let all of the
crunch boil away. I pulled them, vibrant and sweet, from the water and
let them cool.

Benedetta came downstairs. She cocked her head and picked up a green
bean. ``Who cooked these beans raw?'' she asked, her voice incredulous,
while the inch-long ash from her dangling cigarette threatened to fall
onto the tray.

Mortified, I took responsibility. She chuckled and poked fun at my
American way with vegetables. ``The only thing that should be cooked al
dente here,'' she said, ``is pasta.'' Then she heated a big, shallow
pot, added a generous splash of olive oil and garlic, tossed in the
green beans and lightly browned them. She turned down the heat, handed
me the wooden spoon and told me to keep an eye on the pot for two hours.
I was simultaneously horrified about how overcooked they'd be in that
time and deeply ashamed about how far off I'd been.

But I did what Benedetta asked and tended to the beans. As they cooked,
they changed from firm and bright to limp and gray, just as I'd feared.
For over an hour, the beans tasted forgettable. I worried I'd ruined
them a second time. But right around the two-hour mark, they transformed
again, into a dark, tangled mess, soft but defined. They were
extraordinarily rich, deliriously sweet and dense with flavor. I'd never
tasted anything like them. I wondered why.

The classic French blanch-and-cool technique I learned at Chez Panisse
yields the kind of brilliant, picturesque vegetables we all want to see
on restaurant plates. Long-cooked foods, on the other hand, fall firmly
into the ``ugly but good'' camp of the Tuscan \emph{cucina povera},
where flavor far outshines looks. Peasant cooks developed methods like
long cooking to turn the overlooked into the irresistible. They knew
that the best cooking is guided by all the senses, but if one must trump
the rest, it should be taste.

\includegraphics{https://static01.graylady3jvrrxbe.onion/images/2017/08/27/magazine/27eat2/27eat2-articleInline.jpg?quality=75\&auto=webp\&disable=upscale}

Whenever you do get your hands on immaculate baby carrots or fennel,
preserve their flavor. Boil them until they're just barely cooked, then
serve them with flaky salt and melted butter or good olive oil. The
delicate sweetness of just-picked vegetables is always worth savoring.
But on all the many other days of the year, when you're cooking with
whatever you've got, perfect or not, know you can manufacture your own
sweetness by long cooking.

While you can use the technique with almost any vegetable, it works
particularly well with the shunned, the fibrous and the
forgotten-in-the-fridge. All it takes is time and courage. Since
browning begets browning, wait until the end to gently caramelize the
vegetables; that way you won't have to constantly stir the pot. Heat a
little oil with some garlic and a sliced shallot, throw in whichever
vegetables you have on hand and add a tiny splash of water. Set the pot
over low heat. Cover it, and do your best to step away.

When your curiosity inevitably gets the best of you, don't panic.
Without any initial browning, the pale, bland, half-crunchy, half-tender
broccoli or green beans will shock you. Replace the lid, and give
yourself a pep talk, knowing that even experienced cooks usually become
alarmed at this point, too. Every time I employ this method, I spend at
least an hour convinced I've completely forgotten how to cook.

But reliably, that incredible transformation will eventually occur.
Overgrown fennel will grow buttery and soft enough to eat with a spoon.
Broccoli rabe, stems and all, will become mildly bittersweet. Time and
gentle heat will weave even celery --- hardly ever considered worthy of
its own platter --- into velvet. Standing at the stove, you'll eat
forkful after forkful of these vegetables, marveling as you think, ``If
only vegetables had tasted like this when I was a kid.''

\textbf{Recipe:}
\href{https://cooking.nytimes3xbfgragh.onion/recipes/1018907-long-cooked-vegetables}{Long-Cooked
Vegetables}

Advertisement

\protect\hyperlink{after-bottom}{Continue reading the main story}

\hypertarget{site-index}{%
\subsection{Site Index}\label{site-index}}

\hypertarget{site-information-navigation}{%
\subsection{Site Information
Navigation}\label{site-information-navigation}}

\begin{itemize}
\tightlist
\item
  \href{https://help.nytimes3xbfgragh.onion/hc/en-us/articles/115014792127-Copyright-notice}{©~2020~The
  New York Times Company}
\end{itemize}

\begin{itemize}
\tightlist
\item
  \href{https://www.nytco.com/}{NYTCo}
\item
  \href{https://help.nytimes3xbfgragh.onion/hc/en-us/articles/115015385887-Contact-Us}{Contact
  Us}
\item
  \href{https://www.nytco.com/careers/}{Work with us}
\item
  \href{https://nytmediakit.com/}{Advertise}
\item
  \href{http://www.tbrandstudio.com/}{T Brand Studio}
\item
  \href{https://www.nytimes3xbfgragh.onion/privacy/cookie-policy\#how-do-i-manage-trackers}{Your
  Ad Choices}
\item
  \href{https://www.nytimes3xbfgragh.onion/privacy}{Privacy}
\item
  \href{https://help.nytimes3xbfgragh.onion/hc/en-us/articles/115014893428-Terms-of-service}{Terms
  of Service}
\item
  \href{https://help.nytimes3xbfgragh.onion/hc/en-us/articles/115014893968-Terms-of-sale}{Terms
  of Sale}
\item
  \href{https://spiderbites.nytimes3xbfgragh.onion}{Site Map}
\item
  \href{https://help.nytimes3xbfgragh.onion/hc/en-us}{Help}
\item
  \href{https://www.nytimes3xbfgragh.onion/subscription?campaignId=37WXW}{Subscriptions}
\end{itemize}
