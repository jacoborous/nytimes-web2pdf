How to Stay Sane During a Solar Eclipse

\url{https://nyti.ms/2vINsEd}

\begin{itemize}
\item
\item
\item
\item
\item
\item
\end{itemize}

\includegraphics{https://static01.graylady3jvrrxbe.onion/images/2017/09/06/magazine/06mag-watching-top/06mag-watching-top-articleLarge.gif?quality=75\&auto=webp\&disable=upscale}

Sections

\protect\hyperlink{site-content}{Skip to
content}\protect\hyperlink{site-index}{Skip to site index}

\hypertarget{how-to-stay-sane-during-a-solar-eclipse}{%
\section{How to Stay Sane During a Solar
Eclipse}\label{how-to-stay-sane-during-a-solar-eclipse}}

The terrifying beauty of totality is best confronted in a crowd.

Credit...Illustration by Geoff McFetridge

Supported by

\protect\hyperlink{after-sponsor}{Continue reading the main story}

By Helen Macdonald

\begin{itemize}
\item
  Aug. 18, 2017
\item
  \begin{itemize}
  \item
  \item
  \item
  \item
  \item
  \item
  \end{itemize}
\end{itemize}

In March 2006, I stood on a crowded beach in Turkey and waited until, at
the allotted time, with a chorus of screams and cheers and whistles and
applause, the sun slid away, and impossibly, impossibly, we saw above us
a stretch of black sky and in the middle of it a hole, blacker than
anything I'd ever seen, fringed with a ring of soft white fire. My heart
jumped up to my throat, and my eyes grew hot with tears. I fell to my
knees, feeling tiny and huge, and as lonely as I've ever been, but also
astonishingly close to the crowds around me.

Totality --- that point of a solar eclipse when the sun is entirely
covered by the moon --- is incomprehensible. Your mind can't grasp any
of it: not the dark, nor the sunset clouds on the horizon, nor the
stars; just that extraordinary wrongness, up there, that pulls the eyes
toward it. I stared up at the hole in the sky and then at the figures
around me, and became gripped by the conviction that my life was over;
that I was kneeling in the underworld in the company of all the shades
of the dead. It was bitterly cold. A loose wind blew through the
darkness. But then came third contact. From the lower edge of the blank,
black disk of the dead sun burst a perfect point of brilliance. It leapt
and burned, unthinkably fierce and bright, something absurdly like a
word. I'm not a person of faith, but even so, the sun's reappearance as
the moon drew away seemed like the first line of Genesis retold. Is it
all set to rights, now? I thought. Is all remade? From a bay tree,
struck into existence a moment ago, a songbird, a white-spectacled
bulbul, called a greeting to the new dawn.

\textbf{Long ago}, when I first decided I wanted to see a total solar
eclipse, I planned to do so in romantic solitude. I was in my 20s,
inclined to think myself the center of the universe and imagining the
eclipse to be an event in which the sun and moon --- and I --- would
line up to provoke some deep and abiding revelation. The presence of
other people would detract from the meaningfulness of it all, I thought,
convinced that the best way to experience the natural world was to seek
private communion with it. It's embarrassing to recall this conviction,
for in the years since, I've seen three total eclipses, and now the
thought of watching one alone horrifies me. Only among a crowd, a band
of others, can you fully experience an eclipse's atmosphere. When
totality begins, you feel a wordless solidarity with the people around
you as all language is ripped away. You communicate through yells,
whoops, wolf-howls, screams, wild laughter. Eclipses are properly
sublime events in the philosophical sense: That is, they are an
encounter with something so astonishing and terrifying that you lose all
capacity for reason. Only afterward is it possible to speak lucidly of
what you have seen.

When I knelt on the beach in Turkey under an incomprehensible sky, it
was only the certainty that the people around me were seeing this, too,
that kept me clinging to any sense of reality. Witnessing an eclipse
wreaks havoc on your sense of self, on rational individuality. Back in
the 19th century, scientists on eclipse-viewing expeditions saw them as
a test of self-control. They were beset with anxiety that they might
fail to maintain their objectivity in the face of the overwhelming
emotions totality would bring. As the historian Alex Soojun-Kim Pang has
described, their hands shook so much that many could barely record their
data, and one observer was so overwhelmed by the 1871 eclipse in India
that he was forced to retreat to his room and plunge his head into
water. Charles Piazzi Smyth, the Edinburgh astronomer royal, wrote in
surprise that during the eclipse of 1851, it was not just the ``volatile
Frenchman'' who was ``carried away in the impulses of the moment'' but
also the ``staid Englishman'' and the ``stolid German.'' National
stereotypes aside, his concerns point to the exquisite contradiction of
solar eclipses. While their paths and timings can be predicted with
astonishing mathematical accuracy, their action is always to instill the
very opposite of empirical description and objective science; they
provoke a flood of primal awe. I knew, down to the precise second, when
the moon would obscure the sun over that Turkish beach. I did not know I
would fall to my knees and weep, or hear a man behind me whispering ``I
can't. . . . I can't. . . .'' over and over again.

Before my first eclipse, I was always afraid of crowds; growing up
watching television in Britain in the 1970s and 1980s was an education
in their dangers. The ones on-screen were mostly political
demonstrations, rock festivals or riots, and they were framed as things
to be feared for the same reason that 19th-century scientists feared
eclipses. Crowds made you forget yourself. Dissolving all individual
restraint, they coursed with uncontrollable instinct and emotion. This
conception of crowds as irrational and contagiously violent entities was
the legacy of European theorists like Gustave Le Bon, whose own views
were shaped by the political turmoil of late-19th-century France. To
him, crowds were barbarous agents of destruction. All this fed into the
nervousness I already felt about being in groups of people. I used to
spend a lot of time out on my own in woods and fields mostly because I
wanted to watch wild animals, which are hard to sneak up on as part of a
crowd. But there were more troubling reasons behind my desire to be
alone. It's reassuring to view the world on your own. You can gaze at a
landscape and see it peopled by things --- trees, clouds, hills and
valleys --- that have no voice except the ones you give them in your
imagination; none can challenge who you are. So often we see solitary
contemplation as simply the correct way to engage with nature. But it is
always a political act, bringing freedom from the pressures of other
minds, other interpretations, other consciousnesses competing with your
own.

\textbf{Another way} to escape social conflict is to make yourself part
of a crowd of people who see the world the same way you do and value the
same things as you. We're familiar with the notion that America is a
land of rugged individualists, but it turns out that it has a long
tradition of sociability when it comes to seeking out the sublime. As
the historian David Nye has argued, groups of tourists who traveled to
natural sites like the Grand Canyon or to witness awe-inspiring events
like space-program launches were engaged in a distinctly American form
of secular pilgrimage. Their experience of the sublime supported the
idea of American exceptionalism, the marveling crowds newly assured of
the singular grandeur and importance of their nation. But the millions
of tourists who will flock to this summer's event won't see something
that time has fashioned from American rock and earth, or something
wrought by American ingenuity, but a passing shadow cast across the
nation from celestial bodies above.

The event this August has been called the Great American Eclipse, and it
seems to me to chime with the country's current struggles: between
reason and unreason, individuality and crowd consciousness, belonging
and difference. The most distressing present-day crowds are those whose
politics are built from fear and outrage against otherness. They are
entities that define themselves by virtue of what they are against. Yet
the simple fact about an eclipse crowd is that it cannot work in this
way. Confronting something like the absolute, all our differences are
moot. When you stand and watch the death of the sun and see it reborn,
there can be no them, only us. In Turkey, the crowd around me changed as
the eclipse progressed. At first I stood there amid a collection of
strangers. Then, after first contact, as the moon began to eat away at
the sun, I began to feel I was part of something resembling a religious
congregation. During the gloom of second contact, I felt lost and alone,
but also completely merged with the crowd that shared my experience ---
a contradiction rooted in the overwhelming recognition of human
mortality the eclipse had provoked. Treading my way across the sand
after it was over, I couldn't stop myself grinning at everyone I passed,
and several times exchanged hugs with complete strangers who no longer
seemed strangers at all.

\textbf{On that day,} thousands of pairs of eyes had been focused upon a
single point in the sky; we shared a vision that made us one. But even
when the weather is poor and the sun obscured by clouds, the crowds
under an eclipse can work a rare and life-changing magic. Seven years
before the event in Turkey, my father and I walked onto a packed beach
in Cornwall to witness the first total eclipse to cross Britain in more
than 70 years. We found ourselves standing-room among milling tour
guides, eclipse chasers, schoolchildren, camera crews, teenagers waving
glow sticks, New Age travelers and folks in fancy dress. It was my first
eclipse. I was anxious about all the people and still clinging to that
sophomoric intuition that a revelation would come only if none of them
were there. Depressingly, the sky was thick with clouds; I knew none of
us would see anything other than darkness when totality came. But as the
light dimmed after second contact, the atmosphere grew electric, and the
crowd became suddenly important, a palpable presence in my mind. I felt
a fleeting, urgent concern for the safety of everyone around me as the
world rolled, and the moon, too, and night slammed down on us. Though I
could hardly see a hand held in front of my face, far out across the sea
hung clouds tinted the eerie sunset shade of faded photographs of atomic
tests, and beyond them clear blue day. Shock, then, and a sense of
creeping dread.

And then the revelation came. It wasn't what I'd expected, for it wasn't
focused up there in the sky, but down here with us all, as the crowds
that lined the Atlantic shore raised cameras to commemorate totality,
and as they flashed a wave of particulate light crashed along the dark
beach and flooded across to the other side of the bay, making the whole
coast a glittering field of stars. Each fugitive point of light was a
different person. I remember laughing out loud. I'd wanted a solitary
revelation, but I was given something else. An overwhelming sense of
humanity, and of what it is made --- a host of individual lights shining
briefly against the oncoming darkness.

Advertisement

\protect\hyperlink{after-bottom}{Continue reading the main story}

\hypertarget{site-index}{%
\subsection{Site Index}\label{site-index}}

\hypertarget{site-information-navigation}{%
\subsection{Site Information
Navigation}\label{site-information-navigation}}

\begin{itemize}
\tightlist
\item
  \href{https://help.nytimes3xbfgragh.onion/hc/en-us/articles/115014792127-Copyright-notice}{©~2020~The
  New York Times Company}
\end{itemize}

\begin{itemize}
\tightlist
\item
  \href{https://www.nytco.com/}{NYTCo}
\item
  \href{https://help.nytimes3xbfgragh.onion/hc/en-us/articles/115015385887-Contact-Us}{Contact
  Us}
\item
  \href{https://www.nytco.com/careers/}{Work with us}
\item
  \href{https://nytmediakit.com/}{Advertise}
\item
  \href{http://www.tbrandstudio.com/}{T Brand Studio}
\item
  \href{https://www.nytimes3xbfgragh.onion/privacy/cookie-policy\#how-do-i-manage-trackers}{Your
  Ad Choices}
\item
  \href{https://www.nytimes3xbfgragh.onion/privacy}{Privacy}
\item
  \href{https://help.nytimes3xbfgragh.onion/hc/en-us/articles/115014893428-Terms-of-service}{Terms
  of Service}
\item
  \href{https://help.nytimes3xbfgragh.onion/hc/en-us/articles/115014893968-Terms-of-sale}{Terms
  of Sale}
\item
  \href{https://spiderbites.nytimes3xbfgragh.onion}{Site Map}
\item
  \href{https://help.nytimes3xbfgragh.onion/hc/en-us}{Help}
\item
  \href{https://www.nytimes3xbfgragh.onion/subscription?campaignId=37WXW}{Subscriptions}
\end{itemize}
