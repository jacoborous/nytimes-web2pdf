Sections

SEARCH

\protect\hyperlink{site-content}{Skip to
content}\protect\hyperlink{site-index}{Skip to site index}

\href{https://myaccount.nytimes3xbfgragh.onion/auth/login?response_type=cookie\&client_id=vi}{}

\href{https://www.nytimes3xbfgragh.onion/section/todayspaper}{Today's
Paper}

Can I Let My Friend Pay Off My Mortgages?

\url{https://nyti.ms/2hO6nJR}

\begin{itemize}
\item
\item
\item
\item
\item
\item
\end{itemize}

Advertisement

\protect\hyperlink{after-top}{Continue reading the main story}

Supported by

\protect\hyperlink{after-sponsor}{Continue reading the main story}

\href{/column/the-ethicist}{The Ethicist}

\hypertarget{can-i-let-my-friend-pay-off-my-mortgages}{%
\section{Can I Let My Friend Pay Off My
Mortgages?}\label{can-i-let-my-friend-pay-off-my-mortgages}}

By Kwame Anthony Appiah

\begin{itemize}
\item
  Nov. 20, 2017
\item
  \begin{itemize}
  \item
  \item
  \item
  \item
  \item
  \item
  \end{itemize}
\end{itemize}

\includegraphics{https://static01.graylady3jvrrxbe.onion/images/2017/11/26/magazine/26ethicist/26ethicist-articleLarge.jpg?quality=75\&auto=webp\&disable=upscale}

\emph{My closest American friend here in Japan, of more than 30 years,
is worried about me and wants to pay off my mortgages. He says he
doesn't want to be paid back; he just wants to make sure I am out of
debt before he dies. He is not dying, but he is 98. He has been
mentioning this more and more, and says he wants to write a check the
next time we meet. I never talk about this with him unless he brings up
the subject. The amount he would give me would come to about 3 percent
of his assets. It would have no impact on his financial needs. And
frankly, it would be helpful for me.}

\emph{Yet, I have a gnawing feeling that I would be taking advantage of
him. Or that I have unconsciously manipulated him. But I can't think of
anything I did. I've never asked for or taken money from him. I got
myself out of credit-card and student-loan debt --- he offered to help,
but I declined. My only remaining debts are my mortgages. I think he is
pleased that I managed to get my finances in shape.}

\emph{Even though he is 98, he is not suffering from dementia. However,
he isn't as capable of doing things as he once was, and he depends on my
help more and more --- with his computer and finances, and to serve as a
translator.}

\emph{Should I decline and feel noble? Or should I be practical and take
the offer in the spirit he intends?} Name Withheld

\textbf{You don't mention} your age, but the relationship you have with
this man sounds very like the relationship between a father and a son.
(We've withheld your name but know that you're a man.) His generous
offer is in the spirit of paternal love, and your services to him, your
clear concern for his interests, your scruples and your wish not to take
advantage of him are like those of a loving son. You are lucky to have
this relationship, as is he. He can afford to do what he proposes to do.
It will allow him to express his gratitude and his friendship. It will
make him happy. And it will permit you to have one less thing to worry
about.

If you were preying on some emotional or physical or mental
vulnerability, you'd be guilty of exploitation. But nothing in your
letter suggests that. And even if the parental model isn't quite right
--- even if, as sounds possible, he is a little in love with you ---
your reciprocated caring and affection mean that you're not taking
advantage of him. Go ahead. In accepting his gift, you'll be making a
gift of your own.

\emph{I recently spoke on the phone with an old friend from college.
During the call she mentioned that her son is taking a drug for A.D.H.D.
and that it really helps him focus. I know there is controversy
surrounding this class of drugs, but I didn't feel comfortable bringing
that up. I assume she has looked into the pros and cons, and I know her
mother is a psychiatrist. But should I mention my concerns nevertheless?
Or should my concerns about seeming a busybody outweigh concerns about
her son's future health?} Name Withheld

\textbf{The answer to} your last question is easy: No. If the only
issues were being thought to be a busybody and the possibility that a
child would be seriously harmed, the latter would win out. But you'd be
raising this issue with a mother who has a psychiatrist parent and,
presumably, another doctor writing the prescriptions. You don't have a
basis for thinking that you're better-placed to see the risks than your
friend is. So one reason you might have hesitated to say something is
that you didn't want to insult her by suggesting she was either ignorant
or careless about her child's welfare.

Still, I doubt she'd have taken affront had you said at the time, ``Oh,
I thought I'd read there could be bad long-term consequences with those
drugs.'' Your friend could have told you if she'd looked into them (or
rushed off to do research if she hadn't). The more time that passes,
though, the more awkward your interjection becomes.

You can reassure yourself that parents these days are likely to Google
the names of drugs prescribed for their children and look up side
effects. If you do this for two of the major drugs prescribed for
A.D.H.D. --- methylphenidate (e.g., Concerta) and amphetamines (e.g.,
Adderall) --- you find arguments suggesting or denying significant risks
to long-term use. It's an issue that a concerned mother would take up
with a doctor. At this point, you should probably direct your attentions
elsewhere.

\emph{I teach at a prestigious private art school. Every year, we take
in 600 or so young people with little understanding of how the arts work
as an industry. We charge a very high tuition, offer almost no
scholarships and load them up with a lot of debt. Even though we claim
to offer ``career planning,'' the illusions of our students are not
addressed. Our graduates, even those with a degree in design, rarely
find a job in their field. Those who do rarely last long before
realizing that they are in a hopeless situation. Most have given up on
art within a few years of graduation. When I encounter them, they convey
a considerable amount of bitterness about student loans and the
education they received. By preying on their naïveté and ignorance, I
feel that we are essentially robbing our students. Some colleagues argue
that we are not doing anything that Harvard or N.Y.U. isn't doing ---
that we are simply a ``special place for a certain kind of young
adult.'' I do not have tenure and have no influence in admissions or
tuition policy. Without this job, I am virtually unemployable at my age.
Is it wrong to take the ``caveat emptor'' approach and let these naïve
young people continue to pay me through their student loans?} Name
Withheld

\textbf{Having taught at} Harvard and N.Y.U., I confess to thinking that
the education they offer is not a matter of preying on ``naïveté and
ignorance.'' But even if it were, the bad behavior of other institutions
wouldn't excuse that of your own. The school would do well to provide,
and draw attention to, reliable information about the career prospects
of graduates. It might lose some students this way, but it would gain
something as well; a trail of bitter graduates is not a sound basis for
seeking alumni support.

Are you obliged to take a public stand on this, at the expense of your
career? You are not. Especially because it's highly unlikely that the
school will change its practices as a result. But you certainly
shouldn't mislead any students who ask you about their prospects. The
trope of the ``starving artist'' got established for a reason.

Advertisement

\protect\hyperlink{after-bottom}{Continue reading the main story}

\hypertarget{site-index}{%
\subsection{Site Index}\label{site-index}}

\hypertarget{site-information-navigation}{%
\subsection{Site Information
Navigation}\label{site-information-navigation}}

\begin{itemize}
\tightlist
\item
  \href{https://help.nytimes3xbfgragh.onion/hc/en-us/articles/115014792127-Copyright-notice}{©~2020~The
  New York Times Company}
\end{itemize}

\begin{itemize}
\tightlist
\item
  \href{https://www.nytco.com/}{NYTCo}
\item
  \href{https://help.nytimes3xbfgragh.onion/hc/en-us/articles/115015385887-Contact-Us}{Contact
  Us}
\item
  \href{https://www.nytco.com/careers/}{Work with us}
\item
  \href{https://nytmediakit.com/}{Advertise}
\item
  \href{http://www.tbrandstudio.com/}{T Brand Studio}
\item
  \href{https://www.nytimes3xbfgragh.onion/privacy/cookie-policy\#how-do-i-manage-trackers}{Your
  Ad Choices}
\item
  \href{https://www.nytimes3xbfgragh.onion/privacy}{Privacy}
\item
  \href{https://help.nytimes3xbfgragh.onion/hc/en-us/articles/115014893428-Terms-of-service}{Terms
  of Service}
\item
  \href{https://help.nytimes3xbfgragh.onion/hc/en-us/articles/115014893968-Terms-of-sale}{Terms
  of Sale}
\item
  \href{https://spiderbites.nytimes3xbfgragh.onion}{Site Map}
\item
  \href{https://help.nytimes3xbfgragh.onion/hc/en-us}{Help}
\item
  \href{https://www.nytimes3xbfgragh.onion/subscription?campaignId=37WXW}{Subscriptions}
\end{itemize}
