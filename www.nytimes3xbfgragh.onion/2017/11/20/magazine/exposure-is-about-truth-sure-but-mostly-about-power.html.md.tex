Sections

SEARCH

\protect\hyperlink{site-content}{Skip to
content}\protect\hyperlink{site-index}{Skip to site index}

\href{https://myaccount.nytimes3xbfgragh.onion/auth/login?response_type=cookie\&client_id=vi}{}

\href{https://www.nytimes3xbfgragh.onion/section/todayspaper}{Today's
Paper}

`Exposure' Is About Truth, Sure, but Mostly About Power

\url{https://nyti.ms/2hMPmzD}

\begin{itemize}
\item
\item
\item
\item
\item
\item
\end{itemize}

Advertisement

\protect\hyperlink{after-top}{Continue reading the main story}

Supported by

\protect\hyperlink{after-sponsor}{Continue reading the main story}

\href{/column/first-words}{First Words}

\hypertarget{exposure-is-about-truth-sure-but-mostly-about-power}{%
\section{`Exposure' Is About Truth, Sure, but Mostly About
Power}\label{exposure-is-about-truth-sure-but-mostly-about-power}}

\includegraphics{https://static01.graylady3jvrrxbe.onion/images/2017/11/26/magazine/26mag-firstwords1/26mag-26firstwords-t_CA0-articleLarge.jpg?quality=75\&auto=webp\&disable=upscale}

By Carina Chocano

\begin{itemize}
\item
  Nov. 20, 2017
\item
  \begin{itemize}
  \item
  \item
  \item
  \item
  \item
  \item
  \end{itemize}
\end{itemize}

For a town so in love with doomsday scenarios, Hollywood has never been
big on revelations about itself. That's what apocalypse is, literally:
disclosure of knowledge. In biblical Greek, ``apokalypsis'' means
``uncovering'' or ``unveiling.'' It refers to the moment when a
long-buried truth is finally exposed. The secret could be anything --- a
plague of worms; corruption; a culture of flagrant harassment,
exploitation and abuse. What actually ends the world as we know it is
the revelation itself, being shown the thing we had agreed not to see.
Once we see that civilization had it coming, we can consign everything
to the purifying flames.

Exposure is terrifying. You can die of it. It suggests that there is
some sinister, lurking thing, shrouded in pretense: We know it's in
there, and we know it might come out. It's as though we're dwelling in
two parallel realities --- one hidden but real, the other visible but
false --- and this unstable doubleness is poised to blow up in our faces
at any moment. Hollywood's entire business model is built on this kind
of explosion. Movies are made out of well-timed revelations ---
third-act eruptions, secrets unveiled, faces unmasked, true identities
uncovered. And popular entertainment tucks apocalyptic revelations into
even the most anodyne entertainments. On home-remodeling shows, couples
are escorted to different houses, made to choose one, then enjoined to
sledgehammer the walls and expose the structure's commendable bones ---
and its alarming rot. ``RuPaul's Drag Race'' has helped turn cries of
``Expose her!'' into a viral phenomenon --- a call to zero in on
someone's flaws and show the world who he or she \emph{really} is.
Journalism, too, craves the exposé of hidden wrongdoing --- when it's
not reporting on how, say, White House aides are worrying over their
legal exposure in a special-counsel investigation. Exposure always means
vulnerability: You can be exposed to a fatal chill, a deadly virus, a
sunburn, a lawsuit. Sometimes young actors and writers are offered
favorable exposure in lieu of payment, but, of course, you can die of
that kind, too.

Lately, each day has been a carnival of exposure, as we've watched the
stones overturned to reveal more and more supposedly great men as
criminals, perverts or frauds. Some people are surprised by this, but
many are not. In several cases, the big revelation was already old news,
the kind of thing we used to call an ``open secret'' --- something that
wasn't O.K. to acknowledge on the record, lest it come back to bite you.
Open secrets are par for the course for women in the workplace; they are
the pact we make in order to be able to participate. Why do we do it?
Have we all been coerced or threatened? Have we all signed nondisclosure
agreements and waivers? Have we all tacitly agreed to participate in
this weird hierarchical system, without ever being asked how we like it,
to preserve the status quo and the narrative that everything is fine,
because familiar narratives are comforting, and exposure is
destabilizing, and we've been given no choice? Are we brainwashed? Sure.
Of course. Some, maybe all, of the above.

If the wrong people weren't so often protected, there would be no need
to expose them --- no need to expose what hadn't been concealed in the
first place. In the 2007 documentary ``Girl 27,'' the writer David Stenn
tells the story of an underage dancer named Patricia Douglas who was
raped at an MGM stag party by a studio salesman in 1937. When she went
public, filing a criminal complaint and suing the studio, she was
smeared by the press. She spent the rest of her life in hiding, while
the infamous MGM executive, producer and fixer Eddie Mannix is said to
have boasted that ``we had her killed.'' Women like Douglas were taught
that their attempts at exposing the truth would fail and that their
lives would be ruined. They were taught to take it or leave it, but
mostly they were taught to shut up about it. They were still being
taught that more than 70 years later.

Exposure is about truth, sure, but it's mostly about power --- about the
relationship of truth to power. When a powerful man exposes himself by
forcing his nakedness on others, he's commanding their attention in a
violent way, making them see what they don't want to see. In the moment
of exposure, he's not the one who feels vulnerable; they are.
Conversely, when that man is exposed as a monster, he is shown in a
different light. The perspective is forced, and he is revealed to be
something else.

In the case of the producer Harvey Weinstein, now accused of assault or
harassment by more than 80 women, it may be that his victims became more
powerful over time. (When you harass seemingly every young actress who
crosses your path, it's inevitable that at least some of them will rise
to a position of power.) When it seemed that women might come forward to
expose him to the authorities, Weinstein paid informants to gather
intelligence on his accusers' psychological and sexual histories --- the
usual scramble to accuse the accuser herself of being a fraud, a gold
digger, a disgruntled loser, ``disturbed.'' Exposing others is scary
because we fear our own exposure; when we say, ``Someone did this to
me,'' we are asking to be looked at in a new, unpleasant, disempowered
way. Who will get the last word? Who will get to delegitimize whom?

In Hollywood, as in politics, exposure is something you want coming in
but never going out. The fresh face --- say, Mariel Hemingway at the
start of her career, hoping for a huge role in Woody Allen's
``Manhattan'' --- needs exposure. The predatory potentate --- say, Allen
himself, when he started dating Mia Farrow's adopted daughter Soon-Yi
Previn --- does not. What he needs is cover, leverage and all the good
will he has banked over the years. He may not even need to withdraw very
large amounts. He could get himself elected president, if he wanted. He
could do anything. When you're a star, they let you do it. The comedian
Louis C.K., who has copped to masturbating in front of freaked-out
colleagues, not only exposed himself but also toyed with artistic
self-exposure: He pulled a Raskolnikov, giving himself away in his
comedy and in a dangerously self-referential movie. How much could he
reveal in public before being exposed? Until recently? Almost
everything.

The sheer volume of recent exposures --- from what happens between two
people in a hotel room to what happens in a huge political campaign ---
certainly makes it feel as though we're living through an apocalyptic
moment. It's elating, or overwhelming, depending on your experience.
Allegations ranging from gross harassment to assault have been leveled
against Weinstein, against the writer and director James Toback, against
the former New Republic editor Leon Wieseltier, against the former NPR
and New York Times editor Michael Oreskes, against the former Amazon
Studios chief Roy Price, against the journalist and author Mark
Halperin, against actors and talent agents, against a senator and a
Senate candidate, against ``media men'' listed in a crowdsourced
spreadsheet.

Countless streams of revelations are coursing everywhere around us ---
from the Paradise Papers, which expose the flow of wealthy people's
capital through secret banking channels, to the steady drip of emails
and direct messages and surprising meetings uncovered in the
Trump-Russia probe. It is, in many ways, exhilarating. It's as if all
the rugs are being lifted, all the demons released.

It's a reminder, once again, that language is power, that storytelling
is power, that mythmaking is control. The postwar French critic Roland
Barthes was big on exposure, too --- only he was interested in revealing
the lies in our most innocuous pop-culture images and ``common sense''
beliefs. ``I wanted to expose in the decorative display of
what-goes-without-saying the ideological abuse I believed was hidden
there,'' he wrote. It takes power to expose abuse, and courage. It takes
a culture that believes the powerless aren't exposing for the exposure.

In ancient Greece, Pandora's box was not actually a box but a jar, or a
clay pot with a lid, that was kept in the kitchen, where the women were
also kept. Maybe it contained evil --- or maybe it just concealed it.
Maybe Pandora let the evil out, or maybe she blew the lid off what was
really going on back there, where nobody else could see it. Anyway, the
truth got out, and all hell broke loose, leaving behind only hope.

Advertisement

\protect\hyperlink{after-bottom}{Continue reading the main story}

\hypertarget{site-index}{%
\subsection{Site Index}\label{site-index}}

\hypertarget{site-information-navigation}{%
\subsection{Site Information
Navigation}\label{site-information-navigation}}

\begin{itemize}
\tightlist
\item
  \href{https://help.nytimes3xbfgragh.onion/hc/en-us/articles/115014792127-Copyright-notice}{©~2020~The
  New York Times Company}
\end{itemize}

\begin{itemize}
\tightlist
\item
  \href{https://www.nytco.com/}{NYTCo}
\item
  \href{https://help.nytimes3xbfgragh.onion/hc/en-us/articles/115015385887-Contact-Us}{Contact
  Us}
\item
  \href{https://www.nytco.com/careers/}{Work with us}
\item
  \href{https://nytmediakit.com/}{Advertise}
\item
  \href{http://www.tbrandstudio.com/}{T Brand Studio}
\item
  \href{https://www.nytimes3xbfgragh.onion/privacy/cookie-policy\#how-do-i-manage-trackers}{Your
  Ad Choices}
\item
  \href{https://www.nytimes3xbfgragh.onion/privacy}{Privacy}
\item
  \href{https://help.nytimes3xbfgragh.onion/hc/en-us/articles/115014893428-Terms-of-service}{Terms
  of Service}
\item
  \href{https://help.nytimes3xbfgragh.onion/hc/en-us/articles/115014893968-Terms-of-sale}{Terms
  of Sale}
\item
  \href{https://spiderbites.nytimes3xbfgragh.onion}{Site Map}
\item
  \href{https://help.nytimes3xbfgragh.onion/hc/en-us}{Help}
\item
  \href{https://www.nytimes3xbfgragh.onion/subscription?campaignId=37WXW}{Subscriptions}
\end{itemize}
