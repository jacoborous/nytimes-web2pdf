Sections

SEARCH

\protect\hyperlink{site-content}{Skip to
content}\protect\hyperlink{site-index}{Skip to site index}

\href{https://www.nytimes3xbfgragh.onion/section/world/asia}{Asia
Pacific}

\href{https://myaccount.nytimes3xbfgragh.onion/auth/login?response_type=cookie\&client_id=vi}{}

\href{https://www.nytimes3xbfgragh.onion/section/todayspaper}{Today's
Paper}

\href{/section/world/asia}{Asia Pacific}\textbar{}Leading Western
Publisher Bows to Chinese Censorship

\url{https://nyti.ms/2z6v0Wr}

\begin{itemize}
\item
\item
\item
\item
\item
\end{itemize}

Advertisement

\protect\hyperlink{after-top}{Continue reading the main story}

Supported by

\protect\hyperlink{after-sponsor}{Continue reading the main story}

\hypertarget{leading-western-publisher-bows-to-chinese-censorship}{%
\section{Leading Western Publisher Bows to Chinese
Censorship}\label{leading-western-publisher-bows-to-chinese-censorship}}

\includegraphics{https://static01.graylady3jvrrxbe.onion/images/2017/11/02/world/02china-censor/02china-censor-articleLarge.jpg?quality=75\&auto=webp\&disable=upscale}

By
\href{https://www.nytimes3xbfgragh.onion/by/javier-c-hernandez}{Javier
C. Hernández}

\begin{itemize}
\item
  Nov. 1, 2017
\item
  \begin{itemize}
  \item
  \item
  \item
  \item
  \item
  \end{itemize}
\end{itemize}

\href{https://cn.nytimes3xbfgragh.onion/china/20171102/china-springer-nature-censorship/}{阅读简体中文版}\href{https://cn.nytimes3xbfgragh.onion/china/20171102/china-springer-nature-censorship/zh-hant/}{閱讀繁體中文版}

BEIJING --- One of the world's largest academic publishers was
criticized on Wednesday for bowing to pressure from the Chinese
government to block access to hundreds of articles on its Chinese
website.

\href{http://www.springernature.com/us/}{Springer Nature}, whose
publications include Nature and Scientific American, acknowledged that
at the government's request, it had removed articles from its mainland
site that touch on topics the ruling Communist Party considers
sensitive, including Taiwan, Tibet, human rights and elite politics.

The publisher defended its decision, saying that only 1 percent of its
content was inaccessible in mainland China.

Under President Xi Jinping, China has grown increasingly confident in
using its vast market as a bargaining chip, forcing foreign firms to
acquiesce to strict demands on free speech.

Academic publishers have become a popular target, part of Mr. Xi's
efforts to restrict the flow of ideas at universities.

In August, Cambridge University Press, one of the oldest publishing
houses,
\href{https://www.nytimes3xbfgragh.onion/2017/08/18/world/asia/cambridge-university-press-academic-freedom.html}{said
it had removed} more than 300 articles from the Chinese site of the
journal China Quarterly. The articles mentioned the 1989 Tiananmen
Square massacre, the Cultural Revolution and other topics deemed
inappropriate by the authorities. The publisher later
\href{https://www.nytimes3xbfgragh.onion/2017/08/21/world/asia/china-quarterly-cambridge-university-press-censorship-publisher-reverses-decision-to-bow-to-chinas-censors.html?_r=0}{reversed
course} after an outcry.

Several scholars on Wednesday denounced Springer Nature's censorship in
the mainland, which was first reported by The Financial Times. They
accused the company of prioritizing profit over free speech.

``Springer's censorship is a disservice to everyone,'' said
\href{https://www.mq.edu.au/about_us/faculties_and_departments/faculty_of_arts/department_of_international_studies/staff/staff_chinese_studies/dr_kevin_carrico}{Kevin
Carrico}, a China scholar at Macquarie University in Sydney, Australia.
``Springer's success relies on its authors and its readers, and both are
being cheated in this arrangement.''

\href{http://www.lse.ac.uk/researchAndexpertise/experts/profile.aspx?KeyValue=m.e.cox\%40lse.ac.uk}{Michael
Cox}, a scholar who serves as editor of the International Politics
journal, one of the Springer Nature publications that is being censored
in China, said he would press the publisher to reconsider.

``My first priority is to maintain and defend the principle of academic
freedom,'' said Mr. Cox, who is also professor emeritus at the London
School of Economics and Political Science.

Since coming to power in 2012, Mr. Xi has significantly tightened
control of the internet. He has also encouraged universities to be more
vigilant about the spread of Western influences. While foreign news
sites and social media portals are widely blocked in China, overseas
academic journals had largely avoided mass censorship until recently.

Susie Winter, the publisher's director of communications and engagement,
called the company's action ``deeply regrettable,'' but said that it had
been taken ``to prevent a much greater impact on our customers and
authors.''

``This is not editorial censorship and does not affect the content we
publish or make accessible elsewhere in the world,'' she said.

Another Springer Nature publication, the Journal of Chinese Political
Science, is also being censored in mainland China.

Many of the censored articles focus on issues that government has long
deemed sensitive, including human rights. But even articles that only
briefly touch on these topics appear to be blocked, suggesting that
Springer Nature is using broad criteria in deciding which content to
censor. For example, one censored article focuses on the disputed South
China Sea, a topic widely covered in China's state-run news media.

Springer Nature did not elaborate on its methods, saying only that it
deferred to the local authorities in deciding which articles to block.

Several scholars expressed concern that Springer Nature had seemingly
given the Chinese authorities such expansive authority.

``This is not even effective censorship,'' Professor Carrico said. ``It
takes such a clumsy broad-brush approach that even completely
uncontroversial articles could be blocked.''

Advertisement

\protect\hyperlink{after-bottom}{Continue reading the main story}

\hypertarget{site-index}{%
\subsection{Site Index}\label{site-index}}

\hypertarget{site-information-navigation}{%
\subsection{Site Information
Navigation}\label{site-information-navigation}}

\begin{itemize}
\tightlist
\item
  \href{https://help.nytimes3xbfgragh.onion/hc/en-us/articles/115014792127-Copyright-notice}{©~2020~The
  New York Times Company}
\end{itemize}

\begin{itemize}
\tightlist
\item
  \href{https://www.nytco.com/}{NYTCo}
\item
  \href{https://help.nytimes3xbfgragh.onion/hc/en-us/articles/115015385887-Contact-Us}{Contact
  Us}
\item
  \href{https://www.nytco.com/careers/}{Work with us}
\item
  \href{https://nytmediakit.com/}{Advertise}
\item
  \href{http://www.tbrandstudio.com/}{T Brand Studio}
\item
  \href{https://www.nytimes3xbfgragh.onion/privacy/cookie-policy\#how-do-i-manage-trackers}{Your
  Ad Choices}
\item
  \href{https://www.nytimes3xbfgragh.onion/privacy}{Privacy}
\item
  \href{https://help.nytimes3xbfgragh.onion/hc/en-us/articles/115014893428-Terms-of-service}{Terms
  of Service}
\item
  \href{https://help.nytimes3xbfgragh.onion/hc/en-us/articles/115014893968-Terms-of-sale}{Terms
  of Sale}
\item
  \href{https://spiderbites.nytimes3xbfgragh.onion}{Site Map}
\item
  \href{https://help.nytimes3xbfgragh.onion/hc/en-us}{Help}
\item
  \href{https://www.nytimes3xbfgragh.onion/subscription?campaignId=37WXW}{Subscriptions}
\end{itemize}
