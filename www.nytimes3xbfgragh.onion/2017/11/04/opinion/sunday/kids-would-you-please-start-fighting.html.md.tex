Sections

SEARCH

\protect\hyperlink{site-content}{Skip to
content}\protect\hyperlink{site-index}{Skip to site index}

\href{https://www.nytimes3xbfgragh.onion/section/opinion/sunday}{Sunday
Review}

\href{https://myaccount.nytimes3xbfgragh.onion/auth/login?response_type=cookie\&client_id=vi}{}

\href{https://www.nytimes3xbfgragh.onion/section/todayspaper}{Today's
Paper}

\href{/section/opinion/sunday}{Sunday Review}\textbar{}Kids, Would You
Please Start Fighting?

\url{https://nyti.ms/2j358Xb}

\begin{itemize}
\item
\item
\item
\item
\item
\item
\end{itemize}

Advertisement

\protect\hyperlink{after-top}{Continue reading the main story}

Supported by

\protect\hyperlink{after-sponsor}{Continue reading the main story}

\href{/section/opinion}{Opinion}

\hypertarget{kids-would-you-please-start-fighting}{%
\section{Kids, Would You Please Start
Fighting?}\label{kids-would-you-please-start-fighting}}

\href{https://www.nytimes3xbfgragh.onion/column/adam-grant}{\includegraphics{https://static01.graylady3jvrrxbe.onion/images/2015/03/16/opinion/Grant-Adam-circular/Grant-Adam-circular-thumbLarge-v4.jpg}}

By \href{https://www.nytimes3xbfgragh.onion/column/adam-grant}{Adam
Grant}

\begin{itemize}
\item
  Nov. 4, 2017
\item
  \begin{itemize}
  \item
  \item
  \item
  \item
  \item
  \item
  \end{itemize}
\end{itemize}

\href{https://cn.nytimes3xbfgragh.onion/opinion/20171115/kids-would-you-please-start-fighting/}{阅读简体中文版}\href{https://cn.nytimes3xbfgragh.onion/opinion/20171115/kids-would-you-please-start-fighting/zh-hant/}{閱讀繁體中文版}

\includegraphics{https://static01.graylady3jvrrxbe.onion/images/2017/11/05/opinion/sunday/05grant/05grant-articleLarge.jpg?quality=75\&auto=webp\&disable=upscale}

When Wilbur and Orville Wright finished their flight at Kitty Hawk,
Americans celebrated the brotherly bond. The brothers had grown up
playing together, they had been in the newspaper business together, they
had built an airplane together. They even
\href{https://www.pbs.org/wnet/historyofus/web11/segment1_p.html}{said}
they ``thought together.''

These are our images of creativity: filled with harmony. Innovation, we
think, is something magical that happens when people find synchrony
together. The melodies of Rodgers blend with the lyrics of Hammerstein.
It's why one of the cardinal rules of
\href{https://hbr.org/2017/05/your-team-is-brainstorming-all-wrong}{brainstorming}
is ``withhold criticism.'' You want people to build on one another's
ideas, not shoot them down. But that's not how creativity really
happens.

When the Wright brothers said they thought together, what they really
meant is that they argued together. One of their pivotal decisions was
the design of a propeller for their plane. They squabbled for weeks,
often shouting back and forth for hours. ``After long arguments we often
found ourselves in the ludicrous position of each having been converted
to the other's side,'' Orville
\href{https://www.smithsonianbooks.com/store/aviation-military-history/published-writings-wilbur-and-orville-wright/}{reflected},
``with no more agreement than when the discussion began.'' Only after
thoroughly decimating each other's arguments did it dawn on them that
they were both wrong. They needed not one but two propellers, which
could be spun in opposite directions to create a kind of rotating wing.
``I don't think they really got mad,'' their mechanic marveled, ``but
they sure got awfully hot.''

The skill to get hot without getting mad --- to have a good argument
that doesn't become personal --- is critical in life. But it's one that
few parents teach to their children. We want to give kids a stable home,
so we stop siblings from quarreling and we have our own arguments behind
closed doors. Yet if kids never get exposed to disagreement, we'll end
up limiting their creativity.

We've known groupthink is a problem for a long time: We've watched
ill-fated wars unfold after dissenting voices were silenced. But
teaching kids to argue is more important than ever. Now we live in a
time when voices that might offend are silenced on college campuses,
when politics has become an untouchable topic in many circles, even more
fraught than religion or race. We should know better: Our legal system
is based on the idea that arguments are necessary for justice. For our
society to remain free and open, kids need to learn the value of open
disagreement.

It turns out that highly creative adults often grow up in families full
of tension. Not fistfights or personal insults, but real disagreements.
When adults in their early 30s were
\href{http://www.sciencedirect.com/science/article/pii/S0092656698922401}{asked}
to write imaginative stories, the most creative ones came from those
whose parents had the most conflict a quarter-century earlier. Their
parents had clashing views on how to raise children. They had different
values and attitudes and interests. And when highly creative architects
and scientists were compared with their technically skilled but less
original peers, the innovators often had
\href{http://psycnet.apa.org/record/1965-15301-001}{more friction} in
their families. As the psychologist Robert Albert
\href{https://books.google.com/books?id=MrFyrO_ZWPgC\&pg=PA175\&lpg=PA175\&dq=robert+s.+albert+creativity+wobble\&source=bl\&ots=qlsoQY7Uhm\&sig=CjY-9g1f0T-UCiNssGbvI1duOqg\&hl=en\&sa=X\&ved=0ahUKEwiCg7XtsvfUAhVMaD4KHQhTCvMQ6AEIJzAB\#v=onepage\&q=robert\%20s.\%20albert\%20creativity\%20wobble\&f=false}{put
it}, ``the creative person-to-be comes from a family that is anything
but harmonious, one with a `wobble.' ''

Wilbur and Orville Wright came from a wobbly family.
\href{http://articles.latimes.com/1989-09-03/books/bk-2324_1_wright-brothers}{Their
father}, a preacher, never met a moral fight he wasn't willing to pick.
They watched him clash with school authorities who weren't fond of his
decision to let his kids miss a half-day of school from time to time to
learn on their own. Their father believed so much in embracing arguments
that despite being a bishop in the local church, he had multiple books
by atheists in his library --- and encouraged his children to read them.

If we rarely see a spat, we learn to shy away from the threat of
conflict. Witnessing arguments --- and participating in them --- helps
us grow a thicker skin. We develop the will to fight uphill battles and
the skill to win those battles, and the resilience to lose a battle
today without losing our resolve tomorrow. For the Wright brothers,
argument was the family trade and a fierce one was something to be
savored. Conflict was something to embrace and resolve. ``I like
scrapping with Orv,'' Wilbur said.

The Wright brothers weren't alone. The Beatles fought over instruments
and lyrics and melodies. Elizabeth Cady Stanton and Susan B. Anthony
clashed over the right way to win the right to vote. Steve Jobs and
Steve Wozniak argued incessantly while designing the first Apple
computer. None of these people succeeded in spite of the drama --- they
flourished because of it. Brainstorming groups
\href{http://onlinelibrary.wiley.com/doi/10.1002/ejsp.210/full}{generate
16 percent more ideas} when the members are encouraged to criticize one
another. The most creative ideas in Chinese technology companies and the
best decisions in
\href{http://amj.aom.org/content/42/4/389.short}{American hospitals}
come from teams that have real disagreements early on. Breakthrough labs
in
\href{http://citeseerx.ist.psu.edu/viewdoc/summary?doi=10.1.1.110.5562}{microbiology}
aren't full of enthusiastic collaborators cheering one another on but of
skeptical scientists challenging one another's interpretations.

If no one ever argues, you're not likely to give up on old ways of doing
things, let alone try new ones. Disagreement is the antidote to
groupthink. We're at our most imaginative when we're out of sync.
There's no better time than childhood to learn how to dish it out ---
and to take it.

As Samuel Johnson was growing up, his parents
\href{https://books.google.com/books?id=bMlDB2Cor1EC\&pg=PT49\&dq=biography+\%22parents+argued+constantly\%22\&hl=en\&sa=X\&ved=0ahUKEwi07omCpvfUAhXCrD4KHWodDaEQ6AEIJjAA\#v=snippet\&q=argued\&f=false}{argued
constantly}. He described a family as ``a little kingdom, torn with
factions and exposed to revolutions.'' He went on to
\href{https://www.neh.gov/humanities/2009/septemberoctober/feature/what-samuel-johnson-really-did}{write}
one of the greatest dictionaries in history, one that had a lasting
impact on the English language and wasn't supplanted until the Oxford
English Dictionary appeared more than a century later. Who would be more
motivated and qualified to clean up a messy language than someone whose
household was filled with it?

Children need to learn the value of thoughtful disagreement. Sadly, many
parents teach kids that if they disagree with someone, it's polite to
hold their tongues. Rubbish. What if we taught kids that silence is bad
manners? It disrespects the other person's ability to have a civil
argument --- and it disrespects the value of your own viewpoint and your
own voice. It's a sign of respect to care enough about someone's opinion
that you're willing to challenge it.

We can also help by having disagreements openly in front of our kids.
Most parents hide their conflicts: They want to present a united front,
and they don't want kids to worry. But when parents disagree with each
other, kids learn to think for themselves. They discover that no
authority has a monopoly on truth. They become more tolerant of
ambiguity. Rather than conforming to others' opinions, they come to rely
on their own independent judgment.

It doesn't seem to matter how often parents argue; what counts is how
they
\href{https://www.guilford.com/books/Marital-Conflict-and-Children/Cummings-Davies/9781462503292}{handle
arguments} when they happen. Creativity tends to flourish, Mr. Albert,
the psychologist,
\href{http://journals.sagepub.com/doi/abs/10.1177/001698627802200214?journalCode=gcqb}{found},
in families that are ``tense but secure.'' In a
\href{http://onlinelibrary.wiley.com/doi/10.1111/j.1469-7610.2008.01945.x/full}{recent
study} of children ages 5 to 7, the ones whose parents argued
constructively felt more emotionally safe. Over the
\href{http://www.nbcnews.com/id/29959807/ns/health-childrens_health/t/how-dare-you-when-mom-dad-disagree/\#.WV-rfumQw2x}{next
three years}, those kids showed greater empathy and concern for others.
They were friendlier and more helpful toward their classmates in school.

Instead of trying to prevent arguments, we should be modeling courteous
conflict and teaching kids how to have healthy disagreements. We can
start with four rules:

• Frame it as a
\href{http://pubsonline.informs.org/doi/pdf/10.1287/orsc.2015.1025}{debate},
rather than a conflict.

• Argue as if you're right but
\href{https://hbr.org/2010/08/its-up-to-you-to-start-a-good}{listen as
if you're wrong}.

• Make the
\href{http://www.nytimes3xbfgragh.onion/2011/01/16/business/16corner.html}{most
respectful interpretation} of the other person's
\href{https://mobile.nytimes3xbfgragh.onion/2017/09/24/opinion/dying-art-of-disagreement.html}{perspective}.

• Acknowledge where you agree with your critics and what you've
\href{https://www.brainpickings.org/2014/03/28/daniel-dennett-rapoport-rules-criticism/}{learned
from them}.

Good arguments are wobbly: a team or family might rock back and forth
but it never tips over. If kids don't learn to wobble, they never learn
to walk; they end up standing still.

Advertisement

\protect\hyperlink{after-bottom}{Continue reading the main story}

\hypertarget{site-index}{%
\subsection{Site Index}\label{site-index}}

\hypertarget{site-information-navigation}{%
\subsection{Site Information
Navigation}\label{site-information-navigation}}

\begin{itemize}
\tightlist
\item
  \href{https://help.nytimes3xbfgragh.onion/hc/en-us/articles/115014792127-Copyright-notice}{©~2020~The
  New York Times Company}
\end{itemize}

\begin{itemize}
\tightlist
\item
  \href{https://www.nytco.com/}{NYTCo}
\item
  \href{https://help.nytimes3xbfgragh.onion/hc/en-us/articles/115015385887-Contact-Us}{Contact
  Us}
\item
  \href{https://www.nytco.com/careers/}{Work with us}
\item
  \href{https://nytmediakit.com/}{Advertise}
\item
  \href{http://www.tbrandstudio.com/}{T Brand Studio}
\item
  \href{https://www.nytimes3xbfgragh.onion/privacy/cookie-policy\#how-do-i-manage-trackers}{Your
  Ad Choices}
\item
  \href{https://www.nytimes3xbfgragh.onion/privacy}{Privacy}
\item
  \href{https://help.nytimes3xbfgragh.onion/hc/en-us/articles/115014893428-Terms-of-service}{Terms
  of Service}
\item
  \href{https://help.nytimes3xbfgragh.onion/hc/en-us/articles/115014893968-Terms-of-sale}{Terms
  of Sale}
\item
  \href{https://spiderbites.nytimes3xbfgragh.onion}{Site Map}
\item
  \href{https://help.nytimes3xbfgragh.onion/hc/en-us}{Help}
\item
  \href{https://www.nytimes3xbfgragh.onion/subscription?campaignId=37WXW}{Subscriptions}
\end{itemize}
