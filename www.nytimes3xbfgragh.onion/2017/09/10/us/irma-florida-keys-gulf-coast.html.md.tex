Sections

SEARCH

\protect\hyperlink{site-content}{Skip to
content}\protect\hyperlink{site-index}{Skip to site index}

\href{https://www.nytimes3xbfgragh.onion/section/us}{U.S.}

\href{https://myaccount.nytimes3xbfgragh.onion/auth/login?response_type=cookie\&client_id=vi}{}

\href{https://www.nytimes3xbfgragh.onion/section/todayspaper}{Today's
Paper}

\href{/section/us}{U.S.}\textbar{}Irma Roars In, and All of Florida
Shakes and Shudders

\url{https://nyti.ms/2xURVls}

\begin{itemize}
\item
\item
\item
\item
\item
\item
\end{itemize}

Advertisement

\protect\hyperlink{after-top}{Continue reading the main story}

Supported by

\protect\hyperlink{after-sponsor}{Continue reading the main story}

\hypertarget{irma-roars-in-and-all-of-florida-shakes-and-shudders}{%
\section{Irma Roars In, and All of Florida Shakes and
Shudders}\label{irma-roars-in-and-all-of-florida-shakes-and-shudders}}

\includegraphics{https://static01.graylady3jvrrxbe.onion/images/2017/09/11/us/11storm-05/11storm-05-articleLarge.jpg?quality=75\&auto=webp\&disable=upscale}

By \href{http://www.nytimes3xbfgragh.onion/by/frances-robles}{Frances
Robles},
\href{http://www.nytimes3xbfgragh.onion/by/lizette-alvarez}{Lizette
Alvarez} and
\href{http://www.nytimes3xbfgragh.onion/by/vivian-yee}{Vivian Yee}

\begin{itemize}
\item
  Sept. 10, 2017
\item
  \begin{itemize}
  \item
  \item
  \item
  \item
  \item
  \item
  \end{itemize}
\end{itemize}

MIAMI --- Ready or not, Florida found itself face to face with Hurricane
Irma's galloping winds and rains on Sunday, as evacuees and holdouts
alike marked uneasy time in homes and shelters from the Keys to the
Panhandle, tap-tapping their nearly dead cellphones for news they were
frantic to hear but helpless to change.

The hurricane rammed ashore at Cudjoe Key before whirling on the state's
southwest and west coast on the first day of its sodden chug north,
\href{https://www.nytimes3xbfgragh.onion/2017/09/10/us/crane-collapse-miami-irma.html?smid=tw-nytnational\&smtyp=cur\&_r=0}{buckling
two giant construction cranes} in Miami and rotating others like clock
hands, snacking on trees and power lines, and interrupting millions of
lives.

An apocalyptic forecast had already forced one of the largest
evacuations in American history. Now it was time to find out what the
storm would do --- and whether the heavily populated cities of Naples,
Fort Myers, St. Petersburg and Tampa were prepared.

``Everybody has a plan until they get punched in the face,'' Mayor Bob
Buckhorn of Tampa said at a Sunday news conference, paraphrasing the
boxer Mike Tyson. ``Well, we're about to get punched in the face.''

\includegraphics{https://static01.graylady3jvrrxbe.onion/images/2017/09/12/us/12storm-video-hp/12storm-video-hp-videoSixteenByNine3000-v2.jpg}

Having flattened a string of Caribbean islands and strafed Puerto Rico
and Cuba over the last week as a dangerous Category 4 and 5 storm, Irma
was downgraded on Sunday afternoon to Category 2,
\href{http://www.nhc.noaa.gov/text/refresh/MIATCPAT1+shtml/101757.shtml}{according
to the National Hurricane Center}. The center said that while the storm
was weakening, it was ``expected to remain a powerful hurricane,'' with
maximum sustained winds near 110 miles per hour, down from 130 m.p.h. On
Monday, it was set to spin over northern Florida, with Georgia next in
line.

The sea was Irma's ally in destruction. In Key Largo, it annexed
backyard pools. In Miami, it poured a salt river down Biscayne
Boulevard, the city's main artery. In Naples and Tampa Bay, it pulled
back from the shoreline, leaving waters so shallow that unwary dogs
could splash around what remained. But that was only a prelude to a
violent return: When the wind changed, scientists warned, the water
would hurl itself right back to where it was, and then some.

At least four deaths were reported in Florida after the storm's arrival
on Sunday, adding to a death toll of at least 27 from its Caribbean
rampage. More than three million people in Florida were
\href{https://www.fpl.com/storm/customer-outages.html}{without power},
officials said on Sunday night.

Officials along the Gulf Coast had believed they would be spared the
worst of the assault until the storm's trajectory took an unfavorable
westward bounce late in the week. After a Saturday spent hastily
converting fortified buildings into shelters, they were hurrying the
final preparations into place on Sunday.

Curfews were declared in Collier County, which includes Naples; Lee
County, which includes Fort Myers; and in Tampa, and officials said they
would not be lifted until the storm cleared. Shortly before 5 p.m.
Sunday, the Tampa police called officers off the streets as the city
confronted consistent wind gusts of more than 40 m.p.h. The westbound
lanes on two of the three bridges connecting Tampa with St. Petersburg
were closed.

Lest any humans decide to take the weather into their own hands, the
sheriff's office in Pasco County, north of Tampa Bay, was telling local
residents not to shoot weapons at the hurricane.

``You won't make it turn around,'' the sheriff's office
\href{https://twitter.com/PascoSheriff/status/906712903868469249}{tweeted},
``\& it will have very dangerous side effects.''

Midafternoon in Fort Myers, it was hard to tell which was worse, the
wind or the rain.

The wind whipped the tops of palm trees around like pompoms in the hands
of a cheerleader. At one Fort Myers hotel, the rain pelted the building
with such force that it came into rooms around window frames, stains
spreading ever wider on the carpet.

\includegraphics{https://static01.graylady3jvrrxbe.onion/images/2017/09/11/us/11storm-02/11storm-02-articleInline.jpg?quality=75\&auto=webp\&disable=upscale}

But the Keys, a collection of islands off Florida's southern tip, met
Irma first.

Images showed entire houses underwater. The flooding in Key Largo had
small boats bobbing in the streets next to furniture and refrigerators
like rubber toys in a bathtub. Shingles were kidnapped from roofs;
swimming pools dissolved into the ocean.

``Still whiteout,'' John Huston, a resident who had stayed, wrote in a
text message to The Associated Press around lunchtime on Sunday. ``Send
cold beer.''

Local authorities were still waiting out the storm before determining
the extent of the flooding and damage. But one of Irma's casualties was
indisputable: The roof of the Key Largo building that local emergency
operations officials were using after they fled their headquarters in
Marathon had blown off.

On Key West, by contrast, one resident who was able to speak to a
reporter by landline described streets pocked with shutters, windows and
branches, but no flooding or ravaged houses. The resident, an
81-year-old artist named Richard Peter Matson who has lived in an old
townhouse there since 1980, had decided to shelter in his home against
all advice.

Image

A FEMA urban search and rescue team from California organized pallets of
supplies at the Orange County Convention Center in Orlando,
Fla.Credit...Sam Hodgson for The New York Times

``If anything was going to happen,'' Mr. Matson said, ``I wanted to be
here to take care of it.''

Those who did evacuate should not come back until local officials had
had a chance to inspect the 42 bridges that connect the Keys to each
other and to the mainland, said Cammy Clark, a county spokeswoman. As a
precaution, officials were asking residents to boil water.

Irma was capricious. The residents of the Miami area, once projected to
bear the worst of it, seemed at some points on Sunday to be suffering
more from the fidgets than anything else.

As power vanished, their cellphones became their only tether to news,
family and friends. When their cellphone batteries died, they dashed out
to their cars to recharge.

Yamile Castella and her husband, Ramon, both Miami natives, spent Sunday
reading, listening to ``Hamilton'' and watching ``Wonder Woman'' until
the wind gusts intensified enough to throw half an avocado tree at their
house. All the while, Ms. Castella was juggling four chats on WhatsApp
--- a rowing group, a running group, and two family groups, everyone
trading stories about the highest gusts, who was eating what, who was
doing what.

Image

Hotel guests rode out the storm in the lobby of Fairfield Inn \& Suites
in Miami.Credit...Eric Thayer for The New York Times

``We feel like we're not alone,'' she said.

To the north, most could not yet afford to relax.

By Sunday afternoon, more than half of the 45 shelters in Hillsborough
County, which contains Tampa, had filled, including a shelter for people
with special medical needs that had sprung up on the floor of the Sun
Dome arena at the University of South Florida. There were nearly 800
people there, including patients, volunteers, nurses and doctors, and
they were out of cots and pillows. Mike Wagner, the shelter's manager,
had to tell a woman and her family that there was no room.

``We just had to tell her, you have to go back home and hunker down,''
Mr. Wagner said. ``It's a patient with five family members and a pet.
It's a sad state of affairs, but you have to draw some limits.''

The floor of the stadium, which is usually the home of the university's
basketball and volleyball teams, was now a patchwork of cots --- 435 of
them --- and medical devices. Patients were hooked into oxygen machines
and tucked under plaid or striped blankets. There was a special section
for hospice patients, and more cots lined the hallways.

Mr. Wagner's main worry was trying to ration precious time with the
electrical outlets. It was becoming nearly impossible to accommodate new
patients who needed electricity around the clock to power their medical
equipment.

Image

Trees bent in the wind as Hurricane Irma hit Naples, Fla., on
Sunday.Credit...David Goldman/Associated Press

``We're physically going to have to unplug someone, we're telling them,
you have to go back home,'' Mr. Wagner said. ``I don't even know how
that works for them. They'll have to find some place. But I can't unplug
you, if you need oxygen, just to plug someone else in.''

John Hawrsk, 67, was caring for his 96-year-old mother, whom he was
keeping slightly sedated so she would stay calm.

``She gets kind of panicky, there's a little confusion,'' Mr. Hawrsk
said. ``Try to keep her eyes closed, try to get her to sleep as much as
she can on her own.''

North of Irma's swirl, in Orlando, searchers, canine handlers, doctors
and communications experts had come from as far as Los Angeles to help.

Warn your families that Hurricane Irma could end communications home for
days, Chuck Ruddell, a member of California Task Force 1, told his
teammates. Accept that the team, which worked the aftermath of Hurricane
Harvey in Texas, might be sleeping at high schools and fairgrounds for
weeks more. And prepare to make snap decisions about who to save first.

Speaking in shorthand, the men and women checked their eight boats,
three tractor-trailers and other equipment. They scanned maps of Florida
communities. They watched the news.

Then they, too, had nothing more to do but wait.

Advertisement

\protect\hyperlink{after-bottom}{Continue reading the main story}

\hypertarget{site-index}{%
\subsection{Site Index}\label{site-index}}

\hypertarget{site-information-navigation}{%
\subsection{Site Information
Navigation}\label{site-information-navigation}}

\begin{itemize}
\tightlist
\item
  \href{https://help.nytimes3xbfgragh.onion/hc/en-us/articles/115014792127-Copyright-notice}{©~2020~The
  New York Times Company}
\end{itemize}

\begin{itemize}
\tightlist
\item
  \href{https://www.nytco.com/}{NYTCo}
\item
  \href{https://help.nytimes3xbfgragh.onion/hc/en-us/articles/115015385887-Contact-Us}{Contact
  Us}
\item
  \href{https://www.nytco.com/careers/}{Work with us}
\item
  \href{https://nytmediakit.com/}{Advertise}
\item
  \href{http://www.tbrandstudio.com/}{T Brand Studio}
\item
  \href{https://www.nytimes3xbfgragh.onion/privacy/cookie-policy\#how-do-i-manage-trackers}{Your
  Ad Choices}
\item
  \href{https://www.nytimes3xbfgragh.onion/privacy}{Privacy}
\item
  \href{https://help.nytimes3xbfgragh.onion/hc/en-us/articles/115014893428-Terms-of-service}{Terms
  of Service}
\item
  \href{https://help.nytimes3xbfgragh.onion/hc/en-us/articles/115014893968-Terms-of-sale}{Terms
  of Sale}
\item
  \href{https://spiderbites.nytimes3xbfgragh.onion}{Site Map}
\item
  \href{https://help.nytimes3xbfgragh.onion/hc/en-us}{Help}
\item
  \href{https://www.nytimes3xbfgragh.onion/subscription?campaignId=37WXW}{Subscriptions}
\end{itemize}
