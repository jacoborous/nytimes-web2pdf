Sections

SEARCH

\protect\hyperlink{site-content}{Skip to
content}\protect\hyperlink{site-index}{Skip to site index}

\href{https://www.nytimes3xbfgragh.onion/section/politics}{Politics}

\href{https://myaccount.nytimes3xbfgragh.onion/auth/login?response_type=cookie\&client_id=vi}{}

\href{https://www.nytimes3xbfgragh.onion/section/todayspaper}{Today's
Paper}

\href{/section/politics}{Politics}\textbar{}Trump Jr. Says He Wanted
Russian Dirt to Determine Clinton's `Fitness' for Office

\url{https://nyti.ms/2xR5hjp}

\begin{itemize}
\item
\item
\item
\item
\item
\item
\end{itemize}

Advertisement

\protect\hyperlink{after-top}{Continue reading the main story}

Supported by

\protect\hyperlink{after-sponsor}{Continue reading the main story}

\hypertarget{trump-jr-says-he-wanted-russian-dirt-to-determine-clintons-fitness-for-office}{%
\section{Trump Jr. Says He Wanted Russian Dirt to Determine Clinton's
`Fitness' for
Office}\label{trump-jr-says-he-wanted-russian-dirt-to-determine-clintons-fitness-for-office}}

\includegraphics{https://static01.graylady3jvrrxbe.onion/images/2017/09/08/us/08dc-tower1/merlin-to-scoop-126964676-207557-articleLarge.jpg?quality=75\&auto=webp\&disable=upscale}

By \href{https://www.nytimes3xbfgragh.onion/by/nicholas-fandos}{Nicholas
Fandos} and
\href{http://www.nytimes3xbfgragh.onion/by/maggie-haberman}{Maggie
Haberman}

\begin{itemize}
\item
  Sept. 7, 2017
\item
  \begin{itemize}
  \item
  \item
  \item
  \item
  \item
  \item
  \end{itemize}
\end{itemize}

\href{https://cn.nytimes3xbfgragh.onion/usa/20170908/trump-russia-investigation/}{阅读简体中文版}\href{https://cn.nytimes3xbfgragh.onion/usa/20170908/trump-russia-investigation/zh-hant/}{閱讀繁體中文版}

WASHINGTON --- Donald Trump Jr. told Senate investigators on Thursday
that he set up
\href{https://www.nytimes3xbfgragh.onion/2017/07/09/us/politics/trump-russia-kushner-manafort.html}{a
June 2016 meeting with a Russian lawyer} because he was intrigued that
she might have damaging information about Hillary Clinton, saying it was
important to learn about Mrs. Clinton's ``fitness'' to be president.

But nothing came of the Trump Tower meeting, he said, and he was adamant
that he never colluded with the Russian government's campaign to disrupt
last year's presidential election.

During five hours of questioning, investigators for the Senate Judiciary
Committee pressed Mr. Trump on numerous topics related to the meeting
with the Russian lawyer, including how the president's aides this summer
drafted a statement aboard Air Force One in response to queries from The
New York Times about the meeting.

Mr. Trump said he did not speak to his father about the draft statement
because he did not want to involve him in something he ``knew nothing
about,'' according to one person briefed about parts of his testimony.
Lawmakers have wanted to know what, if anything, President Trump knew
about the June 2016 meeting and whether he was involved in preparing the
draft statement to The Times.

In his prepared remarks on Thursday to Congress, the younger Mr. Trump
said he was initially conflicted when he heard that the Russian lawyer,
Natalia Veselnitskaya, might have damaging information about Mrs.
Clinton. Despite his interest, he said, he always intended to consult
with his lawyers about the propriety of using any information that Ms.
Veselnitskaya, who has links to the Kremlin, gave him at the meeting.

A copy of Mr. Trump's statement was obtained by The New York Times.

The acknowledgment by the president's eldest son that he intended to
seek legal counsel after the meeting suggests that he knew, or at least
suspected, that accepting potentially damaging information about a rival
campaign from a foreign country raised thorny legal issues.

``To the extent they had information concerning the fitness, character
or qualifications of a presidential candidate, I believed that I should
at least hear them out,'' he said. ``Depending on what, if any,
information they had, I could then consult with counsel to make an
informed decision as to whether to give it further consideration.''

Mr. Trump's lengthy interview took place in the basement of the Capitol,
with Mr. Trump successfully evading reporters as he left and returned
for bathroom breaks. The interview was conducted by committee staff,
with Democratic and Republican teams of investigators taking turns
questioning Mr. Trump in one-hour blocks. A handful of senators also
attended portions of the meeting.

The June 2016 meeting was arranged after the younger Mr. Trump received
an email from a family associate saying that potentially damaging
information was being provided as part of the Russian government's
support for his father. But in his statement on Thursday, he described
his decision to agree to the meeting as the byproduct of the chaotic,
seat-of-the-pants campaign assembled by his father, rather than any
attempt to collude with Russia.

Mr. Trump has given differing accounts of his contacts last year with
Russians. He told The Times in March that he never met with Russians on
behalf of the campaign, a statement his lawyer has since said was meant
to refer to Russian government officials.
\href{https://www.nytimes3xbfgragh.onion/2017/07/08/us/politics/trump-russia-kushner-manafort.html}{In
July, he described} the Trump Tower meeting as primarily focused on the
issue of Russian adoptions, before eventually acknowledging that he took
the meeting because he was told Ms. Veselnitskaya had damaging
information about Mrs. Clinton.

But intentionally misspeaking to Congress is a crime, giving his
statement on Thursday added weight. If there were any doubt about the
stakes, the office of Senator Chris Coons, Democrat of Delaware and a
member of the panel, made them clear in an email to reporters on
Thursday afternoon that included the text of the so-called False
Statements statute.

Mr. Trump told investigators that working for his father's campaign
consumed his life. ``I had never worked on a campaign before, and it was
an exhausting, all-encompassing, life-changing experience. Every single
day I fielded dozens, if not hundreds, of emails and phone calls.''

He is the second person connected to the Trump campaign to tell
congressional investigators that the campaign was, essentially, too
inexperienced and too unfamiliar with politics to pull off a master
strategy --- let alone coordinate with the Russian government. Mr.
Trump's brother-in-law, Jared Kushner, painted a similar picture during
an interview with the Senate Intelligence Committee.

In his statement, Mr. Trump said he had some reservations about the June
2016 proposal from the meeting's facilitator, Rob Goldstone, whom he
described as a ``colorful'' music promoter he had come to know through
the son of a Russian oligarch. Mr. Goldstone asked Mr. Trump to take a
meeting that would include potentially damaging information about Mrs.
Clinton.

``Since I had no additional information to validate what Rob was saying,
I did not quite know what to make of his email,'' he said. ``I had no
way to gauge the reliability, credibility or accuracy of any of the
things he was saying.''

``As it later turned out, my skepticism was justified,'' Mr. Trump
added. ``The meeting provided no meaningful information and turned out
not to be about what had been represented.''

In an email response to Mr. Goldstone, Mr. Trump wrote that if the
promised information about Mrs. Clinton was as advertised, ``I love
it.''

``As much as some have made of my using the phrase `I love it,' it was
simply a colloquial way of saying that I appreciated Rob's gesture,'' he
said in his statement on Thursday.

When asked why, shortly after the Trump Tower meeting was set up, his
father promised to deliver a ``major speech'' about Mrs. Clinton's
``corrupt dealings,'' Mr. Trump said that that was merely the way his
father speaks, according to a person familiar with the interview.

The Senate Judiciary Committee is one of three congressional panels
investigating aspects of President Trump's links to Russia and related
matters. The committee, which has oversight of the Justice Department,
is particularly interested in the circumstances surrounding President
Trump's abrupt
\href{https://www.nytimes3xbfgragh.onion/2017/05/09/us/politics/james-comey-fired-fbi.html?_r=0}{firing
in May of James B. Comey} as F.B.I. director.

Democrats have repeatedly said that they still expect Donald Trump Jr.
to appear at a public committee hearing and were careful to cast the
Thursday sit-down as a staff-driven interview, where senators breezed in
and out merely to observe. It was unclear if Mr. Trump had agreed to
such testimony, and the committee's chairman, Charles E. Grassley, did
not say one way or another whether there would be a hearing.

Spokesmen for Mr. Grassley and Senator Dianne Feinstein of California,
the committee's top Democrat, declined to comment on the interview after
it concluded. Committee members who did not attend are expected to be
briefed on its contents in the coming days.

Senator Richard Blumenthal, Democrat of Connecticut, said the mood
behind closed doors was ``cordial'' and that investigators were asking
primarily factual questions.

He said that what he had heard from Mr. Trump made him only more certain
that the committee needed to hear from other attendees of the Trump
Tower meeting, including Mr. Kushner and Paul J. Manafort, who was the
Trump campaign chairman at the time. Mr. Blumenthal said that the
committee needs to look further into how Mr. Trump's initial statements
to the news media about that meeting were put together.

``We covered a good deal of ground,'' Mr. Blumenthal said. ``There is
still a lot of questioning to be covered.''

Advertisement

\protect\hyperlink{after-bottom}{Continue reading the main story}

\hypertarget{site-index}{%
\subsection{Site Index}\label{site-index}}

\hypertarget{site-information-navigation}{%
\subsection{Site Information
Navigation}\label{site-information-navigation}}

\begin{itemize}
\tightlist
\item
  \href{https://help.nytimes3xbfgragh.onion/hc/en-us/articles/115014792127-Copyright-notice}{©~2020~The
  New York Times Company}
\end{itemize}

\begin{itemize}
\tightlist
\item
  \href{https://www.nytco.com/}{NYTCo}
\item
  \href{https://help.nytimes3xbfgragh.onion/hc/en-us/articles/115015385887-Contact-Us}{Contact
  Us}
\item
  \href{https://www.nytco.com/careers/}{Work with us}
\item
  \href{https://nytmediakit.com/}{Advertise}
\item
  \href{http://www.tbrandstudio.com/}{T Brand Studio}
\item
  \href{https://www.nytimes3xbfgragh.onion/privacy/cookie-policy\#how-do-i-manage-trackers}{Your
  Ad Choices}
\item
  \href{https://www.nytimes3xbfgragh.onion/privacy}{Privacy}
\item
  \href{https://help.nytimes3xbfgragh.onion/hc/en-us/articles/115014893428-Terms-of-service}{Terms
  of Service}
\item
  \href{https://help.nytimes3xbfgragh.onion/hc/en-us/articles/115014893968-Terms-of-sale}{Terms
  of Sale}
\item
  \href{https://spiderbites.nytimes3xbfgragh.onion}{Site Map}
\item
  \href{https://help.nytimes3xbfgragh.onion/hc/en-us}{Help}
\item
  \href{https://www.nytimes3xbfgragh.onion/subscription?campaignId=37WXW}{Subscriptions}
\end{itemize}
