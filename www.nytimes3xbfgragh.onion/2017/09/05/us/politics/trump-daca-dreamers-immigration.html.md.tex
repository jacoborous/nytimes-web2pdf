Sections

SEARCH

\protect\hyperlink{site-content}{Skip to
content}\protect\hyperlink{site-index}{Skip to site index}

\href{https://www.nytimes3xbfgragh.onion/section/politics}{Politics}

\href{https://myaccount.nytimes3xbfgragh.onion/auth/login?response_type=cookie\&client_id=vi}{}

\href{https://www.nytimes3xbfgragh.onion/section/todayspaper}{Today's
Paper}

\href{/section/politics}{Politics}\textbar{}Trump Moves to End DACA and
Calls on Congress to Act

\url{https://nyti.ms/2x7xOo2}

\begin{itemize}
\item
\item
\item
\item
\item
\item
\end{itemize}

Advertisement

\protect\hyperlink{after-top}{Continue reading the main story}

Supported by

\protect\hyperlink{after-sponsor}{Continue reading the main story}

\hypertarget{trump-moves-to-end-daca-and-calls-on-congress-to-act}{%
\section{Trump Moves to End DACA and Calls on Congress to
Act}\label{trump-moves-to-end-daca-and-calls-on-congress-to-act}}

\includegraphics{https://static01.graylady3jvrrxbe.onion/images/2017/09/05/us/politics/daca_still/daca_still-videoSixteenByNineJumbo1600.jpg}

By \href{https://www.nytimes3xbfgragh.onion/by/michael-d-shear}{Michael
D. Shear} and
\href{https://www.nytimes3xbfgragh.onion/by/julie-hirschfeld-davis}{Julie
Hirschfeld Davis}

\begin{itemize}
\item
  Sept. 5, 2017
\item
  \begin{itemize}
  \item
  \item
  \item
  \item
  \item
  \item
  \end{itemize}
\end{itemize}

\href{https://www.nytimes3xbfgragh.onion/es/2017/09/05/donald-trump-revoca-daca-dreamers/}{Leer
en español}

WASHINGTON --- \emph{{[}Read more on the Supreme Court's decision on
President}
\href{https://www.nytimes3xbfgragh.onion/2020/06/18/us/trump-daca-supreme-court.html}{\emph{Trump
and DACA}}\emph{.{]}}

President Trump on Tuesday ordered an end to the Obama-era program that
shields young undocumented immigrants from deportation, calling it an
``amnesty-first approach'' and urging Congress to pass a replacement
before he begins phasing out its protections in six months.

As early as March, officials said, some of the 800,000 young adults
brought to the United States illegally as children who qualify for the
program, Deferred Action for Childhood Arrivals, will become eligible
for deportation. The five-year-old policy allows them to remain without
fear of immediate removal from the country and gives them the right to
work legally.

Mr. Trump and Attorney General Jeff Sessions, who announced the change
at the Justice Department, both used the aggrieved language of
anti-immigrant activists, arguing that those in the country illegally
are lawbreakers who hurt native-born Americans by usurping their jobs
and pushing down wages.

Mr. Trump said in a statement that he was driven by a concern for ``the
millions of Americans victimized by this unfair system.'' Mr. Sessions
said the program had ``denied jobs to hundreds of thousands of Americans
by allowing those same illegal aliens to take those jobs.''

Protests broke out in front of the White House and the Justice
Department and in cities across the country soon after Mr. Sessions's
announcement. Democrats and some Republicans,
\href{https://www.nytimes3xbfgragh.onion/2017/09/05/business/chief-executives-see-a-sad-day-after-trumps-daca-decision.html?hp\&action=click\&pgtype=Homepage\&clickSource=story-heading\&module=b-lede-package-region\&region=top-news\&WT.nav=top-news}{business
executives}, college presidents and immigration activists condemned the
move as a coldhearted and shortsighted effort that was unfair to the
young immigrants and could harm the economy.

``This is a sad day for our country,'' Mark Zuckerberg, the Facebook
founder,
\href{https://www.facebookcorewwwi.onion/zuck/posts/10104016069261801}{wrote
on his personal page}. ``It is particularly cruel to offer young people
the American dream, encourage them to come out of the shadows and trust
our government, and then punish them for it.''

\includegraphics{https://static01.graylady3jvrrxbe.onion/images/2017/09/06/us/06daca-vid/06daca-vid-videoSixteenByNine3000.jpg}

Former President Barack Obama, who had warned that any threat to the
program would prompt him to speak out, called his successor's decision
``wrong,'' ``self-defeating'' and ``cruel.''

``Whatever concerns or complaints Americans may have about immigration
in general, we shouldn't threaten the future of this group of young
people who are here through no fault of their own, who pose no threat,
who are not taking away anything from the rest of us,'' Mr. Obama
\href{https://www.facebookcorewwwi.onion/barackobama/posts/10155227588436749}{wrote
on Facebook}.

Both he and Mr. Trump said the onus was now on lawmakers to protect the
young immigrants as part of a broader overhaul of the immigration system
that would also toughen enforcement.

But despite broad and longstanding bipartisan support for measures to
legalize unauthorized immigrants brought to the United States as
children, the odds of a sweeping immigration deal in a deeply divided
Congress
\href{https://www.nytimes3xbfgragh.onion/2017/09/05/us/politics/dream-act-daca-trump-congress-dreamers.html}{appeared
long}. Legislation to protect the ``dreamers'' has also repeatedly died
in Congress.

Just hours after the angry reaction to Mr. Trump's decision, the
president appeared to have second thoughts. In a
\href{https://twitter.com/realDonaldTrump/status/905228667336499200}{late-evening
tweet}, Mr. Trump specifically called on Congress to ``legalize DACA,''
something his administration's officials had declined to do earlier in
the day.

Mr. Trump also warned lawmakers that if they do not legislate a program
similar to the one Mr. Obama created through executive authority, he
will ``revisit this issue!'' --- a statement sure to inject more
uncertainty into the ultimate fate of the young, undocumented immigrants
who have been benefiting from the program since 2012.

\includegraphics{https://static01.graylady3jvrrxbe.onion/images/2017/09/06/us/06dc-daca-2/06dc-daca-2-articleLarge.jpg?quality=75\&auto=webp\&disable=upscale}

Conservatives praised Mr. Trump's move, though some expressed
frustration that he had taken so long to rescind the program and that
the gradual phaseout could mean that some immigrants retained protection
from deportation until October 2019.

The White House portrayed the decision as a matter of legal necessity,
given that nine Republican state attorneys general had threatened to sue
to halt the program immediately if Mr. Trump did not act.

Months of internal White House debate preceded the move, as did the
president's public display of his own conflicted feelings. He once
referred to DACA recipients as ``incredible kids.''

The president's wavering was reflected in a day of conflicting messages
from him and his team. Hours after his statement was released, Mr. Trump
told reporters that he had ``great love'' for the beneficiaries of the
program he had just ended.

``I have a love for these people, and hopefully now Congress will be
able to help them and do it properly,'' he said. But he notably did not
endorse bipartisan legislation to codify the program's protections,
leaving it unclear whether he would back such a solution.

Mr. Trump's aides were negotiating late into Monday evening with one
another about precisely how the plan to wind down the program would be
executed. Until Tuesday morning, some aides believed the president had
settled on a plan that would be more generous, giving more of the
program's recipients the option to renew their protections.

But even taking into account Mr. Trump's contradictory language, the
rollout of his decision was smoother than his early moves to crack down
on immigration, particularly the
\href{https://www.nytimes3xbfgragh.onion/2017/01/29/us/politics/donald-trump-rush-immigration-order-chaos.html}{botched
execution in January of his ban on travelers} from seven predominantly
Muslim countries.

In addition to the public statement from Mr. Sessions and a White House
question-and-answer session, the president was ready on Tuesday with the
lengthy written statement, and officials at the Justice and Homeland
Security Departments provided detailed briefings and distributed
information to reporters in advance.

Mr. Trump sought to portray his move as a compassionate effort to head
off the expected legal challenge that White House officials said would
have forced an immediate and highly disruptive end to the program. But
he also denounced the policy, saying it helped spark a ``massive surge''
of immigrants from Central America, some of whom went on to become
members of violent gangs like MS-13. Some immigration critics contend
that programs like DACA, started under Mr. Obama, encouraged Central
Americans to enter the United States, hoping to stay permanently. Tens
of thousands of migrants surged across America's southern border in the
summer of 2014, many of them children fleeing dangerous gangs.

Sarah Huckabee Sanders, the White House press secretary, indicated that
Mr. Trump would support legislation to ``fix'' the DACA program, as long
as Congress passed it as part of a broader immigration overhaul to
strengthen the border, protect American jobs and enhance enforcement.

``The president wants to see responsible immigration reform, and he
wants that to be part of it,'' Ms. Sanders said, referring to a
permanent solution for the young immigrants. ``Something needs to be
done. It's Congress's job to do that. And we want to be part of that
process.''

Later on Tuesday, Marc Short, Mr. Trump's top legislative official, told
reporters on Capitol Hill that the White House would release principles
for such a plan in the coming days, input that at least one key member
of Congress indicated would be crucial.

Image

Three DACA recipients --- Sofia Ruales, left; her sister Erica Ruales;
and their cousin Marlon Ruales --- listened on Tuesday near Trump Tower
in Manhattan to Attorney General Jeff Sessions's
announcement.Credit...Todd Heisler/The New York Times

``It is important that the White House clearly outline what kind of
legislation the president is willing to sign,'' Senator Marco Rubio,
Republican of Florida, said in a statement. ``We have no time to waste
on ideas that do not have the votes to pass or that the president won't
sign.''

The announcement was an effort by Mr. Trump to honor the law-and-order
message of his campaign, which included a repeated pledge to end Mr.
Obama's immigration policy, while seeking to avoid the emotionally
charged and politically perilous consequences of targeting a sympathetic
group of immigrants.

Mr. Trump's decision came less than two weeks after he
\href{https://www.nytimes3xbfgragh.onion/2017/08/25/us/politics/joe-arpaio-trump-pardon-sheriff-arizona.html}{pardoned
Joe Arpaio}, the former Arizona sheriff who drew intense criticism for
his aggressive pursuit of unauthorized immigrants, which earned him a
criminal contempt conviction.

The blame-averse president told a confidante over the past few days that
he realized that he had gotten himself into a politically untenable
position. As late as one hour before the decision was to be announced,
administration officials privately expressed concern that Mr. Trump
might not fully grasp the details of the steps he was about to take, and
when he discovered their full impact, would change his mind, according
to a person familiar with their thinking who was not authorized to
comment on it and spoke on condition of anonymity.

\includegraphics{https://static01.graylady3jvrrxbe.onion/images/2017/09/06/us/06dc-daca-sessions/06dc-daca-sessions-videoSixteenByNine3000-v2.jpg}

But ultimately, the president followed through on his campaign pledge at
the urging of Mr. Sessions and other hard-line members inside his White
House, including Stephen Miller, his top domestic policy adviser.

The announcement started the clock on revoking legal status from those
protected under the program.

Officials said DACA recipients whose legal status expires on or before
March 5 would be able to renew their two-year period of legal status as
long as they apply by Oct. 5. But the announcement means that if
Congress fails to act, immigrants who were brought to the United States
illegally as children could face deportation as early as March 6 to
countries where many left at such young ages that they have no memory of
them.

\href{https://www.nytimes3xbfgragh.onion/interactive/2017/09/05/us/politics/document-Jeff-Sessions-DACA.html}{}

\includegraphics{https://static01.graylady3jvrrxbe.onion/images/2017/09/05/us/image-Jeff-Sessions-DACA/image-Jeff-Sessions-DACA-thumbLarge.gif}

\hypertarget{read-jeff-sessionss-letter-advising-an-end-to-daca}{%
\subsection{Read Jeff Sessions's Letter Advising an End to
DACA}\label{read-jeff-sessionss-letter-advising-an-end-to-daca}}

Attorney General Jeff Sessions on Tuesday sent a letter to the
Department of Homeland Security recommending an end to Deferred Action
for Childhood Arrivals, or DACA, the Obama-era executive action that
shields young undocumented immigrants from deportation.

Immigration officials said they did not intend to actively target the
young immigrants as priorities for deportation, though without the
program's protection, they would be considered subject to removal from
the United States and would no longer be able to work legally.

Officials said some of the young immigrants could be prevented from
returning to the United States if they traveled abroad.

Immigration advocates took little comfort from the administration's
assurances, describing the president's decision as deeply disturbing and
vowing to shift their demands for protections to Capitol Hill.

Marielena Hincapié, the executive director of the National Immigration
Law Center, called Mr. Trump's decision ``nothing short of hypocrisy,
cruelty and cowardice.'' Maria Praeli, a recipient of protection under
the program, criticized Mr. Sessions and Mr. Trump for talking ``about
us as if we don't matter and as if this isn't our home.''

The Mexican foreign ministry issued a statement saying the ``Mexican
government deeply regrets'' Mr. Trump's decision.

As recently as July, Mr. Trump expressed skepticism about the prospect
of a broad legislative deal.

``What I'd like to do is a comprehensive immigration plan,'' he told
reporters. ``But our country and political forces are not ready yet.''

As for DACA, he said: ``There are two sides of a story. It's always
tough.''

Advertisement

\protect\hyperlink{after-bottom}{Continue reading the main story}

\hypertarget{site-index}{%
\subsection{Site Index}\label{site-index}}

\hypertarget{site-information-navigation}{%
\subsection{Site Information
Navigation}\label{site-information-navigation}}

\begin{itemize}
\tightlist
\item
  \href{https://help.nytimes3xbfgragh.onion/hc/en-us/articles/115014792127-Copyright-notice}{©~2020~The
  New York Times Company}
\end{itemize}

\begin{itemize}
\tightlist
\item
  \href{https://www.nytco.com/}{NYTCo}
\item
  \href{https://help.nytimes3xbfgragh.onion/hc/en-us/articles/115015385887-Contact-Us}{Contact
  Us}
\item
  \href{https://www.nytco.com/careers/}{Work with us}
\item
  \href{https://nytmediakit.com/}{Advertise}
\item
  \href{http://www.tbrandstudio.com/}{T Brand Studio}
\item
  \href{https://www.nytimes3xbfgragh.onion/privacy/cookie-policy\#how-do-i-manage-trackers}{Your
  Ad Choices}
\item
  \href{https://www.nytimes3xbfgragh.onion/privacy}{Privacy}
\item
  \href{https://help.nytimes3xbfgragh.onion/hc/en-us/articles/115014893428-Terms-of-service}{Terms
  of Service}
\item
  \href{https://help.nytimes3xbfgragh.onion/hc/en-us/articles/115014893968-Terms-of-sale}{Terms
  of Sale}
\item
  \href{https://spiderbites.nytimes3xbfgragh.onion}{Site Map}
\item
  \href{https://help.nytimes3xbfgragh.onion/hc/en-us}{Help}
\item
  \href{https://www.nytimes3xbfgragh.onion/subscription?campaignId=37WXW}{Subscriptions}
\end{itemize}
