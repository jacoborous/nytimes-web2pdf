Sections

SEARCH

\protect\hyperlink{site-content}{Skip to
content}\protect\hyperlink{site-index}{Skip to site index}

\href{https://www.nytimes3xbfgragh.onion/section/world/asia}{Asia
Pacific}

\href{https://myaccount.nytimes3xbfgragh.onion/auth/login?response_type=cookie\&client_id=vi}{}

\href{https://www.nytimes3xbfgragh.onion/section/todayspaper}{Today's
Paper}

\href{/section/world/asia}{Asia Pacific}\textbar{}Mongolian Warriors and
Communist Soldiers: A Frontier Town in China

\url{https://nyti.ms/2tBW8sf}

\begin{itemize}
\item
\item
\item
\item
\item
\end{itemize}

Advertisement

\protect\hyperlink{after-top}{Continue reading the main story}

Supported by

\protect\hyperlink{after-sponsor}{Continue reading the main story}

Wenquan Journal

\hypertarget{mongolian-warriors-and-communist-soldiers-a-frontier-town-in-china}{%
\section{Mongolian Warriors and Communist Soldiers: A Frontier Town in
China}\label{mongolian-warriors-and-communist-soldiers-a-frontier-town-in-china}}

\includegraphics{https://static01.graylady3jvrrxbe.onion/images/2017/06/17/world/19mongols-1/19mongols-1-articleLarge.jpg?quality=75\&auto=webp\&disable=upscale}

By \href{http://www.nytimes3xbfgragh.onion/by/edward-wong}{Edward Wong}

\begin{itemize}
\item
  June 18, 2017
\item
  \begin{itemize}
  \item
  \item
  \item
  \item
  \item
  \end{itemize}
\end{itemize}

\href{https://cn.nytimes3xbfgragh.onion/china/20170619/mongolian-warriors-and-communist-soldiers-a-frontier-town-in-china/}{阅读简体中文版}

WENQUAN, China --- Wenquan means ``hot springs,'' and the town, nestled
in a fertile swath of Central Asia, certainly has its share. Alpine
forests cloak the surrounding mountains. To the south is a wide lake
where azure waters lap at stony shores. Horses and sheep roam the
pastures.

But an abundance of natural beauty is not what brought the Mongolian
warriors to this land of broad valleys in China, next to present-day
Kazakhstan.

The Chahar made the long journey in horse and camel caravans in the 18th
century, under orders from the Qianlong Emperor and his court in
Beijing. Qianlong, one of the greatest of the Manchu rulers of the Qing
dynasty, cobbled together a vast multiethnic Chinese empire through
conquests and alliances. Mongolian khanates, armies and tribes fell
under his rule, often after vicious battles.

Qianlong dispatched a Chahar army from near the Mongolian steppe to
newly conquered territories along the empire's northwestern rim, where
the Chahar were to form a border garrison.

\includegraphics{https://static01.graylady3jvrrxbe.onion/images/2017/06/17/world/19mongols-2/19mongols-2-articleLarge.jpg?quality=75\&auto=webp\&disable=upscale}

``My father was here from a young age,'' said Xiu Yun, 48, a manager at
a modest resort hotel built around the hot springs in the town center.
``His parents were from here too. My family members are descendants of
the Mongolians who came under the Qing. We Mongols are very proud of
this history.''

She added: ``I can speak Mongolian and read and write it. Most
Mongolians here can do the same.''

In recent years, Wenquan officials have begun highlighting the town's
Mongolian heritage. Street signs have both Chinese and Mongolian script.
A new concrete mural on the main road next to the hot springs hotel
depicts the ancient caravans that traveled west. A museum at the other
end of the main street has a large map showing the three waves of Chahar
migration. On one wall, a poem called ``The Door of the Rainbow'' pays
tribute to that history.

For centuries, the Chahar claimed to have a seal from Genghis Khan,
which conferred legitimacy. So their alliance with the Qing --- they
were incorporated into the banner system after a failed rebellion in
1675 --- was important for the Manchu rulers. It bolstered Manchu
standing in the eyes of other Mongolian tumen, or tribes.

``From Genghis Khan to the emperors of the Yuan dynasty to the khans of
the Chahar tumen, there was this single lineage,'' said Oyunbilig
Borjigidai, a professor of Manchu and Mongolian history at Renmin
University of China in Beijing. ``Its status and influence were much
greater than those of the other tumen.''

Image

Uighur, Kazakh and Han Chinese on a street in Wenquan.Credit...Gilles
Sabrié for The New York Times

Another Mongolian group, the Dzungars, ruled
\href{https://www.nytimes3xbfgragh.onion/2014/05/31/world/asia/chinas-leader-lays-out-plan-to-pacify-restive-region.html?mcubz=2}{northwest
Xinjiang}, on the Central Asian frontier, before Qianlong decimated them
in a famous series of campaigns. Qianlong then wanted to build garrisons
on the borderlands, and so armies from the Chahar, Xibe and Solon ethnic
groups were dispatched.

``They truly played an important role in helping the Qing court
establish a foothold in the northwestern border area and develop it,''
Mr. Oyunbilig said of the Chahar. ``They have a reason to be proud.''

Wenquan is part of the Bortala Mongol Autonomous Prefecture, the base of
the Chahar in Xinjiang. (Their ancestral home is in present-day Inner
Mongolia, where the majority of Chahar in China live.) The prefecture is
one of several scattered enclaves that arose from Qing-era garrisons.
Farther south, in
\href{https://www.farwestchina.com/2016/07/yili-xinjiang-top-5-places-visit.html}{the
fertile Ili Valley}, is another --- that of the
\href{https://www.nytimes3xbfgragh.onion/2016/01/12/world/asia/china-xinjiang-manchu-xibe-language.html?mcubz=2}{Xibe},
who speak a language similar to Manchu and are one of China's 56
official ethnic groups.

Wenquan is a quiet town beyond a pass north of Sayram Lake, the largest
alpine lake in Xinjiang and where Kazakh herders graze sheep and offer
horseback rides to tourists in summer. Wenquan has one short commercial
strip --- at one end is the hot springs hotel and at the other is the
museum.

Image

A statue commemorating the westward migration that sent Chahar Mongols
to Wenquan three centuries ago.Credit...Gilles Sabrié for The New York
Times

Half the town is taken up by a bingtuan, a term for an agricultural
production center that originated in the Mao era as a
\href{http://www.nytimes3xbfgragh.onion/2009/08/07/world/asia/07xinjiang.html?mcubz=2}{garrison
project} of the People's Liberation Army. It is a modern-day variation
on the mission of the Chahar.

It is hard to tell where the town ends and the bingtuan begins. The two
merge seamlessly. The bingtuan, the 88th Regiment of the Fifth Division,
has streets, homes, schools, shops and office buildings.

While on a reporting trip to Xinjiang, I drove to Wenquan to spend a
night here. I was curious about the town because my father, as
\href{https://www.nytimes3xbfgragh.onion/2016/12/29/world/asia/great-wall-china.html?mcubz=2\&_r=0}{a
member of the Chinese Army}, had been posted here from 1955 to 1957 to
work in an earlier bingtuan, as an aide to the party chief.

He bunked with two others in a room with a coal stove. The town had a
dirt road lined with homes. There were no shops then. The hot springs
bathhouse stood alone, not as part of a hotel, and my father enjoyed
dips there. The mud-walled home where my father lived was uphill at one
end of the road, on the grounds of the headquarters of the Fifth
Regiment.

Image

Guomu Jiafu, 59, a descendant of the Mongols sent by the Qing dynasty,
in prayer by an outdoor altar.Credit...Gilles Sabrié for The New York
Times

``Every day, I would look at the mountains,'' he told me. ``Someone
said, `You cross the mountains and you are in Russia.' Kazakhstan was on
the other side.''

Wenquan is a palimpsest of military conquest. In a sense, my father and
the handful of other ethnic Han soldiers posted here in the first decade
of Communist Party rule were spiritual descendants of the Chahar.

Though there were Mongolians in the area back then, most people in town
and in the surrounding hills were Kazakhs, and they remain the largest
ethnic group in Wenquan. On occasion, my father would ride a horse for a
day to visit nomads in the high pastures and spend a night or two in
their felt yurts. He learned to speak some Kazakh.

To visit Mongolians in the nearby prefecture seat of Bortala, my father
and his comrades rode in a horse-drawn wagon. In Wenquan, there was no
emphasis on Mongolian language or culture. After my trip, my father was
surprised to hear of the displays of Mongolian culture I had seen.

Image

A tourist from Shanghai at Sayram Lake, about 3,000 miles away from her
home.Credit...Gilles Sabrie for The New York Times

Though the party's
\href{https://www.nytimes3xbfgragh.onion/2015/11/29/world/asia/china-tibet-language-education.html?mcubz=2}{ethnic
policies} are contentious, there has been a revived interest in some
parts of China in the languages and traditions of smaller ethnic groups.
Sometimes this has strong support by the national government, as in the
case of the Manchus. In other instances, ordinary people or community
officials drive the revival.

``There is this new sub-ethnic consciousness,'' said
\href{http://history.yale.edu/people/peter-c-perdue}{Peter C. Perdue}, a
historian at Yale University who has studied the Qing conquest of
Xinjiang. ``The Chahar want to say they are a separate ethnic group, not
mixed in with the other Mongolians there.''

``You hear about the Uighurs all the time there,'' he added, referring
to a Turkic-speaking group in Xinjiang. ``The other minority people are
also trying to regenerate a sense of their identity, in a somewhat
different sense than the way the People's Republic of China assigns
ethnic labels to people.''

The evening I stayed in town, Mongolians, Kazakh, Han and Uighurs all
showed up at the bathhouse in the hot springs hotel. In recent years,
violence involving Uighurs and Han
\href{http://www.nytimes3xbfgragh.onion/2009/07/12/weekinreview/12wong.html?mcubz=2}{has
erupted} in oasis towns in southern Xinjiang, the Uighur heartland, and
in the regional capital, Urumqi. There did not appear to be much tension
in Wenquan.

Ms. Xiu, the hotel's manager, has an uncle who writes Chahar poems for
local newspapers and plays the topshur, a two-stringed instrument
popular among western Mongolian tribes. The uncle, Madega, 66, runs a
local company that makes the instrument. He lives with a daughter,
Wuyunhua'er.

``The Chahar dialect is still widely spoken in Chahar families in
Wenquan,'' the daughter said. ``But over all, I can't say how well
preserved the Chahar culture is in Wenquan.''

For the last decade, the prefecture has held a summertime festival
called
\href{http://www.nytimes3xbfgragh.onion/2008/07/11/world/asia/11mongolia.html?mcubz=2}{Naadam},
in which Mongolians celebrate traditional sports that include wrestling,
archery and horse racing. Wenquan began hosting it about four years ago.
Last year, officials changed the name from Naadam, a common Mongolian
word for such a festival, to the Hot Springs Festival, in a bid to
attract more tourists.

But in other ways, the festival has been expanding the spotlight on the
area's Mongolian heritage.

At last summer's event, Ms. Xiu said, ``local Kazakhs and Mongolians
sold costumes, handicrafts and relics for the first time, and this was
popular with tourists.''

Advertisement

\protect\hyperlink{after-bottom}{Continue reading the main story}

\hypertarget{site-index}{%
\subsection{Site Index}\label{site-index}}

\hypertarget{site-information-navigation}{%
\subsection{Site Information
Navigation}\label{site-information-navigation}}

\begin{itemize}
\tightlist
\item
  \href{https://help.nytimes3xbfgragh.onion/hc/en-us/articles/115014792127-Copyright-notice}{©~2020~The
  New York Times Company}
\end{itemize}

\begin{itemize}
\tightlist
\item
  \href{https://www.nytco.com/}{NYTCo}
\item
  \href{https://help.nytimes3xbfgragh.onion/hc/en-us/articles/115015385887-Contact-Us}{Contact
  Us}
\item
  \href{https://www.nytco.com/careers/}{Work with us}
\item
  \href{https://nytmediakit.com/}{Advertise}
\item
  \href{http://www.tbrandstudio.com/}{T Brand Studio}
\item
  \href{https://www.nytimes3xbfgragh.onion/privacy/cookie-policy\#how-do-i-manage-trackers}{Your
  Ad Choices}
\item
  \href{https://www.nytimes3xbfgragh.onion/privacy}{Privacy}
\item
  \href{https://help.nytimes3xbfgragh.onion/hc/en-us/articles/115014893428-Terms-of-service}{Terms
  of Service}
\item
  \href{https://help.nytimes3xbfgragh.onion/hc/en-us/articles/115014893968-Terms-of-sale}{Terms
  of Sale}
\item
  \href{https://spiderbites.nytimes3xbfgragh.onion}{Site Map}
\item
  \href{https://help.nytimes3xbfgragh.onion/hc/en-us}{Help}
\item
  \href{https://www.nytimes3xbfgragh.onion/subscription?campaignId=37WXW}{Subscriptions}
\end{itemize}
