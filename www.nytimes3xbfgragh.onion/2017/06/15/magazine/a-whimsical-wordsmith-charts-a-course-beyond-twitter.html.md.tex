A Whimsical Wordsmith Charts a Course Beyond Twitter

\url{https://nyti.ms/2srzVQl}

\begin{itemize}
\item
\item
\item
\item
\item
\item
\end{itemize}

\includegraphics{https://static01.graylady3jvrrxbe.onion/images/2017/06/18/magazine/18sun1/18sun1-articleLarge-v2.jpg?quality=75\&auto=webp\&disable=upscale}

Sections

\protect\hyperlink{site-content}{Skip to
content}\protect\hyperlink{site-index}{Skip to site index}

Feature

\hypertarget{a-whimsical-wordsmith-charts-a-course-beyond-twitter}{%
\section{A Whimsical Wordsmith Charts a Course Beyond
Twitter}\label{a-whimsical-wordsmith-charts-a-course-beyond-twitter}}

Jonny Sun's online personality --- a sentimental alien --- has attracted
a huge following. Now he's trying to figure out what comes next.

Jonny Sun.Credit...Chris Buck for The New York Times

Supported by

\protect\hyperlink{after-sponsor}{Continue reading the main story}

By Jesse Lichtenstein

\begin{itemize}
\item
  June 15, 2017
\item
  \begin{itemize}
  \item
  \item
  \item
  \item
  \item
  \item
  \end{itemize}
\end{itemize}

\textbf{T}here are a few things we should clear up about the Twitter
humorist known as ``jomny sun.'' Although sun's Twitter bio describes
him as an ``aliebn confuesed abot humamn lamgauge,'' and his avatar is a
line drawing of a smiling alien in a T-shirt, he is, in fact, a real
person, and that really is his name. Also, he knows how to spell.

Twitter is layered with niche and broadly popular humor accounts with
handles like @woodmuffin, @pixelatedboat and @weedhitler. Many of Sun's
half-million followers can be forgiven for assuming that, even
correcting for typos, ``\href{https://twitter.com/jonnysun}{Jonny Sun}''
is a nom de tweet. ``I'm fortunate that my name sounds like a fake
name,'' Sun says.

The \href{https://www.jomnysun.com/}{jokes Sun tweets} are sometimes
absurd, sometimes clever, often both. They wouldn't necessarily kill at
the Laugh Factory, but Sun's Twitter account offers comfort, whimsy and
an alternative to the rage/panic/schadenfreude/political flame-warring
of much online discourse.

\begin{quote}
on earth: a magiciam puts his hand in his hat\\
in the rabbit realm: The Hand emerges. it is time. the rabit council
must chose a sacrifice

LIFE HACK: give ur next child a normal name\\
ME: are u still mad that ur mother and i named u Life Hack
\end{quote}

And many of the tweets are not jokes at all --- they're unironically
joyful and lonely, a little bit bewildered by humanity and terminally
confused about English spelling.

\begin{quote}
look. life is bad. evryones sad. we're all gona die. but i alredy bought
this inflatable boumcy castle so r u gona take ur shoes off or wat
\end{quote}

Sun's recipe of amusement tinged with melancholy has won him the fandom
of big names in comedy and entertainment. The comedian Will Arnett told
me, ``I just latched onto his particular brand of --- it's not just
humor but, beneath the surface, a very resonating life lesson.'' When
Lin-Manuel Miranda first met Sun backstage at ``Hamilton,'' he joyously
tweeted: ``I met JOMNY SUN! It's a goob night!'' Sun frequently
converses with Miranda and Mara Wilson; he'll chat with Arnett, Patton
Oswalt and Joss Whedon and has become close with an array of Twitter and
YouTube stars.

But Sun's most devoted following probably skews younger. A certain kind
of jomny sun tweet makes a perfect high-school-yearbook quote, and Sun's
Twitter feed has overflowed with fans posting yearbook pages, often
noting that some yearbook staff member corrected Sun's wayward spelling.
At Sun's few public appearances, young fans have brought printouts of
his tweets for him to sign. Some of his most popular tweets aim for a
kind of resonant mystery without attempting a punch line. These have
inspired fans to create drawings, paintings and hand-glazed coffee mugs,
to say nothing of tattoos. One tweet, in particular, has birthed a fair
number of inky astronauts floating across a sidereal arm.

\begin{quote}
\end{quote}

The minor-key wonder of such tweets belies a more ambitious agenda. Sun,
like many young artists who have earned an audience through tireless
production on social media, is trying to figure out how that material
translates to other realms: in his case, playwriting, hip-hop, a spec
script for TV and, this month, the publication of his first book, an
illustrated story based loosely on his Twitter feed: ``everyone's a
aliebn when ur a aliebn too.''

At the same time, as a \href{http://www.jonathan-sun.com/}{Ph.D. student
in urban studies at M.I.T.}, he is trying to understand, in a more
objective way, the communities that social media fosters. His research
focuses on the question of how Twitter and YouTube and Instagram stars
(or viral content from nonstars) influence the broader world --- how
meaning is made online and how it spreads, changes the news or culture
and is itself altered as it spreads. As one popular Twitter comedian
remarked about him, ``he's like jane goodall and we're the apes.''

\textbf{On a bright day} in late February, I met Sun, now 27, at a
coffee shop in Brooklyn. He was dressed casually but crisply,
tortoiseshell glasses framing his face and a giant analog watch
devouring his right wrist. He didn't appear to be wearing socks. Sun
greeted me warmly. ``I sometimes get really intense social anxiety, so
it's difficult for me to get to know someone in person,'' he revealed,
but he has a generous laugh that puts people at ease.

\includegraphics{https://static01.graylady3jvrrxbe.onion/images/2017/06/18/magazine/18sun2/18sun2-articleLarge.jpg?quality=75\&auto=webp\&disable=upscale}

Sun had just emerged from the ``cave'' of finishing the book's
illustrations. He spent a year completing 180 drawings, pen on vellum,
and managed to damage his shoulder in the process. It was his attempt to
bring a sensibility reminiscent of some of his favorite
writer-illustrators --- Bill Watterson of ``Calvin and Hobbes,'' Shel
Silverstein, Maurice Sendak --- into the social-media age. In the book,
an alien comes to Earth and encounters animals and trees, which he
assumes are people. There is an otter that spouts art theory. An egg
enduring an existential crisis. A melancholic tree. A bee who offers
therapy to a bear. The story lines intersect, vanish, reappear.

Sun explains, ``It's supposed to read like the experience of joining
Twitter and following a bunch of people and seeing all their tweets on
your timeline.'' The animals are deliberately cute, rendered in an
unadorned style. Sun thinks of them as icons --- or avatars --- and
compares them to the people we view on social media. ``You don't
necessarily engage with a tweet the same way you'd engage with a real
person,'' he says. ``Twitter is often thought of as a shallow,
superficial thing. In reality, there's a lot of honest pathos and
humanity in it.''

When Sun first started paying attention to Twitter, in 2011, he followed
several professional comedians but quickly grew bored with their
self-promotion. Then he stumbled on a demimonde that has come to be
known as Weird Twitter. ``It was funny in a way I didn't know you could
be funny,'' Sun told me. These loosely affiliated, offbeat miniaturists
were sending up the format of traditional jokes by undercutting the
expectation of a punch line, using textual elements (odd spacing,
spelling, punctuation), creating tiny scripts or warping grammar to
comedic ends. Sun began responding to their jokes, and then writing his
own. Weird Twitter began to retweet him; he was weird by association,
and then by commission. Now his account has far more followers than many
of the humorists he admired --- often by a factor of 10.

``Part of my working theory on comedy --- and maybe just all art --- is
that it's supposed to feel like an inside joke, but you're supposed to
try to get everyone to feel like they're in on the inside joke,'' he
said. One feature of humor on Twitter is how little is typically known
of the author. The platform opens the door for a lot of chilling,
dispiriting, consequence-free vitriol, but it also removes some of the
preliminary gatekeepers to reaching an audience. If you're consistently
funny, you're consistently funny. People will share what you write.
``You can't talk about humor online without talking about
underrepresented communities,'' Sun says.

Twitter offered an alternative route for Sun, who is Asian, to catch the
eye of traditional media, just as it has for many others. Professional
comedy writing has long been a white-male preserve, but Twitter has
helped quite a few people who don't fit that mold --- people like Megan
Amram, Shelby Fero and Demi Adejuyigbe --- to gain admission to the
club. ``With Asian-American representation,'' Sun says, ``we still
aren't at the level where we have that many people to look up to yet.
Someone tweeted me and said, `When I found out you were Asian, I started
crying, because I'd never seen anyone like that doing what you do.' ''

Fans like Lin-Manuel Miranda admire what Sun has done. ``Establishing a
distinctive worldview, an indelible character, and adapting his voice to
the world as it unfolds on Twitter: That's \emph{a lot} to accomplish in
140 characters,'' Miranda told me. Another well-known Twitter comedian,
known simply as @darth, told me over chat that jomny sun ``just makes
reading twitter bearable. ... all the terrible news and then an
aliebn.''

Sun has his share of dissenters too. There is a fine line in comedy ---
in writing, in art --- between voice and tic. One Weird Twitter stalwart
publicly wondered how ``that Jonny Sun guy'' could post something like
``fimding a melen :)'' and get thousands of retweets. Another comedian
with a popular Twitter account lampooned Sun's sensibility:

\begin{quote}
doctor: sir ur dying\\
jomny sun: flowmers are beautifel n we shld kiss all the cloumds
\end{quote}

Sun is aware of the limits of this tactic, but as he points out, Twitter
is a very crowded room in which to be heard. ``When you see the
`gimmicky' stuff online,'' he told me, ``it's not necessarily a gimmick,
it's just something with a strong voice that cuts through everything
else.'' But the voice that holds your attention on social media can
sound like shouting in an empty room when it has to carry a book or a TV
show. (See ``\$\#*! My Dad Says.'') ``That's the balance to strike,''
Sun acknowledges.

\textbf{Sun's parents are} medical researchers who emigrated from China
to Calgary in the 1980s. ``I think my feelings of being an outsider and
not being able to connect --- those come down to me as an Asian-Canadian
and not really feeling at home anywhere,'' Sun told me. The family moved
to Toronto when Sun was 11, which offered a more cosmopolitan
environment. Still, Sun remained socially anxious and awkward. By the
time he got to high school, he felt virtually invisible. He evaporated
in groups, never spoke in class. But after school, everyone got on MSN
Chat, and it was as if his high school recreated itself online. Here he
felt liberated. People who would walk past him without acknowledgment
seemed to enjoy his personality. ``I was able to be funny and natural
through the keyboard and awkward and quiet and weird in person,'' he
told me. ``I wonder how many writers of this generation figured out how
to write through chat.''

In an effort to change his fate, Sun enrolled in drama class. Acting and
hanging around actors gave him ``permission to perform the way you wish
you could be.'' He began reading drama too --- he gravitated to the
short, comic plays of David Ives and fell in love with the work of the
Irish playwright Martin McDonagh, as well as with Beckett and the
theater of the absurd. But that path appeared to end with high school.
``My parents wanted me to do something that would offer more
stability,'' he says. ``I was good at math.''

He enrolled at the University of Toronto's engineering school, but after
graduation, looking to wed his technical skills to more creative
pursuits, Sun went to the Yale School of Architecture. ``His drawings
had a very distinct sensibility --- more Marimekko than Michelangelo,''
Sunil Bald, one of his professors there, says. ``There was an obvious
interest in cities over buildings, patterns over singular figures.''

It was in architecture school that he began a regimen of tweeting at
least one joke every day. It became a new discipline, a way to keep his
comedy brain sharp, almost an obsession. He also took classes in
playwriting at Yale's drama school. One of his plays, a sort of ``No
Exit'' with zombies, was produced in Toronto last fall. Another, ``Fried
Mussels,'' captured the imagination of Taylor Norton, a theater director
and independent producer who had fallen in love with the voice of Sun's
tweets and brought him to New York for a table read. ``Fried Mussels''
involves three pairs of characters at a diner, all engaged in somewhat
archetypal, emotionally charged situations. Their dialogue cascades down
the page in three separate columns, which makes reading the play
something of a choose-your-own-adventure experience.

Image

When Sun discovered Twitter, he says, ``it was funny in a way I didn't
know you could be funny.''Credit...Chris Buck for The New York Times

At the table reading, the actors got lost several times, distracted by
others' lines. But there were moments of harmony, when a single word
echoed through the room, and there were intense chords of dialogue when
everyone seemed to speak at once, impossible to follow but still
carrying meaning.

\textbf{Not long ago,} Sun was invited to the wedding of two people he
had never met. The couple had found each other online through their
shared loved of jomny son tweets. As Sun sees it, social-media platforms
are like urban landscapes, in which popular accounts function almost
like landmarks. They are spaces where people go to interact and
encounter one another; people imbue them with meaning and, over time, a
shared history.

Sun has made fruitful use of such connections with ``landmarks'' in his
collaboration with Susan Benesch at the Berkman Klein Center for
Internet and Society at Harvard. They make an interesting pair. Sun's
research looks at how online communities form and then develop rules of
engagement. Benesch, 53, is an expert in human rights law and the
founder of the Dangerous Speech Project; she is working to identify
examples of ``counterspeech'' on social media --- some of which
researchers call ``golden conversations,'' in which responses to
``dangerous'' or venomous speech can help change someone's mind or means
of expression. Humor seems to be a promising avenue.

At M.I.T., over the spring semester, Sun, Benesch and colleagues
interviewed a diverse roster of activists, podcasters, YouTubers, Vine
comedians and others, like a TV writer whose joke tweet helped inspire
the Tax March in April. But the most popular conversation was with Matt
Nelson, a student at Campbell University in North Carolina, whose
Twitter account WeRateDogs --- the handle is @dog\_rates --- boasts more
than two million followers. Students filled an auditorium at M.I.T.'s
Media Lab, and 55,000 people watched a video stream of the conversation
online.

WeRateDogs is not exactly edgy. With gently funny captions, the account
does what it promises: It rates pictures of cute dogs, referred to as
``doggos'' or ``puppers.'' All dogs get high scores --- usually at least
12 out of 10. WeRateDogs generally steers clear of politics, in part
because there's a financial risk: Nelson now employs two people and
sells hats. But dog-rating courted controversy when the account tweeted
pictures of dogs from the Women's March and travel-ban protests. Sun
became fascinated by how something so innocuous took on a political
cast. ``It's a Trojan-horse approach,'' he told me before speaking with
Nelson. ``People come for the cute animals,'' and what they find are
protesters who, in the presence of pets, seem more ``normalized,'' more
human. Yet WeRateDogs had publicly mocked an antagonist --- not exactly
golden conversation --- and that response also went viral.

Recently, Nelson began selling a hat featuring a presidential Twitter
typo (``covfefe''). This annoyed some left-wing dog lovers --- why make
money from this president's gibberish? Then he announced the account
would donate half the profits to Planned Parenthood. That angered
right-wing dog lovers. Nelson quickly promised to donate an equal amount
to a dog shelter. ``Dogs are bipartisan and that's why my following is
diverse and my account brings people together,'' he wrote in a lengthy
mea culpa. ``Last night I jeopardized that, and I'm sorry.'' The
response to this was the most brutal of all: As a furious segment of his
followers saw it, Nelson had just apologized for supporting women's
health. He took the statement down.

At M.I.T., Sun and Benesch asked Nelson about his political influence on
his followers, and theirs on him. It was hard for Nelson to say. The
discussion turned, inevitably, to how the tweeter in chief had changed
the landscape of social media. ``People are angrier,'' Sun said. ``I'm
getting more people mad at me. ... It has turned into a worse space over
all.''

Most of us are probably not equipped to handle the negativity that can
build up online --- the way a remark shared with a small circle can slip
its context and bring the swift and lasting condemnation of strangers,
the way public posts march down our screens in an endless conga line of
rage. Tweets like Sun's are landmarks that let us catch our breath. As
Sherry Turkle, a professor in the department of science, technology and
society at M.I.T., said to me describing the appeal of jomny sun: ``In
part we are smiling because we feel reassured. We are with our people.''

All the terrible news, in other words, and then an aliebn.

Advertisement

\protect\hyperlink{after-bottom}{Continue reading the main story}

\hypertarget{site-index}{%
\subsection{Site Index}\label{site-index}}

\hypertarget{site-information-navigation}{%
\subsection{Site Information
Navigation}\label{site-information-navigation}}

\begin{itemize}
\tightlist
\item
  \href{https://help.nytimes3xbfgragh.onion/hc/en-us/articles/115014792127-Copyright-notice}{©~2020~The
  New York Times Company}
\end{itemize}

\begin{itemize}
\tightlist
\item
  \href{https://www.nytco.com/}{NYTCo}
\item
  \href{https://help.nytimes3xbfgragh.onion/hc/en-us/articles/115015385887-Contact-Us}{Contact
  Us}
\item
  \href{https://www.nytco.com/careers/}{Work with us}
\item
  \href{https://nytmediakit.com/}{Advertise}
\item
  \href{http://www.tbrandstudio.com/}{T Brand Studio}
\item
  \href{https://www.nytimes3xbfgragh.onion/privacy/cookie-policy\#how-do-i-manage-trackers}{Your
  Ad Choices}
\item
  \href{https://www.nytimes3xbfgragh.onion/privacy}{Privacy}
\item
  \href{https://help.nytimes3xbfgragh.onion/hc/en-us/articles/115014893428-Terms-of-service}{Terms
  of Service}
\item
  \href{https://help.nytimes3xbfgragh.onion/hc/en-us/articles/115014893968-Terms-of-sale}{Terms
  of Sale}
\item
  \href{https://spiderbites.nytimes3xbfgragh.onion}{Site Map}
\item
  \href{https://help.nytimes3xbfgragh.onion/hc/en-us}{Help}
\item
  \href{https://www.nytimes3xbfgragh.onion/subscription?campaignId=37WXW}{Subscriptions}
\end{itemize}
