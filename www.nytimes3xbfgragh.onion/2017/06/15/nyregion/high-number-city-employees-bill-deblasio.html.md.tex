Sections

SEARCH

\protect\hyperlink{site-content}{Skip to
content}\protect\hyperlink{site-index}{Skip to site index}

\href{https://www.nytimes3xbfgragh.onion/section/nyregion}{New York}

\href{https://myaccount.nytimes3xbfgragh.onion/auth/login?response_type=cookie\&client_id=vi}{}

\href{https://www.nytimes3xbfgragh.onion/section/todayspaper}{Today's
Paper}

\href{/section/nyregion}{New York}\textbar{}With Largest Staff Ever, New
York City Reimagines How It Works

\url{https://nyti.ms/2tsOMqR}

\begin{itemize}
\item
\item
\item
\item
\item
\item
\end{itemize}

Advertisement

\protect\hyperlink{after-top}{Continue reading the main story}

Supported by

\protect\hyperlink{after-sponsor}{Continue reading the main story}

\hypertarget{with-largest-staff-ever-new-york-city-reimagines-how-it-works}{%
\section{With Largest Staff Ever, New York City Reimagines How It
Works}\label{with-largest-staff-ever-new-york-city-reimagines-how-it-works}}

\includegraphics{https://static01.graylady3jvrrxbe.onion/images/2017/06/16/nyregion/16HEADCOUNT1/00HEADCOUNT1-articleLarge.jpg?quality=75\&auto=webp\&disable=upscale}

By \href{http://www.nytimes3xbfgragh.onion/by/j-david-goodman}{J. David
Goodman}

\begin{itemize}
\item
  June 15, 2017
\item
  \begin{itemize}
  \item
  \item
  \item
  \item
  \item
  \item
  \end{itemize}
\end{itemize}

After serving a tour in the sticky rice and fruit fields of northeast
Thailand for the Peace Corps, Leanne Spaulding landed a job at a
Virginia-based trade association, working her way to a master's degree
from Duke University in environmental management.

Now Ms. Spaulding is in New York, where she was recently hired by the
city's Sanitation Department.

Her duties, naturally, involve garbage, but not in the traditional
sense: Ms. Spaulding is trying to help sell residents of the nation's
largest city on
\href{https://www.nytimes3xbfgragh.onion/2017/06/02/nyregion/compost-organic-recycling-new-york-city.html}{its
ambitious composting effort}. In that respect, her job is like thousands
of others added in recent years that are slowly changing the day-to-day
face of government service.

There are now nearly 294,000 full-time city employees, more than at any
point in the city's history.
\href{https://www.nytimes3xbfgragh.onion/2016/10/12/nyregion/bill-de-blasio-government-jobs.html}{The
growth} under Mayor Bill de Blasio comes at a time of record revenues in
a booming city, and has been across the board; nearly every city agency
now employs more workers than it did in 2014, when the mayor took
office.

The hiring has allowed the de Blasio administration to restaff agencies
that were cut back by Mayor Michael R. Bloomberg after the economic
downturn of 2008. But Mr. de Blasio has gone far further, expanding the
work force beyond its pre-recession peak, a costly investment that is
not without risk: the city could be vulnerable to an economic downturn.

A
\href{https://www.moodys.com/research/Moodys-New-York-Citys-robust-and-diverse-economy-drives-growth--PR_361053}{report
from Moody's} earlier this year heralded the diversity in the city's
economy, but noted that the city's debt service, pension and retiree
health care costs were growing rapidly. ``Increasing headcount brings
added costs with it in the future,'' said Nick Samuels, a senior credit
officer and the author of the report. ``Keeping up with that over time
will require additional economic growth.''

\includegraphics{https://static01.graylady3jvrrxbe.onion/images/2017/06/16/nyregion/16HEADCOUNT2/00HEADCOUNT2-articleLarge.jpg?quality=75\&auto=webp\&disable=upscale}

Carol Kellermann, the president of the nonprofit Citizens Budget
Commission, a fiscal watchdog group, questioned Mr. de Blasio's decision
to rapidly grow the city's head count during flush times, saying that it
made it more likely that new rounds of painful layoffs could be
necessary in the city's future.

``You don't have to keep adding people every year,'' she said. ``You
could manage what you have and use the staff that you have to run
programs. Find a way to do the things you want to do with the existing
work force.''

Mr. de Blasio is instead betting that the city can weather future
economic shoals by setting aside reserves. And he remains committed to
an expansive vision of government --- perhaps most evident in the surge
in hiring among the city's uniformed and law enforcement agencies: the
Police, Fire, Correction and Sanitation Departments,
\href{https://www.nytimes3xbfgragh.onion/2017/05/22/nyregion/new-york-jails-staffing.html}{as
well as the anticorruption} Department of Investigation.

The Fire Department has expanded its civilian work force by 20 percent
since June 2014, including more than 800 new emergency medical
technicians. The new hiring has allowed the department to refocus on its
\href{https://cbcny.org/advocacy/cbc-report-calls-transforming-fdnys-response-medical-emergencies}{most
pressing calls} --- cardiac arrests and other medical emergencies --- at
a time when New York had fewer fatal fires last year than at any point
in 100 years.

Police officers have been given unstructured time to engage with
neighborhood residents and hear complaints, eschewing the blare of the
radio for conversations.

Image

Capt. Jason Saffon of the Fire Department's Emergency Medical Service,
second from left, and Fritz Joseph, a paramedic, far right, helped a
woman who was struck by a car last month in the Bronx.Credit...Chang W.
Lee/The New York Times

Sanitation workers are now flanked by civilian outreach teams in blue
dress shirts with expertise in project management and mulch, helping New
Yorkers make sense of its new composting plan.

And the Department of Investigation has added inspectors general for the
police as well as the city's hospitals, growing faster than every other
area of city government. Its full-time staff of 352 is up almost 70
percent from where it was three years ago.

``Additional police officers. Additional traffic enforcement agents.
Additional pre-K teachers. Additional correction officers,'' Mr. de
Blasio said at a recent news conference. ``All that makes sense.''

All that also costs money: The city spent almost \$44 billion on its
work force in the 2016 fiscal year,
\href{http://www1.nyc.gov/assets/omb/downloads/pdf/exec17-fp.pdf}{according
to budget documents}, and that is expected to rise to \$51 billion by
2020.

Mr. de Blasio has been sensitive to the long-term implications of the
city's growing work force, from pensions to benefits: The average
pension cost per employee of the city government is about \$29,500 a
year out of a total average compensation of \$140,000, according to an
analysis by the Citizens Budget Commission of the current fiscal year's
personnel spending and actual city head count.

Image

Mr. Joseph, a Bronx paramedic since 1997, is part of a program that uses
a so-called fly car, which is meant to reach patients on the street more
quickly than an ambulance.Credit...Chang W. Lee/The New York Times

The mayor's latest budget set aside \$1.2 billion in reserves and adding
hundreds of millions to a retiree health benefit trust fund. Mr. de
Blasio has said that, if and when hard times return, he would not oppose
cutting back.

The Bloomberg administration laid off thousands of workers after the
recessions of 2002 and 2008, refilling those positions when the economy
turned around. That created dramatic swings that ``are not conducive to
stable, consistent delivery of city services,'' Ms. Kellermann, of the
Citizens Budget Commission, said.

The trouble is that the mayor has few levers to pull on his own should
good times end. Under Mayor Bloomberg, the city raised property tax
rates in response to the recession. (A spokeswoman for Mr. de Blasio
said he had no plans to raise property taxes, but did not rule out ever
doing so.)

For the moment, more employees mean more and new tasks. But how that
plays out day to day is not always readily apparent.

On a recent Thursday morning, a call came in requesting assistance for a
pedestrian hit by a car in the Bronx. Fritz Joseph, 45, a Bronx
paramedic since 1997, and his partner, Capt. Jason Saffon of the Fire
Department's Emergency Medical Service, were the first to arrive --- not
in an ambulance, but in a so-called fly car, a specially outfitted truck
that rushes to serious emergencies but does not transport patients.

Image

Ms. Spaulding, the composting expert, during the distribution of bins in
Brooklyn last month. The composting program has expanded to reach
hundreds of thousands of New Yorkers under Mayor Bill de
Blasio.Credit...Sam Hodgson for The New York Times

The fly car --- one of 10 in use in the Bronx, the first borough to try
the approach --- is meant to reach patients on the street more quickly
than an ambulance, and to stay on the street when the emergency has
passed. It's a system that has been used in other American cities as
well as in Europe.

``We definitely do more jobs,'' Mr. Joseph said. ``Almost one an hour.''

At the scene of the pedestrian accident, Mr. Joseph and Captain Saffon
arrived just before the ambulance. A woman had been struck at slow speed
in the parking lot of a Stop-and-Shop supermarket, and they helped her
up. It seemed that she would not need further help, but after learning
she was diabetic and testing her blood sugar, they decided to ride with
her in the ambulance to the emergency room, to begin an IV.

At the Sanitation Department, staff members have been hired to work at
its new transfer stations, to beef up its data-driven approach to snow
removal and garbage routes, to upgrade its garages to include bathrooms
for women, and to promote its composting initiative.

On a recent Friday morning in brownstone Brooklyn, some of the
Sanitation Department's nonuniformed workers handed out small brown bags
filled with dirt from the city's landfill that look like --- and have
been mistaken for --- artisanal coffee beans. The bags showcase the end
product of the composting process, and the dirt can be used in
flowerpots and gardens.

The Sanitation Department has grown its civilian ranks by 14 percent
since 2014 --- to 2,150 workers, in addition to the 7,600 uniformed
sanitation employees --- by recruiting people like Ms. Spaulding, the
composting expert.

On that Friday morning, Ms. Spaulding crisscrossed tree-lined streets in
Clinton Hill, as brown bins for food waste and yard clippings clattered
onto the doorsteps of unsuspecting homeowners, part of the city's
efforts to meet its
\href{https://www.nytimes3xbfgragh.onion/2014/05/31/nyregion/composting-in-new-york-city-pilot-program-expands.html}{goal}
of offering composting collection to all households by the end of 2018.

Outreach workers passed by in Zipcars with magnetic city logos stuck to
the sides while rented box trucks double parked to deposit bins. Nearby,
two other workers set up at a table by Fort Greene Park to explain the
composting initiative,
\href{https://www.nytimes3xbfgragh.onion/2015/05/22/nyregion/with-compost-program-keeping-waste-from-going-to-waste.html}{which
has expanded} to reach hundreds of thousands of New Yorkers under Mr. de
Blasio.

``The unit that oversees this, it was three people,'' Ms. Spaulding
said, wearing a shirt with the agency's logo and the words Recycling and
Sustainability embroidered below. ``It's really exploded. I hope we grow
more. There's a need for it.''

Advertisement

\protect\hyperlink{after-bottom}{Continue reading the main story}

\hypertarget{site-index}{%
\subsection{Site Index}\label{site-index}}

\hypertarget{site-information-navigation}{%
\subsection{Site Information
Navigation}\label{site-information-navigation}}

\begin{itemize}
\tightlist
\item
  \href{https://help.nytimes3xbfgragh.onion/hc/en-us/articles/115014792127-Copyright-notice}{©~2020~The
  New York Times Company}
\end{itemize}

\begin{itemize}
\tightlist
\item
  \href{https://www.nytco.com/}{NYTCo}
\item
  \href{https://help.nytimes3xbfgragh.onion/hc/en-us/articles/115015385887-Contact-Us}{Contact
  Us}
\item
  \href{https://www.nytco.com/careers/}{Work with us}
\item
  \href{https://nytmediakit.com/}{Advertise}
\item
  \href{http://www.tbrandstudio.com/}{T Brand Studio}
\item
  \href{https://www.nytimes3xbfgragh.onion/privacy/cookie-policy\#how-do-i-manage-trackers}{Your
  Ad Choices}
\item
  \href{https://www.nytimes3xbfgragh.onion/privacy}{Privacy}
\item
  \href{https://help.nytimes3xbfgragh.onion/hc/en-us/articles/115014893428-Terms-of-service}{Terms
  of Service}
\item
  \href{https://help.nytimes3xbfgragh.onion/hc/en-us/articles/115014893968-Terms-of-sale}{Terms
  of Sale}
\item
  \href{https://spiderbites.nytimes3xbfgragh.onion}{Site Map}
\item
  \href{https://help.nytimes3xbfgragh.onion/hc/en-us}{Help}
\item
  \href{https://www.nytimes3xbfgragh.onion/subscription?campaignId=37WXW}{Subscriptions}
\end{itemize}
