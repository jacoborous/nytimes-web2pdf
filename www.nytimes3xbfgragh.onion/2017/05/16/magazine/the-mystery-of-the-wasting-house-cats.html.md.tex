The Mystery of the Wasting House-Cats

\url{https://nyti.ms/2rllKIE}

\begin{itemize}
\item
\item
\item
\item
\item
\item
\end{itemize}

\includegraphics{https://static01.graylady3jvrrxbe.onion/images/2017/05/21/magazine/21cat1/21cat1-articleLarge-v2.jpg?quality=75\&auto=webp\&disable=upscale}

Sections

\protect\hyperlink{site-content}{Skip to
content}\protect\hyperlink{site-index}{Skip to site index}

The Health Issue

\hypertarget{the-mystery-of-the-wasting-house-cats}{%
\section{The Mystery of the Wasting
House-Cats}\label{the-mystery-of-the-wasting-house-cats}}

Forty years ago, feline hyperthyroidism was virtually nonexistent. Now
it's an epidemic --- and some scientists think a class of everyday
chemicals might be to blame.

Credit...Mark Peckmezian for The New York Times

Supported by

\protect\hyperlink{after-sponsor}{Continue reading the main story}

By Emily Anthes

\begin{itemize}
\item
  May 16, 2017
\item
  \begin{itemize}
  \item
  \item
  \item
  \item
  \item
  \item
  \end{itemize}
\end{itemize}

Most days, the back room of the Animal Endocrine Clinic in Manhattan is
home to half a dozen cats convalescing in feline luxury. They lounge in
their own individual ``condos,'' each equipped with a plush bed, a
raised perch and a cozy box for hiding. Classical music plinks softly
from speakers overhead. A television plays cat-friendly videos --- birds
chirping, squirrels scampering. Patients can also tune in to the live
version: A seed-stuffed bird feeder hangs directly outside each window.

One afternoon in April, a jet-black cat named Nubi assumed a predatory
crouch in his condo as a brawny pigeon landed on a feeder. Dr. Mark
Peterson, the soft-spoken veterinarian who runs the clinic, opened the
door to Nubi's condo and greeted the 12-year-old tom in a lilting,
high-pitched voice. ``How are you?'' Peterson asked, reaching in to
scratch his patient's soft chin. Nubi, who typically is so temperamental
that his owner jokes about needing a priest to perform an exorcism,
gently acquiesced, then turned back to the bird. Peterson seemed eager
to linger with each of Nubi's four feline neighbors --- Maggie, Biggie,
Fiji and Napoleon --- but, he warned, ``these cats back here are
radioactive.''

He meant that literally. The previous day, all five animals received
carefully titrated doses of radioactive iodine, designed to destroy the
overactive cells that had proliferated in their thyroid glands and
flooded their bodies with hormones. These cats are among the millions
suffering from hyperthyroidism, one of the most mysterious diseases in
veterinary medicine. When Peterson entered veterinary school in 1972,
feline hyperthyroidism seemingly didn't exist; today, he treats nothing
else. In the intervening decades, hyperthyroidism somehow became an
epidemic in cats, and no one knows why. ``I've devoted most of my time
in the last 35 years to this,'' said Peterson, who noted that he has
treated more than 10,000 hyperthyroid cats, ``and I still have more
questions than I have answers.''

Although definitive answers remain elusive, scientists are narrowing in
on one possible explanation: A steady drumbeat of research links the
strange feline disease to a common class of flame retardants that have
blanketed the insides of our homes for decades. But even as the findings
may answer one epidemiological question, they raise another in its
place. If household chemicals are wreaking havoc on the hormones of
cats, what are they doing to us?

\textbf{By the time Peterson} met Sasha in the fall of 1978, the scrawny
tuxedo cat was a regular at the Animal Medical Center in Manhattan. The
15-year-old had lost a profound amount of weight, despite a seemingly
insatiable appetite. Her case stumped veterinarians, who had already
ruled out many of the obvious culprits, including parasites,
irritable-bowel disease and diabetes.

Peterson, who had become restless with his veterinary residency, was
spending his time off attending endocrine rounds at New York Hospital.
When he heard about Sasha's symptoms, he thought of the thyroid, a gland
that sits at the base of the neck and secretes hormones that regulate
metabolism. In humans, weight loss and increased appetite are among the
hallmark symptoms of hyperthyroidism, in which the gland churns out huge
quantities of hormones, sending the body's internal systems into
overdrive.

Although cats weren't known to develop the condition, Peterson thought
the possibility was worth at least investigating. And so, one afternoon,
he ferried Sasha to the hospital, where a sympathetic doctor had agreed
to give the cat a thyroid scan. The image was unambiguous: There was a
large mass on Sasha's thyroid. The tumor was benign, but its
inexhaustible cells were dumping thyroid hormones into her bloodstream.
``We got all excited, and we didn't know exactly what we were doing, but
we removed the tumor,'' Peterson says. ``And the cat got better and
gained like five pounds in six months.''

It was a happy ending for Sasha, but for Peterson and Gerald Johnson,
the gastroenterologist at the Animal Medical Center, it was just the
beginning. ``Dr. Johnson said, `You know, I have these other cases that
I haven't been able to figure out,' '' Peterson recalls. ``So we
thought: We'll get them back. Let's test them.'' They quickly found four
more cats with benign thyroid tumors and elevated levels of thyroid
hormones. And the more they looked, the more hyperthyroid cats they
found. ``It didn't take very long to get a dozen cases, and then 30
cases, and then 100 cases,'' Peterson says. It was an astonishing
discovery --- dozens of pets wasting away from a disease that nobody
knew existed.

In the summer of 1979, Peterson presented the first five cases of feline
hyperthyroidism to a standing-room-only crowd at a veterinary conference
in Seattle. There, he learned that hyperthyroid cats had recently begun
turning up in Boston; the vets at Angell Memorial Animal Hospital would
soon publish a paper on their first 10 patients. The reports set the
veterinary world abuzz and raised some unsettling questions. ``The
first, among specialists, was, `How did we miss this?' '' recalls Duncan
Ferguson, a veterinarian and pharmacologist who was a co-author of a
1982 paper on the first cluster of cases to appear in Philadelphia. ``We
can't believe it just sort of appeared. Is this a new disease?''

It seemed to be. When Peterson later combed through old pathology
reports for 7,000 feline necropsies, he found that the thyroid
abnormalities he was seeing were rare until the late 1970s. But once the
outbreak started, it spread fast. From 1979 to 1983, the vets at the
Animal Medical Center saw three cases a month on average; by 1993, they
were seeing more than 20. The disease hopscotched across the United
States and then the world, striking cats in Canada, Europe, Japan,
Australia and New Zealand.

Today, senior cats are routinely screened for hyperthyroidism, and about
10 percent will be found to have the disease. Owners can choose from a
variety of treatments, including drugs, surgery or radioactive iodine,
which destroys the hyperactive thyroid cells while sparing the healthy
tissue. At his two clinics --- the one in Manhattan and another in
Bedford Hills, N.Y. --- Peterson administers radioiodine to more than
300 cats each year. But for all the progress veterinarians have made in
diagnosing and treating the disorder, it has been far trickier to
determine its origin.

\textbf{When hyperthyroidism} first surfaced in cats, Peterson was
confident that scientists would soon make sense of the curious
condition. A number of researchers, including Peterson, became
epidemiological detectives, searching for dietary, environmental and
lifestyle factors that distinguished the hyperthyroid cats from healthy
ones, and they turned up many leads. Among the many behaviors that
appeared to put cats at risk: spending time indoors, using cat litter,
eating canned food, eating fish-flavored canned food, eating
liver-and-giblet-flavored canned food, drinking puddle water, sleeping
on the floor, sleeping on bedding treated with flea-control products and
living in a home with a gas fireplace.

\includegraphics{https://static01.graylady3jvrrxbe.onion/images/2017/05/21/magazine/21cat2/21cat2-articleLarge.jpg?quality=75\&auto=webp\&disable=upscale}

It was a long and eclectic list, and from the 1980s to the early 2000s,
scientists proposed a wide range of potential culprits, including
chemicals used in canning and a toxic mystery substance in cat litter.
Eventually, researchers homed in on another possibility: a class of
flame retardants known as polybrominated diphenyl ethers (PBDEs).
Beginning in the 1970s, large quantities of the chemicals were routinely
added to many household goods, including couch cushions, carpet padding
and electronics. PBDEs can be itinerant compounds; they leach from our
sofas and TVs and latch onto particles of house dust, coating our floors
and furniture. They drift into soil, water and air and slip into the
bodies of animals, collecting in everything from the eggs of peregrine
falcons to the blubber of beluga whales.

PBDEs also happen to have a chemical structure that resembles thyroid
hormones and may mimic or compete with these hormones in the body,
binding to their receptors and interfering with their transport and
metabolism. By the mid-2000s, it was clear that they could alter thyroid
function in rodents, birds and fish, and the United States and the
European Union have now largely phased the chemicals out. (They remain
ubiquitous, however; PBDEs take years to degrade, and many people still
own products manufactured before they were taken off the market.)

As the health risks of PBDEs became clear, two scientists at the
Environmental Protection Agency --- Linda Birnbaum, a toxicologist, and
Janice Dye, a veterinarian --- began to wonder whether the chemicals
might also be responsible for the rise of hyperthyroidism in cats. ``How
do cats behave?'' says Birnbaum, who now directs the National Institute
of Environmental Health Sciences and its National Toxicology Program.
``They crawl on the floor. They sit on the couch. They lick their paws
all the time. So anything in the dust, they're going to end up
ingesting.'' If PBDEs were to blame, it would explain why the disease
didn't appear until the late 1970s, why it first emerged in the United
States --- where use of the chemicals was especially heavy --- and why
indoor cats seemed to be at particular risk.

Birnbaum and Dye started a small pilot study to scour the blood of 23
cats, including 11 with hyperthyroidism, for traces of PBDEs. The
volumes of blood they collected were so small that the graduate student
conducting the lab work worried that she might not detect anything. Her
fears were unfounded: The cats had PBDE levels 20 to 100 times as high
as those typically observed in American adults. Birnbaum and Dye, who
reported their results in a 2007 paper, also found relatively large
quantities of PBDEs in several types of cat food, particularly
seafood-flavored canned foods.

Several years later, a group in Illinois discovered that pet cats had
higher PBDE levels than feral ones and that hyperthyroid cats tended to
live in homes that were particularly saturated with the flame
retardants. In 2015, a Swedish team found that hyperthyroid cats had
significantly higher levels of three types of PBDEs in their blood than
healthy cats did. Last year, researchers in California reported a
similar result: Total PBDE levels were higher in cats with
hyperthyroidism than those without.

The findings are tantalizing but not definitive. Cats' lengthening life
spans may explain some of the increased incidence of the disease, and
it's possible that high PBDE levels are a result of hyperthyroidism,
rather than a cause; the compounds, which are stored in fat, may be
released into the bloodstream when cats lose weight. Even if flame
retardants do contribute to the disease, they may not be the sole cause.
Researchers at the California Department of Toxic Substances Control
recently identified more than 70 different compounds that seem to be
present in especially high concentrations in hyperthyroid cats. ``It's
terribly complicated to nail,'' says Ake Bergman, who led the Swedish
study and is the director of the Swedish Toxicology Sciences Research
Center. ``Because you are, and I am, and we are all, including the cats,
exposed to such a mixture of chemicals.''

Image

The cat condos at the Animal Endocrine Clinic in Manhattan.Credit...Mark
Peckmezian for The New York Times

\textbf{In the early 1950s,} the cats of Minamata, Japan, seemed to go
mad all at once. They began to stagger, stumble and convulse, limbs
flailing in every direction. They salivated uncontrollably. They hurled
themselves at stone walls and drowned themselves in the sea. This
``dancing-cat disease,'' as it came to be known, was a warning --- one
that went unheeded.

In the spring of 1956, a 5-year-old Minamata girl suddenly lost control
of her body. She dropped her food, wobbled when she walked and shuddered
with convulsions, biting her tongue until it bled. Other city residents,
including the girl's 2-year-old sister, soon began to exhibit similar
symptoms. Thousands of people eventually fell ill; many slipped into
comas and died. In 1959, a physician identified the cause of the
catastrophe: A local chemical plant had been dumping methylmercury into
the bay, poisoning the fish and, ultimately, the humans and cats who ate
them. ``In retrospect, if we'd paid more attention to the dancing cats,
we might have prevented some of the problems of mercury poisoning in the
people,'' says Peter Rabinowitz, who directs the University of
Washington's Center for One Health Research, which explores connections
among human, animal and environmental health.

Environmental toxicants are equal-opportunity hazards; mercury,
asbestos, pesticides and other compounds can cause health problems in
humans and animals alike. For at least a century --- since coal miners
began using caged canaries to alert them to the presence of toxic gases
--- we have known that we can put these shared vulnerabilities to
practical use. Sick animals can be sentinels, warning of looming threats
to human health. For household chemicals, cats and dogs, which tend to
spend nearly all their time in the home and happily hoover up whatever
detritus falls on the floor, may be particularly useful sentinels. ``Our
household pets are exposed to many of the same kinds of chemicals that
we are,'' Birnbaum says. ``I think if we see a health problem in our
animals, especially one that has arisen very recently --- genetics
doesn't change that quickly --- I think it's kind of raising the
canary-in-the-coal-mine issue.''

Could hyperthyroid cats be modern-day canaries? We know that flame
retardants accumulate in our own bodies; scientists find PBDEs in nearly
every person they test, including newborns. ``It's almost 100 percent
detection,'' says Heather Stapleton, an environmental chemist and
exposure scientist at Duke University. The compounds turn up in human
blood, breast milk and tissue and can persist for years in fat.

Over the course of decades, human PBDE levels skyrocketed, increasing
100-fold from the 1970s to the early 2000s. (These levels now appear to
be declining, most likely as a result of the phasing out of the
chemicals.) The rate of human thyroid cancer more than doubled over the
same time period. These parallel trends may be more than coincidence:
Multiple studies have shown that men and women with high concentrations
of PBDEs in their bodies tend to have altered levels of thyroid hormones
circulating in their bloodstreams. Last year, researchers reported that
thyroid problems were more common among American women with elevated
levels of PBDEs in their blood. And at a conference this spring,
Stapleton and her colleagues presented findings suggesting that
long-term exposure to PBDEs may be a risk factor for papillary thyroid
cancer; according to the unpublished data, living in a home with high
levels of one type of PBDE in the dust more than doubled the odds of
having the disease.

Thyroid hormones also play a crucial role in brain development; a
deficiency of these hormones, known as hypothyroidism, may cause
neurological abnormalities. If PBDEs cause unusual fluctuations in
hormone levels in early life, they may do lasting damage. Scientists
have found that those who are exposed to high concentrations of PBDEs in
utero or during early childhood score lower on tests of motor skills and
cognition. These findings are particularly worrisome given that young
children --- who are not uncatlike in their behavior, ingesting up to
200 milligrams of dust a day --- tend to have higher body burdens of
PBDEs than adults. The data are not conclusive, and the underlying
mechanisms remain unclear. But further studies of cats could help
scientists clarify what's happening. ``I remain convinced that paying
more attention to what the animals are trying to tell us is a really
good idea,'' Rabinowitz says. ``There are still many disease outbreaks
in animals that remain sort of unexplored or unexplained.''

Rabinowitz, who created the online Canary Database to index papers on
animal outbreaks that may be relevant to human health, thinks scientists
and clinicians could be more strategic about connecting the dots between
species. When he and his colleagues recently investigated the potential
health risks of hydraulic fracturing, they discovered that skin problems
were common in both the people and the dogs living near gas-extraction
sites. ``We're finding that there was really some utility in asking
about both people and animals when looking at a new hazard,'' Rabinowitz
says. He suggests that we consider linking the health records of pets
and their owners.

For his part, Peterson remains steadfastly focused on cats, which keep
showing up with thyroid hot spots that need to be injected with
radiation. He will keep them as comfortable as possible during their
stay at the ``Hypurrcat Spa,'' which is why he has converted the
floor-to-ceiling pipe into a scratching post and keeps towel-lined
baskets on the cold exam table. At his Bedford Hills clinic, which lacks
windows for bird-watching, he has even installed a cage of gerbils in
the cats' line of sight. (``People always say, `Are the gerbils
upset?' '' he told me. ``I think the gerbils like it, because they get
to see new cats.'') Sitting in his Manhattan clinic's waiting room,
where the cats are encouraged to relax on the furniture, he said: ``I
love the animals. I love the animals more than people, I think.''

Advertisement

\protect\hyperlink{after-bottom}{Continue reading the main story}

\hypertarget{site-index}{%
\subsection{Site Index}\label{site-index}}

\hypertarget{site-information-navigation}{%
\subsection{Site Information
Navigation}\label{site-information-navigation}}

\begin{itemize}
\tightlist
\item
  \href{https://help.nytimes3xbfgragh.onion/hc/en-us/articles/115014792127-Copyright-notice}{©~2020~The
  New York Times Company}
\end{itemize}

\begin{itemize}
\tightlist
\item
  \href{https://www.nytco.com/}{NYTCo}
\item
  \href{https://help.nytimes3xbfgragh.onion/hc/en-us/articles/115015385887-Contact-Us}{Contact
  Us}
\item
  \href{https://www.nytco.com/careers/}{Work with us}
\item
  \href{https://nytmediakit.com/}{Advertise}
\item
  \href{http://www.tbrandstudio.com/}{T Brand Studio}
\item
  \href{https://www.nytimes3xbfgragh.onion/privacy/cookie-policy\#how-do-i-manage-trackers}{Your
  Ad Choices}
\item
  \href{https://www.nytimes3xbfgragh.onion/privacy}{Privacy}
\item
  \href{https://help.nytimes3xbfgragh.onion/hc/en-us/articles/115014893428-Terms-of-service}{Terms
  of Service}
\item
  \href{https://help.nytimes3xbfgragh.onion/hc/en-us/articles/115014893968-Terms-of-sale}{Terms
  of Sale}
\item
  \href{https://spiderbites.nytimes3xbfgragh.onion}{Site Map}
\item
  \href{https://help.nytimes3xbfgragh.onion/hc/en-us}{Help}
\item
  \href{https://www.nytimes3xbfgragh.onion/subscription?campaignId=37WXW}{Subscriptions}
\end{itemize}
