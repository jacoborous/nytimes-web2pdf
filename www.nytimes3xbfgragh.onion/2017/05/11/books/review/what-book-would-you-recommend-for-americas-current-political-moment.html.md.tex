Sections

SEARCH

\protect\hyperlink{site-content}{Skip to
content}\protect\hyperlink{site-index}{Skip to site index}

\href{https://www.nytimes3xbfgragh.onion/section/books/review}{Book
Review}

\href{https://myaccount.nytimes3xbfgragh.onion/auth/login?response_type=cookie\&client_id=vi}{}

\href{https://www.nytimes3xbfgragh.onion/section/todayspaper}{Today's
Paper}

\href{/section/books/review}{Book Review}\textbar{}What Book Would You
Recommend for America's Current Political Moment?

\url{https://nyti.ms/2pAwjGx}

\begin{itemize}
\item
\item
\item
\item
\item
\item
\end{itemize}

Advertisement

\protect\hyperlink{after-top}{Continue reading the main story}

Supported by

\protect\hyperlink{after-sponsor}{Continue reading the main story}

\href{/column/bookends}{Bookends}

\hypertarget{what-book-would-you-recommend-for-americas-current-political-moment}{%
\section{What Book Would You Recommend for America's Current Political
Moment?}\label{what-book-would-you-recommend-for-americas-current-political-moment}}

By Francine Prose and Thomas Mallon

\begin{itemize}
\item
  May 11, 2017
\item
  \begin{itemize}
  \item
  \item
  \item
  \item
  \item
  \item
  \end{itemize}
\end{itemize}

\emph{In Bookends, two writers take on questions about the world of
books. This week, Francine Prose and Thomas Mallon suggest reading that
suits the current moment and tells us what we can expect from it.}

\textbf{By Francine Prose}

\emph{Jane Mayer's ``Dark Money'' explores the roots of the intense
class divisions and gross inequalities that beset us.}

Image

Francine ProseCredit...Illustration by R. Kikuo Johnson

Jane Mayer's harrowing ``Dark Money'' should be required reading for
anyone who wants to know how our country got into the mess we're in.
Mayer shows how the intense class divisions and gross inequalities that
beset us now were neither inevitable nor accidental, but carefully
orchestrated by a handful of billionaires determined to transform the
United States into a cash machine from which only the wealthiest
citizens could withdraw.

Beginning in the 1970s, alarmed by Richard Nixon's tough new
restrictions aimed at protecting the environment, these oil, banking and
industrial moguls spent fortunes working behind the scenes, fighting
against anything that would increase their costs or decrease their
profits. Perhaps the most famous of these Wizard of Oz-like figures are
Charles and David Koch, but there are others (among them the family of
Secretary of Education Betsy DeVos) whose stories Mayer tells: tycoons
who silenced whistle-blowers, intimidated rivals, sued and blackmailed
their own siblings, and whose plans for us involved polluting our air
and water, lowering taxes for the wealthy, and drastically decreasing
the amount we spend on education, health and welfare. To implement their
agenda, they started foundations, think tanks and university programs;
they financed libertarian political campaigns and the Tea Party
movement, all the while using these ``charitable'' donations as
loopholes to avoid paying taxes. According to Mayer, President Obama
never stood much of a chance against the power of his opponents' money,
determination and resistance --- and their untiring and ultimately
successful efforts to turn the Republican Party into the ``party of
no.''

Greed isn't pretty, and ``Dark Money'' won't make you feel differently
about the deadly sin that is currently ruling our society. But the book
answers some questions. In its final paragraph David Koch tells us
everything we need to know about what motivates him and others like him.
When, as a child, he was asked to share a treat, he would say, ``I just
want my fair share --- which is all of it.''

If ``Dark Money'' confirms our suspicions about the past and the
present, Nadezhda Mandelstam's ``Hope Against Hope'' --- a memoir about
what she and her husband, the great poet Osip Mandelstam, endured during
the worst years of Stalin's regime --- reflects our darkest fears about
a possible future. And yet, despite the terror and suffering it
describes, I am always heartened and consoled by the beauty and
intelligence with which Mandelstam tells us how a group of decent,
resourceful, gifted and complicated human beings braved a terrible era.

Having written a poem comparing Stalin's fingers to fat grubs, Osip was
arrested by the despot, who (as seems typical of despots) had no sense
of humor and appears to have been particularly offended by the line
about his fingers. Nadezhda had a prodigious memory, and her book is
full of sharply observed moments: We see the poet Anna Akhmatova
insisting that Osip, on his way to prison, eat the egg (a great delicacy
in those days) that the Mandelstams have procured and prepared for her,
in honor of her visit. Dense and detailed, the book tells us what it was
like to cope with the lack of basic necessities and to live under a
dictatorship: learning to lie, watching every word, discovering who can
and cannot be trusted, struggling to preserve one's dignity. And
everywhere are astute and moving observations about poetry, love and
life.

One more book I might add: For those days when you feel the need to take
a break, to experience a different time and place, to take the kind of
vacation from reality that only a great book can provide, pick up
Charles Dickens's ``David Copperfield.'' Read the first sentence, then
the second. See what happens next. At this current moment, perhaps even
more than most, there's a great deal to be said for literature as a
passport, as an inexpensive, trustworthy and temporary means of escape.

\emph{\textbf{Francine Prose}} \emph{is the author of more than 20 works
of fiction and nonfiction, among them the novel ``Blue Angel,'' a
National Book Award nominee, and the guide ``Reading Like a Writer,'' a
New York Times best seller. Her most recent novel is ``Mister Monkey.''
Currently a distinguished visiting writer at Bard College, she is the
recipient of numerous grants and awards; a contributing editor at
Harper's, Saveur and Bomb; a former president of the PEN American
Center; and a member of the American Academy of Arts and Letters and the
American Academy of Arts and Sciences.}

◆ ◆ ◆

\textbf{By Thomas Mallon}

\emph{There are flashes of recognition in H. R. Haldeman's chronicle of
Richard Nixon's presidency.}

Image

Thomas MallonCredit...Illustration by R. Kikuo Johnson

Trump-Nixon comparisons are in vogue, and any consideration of them
requires ``The Haldeman Diaries,'' the indispensable narrative that
Richard Nixon's chief of staff, Harry Robbins (H. R.) Haldeman, began
keeping on Jan. 18, 1969, two days before the start of a presidency
distinguished by bold foreign overtures; beset by Vietnam and domestic
dissent; and destroyed, finally, by Watergate. Haldeman switched from
handwriting the diaries to taping them late in 1970, only months before
the White House, interested in better record-keeping, began taping
itself.

As Nixon's ``lord high executioner,'' Haldeman presented a public image
as unyielding as his brush cut. While hardly literary, his chronicle of
``the P'' displays keen and constant powers of observation. The book is
in some ways \emph{Nixon's} diary, presented in the third person, with
Haldeman acting as superego to the president's id, preserving the boss's
utterances with the fidelity of a Boswell or an Eckermann. Haldeman
showcases the chief executive's lack of proportion and tendency to go on
``talking for hours,'' occasionally ``in one of his sort of mystic
moods'' but more often just profanely bitching.

Nixon seeks to create for himself a personal ``mystique,'' a spooky
charisma that will wean the public from its memory of John F. Kennedy's
glamour and flash: ``The P made the point that in describing Nixon you
have to make the point that he's always like the iceberg, you see only
the tip.'' But if he can be ``cool, tough, firm and totally in command''
during a crisis (Haldeman's own view of him), Nixon remains, day to day,
painfully maladroit with other people. The chief of staff shrewdly
monitors his psychology. When a reporter describes one presidential
statement as having been ``bitter,'' Nixon descends into an angry mood
that Haldeman deconstructs: ``I think he knows it was bitter and thus
doubly resents it.'' In May 1970, when the P ventures out to the Lincoln
Memorial around dawn, to talk with young antiwar protesters, Haldeman
writes: ``I am concerned about his condition.'' Seven months later,
Nixon decides (not uninterestingly) ``that the major effect of the youth
revolution is its effect on parents and their guilt complexes.''

The days grind on with problems and personalities that never seem to
evolve. How can the White House best deploy the quotable and incendiary
Vice President Spiro Agnew? Will Henry Kissinger, the national security
adviser, and William Rogers, the secretary of state, ever stop
complaining about each other? Haldeman curates the administration's
flamboyant gestures (P ``ordered me to have `Portnoy's Complaint'
\ldots{} removed from White House library''); its small screw-ups (``P
launched his Indian message with meeting with Taos tribe. No one
mentioned the whole deal would have to be translated''); its daily
little lies (``The Apollo shot was this morning; the P slept through it,
but we put out an announcement that he had watched it with great
interest'').

Take Haldeman's book down from the shelf in the spring of 2017, and
correspondences will abound: ``The P said that he thought he was the
only man who could be P.'' \emph{(I alone can fix it.)} Nixon sees
himself leading a movement of ``forgotten minorities'' (white ethnics)
that is bigger and more important than the Republican Party. The press
and the establishment are despised, and even a form of the Deep State
excites worry: ``Were the Camp David phones tapped for the Defense
Department?'' Haldeman wonders in a note to his text.

Comforting? \emph{Plus ça change?} Maybe. But such flashes of
recognition, however plentiful, remain incidental. There are really no
political precedents for the way we have to live now. In 1974, a
drifting ship of state foundered on the gifted, garrulous iceberg that
was Nixon himself. Today we are steered by a bellowing captain who has
never looked beneath the surface of anything. Is someone, after coming
home from the White House each night, secretly logging the voyage toward
the crash?

\emph{\textbf{Thomas Mallon's}} \emph{nine novels include ``Finale,''
``Henry and Clara,'' ``Fellow Travelers'' and ``Watergate,'' a finalist
for the PEN/Faulkner Award. He has also published nonfiction about
plagiarism (``Stolen Words''), diaries (``A Book of One's Own''),
letters (``Yours Ever'') and the Kennedy assassination (``Mrs. Paine's
Garage''), as well as two books of essays. His work appears in The New
Yorker, The Atlantic Monthly and other publications. A recipient of the
Vursell prize of the American Academy of Arts and Letters, for
distinguished prose style, he is Professor Emeritus of English at The
George Washington University.}

Advertisement

\protect\hyperlink{after-bottom}{Continue reading the main story}

\hypertarget{site-index}{%
\subsection{Site Index}\label{site-index}}

\hypertarget{site-information-navigation}{%
\subsection{Site Information
Navigation}\label{site-information-navigation}}

\begin{itemize}
\tightlist
\item
  \href{https://help.nytimes3xbfgragh.onion/hc/en-us/articles/115014792127-Copyright-notice}{©~2020~The
  New York Times Company}
\end{itemize}

\begin{itemize}
\tightlist
\item
  \href{https://www.nytco.com/}{NYTCo}
\item
  \href{https://help.nytimes3xbfgragh.onion/hc/en-us/articles/115015385887-Contact-Us}{Contact
  Us}
\item
  \href{https://www.nytco.com/careers/}{Work with us}
\item
  \href{https://nytmediakit.com/}{Advertise}
\item
  \href{http://www.tbrandstudio.com/}{T Brand Studio}
\item
  \href{https://www.nytimes3xbfgragh.onion/privacy/cookie-policy\#how-do-i-manage-trackers}{Your
  Ad Choices}
\item
  \href{https://www.nytimes3xbfgragh.onion/privacy}{Privacy}
\item
  \href{https://help.nytimes3xbfgragh.onion/hc/en-us/articles/115014893428-Terms-of-service}{Terms
  of Service}
\item
  \href{https://help.nytimes3xbfgragh.onion/hc/en-us/articles/115014893968-Terms-of-sale}{Terms
  of Sale}
\item
  \href{https://spiderbites.nytimes3xbfgragh.onion}{Site Map}
\item
  \href{https://help.nytimes3xbfgragh.onion/hc/en-us}{Help}
\item
  \href{https://www.nytimes3xbfgragh.onion/subscription?campaignId=37WXW}{Subscriptions}
\end{itemize}
