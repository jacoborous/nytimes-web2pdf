Sections

SEARCH

\protect\hyperlink{site-content}{Skip to
content}\protect\hyperlink{site-index}{Skip to site index}

\href{https://www.nytimes3xbfgragh.onion/section/business}{Business}

\href{https://myaccount.nytimes3xbfgragh.onion/auth/login?response_type=cookie\&client_id=vi}{}

\href{https://www.nytimes3xbfgragh.onion/section/todayspaper}{Today's
Paper}

\href{/section/business}{Business}\textbar{}Behind China's \$1 Trillion
Plan to Shake Up the Economic Order

\url{https://nyti.ms/2rc1mKd}

\begin{itemize}
\item
\item
\item
\item
\item
\end{itemize}

Advertisement

\protect\hyperlink{after-top}{Continue reading the main story}

Supported by

\protect\hyperlink{after-sponsor}{Continue reading the main story}

\hypertarget{behind-chinas-1-trillion-plan-to-shake-up-the-economic-order}{%
\section{Behind China's \$1 Trillion Plan to Shake Up the Economic
Order}\label{behind-chinas-1-trillion-plan-to-shake-up-the-economic-order}}

\includegraphics{https://static01.graylady3jvrrxbe.onion/images/2017/05/13/world/14china-road-1/14china-road-1-articleLarge.jpg?quality=75\&auto=webp\&disable=upscale}

By \href{http://www.nytimes3xbfgragh.onion/by/jane-perlez}{Jane Perlez}
and Yufan Huang

\begin{itemize}
\item
  May 13, 2017
\item
  \begin{itemize}
  \item
  \item
  \item
  \item
  \item
  \end{itemize}
\end{itemize}

\href{https://cn.nytimes3xbfgragh.onion/business/20170515/china-railway-one-belt-one-road-1-trillion-plan/}{阅读简体中文版}

VANG VIENG, Laos --- Along the jungle-covered mountains of Laos, squads
of Chinese engineers are drilling hundreds of tunnels and bridges to
support a 260-mile railway, a \$6 billion project that will eventually
connect eight Asian countries.

Chinese money is building power plants in Pakistan to address chronic
electricity shortages, part of an expected \$46 billion worth of
investment.

Chinese planners are mapping out train lines from Budapest to Belgrade,
Serbia, providing another artery for Chinese goods flowing into Europe
through a Chinese-owned port in Greece.

The massive infrastructure projects, along with hundreds of others
across Asia, Africa and Europe, form the backbone of China's ambitious
economic and geopolitical agenda. President Xi Jinping of China is
literally and figuratively forging ties, creating new markets for the
country's construction companies and exporting its model of state-led
development in a quest to create deep economic connections and strong
diplomatic relationships.

The initiative, called ``One Belt, One Road,'' looms on a scope and
scale with little precedent in modern history, promising more than \$1
trillion in infrastructure and spanning more than 60 countries. To
celebrate China's new global influence, Mr. Xi is gathering dozens of
state leaders, including President Vladimir V. Putin of Russia, in
Beijing on Sunday.

It is global commerce on China's terms.

Mr. Xi is aiming to use China's wealth and industrial know-how to create
a new kind of globalization that will dispense with the rules of the
aging Western-dominated institutions. The goal is to refashion the
global economic order, drawing countries and companies more tightly into
China's orbit.

\includegraphics{https://static01.graylady3jvrrxbe.onion/images/2017/05/13/world/14china-road-2/14china-road-2-articleLarge.jpg?quality=75\&auto=webp\&disable=upscale}

The projects inherently serve China's economic interests. With growth
slowing at home, China is producing more steel, cement and machinery
than the country needs. So Mr. Xi is looking to the rest of the world,
particularly developing countries, to keep its economic engine going.

``President Xi believes this is a long-term plan that will involve the
current and future generations to propel Chinese and global economic
growth,'' said Cao Wenlian, director general of the International
Cooperation Center of the National Development and Reform Commission, a
group dedicated to the initiative. ``The plan is to lead the new
globalization 2.0.''

Mr. Xi is rolling out a more audacious version of the
\href{http://marshallfoundation.org/marshall/the-marshall-plan/history-marshall-plan/}{Marshall
Plan}, America's postwar reconstruction effort. Back then, the United
States extended vast amounts of aid to secure alliances in Europe. China
is deploying hundreds of billions of dollars of state-backed loans in
the hope of winning new friends around the world, this time without
requiring military obligations.

Mr. Xi's plan stands in stark contrast to President Trump and his
``America First'' mantra. The Trump administration walked away from the
Trans-Pacific Partnership, the American-led trade pact that was
envisioned as a buttress against China's growing influence.

``Pursuing protectionism is just like locking oneself in a dark room,''
Mr. Xi
\href{https://www.nytimes3xbfgragh.onion/2017/01/17/business/dealbook/world-economic-forum-davos-china-xi-globalization.html?_r=0}{told
business leaders} at the World Economic Forum in January.

As head of the Communist Party, Mr. Xi is promoting global leadership in
China's own image, emphasizing economic efficiency and government
intervention. And China is corralling all manner of infrastructure
projects under the plan's broad umbrella, without necessarily ponying up
the funds.

Image

The bridge site near Vang Vieng. The work is a small piece of China's
``One Belt, One Road'' initiative, whose scope and scale has little
precedent in modern history.Credit...Adam Dean for The New York Times

China is moving so fast and thinking so big that it is willing to make
short-term missteps for what it calculates to be long-term gains. Even
financially dubious projects in corruption-ridden countries like
Pakistan and Kenya make sense for military and diplomatic reasons.

The United States and many of its major European and Asian allies have
taken a cautious approach to the project, leery of bending to China's
strategic goals. Some, like Australia, have rebuffed Beijing's requests
to sign up for the plan. Despite projects on its turf, India is uneasy
because Chinese-built roads will run through disputed territory in
Pakistan-occupied Kashmir.

But it is impossible for any foreign leader, multinational executive or
international banker to ignore China's push to remake global trade.

Germany's minister of economics and energy, Brigitte Zypries, plans to
attend the meeting in Beijing. Western industrial giants like General
Electric and Siemens are coming, as they look for lucrative contracts
and try to stay in China's good graces.

The Trump administration just upgraded its participation.

Originally, it planned to send a Commerce Department official, Eric
Branstad, the son of the incoming American ambassador to Beijing, Terry
Branstad. Now,
\href{https://www.nytimes3xbfgragh.onion/2017/04/04/world/asia/matthew-pottinger-trump-china.html}{Matthew
Pottinger}, senior director for Asia at the National Security Council,
will attend instead --- a signal that the White House is enhancing its
warm relationship with Mr. Xi by honoring his favorite endeavor with the
presence of a top official.

\hypertarget{influence-via-infrastructure}{%
\subsection{Influence Via
Infrastructure}\label{influence-via-infrastructure}}

As the sun beat down on Chinese workers driving bulldozers, four huge
tractor-trailers rolled into a storage area here in Vang Vieng, a
difficult three-hour drive over potholed roads from the capital,
Vientiane. They each carried massive coils of steel wire.

Half a mile away, a Chinese cement mixing plant with four bays glistened
in the sun. Nearby, along a newly laid road, another Chinese factory was
providing cement for tunnel construction.

Nearly everything for the Laos project is made in China. Almost all the
labor force is Chinese. At the peak of construction, there will be an
estimated 100,000 Chinese workers.

When Mr. Xi announced the ``One Belt, One Road'' plan in September 2013,
it was clear that Beijing needed to do something for the industries that
had succeeded in building China's new cities, railways and roads ---
state-led investment that turned it into an economic powerhouse. China
did not have a lot left to build, and growth started to sputter.

Along with the economic boost, tiny Laos, a landlocked country with six
million people, is a linchpin in Beijing's strategy to chip away at
American power in Southeast Asia. After
\href{https://www.nytimes3xbfgragh.onion/2017/01/23/us/politics/tpp-trump-trade-nafta.html}{Mr.
Trump abandoned the Trans-Pacific Partnership} in January, American
influence in the region is seen to be waning. The rail line through Laos
would provide a link to countries that China wants to bring firmly into
its fold.

Each nation in Mr. Xi's plan brings its own strategic advantages.

The power plants in Pakistan, as well as upgrades to a major highway and
a \$1 billion port expansion, are a political bulwark. By prompting
growth in Pakistan, China wants to blunt the spread of Pakistan's
terrorists across the border into the Xinjiang region, where a restive
Muslim population of Uighurs resides. It has military benefits,
providing China's navy future access to a remote port at Gwadar managed
by a state-backed Chinese company with a 40-year contract.

Many countries in the program have serious needs. The Asian Development
Bank estimated that
\href{https://www.adb.org/publications/asia-infrastructure-needs}{emerging
Asian economies} need \$1.7 trillion per year in infrastructure to
maintain growth, tackle poverty and respond to climate change.

Image

A new road to a construction site that is part of the Chinese railway
project near Luang Prabang. Laos is a linchpin in Beijing's strategy to
chip away at American power in Southeast Asia.Credit...Adam Dean for The
New York Times

In Kenya, China is upgrading a railway from the port of Mombasa to
Nairobi that will make it easier to get Chinese goods into the country.
The Kenyan government had been unable to persuade others to do the job,
whereas China has been transforming crumbling infrastructure in Africa
for more than a decade.

The rail line, which is set to start running next month, is the first to
be built to Chinese standards outside China. The country will benefit
for years from maintenance contracts.

``China's Belt and Road initiative is starting to deliver useful
infrastructure, bringing new trade routes and better connectivity to
Asia and Europe,'' said Tom Miller, author of ``China's Asian Dream:
Empire Building Along the New Silk Road.'' ``But Xi will struggle to
persuade skeptical countries that the initiative is not a smokescreen
for strategic control.''

\hypertarget{calculating-the-risks}{%
\subsection{Calculating the Risks}\label{calculating-the-risks}}

Although Chinese engineers just started arriving in
\href{https://www.nytimes3xbfgragh.onion/2016/10/02/travel/vang-vieng-laos-river-tubing-haven-offers-more-peace.html}{this
tourist town} several months ago, they have started punching three
tunnels into mountains that slope down to roiling river water. They are
in a race to get as much done as possible before the monsoon rains next
month slow down work.

It is a fast start to a much-delayed program that may bring only limited
benefits to the agrarian country.

For years, Laos and China sparred over financing. With the cost running
at nearly \$6 billion, officials in Laos wondered how they would afford
their share. The country's output is just \$12 billion annually. A
feasibility study by a Chinese company said the railway would lose money
for the first 11 years.

Image

A Chinese worker from Sichuan at the railway project near Luang
Prabang.Credit...Adam Dean for The New York Times

Such friction is characteristic.

In Indonesia, construction of a high-speed railway between Jakarta and
Bandung finally began last month after arguments over land acquisition.
In Thailand, the government is demanding better terms for a vital
railway.

China's outlays for the plan so far have been modest: Only \$50 billion
has been spent, an ``extremely small'' amount relative to China's
domestic investment program, said Nicholas R. Lardy, a China specialist
at the Peterson Institute for International Economics in Washington.

Even China's good friends so far are left wanting. Mr. Xi attended a
groundbreaking ceremony in 2014 in Tajikistan for a gas pipeline, but
the project stalled after Beijing's demand waned.

Mr. Putin will be at the center of the Beijing conference. While two
companies owned by one of his closest friends, Gennady Timchenko, have
benefited from projects, there has not been much else for Russia.

``Russia's elites' high expectations regarding Belt and Road have gone
through a severe reality check, and now oligarchs and officials are
skeptical about practical results,'' said Alexander Gabuev, senior
associate at the Carnegie Center in Moscow.

China is making calculations that the benefits will outweigh the risks.

The investments could complicate Beijing's effort to stem the exodus of
capital outflow that have been weighing on the economy. The cost could
also come back to haunt China, whose banks are being pressed to lend to
projects that they find less than desirable. By some estimates, over
half the countries that have accepted Belt and Road projects have credit
ratings below investment grade.

Image

A poster for a Chinese high-speed train at the construction site for a
bridge over the Mekong River near Luang Prabang.Credit...Adam Dean for
The New York Times

``A major constraint in investor enthusiasm,'' said Eswar Prasad,
professor of trade policy at Cornell University, ``is that many
countries in the Central Asian region, where the initial thrust of the
initiative is focused, suffer from weak and unstable economies, poor
public governance, political stability and corruption.''

Laos is one of the risky partners. The Communist government is a
longstanding friend of China. But fearing China's domination, Laos is
casting around for other friends as well, including China's regional
rivals Japan and Vietnam.

After five years of negotiations over the rail line, Laos finally got a
better deal. Laos has an \$800 million loan from China's Export-Import
Bank and agreed to form a joint venture with China that will borrow much
of the rest.

Still, Laos faces a huge debt burden. The International Monetary Fund
warned this year that the country's reserves stood at two months of
prospective imports of goods and services. It also expressed concerns
that public debt could rise to around 70 percent of the economy.

As construction gathers steam, nearby communities are starting to
rumble.

Farmers are balking at giving up their land. Some members of the
national assembly have raised questions about property rights.

At Miss Mai's Noodle Shop here, a customer, Mr. Sipaseuth, who said he
used only one name, pondered the project over a glass of icy
\href{http://www.nytimes3xbfgragh.onion/2009/05/26/business/global/26beer.html}{Beer
Lao}.

In the past, he said, the government had promised \$10 for an acre of
land worth about \$100. ``But then they never paid it,'' Mr. Sipaseuth
added.

Was the rail project good for Laos?

``We need civilization. Laos is very poor, very underdeveloped,'' he
said. ``But how many Chinese will come here? Too many is not a good
idea.''

Advertisement

\protect\hyperlink{after-bottom}{Continue reading the main story}

\hypertarget{site-index}{%
\subsection{Site Index}\label{site-index}}

\hypertarget{site-information-navigation}{%
\subsection{Site Information
Navigation}\label{site-information-navigation}}

\begin{itemize}
\tightlist
\item
  \href{https://help.nytimes3xbfgragh.onion/hc/en-us/articles/115014792127-Copyright-notice}{©~2020~The
  New York Times Company}
\end{itemize}

\begin{itemize}
\tightlist
\item
  \href{https://www.nytco.com/}{NYTCo}
\item
  \href{https://help.nytimes3xbfgragh.onion/hc/en-us/articles/115015385887-Contact-Us}{Contact
  Us}
\item
  \href{https://www.nytco.com/careers/}{Work with us}
\item
  \href{https://nytmediakit.com/}{Advertise}
\item
  \href{http://www.tbrandstudio.com/}{T Brand Studio}
\item
  \href{https://www.nytimes3xbfgragh.onion/privacy/cookie-policy\#how-do-i-manage-trackers}{Your
  Ad Choices}
\item
  \href{https://www.nytimes3xbfgragh.onion/privacy}{Privacy}
\item
  \href{https://help.nytimes3xbfgragh.onion/hc/en-us/articles/115014893428-Terms-of-service}{Terms
  of Service}
\item
  \href{https://help.nytimes3xbfgragh.onion/hc/en-us/articles/115014893968-Terms-of-sale}{Terms
  of Sale}
\item
  \href{https://spiderbites.nytimes3xbfgragh.onion}{Site Map}
\item
  \href{https://help.nytimes3xbfgragh.onion/hc/en-us}{Help}
\item
  \href{https://www.nytimes3xbfgragh.onion/subscription?campaignId=37WXW}{Subscriptions}
\end{itemize}
