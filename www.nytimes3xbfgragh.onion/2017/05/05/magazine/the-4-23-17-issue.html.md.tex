Sections

SEARCH

\protect\hyperlink{site-content}{Skip to
content}\protect\hyperlink{site-index}{Skip to site index}

\href{https://myaccount.nytimes3xbfgragh.onion/auth/login?response_type=cookie\&client_id=vi}{}

\href{https://www.nytimes3xbfgragh.onion/section/todayspaper}{Today's
Paper}

The 4.23.17 Issue

\url{https://nyti.ms/2pNo7GO}

\begin{itemize}
\item
\item
\item
\item
\item
\end{itemize}

Advertisement

\protect\hyperlink{after-top}{Continue reading the main story}

Supported by

\protect\hyperlink{after-sponsor}{Continue reading the main story}

The Thread

\hypertarget{the-42317-issue}{%
\section{The 4.23.17 Issue}\label{the-42317-issue}}

\includegraphics{https://static01.graylady3jvrrxbe.onion/images/2017/05/07/magazine/07thread1/07mag-07thread-t_CA1-articleLarge.jpg?quality=75\&auto=webp\&disable=upscale}

May 5, 2017

\begin{itemize}
\item
\item
\item
\item
\item
\end{itemize}

\textbf{RE: THE CLIMATE ISSUE}

\emph{The April 23 issue focused on how people will adapt to our
changing environment. Jon Gertner wrote about}
\href{https://www.nytimes3xbfgragh.onion/2017/04/18/magazine/is-it-ok-to-engineer-the-environment-to-fight-climate-change.html}{\emph{geoengineering}}
\emph{and the attempt to intentionally alter the environment in order to
fight climate change, and Jon Mooallem wrote about how the}
\href{https://www.nytimes3xbfgragh.onion/2017/04/19/magazine/our-climate-future-is-actually-our-climate-present.html}{\emph{future
is already here}}\emph{.}

\textbf{Your climate issue} is very sobering. We have the information we
need to tackle the existential challenge of climate change. What we
don't know is if we have the will to make sure that the most severe
effects of climate change don't permanently damage society. As important
as individual actions are, only collective action will save us from
ourselves. \emph{Edwin Andrews, Malden, Mass.}

\includegraphics{https://static01.graylady3jvrrxbe.onion/images/2017/05/07/magazine/07thread2/07thread2-articleLarge.jpg?quality=75\&auto=webp\&disable=upscale}

\textbf{Honestly, at this point,} what have we got to lose? The
situation is already on the brink of hopeless. We need geoengineering as
a stopgap. Preliminary research can minimize the risks. If this can be
done for a billion dollars per year, there is nothing else we can do for
so little cost, and we are too collectively stupid to make the
investments needed to produce an optimal solution. We'd much rather
spend our money on preparing to fight with each other, and we are not
good at optimal, complex solutions. We're much more likely to have at
least some success with simple and cheap.

Further, we still have some 4 billion people to add to our population
this century. Even if we stopped emitting greenhouse gases tomorrow, our
climate would continue to warm until a new equilibrium is reached. This
is referred to as committed warming. As the article points out, the
amount of carbon dioxide in the atmosphere now is equal to the amount 3
million years ago when camels roamed above the Arctic Circle. Yet we are
far from done mining carbon from the earth's crust and pumping it into
the atmosphere. Without geoengineering, the best we can do in the
process of converting to renewable energy sources is slow down warming
somewhat, but that is not nearly enough. \emph{Michael Rodriguez,
Corvallis, Ore.}

Image

Credit...Illustration by Giacomo Gambineri

\textbf{Jon Mooallem's essay} is reminiscent to me of Azar Nafisi's
``Reading Lolita in Tehran,'' where she highlights the intricacies of
obtaining freedom. That sense of knowing something bigger and better
might be there, but also of being unable to observe the hole left over
in the space it once occupied. It is only a hole in the minds of those
who held it, but since there is no hole in reality, newer humans move
on.

Elizabeth Kolbert touched on this idea, too, in ``The Sixth
Extinction,'' where she is visiting the reef at One Tree Island, being
simultaneously aware of the vastness of the ocean before her but also
that that feeling is what tricks us into complacency, of feeling that we
are alone and ineffective, when in fact we are so numerous as to be a
scourge on the earth.

We must live in the reality we inhabit. We must refuse to look away from
what we know of the past through our own experience but also through the
accounts of those who came before us. We must acknowledge our collective
wrongs and each take responsibility for more than our part. If we fail,
we condemn ourselves to our own future (and present) condemnations.
\emph{Melanie H., Chicago}

Image

Credit...Illustration by Giacomo Gambineri

\textbf{In 2009, researchers} from the Netherlands Environmental
Assessment Agency and elsewhere published their projections of the
greenhouse gas consequences if humanity came to eat less meat, no meat
or no animal products at all. The researchers calculated that a
no-animal-product diet could produce a cumulative reduction in 2010 to
2050 of carbon dioxide to 17 percent, methane emissions to 24 percent
and nitrous oxide emissions to 21 percent. What's more, the researchers
found that these gains would be achieved at a much lower cost than a
purely energy-focused intervention involving carbon taxes and renewable
energy technology.

To dedicate an entire magazine to climate change without providing
readers who may want to do something effective to combat it with a
simple change in diet is a grave missed opportunity. That meat recipe in
the Eat column could have been replaced with one for the heart-healthy,
environmentally friendly lentil. \emph{Claire Stadtmueller, New York}

Advertisement

\protect\hyperlink{after-bottom}{Continue reading the main story}

\hypertarget{site-index}{%
\subsection{Site Index}\label{site-index}}

\hypertarget{site-information-navigation}{%
\subsection{Site Information
Navigation}\label{site-information-navigation}}

\begin{itemize}
\tightlist
\item
  \href{https://help.nytimes3xbfgragh.onion/hc/en-us/articles/115014792127-Copyright-notice}{©~2020~The
  New York Times Company}
\end{itemize}

\begin{itemize}
\tightlist
\item
  \href{https://www.nytco.com/}{NYTCo}
\item
  \href{https://help.nytimes3xbfgragh.onion/hc/en-us/articles/115015385887-Contact-Us}{Contact
  Us}
\item
  \href{https://www.nytco.com/careers/}{Work with us}
\item
  \href{https://nytmediakit.com/}{Advertise}
\item
  \href{http://www.tbrandstudio.com/}{T Brand Studio}
\item
  \href{https://www.nytimes3xbfgragh.onion/privacy/cookie-policy\#how-do-i-manage-trackers}{Your
  Ad Choices}
\item
  \href{https://www.nytimes3xbfgragh.onion/privacy}{Privacy}
\item
  \href{https://help.nytimes3xbfgragh.onion/hc/en-us/articles/115014893428-Terms-of-service}{Terms
  of Service}
\item
  \href{https://help.nytimes3xbfgragh.onion/hc/en-us/articles/115014893968-Terms-of-sale}{Terms
  of Sale}
\item
  \href{https://spiderbites.nytimes3xbfgragh.onion}{Site Map}
\item
  \href{https://help.nytimes3xbfgragh.onion/hc/en-us}{Help}
\item
  \href{https://www.nytimes3xbfgragh.onion/subscription?campaignId=37WXW}{Subscriptions}
\end{itemize}
