Sections

SEARCH

\protect\hyperlink{site-content}{Skip to
content}\protect\hyperlink{site-index}{Skip to site index}

\href{https://myaccount.nytimes3xbfgragh.onion/auth/login?response_type=cookie\&client_id=vi}{}

\href{https://www.nytimes3xbfgragh.onion/section/todayspaper}{Today's
Paper}

A Taste of Home in One Pot

\url{https://nyti.ms/2rXL5sI}

\begin{itemize}
\item
\item
\item
\item
\item
\item
\end{itemize}

Advertisement

\protect\hyperlink{after-top}{Continue reading the main story}

Supported by

\protect\hyperlink{after-sponsor}{Continue reading the main story}

\href{/column/magazine-eat}{Eat}

\hypertarget{a-taste-of-home-in-one-pot}{%
\section{A Taste of Home in One Pot}\label{a-taste-of-home-in-one-pot}}

\includegraphics{https://static01.graylady3jvrrxbe.onion/images/2017/05/28/magazine/28eat1/28eat1-articleInline.jpg?quality=75\&auto=webp\&disable=upscale}

By \href{https://www.nytimes3xbfgragh.onion/by/tejal-rao}{Tejal Rao}

\begin{itemize}
\item
  May 25, 2017
\item
  \begin{itemize}
  \item
  \item
  \item
  \item
  \item
  \item
  \end{itemize}
\end{itemize}

Gonzalo Guzmán was a teenager, working two jobs as a dishwasher in San
Francisco, when he started taking English classes, bulking up on new
words, exercising them in increasingly complex sequences. On his days
off, he'd slide a pen between his teeth and work on his pronunciation,
repeating the words that tripped him up.

Guzmán, now 35, is the chef in the bustling, bilingual kitchens of
Nopalito, a popular Mexican restaurant with two locations. Within his
menus, he has preserved memories of the small town outside Veracruz,
Mexico, where he grew up. He started cooking there when he was just 6
--- boiling field beans in a clay pot over a fire. ``We always had beans
around,'' he said. ``They were like my homework.'' But Guzmán wasn't
dreaming about life as a chef; he was just pitching in with the day's
chores, helping out his mother and two younger sisters.

His mother worked in the fields, harvesting coffee, corn and peanuts,
but there wasn't enough money for the books and uniform he needed for
secondary school, or the bicycle that could cut his 40-minute walk to
class in half. ``For a long time, the only thing in my head was, how can
I help?'' Guzmán said. By the time he was 13, he felt he needed to earn
money. So he moved to Puebla and stayed with relatives, fixing cars,
working the line at a string-ball factory. He wanted to do more. Guzmán
crossed the border from Mexico to the United States in 1998, walking for
days in the desert, away from his family, toward uncertainty.

In San Francisco, he found his father and a job washing dishes. The work
is essential to every smooth-running restaurant service but is
physically exhausting and offers little opportunity for promotion. ``It
took me a couple of years to even think, I want to be a chef,'' he said.
Guzmán noticed who was mentored and moved into more powerful positions
and who wasn't. He thought that a fluency in English, combined with his
creativity, technical skill and ambition, might help him connect with a
mentor and move up. He took classes for 10 years as he slowly made his
way through the kitchen ranks. Prep cook, line cook, sous chef, chef.

In his new cookbook, ``Nopalito: A Mexican Kitchen,'' which he wrote
with Stacy Adimando, Guzmán shares recipes for the food he serves at the
restaurants, like a glorious soup his mother used to make for him, rich
with fried noodles and chicken on the bone. And he devotes several pages
to the restaurant's daily ritual of making masa --- dried corn,
rehydrated in an alkaline solution, then ground into the dough that
forms the base of fresh tortillas. ``I want to pass this process along
to all my cooks,'' he said, ``and hopefully if they ever open their own
Mexican restaurants, they'll keep this old-school cooking alive.''

\includegraphics{https://static01.graylady3jvrrxbe.onion/images/2017/05/28/magazine/28eat2/28eat2-articleLarge.jpg?quality=75\&auto=webp\&disable=upscale}

As a chef, he updates and riffs too. Guzmán once tasted a version of
\emph{frijoles puercos,} a dish of beans and pork, in northern Mexico
and adapted it into an archetypal brunch dish at Nopalito, optimized for
the weekend crowd. It's a delicious and satisfying one-pot meal, built
by bridging the two main ingredients with scrambled eggs.

To make it, Guzmán boils dried butter beans with onion and guajillo
chiles, until the beans swell up and go tender all the way through, then
seasons them while they're still hot. He sweats bacon and chorizo a
little, and scrambles eggs right in, folding in the crumbled sausage and
its rendered red-stained fat. The beans go in once the eggs are just
set, along with all the fragrant cooking liquid, and the whole lot
bubbles along together. In just a few minutes, the dish is ready. Guzmán
tosses some herbs, \emph{queso fresco} and dabs of vinegary salsa
\emph{escabeche} over the top when it's time to eat.

Of course, Guzmán cooks with his own homemade chorizo, a mix of cold
pork fat and shoulder meat, seasoned with guajillo, ancho and puya
chiles, a load of raw garlic and many spices. Kitchen shortcuts have
never been his style, but maybe they're yours. Look for Mexican-style
chorizo already prepared at a grocery store, and squeeze the raw meat
out of its casing, directly into the hot pan. ``I understand, it's a lot
of work and some people don't have the time,'' Guzmán said. ``This is
life!''

\textbf{Recipe:}
\href{https://cooking.nytimes3xbfgragh.onion/recipes/1018774-gonzalo-guzmans-pork-braised-butter-beans-with-eggs}{Gonzalo
Guzmán's Pork-Braised Butter Beans With Eggs}

Advertisement

\protect\hyperlink{after-bottom}{Continue reading the main story}

\hypertarget{site-index}{%
\subsection{Site Index}\label{site-index}}

\hypertarget{site-information-navigation}{%
\subsection{Site Information
Navigation}\label{site-information-navigation}}

\begin{itemize}
\tightlist
\item
  \href{https://help.nytimes3xbfgragh.onion/hc/en-us/articles/115014792127-Copyright-notice}{©~2020~The
  New York Times Company}
\end{itemize}

\begin{itemize}
\tightlist
\item
  \href{https://www.nytco.com/}{NYTCo}
\item
  \href{https://help.nytimes3xbfgragh.onion/hc/en-us/articles/115015385887-Contact-Us}{Contact
  Us}
\item
  \href{https://www.nytco.com/careers/}{Work with us}
\item
  \href{https://nytmediakit.com/}{Advertise}
\item
  \href{http://www.tbrandstudio.com/}{T Brand Studio}
\item
  \href{https://www.nytimes3xbfgragh.onion/privacy/cookie-policy\#how-do-i-manage-trackers}{Your
  Ad Choices}
\item
  \href{https://www.nytimes3xbfgragh.onion/privacy}{Privacy}
\item
  \href{https://help.nytimes3xbfgragh.onion/hc/en-us/articles/115014893428-Terms-of-service}{Terms
  of Service}
\item
  \href{https://help.nytimes3xbfgragh.onion/hc/en-us/articles/115014893968-Terms-of-sale}{Terms
  of Sale}
\item
  \href{https://spiderbites.nytimes3xbfgragh.onion}{Site Map}
\item
  \href{https://help.nytimes3xbfgragh.onion/hc/en-us}{Help}
\item
  \href{https://www.nytimes3xbfgragh.onion/subscription?campaignId=37WXW}{Subscriptions}
\end{itemize}
