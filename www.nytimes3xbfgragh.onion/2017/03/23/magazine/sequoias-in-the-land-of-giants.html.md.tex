In the Land of Giants

\url{https://nyti.ms/2mTrEOR}

\begin{itemize}
\item
\item
\item
\item
\item
\item
\end{itemize}

\includegraphics{https://static01.graylady3jvrrxbe.onion/images/2017/03/26/magazine/26sequoias1/26mag-26sequoias-t_CA0-articleLarge.jpg?quality=75\&auto=webp\&disable=upscale}

Sections

\protect\hyperlink{site-content}{Skip to
content}\protect\hyperlink{site-index}{Skip to site index}

The Voyages Issue

\hypertarget{in-the-land-of-giants}{%
\section{In the Land of Giants}\label{in-the-land-of-giants}}

Communing with some of the biggest trees on Earth.

Titans in the fog in Sequoia National Park, California.Credit...David
Benjamin Sherry for The New York Times

Supported by

\protect\hyperlink{after-sponsor}{Continue reading the main story}

By Jon Mooallem

\begin{itemize}
\item
  March 23, 2017
\item
  \begin{itemize}
  \item
  \item
  \item
  \item
  \item
  \item
  \end{itemize}
\end{itemize}

The trees are so big that it would be cowardly not to deal with their
bigness head on. They are very, very big. You already knew this ---
they're called ``giant sequoias'' --- and I knew it, too. But in person,
their bigness still feels unexpected, revelatory. And the delirium of
their size is enhanced by their age, by the knowledge that some of the
oldest sequoias predate our best tools for processing and communicating
phenomena like sequoias, that the trees are older than the English
language and most of the world's major religions --- older by centuries,
easily, even millenniums. The physical appearance of a tree cannot be
deafening, and yet with these trees, it is. Facing down a sequoia, the
most grammatically scrambled thoughts wind up feeling right. Really,
there's only so much a person can do or say. Often I found myself
expelling a quivering, involuntary \emph{Whoa.}

The first time that happened, I was driving into Sequoia National Park
from the foothills of central California's Sierra Nevada, south of
Yosemite. Suddenly, the Four Guardsmen came into view: a tight quartet
of elephantine sequoia trunks through which the road passes. The trees
have tops, too --- those trunks lead to crowns --- but that's
immaterial; the trunks are all you have a hope of registering from
inside your car. They fill the windows and function as a gateway. They
were like living infrastructure, rising out of the snow.

The rental-car company had given me a squat Fiat micro-S.U.V., which,
though it was equipped with all-wheel drive and seemed to be handling
capably enough, was so strikingly unbrawny in appearance that crunching
up the icy, winding mountain road, I wasn't brave enough to push it any
faster than a feeble crawl. Now, with the squeeze through the Guardsmen
ahead of me tightening, I slowed even more. I heard myself letting out
an anticipatory holler, like a Hollywood fighter pilot banking through a
dogfight, and threaded the needle at nine miles an hour.

There are more than 8,000 sequoias in the Giant Forest, the
three-and-a-half-square-mile centerpiece of the park. The largest grow
more than 300 feet tall and 30 feet across, barely tapering as they rise
until, about two-thirds of the way up, the scrambling madness of their
branches starts. The branches are crooked and gnarled, while the rest of
the tree is stoic and straight. The branches are grayish and brownish
--- average American tree colors --- while the trunk, particularly in
sunlight reflected off snow, hums with a dreamy reddish-orange glow. The
branches often seem to have nothing to do with the sequoia they're
attached to; they are trees themselves. In 1978, a branch broke off a
sequoia called the General Sherman. It was 150 feet long and nearly
seven feet thick. All by itself, that branch would have been one of the
tallest trees east of the Mississippi.

The General Sherman Tree is one of the park's primary attractions. It's
275 feet tall, 100 feet in circumference, and known to be the largest
tree on Earth, by volume. (The National Park Service drives home its
massiveness on a sign in front of its trunk this way: If the General
Sherman were hollowed out and filled with water, it'd be enough water
for you to take a bath every day for 27 years.) The General Sherman is
not far off the Generals Highway, which runs through the park. It is a
tree with its own parking lot. Though the pathways were ice-crusted or
snowed under when I visited last month, I watched tourists of all shapes
and sizes hobble and skitter over them toward the tree for photographs:
the Italian dude with the soul patch posing with double thumbs up; the
overweight couple huffing, ``You make it to the tree?'' to a few young
women returning to their car; the young man looking up at the tree, eyes
closed and still, face in the sun --- a tranquil image of cosmic,
momentary oneness were it not for his self-aggrandizing sweatshirt,
which read, I AM NOT A GOD BUT SOMETHING SIMILAR. And then there was the
woman with a moaning child in her arms. She was whispering, ``Last one,
I promise,'' while her husband set up a tripod and timer, far, far away,
struggling to frame his teensy family against the universe of the tree.
Eventually the man found he had to reposition and walked right in front
of me. When we made eye contact, he said, ``It's big!''

Exactly, yes. And still, it's not just that the trees are big; it's that
everything about them is also big. The raised columns of bark running
down their trunks are bigger than the bark on ordinary trees. The
gullies between those columns are wider and deeper. The fire scars are
bigger. (Sequoias are mostly fire-resistant, even when wildfires or
lightning burn away at their bases, opening triangular, vaulting caverns
in their trunks, like grottos in a sea cliff.) The burls on the trees
are bigger. Even the woodpecker holes are bigger, which seems illogical
--- you'd expect woodpeckers to hammer out the same size holes,
regardless --- but honestly, they are. Every element of a sequoia is
freakishly, but also flawlessly, proportionally big. And this creates a
subconscious sense that you're not looking at a normal tree that just
kept growing until it became very tall but a tree that was somehow
supernaturally inflated to unimaginable dimensions, all of its features
swelling like some fantastically transformed mushroom or a cursed
cartoon man bloating into a giant. This aspect of the sequoia's size is
also a tricky thing to pick up from photographs. Even if there's a fence
or person in the shot for scale, the human eye can find a way to correct
for the sequoias' unacceptable gigantism: It reads the fir trees near
the sequoias as bushes, to make the sequoias seem like ordinary trees;
or it flattens the perspective, so that, say, four far-off sequoias
appear to be right alongside six cedars in the foreground --- fusing all
of them into a single line of 10 perfectly boring-size trees. In one of
these ways or another, virtually every sequoia picture I took wound up a
dud. Later, when I texted a friend what I thought was the best one, she
mistook it for a shot of my backyard.

``I feel like I'm in a fairy tale!'' a woman named Angela Fitzpatrick
announced one afternoon. Fitzpatrick and I were the only two people who
had shown up for a snowshoe hike led by a nonprofit group called the
Sequoia Parks Conservancy. The park's sparse winter crowds heightened
the otherworldliness of the trees. So did all the snow. The woods were
hushed around us, a cradle of pure whites, reds and greens.

Fitzpatrick was an information-security analyst from Tampa, Fla., who
had been flown out to audit a credit union in a nearby town, then
planned an extra day to see the trees. She was excellent company,
equally not-shy when it came to fumbling expressions of stupefaction and
delight. At one point, falling behind, I realized I hadn't yet touched a
sequoia, so I veered off and patted one. ``It's soft!'' I shrieked.
``What the hell?'' (The trees' outer layer is spongy and fibrous --- a
defense against burrowing bugs.) ``That's crazy!'' Fitzpatrick said. She
hustled back to put a hand on the tree. We stood side by side for a
second, pressing and kneading it. ``I'm so glad you touched that!'' she
said.

Later we stopped short in front of another sequoia that looked perfectly
healthy on one side, but was chewed up by fire on the other, leaving a
150-foot-tall concave husk from ground to crown --- a pillar of
charcoal. It was shocking: a baleful black chamber the color of new
asphalt, or volcanic rock, or Mordor. Deep in, at the rear, I could see
another opening, a twisting pit through the mulchy ground toward its
roots.

Is that even alive? we asked our guide, Katie Wightman. Of course it
was, she said; a tree like this might endure for centuries. Then she
asked, ``You guys wanna get inside?'' We did.

\textbf{There's a type} of enchantment we feel from afar, for certain
places and things, that's hard to pick apart or defend after years of
feeling it. I used to live in San Francisco and had encountered the
sequoias' cousins, the coast redwoods, many times. They were, in my
mind, the slightly less spectacular of America's two spectacularly large
tree species: taller than sequoias, in many cases, but plainer --- more
conventionally treelike and slender, with pinnacled, Christmas-tree tops
and duller, browner bark. But mostly they were just more accessible, at
least to me. Their range runs from south of Monterey up the coast into
Oregon. One of the most famous groves, Muir Woods, was close enough to
the city that I once chaperoned my daughter's preschool field trip
there.

Sequoias, on the other hand, existed only at the edges of my personal
geography. In all the world, there were only about 70 native groves of
them, flecked across a relatively thin stretch of the Sierra, far east
of San Francisco and Los Angeles, beyond the Central Valley's citrus
groves and almond fields. It was arbitrary, but I'd lived my life in
California predominantly on a north-south axis, road-tripping more often
along the coast than inland to the mountains. Redwoods were creatures I
ran into from time to time without trying, while sequoias remained
effectively hidden. They were the giants I needed to search out and
pursue. And this implied something, too, about the alluring enormousness
of the world that contained them.

Now I wanted to go see some of the oldest, biggest trees on Earth so I
could feel small. The literature of sequoias is, counterintuitively,
also a celebration of smallness. There's a promise of renewal and
transcendence in the juxtaposition of self and tree. The ecstatic
naturalist John Muir, among the first to go gaga for ``King Sequoia,''
wrote that ``one naturally walked softly and awe-stricken among them ...
subdued in the general calm, as if in some vast hall pervaded by the
deepest sanctities and solemnities that sway human souls.'' (Muir also
made ``wine'' by soaking the trees' cones in water and drank it as a
``sacrament.'' He wrote, ``I wish I were so drunk and Sequoical that I
could preach the green brown woods to all the juiceless world.'')

It seemed like a particularly good moment in America for humility, for
perspective-taking, for recalibrating my sense of scale and time. But
the night before I was supposed to fly out, a snowstorm unexpectedly hit
the Sierra, provoking a long and brutally disincentivizing warning on
the National Park Service's website. ``Roads may close,'' it said, and
tire chains were now mandatory --- equipment I'd always found
irrationally intimidating, even more so, perhaps, than the prospect of
skidding off a mountainside. The alert concluded: ``If you're
uncomfortable driving in the mountains during winter storms, consider
postponing your visit.''

I suppose it was typical winter-mountain stuff. But in my inexperience,
I panicked. And I continued panicking until I eventually reached the
park --- a day later than I had planned, after deciding to indulge that
panic and spend a night at the base of the mountain, betting the roads
would at least partly thaw in the morning. ``Snow panic,'' a friend
called it, a friend who had been considering meeting me in the sequoias
and was now bowing out. It was a familiar strain of jittery duress and
intensifying fragility that comes from trying with all your energy to
figure out exactly how bad the future will be.

\includegraphics{https://static01.graylady3jvrrxbe.onion/images/2017/03/26/magazine/26sequoias2/26mag-26sequoias-t_CA2-articleInline.jpg?quality=75\&auto=webp\&disable=upscale}

In retrospect, I recognize that the weather was just one more
uncertainty --- and one too many --- to withstand; already, I worried
that a strange-but-minor injury on the ball of my foot might become
inflamed and keep me from hiking around the park, and that a scratch in
my throat was the beginnings of my daughter's flu. And beneath the foot
and the flu were other worries --- namely, about the recklessly
accelerating gush of world events that I'd been pummeling myself with
many times an hour online. All it took was returning after a few hours
away from Twitter to discover a long record of outrages stacked up and
hardened like signs of ancient droughts or fires preserved in the rings
of a tree. The timeline was quickening, tightening; there were certain
days on which we'd all lived through centuries. When I called William C.
Tweed, a former ranger at the park, he told me, ``On a good day, the
sequoias remind us that we're not really in charge of the world.'' I
wanted \emph{that.} But the snow was a reminder that not being in charge
also means being powerless. That kind of smallness didn't feel
liberating at all. I hated it.

\textbf{Sequoia National Park} was established in 1890, at a moment in
America not so wildly different from our own. It was an era of
intensifying inequality, vulnerability and dislocation. Urban
industrialization upended rural tradition, and populist uprisings, like
the Pullman Strike and the Haymarket Riot, pitted an exasperated working
class against a government that seemed to collude with the corporations
exploiting it. As a labor leader in San Francisco named James Martin
wrote, with society seemingly in ``chaotic condition, there is ample
scope for the most dismal speculation.'' And so, in 1885, a collective
of radicals, including Martin, decided to build an alternate society,
applying to purchase government land in the Sierra where they could
construct a glimmering socialist utopia. Kaweah Colony, they called it.
Fifty-three individuals filed claims for 8,000 adjoining acres, centered
in the Giant Forest.

American settlers had been enraptured by the giant sequoias since they
first stumbled onto them 30 years earlier, and yet the government had
never seen any reason to protect the land; in fact, the federal Timber
and Stone Act, under which the Kaweah colonists were purchasing their
acreage, was meant to encourage logging in the West. And this was the
colonists' plan: They'd be lumberjacks, bankrolling their utopia with
that enormous storehouse of wood. All they had to do was build a road in
and out of the forest --- 20 grueling miles straight up a mountainside
pocked with jagged eruptions of granite. A tremendous job, but doable,
they decided. They were optimists, after all.

By the end of the following year, there were 160 Kaweah colonists on
site, throwing themselves at the road-cutting project and establishing
the structures of their new civic life. The colonists split into
``divisions,'' then subdivided the divisions into hundreds of different
``departments,'' like a Hand-Craft Department and an Amusements
Department. They exchanged man-hours as currency and got a lot done;
Kaweah quickly turned into an egalitarian cooperative. ``Brute
passions,'' Martin reported, were ``surrendering to moral restraint,''
and an ``inoffensive and charming rivalry exists to outdo the other in
neighborly acts.'' Colonists picnicked together, dried fruit, sewed
clothes and never spanked their children. One photo shows dozens of them
posing in front of one phenomenally large sequoia --- a tree so
unmistakably magnificent they named it the Karl Marx Tree.

By the summer of 1890, the colonists had pushed their road within a few
miles of the sequoias. They decided to pause there and start felling
pine trees, to scratch together the money they needed to finish. But
that fall, Congress created Sequoia National Park, only the second in
what would become America's national park system. The government didn't
try to seize private land for the park; in this case, the Kaweah
colonists didn't technically own the acreage. Their application to buy
it had never been officially approved. Only private citizens were
allowed to purchase land under the Timber and Stone Act, and because all
53 original Kaweah claimants had used the same San Francisco address on
their paperwork, officials had flagged it, suspecting they were a large
and devious corporation. (Logging companies were, in fact, grossly
abusing the law, coordinating groups of locals --- sometimes just by
buying rounds at the local saloon --- to claim chunks of land on their
behalf.) The colonists were aware of this bureaucratic hiccup, but had
gone ahead, expecting it would eventually be resolved. In the end, it
wasn't. They were stripped of the land, and the government claimed the
road they built as well. Several members were charged with federal
``timber trespass.'' America renamed the Karl Marx Tree after General
Sherman.

Image

The General Sherman Tree, 275 feet tall, 100 feet around.Credit...David
Benjamin Sherry for The New York Times

Historians now see evidence that the government's actions were
influenced by the Southern Pacific Railroad, which was moving to protect
its own interests in the area. That is, the Kaweah colonists spent four
years working as unpaid labor on a nightmarish infrastructure project to
improve land for the same exploitative governmental-industrial complex
from which they thought they were breaking free. They had tried to
resize themselves --- to create a smaller, separate and more perfect
world in which their lives and values could be bigger --- but the real
world was still all around them, and in it, they were still painfully,
negligibly small.

It's hard to diagram the Kaweah story as an allegory of any contemporary
ideology of good and evil, heroism and villainy. It gets confusing: The
federal government, partly at the behest of an underhanded corporation,
sabotaged a community of hardworking and benevolent utopians --- but
only to create something fundamentally idealistic and to protect an
irreplaceable ecological wonder from capitalistic loggers. And yet, the
loggers were the utopians. The capitalists were socialists! Which would
have been fine, except that the government had mistaken them for an
underhanded corporation.

Baffled, I called William Tweed, the retired Sequoia park ranger, who
has also written about the colony. ``You reach a stage in life where
what you most frequently see in history is irony,'' Tweed told me
sagely. ``Perhaps the lesson for 2017 is that ideology rarely explains
what happens.''

\textbf{It was almost} dusk on the first evening by the time I rented my
Fiat at the San Jose airport and reached the entrance to Sequoia
National Park. I pulled into the tiny outpost of Three Rivers, Calif.,
and headed straight to a place called the Totem Market to rent a set of
tire chains, still toying with the idea of pushing up the mountain that
night.

The market is a combination gift shop, bar, deli and full-service
tire-chain-rental depot --- a sleepy-seeming establishment with wagon
wheels and barrels on its roof. But inside, the scene was incongruously
lively. A couple dozen mostly younger people stood around the bar,
shouting conversation over that song that goes ``Amber is the color of
my energy'' again and again. It felt like a rehearsal dinner; I couldn't
figure out how everyone knew one another. Then a woman in full Park
Service garb --- green wool pants, khaki shirt, government-issue leather
boots --- stepped out of my peripheral vision to order a beer.

Image

The canopy in Sequoia National Park, where the branches are large enough
to be impressive trees in their own rightCredit...David Benjamin Sherry
for The New York Times

Almost all of them were ``parkies,'' as one man eventually put it. They
were giving a going-away party for one of their supervisors, who was
leaving for a new detail at a park near San Diego. Someone pointed him
out: an older, smiley, muscular man in a T-shirt that said, ``Yard
Sale.'' They eventually sang ``Happy Birthday'' to someone, too --- a
younger guy in a camouflage hat, holding a generous glass of red wine
lazily aloft and squinting. At one point, another man dropped a pint,
and it shattered. The entire room shouted and applauded. Then Yard Sale
graciously, dutifully appeared with a broom and --- maybe, I wanted to
imagine, just to leave his troops with one final image of how a true
leader behaved --- swept up the glass.

Off in a corner, I struck up a conversation with Thor Riksheim, a
tree-size Park Service veteran with an impressive mustache. Riksheim
directs historical preservation at Sequoia. He had recently restored the
only Kaweah Colony building remaining in the park, a remote cabin that
the government calls, a little ruthlessly, Squatter's Cabin. The colony
had been conspicuously written out of the official story of Sequoia
National Park, and its road has long since reverted to a trail. But
Riksheim spoke affectionately of the cabin, which he called
``Squatty's,'' and the colonists, too. (He also called the General
Sherman Tree ``Sherm,'' as if they'd gone to high school together.)
Right away, I liked him immensely. It was clear his connection to the
trees was deep and singular. He was currently living in another historic
building he had restored in the heart of the Giant Forest --- in the
shadow of the famed Sentinel Tree, a cluster known as the Bachelor and
the Three Graces and other sequoias. It was touching how privileged he
seemed to feel, how proud. ``I'm Giant Forest, population 1,'' he told
me.

To a human being, a 2,000-year-old sequoia seems immortal. But I noticed
that people like Riksheim who have lived closely with the trees aren't
prone to mistaking their longevity and resilience for indestructibility.
To know sequoias means being cognizant of their weaknesses,
understanding them as provisional objects in some vaster, slower-moving
natural flux. In fact, there's a prominent exhibit at the park's Giant
Forest Museum chronicling how the government nearly undid the trees'
entire ecosystem through misunderstandings and mismanagement. By the
1930s, the Park Service had constructed a small resort town for tourists
in the center of Giant Forest. There were restaurants, cabins, a gas
station, a hotel and a grocery store --- nearly 300 buildings, erected
over the sensitive and shallow root systems of the sequoias, which never
reach more than about six feet below the surface. The Park Service
vigilantly fought back the beginnings of forest fires; this seemed wise,
fire being a reckless and destructive force, but it actually kept the
sequoias from reproducing. (It was not yet understood that, among other
ecological benefits, heat from wildfires opens the trees' cones and
allows them to spread their seeds.)

All of this was gradually corrected. Then droughts started intensifying.
The climate was shifting. The Park Service is now contemplating
``assisted migration'' of the sequoias: manually planting them farther
north to keep pace with climate change. But of course, Tweed, told me,
it's now conceivable that the Trump administration might not allow
climate change even to be mentioned at national parks' visitor centers.
Or that the administration, which picked a Twitter fight with the
National Park Service on Day 1, might decide to privatize management of
those lands. Who knows, Tweed said: ``The worries are deep and
profound.''

That is, there is another time scale on which the trees are vulnerable,
on which the trees are small and come and go as we do: sprouting,
growing up, suffering through storms, receiving scars, losing limbs,
before they finally drop. Every so often, the imperceptible turbulence
and instability in which the trees exist does upend them. Apparently,
the first thing you hear when one is falling is a blistering and
percussive crackle --- the roots snapping, one at a time, underground.
It may be far less likely, at any given moment, that one of the sequoias
in the park will keel over than that one of the tourists will, but it
could happen. It must happen, every now and again. Earlier this year, a
famous sequoia with a road tunneled through its base, known as the
Pioneer Cabin Tree, farther north, near Sacramento, toppled over in a
storm. At the Giant Forest Museum, I saw photos of another one that
flattened a parked Jeep in August 2003.

I don't know why, but I could not stop thinking about this while
trundling around the park that weekend: I kept privately picturing them
cracking and crashing down. It was a tremendously upsetting image, but
still never felt possible enough to scare me.

Late one afternoon, I lay down in the snow at the base of one for a
while, watching as the fog poured in through its crown, and I remembered
how untroubled Riksheim sounded at the bar the previous evening when,
lowering his voice, he mentioned that there was a particular sequoia
near his house that he was keeping an eye on. He could wake up dead
tomorrow, he said. ``It's just that flying, fickle finger of Fate. Every
once in a while, it's going to point at you.'' Then he fluttered his
long, bony index finger through the air and lowered it with a sudden
whoosh. Out of nowhere: crash. And I realized that his experience of it
--- a feeling of forsakenness, of arbitrary cruelty --- would be
essentially the same as the tree's.

Two days later, I was snowshoeing around alone when I discovered I was
standing in front of the same sequoia I had lain under. There, in the
sloping snow at its roots, I saw my imprint. My back and legs and arms
were joined into a wispy column, with the perfectly ovular hood of my
parka rounding off the top. It looked like a snow angel, but also like a
mummy --- an image of both levity and dolefulness, neither all good nor
all bad. I took a picture of it: what little of myself was left after
I'd gone. The figure looked smaller and more delicate than I thought it
should, but the Giant Forest was so quiet that I couldn't imagine who
else it could be.

Advertisement

\protect\hyperlink{after-bottom}{Continue reading the main story}

\hypertarget{site-index}{%
\subsection{Site Index}\label{site-index}}

\hypertarget{site-information-navigation}{%
\subsection{Site Information
Navigation}\label{site-information-navigation}}

\begin{itemize}
\tightlist
\item
  \href{https://help.nytimes3xbfgragh.onion/hc/en-us/articles/115014792127-Copyright-notice}{©~2020~The
  New York Times Company}
\end{itemize}

\begin{itemize}
\tightlist
\item
  \href{https://www.nytco.com/}{NYTCo}
\item
  \href{https://help.nytimes3xbfgragh.onion/hc/en-us/articles/115015385887-Contact-Us}{Contact
  Us}
\item
  \href{https://www.nytco.com/careers/}{Work with us}
\item
  \href{https://nytmediakit.com/}{Advertise}
\item
  \href{http://www.tbrandstudio.com/}{T Brand Studio}
\item
  \href{https://www.nytimes3xbfgragh.onion/privacy/cookie-policy\#how-do-i-manage-trackers}{Your
  Ad Choices}
\item
  \href{https://www.nytimes3xbfgragh.onion/privacy}{Privacy}
\item
  \href{https://help.nytimes3xbfgragh.onion/hc/en-us/articles/115014893428-Terms-of-service}{Terms
  of Service}
\item
  \href{https://help.nytimes3xbfgragh.onion/hc/en-us/articles/115014893968-Terms-of-sale}{Terms
  of Sale}
\item
  \href{https://spiderbites.nytimes3xbfgragh.onion}{Site Map}
\item
  \href{https://help.nytimes3xbfgragh.onion/hc/en-us}{Help}
\item
  \href{https://www.nytimes3xbfgragh.onion/subscription?campaignId=37WXW}{Subscriptions}
\end{itemize}
