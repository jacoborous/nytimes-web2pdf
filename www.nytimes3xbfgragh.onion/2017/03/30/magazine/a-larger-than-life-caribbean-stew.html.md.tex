Sections

SEARCH

\protect\hyperlink{site-content}{Skip to
content}\protect\hyperlink{site-index}{Skip to site index}

\href{https://myaccount.nytimes3xbfgragh.onion/auth/login?response_type=cookie\&client_id=vi}{}

\href{https://www.nytimes3xbfgragh.onion/section/todayspaper}{Today's
Paper}

A Larger-Than-Life Caribbean Stew

\url{https://nyti.ms/2oBioAa}

\begin{itemize}
\item
\item
\item
\item
\item
\item
\end{itemize}

Advertisement

\protect\hyperlink{after-top}{Continue reading the main story}

Supported by

\protect\hyperlink{after-sponsor}{Continue reading the main story}

\href{/column/magazine-eat}{Eat}

\hypertarget{a-larger-than-life-caribbean-stew}{%
\section{A Larger-Than-Life Caribbean
Stew}\label{a-larger-than-life-caribbean-stew}}

\includegraphics{https://static01.graylady3jvrrxbe.onion/images/2017/04/02/magazine/02eat/02eat-articleInline.jpg?quality=75\&auto=webp\&disable=upscale}

By \href{http://www.nytimes3xbfgragh.onion/by/sam-sifton}{Sam Sifton}

\begin{itemize}
\item
  March 30, 2017
\item
  \begin{itemize}
  \item
  \item
  \item
  \item
  \item
  \item
  \end{itemize}
\end{itemize}

I was staying in a square cinder-block house on a dirt road outside the
Lovely Bay settlement on Acklins Island in the southern Bahamas. This
was a number of years ago, a fishing trip, long days spent out on the
flats of the Bight of Acklins with a local guide named Fidel Johnson.
Each night we returned to eat fiery stews prepared by his sister,
Doramae Johnson. They were bright crimson affairs, perhaps familiar to
those who have spent time eating in the islands of the Bahamas and
Lesser Antilles. Doramae ginned them up from canned goods and fresh
conchs, lobster, sometimes chicken or pork shipped in from Nassau, and
served them on a small table in her brother's kitchen. They were
bewilderingly delicious. I begged her for a recipe. She only smiled. I
tried to come off the water early to cook with her. She was onto me. The
stew was finished, left warm in the oven, and she was gone, her secrets
kept.

The writer Jim Harrison, who died last year at 78, did not to my
knowledge ever eat or cook with Doramae. In 1981, when he published a
recipe for what he called Caribbean stew in the literary magazine Smoke
Signals, she was only a child. But when I came across that recipe in
Harrison's new posthumous collection of food writing, ``A Really Big
Lunch,'' published last month, it sent me directly back to that kitchen
on Acklins, to the aromas and flavors of the place, and to the deep
satisfaction I had eating there, night after night.

It was not easy cooking, at first, to make it right. In Smoke Signals,
Harrison shared pages with Terry Southern, Charles Bukowski and Patti
Smith. His job was to be ribald and wild. He'd get to the recipe
eventually. In his essay ``Eat Your Heart Out,'' he first outlined a
fantasy about Meryl Streep (``Then I slipped on my fifty-dollar Key West
pig mask and stalked her pealing laughter through the penthouse etc.'')
and delivered the address, in Ann Arbor, Mich., of a woman who made a
hot sauce he admired (Clancy's Fancy, still around).

And, a point of order: ``No one is allowed to use cocaine before the
meal when I cook,'' he wrote. ``Afterward, O.K. Cocaine creates a sort
of bubblegum nimbus that slaughters the palate and sensuous capacities,
in addition to shrinking the wee-wee and tearing holes in the social
fabric.'' As for the recipe, he wrote: ``Do not change or substitute!
Above my desk hang a crow wing and a pink rubber piglet with a green
drake trout fly stuck in its ass, and a coyote tooth in its mouth. I've
written a new novel called `Warlock.' You tamper with my recipes at your
peril.''

I had to, though! The recipe didn't really work. Ingredients were
missing. (``Add onions.'' What onions?) Or they were crazily specific
--- not just that Clancy's Fancy, but tablespoons of basil vinegar from
a particular shop in Paris. (Harrison liked his specialty groceries.)
Fiddling was needed. Adjustments were made.

I believe that Harrison would be all right with that. ``Cooking is in
the details and is not for those who think they must spend all of their
time thinking large,'' he wrote in a 2011 essay for Playboy, which is
also in the collection. ``This morning I burned my Jimmy Dean hot-pepper
sausage patty because I was on the phone speaking with a friend about
another friend's cancer. Yesterday morning I ruined a quesadilla by
adding too much salsa because I was busy revising a poem.''

So yes, I've made some revisions myself. What hasn't changed, though, is
the sheer exuberance of the resulting stew, and the base-line excellence
of the flavors within it --- onions and garlic sautéed in chicken and
pork fat, then caramelized with a full can of tomato paste and the
scorching sweetness of hot-pepper sauce, the whole lot mixed with lemon
juice and Worcestershire, chile powder and paprika, then poured over the
meats and baked covered in the oven until it is fragrant beyond measure
and easily spooned over rice.

The meats, plural. In the islands you might only have conch, or a tray
of frozen chicken purchased in town at an exorbitant price. For
Harrison, though, you'll buy out the butcher for ribs, Italian sausages,
chicken thighs. These offer a delightful variety of textures and flavors
in the pot.

Also, it must be said, more than a little fat, for Harrison cooked as he
lived, largely and with little heed for consequences. Spoon off the
excess when you're done cooking, he advised, ``or suck it off with a
straw.'' Vintage Harrison: Eat your heart out.

\textbf{Recipe:}
\href{https://cooking.nytimes3xbfgragh.onion/recipes/1018697-jim-harrisons-caribbean-stew}{Jim
Harrison's Caribbean Stew}

Advertisement

\protect\hyperlink{after-bottom}{Continue reading the main story}

\hypertarget{site-index}{%
\subsection{Site Index}\label{site-index}}

\hypertarget{site-information-navigation}{%
\subsection{Site Information
Navigation}\label{site-information-navigation}}

\begin{itemize}
\tightlist
\item
  \href{https://help.nytimes3xbfgragh.onion/hc/en-us/articles/115014792127-Copyright-notice}{©~2020~The
  New York Times Company}
\end{itemize}

\begin{itemize}
\tightlist
\item
  \href{https://www.nytco.com/}{NYTCo}
\item
  \href{https://help.nytimes3xbfgragh.onion/hc/en-us/articles/115015385887-Contact-Us}{Contact
  Us}
\item
  \href{https://www.nytco.com/careers/}{Work with us}
\item
  \href{https://nytmediakit.com/}{Advertise}
\item
  \href{http://www.tbrandstudio.com/}{T Brand Studio}
\item
  \href{https://www.nytimes3xbfgragh.onion/privacy/cookie-policy\#how-do-i-manage-trackers}{Your
  Ad Choices}
\item
  \href{https://www.nytimes3xbfgragh.onion/privacy}{Privacy}
\item
  \href{https://help.nytimes3xbfgragh.onion/hc/en-us/articles/115014893428-Terms-of-service}{Terms
  of Service}
\item
  \href{https://help.nytimes3xbfgragh.onion/hc/en-us/articles/115014893968-Terms-of-sale}{Terms
  of Sale}
\item
  \href{https://spiderbites.nytimes3xbfgragh.onion}{Site Map}
\item
  \href{https://help.nytimes3xbfgragh.onion/hc/en-us}{Help}
\item
  \href{https://www.nytimes3xbfgragh.onion/subscription?campaignId=37WXW}{Subscriptions}
\end{itemize}
