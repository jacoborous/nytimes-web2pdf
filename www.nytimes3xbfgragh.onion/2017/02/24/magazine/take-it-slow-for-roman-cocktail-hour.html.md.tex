Sections

SEARCH

\protect\hyperlink{site-content}{Skip to
content}\protect\hyperlink{site-index}{Skip to site index}

\href{https://myaccount.nytimes3xbfgragh.onion/auth/login?response_type=cookie\&client_id=vi}{}

\href{https://www.nytimes3xbfgragh.onion/section/todayspaper}{Today's
Paper}

Take It Slow for Roman Cocktail Hour

\url{https://nyti.ms/2lCs4tX}

\begin{itemize}
\item
\item
\item
\item
\item
\item
\end{itemize}

Advertisement

\protect\hyperlink{after-top}{Continue reading the main story}

Supported by

\protect\hyperlink{after-sponsor}{Continue reading the main story}

\href{/column/magazine-drink}{Drink}

\hypertarget{take-it-slow-for-roman-cocktail-hour}{%
\section{Take It Slow for Roman Cocktail
Hour}\label{take-it-slow-for-roman-cocktail-hour}}

\includegraphics{https://static01.graylady3jvrrxbe.onion/images/2017/02/26/magazine/26drink/26mag-26drink-t_CA0-articleInline.jpg?quality=75\&auto=webp\&disable=upscale}

By
\href{http://query.nytimes3xbfgragh.onion/search/sitesearch?query=Schaap\%2C+Rosie\&more=date_all}{Rosie
Schaap}

\begin{itemize}
\item
  Feb. 24, 2017
\item
  \begin{itemize}
  \item
  \item
  \item
  \item
  \item
  \item
  \end{itemize}
\end{itemize}

Last month, as I sat in an elegant hotel bar in Rome, I caught myself
wondering if there can be such a thing as too much magnificence in a
single place. With its antiquity (``Just go to the Colosseum, face east
and shout, `Moira,' '' my friend joked when I asked how to get to her
house), its art and its food, Rome nearly overwhelmed me with its
aesthetic and alimentary too-much-ness.

\emph{Aperitivo} time is the perfect antidote to all that. During that
blessed and blissful hour (or two, or three) of easygoing snacking and
drinking and talking, time slows, and stress slips away. In Rome, I met
the local author --- and aperitivo enthusiast --- Elizabeth Minchilli,
whose most recent book is ``Eating Rome.'' ``I love aperitivo time in
Rome because it can be anything you want,'' she told me. ``It's that
in-between time of day --- not work, not dinner --- when you can meet as
few or as many friends for as long or as short as you like and drink as
much or as little as you like. Many other `meals' in Italy have so many
rules. But with aperitivo, it's a social opportunity that you can put
your own spin on.''

Yes, you may drink as much or as little as you wish, but during
\emph{aperitivo} time, no one seems to drink too much. Drunkenness is
not the objective, and it helps that many of the favored drinks are
relatively low in alcohol: a glass of wine, a spritz of some kind. In
the latter category, Moira's husband, Damiano Abeni, introduced me to
the Pirlo, a classic from his native Brescia in northern Italy. The
recipe is as simple as it gets: Campari and a dry, sparkling white wine
--- ideally Pignoletto frizzante.

\includegraphics{https://static01.graylady3jvrrxbe.onion/images/2017/02/26/magazine/26drink2/26drink2-articleInline.jpg?quality=75\&auto=webp\&disable=upscale}

Among \emph{aperitivi}, there's a place for stronger stuff, too, in the
form of classic Italian cocktails --- just sipped a bit more slowly.
That elegant hotel bar I found myself in was the Stravinskij at the
Hotel de Russie, where, with exceptional grace and premium ingredients,
a bartender made me an exemplary Negroni that cured me of my Negroni
fatigue and reminded me of the drink's charms. I also like the Negroni
variant called the Cyn-Cyn, which calls for Cynar instead of Campari,
and is a favorite of Elizabeth's.

And the effects of drink are always mitigated by the presence of food:
Good, salty cheese, cured meats, bread and olives frequently appear, but
I had pizza, and even sashimi, at \emph{aperitivo} gatherings. The
ritual is equally unfussy at home in a small group before dinner or in a
crowd at a bar. The snacks at the Stravinskij were more rarefied ---
salmon and other seafood --- than the usual \emph{aperitivo} offerings,
but the relaxed, leisurely pace remained.

\emph{Aperitivo} time is among the most civilized drinking traditions
I've ever witnessed. There is no pressure, no pretension: It's all
unhurried, unforced pleasure. ``Piano, piano,'' they say in Rome.
Slowly, gently.

\textbf{Recipes:}
\href{https://cooking.nytimes3xbfgragh.onion/recipes/1018598-cyn-cyn}{Cyn-Cyn}
\textbar{}
\href{https://cooking.nytimes3xbfgragh.onion/recipes/1018597-pirlo}{Pirlo}

Advertisement

\protect\hyperlink{after-bottom}{Continue reading the main story}

\hypertarget{site-index}{%
\subsection{Site Index}\label{site-index}}

\hypertarget{site-information-navigation}{%
\subsection{Site Information
Navigation}\label{site-information-navigation}}

\begin{itemize}
\tightlist
\item
  \href{https://help.nytimes3xbfgragh.onion/hc/en-us/articles/115014792127-Copyright-notice}{©~2020~The
  New York Times Company}
\end{itemize}

\begin{itemize}
\tightlist
\item
  \href{https://www.nytco.com/}{NYTCo}
\item
  \href{https://help.nytimes3xbfgragh.onion/hc/en-us/articles/115015385887-Contact-Us}{Contact
  Us}
\item
  \href{https://www.nytco.com/careers/}{Work with us}
\item
  \href{https://nytmediakit.com/}{Advertise}
\item
  \href{http://www.tbrandstudio.com/}{T Brand Studio}
\item
  \href{https://www.nytimes3xbfgragh.onion/privacy/cookie-policy\#how-do-i-manage-trackers}{Your
  Ad Choices}
\item
  \href{https://www.nytimes3xbfgragh.onion/privacy}{Privacy}
\item
  \href{https://help.nytimes3xbfgragh.onion/hc/en-us/articles/115014893428-Terms-of-service}{Terms
  of Service}
\item
  \href{https://help.nytimes3xbfgragh.onion/hc/en-us/articles/115014893968-Terms-of-sale}{Terms
  of Sale}
\item
  \href{https://spiderbites.nytimes3xbfgragh.onion}{Site Map}
\item
  \href{https://help.nytimes3xbfgragh.onion/hc/en-us}{Help}
\item
  \href{https://www.nytimes3xbfgragh.onion/subscription?campaignId=37WXW}{Subscriptions}
\end{itemize}
