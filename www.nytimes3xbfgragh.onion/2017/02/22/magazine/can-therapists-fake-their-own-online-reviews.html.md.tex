Sections

SEARCH

\protect\hyperlink{site-content}{Skip to
content}\protect\hyperlink{site-index}{Skip to site index}

\href{https://myaccount.nytimes3xbfgragh.onion/auth/login?response_type=cookie\&client_id=vi}{}

\href{https://www.nytimes3xbfgragh.onion/section/todayspaper}{Today's
Paper}

Can Therapists Fake Their Own Online Reviews?

\url{https://nyti.ms/2luioBz}

\begin{itemize}
\item
\item
\item
\item
\item
\item
\end{itemize}

Advertisement

\protect\hyperlink{after-top}{Continue reading the main story}

Supported by

\protect\hyperlink{after-sponsor}{Continue reading the main story}

\href{/column/the-ethicist}{The Ethicist}

\hypertarget{can-therapists-fake-their-own-online-reviews}{%
\section{Can Therapists Fake Their Own Online
Reviews?}\label{can-therapists-fake-their-own-online-reviews}}

By Kwame Anthony Appiah

\begin{itemize}
\item
  Feb. 22, 2017
\item
  \begin{itemize}
  \item
  \item
  \item
  \item
  \item
  \item
  \end{itemize}
\end{itemize}

\includegraphics{https://static01.graylady3jvrrxbe.onion/images/2017/02/26/magazine/26ethicist/26ethicist-articleInline.jpg?quality=75\&auto=webp\&disable=upscale}

\emph{I am caught between competing ethical requirements. I am a
mental-health professional with substantial experience; some people call
me an expert. I recently expanded my private-practice hours. To increase
patient numbers, I joined an online referral service (at considerable
expense). Here's where the ethical dilemma comes in: Like most
``locator'' sites, the service includes ``customer'' ratings. The site
reps instructed me to have current patients complete the ratings. My
professional training (I'm a psychotherapist) made it very clear that it
is a big no-no to solicit testimonials from patients; doing so can badly
interfere with the treatment relationship. Patients are in treatment for
their own needs and should not be required to consider the therapist's
other actual or prospective patients. (Psychotherapy isn't a simple
commercial transaction.) So my professional-ethics training tells me to
leave any ratings up to any patients who find me through the site, which
then asks them to rate me. This would mean virtually no traffic through
the site.}

\emph{Discussions with fellow clinicians have revealed that many if not
most have ``primed the pump'' with favorable ``reviews,'' written by
friends or family members or by the therapists themselves. This thought
makes me very queasy! But it seems to be a necessary action in the
online marketplace. Basic ethics say not to lie, especially
self-servingly. Still, I'm wondering about the ethics of depriving
potential patients of the ability to find me (by remaining essentially
invisible on the site) and to see if I might be able to help them.}

\emph{So what do you think of this solution? I have submitted a few
ratings to the site, directly quoting my actual, satisfied patients but
using made-up names. My thinking is that the patients' spontaneous
comments about our work are real, but I haven't made an improper demand
of anyone. Furthermore, because I genuinely believe I may be able to
help a potential patient who might read the reviews, fudging their
origins doesn't seem like too bad a con. I think the worst harm my
actions might cause is that someone meets with me once and determines
that I'm not the right therapist for him or her; that's actually fine
and a pretty common event. And readers know better than to take customer
reviews too seriously, right?}

\emph{Please tell me my approach isn't too grievous a wrong; otherwise,
I'm out a big chunk of change on the site fee.} Name Withheld

\textbf{You speak of} competing ethical requirements. I understand what
one of them is: honesty. What I don't get is what the countervailing
ethical requirement is supposed to be. The only candidate you offer is a
supposed ethical duty to make your powers as a healer known to people
who need them. If there were such a duty, talented psychotherapists
would mostly be violating it. So what you have, on the one side, is a
wrong; on the other side, a bunch of excuses.

This is a common form of dishonesty, you point out. ``But everybody does
it'' is an excuse we learn in grade school. Parents can reply, with the
Bible: ``Thou shalt not follow a multitude to do evil.'' (That's Exodus
23:2. Exodus 23:1 begins, ``Thou shalt not raise a false report.'') Now,
``evil'' seems a bit tough here, because what you're doing is less
harmful than it might be. Many people discount these customer ratings,
because they are aware that these reports, like yours, are often fakes.
They indeed ``know better than to take customer reviews too seriously.''
But then your reports are either going to have little effect or they'll
selectively persuade the ignorant and the credulous. Taking advantage of
people with these epistemic weaknesses is exploiting the vulnerable.

You maintain that your form of fakery is better than the straight-out
inventions of others, because your ratings are based on things that
clients have actually said. But because these are not real reports,
readers are not getting a reflection of the real views of your
clientele: What if a fair sampling would include some critics? You
suggest that it's a ``fairly common event'' for people to decide that
you're not the right therapist for them. Bothering to rate someone
positively is a sign of satisfaction; it's conceivable that the fact
your clients haven't done so is itself evidence of something. I'm
putting aside the issue of whether metrics of consumer enthusiasm are
entirely appropriate in the realm of psychotherapy. (Imagine Dora on
Sigmund Freud: ``Worst. Analysis. Ever.'')

That you are embedded in this ethical morass is not, of course, your
fault. It sounds as if the people who created the website you signed up
for have invented a permanent temptation to dishonesty and done little
to obviate it. (A ``closed-loop'' system --- which aims to restrict
comments to registered, verified patients who have seen the
practitioners --- is harder to game in the way you describe.) The web,
like every technology, creates new opportunities both for doing wrong
and for doing right. Print made possible the wide circulation of lies as
well as of truths; so, too, did the telegraph, the radio and television.
Indeed, language itself is like this: no lies, no truths. There are
three mechanisms for counteracting falsehoods: exposure, the education
of consumers and the conscience of the producers. The last of these, as
your letter suggests, isn't to be relied upon. Your one consolation, and
ours, is that your dishonesty is a mere grain of sand on the great
mountain of falsehood. Still, you should take these fake ratings down.
If you want to replace them, why not write, under your own name, a
paragraph summarizing the comments of satisfied patients?

\emph{This past week, my primary-care physician called me with some
startling news: iron-deficiency anemia. She was so concerned with my
results that she ordered a colonoscopy and upper endoscopy to look for
internal bleeding and recommended I take ferrous gluconate to increase
my iron levels.}

\emph{I have no history of iron deficiency or anemia. The more I thought
about it, the more I thought of a possible cause. I have been donating
blood on a regular basis for the last several years at a local
bloodmobile. After the first few times, I was turned away because my
iron level was found to be too low. Next blood drive, no problem. The
latest was another story. The staff nurse pricked my finger and told me
my iron was too low but then said something along these lines, ``Oh, let
me get so and so, she can always get the proper reading.'' Just like
that, my iron level was high enough to donate, which I did. When I asked
how that could be, she said, ``She knows how to get the proper reading,
she has to poke a little deeper.'' Hmm.}

\emph{My doctor now thinks that donating blood could be the reason for
my iron-deficient anemia. She was shocked to learn that the staff in the
bloodmobile neglected to suggest I contact my doctor and blatantly
manipulated the results to make me eligible to donate blood.}

\emph{Is it my responsibility to alert the teaching hospital that
operates these blood drives? I feel horrible that someone has possibly
been given my iron-deficient blood.} Maura Toomey, Brookline, Mass.

\textbf{The helping professions} may themselves be in need of help: That
seems to be the lesson of the day. It looks as if you have important
information about the way some blood donations are conducted in your
area. What the staff nurse said suggests that what happened to you may
have happened to others. A large-scale 2011 study found iron deficiency
in a large portion of regular donors --- about two-thirds of the women
and half of the men --- and those were just people whose donations had
been accepted. As your doctor is aware, regular donation can result in
(and worsen) iron deficiency and anemia. And of course, there are good
recipient-side reasons iron-deficient blood, which doesn't carry oxygen
very well, should be avoided. (Anemia can also be a symptom of
transmissible diseases.) So for the sake of both donors and recipients,
it's a bad idea to ignore signs of anemia in those who donate at blood
drives. You should indeed notify the hospital that runs the bloodmobile.
It may be too late to stop your blood from being used, because it's not
going to be stored for more than six weeks. But sharing your experience
with the relevant officials could help prevent this abuse of the proper
protocols from continuing.

Advertisement

\protect\hyperlink{after-bottom}{Continue reading the main story}

\hypertarget{site-index}{%
\subsection{Site Index}\label{site-index}}

\hypertarget{site-information-navigation}{%
\subsection{Site Information
Navigation}\label{site-information-navigation}}

\begin{itemize}
\tightlist
\item
  \href{https://help.nytimes3xbfgragh.onion/hc/en-us/articles/115014792127-Copyright-notice}{©~2020~The
  New York Times Company}
\end{itemize}

\begin{itemize}
\tightlist
\item
  \href{https://www.nytco.com/}{NYTCo}
\item
  \href{https://help.nytimes3xbfgragh.onion/hc/en-us/articles/115015385887-Contact-Us}{Contact
  Us}
\item
  \href{https://www.nytco.com/careers/}{Work with us}
\item
  \href{https://nytmediakit.com/}{Advertise}
\item
  \href{http://www.tbrandstudio.com/}{T Brand Studio}
\item
  \href{https://www.nytimes3xbfgragh.onion/privacy/cookie-policy\#how-do-i-manage-trackers}{Your
  Ad Choices}
\item
  \href{https://www.nytimes3xbfgragh.onion/privacy}{Privacy}
\item
  \href{https://help.nytimes3xbfgragh.onion/hc/en-us/articles/115014893428-Terms-of-service}{Terms
  of Service}
\item
  \href{https://help.nytimes3xbfgragh.onion/hc/en-us/articles/115014893968-Terms-of-sale}{Terms
  of Sale}
\item
  \href{https://spiderbites.nytimes3xbfgragh.onion}{Site Map}
\item
  \href{https://help.nytimes3xbfgragh.onion/hc/en-us}{Help}
\item
  \href{https://www.nytimes3xbfgragh.onion/subscription?campaignId=37WXW}{Subscriptions}
\end{itemize}
