Sections

SEARCH

\protect\hyperlink{site-content}{Skip to
content}\protect\hyperlink{site-index}{Skip to site index}

\href{https://www.nytimes3xbfgragh.onion/section/politics}{Politics}

\href{https://myaccount.nytimes3xbfgragh.onion/auth/login?response_type=cookie\&client_id=vi}{}

\href{https://www.nytimes3xbfgragh.onion/section/todayspaper}{Today's
Paper}

\href{/section/politics}{Politics}\textbar{}Donald Trump and Lawmakers
Confer on Imports and Tax Code

\url{https://nyti.ms/2k6yx0K}

\begin{itemize}
\item
\item
\item
\item
\item
\end{itemize}

Advertisement

\protect\hyperlink{after-top}{Continue reading the main story}

Supported by

\protect\hyperlink{after-sponsor}{Continue reading the main story}

\hypertarget{donald-trump-and-lawmakers-confer-on-imports-and-tax-code}{%
\section{Donald Trump and Lawmakers Confer on Imports and Tax
Code}\label{donald-trump-and-lawmakers-confer-on-imports-and-tax-code}}

By \href{https://www.nytimes3xbfgragh.onion/by/alan-rappeport}{Alan
Rappeport}

\begin{itemize}
\item
  Feb. 2, 2017
\item
  \begin{itemize}
  \item
  \item
  \item
  \item
  \item
  \end{itemize}
\end{itemize}

WASHINGTON --- President Trump, at loggerheads with congressional
Republicans over the best way to overhaul the tax code, may have come
toward Capitol Hill on a key sticking point, the way imports should be
taxed, after a meeting at the White House on Thursday, said
Representative Kevin Brady, chairman of the House Ways and Means
Committee.

Congressional Republicans are trying to nudge Mr. Trump away from one of
his big campaign promises: a 35 percent tariff on any product made by an
American company at an overseas factory, then imported into the United
States market. Success in getting the president off that applause line
will be critical to getting tax legislation passed in 2017 and
fulfilling a central promise that Republicans have been making to voters
for years.

At the same time, both sides want to end what they see as a
\href{http://www.nationalreview.com/article/444240/donald-trump-trade-border-adjustment-best-way-level-playing-field}{penalty
on American exporters}, who compete with overseas factories that can
game the system for lower tax rates. ``Like us, they're just not
satisfied with a tax code today that favors foreign products over U.S.
products, and I think they are looking carefully at ending that `made in
America' export tax so that we can compete around the world,'' Mr.
Brady, Republican of Texas, said of the Trump administration.

At the White House meeting, Mr. Brady was joined by Senator Orrin Hatch
of Utah, chairman of the Finance Committee, Senator Ron Wyden of Oregon,
the ranking Democrat, and Representative Richard Neal of Massachusetts,
the ranking Democrat on Ways and Means. The conversation with Mr. Trump
also focused on trade and renegotiating the North American Free Trade
Agreement.

Mr. Brady's plan would also essentially levy a 20 percent tax on all
imports, to create revenue and spur domestic production and exports ---
a notion Mr. Trump has said is overly complicated. The prospect of such
a sweeping change has stirred fears among big companies in the retail
and energy sectors, which rely heavily on imports, and created a
backlash that could scuttle plans to slash tax rates without adding to
the deficit.

The split within corporate America over the shape of a tax code rewrite
was on display Thursday as Freedom Partners, an advocacy group funded by
the libertarian billionaires Charles and David Koch, called on
Republicans to ``scrap'' the border tax idea, which it warned would mean
higher prices for consumers. Also Thursday, a new business group, the
American Made Coalition, began to rally support around the tax plan
championed by Mr. Brady and the House speaker, Paul D. Ryan of
Wisconsin.

Mr. Brady said he was confident that the White House and worried
companies were starting to come around. ``The more people learn about
why we need to tax equally within the U.S., at this new low business
rate of 20 percent, they come on board in gangbusters,'' he said in an
interview.

Mr. Brady is still engaged in negotiations and salesmanship. He said on
Thursday that although there should be no exemptions for importers of
any kind, Republicans were open to working with businesses that rely
heavily on imports to develop ways to ease new tax burdens. He also said
the United States did not intend to flout the rules of the World Trade
Organization.

House Republicans will also need to hash out their final tax plan with
Republicans in the Senate, and rumblings this week from Mr. Hatch
suggested that they might not be on the same page. Speaking at the
Chamber of Commerce on Wednesday, Mr. Hatch said he had concerns about
the border tax and that the Senate's legislation would almost certainly
end up looking different from what the House produces.

Mr. Trump and his staff have continued to publicly float provocative
ideas, such as taxing companies that move their manufacturing operations
abroad and sell products in the United States, and taxing products from
Mexico to pay for the construction of a wall along the border. ``The
president is clearly laying out a bunch of different options that he
could use, including those tariffs,'' Mr. Brady said, suggesting that
his plan would be more ``pro-growth.''

As for using revenues from the tax overhaul to pay for the wall, Mr.
Brady insisted that such projects are not his focus. ``Our focus right
now is all on growth within tax reform,'' Mr. Brady said. ``Discussions
on how new revenues are spent I'm sure will be part of future
discussions.''

Advertisement

\protect\hyperlink{after-bottom}{Continue reading the main story}

\hypertarget{site-index}{%
\subsection{Site Index}\label{site-index}}

\hypertarget{site-information-navigation}{%
\subsection{Site Information
Navigation}\label{site-information-navigation}}

\begin{itemize}
\tightlist
\item
  \href{https://help.nytimes3xbfgragh.onion/hc/en-us/articles/115014792127-Copyright-notice}{©~2020~The
  New York Times Company}
\end{itemize}

\begin{itemize}
\tightlist
\item
  \href{https://www.nytco.com/}{NYTCo}
\item
  \href{https://help.nytimes3xbfgragh.onion/hc/en-us/articles/115015385887-Contact-Us}{Contact
  Us}
\item
  \href{https://www.nytco.com/careers/}{Work with us}
\item
  \href{https://nytmediakit.com/}{Advertise}
\item
  \href{http://www.tbrandstudio.com/}{T Brand Studio}
\item
  \href{https://www.nytimes3xbfgragh.onion/privacy/cookie-policy\#how-do-i-manage-trackers}{Your
  Ad Choices}
\item
  \href{https://www.nytimes3xbfgragh.onion/privacy}{Privacy}
\item
  \href{https://help.nytimes3xbfgragh.onion/hc/en-us/articles/115014893428-Terms-of-service}{Terms
  of Service}
\item
  \href{https://help.nytimes3xbfgragh.onion/hc/en-us/articles/115014893968-Terms-of-sale}{Terms
  of Sale}
\item
  \href{https://spiderbites.nytimes3xbfgragh.onion}{Site Map}
\item
  \href{https://help.nytimes3xbfgragh.onion/hc/en-us}{Help}
\item
  \href{https://www.nytimes3xbfgragh.onion/subscription?campaignId=37WXW}{Subscriptions}
\end{itemize}
