The Retraining Paradox

\url{https://nyti.ms/2lyo5hC}

\begin{itemize}
\item
\item
\item
\item
\item
\item
\end{itemize}

\includegraphics{https://static01.graylady3jvrrxbe.onion/images/2017/02/26/magazine/26retraining/26retraining-articleLarge-v4.jpg?quality=75\&auto=webp\&disable=upscale}

Sections

\protect\hyperlink{site-content}{Skip to
content}\protect\hyperlink{site-index}{Skip to site index}

The Future of Work

\hypertarget{the-retraining-paradox}{%
\section{The Retraining Paradox}\label{the-retraining-paradox}}

Many Americans need jobs, or want better jobs, while employers have good
jobs they can't fill. Matching them up is the tricky part.

Credit...Illustration by Kristian Hammerstad

Supported by

\protect\hyperlink{after-sponsor}{Continue reading the main story}

By Ruth Graham

\begin{itemize}
\item
  Feb. 23, 2017
\item
  \begin{itemize}
  \item
  \item
  \item
  \item
  \item
  \item
  \end{itemize}
\end{itemize}

\textbf{When Nathan Kecy} graduated from Plymouth State University in
New Hampshire a decade ago with a bachelor's degree in communications,
he found himself with about \$10,000 in debt and few clear career
options. He first found work as a door-to-door salesman (``a pyramid
scheme,'' he recalls) and then in telemarketing. Finally he landed a job
as an infrastructure specialist for Datamatic, a Texas-based
water-meter-technology company. He was traveling across the country
installing meters, making a decent salary. But he lost his job after the
company restructured in 2012, he said, and soon he found that his skills
weren't easily transferable to a new field; Datamatic's technology was
proprietary, and his expertise in the company's installation program
wasn't appealing to employers outside that particular industry. He tried
going into business with a friend, but the relationship soured. By then
he had a baby and a fiancée, and he felt stuck.

Now 32, Kecy is a few months away from finishing a six-month certificate
program in advanced composites manufacturing at Great Bay Community
College in Rochester, N.H. The program operates out of a satellite
campus that opened in 2013, with aid from a Labor Department grant meant
to help community colleges reach ``trade displaced'' workers who need
help training for new careers. The unemployment rate in southern New
Hampshire is low, less than 3 percent. At one state job fair last
summer, just 350 people showed up for 1,200 available jobs. In Strafford
County, where Rochester is located, the largest employers include the
University of New Hampshire and Liberty Mutual, but also manufacturers
like Turbocam and Contitech. Kecy's classmates include veterans, recent
high-school graduates and older workers whose careers had reached dead
ends. All of them are looking for hope and a decent paycheck by
acquiring a new set of skills. ``Within six months, I'm going to go from
regular guy to working in the aerospace community,'' says Tommy
Florentino, a disabled veteran with a background in construction and
automotive manufacturing. He has friends who went to Boston College or
Suffolk University, ``and they're waiters and waitresses.''

The college's 27,000-square-foot Advanced Technology and Academic Center
is at the edge of a nondescript shopping center. The complex also houses
a Dollar Tree, a J.C. Penney and a Kmart, where a banner out front
reads, ``Now hiring.'' Cashiers there earn close to minimum wage. But
Kecy expects to earn at least \$16 an hour when he graduates and to move
up quickly from there. Composites is a broad field in manufacturing,
with applications including automotive parts, sporting goods and
prosthetics, as well as in the locally prominent aerospace industry. The
state's department of economic development bills its seacoast region as
``the emerging composites region,'' and it points to Great Bay's program
as a reason for more aerospace and defense businesses in particular to
relocate there. ``I've got some options, which is something I've never
really experienced before,'' Kecy says.

\textbf{There's a strange} disconnect between two of the big narratives
about the American blue-collar work force right now. In one story, there
is a population of unemployed and underemployed working-class adults for
whom well-paying work seems increasingly out of reach; their jobs have
gone overseas or become automated, and they find themselves working
retail, or not working at all. But an apparently conflicting story comes
from American employers, which have been insisting for years that they
have a hard time finding workers to fill many skilled blue-collar jobs.
A 2015 report from the Manufacturing Institute, for example, found that
seven in 10 manufacturing executives said they faced shortages of
workers with adequate tech skills. A high proportion of existing skilled
workers is also nearing retirement, which means a bigger gap is looming
soon. By 2025, the report warned, two million jobs will be going
unfilled. (Health care, also a big focus of retraining programs, is
another rapidly expanding field.)

The tantalizing promise of government-funded job training is that it can
bridge the gap between those narratives in a way that benefits
individual workers, employers and the country as a whole. Hard-working
Americans get good jobs, employers get skilled labor and the economy
benefits from their mutual good fortune. The image of that virtuous
cycle has made the promotion of training programs appealing for
politicians on the left and the right. Hillary Clinton proposed
retraining former coal-industry workers in new careers as part of a \$30
billion package meant ``to ensure that coal miners and their families
get the benefits they've earned and respect they deserve.'' Even as
Republicans have voted to cut funding for training in recent years, they
have paid it lip service as a way to put Americans back to work.

It's perhaps not surprising, though, that so much of the working class
gravitated in the last election to Donald Trump, whose rhetoric about
displaced workers was very different: blunt (if unrealistic) promises to
stop old careers from disappearing, to ``bring back our jobs.'' In its
zeal for retraining, the federal government's approach to the problem
has become increasingly byzantine, a dizzying constellation of programs
to help struggling workers prepare for new careers. Some of them are
intended for employees laid off en masse when their jobs went overseas,
and others are for those who are simply unemployed and underqualified
for well-paying work. In the 2009 fiscal year, the Government
Accountability Office counted 47 different federal training-related
programs administered by nine agencies, numbers Republicans have since
used to argue that many of the programs were redundant. In his 2012
State of the Union address, even President Obama criticized the ``maze
of confusing training programs'' unemployed workers had to navigate to
get help. The Workforce Innovation and Opportunity Act, signed into law
in 2014 with bipartisan support, was designed in part to streamline the
government's approach.

Critics also say that job training is costly and too often ineffective.
Take the primary federal effort specifically aimed at workers affected
by global trade, the Labor Department's Trade Adjustment Assistance
program. Through T.A.A., qualified workers can receive free retraining,
typically through a community-college program like Great Bay's. The
program is generous, spending more than \$11,500 on each person who
participated in retraining in the 2015 fiscal year. But it serves
relatively few people, and recent analysis has shown iffy results: A
2012 evaluation prepared for the Labor Department found that while 85
percent of those who went through T.A.A.-funded training eventually
received a certificate or degree, only 37 percent of them were working
in that field four years later. (The program was later amended to
include more individualized support.)

All too often, skeptics say, publicly funded training programs are a sop
to well-connected companies who want taxpayers to foot the bill to train
their workers. Critics also point at research suggesting that on-the-job
training by employers themselves has been declining in recent years. But
it simply doesn't make economic sense for most employers to do all of
their own training anymore. In part, this is because of technology: Jobs
in advanced manufacturing and health care require intense technological
instruction, usually accompanied by classroom time. At the same time,
standardization means employers often poach skilled workers from one
another, which discourages them from investing a lot of time and money
in training their own workers. ``It's unrealistic today to think of
traditional, very idiosyncratic manufacturing jobs where you're going to
walk in, get a job, get trained in a bunch of very specific skills, and
they'll hold onto you for decades,'' says Lawrence Katz, an economist at
Harvard University. ``That's just not the trajectory of employment
anymore.''

\textbf{After completing the} certificate program in April, Kecy will
have specializations in ``nondestructive testing'' and ``bonding and
finishing,'' skills that set him up for specific positions that local
employers have been struggling to fill. The simplest description of
composites manufacturing is that it is the process of putting two
materials together; adobe, for example, is a composite of straw and mud.
``Advanced'' composites manufacturing typically involves adding
high-tech resin to woven fibers. The strong, lightweight finished
products are replacing metal in many manufacturing areas, including
aerospace. Great Bay students further specialize in areas like quality
inspection or resin-transfer molding; the goal is that when they
graduate, they are ready for high-end entry-level jobs. Advanced
manufacturing in general is a strong industry in New England; a recent
analysis by Deloitte and the New England Council found that in 2012, 59
percent of the region's 641,000 manufacturing jobs were ``advanced.''

With his certificate, Kecy is confident that he will find a job locally,
and he's probably right. Great Bay's composites program was developed in
a close relationship with Safran Aerospace Composites and Albany
Engineered Composites, two companies that opened a shared plant in
Rochester in 2014. Safran helped develop the program's curriculum and
stays in touch about which specializations the company will be needing
in the coming months. It guarantees interviews to all graduates of the
program and has hired about 30 of the more than 170 participants so far.
Over all, more than half the program's graduates have been hired by five
large local manufacturers, according to its director, Debra Mattson.

That level of coordination with local industry, ideally touching on
everything from curriculum to recruitment, is now seen by policy experts
as a crucial dividing line between programs that work and those that
don't. The federal government now emphasizes this kind of ``demand
driven'' training in part to ensure that workers aren't being retrained
with new skills as obsolete as their old ones. ``A good sign is if the
program was co-developed with the firm,'' says Mark Muro, a senior
fellow at the Brookings Institution's Metropolitan Policy Program. ``One
of the fundamental problems is training divorced from labor-market
dynamics --- people being trained without the presence of jobs they
could actually arrive in.'' (The Nordic countries, which spend more on
job training in general, have a strong record in developing training
with input from both industry and labor.) The evidence in the United
States for demand-driven training is promising so far. A 2010 study of
three such programs found that enrollees were earning almost 30 percent
more than a control group two years after they began the program and
were significantly more likely to be employed.

The Great Bay program has relationships with Safran, A.E.C. and other
area employers, including BAE Systems, Turbocam International and the
gun manufacturer Sig Sauer, which recently landed a \$580 million
contract with the Army. The program is short by design, and new cohorts
start three times a year to ensure a steady stream of graduates for
local employers. ``Industry is dying for bodies, just dying for skilled
workers,'' says Will Arvelo, Great Bay's president. ``They can't wait
two years.''

On a snowy afternoon a few weeks ago, Kecy and his classmates in his
Fundamentals of Composites Manufacturing class were at work in the
``clean room.'' The setting looked more like a science lab than a
factory. A large cooler stacked with vacuum-sealed bags of thick fabric
pieces stood in the corner, and work tables held clusters of metal
tubes. The class instructor, Peter Dow, watched as two teams of students
worked on a project they had been planning for several weeks:
constructing a three-inch carbon-fiber tube with a finished exterior.
Later they would have a chance to tweak their plans and try it all over
again, a lesson in the manufacturing principle of ``continuous
improvement.''

For all the ways in which technology has changed the manufacturing
industry, one of the most striking to an outsider is the appearance of
the work space itself. The students in the clean room wore white coats
and safety glasses as they used hair dryers and refrigerant spray to
fiddle with the sticky material. Outside their small work area, the
facility's spotless manufacturing lab offered the capacity to build a
product from start to finish: a huge, three-dimensional loom for weaving
carbon fiber, a five-axis machining center, an automatic autoclave.
Practically every piece of equipment seemed to feature a keyboard or
touch screen.

But manufacturing's new high-tech, high-skill profile is also what makes
it daunting for many older workers looking for new careers. The dilemma
illustrates some of the broader challenges of retraining later in life.
Kerri Uyeno, a 43-year-old single mother of three who graduated in the
Great Bay program's first cohort in 2014, began working at Safran as a
bonding operator three weeks after earning her certificate. It was such
a happy ending that she featured prominently in early publicity
materials for the program. But she had conflicts with her supervisors
and lasted just over a year in the job before quitting. She didn't work
again for six months; her house went into foreclosure. An administrator
at Great Bay tried to persuade her to come back and work toward her
associate degree, but the prospect was exhausting. ``It was so hard to
get through that six months to my certificate,'' she said, ``I just
didn't have it in me to get more schooling.'' Today she is an office
manager at a flooring showroom nearby. She still exudes pride when she
talks about earning her certificate, but she also calls the experience
``one of the biggest heartbreaks I've ever gone through.''

At 49, Dean Kandilakis is one of the oldest students in the program's
current cohort. He has a master's degree in international relations, but
he spent most of his career doing administrative work. ``There's a
really large learning curve for someone who's just re-entering from a
different field,'' he said during a break from class. ``It's been a very
stressful time for me, because it's an adjustment in my identity as a
human being.'' But he says it's worth it to feel as if he's finally
becoming a specialist in something.

It can take enormous intellectual and emotional efforts to pursue
retraining, especially for people who have been rattled by sudden job
loss or depressed by declining career prospects. For all his
grandiosity, Donald Trump's approach to working-class voters was
characterized by relentless pessimism: dark visions of ``poverty and
heartache,'' warnings about Mexicans ``taking our manufacturing jobs.''
Nostalgia, with its disdain for the present and mistrust of the future,
is actually quite a gloomy sentiment. Job training, by contrast, makes
the smaller-but-sunnier assurance that starting over is possible with
help and time. It takes optimism on the part of both policy makers and
workers. Back in the lab, Kandilakis's team had been having some
difficulty with their tube; the material was too warm, and it was
thickening too quickly as they molded it. ``We're having some problems
today,'' he said, but he didn't sound concerned. ``Thankfully we'll have
another run.''

Advertisement

\protect\hyperlink{after-bottom}{Continue reading the main story}

\hypertarget{site-index}{%
\subsection{Site Index}\label{site-index}}

\hypertarget{site-information-navigation}{%
\subsection{Site Information
Navigation}\label{site-information-navigation}}

\begin{itemize}
\tightlist
\item
  \href{https://help.nytimes3xbfgragh.onion/hc/en-us/articles/115014792127-Copyright-notice}{©~2020~The
  New York Times Company}
\end{itemize}

\begin{itemize}
\tightlist
\item
  \href{https://www.nytco.com/}{NYTCo}
\item
  \href{https://help.nytimes3xbfgragh.onion/hc/en-us/articles/115015385887-Contact-Us}{Contact
  Us}
\item
  \href{https://www.nytco.com/careers/}{Work with us}
\item
  \href{https://nytmediakit.com/}{Advertise}
\item
  \href{http://www.tbrandstudio.com/}{T Brand Studio}
\item
  \href{https://www.nytimes3xbfgragh.onion/privacy/cookie-policy\#how-do-i-manage-trackers}{Your
  Ad Choices}
\item
  \href{https://www.nytimes3xbfgragh.onion/privacy}{Privacy}
\item
  \href{https://help.nytimes3xbfgragh.onion/hc/en-us/articles/115014893428-Terms-of-service}{Terms
  of Service}
\item
  \href{https://help.nytimes3xbfgragh.onion/hc/en-us/articles/115014893968-Terms-of-sale}{Terms
  of Sale}
\item
  \href{https://spiderbites.nytimes3xbfgragh.onion}{Site Map}
\item
  \href{https://help.nytimes3xbfgragh.onion/hc/en-us}{Help}
\item
  \href{https://www.nytimes3xbfgragh.onion/subscription?campaignId=37WXW}{Subscriptions}
\end{itemize}
