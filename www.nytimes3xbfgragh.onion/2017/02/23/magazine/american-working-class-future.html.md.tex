Divisions of Labor

\url{https://nyti.ms/2maIjRW}

\begin{itemize}
\item
\item
\item
\item
\item
\item
\end{itemize}

\includegraphics{https://static01.graylady3jvrrxbe.onion/images/2017/02/26/magazine/26cover/26cover-articleLarge-v2.jpg?quality=75\&auto=webp\&disable=upscale}

Sections

\protect\hyperlink{site-content}{Skip to
content}\protect\hyperlink{site-index}{Skip to site index}

The Future of Work

\hypertarget{divisions-of-labor}{%
\section{Divisions of Labor}\label{divisions-of-labor}}

New kinds of work require new ideas --- and new ways of organizing.

Credit...Craig Cutler for The New York Times

Supported by

\protect\hyperlink{after-sponsor}{Continue reading the main story}

By Barbara Ehrenreich

\begin{itemize}
\item
  Feb. 23, 2017
\item
  \begin{itemize}
  \item
  \item
  \item
  \item
  \item
  \item
  \end{itemize}
\end{itemize}

\textbf{The working class,} or at least the white part, has emerged as
our great national mystery. Traditionally Democratic, they helped elect
a flamboyantly ostentatious billionaire to the presidency. ``What's
wrong with them?'' the liberal pundits keep asking. Why do they believe
Trump's promises? Are they stupid or just deplorably racist? Why did the
working class align itself against its own interests?

I was born into this elusive class and remain firmly connected to it
through friendships and family. In the 1980s, for example, I personally
anchored a working-class cultural hub in my own home on Long Island. The
attraction was not me but my husband (then) and longtime friend Gary
Stevenson, a former warehouse worker who had become an organizer for the
Teamsters union. You may think of the Long Island suburbs as a bedroom
community for Manhattan commuters or a portal to the Hamptons, but they
were then also an industrial center, with more than 20,000 workers
employed at Grumman alone. When my sister moved into our basement from
Colorado, she quickly found a job in a factory within a mile of our
house, as did thousands of other people, some of them bused in from the
Bronx. Mostly we hosted local residents who passed through our house for
evening meetings or weekend gatherings --- truck drivers, factory
workers, janitors and eventually nurses. My job was to make chili and
keep room in the fridge for the baked ziti others would invariably
bring. I once tried to explain the concept of ``democratic socialism''
to some machine-shop workers and went off on a brief peroration against
the Soviet Union. They stared at me glumly across the kitchen counter
until one growled, ``At least they have health care over there.''

By the time my little crew was gathering in the ranch house,
working-class aspirations were everywhere being trampled underfoot. In
1981, President Reagan busted the air traffic controllers' union by
firing more than 11,000 striking workers --- a clear signal of what was
to come. A few years later, we hosted a picnic for Jim Guyette, the
leader of a militant meatpacking local in Minnesota that had undertaken
a wildcat strike against Hormel (and of course no Hormel products were
served at our picnic). But labor had entered into an age of givebacks
and concessions. Grovel was the message, or go without a job. Even the
``mighty mighty'' unions of the old labor chant, the ones that our
little group had struggled both to build and to democratize, were
threatened with extinction. Within a year, the wildcat local was crushed
by its own parent union, the United Food and Commercial Workers.

Steel mills went quiet, the mines where my father and grandfather had
worked shut down, factories fled south of the border. Much more was lost
in the process than just the jobs; an entire way of life, central to the
American mythos, was coming to an end. The available jobs, in fields
like retail sales and health care, were ill paid, making it harder for a
man without a college education to support a family on his own. I could
see this in my own extended family, where the grandsons of miners and
railroad workers were taking jobs as delivery-truck drivers and
fast-food restaurant managers or even competing with their wives to
become retail workers or practical nurses. As Susan Faludi observed in
her 1999 book ``Stiffed,'' the deindustrialization of America led to a
profound masculinity crisis: What did it mean to be a man when a man
could no longer support a family?

It wasn't just a way of life that was dying but also many of those who
had lived it. Research in 2015 by Angus Deaton, a Nobel laureate in
economics, with his wife, Anne Case, showed that the mortality gap
between college-educated whites and non-college-educated whites had been
widening rapidly since 1999. A couple of months later, economists at the
Brookings Institution found that for men born in 1920, there was a
six-year difference in life expectancy between the top 10 percent of
earners and the bottom 10 percent. For men born in 1950, that difference
more than doubled, to 14 years. Smoking, which is now mostly a
working-class habit, could account for only a third of the excess
deaths. The rest were apparently attributable to alcoholism, drug
overdoses and suicide, usually by gunshot --- what are often called
``diseases of despair.''

\textbf{In the new} economic landscape of low-paid service jobs, some of
the old nostrums of the left have stopped making sense. ``Full
employment,'' for example, was the mantra of the unions for decades, but
what did it mean when so many jobs no longer paid enough to live on? The
idea had been that if everyone who wanted a job could get one, employers
would have to raise wages to attract new workers. But when I went out as
an undercover journalist in the late 1990s to test the viability of
entry-level jobs, I found my co-workers --- waitstaff, nursing-home
workers, maids with a cleaning service, Walmart ``associates'' ---
living for the most part in poverty. As I reported in the resulting
book, ``Nickel and Dimed,'' some were homeless and slept in their cars,
while others skipped lunch because they couldn't afford anything more
than a snack-size bag of Doritos. They were full-time workers, and this
was a time, like the present, of nearly full employment.

\includegraphics{https://static01.graylady3jvrrxbe.onion/images/2017/02/26/magazine/26openernewpolitics-alt/26openernewpolitics-alt-articleInline.jpg?quality=75\&auto=webp\&disable=upscale}

The other popular solution to the crisis of the working class was job
retraining. If ours is a ``knowledge economy'' --- which sounds so much
better than a ``low-wage economy'' --- unemployed workers would just
have to get their game on and upgrade to more useful skills. President
Obama promoted job retraining, as did Hillary Clinton as a presidential
candidate, along with many Republicans. The problem was that no one was
sure what to train people in; computer skills were in vogue in the '90s,
welding has gone in and out of style and careers in the still-growing
health sector are supposed to be the best bets now. Nor is there any
clear measure of the effectiveness of existing retraining programs. In
2011, the Government Accountability Office found the federal government
supporting 47 job-training projects as of 2009, of which only five had
been evaluated in the previous five years. Paul Ryan has repeatedly
praised a program in his hometown, Janesville, Wis., but a 2012
ProPublica study found that laid-off people who went through it were
less likely to find jobs than those who did not.

No matter how good the retraining program, the idea that people should
be endlessly malleable and ready to recreate themselves to accommodate
every change in the job market is probably not realistic and certainly
not respectful of existing skills. In the early '90s, I had dinner at a
Pizza Hut with a laid-off miner in Butte, Mont. (actually, there are no
other kinds of miners in Butte). He was in his 50s, and he chuckled when
he told me that he was being advised to get a degree in nursing. I
couldn't help laughing too --- not at the gender incongruity but at the
notion that a man whose tools had been a pickax and dynamite should now
so radically change his relation to the world. No wonder that when
blue-collar workers were given the choice between job retraining, as
proffered by Clinton, and somehow, miraculously, bringing their old jobs
back, as proposed by Trump, they went for the latter.

Now when politicians invoke ``the working class,'' they are likely to
gesture, anachronistically, to an abandoned factory. They might more
accurately use a hospital or a fast-food restaurant as a prop. The new
working class contains many of the traditional blue-collar occupations
--- truck driver, electrician, plumber --- but by and large its members
are more likely to wield mops than hammers, and bedpans rather than
trowels. Demographically, too, the working class has evolved from the
heavily white male grouping that used to assemble at my house in the
1980s; black and Hispanic people have long been a big, if
unacknowledged, part of the working class, and now it's more female and
contains many more immigrants as well. If the stereotype of the old
working class was a man in a hard hat, the new one is better represented
as a woman chanting, ``El pueblo unido jamás será vencido!'' (The people
united will never be defeated!)

The old jobs aren't coming back, but there is another way to address the
crisis brought about by deindustrialization: Pay all workers better. The
big labor innovation of the 21st century has been campaigns seeking to
raise local or state minimum wages. Activists have succeeded in passing
living-wage laws in more than a hundred counties and municipalities
since 1994 by appealing to a simple sense of justice: Why should someone
work full time, year-round, and not make enough to pay for rent and
other basics? Surveys found large majorities favoring an increase in the
minimum wage; college students, church members and unions rallied to
local campaigns. Unions started taking on formerly neglected
constituencies like janitors, home health aides and day laborers. And
where the unions have faltered, entirely new kinds of organizations
sprang up: associations sometimes backed by unions and sometimes by
philanthropic foundations --- Our Walmart, the National Domestic Workers
Alliance and the Restaurant Opportunities Centers United.

\textbf{Our old scene} on Long Island is long gone: the house sold, the
old friendships frayed by age and distance. I miss it. As a group, we
had no particular ideology, but our vision, which was articulated
through our parties rather than any manifesto, was utopian, especially
in the context of Long Island, where if you wanted any help from the
county, you had to be a registered Republican. If we had a single theme,
it could be summed up in the old-fashioned word ``solidarity'': If you
join my picket line, I'll join yours, and maybe we'll all go protest
together, along with the kids, at the chemical plant that was oozing
toxins into our soil --- followed by a barbecue in my backyard. We were
not interested in small-P politics. We wanted a world in which
everyone's work was honored and every voice heard.

I never expected to be part of anything like that again until, in 2004,
I discovered a similar, far-better-organized group in Fort Wayne, Ind.
The Northeast Indiana Central Labor Council, as it was then called,
brought together Mexican immigrant construction workers and the
native-born building-trade union members they had been brought in to
replace, laid-off foundry workers and Burmese factory workers, adjunct
professors and janitors. Their goal, according to the president at the
time, Tom Lewandowski, a former General Electric factory worker who
served in the 1990s as the A.F.L.-C.I.O.'s liaison to the Polish
insurgent movement Solidarnosc, was to create a ``culture of
solidarity.'' They were inspired by the realization that it's not enough
to organize people with jobs; you have to organize the unemployed as
well as the ``anxiously employed'' --- meaning potentially the entire
community. Their not-so-secret tactic was parties and picnics, some of
which I was lucky enough to attend.

The scene in Fort Wayne featured people of all colors and collar colors,
legal and undocumented workers, liberals and political conservatives,
some of whom supported Trump in the last election. It showed that a new
kind of solidarity was in reach, even if the old unions may not be
ready. In 2016, the ailing A.F.L.-C.I.O., which for more than six
decades has struggled to hold the labor movement together, suddenly
dissolved the Northeast Indiana Central Labor Council, citing obscure
bureaucratic imperatives. But the labor council was undaunted. It
promptly reinvented itself as the Workers' Project and drew more than
6,000 people to the local Labor Day picnic, despite having lost its
internet access and office equipment to the A.F.L.-C.I.O.

When I last talked to Tom Lewandowski, in early February, the Workers'
Project had just succeeded in organizing 20 Costco contract workers into
a collective unit of their own and were planning to celebrate with, of
course, a party. The human urge to make common cause --- and have a good
time doing it --- is hard to suppress.

Advertisement

\protect\hyperlink{after-bottom}{Continue reading the main story}

\hypertarget{site-index}{%
\subsection{Site Index}\label{site-index}}

\hypertarget{site-information-navigation}{%
\subsection{Site Information
Navigation}\label{site-information-navigation}}

\begin{itemize}
\tightlist
\item
  \href{https://help.nytimes3xbfgragh.onion/hc/en-us/articles/115014792127-Copyright-notice}{©~2020~The
  New York Times Company}
\end{itemize}

\begin{itemize}
\tightlist
\item
  \href{https://www.nytco.com/}{NYTCo}
\item
  \href{https://help.nytimes3xbfgragh.onion/hc/en-us/articles/115015385887-Contact-Us}{Contact
  Us}
\item
  \href{https://www.nytco.com/careers/}{Work with us}
\item
  \href{https://nytmediakit.com/}{Advertise}
\item
  \href{http://www.tbrandstudio.com/}{T Brand Studio}
\item
  \href{https://www.nytimes3xbfgragh.onion/privacy/cookie-policy\#how-do-i-manage-trackers}{Your
  Ad Choices}
\item
  \href{https://www.nytimes3xbfgragh.onion/privacy}{Privacy}
\item
  \href{https://help.nytimes3xbfgragh.onion/hc/en-us/articles/115014893428-Terms-of-service}{Terms
  of Service}
\item
  \href{https://help.nytimes3xbfgragh.onion/hc/en-us/articles/115014893968-Terms-of-sale}{Terms
  of Sale}
\item
  \href{https://spiderbites.nytimes3xbfgragh.onion}{Site Map}
\item
  \href{https://help.nytimes3xbfgragh.onion/hc/en-us}{Help}
\item
  \href{https://www.nytimes3xbfgragh.onion/subscription?campaignId=37WXW}{Subscriptions}
\end{itemize}
