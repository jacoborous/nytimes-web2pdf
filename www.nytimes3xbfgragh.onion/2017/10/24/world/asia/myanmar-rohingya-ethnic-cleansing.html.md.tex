Sections

SEARCH

\protect\hyperlink{site-content}{Skip to
content}\protect\hyperlink{site-index}{Skip to site index}

\href{https://www.nytimes3xbfgragh.onion/section/world/asia}{Asia
Pacific}

\href{https://myaccount.nytimes3xbfgragh.onion/auth/login?response_type=cookie\&client_id=vi}{}

\href{https://www.nytimes3xbfgragh.onion/section/todayspaper}{Today's
Paper}

\href{/section/world/asia}{Asia Pacific}\textbar{}Across Myanmar, Denial
of Ethnic Cleansing and Loathing of Rohingya

\url{https://nyti.ms/2zxkBly}

\begin{itemize}
\item
\item
\item
\item
\item
\item
\end{itemize}

Advertisement

\protect\hyperlink{after-top}{Continue reading the main story}

Supported by

\protect\hyperlink{after-sponsor}{Continue reading the main story}

\hypertarget{across-myanmar-denial-of-ethnic-cleansing-and-loathing-of-rohingya}{%
\section{Across Myanmar, Denial of Ethnic Cleansing and Loathing of
Rohingya}\label{across-myanmar-denial-of-ethnic-cleansing-and-loathing-of-rohingya}}

\includegraphics{https://static01.graylady3jvrrxbe.onion/images/2017/10/25/world/25rakhine-1/25rakhine-1-articleLarge.jpg?quality=75\&auto=webp\&disable=upscale}

By \href{https://www.nytimes3xbfgragh.onion/by/hannah-beech}{Hannah
Beech}

\begin{itemize}
\item
  Oct. 24, 2017
\item
  \begin{itemize}
  \item
  \item
  \item
  \item
  \item
  \item
  \end{itemize}
\end{itemize}

\href{https://cn.nytimes3xbfgragh.onion/asia-pacific/20171025/myanmar-rohingya-ethnic-cleansing/}{阅读简体中文版}\href{https://cn.nytimes3xbfgragh.onion/asia-pacific/20171025/myanmar-rohingya-ethnic-cleansing/zh-hant/}{閱讀繁體中文版}

SITTWE, Myanmar --- The Buddhist abbot tucked his legs under his robes
and began to explain.

Rohingya Muslims do not belong in Myanmar, and they never have, he said.
Their fertility allowed them to overwhelm the local Buddhist population.
But now, somehow, many Rohingya seemed to be gone.

``We thank the Lord Buddha for this,'' said U Thu Min Gala, the
57-year-old abbot of the Damarama Monastery in Sittwe, the capital of
Rakhine State in western Myanmar. ``They stole our land, our food and
our water. We will never accept them back.''

An overwhelming body of published accounts has detailed the
\href{https://www.nytimes3xbfgragh.onion/2017/10/11/world/asia/rohingya-myanmar-atrocities.html?rref=collection\%2Ftimestopic\%2FMyanmar}{Myanmar
Army's campaign of killing, rape and arson} in Rakhine, which has driven
more than
\href{https://www.nytimes3xbfgragh.onion/2017/09/29/world/asia/rohingya-refugees-myanmar-bangladesh.html?action=click\&contentCollection=Asia\%20Pacific\&module=RelatedCoverage\&region=Marginalia\&pgtype=article}{600,000
Rohingya out of the country} since late August, in what the United
Nations says is the fastest displacement of a people since the Rwanda
genocide.

But in Myanmar, and even in Rakhine itself, there is stark denial that
any ethnic cleansing is taking place.

The divergence between how Myanmar and much of the outside world see the
Rohingya is not limited to one segment of local society. Nor can hatred
in Myanmar of the largely stateless Muslim group be dismissed as a
fringe attitude.

\emph{{[}Read:}
\emph{\href{http://www.nytimes3xbfgragh.onion/2020/01/23/world/asia/myanmar-rohingya-genocide.html}{U.N.
court orders Myanmar to protect Rohingya muslims}.{]}}

Government officials, opposition politicians, religious leaders and even
local human-rights activists have become unified behind this narrative:
The Rohingya are not rightful citizens of Buddhist-majority Myanmar, and
now, through the power of a globally resurgent Islam,
\href{https://www.nytimes3xbfgragh.onion/2017/09/18/world/asia/myanmar-rohingya-ethnic-cleansing.html}{the
minority is falsely trying to hijack the world's sympathy}.

\includegraphics{https://static01.graylady3jvrrxbe.onion/images/2017/10/25/world/25rakhine-2/25rakhine-2-articleLarge.jpg?quality=75\&auto=webp\&disable=upscale}

Social media postings have amplified the message, claiming that
international aid workers are openly siding with the Rohingya.
Accordingly, the Myanmar government has blocked aid agencies' access to
Rohingya still trapped in Myanmar --- about 120,000 confined to camps in
central Rakhine and tens of thousands more in desperate conditions in
the north.

The official answer to United Nations accounts of the military's mass
burning of villages and targeting of civilians has been to insist that
the Rohingya have been doing it to themselves.

``There is no case of the military killing Muslim civilians,'' said Dr.
Win Myat Aye, the country's social welfare minister and the governing
National League for Democracy party's point person on Rakhine. ``Muslim
people killed their own Muslim people.''

When asked in an interview about the evidence against the military, the
minister noted that the Myanmar government had not sent any
investigators to Bangladesh to vet the testimony of fleeing Rohingya,
but that he would raise the possibility of doing so in a future meeting.

``Thank you for advising us on this idea,'' he said.

The Rohingya, who speak a Bengali dialect and tend to look distinct from
most of Myanmar's other ethnic groups, have had roots in Rakhine for
generations. Communal tensions between the Rohingya and ethnic Rakhine
Buddhists exploded in World War II, when the Rakhine aligned themselves
with the Japanese, while the Rohingya chose the British.

Although many Rohingya were considered citizens when Myanmar, also known
as Burma, became independent in 1948, the military junta that wrested
power in 1962 began stripping them of their rights. After a restrictive
citizenship law was introduced in 1982, most Rohingya became stateless.

Even the name Rohingya, which the ethnic group has identified with more
vocally in recent years, has been taken from them. The Myanmar
government usually refers to the Rohingya as Bengalis, implying they
belong in Bangladesh. The public tends to call them an epithet used for
all Muslims in Myanmar: kalar.

Image

Daw Soe Chay, an ethnic Rakhine Buddhist from Myebon Township, was
beaten and publicly shamed after her husband delivered aid to Rohingya
Muslims in their camp in Sittwe.Credit...Adam Dean for The New York
Times

The nomenclature is so sensitive that in a speech this month, Daw Aung
San Suu Kyi, the Nobel Peace Prize laureate and de facto leader of the
government, referred only to ``those who have crossed over to
Bangladesh.''

Some ethnic Rakhine politicians are hailing the Rohingya exodus as a
good thing.

``All the Bengalis learn in their religious schools is to brutally kill
and attack,'' said Daw Khin Saw Wai, a Rakhine member of Parliament from
Rathedaung Township. ``It is impossible to live together in the
future.''

Buddhist monks, moral arbiters in a pious land, have been at the
forefront of a campaign to dehumanize the Rohingya. In popular videos,
extremist monks refer to the Rohingya as ``snakes'' or ``worse than
dogs.''

Outside Mr. Thu Min Gala's monastery in Sittwe, a pair of signs
reflected an alternate sense of reality. One said that the monastery,
which is sheltering ethnic Rakhine who fled the conflict zone, would not
accept any donations from international agencies. The other warned that
multifaith groups were not welcome.

The abbot claimed that the authorities in Rakhine had stopped a car
owned by the International Committee of the Red Cross that was filled
with weaponry destined for Rohingya militants who carried out attacks
against the security forces in August. Mr. Thu Min Gala claimed that
sticks of dynamite had been wrapped in paper with the Red Cross logo.
The Red Cross denied these accusations.

``We don't trust the international society,'' the abbot said. ``They are
only on the side of the terrorists.''

At another monastery in Sittwe, an elderly abbot, U Baddanta Thaw Ma,
halted my conversation with a young monk by slapping the air in front of
my face. ``Go! Go! Go!'' he yelled in English, before switching to the
local Rakhine dialect. ``Go away, you foreigner! Go away, you kalar
lover.''

Public sentiment against Muslims --- who are about 4 percent of
Myanmar's population, encompassing several ethnic groups, including the
Rohingya --- has spread beyond Rakhine. In 2015 elections, no major
political party fielded a Muslim candidate. Today no Muslims serve in
Parliament, the first time since the country's independence.

A couple of hours outside Yangon, the country's largest city, U Aye Swe,
an administrator for Sin Ma Kaw village, said he was proud to oversee
one of Myanmar's ``Muslim-free'' villages, which bar Muslims from
spending the night, among other restrictions.

``Kalar are not welcome here because they are violent and they multiply
like crazy, with so many wives and children,'' he said.

Mr. Aye Swe admitted he had never met a Muslim before, adding, ``I have
to thank Facebook because it is giving me the true information in
Myanmar.''

Social media messaging has driven much of the rage in Myanmar. Though
widespread access to cellphones only started a few years ago, mobile
penetration is now about 90 percent. For many people, Facebook is their
only source of news, and they have little experience in sifting fake
news from credible reporting.

One widely shared message on Facebook, from a spokesman for Ms. Aung San
Suu Kyi's office, emphasized that biscuits from the World Food Program,
a United Nations agency, had been found at a Rohingya militant training
camp. The United Nations called the post ``irresponsible.''

The Myanmar government, however, insists the public needs to be guided.

``We do something that we call educating the people,'' said U Pe Myint,
the nation's information minister. He acknowledged, ``It looks rather
like indoctrination, like in an authoritarian or totalitarian state.''

Image

A Buddhist woman and her son were staying at the Damarama Monastery, in
Sittwe, after being displaced by violence in northern
Rakhine.Credit...Adam Dean for The New York Times

In Yangon, Mr. Pe Myint this month gathered local journalists to discuss
what he called ``fabricated news'' by foreign reporters and a
``political war'' in which international aid groups favored the
Rohingya.

Last month, a mob in Sittwe attacked Red Cross workers, who were loading
a boat with supplies that locals believed would only go to the Rohingya.

Even among officials who might otherwise champion human rights,
frustration has been directed at foreign critics. Quietly, some defend
Ms. Aung San Suu Kyi's failure to call out the military and protect the
Rohingya by saying it would be political suicide in a country where
hatred of the Rohingya is so widespread. They see the recent
international pressure, at best, as ignorant of domestic complexities
and, at worst, as intent on hindering Myanmar's development.

``We ask the international community to acknowledge that these Muslims
are illegal immigrants from Bangladesh and that this crisis is an
infringement of our sovereignty,'' said U Nyan Win, a spokesman for the
National League for Democracy, which shares power with Myanmar's
military. ``This is the most important thing with the Rakhine issue.''

U Ko Ko Gyi, a democracy advocate who was jailed for 17 years by the
military when it ruled Myanmar, also evoked national interest.

``We have been human-rights defenders for many years and suffered for a
long time but we are standing together on this issue because we need to
support our national security,'' he said.

``We are a small country that lies between India and China, and the DNA
of our ancestors is to try to struggle for our survival,'' Mr. Ko Ko Gyi
said. ``If you in the West criticize us too much, then you will push us
into the arms of China and Russia.''

Image

Sin Ma Kaw, where an official said "he was proud to oversee one of
Myanmar's ``Muslim-free'' villages.Credit...Adam Dean for The New York
Times

Last month, those two permanent members of the United Nations Security
Council shielded Myanmar from an attempt by other nations to condemn the
Myanmar military for its offensive in Rakhine.

The humanitarian situation has grown desperate within Rakhine while the
official block on aid largely continues.

Throughout the state, ethnic Rakhine have been warned by community
leaders not to break the blockade. Last month in Myebon Township, in
central Rakhine, women's activists prevented international aid groups
from delivering assistance to an internment camp where thousands of
Rohingya have been sequestered since the 2012 sectarian violence,
according to foreign staff.

But U Tun Tin, a Rakhine trishaw driver, needed the money and delivered
food to the Rohingya camp. Shortly after, his wife, Daw Soe Chay, said
she was accosted by a crowd that forced her to a nearby monastery.

Inside the religious compound, they beat her and sheared her hair. Then
the mob marched her through Myebon, wearing a sign calling her a
``national traitor.''

Despite his wife's ordeal, Mr. Tun Tin said he did not regret having
sent supplies to the camp, where Rohingya say their rations are running
low.

``They are human,'' he said. ``They need to eat, just like us.''

Advertisement

\protect\hyperlink{after-bottom}{Continue reading the main story}

\hypertarget{site-index}{%
\subsection{Site Index}\label{site-index}}

\hypertarget{site-information-navigation}{%
\subsection{Site Information
Navigation}\label{site-information-navigation}}

\begin{itemize}
\tightlist
\item
  \href{https://help.nytimes3xbfgragh.onion/hc/en-us/articles/115014792127-Copyright-notice}{©~2020~The
  New York Times Company}
\end{itemize}

\begin{itemize}
\tightlist
\item
  \href{https://www.nytco.com/}{NYTCo}
\item
  \href{https://help.nytimes3xbfgragh.onion/hc/en-us/articles/115015385887-Contact-Us}{Contact
  Us}
\item
  \href{https://www.nytco.com/careers/}{Work with us}
\item
  \href{https://nytmediakit.com/}{Advertise}
\item
  \href{http://www.tbrandstudio.com/}{T Brand Studio}
\item
  \href{https://www.nytimes3xbfgragh.onion/privacy/cookie-policy\#how-do-i-manage-trackers}{Your
  Ad Choices}
\item
  \href{https://www.nytimes3xbfgragh.onion/privacy}{Privacy}
\item
  \href{https://help.nytimes3xbfgragh.onion/hc/en-us/articles/115014893428-Terms-of-service}{Terms
  of Service}
\item
  \href{https://help.nytimes3xbfgragh.onion/hc/en-us/articles/115014893968-Terms-of-sale}{Terms
  of Sale}
\item
  \href{https://spiderbites.nytimes3xbfgragh.onion}{Site Map}
\item
  \href{https://help.nytimes3xbfgragh.onion/hc/en-us}{Help}
\item
  \href{https://www.nytimes3xbfgragh.onion/subscription?campaignId=37WXW}{Subscriptions}
\end{itemize}
