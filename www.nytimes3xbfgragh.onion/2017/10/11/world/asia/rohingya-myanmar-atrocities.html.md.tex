Sections

SEARCH

\protect\hyperlink{site-content}{Skip to
content}\protect\hyperlink{site-index}{Skip to site index}

\href{https://www.nytimes3xbfgragh.onion/section/world/asia}{Asia
Pacific}

\href{https://myaccount.nytimes3xbfgragh.onion/auth/login?response_type=cookie\&client_id=vi}{}

\href{https://www.nytimes3xbfgragh.onion/section/todayspaper}{Today's
Paper}

\href{/section/world/asia}{Asia Pacific}\textbar{}Rohingya Recount
Atrocities: `They Threw My Baby Into a Fire'

\url{https://nyti.ms/2z00lsf}

\begin{itemize}
\item
\item
\item
\item
\item
\item
\end{itemize}

Advertisement

\protect\hyperlink{after-top}{Continue reading the main story}

Supported by

\protect\hyperlink{after-sponsor}{Continue reading the main story}

\hypertarget{rohingya-recount-atrocities-they-threw-my-baby-into-a-fire}{%
\section{Rohingya Recount Atrocities: `They Threw My Baby Into a
Fire'}\label{rohingya-recount-atrocities-they-threw-my-baby-into-a-fire}}

\includegraphics{https://static01.graylady3jvrrxbe.onion/images/2017/10/12/world/12massacres-1/09massacres-1-articleLarge.jpg?quality=75\&auto=webp\&disable=upscale}

By \href{http://www.nytimes3xbfgragh.onion/by/jeffrey-gettleman}{Jeffrey
Gettleman}

\begin{itemize}
\item
  Oct. 11, 2017
\item
  \begin{itemize}
  \item
  \item
  \item
  \item
  \item
  \item
  \end{itemize}
\end{itemize}

COX'S BAZAR, Bangladesh --- Hundreds of women stood in the river, held
at gunpoint, ordered not to move.

A pack of soldiers stepped toward a petite young woman with light brown
eyes and delicate cheekbones. Her name was Rajuma, and she was standing
chest-high in the water, clutching her baby son, while her village in
Myanmar burned down behind her.

``You,'' the soldiers said, pointing at her.

She froze.

``You!''

She squeezed her baby tighter.

In the next violent blur of moments, the soldiers clubbed Rajuma in the
face, tore her screaming child out of her arms and hurled him into a
fire. She was then dragged into a house and gang-raped.

By the time the day was over, she was running through a field naked and
covered in blood. Alone, she had lost her son, her mother, her two
sisters and her younger brother, all wiped out in front of her eyes, she
says.

Rajuma is a
\href{https://www.nytimes3xbfgragh.onion/2017/09/13/world/asia/myanmar-rohingya-muslim.html}{Rohingya}
Muslim, one of the most persecuted ethnic groups on earth, and she now
spends her days drifting through a refugee camp in Bangladesh in a daze.

She relayed her story to me during a recent reporting trip I made to the
camps, where hundreds of thousands of Rohingya like her have rushed for
safety. Her deeply disturbing account of what happened in her village,
in late August, was corroborated by dozens of other survivors, whom I
spoke with at length, and by human rights groups gathering evidence of
atrocities.

Survivors said they saw government soldiers stabbing babies, cutting off
boys' heads, gang-raping girls, shooting 40-millimeter grenades into
houses, burning entire families to death, and rounding up dozens of
unarmed male villagers and summarily executing them.

\includegraphics{https://static01.graylady3jvrrxbe.onion/images/2017/10/12/world/12massacres-7/10massacres-7-articleLarge.jpg?quality=75\&auto=webp\&disable=upscale}

Much of the violence was flamboyantly brutal, intimate and personal ---
the kind that is detonated by a long, bitter history of ethnic hatred.

``People were holding the soldiers' feet, begging for their lives,''
Rajuma said. ``But they didn't stop, they just kicked them off and
killed them. They chopped people, they shot people, they raped us, they
left us senseless.''

Human rights investigators said that Myanmar's military killed more than
1,000 civilians in the state of Rakhine, and possibly as many as 5,000,
though it will be hard to ever know because Myanmar is not allowing the
United Nations or anyone else into the affected areas.

Peter Bouckaert, a veteran investigator with Human Rights Watch, said
there was growing evidence of organized massacres, like the one Rajuma
survived, in which government soldiers methodically slaughtered more
than 100 civilians in a single location. He called them crimes against
humanity.

On Wednesday, the United Nations human rights office said that
government troops had targeted ``houses, fields, food-stocks, crops,
livestock and even trees,'' making it ``almost impossible'' for the
Rohingya to return home.

Myanmar's army has claimed it was responding to an attack by
\href{https://www.nytimes3xbfgragh.onion/2017/09/17/world/asia/myanmar-rohingya-militants.html}{Rohingya
militants} on Aug. 25 and targeting only the insurgents. But according
to dozens of witnesses, almost all of the people killed were unarmed
villagers, and many had their hands bound.

Satellite imagery has revealed
\href{https://www.nytimes3xbfgragh.onion/interactive/2017/09/18/world/asia/rohingya-villages.html}{288
separate villages burned}, some down to the last post.

Human rights groups said the government troops had one goal: to erase
entire Rohingya communities. The unsparing destruction drove more than
half a million people into Bangladesh in recent weeks. United Nations
officials called the campaign against the Rohingya a ``textbook
example''
\href{https://www.nytimes3xbfgragh.onion/2017/09/11/world/asia/myanmar-rohingya-ethnic-cleansing.html}{of
ethnic cleansing}.

\href{https://www.nytimes3xbfgragh.onion/interactive/2017/09/18/world/asia/rohingya-villages.html}{}

\includegraphics{https://static01.graylady3jvrrxbe.onion/images/2017/09/18/world/asia/rohingya-villages-1505780683109/rohingya-villages-1505780683109-articleLarge.jpg}

\hypertarget{satellite-images-show-more-than-200-rohingya-villages-burned-in-myanmar}{%
\subsection{Satellite Images Show More Than 200 Rohingya Villages Burned
in
Myanmar}\label{satellite-images-show-more-than-200-rohingya-villages-burned-in-myanmar}}

Satellite images show that at least 200 villages of Myanmar's Rohingya
ethnic minority have been set ablaze since late August.

Nearly each night here in coastal Bangladesh, up the Bay of Bengal from
Myanmar,
\href{https://bdnews24.com/bangladesh/2017/09/28/bodies-of-16-rohingyas-mostly-children-wash-up-on-bangladesh-shore}{bodies
wash up} in the foamy brown tide --- children, men, old women who tried
to escape on leaking boats, their faces bloated from seawater.

Rajuma barely made it to Bangladesh, escaping on a small wooden boat a
few weeks ago. She cannot read or write. She does not have a single
piece of paper to prove who she is or that she was born in Myanmar. This
may be a problem if she applies for refugee status in Bangladesh, which
has been reluctant to give it, or ever tries to go home to Myanmar. She
thinks she is around 20, but she could pass for 14 --- painfully thin,
with wrists that look as if they could easily break.

She grew up in a rice farming hamlet called Tula Toli, and said the
place had never known peace.

Image

Hundreds of thousands of Rohingya have fled to rough camps in Bangladesh
like this one, Balukhali.Credit...Sergey Ponomarev for The New York
Times

The two main ethnic groups in her village, the Buddhist Rakhines and the
Muslim Rohingya, were like two planes drawn to never touch. They
followed different religions, spoke different languages, ate different
foods and have always distrusted each other.

A community of Buddhists lived just a few minutes from Rajuma's house,
but she had never spoken with any of them.

``They hate us,'' she said.

Azeem Ibrahim, a Scottish academic who recently wrote
\href{http://rohingyabook.com}{a book on the Rohingya}, explained that
much of the animosity could be traced to World War II, when the Rohingya
fought on the British side and many Buddhists in Rakhine fought for the
occupying Japanese. Both sides massacred civilians.

After the Allies won, the Rohingya hoped to win independence or join
East Pakistan (today's Bangladesh), which was also majority Muslim and
ethnically similar to the Rohingya. But the British, eager to appease
Myanmar's Buddhist majority, decreed that the Rohingya areas would
become part of newly independent Myanmar (then called Burma), setting
the Rohingya up for decades of discrimination.

Myanmar's leaders soon began stripping their rights and blaming them for
the country's shortcomings, claiming the Rohingya were illegal migrants
from Bangladesh who had stolen good land.

``Year after year, they were demonized,'' Mr. Ibrahim said.

Some influential Buddhist monks said the Rohingya were the reincarnation
of snakes and insects and should be exterminated, like vermin.

The persecution fueled a new Rohingya militant movement, which staged
attacks against Myanmar security outposts on Aug. 25.

Image

Rohingya men praying at a makeshift mosque in the Noapara camp, outside
Cox's Bazar, Bangladesh.Credit...Sergey Ponomarev for The New York Times

From her village, Rajuma said, she heard explosions from one of those
attacks --- or at least from the government response to it.

Over the next few days, Rajuma watched huge fires burn on the horizon.
The military was beginning what it called ``clearance operations.''
Rohingya villages all around Tula Toli were burned to the ground, and on
the night of Aug. 29, an elder came from the mosque to Rajuma's house to
deliver a message: The Buddhists say we should go to the river, for our
safety.

Her family decided to stay put. ``Nobody trusts a Buddhist,'' Rajuma
said.

The next morning, Rajuma was busy making potato curry. As she sprinkled
ginger and chiles into a big pot, she sensed something and stopped.

She crept to the window and peeked out: soldiers, dozens of them,
jogging toward Tula Toli.

Rajuma and her family tried to run but were quickly captured and marched
to a riverbank where hundreds of other terrified villagers had been
taken prisoner.

The soldiers separated the men from the women. The villagers pleaded for
their lives and dropped to their knees, hugging the soldiers' boots. The
soldiers kicked them off and methodically killed all the men, said
Rajuma and several other survivors from Tula Toli, all interviewed
separately.

Image

Rajuma, 20, saw her 18-month-old son and most of the rest of her family
killed by soldiers in Myanmar. She is now in the Kutupalong refugee camp
in Bangladesh.Credit...Sergey Ponomarev for The New York Times

The women and young children were sent into the water and told to wait.

In terms of the tactics used, the weapons fired, the openness of the
killings, the gang rapes and the level of military organization, the
accounts from many different Rohingya areas present a distressing
harmony.

``Stories of atrocities are universal,'' said Anthony Lake, the
executive director of Unicef.

He said he was profoundly troubled by what Rohingya children had been
drawing in the camps ---
\href{https://weshare.unicef.org/C.aspx?VP3=SearchResult\&LBID=2AMZKTG1WYH\&IT=Thumb_FixedHeight_M_Details_ToolTip}{guns,
fires, machetes and people on the ground with red streaming out of
them}.

In a hospital bed near Cox's Bazar, the biggest town in this part of
Bangladesh, Muhamedul Hassan, a Rohingya shopkeeper from a village
called Monu Para, lies on a clean white sheet. Doctors say the fact he
is still alive is a miracle.

On Aug. 27, Mr. Hassan said, around 20 soldiers from a nearby army base
stormed into Monu Para and ordered all the men and any boys older than
10 to report to the house of a prominent Rohingya cattle trader.

The soldiers tied everyone's hands behind their backs. They made them
sit in the yard, heads down.

Around 400 men and boys were hunched over, Mr. Hassan said. They were
sweating through their shirts. An army sergeant whom the villagers knew
then pulled out a long, thin knife.

Image

A Rohingya family at the newly erected Balukhali camp.Credit...Sergey
Ponomarev for The New York Times

``People were calling for help,'' Mr. Hassan said. ``The boys were
screaming out their mother's name, their father's name.''

Mr. Hassan said that in front of his eyes, dozens of people were
decapitated or shot. He was shot three times --- twice in the back and
once in the chest --- but all the bullets missed vital organs.

After the soldiers left, Mr. Hassan said, he stumbled away to his house,
where his sister stuffed turmeric powder, the best they could do for an
antiseptic, into his wounds.

Human rights investigators said the gravest atrocities they have
documented were committed from Aug. 25 to Sept. 1, the period right
after the militant attacks. Many witnesses described government troops
wantonly killing anyone they could get their hands on.

In Tula Toli, Rajuma fought as hard as she could to hold onto her baby,
Muhammad Sadeque, about 18 months old.

But one soldier grabbed her hands, another grabbed her body, and another
slugged her in the face with a club. A jagged scar now runs along her
jaw.

The child was lifted away from her, his legs wiggling in the air.

``They threw my baby into a fire --- they just flung him,'' she said.

Image

A drawing by a Rohingya boy about his experiences while fleeing
Myanmar.Credit...Sergey Ponomarev for The New York Times

Rajuma said two soldiers then pulled her into a house, tore off her veil
and dress and raped her. She said that her two sisters were raped and
killed in the same room, and that in the next room, her mother and
10-year-old brother were shot.

At some point, Rajuma thought she had died. She lost consciousness. When
she woke, the soldiers were gone, but the house was on fire.

She sprinted out naked, past her family's bodies, past burning homes,
and hid in a forest. Night fell, but she did not sleep.

In the morning she found an old T-shirt to wear and kept running.

Many people in the refugee camps have been eerily stoic --- seemingly
traumatized past the ability to feel. In dozens of interviews with
survivors who said their loved ones had been killed in front of them,
not a single tear was shed.

But as she reached the end of her horrible testimony, Rajuma broke down.

``I can't explain how hard it hurts,'' she said, tears rolling off her
cheeks, ``to no longer hear my son call me ma.''

She hunched over on a plastic stool in another family's hut, covered her
mouth with a red veil and started sobbing so hard she could barely
breathe.

A few other refugees looked over at her but went on cooking or cleaning.
Outside, on a road not far away, trucks blared their horns, fighting
through traffic.

Image

The body of a Rohingya woman washed up in Bangladesh after her boat sank
in rough seas in late September.Credit...Sergey Ponomarev for The New
York Times

Advertisement

\protect\hyperlink{after-bottom}{Continue reading the main story}

\hypertarget{site-index}{%
\subsection{Site Index}\label{site-index}}

\hypertarget{site-information-navigation}{%
\subsection{Site Information
Navigation}\label{site-information-navigation}}

\begin{itemize}
\tightlist
\item
  \href{https://help.nytimes3xbfgragh.onion/hc/en-us/articles/115014792127-Copyright-notice}{©~2020~The
  New York Times Company}
\end{itemize}

\begin{itemize}
\tightlist
\item
  \href{https://www.nytco.com/}{NYTCo}
\item
  \href{https://help.nytimes3xbfgragh.onion/hc/en-us/articles/115015385887-Contact-Us}{Contact
  Us}
\item
  \href{https://www.nytco.com/careers/}{Work with us}
\item
  \href{https://nytmediakit.com/}{Advertise}
\item
  \href{http://www.tbrandstudio.com/}{T Brand Studio}
\item
  \href{https://www.nytimes3xbfgragh.onion/privacy/cookie-policy\#how-do-i-manage-trackers}{Your
  Ad Choices}
\item
  \href{https://www.nytimes3xbfgragh.onion/privacy}{Privacy}
\item
  \href{https://help.nytimes3xbfgragh.onion/hc/en-us/articles/115014893428-Terms-of-service}{Terms
  of Service}
\item
  \href{https://help.nytimes3xbfgragh.onion/hc/en-us/articles/115014893968-Terms-of-sale}{Terms
  of Sale}
\item
  \href{https://spiderbites.nytimes3xbfgragh.onion}{Site Map}
\item
  \href{https://help.nytimes3xbfgragh.onion/hc/en-us}{Help}
\item
  \href{https://www.nytimes3xbfgragh.onion/subscription?campaignId=37WXW}{Subscriptions}
\end{itemize}
