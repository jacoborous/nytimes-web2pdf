Sections

SEARCH

\protect\hyperlink{site-content}{Skip to
content}\protect\hyperlink{site-index}{Skip to site index}

\href{https://www.nytimes3xbfgragh.onion/section/world/asia}{Asia
Pacific}

\href{https://myaccount.nytimes3xbfgragh.onion/auth/login?response_type=cookie\&client_id=vi}{}

\href{https://www.nytimes3xbfgragh.onion/section/todayspaper}{Today's
Paper}

\href{/section/world/asia}{Asia Pacific}\textbar{}Bloomberg News Is Said
to Curb Articles That Might Anger China

\url{https://nyti.ms/1dTodhS}

\begin{itemize}
\item
\item
\item
\item
\item
\item
\end{itemize}

Advertisement

\protect\hyperlink{after-top}{Continue reading the main story}

Supported by

\protect\hyperlink{after-sponsor}{Continue reading the main story}

\hypertarget{bloomberg-news-is-said-to-curb-articles-that-might-anger-china}{%
\section{Bloomberg News Is Said to Curb Articles That Might Anger
China}\label{bloomberg-news-is-said-to-curb-articles-that-might-anger-china}}

By \href{https://www.nytimes3xbfgragh.onion/by/edward-wong}{Edward Wong}

\begin{itemize}
\item
  Nov. 8, 2013
\item
  \begin{itemize}
  \item
  \item
  \item
  \item
  \item
  \item
  \end{itemize}
\end{itemize}

BEIJING --- The decision came in an early evening call to four
journalists huddled in a Hong Kong conference room. On the line 12 time
zones away in New York was their boss,
\href{http://www.bloomberg.com/now/people/matthew-winkler/}{Matthew
Winkler}, the longtime editor in chief of Bloomberg News. And they were
frustrated by what he was telling them.

The investigative report they had been working on for the better part of
a year, which detailed the hidden financial ties between one of the
wealthiest men in China and the families of top Chinese leaders, would
not be published.

In the call late last month, Mr. Winkler defended his decision,
comparing it to the self-censorship by foreign news bureaus trying to
preserve their ability to report inside Nazi-era Germany, according to
Bloomberg employees familiar with the discussion.

``He said, `If we run the story, we'll be kicked out of China,'~'' one
of the employees said. Less than a week later, a second article, about
the children of senior Chinese officials employed by foreign banks, was
also declared dead, employees said.

Mr. Winkler said in an email on Friday that the articles in question
were not killed. ``What you have is untrue,'' he said. ``The stories are
active and not spiked.''

His statement was echoed by the senior editor on the articles, Laurie
Hays.

Mr. Winkler and several other senior executives at Bloomberg declined to
discuss his conference calls with reporters and editors in Hong Kong.

Several Bloomberg employees in Hong Kong said Mr. Winkler made clear in
his call that his concerns were primarily about continuing to have
reporters work in China, not protecting company revenues. Even so, they
said, he gave the listeners a clear impression that the company was in
retreat on aspects of its coverage of the world's second-largest
economy, a little more than a year after it locked horns with a
confident Chinese leadership that has shown itself willing to punish
foreign news organizations that cross it.

Bloomberg News infuriated the government in 2012 by publishing
\href{http://www.bloomberg.com/news/2012-06-29/xi-jinping-millionaire-relations-reveal-fortunes-of-elite.html}{a
series of articles} on the personal wealth of the families of Chinese
leaders, including the new Communist Party chief, Xi Jinping.
Bloomberg's operations in China have suffered since, as new journalists
have been denied residency and sales of its financial terminals to state
enterprises have slowed. Chinese officials have said repeatedly that
news coverage on the wealth and personal lives of Chinese leaders
crosses a red line.

The perception among some Bloomberg employees that the company is now
unwilling to cross such lines has left them unsettled. More broadly, it
has cast new light on the dilemma that numerous foreign news
organizations confront as they navigate the pressures of doing both
journalism and business in China.

As the article on Mr. Xi's family was published, in June 2012, Chinese
officials ordered the Bloomberg News website blocked. Today, it remains
inaccessible on Chinese servers. No Bloomberg journalist trying to enter
China on a new long-term assignment has received a residency visa.

Most important for the larger Bloomberg company's bottom line, financial
news terminal subscriptions, which cost more than \$20,000 per year and
are the main revenue generator for Bloomberg, slowed for a spell in
China, after officials issued orders to some Chinese companies to avoid
buying subscriptions. Despite all that,
\href{http://www.scio.gov.cn/jrxx/xkmd/1/Document/1063839/1063839.htm}{Bloomberg
got a license renewal this July} from the State Council, China's
cabinet, that allows it to continue providing financial news for two
more years.

Other news organizations have come under similar pressure. The websites
of The New York Times, including a new Chinese-language edition, were
blocked when it published
\href{http://www.nytimes3xbfgragh.onion/2012/10/26/business/global/family-of-wen-jiabao-holds-a-hidden-fortune-in-china.html}{an
article in October 2012} on the family wealth of Wen Jiabao, then the
prime minister. Like Bloomberg, The Times has also not received
residency visas for new journalists.

In recent years, some editors at Bloomberg have encouraged reporters to
tackle ambitious investigative reports, in order to broaden the company
beyond its foundation as a speed-driven financial news service. At
times, that aggressiveness has resulted in ethical breaches, as when
Bloomberg was forced to disclose in May that its journalists had
\href{http://www.nytimes3xbfgragh.onion/2013/05/13/business/media/bloomberg-admits-terminal-snooping.html}{gained
access to the log-in data of terminal customers} to gain an edge in
reporting.

But the investigative work has also won top prizes, most notably for the
China family wealth series in 2012. Two of the main writers on that
series, Michael Forsythe and Shai Oster, were the lead reporters on the
recent tycoon story.

Editors at Bloomberg have long been aware of the need to tread carefully
in China. A system has been in place that allows editors to add an
internal prepublication code to some articles to ensure that they do not
appear on terminals in China, two employees said. This has been used
regularly with articles on Chinese politics, including the one on Mr.
Xi's family.

The debates within Bloomberg over the two recent China stories have
taken place right before scheduled trips to China by two senior
Bloomberg figures. Daniel L. Doctoroff, the chief executive of Bloomberg
L.P., the parent company, is expected to travel to China in the coming
weeks, employees said. The company's billionaire founder, Michael R.
Bloomberg, told Forbes this fall that he plans to go to China soon after
stepping down as New York City's mayor in January to ``give some
speeches on behalf of the company.''

Bloomberg News has already come under criticism as word of the uncertain
fate of the investigative China articles has slowly leaked out in recent
days. An animation arm of Next Media, a powerful Hong Kong media company
critical of the Chinese Communist Party, released an
\href{http://www.youtube.com/watch?v=DQGMLlihZ1I\&feature=youtu.be}{online
video cartoon} on Friday evening mocking Bloomberg for self-censorship.

The turmoil since October was described to The Times by four Bloomberg
employees who spoke on the condition of anonymity for fear of losing
their jobs.

The recent article by Mr. Forsythe and Mr. Oster was focused on a
Chinese billionaire entrepreneur who had financial ties to relatives of
current and former members of the Standing Committee of the ruling
Politburo, the top political body in China, said employees who had read
the article.

Until late October, no editor had raised any serious objections to the
tycoon story, even though it had gone through extensive editing and
fact-checking, the employees said. The final editing stage began in
September. Two senior editors in New York, Ms. Hays and Jonathan
Kaufman, shepherded the story and were enthusiastic about it, employees
said. So was a company lawyer who, after reviewing it in early October,
suggested some minor wording changes.

Mr. Winkler also looked at the article and made some small suggestions.

Mr. Kaufman flew to Hong Kong in October and raised no serious
objections to the article, employees said. From mid-October onward, the
reporters and editors in Hong Kong did not hear much from New York. Then
Ms. Hays and Mr. Kaufman told editors in Hong Kong that the story would
not be published, employees said. The next day, Mr. Forsythe and Mr.
Oster, the story's two main reporters, took part in a conference call
with New York. The editors there said that the article had no ``smoking
gun,'' that billionaires around the world had close ties to governments,
and that the article did not provide enough new information beyond what
Bloomberg had reported in its 2012 series, employees said.

``They were adamant that the reasons for killing the story were
editorial reasons, not political reasons,'' an employee added. When Ms.
Hays was asked who had made the decision to shelve the article, she said
she and four other editors had, including Tim Quinson, who in September
was assigned to the new position of standards editor after an internal
review of the May reporting scandal. Ms. Hays said Mr. Doctoroff, the
chief executive, had not seen the article.

Mr. Winkler, the top editor, then spoke to four reporters and editors in
Hong Kong on a final conference call, which was held on the night of
Oct. 29.

The strongest reason he presented was the possibility of Bloomberg's
being evicted from China, employees said.

``He was speaking from a news perspective, not a sales perspective,'' an
employee said. Mr. Winkler also said he had read a lot about foreign
journalists working under the Third Reich and wanted to formulate a
strategy for staying in China as long as possible. ``He said he was
looking at the example of how news organizations worked in Nazi Germany,
how they were able to stay there, how they were able to write in that
environment,'' the employee said.

The message was clear to those in Hong Kong: There was no chance of
publication for now.

An article by Cathy Chan, another reporter in Hong Kong, ran into
similar problems within a week. The article was halted after a
conference call with New York, employees said. The article outlined how
children of Chinese leaders, or ``princelings,'' had secured jobs at
foreign banks. The hiring practice has come under scrutiny: In August,
American newspapers reported that the Securities and Exchange Commission
was
\href{http://dealbook.nytimes3xbfgragh.onion/2013/08/20/many-wall-st-banks-woo-children-of-chinese-leaders/?_r=0}{investigating
whether JPMorgan Chase} had hired the children of senior Chinese
officials to win business in China.

Ms. Chan had been told to get more documentary evidence for the article
and rely less on human sources, said one employee, even though that
demand is very difficult for such reporting. The employee said, ``Some
people at the top are setting a much higher bar for stories.''

Advertisement

\protect\hyperlink{after-bottom}{Continue reading the main story}

\hypertarget{site-index}{%
\subsection{Site Index}\label{site-index}}

\hypertarget{site-information-navigation}{%
\subsection{Site Information
Navigation}\label{site-information-navigation}}

\begin{itemize}
\tightlist
\item
  \href{https://help.nytimes3xbfgragh.onion/hc/en-us/articles/115014792127-Copyright-notice}{©~2020~The
  New York Times Company}
\end{itemize}

\begin{itemize}
\tightlist
\item
  \href{https://www.nytco.com/}{NYTCo}
\item
  \href{https://help.nytimes3xbfgragh.onion/hc/en-us/articles/115015385887-Contact-Us}{Contact
  Us}
\item
  \href{https://www.nytco.com/careers/}{Work with us}
\item
  \href{https://nytmediakit.com/}{Advertise}
\item
  \href{http://www.tbrandstudio.com/}{T Brand Studio}
\item
  \href{https://www.nytimes3xbfgragh.onion/privacy/cookie-policy\#how-do-i-manage-trackers}{Your
  Ad Choices}
\item
  \href{https://www.nytimes3xbfgragh.onion/privacy}{Privacy}
\item
  \href{https://help.nytimes3xbfgragh.onion/hc/en-us/articles/115014893428-Terms-of-service}{Terms
  of Service}
\item
  \href{https://help.nytimes3xbfgragh.onion/hc/en-us/articles/115014893968-Terms-of-sale}{Terms
  of Sale}
\item
  \href{https://spiderbites.nytimes3xbfgragh.onion}{Site Map}
\item
  \href{https://help.nytimes3xbfgragh.onion/hc/en-us}{Help}
\item
  \href{https://www.nytimes3xbfgragh.onion/subscription?campaignId=37WXW}{Subscriptions}
\end{itemize}
