Sections

SEARCH

\protect\hyperlink{site-content}{Skip to
content}\protect\hyperlink{site-index}{Skip to site index}

\href{https://myaccount.nytimes3xbfgragh.onion/auth/login?response_type=cookie\&client_id=vi}{}

\href{https://www.nytimes3xbfgragh.onion/section/todayspaper}{Today's
Paper}

\begin{itemize}
\item
  \href{https://www.nytimes3xbfgragh.onion/live/2020/09/08/us/trump-vs-biden?action=click\&pgtype=Article\&state=default\&region=TOP_BANNER\&context=storylines_menu}{Election
  Updates}
\item
  \href{https://www.nytimes3xbfgragh.onion/interactive/2020/us/elections/election-states-biden-trump.html?action=click\&pgtype=Article\&state=default\&region=TOP_BANNER\&context=storylines_menu}{Paths
  to 270}
\item
  \href{https://www.nytimes3xbfgragh.onion/interactive/2020/08/31/us/politics/vote-by-mail-deadlines.html?action=click\&pgtype=Article\&state=default\&region=TOP_BANNER\&context=storylines_menu}{Voting
  by Mail}
\item
  \href{https://www.nytimes3xbfgragh.onion/interactive/2019/us/elections/2020-presidential-election-calendar.html?action=click\&pgtype=Article\&state=default\&region=TOP_BANNER\&context=storylines_menu}{Key
  Dates}
\item
  \href{https://www.nytimes3xbfgragh.onion/newsletters/politics?action=click\&pgtype=Article\&state=default\&region=TOP_BANNER\&context=storylines_menu}{Politics
  Newsletter}
\end{itemize}

\hypertarget{coronavirus-impact-how-a-crisis-is-changing-the-us}{%
\section{Coronavirus Impact: How a Crisis Is Changing the
U.S.}\label{coronavirus-impact-how-a-crisis-is-changing-the-us}}

Last Updated

June 9, 2020, 7:05 a.m. ET

June 9, 2020, 7:05 a.m. ET

\includegraphics{https://static01.graylady3jvrrxbe.onion/images/2020/03/27/multimedia/27impact-header2/merlin_171040551_8d8090c3-b23c-44fb-a634-1931a5e741ae-articleLarge.jpg?quality=75\&auto=webp\&disable=upscale}

The coronavirus is changing how we live our daily lives. Taking a look
at how the global pandemic has affected various aspects of life in the
United States reveals the unique nature of this crisis.

\begin{itemize}
\item
  In the latest installment of ``New York Shuttered,'' the photographer
  Todd Heisler was there for a
  \href{https://www.nytimes3xbfgragh.onion/2020/03/27/nyregion/queens-horses-coronavirus.html}{tearful
  goodbye} as horses used for therapeutic riding lessons were sent
  upstate.
\item
  Hoda Kotb, the co-anchor of ``Today,''
  \href{https://www.nytimes3xbfgragh.onion/live/2020/coronavirus-usa-03-27\#today-anchors-on-air-emotions-resonate-for-many}{showed
  the same emotions many people at home are feeling} in a brief on-air
  breakdown that resonated for many online.
\item
  In an adaptation to the coronavirus,
  \href{https://www.nytimes3xbfgragh.onion/2020/03/27/style/black-girl-women-hair-styling-live-tutorials-coronavirus.html}{black
  women are turning to online tutorials} for advice on how to braid
  their own hair.
\end{itemize}

\hypertarget{heres-what-you-need-to-know}{%
\subsubsection{Here's what you need to
know:}\label{heres-what-you-need-to-know}}

\begin{itemize}
\item
  March 27, 2020, 5:45 p.m. ET

  \protect\hyperlink{after-12-days-on-a-ventilator-a-chance-to-facetime-hello}{}

  After 12 days on a ventilator, a chance to FaceTime hello.
\item
  March 27, 2020, 5:00 p.m. ET

  \protect\hyperlink{a-small-shop-in-rhode-island-keeps-quahoggers-out-on-the-water}{}

  A small shop in Rhode Island keeps quahoggers out on the water.
\item
  March 27, 2020, 3:30 p.m. ET

  \protect\hyperlink{for-college-seniors-the-job-market-adds-to-frustrations}{}

  For college seniors, the job market adds to frustrations.
\end{itemize}

\includegraphics{https://static01.graylady3jvrrxbe.onion/images/icons/t_logo_291_black.png}

March 27, 2020, 6:30 p.m. ET

March 27, 2020, 6:30 p.m. ET

By The New York Times

\hypertarget{we-want-to-see-your-acts-of-kindness-in-a-difficult-time}{%
\subsection{\texorpdfstring{\protect\hyperlink{we-want-to-see-your-acts-of-kindness-in-a-difficult-time}{We
want to see your acts of kindness in a difficult
time.}}{We want to see your acts of kindness in a difficult time.}}\label{we-want-to-see-your-acts-of-kindness-in-a-difficult-time}}

``Love each other,'' someone recently wrote in pink chalk on a Manhattan
sidewalk.

It's a simple command. But even the smallest gestures, like offering a
hug or watching someone's child, now
\href{https://www.nytimes3xbfgragh.onion/2020/03/24/nyregion/ny-coronavirus-help.html}{carries
health risks} as communities struggle to contain the spread of the
coronavirus.

But people are finding a way.

\href{https://www.nytimes3xbfgragh.onion/2020/03/25/business/coronavirus-masks-sewers.html}{An
army of sewers} across the country has banded together to make
desperately needed masks for hospital workers. In rural Oregon, bus
drivers are delivering lunches in
\href{https://time.com/5808475/coronavirus-school-lunches-bus-drivers/}{brown
paper bags to children in need}. Landlords are
\href{https://nypost.com/2020/03/22/milwaukee-landlord-slashes-rent-to-100-amid-coronavirus-outbreak/}{reducing
rent}. College students are
\href{https://www.nytimes3xbfgragh.onion/2020/03/24/nyregion/ny-coronavirus-help.html}{fetching
groceries for older neighbors} or those with compromised immune systems
--- and wiping them down with bleach before depositing them on a
doorstep. People are rescuing
\href{https://www.nytimes3xbfgragh.onion/2020/03/19/us/coronavirus-foster-pets.html}{cats
and dogs from shelters}.

To make it through overwhelming and uncertain times, experts say
\href{https://www.nytimes3xbfgragh.onion/2020/03/27/us/coronavirus-good-news-kindness.html}{it's
critical for people to remember there are glimmers of} hope and
goodness.

We would like to hear from you.

\textbf{What acts of kindness have you witnessed or showed?}

Please take a video of yourself answering the following questions:

\begin{itemize}
\item
  Tell us your name, age, and the city or town and state where you're
  living, and who you are living with in your house or apartment.
\item
  How has your community been impacted by the spread of the virus?
\item
  Tell us one thoughtful thing you've done for your neighbors or family,
  or one act of generosity that has been done for you or someone else
  during the coronavirus outbreak.
\item
  How did that act change your overall feelings about your personal
  situation, or your community's situation?
\end{itemize}

\textbf{When taking your video, please:}

\begin{itemize}
\item
  Shoot vertically, leaving room at the top of the frame.
\item
  Keep your response to each question under 15 seconds.
\item
  Make sure you're in a brightly lit area (but not too bright).
\item
  Record somewhere with limited background noise so we can hear you
  clearly.
\end{itemize}

We may feature your video in an
\href{https://www.instagram.com/nytimes/}{Instagram Story on @nytimes}.
To participate, please upload your video and complete the form below.

Read more

\href{https://www.nytimes3xbfgragh.onion/by/corey-kilgannon}{\includegraphics{https://static01.graylady3jvrrxbe.onion/images/2018/02/20/multimedia/author-corey-kilgannon/author-corey-kilgannon-thumbLarge.jpg}}

March 27, 2020, 6:00 p.m. ET

March 27, 2020, 6:00 p.m. ET

By \href{https://www.nytimes3xbfgragh.onion/by/corey-kilgannon}{Corey
Kilgannon}

\hypertarget{this-is-my-piece-of-paradise}{%
\subsection{\texorpdfstring{\protect\hyperlink{this-is-my-piece-of-paradise}{`This
is my piece of
paradise.'}}{`This is my piece of paradise.'}}\label{this-is-my-piece-of-paradise}}

\includegraphics{https://static01.graylady3jvrrxbe.onion/images/2020/03/21/nyregion/00nyvirus-dogwalking40/00nyvirus-dogwalking40-articleLarge.jpg?quality=75\&auto=webp\&disable=upscale}

The coronavirus has warped life in New York City, which has 23,000 cases
and at least 365 deaths, making it the epicenter of the outbreak in the
United States. For some city dwellers, the necessary act of walking the
dog has become a glimmer of solace during a dark time.

``We're bombarded with gloom and doom every minute on the TV, but this
is my piece of paradise,'' said Roberta Strugger, who recently watched
her Labradoodle, Harvey, romp in a dog run in the Bronx.

Professional dog walkers, however, are experiencing much more troubling
consequences from this scenario, including loss of income and jobs.

\hypertarget{advertisement}{%
\subsubsection{Advertisement}\label{advertisement}}

\protect\hyperlink{after-dfp-ad-mid1}{Continue reading the main story}

\href{https://www.nytimes3xbfgragh.onion/by/elaina-plott}{\includegraphics{https://static01.graylady3jvrrxbe.onion/images/2020/02/19/reader-center/author-Elaina-Plott/author-Elaina-Plott-thumbLarge.png}}

March 27, 2020, 5:45 p.m. ET

March 27, 2020, 5:45 p.m. ET

By \href{https://www.nytimes3xbfgragh.onion/by/elaina-plott}{Elaina
Plott}

\hypertarget{after-12-days-on-a-ventilator-a-chance-to-facetime-hello}{%
\subsection{\texorpdfstring{\protect\hyperlink{after-12-days-on-a-ventilator-a-chance-to-facetime-hello}{After
12 days on a ventilator, a chance to FaceTime
hello.}}{After 12 days on a ventilator, a chance to FaceTime hello.}}\label{after-12-days-on-a-ventilator-a-chance-to-facetime-hello}}

\includegraphics{https://static01.graylady3jvrrxbe.onion/images/2020/03/19/us/politics/19louisiana-family-lives-blog/19louisiana-family-03-articleLarge.jpg?quality=75\&auto=webp\&disable=upscale}

On Friday, for the first time in the 12 days since he was diagnosed with
the coronavirus, Mark Frilot got to speak to his family.

For 12 days he had been hooked up to a ventilator in the intensive care
unit of a Kenner, La., hospital, his wife, Heaven, and their son, Ethan,
quarantined in their home nearby.

On Friday afternoon, doctors were at last able to remove the ventilator.
``The nurse FaceTimed us and we talked to him!'' Ms. Frilot shared in a
text message. ``He's himself joking with us already. Words cannot
express my joy!'' (She added that Ethan made sure to tell his dad it was
time to shave.)

Ms. Frilot said that a two-month recovery likely awaits as doctors
continue to treat his pneumonia, a journey that will be followed by far
more than just his family: Since first sharing her husband's story, Ms.
Frilot has become a light of sorts for her conservative community in
Louisiana and beyond, in which many had written off the pandemic as
partisan fear-mongering.

She's spent the last several days responding to dozens of messages from
strangers whose families are undergoing their own trials with the virus.
For now, though, she just wants to relish in her husband's improvement.
``Right now I'm soaking in that my hubby is alive, awake and himself,''
she said.

Read more

\href{https://www.nytimes3xbfgragh.onion/by/elizabeth-paton}{\includegraphics{https://static01.graylady3jvrrxbe.onion/images/2019/12/05/reader-center/author-elizabeth-paton/author-elizabeth-paton-thumbLarge.png}}

March 27, 2020, 5:30 p.m. ET

March 27, 2020, 5:30 p.m. ET

By
\href{https://www.nytimes3xbfgragh.onion/by/elizabeth-paton}{Elizabeth
Paton}

\hypertarget{luxury-brands-are-boarding-up-their-stores}{%
\subsection{\texorpdfstring{\protect\hyperlink{luxury-brands-are-boarding-up-their-stores}{Luxury
brands are boarding up their
stores.}}{Luxury brands are boarding up their stores.}}\label{luxury-brands-are-boarding-up-their-stores}}

\includegraphics{https://static01.graylady3jvrrxbe.onion/images/2020/03/28/business/28virus-luxurystores1-refer-print/merlin_170948553_73d4e1f9-a9d3-476c-b7c0-3798b8959757-articleLarge.jpg?quality=75\&auto=webp\&disable=upscale}

In Shanghai, day-to-day life for many luxury retailers has started ---
slowly --- returning to normal. In Europe, where millions of citizens
have been
\href{https://www.nytimes3xbfgragh.onion/2020/03/17/world/europe/paris-coronavirus-lockdown.html}{living
under national shutdowns} for more than a week, stores in famous retail
destinations have bolted their doors. In London, department stores like
Harrods and Selfridges, and Bond Street boutiques like Burberry and
Chopard, have cleared jewels and stock from plain sight.

But in New York, where the cobbled streets of SoHo have shuddered to a
standstill as state measures to slow the spread of the virus have taken
hold, a number of elegant luxury boutiques, including Fendi, Celine and
Chanel, did not just shutter storefronts this week; they had them
boarded up with vast sheets of plywood, as if in anticipation of riots
and civil disobedience, similar to how they react to
\href{https://www.nytimes3xbfgragh.onion/2018/12/17/business/paris-yellow-vests-luxury-retail.html}{European
protests}. But some are cautioning against the practice.

``Boarding up your storefront makes it so that people on the street
can't see inside,'' said Mark Dicus, the executive director of the SoHo
Broadway Initiative business improvement district. ``That might be more
appealing to those looking for break in opportunities.''

\href{https://www.nytimes3xbfgragh.onion/by/stefanos-chen}{\includegraphics{https://static01.graylady3jvrrxbe.onion/images/2018/06/13/multimedia/author-stefanos-chen/author-stefanos-chen-thumbLarge-v2.png}}

March 27, 2020, 5:15 p.m. ET

March 27, 2020, 5:15 p.m. ET

By \href{https://www.nytimes3xbfgragh.onion/by/stefanos-chen}{Stefanos
Chen}

\hypertarget{real-estate-listings-in-manhattan-plummet}{%
\subsection{\texorpdfstring{\protect\hyperlink{real-estate-listings-in-manhattan-plummet}{Real
estate listings in Manhattan
plummet.}}{Real estate listings in Manhattan plummet.}}\label{real-estate-listings-in-manhattan-plummet}}

In the first week since New York State announced a stay-at-home order to
help fight coronavirus, real estate listings in Manhattan have plunged
and the spring buying season has ground to a halt.

Since March 20, the day Gov. Andrew M. Cuomo signed the executive order,
just 66 homes were listed for sale in Manhattan, an 85 percent drop
compared to the same period last year, when 428 listings came to market,
according to UrbanDigs, a real-estate data company.

Real estate agents, who have been deemed nonessential workers, have been
unable to schedule showings and in many cases are barred from co-ops and
condos, where the buildings have adopted strict entry policies.

The virus is not only keeping new sellers on the sidelines, but leading
many to pull their listings from public view altogether, said Noah
Rosenblatt, the chief executive and founder of UrbanDigs.

With just a few days left to the month, 1,074 listings had been taken
off the market in Manhattan, compared to just 417 in all of March 2019.
There were 5,882 active listings for sale in Manhattan on March 26, down
12.8 percent from the same time last year.

``If you look at 2009, the market did the same exact thing,'' Mr.
Rosenblatt said, referring to the high number of sellers who simply gave
up when the Great Recession took hold.

``Everything came to a screeching halt last week,'' said Barbara Fox,
the president of Fox Residential, a New York brokerage. While measures
have been taken by the state to ensure that closings can proceed --- for
instance, allowing virtual alternatives for typically in-person
requirements, like appraisals and notarization --- there are still
several steps in the sales process without simple solutions.

``I just can't imagine people are going to be buying apartments from a
video,'' Ms. Fox said, referring to virtual house tours via FaceTime and
other apps.

Agents say the extent of the damage to the real estate industry will
depend largely on how long the stay-at-home protocol is enforced, but
added that the timing is terrible.

The real estate market, especially the high-end, has been softening
since prices peaked around 2016. The first quarter of the year showed
signs of improvement, before the virus arrived, said Jonathan Miller, a
New York real estate appraiser. Many agents expect the second quarter,
typically a bright spot for sellers, to erase those gains.

``It's like a retail store losing Christmas,'' said Mr. Miller. ``That's
really what this is.''

Read more

\hypertarget{advertisement-1}{%
\subsubsection{Advertisement}\label{advertisement-1}}

\protect\hyperlink{after-dfp-ad-mid2}{Continue reading the main story}

\href{https://www.nytimes3xbfgragh.onion/by/c-j-chivers}{\includegraphics{https://static01.graylady3jvrrxbe.onion/images/2018/07/12/multimedia/author-c-j-chivers/author-c-j-chivers-thumbLarge.png}}

March 27, 2020, 5:00 p.m. ET

March 27, 2020, 5:00 p.m. ET

By \href{https://www.nytimes3xbfgragh.onion/by/c-j-chivers}{C. J.
Chivers}

\hypertarget{a-small-shop-in-rhode-island-keeps-quahoggers-out-on-the-water}{%
\subsection{\texorpdfstring{\protect\hyperlink{a-small-shop-in-rhode-island-keeps-quahoggers-out-on-the-water}{A
small shop in Rhode Island keeps quahoggers out on the
water.}}{A small shop in Rhode Island keeps quahoggers out on the water.}}\label{a-small-shop-in-rhode-island-keeps-quahoggers-out-on-the-water}}

\includegraphics{https://static01.graylady3jvrrxbe.onion/images/2020/04/01/dining/27virus-clams5/merlin_171027348_9894c810-e554-480f-9f22-821e2eb02172-articleLarge.jpg?quality=75\&auto=webp\&disable=upscale}

Many fishing ports across the United States, long imperiled and
struggling under strict regulations and the declines of valuable fish
and shellfish stocks, have fallen even quieter during the pandemic.

For Rhode Island's quahoggers, as the harvesters of wild hard-shelled
clams are known, the circumstances have gone past difficult to bizarre.
While their neighbors struggled to buy food during surges of panic
shopping that emptied grocery store shelves, quahoggers found the market
for fresh clams --- a food rich in protein and minerals --- abruptly
shut down.

In Rhode Island, where state regulations forbid quahoggers from selling
clams directly to consumers, the result is that the fleet has all but
stopped working --- even though catches were high and people, wary of
going into crowded and picked-over grocery stores, are eager for healthy
meals.

Andrade's Catch in Bristol has managed to support quahog sales, at least
at a small scale. While the shop does a robust wholesale business, it
also runs a retail shop out front. By shifting operations almost
entirely to retail, it has kept a few boats on the water.

``I've got about six guys I am buying from,'' Mr. Andrade said, and he
rotates their days. ``We want to keep the guys going.''

Read more

\href{https://www.nytimes3xbfgragh.onion/by/hilarie-m-sheets}{\includegraphics{https://static01.graylady3jvrrxbe.onion/images/2019/04/03/multimedia/author-hilarie-m-sheets/author-hilarie-m-sheets-thumbLarge.png}}

March 27, 2020, 4:30 p.m. ET

March 27, 2020, 4:30 p.m. ET

By \href{https://www.nytimes3xbfgragh.onion/by/hilarie-m-sheets}{Hilarie
M. Sheets}

\hypertarget{an-artist-hopes-a-communal-project-can-weave-us-all-together}{%
\subsection{\texorpdfstring{\protect\hyperlink{an-artist-hopes-a-communal-project-can-weave-us-all-together}{An
artist hopes a communal project can weave us all
together.}}{An artist hopes a communal project can weave us all together.}}\label{an-artist-hopes-a-communal-project-can-weave-us-all-together}}

\includegraphics{https://static01.graylady3jvrrxbe.onion/images/2020/03/31/arts/27apartogether-virus1/27apartogether-virus1-articleLarge-v2.jpg?quality=75\&auto=webp\&disable=upscale}

Looking to create beauty and build community in the time of social
distancing, the artist \href{http://lizalou.com/}{Liza Lou} is inviting
other artists along with the general public to join her in a communal
art project called
\href{http://www.apartogether.com/}{``Apartogether.''} She introduced
the concept on her Instagram page last week, cuing people to begin
gathering old clothes and materials around the house from which to piece
together a quilt or what she's calling a ``comfort blanket.'' (Ms. Lou
\href{https://www.instagram.com/p/B969GxWFGqu/}{showed herself hugging
her own baby blanket}.)

``The idea that an object can protect is, of course, a childlike idea,''
she said in her posted video. ``I think that making is a form of
protection.'' Known for her monumental sculptures and wall pieces
encrusted with mosaics of individually applied beads, the 50-year-old
artist has long explored the meaning found in process and labor
traditionally associated with craft and performed by women.

Ms. Lou is rolling out more details of ``Apartogether'' in a virtual
studio visit on Friday at 12:00 EST on Instagram Live, where viewers can
comment and ask questions. From this hub, using the handle
\href{https://www.instagram.com/liza_lou_studio/?hl=en}{@liza\_lou\_studio},
she will post regular prompts and live videos over the coming weeks. She
is encouraging people to share their progress by tagging it
@apartogether\_art so that it can be seen and archived on the website
apartogether.com. She hopes that groups will gather on Zoom to talk and
work on their projects in real time.

\href{https://www.nytimes3xbfgragh.onion/by/jodi-kantor}{\includegraphics{https://static01.graylady3jvrrxbe.onion/images/2018/02/16/multimedia/author-jodi-kantor/author-jodi-kantor-thumbLarge-v2.png}}

March 27, 2020, 4:00 p.m. ET

March 27, 2020, 4:00 p.m. ET

By \href{https://www.nytimes3xbfgragh.onion/by/jodi-kantor}{Jodi Kantor}

\hypertarget{the-debate-about-going-outside-is-intensifying}{%
\subsection{\texorpdfstring{\protect\hyperlink{the-debate-about-going-outside-is-intensifying}{The
debate about going outside is
intensifying.}}{The debate about going outside is intensifying.}}\label{the-debate-about-going-outside-is-intensifying}}

\includegraphics{https://static01.graylady3jvrrxbe.onion/images/2020/03/28/multimedia/27dilemmas-01/merlin_170976999_de5b2a92-274c-4d81-b885-2966606f9636-articleLarge.jpg?quality=75\&auto=webp\&disable=upscale}

\emph{In the Dilemmas series, Jodi Kantor is helping answer questions
from readers about how to deal with the changes to our life as a result
of the coronavirus.}

Barri Motola, a reader from New York City, wrote: **

\begin{quote}
I'm 77 years old and I want/need to walk. The two buildings in my
complex have a basketball court between them. I have previously taken
the freight elevator down 36 stories at 5:30 a.m., meeting no one but
armed anyway with mask, gloves, wipes and hand sanitizer. I walked for
35 minutes and went back upstairs, again meeting nobody. Should I force
myself to continue? I am simply afraid to go outside.
\end{quote}

Ms. Motola's world has mostly shrunk to one room. She lives by herself
in a studio apartment high above Manhattan, with a piano, books and a
narrowing set of routines. Her longtime habit of swimming laps is on
pause. So are her dates with her children and grandchildren.

``The walking was truly helping me keep it together,'' she said on the
telephone. But she stopped a week ago and hasn't left her building
since. ``As this ramped up, I kept weighing anything and everything I
was thinking about doing outside, and saying: `Is it worth getting sick
for? Is it worth dying for?'''

She's not the only one asking. The outdoors is now contested ground.
\href{https://www.latimes.com/travel/story/2020-03-23/where-to-go-outside-in-southern-california-as-options-decrease}{Parks}
and trails from Los Angeles to the Great Smokies are being closed. (Too
many people were socially distancing in the same places, and therefore
not at all.) Authorities are patrolling others,
\href{https://www.seattletimes.com/seattle-news/health/outdoor-crowds-test-the-limits-of-social-distancing-in-fight-against-coronavirus/?utm_source=marketingcloud\&utm_medium=email\&utm_campaign=TSA_032320143559\%20Police\%20turn\%20loudspeakers\%20on\%20crowds\%20at\%20parks.\%20What\%EF\%BF\%BD\%EF\%BF\%BD\%EF\%BF\%BDs\%20next_3_23_2020\&utm_term=Active\%20subscriber\&mod=article_inline}{warning}
people to disperse. This week, India's prime minister told 1.3 billion
people not to
\href{https://www.nytimes3xbfgragh.onion/2020/03/25/world/asia/india-lockdown-coronavirus.html}{set
foot} outside their homes. ``Stay Home Save Lives'' has become a
rallying cry and a pressure point on social media.

Read more

\hypertarget{advertisement-2}{%
\subsubsection{Advertisement}\label{advertisement-2}}

\protect\hyperlink{after-dfp-ad-mid3}{Continue reading the main story}

\includegraphics{https://static01.graylady3jvrrxbe.onion/images/icons/t_logo_291_black.png}

March 27, 2020, 3:30 p.m. ET

March 27, 2020, 3:30 p.m. ET

By \href{http://www.nytimes3xbfgragh.onion/by/david-yaffe-bellany}{David
Yaffe-Bellany} and
\href{https://www.nytimes3xbfgragh.onion/by/jaclyn-peiser}{Jaclyn
Peiser}

\hypertarget{for-college-seniors-the-job-market-adds-to-frustrations}{%
\subsection{\texorpdfstring{\protect\hyperlink{for-college-seniors-the-job-market-adds-to-frustrations}{For
college seniors, the job market adds to
frustrations.}}{For college seniors, the job market adds to frustrations.}}\label{for-college-seniors-the-job-market-adds-to-frustrations}}

\includegraphics{https://static01.graylady3jvrrxbe.onion/images/2020/03/29/business/27VIRUS-CLASSOF2020-02/merlin_171000612_bcc4703d-75f1-4ab0-83dd-798d28d30666-articleLarge.jpg?quality=75\&auto=webp\&disable=upscale}

They hoped to secure jobs on political campaigns, at fashion brands and
law offices, and in sales and finance. Instead, they've had internships
canceled and interviews postponed, wandered through empty job fairs and
seen recruiters ignore their anxious emails.

When the coronavirus pandemic forced college students across the country
to leave campus in early March, the abrupt departure was especially
painful for seniors. It meant rushed goodbyes, canceled graduation
ceremonies
---\href{https://www.nytimes3xbfgragh.onion/2020/03/15/nyregion/cornell-university-coronavirus.html}{an
overwhelming sense of loss}.

Now, many of those seniors are home with their families, contemplating
an even worse prospect: a job market more grim than any in recent
history. Last week, according to the Labor Department,
\href{https://www.nytimes3xbfgragh.onion/2020/03/26/business/economy/coronavirus-unemployment-claims.html}{nearly
3.3 million people} filed for unemployment benefits, more than quadruple
the previous record.

\includegraphics{https://static01.graylady3jvrrxbe.onion/images/icons/t_logo_291_black.png}

March 27, 2020, 2:00 p.m. ET

March 27, 2020, 2:00 p.m. ET

By \href{https://www.nytimes3xbfgragh.onion/by/sandra-e-garcia}{Sandra
E. Garcia}

\hypertarget{online-videos-help-black-women-learn-to-braid-while-social-distancing}{%
\subsection{\texorpdfstring{\protect\hyperlink{online-videos-help-black-women-learn-to-braid-while-social-distancing}{Online
videos help black women learn to braid while social
distancing.}}{Online videos help black women learn to braid while social distancing.}}\label{online-videos-help-black-women-learn-to-braid-while-social-distancing}}

\includegraphics{https://static01.graylady3jvrrxbe.onion/images/2020/03/26/us/26xp-virus-blackhair-1/26xp-virus-blackhair-1-articleLarge.jpg?quality=75\&auto=webp\&disable=upscale}

Niani Barracks usually tends to clients at a salon in Detroit, but now
that she must stay indoors because of the coronavirus pandemic, she has
been running her fingers through the hair of a mannequin head affixed to
a stand in her home, as a dozen other black women, who paid \$5 each,
watch her on Facebook Live.

In one video, Ms. Barracks gently cradles three strands of hair between
her fingers as she explains how to start a braid.

The skill is essential for many black women trying to keep their hair
healthy while they practice social distancing. Braids are the foundation
of many protective hairstyles, like wigs and hair extensions.

With nonessential businesses closing and nearly two dozen states urging
at least
\href{https://www.nytimes3xbfgragh.onion/interactive/2020/us/coronavirus-stay-at-home-order.html}{212
million Americans to stay home}, Facebook has experienced a sharp
increase in the use of its Live feature, which lets users broadcast
videos. Most of the students in Ms. Barracks's class are black women
hoping to learn how to braid while salons and barbershops have shuttered
to prevent the spread of the coronavirus.

After she started staying home with her son when his school closed, Ms.
Barracks got the idea to start the class.

``There were some moments of anxiety when I realized I don't have
another job and that I won't be making any money,'' Ms. Barracks said.
``Everything started shutting down except the bills.''

Read more

\href{https://www.nytimes3xbfgragh.onion/by/benjamin-hoffman}{\includegraphics{https://static01.graylady3jvrrxbe.onion/images/2018/10/17/multimedia/author-benjamin-hoffman/author-benjamin-hoffman-thumbLarge.png}}

March 27, 2020, 1:32 p.m. ET

March 27, 2020, 1:32 p.m. ET

By
\href{https://www.nytimes3xbfgragh.onion/by/benjamin-hoffman}{Benjamin
Hoffman}

\hypertarget{today-anchors-on-air-emotions-resonate-for-many}{%
\subsection{\texorpdfstring{\protect\hyperlink{today-anchors-on-air-emotions-resonate-for-many}{`Today'
anchor's on-air emotions resonate for
many.}}{`Today' anchor's on-air emotions resonate for many.}}\label{today-anchors-on-air-emotions-resonate-for-many}}

\begin{quote}
We love you, Hoda. ❤️
\href{https://t.co/rocZr8J4EE}{pic.twitter.com/rocZr8J4EE}

--- TODAY (@TODAYshow)
\href{https://twitter.com/TODAYshow/status/1243505483568353281?ref_src=twsrc\%5Etfw}{March
27, 2020}
\end{quote}

In a moment that was instantly relatable for people who are struggling
emotionally with the coronavirus pandemic, Hoda Kotb, one of the
co-anchors of ``Today,'' broke down in tears on Friday, shortly after
interviewing Drew Brees, the quarterback of the New Orleans Saints, who
was appearing on the show to discuss his charitable work.

Ms. Kotb discussed with Mr. Brees that he and his wife, Brittany, had
donated \$5 million to the state of Louisiana to help with the outbreak
that has been particularly severe in New Orleans. But shortly after she
told Mr. Brees, ``We love ya,'' she was briefly unable to continue. Her
co-anchor, Savannah Guthrie, quickly jumped in to tell Ms. Kotb to take
a moment, and a video of the interaction was tweeted out by ``Today''
and drew an intense reaction online.

\begin{quote}
May sound weird, but Hoda Kotb crying on the
\href{https://twitter.com/TODAYshow?ref_src=twsrc\%5Etfw}{@TODAYshow}
after interviewing Drew Brees was what I needed this morning. She kept
apologizing. She didn't need to. We're all in this, and it's hurting us
all.

--- Ed Bottomley (@EdBottomley)
\href{https://twitter.com/EdBottomley/status/1243504613065703425?ref_src=twsrc\%5Etfw}{March
27, 2020}
\end{quote}

\begin{quote}
Watching
\href{https://twitter.com/hodakotb?ref_src=twsrc\%5Etfw}{@hodakotb} show
emotion after a piece on her home is the most human thing I've seen. We
love you Hoda!
\href{https://twitter.com/TODAYshow?ref_src=twsrc\%5Etfw}{@TODAYshow}

--- Michelle Petrovic (@Michelle\_Petrov)
\href{https://twitter.com/Michelle_Petrov/status/1243503622987616256?ref_src=twsrc\%5Etfw}{March
27, 2020}
\end{quote}

\begin{quote}
Watching the today show and Hoda just finished a segment with Drew Brees
and New Orleans and football and just absolutely lost it crying. The
worst is knowing how many people could use a hug and not being able to
do anything.

--- Caitlin Moroney (@cmjmoroney)
\href{https://twitter.com/cmjmoroney/status/1243503568809799686?ref_src=twsrc\%5Etfw}{March
27, 2020}
\end{quote}

Part of Ms. Kotb's reaction likely was a result of her connection to New
Orleans, where she worked for a number of years. As The Times reported
on Thursday,
\href{https://www.nytimes3xbfgragh.onion/2020/03/26/us/coronavirus-louisiana-new-orleans.html}{the
city has been inundated by coronavirus cases}, with the issue likely
exacerbated by recent Mardi Gras celebrations.

Read more

\hypertarget{advertisement-3}{%
\subsubsection{Advertisement}\label{advertisement-3}}

\protect\hyperlink{after-dfp-ad-mid4}{Continue reading the main story}

\href{https://www.nytimes3xbfgragh.onion/by/kate-conger}{\includegraphics{https://static01.graylady3jvrrxbe.onion/images/icons/t_logo_291_black.png}}\href{https://www.nytimes3xbfgragh.onion/by/erin-griffith}{\includegraphics{https://static01.graylady3jvrrxbe.onion/images/2019/06/18/reader-center/author-erin-griffith/author-erin-griffith-thumbLarge.png}}

March 27, 2020, 12:30 p.m. ET

March 27, 2020, 12:30 p.m. ET

By \href{https://www.nytimes3xbfgragh.onion/by/kate-conger}{Kate Conger}
and \href{https://www.nytimes3xbfgragh.onion/by/erin-griffith}{Erin
Griffith}

\hypertarget{the-tech-unsavvy-struggle-to-stay-connected}{%
\subsection{\texorpdfstring{\protect\hyperlink{the-tech-unsavvy-struggle-to-stay-connected}{The
tech-unsavvy struggle to stay
connected.}}{The tech-unsavvy struggle to stay connected.}}\label{the-tech-unsavvy-struggle-to-stay-connected}}

As life has increasingly moved online during the pandemic, an older
generation that grew up in an analog era is facing a digital divide.
Often unfamiliar or uncomfortable with apps, gadgets and the internet,
many are struggling to keep up with friends and family through digital
tools when some of them are craving those connections the most.

While teenagers are
\href{https://www.nytimes3xbfgragh.onion/2020/03/17/style/zoom-parties-coronavirus-memes.html}{celebrating
birthdays over Zoom} with one another, children are chatting with
friends over online games and young adults are ordering food via
delivery apps, some older people are intimidated by such technology.
According to a 2017
\href{https://www.pewresearch.org/internet/2017/05/17/barriers-to-adoption-and-attitudes-towards-technology/}{Pew
Research study}, three-quarters of those older than 65 said they needed
someone else to set up their electronic devices. A third also said they
were only a little or not at all confident in their ability to use
electronics and to navigate the web.

That is problematic now when many people 65 and older, who are regarded
by the
\href{https://www.cdc.gov/coronavirus/2019-ncov/specific-groups/high-risk-complications/older-adults.html}{Centers
for Disease Control and Prevention} as most at risk of severe illness
related to the coronavirus, are shutting themselves in. Many
\href{https://www.nytimes3xbfgragh.onion/2020/03/10/us/coronavirus-nursing-homes-washington-seattle.html}{nursing
homes have closed off to visitors} entirely. Yet people are seeking
human interaction and communication through the web or their devices to
stave off loneliness and to stay positive.

Read more

\includegraphics{https://static01.graylady3jvrrxbe.onion/images/icons/t_logo_291_black.png}

March 27, 2020, 12:00 p.m. ET

March 27, 2020, 12:00 p.m. ET

By Jennifer Miller

\hypertarget{it-wasnt-only-toilet-paper}{%
\subsection{\texorpdfstring{\protect\hyperlink{it-wasnt-only-toilet-paper}{It
wasn't `only' toilet
paper.}}{It wasn't `only' toilet paper.}}\label{it-wasnt-only-toilet-paper}}

\includegraphics{https://static01.graylady3jvrrxbe.onion/images/2020/03/29/business/00virus-vignettes1-illo/00virus-vignettes1-illo-articleLarge.jpg?quality=75\&auto=webp\&disable=upscale}

\emph{The}
\href{https://www.nytimes3xbfgragh.onion/coronavirus}{\emph{Covid-19
pandemic}} \emph{is gutting the global economy and forcing entire
nations into quarantine. While reporting on devastation that is
incomprehensibly big, our reporters have been accumulating in their
notebooks some moments that are compelling because they are small.}

On Monday night, three police cars sat outside a CVS in northwest
Washington, lights flashing, while officers stood in the doorway of the
drugstore, watching the unboxing of a delivery of Cottonelle.

The mood among shoppers was calm enough, even collegial. There were
maybe 20, and they took the goods directly from the deliverymen.
``There's no Charmin!'' one said. Another shopper --- my father ---
thought to himself, ``Lady, it's only toilet paper.''

It wasn't ``only'' toilet paper. Not when its delivery required the
cops. Not when holding half a dozen rolls in your arms felt like
pressing a security blanket to your chest.

My dad needed to make sense of things, so he walked up to the police.
``You guys are here just ---''

A policewoman smiled at him. ``Just to ensure that everything is going
to be OK.''

Read more

\includegraphics{https://static01.graylady3jvrrxbe.onion/images/icons/t_logo_291_black.png}

March 27, 2020, 11:00 a.m. ET

March 27, 2020, 11:00 a.m. ET

By \href{https://www.nytimes3xbfgragh.onion/by/todd-heisler}{Todd
Heisler}

\hypertarget{a-tearful-goodbye-for-horses-used-for-therapeutic-riding-lessons}{%
\subsection{\texorpdfstring{\protect\hyperlink{a-tearful-goodbye-for-horses-used-for-therapeutic-riding-lessons}{A
tearful goodbye for horses used for therapeutic riding
lessons.}}{A tearful goodbye for horses used for therapeutic riding lessons.}}\label{a-tearful-goodbye-for-horses-used-for-therapeutic-riding-lessons}}

\includegraphics{https://static01.graylady3jvrrxbe.onion/images/2020/03/27/nyregion/27nyvirus-photo-horses/merlin_170959680_ac9e178d-cd4e-40d7-bbff-71b3d310c69c-articleLarge.jpg?quality=75\&auto=webp\&disable=upscale}

\emph{In the}
\href{https://www.nytimes3xbfgragh.onion/spotlight/new-york-shuttered}{\emph{``New
York Shuttered''}} \emph{series, the photographer Todd Heisler --- with
occasional help from some reporters --- is capturing what it is like to
live in New York City during the coronavirus pandemic.}

Amid the shutdown of large gatherings and nonessential businesses,
Gallop NYC, which specializes in therapeutic riding for people with
emotional, developmental and physical challenges,
\href{https://www.nytimes3xbfgragh.onion/2020/03/27/nyregion/queens-horses-coronavirus.html}{had
to cancel all its programs}. The measures being taken to fight the
spread of coronavirus, specifically the orders to stay at home, are
``going to have a huge toll on people with disabilities,'' said the
executive director, James Wilson.

Image

\hypertarget{advertisement-4}{%
\subsubsection{Advertisement}\label{advertisement-4}}

\protect\hyperlink{after-dfp-ad-mid5}{Continue reading the main story}

\href{https://www.nytimes3xbfgragh.onion/by/john-eligon}{\includegraphics{https://static01.graylady3jvrrxbe.onion/images/2018/06/12/multimedia/author-john-eligon/author-john-eligon-thumbLarge.png}}

March 27, 2020, 10:00 a.m. ET

March 27, 2020, 10:00 a.m. ET

By \href{https://www.nytimes3xbfgragh.onion/by/john-eligon}{John Eligon}

\hypertarget{a-spirit-of-forgiveness-has-emerged-its-longevity-is-in-question}{%
\subsection{\texorpdfstring{\protect\hyperlink{a-spirit-of-forgiveness-has-emerged-its-longevity-is-in-question}{A
spirit of forgiveness has emerged. Its longevity is in
question.}}{A spirit of forgiveness has emerged. Its longevity is in question.}}\label{a-spirit-of-forgiveness-has-emerged-its-longevity-is-in-question}}

\includegraphics{https://static01.graylady3jvrrxbe.onion/images/2020/03/27/us/27virus-forgiveness01/merlin_170884980_b1573e9b-e0e7-45f3-9102-96075ffd0921-articleLarge.jpg?quality=75\&auto=webp\&disable=upscale}

The coronavirus, for all its devastation, is spreading a spirit of
forgiveness across America and softening the country's often
uncompromising lock-'em-up ways.

Dozens of states and localities have suspended evictions and utility
shut-offs.
\href{https://www.nytimes3xbfgragh.onion/2020/03/25/us/politics/coronavirus-senate-deal.html}{The
\$2 trillion stimulus bill} that passed the Senate this week included
provisions to halt evictions in some federally funded housing, defer
federal student loan payments interest-free and stop collections on
those who are in default. Law enforcement officials in numerous
jurisdictions are refusing to send people accused of low-level offenses
to jail or releasing some who are already locked up.

The efforts at leniency have bipartisan backing, with the biggest debate
over just how long the generosity ought to extend. Those who have long
been fighting for tenant rights or criminal justice reform all of a
sudden see their views in the mainstream and argue that this is not
forgiveness, but justice. Law-and-order and small-government types
shudder to think of the consequences if the current mood is
longstanding.

``We're winning stuff that last week sounded radical,'' said Tara
Raghuveer, a tenant rights advocate in Kansas City, Mo. ``We have to
start demanding more.''

The calculation for public officials may be as much about practicality
as good will.

How can they ask people to stay at a distance, yet pack them into
crowded jail cells? How can they demand that residents hunker down at
home and maintain good hygiene, yet shut off their water and kick them
out of their residences?

Read more

\includegraphics{https://static01.graylady3jvrrxbe.onion/images/icons/t_logo_291_black.png}

March 27, 2020, 9:03 a.m. ET

March 27, 2020, 9:03 a.m. ET

By Priya Krishna

\hypertarget{distance-will-temporarily-reinvent-some-holidays}{%
\subsection{\texorpdfstring{\protect\hyperlink{distance-will-temporarily-reinvent-some-holidays}{Distance
will temporarily reinvent some
holidays.}}{Distance will temporarily reinvent some holidays.}}\label{distance-will-temporarily-reinvent-some-holidays}}

\includegraphics{https://static01.graylady3jvrrxbe.onion/images/2020/04/01/dining/26virus-holidays11/26virus-holidays11-articleLarge-v3.jpg?quality=75\&auto=webp\&disable=upscale}

Over the next several weeks, Americans will conduct virtual Seders, cook
Easter brunch for just one or two, enjoy scaled-down Nowruz feasts at a
six-foot distance from one another, and break their Ramadan fasts while
isolating at home.

All of these holy days and celebrations, which promote renewal and
reflection, involve gathering around meals. Social distancing has
suddenly posed a big barrier, and some festivities have been called off.

Yet despite widespread restrictions on travel, congregating in large
groups or attending religious services, many people are finding creative
ways to stage their holidays.

A
\href{https://www.nytimes3xbfgragh.onion/2020/03/27/dining/holidays-coronavirus.html}{look
at how various religious groups will handle the restrictions} reveals
some common ground between a wide mix of people. The Washington Post
\href{https://www.washingtonpost.com/nation/2020/03/27/coronavirus-lent-meat/}{reported
on other adaptations} by religious groups, including a Catholic diocese
in New Jersey that has lifted the restriction on eating meat on Fridays
during lent.

\href{https://twitter.com/diocesemetuchen/status/1243227989250359298}{In
a tweet}, Bishop James F. Checchio acknowledged that his parishioners
were sacrificing enough during this crisis, and said sacrificing meat
would not be necessary.

``I have granted a dispensation from abstaining from meat on Fridays for
the rest of Lent, except Good Friday which is universal law,'' he said.

Read more

\href{https://www.nytimes3xbfgragh.onion/by/laura-collins-hughes}{\includegraphics{https://static01.graylady3jvrrxbe.onion/images/2019/04/03/multimedia/author-laura-collins-hughes/author-laura-collins-hughes-thumbLarge.png}}

March 27, 2020, 9:02 a.m. ET

March 27, 2020, 9:02 a.m. ET

By
\href{https://www.nytimes3xbfgragh.onion/by/laura-collins-hughes}{Laura
Collins-Hughes}

\hypertarget{a-theater-critic-tries-to-get-her-drama-fix-online}{%
\subsection{\texorpdfstring{\protect\hyperlink{a-theater-critic-tries-to-get-her-drama-fix-online}{A
theater critic tries to get her drama fix
online.}}{A theater critic tries to get her drama fix online.}}\label{a-theater-critic-tries-to-get-her-drama-fix-online}}

\includegraphics{https://static01.graylady3jvrrxbe.onion/images/2020/03/27/arts/25SIBLINGS-STREAM-1/merlin_170934651_534d8bf2-9574-4e57-a099-25716e7d7f91-articleLarge.jpg?quality=75\&auto=webp\&disable=upscale}

I wanted to climb right through the screen into a seat a few rows from
the stage.

It was a visceral impulse, not a rational one. It paid no heed to
impossibility, or even recent memory --- the feeling of unease that
nagged in the last days before the theaters shut down, when from avidity
and habit we kept packing the houses, breathing communal air.

That was a scant two weeks ago, but it feels eons longer since we've
learned to keep a cautious social distance. For now we seek our drama
fixes online, trying to fill the empty space left by the temporary
absence of the empty spaces.

So a couple of recent evenings lately have found me peering into my
laptop, watching plays that were digitally immortalized before the
coronavirus thwarted their respective opening nights, and the runs that
would have followed.

Both Ren Dara Santiago's ``The Siblings Play,'' a world premiere at
\href{https://www.rattlestick.org/}{Rattlestick Playwrights Theater} in
New York, and Mike Lew's ``Teenage Dick,'' a Chicago premiere at
\href{https://www.theaterwit.org/}{Theater Wit}, are available for
ticketed streaming --- a scrappily defiant, even noble insistence on the
part of producers that the artists' work on their small stages not
simply disappear.

It is heartening that these recordings are there, and that the companies
can earn some box office from them. But if you watch, you may arrive as
I did at the conviction that you have not truly seen the plays.

That's not the fault of the shows; it's a function of experiencing them
through the filter of a medium they weren't constructed for.

Glowing at me through my screen, they felt flattened and far away ---
and, oh, how they made me wish I were in the room to sense all of their
dimensions. Recordings of stage productions are frustrating by nature,
pale relics of theater rather than theater itself.

I hesitate to say that, because it is a brutal truth, and don't we all
have enough of those in this strange pandemic time? The upside is the
forceful argument these videos inadvertently make for the live
experience --- for the undiminished necessity of coming together in
person to see a story unfold.

Read more

\hypertarget{our-2020-election-guide}{%
\section{Our 2020 Election Guide}\label{our-2020-election-guide}}

Updated ~Sept. 8, 2020

\begin{center}\rule{0.5\linewidth}{\linethickness}\end{center}

\begin{itemize}
\item ~
  \hypertarget{the-latest}{%
  \subsection{The Latest}\label{the-latest}}

  \begin{itemize}
  \item
    The campaign
    \href{https://www.nytimes3xbfgragh.onion/live/2020/09/08/us/trump-vs-biden?action=click\&pgtype=Article\&state=default\&region=BELOW_MAIN_CONTENT\&context=storylines_guide}{shifts
    to a higher gear this week}, with President Trump set to visit
    Florida and North Carolina today and Joseph R. Biden heading to
    Michigan tomorrow.
  \end{itemize}
\item ~
  \hypertarget{how-to-win-270}{%
  \subsection{How to Win 270}\label{how-to-win-270}}

  \begin{itemize}
  \item
    Joe Biden and Donald Trump need 270 electoral votes to reach the
    White House. Try building
    \href{https://www.nytimes3xbfgragh.onion/interactive/2020/us/elections/election-states-biden-trump.html?action=click\&pgtype=Article\&state=default\&region=BELOW_MAIN_CONTENT\&context=storylines_guide}{your
    own coalition of battleground states}~to see potential outcomes.
  \end{itemize}
\item ~
  \hypertarget{voting-by-mail}{%
  \subsection{Voting by Mail}\label{voting-by-mail}}

  \begin{itemize}
  \item
    Will you have enough time to vote by mail in your state? Yes, but
    it's risky to procrastinate.
    \href{https://www.nytimes3xbfgragh.onion/interactive/2020/08/31/us/politics/vote-by-mail-deadlines.html?action=click\&pgtype=Article\&state=default\&region=BELOW_MAIN_CONTENT\&context=storylines_guide}{Check
    your state's deadline.}
  \item
    \href{https://www.nytimes3xbfgragh.onion/interactive/2020/us/elections/joe-biden.html?action=click\&pgtype=Article\&state=default\&region=BELOW_MAIN_CONTENT\&context=storylines_guide}{}

    \hypertarget{joe-biden}{%
    \section{Joe Biden}\label{joe-biden}}

    \hypertarget{democrat}{%
    \subsection{Democrat}\label{democrat}}

    \href{https://www.nytimes3xbfgragh.onion/interactive/2020/us/elections/donald-trump.html?action=click\&pgtype=Article\&state=default\&region=BELOW_MAIN_CONTENT\&context=storylines_guide}{}

    \hypertarget{donald-trump}{%
    \section{Donald Trump}\label{donald-trump}}

    \hypertarget{republican}{%
    \subsection{Republican}\label{republican}}
  \end{itemize}
\item
  \hypertarget{keep-up-with-our-coverage}{%
  \subsection{Keep Up With Our
  Coverage}\label{keep-up-with-our-coverage}}

  \begin{itemize}
  \item
    Get an
    \href{https://www.nytimes3xbfgragh.onion/newsletters/politics?action=click\&pgtype=Article\&state=default\&region=BELOW_MAIN_CONTENT\&context=storylines_guide}{email}~recapping
    the day's news
  \item
    Download our mobile app on
    \href{https://apps.apple.com/us/app/nytimes/id284862083?ls=1\&mat_click_id=5c79ae7455014fd1bd66b5610c05b8f2-20191112-16948\&referrer=mat_click_id\%3D5c79ae7455014fd1bd66b5610c05b8f2-20191112-16948\%26link_click_id\%3D722930677036718082}{iOS}~and
    \href{http://a.localytics.com/android?id=com.nytimes.android\&referrer=utm_source\%3Dother_nyt_mobile_web\%26utm_medium\%3DWeb\%2520page\%26utm_term\%3DGeneral\%2520Mobile\%2520Page\%26utm_campaign\%3DNYT\%2520Mobile\%2520General\%2520Page}{Android}~and
    turn on Breaking News and Politics alerts
  \end{itemize}
\end{itemize}

\hypertarget{site-index}{%
\subsection{Site Index}\label{site-index}}

\hypertarget{site-information-navigation}{%
\subsection{Site Information
Navigation}\label{site-information-navigation}}

\begin{itemize}
\tightlist
\item
  \href{https://help.nytimes3xbfgragh.onion/hc/en-us/articles/115014792127-Copyright-notice}{©~2020~The
  New York Times Company}
\end{itemize}

\begin{itemize}
\tightlist
\item
  \href{https://www.nytco.com/}{NYTCo}
\item
  \href{https://help.nytimes3xbfgragh.onion/hc/en-us/articles/115015385887-Contact-Us}{Contact
  Us}
\item
  \href{https://www.nytco.com/careers/}{Work with us}
\item
  \href{https://nytmediakit.com/}{Advertise}
\item
  \href{http://www.tbrandstudio.com/}{T Brand Studio}
\item
  \href{https://www.nytimes3xbfgragh.onion/privacy/cookie-policy\#how-do-i-manage-trackers}{Your
  Ad Choices}
\item
  \href{https://www.nytimes3xbfgragh.onion/privacy}{Privacy}
\item
  \href{https://help.nytimes3xbfgragh.onion/hc/en-us/articles/115014893428-Terms-of-service}{Terms
  of Service}
\item
  \href{https://help.nytimes3xbfgragh.onion/hc/en-us/articles/115014893968-Terms-of-sale}{Terms
  of Sale}
\item
  \href{https://spiderbites.nytimes3xbfgragh.onion}{Site Map}
\item
  \href{https://help.nytimes3xbfgragh.onion/hc/en-us}{Help}
\item
  \href{https://www.nytimes3xbfgragh.onion/subscription?campaignId=37WXW}{Subscriptions}
\end{itemize}
