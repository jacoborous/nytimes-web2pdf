Sections

SEARCH

\protect\hyperlink{site-content}{Skip to
content}\protect\hyperlink{site-index}{Skip to site index}

\href{https://myaccount.nytimes3xbfgragh.onion/auth/login?response_type=cookie\&client_id=vi}{}

\href{https://www.nytimes3xbfgragh.onion/section/todayspaper}{Today's
Paper}

\begin{itemize}
\item
  \href{https://www.nytimes3xbfgragh.onion/live/2020/09/07/us/trump-vs-biden?action=click\&pgtype=Article\&state=default\&region=TOP_BANNER\&context=storylines_menu}{Election
  Updates}
\item
  \href{https://www.nytimes3xbfgragh.onion/interactive/2020/us/elections/election-states-biden-trump.html?action=click\&pgtype=Article\&state=default\&region=TOP_BANNER\&context=storylines_menu}{Paths
  to 270}
\item
  \href{https://www.nytimes3xbfgragh.onion/interactive/2020/08/31/us/politics/vote-by-mail-deadlines.html?action=click\&pgtype=Article\&state=default\&region=TOP_BANNER\&context=storylines_menu}{Voting
  by Mail}
\item
  \href{https://www.nytimes3xbfgragh.onion/interactive/2019/us/elections/2020-presidential-election-calendar.html?action=click\&pgtype=Article\&state=default\&region=TOP_BANNER\&context=storylines_menu}{Key
  Dates}
\item
  \href{https://www.nytimes3xbfgragh.onion/newsletters/politics?action=click\&pgtype=Article\&state=default\&region=TOP_BANNER\&context=storylines_menu}{Politics
  Newsletter}
\end{itemize}

\hypertarget{highlights-from-the-south-carolina-primary-and-joe-bidens-big-win}{%
\section{Highlights From the South Carolina Primary and Joe Biden's Big
Win}\label{highlights-from-the-south-carolina-primary-and-joe-bidens-big-win}}

Last Updated

Feb. 29, 2020, 11:05 p.m. ET

Feb. 29, 2020, 11:05 p.m. ET

\includegraphics{https://static01.graylady3jvrrxbe.onion/images/2020/02/29/us/politics/29-live-wilson-17541/merlin_169806657_1966e1a6-8c2a-400d-b896-cea09d76f0a6-articleLarge.jpg?quality=75\&auto=webp\&disable=upscale}

\begin{itemize}
\item
  \href{https://www.nytimes3xbfgragh.onion/2020/02/29/us/politics/joe-biden-south-carolina-primary.html}{Former
  Vice President Joseph R. Biden Jr. was declared the winner of the
  South Carolina primary} shortly after polls closed at 7 p.m. Eastern.
  Early results show Senator Bernie Sanders in a distant second place,
  followed by the hedge fund billionaire
  \href{http://www.nytimes3xbfgragh.onion/2020/02/29/us/politics/tom-steyer-drops-out.html}{Tom
  Steyer, who dropped out of the race} Saturday night.
\item
  Black voters were the decisive factor for Mr. Biden. Exit polls
  indicated that they made up about 55 percent of the electorate, and a
  majority of them voted for Mr. Biden. Black voters who are 29 or
  younger favored Mr. Sanders.
\item
  \href{https://www.nytimes3xbfgragh.onion/interactive/2020/us/elections/joe-biden.html}{Mr.
  Biden},
  \href{https://www.nytimes3xbfgragh.onion/interactive/2020/us/elections/bernie-sanders.html}{Mr.
  Sanders},
  \href{https://www.nytimes3xbfgragh.onion/2020/02/28/us/politics/tom-steyer-south-carolina-campaign-spending.html}{Mr.
  Steyer}, Senator
  \href{https://www.nytimes3xbfgragh.onion/interactive/2020/us/elections/elizabeth-warren.html}{Elizabeth
  Warren} and former Mayor
  \href{https://www.nytimes3xbfgragh.onion/interactive/2020/us/elections/pete-buttigieg.html}{Pete
  Buttigieg} campaigned most aggressively in the state. Mr. Sanders
  currently has the most delegates after votes in Iowa, New Hampshire
  and Nevada.
\item
  South Carolina is important for two reasons: It's the first primary
  with a
  \href{https://www.nytimes3xbfgragh.onion/2020/02/29/us/politics/black-voters-south-carolina-primary.html}{large
  black electorate}, and the results will give Mr. Biden momentum
  heading into the hugely influential primaries in California, Texas and
  other states on
  \href{https://www.nytimes3xbfgragh.onion/2020/02/28/us/politics/latest-democratic-polls.html}{Super
  Tuesday}.
\item
  The results also point to
  \href{https://www.nytimes3xbfgragh.onion/2020/02/29/us/politics/buttigieg-black-voters-south-carolina.html}{the
  weak support from black voters for Mr. Buttigieg}; exit polls showed
  him winning votes from just three percent of African-Americans. Mr.
  Buttigieg won Iowa and is running second in delegates, but any
  Democrat will have a hard time winning the nomination without solid
  support from black voters.
\end{itemize}

Stay up to date on primaries and caucuses. Subscribe to ``On Politics,''
and we'll send you a link to the live results.

\href{https://www.nytimes3xbfgragh.onion/newsletters/politics}{Sign up
for our politics newsletter}

\href{https://www.nytimes3xbfgragh.onion/by/nick-corasaniti}{\includegraphics{https://static01.graylady3jvrrxbe.onion/images/2018/06/13/multimedia/author-nick-corasaniti/author-nick-corasaniti-thumbLarge-v2.png}}

Feb. 29, 2020, 10:26 p.m. ET

Feb. 29, 2020, 10:26 p.m. ET

By \href{https://www.nytimes3xbfgragh.onion/by/nick-corasaniti}{Nick
Corasaniti}

\hypertarget{a-delegate-math-silver-lining-for-sanders}{%
\subsection{\texorpdfstring{\protect\hyperlink{a-delegate-math-silver-lining-for-sanders}{A
delegate math silver lining for
Sanders.}}{A delegate math silver lining for Sanders.}}\label{a-delegate-math-silver-lining-for-sanders}}

It is a big night for Joseph R. Biden Jr., but Bernie Sanders can count
on netting at least a few delegates, as he will finish above the 15
percent threshold statewide, according to The Associated Press.

Though results are still coming in, The A.P. estimated that Mr. Sanders
would receive at least seven delegates from South Carolina, preventing
an outright shutout by Mr. Biden in South Carolina.

The senator from Vermont can earn delegates based on his statewide vote
share, as well as by finishing above 15 percent in any of the seven
congressional districts around the state.

Mr. Sanders, who congratulated Mr. Biden on his victory at a rally in
Virginia, saying, ``you can't win them all,'' had jumped to an early
delegate lead in the primary race after a close victory in New Hampshire
and a big win in Nevada.

Tom Steyer, the self-funding billionaire candidate who spent lavishly on
his effort nationally and in South Carolina, had not yet earned a
delegate, with nearly 90 percent of the results in. He dropped out of
the race on Saturday night.

Read more

\href{https://www.nytimes3xbfgragh.onion/by/stephanie-saul}{\includegraphics{https://static01.graylady3jvrrxbe.onion/images/2020/02/06/reader-center/author-stephanie-saul/author-stephanie-saul-thumbLarge.png}}

Feb. 29, 2020, 9:45 p.m. ET

Feb. 29, 2020, 9:45 p.m. ET

By \href{https://www.nytimes3xbfgragh.onion/by/stephanie-saul}{Stephanie
Saul}

\hypertarget{steyer-drops-out-i-honestly-dont-see-a-path}{%
\subsection{\texorpdfstring{\protect\hyperlink{tom-steyer}{Steyer drops
out: `I honestly don't see a
path.'}}{Steyer drops out: `I honestly don't see a path.'}}\label{steyer-drops-out-i-honestly-dont-see-a-path}}

Image

Tom Steyer at Allen University in Columbia on Friday.Credit...Hilary
Swift for The New York Times

COLUMBIA, S.C. ---
\href{https://www.nytimes3xbfgragh.onion/2020/02/29/us/politics/tom-steyer-drops-out.html}{Tom
Steyer dropped out of the presidential race} Saturday night following a
poor showing in the South Carolina primary. He spent more than \$191
million, most of it drawn from his personal fortune, on his long-shot
presidential bid.

A billionaire former hedge fund operator who has devoted the last decade
of his life to issues ranging from climate change to impeaching
President Trump, his campaign never fully caught on. He had focused much
of strategy on South Carolina, but with more than 50 percent of
precincts reporting he had garnered less than 12 percent of the vote.

``I didn't get into this race to start talking about things to get
votes,'' Mr. Steyer said, appearing before supporters at an election
night party. ``I was in this race to talk about things that I care
about.''

He added: ``We were disappointed with how we came out. I think we got
one or two delegates from South Carolina, but there's no question today
that we were disappointed. I said that if I didn't see a path to
winning, I would suspend my campaign. And I honestly don't see a path.''

In withdrawing from the race, Mr. Steyer did not endorse a candidate,
but he pledged to work to support the eventual Democratic nominee.

Read more

\hypertarget{advertisement}{%
\subsubsection{Advertisement}\label{advertisement}}

\protect\hyperlink{after-dfp-ad-mid1}{Continue reading the main story}

\href{https://www.nytimes3xbfgragh.onion/by/reid-j-epstein}{\includegraphics{https://static01.graylady3jvrrxbe.onion/images/2019/06/25/reader-center/author-reid-epstein/9e877853d8234217b58e5762253aa771-thumbLarge.png}}

Feb. 29, 2020, 9:41 p.m. ET

Feb. 29, 2020, 9:41 p.m. ET

By \href{https://www.nytimes3xbfgragh.onion/by/reid-j-epstein}{Reid J.
Epstein}

\hypertarget{buttigieg-in-raleigh-reacts-to-fourth-place-finish}{%
\subsection{\texorpdfstring{\protect\hyperlink{pete-buttigieg-speech}{Buttigieg,
in Raleigh, reacts to fourth-place
finish.}}{Buttigieg, in Raleigh, reacts to fourth-place finish.}}\label{buttigieg-in-raleigh-reacts-to-fourth-place-finish}}

RALEIGH, N.C. --- Throwing a victory party after what appears to be a
fourth-place finish in South Carolina's primary, Pete Buttigieg said he
had been humbled yet remained hopeful.

``Running for president is an exercise in hope and humility and we have
come down South filled with both,'' the former mayor of South Bend,
Ind., told a high school gymnasium packed with a few thousand
supporters. ``I am proud of the votes we earned and I am determined to
earn every vote on the road ahead.''

Looking forward to the contests coming Tuesday --- including North
Carolina's primary --- Mr. Buttigieg took shots at Joseph R. Biden Jr.
and Bernie Sanders, who are now the delegate leaders in the Democratic
presidential contest, without naming them.

``We cannot go on with the politics that has us at each others' throats
rather than at each other's backs,'' he said of Mr. Sanders.

And Mr. Buttigieg repeated a point he's been stressing since the outset
of his campaign: that Mr. Biden's decades of Washington experience were
a detriment to his chances of defeating President Trump in November.

``Democrats win,'' he said, ``when we offer a new vision, a new
generation of leadership.''

Read more

\href{https://www.nytimes3xbfgragh.onion/by/katie-glueck}{\includegraphics{https://static01.graylady3jvrrxbe.onion/images/2020/01/29/reader-center/author-katie-glueck/author-katie-glueck-thumbLarge.png}}

Feb. 29, 2020, 9:19 p.m. ET

Feb. 29, 2020, 9:19 p.m. ET

By \href{https://www.nytimes3xbfgragh.onion/by/katie-glueck}{Katie
Glueck}

\hypertarget{biden-declares-victory-and-swipes-at-sanders}{%
\subsection{\texorpdfstring{\protect\hyperlink{biden-declares-victory-and-swipes-at-sanders}{Biden
declares victory and swipes at
Sanders.}}{Biden declares victory and swipes at Sanders.}}\label{biden-declares-victory-and-swipes-at-sanders}}

\includegraphics{https://static01.graylady3jvrrxbe.onion/images/2020/02/29/us/politics/29vid-sc-primary-biden/29vid-sc-primary-biden-videoSixteenByNine3000.jpg}

COLUMBIA, S.C. --- Joseph R. Biden Jr., who has run three times for the
White House, won his first state ever as a presidential candidate on
Saturday, declaring victory in South Carolina and using the moment to
obliquely rebuke his chief rival for the nomination, Senator Bernie
Sanders.

``This is the moment to choose the path forward for our party,'' Mr.
Biden said, speaking at his victory party here. ``Folks, win big or
lose, that's the choice. Most Americans don't want the promise of
revolution. They want more than promises. They want results.''

Mr. Biden's decisive victory here comes after crippling losses in Iowa
and New Hampshire, and as Mr. Sanders has made significant inroads in
the many states that will vote on Super Tuesday, three days from now.

The former vice president's allies hope that Mr. Biden's win here will
inject enough momentum into his campaign to revive his standing in those
states, and that he will emerge as the clear moderate alternative to Mr.
Sanders, though former Mayor Michael R. Bloomberg of New York in
particular has complicated those efforts.

``Talk is cheap, false promises are deceptive, talk about revolution
isn't changing anyone's life,'' he said, another jab at Mr. Sanders, a
democratic socialist from Vermont who has spoken of revolution. ``We
need real changes right now.''

Mr. Biden spoke at a raucous rally held in a cavernous gym here, a stark
departure from the small, subdued venues he often encountered in the
first two states. He spoke in sweeping terms about unity and championing
the American values that, he suggested, President Trump had shredded.
And he worked in some implicit swipes at Mr. Bloomberg, who has been a
Republican and an independent as well as a Democrat.

``If the Democrats want a nominee who's a Democrat --- A lifelong
Democrat! A proud Democrat! An Obama-Biden Democrat! --- then join us,''
Mr. Biden said with a grin.

Mr. Biden discussed the crucial role South Carolina played in the
presidential campaigns of Bill Clinton and Barack Obama, and cast
himself as the next candidate to be propelled to the nomination out of
this state. He also appeared to nod to African-American voters, a
critical part of the Democratic base. Many of those voters helped propel
Mr. Biden to victory.

``Just days ago, the press and the pundits declared this candidacy
dead,'' he said. ``Now thanks to all to you, the heart of the Democratic
Party, we've just won and we won big.''

``And,'' he added, ``we are very much alive.''

Read more

\includegraphics{https://static01.graylady3jvrrxbe.onion/images/icons/t_logo_291_black.png}

Feb. 29, 2020, 9:12 p.m. ET

Feb. 29, 2020, 9:12 p.m. ET

By Michael Hardy

\hypertarget{warren-speaking-in-houston-picks-up-an-endorsement-and-looks-to-super-tuesday}{%
\subsection{\texorpdfstring{\protect\hyperlink{elizabeth-warren-houston}{Warren,
speaking in Houston, picks up an endorsement and looks to Super
Tuesday.}}{Warren, speaking in Houston, picks up an endorsement and looks to Super Tuesday.}}\label{warren-speaking-in-houston-picks-up-an-endorsement-and-looks-to-super-tuesday}}

HOUSTON --- With votes still being counted in South Carolina but a
disappointing finish in the single digits all but certain, Elizabeth
Warren took the stage in Houston tonight to rally a crowd of about 2,000
raucous supporters ahead of Texas' March 3 primary.

Ms. Warren started the day in South Carolina before flying to Little
Rock, Ark., and ending the day with a town hall here. Standing in front
of an American flag backdrop at a downtown park, Ms. Warren declared her
intention to fight through Super Tuesday and beyond.

``I'll be the first to say that the first four contests haven't gone
exactly as I'd hoped,'' Ms. Warren said, after congratulating Joseph R.
Biden Jr. for his South Carolina victory.

``Super Tuesday is three days away and we want to gain as many delegates
to the convention as we can --- from California to right here in
Texas.''

Ms. Warren was preceded onstage by Randi Weingarten, president of the
American Federation of Teachers, who announced her endorsement of Ms.
Warren --- not on behalf of the A.F.T. but in her personal capacity.
(The AFT has encouraged its 1.8 million members to support Ms. Warren,
Mr. Biden, or Bernie Sanders.)

Image

Randi Weingarten spoke a rally for Senator Elizabeth Warren in
Houston.~Credit...Ruth Fremson/The New York Times

Ms. Warren touted her connections to Texas, noting that she had earned
her undergraduate degree from the University of Houston, the school
where she later took her first job as a law professor. She later taught
at the University of Texas at Austin.

Ms. Warren also addressed the coronavirus outbreak, harshly criticizing
President Trump's handling of the health crisis. She described the
virus's spread as a potential dagger in the heart of what she described
as a ``shaky'' economy overly dependent on household and corporate debt.

``The impact on our economy could be brutal, putting jobs at risk,
threatening savings, undermining economic stability, and even
potentially destabilizing our giant, globally interconnected banks,''
Ms. Warren said.

The candidate who famously has a plan for everything, and who several
weeks ago
\href{https://elizabethwarren.com/plans/combating-infectious-disease-outbreaks?s}{released
a proposal to contain and treat infectious disease outbreaks}, said that
in the coming days she would release an additional plan to deal with the
public health and economic effects of the virus.

Among other proposals, the plan will call for free coronavirus testing,
free medical care for those put into mandatory quarantine, and paid time
off for employees to get tested and treated.

She also called for a major fiscal stimulus bill and for the Federal
Reserve to offer low-cost loans to struggling companies.

Ms. Warren once again slammed Michael R. Bloomberg, saying that the
coronavirus outbreak ``demands more than a billionaire mayor who
believes that since he's rich enough to buy network airtime to pretend
he's the president, that entitles him to be president --- a mayor whose
track record shows he'll govern to protect himself and his rich friends
over everyone else.''

She also took a swipe at Bernie Sanders, ``a senator who has good ideas,
but whose 30-year track record shows he consistently calls for things
that fail to get done, and consistently opposes things he nevertheless
fails to stop.''

To qualify for a share of Texas's 261 delegates, Ms. Warren needs to win
at least 15 percent of the vote. Polls currently show her at or just
below 15 percent in Texas --- a big reason she chose the state for her
Saturday night address.

It's her sixth visit to the state since announcing her candidacy.

``I just hope she gets enough votes here to give her a chunk of the
delegates,'' said Lisa Devereaux, who drove 45 minutes from Texas City
to attend the Warren event. ``That way she has some momentum going
forward.''

Tomorrow Ms. Warren will travel to Selma, Ala., to help commemorate the
55th anniversary of the city's landmark civil rights march, known as
Bloody Sunday.

Read more

\hypertarget{advertisement-1}{%
\subsubsection{Advertisement}\label{advertisement-1}}

\protect\hyperlink{after-dfp-ad-mid2}{Continue reading the main story}

\href{https://www.nytimes3xbfgragh.onion/by/sydney-ember}{\includegraphics{https://static01.graylady3jvrrxbe.onion/images/2018/06/12/multimedia/author-sydney-ember/author-sydney-ember-thumbLarge.png}}

Feb. 29, 2020, 9:01 p.m. ET

Feb. 29, 2020, 9:01 p.m. ET

By \href{https://www.nytimes3xbfgragh.onion/by/sydney-ember}{Sydney
Ember}

\hypertarget{sanders-congratulates-biden-and-looks-ahead}{%
\subsection{\texorpdfstring{\protect\hyperlink{bernie-sanders-speech}{Sanders
congratulates Biden, and looks
ahead.}}{Sanders congratulates Biden, and looks ahead.}}\label{sanders-congratulates-biden-and-looks-ahead}}

\includegraphics{https://static01.graylady3jvrrxbe.onion/images/2020/02/29/us/politics/29vid-sc-primary-sanders/29vid-sc-primary-sanders-videoSixteenByNine3000.jpg}

VIRGINIA BEACH --- Bernie Sanders congratulated Joseph R. Biden Jr. at a
rally on Saturday night, acknowledging he had not won South Carolina but
projecting optimism about the primaries on Super Tuesday.

``A lot of states out there, and tonight we did not win in South
Carolina,'' he said, prompting boos. ``That will not be the only defeat
--- there are a lot of states in this country. Nobody wins them all.''

Sounding somewhat drained after campaigning in two states --- earlier in
the day, he was in Massachusetts --- he nevertheless struck an upbeat
tone about the primary races on March 3, when 15 states and territories,
including California and Texas, will vote.

``I want to congratulate Joe Biden on his victory tonight,'' he said.

``And now,'' he said dramatically, as if announcing the next heavyweight
title bout, ``We head to Super Tuesday and Virginia!''

Near the stage was a screen. But it was not showing any results, as it
had during his rallies after the New Hampshire and Nevada nominating
contests. Instead, it showed a static campaign logo.

Read more

\href{https://www.nytimes3xbfgragh.onion/by/reid-j-epstein}{\includegraphics{https://static01.graylady3jvrrxbe.onion/images/2019/06/25/reader-center/author-reid-epstein/9e877853d8234217b58e5762253aa771-thumbLarge.png}}

Feb. 29, 2020, 8:09 p.m. ET

Feb. 29, 2020, 8:09 p.m. ET

By \href{https://www.nytimes3xbfgragh.onion/by/reid-j-epstein}{Reid J.
Epstein}

\hypertarget{this-is-how-a-race-is-called-the-moment-polls-close}{%
\subsection{\texorpdfstring{\protect\hyperlink{this-is-how-a-race-is-called-the-moment-polls-close}{This
is how a race is called the moment polls
close.}}{This is how a race is called the moment polls close.}}\label{this-is-how-a-race-is-called-the-moment-polls-close}}

RALEIGH, N.C. --- South Carolina's primary didn't bring much suspense:
The Associated Press called the race for former Vice President Joseph R.
Biden Jr. as soon as polls closed at 7 p.m.

How can they know who won the moment the polls closed, before official
results are in?

The A.P. made its call based on A.P. VoteCast, a survey of the American
electorate conducted by the research organization NORC at the University
of Chicago.

It's the same system the A.P. will use to call noncompetitive states in
the general election come November, and similar to the exit polling the
television networks use to call races.

\href{https://www.nytimes3xbfgragh.onion/by/thomas-kaplan}{\includegraphics{https://static01.graylady3jvrrxbe.onion/images/2019/08/28/reader-center/author-thomas-kaplan/author-thomas-kaplan-thumbLarge-v2.png}}\href{https://www.nytimes3xbfgragh.onion/by/katie-glueck}{\includegraphics{https://static01.graylady3jvrrxbe.onion/images/2020/01/29/reader-center/author-katie-glueck/author-katie-glueck-thumbLarge.png}}

Feb. 29, 2020, 8:01 p.m. ET

Feb. 29, 2020, 8:01 p.m. ET

By \href{https://www.nytimes3xbfgragh.onion/by/thomas-kaplan}{Thomas
Kaplan} and
\href{https://www.nytimes3xbfgragh.onion/by/katie-glueck}{Katie Glueck}

\hypertarget{clyburn-says-the-biden-campaign-needs-retooling}{%
\subsection{\texorpdfstring{\protect\hyperlink{clyburn-says-the-biden-campaign-needs-retooling}{Clyburn
says the Biden campaign needs
`retooling.'}}{Clyburn says the Biden campaign needs `retooling.'}}\label{clyburn-says-the-biden-campaign-needs-retooling}}

Image

Representative James E. Clyburn on Capitol Hill in Washington on
Thursday.Credit...T.J. Kirkpatrick for The New York Times

In the hours before Joseph R. Biden Jr. was declared the winner of the
South Carolina primary, Representative James E. Clyburn of South
Carolina said Mr. Biden's presidential campaign needed ``retooling'' and
promised to personally intervene now that he had endorsed Mr. Biden in
the primary race.

``I'm not going to sit idly by and watch people mishandle this
campaign,'' Mr. Clyburn said in an interview on CNN. ``We are going to
get it right.''

Mr. Clyburn, the House majority whip and the highest-ranking
African-American in Congress, endorsed Mr. Biden on Wednesday, giving
him a major boost in the final days before Saturday's primary in South
Carolina. Nearly half of primary voters said Mr. Clyburn's support was
an important factor in their decision, according to early exit polls.

During the interview, the CNN host Ana Cabrera noted that Mr. Biden, the
former vice president, had been vastly outspent on advertising in Super
Tuesday states, and she read a series of quotations from Democratic
officials in
\href{https://www.nytimes3xbfgragh.onion/2020/02/26/us/politics/joe-biden-california-super-tuesday.html}{a
New York Times article} from this past week about Mr. Biden's limited
presence on the ground in those states.

Party leaders from the Arkansas Democratic chairman to the Democratic
minority leader in the Alabama House of Representatives --- who is now
supporting former Mayor Michael R. Bloomberg of New York --- had been
sharply critical of Mr. Biden's campaign organization.

``So, congressman, how does Biden compete moving forward?'' Ms. Cabrera
asked.

``Well, I think he will have to do better, no question about that,'' Mr.
Clyburn responded. ``I have those same concerns.''

He added: ``If we are successful tonight in this campaign, if he has a
relaunch, I think we will have to sit down and get serious about how we
retool this campaign, how we retool the fund-raising, how we do the
G.O.T.V. And at that point in time, many of us around the country will
be able to join with him and help him get it right. We need to do some
retooling in the campaign, no question about that.''

Also on Saturday, the Biden campaign announced that Mr. Clyburn would
campaign for Mr. Biden on Sunday in North Carolina, one of the 14 states
with primaries on Super Tuesday.

Read more

\hypertarget{advertisement-2}{%
\subsubsection{Advertisement}\label{advertisement-2}}

\protect\hyperlink{after-dfp-ad-mid3}{Continue reading the main story}

\href{https://www.nytimes3xbfgragh.onion/by/nick-corasaniti}{\includegraphics{https://static01.graylady3jvrrxbe.onion/images/2018/06/13/multimedia/author-nick-corasaniti/author-nick-corasaniti-thumbLarge-v2.png}}

Feb. 29, 2020, 7:03 p.m. ET

Feb. 29, 2020, 7:03 p.m. ET

By \href{https://www.nytimes3xbfgragh.onion/by/nick-corasaniti}{Nick
Corasaniti}

\hypertarget{joe-biden-wins-south-carolina-primary-with-overwhelming-support}{%
\subsection{\texorpdfstring{\protect\hyperlink{joe-biden-wins-south-carolina-primary}{Joe
Biden wins South Carolina primary with overwhelming
support.}}{Joe Biden wins South Carolina primary with overwhelming support.}}\label{joe-biden-wins-south-carolina-primary-with-overwhelming-support}}

Joseph R. Biden Jr. won the South Carolina primary on Saturday, giving
the former vice president a much-needed victory that demonstrated his
deep connections with black voters in the first state to vote with a
significant black population.

The mood at the Biden campaign party was instantly ebullient, as
attendees arriving around 7 p.m. were greeted with news of the race
call. ``Celebration'' blasted on the speakers.

The victory also brought in some much needed money; the campaign's
director of online fund-raising announced that the campaign had its best
hour of fund-raising ever in the ``48 minutes'' after the race was
called.

Though he was expected to have a strong showing in South Carolina, Mr.
Biden's struggles in Iowa, New Hampshire and Nevada shook his status as
the consensus alternative to Senator Bernie Sanders of Vermont.

But his decisive victory here appears to have calmed some establishment
Democrats. Former Gov. Terry McAuliffe of Virginia announced his
endorsement of the former vice president moments after Mr. Biden was
declared the winner in South Carolina, saying he had the ``broad appeal
to win Virginia and the White House in 2020.''

Mr. Biden has run for president three times, and this is the first time
he has ever won a state. He had been pointing to a victory in South
Carolina as the launching point for the next phase for his campaign,
signaling that a meaningful victory would bring in a new influx of cash,
and that a sweeping victory would propel him onto the nomination.

``Today is a great day because, I tell you what, the full comeback
starts in South Carolina,'' Mr. Biden said at a campaign rally in North
Carolina earlier in the day. ``We're going to win South Carolina. And
the next step is North Carolina.''

Then, he said, ``it's a straight path to the nomination for president of
the United States of America.''

While the victory is sure to ease some anxiety at his campaign
headquarters in Philadelphia, Mr. Biden now turns to a four-day blitz to
make up for lost time in Super Tuesday states, where a lack of resources
and thin campaign organization present a daunting challenge to his
shaken candidacy.

But Mr. Biden was confident in his path ahead on Saturday.

``I don't think it will even be over after Super Tuesday,'' he said at a
campaign event early in the day in Greenville, S.C. ``I think it's still
going to go on to states that are ones I feel very good about.''

Read more

\href{https://www.nytimes3xbfgragh.onion/by/lisa-lerer}{\includegraphics{https://static01.graylady3jvrrxbe.onion/images/2018/09/11/us/politics/author-lisa-lerer/lisa-lerer-headshot-thumbLarge.png}}

Feb. 29, 2020, 7:00 p.m. ET

Feb. 29, 2020, 7:00 p.m. ET

By \href{https://www.nytimes3xbfgragh.onion/by/lisa-lerer}{Lisa Lerer}

\hypertarget{after-12-hours-the-polls-in-south-carolina-have-closed}{%
\subsection{\texorpdfstring{\protect\hyperlink{after-12-hours-the-polls-in-south-carolina-have-closed}{After
12 hours, the polls in South Carolina have
closed.}}{After 12 hours, the polls in South Carolina have closed.}}\label{after-12-hours-the-polls-in-south-carolina-have-closed}}

Image

Shortly after polls closed at~East Cooper Montessori Charter School in
Charleston.Credit...Hilary Swift for The New York Times

Polls have closed in South Carolina, ending voting in a state that will
test the ability of the Democratic presidential candidates to win
support from African-Americans, a core contingent of the party's base.

After a string of losses, a strong showing in the fourth nominating
contest is crucial for former Vice President Joseph R. Biden Jr., who
has staked his candidacy on victory in the Palmetto State. Mr. Biden and
his team hope that a Wednesday
\href{https://www.nytimes3xbfgragh.onion/live/2020/south-carolina-debate-primary-02-26}{endorsement
from Representative James Clyburn}, the state's most influential
Democrat, will help power a decisive win. A loss in the state could mark
the beginning of the end for Mr. Biden's effort.

Senator Bernie Sanders of Vermont and the billionaire former hedge fund
investor Tom Steyer have stayed within striking distance of Mr. Biden in
some recent polls. Four years ago, Mr. Sanders lost the state by nearly
50 points. Since then, he's made inroads into the electorate, developing
a base of support among younger black voters.

Senator Elizabeth Warren of Massachusetts, Pete Buttigieg, the former
mayor of South Bend, Ind., and Senator Amy Klobuchar of Minnesota have
struggled to break into double digits among black voters, a weakness
that portends a weak showing tonight.

Missing from the ballot was former Mayor Michael R. Bloomberg of New
York, who filed to compete in his first contests on Super Tuesday next
week, when 15 states and territories, along with Democrats abroad, will
cast ballots. Hoping to step on any momentum from the contest, Mr.
Bloomberg announced that he would make
\href{https://www.nytimes3xbfgragh.onion/2020/02/29/us/politics/michael-bloomberg-coronavirus-ad.html}{a
three-minute address to the nation} Sunday night on the growing crisis
surrounding the coronavirus --- an unusual and pricey political play
during a primary race.

The South Carolina primary is the first election in the country where
voters will use all-new voting machines, part of an effort by about a
dozen states to replace their equipment amid security concerns after
Russian interference in the 2016 election.

A strong performance in South Carolina will likely give the winner a
boost of momentum heading into the Super Tuesday contests. Already,
candidates have turned their focus to those delegate-rich primaries,
with most spending the day holding rallies and town hall events in
Texas, Massachusetts, Virginia, North Carolina and other states that
will vote next week.

\href{https://www.nytimes3xbfgragh.onion/interactive/2020/02/29/us/elections/results-south-carolina-primary-election.html}{We
are tracking live results here}.

Read more

\includegraphics{https://static01.graylady3jvrrxbe.onion/images/icons/t_logo_291_black.png}

Feb. 29, 2020, 6:59 p.m. ET

Feb. 29, 2020, 6:59 p.m. ET

By Michael Hardy

\hypertarget{a-crowd-waits-for-warren-in-houston-i-just-love-her-energy}{%
\subsection{\texorpdfstring{\protect\hyperlink{a-crowd-waits-for-warren-in-houston-i-just-love-her-energy}{A
crowd waits for Warren in Houston: `I just love her
energy.'}}{A crowd waits for Warren in Houston: `I just love her energy.'}}\label{a-crowd-waits-for-warren-in-houston-i-just-love-her-energy}}

HOUSTON --- Hours before Elizabeth Warren was scheduled to hold a town
hall at a park in downtown Houston, hundreds of her supporters were
already in a festive mood, packed in front of a stage with an American
flag backdrop. Many spread blankets on the grass and had a picnic,
enjoying the unseasonably warm weather.

Lisa Devereaux, wearing a t-shirt that read ``Erase Hate,'' had driven
45 minutes from Texas City to see her favorite candidate.

``I just love her energy,'' said Ms. Devereaux, 51, a recruiter for the
University of Texas Medical Branch. ``All the candidates have pretty
similar policies, with the exception of something like Medicare for all.
I like Elizabeth because she has actual plans for getting things
passed.''

Amanda Heathco, a non-profit arts administrator from Houston, brought
her 10-year-old daughter.

``I think it's super important for her to see women in important
positions, especially as she enters middle school and high school,'' Ms.
Heathco, 36, said. ``It's very inspirational.''

Read more

\hypertarget{advertisement-3}{%
\subsubsection{Advertisement}\label{advertisement-3}}

\protect\hyperlink{after-dfp-ad-mid4}{Continue reading the main story}

\includegraphics{https://static01.graylady3jvrrxbe.onion/images/icons/t_logo_291_black.png}

Feb. 29, 2020, 6:55 p.m. ET

Feb. 29, 2020, 6:55 p.m. ET

By Daniel Jackson

\hypertarget{looking-to-super-tuesday-buttigieg-campaigns-in-nashville}{%
\subsection{\texorpdfstring{\protect\hyperlink{pete-buttigieg-nashville}{Looking
to Super Tuesday, Buttigieg campaigns in
Nashville.}}{Looking to Super Tuesday, Buttigieg campaigns in Nashville.}}\label{looking-to-super-tuesday-buttigieg-campaigns-in-nashville}}

Image

Pete Buttigieg at a campaign rally in Nashville, Tenn.Credit...Mark
Humphrey/Associated Press

NASHVILLE --- A woman shouted out from the audience amid a sea of blue
and yellow signs, puffed jackets and apparel from the local ice hockey
team, the Nashville Predators, thanking Pete Buttigieg for coming to
Tennessee.

``Look, I'm a red-state Democrat too,'' Mr. Buttigieg said, responding
to the woman. ``And we've got to go everywhere. We've got to campaign in
every part of the country. Especially the parts of the country the
others just look out the window of their airplanes at. That's home for
us.''

It was three days before the state's primary, set for Super Tuesday. The
day before, Senator Amy Klobuchar of Minnesota had made an appearance in
the city. Michael R. Bloomberg, the former New York City mayor, had also
worked his way across Tennessee.

Mr. Buttigieg, speaking on a cold, clear Saturday from a stage set up in
front of Nashville's monolithic courthouse and city hall, criticized the
way Senator Bernie Sanders has presented the ideals he said both
candidates share.

``I also believe the best way to build the majority that will defeat
Donald Trump is to call people in, not call people names online,'' Mr.
Buttigieg said.

But he leveled the majority of his attacks against President Trump,
relishing the opportunity to, as a former member of the armed forces,
confront him on issuing pardons to ``war criminals.''

``Let's have that debate. I'm looking forward to that debate,'' Mr.
Buttigieg said to cheers from the 2,700 gathered.

Mr. Buttigieg pulled questions from the fishbowl from the Bible belt
audience and answered queries on his stance on religious liberty and
favorite piece of scripture. The questions came from attendees who had
traveled from Kentucky and northwestern Mississippi.

Renee Hall, 62, brought her Staffordshire terrier, Buster, to the rally.
While waiting for the rally to begin, Hall's homemade cardboard sign
with the words ``Go Pete Win'' leaned against her folding law chair.

Four years ago, she supported Mr. Sanders in that primary because she
thought he was the candidate needed in that moment. But she switched to
Mr. Buttigieg because she believes his understanding and demeanor is
better.

``He's got fight in a nice way,'' Ms. Hall said.

She said she had cast her ballot for Mr. Buttigieg in Tennessee's early
voting period, and showed off the sticker affixed to the back of her
phone.

Ms. Hall, a registered nurse from Goodlettsville, Tenn., sees people in
the hospital who can't afford insurance.

``I'm for `Medicare for all,' but Pete knows how to get there,'' she
said.

Meanwhile, Caleb Thomas, 33, worries about unqualified judges placed in
the federal judiciary and a weaponized Department of Justice. Though he
supported Mr. Sanders in 2016, a red-faced Mr. Sanders yelling during a
recent debate changed his mind.

``I want somebody who cares about process and who cares about rule of
law and can comfort and unify people,'' said Mr. Thomas, a video editor
from Nashville. ``And so that's why I'm leaning Pete, because I think he
has a very comforting way of talking, as opposed to Bernie: `Let's burn
the whole thing down.'''

Mr. Thomas is still undecided, though he's leaning toward Mr. Buttigieg.

``I'm trying to make my mind up and I love that Pete's coming here,''
Mr. Thomas said, adding he believes the moderates and progressives in
Tennessee are often overlooked by presidential campaigns.

But he's waiting for the results from South Carolina to see what they
indicate before he ultimately decides.

Read more

\href{https://www.nytimes3xbfgragh.onion/by/nick-corasaniti}{\includegraphics{https://static01.graylady3jvrrxbe.onion/images/2018/06/13/multimedia/author-nick-corasaniti/author-nick-corasaniti-thumbLarge-v2.png}}

Feb. 29, 2020, 5:59 p.m. ET

Feb. 29, 2020, 5:59 p.m. ET

By \href{https://www.nytimes3xbfgragh.onion/by/nick-corasaniti}{Nick
Corasaniti}

\hypertarget{klobuchar-more-than-1000-miles-from-south-carolina-amps-up-criticism-of-medicare-for-all}{%
\subsection{\texorpdfstring{\protect\hyperlink{klobuchar-more-than-1000-miles-from-south-carolina-amps-up-criticism-of-medicare-for-all}{Klobuchar,
more than 1,000 miles from South Carolina, amps up criticism of
`Medicare for
all.'}}{Klobuchar, more than 1,000 miles from South Carolina, amps up criticism of `Medicare for all.'}}\label{klobuchar-more-than-1000-miles-from-south-carolina-amps-up-criticism-of-medicare-for-all}}

Image

Senator Amy Klobuchar at a rally in Portland.~Credit...Robert F.
Bukaty/Associated Press

PORTLAND, Maine --- Senator Amy Klobuchar has not been shy about her
disagreements with Senators Bernie Sanders and Elizabeth Warren on
``Medicare for all.'' But here in Portland, she was blunter in her
assessment of the health care plan.

``I did not get on the Medicare for all bill,'' she said. ``Why? Because
I read it.''

As she swings through Super Tuesday states, having given up on South
Carolina, Ms. Klobuchar has been trying to draw more direct contrasts
with the two senators to her left rather than with her more moderate
rivals like Pete Buttigieg, the former mayor of South Bend, Ind., or
former Vice President Joseph R. Biden Jr.

In Maine, her fourth Super Tuesday state in the past 48 hours, Ms.
Klobuchar made plays to the local crowd, pointing to the impact of
climate change on the lobster fishermen and oyster farmers and the lapse
of rural broadband in northern Maine as being worse than in Iceland.

She continued to criticize the president's handling of the coronavirus,
saying he ``bungled lot of things,'' including his proposed cuts to the
Centers for Disease Control and Prevention early on in his
administration, and the decision to run all of the messaging about the
virus through the vice president's office.

Amid the growing concern over the virus, Ms. Klobuchar is keeping a
frenetic schedule ahead of Super Tuesday, holding events in 11 of the 14
Super Tuesday states in a four-day blitz. Her trip here, however, was a
bit of a zag, jetting hundreds of miles out of the way to Maine between
Virginia and North Carolina.

``Of course you might think we would go from North Carolina from
there,'' she said. ``But I said, `I want to go to Maine!'''

Yet in her final event before polls closed in South Carolina, Ms.
Klobuchar made no mention of the state currently voting, more than 1,000
miles away.

Read more

\href{https://www.nytimes3xbfgragh.onion/by/matt-stevens}{\includegraphics{https://static01.graylady3jvrrxbe.onion/images/2019/04/03/multimedia/author-matt-stevens/author-matt-stevens-thumbLarge.png}}

Feb. 29, 2020, 5:13 p.m. ET

Feb. 29, 2020, 5:13 p.m. ET

By \href{https://www.nytimes3xbfgragh.onion/by/matt-stevens}{Matt
Stevens}

\hypertarget{elizabeth-warren-makes-first-trip-to-arkansas}{%
\subsection{\texorpdfstring{\protect\hyperlink{elizabeth-warren-makes-first-trip-to-arkansas}{Elizabeth
Warren makes first trip to
Arkansas.}}{Elizabeth Warren makes first trip to Arkansas.}}\label{elizabeth-warren-makes-first-trip-to-arkansas}}

Saturday was Elizabeth Warren's
\href{https://twitter.com/KristenOrthman/status/1233839092317802496}{first
trip to Arkansas} during her monthslong presidential campaign --- and
she felt compelled to introduce herself.

``I'm Elizabeth Warren and I'm the woman who's going to beat Donald
Trump,'' she said, throwing supporters at the Little Rock, Ark., rally
into a frenzy.

Ms. Warren, a senator from Massachusetts, proceeded to spend much of her
address recounting standard highlights of her biography as a preface and
backdrop to her argument about the need to make government work for
average Americans and their families.

As a part of that argument, she once again offered thinly veiled attacks
at the rich, including the millionaires and billionaires she has
proposed taxing, like her Democratic primary rival Michael R. Bloomberg.

``Some billionaires don't like this plan. You may have noticed. Some
have gone on TV and cried --- it was \emph{so} sad,'' she said. ``Others
have run for president.''

She was also asked during a question-and-answer session to address why,
given her opposition to super PACs,
\href{https://www.nytimes3xbfgragh.onion/2020/02/27/us/politics/elizabeth-warren-super-pac.html}{one
backing her has spent millions of dollars on ads} supporting her.

``There's a super PAC now that's come in for me, and I get it --- there
are people who want to try to get women elected. They feel really
frustrated that they haven't had an opportunity to do that,'' she said.

``But my view on this is we can keep super PACs out, but it takes
everybody following the same set of rules. So as soon as everybody's
ready, I'll lead the charge and we'll keep the super PACs out because I
think that's the right way to do it.''

Read more

\hypertarget{advertisement-4}{%
\subsubsection{Advertisement}\label{advertisement-4}}

\protect\hyperlink{after-dfp-ad-mid5}{Continue reading the main story}

\includegraphics{https://static01.graylady3jvrrxbe.onion/images/icons/t_logo_291_black.png}

Feb. 29, 2020, 5:04 p.m. ET

Feb. 29, 2020, 5:04 p.m. ET

The New York Times

\hypertarget{civil-rights-leader-greets-admirers}{%
\subsection{\texorpdfstring{\protect\hyperlink{civil-rights-leader-greets-admirers}{Civil
rights leader greets
admirers.}}{Civil rights leader greets admirers.}}\label{civil-rights-leader-greets-admirers}}

Image

Credit...Maddie McGarvey for The New York Times

The Rev. Jesse L. Jackson greeted restaurant goers at Railroad BBQ in
Columbia, S.C., where his team has set up shop this week.

\href{https://www.nytimes3xbfgragh.onion/by/stephanie-saul}{\includegraphics{https://static01.graylady3jvrrxbe.onion/images/2020/02/06/reader-center/author-stephanie-saul/author-stephanie-saul-thumbLarge.png}}

Feb. 29, 2020, 4:32 p.m. ET

Feb. 29, 2020, 4:32 p.m. ET

By \href{https://www.nytimes3xbfgragh.onion/by/stephanie-saul}{Stephanie
Saul}

\hypertarget{sc-democrats-predict-high-voter-turnout-and-say-no-evidence-so-far-of-republicans-infiltrating-primary}{%
\subsection{\texorpdfstring{\protect\hyperlink{voter-turnout}{S.C.
Democrats predict high voter turnout and say no evidence, so far, of
Republicans `infiltrating'
primary.}}{S.C. Democrats predict high voter turnout and say no evidence, so far, of Republicans `infiltrating' primary.}}\label{sc-democrats-predict-high-voter-turnout-and-say-no-evidence-so-far-of-republicans-infiltrating-primary}}

Image

Bobbie Keitt, right, voted at the Shandon Fire Station in Columbia,
S.C.Credit...Maddie McGarvey for The New York Times

COLUMBIA, S.C. --- South Carolina Democratic Party officials predicted
heavy turnout during Saturday's primary, possibly matching the 500,000
people who voted in 2008 when Barack Obama's popularity drove the
state's black voters to polls in record numbers.

Speaking at a news conference here eight hours after the polls opened,
Trav Robertson, the chairman of the state party, said that nearly 80,000
voters had filed absentee ballots, a significant increase from the last
two elections, indicating strong participation.

``In the past, absentee balloting has always been an indicator of what
turnout is going to be in South Carolina,'' Mr. Robertson said. ``And so
our hopes are that we will see a very good turnout.''

Separately, there was little indication so far
\href{https://www.nytimes3xbfgragh.onion/2020/02/25/us/south-carolina-primary-operation-chaos.html}{that
Republicans had tried to infiltrate the Democratic Party process},
despite calls from President Trump, during a rally in North Charleston
on Friday night, that Republicans vote to alter the outcome of the
election.

That strategy had also been proposed by what Jay Parmley, the executive
director of the state Democrats, called some ``rogue chairmen'' on the
Republican side. South Carolina is an open primary state where voters do
not have to be registered in a party to vote in its primary.

``We have not seen any sort of mass evidence, based on absentee ballots,
of the Republicans infiltrating our primary,'' Mr. Parmley said. Of
nearly 80,000 absentee ballots, only 2,500 people had previously voted
in more than one Republican primary.

``This number is very insignificant in terms of the total number of
absentee ballots,'' he said. ``And some of them could be legitimate
disaffected Republican voters who want to vote in our primary.''

Some of the increase, Mr. Robertson believes, has been driven by
campaign efforts to get voters to the polls. The party's emphasis on
increasing voter registration also contributed to high turnout, adding
70,000 voters in the last six months alone, he said.

``This year we hit a significant milestone,'' Mr. Robertson said. ``We
now have over a million nonwhite registered voters.''

Particularly encouraging were absentee ballots from traditionally
Republican areas of the state, he said. ``Greenville County absentee
ballots have almost doubled, proving that in the upstate of South
Carolina which has been written off, the Democratic Party is working
very hard to make inroads there,'' Mr. Robertson said.

About 370,000 ballots were cast in the state's last presidential primary
in 2016. Mr. Parmley said that the election had been going smoothly so
far, with only minor complaints from voters.

As for the big question --- who benefits from heavy turnout --- the
Democratic leaders declined to speculate.

Mr. Parmley said four of the candidates --- former Vice President Joseph
R. Biden Jr., the former hedge fund investor Tom Steyer, Senator
Elizabeth Warren of Massachusetts and former Mayor Pete Buttigieg of
South Bend, Ind. --- had efficient ground operations that he thinks
might help them.

Read more

\href{https://www.nytimes3xbfgragh.onion/by/matt-stevens}{\includegraphics{https://static01.graylady3jvrrxbe.onion/images/2019/04/03/multimedia/author-matt-stevens/author-matt-stevens-thumbLarge.png}}

Feb. 29, 2020, 3:38 p.m. ET

Feb. 29, 2020, 3:38 p.m. ET

By \href{https://www.nytimes3xbfgragh.onion/by/matt-stevens}{Matt
Stevens}

\hypertarget{bloomberg-will-address-coronavirus-in-3-minute-prime-time-tv-ad}{%
\subsection{\texorpdfstring{\protect\hyperlink{bloomberg-coronavirus-ad}{Bloomberg
will address coronavirus in 3-minute prime-time TV
ad.}}{Bloomberg will address coronavirus in 3-minute prime-time TV ad.}}\label{bloomberg-will-address-coronavirus-in-3-minute-prime-time-tv-ad}}

Image

Michael R. Bloomberg at a campaign stop in Blountville, Tenn., on
Friday.Credit...Brittainy Newman/The New York Times

Seeking to draw a direct contrast with President Trump and present
himself as the best Democratic alternative, Michael R. Bloomberg will
deliver a three-minute prerecorded address on the coronavirus outbreak
in an ad on network television Sunday night, his campaign announced.

It was not immediately clear how much of Mr. Bloomberg's personal
fortune he spent to provide himself with the elevated platform. The ad
is set to air around 8:30 p.m. Eastern time Sunday on CBS and NBC, and
media executives estimated that it could cost the campaign anywhere from
\$1.25 million to \$3 million, depending on whether the networks charged
a premium to accommodate a last-minute purchase.

Ahead of its scheduled airing, the Bloomberg campaign released the video
of his remarks on Saturday, pitching the ad buy as an ``unprecedented
candidate address.''

``At times like this, it is the job of the president to reassure the
public that he or she is taking all the steps necessary to protect the
health and well-being of every citizen,'' Mr. Bloomberg says in the
taped address. ``The public wants to know their leader is trained,
informed and respected. When a problem arises, they want someone in
charge who can marshal facts and expertise to confront the problem.''

As Mr. Bloomberg has sought to rebound from his performances in two
Democratic debates and build support in the run-up to Super Tuesday,
\href{https://www.nytimes3xbfgragh.onion/2020/02/28/us/politics/michael-bloomberg-mayor-experience-2020.html}{he
has increasingly focused his campaign on his experience handling
disasters} as the mayor of New York, while contrasting himself with Mr.
Trump and his handling of the coronavirus outbreak.

In the ad that will air Sunday, Mr. Bloomberg does not mention Mr. Trump
by name. Instead, he cites his own leadership experience rebuilding New
York City after the Sept. 11 terrorist attacks, dealing with the
aftermath of Hurricane Sandy, and grappling with public health problems
like West Nile virus and swine flu.

``Each crisis is different, but they all require steady leadership, team
building and preparation,'' he says in the address. ``As Americans we
have faced many challenges before, and we have overcome them together by
looking out for one another --- and I am confident that is how we will
get through this one as well.''

The deal between NBC and the Bloomberg campaign was completed on Friday,
according to a person familiar with the negotiations who was not
authorized to speak about them publicly. The three-minute spot will run
during ``Little Big Shots,'' a variety show featuring child performers
hosted by the actress Melissa McCarthy.

Tiffany Hsu contributed reporting.

Read more

\hypertarget{advertisement-5}{%
\subsubsection{Advertisement}\label{advertisement-5}}

\protect\hyperlink{after-dfp-ad-mid6}{Continue reading the main story}

\includegraphics{https://static01.graylady3jvrrxbe.onion/images/icons/t_logo_291_black.png}

Feb. 29, 2020, 3:07 p.m. ET

Feb. 29, 2020, 3:07 p.m. ET

The New York Times

\hypertarget{vote--then-get-married}{%
\subsection{\texorpdfstring{\protect\hyperlink{vote-then-get-married}{Vote
✅, then get
married.}}{Vote ✅, then get married.}}\label{vote--then-get-married}}

Image

Credit...Hilary Swift for The New York Times

Maire McMahon voted in the South Carolina Democratic Primary right
before getting married at Hazel Parker Playground in Charleston.

\href{https://www.nytimes3xbfgragh.onion/by/matt-stevens}{\includegraphics{https://static01.graylady3jvrrxbe.onion/images/2019/04/03/multimedia/author-matt-stevens/author-matt-stevens-thumbLarge.png}}

Feb. 29, 2020, 2:32 p.m. ET

Feb. 29, 2020, 2:32 p.m. ET

By \href{https://www.nytimes3xbfgragh.onion/by/matt-stevens}{Matt
Stevens}

\hypertarget{iowa-democrats-vote-to-certify-caucus-results}{%
\subsection{\texorpdfstring{\protect\hyperlink{iowa-democrats-certify-caucus-results}{Iowa
Democrats vote to certify caucus
results.}}{Iowa Democrats vote to certify caucus results.}}\label{iowa-democrats-vote-to-certify-caucus-results}}

Image

The Iowa Democratic Party headquarters in Des Moines on Feb. 4, the day
after the disastrous caucuses.Credit...Hilary Swift for The New York
Times

The counting in Iowa is over. Or at least it's as over as it's ever
going to be.

The Iowa Democratic Party voted on Saturday to certify the results of
the caucuses that took place almost a month ago, effectively ending a
saga that threw the Democratic presidential race into disarray.

In
\href{https://iowademocrats.org/idp-state-central-committee-certifies-2020-iowa-caucus-results/}{a
statement}, the party said its state central committee had voted to
certify the results from the Feb. 3 caucuses and send them to the
Democratic National Committee. The news outlet
\href{https://twitter.com/IAStartingLine/status/1233804386259886080}{Iowa
Starting Line reported} that the committee voted 22 to 13 in favor of
certification.

\href{https://results.thecaucuses.org/}{The Iowa Democratic Party
results} awarded Pete Buttigieg 14 pledged delegates to the national
convention and Bernie Sanders 12; those delegates will help determine
the Democratic presidential nominee.

By another metric, the Iowa outcome was extraordinarily close: The
results showed Mr. Buttigieg had amassed 562.954 ``state delegate
equivalents,'' just ahead of Mr. Sanders, who earned 562.021. (The state
delegate equivalents help determine the national pledged delegates won
by each candidate.)

The Associated Press has said it will not call a winner in Iowa because
of concerns over the accuracy of the results.
\href{https://www.nytimes3xbfgragh.onion/2020/02/06/upshot/iowa-caucuses-errors-results.html}{A
review by The New York Times} found that the results were riddled with
errors, though they did not appear intentional or tilted toward a
particular candidate.

But The A.P. allowed that it would allocate the final outstanding
delegate to Mr. Buttigieg such that its results would reflect those from
the Iowa Democratic Party. Previously, The A.P. had awarded only 13
delegates to Mr. Buttigieg and withheld the final delegate that goes to
the statewide winner.

The certification vote came after a partial recount,
\href{https://www.nytimes3xbfgragh.onion/live/2020/south-carolina-primary-02-27\#iowa-is-sort-of-maybe-actually-over}{the
results of which were released Thursday night}, and a recanvass that
took place earlier this month. During those processes, officials
reviewed preference cards completed by individual caucusgoers in 23
precincts and reviewed precinct leaders' math.

Shortly after the Iowa Democratic Party announced that it had completed
the recounts this week, Jeff Weaver, a senior adviser to Mr. Sanders,
\href{https://www.politico.com/news/2020/02/27/buttigieg-iowa-caucus-recount-118044}{told
Politico} that the campaign had filed an ``implementation challenge''
with the Democratic National Committee ``stating that the Iowa
Democratic Party conducted its recanvass and recount in a way that
violated their delegate selection plan.''

Mr. Sanders's campaign did not immediately responded to a request for
comment on Saturday.

In a statement, Mr. Buttigieg's campaign said: ``Yet again, these
results confirm Pete won the Iowa caucuses. Pete was the only candidate
that was able to form a broad-based coalition across the state and
across ideological differences.''

Read more

\href{https://www.nytimes3xbfgragh.onion/by/katie-glueck}{\includegraphics{https://static01.graylady3jvrrxbe.onion/images/2020/01/29/reader-center/author-katie-glueck/author-katie-glueck-thumbLarge.png}}

Feb. 29, 2020, 2:28 p.m. ET

Feb. 29, 2020, 2:28 p.m. ET

By \href{https://www.nytimes3xbfgragh.onion/by/katie-glueck}{Katie
Glueck}

\hypertarget{biden-banks-on-a-carolina-comeback}{%
\subsection{\texorpdfstring{\protect\hyperlink{joe-biden-north-carolina}{Biden
banks on a Carolina
comeback.}}{Biden banks on a Carolina comeback.}}\label{biden-banks-on-a-carolina-comeback}}

Image

Former Vice President Jospeh R. Biden Jr. at a community event in
Raleigh, N.C.Credit...Chang W. Lee/The New York Times

RALEIGH, N.C. --- An energized Joseph R. Biden Jr. bounded onstage at
St. Augustine's University, a historically black college, taking a break
from the South Carolina campaign trail on Saturday afternoon to pitch
Democrats whose votes will be critical to his political future on
Tuesday.

``Today is a great day, because I tell you what, the full comeback
starts in South Carolina,'' he said to applause. ``We're going to win
South Carolina. And the next step is North Carolina.''

Then, he said, ``it's a straight path to the nomination for president of
the United States of America.''

North Carolina will vote on Super Tuesday, when a trove of delegates
will be up for grabs. He and his campaign are hoping that he achieves a
decisive enough victory in South Carolina on Saturday to propel him
forward on Tuesday --- but early voting has already begun in some
states, and Mr. Biden, who has yet to win a contest,
\href{https://www.nytimes3xbfgragh.onion/2020/02/26/us/politics/joe-biden-california-super-tuesday.html}{faces
real competition across the country from rivals who appear to have
vastly outpaced him in money and organization}.

His rally here on Saturday was high-energy --- but not every seat in the
bleachers was filled.

North Carolina, he went on, was ``critical in delivering the presidency
to Barack Obama and me 12 years ago, and it's going to be critical this
time as well.''

Mr. Biden, who was flanked by two members of Congress ---
Representatives David Price and G.K. Butterfield, both Democrats of
North Carolina --- emphasized what they view as Mr. Biden's ability to
help candidates running in tough races down-ballot.

``Not only is majority leader of the United States Senate Mitch
McConnell not going to be majority leader, he's not going to be a
senator,'' Mr. Biden said to thunderous applause.

In an unusually brief address for the voluble former vice president,
clocking in at under 20 minutes, he razzed attendees by talking up a
historically black school in Delaware and ripped President Trump over a
range of issues, including what Mr. Biden cast as a dismissive approach
to coronavirus.

``This man, my God, he has no shame,'' he said. ``None whatsoever, and
we must defeat him. We must defeat him. Once we defeat him, once he's
gone, the days of relentless attacks on your health care, they will be
over.''

Mr. Biden appeared delighted by the raucous reception he received, which
stood in stark contrast to the chilly, subdued rooms he often faced in
Iowa and New Hampshire.

He gave a familiar riff in closing --- but this time, the crowd
participated.

``We choose hope over fear!'' he said to applause. ``We choose science
over fiction!''

``Yes!'' the crowd responded.

``We choose unity over division!'' Mr. Biden declared.

``Yes!'' the audience responded.

He continued to applause, ``we choose truth over lies!''

And when Mr. Biden delivered a standard campaign line, about how it's
time for Americans to ``get up'' --- the crowd was ready to oblige.
Attendees were on their feet.

Read more

\hypertarget{our-2020-election-guide}{%
\section{Our 2020 Election Guide}\label{our-2020-election-guide}}

Updated ~Sept. 7, 2020

\begin{center}\rule{0.5\linewidth}{\linethickness}\end{center}

\begin{itemize}
\item ~
  \hypertarget{the-latest}{%
  \subsection{The Latest}\label{the-latest}}

  \begin{itemize}
  \item
    The unofficial Labor Day kickoff to the fall presidential campaign
    centered on Pennsylvania and Wisconsin,
    \href{https://www.nytimes3xbfgragh.onion/2020/09/07/us/politics/wisconsin-biden-harris-trump-pence.html?action=click\&pgtype=Article\&state=default\&region=BELOW_MAIN_CONTENT\&context=storylines_guide}{two
    pivotal states for both President Trump and Joseph R. Biden Jr}.
  \end{itemize}
\item ~
  \hypertarget{how-to-win-270}{%
  \subsection{How to Win 270}\label{how-to-win-270}}

  \begin{itemize}
  \item
    Joe Biden and Donald Trump need 270 electoral votes to reach the
    White House. Try building
    \href{https://www.nytimes3xbfgragh.onion/interactive/2020/us/elections/election-states-biden-trump.html?action=click\&pgtype=Article\&state=default\&region=BELOW_MAIN_CONTENT\&context=storylines_guide}{your
    own coalition of battleground states}~to see potential outcomes.
  \end{itemize}
\item ~
  \hypertarget{voting-by-mail}{%
  \subsection{Voting by Mail}\label{voting-by-mail}}

  \begin{itemize}
  \item
    Will you have enough time to vote by mail in your state? Yes, but
    it's risky to procrastinate.
    \href{https://www.nytimes3xbfgragh.onion/interactive/2020/08/31/us/politics/vote-by-mail-deadlines.html?action=click\&pgtype=Article\&state=default\&region=BELOW_MAIN_CONTENT\&context=storylines_guide}{Check
    your state's deadline.}
  \item
    \href{https://www.nytimes3xbfgragh.onion/interactive/2020/us/elections/joe-biden.html?action=click\&pgtype=Article\&state=default\&region=BELOW_MAIN_CONTENT\&context=storylines_guide}{}

    \hypertarget{joe-biden}{%
    \section{Joe Biden}\label{joe-biden}}

    \hypertarget{democrat}{%
    \subsection{Democrat}\label{democrat}}

    \href{https://www.nytimes3xbfgragh.onion/interactive/2020/us/elections/donald-trump.html?action=click\&pgtype=Article\&state=default\&region=BELOW_MAIN_CONTENT\&context=storylines_guide}{}

    \hypertarget{donald-trump}{%
    \section{Donald Trump}\label{donald-trump}}

    \hypertarget{republican}{%
    \subsection{Republican}\label{republican}}
  \end{itemize}
\item
  \hypertarget{keep-up-with-our-coverage}{%
  \subsection{Keep Up With Our
  Coverage}\label{keep-up-with-our-coverage}}

  \begin{itemize}
  \item
    Get an
    \href{https://www.nytimes3xbfgragh.onion/newsletters/politics?action=click\&pgtype=Article\&state=default\&region=BELOW_MAIN_CONTENT\&context=storylines_guide}{email}~recapping
    the day's news
  \item
    Download our mobile app on
    \href{https://apps.apple.com/us/app/nytimes/id284862083?ls=1\&mat_click_id=5c79ae7455014fd1bd66b5610c05b8f2-20191112-16948\&referrer=mat_click_id\%3D5c79ae7455014fd1bd66b5610c05b8f2-20191112-16948\%26link_click_id\%3D722930677036718082}{iOS}~and
    \href{http://a.localytics.com/android?id=com.nytimes.android\&referrer=utm_source\%3Dother_nyt_mobile_web\%26utm_medium\%3DWeb\%2520page\%26utm_term\%3DGeneral\%2520Mobile\%2520Page\%26utm_campaign\%3DNYT\%2520Mobile\%2520General\%2520Page}{Android}~and
    turn on Breaking News and Politics alerts
  \end{itemize}
\end{itemize}

\hypertarget{site-index}{%
\subsection{Site Index}\label{site-index}}

\hypertarget{site-information-navigation}{%
\subsection{Site Information
Navigation}\label{site-information-navigation}}

\begin{itemize}
\tightlist
\item
  \href{https://help.nytimes3xbfgragh.onion/hc/en-us/articles/115014792127-Copyright-notice}{©~2020~The
  New York Times Company}
\end{itemize}

\begin{itemize}
\tightlist
\item
  \href{https://www.nytco.com/}{NYTCo}
\item
  \href{https://help.nytimes3xbfgragh.onion/hc/en-us/articles/115015385887-Contact-Us}{Contact
  Us}
\item
  \href{https://www.nytco.com/careers/}{Work with us}
\item
  \href{https://nytmediakit.com/}{Advertise}
\item
  \href{http://www.tbrandstudio.com/}{T Brand Studio}
\item
  \href{https://www.nytimes3xbfgragh.onion/privacy/cookie-policy\#how-do-i-manage-trackers}{Your
  Ad Choices}
\item
  \href{https://www.nytimes3xbfgragh.onion/privacy}{Privacy}
\item
  \href{https://help.nytimes3xbfgragh.onion/hc/en-us/articles/115014893428-Terms-of-service}{Terms
  of Service}
\item
  \href{https://help.nytimes3xbfgragh.onion/hc/en-us/articles/115014893968-Terms-of-sale}{Terms
  of Sale}
\item
  \href{https://spiderbites.nytimes3xbfgragh.onion}{Site Map}
\item
  \href{https://help.nytimes3xbfgragh.onion/hc/en-us}{Help}
\item
  \href{https://www.nytimes3xbfgragh.onion/subscription?campaignId=37WXW}{Subscriptions}
\end{itemize}
