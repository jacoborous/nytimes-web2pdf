Sections

SEARCH

\protect\hyperlink{site-content}{Skip to
content}\protect\hyperlink{site-index}{Skip to site index}

\href{https://myaccount.nytimes3xbfgragh.onion/auth/login?response_type=cookie\&client_id=vi}{}

\href{https://www.nytimes3xbfgragh.onion/section/todayspaper}{Today's
Paper}

\hypertarget{the-coronavirus-outbreak}{%
\subsubsection{\texorpdfstring{\href{https://www.nytimes3xbfgragh.onion/news-event/coronavirus?name=styln-coronavirus-markets\&region=TOP_BANNER\&block=storyline_menu_recirc\&action=click\&pgtype=LegacyCollection\&impression_id=4a82d730-f4b8-11ea-a2db-0bf9b73f0b42\&variant=undefined}{The
Coronavirus
Outbreak}}{The Coronavirus Outbreak}}\label{the-coronavirus-outbreak}}

\begin{itemize}
\tightlist
\item
  live\href{https://www.nytimes3xbfgragh.onion/2020/09/11/world/covid-19-coronavirus.html?name=styln-coronavirus-markets\&region=TOP_BANNER\&block=storyline_menu_recirc\&action=click\&pgtype=LegacyCollection\&impression_id=4a82d731-f4b8-11ea-a2db-0bf9b73f0b42\&variant=undefined}{Latest
  Updates}
\item
  \href{https://www.nytimes3xbfgragh.onion/interactive/2020/us/coronavirus-us-cases.html?name=styln-coronavirus-markets\&region=TOP_BANNER\&block=storyline_menu_recirc\&action=click\&pgtype=LegacyCollection\&impression_id=4a82d732-f4b8-11ea-a2db-0bf9b73f0b42\&variant=undefined}{Maps
  and Cases}
\item
  \href{https://www.nytimes3xbfgragh.onion/interactive/2020/science/coronavirus-vaccine-tracker.html?name=styln-coronavirus-markets\&region=TOP_BANNER\&block=storyline_menu_recirc\&action=click\&pgtype=LegacyCollection\&impression_id=4a82fe40-f4b8-11ea-a2db-0bf9b73f0b42\&variant=undefined}{Vaccine
  Tracker}
\item
  \href{https://www.nytimes3xbfgragh.onion/2020/09/10/us/politics/fda-coronavirus-vaccine.html?name=styln-coronavirus-markets\&region=TOP_BANNER\&block=storyline_menu_recirc\&action=click\&pgtype=LegacyCollection\&impression_id=4a82fe41-f4b8-11ea-a2db-0bf9b73f0b42\&variant=undefined}{F.D.A.
  Regulators' Self-Defense}
\item
  \href{https://www.nytimes3xbfgragh.onion/2020/09/09/upshot/coronavirus-surprise-test-fees.html?name=styln-coronavirus-markets\&region=TOP_BANNER\&block=storyline_menu_recirc\&action=click\&pgtype=LegacyCollection\&impression_id=4a82fe42-f4b8-11ea-a2db-0bf9b73f0b42\&variant=undefined}{Surprise
  Test Fees}
\end{itemize}

\hypertarget{powell-says-rates-are-likely-to-stay-low-for-years}{%
\section{Powell Says Rates Are Likely to Stay Low for
Years}\label{powell-says-rates-are-likely-to-stay-low-for-years}}

Last Updated

Sept. 4, 2020, 10:39 p.m. ET

Sept. 4, 2020, 10:39 p.m. ET

The latest economic and business news during the coronavirus pandemic.

\hypertarget{heres-what-you-need-to-know}{%
\subsubsection{Here's what you need to
know:}\label{heres-what-you-need-to-know}}

\begin{itemize}
\item
  \protect\hyperlink{fed-chair-powell-says-interest-rates-are-likely-to-stay-low-for-years}{}

  Fed Chair Powell says interest rates are likely to stay low for years.
\item
  \protect\hyperlink{stocks-swing-sharply-and-close-down-again}{}

  Stocks swing sharply, and close down again.
\item
  \protect\hyperlink{jobs-report-august-2020}{}

  The U.S. added 1.4 million jobs in August as unemployment fell to 8.4
  percent.
\item
  \protect\hyperlink{after-taking-the-heaviest-employment-hit-women-make-a-sharp-rebound}{}

  After taking the heaviest employment hit, women make a sharp rebound.
\item
  \protect\hyperlink{wage-data-remain-distorted-by-big-shifts-in-the-labor-market}{}

  Wage data remain distorted by big shifts in the labor market.
\end{itemize}

\hypertarget{fed-chair-powell-says-interest-rates-are-likely-to-stay-low-for-years}{%
\subsection{\texorpdfstring{\protect\hyperlink{fed-chair-powell-says-interest-rates-are-likely-to-stay-low-for-years}{Fed
Chair Powell says interest rates are likely to stay low for
years.}}{Fed Chair Powell says interest rates are likely to stay low for years.}}\label{fed-chair-powell-says-interest-rates-are-likely-to-stay-low-for-years}}

\includegraphics{https://static01.graylady3jvrrxbe.onion/images/2020/09/04/business/04markets-brf-powell/merlin_148327032_316d5fd9-2e5e-4202-9eb3-58e2439d798c-articleLarge.jpg?quality=75\&auto=webp\&disable=upscale}

It could be years before the central bank raises interest rates above a
``low'' level, Jerome H. Powell, the chair of the Federal Reserve, said
on Friday, a sign of the Fed's steadfast view that the economy, while
slowly recovering, will need extraordinary support for an extended
amount of time given the pandemic.

His comments,
\href{https://www.npr.org/2020/09/04/909627932/feds-jerome-powell-jobless-rate-better-than-expected-recovery-to-take-a-long-tim}{in
an interview with NPR}, were a contrast to those from the White House,
which hailed Friday's jobs report showing the unemployment rate
\href{https://www.nytimes3xbfgragh.onion/live/2020/09/04/business/stock-market-today-coronavirus\#jobs-report-august-2020}{dipping
to 8.4 percent} as a sign of a continued, rapid recovery from the depths
of the pandemic recession.

``The economy is now recovering,'' Mr. Powell said. ``But it's going to
be a long time, we think. We think that the economy's going to need low
interest rates, which support economic activity, for an extended period
of time. It will be measured in years.''

The Fed slashed interest rates to near zero in March and has
increasingly suggested that it is in no rush to raise them, even if the
unemployment rate drops and the labor market is running hot.

Last month, Mr. Powell announced a major shift in how the central bank
guides the economy, signaling it would make job growth pre-eminent and
would not raise interest rates to guard against expected inflation just
because the
\href{https://www.nytimes3xbfgragh.onion/2020/08/27/business/economy/unemployment-claims.html}{unemployment}
rate is low.

``What we've learned is that unemployment can be even lower than we
thought and not result in troubling levels of inflation,'' Mr. Powell
told NPR. Later, he added: ``We're not going to prematurely withdraw the
support that we think the economy needs.''

While Mr. Powell expressed concern over the long-run sustainability of
federal debt, which will nearly exceed the size of the economy this
fiscal year, he said now is not the moment to worry about cutting
spending and limiting borrowing. ``The time to start working on fiscal
sustainability is not right now when we have so many people in need,''
he said.

--- \href{https://www.nytimes3xbfgragh.onion/by/jim-tankersley}{Jim
Tankersley}

\hypertarget{stocks-swing-sharply-and-close-down-again}{%
\subsection{\texorpdfstring{\protect\hyperlink{stocks-swing-sharply-and-close-down-again}{Stocks
swing sharply, and close down
again.}}{Stocks swing sharply, and close down again.}}\label{stocks-swing-sharply-and-close-down-again}}

\begin{itemize}
\item
  Stocks were down sharply in early trading on Friday, but recovered
  most of their losses by the afternoon. The S\&P 500 ended the day down
  0.8 percent after being down as much as 3 percent. The tech-heavy
  Nasdaq composite closed down 1.3 percent.
\item
  The S\&P started the week by marching upward and reaching another high
  on Wednesday, but ended it by sliding down 2.3 percent from where it
  had started. Given that the S\&P 500 has rallied 50 percent since its
  low in March, market analysts said \textbf{some big drops were almost
  inevitabl}e. ``You knew you had one of these coming at some point,
  perhaps more than one,'' said Jim Paulsen, chief investment strategist
  at the Leuthold Group. ``The moves upward, especially in the last few
  days, were something.''

  Still, the steepness of the recent declines could stoke more fear and
  further selling.
\item
  The continued selling occurred after a government report on Friday
  showed that the U.S. economy added 1.4 million jobs in August. Stocks
  have moved steadily higher since March even as the pandemic-hit
  economy struggles to recover. The Federal Reserve and Congress have
  helped stimulate the economy, giving investors confidence, but some
  analysts say the economy has to show more strength to keep the rally
  going.
\item
  Tech companies hold
  \href{https://www.nytimes3xbfgragh.onion/2020/04/28/business/coronavirus-stocks.html}{significant
  sway over the S\&P 500 index} by virtue of their size. Investors have
  been optimistic about the tech firms, whose
  \href{https://www.nytimes3xbfgragh.onion/2020/03/23/technology/coronavirus-facebook-amazon-youtube.html}{market
  dominance and online business models} appear poised to benefit from
  the prospect of a work-from-home world.
\item
  And big tech companies helped to pull the index down: \textbf{Amazon}
  ended down 2.2 percent, and \textbf{Facebook} and \textbf{Alphabet}
  both fell about 3 percent.
\item
  A day earlier the S\&P 500 suffered its worst drop since June: 3.5
  percent. The Dow Jones industrial average closed down more than 800
  points.
\item
  Stocks had been on \textbf{a significant tear}. Before Thursday, the
  S\&P 500 had been up
  \href{https://www.nytimes3xbfgragh.onion/live/2020/09/03/business/stock-market-today-coronavirus/wall-street-has-been-on-an-extraordinary-rally-since-the-depths-of-march}{in
  nine of the last 10 sessions}.
\item
  \textbf{European markets also fell} on Friday, and Asian markets
  closed lower after Wall Street's plunge.
\end{itemize}

\hypertarget{advertisement}{%
\subsubsection{Advertisement}\label{advertisement}}

\protect\hyperlink{after-dfp-ad-mid1}{Continue reading the main story}

\hypertarget{the-us-added-14-million-jobs-in-august-as-unemployment-fell-to-84-percent}{%
\subsection{\texorpdfstring{\protect\hyperlink{jobs-report-august-2020}{The
U.S. added 1.4 million jobs in August as unemployment fell to 8.4
percent.}}{The U.S. added 1.4 million jobs in August as unemployment fell to 8.4 percent.}}\label{the-us-added-14-million-jobs-in-august-as-unemployment-fell-to-84-percent}}

\hypertarget{jobs-remain-far-below-pre-pandemic-levels}{%
\subsubsection{Jobs remain far below pre-pandemic
levels}\label{jobs-remain-far-below-pre-pandemic-levels}}

\hypertarget{cumulative-change-in-all-jobs-since-august-2016}{%
\paragraph{Cumulative change in all jobs since August
2016}\label{cumulative-change-in-all-jobs-since-august-2016}}

By Ella Koeze·Data is seasonally adjusted.·Source: Bureau of Labor
Statistics

Employers continued to bring back furloughed workers last month, but at
a far slower pace than in the spring, and millions of Americans remain
out of work.

The U.S. economy added 1.4 million jobs in August, the Labor Department
said Friday, down from 1.7 million in July and down sharply from
\href{https://www.nytimes3xbfgragh.onion/2020/07/02/business/economy/jobs-unemployment-coronavirus.html}{the
4.8 million added in June}. Payrolls are still more than 11 million jobs
below their pre-pandemic level.

The unemployment rate fell to 8.4 percent, down significantly from 14.7
percent in April and 10.2 percent in July. The drop brings the rate
below the peak of the last recession a decade ago, when unemployment
briefly hit 10 percent, but joblessness is still higher than the peak of
many past recessions.

``We still have a long way to go,'' said Beth Ann Bovino, chief U.S.
economist for S\&P Global.

\hypertarget{unemployment-rate}{%
\subsubsection{Unemployment rate}\label{unemployment-rate}}

By Ella Koeze·Unemployment rates are seasonally adjusted.·Source: Bureau
of Labor Statistics

The figure for August job growth
\href{https://www.nytimes3xbfgragh.onion/live/2020/09/04/business/stock-market-today-coronavirus/census-hiring-helped-increase-the-august-jobs-tally}{was
buoyed by the hiring of close to 240,000 temporary workers for the 2020
census}, most of whom will be laid off when census canvassing ends later
this month. Private-sector payrolls, which were not affected by the
census hires, rose by one million in August, down from 1.5 million in
July.

The report on Friday provides some of the first clear data on the state
of the economy as emergency federal spending winds down, including a
\$600 weekly supplement to unemployment benefits that helped keep many
households afloat early in the pandemic. Economists warn that without
that supplement, which expired at the end of July, millions of families
will struggle to pay rent and buy food, reining in the broader economy.

\begin{itemize}
\tightlist
\item
  Because the August jobs data was collected early in the month, it may
  not reflect the full impact of the loss of benefits, economists warn.
  That quirk of the calendar could have political ramifications: The
  relatively solid jobs report
  \href{https://www.nytimes3xbfgragh.onion/2020/09/04/business/economy/jobs-report.html}{could
  ease pressure on Congress to agree on a new round of emergency
  spending}. Economists warn that could set the stage for a big drop in
  spending in the fall, leading to more job losses and a
  \href{https://www.nytimes3xbfgragh.onion/2020/09/01/business/economy/small-businesses-coronavirus.html}{wave
  of small-business failures}.
\end{itemize}

\begin{itemize}
\item
  Even as people return to work,
  \href{https://www.nytimes3xbfgragh.onion/2020/09/04/business/even-as-people-return-to-work-more-find-that-layoffs-are-permanent.html}{more
  are finding that layoffs are permanent}. In August, less than half of
  unemployed workers reported being on temporary layoff or furlough.
  Back in April, that figure was nearly 80 percent. That development is
  the result of a combination of good news and bad.
\item
  Signs of a slowing recovery
  \href{https://www.nytimes3xbfgragh.onion/2020/09/04/business/augusts-slowdown-in-job-growth-spanned-many-industries.html}{were
  seen across many industries}. For example, the construction sector is
  fewer than half a million jobs short of its level before the pandemic,
  but only 16,000 construction jobs were added in August.
\end{itemize}

--- \href{https://www.nytimes3xbfgragh.onion/by/ben-casselman}{Ben
Casselman}

\hypertarget{after-taking-the-heaviest-employment-hit-women-make-a-sharp-rebound}{%
\subsection{\texorpdfstring{\protect\hyperlink{after-taking-the-heaviest-employment-hit-women-make-a-sharp-rebound}{After
taking the heaviest employment hit, women make a sharp
rebound.}}{After taking the heaviest employment hit, women make a sharp rebound.}}\label{after-taking-the-heaviest-employment-hit-women-make-a-sharp-rebound}}

\hypertarget{black-unemployment-rates-are-higher-than-those-for-other-demographics}{%
\subsubsection{Black unemployment rates are higher than those for other
demographics}\label{black-unemployment-rates-are-higher-than-those-for-other-demographics}}

\hypertarget{unemployment-rates-by-race-for-men-women-and-over-all}{%
\paragraph{Unemployment rates by race for men, women and over
all}\label{unemployment-rates-by-race-for-men-women-and-over-all}}

Black

Hispanic

Asian

White

By Ella Koeze·Rates are seasonally adjusted except those for Asian men
and women.·Source: Bureau of Labor Statistics

The pandemic-induced downturn sent the unemployment rate for women
skyrocketing, but the figure is also falling relatively quickly as
business lockdowns ease and the job market recovers.

The unemployment rate for women 16 or older declined to 8.6 percent in
August, a \href{https://www.bls.gov/news.release/empsit.t01.htm}{report
from the Labor Department showed}, down from 16.2 percent at its peak in
April. Women have staged a faster rebound than men, albeit from much
worse levels.

Male
\href{https://beta.bls.gov/dataViewer/view/timeseries/LNS14000001}{unemployment}
fell to 8.3 percent in August, down from 13.5 percent in April. Women's
\href{https://beta.bls.gov/dataViewer/view/timeseries/LNS14000002}{joblessness}
remains 5.2 percentage points above February levels, while male
joblessness is about 4.7 percentage points higher than it was before the
crisis.

Image

Women have staged a faster rebound than men, albeit from much worse
levels.Credit...Hiroko Masuike/The New York Times

Along other demographic lines, people of color continue to face a worse
job-market impact from the pandemic recession than their white
counterparts. The unemployment
\href{https://beta.bls.gov/dataViewer/view/timeseries/LNS14000006}{rate}
for Black workers remains the highest among large racial groups at 13
percent, which up from 5.8 percent in February.
\href{https://beta.bls.gov/dataViewer/view/timeseries/LNS14032183}{Asian}workers
have seen a particularly large sustained unemployment hit: Their jobless
rate is now 10.7 percent, up from 2.5 percent before lockdowns started.

For
\href{https://beta.bls.gov/dataViewer/view/timeseries/LNS14000009}{Hispanic}
and Latino employees, unemployment fell to 10.5 percent in August, up
from 4.4 percent in February. And for
\href{https://beta.bls.gov/dataViewer/view/timeseries/LNS14000003}{white}
workers, it eased to 7.3 percent in August, up from 3.1 percent before
the crisis.

--- \href{https://www.nytimes3xbfgragh.onion/by/jeanna-smialek}{Jeanna
Smialek}

\hypertarget{a-furlough-turned-into-a-layoff-she-drew-down-her-401k-then-she-got-lucky}{%
\subsection{\texorpdfstring{\protect\hyperlink{a-furlough-turned-into-a-layoff-she-drew-down-her-401-k-then-she-got-lucky}{A
furlough turned into a layoff. She drew down her 401(k). Then she got
lucky.}}{A furlough turned into a layoff. She drew down her 401(k). Then she got lucky.}}\label{a-furlough-turned-into-a-layoff-she-drew-down-her-401k-then-she-got-lucky}}

\includegraphics{https://static01.graylady3jvrrxbe.onion/images/2020/09/04/business/04markets-brf-furloughed/merlin_176530929_29276884-14c2-4ea4-a4b0-4f5f3bbf68a7-articleLarge.jpg?quality=75\&auto=webp\&disable=upscale}

When Sharmah Wardlaw lost her job as a receptionist at an Atlanta
convention center in March, she was meant to be on a temporary furlough.
But as the weeks and months went by, she was not recalled to work. Then,
on Aug. 10, the hammer fell: She got a letter from her employer telling
her she was permanently laid off.

Ms. Wardlaw, 55, had been making ends meet thanks to the \$600-a-week
federal supplement to her unemployment benefits. That allowed her to pay
the \$1,100 rent and other bills for her apartment in Stonecrest, Ga.,
where she lives with her 19-year-old daughter.

When the supplement ceased at the end of July, she was left with \$300 a
week in state benefits. She had tucked away enough from each
unemployment check to pay her rent for August, but not for September. So
in mid-August, she withdrew \$25,000 from her 401(k) account to pay her
coming bills.

Then, on a whim last Sunday, she bought three lottery tickets, all with
the same number. She woke up on Tuesday to the news that all of her
tickets were winners. She had won \$15,000 --- enough to pay her rent
through the end of the year.

That stroke of luck changed everything.

``I can breathe a sigh of relief,'' she said. ``If I hadn't won, I'd be
draining my retirement savings to pay my rent.''

Ms. Wardlaw hopes to find a new job before the lottery money runs out.

--- \href{http://nytimes3xbfgragh.onion/by/gillian-friedman}{Gillian
Friedman}

\hypertarget{advertisement-1}{%
\subsubsection{Advertisement}\label{advertisement-1}}

\protect\hyperlink{after-dfp-ad-mid2}{Continue reading the main story}

\hypertarget{wage-data-remain-distorted-by-big-shifts-in-the-labor-market}{%
\subsection{\texorpdfstring{\protect\hyperlink{wage-data-remain-distorted-by-big-shifts-in-the-labor-market}{Wage
data remain distorted by big shifts in the labor
market.}}{Wage data remain distorted by big shifts in the labor market.}}\label{wage-data-remain-distorted-by-big-shifts-in-the-labor-market}}

\includegraphics{https://static01.graylady3jvrrxbe.onion/images/2020/09/04/business/04markets-brf-wages/merlin_174644931_a47c9359-8064-48f5-b81b-a1d6b96dc75a-articleLarge.jpg?quality=75\&auto=webp\&disable=upscale}

Hourly wage growth remained high in August, as shifts in the composition
of the labor force that have been distorting the figures continued to
muddle the numbers.

While wage data in the United States have risen rapidly during much of
the pandemic era, the trend reflects a statistical quirk. Workers at the
lower end of the earning spectrum have disproportionately lost jobs,
taking smaller data points out of the pool and pushing up the overall
average in what economists call a ``compositional shift.''

``The large employment fluctuations over the past several months ---
especially in industries with lower-paid workers --- complicate the
analysis of recent trends in average hourly earnings,'' the Labor
Department report said.

Average hourly wages were 4.7 percent higher in August than they were a
year earlier. That is slightly weaker than the July figure, but still
sharply elevated from the 3.3 percent average gain for the data series
in 2019.

On the ground, the wage story has been complicated, based on anecdotal
evidence. While some employees have received hazard pay for coming to
the workplace while infection remains a risk, others have taken wage
cuts as companies tried to avoid furloughing workers even as revenues
sank. Employers have at times reporting raising pay to compete with
expanded unemployment insurance, which
\href{https://www.nytimes3xbfgragh.onion/2020/08/27/business/since-a-600-a-week-benefit-lapsed-her-savings-have-been-dwindling.html}{lapsed
in late July}.

``A number of staffing agencies reported that before enhanced
unemployment benefits had expired, the benefits motivated them to raise
wages to attract workers,'' according to the Federal Reserve's
\href{https://www.federalreserve.gov/monetarypolicy/beigebook202009.htm}{Beige
Book business survey} for August, based on interviews from the Cleveland
district. In the Philadelphia area, companies reported retaining
so-called ``hero'' pay, while some in the Atlanta region reported
rescinding salary cuts even as others made them permanent.

--- \href{https://www.nytimes3xbfgragh.onion/by/jeanna-smialek}{Jeanna
Smialek}

\hypertarget{people-are-coming-back-into-the-labor-force-as-unemployment-sinks}{%
\subsection{\texorpdfstring{\protect\hyperlink{people-are-coming-back-into-the-labor-force-as-unemployment-sinks}{People
are coming back into the labor force as unemployment
sinks.}}{People are coming back into the labor force as unemployment sinks.}}\label{people-are-coming-back-into-the-labor-force-as-unemployment-sinks}}

\includegraphics{https://static01.graylady3jvrrxbe.onion/images/2020/09/04/business/04markets-brf-participation/merlin_174644955_37dbd1d0-13b4-497d-8078-8d54d93c3b09-articleLarge.jpg?quality=75\&auto=webp\&disable=upscale}

The share of Americans who are either working or looking for a job
continued to rebound in August as the labor market heals from the depths
of the pandemic recession.

The labor force participation rate increased to 61.7 percent,
\href{https://www.bls.gov/news.release/empsit.nr0.htm}{a report} from
the Labor Department showed, up from 61.4 percent the prior month and
roughly in line with the median forecast in a Bloomberg survey of
economists.

The rate is down from 63.4 percent in February, before the
pandemic-induced downturn.

Economists closely watch participation measures, which provide a more
complete picture of labor market strength than the jobless rate alone. A
rise in the unemployment rate that happens because more people have
chosen to look for work can actually be a sign that job market prospects
are improving, for example. That participation is rising even as
joblessness falls signals that the job market is staging a genuine
rebound.

A closely watched measure of
\href{https://beta.bls.gov/dataViewer/view/timeseries/LNS11300060}{labor
force} attachment among prime-working-age Americans, the participation
rate for those 25 to 54 years old had been holding roughly steady over
recent months. It stood at 81.4 percent in August, little changed from
81.3 percent in July. That is better than the recent low of 79.9 percent
in April, but still down sharply from 83 percent before the crisis.

A long rebound in prime-age participation before the pandemic had been a
bright spot in the broader U.S. labor market performance.

--- \href{https://www.nytimes3xbfgragh.onion/by/jeanna-smialek}{Jeanna
Smialek}

\hypertarget{the-uk-economy-is-quite-likely-to-need-more-stimulus-central-banker-says}{%
\subsection{\texorpdfstring{\protect\hyperlink{the-uk-economy-is-quite-likely-to-need-more-stimulus-central-banker-says}{The
U.K. economy is `quite likely' to need more stimulus, central banker
says.}}{The U.K. economy is `quite likely' to need more stimulus, central banker says.}}\label{the-uk-economy-is-quite-likely-to-need-more-stimulus-central-banker-says}}

\includegraphics{https://static01.graylady3jvrrxbe.onion/images/2020/09/04/business/04markets-brf-uk/merlin_175146690_3543df6f-eff4-48a2-a694-808a80592d88-articleLarge.jpg?quality=75\&auto=webp\&disable=upscale}

The British economy has rebounded faster than expected out of its
\href{https://www.nytimes3xbfgragh.onion/2020/08/12/business/britain-economy-recession-coronavirus.html}{deepest
recession on record}, but the circumstances that let that happen are
already disappearing, a central banker warned.

It is ``quite likely'' that the U.K. will need more monetary stimulus,
Michael Saunders, a member of the Bank of England's interest
rate-setting
committee,\href{https://www.bankofengland.co.uk/speech/2020/michael-saunders-speech-the-economy-and-covid-19-looking-back-and-looking-forward}{said
in a speech on Friday}.

``The economy in June, July and August has benefited from a relatively
benign confluence of factors,'' Mr. Saunders said. Government spending
and other fiscal support were very high, while the easing of lockdown
restrictions boosted spending. ``Even that very limited sweet spot may
now be fading,'' he said.

\href{https://www.nytimes3xbfgragh.onion/2020/08/21/world/europe/coronavirus-second-wave.html}{Coronavirus
infection rates} in the U.K. and Europe are rising again, and consumer
confidence has stalled, Mr. Saunders added. Meanwhile, the government's
plans to roll back spending measures will reduce net fiscal support to
an average of £15 billion (\$19.9 billion) in each of the next two
quarters, about 2 to 3 percent of gross domestic product. Between April
and June, fiscal stimulus amounted to 19 percent of G.D.P.

``I do not interpret the economy's recovery in the last few months as a
strong signal that further upside surprises lie ahead,'' Mr. Saunders
said.

Earlier this week,
\href{https://www.nytimes3xbfgragh.onion/live/2020/09/02/business/stock-market-today-coronavirus\#bank-of-england-policymakers-warn-of-even-more-economic-damage-than-is-forecast}{two
other policymakers} said the British economy might experience
longer-term impacts and a slower recovery than the central bank had most
recently forecast.

--- \href{https://www.nytimes3xbfgragh.onion/by/eshe-nelson}{Eshe
Nelson}

\hypertarget{advertisement-2}{%
\subsubsection{Advertisement}\label{advertisement-2}}

\protect\hyperlink{after-dfp-ad-mid3}{Continue reading the main story}

\hypertarget{what-we-heard-on-earnings-calls-this-week}{%
\subsection{\texorpdfstring{\protect\hyperlink{what-we-heard-on-earnings-calls-this-week}{What
we heard on earnings calls this
week.}}{What we heard on earnings calls this week.}}\label{what-we-heard-on-earnings-calls-this-week}}

The editors and reporters for the
\href{https://www.nytimes3xbfgragh.onion/2020/09/04/business/dealbook/stock-markets-tech.html}{DealBook
newsletter} sift through a lot of company reports and listen to many
earnings conference calls. These are some of the things that caught our
notice this week:

🥫 ``We all knew that there would be a pivot eventually back to healthier
recipes \ldots{} a little more comfort-oriented initially, a little more
healthier now.'' \emph{--- Mark Clouse,} \emph{\textbf{Campbell
Soup}}*'s chief executive*

⚠️ ``A large portion of the demand is driven by folks who are just
fearful of their personal protection and safety, starting with the
pandemic and moving on to the civil unrest.'' \emph{--- Mark Smith, the}
\emph{\textbf{Smith \& Wesson}} \emph{chief executive}

😷 ``Yesterday I bought a good amount of spectacular Christmas-decorated
KN95 masks. So hopefully those masks will encourage people to get close
and hug their families while still protecting themselves.'' \emph{---
Michael Ross,} \emph{\textbf{Dollarama}}*'s chief financial officer*

💻 \textbf{Zoom} reported
\href{https://www.cnbc.com/2020/09/01/zooms-stock-surges-41percent-on-earnings-adding-over-37-billion-in-value.html}{a
huge rise in profit} this week, generating effusive praise from analysts
on its conference call, including ``another incredible quarter,'' ``a
truly outstanding quarter'' and ``another just phenomenal quarter.''
Others thanked the videoconferencing company for more fundamental
reasons:

\begin{itemize}
\item
  ``Just thank you for keeping everybody connected.'' \emph{--- Heather
  Bellini of} \emph{\textbf{Goldman Sachs}}
\item
  ``I echo my congratulations and gratitude all around, and it's nice to
  see everybody.'' \emph{--- Brad Zelnick of} \emph{\textbf{Credit
  Suisse}}
\item
  ``First, I want to say thank you from the analyst community and as a
  parent, as a husband. Yes, you've made a substantive difference in all
  our lives.'' \emph{--- Alex Zukin of} \emph{\textbf{RBC Capital
  Markets}}
\end{itemize}

--- \href{https://www.nytimes3xbfgragh.onion/by/jason-karaian}{Jason
Karaian}

\hypertarget{the-latest-neiman-marcuss-bankruptcy-plan-is-approved}{%
\subsection{\texorpdfstring{\protect\hyperlink{the-latest-neiman-marcuss-bankruptcy-plan-is-approved}{The
latest: Neiman Marcus's bankruptcy plan is
approved.}}{The latest: Neiman Marcus's bankruptcy plan is approved.}}\label{the-latest-neiman-marcuss-bankruptcy-plan-is-approved}}

\begin{itemize}
\item
  A bankruptcy judge signed off on \textbf{Neiman Marcus}' restructuring
  plan on Friday, marking the end of one of the first major
  \href{https://www.nytimes3xbfgragh.onion/2020/05/07/business/neiman-marcus-bankruptcy.html}{bankruptcy
  filings during the Covid-19 pandemic}, which has caused a wave of
  major retailers to seek Chapter 11 protection. The news comes a day
  after U.S. prosecutors
  \href{https://www.nytimes3xbfgragh.onion/live/2020/09/03/business/stock-market-today-coronavirus/a-hedge-fund-manager-is-accused-of-securities-fraud-related-to-the-neiman-marcus-bankruptcy}{charged
  hedge-fund founder Daniel Kamensky} with fraud for suppressing a
  competing bid for a piece of the bankrupt retailer.
\item
  \textbf{Virgin Atlantic} said it planned to cut another 1,150 jobs as
  the airline received court approval for its 1.2 billion euro private
  rescue deal this week. It brings the total job losses since May to
  4,700, or nearly half of the airline's work force. Virgin also plans
  to put 600 cabin crew on a company-financed furlough program.
\item
  \textbf{FedEx} said Thursday that
  \href{https://www.nytimes3xbfgragh.onion/live/2020/09/03/business/stock-market-today-coronavirus/fedex-to-hire-27-percent-more-workers-than-last-year-for-holiday-season}{it
  plans to hire 70,000 U.S. workers} to prepare for an upcoming holiday
  season\href{https://www.nytimes3xbfgragh.onion/2020/09/02/business/retailers-holiday-shopping.html}{in
  which many consumers will be housebound} and reliant on online
  shopping --- and package delivery. That's a 27 percent increase from
  last year, when the company brought on 55,000 workers to prepare for
  the holidays. FedEx also
  \href{https://newsroom.fedex.com/newsroom/fedex-enhancements-position-company-ahead-of-a-record-setting-peak-season/}{announced}plans
  to expand year-round Sunday residential coverage for its FedEx Ground
  service to nearly 95 percent of the U.S. population, effective
  September 13.
\end{itemize}

\hypertarget{site-index}{%
\subsection{Site Index}\label{site-index}}

\hypertarget{site-information-navigation}{%
\subsection{Site Information
Navigation}\label{site-information-navigation}}

\begin{itemize}
\tightlist
\item
  \href{https://help.nytimes3xbfgragh.onion/hc/en-us/articles/115014792127-Copyright-notice}{©~2020~The
  New York Times Company}
\end{itemize}

\begin{itemize}
\tightlist
\item
  \href{https://www.nytco.com/}{NYTCo}
\item
  \href{https://help.nytimes3xbfgragh.onion/hc/en-us/articles/115015385887-Contact-Us}{Contact
  Us}
\item
  \href{https://www.nytco.com/careers/}{Work with us}
\item
  \href{https://nytmediakit.com/}{Advertise}
\item
  \href{http://www.tbrandstudio.com/}{T Brand Studio}
\item
  \href{https://www.nytimes3xbfgragh.onion/privacy/cookie-policy\#how-do-i-manage-trackers}{Your
  Ad Choices}
\item
  \href{https://www.nytimes3xbfgragh.onion/privacy}{Privacy}
\item
  \href{https://help.nytimes3xbfgragh.onion/hc/en-us/articles/115014893428-Terms-of-service}{Terms
  of Service}
\item
  \href{https://help.nytimes3xbfgragh.onion/hc/en-us/articles/115014893968-Terms-of-sale}{Terms
  of Sale}
\item
  \href{https://spiderbites.nytimes3xbfgragh.onion}{Site Map}
\item
  \href{https://help.nytimes3xbfgragh.onion/hc/en-us}{Help}
\item
  \href{https://www.nytimes3xbfgragh.onion/subscription?campaignId=37WXW}{Subscriptions}
\end{itemize}
