Sections

SEARCH

\protect\hyperlink{site-content}{Skip to
content}\protect\hyperlink{site-index}{Skip to site index}

\href{https://myaccount.nytimes3xbfgragh.onion/auth/login?response_type=cookie\&client_id=vi}{}

\href{https://www.nytimes3xbfgragh.onion/section/todayspaper}{Today's
Paper}

\href{/section/opinion}{Opinion}\textbar{}It's Time to Rope InChina's
Rogue-Elephant Act

\begin{itemize}
\item
\item
\item
\item
\item
\end{itemize}

Advertisement

\protect\hyperlink{after-top}{Continue reading the main story}

Supported by

\protect\hyperlink{after-sponsor}{Continue reading the main story}

\href{/section/opinion}{Opinion}

\hypertarget{its-time-to-rope-inchinas-rogue-elephant-act}{%
\section{It's Time to Rope InChina's Rogue-Elephant
Act}\label{its-time-to-rope-inchinas-rogue-elephant-act}}

By Joseph R. Biden, International Herald Tribune

\begin{itemize}
\item
  Nov. 15, 1991
\item
  \begin{itemize}
  \item
  \item
  \item
  \item
  \item
  \end{itemize}
\end{itemize}

The writer, a Democrat of Delaware, is a member of the Senate Foreign
Relations Committee. He authored proposed legislation to link U.S.-China
trade to Chinese missile and nuclear technology transfers to the Middle
East.

Secretary of State James Baker's principal message in Beijing should be
an ultimatum: If China persists in selling technologies of mass
destruction to the Middle East, America will revoke its
most-favored-nation trade status, which yields China a trade surplus of
\$15 billion a year.

Geopolitical "realists" claim the proud Chinese would never succumb to
threats. It may depend on which threats. With President George Bush's
indulgence, Chinese leaders have come to expect the best of both worlds:
hard currency from the U.S. retail market (clothes, shoes, toys),and
from the Middle East arms market as well. They must be forced to choose.

As yet no Chinese misdeed, however detrimental to international
security, has swayed Mr. Bush's determined passivity. In June, amid
reports of advanced Chinese ballistic missile sales to Syria and
Pakistan, the president renewed China's favored trade status. Now,
despite news that Beijing is dispensing technology that could give Iran
an "Islamic bomb," Mr. Bush still turns a blind eye.

The administration's contradictions comprise a China syndrome.
Proclaiming Saddam Hussein "worse than Hitler," in part because of his
awesome arsenal, the president went to war to protect Gulf stability.
The administration has also revived a Middle East peace process that
entails implicit American pressure on Israel to trade land for peace.
Yet the United States stands idle as China sells weapons equally as
lethal to Syria and Iran, regimes no less radical than Iraq and just as
uncompromising toward Israel.

Bombing these new buyers may be out of bounds; refusing to sanction the
seller is unconscionable.

A wispy Chinese pledge to curb weapons sales is not enough. A stack of
empty promises, solemnly given, precedes it. Mr. Baker must present
Beijing with a choice: the American market (worth billions) or the arms
bazaar (worth millions).

Legislation pending in the U.S. Congress would link favored trade status
to a Chinese turnabout on forced abortion, slave labor, democratic
freedoms, arms proliferation, unfair trade practices and Chinese
policies on Cambodia, Hong Kong, Taiwan and Tibet. China's record on
these issues is reprehensible. But a multitude of conditions - demanding
all - can be as futile as demanding nothing.

In foreign policy as in all else, we must set priorities. One realistic
goal, integral to the administration's Madrid initiative and vital to
American interests, is to ensure that a future Middle East conflict
could not be launched with modern missiles and nuclear warheads.

"Constructive engagement" with China may be feasible, but only if
Beijing sees that irresponsibility has its price. By acting now, and
speaking in one voice with a single compelling objective, Congress and
President Bush can focus Beijing squarely on its own self-interest.

Tying favored trade status solely to arms sales implies no lessening of
concern for human rights. Indeed, I sponsored a law, just enacted, that
will convene an experts commission to examine the start-up of a "freedom
radio" for China, modeled on the Radio Free Europe broadcasts.

The key is using tools that work. Rogue behavior on arms can be
deterred, or punished, by clear economic leverage.

Advertisement

\protect\hyperlink{after-bottom}{Continue reading the main story}

\hypertarget{site-index}{%
\subsection{Site Index}\label{site-index}}

\hypertarget{site-information-navigation}{%
\subsection{Site Information
Navigation}\label{site-information-navigation}}

\begin{itemize}
\tightlist
\item
  \href{https://help.nytimes3xbfgragh.onion/hc/en-us/articles/115014792127-Copyright-notice}{©~2020~The
  New York Times Company}
\end{itemize}

\begin{itemize}
\tightlist
\item
  \href{https://www.nytco.com/}{NYTCo}
\item
  \href{https://help.nytimes3xbfgragh.onion/hc/en-us/articles/115015385887-Contact-Us}{Contact
  Us}
\item
  \href{https://www.nytco.com/careers/}{Work with us}
\item
  \href{https://nytmediakit.com/}{Advertise}
\item
  \href{http://www.tbrandstudio.com/}{T Brand Studio}
\item
  \href{https://www.nytimes3xbfgragh.onion/privacy/cookie-policy\#how-do-i-manage-trackers}{Your
  Ad Choices}
\item
  \href{https://www.nytimes3xbfgragh.onion/privacy}{Privacy}
\item
  \href{https://help.nytimes3xbfgragh.onion/hc/en-us/articles/115014893428-Terms-of-service}{Terms
  of Service}
\item
  \href{https://help.nytimes3xbfgragh.onion/hc/en-us/articles/115014893968-Terms-of-sale}{Terms
  of Sale}
\item
  \href{https://spiderbites.nytimes3xbfgragh.onion}{Site Map}
\item
  \href{https://help.nytimes3xbfgragh.onion/hc/en-us}{Help}
\item
  \href{https://www.nytimes3xbfgragh.onion/subscription?campaignId=37WXW}{Subscriptions}
\end{itemize}
