Sections

SEARCH

\protect\hyperlink{site-content}{Skip to
content}\protect\hyperlink{site-index}{Skip to site index}

\href{https://www.nytimes3xbfgragh.onion/section/arts}{Arts}

\href{https://myaccount.nytimes3xbfgragh.onion/auth/login?response_type=cookie\&client_id=vi}{}

\href{https://www.nytimes3xbfgragh.onion/section/todayspaper}{Today's
Paper}

\href{/section/arts}{Arts}\textbar{}STAGE: 'COLORED MUSEUM,' SATIRE BY
GEORGE C. WOLFE

\url{https://nyti.ms/29w9hLr}

\begin{itemize}
\item
\item
\item
\item
\item
\end{itemize}

Advertisement

\protect\hyperlink{after-top}{Continue reading the main story}

Supported by

\protect\hyperlink{after-sponsor}{Continue reading the main story}

\hypertarget{stage-colored-museum-satire-by-george-c-wolfe}{%
\section{STAGE: 'COLORED MUSEUM,' SATIRE BY GEORGE C.
WOLFE}\label{stage-colored-museum-satire-by-george-c-wolfe}}

By \href{https://www.nytimes3xbfgragh.onion/by/frank-rich}{Frank Rich}

\begin{itemize}
\item
  Nov. 3, 1986
\item
  \begin{itemize}
  \item
  \item
  \item
  \item
  \item
  \end{itemize}
\end{itemize}

\includegraphics{https://s1.graylady3jvrrxbe.onion/timesmachine/pages/1/1986/11/03/815086_360W.png?quality=75\&auto=webp\&disable=upscale}

See the article in its original context from\\
November 3, 1986, Section C, Page
17\href{https://store.nytimes3xbfgragh.onion/collections/new-york-times-page-reprints?utm_source=nytimes\&utm_medium=article-page\&utm_campaign=reprints}{Buy
Reprints}

\href{http://timesmachine.nytimes3xbfgragh.onion/timesmachine/1986/11/03/815086.html}{View
on timesmachine}

TimesMachine is an exclusive benefit for home delivery and digital
subscribers.

About the Archive

This is a digitized version of an article from The Times's print
archive, before the start of online publication in 1996. To preserve
these articles as they originally appeared, The Times does not alter,
edit or update them.

Occasionally the digitization process introduces transcription errors or
other problems; we are continuing to work to improve these archived
versions.

THERE comes a time when a satirical writer, if he's really out for
blood, must stop clowning around and move in for the kill. That
unmistakable moment of truth arrives about halfway through ''The Colored
Museum,'' the wild new evening of black black humor at the Public
Theater. In a sketch titled ''The Last Mama-on-the-Couch Play,'' the
author, George C. Wolfe, says the unthinkable, says it with
uncompromising wit and leaves the audience, as well as a sacred target,
in ruins. The devastated audience, one should note, includes both blacks
and whites. Mr. Wolfe is the kind of satirist, almost unheard of in
today's timid theater, who takes no prisoners.

The target in this particular sketch, the longest of the 90-minute
entertainment's 11 playlets, is nothing less than ''A Raisin in the
Sun,'' Lorraine Hansberry's breakthrough play of 1959 and a model for
much black American drama that came later. Mr. Wolfe's new version of it
is introduced by an elegant ''Masterpiece Theater''-type announcer
(Tommy Hollis) in black tie, who promises us ''a searing, domestic drama
that tears at the very fabric of racist America.'' The principal
inhabitants of the tenement setting are a ''well-worn mama'' (Vickilyn
Reynolds), forever sitting with her Bible on her ''well-worn couch,''
and her son Walter-Lee-Beau-Willy (Reggie Montgomery), whose ''brow is
heavy with 300 years of oppression.''

Ms. Reynolds and Mr. Montgomery provide stinging parodies of both the
lines and performances of Claudia McNeil and Sidney Poitier in
''Raisin'' - with Mama forever instructing the angry son to let God
settle his grievances with ''the Man.'' But if Mr. Wolfe is merciless in
mocking the well-made plays, ''shattering'' acting and generational
conflicts of a 1950's black American drama preoccupied with
''middle-class aspirations,'' he doesn't stop there. ''The Last
Mama-on-the-Couch Play'' eventually satirizes a more pretentious,
latter-day form of black theater (blamed on Juilliard in this case) -
and finally turns into an all-black Broadway musical that spirals into a
nightmarish indictment of the white audience's eternal relationship to
black performers. By then, Mr. Wolfe, too, has torn ''at the very fabric
of racist America'' - but not before he has revealed the cultural blind
spots of blacks and whites alike. Although the letter of Miss
Hansberry's work has been demolished, the spirit of its political punch
is upheld.

Not all of ''The Colored Museum'' is as funny as ''The Last
Mama-on-the-Couch Play,'' but even the lesser bits are written at the
same high level of sophistication. The evening as a whole amounts to
more than its uneven, revue-like parts. The director L. Kenneth
Richardson, working with an exceptional cast and elegant designers, has
given ''The Colored Museum'' the pacing and unity of a sustained play in
the pop-art absurdist mode. Mr. Wolfe's themes are also sustained. The
issue raised by his Hansberry parody percolates in every sketch: How do
American black men and women at once honor and escape the legacy of
suffering that is the baggage of their past?

This dilemma is drawn before the first sketch begins. Brian Martin's
sleek all-purpose set, of an antiseptic modern museum displaying
exhibits of ''colored'' history, is a civilized repository of black
America's ancestral baggage. But the past can't be so neatly stored
away. There is nothing civilized about the gruesome slide show,
depicting the horrors of slavery, that overwhelms the museum's
impersonal walls in the prologue. The opening sketch sharpens Mr.
Wolfe's conflict. The actress Danitra Vance, wearing a hot-pink
stewardess outfit and a hideously perky grin, welcomes us to a
''celebrity slaveship'' whose Savannah-bound passengers are forced to
obey a ''Fasten Shackles'' sign and are warned that they will have to
''suffer for a few hundred years'' in exchange for receiving a ''complex
culture'' that will ultimately embrace the Watusi, ''Miss Diahann
Carroll in 'Julia,' '' and such white hangers-on as Gershwin and
Faulkner.

The other ''exhibits'' in Mr. Wolfe's museum are contemporary blacks
torn between the cultural legacy of oppression and revolt and the
exigencies of living in the present. Perhaps the prototypical Wolfe
character is a pin-stripe-suited businessman who tries to throw away his
past (''Free Huey'' buttons, Sly Stone records, his first dashiki) only
to discover that his rebellious younger self refuses to be trashed
without a fight. A Josephine Baker-like chanteuse named LaLa (Loretta
Devine) similarly finds that her carefully created, Gallicized show-biz
image is haunted by the ''little girl'' she thought she'd left for dead
in backwoods Mississippi. A woman dressing for a date is traumatized
when her two wigs - one a 1960's Afro, the other that of ''a Barbie doll
dipped in chocolate'' -come alive to debate the ideological identity
conflicts they have represented in their owner's life for 20 years.

Mr. Wolfe's characters ''can't live inside yesterday's pain,'' and yet
they can't bury it, either. When two Ebony magazine fashion models try
to retreat from their past into a world of narcissistic glamour, they
find only ''the kind of pain that comes from feeling no pain at all.''
Another would-be escapee, a campy homosexual nightclub denizen known as
Miss ROJ (Mr. Montgomery), looks beneath the surface of his glittery
nocturnal existence to find maggot-laced visions of ''a whole race
trashed and debased.''

While ''The Colored Museum'' may sound depressing, it's not. Mr. Wolfe
is always lobbing in wisecracks about Michael Jackson's nose or ''The
Color Purple.'' Mr. Richardson's staging, which uses a turntable to
bring exhibits of black experience past and present into chilling
juxtapositions, adds energy even to the more lugubrious sketches.
Although the whole cast deserves credit for injecting still more
vitality, one is especially pleased to see Ms. Devine and Ms. Vance used
to such fine effect after having watched them suffer last season in,
respectively, the musical ''Big Deal'' and television's ''Saturday Night
Live.'' Ms. Vance is not only funny but also noble in a monologue in
which Mr. Wolfe retrieves the dignity of a very innocent, very pregnant
teen-ager.

It's left to Ms. Reynolds, though, to resolve the playwright's themes in
a lyrical final monologue. The speech belongs to a woman named Topsy
Washington who imagines a big blowout of a party, ''somewhere between
125th Street and infinity,'' where ''Nat Turner sips champagne out of
Eartha Kitt's slipper'' and Angela Davis and Aunt Jemima sit around
talking about South Africa. As this fantasy merging of present and past
grows and grows, ''defying logic and limitations,'' Topsy decides to put
her rage about the past behind her so she can ''go about the business of
being me'' and celebrate her own ''madness and colored contradictions.''
Given that history soon rises up around Topsy in the form of music,
other characters and projected images, it's clear that the baggage of
slavery cannot really be banished from ''The Colored Museum.'' But the
shackles of the past have at the very least been defied by Mr. Wolfe's
fearless humor, and it's a most liberating revolt. PAST AND PRESENT -
THE COLORED MUSEUM, by George C. Wolfe; directed by L. Kenneth
Richardson; scenery by Brian Martin; costumes by Nancy L. Konrardy;
lighting by Victor En Yu Tan and William H. Grant 3d; sound design by
Rob Gorton; composer and arranger, Kysia Bostic; choreographer, Hope
Clarke; slide projections, Anton Nelessen; associate producer, Jason
Steven Cohen. Presented by Joseph Papp. At the Public Theater, 425
Lafayette Street. WITH: Loretta Devine, Tommy Hollis, Reggie Montgomery,
Vickilyn Reynolds, Colette Baptiste and Danitra Vance.

Advertisement

\protect\hyperlink{after-bottom}{Continue reading the main story}

\hypertarget{site-index}{%
\subsection{Site Index}\label{site-index}}

\hypertarget{site-information-navigation}{%
\subsection{Site Information
Navigation}\label{site-information-navigation}}

\begin{itemize}
\tightlist
\item
  \href{https://help.nytimes3xbfgragh.onion/hc/en-us/articles/115014792127-Copyright-notice}{©~2020~The
  New York Times Company}
\end{itemize}

\begin{itemize}
\tightlist
\item
  \href{https://www.nytco.com/}{NYTCo}
\item
  \href{https://help.nytimes3xbfgragh.onion/hc/en-us/articles/115015385887-Contact-Us}{Contact
  Us}
\item
  \href{https://www.nytco.com/careers/}{Work with us}
\item
  \href{https://nytmediakit.com/}{Advertise}
\item
  \href{http://www.tbrandstudio.com/}{T Brand Studio}
\item
  \href{https://www.nytimes3xbfgragh.onion/privacy/cookie-policy\#how-do-i-manage-trackers}{Your
  Ad Choices}
\item
  \href{https://www.nytimes3xbfgragh.onion/privacy}{Privacy}
\item
  \href{https://help.nytimes3xbfgragh.onion/hc/en-us/articles/115014893428-Terms-of-service}{Terms
  of Service}
\item
  \href{https://help.nytimes3xbfgragh.onion/hc/en-us/articles/115014893968-Terms-of-sale}{Terms
  of Sale}
\item
  \href{https://spiderbites.nytimes3xbfgragh.onion}{Site Map}
\item
  \href{https://help.nytimes3xbfgragh.onion/hc/en-us}{Help}
\item
  \href{https://www.nytimes3xbfgragh.onion/subscription?campaignId=37WXW}{Subscriptions}
\end{itemize}
