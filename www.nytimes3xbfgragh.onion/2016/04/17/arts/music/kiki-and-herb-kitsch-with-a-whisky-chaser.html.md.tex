Sections

SEARCH

\protect\hyperlink{site-content}{Skip to
content}\protect\hyperlink{site-index}{Skip to site index}

\href{https://www.nytimes3xbfgragh.onion/section/arts/music}{Music}

\href{https://myaccount.nytimes3xbfgragh.onion/auth/login?response_type=cookie\&client_id=vi}{}

\href{https://www.nytimes3xbfgragh.onion/section/todayspaper}{Today's
Paper}

\href{/section/arts/music}{Music}\textbar{}Kiki and Herb: Kitsch, With a
Whisky Chaser

\url{https://nyti.ms/1qusfIi}

\begin{itemize}
\item
\item
\item
\item
\item
\end{itemize}

Advertisement

\protect\hyperlink{after-top}{Continue reading the main story}

Supported by

\protect\hyperlink{after-sponsor}{Continue reading the main story}

\hypertarget{kiki-and-herb-kitsch-with-a-whisky-chaser}{%
\section{Kiki and Herb: Kitsch, With a Whisky
Chaser}\label{kiki-and-herb-kitsch-with-a-whisky-chaser}}

\includegraphics{https://static01.graylady3jvrrxbe.onion/images/2016/04/17/arts/17KIKIHERB1/17KIKIHERB1-articleLarge.jpg?quality=75\&auto=webp\&disable=upscale}

By \href{http://www.nytimes3xbfgragh.onion/by/joshua-barone}{Joshua
Barone}

\begin{itemize}
\item
  April 15, 2016
\item
  \begin{itemize}
  \item
  \item
  \item
  \item
  \item
  \end{itemize}
\end{itemize}

At first glance, a Kiki and Herb cabaret show could come off as a
bizarre and politically incorrect, yet addictive, mess. Kiki, with the
faded glamour of Norma Desmond, sings deranged takes on Top 40 hits and
slurs through tactless banter about current events. Herb, with
sprayed-on gray hair and a kitschy suit, hammers a keyboard and sips
Canadian Club. A medley might start with ``Frosty the Snowman'' and end
with Patti Smith.

From 1993 until 2008, Kiki and Herb, known offstage as Justin Vivian
Bond and Kenny Mellman, performed under the guise of failed,
septuagenarian lounge singers, earning a devoted audience that included
luminaries from the pop, film and fashion worlds. What started small in
San Francisco evolved into a Broadway show and ended with a sold-out
farewell at Carnegie Hall. And now, after eight years, Kiki and Herb are
reuniting for a monthlong engagement beginning Thursday, April 21, at
Joe's Pub, as part of the commissioning program New York Voices. The
show,
``\href{http://www.publictheater.org/en/tickets/calendar/playdetailscollection/joes-pub/2016/k/kiki—herb/?SiteTheme=JoesPub}{Seeking
Asylum!},'' sold out within minutes.

``I guess this is how `Star Wars' fans must feel when a new movie comes
out,'' said Kathleen Hanna, a feminist punk musician and a longtime fan.

Ms. Hanna first came across Kiki and Herb in the late 1990s when
visiting her boyfriend, Adam Horovitz of the Beastie Boys, in New York.
She was living in North Carolina then and considering a move to either
New York or Chicago. She thought the cabaret act was weird, brilliant,
even devastating. ``This is the New York of my fantasies,'' she
remembered thinking.

Forget Chicago, Ms. Hanna told her boyfriend, ``I want to be where Kiki
and Herb are.'' Now, she plays with Mr. Mellman in her band the Julie
Ruin, and Mr. Horovitz, now her husband, accompanies another downtown
cabaret fixture, Bridget Everett.

Kiki and Herb fans often have stories like this, in which they stumble
on a new cabaret act and end up obsessed.

Doug Wright, the Pulitzer Prize-winning playwright and an occasional
director of Kiki and Herb shows, called them ``Greek tragedians in tacky
lapels and stiletto heels'' who turn familiar songs into ``soul-shaking
lamentation.'' After seeing their act for the first time, he said, ``In
some dark little corner of my heart, I wanted them to adopt me.''

But when the act started in San Francisco, Mr. Mellman and Bond, who
uses the gender-neutral honorific Mx., counted only a small number of
regulars in the audience (a ``boutique audience,'' Mx. Bond called
them). They first performed together in 1989, in Mx. Bond's cabaret show
``Dixie McCall's Patterns for Living.'' At the time, Mx. Bond was 26 and
considering a career as a professor of art history. Mr. Mellman was 21,
in college and writing poetry while living on Market Street.

One night, during a later engagement at Cafe du Nord, Mx. Bond was sick
and told Mr. Mellman: ``I'm not going to be able to sing. So I'm going
to be Kiki, and you're going to be Herb, and we're going to do this
material in the style of old vaudeville people.'' Herb drank Scotch,
they decided, and Kiki drank Canadian Club and ginger ale. (Later, Herb
took up Canadian Club because they couldn't afford two bottles of
liquor.)

The act was improvised but well received. ``From the moment this
coalesced, I felt it was artistically right,'' Mr. Mellman said.

They found inspiration in real people. Herb, for example, was a lot like
Eddie, who played piano at Athens by Night, the tiny Greek restaurant
that first hosted ``Dixie McCall's.'' ``Eddie would get drunk and talk
about his cat,'' Mr. Mellman said. ``And he'd weep on his piano keys. I
was like, `Oh, that's what happens to an old piano player.'''

Kiki was based on the mother of Mx. Bond's childhood friend Nancy. The
woman had been a showgirl in Baltimore in the 1950s, but gave up
performance for parenthood and later had terminal cancer. Mx. Bond
remembers radiation pencil marks on her neck, but also blood-red
fingernails and a turban on her head. ``When I first met her, she
started doing a soft-shoe in my friend's bedroom, and she said, `I
started out as a dancer in the 50s, and I still got it,''' Mx. Bond
said. ``And that became one of Kiki's signature lines.''

The woman, like Kiki, was also very opinionated about politics. One day,
she was sick and lying in bed, Mx. Bond said, and watching President
Reagan on TV. ``She looked at me and said, `You know, the saddest day of
my life was the day John Hinckley missed.'''

\includegraphics{https://static01.graylady3jvrrxbe.onion/images/2016/04/17/arts/17KIKIHERB2/17KIKIHERB2-articleLarge.jpg?quality=75\&auto=webp\&disable=upscale}

When Kiki and Herb first began performing regularly, Mr. Mellman said,
``San Francisco was one big protest.'' People were angry about AIDS and
fighting for gay rights, and Kiki's banter catered to them. When the act
moved to New York in the late 90s, Mx. Bond wove humor with empowering
political statements and frank talk about race and other issues.

In \href{https://www.youtube.com/watch?v=LXrDCMb6HtA}{a video} from 1999
at Flamingo East in Manhattan, Kiki tells the story of her fictional
son, Bradford. It begins with a joke: ``He would say we're estranged,
perhaps.'' Then Kiki says that Bradford called to say he has AIDS. ``I
said, `You're calling me on Mother's Day to say you're dying of AIDS,'''
Kiki slurs, ``and he says, `No, Mother, I'm \emph{living} with AIDS.'''
The audience whistles and cheers, but Kiki doesn't let the political
message linger. She ends with another joke: ``I sent him a card and
wrote, `I hope you're feeling chipper, get well soon.' He could have
picked a better day.''

The act works only because of Mx. Bond and Mr. Mellman's ``100 percent
commitment'' to the characters, said the fashion designer Isaac Mizrahi,
a friend and occasional collaborator. The actress Molly Ringwald,
another friend, recalled, ``The transformation to Kiki was so complete
that it felt like she existed somewhere in the ether, and Justin was
simply channeling her.''

This method acting extends to their seemingly misguided musicianship.
Both Mx. Bond and Mr. Mellman are classically trained, and ``the way I
sing as Kiki is very difficult,'' Mx. Bond said. ``Dolly Parton said,
`It takes a lot of money to look this cheap.' Well, it takes a lot of
work to look this haphazard.''

The actress Tilda Swinton referred to their performance as a ``truly
beautiful'' partnership. ``It's a love story,'' she said, ``if one
played out through the tantalizing veils of mutual frustration and
high-decibel screaming.''

Over the years, Kiki and Herb moved from downtown basements to the
mainstream, even
\href{http://www.nytimes3xbfgragh.onion/2003/05/25/theater/theater-excerpt-kiki-herb-coup-de-theatre.html}{Off
Broadway} at the Cherry Lane Theater. In 2004, they gave a
\href{http://www.nytimes3xbfgragh.onion/2004/09/18/arts/music/swan-songs-as-a-duo-plan-lifes-second-act.html}{so-called
farewell performance} at Carnegie Hall, with guests like Mr. Mizrahi and
Jake Shears of the Scissor Sisters.

But they were back onstage the next year, touring and eventually
\href{http://www.nytimes3xbfgragh.onion/2006/08/16/theater/reviews/16kiki.html}{arriving
on Broadway}. The show was nominated for a Tony Award but lost to the
ventriloquism act
``\href{http://www.nytimes3xbfgragh.onion/2006/09/29/theater/reviews/29john.html}{Jay
Johnson: The Two and Only!}'' --- or, as Mx. Bond and Mr. Mellman said,
``We lost to a puppet.'' Then, in 2008, they gave another performance at
Carnegie Hall that would turn out to be their last in New York.

``We reached a crossroads,'' Mx. Bond said. ``To make a commitment to
those characters would have been a career choice. Joan Rivers could
never not be Joan Rivers. Robin Williams was very popular as Mork. And
he could have been that character for his entire life, but he wanted to
do other things.''

Mx. Bond wanted to be more like Mr. Williams; Mr. Mellman was fine as
Rivers. So Kiki and Herb ended suddenly, and not amicably. They didn't
speak for five years.

In their time apart, they thrived individually, pursuing projects that
for years had taken a back seat to Kiki and Herb. Mx. Bond wrote
\href{http://www.nytimes3xbfgragh.onion/2011/09/04/books/review/tango-my-childhood-backwards-and-in-high-heels-by-justin-vivian-bond-book-review.html}{a
memoir}, has had three solo art-gallery shows and is behind a fragrance,
``Afternoon of a Faun'' (named after the 1912 Nijinsky ballet, which Mx.
Bond called ``the first queer performance''). Mr. Mellman contributed to
other cabaret acts, like Ms. Everett's, and with Ms. Hanna founded the
Julie Ruin, which has a new album scheduled for release on July 8.

In 2013, Mx. Bond had a 50th birthday party at Le Poisson Rouge and
wanted to invite Mr. Mellman, but didn't. After a bittersweet email
exchange, Mx. Bond said, ``We acknowledged to each other that we missed
being together, and that we meant a lot to each other.''

Their renewed friendship, combined with Mx. Bond's 25th-anniversary
retrospective underway at Joe's Pub, made for a natural opportunity to
reunite as Kiki and Herb. And not a moment too soon for some fans, like
Ms. Swinton, who said, ``I'm longing for the sight and rhythm of Kiki's
loose-slung embonpoint and the deathless glamour of her style, updates
on Herb's mental health and her ever-enlightening analysis of the world
of political cut and thrust.''

For their reunion, Mx. Bond and Mr. Mellman have also brought on board
their longtime costume designer, Marc Happel, the director of costumes
at New York City Ballet. He has been working on Kiki and Herb's outfits
in his off hours because, he said, ``How often do you get to dress a
failed, drunk cabaret singer?''

``Seeking Asylum!'' will address where Kiki and Herb have been since
Carnegie Hall. Without giving too much away, Mx. Bond said, ``Herb had a
scenario that got complicated when he was visiting Southeast Asia as a
sexual tourist.'' Kiki has been in war-torn Syria, where ``she has a
long history with the Assad family,'' Mx. Bond said. ``But you don't
know what people are like.''

With a signature coy smirk, Mx. Bond added, ``Some of the most evil
people in the world are charming.''

Advertisement

\protect\hyperlink{after-bottom}{Continue reading the main story}

\hypertarget{site-index}{%
\subsection{Site Index}\label{site-index}}

\hypertarget{site-information-navigation}{%
\subsection{Site Information
Navigation}\label{site-information-navigation}}

\begin{itemize}
\tightlist
\item
  \href{https://help.nytimes3xbfgragh.onion/hc/en-us/articles/115014792127-Copyright-notice}{©~2020~The
  New York Times Company}
\end{itemize}

\begin{itemize}
\tightlist
\item
  \href{https://www.nytco.com/}{NYTCo}
\item
  \href{https://help.nytimes3xbfgragh.onion/hc/en-us/articles/115015385887-Contact-Us}{Contact
  Us}
\item
  \href{https://www.nytco.com/careers/}{Work with us}
\item
  \href{https://nytmediakit.com/}{Advertise}
\item
  \href{http://www.tbrandstudio.com/}{T Brand Studio}
\item
  \href{https://www.nytimes3xbfgragh.onion/privacy/cookie-policy\#how-do-i-manage-trackers}{Your
  Ad Choices}
\item
  \href{https://www.nytimes3xbfgragh.onion/privacy}{Privacy}
\item
  \href{https://help.nytimes3xbfgragh.onion/hc/en-us/articles/115014893428-Terms-of-service}{Terms
  of Service}
\item
  \href{https://help.nytimes3xbfgragh.onion/hc/en-us/articles/115014893968-Terms-of-sale}{Terms
  of Sale}
\item
  \href{https://spiderbites.nytimes3xbfgragh.onion}{Site Map}
\item
  \href{https://help.nytimes3xbfgragh.onion/hc/en-us}{Help}
\item
  \href{https://www.nytimes3xbfgragh.onion/subscription?campaignId=37WXW}{Subscriptions}
\end{itemize}
