\href{/section/theater}{Theater}\textbar{}The Gaudy, Glittery,
Gorgeously Subversive World of Taylor Mac

\url{https://nyti.ms/2cMB5vD}

\begin{itemize}
\item
\item
\item
\item
\item
\end{itemize}

\includegraphics{https://static01.graylady3jvrrxbe.onion/images/2016/09/18/arts/18COVER1/18COVER1-articleLarge-v2.jpg?quality=75\&auto=webp\&disable=upscale}

Sections

\protect\hyperlink{site-content}{Skip to
content}\protect\hyperlink{site-index}{Skip to site index}

\hypertarget{the-gaudy-glittery-gorgeously-subversive-world-of-taylor-mac}{%
\section{The Gaudy, Glittery, Gorgeously Subversive World of Taylor
Mac}\label{the-gaudy-glittery-gorgeously-subversive-world-of-taylor-mac}}

This downtown performer is hellbent on making us look at the hidden
history of our favorite songs. In platform heels. For 24 hours.

Credit...Jesse Dittmar for The New York Times

Supported by

\protect\hyperlink{after-sponsor}{Continue reading the main story}

By
\href{https://www.nytimes3xbfgragh.onion/by/alexandra-jacobs}{Alexandra
Jacobs}

\begin{itemize}
\item
  Sept. 14, 2016
\item
  \begin{itemize}
  \item
  \item
  \item
  \item
  \item
  \end{itemize}
\end{itemize}

BREWSTER, N.Y. --- A baby was crying in a makeshift concert hall here, a
wooden barn that felt like a Swedish sauna in the August heat. Already
ringed in purple glitter, the eyes of the entertainer
\href{http://www.taylormac.org/}{Taylor Mac} lit up further, with
opportunity.

``If you feel agitated that a baby is talking back --- leave,'' he said,
alluding to Donald J. Trump's
\href{http://www.nytimes3xbfgragh.onion/2016/08/03/us/politics/donald-trump-baby.html}{tot
showdown} a few days earlier. ``When I say that I love it when a baby
makes noise in the house, I actually mean it.''

His outfit gave his words extra force: white platform pumps, an
asymmetrical bodysuit made of clashing patterned fabric strips, a hooded
black coat appliquéd with cassette tapes, a headdress of three
cantilevered Styrofoam skulls whose eye sockets dripped with tinsel
(meant to symbolize the tears caused by the AIDS epidemic) and a faceful
of melting makeup.

Though his performances are not generally recommended for minors, Mr.
Mac has a penchant for welcoming --- nay, demanding --- audience
participation. Soon he was instructing several strangers to improvise a
group interpretive dance in a hayloft. They complied with surprising
gusto, arms waving, one pair of legs cycling in the air above the bales.
Over the course of three hours, he also sang a mash-up of Robert
Palmer's ``Addicted to Love'' and Diamanda Galás's ``Let's Not Chat
About Despair''; accompanied ballads of his own composition on a
cigar-box ukulele; and described the dream, involving Pegasus and the
actress Maggie Smith, that supposedly provoked his first nocturnal
emission.

By any conventional measure, this would be a long night in the theater,
but Mr. Mac, who is 43 but has the stamina of a man half his age, is not
only defying conventional measures but also breaking the yardstick over
his knee --- and then, perhaps, upcycling its fragments for an
outlandish costume.

He is both a throwback to an era when artists prided themselves on being
outsiders challenging the establishment, and an ideal avatar of current
cut-and-paste aesthetics. Queer in pretty much every sense of the word,
he has emerged as a high priest of nonconformity with a devoted
congregation of the disenfranchised, as well as a circus ringleader who
wants everyone, even the frat boy, to find the freak within.

\includegraphics{https://static01.graylady3jvrrxbe.onion/images/2016/09/13/multimedia/taylor-mac-civil-war/taylor-mac-civil-war-videoSixteenByNineJumbo1600.jpg}

``This is what I hope to be: a bridge between the insane and the
normative,'' he said.

Supersizing performances has become something of a fad among downtown
theater artists and other experimentalists. But Mr. Mac's latest project
makes recent examples of cultural endurance, like Marina Abramovic's
sit-fest at the Museum of Modern Art, or the six-and-a-half-hour
``Gatz'' production, seem like a hike in the park.

For five years, Mr. Mac has been preparing ``A 24-Decade History of
Popular Music,'' alloting an hour to each decade, from the American
Revolution to the present day. It has been touring in three-hour
segments, which will begin playing at
\href{http://stannswarehouse.org/show/taylor-mac-24-decade-history-popular-music-1776-2016/}{St.
Ann's Warehouse} starting on Thursday, Sept. 15. On Oct. 8, beginning at
noon, the ticket buyer paying \$400 and pledging not to leave will be
treated to an all-day, all-night extravaganza of acrobats, burlesque
performers, puppets, a choir, a marching band and 24 back-up musicians
drifting away, one per decade, till Mr. Mac stands alone. (There will
also be vegetarian meals, bathrooms, a medical tent, and a smattering of
cots and hammocks).

She, he, or they --- Mr. Mac prefers the pronoun ``judy,'' as in
Garland, though that gets awkward fast --- will be pressured into
activities that might include, more mildly, leaning your head on the
shoulder of the person one seat over; playing beer pong with a stranger
across the aisle; or (during a segment about white flight after World
War II) rearranging seats to simulate racial segregation.

It is all part of a rigorous interrogation of who and what determines
what is ``popular,'' from ``Yankee Doodle'' to Tin Pan Alley to Top 40
and beyond.

``There's some sedition in it,'' Mr. Mac said the morning after the barn
performance, his face scrubbed fresh, though bits of glitter clung to it
still. ``It's all about inspiring you to revolt against a government, or
an institution, or your obstinate sense of self.''

He calls the ``24-Decade History'' a ``radical faerie realness ritual,''
drawing elements from the \href{http://www.radfae.org/}{countercultural
movement} known for its ``heart circles'' conducted in natural settings.

``We're making a community event,'' said his co-director,
\href{http://www.nytimes3xbfgragh.onion/2015/09/19/theater/the-fleas-new-artistic-director-is-dreaming-big.html}{Niegel
Smith}. ``This community is anyone who is interested and engaged with
the question of America. Anyone who has a deep interest in how a nation
has formed and struggled with itself.''

\includegraphics{https://static01.graylady3jvrrxbe.onion/images/2016/09/18/arts/18TAYLORMACSUB1/0918TaylorMacSub1-articleLarge.jpg?quality=75\&auto=webp\&disable=upscale}

That the community is likely to be exhausted, or even to break down,
during the St. Ann's marathon is part of the point, Mr. Mac said.

But there is also space, Mr. Smith added, for those ``who just want a
good time.''

Mr. Mac's flamboyance --- his plumage --- and his fierce political
convictions sometimes threaten to overshadow his more conventional
talents.

He has a recurring act with Mandy Patinkin, to whom he has taught the
Queen song ``Bohemian Rhapsody.'' (Directed by Susan Stroman, ``The Last
Two People on Earth: An Apocalyptic Vaudeville'' will play at the
\href{https://www.kennedy-center.org/calendar/event/TRTSM}{Kennedy
Center} in the spring.) He is a classically trained actor who has logged
starring roles in productions of
\href{http://www.nytimes3xbfgragh.onion/2013/02/09/theater/reviews/good-person-of-szechwan-with-taylor-mac-at-la-mama.html?_r=0}{Brecht}
and
\href{http://www.nytimes3xbfgragh.onion/2012/04/30/theater/reviews/a-midsummer-nights-dream-at-classic-stage-company.html}{Shakespeare}.
And he is a prolific playwright who wrote one of the most acclaimed Off
Broadway works of last season:
\href{http://www.nytimes3xbfgragh.onion/2015/11/09/theater/review-hir-sorts-through-a-family-in-transition.html}{``Hir,''}
a kitchen-sink piece about a nuclear suburban family confronting the
sudden plasticity of gender in modern culture (one character vomits into
a kitchen sink). \emph{Judy, Judy, Judy!}

Loretta Greco, the artistic director of the Magic Theater in San
Francisco, which developed ``Hir,'' likened its scathing indictment of
the American dream to that in Sam Shepard's
\href{http://www.nytimes3xbfgragh.onion/2016/02/18/theater/review-in-shepards-buried-child-a-father-and-family-dissolve-into-darkness.html}{``Buried
Child.''} But ``he's not pissed off,'' she said of Mr. Mac. ``He's
all-inclusive. He doesn't want to leave people out of the party; he
knows what that feels like all too well.''

Pigeonhole Mr. Mac at your peril. After a writer called him ``Ziggy
Stardust meets Tiny Tim,'' Mr. Mac responded with a 2010 cabaret show,
``Comparison Is Violence,'' in which he played only their songs.

That hasn't stopped his admirers from trying. So various are his effects
that RoseLee Goldberg, the
\href{http://www.nytimes3xbfgragh.onion/2005/11/04/arts/design/performance-art-gets-its-biennial.html}{performance
art doyenne}, places him in the same tradition as the downtown pioneers
Ethyl Eichelberger and Jack Smith; the dancer-choreographer Bill T.
Jones invokes Andy Warhol, Lucille Ball and Bozo the Clown; Tim Sanford,
the artistic director of Playwrights Horizons which staged ``Hir'' in
New York, likens him to Bruce Springsteen. (Cabaret \emph{that}, sister
girlfriend.)

``He's like the Boss Lady in that he has this incredible stamina, not to
mention the training to be able to sing like that so beautifully,'' Mr.
Sanford said. ``In his concerts, Taylor's conscious of the fact that
he's got all walks of life with him, and he doesn't put people down.''

Image

The costume designer Matthew Flower, a.k.a. Machine Dazzle, left, with
Taylor Mac at St. Ann's Warehouse.Credit...Jesse Dittmar for The New
York Times

Perhaps surprisingly for those who have watched Mr. Mac manhandle minors
onstage --- at one performance in the Prospect Park bandshell in
Brooklyn, he ordered two young boys to call each other girls before a
crowd of thousands --- he has also been commissioned by Peter Brosius,
the artistic director of the
\href{https://www.childrenstheatre.org/}{Children's Theater Company} of
Minneapolis, to create a play for teenagers. Derived from Aristophanes'
``The Frogs'' and scheduled for 2017, ``The Fre'' will explore empathy
and take place in a mud pit, with flinging encouraged. ``It is going to
be a wildly immersive, wildly participatory piece,'' Mr. Brosius said.
``And a messy one.''

Mr. Patinkin called Mr. Mac ``a giant writer, an incredible writer.''
After performing their first duet, he said, ``I was instantly sure that
this was not, as he is sometimes referred to, a `drag queen' --- this
was the furthest thing from a drag queen. He is Lear's fool. All of this
veneer and everything is to make him look the fool. Because the fool is
the smartest man in the room.''

Mr. Mac might be the smartest man in the room, but he has often felt the
least-educated one, and the ``24-Decade History'' is, among other
things, a disgorgement of his committed autodidacticism. This was made
plain during a visit last spring to his quarters on the second floor of
the Park Avenue Armory, where he was then an artist in residence. (After
some lean years, he is now feasting at the banquet of awards,
fellowships, grants and retreats that sustain the alternative theater.)

Currently Mandarin is on the curriculum. ``How can you live in this
world and really understand it if you're not bilingual?'' Mr. Mac asked.
``I feel like I'm always catching up from an education that really I
could have learned more from if I hadn't been gay and under siege.
Constantly going to class and having people threaten my life pretty much
every day --- and you wonder why I didn't listen to what the teacher was
talking about!''

Mr. Mac's bald head, blue eyes and ready smile can call to mind Mr.
Clean. His work, though, is essentially, intentionally dirty, in both
senses of the word: frank about sex in all its manifestations and
preoccupied with the discarded and disordered. ``There's always an
element of chaos,'' he said at the Armory. Around him, suitcases spilled
over with various materials and props required for the ``24-Decade
History,'' which in various configurations has visited cities including
Dublin; Melbourne, Australia; Hanover, N.H.; and Charleston, S.C., where
there was some
\href{http://www.charlestoncitypaper.com/charleston/preaching-to-the-choir-a-flamboyant-and-gifted-artist-manages-to-be-boring/Content?oid=5220392}{heckling}.

Rainbow-colored tulle was piled onto the head of an otherwise naked
mannequin, feather boas were slung over a rolling rack, and a lace
collar awaited its fate near a Singer sewing machine.

Mr. Mac described with some reluctance his upbringing as a Christian
Scientist in Stockton, Calif., a city better known as the birthplace of
tank tread than creative professionals (though the 1990s indie-rock band
Pavement originated there).

Image

Cameron Scoggins, Kristine Nielsen and Daniel Oreskes in Mr. Mac's
``Hir'' at Playwrights Horizons.Credit...Sara Krulwich/The New York
Times

His father, Robert Mac Bowyer, who had been an Army lieutenant during
the Vietnam War, died in a motorcycle accident when Taylor was almost 4
and his sister, Robin, was 6. (Letters Mr. Bowyer received from women
while stationed in Vietnam were the basis for his son's
\href{http://www.nytimes3xbfgragh.onion/2007/09/29/theater/reviews/29ladi.html}{2007
play}, ``The Young Ladies Of.'')

His mother, Joy Aldrich, remarried, but according to both children,
their relationship with the now ex-stepfather was unhappy. ``He was
struggling; he was doing the best he could at the time, and he wasn't
very good at it,'' Mr. Mac said.

Dismayed by the elimination of arts programs in the local schools, Ms.
Aldrich began inviting neighborhood children over for creative, messy
projects like papier-mâché and collages, eventually leasing a building
to teach them. She encouraged reading. ``We were big fans of Shel
Silverstein's `Where the Sidewalk Ends,''' Robin Bowyer said in a
telephone interview. She also fondly recalled the first time he borrowed
a dress from her: ``More of a Hallmark moment than his coming out!''

In a more formal learning environment, Taylor suffered from bullying,
though he found refuge in community theater. At 13, he saw ``Cats'' on a
class trip to New York. ``All the Andrew Lloyd Webber ones were kind of
semisafe,'' Mr. Mac said of his teachers' attitudes toward musicals.
``We knew about `Chorus Line,' but even that was --- there were
\emph{homosexuals} in that! That was dangerous. Healthy theater was
Rodgers and Hammerstein.\emph{''}

Mr. Mac attended San Francisco State University for about a year, then
dropped out and joined a group of activists whose long protest march
would become the basis for his 2011
play,\href{http://www.nytimes3xbfgragh.onion/2011/01/21/theater/reviews/21walk.html}{``The
Walk Across America for Mother Earth.''} After eight months singing
``Runaround Sue'' in a poodle costume for the long-running revue ``Beach
Blanket Babylon,'' he applied and was accepted to the American Academy
of Dramatic Arts in New York.

Arriving the day before his 21st birthday, he proceeded straight to
Times Square and got a half-price ticket to see Stephen Sondheim's
``Passion.'' It was the start of a long remedial binge on the city's
theater offerings, from La MaMa to Lincoln Center.

Moored for a while in Park Slope, Brooklyn, he cater-waited high-end bar
mitzvahs, weddings and benefits, once serving the Dalai Lama and the
socialite Jocelyn Wildenstein at Cipriani in the same night. He temped
in law offices, spending hours highlighting documents pertaining to the
CBS-Viacom merger. And he mopped floors at
\href{http://www.dramabookshop.com/}{the Drama Book Shop}, where Bette
Midler and the playwright Dael Orlandersmith came in one day at the same
time and began to chat. ``I saw them and thought, `Well, I want to be
some kind of combination of what those two people are, and here I am
cleaning the bathroom after they use it,''' Mr. Mac remembered. ``So I
quit.''

Image

Taylor Mac performing part of ``A 24-Decade History of Popular Music''
at New York Live Arts in 2015.Credit...Sara Krulwich/The New York Times

He'd been discovering New York's drag scene at places like the Pyramid
Club, guided by the Mother Flawless Sabrina, star of the 1968
documentary ``The Queen.'' Seeing John Cameron Mitchell's ``Hedwig and
the Angry Inch'' --- which combined furious glam rock and a poignant
coming-of-age story --- was another aha moment. ``I was kind of sobbing
because I thought `Oh, this is my calling,''' Mr. Mac said.

\href{http://www.nytimes3xbfgragh.onion/2016/01/06/arts/elizabeth-swados-creator-of-socially-conscious-musicals-is-dead-at-64.html?_r=0}{Elizabeth
Swados} cast him in a musical called ``Jabu'' and helped him get his
first grant, for \$7,000. ``It changed everything in my life,'' he said.
Festivals followed --- Edinburgh, Under the Radar--- and before long Mr.
Mac was commanding the microphone at fancy wingdings rather than
hoisting a tray.

Along with Mr. Smith and Matt Ray, his music director, arranger and
patient ``straight man,'' Mr. Mac's main professional sidekick is the
costume designer Matthew Flower, a.k.a. Machine Dazzle, who first worked
with him on a play called ``The Lily's Revenge'' (2009), a consideration
of gay marriage, nostalgia and mourning with a comparatively brisk
running time of five hours.

Arriving at the Armory, Mr. Flower opened some windows and showed off
some telling details of his creations for the coming St. Ann's show. The
``white flight'' costume somehow incorporated Mr. Potato Head, long
strips of paper 3-D glasses and bumblebees affixed to a picket fence.

``There's a penis there,'' he pointed out, ``and a vagina, too.''

Is anything ever too outré?

The two men said they were only concerned with that
\href{http://www.nytimes3xbfgragh.onion/2016/09/13/books/lionel-shriver-cultural-appropriation-brisbane-writers-festival.html}{hot-button
term} of the moment, ``appropriation.'' A friend had expressed concern
about a flapper costume whose feathers and silhouette seemed Native
American.

And Mr. Flower was fretting about an 1890s get-up themed to Jewish
tenements. ``I did ask a few Jewish people, but \ldots{} ''

Mr. Mac interjected: ``Some people felt problematic about the payot
earrings.''

Though he seems to champion an anything-goes sexuality, Mr. Mac is in a
committed relationship with Patt Scarlett, an architect with whom he
shares a one-bedroom apartment near Gramercy Park. He cares little for
possessions --- ``I always say, you can make a living in the theater if
you're willing to fall in love with verbs more than nouns,'' he said ---
and doesn't have a credit card.

Image

Taylor Mac backstage at St. Ann's Warehouse.Credit...Jesse Dittmar for
The New York Times

He is not too Spartan, however, to request a massage, as he did the
morning after the barn performance in Brewster. He had hurt his back
while doing squats to build up cardiovascular endurance for a 12-hour
warm-up concert in Poughkeepsie, N.Y., spent a night in the hospital,
gotten a walker and then done two shows from a bed on the stage.

Art isn't easy, as Mr. Sondheim wrote. But celebrity is even harder, Mr.
Mac believes, having observed a swarm descend on Meryl Streep during a
reading they did in honor of Ms. Swados. ``I don't want that in my
life,'' he said. ``I have a lovely level of celebrity, which is if
somebody recognizes me on the street and wants to talk to me, it's
usually somebody that I want to talk to.''

And for those lacking the patience to hear what he has to say?

``I grew up in your government, I grew up in your living institutions, I
grew up in your churches and stuff --- I'm asking you to spend three to
24 hours at \emph{my} church at one point in your life,'' Mr. Mac said.
``I'm not asking you to convert. Look, I spent all this time in your
environment. Spend a little time in mine.''

\hypertarget{taylor-macs-5-picks-from-his-24-decade-history}{%
\subsection{Taylor Mac's 5 Picks From His `24-Decade
History'}\label{taylor-macs-5-picks-from-his-24-decade-history}}

In ``A 24-Decade History of Popular Music'' --- which can be seen in
pieces or its 24-hour-entirety --- Taylor Mac asks his audience to
reconsider the American canon. Here he explains some selections from his
marathon set list:

\textbf{`THE RIGHTS OF WOMEN'} (1780s) ``That's from the early women's
liberation movement. Basically `God Save the Queen,' but they rewrote
the lyrics.
\href{https://www.nwhm.org/education-resources/biography/biographies/judith-sargent-murray/}{Judith
Sargent Murray} wrote this essay called `On the Equality of the Sexes,'
and we pass it out to the audience.''

\textbf{`ORPHAN CHILD'} (1830s) ``I started thinking about when people
were on that Trail of Tears. That led me to this song, a Cherokee song;
I decided to do a whole decade of children's songs.''

\textbf{`THE MIKADO'} (1880s) ``We perform all the songs from `The
Mikado,' but we set them on Mars. I felt like Gilbert and Sullivan just
wanted to make people happy with their work, and they also wanted to
address British culture, and they'd gotten into a rut, so they set it in
Japan. But what they ended up doing was creating a new kind of
minstrelsy.''

\textbf{`THE SURREY WITH THE FRINGE ON TOP'} (1940s) ``We do a ballad
version of it, and it's just me and two Nazis riding through the German
countryside together. Just my way of showing the ridiculousness of our
fascination with blondes.''

\textbf{`BIRDLAND,' BY PATTI SMITH} (1970s) ``Wilhelm Reich was the man
who invented the sex machine, and he was put in prison, and he wound up
dying in prison, and his son wrote a book called `A Book of Dreams,' and
Patti Smith wrote a song about it. Somebody that makes something, and
then another person makes something, and then another person makes
something --- I like to address that lineage. Also, I'm talking about
Stonewall, and I use it as an opportunity to talk about people sitting
in jail.''

Advertisement

\protect\hyperlink{after-bottom}{Continue reading the main story}

\hypertarget{site-index}{%
\subsection{Site Index}\label{site-index}}

\hypertarget{site-information-navigation}{%
\subsection{Site Information
Navigation}\label{site-information-navigation}}

\begin{itemize}
\tightlist
\item
  \href{https://help.nytimes3xbfgragh.onion/hc/en-us/articles/115014792127-Copyright-notice}{©~2020~The
  New York Times Company}
\end{itemize}

\begin{itemize}
\tightlist
\item
  \href{https://www.nytco.com/}{NYTCo}
\item
  \href{https://help.nytimes3xbfgragh.onion/hc/en-us/articles/115015385887-Contact-Us}{Contact
  Us}
\item
  \href{https://www.nytco.com/careers/}{Work with us}
\item
  \href{https://nytmediakit.com/}{Advertise}
\item
  \href{http://www.tbrandstudio.com/}{T Brand Studio}
\item
  \href{https://www.nytimes3xbfgragh.onion/privacy/cookie-policy\#how-do-i-manage-trackers}{Your
  Ad Choices}
\item
  \href{https://www.nytimes3xbfgragh.onion/privacy}{Privacy}
\item
  \href{https://help.nytimes3xbfgragh.onion/hc/en-us/articles/115014893428-Terms-of-service}{Terms
  of Service}
\item
  \href{https://help.nytimes3xbfgragh.onion/hc/en-us/articles/115014893968-Terms-of-sale}{Terms
  of Sale}
\item
  \href{https://spiderbites.nytimes3xbfgragh.onion}{Site Map}
\item
  \href{https://help.nytimes3xbfgragh.onion/hc/en-us}{Help}
\item
  \href{https://www.nytimes3xbfgragh.onion/subscription?campaignId=37WXW}{Subscriptions}
\end{itemize}
