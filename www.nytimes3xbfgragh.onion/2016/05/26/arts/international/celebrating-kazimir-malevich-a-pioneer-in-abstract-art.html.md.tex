Sections

SEARCH

\protect\hyperlink{site-content}{Skip to
content}\protect\hyperlink{site-index}{Skip to site index}

\href{https://www.nytimes3xbfgragh.onion/section/arts/international}{International
Arts}

\href{https://myaccount.nytimes3xbfgragh.onion/auth/login?response_type=cookie\&client_id=vi}{}

\href{https://www.nytimes3xbfgragh.onion/section/todayspaper}{Today's
Paper}

\href{/section/arts/international}{International Arts}\textbar{}Europe
Celebrates Kazimir Malevich, a Pioneer in Abstract Art

\url{https://nyti.ms/25gPsxd}

\begin{itemize}
\item
\item
\item
\item
\item
\end{itemize}

Advertisement

\protect\hyperlink{after-top}{Continue reading the main story}

Supported by

\protect\hyperlink{after-sponsor}{Continue reading the main story}

\hypertarget{europe-celebrates-kazimir-malevich-a-pioneer-in-abstract-art}{%
\section{Europe Celebrates Kazimir Malevich, a Pioneer in Abstract
Art}\label{europe-celebrates-kazimir-malevich-a-pioneer-in-abstract-art}}

\includegraphics{https://static01.graylady3jvrrxbe.onion/images/2016/05/26/arts/26iht-malevich26/26iht-malevich26-articleLarge.jpg?quality=75\&auto=webp\&disable=upscale}

By Kevin Holden Platt

\begin{itemize}
\item
  May 25, 2016
\item
  \begin{itemize}
  \item
  \item
  \item
  \item
  \item
  \end{itemize}
\end{itemize}

One hundred years after the Russian artist Kazimir Malevich shook up the
art world with the first exhibition of his Suprematist paintings,
museums across Europe, from Austria to France to Spain, are celebrating
his legacy with a series of exhibitions.

In his works, Malevich depicted planes of color speeding across an
expanding cosmos, seeking to render moments of what he called
``supreme'' worlds floating through time and space. His use of purely
abstract forms broke with centuries of artistic tradition. Malevich
introduced his vision to the world in the winter of 1915-16 with a show
in Petrograd (now St. Petersburg) called ``The Last Futurist Exhibition
of Paintings 0.10.''

``The `Zero-Ten' show is one of the most important exhibitions in the
history of Modernism,'' said Matthew Drutt, who last winter
\href{http://www.fondationbeyeler.ch/en/exhibitions/search-010}{recreated
it} for the Fondation Beyeler museum in Switzerland. As the great powers
sought to destroy each other in World War I with fearsome new weapons
like tanks, poison gas and armed airplanes, Malevich was envisioning a
postwar utopia visible only in the new world of abstract art, Mr. Drutt
said.

The Swiss show also traced the arc of Malevich's influence on
generations of artists, including Piet Mondrian, Mark Rothko and the
installation designer \href{http://www.olafureliasson.net/}{Olafur
Eliasson}.

Other museums featuring Malevich works this year include the
\href{http://www.albertina.at/jart/prj3/albertina/main.jart?rel=en\&reserve-mode=active\&content-id=1202307119323\&j-cc-node=item\&j-cc-id=1435222262094\&j-cc-item=ausstellungen\&ausstellungen_id=1435222262094}{Albertina}
in Vienna (through June 26) and a new satellite of the State
\href{http://en.rusmuseum.ru/exhibitions/mobile/jack-of-diamonds/}{Russian
Museum} in Málaga, Spain (through July 31). This fall, Malevich and
other Russian avant-garde artists will be an important part of ``Icons
of Modern Art,''
\href{http://www.theguardian.com/artanddesign/2016/feb/10/treasures-modern-art-seen-outside-russia-first-time-sergei-shchukin}{an
exhibition} at the Fondation Louis Vuitton in Paris that will display
the collection of Sergei Shchukin, a Russian textile merchant who opened
the first private galleries for modern art in Moscow in the twilight of
czarist rule.

That Malevich's provocative works survived the turbulent world in which
they were created is something of an art-world miracle. It is a story of
how art can become entangled in political movements, and the difficulty,
decades later, of determining who has a legitimate claim to that art.

In 1927 Malevich, who had been well received in Germany, brought more
than 100 of his abstract masterpieces to the Great Berlin Art
Exhibition. While there, he received a letter from his wife warning that
Stalin was escalating attacks on experimental artists, and pleading that
he defect in Berlin.

Instead, Malevich decided to save his paintings by leaving them in
Berlin and risked his life by returning to the Soviet Union to try to
get his family out.

Branded a counterrevolutionary, Malevich was jailed and barred from
leaving the Soviet Union. He died in St. Petersburg (then called
Leningrad) in 1935.

Malevich's descendants were themselves the target of the Soviet
leadership. They remained behind the Iron Curtain until the collapse of
the Soviet government in 1991 freed them to begin searching for the
paintings Malevich had hidden in Berlin.

Before leaving Germany, Malevich appointed two friends --- Alexander
Dorner, a museum director in Hanover, and Hugo Häring, an architect ---
to safeguard his artworks.

Dorner displayed the paintings until pressure from the Nazi Party
impelled him to hide the art. He risked imprisonment by showing them in
1935 to Alfred H. Barr Jr., founding director of the Museum of Modern
Art in New York, according to
\href{http://www.nytimes3xbfgragh.onion/2006/03/26/arts/design/26ridi.html?pagewanted=all\&_r=0}{Clemens
Toussaint}, a German art historian and expert on Malevich who has been
helping the artist's descendants trace and stake a claim to some of his
works.

\includegraphics{https://static01.graylady3jvrrxbe.onion/images/2016/05/26/arts/26iht-malevich26-b/26iht-malevich26-b-articleLarge.jpg?quality=75\&auto=webp\&disable=upscale}

Persuaded by Barr to lend 16 works to MoMA, Dorner smuggled them out of
Germany in shipping crates filled with technical drawings. When the
Nazis expelled Dorner from his museum post, Barr then helped Dorner
escape from Germany, Mr. Toussaint said.

Dorner entrusted the remaining canvases to Häring for safekeeping.

After the war, Häring agreed to lend 80 works to the Stedelijk Museum in
Amsterdam, which promised to restore and exhibit them. The museum said
later --- in testimony in an American lawsuit --- that Häring had agreed
to sell it the works.

In 1999, the Museum of Modern Art agreed to return one of the works to
the Malevich descendants, in a settlement brokered by Mr. Toussaint. He
said that parallel efforts to recover paintings from the Stedelijk were
unsuccessful.

Then, the Stedelijk sent 14 Malevich paintings to the United States for
a show at the Guggenheim in 2003 organized by Mr. Drutt. Just before the
paintings were to be sent back to the Netherlands, some of Malevich's
descendants filed suit in Federal District Court for the District of
Columbia to recover them.

Howard Spiegler, a lawyer who represented the descendants in the case,
Malewicz vs. City of Amsterdam, said that heads of state and even
government-owned museums like the Stedelijk had traditionally been
protected from such lawsuits by the doctrine of sovereign immunity, but
that American court rulings had whittled away at that shield.

In its
\href{https://scholar.google.com/scholar_case?case=18213435335536066321\&q=malewicz+v.+city+of+amsterdam+2007\&hl=en\&as_sdt=6,33}{decision
in the case}, the court ruled that the exhibition in the United States
of artworks taken in violation of international law could open the
foreign lending museum to a lawsuit. In rejecting the Stedelijk's claim
of immunity, Judge Rosemary M. Collyer found that there was nothing
sovereign about Amsterdam's acquisition of the Malevich paintings
``other than that it was performed by a sovereign entity.''

Evgeny Bykov, a great-grandson of Malevich, said that the decision
``gave us hope that in our case justice would be met and that in the
future it will help other families who are struggling for restitution of
masterpieces that once belonged to them.''

While preparing to appeal the district court's decision --- but facing
the prospect of losing all 14 paintings it had sent to the United States
--- the Stedelijk agreed in a settlement to return five artworks to the
Malevich family.

Museum directors across America have been pressing Congress to enact
legislation that would counter the Malevich ruling, arguing that
exposing their foreign counterparts to suits by dispossessed owners
could halt cultural exchanges.

The Senate is reviewing a bill that could again block access to the
courts by those who seek restitution for what they say are illegally
confiscated artworks, including works taken during the Cuban revolution,
said Mari-Claudia Jiménez, an expert in art law in New York. Similar
legislation was proposed in 2012 and
\href{http://www.nytimes3xbfgragh.onion/2012/05/22/arts/design/dispute-over-bill-to-protect-art-lent-to-museums.html}{met
resistance} from Jewish groups and others who believed an exemption for
Nazi-era art was too narrowly drawn.

Mr. Toussaint said that while Malevich's descendants continue their
search for canvases, they have recovered eight paintings of the 100 left
in Berlin, with one
\href{http://www.nytimes3xbfgragh.onion/2008/11/06/arts/06iht-melik5.html}{auctioned}
for \$60 million.

``For the avant-garde art movement, Kazimir Malevich is a guru,'' Mr.
Toussaint said. ``His works disappeared into the dictatorships but later
re-emerged --- like a phoenix --- in a very mystical way.''

Showcasing these masterpieces in Paris will fulfill one of the artist's
most powerful dreams, he said: Malevich was planning to spirit his
family and canvases off to a new life in Paris just before being
detained in one of Stalin's political prisons.

Advertisement

\protect\hyperlink{after-bottom}{Continue reading the main story}

\hypertarget{site-index}{%
\subsection{Site Index}\label{site-index}}

\hypertarget{site-information-navigation}{%
\subsection{Site Information
Navigation}\label{site-information-navigation}}

\begin{itemize}
\tightlist
\item
  \href{https://help.nytimes3xbfgragh.onion/hc/en-us/articles/115014792127-Copyright-notice}{©~2020~The
  New York Times Company}
\end{itemize}

\begin{itemize}
\tightlist
\item
  \href{https://www.nytco.com/}{NYTCo}
\item
  \href{https://help.nytimes3xbfgragh.onion/hc/en-us/articles/115015385887-Contact-Us}{Contact
  Us}
\item
  \href{https://www.nytco.com/careers/}{Work with us}
\item
  \href{https://nytmediakit.com/}{Advertise}
\item
  \href{http://www.tbrandstudio.com/}{T Brand Studio}
\item
  \href{https://www.nytimes3xbfgragh.onion/privacy/cookie-policy\#how-do-i-manage-trackers}{Your
  Ad Choices}
\item
  \href{https://www.nytimes3xbfgragh.onion/privacy}{Privacy}
\item
  \href{https://help.nytimes3xbfgragh.onion/hc/en-us/articles/115014893428-Terms-of-service}{Terms
  of Service}
\item
  \href{https://help.nytimes3xbfgragh.onion/hc/en-us/articles/115014893968-Terms-of-sale}{Terms
  of Sale}
\item
  \href{https://spiderbites.nytimes3xbfgragh.onion}{Site Map}
\item
  \href{https://help.nytimes3xbfgragh.onion/hc/en-us}{Help}
\item
  \href{https://www.nytimes3xbfgragh.onion/subscription?campaignId=37WXW}{Subscriptions}
\end{itemize}
