Sections

SEARCH

\protect\hyperlink{site-content}{Skip to
content}\protect\hyperlink{site-index}{Skip to site index}

\href{https://www.nytimes3xbfgragh.onion/section/books/review}{Book
Review}

\href{https://myaccount.nytimes3xbfgragh.onion/auth/login?response_type=cookie\&client_id=vi}{}

\href{https://www.nytimes3xbfgragh.onion/section/todayspaper}{Today's
Paper}

\href{/section/books/review}{Book Review}\textbar{}What's the Best Book,
New or Old, You Read This Year?

\url{https://nyti.ms/2ig0sYH}

\begin{itemize}
\item
\item
\item
\item
\item
\item
\end{itemize}

Advertisement

\protect\hyperlink{after-top}{Continue reading the main story}

Supported by

\protect\hyperlink{after-sponsor}{Continue reading the main story}

\href{/column/bookends}{Bookends}

\hypertarget{whats-the-best-book-new-or-old-you-read-this-year}{%
\section{What's the Best Book, New or Old, You Read This
Year?}\label{whats-the-best-book-new-or-old-you-read-this-year}}

Dec. 21, 2016

\begin{itemize}
\item
\item
\item
\item
\item
\item
\end{itemize}

In this special year-end edition of Bookends, our columnists share their
favorite reading experience of 2016.

What was the best book you read this year? Let us know your answers in
the comments below. We'll share some of them on a future episode of the
Book Review's podcast.

\includegraphics{https://static01.graylady3jvrrxbe.onion/images/2015/12/07/books/review/07bookends-one/07bookends-one-articleLarge.jpg?quality=75\&auto=webp\&disable=upscale}

\textbf{Siddhartha Deb:}

At year's end, I'm remembering Cormac McCarthy's ``Blood Meridian.''
Unflinching in its portrayal of settler colonialism and so familiar in
its violence, racism and twisted masculinity, it is most memorable for
me in its portrait of Judge Holden, the Devil incarnate, perched on a
rock and waiting for us to pass by.

\emph{Siddhartha Deb's most recent book is ``The Beautiful and the
Damned: A Portrait of the New India.''}

\textbf{Rivka Galchen:}

The Italian writer Natalia Ginzburg lived through the rise of fascism,
which left her widowed with three small children. Among my favorite of
her works translated by Lynne Sharon Schwartz and collected in ``A Place
to Live: Selected Essays of Natalia Ginzburg'' are ``Winter in the
Abruzzi'' and ``The Baby Who Saw Bears.''

\emph{Rivka Galchen's most recent book is ``Little Labors.''}

\textbf{Alice Gregory:}

Rebecca Solnit's ``River of Shadows: Eadweard Muybridge and the
Technological Wild West'' is the most impressive book I read this year
or maybe any year. It somehow manages to deploy the most specific and
peculiar facts while telling a story that's about everything --- art,
politics, history, science, philosophy. It blows my mind that one person
wrote it.

\emph{Alice Gregory is a contributing editor at T: The New York Times
Style Magazine.}

\textbf{Zoë Heller:}

I really enjoyed Emma Cline's debut novel, ``The Girls.'' Cline writes
lovely, noticing sentences, and her story about the charismatic power of
an evil cult leader turned out to be a not altogether inappropriate
fable for 2016.

\emph{Zoë Heller is the author of ``Everything You Know,'' ``Notes on a
Scandal'' and ``The Believers.''}

Image

From left: Anna Holmes, Leslie Jamison, Adam Kirsch, Thomas
Mallon.Credit...Illustrations by R. Kikuo Johnson

\textbf{Anna Holmes:}

Colson Whitehead's slave-narrative novel, ``The Underground Railroad'':
a defiant, gorgeous triumph of human imagination and empathy that can be
interpreted as a commentary on the past, a reckoning with the present or
a provocation of the future. (It's probably all three.)

\emph{Anna Holmes is an editorial executive at First Look Media and the
editor of two books, including ``The Book of Jezebel.''}

\textbf{Leslie Jamison:}

This year I reread Michelle Alexander's ``The New Jim Crow,'' a
necessary account of the systemic racism embedded in our justice system.
Alexander's book feels more vital now than ever. It's a protest against
the persecution that has persistently operated under alibis of security
and justice --- a protest we need to keep making as powerfully as we
can.

\emph{Leslie Jamison is the author of ``The Empathy Exams.''}

\textbf{Adam Kirsch:}

``Moonglow,'' Michael Chabon's new novel, is my favorite of his books
and one of the most memorable novels I read this year. Using
autobiography as a launchpad and then taking off for the moon, Chabon
offers a funny, moving and dramatic tribute to his grandparents and
their American generation.

\emph{Adam Kirsch is a poet, critic and columnist for Tablet magazine.}

\textbf{Thomas Mallon:}

In ``The Dream Life of Astronauts,'' Patrick Ryan flies further into a
little fictional empyrean he's made all his own. Peopled by kookily sad
denizens of Florida's Space Coast, whose dreams rarely achieve liftoff
without crashing and burning, Ryan's stories are filled with a wan
tenderness and a spectacular lack of condescension.

\emph{Thomas Mallon's most recent book is ``Finale: A Novel of the
Reagan Years.''}

Image

From left: Ayana Mathis, Charles McGrath, Pankaj Mishra, Benjamin
Moser.Credit...Illustrations by R. Kikuo Johnson

\textbf{Ayana Mathis:}

``The Plague of Doves'' takes as its subject the residents of a North
Dakota town abutting an Ojibwe reservation. The novel is a kaleidoscope
of voices, imagery and memories. Louise Erdrich's prose evokes the
tumult of lived experience and ancestral trauma. She reminds us we are
all yearning creatures, subject to forces set in motion long before we
were born.

\emph{Ayana Mathis is the author of ``The Twelve Tribes of Hattie.''}

\textbf{Charles McGrath:}

``Caught,'' by Henry Green. First published in 1943 and now reissued in
the New York Review Classics series, ``Caught'' manages the improbable
feat of being both a harrowing war story of London during the Blitz and
a sharply observed comedy about social class. Green was a silver-spoon
aristocrat, but his ear for common speech was as keen as Dickens's.

\emph{Charles McGrath was the editor of the Book Review from 1995 to
2004.}

\textbf{Pankaj Mishra:}

David Kennedy's ``A World of Struggle: How Power, Law, and Expertise
Shape Global Political Economy'' describes our world more accurately
than any book I have read this year. Kennedy offers no clear
prescriptions. Yet he clarifies that understanding how this world of
injustice and inequality came about is the essential first step toward a
democratic alternative.

\emph{Pankaj Mishra's next book, ``Age of Anger,'' will be published in
February.}

\textbf{Benjamin Moser:}

In ``The Fall Of Language in the Age of English,'' Minae Mizumura shows,
better than anyone ever has, how English is wrecking other languages ---
reducing even great literary languages, including Japanese and French,
to local dialects --- and makes a vigorous case for the superiority of
the written over the spoken word.

\emph{Benjamin Moser is the author of ``Why This World: A Biography of
Clarice Lispector.''}

Image

From left: James Parker, Francine Prose, Liesl Schillinger, Dana
Stevens.Credit...Illustrations by R. Kikuo Johnson

\textbf{James Parker:}

How I wish I'd written Max Porter's ugly-beautiful post-Ted Hughes
polyphonic spree of a novel ``Grief Is the Thing With Feathers''; I
listened to an interview with the author and could barely hear his
(cultured, friendly) voice through the electrical envy-storm that was
writhing in purple bands across my forebrain.

\emph{James Parker is a contributing editor at The Atlantic.}

\textbf{Francine Prose:}

``Pedro Páramo,'' --- Juan Rulfo's 1955 masterpiece --- packs the scope
and sweep of an epic into just over 120 pages. It has the beauty of a
lyric poem and manages the dazzling magic trick of blurring the line
between life and death. Set in a rural Mexican ghost town, Rulfo's book
shows us how seamlessly fiction can combine the regional and the
universal.

\emph{Francine Prose's most recent novel is ``Mister Monkey.''}

\textbf{Liesl Schillinger:}

In July, during the national conventions, I read a stunning debut that
resurrects the violence and anger of the 1968 Chicago riots, carrying it
forward into the present day through the story of two Midwesterners
addicted to virtual-reality games. It's ``The Nix,'' by Nathan Hill, the
first book I've read in two decades that earns the title Great American
Novel.

\emph{Liesl Schillinger is a critic and translator and the author of
``Wordbirds.''}

\textbf{Dana Stevens:}

I read Rebecca Solnit's ``A Field Guide to Getting Lost'' under ideal
conditions: alone in a strange city, at bus stops and in public parks,
surrounded by families speaking foreign yet familiar languages. Solnit's
meditation on lostness as a peculiarly American experience animated my
thinking and writing for the rest of what turned out to be a very long
year.

\emph{Dana Stevens is the film critic at Slate and a cohost of the Slate
Culture Gabfest.}

Advertisement

\protect\hyperlink{after-bottom}{Continue reading the main story}

\hypertarget{site-index}{%
\subsection{Site Index}\label{site-index}}

\hypertarget{site-information-navigation}{%
\subsection{Site Information
Navigation}\label{site-information-navigation}}

\begin{itemize}
\tightlist
\item
  \href{https://help.nytimes3xbfgragh.onion/hc/en-us/articles/115014792127-Copyright-notice}{©~2020~The
  New York Times Company}
\end{itemize}

\begin{itemize}
\tightlist
\item
  \href{https://www.nytco.com/}{NYTCo}
\item
  \href{https://help.nytimes3xbfgragh.onion/hc/en-us/articles/115015385887-Contact-Us}{Contact
  Us}
\item
  \href{https://www.nytco.com/careers/}{Work with us}
\item
  \href{https://nytmediakit.com/}{Advertise}
\item
  \href{http://www.tbrandstudio.com/}{T Brand Studio}
\item
  \href{https://www.nytimes3xbfgragh.onion/privacy/cookie-policy\#how-do-i-manage-trackers}{Your
  Ad Choices}
\item
  \href{https://www.nytimes3xbfgragh.onion/privacy}{Privacy}
\item
  \href{https://help.nytimes3xbfgragh.onion/hc/en-us/articles/115014893428-Terms-of-service}{Terms
  of Service}
\item
  \href{https://help.nytimes3xbfgragh.onion/hc/en-us/articles/115014893968-Terms-of-sale}{Terms
  of Sale}
\item
  \href{https://spiderbites.nytimes3xbfgragh.onion}{Site Map}
\item
  \href{https://help.nytimes3xbfgragh.onion/hc/en-us}{Help}
\item
  \href{https://www.nytimes3xbfgragh.onion/subscription?campaignId=37WXW}{Subscriptions}
\end{itemize}
