Sections

SEARCH

\protect\hyperlink{site-content}{Skip to
content}\protect\hyperlink{site-index}{Skip to site index}

\href{https://www.nytimes3xbfgragh.onion/section/us}{U.S.}

\href{https://myaccount.nytimes3xbfgragh.onion/auth/login?response_type=cookie\&client_id=vi}{}

\href{https://www.nytimes3xbfgragh.onion/section/todayspaper}{Today's
Paper}

\href{/section/us}{U.S.}\textbar{}All This Talk of Voter Fraud? Across
U.S., Officials Found Next to None

\url{https://nyti.ms/2hXjTZS}

\begin{itemize}
\item
\item
\item
\item
\item
\end{itemize}

Advertisement

\protect\hyperlink{after-top}{Continue reading the main story}

Supported by

\protect\hyperlink{after-sponsor}{Continue reading the main story}

\hypertarget{all-this-talk-of-voter-fraud-across-us-officials-found-next-to-none}{%
\section{All This Talk of Voter Fraud? Across U.S., Officials Found Next
to
None}\label{all-this-talk-of-voter-fraud-across-us-officials-found-next-to-none}}

\includegraphics{https://static01.graylady3jvrrxbe.onion/images/2016/12/16/us/16FRAUD/15FRAUD-articleInline.jpg?quality=75\&auto=webp\&disable=upscale}

By \href{https://www.nytimes3xbfgragh.onion/by/michael-wines}{Michael
Wines}

\begin{itemize}
\item
  Dec. 18, 2016
\item
  \begin{itemize}
  \item
  \item
  \item
  \item
  \item
  \end{itemize}
\end{itemize}

\emph{{[}Read more about}
\href{https://www.nytimes3xbfgragh.onion/article/mail-in-vote-fraud-ballot.html}{\emph{fake
ballots, mail-in voting and voter fraud}}\emph{.{]}}

After all the allegations of rampant voter fraud and claims that
millions had voted illegally, the people who supervised the general
election last month in states around the nation have been adding up how
many credible reports of fraud they actually received. The overwhelming
consensus: next to none.

In an election in which \href{http://www.electproject.org/2016g}{more
than 137.7 million} Americans cast ballots, election and law enforcement
officials in 26 states and the District of Columbia ---
Democratic-leaning, Republican-leaning and in-between --- said that so
far they knew of no credible allegations of fraudulent voting. Officials
in another eight states said they knew of only one allegation.

A few states reported somewhat larger numbers of fraud claims that were
under review. Tennessee counted 40 credible allegations out of some 4.3
million primary and general election votes. In Georgia, where more than
4.1 million ballots were cast, officials said they had opened 25
inquiries into ``suspicious voting or election-related activity.''

But inquiries to all 50 states (every one but Kansas responded) found no
states that reported indications of widespread fraud. And while
additional allegations could surface as states wind up postelection
reviews, their conclusions are unlikely to change significantly.

The findings unambiguously debunk repeated statements by President-elect
Donald J. Trump that
\href{https://www.nytimes3xbfgragh.onion/2016/11/27/us/politics/trump-adviser-steps-up-searing-attack-on-romney.html}{millions
of illegal voters} backed his Democratic opponent, Hillary Clinton. They
also refute warnings by Republican governors in Maine and North Carolina
that election results could not be trusted.

And they underscore what
\href{http://www.scholarsstrategynetwork.org/sites/default/files/ssn_key_findings_minnite_on_the_myth_of_voter_fraud.pdf}{researchers
and scholars} have said for years: Fraud by voters casting ballots
illegally is a minuscule problem, but a potent political weapon.

``The old notion that somehow there are all these impostors out there,
people not eligible to vote that are voting --- it's a lie,'' said
\href{https://www.brookings.edu/experts/thomas-e-mann/}{Thomas E. Mann},
a resident scholar at the Institute of Governmental Studies at the
University of California, Berkeley. ``But it's what's being used in the
states now to impose increased qualifications and restrictions on
voting.''

In a year that unfolded amid wild fraud claims, the reports from
election officials were strikingly humdrum.

``Nothing at all, really,'' said Jim Tenuto, the assistant executive
director of the Illinois State Board of Elections.

``We only had one,'' said Laura Strimple, Nebraska's assistant secretary
of state. ``It hasn't been confirmed.''

``We haven't received any complaints to our office or any word of
suspicious activity, and we would definitely hear it,'' said Matt
Roberts, the spokesman for Arizona's secretary of state.

Some state officials qualified their estimates, saying they had not yet
reviewed all questionable ballots, or that voter fraud was a local
matter that was usually --- but not always --- reported to them. Ohio
officials declined to offer totals, saying they were still assessing
complaints; Pennsylvania and Mississippi officials said they did not
track fraud cases.

Many Republicans insist significant problems persist, and that much
fraud goes undetected. The conservative Heritage Foundation has
published \href{http://bit.ly/2ekpokZ}{online} what it calls an
incomplete list of voter fraud and other election-law violations dating
to 1982, roughly 450 cases involving both voters and public officials.
Properly written, laws requiring voters to display IDs ``could increase
the fairness of the election process for everyone, regardless of
party,'' Hans von Spakovsky, the manager of the foundation's Election
Law Reform Initiative, said.

Voting-rights advocates note that the current system caught those
violations --- and that the numbers, less than one per state per year
--- constitute a tiny sliver of the millions of votes cast in any
election cycle.

No one doubts that election fraud has occurred and needs to be
monitored. Election outcomes have been changed by officials who altered
vote tallies, and in theory hackers could pick winners by playing havoc
with voter rolls, voting machines or electronic reporting networks. But
voter fraud, in which someone deliberately casts an invalid ballot or a
ballot under someone else's name, is exceedingly rare.

Its prevalence is at the heart of the debate on restrictions like voter
ID. Critics say that cracking down on abuses that barely exist can cost
hundreds of thousands of people or more --- often the poor and
minorities --- their ability to vote.

For example, a federal court in 2014 found that in Wisconsin an
estimated 300,000 voters who had already registered did not have any of
the required IDs.

Federal courts have altered or nullified the strictest voter-ID laws,
saying they suppress turnout among minorities, who are most likely to
lack a required ID.

This year has set new benchmarks for accusations about tainted
elections.

In Maine, Gov. Paul LePage, a Republican, this month certified state
elections, in which Mrs. Clinton won, but refused to call the vote count
accurate. (Maine's secretary of state says no voter fraud was detected.)

In North Carolina, Gov. Pat McCrory, also a Republican, charged that
Democratic-driven fraud in more than half the state's 100 counties
contributed to his re-election defeat by the state attorney general, Roy
Cooper.

Mr. McCrory conceded on Dec. 6. But for three weeks before that, he and
others repeatedly accused Democrats of concocting illegal absentee
ballots and relying on votes by criminals, the dead and two-time voters.

The accusations proved largely spurious. Of more than 4.7 million
ballots cast, election officials uncovered 25 apparently invalid
\href{http://www.newsobserver.com/news/politics-government/election/article116789083.html}{votes
by felons}; whether they knew they were ineligible to vote is unclear.
State and county election boards, all led by Republican majorities,
threw out most of the remaining challenges. So-called dead voters
actually had died after casting early votes; two-time voters turned out
to be people with similar or identical names.

Mr. Trump falsely asserted on Twitter that he would have won the popular
vote --- Mrs. Clinton received
\href{http://www.nytimes3xbfgragh.onion/elections/results/president}{some
2.8 million more votes} --- ``if you deduct the millions of people who
voted illegally.''

But even Republican leaders who once disavowed Mr. Trump's fraud remarks
have fallen silent. In October, the House speaker, Paul D. Ryan,
countered Mr. Trump's rigged-election claims by noting
\href{http://thehill.com/blogs/ballot-box/presidential-races/301200-paul-ryan-confident-the-election-will-not-be-rigged}{through
a spokeswoman} that he was ``fully confident'' of an honest vote.

Asked this month about Mr. Trump's claim that Mrs. Clinton won the
popular vote with illegal ballots, Mr. Ryan demurred. ``I don't know.
I'm not really focused on these things,'' he told CBS News's ``60
Minutes.''

Reince Priebus, the Republican National Committee chairman and Mr.
Trump's chief of staff, went further. ``I don't know that it's not
true,'' he said on CBS's ``Face the Nation.'' ``It's possible.''

Bogus fraud claims are not new. ``You call up and say there are busloads
of people being dropped off in multiple parishes, and we have to check
it out, even though we hear it every election,'' said Meg Casper, the
spokeswoman for Louisiana's secretary of state. ``We call up the
precinct office and they say, `No, we haven't seen anything like
that.'''

Still, almost every election has some irregularities, including last
month's.

In Walterboro, S.C., records showed that a woman cast an absentee
ballot, but voted again on Election Day. ``She indicated that she was
concerned her absentee ballot wouldn't count,'' said Chris Whitmire, a
spokesman for the State Election Commission.

In North Dakota, said the deputy secretary of state, Jim Silrum, ``one
of our county auditors was called the day after the election by a voter
who said: `Hey, my name is so-and-so. I'm from Minnesota but I voted in
the election and to do that I filled out an affidavit. Can you make me a
Minnesotan again?'''

In New Hampshire, said Brian W. Buonamano, an assistant attorney
general, officials are examining four to six unconfirmed complaints ---
``voting more than once, lying on your affidavit, things like that.'' In
Kentucky, a voter hotline recorded 18 complaints of ``general election
fraud;'' upon investigation, none were deemed credible.

In Idaho, a single voter was found to have cast another ballot in
Oregon. In Delaware, ``one voter voted absentee ballot in Suffolk County
and then went to Kent County, changed his address, and voted again,''
said Secretary of State Elaine Manlove. ``First time that's ever been
reported to me.''

Colorado, where most residents vote by mail, has sequestered 20,000
ballots for review --- some lacking signatures or the ID that first-time
voters are required to mail in, most with signatures that don't match
those on file.

But ``we're not saying there are 20,000 cases of fraud,'' said Lynn
Bartels, a spokeswoman for Secretary of State Wayne W. Williams. Indeed,
she said, a ballot cast by Mr. Williams's daughter was once rejected
because of a mismatched signature. Voting experts say the vast bulk of
mismatches arise from handwriting changes or signatures poorly entered
on touchpads.

As for noncitizens casting invalid ballots, Mr. Trump was right: It did
happen. Not millions of times, but at least once. Tennessee is still
investigating one allegation of noncitizen voting. And in Oregon, an
American citizen registered her noncitizen husband to vote, which he did
--- until he discovered it was illegal. The man reported his mistake to
county election officials, the secretary of state's office said.

He asked that his ballot not be counted.

Advertisement

\protect\hyperlink{after-bottom}{Continue reading the main story}

\hypertarget{site-index}{%
\subsection{Site Index}\label{site-index}}

\hypertarget{site-information-navigation}{%
\subsection{Site Information
Navigation}\label{site-information-navigation}}

\begin{itemize}
\tightlist
\item
  \href{https://help.nytimes3xbfgragh.onion/hc/en-us/articles/115014792127-Copyright-notice}{©~2020~The
  New York Times Company}
\end{itemize}

\begin{itemize}
\tightlist
\item
  \href{https://www.nytco.com/}{NYTCo}
\item
  \href{https://help.nytimes3xbfgragh.onion/hc/en-us/articles/115015385887-Contact-Us}{Contact
  Us}
\item
  \href{https://www.nytco.com/careers/}{Work with us}
\item
  \href{https://nytmediakit.com/}{Advertise}
\item
  \href{http://www.tbrandstudio.com/}{T Brand Studio}
\item
  \href{https://www.nytimes3xbfgragh.onion/privacy/cookie-policy\#how-do-i-manage-trackers}{Your
  Ad Choices}
\item
  \href{https://www.nytimes3xbfgragh.onion/privacy}{Privacy}
\item
  \href{https://help.nytimes3xbfgragh.onion/hc/en-us/articles/115014893428-Terms-of-service}{Terms
  of Service}
\item
  \href{https://help.nytimes3xbfgragh.onion/hc/en-us/articles/115014893968-Terms-of-sale}{Terms
  of Sale}
\item
  \href{https://spiderbites.nytimes3xbfgragh.onion}{Site Map}
\item
  \href{https://help.nytimes3xbfgragh.onion/hc/en-us}{Help}
\item
  \href{https://www.nytimes3xbfgragh.onion/subscription?campaignId=37WXW}{Subscriptions}
\end{itemize}
