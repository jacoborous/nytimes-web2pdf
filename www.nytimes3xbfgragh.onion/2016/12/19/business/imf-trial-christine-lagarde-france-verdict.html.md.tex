Sections

SEARCH

\protect\hyperlink{site-content}{Skip to
content}\protect\hyperlink{site-index}{Skip to site index}

\href{https://www.nytimes3xbfgragh.onion/section/business}{Business}

\href{https://myaccount.nytimes3xbfgragh.onion/auth/login?response_type=cookie\&client_id=vi}{}

\href{https://www.nytimes3xbfgragh.onion/section/todayspaper}{Today's
Paper}

\href{/section/business}{Business}\textbar{}I.M.F. Stands by Christine
Lagarde, Convicted of Negligence

\url{https://nyti.ms/2i6JkV5}

\begin{itemize}
\item
\item
\item
\item
\item
\item
\end{itemize}

Advertisement

\protect\hyperlink{after-top}{Continue reading the main story}

Supported by

\protect\hyperlink{after-sponsor}{Continue reading the main story}

\hypertarget{imf-stands-by-christine-lagarde-convicted-of-negligence}{%
\section{I.M.F. Stands by Christine Lagarde, Convicted of
Negligence}\label{imf-stands-by-christine-lagarde-convicted-of-negligence}}

\includegraphics{https://static01.graylady3jvrrxbe.onion/images/2016/12/20/world/20Lagarde2/20Lagarde2-videoSixteenByNineJumbo1600-v2.jpg}

By \href{http://www.nytimes3xbfgragh.onion/by/landon-thomas-jr}{Landon
Thomas Jr.},
\href{https://www.nytimes3xbfgragh.onion/by/liz-alderman}{Liz Alderman}
and
\href{https://www.nytimes3xbfgragh.onion/by/aurelien-breeden}{Aurelien
Breeden}

\begin{itemize}
\item
  Dec. 19, 2016
\item
  \begin{itemize}
  \item
  \item
  \item
  \item
  \item
  \item
  \end{itemize}
\end{itemize}

The International Monetary Fund threw its support behind its leader,
Christine Lagarde, on Monday despite her conviction in a French court on
charges of misusing public funds.

With international elites and their institutions facing populist
criticism amid political and social change in the United States and
Europe, the
\href{https://www.imf.org/external/np/sec/memdir/eds.aspx}{24 directors}
of the fund decided that this was not the time to leave the I.M.F.
rudderless.

Earlier on Monday, the Cour de Justice de la République, a French court
that considers cases against current and former government ministers,
found Ms. Lagarde guilty of
\href{https://www.nytimes3xbfgragh.onion/2016/12/19/business/lagarde-imf-verdict-france-questions.html}{criminal
charges linked to the misuse of public funds} when she was France's
finance minister nearly a decade ago. But the court did not impose a
fine or a sentence.

In a statement, the directors of the I.M.F. said they had considered the
court's decisions and had ``full confidence in the managing director's
ability to continue to effectively carry out her duties.''

Yet the verdict --- with its potential to tarnish Ms. Lagarde as a
leader --- came at a critical juncture for the I.M.F.

Founded and largely managed by Europeans and Americans, the fund
oversees a global economy in which faster-growing countries like China
are seeking a greater role.

Ms. Lagarde is the fourth of the last six leaders to come from France,
and the
\href{https://www.nytimes3xbfgragh.onion/2016/10/21/business/dealbook/imf-assessment-hints-at-internal-struggles.html}{difficult
time}the I.M.F. has had in anticipating, as well as reacting to, the
European debt crisis has caused some to wonder whether the time has come
to appoint a non-European leader.

I.M.F. doctrine, which advocates free trade and austerity for countries
in financial difficulties, has been criticized as elitist and not
sufficiently in tune with the populist movements sweeping the globe.

For now, the fund must confront more immediate challenges. With Britain
leaving the European Union, Italy's future in the eurozone perhaps in
doubt and the possibility of global trade wars being set off by
President-elect Donald J. Trump, some have said the steady, experienced
hand of Ms. Lagarde was needed more than ever.

``She has been a very effective leader,'' said Edwin M. Truman, a
specialist in international finance formerly at the Federal Reserve and
the United States Treasury. ``Yes, there are big questions about the
fund's future. But for her to have to step down now --- well, that would
be complicated.''

Jacob J. Lew, the Treasury secretary, expressed the Obama
administration's support for Ms. Lagarde, saying that ``she is a strong
leader of the I.M.F., and we have every confidence in her ability to
guide the fund at a critical time for the global economy.''

For the Trump administration, ``I don't think this kind of ethical
question is likely to be the highest priority,'' Mr. Truman said. While
the I.M.F. and other global institutions did not figure in the
presidential debate, Mr. Trump repeatedly criticized a
\href{http://www.nytimes3xbfgragh.onion/2016/10/14/us/politics/trump-comments-linked-to-antisemitism.html?_r=0}{``global
power structure''} that fixed the economy against workers.

``At bottom, it's all about French politics,'' Mr. Truman said.

Members of the I.M.F. board were well aware that Ms. Lagarde was facing
trial in her native France over allegations that occurred when she was
the finance minister in the administration of Nicolas Sarkozy.

The consensus among the directors was that Ms. Lagarde's transgressions
occurred when she was not at the fund --- in contrast to those of her
predecessor, Dominique Strauss-Kahn --- and since taking charge in 2011,
she had proved to be a leader capable of presenting a softer side of the
fund while fighting hard to bolster its legitimacy in the aftermath of
the financial crisis.

More so than her predecessors, Ms. Lagarde has pushed the fund to be
more aggressive in taking up the cause of women and focusing attention
on growing issues of inequality around the world.

Over the last year and a half, she has also led a forceful public
critique of Europe's refusal to offer Greece debt relief in return for
the difficult economic changes the country has been making.

Nevertheless, while Ms. Lagarde may have retained the backing of her
board for the moment, over the longer term, her French legal problems
may have hurt her most valuable asset --- her carefully constructed
public persona.

``She was a breath of fresh air, someone representing true change from
the past,'' said Peter Doyle, a former economist at the fund and now an
outspoken critic. ``Now she is just another tainted European leader.''

Those are tough words. But economists note that the fund's core mission
of requiring financially ailing countries to reform their economies and
fight corruption demands credibility and reputation of the highest
order.

And that starts at the top, with its leader --- especially one who is as
widely known as Ms. Lagarde.

``It would be complacent if not delusional to say there will be no
impact on the institution,'' said Nicolas Véron, a specialist on
international economics at the Bruegel Institute in Brussels. ``The only
question is how big is the impact --- and how does it compare with the
need for stability.''

Ms. Lagarde's legal issues in France have dogged her work at the fund
since she was appointed in 2011. She took over as managing director
after Mr. Strauss-Kahn resigned following accusations that he sexually
assaulted a maid in a New York City hotel.

The case against Ms. Lagarde centered on Bernard Tapie, a former
entertainer and owner of Adidas who had previously been jailed on
corruption charges. Mr. Tapie accused the lender Crédit Lyonnais, in
which the French state had a stake at the time, of cheating him when it
oversaw the sale of his share in the sportswear empire in 1993. Years of
costly legal battles ensued.

In 2007, Ms. Lagarde sent the dispute to a three-person private
arbitration authority that awarded Mr. Tapie more than 400 million
euros, or \$420 million at current exchange rates, in damages and
interest, to be paid by the state.

The court did not fault Ms. Lagarde for approving the arbitration, but
it ruled that she had been negligent for not appealing the decision. The
court, noting that a judge had previously invalidated the payout in a
2015 ruling and that she had ``national and international'' stature,
decided not to punish Ms. Lagarde and spared her a criminal record.

Speaking to reporters after the hearing, Ms. Lagarde's lawyer, Patrick
Maisonneuve, said he had a ``mixed'' reaction to the verdict.

``On the one hand, she is found responsible, but given the
circumstances, given the responsibilities that Ms. Lagarde had at the
time --- in 2008, we were in a major economic crisis --- the court
decided that it would not sentence Ms. Lagarde to anything,'' he said.

Ms. Lagarde's lawyers can appeal the verdict before France's highest
criminal court, the Cour de Cassation, on procedural grounds. But Mr.
Maisonneuve suggested she might not, because no punishment was meted
out.

Ms. Lagarde did not attend the latest hearing on Monday, but was in
Paris last week for the case.

Advertisement

\protect\hyperlink{after-bottom}{Continue reading the main story}

\hypertarget{site-index}{%
\subsection{Site Index}\label{site-index}}

\hypertarget{site-information-navigation}{%
\subsection{Site Information
Navigation}\label{site-information-navigation}}

\begin{itemize}
\tightlist
\item
  \href{https://help.nytimes3xbfgragh.onion/hc/en-us/articles/115014792127-Copyright-notice}{©~2020~The
  New York Times Company}
\end{itemize}

\begin{itemize}
\tightlist
\item
  \href{https://www.nytco.com/}{NYTCo}
\item
  \href{https://help.nytimes3xbfgragh.onion/hc/en-us/articles/115015385887-Contact-Us}{Contact
  Us}
\item
  \href{https://www.nytco.com/careers/}{Work with us}
\item
  \href{https://nytmediakit.com/}{Advertise}
\item
  \href{http://www.tbrandstudio.com/}{T Brand Studio}
\item
  \href{https://www.nytimes3xbfgragh.onion/privacy/cookie-policy\#how-do-i-manage-trackers}{Your
  Ad Choices}
\item
  \href{https://www.nytimes3xbfgragh.onion/privacy}{Privacy}
\item
  \href{https://help.nytimes3xbfgragh.onion/hc/en-us/articles/115014893428-Terms-of-service}{Terms
  of Service}
\item
  \href{https://help.nytimes3xbfgragh.onion/hc/en-us/articles/115014893968-Terms-of-sale}{Terms
  of Sale}
\item
  \href{https://spiderbites.nytimes3xbfgragh.onion}{Site Map}
\item
  \href{https://help.nytimes3xbfgragh.onion/hc/en-us}{Help}
\item
  \href{https://www.nytimes3xbfgragh.onion/subscription?campaignId=37WXW}{Subscriptions}
\end{itemize}
