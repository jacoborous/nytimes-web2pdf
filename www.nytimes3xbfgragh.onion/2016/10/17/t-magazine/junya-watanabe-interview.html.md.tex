Junya Watanabe, One of Fashion's Foremost Thinkers

\url{https://nyti.ms/2eb0SB4}

\begin{itemize}
\item
\item
\item
\item
\item
\end{itemize}

\includegraphics{https://static01.graylady3jvrrxbe.onion/images/2016/10/17/t-magazine/watanabe-slide-H7LC/watanabe-slide-H7LC-articleLarge.jpg?quality=75\&auto=webp\&disable=upscale}

Sections

\protect\hyperlink{site-content}{Skip to
content}\protect\hyperlink{site-index}{Skip to site index}

\hypertarget{junya-watanabe-one-of-fashions-foremost-thinkers}{%
\section{Junya Watanabe, One of Fashion's Foremost
Thinkers}\label{junya-watanabe-one-of-fashions-foremost-thinkers}}

The designer rips apart and reconstructs fashion with obsessive
intellect. In a rare interview, he illuminates the ideas behind his
radical clothing.

Junya Watanabe in his Tokyo office.Credit...Jamie Hawkesworth

Supported by

\protect\hyperlink{after-sponsor}{Continue reading the main story}

By Alexander Fury

\begin{itemize}
\item
  Oct. 17, 2016
\item
  \begin{itemize}
  \item
  \item
  \item
  \item
  \item
  \end{itemize}
\end{itemize}

The fashion designer Junya Watanabe uses the word \emph{monozukuri} a
lot when discussing his work. ``I think it's very specific to Japanese
culture,'' says Watanabe's American-­born assistant and longtime
interpreter, Ikuko Ichihashi. ``You could translate it to craftsmanship,
but it's more than that. It has more depth. It's more about the design
aspect, the aesthetics. How do you create something?''

The word ``craftsmanship'' references the person behind the craft. By
contrast, ``monozukuri'' is formed from the words ``mono'' (thing) and
­``zukuri'' (to make, to manufacture, to grow). The individual behind
the craft is subjugated to the act of making. For Watanabe, who
de­-emphasizes himself often, the use of the word is pointed. He doesn't
appear for the customary bow at the end of his runway shows, presented
four times a year in Paris. He rarely grants interviews, refuses to
discuss his personal life and is reticent even to talk about his work.
Many of his own employees have never been to his studio. ``He doesn't
have a problem with talking about his clothes and creation,'' Ichihashi
tells me, before our interview begins. ``But he's a little hesitant
about talking about personal interests and just personal ...'' She
trails off a little. Personal anything, then.

\includegraphics{https://static01.graylady3jvrrxbe.onion/images/2016/10/17/t-magazine/watanabe-slide-PYUU/watanabe-slide-PYUU-articleInline.jpg?quality=75\&auto=webp\&disable=upscale}

I glean from outside sources: Watanabe was born in Fukushima in 1961, he
is divorced, he studied at Bunka Fashion College in Tokyo before joining
Rei Kawakubo's Comme des Garçons in 1984 as a pattern cutter. His own
line was formed in 1992 under the umbrella of Comme des Garçons. His
debut show was held at the concourse of Tokyo's Ryogoku Station the same
year, and in 1993 he presented his first women's wear show in Paris.
When asked about what influenced him to become a designer, Watanabe
says, ``There's nothing in particular that made me want to start fashion
and create clothes. But if I were to mention something, it would be the
fact that my mother used to have a little made-­to-­order shop. That may
have been an influence.'' A question about Watanabe's father is politely
rebuffed.

Despite his insistence on privacy, his name is on the label, and it's
his singular imagination that has made that label so remarkably
influential in global fashion. Watanabe has created garments that have
shifted the way people think about clothing, not just fashion. His work
is about experimentation, endlessly reworking garments into fresh
constructions. In an industry where referencing --- of other cultures,
of other historical styles --- runs rife, Watanabe's pieces have the
rare, almost unique attribute of seeming like stuff we've never seen
before. It's all the more striking because Watanabe works with what he
calls ``dumb'' clothes: trench coats, biker jackets, the white shirt.
The ordinary becomes extraordinary.

There was the 1999 Watanabe show where fabrics reversed to become
waterproof, as demonstrated by an isolated rain shower mid­runway; a
2001 show that elevated denim to couture level and prompted a barrage of
high­-fashion homages (read: copies); a 2006 collection whose endless
reiterations of the trench gave new dimensions to a garment considered
staid and classic. A memorable sequence of women's wear collections,
from a half-decade ago, explored elements of nearly any basic wardrobe:
army fatigues; puffer coats; sailor stripes. They were remarkable for
gleaning such richness and breadth from simple staples. Today, they have
become flash points for other designers. It's difficult to imagine a
designer sitting down to create one of those garments without looking at
what Watanabe did first.

Image

Referencing classical haute couture shapes and cutting-edge fabrication
technology, Watanabe pairs a nylon sweater with an origami pleated skirt
in the same fabric.Credit...Photograph by Jamie Hawkesworth. Styled by
Marie-Amélie Sauvé

Watanabe is 55; his company is 24. The day we met, he wore a black
Lacoste polo shirt, shorts and horn­-rimmed glasses that gave a decided
intellectual slant to his appearance. ``Intellectual'' is an adjective
often used to describe Watanabe's clothes, usually by journalists. What
they mean is that his clothes are complex, complicated to make,
sometimes complicated to wear, intriguing and experimental. He often
uses one fabric for a collection, his approach almost scientific in the
dissection and cataloging of the material's various forms. His fall
collection explored geometric structures rendered in polyurethane bonded
with nylon tricot, a material more commonly used for industrial
purposes, like car interiors. The folded, pinched and corrugated fabric
spiraled around the models' bodies, abstracting them, an exercise in
shape and construction that just happened to become clothing.

Some of it was mad, in the sense of the highly abstract: A dull red
geodesic cape peppered with holes like Gruyère cheese could only be
dubbed a ``garment'' because it was fabric that, in that moment, sat on
a body. It bore no fastenings, no extraneous details like sleeves or a
collar. The mathematical precision of its structure held the same
fascination as a complex equation chalked on a board: the observation of
another's processes, all that sculpted, folded fabric, that painstaking
technique. Oddly enough, Watanabe counts Pierre Cardin as an early
influence, but there's a similarity in their uncompromising shapes and
obsession with geometric forms. There's something of Issey Miyake, too:
Leafing through a magazine and coming across his work made Watanabe
follow fashion in the first place. ``I was drawn to the fact that
designers before Miyake, like Dior and big names, would create clothes
that were form­fitting,'' he says. ``Issey totally changed the idea,
completely different, and that impact was profound on me. Of making me
want to create something, the idea of clothing much different from
previous designers.''

Another epithet frequently used when referring to Watanabe's designs is
``Japanese.'' ``Oftentimes, interviewers or people in the fashion world
like to refer to garments I make as Japanese or having a Japanese style
or taste,'' Watanabe says. ``I want to ask you, why is that? To
categorize, or do they really, truly feel a connection?'' Watanabe likes
to ask questions more than answer them. Curiosity is part of what makes
him a great designer. The idea of ``Japanese'' alludes, perhaps, to
``otherness,'' to an enduring occidental fascination with the
obliqueness of the Far East, of words that look like pictures and
ancient ceremonies with complex rules. There is an otherness to
Watanabe's clothes, a removal from the norm.

The designer he's most frequently compared to is Rei Kawakubo. It's
understandable, if not necessarily correct. His company is owned by
Comme des Garçons Co. Ltd. He worked alongside her for eight years. But
while Comme des Garçons' runway collections frequently approach clothing
as a contextual conceit --- Kawakubo's latest pieces don't resemble
garments so much as site-specific soft sculpture --- Watanabe's
collections are more pragmatic. Kawakubo has focused on pulling fashion
apart, literally. Her first Paris collection in 1981 challenged and
inverted fashion's established norms, centering on holey sweaters and
knits scarred with enormous, random perforations that she anarchically
described as ``lace.'' Watanabe is also anarchic and challenging, but he
works within fashion's rule book. Kawakubo is the mistress of the
four-sleeved jacket. Watanabe's have two, and they both work.

\href{https://www.nytimes3xbfgragh.onion/slideshow/2016/10/17/t-magazine/fashion/junya-watanabes-spring-summer-2017-show.html}{}

\hypertarget{junya-watanabes-springsummer-2017-show}{%
\subsection{Junya Watanabe's Spring/Summer 2017
Show}\label{junya-watanabes-springsummer-2017-show}}

17 Photos

View Slide Show ›

Molly SJ Lowe

``My idea of something being beautiful or aesthetically pleasing is
completely different from what Rei Kawakubo's vision of beauty is,''
Watanabe allows. ``To this day, seeing Rei Kawakubo's work, I feel the
same. I understand certain points and I can relate to certain areas ...
That doesn't mean that I completely agree. As a person-to-person
relationship, I feel that I have a different idea, and I'll always have
a different vision of what is beautiful. Another reason, perhaps, I
didn't end up working right alongside Kawakubo is perhaps she felt that
I had a different vision of my own. Maybe that's why we parted, in terms
of creating something that was different.''

When we meet, it is in the Comme des Garçons building in Tokyo's Aoyama,
a district whose fashionable identity was essentially invented when
Comme des Garçons opened its first store here in 1975. Now, there's a
Miu Miu, a Moncler, a Herzog \& de Meuron Prada building with faceted
windows bulging like bug eyes. Comme des Garçons' own headquarters are
rather more sedate. Inside the nondescript red-brick office building,
black and white offices are stacked like humbugs. Black floor, white
walls, black window frames. Black chairs, white chairs and black tables.

On the second floor of their office building, Watanabe's team of 30
works to pull the upcoming collection together. The studio looks similar
to those of many fashion designers: Gargantuan tables that could seat 20
are layered with tissuey mille-feuilles of paper patterns, the building
blocks of collections, people hunched over and slicing out cloth pieces.
His design staff, generally dressed in Watanabe, Comme des Garçons or a
mixture of both, quietly buzz about the otherwise silent space. In
another room, the production staff are seated at ranks of computer
terminals. (The company deals with the manufacture and distribution of
Watanabe's clothes worldwide.) Watanabe himself has a small office in
back: a rare space held off-limits. I wasn't allowed to glimpse
Watanabe's upcoming spring collection, which the team was working on
when I visited. The dressmaking dummies were covered with black nylon
tarps, shrouding shapes and materials. The team evidently had been
warned of my arrival.

But when I ask about his process of creation, Watanabe leaves the room,
returning a few moments later with a paper prototype of folded,
ferocious spikes, and a series of photographs of similar spikes and
bumps, like stalactites or microscopic images of bacteria. ``It all
begins inside my head,'' he says. ``I start to look for strings of ideas
that interest me. From then, I put my ideas to words. I work alongside
my pattern makers, trying to put my words into creating and actually
seeing it come to life. Photographs, artist's work, anything that may
seem relatable to what I am speaking about,'' he says. ``After looking
at visual elements, they start to craft.''

He grabs the paper model, toying with it like a cootie catcher. ``This
is one example of paper craft. It doesn't always turn out to be like
this, but sometimes it's easier to take the form of paper craft. After
creating pieces and little constructed elements, then I start to think
about the relationship of these pieces with the body and how it could be
formed onto the body ... all these ideas, little by little, they form
garments, clothing. Then I create the collection.''

It's not the way designers conventionally work, typically pulling from
fashion's past, toying with these examples in cloth, finding ways to
make them work on the proportions of today. Watanabe's approach, like
his clothing, is more abstract. ``He doesn't consider himself to be a
major designer,'' Ichihashi tells me. Possibly that has less to do with
stature, and more with the fashion industry as a whole. Physically,
Watanabe is markedly disconnected. His base is in Tokyo, which, while
not exactly the boondocks, is removed from the four-city fashion circus
where most designers not only work, but also live. He says he no longer
looks at the work of his peers. ``I don't really know what they're
doing,'' he admits. Instead, he acts as his own frame of reference.

Image

The radical structures that characterized Watanabe's fall collection
were partly inspired by advanced mathematics.Credit...Photograph by
Jamie Hawkesworth. Styled by Marie-Amélie Sauvé.

Ultimately, Watanabe wants to create something different. He doesn't
reference ``fashion,'' or sometimes even clothing. Recent collections
have moved away from specific garments, into abstraction around the
body. ``I don't know how others see `fashion,' '' Watanabe says. (The
quotations are his.) ``But to me fashion is creating something, creating
something new through clothes. That's what really drew me, in the
beginning.'' I ask him if he's achieved his goal. ``To this day, I feel
that I haven't quite been able to portray the new. That's something
constant that I'm trying to work towards.''

And yet, oddly, his clothes often chime with the mood of the times, a
collective unconscious. Watanabe's '90s ``techno couture'' reflected a
general thrust toward space-age futurism incited by the millennium; his
fall 2015 collection, a symphony of accordion pleats, was the most
extreme and accomplished example of the technique in a season awash with
fabric folding. That Watanabe, an intentional outsider, can pinpoint
exactly the axis around which the rest of the insular fashion world is
turning gives his collections a prophetic quality.

The word monozukuri, incidentally, is a relatively new invention, barely
20 years old. Professor Takahiro Fujimoto, of the Manufacturing
Management Research Center at the University of Tokyo, categorizes
monozukuri as the ``art, science and craft of making things.'' Junya
Watanabe's craft is both a science and an art. It's what makes his
clothing great.

Advertisement

\protect\hyperlink{after-bottom}{Continue reading the main story}

\hypertarget{site-index}{%
\subsection{Site Index}\label{site-index}}

\hypertarget{site-information-navigation}{%
\subsection{Site Information
Navigation}\label{site-information-navigation}}

\begin{itemize}
\tightlist
\item
  \href{https://help.nytimes3xbfgragh.onion/hc/en-us/articles/115014792127-Copyright-notice}{©~2020~The
  New York Times Company}
\end{itemize}

\begin{itemize}
\tightlist
\item
  \href{https://www.nytco.com/}{NYTCo}
\item
  \href{https://help.nytimes3xbfgragh.onion/hc/en-us/articles/115015385887-Contact-Us}{Contact
  Us}
\item
  \href{https://www.nytco.com/careers/}{Work with us}
\item
  \href{https://nytmediakit.com/}{Advertise}
\item
  \href{http://www.tbrandstudio.com/}{T Brand Studio}
\item
  \href{https://www.nytimes3xbfgragh.onion/privacy/cookie-policy\#how-do-i-manage-trackers}{Your
  Ad Choices}
\item
  \href{https://www.nytimes3xbfgragh.onion/privacy}{Privacy}
\item
  \href{https://help.nytimes3xbfgragh.onion/hc/en-us/articles/115014893428-Terms-of-service}{Terms
  of Service}
\item
  \href{https://help.nytimes3xbfgragh.onion/hc/en-us/articles/115014893968-Terms-of-sale}{Terms
  of Sale}
\item
  \href{https://spiderbites.nytimes3xbfgragh.onion}{Site Map}
\item
  \href{https://help.nytimes3xbfgragh.onion/hc/en-us}{Help}
\item
  \href{https://www.nytimes3xbfgragh.onion/subscription?campaignId=37WXW}{Subscriptions}
\end{itemize}
