Sections

SEARCH

\protect\hyperlink{site-content}{Skip to
content}\protect\hyperlink{site-index}{Skip to site index}

\href{https://www.nytimes3xbfgragh.onion/section/world/africa}{Africa}

\href{https://myaccount.nytimes3xbfgragh.onion/auth/login?response_type=cookie\&client_id=vi}{}

\href{https://www.nytimes3xbfgragh.onion/section/todayspaper}{Today's
Paper}

\href{/section/world/africa}{Africa}\textbar{}In South Sudan, City of
Hope Is Now City of Fear

\url{https://nyti.ms/1UYGZuW}

\begin{itemize}
\item
\item
\item
\item
\item
\end{itemize}

Advertisement

\protect\hyperlink{after-top}{Continue reading the main story}

Supported by

\protect\hyperlink{after-sponsor}{Continue reading the main story}

Juba Journal

\hypertarget{in-south-sudan-city-of-hope-is-now-city-of-fear}{%
\section{In South Sudan, City of Hope Is Now City of
Fear}\label{in-south-sudan-city-of-hope-is-now-city-of-fear}}

\includegraphics{https://static01.graylady3jvrrxbe.onion/images/2016/03/11/world/southsudan-web/southsudan-web-articleLarge.jpg?quality=75\&auto=webp\&disable=upscale}

By \href{http://www.nytimes3xbfgragh.onion/by/jeffrey-gettleman}{Jeffrey
Gettleman}

\begin{itemize}
\item
  March 11, 2016
\item
  \begin{itemize}
  \item
  \item
  \item
  \item
  \item
  \end{itemize}
\end{itemize}

JUBA, South Sudan --- Diu Tut glanced up at the gates of the displaced
persons camp where he lives and shook his head.

``I can't go out there,'' he said.

``Why not?'' he was asked.

``Because of this,'' he said, rubbing the soft tribal scars on his
forehead that mark him as a member of the Nuer ethnic group. ``It's my
death certificate.''

A lot of people in this town feel the same way. Juba, South Sudan's
capital --- and this whole country, for that matter --- has slid so far
from where its
\href{http://www.nytimes3xbfgragh.onion/2011/07/10/world/africa/10sudan.html}{people
dreamed it would go}.

The Republic of South Sudan is not even five years old, but already
50,000 people have been killed in an ethnically driven
\href{http://www.nytimes3xbfgragh.onion/2013/12/25/world/africa/south-sudan-crisis.html}{civil
war} replete with mass rape,
\href{http://www.nytimes3xbfgragh.onion/2015/06/23/world/africa/as-south-sudan-crisis-worsens-there-is-no-more-country.html}{civilian
massacres}, countless people displaced, killings at hospitals and now
children starving to death in sunshine-flooded pediatric wards, skin
peeling off their little backs like paint chips flaking off old wood.

On paper, the war is supposed to be over. An agreement
\href{http://www.nytimes3xbfgragh.onion/2016/02/12/world/africa/south-sudan-leader-takes-major-step-to-ending-conflict.html}{solidified
last month} calls for a cessation of hostilities, the formation of a
joint security force and the end of
\href{http://www.nytimes3xbfgragh.onion/2014/01/01/world/africa/old-rivalries-reignited-a-fuse-in-south-sudan.html}{the
political feud that started the war}. Riek Machar, the rebel leader
accused of staging a coup attempt more than two years ago, is supposed
to return to his position as vice president.

But most people here do not believe what's on paper.

In the camp where Mr. Tut and tens of thousands of other displaced Nuer
live, few feel safe enough to venture out, peace deal or not. The camp
is on Juba's outskirts and many camp dwellers have homes less than two
miles away. They fled at the outbreak of the civil war, and even now
they say they will be killed if they try to return.

\includegraphics{https://static01.graylady3jvrrxbe.onion/images/2016/03/11/world/southsudan-web2/southsudan-web2-articleLarge.jpg?quality=75\&auto=webp\&disable=upscale}

Juba has turned into a messy city of great fear. Just five years ago, on
the eve of independence, it was ebullient. After
\href{http://www.nytimes3xbfgragh.onion/2011/01/09/world/africa/09sudan.html?pagewanted=all}{decades
of war} and millions of lives lost to free itself from Sudan, the nation
was burdened with daunting problems, of course, but it was also
\href{http://www.nytimes3xbfgragh.onion/2011/01/10/world/africa/10sudan.html}{brimming
with promise}. Flags flapped proudly in the air and a clock
\href{http://www.nytimes3xbfgragh.onion/2011/07/08/world/africa/08sudan.html}{counted
down the days to independence}, decorated with the words ``Free at
Last.''

But even after billions of aid dollars, many roads remain dirt, covered
by a thin, crunchy layer of crushed plastic water bottles. Boys on
motorbikes putter past smoking piles of garbage. Pickup trucks packed
with soldiers careen recklessly into traffic, the young men in back
angrily shaking their machine guns at anyone unfortunate enough to cross
their paths.

At night, shots ring out and people are killed.

The government chalks up Juba's continuing violence to what it calls
``unknown gunmen,'' but there is little investigation. Most people
believe the culprits are government soldiers, anyway. They just can't
say that too loudly for fear that they will be next.

On Friday, the
\href{http://www.nytimes3xbfgragh.onion/2016/03/12/world/africa/un-reports-systematic-rape-in-south-sudan-conflict.html?hp\&action=click\&pgtype=Homepage\&clickSource=story-heading\&module=first-column-region\&region=top-news\&WT.nav=top-news}{United
Nations} said that all parties to the conflict had committed serious and
systematic violence against civilians, but it singled out forces loyal
to the government as the worst offenders.

The United Nations recorded gruesome accounts from civilians, including
of women and children being hanged from trees, burned alive or shot and
hacked to pieces with machetes. Churches, mosques and hospitals have
come under attack, the United Nations said.

Much of the bloodshed is caused by the bitter, ethnically tinged rivalry
between the government, led by members of the Dinka ethnic group, and
Mr. Machar's rebel alliance, made up of mostly Nuer. But some of the
worst violence, such as the fighting last year that left chopped up body
parts scattered across the swamps, was Nuer versus Nuer, grisly evidence
of a society splintering deeper and deeper.

\includegraphics{https://static01.graylady3jvrrxbe.onion/images/2016/03/12/world/12Sudan-video/12Sudan-video-videoSixteenByNine1050.jpg}

Before the civil war broke out in December 2013, South Sudan's economy
had one thing going for it: oil. But the oil fields soon became battle
zones, and then the world oil price plummeted. On top of that, the oil
deal South Sudan triumphantly signed with Sudan after splitting off from
it in 2011 is not looking so great anymore.

When it comes to oil production, the two Sudans are hopelessly linked.
The south is home to the biggest reserves; the north has the pipelines
to export it.

Instead of splitting profits, South Sudan agreed to pay Sudan a flat fee
for every barrel pumped through the northern pipeline. That transit fee,
along with some other debt South Sudan has to repay to Sudan, amounts to
around \$25 a barrel. But with the global oil price hovering around \$40
a barrel, South Sudan is now steadily draining its most precious
resource for very little in return.

But it has few options. Nearly all of the government's revenue is
generated by oil, and at a time like this, the government can't afford
not to pay its soldiers.

Before South Sudan became independent, Sudan's leaders tried everything
in their power to prevent the south from breaking off. Oppression.
Neglect. A vicious civil war, characterized by scorched villages and
slave raids. When it became clear by the early 2000s that it could not
win, Khartoum, Sudan's capital, appealed to the court of world opinion,
trying to persuade Western powers that the south was too poor, too
uneducated and too divided to govern itself.

Sadly, many observers say, the chaos that has broken out in South Sudan
since independence is very similar to what the northerners had been
warning about all along. More than half a dozen cease-fires have been
shattered. Though Mr. Machar has agreed to return as vice president, he
has yet to do so, citing security concerns. Many people believe he is
just stalling.

Image

Marco Freddi, a Franciscan friar from a parish in Juba, South Sudan,
visited a patient at hospital in a camp for internally displaced
people.Credit...Tyler Hicks/The New York Times

At the same time, South Sudan's president, Salva Kiir, seems to be
losing control.

Dressed in dark suits with his signature black cowboy hat (a gift from
President George W. Bush), he is known as a decent and religious man but
also as a bit of an enigma.

``It's very difficult to pin down the personality of Kiir,'' said James
Solomon Padiet, a professor at Juba University. ``Most of his decisions
depend on who influences him at the moment he makes it.''

He said that Mr. Kiir had urged his generals to respect human rights but
that when government troops committed atrocities, as they have over and
over again, Mr. Kiir did not punish anyone.

``He says: `It's on you. God will punish you,'~'' Mr. Padiet said.

Mr. Kiir also needs each and every one of those generals to stay in
power, the professor added.

On a recent day, the waiting rooms outside Mr. Kiir's office were
crammed with high-ranking officials and other people hoping to meet with
him, but he did not show up to work.

Juba is incredibly uncomfortable these days, with temperatures soaring
into the triple digits and the air thick and sticky. People along the
road hide under mango trees or in the shallow shadows cast by parked
trucks, desperate for shade.

Image

Children crowded into a school at a displaced persons camp near Juba,
South Sudan, established during the country's civil war.Credit...Tyler
Hicks/The New York Times

These are the last days of dizzying heat before the spring rains hit and
shut down many roads. Aid organizations are racing to position emergency
food for the nearly three million South Sudanese edging toward
starvation.

``There's massive access constraints, roadblocks everywhere, widespread
extortion,'' said Jonathan Veitch, head of Unicef's South Sudan office.
He said aid trucks had to run a gantlet of 56 armed checkpoints to
deliver food to Bentiu, a conflict-hit area.

``It costs \$2,000 in bribes,'' he said. ``Per truck!''

Aid organizations are begging for more money, but donors seem to be
getting tired.

``Most people think they have seen this movie before,'' said John
Prendergast, co-founder of the
\href{http://www.enoughproject.org/}{Enough Project}, an anti-genocide
group.

The well-worn narrative of helpless, starving South Sudanese ``wildly
oversimplifies'' the reality, he said.

``The real story,'' he said, ``is one of a falling out among
kleptocratic thieves, whose self-enrichment free-for-all before and
after independence led competing factions to use ethnicity as a
mobilizer, which is the equivalent of aiming a flamethrower at an oil
rig.''

In Juba's displaced persons camp, people have little to do except mourn
their losses --- the loss of family, the loss of homes, the loss of
freedom, the loss of hope.

Mr. Tut recently graduated from a business program. His books ---
``Public Administration,'' an English grammar book and the Chinua Achebe
novel ``No Longer at Ease'' --- lie in a dusty stack on the dirt floor.

He said his best friend at university was a Dinka.

``We still talk on the phone sometimes,'' he said, his voice trailing
off. ``But I haven't seen him for months.''

Advertisement

\protect\hyperlink{after-bottom}{Continue reading the main story}

\hypertarget{site-index}{%
\subsection{Site Index}\label{site-index}}

\hypertarget{site-information-navigation}{%
\subsection{Site Information
Navigation}\label{site-information-navigation}}

\begin{itemize}
\tightlist
\item
  \href{https://help.nytimes3xbfgragh.onion/hc/en-us/articles/115014792127-Copyright-notice}{©~2020~The
  New York Times Company}
\end{itemize}

\begin{itemize}
\tightlist
\item
  \href{https://www.nytco.com/}{NYTCo}
\item
  \href{https://help.nytimes3xbfgragh.onion/hc/en-us/articles/115015385887-Contact-Us}{Contact
  Us}
\item
  \href{https://www.nytco.com/careers/}{Work with us}
\item
  \href{https://nytmediakit.com/}{Advertise}
\item
  \href{http://www.tbrandstudio.com/}{T Brand Studio}
\item
  \href{https://www.nytimes3xbfgragh.onion/privacy/cookie-policy\#how-do-i-manage-trackers}{Your
  Ad Choices}
\item
  \href{https://www.nytimes3xbfgragh.onion/privacy}{Privacy}
\item
  \href{https://help.nytimes3xbfgragh.onion/hc/en-us/articles/115014893428-Terms-of-service}{Terms
  of Service}
\item
  \href{https://help.nytimes3xbfgragh.onion/hc/en-us/articles/115014893968-Terms-of-sale}{Terms
  of Sale}
\item
  \href{https://spiderbites.nytimes3xbfgragh.onion}{Site Map}
\item
  \href{https://help.nytimes3xbfgragh.onion/hc/en-us}{Help}
\item
  \href{https://www.nytimes3xbfgragh.onion/subscription?campaignId=37WXW}{Subscriptions}
\end{itemize}
