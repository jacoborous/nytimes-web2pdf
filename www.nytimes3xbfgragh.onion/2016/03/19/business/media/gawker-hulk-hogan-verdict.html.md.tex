Sections

SEARCH

\protect\hyperlink{site-content}{Skip to
content}\protect\hyperlink{site-index}{Skip to site index}

\href{https://www.nytimes3xbfgragh.onion/pages/business/media/index.html}{Media}

\href{https://myaccount.nytimes3xbfgragh.onion/auth/login?response_type=cookie\&client_id=vi}{}

\href{https://www.nytimes3xbfgragh.onion/section/todayspaper}{Today's
Paper}

\href{/pages/business/media/index.html}{Media}\textbar{}Hulk Hogan
Awarded \$115 Million in Privacy Suit Against Gawker

\url{https://nyti.ms/22sRLMj}

\begin{itemize}
\item
\item
\item
\item
\item
\item
\end{itemize}

Advertisement

\protect\hyperlink{after-top}{Continue reading the main story}

Supported by

\protect\hyperlink{after-sponsor}{Continue reading the main story}

\hypertarget{hulk-hogan-awarded-115-million-in-privacy-suit-against-gawker}{%
\section{Hulk Hogan Awarded \$115 Million in Privacy Suit Against
Gawker}\label{hulk-hogan-awarded-115-million-in-privacy-suit-against-gawker}}

\includegraphics{https://static01.graylady3jvrrxbe.onion/images/2016/03/19/business/19hogan/19hogan-articleLarge.jpg?quality=75\&auto=webp\&disable=upscale}

By Nick Madigan and
\href{http://www.nytimes3xbfgragh.onion/by/ravi-somaiya}{Ravi Somaiya}

\begin{itemize}
\item
  March 18, 2016
\item
  \begin{itemize}
  \item
  \item
  \item
  \item
  \item
  \item
  \end{itemize}
\end{itemize}

The retired wrestler Hulk Hogan was awarded \$115 million in damages on
Friday by a Florida jury in an invasion of privacy case against
Gawker.com over its publication of a sex tape --- an astounding figure
that tops the \$100 million he had asked for, that will probably grow
before the trial concludes, and that could send a cautionary signal to
online publishers despite the likelihood of an appeal by Gawker.

The wrestler, known in court by his legal name, Terry G. Bollea, sobbed
as the verdict was announced in late afternoon, according to people in
the courtroom. The jury had considered the case for about six hours.

Mr. Bollea's team said the verdict represented ``a statement as to the
public's disgust with the invasion of privacy disguised as journalism,''
adding: ``The verdict says, `No more.' ''

The damages awarded to Mr. Bollea on Friday were compensatory: \$55
million for economic harm and \$60 million for emotional distress.
Punitive damages will be established separately, which raises the
prospect that Gawker will have to submit to a detailed examination of
its finances in court so the jury can assess the scale of the damages.

Gawker's founder, Nick Denton, said in his own statement that the jury
did not hear all the facts. ``We feel very positive about the appeal
that we have already begun preparing, as we expect to win this case
ultimately,'' he said.

The meaning of the verdict will not be clear for some time. But the
perception that a Manhattan media company,
\href{http://www.nytimes3xbfgragh.onion/2015/06/14/business/media/gawker-nick-denton-moment-of-truth.html}{noted
for its wry tone} and its insistence that nearly any topic is fair game,
was brought low by a celebrity fighting for privacy is most likely to
resonate widely across the industry.

At issue in the case, in Pinellas County Circuit Court, was a grainy
black-and-white tape made in the mid-2000s, which showed Mr. Bollea
having sex with the wife of a friend of his at the time, Todd Clem, a
radio shock jock who had legally changed his name to Bubba the Love
Sponge Clem.

Gawker posted a brief excerpt in a 2012 post by Albert J. Daulerio, the
site's former editor in chief, that mused on the appeal of celebrity sex
tapes.

The case represented a peculiar clash of worlds, and it was a surreal
spectacle. Mr. Bollea explained his relationship with Mr. Clem, and the
ways in which Mr. Clem had encouraged him to sleep with his wife. He
also drew a distinction between himself and Hulk Hogan, who he suggested
were
\href{http://www.nytimes3xbfgragh.onion/2016/03/09/business/media/when-is-hulk-hogan-not-hulk-hogan.html}{separate
personas}.

Mr. Daulerio, who was named in the suit along with Mr. Denton, decided
to joke about child pornography in his deposition, which shocked the
court. And the jurors had to try and make sense of it all.

Mr. Bollea's lawyers said that the publication of the video was a
gratuitous invasion of privacy, and had no news value. One of them,
Kenneth G. Turkel, took particular aim at the contention that Gawker's
posting of the video was an act of journalism and was therefore
protected under the First Amendment. He described the publication as
``morbid and sensational prying.''

He maintained that had the site's editors been operating under the rules
of professional journalism, they would have contacted Mr. Bollea to ask
his permission to publish the video, or at least to warn him that they
were going to do so. In any case, Mr. Turkel said, it served only as
fodder for readers' clicks and a source for advertising revenue, Mr.
Turkel said.

Gawker had argued that its posting of a brief excerpt of the tape was
protected by the Constitution, and that Mr. Bollea had given up his
right to privacy by talking often in public about his sex life. ``He has
chosen to seek the spotlight,'' a lawyer for Gawker, Michael Sullivan,
said. ``He has consistently chosen to put his private life out there.''

In his closing statement for the defense, Mr. Sullivan insisted that
uncovering the sometimes less-than-laudatory activities of public
figures ``is what journalists do, and at the end of the day it's what we
want journalists to do.''

After accounts of the video and images from it surfaced online --- but
several months before Mr. Daulerio's 2012 post --- Mr. Bollea addressed
it in an appearance on a television show run by the website TMZ and in
other interviews. ``The public discussion was already going on,'' Mr.
Sullivan said.

\includegraphics{https://static01.graylady3jvrrxbe.onion/images/2016/03/19/business/media/19hogan-alt/19hogan-alt-articleLarge.jpg?quality=75\&auto=webp\&disable=upscale}

The verdict is a blow for Gawker, which has rebranded itself as a
politics site since it published the tape and has sought to clean up its
image in the wake of a series of scandals.

The company is technically required to post a bond, which is capped at
\$50 million, before appealing. But it will almost certainly appeal that
bond.

The potential effect on the business, which employs about 250 people
across seven websites, was not immediately clear. The company recently
took significant outside investment for the first time, specifically to
prepare for the case, and has been confident it can meet any financial
demands the case places on it.

In the statement on Friday, Gawker said its team was ``disappointed the
jury was unable to see key evidence and hear testimony from the most
important witness.''

That was an apparent reference to Mr. Clem. According to documents
unsealed on Friday, the radio host initially told federal investigators
that Mr. Bollea was aware that his tryst with Mrs. Clem was being
recorded. But he later changed that account after Mr. Bollea sued him,
saying the former wrestler did not know a camera was present.

Apparently fearing that if he testified in the trial he could be subject
to prosecution for giving differing accounts of the events, Mr. Clem
invoked his right to not incriminate himself and was not called as a
witness.

The plaintiff's legal team issued its own statement, saying that during
the three and a half years since the lawsuit was filed, Mr. Clem had
testified only once under oath and had ``confirmed that Terry Bollea had
no knowledge of being filmed or anything to do with it.''

Samantha Barbas, a law professor at the University at Buffalo whose
research focuses on the intersection of the First Amendment, media and
privacy, said the verdict ``could have a profound impact on privacy
rights and also free press rights'' in the United States.

``For a jury to say that a celebrity has a right to privacy that
outweighs the public's `right to know,' and that a celebrity sex tape is
not newsworthy, represents a real shift in American free press law,''
Professor Barbas said.

Advertisement

\protect\hyperlink{after-bottom}{Continue reading the main story}

\hypertarget{site-index}{%
\subsection{Site Index}\label{site-index}}

\hypertarget{site-information-navigation}{%
\subsection{Site Information
Navigation}\label{site-information-navigation}}

\begin{itemize}
\tightlist
\item
  \href{https://help.nytimes3xbfgragh.onion/hc/en-us/articles/115014792127-Copyright-notice}{©~2020~The
  New York Times Company}
\end{itemize}

\begin{itemize}
\tightlist
\item
  \href{https://www.nytco.com/}{NYTCo}
\item
  \href{https://help.nytimes3xbfgragh.onion/hc/en-us/articles/115015385887-Contact-Us}{Contact
  Us}
\item
  \href{https://www.nytco.com/careers/}{Work with us}
\item
  \href{https://nytmediakit.com/}{Advertise}
\item
  \href{http://www.tbrandstudio.com/}{T Brand Studio}
\item
  \href{https://www.nytimes3xbfgragh.onion/privacy/cookie-policy\#how-do-i-manage-trackers}{Your
  Ad Choices}
\item
  \href{https://www.nytimes3xbfgragh.onion/privacy}{Privacy}
\item
  \href{https://help.nytimes3xbfgragh.onion/hc/en-us/articles/115014893428-Terms-of-service}{Terms
  of Service}
\item
  \href{https://help.nytimes3xbfgragh.onion/hc/en-us/articles/115014893968-Terms-of-sale}{Terms
  of Sale}
\item
  \href{https://spiderbites.nytimes3xbfgragh.onion}{Site Map}
\item
  \href{https://help.nytimes3xbfgragh.onion/hc/en-us}{Help}
\item
  \href{https://www.nytimes3xbfgragh.onion/subscription?campaignId=37WXW}{Subscriptions}
\end{itemize}
