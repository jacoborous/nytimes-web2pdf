Sections

SEARCH

\protect\hyperlink{site-content}{Skip to
content}\protect\hyperlink{site-index}{Skip to site index}

\href{https://myaccount.nytimes3xbfgragh.onion/auth/login?response_type=cookie\&client_id=vi}{}

\href{https://www.nytimes3xbfgragh.onion/section/todayspaper}{Today's
Paper}

A New Pastry Layers Tastes of France and Syria

\url{https://nyti.ms/2uMF20q}

\begin{itemize}
\item
\item
\item
\item
\item
\end{itemize}

Advertisement

\protect\hyperlink{after-top}{Continue reading the main story}

Supported by

\protect\hyperlink{after-sponsor}{Continue reading the main story}

\href{/column/on-dessert}{On Dessert}

\hypertarget{a-new-pastry-layers-tastes-of-france-and-syria}{%
\section{A New Pastry Layers Tastes of France and
Syria}\label{a-new-pastry-layers-tastes-of-france-and-syria}}

\includegraphics{https://static01.graylady3jvrrxbe.onion/images/2018/04/08/magazine/08mag-ondessert1/08mag-08ondessert-t_CA0-articleLarge.jpg?quality=75\&auto=webp\&disable=upscale}

By Dorie Greenspan

\begin{itemize}
\item
  April 3, 2018
\item
  \begin{itemize}
  \item
  \item
  \item
  \item
  \item
  \end{itemize}
\end{itemize}

When my chin is down, my eyes are closed and my hands are cupped over my
ears, it's a sure sign that I'm tasting. I know how odd I look, but the
momentary calm helps me concentrate on flavors when there's a new recipe
at hand or a complex ingredient. I always taste chocolate this way, and
every once in a while I lock myself in my sensory cocoon because I've
come upon something unusual that I want to get to know better. A couple
of months ago in Paris, it was a pastry from a new, strikingly designed
patisserie in the Marais, Maison Aleph.

The morsel was called 1001 Feuilles (\emph{feuilles} is French for
leaves or layers), one layer more than the classic mille-feuille; its
name carries a hint of the tales of the Arabian Nights. It was square,
about the size of a marker for a board game, and it was, all at once,
chewy, buttery, sweet, creamy, a tad chunky, nuttish and precisely
balanced. It straddled the territory between baklava and mille-feuille,
and it spoke of place, although I couldn't be certain where. From the
patisserie's name, I expected something Middle Eastern --- aleph is the
first letter of both the Hebrew and Arabic alphabets --- and knowing
that the pastry was built on phyllo dough should have confirmed its
origins. But the filling seemed French, as did the deep flavor of
butter. The sweet was neither fully French nor Middle Eastern, neither
familiar nor completely new, and it captured my imagination.

When I met Myriam Sabet, Maison Aleph's founder and owner, I learned how
closely the traditions of East and West that shape her desserts mirror
her own history. Now 41, she was born in Aleppo, Syria, went to college
and graduate school in Canada and moved to France almost two decades
ago, when, she says, she decided, ``I was mature enough for Paris.'' In
2014, Sabet, who worked in finance, turned her life upside down to learn
about pastry. She studied with an elderly Syrian baker in Montreal, was
certified in French pastry at a Paris school and, last July, opened
Maison Aleph, a small shop that attracts dessert lovers from around the
world. The day I was there with her, I heard French and English and
Arabic, Hebrew, Japanese, Spanish and snippets of tongues I couldn't
place at all. Once having hoped for a career in foreign affairs, Sabet
looked around the room, laughed and exclaimed, ``Maybe that's exactly
what I'm doing now.''

Sabet makes about a half-dozen variations of the 1001 Feuilles, each
with layers of phyllo sandwiching a nut mixture in the style of baklava.
One has hazelnuts from the Piedmont, one almonds from Valencia, another
pistachios from Iran. The feuilles that called to me was made with
sesame and halvah. As it was for Sabet, halvah was a sweet from my
childhood. In Brooklyn, where I grew up, we would get it sliced to order
from rounds about the size of snare drums. I knew the rich, pressed
sesame-seed confection solely as a candy, but the Maison Aleph recipe
for the squares' filling used it as an ingredient to be smoothed, baked
and transformed by heat.

It was only when I returned home and made the dessert myself that I
grasped how complex it truly was, if not in preparation, then in
conception: It seamlessly fused two disparate heritages. The filling
blends the halvah with tahini, toasted sesame seeds, butter, eggs, sugar
and cornstarch, and traces the contours of a classic crème d'amande, the
mixture a French pâtissier would use to fill a traditional fruit tart or
a galette des rois.

Each of the pastry layers --- of which there are 28, not the poetic 1001
--- is a paper-thin sheet of phyllo brushed with butter then dusted with
powdered sugar. Sabet thinks that the work of buttering layer after
layer of the ancient Arab dough may have inspired the French
mille-feuille. In the moment, it seems right. For me, stacking the
layers, spreading the filling over the plump cushion of dough and then
covering it with more brushed and dusted layers brings the satisfaction
of having, quite literally, constructed something.

When the pastry comes from the oven and I compress the layers, helping
the butter and sugar, the halvah filling and the dough meld with one
another, the dessert leans less toward mille-feuille, more to baklava,
its well-known form a blind for the unexpected filling. This is a pastry
that's best hours after it's made, when the top and bottom layers are at
their crispiest. But as soon as it's cool, I tug a square loose. Now,
back in the United States and an ocean away from Paris, I taste it: Its
flavor and texture are as good as I remember. What's missing is the
element of surprise --- it has been replaced by the sweetness of knowing
every nuance of the dessert, the reward for having made it myself.

\textbf{Recipe:}
\href{https://cooking.nytimes3xbfgragh.onion/recipes/1019257-maison-alephs-sesame-halvah-1001-feuilles}{Maison
Aleph's Sesame Halvah 1001 Feuilles}

Advertisement

\protect\hyperlink{after-bottom}{Continue reading the main story}

\hypertarget{site-index}{%
\subsection{Site Index}\label{site-index}}

\hypertarget{site-information-navigation}{%
\subsection{Site Information
Navigation}\label{site-information-navigation}}

\begin{itemize}
\tightlist
\item
  \href{https://help.nytimes3xbfgragh.onion/hc/en-us/articles/115014792127-Copyright-notice}{©~2020~The
  New York Times Company}
\end{itemize}

\begin{itemize}
\tightlist
\item
  \href{https://www.nytco.com/}{NYTCo}
\item
  \href{https://help.nytimes3xbfgragh.onion/hc/en-us/articles/115015385887-Contact-Us}{Contact
  Us}
\item
  \href{https://www.nytco.com/careers/}{Work with us}
\item
  \href{https://nytmediakit.com/}{Advertise}
\item
  \href{http://www.tbrandstudio.com/}{T Brand Studio}
\item
  \href{https://www.nytimes3xbfgragh.onion/privacy/cookie-policy\#how-do-i-manage-trackers}{Your
  Ad Choices}
\item
  \href{https://www.nytimes3xbfgragh.onion/privacy}{Privacy}
\item
  \href{https://help.nytimes3xbfgragh.onion/hc/en-us/articles/115014893428-Terms-of-service}{Terms
  of Service}
\item
  \href{https://help.nytimes3xbfgragh.onion/hc/en-us/articles/115014893968-Terms-of-sale}{Terms
  of Sale}
\item
  \href{https://spiderbites.nytimes3xbfgragh.onion}{Site Map}
\item
  \href{https://help.nytimes3xbfgragh.onion/hc/en-us}{Help}
\item
  \href{https://www.nytimes3xbfgragh.onion/subscription?campaignId=37WXW}{Subscriptions}
\end{itemize}
