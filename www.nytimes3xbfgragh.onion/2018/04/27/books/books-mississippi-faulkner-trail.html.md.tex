Sections

SEARCH

\protect\hyperlink{site-content}{Skip to
content}\protect\hyperlink{site-index}{Skip to site index}

\href{https://www.nytimes3xbfgragh.onion/section/books}{Books}

\href{https://myaccount.nytimes3xbfgragh.onion/auth/login?response_type=cookie\&client_id=vi}{}

\href{https://www.nytimes3xbfgragh.onion/section/todayspaper}{Today's
Paper}

\href{/section/books}{Books}\textbar{}Marking Mississippi's Literary
Trail, From William Faulkner to Jesmyn Ward

\url{https://nyti.ms/2vQx3Qd}

\begin{itemize}
\item
\item
\item
\item
\item
\end{itemize}

Advertisement

\protect\hyperlink{after-top}{Continue reading the main story}

Supported by

\protect\hyperlink{after-sponsor}{Continue reading the main story}

\hypertarget{marking-mississippis-literary-trail-from-william-faulkner-to-jesmyn-ward}{%
\section{Marking Mississippi's Literary Trail, From William Faulkner to
Jesmyn
Ward}\label{marking-mississippis-literary-trail-from-william-faulkner-to-jesmyn-ward}}

\includegraphics{https://static01.graylady3jvrrxbe.onion/images/2018/04/29/arts/29xp-literary-print/26xp-mississippi-writers-slide-DJMN-articleLarge.jpg?quality=75\&auto=webp\&disable=upscale}

By \href{http://www.nytimes3xbfgragh.onion/by/laura-m-holson}{Laura M.
Holson}

\begin{itemize}
\item
  April 27, 2018
\item
  \begin{itemize}
  \item
  \item
  \item
  \item
  \item
  \end{itemize}
\end{itemize}

Mississippi authors have long stood their ground in the South's
competitive literary landscape.

\href{https://www.nytimes3xbfgragh.onion/2012/07/01/magazine/how-william-faulkner-tackled-race-and-freed-the-south-from-itself.html}{William
Faulkner} won a Nobel Prize in 1949 for his textured examination of
aristocratic decay in small-town Mississippi. Eudora Welty was awarded a
Presidential Medal of Freedom for her novels and essays. More recently,
the National Book Award winner Jesmyn Ward explored the dark moodiness
of race and poverty along Mississippi's Gulf Coast.

Now they and a host of other literary heroes from the Magnolia State
will be celebrated along the newly named Mississippi Writers Trail. This
month, the Mississippi Arts Commission received a \$30,000 grant from
the National Endowment for the Humanities to set up a series of markers
across the state to honor the contributions of its most prominent
writers.

The first place to be acknowledged will be the
\href{https://eudorawelty.org/the-house/}{home of Ms. Welty}, located in
the Belhaven neighborhood of Jackson. She lived there from 1925 until
she died in 2001, and it is where she wrote nearly all of her short
stories, essays and novels, including ``The Optimist's Daughter,'' which
won the Pulitzer Prize for fiction in 1973. The residence is now a
museum for the author's correspondence and photographs.

\includegraphics{https://static01.graylady3jvrrxbe.onion/images/2018/04/25/arts/26xp-mississippi-writers-slide-ECS2/26xp-mississippi-writers-slide-ECS2-articleLarge.jpg?quality=75\&auto=webp\&disable=upscale}

``Mississippi has a big presence in the birth of American culture,''
said Malcolm White, executive director of the Mississippi Arts
Commission. ``The biggest asset is our cultural story, and literature
and writing is part of that.''

Mississippi is perhaps better known for its music, particularly the
contributions of its blues musicians, like B.B. King and Muddy Waters,
who defined the genre inspired by African-American spirituals and folk
music. And like its neighbor Alabama, Mississippi has chronicled its
vivid, often violent, history during the civil rights era.

Mr. White said he had assembled a group of scholars to make
recommendations about which authors should be included. They are
considering 30 to 50 people, he said, among them,
\href{https://www.britannica.com/biography/Richard-Wright-American-writer}{Richard
Wright}, whose autobiography, ``Black Boy,'' is a coming-of-age story in
the Jim Crow South; and the poet
\href{https://www.nytimes3xbfgragh.onion/1998/12/04/books/margaret-walker-alexander-83-professor-and-author-of-jubilee.html}{Margaret
Walker Alexander}, who was born in Birmingham, Ala., but lived in
Jackson on a street named for her. The first of the markers will be
unveiled in August during the 2018 Mississippi Book Festival.

Mr. White said he had a number of conversations with
\href{https://www.nytimes3xbfgragh.onion/2012/06/24/books/review/richard-ford-by-the-book.html}{Richard
Ford}, the Pulitzer Prize winner, about whether the author wanted to
commemorate his childhood home in Jackson. The author grew up in the
same neighborhood as Ms. Welty, and they attended the same high school.

``It's like sacred ground for Mississippi literature,'' Mr. White said.
``I thought he would want his marker there.''

Image

William Faulkner received the Nobel Prize for literature in 1949 and won
Pulitzer Prizes for his novels ``A Fable'' and ``The
Reivers.''Credit...Associated Press

Mr. Ford, though, had other ideas. In the mid-1980s, the novelist, a
former sportswriter, lived in Clarksdale, a small town in the
Mississippi Delta 150 miles from Jackson. There, he spent days at the
Carnegie Public Library writing ``The Sportswriter,'' his 1986 novel
about a failed fiction writer turned sportswriter whose son dies. (He
followed it up with the Pulitzer Prize-winning ``Independence Day'' in
1995.) In the end, Mr. Ford wanted his marker at the library in
Clarksdale, not his childhood home.

``The Delta is where I chose to live,'' Mr. Ford said. ``Carnegie
Library is a refuge. They offered me a haven. I want to be remembered in
a place where people could go read books. Literature can be a way for
society to address what it doesn't want to address.''

One of those issues is systemic racism, which persists in America
despite the gains made in the 1960s. Mr. Ford, who now lives in Maine,
said he recently taught a class in which students had neither read nor
heard of ``Black Boy,'' Mr. Wright's seminal work that painted a grim
picture of race relations when it was published in 1945.

``I have no right to be surprised,'' Mr. Ford said. ``But that is a book
people need to read so that historical anomalies will not be allowed to
persist.''

Image

Richard Ford was born in Jackson, Miss., and lived in the same
neighborhood as Eudora Welty. ``It's like sacred ground for Mississippi
literature,'' said Malcolm White, executive director of the Mississippi
Arts Commission.Credit...Fred R. Conrad for The New York Times

\href{https://www.nytimes3xbfgragh.onion/2017/08/31/books/review/jesmyn-ward-by-the-book.html}{Ms.
Ward} grew up in Delisle, a small rural community not far from Biloxi.
In her most recent books, ``Salvage the Bones'' and ``Sing, Unburied,
Sing,'' she explores the desperation and occasional hopelessness of
rural America in the face of powerful institutions unwilling to address
the impacts of poverty, crime and racism, particularly in Mississippi,
which is among the poorest states in the country.

``We lived with those real and present problems in our lives,'' she said
of herself and her fellow writers. ``We push back against that
narrative. It's a good thing we are trying to tackle the problems we do.
We force readers to be more honest about what it means to be American.''

Not everyone shares Ms. Ward's optimism, however. The author Joyce Carol
Oates recently
\href{https://twitter.com/joycecaroloates/status/920638826971877379?lang=en}{retweeted
a photograph} of a banner hanging at Mississippi State University with
an image of Faulkner and the phrase ``Read, read, read. Read everything
\ldots{}''

Ms. Oates wrote: ``So funny! If Mississippians read, Faulkner would be
banned.''

Ms. Ward took issue with Ms. Oates and
\href{https://twitter.com/jesmimi/status/921112146565586944?lang=en}{posted
a tweet} of her own. ``Some Mississippians even (gasp!) WRITE. Shocking,
I know! Check my previous tweets for references, Joyce.''

When asked about the exchange, Ms. Ward said she was baffled as to why
Ms. Oates, who is not a stranger to
\href{https://www.nytimes3xbfgragh.onion/2015/11/24/books/joyce-carol-oates-celebratory-joyous-islamic-state-twitter.html}{scrapes
on Twitter}, would be so dismissive of her home state. ``We are already
dead last or next to dead last in next to everything,'' she said. ``At
least give us recognition for our literary writers.''

Advertisement

\protect\hyperlink{after-bottom}{Continue reading the main story}

\hypertarget{site-index}{%
\subsection{Site Index}\label{site-index}}

\hypertarget{site-information-navigation}{%
\subsection{Site Information
Navigation}\label{site-information-navigation}}

\begin{itemize}
\tightlist
\item
  \href{https://help.nytimes3xbfgragh.onion/hc/en-us/articles/115014792127-Copyright-notice}{©~2020~The
  New York Times Company}
\end{itemize}

\begin{itemize}
\tightlist
\item
  \href{https://www.nytco.com/}{NYTCo}
\item
  \href{https://help.nytimes3xbfgragh.onion/hc/en-us/articles/115015385887-Contact-Us}{Contact
  Us}
\item
  \href{https://www.nytco.com/careers/}{Work with us}
\item
  \href{https://nytmediakit.com/}{Advertise}
\item
  \href{http://www.tbrandstudio.com/}{T Brand Studio}
\item
  \href{https://www.nytimes3xbfgragh.onion/privacy/cookie-policy\#how-do-i-manage-trackers}{Your
  Ad Choices}
\item
  \href{https://www.nytimes3xbfgragh.onion/privacy}{Privacy}
\item
  \href{https://help.nytimes3xbfgragh.onion/hc/en-us/articles/115014893428-Terms-of-service}{Terms
  of Service}
\item
  \href{https://help.nytimes3xbfgragh.onion/hc/en-us/articles/115014893968-Terms-of-sale}{Terms
  of Sale}
\item
  \href{https://spiderbites.nytimes3xbfgragh.onion}{Site Map}
\item
  \href{https://help.nytimes3xbfgragh.onion/hc/en-us}{Help}
\item
  \href{https://www.nytimes3xbfgragh.onion/subscription?campaignId=37WXW}{Subscriptions}
\end{itemize}
