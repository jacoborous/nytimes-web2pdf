Sections

SEARCH

\protect\hyperlink{site-content}{Skip to
content}\protect\hyperlink{site-index}{Skip to site index}

\href{https://www.nytimes3xbfgragh.onion/section/sports/baseball}{Baseball}

\href{https://myaccount.nytimes3xbfgragh.onion/auth/login?response_type=cookie\&client_id=vi}{}

\href{https://www.nytimes3xbfgragh.onion/section/todayspaper}{Today's
Paper}

\href{/section/sports/baseball}{Baseball}\textbar{} A Big Dealer Holds
His Cards in Atlanta

\url{https://nyti.ms/2KjQ1lz}

\begin{itemize}
\item
\item
\item
\item
\item
\end{itemize}

Advertisement

\protect\hyperlink{after-top}{Continue reading the main story}

Supported by

\protect\hyperlink{after-sponsor}{Continue reading the main story}

Extra Bases

\hypertarget{a-big-dealer-holds-his-cards-in-atlanta}{%
\section{A Big Dealer Holds His Cards in
Atlanta}\label{a-big-dealer-holds-his-cards-in-atlanta}}

The Braves' new general manager, Alex Anthopoulos, made some
high-profile trades when he was with Toronto, but for now he is leery of
giving up prospects too hastily.

By \href{http://www.nytimes3xbfgragh.onion/by/tyler-kepner}{Tyler
Kepner}

\begin{itemize}
\item
  April 27, 2018
\item
  \begin{itemize}
  \item
  \item
  \item
  \item
  \item
  \end{itemize}
\end{itemize}

\includegraphics{https://static01.graylady3jvrrxbe.onion/images/2018/04/29/sports/29extrabases-1/merlin_133863942_9eb4fb8c-8809-4191-b23e-813d1ddd1e40-articleLarge.jpg?quality=75\&auto=webp\&disable=upscale}

When the Atlanta Braves hired Alex Anthopoulos as general manager in
November, action seemed likely to follow. The team needed help, having
just endured a third consecutive season with at least 90 losses, and
Anthopoulos had a knack for high-impact trades.

But there were no deals for stars like Josh Donaldson, David Price or
Troy Tulowitzki, who carried Toronto to a deep playoff run in 2015 after
Anthopoulos acquired them as the Blue Jays' general manager. In Atlanta,
essentially, Anthopoulos simply waited for the season to start. He moved
around some contracts in a deal with the Los Angeles Dodgers, for whom
he had worked as an assistant the last two seasons, but that was about
it.

``I talked to some people that had gone through the same thing and they
all, to a man, said, `In hindsight, I should have waited and given it a
little bit of time,''' Anthopoulos said. ``So the big thing was we
weren't going to trade any prospects at all, unless it was just an
absolute no-brainer. And the challenging part about it was we may miss a
window to sell high on a player. But the flip side was that there may be
a player that wasn't highly regarded that we may have traded who would
have come back to bite us. So I do think it'll balance out.''

As an example, Anthopoulos cited outfielder Ronald Acuña Jr., who soared
to the No. 1 overall spot in this year's Baseball America prospect
rankings, from 67th the year before. Acuña hit .325 with 21 homers and
44 runs batted in at three levels last season, and made his major league
debut on Wednesday. He went 3 for 4 with a homer the next afternoon.

Acuña is only 20, but he tore through spring training and seemed ready
to start in the majors. He struggled in the minors, but hit safely in
seven of eight games before his promotion.

Image

The Braves' Ronald Acuña Jr. in the dugout Thursday, when he played his
second game in the majors and went 3 for 4 with a home run.
Credit...John Minchillo/Associated Press

``You try your best to call these guys up at a time when they're hot and
playing well, so they're in a good place mentally, and also when you
feel, developmentally, they're here to stay,'' Anthopoulos said.
``That'll be up to him, but in terms of expectations: Play good defense,
play hard, have good at-bats. What the results are going to be, who
knows? But if he does those things, he's going to be a productive
player.''

By waiting a few weeks to call up Acuña, the Braves ensured that he
would not collect a full year of service time, meaning they control his
rights through 2024. Acuña is part of a young core that includes four
other everyday players under 30 --- second baseman Ozzie Albies, first
baseman Freddie Freeman, center fielder Ender Inciarte and shortstop
Dansby Swanson --- and several promising pitchers.

But the Braves' flow of young talent came with some sludge. Their
previous general manager, John Coppolella, was
\href{https://www.nytimes3xbfgragh.onion/2017/11/21/sports/baseball/mlb-atlanta-braves-ban.html}{banned
for life in November} for violating international signing rules. Major
League Baseball declared 13 of Atlanta's prospects to be free agents,
removed the Braves' third-round choice in the 2018 draft, and severely
restricted their international bonus allotment until 2022.

``I think everyone views what happened as a one-off,'' Anthopoulos said.
``I don't think people link the organization, specifically, to what
happened. There's a lot of great people here and a lot of great things
have been done here. It was addressed, it was dealt with, and it's over.
You're not going to ever hear me say, `Woe is us,' because every club
has challenges and no one wants to hear excuses. You're hired to get it
right, so embrace the upside, embrace the positives --- and there are a
ton of them here.''

The Braves, who will start a three-game series with Mets at Citi Field
on Tuesday, have lost only one series so far and entered the weekend
with a 14-10 record. Anthopoulos said he was especially encouraged by
the team's run differential (+30 through Thursday), which could be
significant.

Three years ago, when the Blue Jays had a middling record, their strong
run differential persuaded Anthopoulos to fortify the roster in July.
Now, after signing the former Blue Jays slugger Jose Bautista to a minor
league deal in hopes that he can play third base, Anthopoulos may take a
similar approach.

``From a front-office standpoint, it's our responsibility to support the
players,'' he said. ``It's still very early, but as we go on into the
summer, if they continue to show us we're a good team, a competitive
team, we'll look to go out and add, if we can. We're a long way from
that, but if they do, we've shown in the past that we'll be willing to
do that.''

\begin{center}\rule{0.5\linewidth}{\linethickness}\end{center}

Image

Cal Ripken greeting Sachio Kinugasa in Kansas City on June 14, 1996,
after Ripken broke Kinugasa's world record for most consecutive games by
playing in his 2,216th. Kinugasa died on Monday, and Ripken shared
memories of a player he got to know much better while conducting
baseball clinics in Japan in 2011.Credit...Roberto Borea/Associated
Press

\hypertarget{a-brotherhood-of-iron-men}{%
\subsection{A Brotherhood of Iron Men}\label{a-brotherhood-of-iron-men}}

On a bus in Japan in 2011, while conducting baseball clinics on a State
Department good-will tour, Cal Ripken finally found someone who could
relate to his greatest achievement. Ripken, the Hall of Fame infielder
for the Baltimore Orioles, played in a record 2,632 consecutive games.
On the bus was Sachio Kinugasa, who played 2,215 games in a row for the
Hiroshima Carp. They had met before, but they bonded in Kinugasa's home
country, which had just been devastated by a tsunami.

\href{https://www.nytimes3xbfgragh.onion/2018/04/26/obituaries/sachio-kinugasa-japanese-baseballs-iron-man-is-dead-at-71.html}{Kinugasa
died on Monday}, at age 71, and Ripken, who was conducting clinics in
Prague, was eager to discuss their friendship in a phone call.

``There wasn't anybody that had played that many games that I could talk
to, and it was fun to talk about the challenges and how he looked at it,
and how his approach was pretty similar to mine: Basically as a baseball
player, it was his responsibility to come to the ballpark and play ---
and he did, and that was refreshing,'' Ripken said.

``The demands, learning what you're capable of, finding that you could
play with an injury and be successful --- that was part of the battle.
You're always playing at somewhat less than 100 percent, and it was fun
having someone to relate to that --- you get hit with a pitch in the
hand, you get hit with a pitch in the elbow, you don't have feeling in
your fingers when you throw a baseball, and somehow you figure out how
to get through that.''

Ripken passed Lou Gehrig's major league record of 2,130 straight games
in September 1995. The next June in Kansas City, when Ripken played in
his 2,216th consecutive game, Kinugasa was there to greet him.

Ripken fondly recalled Kinugasa's passion for the game.

``There was a kidlike presence about him,'' Ripken said. ``He almost
seemed to be enjoying the clinics and the interaction with the kids as
much as the kids were. He was laughing and he had a big smile on his
face. He had a youthful persona that was really cool to see.''

Kinugasa, who played mostly on the infield corners, hit 504 career home
runs over 23 seasons, all with Hiroshima. He started his streak in late
1970 and played every game of a 130-game schedule through his retirement
in 1987.

Entering the weekend, the majors' longest active streak of consecutive
games, according to the Elias Sports Bureau, was 355 by Alcides Escobar
of the Kansas City Royals. With teams paying more and more attention to
rest, Ripken's record is secure.

``I think nowadays the definition of an everyday player is not 162
games, it's something less,'' Ripken said. ``But that could change. I
feel really good that certain players want to test themselves and see if
they could play 162, and I admire that --- because even if you play 150,
you're still going to have the same mental toughness about playing every
day and the same physical challenges where you're playing less than 100
percent. The 12-game difference is not much of a difference than pushing
through and playing every one.''

\begin{center}\rule{0.5\linewidth}{\linethickness}\end{center}

Image

 Evan Mathis as a Cincinnati Bengal in 2009. Mathis owned a 1952 Topps
Mickey Mantle rookie card for two years, and he recently sold it for
\$2.88 million.Credit...David Kohl/Associated Press

\hypertarget{a-football-player-and-baseball-card-collector}{%
\subsection{A Football Player and Baseball Card
Collector}\label{a-football-player-and-baseball-card-collector}}

Evan Mathis made his mark on professional football over 12 seasons,
earning two Pro Bowl selections as a guard for the Philadelphia Eagles
and winning a Super Bowl with the Denver Broncos. Now he has made
baseball history, in a way.

On Monday, Mathis sold his 1952 Topps Mickey Mantle rookie card for
\$2.88 million through Heritage Auctions, which called it a record price
for a post-World War II trading card. The card was graded a nine by
Professional Sports Authenticator, making it one of only six 1952 Mantle
cards --- out of more than 1,500 graded --- with that distinction. Just
three are known to rate a perfect 10.

``If we had a time machine and went back to 1952 to open boxes of Topps
cards, we'd run into a lot of off-center, mis-cut, and poorly printed
cards,'' Mathis said in an email interview. ``A card making it out of
the factory in mint condition was already hard enough. And for that mint
card to have survived all of this time since then really sets it apart
from the rest. There's just something about pieces of history that have
withstood the test of time that has always been appealing to me.''

Mathis, 36, started collecting at age 6, hunting for Bo Jackson cards.
The hobby was booming back then --- but not in a good way, at least for
return on investment.

Image

Mathis's 1952 Topps Mickey Mantle baseball cardCredit...Heritage
Auctions, HA.com

``The '80s and '90s are referred to as the `junk wax' era because of how
massively overproduced everything was,'' Mathis said. ``There were a lot
of people buying cards, and as a result, a lot of people not getting a
great R.O.I. on what they were putting in. Naturally, this led a lot of
people away from the hobby, as does simply growing up.''

When Mathis grew up, though, he found himself drawn back to the hobby by
purchasing boxes of vintage cards and searching for those with high
grades. The Mantle was his prize, and while he did not say exactly how
much he paid for it, he did not strike it rich with the auction.

``I owned the Mantle for two years, and my cost basis was pretty close
to what it brought at auction, so it's not like I won the lottery or
anything,'' Mathis said. ``I freed up some money to buy a house and gave
myself some liquidity to do some more buying and selling of cards. I
enjoy buying collections and going through them to cherry-pick the cards
that need to get graded.''

Mathis, who interacts with other collectors at
\href{http://selltoevan.com/}{selltoevan.com}, has made collecting his
primary pursuit since retiring after the 2016 season. His current
centerpiece is a 1953 Topps Mantle card with a 10 grade, which is
painfully close to his Holy Grail: the same card he just auctioned, but
with a perfect 10.

``I can't afford it now,'' Mathis said, ``but I've always liked to think
big.''

\begin{center}\rule{0.5\linewidth}{\linethickness}\end{center}

Image

Albert Pujols is close to joining Hank Aaron, Willie Mays and Alex
Rodriguez in one of baseball's most elite clubs.Credit...Chris
Carlson/Associated Press

\hypertarget{a-pujols-distinction}{%
\subsection{A Pujols Distinction}\label{a-pujols-distinction}}

As the Los Angeles Angels returned to Anaheim for a six-game home stand
with the Yankees and the Baltimore Orioles, Albert Pujols needed six
hits to reach 3,000 for his career. That would give him a unique
distinction in major league history.

Pujols, who reached 600 homers last season, will join Hank Aaron, Willie
Mays and Alex Rodriguez as the only players with 600 home runs and 3,000
hits. But Pujols has something the three others do not: multiple World
Series titles. The others won just one apiece: Aaron with the 1957
Milwaukee Braves, Mays with the 1954 New York Giants and Rodriguez with
the 2009 Yankees. Pujols won with the St. Louis Cardinals in 2006 and
2011.

Pujols also has the highest career on-base plus slugging percentage of
the group, at .945, with Mays next at .941. But that figure could drop
before Pujols, 38, retires. He is signed through 2021, and, entering the
weekend, his O.P.S. across seven seasons with the Angels was just .776.

\begin{center}\rule{0.5\linewidth}{\linethickness}\end{center}

Image

The jersey of reliever Danny Farquhar hanging in the White Sox bullpen
the day after he suffered a ruptured brain aneurysm in the dugout.
Credit...Jon Durr/Getty Images

\hypertarget{a-writer-reaches-out-to-a-stricken-player}{%
\subsection{A Writer Reaches Out to a Stricken
Player}\label{a-writer-reaches-out-to-a-stricken-player}}

Peter Gammons was 61 years old in 2006 when he was seized by a terrible
headache on his way to the gym near his home on Cape Cod. Gammons, a
winner of the writers' award at the Baseball Hall of Fame, was airlifted
to a Boston hospital and underwent surgery to treat a brain aneurysm.
Two and a half months later, he was back to work at ESPN.

Gammons, who now writes for The Athletic, reached out recently to Danny
Farquhar, the White Sox reliever who suffered a ruptured brain aneurysm
**** in the dugout during a game in Chicago on April 20. He sent a
message of support on Twitter and contacted the White Sox, offering to
share his experiences with Farquhar's family.

``I know from my wife that the first couple of days, even though the
doctor had gone through my skull and clamped the aneurysm, she and the
rest of my family still didn't know what it was going to be like when
the recovery's over,'' Gammons said. ``There's a period of uncertainty
and a lot of doubt. I just wanted to say to them, `Hey, I've been
through this, this is what happens, and you will come back.'''

Image

The writer Peter Gammons at the 2017 World Series. Gammons has reached
out to Farquhar via social media and the White Sox.Credit...Christian
Petersen/Getty Images

On Monday, in
\href{https://twitter.com/whitesox/status/988497873217228803}{their most
recent update} on his condition, the White Sox said that Farquhar was in
critical, but neurologically stable, condition in the intensive care
unit at Rush University Medical Center after surgery. The team said he
had use of his extremities, could speak and was responding appropriately
to questions and commands from doctors.

Farquhar was expected to remain in the I.C.U. for the next few weeks,
according to the White Sox, who placed him on the 60-day disabled list.
Farquhar is 31, and Gammons believes that his physical condition could
help his chances for recovery.

``I think it will, because he's strong and he's in incredible shape,''
Gammons said. ``People who are in pretty good shape, and whose bodies
can withstand shock, have a better chance of returning to normal.''

Gammons said he would like to visit Farquhar, who has a wife and three
children, when Farquhar is out of the hospital.

``I've never met Lexie, his wife, but I've always liked Danny a lot,''
Gammons said. ``I would just want to tell her that this is not the end
of the world, and he can still have a great life.''

Advertisement

\protect\hyperlink{after-bottom}{Continue reading the main story}

\hypertarget{site-index}{%
\subsection{Site Index}\label{site-index}}

\hypertarget{site-information-navigation}{%
\subsection{Site Information
Navigation}\label{site-information-navigation}}

\begin{itemize}
\tightlist
\item
  \href{https://help.nytimes3xbfgragh.onion/hc/en-us/articles/115014792127-Copyright-notice}{©~2020~The
  New York Times Company}
\end{itemize}

\begin{itemize}
\tightlist
\item
  \href{https://www.nytco.com/}{NYTCo}
\item
  \href{https://help.nytimes3xbfgragh.onion/hc/en-us/articles/115015385887-Contact-Us}{Contact
  Us}
\item
  \href{https://www.nytco.com/careers/}{Work with us}
\item
  \href{https://nytmediakit.com/}{Advertise}
\item
  \href{http://www.tbrandstudio.com/}{T Brand Studio}
\item
  \href{https://www.nytimes3xbfgragh.onion/privacy/cookie-policy\#how-do-i-manage-trackers}{Your
  Ad Choices}
\item
  \href{https://www.nytimes3xbfgragh.onion/privacy}{Privacy}
\item
  \href{https://help.nytimes3xbfgragh.onion/hc/en-us/articles/115014893428-Terms-of-service}{Terms
  of Service}
\item
  \href{https://help.nytimes3xbfgragh.onion/hc/en-us/articles/115014893968-Terms-of-sale}{Terms
  of Sale}
\item
  \href{https://spiderbites.nytimes3xbfgragh.onion}{Site Map}
\item
  \href{https://help.nytimes3xbfgragh.onion/hc/en-us}{Help}
\item
  \href{https://www.nytimes3xbfgragh.onion/subscription?campaignId=37WXW}{Subscriptions}
\end{itemize}
