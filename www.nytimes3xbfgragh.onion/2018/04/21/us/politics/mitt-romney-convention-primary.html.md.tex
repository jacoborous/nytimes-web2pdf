Sections

SEARCH

\protect\hyperlink{site-content}{Skip to
content}\protect\hyperlink{site-index}{Skip to site index}

\href{https://www.nytimes3xbfgragh.onion/section/politics}{Politics}

\href{https://myaccount.nytimes3xbfgragh.onion/auth/login?response_type=cookie\&client_id=vi}{}

\href{https://www.nytimes3xbfgragh.onion/section/todayspaper}{Today's
Paper}

\href{/section/politics}{Politics}\textbar{}Mitt Romney Fails to Bypass
Utah Primary for U.S. Senate

\url{https://nyti.ms/2vyLOqM}

\begin{itemize}
\item
\item
\item
\item
\item
\end{itemize}

Advertisement

\protect\hyperlink{after-top}{Continue reading the main story}

Supported by

\protect\hyperlink{after-sponsor}{Continue reading the main story}

\hypertarget{mitt-romney-fails-to-bypass-utah-primary-for-us-senate}{%
\section{Mitt Romney Fails to Bypass Utah Primary for U.S.
Senate}\label{mitt-romney-fails-to-bypass-utah-primary-for-us-senate}}

\includegraphics{https://static01.graylady3jvrrxbe.onion/images/2018/04/22/business/22romney/merlin_137157948_a9bba598-b7b9-4ba6-99c4-811ebc6d0cbd-articleLarge.jpg?quality=75\&auto=webp\&disable=upscale}

By The Associated Press

\begin{itemize}
\item
  April 21, 2018
\item
  \begin{itemize}
  \item
  \item
  \item
  \item
  \item
  \end{itemize}
\end{itemize}

WEST VALLEY CITY, Utah --- Mitt Romney was forced on Saturday into a
Republican primary for a United States Senate seat in Utah as he looks
to restart his political career by replacing Orrin G. Hatch, a longtime
senator who is retiring.

Mr. Romney, a former governor of Massachusetts and the Republican
candidate for president in 2012, remains the heavy favorite to win the
Senate seat in November. But he could have bypassed a primary altogether
by earning a majority of votes on Saturday at the state's G.O.P.
convention.

Instead, the far-right party delegates preferred State Representative
Mike Kennedy, who got 51 percent of the vote to Mr. Romney's 49 percent.

Voters will decide between the candidates in a June 26 primary. Mr.
Romney had previously secured his spot on the ballot by collecting
28,000 voter signatures, but he said on Saturday that the choice was
partly to blame for his loss.

Gathering signatures is unpopular among many conservative delegates in
the state who say it dilutes their ability to choose a candidate. The
issue prompted hours of debate, shouting and booing at the convention.

Jenny Wilson, a councilwoman in Salt Lake County, is the leading
Democratic candidate in a state that has had only Republican senators
since 1977.

At the convention, Mr. Romney faced 11 other candidates, mostly
political newcomers who questioned his criticism of President Trump and
the depth of his ties to Utah. He had spent two months on the campaign
trail visiting dairy farms, taking photos with college students and
making stump speeches in small towns.

``Some people I've spoken with have said this is a David vs. Goliath
race, but they're wrong,'' Mr. Romney said in his speech. ``I'm not
Goliath. Washington, D.C., is Goliath.''

Mr. Kennedy, a doctor and lawyer who has been a state lawmaker since
2013, received applause from the crowd as he criticized the national
debt, Common Core education standards and President Barack Obama's
health care law. He framed himself as an underdog taking on the ``Romney
machine.''

After Mr. Romney's failed presidential campaign, he moved to Utah, where
he is popular because of his Mormon faith and his role in the 2002
Winter Olympics, which were held in Salt Lake City.

He has worked to keep the focus on state issues rather than his history
of well-documented feuds with Mr. Trump, whom he called a ``con man''
and a phony during the 2016 race. Mr. Trump fired back that Mr. Romney
``choked like a dog'' during his own White House run.

The two men have shown signs of making peace, and Mr. Romney has
accepted Mr. Trump's endorsement. But Mr. Romney said on Saturday that
he has not decided whether he will endorse the president's 2020
re-election bid.

Advertisement

\protect\hyperlink{after-bottom}{Continue reading the main story}

\hypertarget{site-index}{%
\subsection{Site Index}\label{site-index}}

\hypertarget{site-information-navigation}{%
\subsection{Site Information
Navigation}\label{site-information-navigation}}

\begin{itemize}
\tightlist
\item
  \href{https://help.nytimes3xbfgragh.onion/hc/en-us/articles/115014792127-Copyright-notice}{©~2020~The
  New York Times Company}
\end{itemize}

\begin{itemize}
\tightlist
\item
  \href{https://www.nytco.com/}{NYTCo}
\item
  \href{https://help.nytimes3xbfgragh.onion/hc/en-us/articles/115015385887-Contact-Us}{Contact
  Us}
\item
  \href{https://www.nytco.com/careers/}{Work with us}
\item
  \href{https://nytmediakit.com/}{Advertise}
\item
  \href{http://www.tbrandstudio.com/}{T Brand Studio}
\item
  \href{https://www.nytimes3xbfgragh.onion/privacy/cookie-policy\#how-do-i-manage-trackers}{Your
  Ad Choices}
\item
  \href{https://www.nytimes3xbfgragh.onion/privacy}{Privacy}
\item
  \href{https://help.nytimes3xbfgragh.onion/hc/en-us/articles/115014893428-Terms-of-service}{Terms
  of Service}
\item
  \href{https://help.nytimes3xbfgragh.onion/hc/en-us/articles/115014893968-Terms-of-sale}{Terms
  of Sale}
\item
  \href{https://spiderbites.nytimes3xbfgragh.onion}{Site Map}
\item
  \href{https://help.nytimes3xbfgragh.onion/hc/en-us}{Help}
\item
  \href{https://www.nytimes3xbfgragh.onion/subscription?campaignId=37WXW}{Subscriptions}
\end{itemize}
