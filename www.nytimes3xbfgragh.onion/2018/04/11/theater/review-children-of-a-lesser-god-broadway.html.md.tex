Sections

SEARCH

\protect\hyperlink{site-content}{Skip to
content}\protect\hyperlink{site-index}{Skip to site index}

\href{https://www.nytimes3xbfgragh.onion/section/theater}{Theater}

\href{https://myaccount.nytimes3xbfgragh.onion/auth/login?response_type=cookie\&client_id=vi}{}

\href{https://www.nytimes3xbfgragh.onion/section/todayspaper}{Today's
Paper}

\href{/section/theater}{Theater}\textbar{}Review: Sound, or Silence? A
Passionate Debate in `Children of a Lesser God'

\url{https://nyti.ms/2JEZvY8}

\begin{itemize}
\item
\item
\item
\item
\item
\item
\end{itemize}

Advertisement

\protect\hyperlink{after-top}{Continue reading the main story}

Supported by

\protect\hyperlink{after-sponsor}{Continue reading the main story}

\hypertarget{review-sound-or-silence-a-passionate-debate-in-children-of-a-lesser-god}{%
\section{Review: Sound, or Silence? A Passionate Debate in `Children of
a Lesser
God'}\label{review-sound-or-silence-a-passionate-debate-in-children-of-a-lesser-god}}

\includegraphics{https://static01.graylady3jvrrxbe.onion/images/2018/04/12/arts/12childrenofalessergod-1/12childrenofalessergod-1-articleLarge.jpg?quality=75\&auto=webp\&disable=upscale}

\begin{itemize}
\tightlist
\item
  Children of a Lesser God\\
  Broadway, Comedy/Drama, Play 2 hrs. and 20 min. Open Run Studio 54,
  254 W. 54th St. 212-239-6200
\end{itemize}

By \href{http://www.nytimes3xbfgragh.onion/by/jesse-green}{Jesse Green}

\begin{itemize}
\item
  April 11, 2018
\item
  \begin{itemize}
  \item
  \item
  \item
  \item
  \item
  \item
  \end{itemize}
\end{itemize}

Aside from a brief prologue, the first thing the character Sarah says in
``Children of a Lesser God'' is --- well, I can't tell you.

Even if I could, it wouldn't be as funny or as powerful as the way she
delivers it, using a combination of poetically vulgar gestures to paint
her friend Orin as a suck-up.

That's because Sarah speaks American Sign Language, and I have only
polite English at my disposal. A.S.L. wins, hands down --- or, as the
case may be, hands up.

The pungency of sign language is not the subject of Mark Medoff's
\href{http://childrenofalessergodbroadway.com}{``Children of a Lesser
God,''} which opened on Wednesday at Studio 54 in a mixed bag of a
Broadway revival directed by Kenny Leon. But it's a wonderful bonus to
the play's fierce rivalry between those who promote spoken English as
the highest attainable form of communication and those who are staunch
partisans of silence.

Sarah, 26 and deaf since birth, is one of those partisans. For 21 years,
the school where the play takes place has tried to push speech and
lip-reading on her, only hardening her resolve not to learn them. ``I
don't do things that I can't do well,'' she explains.

That would seem like rationalization if Sarah weren't so smart and, in a
knockout
\href{https://www.nytimes3xbfgragh.onion/2018/02/14/t-magazine/lauren-ridloff.html}{professional
debut performance by Lauren Ridloff}, so superb at A.S.L. Her fluency
and expressiveness make the English speakers around her seem, as one of
them actually calls her, dumb.

\includegraphics{https://static01.graylady3jvrrxbe.onion/images/2018/04/11/arts/12childrenofalessergod-2/12childrenofalessergod-2-articleLarge.jpg?quality=75\&auto=webp\&disable=upscale}

Perhaps the novelty of that idea helped to make the original Broadway
production of ``Children of a Lesser God,'' which opened in 1980, a hit.
Today its dramaturgy seems creaky, even when the argument is crackling.

Naturally, the conceit of the play requires Sarah, who now works at the
school as a custodian, to face a persuasive combatant in the form of her
latest teacher, James. Rather less naturally, it requires her to fall in
love with him.

And wouldn't you know it: James is not only a fan of spoken English, but
also a fan of his own voice. He's glib and lordly and --- giving credit
to a lovely performance by Joshua Jackson --- charming, too. He almost
makes you forget that he's browbeating Sarah into the denial of
something she sees as a central, positive element of her personhood.

To her, as to many deaf people today, deafness is an identity, not a
defect or a curse imposed by a lesser god. (The play's title
\href{https://www.poetryfoundation.org/poems/45325/idylls-of-the-king-the-passing-of-arthur}{comes
from Tennyson's poem ``Idylls of the King.''}) And for Sarah, if not
Orin (who speaks and lip-reads), sign language is the unique expression
of that identity.

At least on the page, ``Children of a Lesser God'' seems skeptical of
Sarah's conviction, painting her refusal to speak as a form of hysteria,
with abuse and cold parenting in the background. (Her mother is played,
implacably, by Kecia Lewis.)

But Ms. Ridloff's blistering performance, like Phyllis Frelich's in the
original and Marlee Matlin's
\href{https://www.youtube.com/watch?v=E0AMKG31wME}{in the 1986 movie},
contradicts that skepticism. We see in front of us that A.S.L. is not
just a different language, but also a different way of thinking in which
``veal'' might be rendered as ``cowbaby'' and change the way you feel
about it. (Alexandria Wailes is the sign language director.) Dialogue
that seems banal in English --- as the production's supertitles
frequently make plain --- is as gorgeous and physical as sculpture when
signed.

In any case, the story no longer reads, if it ever did, as simply the
conflict between a smug hearing person and a defensive deaf one. Though
the romance between Sarah and James is mutual, the antiquated
assumptions behind it keep pushing the male-female battle into the
foreground.

Image

A scene from the play ``Children of a Lesser Godl'' at Studio
54.Credit...Sara Krulwich/The New York Times

That would be fine if Sarah were as well armed to fight it as she is to
fight the hegemony of the hearing. But Mr. Medoff allows neither Sarah
nor James any insight into the way their behavior is conditioned by
invented notions of gender, even once they marry. Sarah, otherwise a
paragon of self-possession, frets over a runny quiche and coos over a
blender. She proves herself as a wife by playing bridge. James sulks
manfully, with Bach.

Mr. Leon's color-conscious casting adds yet another level of contrast.
(Ms. Ridloff is of mixed race; Mr. Jackson is white.) With no lines to
address it, this extension of the theme of assumed privilege can only
serve as a descant to the others, but sometimes that's enough. When
Sarah says she wants to have children --- deaf children --- and James
blanches, the parallel to racism gives the moment a tantalizing frisson.

Unfortunately, little else does. When I saw this production's tryout
last summer at the Berkshire Theater Group, in Stockbridge, Mass.,
\href{https://www.nytimes3xbfgragh.onion/2017/07/05/theater/in-a-summer-theater-weekend-odd-couples-and-immigrants-on-the-make.html}{I
admired the leading actors} but hoped the dull staging and plodding pace
might be enhanced for Broadway. I still admire the leading actors and am
even awe-struck by Mr. Jackson, whose television work, including
``\href{https://www.nytimes3xbfgragh.onion/2014/10/10/arts/television/the-affair-on-showtime-stars-dominic-west-and-ruth-wilson.html}{The
Affair},'' did not prepare me for the virtuosity of his simultaneous
rendering of the role in spoken English and A.S.L. while also
interpreting for Ms. Ridloff.

But the uncharacteristically cheap-looking set by Derek McLane --- just
some tree trunks, chairs and portentous, empty door frames --- looks
even worse at Studio 54 and does nothing to help us contextualize the
story. (Mr. Leon's direction seems random.) Conversely, the choice of
music contextualizes it too much, pinning down the late-70s so bluntly
(``\href{https://www.youtube.com/watch?v=ap87QgZKTNw}{Silly Love
Songs}''? Really?) as to undercut its seriousness and timeliness.

Then too, as Walter Kerr wrote
\href{http://www.nytimes3xbfgragh.onion/packages/pdf/theater/112149885.pdf}{in
The New York Times} of the 1980 production, the play falters badly in
its second act, ginning up all sorts of spurious conflict to fill time.
Unlikely jealousies, threats of firing by the school administration
(Anthony Edwards is wasted as the head teacher) and an employment
discrimination lawsuit pursued by Orin (John McGinty) all come to
nothing, or very little.

Eventually you realize that Mr. Medoff simply did not have the
wherewithal to dramatize the fundamental conflict any further because
James is his hero but Sarah is right. The play, written when it was,
can't quite support that --- not because of its deaf politics, but
because of its sexual politics. Like James, Mr. Medoff insists on
speaking for Sarah without fully accepting her independence.

In Ms. Ridloff's performance, though, she answers him, and us, in the
most profound way possible: silence.

Advertisement

\protect\hyperlink{after-bottom}{Continue reading the main story}

\hypertarget{site-index}{%
\subsection{Site Index}\label{site-index}}

\hypertarget{site-information-navigation}{%
\subsection{Site Information
Navigation}\label{site-information-navigation}}

\begin{itemize}
\tightlist
\item
  \href{https://help.nytimes3xbfgragh.onion/hc/en-us/articles/115014792127-Copyright-notice}{©~2020~The
  New York Times Company}
\end{itemize}

\begin{itemize}
\tightlist
\item
  \href{https://www.nytco.com/}{NYTCo}
\item
  \href{https://help.nytimes3xbfgragh.onion/hc/en-us/articles/115015385887-Contact-Us}{Contact
  Us}
\item
  \href{https://www.nytco.com/careers/}{Work with us}
\item
  \href{https://nytmediakit.com/}{Advertise}
\item
  \href{http://www.tbrandstudio.com/}{T Brand Studio}
\item
  \href{https://www.nytimes3xbfgragh.onion/privacy/cookie-policy\#how-do-i-manage-trackers}{Your
  Ad Choices}
\item
  \href{https://www.nytimes3xbfgragh.onion/privacy}{Privacy}
\item
  \href{https://help.nytimes3xbfgragh.onion/hc/en-us/articles/115014893428-Terms-of-service}{Terms
  of Service}
\item
  \href{https://help.nytimes3xbfgragh.onion/hc/en-us/articles/115014893968-Terms-of-sale}{Terms
  of Sale}
\item
  \href{https://spiderbites.nytimes3xbfgragh.onion}{Site Map}
\item
  \href{https://help.nytimes3xbfgragh.onion/hc/en-us}{Help}
\item
  \href{https://www.nytimes3xbfgragh.onion/subscription?campaignId=37WXW}{Subscriptions}
\end{itemize}
