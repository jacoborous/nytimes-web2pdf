Sections

SEARCH

\protect\hyperlink{site-content}{Skip to
content}\protect\hyperlink{site-index}{Skip to site index}

\href{https://www.nytimes3xbfgragh.onion/section/politics}{Politics}

\href{https://myaccount.nytimes3xbfgragh.onion/auth/login?response_type=cookie\&client_id=vi}{}

\href{https://www.nytimes3xbfgragh.onion/section/todayspaper}{Today's
Paper}

\href{/section/politics}{Politics}\textbar{}New Trump Rule Rolls Back
Protections of the Affordable Care Act

\url{https://nyti.ms/2I405ga}

\begin{itemize}
\item
\item
\item
\item
\item
\item
\end{itemize}

Advertisement

\protect\hyperlink{after-top}{Continue reading the main story}

Supported by

\protect\hyperlink{after-sponsor}{Continue reading the main story}

\hypertarget{new-trump-rule-rolls-back-protections-of-the-affordable-care-act}{%
\section{New Trump Rule Rolls Back Protections of the Affordable Care
Act}\label{new-trump-rule-rolls-back-protections-of-the-affordable-care-act}}

\includegraphics{https://static01.graylady3jvrrxbe.onion/images/2018/06/20/us/politics/20dc-healthvid/20dc-healthvid-videoSixteenByNine3000.jpg}

By \href{https://www.nytimes3xbfgragh.onion/by/robert-pear}{Robert Pear}

\begin{itemize}
\item
  June 19, 2018
\item
  \begin{itemize}
  \item
  \item
  \item
  \item
  \item
  \item
  \end{itemize}
\end{itemize}

WASHINGTON --- A sweeping new rule issued Tuesday by the Trump
administration will make it easier for small businesses to join forces
and set up health insurance plans that circumvent many requirements of
the Affordable Care Act, cutting costs but also reducing benefits.

President Trump, speaking at a 75th-anniversary celebration of the
National Federation of Independent Business, said the new rule would
allow small businesses to ``escape some of Obamacare's most burdensome
mandates'' by creating new entities known as association health plans.

``You're going to save massive amounts of money and have much better
health care,'' Mr. Trump said to cheers from the audience of
small-business owners. ``It's going to cost you much less.''

As a result, he said, ``you're going to save a fortune.''

Alexander Acosta, the labor secretary, said the rule would give small
businesses access to insurance options like those available to large
companies, starting as soon as Sept. 1. Millions of people could
benefit, he said.

``As the cost of insurance for small businesses has been increasing,''
Mr. Acosta said, ``the percentage of small businesses offering health
care coverage has been dropping substantially. Today, the Trump
administration helps level the playing field between large companies and
small businesses.''

The new health plans would be exempt from many consumer-protection
mandates in the Affordable Care Act. They may not have to provide
certain ``essential health benefits'' like mental health care, emergency
services, maternity and newborn care, and prescription drugs.

Labor Department officials said association health plans would not be
able to deny coverage or charge higher rates to individual employees
with pre-existing medical conditions.

Still, consumer groups and many state officials opposed the push for
association health plans. They said such plans would draw healthy people
out of the Affordable Care Act marketplace, driving up costs for those
who need comprehensive insurance.

Senator Chuck Schumer of New York, the Democratic leader, said Tuesday
that the new rule would open the door to ``junk health insurance.'' It
is, he said, ``the latest act of sabotage of our health care system by
the Trump administration.''

The new final rule carries out
\href{https://www.nytimes3xbfgragh.onion/2017/10/07/us/politics/trump-association-health-plans.html}{an
executive order} signed by Mr. Trump on Oct. 12.

The rule will allow small-business owners, their employees, sole
proprietors and other self-employed people to join together to buy or
provide insurance in the large-group market through association health
plans.

Because they will be exempt from many requirements of the 2010 health
law, Mr. Trump has said, the association health plans can ``provide more
affordable health insurance options to many Americans, including hourly
wage earners, farmers and the employees of small businesses.''

The new health plans might, for example, appeal to restaurant workers,
real estate agents, dry cleaners, florists, plumbers and painters,
officials said.

Under the rule, Mr. Acosta said, ``business associations from city
chambers of commerce to nationwide industry groups can offer health care
insurance to the employees of their employer members through the
large-group market.''

Trump administration officials said that small businesses and
self-employed people in the same industry, state or region could band
together and obtain health coverage as if they were a single large
employer --- even if they had no other connections to one another.

Until now, the Labor Department has required a much greater
``commonality of interest'' among small businesses that wanted to be
treated as a large group when buying insurance. And the government often
looked at the size of each company, rather than the group as a whole, to
determine if it was a large or small employer.

The new rule takes a step toward fulfilling Mr. Trump's campaign promise
to make it easier for companies to sell insurance across state lines.

``For the first time ever,'' Mr. Trump said Tuesday, ``sole proprietors
will be able to come together and buy lower-cost group insurance instead
of getting ripped off by this disaster that we all know as Obamacare.
These actions will result in very low prices, much more choice, much
more freedom, including in many cases new opportunities to purchase
health insurance. You'll be able to do this across state lines.''

The new rule prohibits certain kinds of discrimination. Association
health plans cannot charge higher premiums to some employers based on
the health status of those companies' employees, and they cannot
discriminate because of any ``health factor.''

The Labor Department gave 10 examples of how the new rule would work. An
association health plan for restaurants could not deny membership to a
business with several employees with large health claims. Nor could it
exclude an employee with diabetes or charge that person more because of
the disease.

But, the Labor Department said, an association plan for agriculture
could charge different premiums to employers in different lines of
business, like growing crops, raising livestock, harvesting seafood or
producing timber.

Likewise, an association plan open to all employers based in a
particular state could charge different premiums to employers in
construction, education, financial services or other sectors, and could
set different rates for part-time and full-time employees, so long as
those distinctions were not based on health factors.

The Labor Department said it did not believe association health plans
would write their membership rules to avoid high-cost areas or high-risk
professions, as some consumer groups fear. An association offering a
health plan is free to define its membership criteria, provided those
standards do not operate as ``a subterfuge for discrimination'' based on
health factors.

Critics were not convinced. ``This rule will erode the availability of
affordable comprehensive coverage in most states' individual and
small-group markets,'' said Keysha Brooks-Coley, a vice president of the
lobbying arm of the American Cancer Society.

Association health plans can either buy commercial insurance or pay
claims out of their own assets. States can regulate both types of health
plans, but the Trump administration said it could, in the future,
pre-empt state insurance laws that ``go too far in regulating''
self-insured plans.

Similar small-business health plans
\href{https://www.nytimes3xbfgragh.onion/2017/10/21/us/politics/trump-association-health-plans-fraud.html}{have
a history of fraud and abuse} that have left employers and employees
with hundreds of millions in unpaid claims.

The Coalition Against Insurance Fraud, representing insurers, consumer
groups and law enforcement officials, met with Trump administration
officials last month and emphasized the need for states to have a strong
role in combating possible fraud.

James Quiggle, a spokesman for the coalition, said Tuesday that the new
rule ``could be an invitation to scam operators to try and slip bogus
health plans through the regulatory system and into the marketplace.''
For this reason, he said, it is imperative that the Labor Department
have more money and personnel to vet and supervise association health
plans.

The Labor Department said it would monitor the new plans to ensure
compliance with the law and to protect consumers. States will share
enforcement authority with the federal government, it said.

Mr. Trump predicted that ``big percentages of this country'' would form
or join association health plans. ``In fact,'' he told small-business
owners, ``while you're in the room together, shake hands, form an
association. Good luck!''

Advertisement

\protect\hyperlink{after-bottom}{Continue reading the main story}

\hypertarget{site-index}{%
\subsection{Site Index}\label{site-index}}

\hypertarget{site-information-navigation}{%
\subsection{Site Information
Navigation}\label{site-information-navigation}}

\begin{itemize}
\tightlist
\item
  \href{https://help.nytimes3xbfgragh.onion/hc/en-us/articles/115014792127-Copyright-notice}{©~2020~The
  New York Times Company}
\end{itemize}

\begin{itemize}
\tightlist
\item
  \href{https://www.nytco.com/}{NYTCo}
\item
  \href{https://help.nytimes3xbfgragh.onion/hc/en-us/articles/115015385887-Contact-Us}{Contact
  Us}
\item
  \href{https://www.nytco.com/careers/}{Work with us}
\item
  \href{https://nytmediakit.com/}{Advertise}
\item
  \href{http://www.tbrandstudio.com/}{T Brand Studio}
\item
  \href{https://www.nytimes3xbfgragh.onion/privacy/cookie-policy\#how-do-i-manage-trackers}{Your
  Ad Choices}
\item
  \href{https://www.nytimes3xbfgragh.onion/privacy}{Privacy}
\item
  \href{https://help.nytimes3xbfgragh.onion/hc/en-us/articles/115014893428-Terms-of-service}{Terms
  of Service}
\item
  \href{https://help.nytimes3xbfgragh.onion/hc/en-us/articles/115014893968-Terms-of-sale}{Terms
  of Sale}
\item
  \href{https://spiderbites.nytimes3xbfgragh.onion}{Site Map}
\item
  \href{https://help.nytimes3xbfgragh.onion/hc/en-us}{Help}
\item
  \href{https://www.nytimes3xbfgragh.onion/subscription?campaignId=37WXW}{Subscriptions}
\end{itemize}
