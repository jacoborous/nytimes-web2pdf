Sections

SEARCH

\protect\hyperlink{site-content}{Skip to
content}\protect\hyperlink{site-index}{Skip to site index}

\href{https://www.nytimes3xbfgragh.onion/section/nyregion}{New York}

\href{https://myaccount.nytimes3xbfgragh.onion/auth/login?response_type=cookie\&client_id=vi}{}

\href{https://www.nytimes3xbfgragh.onion/section/todayspaper}{Today's
Paper}

\href{/section/nyregion}{New York}\textbar{}The ABCs of L.G.B.T.Q.I.A.+

\url{https://nyti.ms/2MICXY5}

\begin{itemize}
\item
\item
\item
\item
\item
\end{itemize}

Advertisement

\protect\hyperlink{after-top}{Continue reading the main story}

Supported by

\protect\hyperlink{after-sponsor}{Continue reading the main story}

\hypertarget{the-abcs-of-lgbtqia}{%
\section{The ABCs of L.G.B.T.Q.I.A.+}\label{the-abcs-of-lgbtqia}}

\includegraphics{https://static01.graylady3jvrrxbe.onion/images/2018/06/24/autossell/24PRIDECOVERWEB/24PRIDECOVERWEB-articleLarge.jpg?quality=75\&auto=webp\&disable=upscale}

By \href{https://www.nytimes3xbfgragh.onion/by/michael-gold}{Michael
Gold}

\begin{itemize}
\item
  June 21, 2018
\item
  \begin{itemize}
  \item
  \item
  \item
  \item
  \item
  \end{itemize}
\end{itemize}

\emph{\textbf{Updated on June 7, 2019}}

When I came out as gay more than 10 years ago, there were only four
letters commonly used to group various sexual and gender minorities: L,
G, B and T.

These letters were an evolution toward inclusion --- an expansion of the
language used to represent a disparate group that had often just been
called ``the gay community.''

Despite their intent, the letters proved to be limiting.

Times and attitudes have changed, and the language used to discuss
sexual orientation and gender identity has also changed. As a result,
the established L.G.B.T. abbreviation has acquired a few extra letters
--- and a cluster of ancillary terminology around both sexuality and
gender. Not everyone has adopted them yet.

Take, for example, the addition of ``Q'' that became increasingly
popular as the 20th century turned into the 21st. Some insisted this
stood for ``questioning,'' representing people who were uncertain of
their sexual orientations or gender identities. Others declared it was
for ``queer,'' a catchall term that has shed its derogatory origins and
is gaining acceptance.

Now there's also I, for intersex; A, for ally (or asexual, depending on
whom you're talking to); and often a plus sign meant to cover anyone
else who's not included: L.G.B.T.Q.I.A.+.

However that's just the beginning. In the year since The New York Times
first published this article in Summer 2018, the language used to
describe the gender and sexuality spectrums has grown, with new terms
becoming more prominent.

As World Pride, the annual celebration of L.G.B.T.Q.I.A.+ identity,
comes to New York City for the month of June, The Times is asking
readers to
\href{https://www.nytimes3xbfgragh.onion/interactive/2019/nyregion/nyc-pride-identity.html}{share
how they identify}. We have updated this list to reflect more common
themes among the responses.

\href{https://www.nytimes3xbfgragh.onion/interactive/2019/nyregion/nyc-pride-identity.html}{}

\includegraphics{https://static01.graylady3jvrrxbe.onion/images/2019/06/10/nyregion/who-am-i-pride-identity-callout-1557170647557/who-am-i-pride-identity-callout-1557170647557-articleLarge-v3.gif}

\hypertarget{tell-us-who-you-are}{%
\subsection{Tell Us Who You Are}\label{tell-us-who-you-are}}

Ahead of World Pride in June, we want to capture the ever-evolving ways
in which we describe ourselves. What labels do you choose for yourself?

What follows is a by-no-means inclusive list of vocabulary.

\textbf{GAY AND LESBIAN} It's important to start with the basics, and
``gay'' and ``lesbian'' are as basic as it gets. As ``homosexual'' began
to feel clinical and pejorative, gay became the de rigueur mainstream
term to refer to same-sex attraction in the late 1960s and early '70s.
Gradually, as what was then called the gay liberation movement gained
steam, the phrase ``gay and lesbian'' became more popular as a way to
highlight the similar-yet-separate issues faced by women in the fight
for tolerance.

Gay is still sometimes used as an umbrella term, but these days, it also
refers specifically to men, as in ``gay men and lesbians.''

\textbf{BISEXUAL} Someone who is attracted to people of their gender or
other gender identities. It is not a way station from straight to gay,
as it had once been described.

The stereotypes around bisexuality --- that it's a transitional stage or
a cover for promiscuity --- have been at the center of fraught
conversation within L.G.B.T.Q. circles
\href{https://www.nytimes3xbfgragh.onion/2005/07/05/health/straight-gay-or-lying-bisexuality-revisited.html}{for
years}. The musical television show ``Crazy Ex-Girlfriend,'' which
features a bisexual male character, had
\href{https://www.youtube.com/watch?v=5e7844P77Is}{an entire song
refuting this}.

As advocates speak out more about what they see as ``bisexual erasure''
--- the persistent questioning or negation of bisexual identity --- the
term has become resurgent. But some people also argue that the prefix
``bi'' reinforces a male/female gender binary that isn't inclusive
enough.

\textbf{PANSEXUAL} Someone who is attracted to people of all gender
identities. Or someone who is attracted to a person's qualities
regardless of their gender identity. (The prefix ``pan'' means ``all,''
rejecting the gender binary that some argue is implied by ``bisexual.'')

Once a more niche term used by academics, pansexual has entered the
mainstream, pushed in part by celebrities bringing it visibility. The
singer Miley Cyrus
\href{https://www.elle.com/uk/life-and-culture/news/a27520/miley-cyrus-interview-october-2015/}{identified
as pansexual in 2015}. In April, after the singer
\href{https://www.rollingstone.com/music/features/cover-story-janelle-monae-prince-new-lp-her-sexuality-w519523}{Janelle
Monàe came out as pansexual} in a ``Rolling Stone'' article, searches
for the word on Merriam-Webster's website rose 11,000 percent, according
to the dictionary.

\textbf{ASEXUAL} Or ``ace.'' Someone who experiences little to no sexual
attraction. They are not to be confused with ``aromantic people,'' who
experience little or no romantic attraction. Asexual people do not
always identify as aromantic; aromantic people do not always identify as
asexual.

More generally, some people (asexual or otherwise) identify as having a
romantic orientation different than their sexual orientation. The
terminology is similar: homoromantic, heteroromantic, biromantic and so
on.

\textbf{DEMISEXUAL} Someone who generally does not experience sexual
attraction unless they have formed a strong emotional, but not
necessarily romantic, connection with someone.

\textbf{GRAYSEXUAL} Someone who occasionally experiences sexual
attraction but usually does not; it covers a kind of gray space between
asexuality and sexual identity.

\textbf{CISGENDER} Someone whose gender identity matches the sex they
were assigned at birth.

\textbf{TRANSGENDER} A wide-ranging term for people whose gender
identity or gender expression differs from the biological sex they were
assigned at birth.

\textbf{TRANSGENDERED} Not a word. Often used as one.

\textbf{TRANS* OR TRANS+} Two umbrella terms for non-cisgender
identities.

\textbf{GENDER NONCONFORMING, OR G.N.C.} One who expresses gender
outside traditional norms associated with masculinity or femininity. Not
all gender-nonconforming people are transgender, and some transgender
people express gender in conventionally masculine or feminine ways.

\textbf{NONBINARY} A person who identifies as neither male nor female
and sees themselves outside the gender binary. This is sometimes
shortened to N.B. or enby. One notable example: Taylor Mason, a
financial analyst on
\href{https://www.nytimes3xbfgragh.onion/column/billions-tv-recaps}{the
show ``Billions,''} who is believed to be the first gender nonbinary
character on television and is played by the nonbinary actor Asia Kate
Dillon.

\textbf{GENDERQUEER} Another term often used to describe someone whose
gender identity is outside the strict male/female binary. They may
exhibit both traditionally masculine and feminine qualities or neither.

\textbf{GENDER FLUID} A term used by people whose identity shifts or
fluctuates. Sometimes these individuals may identify or express
themselves as more masculine on some days, and more feminine on others.

\textbf{GENDER-NEUTRAL} Someone who prefers not to be described by a
specific gender, but prefers ``they'' as a singular pronoun (the
American Dialect Society's
\href{https://www.nytimes3xbfgragh.onion/2016/01/31/fashion/pronoun-confusion-sexual-fluidity.html}{2015
Word of the Year}) or the honorific ``Mx.,'' a substitute for ``Mr.'' or
``Ms.'' that
\href{https://www.nytimes3xbfgragh.onion/2015/06/07/style/me-myself-and-mx.html}{entered
the Oxford English Dictionary in 2015}.

\textbf{M.A.A.B./F.A.A.B./U.A.A.B.} Male-assigned at
birth/female-assigned at birth/unassigned at birth.

\textbf{INTERSEX} A term for someone born with biological sex
characteristics that aren't traditionally associated with male or female
bodies. Intersexuality does not refer to sexual orientation or gender
identity.

\textbf{+} Not just a mathematical symbol anymore, but a denotation of
everything on the gender and sexuality spectrum that letters and words
can't yet describe.

Advertisement

\protect\hyperlink{after-bottom}{Continue reading the main story}

\hypertarget{site-index}{%
\subsection{Site Index}\label{site-index}}

\hypertarget{site-information-navigation}{%
\subsection{Site Information
Navigation}\label{site-information-navigation}}

\begin{itemize}
\tightlist
\item
  \href{https://help.nytimes3xbfgragh.onion/hc/en-us/articles/115014792127-Copyright-notice}{©~2020~The
  New York Times Company}
\end{itemize}

\begin{itemize}
\tightlist
\item
  \href{https://www.nytco.com/}{NYTCo}
\item
  \href{https://help.nytimes3xbfgragh.onion/hc/en-us/articles/115015385887-Contact-Us}{Contact
  Us}
\item
  \href{https://www.nytco.com/careers/}{Work with us}
\item
  \href{https://nytmediakit.com/}{Advertise}
\item
  \href{http://www.tbrandstudio.com/}{T Brand Studio}
\item
  \href{https://www.nytimes3xbfgragh.onion/privacy/cookie-policy\#how-do-i-manage-trackers}{Your
  Ad Choices}
\item
  \href{https://www.nytimes3xbfgragh.onion/privacy}{Privacy}
\item
  \href{https://help.nytimes3xbfgragh.onion/hc/en-us/articles/115014893428-Terms-of-service}{Terms
  of Service}
\item
  \href{https://help.nytimes3xbfgragh.onion/hc/en-us/articles/115014893968-Terms-of-sale}{Terms
  of Sale}
\item
  \href{https://spiderbites.nytimes3xbfgragh.onion}{Site Map}
\item
  \href{https://help.nytimes3xbfgragh.onion/hc/en-us}{Help}
\item
  \href{https://www.nytimes3xbfgragh.onion/subscription?campaignId=37WXW}{Subscriptions}
\end{itemize}
