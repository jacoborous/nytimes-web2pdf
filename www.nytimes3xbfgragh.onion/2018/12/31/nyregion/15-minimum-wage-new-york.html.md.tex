Sections

SEARCH

\protect\hyperlink{site-content}{Skip to
content}\protect\hyperlink{site-index}{Skip to site index}

\href{https://www.nytimes3xbfgragh.onion/section/nyregion}{New York}

\href{https://myaccount.nytimes3xbfgragh.onion/auth/login?response_type=cookie\&client_id=vi}{}

\href{https://www.nytimes3xbfgragh.onion/section/todayspaper}{Today's
Paper}

\href{/section/nyregion}{New York}\textbar{}A \$15 Minimum Wage Seemed
Impossible. Now It's Reality for a Million New Yorkers.

\url{https://nyti.ms/2GQ92hL}

\begin{itemize}
\item
\item
\item
\item
\item
\item
\end{itemize}

Advertisement

\protect\hyperlink{after-top}{Continue reading the main story}

Supported by

\protect\hyperlink{after-sponsor}{Continue reading the main story}

\hypertarget{a-15-minimum-wage-seemed-impossible-now-its-reality-for-a-million-new-yorkers}{%
\section{A \$15 Minimum Wage Seemed Impossible. Now It's Reality for a
Million New
Yorkers.}\label{a-15-minimum-wage-seemed-impossible-now-its-reality-for-a-million-new-yorkers}}

\includegraphics{https://static01.graylady3jvrrxbe.onion/images/2019/01/01/nyregion/01MINWAGE1/merlin_101404822_7a0010b0-ac22-475a-a893-2c200fad3d0e-articleLarge.jpg?quality=75\&auto=webp\&disable=upscale}

By \href{https://www.nytimes3xbfgragh.onion/by/patrick-mcgeehan}{Patrick
McGeehan}

\begin{itemize}
\item
  Dec. 31, 2018
\item
  \begin{itemize}
  \item
  \item
  \item
  \item
  \item
  \item
  \end{itemize}
\end{itemize}

\emph{{[}What you need to know to start the day:}
\href{https://www.nytimes3xbfgragh.onion/newsletters/newyorktoday?module=inline}{\emph{Get
New York Today in your inbox.}}\emph{{]}}

Six years after a group of fast-food workers in New York City ---
earning as little as \$7.25 an hour --- made the seemingly preposterous
demand for a \$15 minimum wage, more than one million of their peers
will get just that starting this week.

On Monday, the lowest legal wage at most companies that employ more than
10 workers rose by \$2, to \$15 an hour. Among those whose pay will
increase are all fast-food workers as well as more than 25,000 workers
at the city's two airports.

The increase is the latest step in a gradual rise in the minimum wage
that labor unions campaigned for and that Gov. Andrew M. Cuomo
eventually endorsed. New York City joins several other cities on the
West Coast where minimum wages have already hit \$15, including San
Francisco and Seattle.

California's minimum, which rises to \$12 an hour for larger employers
on Jan. 1, is
\href{https://www.dir.ca.gov/dlse/faq_minimumwage.htm}{scheduled to
rise} to \$15 over the next few years.

``The `Fight for 15' has gone from a rallying cry to facts on the ground
in just a few short years,'' said Paul Sonn, the state policy program
director with the National Employment Law Project, which advocates for
low-wage workers. ``This demand was from the fast-food workers who
explained that was the minimum they needed for a decent life.''

The movement is no longer confined to high-cost cities on the coasts.
Mr. Sonn cited recent decisions by national retailers like Target and
\href{https://www.nytimes3xbfgragh.onion/2018/10/02/business/amazon-minimum-wage.html}{Amazon
to raise their wages to at least \$15} an hour.

But the country is now a patchwork quilt of wages, with some states
still operating under the federal minimum of \$7.25 an hour. Outside New
York City, Connecticut's minimum wage is \$10.10, while in New Jersey it
is \$8.85, though the state's Democratic governor and the Legislature,
which is also controlled by Democrats, want to raise it to \$15.

For hourly workers, the increase can make a big difference. Rosa Rivera
earned just \$5.15 an hour and relied on government assistance to pay
her rent when she started working at a McDonald's in Manhattan 18 years
ago. Now 53 and a veteran of several rallies for better wages, Ms.
Rivera's eyes teared up as she spoke about attaining one of the main
goals the workers had set.

``When I get my first check with \$15, I'm going to be so happy,'' said
Ms. Rivera, an immigrant from El Salvador with three children. She said
she was proud to pay her rent and help support her grandchildren without
federal benefits.

The ``Fight for 15'' campaign began in 2012, when the minimum wage in
New York State was \$7.25. Fast-food wages had barely budged for many
years, leaving many workers unable to to feed their families and pay
bills.

The campaign grew out of meetings community organizers convened to
discuss the lack of affordable housing in New York, recalled Jonathan
Westin, the executive director of New York Communities for Change.

Talk quickly turned to the poor pay of so many fast-food and retail
jobs, Mr. Westin said.

``They were such temporary jobs and they paid so little,'' Mr. Westin
said. ``They felt like they really didn't have anything to lose.''

Organizers thought \$10 an hour was a reasonable demand, Mr. Westin
said, but workers were unimpressed. Their attitude, he said, was that
``\$10 is not going to get me anything. We need to go much higher.''

So in late 2012,
\href{https://www.nytimes3xbfgragh.onion/2012/11/30/nyregion/fast-food-workers-in-new-york-city-rally-for-higher-wages.html}{workers
at several restaurants in the city walked off the job} and staged the
first rallies for a \$15 minimum wage, with the support of the Service
Employees International Union and some local elected officials. Some of
their bosses tolerated the disobedience; others did not.

\includegraphics{https://static01.graylady3jvrrxbe.onion/images/2018/12/31/nyregion/00MINWAGE2/merlin_146414757_ba791d82-23c0-46d7-b079-000c47f00315-articleLarge.jpg?quality=75\&auto=webp\&disable=upscale}

In one of the more contentious confrontations, City Councilman Jumaane
Williams of Brooklyn intervened to save the job of Shalonda Montgomery,
who worked at a Wendy's in Downtown Brooklyn. After she joined a
walkout, her manager fired her. Mr. Williams said he ``basically staged
an impromptu boycott'' of the restaurant: ``I started telling people why
they shouldn't shop there.''

Ms. Montgomery soon had her job back. ``This is another one of those
great examples of an organized, grass-roots campaign being able to
change people's lives,'' said Mr. Williams, who is running for city
public advocate. He admitted that even he did not believe a \$15 minimum
wage was realistic. ``I actually thought that, O.K., maybe we'll get
\$12,'' he said.

In early 2015, Mr. Cuomo, a Democrat, called for a raise of the hourly
minimum wage to \$11.50 in New York City and \$10.50 for the rest of the
state.

Mayor Bill de Blasio had an even bolder proposal: raise the city minimum
to \$13 that year and to \$15 by 2019. Mr. Cuomo declared the idea a
``non-starter'' with lawmakers in Albany.

But pressured by labor leaders like Hector Figueroa, president of 32BJ
Service Employees International Union, Mr. Cuomo, who is often mentioned
as a possible presidential candidate in 2020, changed his tone. In
mid-2015, he convened a panel known as a wage board to consider raising
the lowest wages of fast-food workers above the state minimum set by
lawmakers in Albany.

After several hearings, the wage board put fast-food workers on a track
toward a \$15 minimum wage. Within a year, Mr. Cuomo had negotiated a
budget that included a series of steps to raise the minimum wage at
different paces for workers in various parts of the state.

That schedule may eventually bring the minimum to \$15 an hour
statewide. At smaller employers in the city, the minimum increased to
\$13.50 from \$12, and will rise again to \$15 on Dec. 31, 2019. On Long
Island and in Westchester County, the minimum rose to \$13 from \$12,
and will rise in two more increments to get to \$15 in 2021.

By then, several other cities across the country will have minimum wages
at or near \$15, though the effort has met opposition. In 2017, Kansas
City residents voted to raise the city's minimum wage to \$15 an hour.
But the state legislature blocked the move, later approving a gradual
increase of the state's hourly minimum wage to \$12 by 2023, starting
with an increase this year to \$8.60 from \$7.85.

Mary Kay Henry, president of the Service Employees International Union,
which adopted the Fight for 15 campaign across the country, said there
would be more resistance like that encountered in Missouri. But she
added: ``We are bound and determined not to stop until all workers get
on a path to \$15 and also have a seat at the table and a voice in their
jobs.''

Advertisement

\protect\hyperlink{after-bottom}{Continue reading the main story}

\hypertarget{site-index}{%
\subsection{Site Index}\label{site-index}}

\hypertarget{site-information-navigation}{%
\subsection{Site Information
Navigation}\label{site-information-navigation}}

\begin{itemize}
\tightlist
\item
  \href{https://help.nytimes3xbfgragh.onion/hc/en-us/articles/115014792127-Copyright-notice}{©~2020~The
  New York Times Company}
\end{itemize}

\begin{itemize}
\tightlist
\item
  \href{https://www.nytco.com/}{NYTCo}
\item
  \href{https://help.nytimes3xbfgragh.onion/hc/en-us/articles/115015385887-Contact-Us}{Contact
  Us}
\item
  \href{https://www.nytco.com/careers/}{Work with us}
\item
  \href{https://nytmediakit.com/}{Advertise}
\item
  \href{http://www.tbrandstudio.com/}{T Brand Studio}
\item
  \href{https://www.nytimes3xbfgragh.onion/privacy/cookie-policy\#how-do-i-manage-trackers}{Your
  Ad Choices}
\item
  \href{https://www.nytimes3xbfgragh.onion/privacy}{Privacy}
\item
  \href{https://help.nytimes3xbfgragh.onion/hc/en-us/articles/115014893428-Terms-of-service}{Terms
  of Service}
\item
  \href{https://help.nytimes3xbfgragh.onion/hc/en-us/articles/115014893968-Terms-of-sale}{Terms
  of Sale}
\item
  \href{https://spiderbites.nytimes3xbfgragh.onion}{Site Map}
\item
  \href{https://help.nytimes3xbfgragh.onion/hc/en-us}{Help}
\item
  \href{https://www.nytimes3xbfgragh.onion/subscription?campaignId=37WXW}{Subscriptions}
\end{itemize}
