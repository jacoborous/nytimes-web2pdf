Sections

SEARCH

\protect\hyperlink{site-content}{Skip to
content}\protect\hyperlink{site-index}{Skip to site index}

\href{https://www.nytimes3xbfgragh.onion/section/books/review}{Book
Review}

\href{https://myaccount.nytimes3xbfgragh.onion/auth/login?response_type=cookie\&client_id=vi}{}

\href{https://www.nytimes3xbfgragh.onion/section/todayspaper}{Today's
Paper}

\href{/section/books/review}{Book Review}\textbar{}The American Past: A
History of Contradictions

\url{https://nyti.ms/2NaJMFH}

\begin{itemize}
\item
\item
\item
\item
\item
\end{itemize}

Advertisement

\protect\hyperlink{after-top}{Continue reading the main story}

Supported by

\protect\hyperlink{after-sponsor}{Continue reading the main story}

Nonfiction

\hypertarget{the-american-past-a-history-of-contradictions}{%
\section{The American Past: A History of
Contradictions}\label{the-american-past-a-history-of-contradictions}}

\includegraphics{https://static01.graylady3jvrrxbe.onion/images/2018/09/16/books/review/16Sullivan/merlin_143300931_8c9e5d00-a114-45a6-90a9-079be0c4800c-articleLarge.jpg?quality=75\&auto=webp\&disable=upscale}

Buy Book ▾

\begin{itemize}
\tightlist
\item
  \href{https://www.amazon.com/gp/search?index=books\&tag=NYTBSREV-20\&field-keywords=These+Truths\%3A+A+History+of+the+United+States+Jill+Lepore}{Amazon}
\item
  \href{https://du-gae-books-dot-nyt-du-prd.appspot.com/buy?title=These+Truths\%3A+A+History+of+the+United+States\&author=Jill+Lepore}{Apple
  Books}
\item
  \href{https://www.anrdoezrs.net/click-7990613-11819508?url=https\%3A\%2F\%2Fwww.barnesandnoble.com\%2Fw\%2F\%3Fean\%3D9780393635249}{Barnes
  and Noble}
\item
  \href{https://www.anrdoezrs.net/click-7990613-35140?url=https\%3A\%2F\%2Fwww.booksamillion.com\%2Fp\%2FThese\%2BTruths\%253A\%2BA\%2BHistory\%2Bof\%2Bthe\%2BUnited\%2BStates\%2FJill\%2BLepore\%2F9780393635249}{Books-A-Million}
\item
  \href{https://bookshop.org/a/3546/9780393635249}{Bookshop}
\item
  \href{https://www.indiebound.org/book/9780393635249?aff=NYT}{Indiebound}
\end{itemize}

When you purchase an independently reviewed book through our site, we
earn an affiliate commission.

By Andrew Sullivan

\begin{itemize}
\item
  Sept. 14, 2018
\item
  \begin{itemize}
  \item
  \item
  \item
  \item
  \item
  \end{itemize}
\end{itemize}

\textbf{THESE TRUTHS}\\
\textbf{A History of the United States}\\
By Jill Lepore\\
Illustrated. 932 pp. W.W. Norton \& Company. \$39.95.

It isn't until you start reading it that you realize how much we need a
book like this one at this particular moment. ``These Truths,'' by
\href{https://www.nytimes3xbfgragh.onion/2018/09/16/books/jill-lepore-on-the-history-of-america-in-1000-pages-or-less.html}{Jill
Lepore} --- a professor at Harvard and a staff writer at The New Yorker
--- is a one-volume history of the United States, constructed around a
traditional narrative, that takes you from the 16th to the 21st century.
It tries to take in almost everything, an impossible task, but I'd be
hard-pressed to think she could have crammed more into these 932 highly
readable pages. It covers the history of political thought, the fabric
of American social life over the centuries, classic ``great man''
accounts of contingencies, surprises, decisions, ironies and character,
and the vivid experiences of those previously marginalized: women,
African-Americans, Native Americans, homosexuals. It encompasses
interesting takes on democracy and technology, shifts in demographics,
revolutions in economics and the very nature of modernity. It's a big
sweeping book, a way for us to take stock at this point in the journey,
to look back, to remind us who we are and to point to where we're
headed.

This is not an account of relentless progress. It's much subtler and
darker than that. It reminds us of some simple facts so much in the
foreground that we must revisit them: ``Between 1500 and 1800, roughly
two and a half million Europeans moved to the Americas; they carried 12
million Africans there by force; and as many as 50 million Native
Americans died, chiefly of disease. \ldots{} Taking possession of the
Americas gave Europeans a surplus of land; it ended famine and led to
four centuries of economic growth.'' Nothing like this had ever happened
in world history; and nothing like it is possible again. The land was
instantly a refuge for religious dissenters, a new adventure in what we
now understand as liberalism and a brutal exercise in slave labor and
tyranny. It was a vast, exhilarating frontier and a giant, torturing
gulag at the same time. Over the centuries, in Lepore's insightful
telling, it represented a giant leap in productivity for humankind:
``Slavery was one kind of experiment, designed to save the cost of labor
by turning human beings into machines. Another kind of experiment was
the invention of machines powered by steam.'' It was an experiment in
the pursuit of happiness, but it was in effect the pursuit of previously
unimaginable affluence.

And, of course, it was and is full of contradictions: A radically new
secularism founded it, and a political-religious fervor came to define
it. As industrialization accelerated, and modernity beckoned, Americans
turned back to God: Before the start of
\href{https://www.u-s-history.com/pages/h1091.html}{the}
\href{https://www.u-s-history.com/pages/h1091.html}{Second Great
Awakening}, at the end of the 18th century, ``a scant one in 10
Americans were church members; by the time it ended, that ratio had
risen to eight in 10.'' And these religious waves advanced the cause of
the spiritual equality of all human beings, which in turn became
political equality. ``The self-evident, secular truths of the
Declaration of Independence became, to evangelical Americans, the truths
of revealed religion'' is Lepore's insight. And argument raged from the
get-go: constant, careening, apocalyptic and at times elevated discourse
about real things, vital things, in primary colors, and with passion.
All these crosscurrents --- reason and faith, truth and propaganda,
black and white, slave and free, immigrant and native, industry and
agriculture --- ripple through this history, with one obvious period
where the country simply came apart in the bloodiest civil conflict in
history.

No country before or since has been this convulsed with conflict and
wealth. No country has been both a republic and effectively an empire
across an entire continent. No country had ever been defined as one of
strangers and travelers, where waves and waves of immigration constantly
churned through society, in what one reformer in 1837 called ``the
boldest experiment upon the stability of government ever made in the
annals of time.'' No people were as passionate both for slavery and for
freedom. The Civil War, in fact, revealed that there were effectively
two countries fighting for supremacy on one continent. The Southern
states showed themselves to be profoundly hostile to democracy and civil
equality, as any system based on white supremacy has to be.
Secessionists, Lepore brutally demonstrates, ``were attempting to build
a modern, pro-slavery, antidemocratic state.'' This meant suppression of
dissent and extirpation of free speech: ``One of the first things the
new state of Georgia did was to pass a law that made dissent'' against
secession ``punishable by death.'' The other country was built on the
First Amendment.

The war itself beggars belief. In one single battle, 24,000 men were
casualties. More than 750,000 Americans died over all, from wounds and
disease. Even today, that number numbs. And yet this cathartic
breakthrough for freedom nonetheless came to be alloyed. Lincoln was
murdered by a white supremacist. Reconstruction --- a surreal and
glorious period when Confederate veterans were barred from voting and
freed slaves exercised real power in the South --- was abandoned in a
petty political deal over a presidential ticket. Jim Crow must count as
the most bitter, resentful and wicked response to defeat by the losing
side in any civil war. It suggested, indeed, that the Civil War would
never end, merely wax and wane. And its toll on the human spirit and the
black body was matched only by its evil. From Jackson's massacre of
Native Americans to the Southern labor camps to the full embrace of
torture in the Bush-Cheney administration is a single, consistent and
evil line.

Image

Lepore's most distinctive theme she refers to as ``the machine'': a
concern that newspapers, and then mass media, especially radio and
television --- in combination with pollsters and political consultants
--- progressively undermined any concept of empirical truth, and thereby
slowly sank the reasoned deliberation essential to republican
government. She seems obsessed with the malignancy of polling; it takes
up more pages than, say, the war on drugs. And she's not wrong about the
cynicism of media and political pros. But dirty campaigning, distortion
of reality and propaganda were there from the very beginning, as indeed
she notes. The Lincoln-Douglas debates were, in some ways, the peak of
political discourse in this country, but they nonetheless were resolved
by mass bloodshed. And the collapse of a common truth in the late 20th
century was as much a function of modernity and post-modernity as of
political malfeasance.

Is our current spasm of authoritarianism unprecedented? Hardly. It was
there in Andrew Jackson's contempt for the Supreme Court; in Lincoln's
suspension of habeas corpus; in Franklin Roosevelt's effective
blackmailing of the Supreme Court to back the New Deal; in the
internment camps for Japanese-Americans; in the crimes of Richard Nixon;
and in the claims of total executive power under Bush-Cheney. Lepore
cites Mencken's spoof Constitution for Roosevelt: ``All governmental
power of whatever sort shall be vested in a president of the United
States.'' Similarly, she exposes Walter Lippmann's advice to the
president: ``The situation is critical. You may have no alternative but
to assume dictatorial powers.''

The same can be said about the rise of white nationalism in the wake of
mass immigration. The last time the foreign-born as a percentage of the
population rivaled ours today, a brutally draconian immigration law was
imposed, with specific racial categories for exclusion, and the Klan
turned not just against blacks but against Catholics and Jews as well.
Ditto the consistency of political extremism: from John Brown to Malcolm
X to Black Lives Matter. Ditto huge economic inequality --- in the 1920s
and 2010s. Rhetorical excess? ``We see dangerous signs of Hitlerism in
the Goldwater campaign,'' opined one Martin Luther King Jr. Social
breakdown? It would be hard to match the late 1960s, when the
achievement of civil rights was followed by an explosion of mass
violence, beginning in Watts, Los Angeles, in 1965, and the 1970s, when
domestic terrorism was everywhere.

Lepore panders a little to liberal sensibilities. And so in her account,
Communism was no real threat at all; Nixon was simply playing the
demagogue in going after Alger Hiss (she doesn't note that Hiss was
indeed a Soviet spy and a traitor). Ronald Reagan gets no credit for the
implosion of the Soviet Union. Clinton's crime bill was a terrible
failure because of mass incarceration, and yet the extraordinary decline
in crime that followed does not earn a mention. But she is withering
about the New Left, and liberalism's turn toward elitism and identity
politics. And she highlights truths that are usually dim-lit: that the
first attempt at a welfare state came in the South, where women secured
a war widow's pension; that the conservative movement was made possible
by women, especially Phyllis Schlafly; that the gay rights movement only
succeeded when it took a conservative turn. She sees John F. Kennedy,
rightly, as a conservative Democrat. She admires in many ways how the
right seized populism as the left abandoned it. This is not an account
conservatives will hate.

She's brilliant at times. She devastates the current maximalist position
of the National Rifle Association (which the N.R.A. itself once strongly
opposed) in the context of gun ownership and the historical debate about
the Second Amendment.
\href{https://www.supremecourt.gov/opinions/07pdf/07-290.pdf}{The 2008
Heller decision} rejecting a District of Columbia handgun ban is quite
obviously bonkers. Similarly, the emergence of abortion as the critical
litmus test for both parties is an entirely novel and polarizing
development: ``Either abortion was murder and guns meant freedom or guns
meant murder and abortion was freedom.'' It is as if complexity has
become a sin. She sees both sides in recent times as corrosive of
liberal norms: ``Both the left and the right, unwilling to brook
dissent, began dismantling structures that nurture fair-minded debate:
the left undermining the university; the right undermining the press.''
Perfect. She notes how recent presidential candidates have declared vast
swaths of the public as ``unworthy of their attention'' (Romney's 47
percent of ``takers'') or beneath their contempt (Hillary's
``deplorables''). They both deserved to lose. And she sees the
deregulation of the airwaves (the end of the Fairness Doctrine under
Reagan) and of Wall Street (under Clinton) as key reasons our politics
is now so nihilist and unequal.

Lepore is also a writer. This book is aimed at a mass audience, driven
by anecdote and statistic, memoir and photograph, with all the giants of
American history in their respective places. There wasn't a moment when
I struggled to keep reading. We know that Washington ordered his slaves
freed once his wife died; I didn't know that in the room where he died,
there were more black people than white. I've always admired Benjamin
Franklin, but he is a glittering star in this account: ``He was the only
man to have signed the Declaration of Independence, the Treaty of Paris
and the Constitution. His last public act was to urge abolition.
Congress would not hear of it.'' There are moments, however, when you
wince at the purple prose. ``The Republic was spreading like ferns on
the floor of a forest.'' Dred Scott was ``suffering from tuberculosis, a
slow sickness, a constitutional weakening, as relentless as the disease
that wracked the nation itself. Frederick Douglass watched, and looked
for a cure, an end to suffering. \ldots{} But it was as if the nation,
like Oedipus of Thebes, had seen that in its origins lay a curse, and
had gouged out its own eyes.'' Oof. The last two paragraphs of the book
amount to one of the most excruciating extended metaphors --- yes, the
ship of state! --- I have ever had the misfortune to struggle through.

But these are quibbles. We need this book. Its reach is long, its
narrative fresh and the arc of its account sobering to say the least.
This is not Whig history. It is a classic tale of a unique country's
astonishing rise and just-as-inevitable fall. And if you reread the book
and ask yourself, what is the period of American history that most
resembles today?, you would have to say, I think, the late 1850s and
early 1860s. Here's Lepore's description of that time: ``A sense of
inevitability fell, as if there were a fate, a dismal dismantlement,
that no series of events or accidents could thwart.'' Lincoln thought of
the nation as a house, and quoted Scripture: ``A house divided against
itself cannot stand.'' And his words, as always, cut through the ages
like a knife.

Advertisement

\protect\hyperlink{after-bottom}{Continue reading the main story}

\hypertarget{site-index}{%
\subsection{Site Index}\label{site-index}}

\hypertarget{site-information-navigation}{%
\subsection{Site Information
Navigation}\label{site-information-navigation}}

\begin{itemize}
\tightlist
\item
  \href{https://help.nytimes3xbfgragh.onion/hc/en-us/articles/115014792127-Copyright-notice}{©~2020~The
  New York Times Company}
\end{itemize}

\begin{itemize}
\tightlist
\item
  \href{https://www.nytco.com/}{NYTCo}
\item
  \href{https://help.nytimes3xbfgragh.onion/hc/en-us/articles/115015385887-Contact-Us}{Contact
  Us}
\item
  \href{https://www.nytco.com/careers/}{Work with us}
\item
  \href{https://nytmediakit.com/}{Advertise}
\item
  \href{http://www.tbrandstudio.com/}{T Brand Studio}
\item
  \href{https://www.nytimes3xbfgragh.onion/privacy/cookie-policy\#how-do-i-manage-trackers}{Your
  Ad Choices}
\item
  \href{https://www.nytimes3xbfgragh.onion/privacy}{Privacy}
\item
  \href{https://help.nytimes3xbfgragh.onion/hc/en-us/articles/115014893428-Terms-of-service}{Terms
  of Service}
\item
  \href{https://help.nytimes3xbfgragh.onion/hc/en-us/articles/115014893968-Terms-of-sale}{Terms
  of Sale}
\item
  \href{https://spiderbites.nytimes3xbfgragh.onion}{Site Map}
\item
  \href{https://help.nytimes3xbfgragh.onion/hc/en-us}{Help}
\item
  \href{https://www.nytimes3xbfgragh.onion/subscription?campaignId=37WXW}{Subscriptions}
\end{itemize}
