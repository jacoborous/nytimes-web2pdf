Sections

SEARCH

\protect\hyperlink{site-content}{Skip to
content}\protect\hyperlink{site-index}{Skip to site index}

\href{https://www.nytimes3xbfgragh.onion/section/fashion}{Fashion}

\href{https://myaccount.nytimes3xbfgragh.onion/auth/login?response_type=cookie\&client_id=vi}{}

\href{https://www.nytimes3xbfgragh.onion/section/todayspaper}{Today's
Paper}

\href{/section/fashion}{Fashion}\textbar{}Brunello Cucinelli, Prince of
All He Surveys

\url{https://nyti.ms/2Nu7dd9}

\begin{itemize}
\item
\item
\item
\item
\item
\end{itemize}

Advertisement

\protect\hyperlink{after-top}{Continue reading the main story}

Supported by

\protect\hyperlink{after-sponsor}{Continue reading the main story}

Critic's notebook

\hypertarget{brunello-cucinelli-prince-of-all-he-surveys}{%
\section{Brunello Cucinelli, Prince of All He
Surveys}\label{brunello-cucinelli-prince-of-all-he-surveys}}

At 65, the Italian designer looms large in the wardrobes of well-dressed
C.E.O.s as well as the hilltop town where he grew up.

By \href{http://www.nytimes3xbfgragh.onion/by/guy-trebay}{Guy Trebay}

\begin{itemize}
\item
  Sept. 21, 2018
\item
  \begin{itemize}
  \item
  \item
  \item
  \item
  \item
  \end{itemize}
\end{itemize}

\includegraphics{https://static01.graylady3jvrrxbe.onion/images/2018/09/27/fashion/22cucinelli-brunello/22cucinelli-brunello-articleLarge.jpg?quality=75\&auto=webp\&disable=upscale}

For Ralph Lauren, it was a wide necktie and an outsized vision that set
in motion
\href{https://www.nytimes3xbfgragh.onion/2018/09/08/fashion/ralph-lauren-50th-anniversary-show.html}{a
five-decade career in fashion}, one that would eventually render a house
painter's son from the Bronx a multibillionaire and a globally legible
brand. For Brunello Cucinelli, a farmer's son from a rural backwater in
Umbria, the path to international success and a great fortune began
almost as improbably --- with a single sweater.

Nowadays, it is too little appreciated how out of the devastation of
World War II came a great reconstruction in Italy, one largely fueled by
funds from the Marshall Plan. The small-scale industrial capacity that
remains a linchpin of the ``Made in Italy'' brand substantially
originated in the late 1940s, with individual regions developing
specific manufacturing specialties: woolen mills in the north,
shoemaking in the Marche region and factories producing specialty
knitwear in the central area where Mr. Cucinelli got his start.

Image

A view from Solomeo. Most of the town's residents are Cucinelli
employees.Credit...Nadia Shira Cohen for The New York Times

Image

The church bell tower in Solomeo. Much of the church was restored by Mr.
Cucinelli.Credit...Nadia Shira Cohen for The New York Times

It was 40 years ago that a 25-year-old Mr. Cucinelli scraped together a
grubstake to pursue his notion of updating the high-quality, although
staid, knitwear made in factories around his hometown. ``I decided to
produce women's cashmere knitwear in pop colors and not the boring grays
and tans that used to happen at that time,'' Mr. Cucinelli said recently
by phone from Solomeo, near Perugia, where the designer still lives full
time --- although in a manner far removed from that of his childhood,
when the family home lacked plumbing and electricity.

In the tiny hilltop town where he first set up shop in 1978 --- which
now includes the sprawling and manicured corporate campus out of which
come impeccable garments some refer to as Gap wear for the One Percent
--- everything Mr. Cucinelli surveys from a frescoed office in a
restored medieval tower belongs to him.

Image

A medieval festival in Perugia, not far from Solomeo.Credit...Nadia
Shira Cohen for The New York Times

Image

The Morlacchi Theater in Perugia. Mr. Cucinelli is supporting the
restoration of part of the structure.Credit...Nadia Shira Cohen for The
New York Times

Like the painter's son from the Bronx, the farmer's kid from Umbria
lives in a lordly manner largely out of reach for Italy's remnant
aristocrats. With a publicly traded company whose market value is in the
billions and two honorary knighthoods awarded by the Italian government,
Mr. Cucinelli has made himself a prince of Solomeo.

To celebrate that --- and, not coincidentally, his 65th birthday ---
early this month, the designer flew in scores of journalists from around
the world (among them the editors of Esquire, Departures, Town \&
Country and GQ) for days of sightseeing, Lucullan feasting and speeches
by the self-styled humanist, who has a penchant for citing Homer,
Hadrian and Cicero.

Image

Mr. Cucinelli also funded the restoration of the Etruscan Arch in
Perugia.Credit...Nadia Shira Cohen for The New York Times

Lost, perhaps, amid all the jollity was the story of how a single
cashmere sweater was eventually transformed into a lifestyle brand
presciently pitched at a population whose growth could hardly have been
predicted when Mr. Cucinelli started out. Not only are there more
billionaires now than at any time in history, according to
\href{https://www.forbes.com/billionaires/\#424da2af251c}{a 2018 Forbes
ranking} of those who have managed to summit capitalism's Everest, but
they are richer than ever before, with an average wealth of \$4.1
billion. Almost from the start of his career, Mr. Cucinelli decided they
would be his clientele.

``He became the uniform for the cool C.E.O.s of the world,'' said the
designer Michael Bastian, who, as men's wear director of Bergdorf
Goodman in the early years of the century, was instrumental in helping
Mr. Cucinelli shape his overall vision and his retail offering. ``That
kind of guy --- Barry Diller, David Geffen --- guys cool enough that
they didn't have to wear a suit to work,'' he added.

There is slightly more at work than cool, however. Mr. Geffen, Mr.
Diller and their fellow moguls have deep enough pockets that they are
unlikely to experience the sticker shock that would afflict an ordinary
mortal when faced with the price tag on, say, the reversible baseball
bomber sweater by Brunello Cucinelli offered at Bergdorf Goodman for
\$3,095.

Image

At the Cucinelli factory in Solomeo, workers pay three euros for lunch,
which often includes vegetables grown onsite and Mr. Cucinelli's own
olive oil, which is not sold to the public.Credit...Nadia Shira Cohen
for The New York Times

Image

A Cucinelli employee uses a special machine to assemble a
sweater.Credit...Nadia Shira Cohen for The New York Times

``Brunello's always had everybody but the ultracool, low-key C.E.O.s
like Steve Jobs,'' Mr. Bastian added. In fact, Mr. Jobs's distinctive
uniform of anonymous-looking cashmere mock turtlenecks was quietly
produced in bulk for the Apple founder by Mr. Cucinelli --- despite the
latter's distaste for both that particular style of collar and his
hatred of black.

In certain ways Mr. Cucinelli's Italian success is also an American one,
since not only is North America his largest market, it was retailers in
the United States who first urged the designer to expand beyond sweaters
into the broader apparel offerings for which he eventually became known.
Had Gene Pressman, the visionary merchant who once headed Barneys New
York, not pushed him to create a full collection, Mr. Cucinelli said, he
might happily have rung changes on the same cashmere sweater for the
rest of his life.

Image

Brunello Cucinelli styles on display at the company
headquarters.Credit...Nadia Shira Cohen for The New York Times

``It was the Americans that started asking for the total look,'' the
designer said, referring to a style largely inspired by how he himself
had dressed practically since, as a fashion-besotted teenager, he began
honing his taste for elevated sportswear. That included rolling his
blazer sleeves, tailoring the legs of his trousers to narrower
proportions and migrating toward a muted palette from which he has
seldom strayed.

``Sometimes when I would visit Solomeo, I would see the designers
talking color,'' Mr. Bastian said. ``And they had to be perfect,
perfect, perfect: Is it warm beige or cool beige? What exact tone of
navy or gray?''

The one shade no one will ever spot in a Cucinelli collection, Mr.
Bastian added, is green: ``He hates green. He wants to make green gone.
Apparently, his mom forced him to wear a green sweater when he was a kid
and he buried it in the backyard.''

In Mr. Cucinelli's idiosyncratic and yet disciplined approach to
business, analysts see a model for other major carriage trade labels
that appear to have lost their way. ``His vision has always been
unwavering and unapologetically for the moneyed,'' said Robert Burke of
Robert Burke Associates, a New York luxury goods consultancy.

``He is not going after the latest, greatest influencers and bloggers,
and I don't think he has any particular high regard for an aspirational
customer,'' Mr. Burke added. ``He's definitely for the wheels-up
crowd.''

Advertisement

\protect\hyperlink{after-bottom}{Continue reading the main story}

\hypertarget{site-index}{%
\subsection{Site Index}\label{site-index}}

\hypertarget{site-information-navigation}{%
\subsection{Site Information
Navigation}\label{site-information-navigation}}

\begin{itemize}
\tightlist
\item
  \href{https://help.nytimes3xbfgragh.onion/hc/en-us/articles/115014792127-Copyright-notice}{©~2020~The
  New York Times Company}
\end{itemize}

\begin{itemize}
\tightlist
\item
  \href{https://www.nytco.com/}{NYTCo}
\item
  \href{https://help.nytimes3xbfgragh.onion/hc/en-us/articles/115015385887-Contact-Us}{Contact
  Us}
\item
  \href{https://www.nytco.com/careers/}{Work with us}
\item
  \href{https://nytmediakit.com/}{Advertise}
\item
  \href{http://www.tbrandstudio.com/}{T Brand Studio}
\item
  \href{https://www.nytimes3xbfgragh.onion/privacy/cookie-policy\#how-do-i-manage-trackers}{Your
  Ad Choices}
\item
  \href{https://www.nytimes3xbfgragh.onion/privacy}{Privacy}
\item
  \href{https://help.nytimes3xbfgragh.onion/hc/en-us/articles/115014893428-Terms-of-service}{Terms
  of Service}
\item
  \href{https://help.nytimes3xbfgragh.onion/hc/en-us/articles/115014893968-Terms-of-sale}{Terms
  of Sale}
\item
  \href{https://spiderbites.nytimes3xbfgragh.onion}{Site Map}
\item
  \href{https://help.nytimes3xbfgragh.onion/hc/en-us}{Help}
\item
  \href{https://www.nytimes3xbfgragh.onion/subscription?campaignId=37WXW}{Subscriptions}
\end{itemize}
