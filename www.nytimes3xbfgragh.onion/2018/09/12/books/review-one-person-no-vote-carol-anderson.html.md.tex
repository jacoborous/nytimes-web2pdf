Sections

SEARCH

\protect\hyperlink{site-content}{Skip to
content}\protect\hyperlink{site-index}{Skip to site index}

\href{https://www.nytimes3xbfgragh.onion/section/books}{Books}

\href{https://myaccount.nytimes3xbfgragh.onion/auth/login?response_type=cookie\&client_id=vi}{}

\href{https://www.nytimes3xbfgragh.onion/section/todayspaper}{Today's
Paper}

\href{/section/books}{Books}\textbar{}Whose Votes Really Count?

\url{https://nyti.ms/2N6k05n}

\begin{itemize}
\item
\item
\item
\item
\item
\item
\end{itemize}

Advertisement

\protect\hyperlink{after-top}{Continue reading the main story}

Supported by

\protect\hyperlink{after-sponsor}{Continue reading the main story}

\href{/column/books-of-the-times}{Books of The Times}

\hypertarget{whose-votes-really-count}{%
\section{Whose Votes Really Count?}\label{whose-votes-really-count}}

By \href{https://www.nytimes3xbfgragh.onion/by/jennifer-szalai}{Jennifer
Szalai}

\begin{itemize}
\item
  Sept. 12, 2018
\item
  \begin{itemize}
  \item
  \item
  \item
  \item
  \item
  \item
  \end{itemize}
\end{itemize}

\includegraphics{https://static01.graylady3jvrrxbe.onion/images/2018/09/13/arts/13bookanderson1/11bookanderson1-articleLarge.jpg?quality=75\&auto=webp\&disable=upscale}

As the elaborate post-mortem of the presidential election drags on ---
amid all the diagnoses of Russian interference, Clintonian blunders and
white-working-class resentment --- Carol Anderson wants to direct your
attention to one simple fact: In November 2016, black voter turnout fell
by 7 percent.

Anderson rebukes anyone who takes this as a facile statement of how
black voters felt about Hillary Clinton. ``Minority voters did not just
refuse to show up,'' she writes in ``One Person, No Vote: How Voter
Suppression Is Destroying Our Democracy.'' ``Republican legislatures and
governors systematically blocked African-Americas, Hispanics and
Asian-Americans from the polls.''

The 2016 presidential election was the first in 50 years to be held
without the full protections of the Voting Rights Act. Three years
before, in 2013,
\href{http://archive.nytimes3xbfgragh.onion/www.nytimes3xbfgragh.onion/interactive/2013/06/25/us/annotated-supreme-court-decision-on-voting-rights-act.html}{the
Supreme Court had revoked} the part of the law that required states with
a history of voting discrimination to get federal approval in order to
change their voting statutes. The ruling effectively left voters at the
mercy of state legislators. Some of these law makers, like the
Republican Party itself, didn't draw significant support from minority
communities. So in the interests of self-preservation, Anderson says,
they changed the laws in ways that made it harder for minorities to
vote. President Donald J. Trump is just one result; a profound and
polarizing distortion of American democracy is another.

Anderson's argument isn't new. Ari Berman, the author of the excellent
\href{http://https//www.nytimes3xbfgragh.onion/2015/08/30/books/review/give-us-the-ballot-by-ari-berman.html}{``Give
Us the Ballot''} (2015), has been making a similar case to anyone who
will listen. His book also included original reporting, and interviews
with lawyers, activists and government officials who experienced the
implementation of the 1965 Voting Rights Act firsthand.

Anderson, a professor of African-American studies at Emory University,
has written a slender volume that is one part historical primer and one
part spirited manifesto, and is clearly timed for the midterms. (Don't
let the number of pages fool you; more than 100 of them are for
endnotes. Anderson has a distracting tendency to quote even basic
factual phrases --- like ``the State's federal public housing
residents'' --- instead of simply stating or rewording them.)

``One Person, No Vote'' reads like a speedy sequel of sorts to her
previous book, the elegant and illuminating best-seller
\href{https://www.nytimes3xbfgragh.onion/2016/06/26/books/review/white-rage-by-carol-anderson.html}{``White
Rage''} (2016), which traced how periods of black progress have so often
triggered a backlash that ``wreaks havoc subtly, almost imperceptibly''
through the legislatures and courts. ``White rage doesn't have to wear
sheets, burn crosses or take to the streets,'' she wrote. ``Working the
halls of power, it can achieve its ends far more effectively, far more
destructively.''

Image

Carol AndersonCredit...Dave Wetty, Cloud Prime Photography

Her new book seems to have been written from a state of emergency, in an
adrenaline-fueled sprint. Anderson is a stinging polemicist; her book
rolls through a condensed history of voting rights and
disenfranchisement, without getting bogged down in legislative minutiae.
This is harder than it looks; as Anderson explains, it's often through
legislative minutiae that voting rights are curtailed.

The lurid violence of voter suppression looms large in the public
imagination, but ever since black men were granted the franchise in the
wake of the Civil War, politicians have also devised discriminatory
rules ``dressed up in the genteel garb of bringing `integrity' to the
voting booth.'' Before the Voting Rights Act took effect, poll taxes and
literacy tests were the favored methods of voter suppression; they have
since been replaced by gerrymandering and extreme measures to combat
\href{https://www.nytimes3xbfgragh.onion/2014/06/11/upshot/vote-fraud-is-rare-but-myth-is-widespread.html}{the
phantom menace of voter fraud}.

The trick is to keep everything constitutional, Anderson says, staying
within the boundaries of the 14th and 15th Amendments, which promised
``equal protection'' and barred discrimination ``on account of race.''
As the Virginia politician Carter Glass put it candidly in 1902,
``Discrimination! Why, that is precisely what we propose.'' It was, he
said, an elected official's duty ``to discriminate to the very extremity
of permissible action under the limitations of the Federal Constitution,
with a view to the elimination of every negro voter who can be gotten
rid of, legally.''

Contemporary rhetoric isn't so frank and incendiary. Anderson describes
Georgia's Exact Match system and the Interstate Crosscheck as modern
incarnations of old efforts to restrict the vote. Cloaked in anodyne
phrases like ``voter roll maintenance,'' those database-matching
programs ``gave the illusion of being clean, clinical, efficient and
fair,'' Anderson writes, when in fact they had a ``horrific effect on
voter registration, especially for minorities.'' Tiny typographical
errors triggered wrongful purges of eligible voters. According to one
team of researchers, the Crosscheck program --- which was vastly
expanded by Kansas Secretary of State Kris Kobach, a staunch Trump ally
currently running to be the state's governor --- had
\href{https://www.washingtonpost.com/news/wonk/wp/2017/07/20/this-anti-voter-fraud-program-gets-it-wrong-over-99-of-the-time-the-gop-wants-to-take-it-nationwide/?utm_term=.f1ecacd402ec}{an
astonishing error rate of 99 percent}.

From the perspective of federal enforcement, Anderson says, the
situation for minority voters is looking even more perilous now than a
couple of years ago. The Department of Justice under Attorney General
Jeff Sessions --- who as a United States attorney in Alabama tried (and
failed) to obtain a conviction of
\href{https://www.nytimes3xbfgragh.onion/2017/01/09/magazine/the-voter-fraud-case-jeff-sessions-lost-and-cant-escape.html}{three
African-American activists for voter fraud} and once called the Voting
Rights Act
\href{https://www.nytimes3xbfgragh.onion/2017/01/08/us/politics/jeff-sessions-attorney-general.html}{``an
intrusive piece of legislation''} --- has demanded that no fewer than 44
states detail their programs for, yes, voter roll maintenance. The
Presidential Advisory Commission on Election Integrity (after reading
this book, you won't be able look at the word ``integrity'' the same way
again) is chaired by two figures who presided over aggressive
anti-voter-fraud measures in their home states: Kobach and Vice
President Mike Pence.

But at the grass-roots level, Anderson believes that things might be
looking up. She offers a surprisingly riveting play-by-play of last
year's special senate election in Alabama, in which Doug Jones, a
Democrat, won a startling upset over the Republican Roy Moore. She
concedes that Moore,
\href{https://www.nytimes3xbfgragh.onion/2017/11/16/us/roy-moore-alabama-coverage.html}{buffeted
by allegations of sexual assault}, was an especially unappealing
candidate, whatever one's politics. But she also shows how groups like
the N.A.A.C.P. mobilized local efforts to help people register to vote
and --- in a state where poll closures made even getting to the voting
booths an issue --- to offer crucial transportation.

Behind the tactics deployed by both sides looms a larger question: What
kind of future should this country pursue? Should it be a democracy that
is, in Anderson's words, ``vibrant, responsive and inclusive''? Or
should it be a system that maximizes ``the frustration of millions of
citizens to minimize their participation in the electoral process''? To
that end, this trenchant little book will push you to think not just
about the vote count but about who counts, too.

Advertisement

\protect\hyperlink{after-bottom}{Continue reading the main story}

\hypertarget{site-index}{%
\subsection{Site Index}\label{site-index}}

\hypertarget{site-information-navigation}{%
\subsection{Site Information
Navigation}\label{site-information-navigation}}

\begin{itemize}
\tightlist
\item
  \href{https://help.nytimes3xbfgragh.onion/hc/en-us/articles/115014792127-Copyright-notice}{©~2020~The
  New York Times Company}
\end{itemize}

\begin{itemize}
\tightlist
\item
  \href{https://www.nytco.com/}{NYTCo}
\item
  \href{https://help.nytimes3xbfgragh.onion/hc/en-us/articles/115015385887-Contact-Us}{Contact
  Us}
\item
  \href{https://www.nytco.com/careers/}{Work with us}
\item
  \href{https://nytmediakit.com/}{Advertise}
\item
  \href{http://www.tbrandstudio.com/}{T Brand Studio}
\item
  \href{https://www.nytimes3xbfgragh.onion/privacy/cookie-policy\#how-do-i-manage-trackers}{Your
  Ad Choices}
\item
  \href{https://www.nytimes3xbfgragh.onion/privacy}{Privacy}
\item
  \href{https://help.nytimes3xbfgragh.onion/hc/en-us/articles/115014893428-Terms-of-service}{Terms
  of Service}
\item
  \href{https://help.nytimes3xbfgragh.onion/hc/en-us/articles/115014893968-Terms-of-sale}{Terms
  of Sale}
\item
  \href{https://spiderbites.nytimes3xbfgragh.onion}{Site Map}
\item
  \href{https://help.nytimes3xbfgragh.onion/hc/en-us}{Help}
\item
  \href{https://www.nytimes3xbfgragh.onion/subscription?campaignId=37WXW}{Subscriptions}
\end{itemize}
