Sections

SEARCH

\protect\hyperlink{site-content}{Skip to
content}\protect\hyperlink{site-index}{Skip to site index}

\href{https://myaccount.nytimes3xbfgragh.onion/auth/login?response_type=cookie\&client_id=vi}{}

\href{https://www.nytimes3xbfgragh.onion/section/todayspaper}{Today's
Paper}

The Young Woman Was a Healthy and Avid Runner. Now She Could Barely
Walk. Why?

\url{https://nyti.ms/2Je7A4K}

\begin{itemize}
\item
\item
\item
\item
\item
\item
\end{itemize}

Advertisement

\protect\hyperlink{after-top}{Continue reading the main story}

Supported by

\protect\hyperlink{after-sponsor}{Continue reading the main story}

\href{/column/diagnosis}{Diagnosis}

\hypertarget{the-young-woman-was-a-healthy-and-avid-runner-now-she-could-barely-walk-why}{%
\section{The Young Woman Was a Healthy and Avid Runner. Now She Could
Barely Walk.
Why?}\label{the-young-woman-was-a-healthy-and-avid-runner-now-she-could-barely-walk-why}}

\includegraphics{https://static01.graylady3jvrrxbe.onion/images/2018/07/15/magazine/15mag-diagnosis-image1/15mag-diagnosis-image1-articleLarge.png?quality=75\&auto=webp\&disable=upscale}

By \href{https://www.nytimes3xbfgragh.onion/by/lisa-sanders-md}{Lisa
Sanders, M.d.}

\begin{itemize}
\item
  July 11, 2018
\item
  \begin{itemize}
  \item
  \item
  \item
  \item
  \item
  \item
  \end{itemize}
\end{itemize}

The two bridesmaids --- sisters of the bride --- walked arm in arm past
the seated guests. The older, 35, held tight to the arm of the younger,
using that single support to replace the crutches on which she had come
to depend. Proud but exhausted by her efforts, the older sister spent
much of the rest of the celebration in a wheelchair.

Until two years before this wedding, the older sister was an avid
runner, competing in long-distance races. Now she could only hobble
around with crutches or on good days with just a cane, barely able to
get out of a chair.

\textbf{↓}

\textbf{Permanent Pain}

It started with a stabbing pain in her left foot just at the point in
her stride where she pushed off from the ground. She lived in rural
Vermont and was training for a half-marathon, her third, when she first
noticed it. She used ice and ibuprofen and got through the race, but the
pain had been with her ever since. The foot wasn't red; it was maybe a
little swollen, and it hurt, initially just when she walked, then all
the time. An X-ray was normal, but a CT scan revealed a stress fracture
in the bone just below her big toe. She wore a surgical boot for months
and was in and out of physical therapy, but the foot only seemed to get
worse. Before this injury, she coached a dance team, kept up with her
two school-age children and worked a full-time job. Now she did all that
while hobbling around on a foot that should have healed but didn't.

And it wasn't just the foot. Her knees, hips and back also hurt. She
figured it was because of her altered gait. And she felt weak. Not being
active seemed to sap her strength.

A follow-up CT scan showed that the fracture had healed but somehow the
pain remained. Her orthopedic surgeon thought she may have developed
something known as Complex Regional Pain Syndrome (C.R.P.S.), an unusual
neurological response to trauma, where after healing, patients develop a
deep and burning pain at the site of the injury that is often
accompanied by swelling and changes in skin color or temperature.
Although rare, it is most commonly seen after fractures, sprains or
surgery. The pain is severe and limits the use of the affected limb.
There may be associated bone changes visible on X-ray or CT scan. Pain
control and physical therapy are the mainstays of treatment.

The woman went to several pain specialists. Nothing helped, and
everything had side effects. She was found to have low vitamin D levels;
this vitamin is essential for bone formation, and deficiencies are known
to cause pain, so she was started on replacement doses. There were no
side effects, but the vitamin didn't seem to help. Her primary-care
physician heard about a clinical trial on a new treatment for C.R.P.S.
The testing required for entry into the study provided a new but
puzzling clue: Her phosphate level turned out to be dangerously low.

\textbf{↓}

\textbf{Too Little Phosphate}

Bone is made, primarily, of calcium and phosphate, absorbed from the
foods we eat. Both are plentiful in most diets, so deficiencies are
rare. Still, her vitamin D was also quite low, and that plays an
essential role in controlling how much calcium and phosphate we absorb.
Her physician put her on high doses of phosphate and vitamin D. When,
after months of treatment, she was still deficient, he referred her to
David Gorson, an endocrinologist and a specialist in bone metabolism.

Just days before her sister's wedding, the patient went to see Gorson.
As she told him her story, he was struck by how uncomfortable she
looked. She shifted from side to side in the chair as if seeking just
the right spot. She walked stiffly, as if her whole body ached. And when
she got to the exam table, she was too sore and too weak to climb up.
She was missing a molar, he noted. She told him it fell out a couple of
years earlier. The only other abnormality he found was that she was
quite weak. When she held out her arms, he could easily push them down.
Her legs were also weak.

\textbf{↓}

\textbf{A New Hormone}

Gorson wasn't surprised. Even before seeing her, he had a theory about
what was going on. The patient's primary-care doctor had called him to
get his thoughts about her unrelenting foot pain and her baffling lab
abnormalities --- persistently low levels of phosphate and vitamin D.
Hearing about her, Gorson recalled a recently discovered hormone called
FGF23 that controls phosphate levels in the body. If a person has too
much of this hormone, her body will eliminate it through the urine. Even
if the person is taking in phosphate in her diet, she will, like this
woman, become phosphate-deficient. Gorson had asked the doctor to check
the woman's FGF23 level. By the time she came to see him, he had the
answer: Her FGF23 level was abnormally high.

Bone is constantly being broken down and rebuilt --- that's how bones
stay strong. This woman didn't have enough phosphate to rebuild her
bones once they'd been broken down, hence the pain in her bones and the
missing tooth. Teeth, and the bone they sit in, need phosphate to stay
strong. Losing her tooth was probably the first hint that she had an
excess of FGF23. A lack of phosphate also made her weak; phosphate is a
key component in the process that gives muscle cells the energy to
function. No phosphate, no function.

\textbf{↓}

\textbf{Searching for an Explanation}

Gorson felt certain that the high level of FGF23 was the cause of her
bone pain and weakness. What wasn't clear to Gorson was why her FGF23
level was so high. In most cases, abnormalities of FGF23 come from one
of several rare, inherited conditions with lifelong effects. This woman
had no history of that.

Once she had a diagnosis, Gorson suggested that the patient contact Dr.
Karl Insogna, a bone specialist at Yale School of Medicine in New Haven.
It would be worth the hourslong trip to see someone with experience with
this hormone and the troubles it could cause. Insogna greeted the
patient, who was accompanied by her mother. He questioned them closely
about the patient's early years. She hadn't had any bone problems as a
child, and her teeth had been fine. She had been quite healthy until
recently.

Insogna, examining the patient, found only what had been noted by Gorson
--- that the young woman was weak, tender and unsteady on her feet ---
classic findings for osteomalacia, a bone disorder linked to an
overproduction of FGF23. Insogna, who had spent decades trying to better
understand diseases of the bone, had seen a handful of patients who
developed osteomalacia from FGF23 excess in adulthood. Those patients
had benign tumors that secreted FGF23. He suspected that this woman did,
too, although it was extremely rare.

\includegraphics{https://static01.graylady3jvrrxbe.onion/images/2018/07/15/magazine/15mag-diagnosis-image2/15mag-diagnosis-image2-articleLarge.png?quality=75\&auto=webp\&disable=upscale}

\textbf{↓}

\textbf{A Tumor Lurking}

The doctor ordered a bone scan and, sure enough, a golfball-size mass
was found in the patient's left thigh. Removing this mass would stop the
bone destruction that had been crippling her for so long, Insogna
explained when he called her with the results. In addition, the scan
identified areas all over her body that had been broken down and not
rebuilt. These were all the places --- her knees, hips and back --- that
were painful and made it difficult to walk. It would take time, he said,
but she would recover.

She had the tumor removed in February. It was benign, and within weeks
she started to feel less weak. Just having the right amount of phosphate
in her system restored much of her strength. But the bone pain got
worse. Insogna had warned her that this was expected while her bones
sucked up the phosphate they had been deprived of for so long and began
rebuilding. After a couple of months, the pain began to ease, and she
started the long process of getting stronger.

This spring the patient's sister, with whom she'd walked arm in arm at
that fall wedding, married. The patient was thrilled to able to move
down the aisle unassisted. She hopes to run another half marathon
sometime next year.

Advertisement

\protect\hyperlink{after-bottom}{Continue reading the main story}

\hypertarget{site-index}{%
\subsection{Site Index}\label{site-index}}

\hypertarget{site-information-navigation}{%
\subsection{Site Information
Navigation}\label{site-information-navigation}}

\begin{itemize}
\tightlist
\item
  \href{https://help.nytimes3xbfgragh.onion/hc/en-us/articles/115014792127-Copyright-notice}{©~2020~The
  New York Times Company}
\end{itemize}

\begin{itemize}
\tightlist
\item
  \href{https://www.nytco.com/}{NYTCo}
\item
  \href{https://help.nytimes3xbfgragh.onion/hc/en-us/articles/115015385887-Contact-Us}{Contact
  Us}
\item
  \href{https://www.nytco.com/careers/}{Work with us}
\item
  \href{https://nytmediakit.com/}{Advertise}
\item
  \href{http://www.tbrandstudio.com/}{T Brand Studio}
\item
  \href{https://www.nytimes3xbfgragh.onion/privacy/cookie-policy\#how-do-i-manage-trackers}{Your
  Ad Choices}
\item
  \href{https://www.nytimes3xbfgragh.onion/privacy}{Privacy}
\item
  \href{https://help.nytimes3xbfgragh.onion/hc/en-us/articles/115014893428-Terms-of-service}{Terms
  of Service}
\item
  \href{https://help.nytimes3xbfgragh.onion/hc/en-us/articles/115014893968-Terms-of-sale}{Terms
  of Sale}
\item
  \href{https://spiderbites.nytimes3xbfgragh.onion}{Site Map}
\item
  \href{https://help.nytimes3xbfgragh.onion/hc/en-us}{Help}
\item
  \href{https://www.nytimes3xbfgragh.onion/subscription?campaignId=37WXW}{Subscriptions}
\end{itemize}
