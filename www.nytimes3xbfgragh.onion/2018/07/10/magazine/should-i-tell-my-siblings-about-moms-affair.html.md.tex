Sections

SEARCH

\protect\hyperlink{site-content}{Skip to
content}\protect\hyperlink{site-index}{Skip to site index}

\href{https://myaccount.nytimes3xbfgragh.onion/auth/login?response_type=cookie\&client_id=vi}{}

\href{https://www.nytimes3xbfgragh.onion/section/todayspaper}{Today's
Paper}

Should I Tell My Siblings About Mom's Affair?

\url{https://nyti.ms/2NEoYmD}

\begin{itemize}
\item
\item
\item
\item
\item
\item
\end{itemize}

Advertisement

\protect\hyperlink{after-top}{Continue reading the main story}

Supported by

\protect\hyperlink{after-sponsor}{Continue reading the main story}

\href{/column/the-ethicist}{The Ethicist}

\hypertarget{should-i-tell-my-siblings-about-moms-affair}{%
\section{Should I Tell My Siblings About Mom's
Affair?}\label{should-i-tell-my-siblings-about-moms-affair}}

\includegraphics{https://static01.graylady3jvrrxbe.onion/images/2018/07/15/magazine/15Mag-Ethicist/15Mag-Ethicist-popup.png?quality=75\&auto=webp\&disable=upscale}

By Kwame Anthony Appiah

\begin{itemize}
\item
  July 10, 2018
\item
  \begin{itemize}
  \item
  \item
  \item
  \item
  \item
  \item
  \end{itemize}
\end{itemize}

\emph{My mother died 18 years ago. Before she died, she told me about an
affair of many years' duration with a family friend. Learning about this
made me re-evaluate my parents' relationship and the decisions they
made. My father is also dead, and I have been struggling with whether
this information is something my siblings have the right to know, or
whether it is not my place to share what she told me.} Name Withheld

\textbf{``We dance round} in a ring and suppose,'' the Robert Frost poem
has it. ``But the Secret sits in the middle and knows.'' In the end, our
parents are likely to remain somewhat inscrutable to us. We're not
likely to have asked them all the questions we would need answered to
make sense of everything important about them. And if we had, they
wouldn't have told us everything we needed to know.

Still, understanding your family and making sense, in particular, of
your parents' lives, is something that many people care about; and even
if they don't, it is going to be up to them what they make of important
facts about their parents. What your mother indicated about what you
should do with this information is relevant, but her post-mortem
interest in the matter fades as time passes. So tell your siblings what
you know --- but be mindful of how much you still don't.

\emph{I am fortunate to be a tenured professor at a state university and
have enjoyed many years of teaching the students enrolled there, many of
whom are first-generation college students, as I was. Attending college
had an enormous impact on my life, and I know that's true for many of my
students, who have done well in careers they never imagined for
themselves. I was proud to be part of an institution having a positive
impact on the lives of many students.}

\emph{However, in the past 10 years, things have changed. Like many
schools in search of money, my university has substantially increased
enrollment for more tuition dollars. In order to do that, it has been
admitting more and more students who are not academically prepared for
college. In addition, many of these students want a diploma because they
have been told that they will make more money if they have one, but they
are not motivated to learn the material. They often work full time and
have time-consuming family obligations but are still taking a full load
of four to five courses per semester. Many of them fail their classes,
but they can repeat a failed course so that the F does not count in
their G.P.A.s. So they are paying for some courses two or three times,
enriching the university. Many of them drop out, though often well after
they should have, with substantial debt and no degree.}

\emph{My colleagues wind up flunking a distressingly large proportion of
their students. Others, to avoid that, simply give everybody passing
grades for just showing up to class, or they teach what they no longer
regard as legitimately ``college level'' courses, thereby shortchanging
the students who wanted, expected to get and would benefit from a
college-level education. The administration's answer to the problem is
for the faculty to give weak students more help. That is not possible
given the high number of unprepared students. And most students either
are not motivated enough or lack the time to come to faculty office
hours. What is an ethical strategy for a faculty member in this
situation? I used to love teaching. I now simply don't know what I
should be trying to do.} Name Withheld

\textbf{Basic questions about} the purpose of higher education are at
stake here. Going to college improves your life prospects in two ways.
One is by giving you a liberal education that allows you to appreciate
what Matthew Arnold called ``the best that has been thought and said.''
This is worthwhile whether or not it helps you (as it well may) earn a
living. A second way that college helps you is by providing training and
credentials for a life of work. At some colleges, one of these benefits
may be emphasized over the other, but it's a virtue of our system of
higher education that it assumes these two rewards to be interconnected.
Someone who is aware only that a diploma can be helpful in the job
market is going to miss a large part of the point of a college
education.

It isn't snobbish to think that a life in which you are in touch with
the best thinking in the humanities and sciences is better than a life
without those experiences. Indeed, what's snobbish is to reserve these
goods for the few. Liberalis in Latin means ``suitable for a free
person'': The liberal arts once implied a contrast between the free and
the unfree --- slaves, serfs, the dependent ``lower orders.'' Today an
education suited for a free person is an education that helps each of us
exercise the responsibility of making a worthwhile life.

This ideal of a liberal education is profoundly democratic, aiming at
enlarging the possibilities and the contributions of the largest
possible number of people. It's good news that there are more
first-generation college students than there used to be. Such an
education enables and encourages people to be capable citizens, better
able to evaluate the arguments that circulate in public life and better
positioned to take the obligations of citizenship seriously.

But all these benefits depend on students' being prepared for the
courses they take and their being engaged with them in a serious way.
The value of education comes not from their mere physical presence in
the classroom or the lab or the library but from their doing course work
as well as they can. If students aren't prepared, if they don't have the
right motivations, if they don't have the necessary time or resources
for study, or if professors don't have the time to give them the
attention they need, the value of a college education is diminished.

Where do you come in? Our obligations to make the world better are
limited by a simple principle: What we owe is only our fair share of the
burden of securing for others what they are owed. What has gone wrong at
your university won't be set right by anything that can reasonably be
expected of you. But there are a couple of things that are within your
professional responsibility. One is to do the best that you can for your
own students. Another is to urge your colleagues --- through
departmental and faculty-senate discussions and through all the other
channels which your university reflects on its policies --- to discuss
the problems you have identified and come up with ways of improving the
situation. Who knows? Maybe having those conversations --- with students
as well as colleagues --- will revive your love of teaching.

Advertisement

\protect\hyperlink{after-bottom}{Continue reading the main story}

\hypertarget{site-index}{%
\subsection{Site Index}\label{site-index}}

\hypertarget{site-information-navigation}{%
\subsection{Site Information
Navigation}\label{site-information-navigation}}

\begin{itemize}
\tightlist
\item
  \href{https://help.nytimes3xbfgragh.onion/hc/en-us/articles/115014792127-Copyright-notice}{©~2020~The
  New York Times Company}
\end{itemize}

\begin{itemize}
\tightlist
\item
  \href{https://www.nytco.com/}{NYTCo}
\item
  \href{https://help.nytimes3xbfgragh.onion/hc/en-us/articles/115015385887-Contact-Us}{Contact
  Us}
\item
  \href{https://www.nytco.com/careers/}{Work with us}
\item
  \href{https://nytmediakit.com/}{Advertise}
\item
  \href{http://www.tbrandstudio.com/}{T Brand Studio}
\item
  \href{https://www.nytimes3xbfgragh.onion/privacy/cookie-policy\#how-do-i-manage-trackers}{Your
  Ad Choices}
\item
  \href{https://www.nytimes3xbfgragh.onion/privacy}{Privacy}
\item
  \href{https://help.nytimes3xbfgragh.onion/hc/en-us/articles/115014893428-Terms-of-service}{Terms
  of Service}
\item
  \href{https://help.nytimes3xbfgragh.onion/hc/en-us/articles/115014893968-Terms-of-sale}{Terms
  of Sale}
\item
  \href{https://spiderbites.nytimes3xbfgragh.onion}{Site Map}
\item
  \href{https://help.nytimes3xbfgragh.onion/hc/en-us}{Help}
\item
  \href{https://www.nytimes3xbfgragh.onion/subscription?campaignId=37WXW}{Subscriptions}
\end{itemize}
