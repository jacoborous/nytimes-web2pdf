Sections

SEARCH

\protect\hyperlink{site-content}{Skip to
content}\protect\hyperlink{site-index}{Skip to site index}

\href{https://myaccount.nytimes3xbfgragh.onion/auth/login?response_type=cookie\&client_id=vi}{}

\href{https://www.nytimes3xbfgragh.onion/section/todayspaper}{Today's
Paper}

How Do You Explain the `Obvious?'

\url{https://nyti.ms/2wj59cu}

\begin{itemize}
\item
\item
\item
\item
\item
\item
\end{itemize}

Advertisement

\protect\hyperlink{after-top}{Continue reading the main story}

Supported by

\protect\hyperlink{after-sponsor}{Continue reading the main story}

\href{/column/first-words}{First Words}

\hypertarget{how-do-you-explain-the-obvious}{%
\section{How Do You Explain the
`Obvious?'}\label{how-do-you-explain-the-obvious}}

\includegraphics{https://static01.graylady3jvrrxbe.onion/images/2018/08/26/magazine/26mag-firstwords-image1/26mag-firstwords-image1-articleLarge.png?quality=75\&auto=webp\&disable=upscale}

By Nausicaa Renner

\begin{itemize}
\item
  Aug. 21, 2018
\item
  \begin{itemize}
  \item
  \item
  \item
  \item
  \item
  \item
  \end{itemize}
\end{itemize}

There's nothing more persuasive than the obvious. To appeal to it is to
ask people to be bigger, better, more noble --- to take a sweeping look
at the facts, admit what is plain and do the right thing. Tell me with a
fixed gaze and an air of confidence that something is obvious. I will be
tempted to believe you, if only to join in the clarity and sense of
purpose that comes with accepting what is staring me in the face.

In July, after President Trump's meeting with Vladimir Putin in
Helsinki, David Remnick, the editor of The New Yorker, called on
congressional Republicans to recognize the obvious. Trump, he wrote, had
spent his trip working ``to humiliate the leaders of Western Europe and
declare them `foes'; to fracture longstanding military, economic and
political alliances; and to absolve Russia of its attempts to undermine
the 2016 election. He did so clearly, repeatedly and with conviction.''
\emph{Use your heads}, Remnick seemed to say, inviting G.O.P. leadership
out of the darkness and into the light, asking which of them would
``stand up not to applaud the Great Leader but to find the capacity to
say what is obvious and what is true.'' New York magazine went further,
using the blunt instrument of obviousness to impugn the Republican
Party: ``G.O.P. Senators: Trump's Obvious Russia Lie Is Good Enough for
Us,'' read
\href{http://nymag.com/daily/intelligencer/2018/07/gop-senators-trumps-obvious-russia-lie-good-enough-for-us.html}{one
headline}, soon after the president claimed that he had, during a news
conference with Putin, accidentally said ``would'' when he meant
``wouldn't.'' (``It should have been obvious,'' he said, defending
himself. ``I thought it would be obvious.'')

The obvious is a common tool in political arguments; there is something
about calling on voters' ``common sense'' that makes the opposition look
like sophists and weasels, waffling and equivocating. The obvious cuts
through nonsense. It asks why we have hundreds of pages of tax law
instead of one; it insists on straightforward fixes for immigration
policy. And part of the appeal of universal health care is simply that
it's universal: no compromises, no complex incentive systems, no
loopholes, less a policy than a statement of rights. In
\href{https://www.vox.com/the-big-idea/2018/7/13/17567952/medicare-for-all-centrists-copycat-plans-water-down-left-center-sanders}{a
recent Vox article}, Tim Higginbotham and Chris Middleman wrote that
Medicare-for-all plans present a ``resolute vision, one in which our
common well-being and dignity take obvious precedence over the profits
of a few.'' The stance is sure of itself; it has the certitude to weigh
health care against profit and reach a decisive answer, while others
remain lost in a mental fog.

But we also appeal to the obvious as a last-ditch effort when, after
decades of conflict, we're further than ever from clarity. After the
2012 shooting at Sandy Hook Elementary School, President Obama gave an
emotional speech at a vigil for the 20 children and six adults who were
killed, asking the nation to look at itself: ``Are we really prepared to
say that we're powerless in the face of such carnage, that the politics
are too hard?'' A few years later, in a speech calling for bipartisan
agreement on gun laws, he noted that after Sandy Hook, 90 percent of
Americans supported a ``common-sense compromise'' bill. But Republicans
had voted that bill down. The speech had a ring of desperation and
defeat: If we can't agree on something this obvious, the president
seemed to ask, what can we agree on?

\textbf{In Edgar Allan} Poe's ``The Purloined Letter,'' the detective
Auguste Dupin is able to find a stolen letter in the apartment of an
unscrupulous government minister --- a letter no one else could find,
because everyone else assumed it would be treated as if it were valuable
and hidden. Instead, the letter was hiding in plain sight, not carefully
preserved but crumpled and torn like trash. It escapes detection ``by
dint of being extremely obvious.'' We prefer our politicians to be like
Dupin: able to rise above the mire of small details and see the whole.

This is harder than it sounds. The letter either pops out or it doesn't.
The obvious can be like a Magic Eye poster, one of those novelties whose
hidden 3-D image only leaps out at you when you look at it just right:
You can't really help someone else see it. It has been a signature move
of the Trump administration to disrupt the obvious, beginning with a
debate over the size of the crowd at the moment the president was sworn
in. The mind is great at coming up with viable alternatives to ideas it
doesn't want to accept, and those unwilling to accept invocations of the
obvious, like Remnick's, find themselves safely tangled in a web of
possibilities. With Trump, ``rather than acknowledge the obvious, the
supporters spin theories of `Art of the Deal,' '' wrote Jim Schutze in
\href{https://www.dallasobserver.com/news/a-more-obvious-trump-theory-the-man-is-just-an-idiot-10918056}{a
column in the Dallas Observer}, ``imputing all kinds of cleverness and
guile, saying he pretends to be an idiot as part of a wily strategy.''
At its least extreme, this entails a belief that there is some cunning
in Trump's most transparent lies and clumsiest public statements; at its
most extreme, it puts him at the center of an elaborate plot to destroy
the ``deep state.'' What is ``obvious'' is taken as false because it's
\emph{too} obvious.

This is because the obvious is, essentially, a shortcut: It appeals to a
set of values we'd formed some consensus around, a set of ideas we once
agreed no serious person would question. To call something ``obvious''
or ``common sense'' is to call it settled and refuse to relitigate it or
revisit all the work that went into determining it was so inarguable in
the first place. In a recent book, ``At War With the Obvious,'' the
psychoanalyst Donald Moss writes that ``the obvious is adaptive. It
mutates under pressure, like cells.'' If you need evidence of this, he
writes, consider the status of gay, queer and trans people over the past
few decades. In the 1990s, the American mainstream found it obvious that
gay people should have no right to marry; today, it's regarded by many
as broadly obvious that they should. An idea that was once marginal
enough to require laborious defense gradually became so self-evident
that it was hardly worth explaining; like the crumpled letter, its
presence was taken for granted.

The difficulty is that, later, when such propositions are threatened,
people may find themselves shocked, out of practice, struggling to
defend their values with the passion or eloquence that first brought
them into existence. Last month, for instance, Michael Anton, a former
national-security official in the Trump administration, published
\href{https://www.washingtonpost.com/opinions/citizenship-shouldnt-be-a-birthright/2018/07/18/7d0e2998-8912-11e8-85ae-511bc1146b0b_story.html}{a
Washington Post op-ed} arguing that, contrary to the understanding of
most readers, birthright citizenship was based in a misreading of the
law and should be ended by executive order. The fury that met this
suggestion was sputtering: For anyone not already immersed in
constitutional law, being horrified by Anton's claims meant arguing in
favor of something that had long been so obvious that it was easy to
forget what made it obvious in the first place. Justin Fox, a columnist
for Bloomberg Opinion,
\href{https://www.bloomberg.com/view/articles/2018-07-24/ignore-fake-arguments-over-birthright-citizenship}{allowed}
that a majority of the world's nations didn't offer birthright
citizenship. But the claim that the authors of the 14th Amendment
intended anything else, he wrote, ``is, to anyone who takes the time to
read a few pages of congressional debate, obviously false.''

\textbf{America is built} on an appeal to the obvious. The Declaration
of Independence holds its truths to be ``self-evident'' --- axiomatic,
irreducible, not needing justification because they justify themselves.
(It was not obvious to the authors that those truths applied to all
Americans, though this seems obvious to most of us now.)

What Americans have confronted lately is a state of affairs in which
many of our most basic paradigms are no longer obvious to everyone.
Appeals to obviousness seem to wilt as soon as they appear. ``Are we
prepared to say that such violence visited on our children, year after
year after year, is somehow the price of our freedom?'' asked Obama in
his Sandy Hook speech. This was a rhetorical question; the obvious
answer is supposed to be ``no.'' But what if some Americans answer with
``yes''?

Politicians and the press still invoke obviousness in the hope of
summoning some conviction we all still share, some bedrock of group
belief we can agree on. To see them fail, repeatedly, is unsettling; it
makes our deepest values seem impotent. It had seemed obvious to some
that a modern presidential administration would not defend white
nationalists or that the United States government would seek to avoid
taking babies from their parents' arms --- or that a man who bragged
about harassing women wouldn't be elected in the first place. Last
summer, NPR celebrated the Fourth of July by tweeting, line by line, the
text of the Declaration of Independence; its account was immediately
attacked by angry Americans accusing the organization of spreading
seditious anti-Trump propaganda. The nation's founding values have come
to seem, somehow, unfamiliar and contentious; we can't recognize the
Declaration of Independence when we see it. Let the obvious sit too long
and it becomes like an animal in a zoo: pointed at, but never exercised,
and idly wandered past by people who have forgotten how powerful it is
in action.

Advertisement

\protect\hyperlink{after-bottom}{Continue reading the main story}

\hypertarget{site-index}{%
\subsection{Site Index}\label{site-index}}

\hypertarget{site-information-navigation}{%
\subsection{Site Information
Navigation}\label{site-information-navigation}}

\begin{itemize}
\tightlist
\item
  \href{https://help.nytimes3xbfgragh.onion/hc/en-us/articles/115014792127-Copyright-notice}{©~2020~The
  New York Times Company}
\end{itemize}

\begin{itemize}
\tightlist
\item
  \href{https://www.nytco.com/}{NYTCo}
\item
  \href{https://help.nytimes3xbfgragh.onion/hc/en-us/articles/115015385887-Contact-Us}{Contact
  Us}
\item
  \href{https://www.nytco.com/careers/}{Work with us}
\item
  \href{https://nytmediakit.com/}{Advertise}
\item
  \href{http://www.tbrandstudio.com/}{T Brand Studio}
\item
  \href{https://www.nytimes3xbfgragh.onion/privacy/cookie-policy\#how-do-i-manage-trackers}{Your
  Ad Choices}
\item
  \href{https://www.nytimes3xbfgragh.onion/privacy}{Privacy}
\item
  \href{https://help.nytimes3xbfgragh.onion/hc/en-us/articles/115014893428-Terms-of-service}{Terms
  of Service}
\item
  \href{https://help.nytimes3xbfgragh.onion/hc/en-us/articles/115014893968-Terms-of-sale}{Terms
  of Sale}
\item
  \href{https://spiderbites.nytimes3xbfgragh.onion}{Site Map}
\item
  \href{https://help.nytimes3xbfgragh.onion/hc/en-us}{Help}
\item
  \href{https://www.nytimes3xbfgragh.onion/subscription?campaignId=37WXW}{Subscriptions}
\end{itemize}
