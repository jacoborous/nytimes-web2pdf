Sections

SEARCH

\protect\hyperlink{site-content}{Skip to
content}\protect\hyperlink{site-index}{Skip to site index}

\href{https://myaccount.nytimes3xbfgragh.onion/auth/login?response_type=cookie\&client_id=vi}{}

\href{https://www.nytimes3xbfgragh.onion/section/todayspaper}{Today's
Paper}

France Meets China in a Luxurious Custard

\url{https://nyti.ms/2M9LQN4}

\begin{itemize}
\item
\item
\item
\item
\item
\end{itemize}

Advertisement

\protect\hyperlink{after-top}{Continue reading the main story}

Supported by

\protect\hyperlink{after-sponsor}{Continue reading the main story}

\href{/column/on-dessert}{On Dessert}

\hypertarget{france-meets-china-in-a-luxurious-custard}{%
\section{France Meets China in a Luxurious
Custard}\label{france-meets-china-in-a-luxurious-custard}}

\includegraphics{https://static01.graylady3jvrrxbe.onion/images/2018/08/12/magazine/12mag-on-dessert-image1/12mag-on-dessert-image1-articleLarge.jpg?quality=75\&auto=webp\&disable=upscale}

By Dorie Greenspan

\begin{itemize}
\item
  Aug. 8, 2018
\item
  \begin{itemize}
  \item
  \item
  \item
  \item
  \item
  \end{itemize}
\end{itemize}

When I was starting out, teaching myself to cook and bake and trying to
make a career in food, I would corner every chef who would give me a
minute and ask what I had to do to get good. From line cooks at
neighborhood joints to chefs who wore starched whites, their advice was
always the same: Learn the basics.

It's what a master would tell a tyro in just about any field, and I took
the counsel seriously. Because my first love was desserts, the basics I
studied most closely were French, the bedrock of pastry. I baked my way
through several books, but my bible was ``Lenôtre's Desserts and
Pastries,'' by Gaston Lenôtre. The book contained dozens of recipes for
basic doughs and batters, creams and syrups. And then I moved on to the
recipes that put these basics together to make cakes, tortes, tarts,
petit fours and other desserts with many parts.

But what I really wanted to learn was how to use these tools to build my
own creations. Then, as now, a fruit, a fragrance, a combination of
flavors or a new texture could set me dreaming about a new sweet. I
wanted to know how to make desserts from the inspirations I found around
me.

The memories of those early days came back to me recently during a
conversation with Daniel Skurnick, the pastry chef at the French
restaurant Le Coucou in New York City. We talked about classics like
gâteau St.-Honoré and Paris-Brest, canonical desserts with layers of
basic batters and creams. It was only later that I discovered that he is
also the pastry chef at the pan-Asian restaurant Buddakan.

I imagined him as a character in ``The Incredibles,'' ducking behind Le
Coucou in SoHo to stash his whisks and ditch the butter, grabbing bamboo
steam baskets and coconut milk before racing to Buddakan in Chelsea. The
truth about how he works turns out to be not so fantastical. Like most
pastry chefs trained in this country, Skurnick was educated in the
French fundamentals, and he uses them at both restaurants.

``When I auditioned at Buddakan,'' Skurnick told me, ``I created a bunch
of desserts using what I knew: the flavors that I'd become familiar with
when I lived in Thailand and traveled throughout Asia, and French
techniques.'' That the basics paved his path to working at Buddakan
convinced me, once again, that they are the Esperanto of the pastry
world.

Although the cuisines seemed disparate to me, Skurnick saw similarities
between them, pointing out that rice pudding with mango, a staple
throughout Asia, is not such a distant relative of \emph{riz à
l'impératrice}, the French rice pudding with candied fruit. ``Add a
poached peach --- its texture is like a mango's --- and they could be
cousins,'' Skurnick said. When I wondered aloud if there was a custard,
one of my favorite desserts, that would be comfortable in both Asia and
France, Skurnick said he would make me one.

What began as a pastry geeks' game of dessert geography turned into a
recipe so good that I've made it repeatedly since Skurnick gave it to
me. The custard relies on egg yolks for structure and has just enough
sugar to legitimize calling it a dessert. Its flavor, subtle but
clarion, is singularly that of ginger. And it's steamed, as many Asian
but few French desserts are; that's the key to its lithe texture.

The ginger custard is lean and spare --- it's made with just five
ingredients --- but it's luxurious, which is what I think custard is
meant to be. It quivers and quakes and makes a gentle ``thuck'' sound
when you dip into it. It jiggles precariously on a spoon, and if you
press a little of it against the roof of your mouth, it disappears in a
moment. It's strikingly similar to crème caramel --- a classic French
custard baked in a caramel-lined mold, not unlike a flan. But this has a
caramel syrup that is flavored with pepper, cloves, nutmeg and ginger
--- spices common to both French and Chinese cooking. It's a perfect
custard, and a perfectly delicious bridge between worlds.

\textbf{Recipe:}
\href{https://cooking.nytimes3xbfgragh.onion/recipes/1019459-daniel-skurnicks-franco-chinese-steamed-ginger-custard}{Daniel
Skurnick's Franco-Chinese Steamed Ginger Custard} \textbar{}
\href{https://cooking.nytimes3xbfgragh.onion/recipes/1019460-spiced-caramel-syrup}{Spiced
Caramel Syrup}

Advertisement

\protect\hyperlink{after-bottom}{Continue reading the main story}

\hypertarget{site-index}{%
\subsection{Site Index}\label{site-index}}

\hypertarget{site-information-navigation}{%
\subsection{Site Information
Navigation}\label{site-information-navigation}}

\begin{itemize}
\tightlist
\item
  \href{https://help.nytimes3xbfgragh.onion/hc/en-us/articles/115014792127-Copyright-notice}{©~2020~The
  New York Times Company}
\end{itemize}

\begin{itemize}
\tightlist
\item
  \href{https://www.nytco.com/}{NYTCo}
\item
  \href{https://help.nytimes3xbfgragh.onion/hc/en-us/articles/115015385887-Contact-Us}{Contact
  Us}
\item
  \href{https://www.nytco.com/careers/}{Work with us}
\item
  \href{https://nytmediakit.com/}{Advertise}
\item
  \href{http://www.tbrandstudio.com/}{T Brand Studio}
\item
  \href{https://www.nytimes3xbfgragh.onion/privacy/cookie-policy\#how-do-i-manage-trackers}{Your
  Ad Choices}
\item
  \href{https://www.nytimes3xbfgragh.onion/privacy}{Privacy}
\item
  \href{https://help.nytimes3xbfgragh.onion/hc/en-us/articles/115014893428-Terms-of-service}{Terms
  of Service}
\item
  \href{https://help.nytimes3xbfgragh.onion/hc/en-us/articles/115014893968-Terms-of-sale}{Terms
  of Sale}
\item
  \href{https://spiderbites.nytimes3xbfgragh.onion}{Site Map}
\item
  \href{https://help.nytimes3xbfgragh.onion/hc/en-us}{Help}
\item
  \href{https://www.nytimes3xbfgragh.onion/subscription?campaignId=37WXW}{Subscriptions}
\end{itemize}
