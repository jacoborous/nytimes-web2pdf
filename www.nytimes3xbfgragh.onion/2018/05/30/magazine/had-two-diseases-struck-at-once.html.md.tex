Sections

SEARCH

\protect\hyperlink{site-content}{Skip to
content}\protect\hyperlink{site-index}{Skip to site index}

\href{https://myaccount.nytimes3xbfgragh.onion/auth/login?response_type=cookie\&client_id=vi}{}

\href{https://www.nytimes3xbfgragh.onion/section/todayspaper}{Today's
Paper}

Had Two Diseases Struck at Once?

\url{https://nyti.ms/2H2UqpZ}

\begin{itemize}
\item
\item
\item
\item
\item
\item
\end{itemize}

Advertisement

\protect\hyperlink{after-top}{Continue reading the main story}

Supported by

\protect\hyperlink{after-sponsor}{Continue reading the main story}

\href{/column/diagnosis}{Diagnosis}

\hypertarget{had-two-diseases-struck-at-once}{%
\section{Had Two Diseases Struck at
Once?}\label{had-two-diseases-struck-at-once}}

\includegraphics{https://static01.graylady3jvrrxbe.onion/images/2018/06/03/magazine/03mag-diagnosis-image-1/03mag-diagnosis-image-1-articleLarge.png?quality=75\&auto=webp\&disable=upscale}

By \href{https://www.nytimes3xbfgragh.onion/by/lisa-sanders-md}{Lisa
Sanders, M.D.}

\begin{itemize}
\item
  May 30, 2018
\item
  \begin{itemize}
  \item
  \item
  \item
  \item
  \item
  \item
  \end{itemize}
\end{itemize}

It was Christmas Day when the 34-year-old woman acknowledged that
something was really wrong. Her husband --- like her, a physician at the
Mayo Clinic in Rochester, Minn. --- drew the short straw and had to work
that day. Home alone with her children --- the baby nearly 1, the other
nearly 3 --- she realized she was literally too tired to put the kids to
bed. When her husband got home, well after their bedtime, he scooped up
the baby and the child and took them upstairs. As they disappeared into
the bedroom, she struggled to follow. She held onto the banister with
both hands. Lifting her legs was real work. She felt as if weights were
attached to her feet. How did she get to be this tired?

Only three months ago, she could take these stairs two at a time.
Indeed, until a few weeks earlier, she was running several miles on her
treadmill each morning at a seven-minute-mile pace. Yes, it was winter
in Minnesota; yes, everyone was tired, maybe even depressed. But as she
worked her way up the stairs, she finally considered the possibility
that maybe there was more to it than the wintertime blues.

\hypertarget{a-worried-neurologist}{%
\subsection{\texorpdfstring{\textbf{A Worried
Neurologist}}{A Worried Neurologist}}\label{a-worried-neurologist}}

She emailed her doctor that night: I'm exhausted, and my muscles ache.
My joints are stiff, but they don't hurt. Can you see me? The answer was
no. Holiday weeks were always crazy for those who worked them. Can we at
least get some testing, she suggested. This wasn't the right way to do
things, the patient knew, but it seemed to be all they had time for.

Within a day, the exhausted doctor had blood drawn and then started
checking for results as they were posted on the patient portal. At first
everything looked fine. She wasn't anemic. Her thyroid-hormone level was
normal. So was her vitamin D. Kidney and liver tests were unrevealing.
There was no sign of an inflammatory disorder. Then finally, a clue. A
protein released when muscles are injured, creatine kinase, was much too
high. Was something attacking her muscles? Her doctor suggested that she
repeat the test, and when that level was still quite high, her doctor
sought advice from a colleague in her practice.

When Dr. Nathan Young, a neurologist, heard about this 30-something
woman with fatigue, muscle pain and lab work suggestive of muscle
injury, he was worried. He called her immediately. She was in her
office, catching up on paperwork. He worked in a practice at the Mayo
Clinic set up to take care of its own employees as well as Rochester
residents, so he was able to evaluate her right there in her office.
Could he come to see her right now? Sure, she told him. Come on up. It
was odd to be the patient in her own exam room, but she was happy to be
able to see a doctor about something that now seemed worrisome.

\hypertarget{excessive-fatigue}{%
\subsection{\texorpdfstring{\textbf{Excessive
Fatigue}}{Excessive Fatigue}}\label{excessive-fatigue}}

It started months ago, she told Young. Thinking back, she realized that
it suddenly got much harder for her to pick up her baby. At 8 months, he
was a big boy, but not that big. Her workouts on the treadmill were
getting slower and slower. She had to stop running altogether. Still,
her muscles hurt all the time. And she couldn't remember ever feeling
this tired. Not when she was pregnant. Not even when she was a resident.

\includegraphics{https://static01.graylady3jvrrxbe.onion/images/2018/06/03/magazine/03mag-diagnosis-image-2/03mag-diagnosis-image-2-articleLarge.png?quality=75\&auto=webp\&disable=upscale}

After getting this history, Young had the young woman climb up on her
own exam table. He started, as he always did, with the nerves of the
head and face. Normal. Then he tested the strength in her shoulders.
Hold your arms out, like chicken wings, he instructed. Pushing down on
her upper arms, he felt her muscles give way a bit. That was odd. Even
with pain, she should have been able to resist his downward pressure.
The test was designed that way. The rest of the upper-body strength was
normal, but he noticed that the backs of her hands were somewhat red and
dry --- particularly over the knuckles.

Then he moved his attention to her legs. He had her lift her thigh off
the exam table. He applied pressure at the knee and was easily able to
overcome the strength of her hip muscle to force her leg back to the
table. Same on the other side. It is one thing to be tired or even in
pain. But actual and measurable weakness worried him.

\hypertarget{an-inflammatory-disorder-of-the-muscles}{%
\subsection{\texorpdfstring{\textbf{An Inflammatory Disorder of the
Muscles}}{An Inflammatory Disorder of the Muscles}}\label{an-inflammatory-disorder-of-the-muscles}}

His exam and the abnormal labs suggested that she had a type of myositis
--- an inflammatory disorder of the muscles. That her weakness was
limited to the largest joints, the shoulders and the hips, along with
this rash on her hands, made Young think of an unusual disorder called
dermatomyositis (DM), which causes inflammation of the skin (derma) and
the muscles (myocytes). It results from an immune system gone haywire,
attacking its own cells rather than invading organisms. It was a
serious, sometimes deadly disease, although there were now effective
treatments. The patient was relieved to finally be able to put a name to
this terrible fatigue and weakness --- but that relief came with a new
concern. DM is usually a disease of middle age. When young people get
DM, it sometimes means they have cancer as well. So first they had to
confirm she had DM. And if she did, they would have to look for cancer.

Young, however, would be out of town the next week. Rather than wait, he
wanted her to follow up with one of his colleagues, Dr. Floranne Ernste,
a rheumatologist, after the New Year holiday, when the clinic reopened.
She was able to see the rheumatologist, who quickly confirmed what the
neurologist had already noted. This woman who seemed so robust was
actually weak. The patient found it mind-boggling that Ernste, a tiny
woman, could so easily overcome her own much-worked-at strength. This
probably was dermatomyositis.

\hypertarget{enlarged-lymph-nodes}{%
\subsection{\texorpdfstring{\textbf{Enlarged Lymph
Nodes}}{Enlarged Lymph Nodes}}\label{enlarged-lymph-nodes}}

Then something else happened. Over the weekend, the patient noticed an
enlarging lymph node on the left side of her neck. She first felt it a
few weeks earlier, but when she showed it to her husband, he thought it
was from some recent virus. Over the course of just the holiday weekend,
the lymph node seemed to double in size beneath her fingers. Now she
could see it in the mirror. She pointed it out to the rheumatologist,
who easily felt the big node. A muscle biopsy would be needed to confirm
the diagnosis of dermatomyositis. A CT scan of her chest, abdomen and
pelvis would be needed to look for an associated cancer. And now the
lymph node in her neck would also need to be biopsied.

The next few days were a blur of appointments and tests. The CT of her
chest showed many enlarged lymph nodes between and around her lungs. It
also revealed a mass in her spleen the size of a Ping-Pong ball. The
spleen is an organ that functions like the largest lymph node in the
body, mostly hidden by the ribs on the upper left side of the abdomen.
The ultrasound of the neck showed other enlarged nodes. The lymph-node
biopsy confirmed that she did have cancer. It was Hodgkin lymphoma. That
was the cause of her enlarged lymph nodes and the splenic mass. It was
probably the trigger for the dermatomyositis too.

\hypertarget{two-diseases-one-treatment}{%
\subsection{\texorpdfstring{\textbf{Two Diseases, One
Treatment}}{Two Diseases, One Treatment}}\label{two-diseases-one-treatment}}

Hodgkin lymphoma is cancer of the antibody-making white blood cells. It
is primarily a cancer of young people, and even in late-stage disease is
quite treatable with chemotherapy. And indeed, the patient started
treatment the following week because her disease appeared to be
progressing rapidly. Ultimately, the DM didn't require treatment.
Because it was caused by the cancer --- in ways that are still not well
understood --- treatment of the cancer treated the DM as well.

The patient finished her chemotherapy last summer. She went back to
seeing patients part time in August. And although she is still working
to get back to her seven-minute mile, she was preparing to run her first
post-treatment race on Memorial Day. She feels lucky to have gotten a
diagnosis so rapidly. Certainly, if you are going to come down with two
deadly diseases, it is good to work at the Mayo Clinic.

Advertisement

\protect\hyperlink{after-bottom}{Continue reading the main story}

\hypertarget{site-index}{%
\subsection{Site Index}\label{site-index}}

\hypertarget{site-information-navigation}{%
\subsection{Site Information
Navigation}\label{site-information-navigation}}

\begin{itemize}
\tightlist
\item
  \href{https://help.nytimes3xbfgragh.onion/hc/en-us/articles/115014792127-Copyright-notice}{©~2020~The
  New York Times Company}
\end{itemize}

\begin{itemize}
\tightlist
\item
  \href{https://www.nytco.com/}{NYTCo}
\item
  \href{https://help.nytimes3xbfgragh.onion/hc/en-us/articles/115015385887-Contact-Us}{Contact
  Us}
\item
  \href{https://www.nytco.com/careers/}{Work with us}
\item
  \href{https://nytmediakit.com/}{Advertise}
\item
  \href{http://www.tbrandstudio.com/}{T Brand Studio}
\item
  \href{https://www.nytimes3xbfgragh.onion/privacy/cookie-policy\#how-do-i-manage-trackers}{Your
  Ad Choices}
\item
  \href{https://www.nytimes3xbfgragh.onion/privacy}{Privacy}
\item
  \href{https://help.nytimes3xbfgragh.onion/hc/en-us/articles/115014893428-Terms-of-service}{Terms
  of Service}
\item
  \href{https://help.nytimes3xbfgragh.onion/hc/en-us/articles/115014893968-Terms-of-sale}{Terms
  of Sale}
\item
  \href{https://spiderbites.nytimes3xbfgragh.onion}{Site Map}
\item
  \href{https://help.nytimes3xbfgragh.onion/hc/en-us}{Help}
\item
  \href{https://www.nytimes3xbfgragh.onion/subscription?campaignId=37WXW}{Subscriptions}
\end{itemize}
