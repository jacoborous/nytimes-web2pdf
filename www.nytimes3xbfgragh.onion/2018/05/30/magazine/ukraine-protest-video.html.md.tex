Who Killed the Kiev Protesters? A 3-D Model Holds the Clues

\url{https://nyti.ms/2xnu5Uk}

\begin{itemize}
\item
\item
\item
\item
\item
\item
\end{itemize}

\includegraphics{https://static01.graylady3jvrrxbe.onion/images/2018/06/03/magazine/03mag-Ukraine-image1/03mag-Ukraine-image1-articleLarge-v2.gif?quality=75\&auto=webp\&disable=upscale}

Sections

\protect\hyperlink{site-content}{Skip to
content}\protect\hyperlink{site-index}{Skip to site index}

\href{/column/at-war}{At War}

\hypertarget{who-killed-the-kiev-protesters-a-3-d-model-holds-the-clues}{%
\section{Who Killed the Kiev Protesters? A 3-D Model Holds the
Clues}\label{who-killed-the-kiev-protesters-a-3-d-model-holds-the-clues}}

A team of civilian investigators used cellphone videos, autopsy reports
and surveillance footage to reconstruct a virtual crime scene.

Credit...Illustration by Michael Houtz. Source materials from SITU
Research.

Supported by

\protect\hyperlink{after-sponsor}{Continue reading the main story}

By Mattathias Schwartz

\begin{itemize}
\item
  May 30, 2018
\item
  \begin{itemize}
  \item
  \item
  \item
  \item
  \item
  \item
  \end{itemize}
\end{itemize}

On Feb. 20, 2014, Evelyn Nefertari, a graduate student living in western
Ukraine, watched as the most violent day in her country's recent history
unfolded. On that morning, paramilitary police forces loyal to President
Viktor Yanukovych
\href{https://www.nytimes3xbfgragh.onion/2014/02/21/world/europe/ukraine.html}{clashed
with} protesters in Kiev, who were demonstrating against the
government's tilt toward Russia and away from economic integration with
Western Europe. When the tear gas and smoke finally cleared, four police
officers and 48 protesters were dead. ``They were highly educated
intellectuals,'' Nefertari told me recently over the phone. ``The nation
paid a very high price for freedom.'' In the aftermath of the
confrontation, she decided to assemble the definitive record of what
happened. ``The whole country was in mourning,'' she remembers. ``I knew
that I should do it.''

Most of the deaths occurred within half an hour along a few hundred feet
of streetscape. The scene unfolded before dozens of cameramen,
smartphones and security cameras. But these recorded fragments from the
day were overshadowed by a fight over what they really showed: The
claims of grief-stricken activists, who blamed the Ukrainian
paramilitary for shooting the protesters,
\href{https://www.nytimes3xbfgragh.onion/video/world/europe/100000002724416/the-ukraine-divide-explained.html}{collided
with denials} from Yanukovych, who would later testify that the killings
were part of a ``planned provocation'' and ``pseudo-operation'' carried
out by the protesters themselves, a U.S.-backed plot to remove him from
power. Pro-Russia sources went even further, pushing the notion that the
Feb. 20 killings were a ``false flag'' operation carried out by snipers
associated with the protesters, or mercenaries from the country of
Georgia, who were said to have shot down from nearby buildings. To this
day, the story continues to circulate on Kremlin-funded media like
Sputnik and RT.

\includegraphics{https://static01.graylady3jvrrxbe.onion/images/2018/06/01/magazine/1-SITU-cover/1-SITU-cover-videoSixteenByNineJumbo1600.jpg}

The killings took place within a few blocks of Kiev's Maidan
Nezalezhnosti, or ``independence square,'' the center of the nationwide
protests. Nefertari began collecting and synchronizing snippets of video
from the internet and from news broadcasts. The task seemed impossible
given that the videos were shot in different places from different
angles and filled with irrelevant noise. Only with great patience, by
picking out sounds and landmarks, could she begin to assign them time
stamps and coordinates and figure out how each related to the others.
The most crucial videos showed civilian protesters in helmets and winter
jackets facing off against masked riflemen who had taken up barricaded
shooting positions. The protesters cowered in groups of five or six
behind makeshift shields; one would suddenly tumble to the ground and be
carried off by comrades on a stretcher. On the first anniversary of the
killings, Nefertari released a 164-minute real-time video on YouTube
showing the standoff from up to nine simultaneous points of view. Within
the first month of being posted, it was watched more than 270,000 times.

Three years later, Nefertari's video trove has turned into the seed of
an even more complicated piece of analysis: a
\href{http://maidan.situplatform.com/}{sophisticated multimedia
presentation} that tries to recreate the deaths of three protesters
using three-dimensional laser scans of the streetscape, ballistics
analysis and autopsy reports. The combination of so many disparate data
sources into a single three-dimensional model has little precedent. The
arduous work of timing and placing individual videos was assisted by
artificial intelligence, which helped organize and synchronize the
enormous quantity of footage. Now assembled on a mini-PC and received as
evidence by a Ukrainian criminal court, the reconstruction project could
prove crucial in the trial of five police officers who, Ukrainian
prosecutors say, are responsible for the killings.

\includegraphics{https://static01.graylady3jvrrxbe.onion/images/2018/06/03/magazine/03mag-Ukraine-image2/03mag-Ukraine-image2-articleLarge-v2.gif?quality=75\&auto=webp\&disable=upscale}

\textbf{The Maidan demonstrations} are one of many cases in which mass
protests, fueled by popular discontent and social media, have threatened
to topple, or at least embarrass, governments around the world. From the
border fence of the Gaza Strip to urban centers in Nicaragua and Turkey,
the response from the international community often hinges on whether
the party in power can effectively make the case that its use of force
was justified. To avoid being cast as authoritarian, governments must do
more than control the crowd; they must control the narrative. Forensic
tools like those used in the Ukraine inquiry can make the difference in
pinning down the truth of what happened. Similar investigations, using
detailed analyses of open-source data, have been conducted into the use
of chemical weapons by the Syrian government and the killing of a
Venezuelan activist.

In Ukraine, the Maidan protests began in a similar way to the Tahrir
Square demonstrations in Egypt and the Occupy Wall Street movement in
the United States --- a few thousand determined protesters battling the
police for a small but highly visible piece of urban terrain. The
anti-Yanukovych crowds toppled a statue of Lenin and cut off the wide
Soviet-era boulevards with piles of debris, pulling up paving stones and
heaping them up into barricades. Russian-backed media quickly set about
framing the protesters as ``fascists'' who posed a threat to Russian
ethnic minorities in eastern Ukraine and Crimea. Ukrainian police
officers tried and failed to scatter the assembly
\href{https://www.nytimes3xbfgragh.onion/2014/01/24/world/europe/ukraine.html}{by
force}, and by December, protesters were occupying and setting fire to
government buildings.

By the third week of February, protesters had spent months living in a
tent city in the heart of Kiev. They built kitchens and brought portable
toilets to the encampment on the central square. Their purpose was to
challenge Yanukovych's rule and to demand reforms: new elections, more
freedom to protest and closer ties with the European Union.

The violence came to a head on Feb. 20, during one of the bloodiest and
most controversial hours of European conflict since the end of the Cold
War. Police officers massed around the protesters, who set their
barricades on fire and tried to march on Ukraine's Parliament.
Protesters threw bricks and Molotov cocktails; the police responded with
tear gas and rubber bullets. A wave of people pushed out of the square
and down Instytutska Street. Defending themselves with helmets and
homemade shields, the protesters were met by barricaded riflemen from
the Berkut, an elite police force loyal to Yanukovych. By the end of the
day, dozens of bodies lay in the streets.

Almost immediately, a disinformation campaign began on social media to
try to reframe the violence. Reporting by The Washington Post has
attributed the effort to the G.R.U., Russia's military-intelligence
agency. On Facebook and the Russian social-media site VKontakte, G.R.U.
operatives created fake accounts, which characterized the Maidan
uprising as a ``coup'' by ``armed nationalists.'' The G.R.U. also set up
online groups that promoted Crimea's secession from Ukraine. The effort,
which also used paid Facebook ads, presaged Russian interference with
the 2016 presidential election in the United States.

Two days after the shootings, Yanukovych fled to Russia. Over the next
few weeks, Ukraine's Parliament held new elections. Pro-Western parties
won at the polls and would later enter into a trade agreement with the
European Union. President Vladimir Putin of Russia claimed that what had
just taken place amounted to a coup. He seized Crimea and made inroads
into Ukraine's eastern provinces. In April, prosecutors in Kiev opened
an investigation into the Feb. 20 killings, searching for someone to
hold responsible for the protesters' deaths. Prosecutors hoped to bring
charges against the unit of the Berkut that appeared to have taken up
shooting positions behind a barricade on Instytutska Street. Most of the
unit's members had fled to Russia. Ukrainian prosecutors tried to
extradite them; the Russian government ignored the request.

The remaining five members of the Berkut unit awaited trial in a Kiev
prison. Their Kalashnikov assault rifles and pump-action shotguns were
discovered at the bottom of a nearby lake, sawed into pieces. In
addition to murder, the officers were accused of terrorism: using
violence to intimidate the population. Because the officers worked as an
organized unit, it wasn't necessary to prove which of them fired the
lethal shots. ``We defend the government elected by the people,'' one
officer told a German TV station. ``None of our commands were illegal,''
he went on. ``We just did our job.'' All five pleaded not guilty. The
bullets that killed many of the protesters had long since disappeared,
and without them it would be impossible to work backward from the bodies
to the individual rifles. The prosecutors would have to find another
way.

\textbf{A few months} after the protests ended, lawyers representing the
families of the dead protesters started working with the prosecution to
try and put together a reconstruction of the shootings that the court
could accept as evidence. They became aware of Nefertari's project and
reached out to her. ``Evelyn's work was the keystone,'' said Pavlo
Dykan, one of the lawyers. ``It allowed this case to proceed.''

In the summer of 2015, Dykan and his colleague Aleksandra Yatsenko
presented Nefertari's video and discussed the challenges of the Maidan
case at a small conference at the Center for Human Rights Science at
Carnegie Mellon University in Pittsburgh. Among the other presenters was
Brad Samuels, a principal of SITU, an architecture firm and research lab
in Brooklyn. The firm combined traditional architecture work with
grant-funded deep dives at the outer limits of human rights research.
Samuels had come to the conference to present a project that SITU did
with
\href{https://www.nytimes3xbfgragh.onion/2018/04/06/arts/design/forensic-architecture-human-rights.html}{Eyal
Weizman}'s forensic-architecture research group on the death of Bassem
Abu Rahmah, a Palestinian protester in the West Bank who was struck in
the chest by a tear-gas canister. Israeli authorities claimed that Abu
Rahmah's death was an accident. SITU reconstructed the event using live
video and ballistics to show that the fatal tear-gas round, which was
supposed to be fired upward at a steep angle, was fired straight at Abu
Rahmah. The case did not lead to an indictment, but SITU's work was
accepted by an Israeli court as evidence. Samuels told me that the
parallels between the Abu Rahmah and Maidan cases were clear.

``The problem was how to take multiple vantage points and put them
together into a coherent analysis,'' he said. ``Maidan was the same
problem set, but on steroids.''

The quantity of raw footage amassed by Nefertari was overwhelming,
running into the thousands of hours. The problem was how to fuse it into
a whole that would persuade the judges and stand up to
cross-examination. Prosecutors would have to do more than cherry-pick a
few convincing moments. Their theory of the case would have to be strong
enough to survive every alternative scenario, from the mysterious
rooftop snipers to the possibility that protesters were killed by
friendly fire. Samuels worked out the basics of a collaboration with
Dykan and Yatsenko over a picnic table. He planned to copy their data
onto his laptop but wound up having to buy an external hard drive when
he saw how much they had.

Multiple cameras recorded the deaths of three protesters whose families
Dykan and Yatsenko were representing, who were selected for the video
presentation. One of the dead was Ihor Dmytriv, a 30-year-old lawyer who
arrived at the protest on Feb. 19. He was crouching behind a makeshift
shield when a bullet pierced it and entered his chest. He fell to the
ground, rolling on his back and clutching his knee. A tall construction
worker in camouflage fatigues rushed forward to help Dmytriv. He was
Andriy Dyhdalovych, a 40-year-old longtime protester who had marched
with Ukrainian veterans of the Afghanistan campaign. He was shot in the
shoulder and died that day in the street. The third victim was Yuriy
Parashchuk, who was 47. He was shot about 15 minutes later on the same
street, in the front of the head. None of the three men were armed.

\textbf{The Maidan reconstruction} is a product of its time, an age when
high-quality video can be recorded from any street corner or citizen's
hand, and when gigabytes of data can easily circulate among experts in
Pittsburgh, Brooklyn and Kiev. The archive the lawyers handed over was
huge --- a folder of more than 400 videos with different naming
conventions and file types. ``This was as robust a data set as we've
ever had the opportunity to work with,'' Samuels says. Nefertari had
spent months going through the footage on her computer, trying to
synchronize the videos and wrangle them together. The Center for Human
Rights Science in Pittsburgh subsequently tried to automate this
process, using an A.I. algorithm that could quickly analyze each file's
audio component and propose possible matches. ``Machine learning is not
a magic bullet,'' says Jay D. Aronson, the center's director. ``It's
just a tool. You still need a lot of human judgment.'' Once the videos
were assembled into a database, SITU narrowed them down to a smaller
number --- fewer than 20 --- that were relevant to the cases. Then,
working with collaborators on the ground in Ukraine, they built a
virtual model of Instytutska Street. The first version, created using
existing site surveys, wasn't sufficiently detailed, but they learned
that Nefertari had organized surveyors with laser scanners who captured
details at what Samuels calls the ``sub-centimeter level.'' The scan was
so fine that it documented paving-stone patterns and individual leaves
of foliage. The surveyors stood in the streets of Kiev with tall white
poles to pinpoint exactly where each victim fell. The granularity was
necessary to get a fix on not only the victims and the supposed shooters
but also on the people holding the cameras. The 70-gigabyte master
layout, known as a ``point cloud,'' was stitched together from 40
individual scans of the street and its environs.

When a bullet breaks the sound barrier, it produces a small sonic boom
that registers as an audible crack. For people positioned downrange, the
crack arrives a fraction of a second before the thump or blast of the
weapon actually firing. In the Maidan case, SITU enlisted a ballistics
expert to measure the time that elapsed between the cracks and the
thumps. The time difference yielded a maximum and minimum distance
between the shooter and the camera, which SITU rendered as a
doughnut-shaped ``area of interest.''

Using the video archive, SITU positioned the victims' bodies within the
virtual space of the point cloud. Autopsy reports noted the locations of
entry and exit wounds, which were joined by thin red lines and extended,
with a five-degree margin of error, forward into space. Viewed from
overhead, these five-degree cones trim the doughnut-shaped area of
interest down into a narrow segment. In the first two cases, the
segments overlapped with a position that the Berkut were defending
behind a barricade; in the third, the segment overlapped with a position
behind a line of supply trucks. A crucial piece of additional footage
obtained by Nefertari arrived in SITU's office more than a year into the
case. Taken from a surveillance camera at the Ukrainian National Bank,
it clearly shows the Berkut positioned behind their front line. In all
three cases, individual officers can be seen aiming and firing their
rifles during the moments leading to the victims' deaths.

The Ukrainian court could accept the findings only if they were entered
into evidence as a physical object. So SITU loaded its
\href{http://maidan.situplatform.com/}{multimedia presentation} onto an
Intel minicomputer and shipped it to Kiev. At the center of the exhibit
are the three cases, with SITU's analysis presented in three silent
videos, each about five minutes in length. Key snippets of live video
are laid on top of the virtual space as footnotes. The analysis arrives
in a clean and linear presentation, and yet SITU manages to show its
work, letting viewers see the crack-thump audio files and the frames of
video used to establish the positions of the bodies. ``It's pretty
banal,'' Samuels says. ``What we're trying to do is take something
that's already pretty obvious and make it abundantly clear, by putting
it into space.''

``The criminal court has never admitted evidence of such technological
complexity,'' Yatsenko told me. ``It's a milestone.'' There are more
than 100 witnesses left to interview in the Feb. 20 case. It could be
more than a year before the court reaches a verdict. Nefertari,
meanwhile, continues to work on her own project, an even more ambitious
reconstruction of the entire Maidan uprising, with all the victims
included. ``I want people in the future to know the truth,'' she says.
``I don't want rumor turned into facts.''

Advertisement

\protect\hyperlink{after-bottom}{Continue reading the main story}

\hypertarget{site-index}{%
\subsection{Site Index}\label{site-index}}

\hypertarget{site-information-navigation}{%
\subsection{Site Information
Navigation}\label{site-information-navigation}}

\begin{itemize}
\tightlist
\item
  \href{https://help.nytimes3xbfgragh.onion/hc/en-us/articles/115014792127-Copyright-notice}{©~2020~The
  New York Times Company}
\end{itemize}

\begin{itemize}
\tightlist
\item
  \href{https://www.nytco.com/}{NYTCo}
\item
  \href{https://help.nytimes3xbfgragh.onion/hc/en-us/articles/115015385887-Contact-Us}{Contact
  Us}
\item
  \href{https://www.nytco.com/careers/}{Work with us}
\item
  \href{https://nytmediakit.com/}{Advertise}
\item
  \href{http://www.tbrandstudio.com/}{T Brand Studio}
\item
  \href{https://www.nytimes3xbfgragh.onion/privacy/cookie-policy\#how-do-i-manage-trackers}{Your
  Ad Choices}
\item
  \href{https://www.nytimes3xbfgragh.onion/privacy}{Privacy}
\item
  \href{https://help.nytimes3xbfgragh.onion/hc/en-us/articles/115014893428-Terms-of-service}{Terms
  of Service}
\item
  \href{https://help.nytimes3xbfgragh.onion/hc/en-us/articles/115014893968-Terms-of-sale}{Terms
  of Sale}
\item
  \href{https://spiderbites.nytimes3xbfgragh.onion}{Site Map}
\item
  \href{https://help.nytimes3xbfgragh.onion/hc/en-us}{Help}
\item
  \href{https://www.nytimes3xbfgragh.onion/subscription?campaignId=37WXW}{Subscriptions}
\end{itemize}
