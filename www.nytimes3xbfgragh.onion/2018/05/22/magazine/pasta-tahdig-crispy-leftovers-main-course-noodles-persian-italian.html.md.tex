Sections

SEARCH

\protect\hyperlink{site-content}{Skip to
content}\protect\hyperlink{site-index}{Skip to site index}

\href{https://myaccount.nytimes3xbfgragh.onion/auth/login?response_type=cookie\&client_id=vi}{}

\href{https://www.nytimes3xbfgragh.onion/section/todayspaper}{Today's
Paper}

The Crispy Leftovers as the Main Course

\url{https://nyti.ms/2ISIjRF}

\begin{itemize}
\item
\item
\item
\item
\item
\end{itemize}

Advertisement

\protect\hyperlink{after-top}{Continue reading the main story}

Supported by

\protect\hyperlink{after-sponsor}{Continue reading the main story}

\href{/column/magazine-eat}{Eat}

\hypertarget{the-crispy-leftovers-as-the-main-course}{%
\section{The Crispy Leftovers as the Main
Course}\label{the-crispy-leftovers-as-the-main-course}}

\includegraphics{https://static01.graylady3jvrrxbe.onion/images/2018/05/27/magazine/27Eat1/27Eat1-articleLarge.jpg?quality=75\&auto=webp\&disable=upscale}

By Samin Nosrat

\begin{itemize}
\item
  May 22, 2018
\item
  \begin{itemize}
  \item
  \item
  \item
  \item
  \item
  \end{itemize}
\end{itemize}

Like all Iranian kids, I grew up feeling strongly that the best part of
dinner was \emph{tahdig}, the crisp, golden crust that forms at the
bottom of every pot of Persian rice --- and sometimes other dishes too.
My mother could make it of almost anything. When she couldn't find thin
flatbreads, she used large flour tortillas --- conveniently already cut
into circles to line the bottom of the pot for bread \emph{tahdig}. My
favorite was her \emph{spaghetti tahdig}. After coating pasta with
tomato-rich meat sauce, my mom would drizzle the bottom of a nonstick
pot with oil and put it all back in to form a dark crust of tangled
noodles. Once she unmolded it at the table like a cake, my brothers and
I would excitedly cut into it, verbally laying claim to our preferred
pieces. I loved the brownest parts where the tomato had threatened to
burn, yielding sharp, sweet bites.

But then I left home and became a ``real'' cook. Eventually I moved to
Italy and focused on learning everything I could about pasta. I made it
from scratch and obsessed over the regional differences in shapes,
fillings and sauces. I watched in awe as my chef pulled bucatini,
spaghetti and penne from the water before the noodles were quite cooked
and took them to al dente in the ever-present pan of simmering sauce.
One thing I never saw, though, was cooked pasta returning to the heat to
crisp up. And because Italians deeply pride themselves on cooking the
exact right amount of pasta, there was never occasion to try a tahdig
with leftover noodles, either. So like a teenager suddenly refusing to
hug her parents in public, I grew embarrassed by what I once loved and
left it behind.

Until I returned to Italy after 14 years, where I found myself
\emph{recently} at a trattoria in Milan facing a disc of crispy risotto
showered with Parmesan, radiant with the glow of saffron. Unlike
leftover pasta, leftover risotto is viewed by Italians as a gift. Cooks
shape it into balls or stuff it with a pinch of stewed meat or cheese.
Then they bread and deep-fry the fritters until golden brown, yielding
arancini, the indulgent ``little oranges'' I can never resist. But this
disc --- \emph{riso al salto} --- was different. No breading, no
stuffing. Just a thin layer of day-old risotto \emph{alla Milanese},
fried in a nonstick pan in liberal amounts of butter. Then came the
\emph{salto} --- a flick of the wrist to flip the rice --- and more
browning, until it was golden and crisp on not just one side but two. My
first bite brought a classic saffron rice tahdig to mind. But with my
second came, unexpectedly, the wistful memory of the pasta tahdig I
shunned two decades earlier. Perhaps now that I'd finally proved myself,
I could reconcile my Iranian heritage with my Italian training and make
a pasta tahdig that did justice to both.

Using what I learned in Italy, I put a huge pot of water over high heat
and seasoned it generously with salt. While it came to a boil, I
defrosted some roasted-tomato sauce and grated a heap of Parmesan. I
cooked a pound of spaghetti until it was barely al dente, then stirred
it into the waiting pan of sauce and added the cheese. Keeping in mind
that I'd be eating at least some of the tahdig at room temperature, I
added a little more salt --- the colder a dish is when served, the more
highly seasoned it should be.

With what my mom and grandmothers taught me about tahdig in mind, I
pulled out a nonstick pan. I set it over a moderate flame and added a
generous amount of good olive oil to keep the pasta from sticking. Once
I piled in the spaghetti, I shaped it into a mound, as my mom does with
rice, until it cooked down and I could press it into a cake. And then,
with my mom's voice echoing in my head, I turned the heat down, lower
than I would have thought, to prevent the crust from burning, as it was
wont to do with all of that sweet tomato sauce. Recalling the way my
grandmother always turned the pan as her rice cooked to ensure even
browning, I rotated the handle, too. With alternating patience and
anxiety, I waited, then checked, waited, then checked the cake before
determining it was ready. Then I took a breath and flipped it with all
the courage I could muster.

\includegraphics{https://static01.graylady3jvrrxbe.onion/images/2018/05/27/magazine/27Eat3/27Eat3-articleLarge.jpg?quality=75\&auto=webp\&disable=upscale}

Equally relieved and thrilled by its lacquered, shiny crust, I continued
the same way with the second side. And though I could hardly wait to cut
into it when it was done, I refrained, giving the cake a chance to set
so that I could get a clean slice. With a shatter, the sweet, crunchy
crust yielded to a mouthful of perfectly seasoned, perfectly sauced
spaghetti. Traveling through time to the suburban kitchen table around
which I grew up, I thought of my mom, knowing she'd love my double
tahdig. Then, I thought of my Italian chef mentor, who intimately
understands the relationship between taste and nostalgia, and I knew
she'd approve, too.

The next day, I took the remainder to my brother's house, where I didn't
bother to heat it up. Standing at the kitchen counter, we gobbled it up,
garnished with yet more Parmesan. ``This is one of the best things
you've ever cooked,'' my brother said, smiling as he wiped tomato sauce
and olive oil from his lips. ``Almost as good as mom's.''

\textbf{Recipe:}
\href{https://cooking.nytimes3xbfgragh.onion/recipes/1019328-pasta-tahdig}{Pasta
Tahdig}

Advertisement

\protect\hyperlink{after-bottom}{Continue reading the main story}

\hypertarget{site-index}{%
\subsection{Site Index}\label{site-index}}

\hypertarget{site-information-navigation}{%
\subsection{Site Information
Navigation}\label{site-information-navigation}}

\begin{itemize}
\tightlist
\item
  \href{https://help.nytimes3xbfgragh.onion/hc/en-us/articles/115014792127-Copyright-notice}{©~2020~The
  New York Times Company}
\end{itemize}

\begin{itemize}
\tightlist
\item
  \href{https://www.nytco.com/}{NYTCo}
\item
  \href{https://help.nytimes3xbfgragh.onion/hc/en-us/articles/115015385887-Contact-Us}{Contact
  Us}
\item
  \href{https://www.nytco.com/careers/}{Work with us}
\item
  \href{https://nytmediakit.com/}{Advertise}
\item
  \href{http://www.tbrandstudio.com/}{T Brand Studio}
\item
  \href{https://www.nytimes3xbfgragh.onion/privacy/cookie-policy\#how-do-i-manage-trackers}{Your
  Ad Choices}
\item
  \href{https://www.nytimes3xbfgragh.onion/privacy}{Privacy}
\item
  \href{https://help.nytimes3xbfgragh.onion/hc/en-us/articles/115014893428-Terms-of-service}{Terms
  of Service}
\item
  \href{https://help.nytimes3xbfgragh.onion/hc/en-us/articles/115014893968-Terms-of-sale}{Terms
  of Sale}
\item
  \href{https://spiderbites.nytimes3xbfgragh.onion}{Site Map}
\item
  \href{https://help.nytimes3xbfgragh.onion/hc/en-us}{Help}
\item
  \href{https://www.nytimes3xbfgragh.onion/subscription?campaignId=37WXW}{Subscriptions}
\end{itemize}
