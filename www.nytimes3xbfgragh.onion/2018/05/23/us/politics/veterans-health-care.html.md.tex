Sections

SEARCH

\protect\hyperlink{site-content}{Skip to
content}\protect\hyperlink{site-index}{Skip to site index}

\href{https://www.nytimes3xbfgragh.onion/section/politics}{Politics}

\href{https://myaccount.nytimes3xbfgragh.onion/auth/login?response_type=cookie\&client_id=vi}{}

\href{https://www.nytimes3xbfgragh.onion/section/todayspaper}{Today's
Paper}

\href{/section/politics}{Politics}\textbar{}Senate Sends Major Overhaul
of Veterans Health Care to Trump

\url{https://nyti.ms/2GJpxH3}

\begin{itemize}
\item
\item
\item
\item
\item
\end{itemize}

Advertisement

\protect\hyperlink{after-top}{Continue reading the main story}

Supported by

\protect\hyperlink{after-sponsor}{Continue reading the main story}

\hypertarget{senate-sends-major-overhaul-of-veterans-health-care-to-trump}{%
\section{Senate Sends Major Overhaul of Veterans Health Care to
Trump}\label{senate-sends-major-overhaul-of-veterans-health-care-to-trump}}

\includegraphics{https://static01.graylady3jvrrxbe.onion/images/2018/05/24/us/24dc-veterans-print/24dc-veterans-articleLarge.jpg?quality=75\&auto=webp\&disable=upscale}

By \href{https://www.nytimes3xbfgragh.onion/by/nicholas-fandos}{Nicholas
Fandos}

\begin{itemize}
\item
  May 23, 2018
\item
  \begin{itemize}
  \item
  \item
  \item
  \item
  \item
  \end{itemize}
\end{itemize}

WASHINGTON --- The Senate gave final passage on Wednesday to a
multibillion-dollar revamp of the veterans health care system,
consolidating seven Veterans Affairs Department health programs into one
and making it far easier for veterans to take their benefits to private
doctors for care.

The legislation, which passed 92 to 5, also expands popular stipends to
family caregivers of veterans who served during the Vietnam War era or
after. And it establishes a nine-member commission to study the
department's current infrastructure to determine where its health system
should expand and contract.

The comprehensive bill had been a year in the making, and ultimately won
the support of Republicans and most Democrats. The House passed it last
week, 347 to 70, and President Trump plans to sign it into law.

The bill rounds out an ambitious legislative agenda on veterans issues
that has bridged the administrations of President Barack Obama and Mr.
Trump, and largely united moderates in both parties who set out to make
major changes to the department after a
\href{https://www.nytimes3xbfgragh.onion/2014/05/29/us/va-report-confirms-improper-waiting-lists-at-phoenix-center.html}{2014
scandal over the manipulation of data} on patient wait times. Mr. Trump
has already signed laws that
\href{https://www.nytimes3xbfgragh.onion/2017/06/23/us/politics/trump-veterans-accountability-bill.html}{make
it easier for the department to remove and fire employees} and for
veterans to appeal benefits decisions, as well as
\href{https://www.nytimes3xbfgragh.onion/2017/08/02/us/politics/gi-bill-senate-expansion.html}{a
rewrite of the G.I. Bill for post-9/11 veterans}.

Wednesday's vote is ``the last piece of a great mosaic to reform the
benefits for our veterans to make them contemporary with the 21st
century,'' said Senator Johnny Isakson of Georgia, the chairman of the
Senate Veterans' Affairs Committee. Dozens of military and veterans
groups, including the largest congressionally chartered organizations,
such as the American Legion and Veterans of Foreign Wars, also touted
the changes.

But beneath the surface, deep mistrust remains over the Trump
administration's mission to increase the use of private care. Many of
the largest veterans groups, as well as Democrats and some moderate
Republicans, fear that the White House's push to unfetter veterans'
ability to choose their care is a backdoor effort to tip the scales in
favor of private medicine and to starve the federal government's
second-largest department and its vast government-run health system.

Liberals, including the Democratic Party's top lawmaker on the House
Veterans' Affairs Committee and a former chairman of the Senate
committee, warned that because the bill lacks a long-term funding
source, the cost of the program --- estimated to be roughly \$50 billion
over five years --- could end up cannibalizing other pieces of the
department's budget.

``It provides nothing to fill the vacancies at the V.A. That is wrong,''
said Senator Bernie Sanders, independent of Vermont, who ran the
committee in the aftermath of the wait-time scandal, when lawmakers
first created the so-called Veterans Choice Program to relieve pressure
on the system. ``My fear is that this bill will open the door to the
draining, year after year, of much-needed resources from the V.A.''

With veterans organizations supporting the measure, labor groups,
including the American Federation of Government Employees, which
represents 260,000 department employees, have led the opposition,
arguing that it will allow the outsourcing of key department health
services without addressing the department's staffing shortages.

``Too much is at stake for veterans, their families and everyone who
benefits from the V.A.'s extraordinary accomplishments to succumb to
political pressures to hurriedly pass potentially damaging changes with
many unknown consequences,'' the groups wrote Tuesday in
\href{https://www.afge.org/globalassets/documents/generalreports/2018/5-22-18-joint-letter-to-senate-opposing-s-2372-the-va-mission-act.pdf}{a
joint letter}.

But advocates of the bill, including Democrats and groups like the
Legion and the V.F.W. that are staunchly opposed to privatization, say
they have merely arrived at a sensible and much-needed compromise. The
Choice program, they say, was designed with good intentions but has
turned into a bureaucratic tangle, burdening department doctors,
straining its balance sheets and delivering little of the relief it was
meant to provide for veterans. They point out that earlier legislation
had already made large investments in improving the department's own
health system.

``The Choice program has been a wreck,'' said Senator Jon Tester of
Montana, the top Democrat on the Senate committee. ``Every veteran up
here will tell you that.''

But, he added, ``The best defense against any effort to privatize the
V.A. or send veterans in a wholesale factor to the private sector is to
make sure the V.A. is living up to its promise.''

Concerned Veterans for America, an advocacy group backed by the
billionaire conservative activists Charles G. and David H. Koch, has led
the push for veterans to have complete control over where they use their
benefits, and it celebrated the vote. Its executive director, Dan
Caldwell, called it ``a big win for those who want to see the V.A.
better integrate with the private sector.''

The new, combined program, called the
\href{https://veterans.house.gov/uploadedfiles/va_mission_act_summary.pdf}{Veterans
Community Care Program}, attempts to integrate the department's 1,300
hospitals and clinics with credentialed private doctors. It also expands
the circumstances under which veterans can elect to go to a private
doctor at government expense.

The current Choice program allows patients facing a wait of at least 30
days to receive treatment to seek private care funded by the government.
Veterans who must travel at least 40 miles can also seek private care.
The new bill scraps those standards.

Instead, eligibility is determined by a mix of factors, including wait
times, distance, the service ratings of government facilities and the
availability of specialists. Veterans will be expected to consult with
their Veterans Affairs Department primary care doctor about the best
course of treatment.

Veterans who have used the department's health system within two years
will also be allowed walk-in visits to private clinics, in some cases
without co-payments.

The department currently sends more than one-third of appointments to
private providers, and the numbers have been ticking up over the past
year. It will have substantial leeway in carrying out the new program
over the next year.

The expansion of caregiver benefits has been a long-term priority of
veterans advocates. The department currently only pays stipends to
family caregivers of veterans who served after Sept. 11, 2001.

The creation of a commission to study the department's physical
footprint and distribution of resources is more divisive. Conservatives
have long fought for the commission as a way to cut what they suspect
are unneeded or redundant facilities. But Democrats and some veterans
advocates fear the panel, in conjunction with the new private care
program, could be used to justify a downsizing of the federal health
system.

The bill includes a long list of other, smaller provisions meant to
improve access to care. It eases some restrictions of the department's
use of telemedicine and establishes a pilot program to test the use of
mobile care teams who will travel to rural or otherwise underserved
areas.

The legislation includes modest incentives to try to build up the
department's work force, including loan repayment and recruitment and
retention bonuses. There are currently more than 30,000 unfilled
positions in a work force of more than 360,000.

Advertisement

\protect\hyperlink{after-bottom}{Continue reading the main story}

\hypertarget{site-index}{%
\subsection{Site Index}\label{site-index}}

\hypertarget{site-information-navigation}{%
\subsection{Site Information
Navigation}\label{site-information-navigation}}

\begin{itemize}
\tightlist
\item
  \href{https://help.nytimes3xbfgragh.onion/hc/en-us/articles/115014792127-Copyright-notice}{©~2020~The
  New York Times Company}
\end{itemize}

\begin{itemize}
\tightlist
\item
  \href{https://www.nytco.com/}{NYTCo}
\item
  \href{https://help.nytimes3xbfgragh.onion/hc/en-us/articles/115015385887-Contact-Us}{Contact
  Us}
\item
  \href{https://www.nytco.com/careers/}{Work with us}
\item
  \href{https://nytmediakit.com/}{Advertise}
\item
  \href{http://www.tbrandstudio.com/}{T Brand Studio}
\item
  \href{https://www.nytimes3xbfgragh.onion/privacy/cookie-policy\#how-do-i-manage-trackers}{Your
  Ad Choices}
\item
  \href{https://www.nytimes3xbfgragh.onion/privacy}{Privacy}
\item
  \href{https://help.nytimes3xbfgragh.onion/hc/en-us/articles/115014893428-Terms-of-service}{Terms
  of Service}
\item
  \href{https://help.nytimes3xbfgragh.onion/hc/en-us/articles/115014893968-Terms-of-sale}{Terms
  of Sale}
\item
  \href{https://spiderbites.nytimes3xbfgragh.onion}{Site Map}
\item
  \href{https://help.nytimes3xbfgragh.onion/hc/en-us}{Help}
\item
  \href{https://www.nytimes3xbfgragh.onion/subscription?campaignId=37WXW}{Subscriptions}
\end{itemize}
