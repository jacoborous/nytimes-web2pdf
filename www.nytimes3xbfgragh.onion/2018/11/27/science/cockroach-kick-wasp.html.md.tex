Sections

SEARCH

\protect\hyperlink{site-content}{Skip to
content}\protect\hyperlink{site-index}{Skip to site index}

\href{https://www.nytimes3xbfgragh.onion/section/science}{Science}

\href{https://myaccount.nytimes3xbfgragh.onion/auth/login?response_type=cookie\&client_id=vi}{}

\href{https://www.nytimes3xbfgragh.onion/section/todayspaper}{Today's
Paper}

\href{/section/science}{Science}\textbar{}The Wasp Wants a Zombie. The
Cockroach Says `No' With a Karate Kick.

\url{https://nyti.ms/2DNKVgH}

\begin{itemize}
\item
\item
\item
\item
\item
\end{itemize}

Advertisement

\protect\hyperlink{after-top}{Continue reading the main story}

Supported by

\protect\hyperlink{after-sponsor}{Continue reading the main story}

sciencetake

\hypertarget{the-wasp-wants-a-zombie-the-cockroach-says-no-with-a-karate-kick}{%
\section{The Wasp Wants a Zombie. The Cockroach Says `No' With a Karate
Kick.}\label{the-wasp-wants-a-zombie-the-cockroach-says-no-with-a-karate-kick}}

\includegraphics{https://static01.graylady3jvrrxbe.onion/images/2018/12/04/autossell/Figure-9/Figure-9-videoSixteenByNineJumbo1600.jpg}

\href{https://www.nytimes3xbfgragh.onion/by/nicholas-st-fleur}{\includegraphics{https://static01.graylady3jvrrxbe.onion/images/2018/02/20/multimedia/author-nicholas-st-fleur/author-nicholas-st-fleur-thumbLarge.jpg}}

By
\href{https://www.nytimes3xbfgragh.onion/by/nicholas-st-fleur}{Nicholas
St. Fleur}

\begin{itemize}
\item
  Nov. 27, 2018
\item
  \begin{itemize}
  \item
  \item
  \item
  \item
  \item
  \end{itemize}
\end{itemize}

Nothing sends people scrambling for a boot faster than the sight of a
scurrying cockroach. But to the pests, there are far scarier dangers out
there.

True terror? That's getting zombified --- and then eaten alive.

When some unlucky American cockroaches encounter the emerald jewel wasp,
the wasp delivers a paralyzing sting to the roach's body. Then, with
surgeon-like precision, it injects a mind-altering cocktail into the
roach's brain. The roach, now a zombie slave, is forced to cater to the
wasp's every whim. But the wasp has only one desire: to reproduce.

Like a handler leading a horse, the wasp grabs hold of the roach's
antenna and steers it into a hole. There, it lays an egg on the roach
that eventually hatches into a hungry larva that chows down on the
cockroach. When the baby matures, it bursts from the roach's chest ready
to continue the gruesome ritual.

``It's kind of straight out of Alien,'' said
\href{https://www.google.com/search?q=Kenneth+Catania+a+biologist+from+Vanderbilt+University+in+Tennessee\&rlz=1C5CHFA_enUS779US779\&oq=Kenneth+Catania+a+biologist+from+Vanderbilt+University+in+Tennessee\&aqs=chrome..69i57.333j0j1\&sourceid=chrome\&ie=UTF-8}{Kenneth
Catania}, a biologist from Vanderbilt University in Tennessee, ``and
it's about the only thing I can think of that'll make you feel sorry for
a cockroach.''

But as Dr. Catania has studied, some roaches defend themselves from the
wasps with a swift and powerful karate kick.

Using high-speed cameras, Dr. Catania recorded scuffles between adult
roaches and wasps in his lab and documented the cockroaches' defensive
techniques in a paper published last month in the journal
\href{https://www.karger.com/Article/FullText/490341}{Brain, Behavior
and Evolution}.

\textbf{\emph{{[}}\href{http://on.fb.me/1paTQ1h}{\emph{Like the Science
Times page on Facebook.}}} ****** \emph{\textbar{} Sign up for the}
\textbf{\href{http://nyti.ms/1MbHaRU}{\emph{Science Times
newsletter.}}\emph{{]}}}

First, some roaches detected an intruding wasp with their antenna. Next,
they raised themselves up, as if on stilts, and lifted their hind legs.
And then they waited until just the right moment when the wasp tried to
strike. In that instant, the roaches knocked back the would-be attacker
with brutal kicks.

``It reminded me in slow motion of one of those old Batman and Robin
videos where you see the words `Pow' appear on the screen,'' said Dr.
Catania.

Roaches' legs are covered in spikes that act like barbed wire, making
the hit extra damaging.

Although the strike doesn't kill the wasp, it makes the attacker back
off. Dr. Catania found that 63 percent of the cockroaches that defended
themselves were able to avoid getting stung by the wasp. But the roaches
that didn't put up a fight nearly always got stung.

Scientists don't know why not every roach fights back, and Dr. Catania
would like to study whether the kick evolved specifically to combat the
wasp or as a general method of self-defense.

Whatever the answer, it seems that striking the first blow keeps the
American cockroach from becoming a mindless zombie.

Advertisement

\protect\hyperlink{after-bottom}{Continue reading the main story}

\hypertarget{site-index}{%
\subsection{Site Index}\label{site-index}}

\hypertarget{site-information-navigation}{%
\subsection{Site Information
Navigation}\label{site-information-navigation}}

\begin{itemize}
\tightlist
\item
  \href{https://help.nytimes3xbfgragh.onion/hc/en-us/articles/115014792127-Copyright-notice}{©~2020~The
  New York Times Company}
\end{itemize}

\begin{itemize}
\tightlist
\item
  \href{https://www.nytco.com/}{NYTCo}
\item
  \href{https://help.nytimes3xbfgragh.onion/hc/en-us/articles/115015385887-Contact-Us}{Contact
  Us}
\item
  \href{https://www.nytco.com/careers/}{Work with us}
\item
  \href{https://nytmediakit.com/}{Advertise}
\item
  \href{http://www.tbrandstudio.com/}{T Brand Studio}
\item
  \href{https://www.nytimes3xbfgragh.onion/privacy/cookie-policy\#how-do-i-manage-trackers}{Your
  Ad Choices}
\item
  \href{https://www.nytimes3xbfgragh.onion/privacy}{Privacy}
\item
  \href{https://help.nytimes3xbfgragh.onion/hc/en-us/articles/115014893428-Terms-of-service}{Terms
  of Service}
\item
  \href{https://help.nytimes3xbfgragh.onion/hc/en-us/articles/115014893968-Terms-of-sale}{Terms
  of Sale}
\item
  \href{https://spiderbites.nytimes3xbfgragh.onion}{Site Map}
\item
  \href{https://help.nytimes3xbfgragh.onion/hc/en-us}{Help}
\item
  \href{https://www.nytimes3xbfgragh.onion/subscription?campaignId=37WXW}{Subscriptions}
\end{itemize}
