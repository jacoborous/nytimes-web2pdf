Sections

SEARCH

\protect\hyperlink{site-content}{Skip to
content}\protect\hyperlink{site-index}{Skip to site index}

\href{https://www.nytimes3xbfgragh.onion/section/world/europe}{Europe}

\href{https://myaccount.nytimes3xbfgragh.onion/auth/login?response_type=cookie\&client_id=vi}{}

\href{https://www.nytimes3xbfgragh.onion/section/todayspaper}{Today's
Paper}

\href{/section/world/europe}{Europe}\textbar{}`Better to Drown': A Greek
Refugee Camp's Epidemic of Misery

\url{https://nyti.ms/2O4u1kk}

\begin{itemize}
\item
\item
\item
\item
\item
\item
\end{itemize}

Advertisement

\protect\hyperlink{after-top}{Continue reading the main story}

Supported by

\protect\hyperlink{after-sponsor}{Continue reading the main story}

\hypertarget{better-to-drown-a-greek-refugee-camps-epidemic-of-misery}{%
\section{`Better to Drown': A Greek Refugee Camp's Epidemic of
Misery}\label{better-to-drown-a-greek-refugee-camps-epidemic-of-misery}}

\includegraphics{https://static01.graylady3jvrrxbe.onion/images/2018/10/03/world/03moria1/merlin_144528777_b0c33e80-1991-4b80-a32c-9bc93e0d8136-articleLarge.jpg?quality=75\&auto=webp\&disable=upscale}

By \href{https://www.nytimes3xbfgragh.onion/by/patrick-kingsley}{Patrick
Kingsley}

\begin{itemize}
\item
  Oct. 2, 2018
\item
  \begin{itemize}
  \item
  \item
  \item
  \item
  \item
  \item
  \end{itemize}
\end{itemize}

\href{https://www.nytimes3xbfgragh.onion/es/2018/10/04/refugiados-moria-grecia}{Leer
en español}

MORIA, Greece --- He survived torture in Congo, and a perilous boat
journey from Turkey. But Michael Tamba, a former Congolese political
prisoner, came closest to death only after he had supposedly found
sanctuary at Europe's biggest refugee camp.

Stuck for months at the camp on the Greek island of Lesbos, Mr. Tamba,
31, tried to end his life by drinking a bottle of bleach. The trigger:
\href{https://www.nytimes3xbfgragh.onion/2018/03/29/world/europe/greece-lesbos-migrant-crisis-moria.html}{Camp
Moria} itself.

``Eleven months in Moria, Moria, Moria,'' said Mr. Tamba, who survived
after being rushed to hospital. ``It's very traumatic.''

Mr. Tamba's experience has become a common one at Moria, a camp of
around 9,000 people living in a space designed for just 3,100, where
squalid conditions and an inscrutable asylum process have led to what
aid groups describe as a mental health crisis.

The overcrowding is so extreme that asylum seekers spend as much as 12
hours a day waiting in line for food that is sometimes moldy. Last week,
there were about 80 people for each shower, and around 70 per toilet,
with aid workers complaining about raw sewage leaking into tents where
children are living. Sexual assaults, knife attacks and suicide attempts
are common.

The conditions have fueled accusations that the camp has been left to
fester in order to deter migration and that European Union funds
provided to help Greece deal with asylum seekers are being misused. In
late September, the European Union's anti-fraud agency announced an
investigation.

At the height of the European migrant crisis in 2015, Moria was merely a
way station as tens of thousands of asylum seekers --- many fleeing wars
in Syria, Iraq and Afghanistan --- poured through the region on their
way to northern Europe. Then, the numbers were so great, the migrants
were effectively waved through.

Gradually, European Union countries tried to gain control over the
situation by closing internal borders and building camps at the bloc's
periphery in places like Lesbos, where so many of the refugees arrived.
Now they are stuck here.

Today, Moria is the most visible symbol of the hardening European stance
toward migrants --- one that has drastically reduced unauthorized
migration, but at what critics see as a deep moral and humanitarian
cost.

\includegraphics{https://static01.graylady3jvrrxbe.onion/images/2018/10/03/world/01moria-3-print/merlin_144526788_28d7c915-c319-4cfc-932d-b16d03ef96e1-articleLarge.jpg?quality=75\&auto=webp\&disable=upscale}

Image

Afghan migrants among the makeshift tents just outside Moria. Designed
for 3,100 people, the camp's population has swelled to 9,000. Many live
under tarpaulins beyond its fences.Credit...Mauricio Lima for The New
York Times

Outside Europe, the European Union has courted authoritarian governments
in
\href{https://www.nytimes3xbfgragh.onion/2016/03/19/world/europe/european-union-turkey-refugees-migrants.html}{Turkey},
\href{https://www.nytimes3xbfgragh.onion/2018/04/22/world/africa/migration-european-union-sudan.html}{Sudan}
and
\href{https://www.nytimes3xbfgragh.onion/aponline/2018/09/20/world/europe/ap-eu-europe-migrants.html}{Egypt},
while Italy has negotiated with
\href{https://www.nytimes3xbfgragh.onion/2017/09/17/world/europe/italy-libya-migrant-crisis.html}{warlords
in Libya}, in
\href{https://www.nytimes3xbfgragh.onion/interactive/2018/06/27/world/europe/europe-migrant-crisis-change.html}{a
successful effort} to stem the flow of migrants toward the
Mediterranean.

Inside Europe itself, those who still make it to the Greek islands ---
about 23,000 have arrived this year, down from 850,000 in 2015 --- must
now stay at camps like Moria until their cases are settled. It can take
as long as two years before the asylum seekers are either sent home or
move on.

``I have been in some pretty horrendous camps and situations,'' said
Louise Roland-Gosselin, who is head of mission in Greece for Doctors
Without Borders and spent five years in crisis zones in Congo and South
Sudan. ``I have to say that Moria is the camp in which I've seen the
highest level of suffering.''

The group's lead psychiatrist on Lesbos, Alessandro Barberio, said he
had never seen such overwhelming numbers of severe mental health cases.
Of the roughly 120 people his team has the capacity to treat, the vast
majority have been prescribed anti-psychotic medication.

``Moria has become a trigger for an acute expression of psychosis and
post-traumatic stress disorder,'' Dr. Barberio said.

The International Rescue Committee, an aid group with a smaller presence
on the island, said that
\href{https://www.rescue.org/sites/default/files/document/3153/unprotectedunsupporteduncertain.pdf}{just
under a third} of the 126 people its psychosocial workers have assessed
at Moria since March have attempted suicide.

The majority of the camp's residents are Syrian, Iraqi and Afghan
refugees, many of whom have suffered wartime traumas that are then
exacerbated by the overcrowded and static conditions, which fuel their
despair.

Image

A doctor checking an 11-month-old Iraqi baby with a respiratory
infection at the Doctors Without Borders outpost.Credit...Mauricio Lima
for The New York Times

Image

Michael Tamba, a former Congolese political prisoner, stuck for months
at the camp, tried to end his life.Credit...Mauricio Lima for The New
York Times

As in Mr. Tamba's case, few suicide attempts result in death, conditions
being so crowded that they are usually discovered quickly, aid workers
say. But the damage can be lasting --- Mr. Tamba's attempt scarred his
stomach, which still pains him.

After finally being identified as a vulnerable case, Mr. Tamba, who was
said he was arrested at a political protest in Congo, has been allowed
to move to another camp on the Greek mainland. But conditions there are
not much better, and Mr. Tamba worries whether he will be able to access
medication now that he has been moved.

The Greek government says it plans to move a third of residents to the
mainland in the coming weeks. The camp's population in September was so
large that many were living under tarpaulins on a windswept spillover
site beyond its fences.

Since 2015, the camp has often relied on smaller overflow encampments,
but in five visits over the past three years, I had never seen the
spillover extend so far.

\hypertarget{a-life-of-lines}{%
\subsection{A life of lines}\label{a-life-of-lines}}

Rahmuddin Ashrafi, an Afghan farmer, arrived here in June with his wife,
Sohaela, and their three small children. In Afghanistan, Mr. Ashrafi,
34, said their house and land were destroyed in fighting between the
Taliban and the Afghan army. At Moria, the five of them now share a
small two-person tent.

The family's typical day begins at 4 a.m., when Mr. Ashrafi joins a line
for water and bread that is usually served four hours later at 8 a.m. At
around 9:30 a.m., he joins the line again for lunch, which tends to
arrive after another four hours of waiting. Two hours later, he joins
another four-hour line for dinner.

Image

At the height of the refugee crisis in 2015, Moria was a stepping stone
to northern Europe. Today refugees can spend a year or more there while
their cases are decided.Credit...Mauricio Lima for The New York Times

On the days when he needs to line up for official paperwork, or to visit
the doctor --- his three-year-old daughter was recently hospitalized
with appendicitis --- he sometimes has to skip meals altogether, or rely
on leftovers from other Afghans.

``Before, I thought that Greece would be one of the best places to
live,'' Mr. Ashrafi said. ``Now I feel it would have been better to
drown while crossing the sea.''

Few residents feel safe. In the privacy of his tent, a 25-year-old Iraqi
student rolled up his sweatshirt to reveal a recent set of stab wounds.
These were the result of an attack by other campers, he said, asking
that his name not be published for fear of further reprisals.

Sexual violence is also common. The International Rescue Committee has
assessed over 70 people since March who have reported being sexually
abused at the camp. Women say they are wary of walking alone through the
camp at night.

Compounding these issues, many residents feel trapped in an endlessly
bureaucratic asylum application process that they do not fully
understand.

Mr. Ashrafi had to miss a scheduled interview because he had to take his
daughter to the hospital. Now he must wait months for another
appointment.

Image

Unaccompanied minors. The majority of the camp's residents are Syrian,
Iraqi and Afghan refugees, many of whom have suffered wartime traumas
that are then exacerbated by the overcrowding at Moria.Credit...Mauricio
Lima for The New York Times

Image

A Syrian family taking pictures in front of the ship that goes to Athens
at the port of Mytilene in March.Credit...Mauricio Lima for The New York
Times

Those arriving at the camp in the coming weeks can expect to wait until
at least March for an interview, said Philip Worthington, managing
director of
\href{https://www.europeanlawyersinlesvos.eu/about/}{European Lawyers in
Lesvos}, a legal aid group operating on the island.

Should an applicant be rejected, there are currently no state-sponsored
lawyers to assist them with their appeal, in contravention of both Greek
and European law, Mr. Worthington said.

\hypertarget{following-the-money}{%
\subsection{Following the money}\label{following-the-money}}

There is growing acrimony --- and now an investigation --- over why the
camp is so bad when so much money has been provided by the European
Union to help improve the Greek asylum system since migration levels
started to rise in 2014.

The European Union has allocated nearly 1.62 billion euros --- about
\$1.9 billion --- to the Greek asylum effort over the past half-decade,
of which €1.1 billion has already been paid out, according to data
supplied to The New York Times by the bloc. Over 20 government
departments and nongovernmental organizations have received European
Union money, a piecemeal approach in which no institution has complete
oversight over how the money is spent.

A spokesman for the Greek migration ministry, Alexis Bouzis, denied any
financial misuse on the part of the government, and attributed the
situation to a small rise in migration flows over the summer, which led
to a backlog.

Image

Afghan youths building a shelter in the makeshift camp.Credit...Mauricio
Lima for The New York Times

Image

Migrants buying food at a stall set up between the makeshift camp and
the military-run Moria camp.Credit...Mauricio Lima for The New York
Times

``No one could have predicted it,'' Mr. Bouzis said.

Aid groups have been warning of a need for expanded facilities for
several years, however. For some, the failure to improve the camp and
hasten the asylum process suggests neglect on the part of the Greek
government and the European Union departments that fund it.

This perception was reinforced at a private meeting of Greek and
European Union officials and aid workers on Sept. 4. According to two
people present, a British official representing the European Commission,
the bloc's civil service, suggested keeping living standards low at
Moria in order to deter future migration to Greece.

Asked to comment on this suggestion, a spokeswoman for the commission
said that it did not represent the service's official stance.

``It is in no one's interest that the conditions in Moria remain as they
are,'' said Natasha Bertaud, a commission spokeswoman, ``which is why
the commission is doing everything in its power to help the Greek
authorities improve them.''

Image

A road near Eftalou beach, where thousands of migrants arrived after
crossing the Aegean Sea from Turkey in 2015.Credit...Mauricio Lima for
The New York Times

Advertisement

\protect\hyperlink{after-bottom}{Continue reading the main story}

\hypertarget{site-index}{%
\subsection{Site Index}\label{site-index}}

\hypertarget{site-information-navigation}{%
\subsection{Site Information
Navigation}\label{site-information-navigation}}

\begin{itemize}
\tightlist
\item
  \href{https://help.nytimes3xbfgragh.onion/hc/en-us/articles/115014792127-Copyright-notice}{©~2020~The
  New York Times Company}
\end{itemize}

\begin{itemize}
\tightlist
\item
  \href{https://www.nytco.com/}{NYTCo}
\item
  \href{https://help.nytimes3xbfgragh.onion/hc/en-us/articles/115015385887-Contact-Us}{Contact
  Us}
\item
  \href{https://www.nytco.com/careers/}{Work with us}
\item
  \href{https://nytmediakit.com/}{Advertise}
\item
  \href{http://www.tbrandstudio.com/}{T Brand Studio}
\item
  \href{https://www.nytimes3xbfgragh.onion/privacy/cookie-policy\#how-do-i-manage-trackers}{Your
  Ad Choices}
\item
  \href{https://www.nytimes3xbfgragh.onion/privacy}{Privacy}
\item
  \href{https://help.nytimes3xbfgragh.onion/hc/en-us/articles/115014893428-Terms-of-service}{Terms
  of Service}
\item
  \href{https://help.nytimes3xbfgragh.onion/hc/en-us/articles/115014893968-Terms-of-sale}{Terms
  of Sale}
\item
  \href{https://spiderbites.nytimes3xbfgragh.onion}{Site Map}
\item
  \href{https://help.nytimes3xbfgragh.onion/hc/en-us}{Help}
\item
  \href{https://www.nytimes3xbfgragh.onion/subscription?campaignId=37WXW}{Subscriptions}
\end{itemize}
