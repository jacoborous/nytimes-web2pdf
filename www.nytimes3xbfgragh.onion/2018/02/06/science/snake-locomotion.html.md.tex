Sections

SEARCH

\protect\hyperlink{site-content}{Skip to
content}\protect\hyperlink{site-index}{Skip to site index}

\href{https://www.nytimes3xbfgragh.onion/section/science}{Science}

\href{https://myaccount.nytimes3xbfgragh.onion/auth/login?response_type=cookie\&client_id=vi}{}

\href{https://www.nytimes3xbfgragh.onion/section/todayspaper}{Today's
Paper}

\href{/section/science}{Science}\textbar{}How the Snake Pours Its Way
Across the Ground

\url{https://nyti.ms/2EkkFeh}

\begin{itemize}
\item
\item
\item
\item
\item
\item
\end{itemize}

Advertisement

\protect\hyperlink{after-top}{Continue reading the main story}

Supported by

\protect\hyperlink{after-sponsor}{Continue reading the main story}

\href{/column/sciencetake}{ScienceTake}

\hypertarget{how-the-snake-pours-its-way-across-the-ground}{%
\section{How the Snake Pours Its Way Across the
Ground}\label{how-the-snake-pours-its-way-across-the-ground}}

\includegraphics{https://static01.graylady3jvrrxbe.onion/images/2018/02/09/science/06SCI-SCITAKE/06SCI-SCITAKE-videoSixteenByNine3000.jpg}

By \href{http://www.nytimes3xbfgragh.onion/by/james-gorman}{James
Gorman}

\begin{itemize}
\item
  Feb. 6, 2018
\item
  \begin{itemize}
  \item
  \item
  \item
  \item
  \item
  \item
  \end{itemize}
\end{itemize}

Snakes move in mysterious ways. Sometimes they slither along in the
grass as you might expect, if snakes are ever expected.

Sometimes they rise straight up as if levitating. They leap across wide
gaps.
\href{https://www.nytimes3xbfgragh.onion/video/science/100000002721454/sciencetake-flying-snakes.html}{They
even fly}, some of them, or at least glide, launching themselves into
the air from trees they have climbed.

And then there's
\href{https://www.nytimes3xbfgragh.onion/2014/10/09/science/secrets-of-the-sidewinder.html}{the
sidewinder, which takes its name from its hypnotic motion}.

Their movements have captured the attention of scientists and poets
alike. The scientists, like Bruce Jayne, at the University of
Cincinnati, observe and measure, deciphering what muscles the animals
use to wind and bend their way along the ground, through the grass or up
a tree.

But poets also have their findings about snakes.
\href{https://www.nytimes3xbfgragh.onion/2016/05/17/science/emily-dickinson-lost-gardens.html}{Emily
Dickinson, whose powers of observation} would put many a field
researcher to shame, wrote in a poem called
``\href{https://www.poetryfoundation.org/poems/49909/a-narrow-fellow-in-the-grass-1096}{A
narrow Fellow in the Grass}'' of ``a Whip Lash Unbraiding in the Sun,''
which, when investigated, ``wrinkled and was gone.''

And Rudyard Kipling, no mean chronicler of animals himself, although he
gave them human personalities, describes the progress of the
\href{https://www.cs.cmu.edu/~rgs/jngl-Hunting.html}{great python Kaa,
as he ``seemed to pour his way across the ground.''}

That description is close to the kind of locomotion that Dr. Jayne and
his graduate student, Steven J. Newman, analyzed in a recent issue of
the
\href{http://jeb.biologists.org/content/early/2017/12/05/jeb.166199}{Journal
of Experimental Biology}.

Unlike most snake forms of movement, this one involves moving in a
straight line, with no bending. It is not swift, as Kaa's progress was
in Kipling's ``Jungle Book,'' but it is eerily liquid. Slyly, slowly,
the snake flows without bending.

The motion had not escaped students of snakes. A scientist named Hans
Lissmann described it about 70 years ago and produced a hypothesis about
what muscles the snakes used to propel themselves, and how those muscles
acted.

Lissmann, Dr. Jayne said,
\href{http://jeb.biologists.org/content/jexbio/26/4/368.full.pdf}{``did
some classic work'' on ``rectilinear locomotion,}'' (scientific language
that suggests why the world needs poets).

Dr. Jayne said, ``We had an extremely good idea of all of the
movements,'' and Lissmann hypothesized how the muscles might work. ``But
what we were really lacking were any direct observations of how the
muscles work,'' Dr. Jayne said.

To do those observations, Mr. Newman inserted fine wire electrodes in
the snakes' muscles to record their activity, something like the way
sensors record heart activity in an electrocardiogram.

What the scientists found was that Lissmann was mostly right, but not
exactly so. Here's how it works:

As the snake is moving forward, a muscle in its belly skin shortens the
skin and stays tensed. Then a muscle running forward from the tip of one
rib to the skin tenses to pull the skeleton and body forward over the
skin.

Another muscle that runs toward the tail from the middle of a rib to the
skin pulls the skin forward.

Lissmann did not expect the skin muscle to stay tensed, and he thought
the skin muscles did the work of pulling the body forward, whereas it is
the muscle running from the tip of a rib to the skin.

Why do the details matter? Scientists and poets alike want to get things
right, whether it's a matter of the precise word or the electrical
activity of the muscle.

And in terms of practicality, the makers of soft, snakelike robots meant
to find their way through nooks and crannies without troublesome legs to
snag on obstacles, might find the exact nature of the motion useful, as
snakes have for millenniums.

Interestingly, Dr. Jayne said, the slow, liquid crawl he and Mr. Newman
studied seems to be used more by the heavier snakes --- boas and
pythons.

Like Kaa.

Advertisement

\protect\hyperlink{after-bottom}{Continue reading the main story}

\hypertarget{site-index}{%
\subsection{Site Index}\label{site-index}}

\hypertarget{site-information-navigation}{%
\subsection{Site Information
Navigation}\label{site-information-navigation}}

\begin{itemize}
\tightlist
\item
  \href{https://help.nytimes3xbfgragh.onion/hc/en-us/articles/115014792127-Copyright-notice}{©~2020~The
  New York Times Company}
\end{itemize}

\begin{itemize}
\tightlist
\item
  \href{https://www.nytco.com/}{NYTCo}
\item
  \href{https://help.nytimes3xbfgragh.onion/hc/en-us/articles/115015385887-Contact-Us}{Contact
  Us}
\item
  \href{https://www.nytco.com/careers/}{Work with us}
\item
  \href{https://nytmediakit.com/}{Advertise}
\item
  \href{http://www.tbrandstudio.com/}{T Brand Studio}
\item
  \href{https://www.nytimes3xbfgragh.onion/privacy/cookie-policy\#how-do-i-manage-trackers}{Your
  Ad Choices}
\item
  \href{https://www.nytimes3xbfgragh.onion/privacy}{Privacy}
\item
  \href{https://help.nytimes3xbfgragh.onion/hc/en-us/articles/115014893428-Terms-of-service}{Terms
  of Service}
\item
  \href{https://help.nytimes3xbfgragh.onion/hc/en-us/articles/115014893968-Terms-of-sale}{Terms
  of Sale}
\item
  \href{https://spiderbites.nytimes3xbfgragh.onion}{Site Map}
\item
  \href{https://help.nytimes3xbfgragh.onion/hc/en-us}{Help}
\item
  \href{https://www.nytimes3xbfgragh.onion/subscription?campaignId=37WXW}{Subscriptions}
\end{itemize}
