Sections

SEARCH

\protect\hyperlink{site-content}{Skip to
content}\protect\hyperlink{site-index}{Skip to site index}

\href{https://www.nytimes3xbfgragh.onion/section/sports}{Sports}

\href{https://myaccount.nytimes3xbfgragh.onion/auth/login?response_type=cookie\&client_id=vi}{}

\href{https://www.nytimes3xbfgragh.onion/section/todayspaper}{Today's
Paper}

\href{/section/sports}{Sports}\textbar{}At Long Last, the Eagles Capture
Their First Super Bowl

\url{https://nyti.ms/2FLgjKa}

\begin{itemize}
\item
\item
\item
\item
\item
\item
\end{itemize}

Advertisement

\protect\hyperlink{after-top}{Continue reading the main story}

Supported by

\protect\hyperlink{after-sponsor}{Continue reading the main story}

Eagles 41, Patriots 33 \textbar{} Eagles win Super Bowl LII

\hypertarget{at-long-last-the-eagles-capture-their-first-super-bowl}{%
\section{At Long Last, the Eagles Capture Their First Super
Bowl}\label{at-long-last-the-eagles-capture-their-first-super-bowl}}

\includegraphics{https://static01.graylady3jvrrxbe.onion/images/2018/02/05/sports/05ertzpic/05ertzpic-articleLarge-v2.jpg?quality=75\&auto=webp\&disable=upscale}

By \href{https://www.nytimes3xbfgragh.onion/by/ben-shpigel}{Ben Shpigel}

\begin{itemize}
\item
  Feb. 4, 2018
\item
  \begin{itemize}
  \item
  \item
  \item
  \item
  \item
  \item
  \end{itemize}
\end{itemize}

MINNEAPOLIS --- Philadelphia is a grand old city with a grand old
football tradition defined not by trophies hoisted but the cocktail of
emotions --- nihilistic despair tinged with unfounded confidence ---
produced by its beloved Eagles coming so close, so many times.

The last 57 years had passed without a championship. It was never Ron
Jaworski's turn or Randall Cunningham's or Donovan McNabb's.
\href{https://www.nytimes3xbfgragh.onion/2018/02/04/sports/football/philadelphia.html?rref=collection\%2Fsectioncollection\%2Fsports\&action=click\&contentCollection=sports\&region=rank\&module=package\&version=highlights\&contentPlacement=2\&pgtype=sectionfront}{It
was always something} --- always.

The paradigm shifted Sunday, when a backup quarterback who nearly
retired two years ago and who had not taken a snap with the first-team
offense until two months ago gave Philadelphia its first
\href{https://www.nytimes3xbfgragh.onion/2019/02/03/sports/super-bowl-patriots-rams.html}{Super
Bowl} title at the expense of the N.F.L.'s enduring dynasty.

For the rest of his life, Nick Foles will be hailed as the savior who
matched
\href{https://www.nytimes3xbfgragh.onion/2018/02/01/sports/football/tom-brady-super-bowl-lii.html}{a
resplendent Tom Brady} and authored a victory against the New England
Patriots for Philadelphia's first football championship since 1960.

The final score --- 41-33 --- will lodge deep into the memory banks of a
frenzied pro-Eagles crowd at U.S. Bank Stadium and the millions of
delirious fans across the Delaware Valley, who will belt it out at work
or family gatherings or watering holes as calculable proof that, yes,
the Eagles did, in fact, win the Super Bowl.

It was Philadelphia's second major sports championship since 1983, after
the Phillies' World Series victory in 2008. The last time the Eagles
earned an N.F.L. championship, in the pre-Super Bowl era, they also
conquered an iconic coach in Vince Lombardi and a Hall of Fame
quarterback in Bart Starr, but that win came before Green Bay's reign of
dominance. In Brady and Bill Belichick, bidding for their sixth title
together, these Eagles outdueled the N.F.L.'s premier comeback artists
to avenge a defeat from their last Super Bowl meeting 13 years ago.

Before that game, Belichick inspired his players by relaying the victory
parade route --- of the Eagles. If Philadelphia has not been reduced to
rubble, that parade will finally meander down Broad Street this week.

What made it possible was an unforgettable performance by Foles, who has
started for Philadelphia since its star quarterback Carson Wentz
\href{https://www.nytimes3xbfgragh.onion/2017/12/11/sports/carson-wentz-philadelphia-eagles.html}{tore
up his knee on Dec. 10 in Los Angeles}. Foles earned the Super Bowl's
Most Valuable Player Award for catching one touchdown and throwing for
three more, including an 11-yarder to Zach Ertz with 2 minutes 21
seconds remaining that put the Eagles ahead, 38-33. The aggressiveness
of Coach Doug Pederson, a former Eagles quarterback, helped outsmart
Belichick en route to winning a championship that Pederson's mentor,
Andy Reid, never could in Philadelphia. And an improbable late stand by
a defense that yielded 613 yards --- 505 passing by Brady --- produced
the most critical play at the most critical moment of the season.

The seconds feel like minutes, and the minutes feel like hours, when
Brady jogs onto the field in the fourth quarter of a Super Bowl with his
team trailing. After an entertaining first half featuring nearly 500
passing yards, a missed field goal, two shanked extra points, a bungled
2-point conversion, a Brady drop and a Foles touchdown reception, the
second half contained multitudes --- if not much defense. The Patriots,
who had scored on their first three second-half possessions, now needed
to go 75 yards in 2:21 for a dynasty-extending touchdown.

On second and 2 from the New England 33, Brady stepped up in the pocket
and either never sensed defensive end Brandon Graham or thought he could
evade him. Graham slapped the ball out, and it bounced into the arms of
Derek Barnett, who recovered at the Patriots' 31. Brady sat on the
ground, his head down, as the Eagles danced around him.

\includegraphics{https://static01.graylady3jvrrxbe.onion/images/2018/02/05/sports/05SUPER-web3/05SUPER-web3-articleLarge.jpg?quality=75\&auto=webp\&disable=upscale}

After the Eagles parlayed that turnover into a field goal, Brady had one
final chance, but his desperation heave from his own 49-yard line fell
incomplete.

Long after the Patriots cleared the field, the theme song to ``Rocky''
blared over the stadium loudspeakers as fans hailed another cherished
underdog from Philadelphia. In a measure of how General Manager Howie
Roseman had revamped the Eagles' roster, the first 85 points of the
playoffs --- and first 32 on Sunday --- were scored by players who were
not on the team last year.

Foremost was Foles, whose circuitous path from an Eagles draft pick in
2012 took him from an unfulfilling stint with the Rams to a season in
Kansas City before Philadelphia made what proved to be a
franchise-defining decision. The Eagles' owner, Jeffrey Lurie, allocated
a hefty contract --- \$11 million over two years --- for a credible
backup.

Without Wentz, the quarterback matchup Sunday seemed more lopsided than
a three-legged table. Brady entered Sunday with more victories and
touchdown passes in the playoffs than Foles has across his entire
career, including the postseason. But Foles started developing timing
with his receivers, Ertz said, during their bye week before their first
playoff game against the Atlanta Falcons, when the Eagles conducted two
practices with training-camp intensity: first-team offense against
first-team defense.

Foles guided them past the Falcons, then flourished in a rout of the
Minnesota Vikings.
\href{https://www.nytimes3xbfgragh.onion/2018/01/12/sports/nfl-playoffs-eagles-falcons.html}{Philadelphia
was a home underdog in both games}. As Wentz looked on Sunday, Foles
threw for 373 yards, directing the Eagles to points on eight of 10
possessions. They punted once.

``I think the big thing helped me was knowing that I didn't have to be
Superman,'' Foles said. ``I have amazing teammates, amazing coaches
around me, and all I had to do was go play as hard as I could.''

Three rules govern the Eagles' quarterback room: be on time, take great
notes and play with swagger. In complying with that final edict, Foles
does not strut or preen. He merely delivers what his position coach,
John DeFilippo, calls mailbox throws --- as in, the ball flies so purely
and precisely that it could land in one.

Those mailboxes materialized at regular intervals Sunday: above Alshon
Jeffery's hands for a 34-yard first-quarter touchdown; in stride on a
wheel route to Corey Clement that went for 55 yards in the second
quarter; between two defenders for Clement's 22-yard score in the third;
a floater across the middle to Ertz in the fourth.

Those mailboxes also materialized for Brady, who commanded an offense
that capitalizes on nameless, faceless positions --- running backs who
catch like receivers, tight ends who run like running backs and
receivers who do both. Brady completed passes long and short and in
between, to Chris Hogan and Rob Gronkowski and Rex Burkhead, gashing the
Eagles for 276 yards by halftime and 404 through three quarters. In all,
New England shredded Philadelphia for eight plays of at least 20 yards.

Asked last week whether Brady had any weaknesses to exploit, Eagles
defensive coordinator Jim Schwartz chuckled. ``Those people that have
tried to find something they can exploit were sent home crying probably
about 2001,'' he said.

That is when the Patriots' dynasty began, with the first of three titles
in four seasons. It was also the last time, before Foles, that a backup
quarterback won a Super Bowl. On Sunday, New England was again seeking a
third title in four seasons against Philadelphia, but all empires must
fall.

Image

Tom Brady, a five-time Super Bowl champion, passed for 505 yards and
three touchdowns.Credit...Ben Solomon for The New York Times

Brady will be 41 next season, but he said Sunday, ``I don't see why I
wouldn't be back.''

So will the Eagles, whose demeanor in the past week evoked the 2013
Seattle Seahawks, who knew how good they were --- and couldn't wait to
prove as much against Peyton Manning and the Denver Broncos. Jeffery
spoke last week in definite terms: when, not if, the Eagles won the
Super Bowl. Of Brady, he said Sunday night, ``I respect him, a great
player, probably one of the greatest ever, but hey, he had not played
the Eagles yet.''

Indeed, he had not. He had not faced the team that choreographed
elaborate touchdown celebrations and
\href{https://www.nytimes3xbfgragh.onion/2018/01/25/sports/football/malcolm-jenkins-eagles-super-bowl.html}{railed
against social injustice} and donned goofy dog masks to embrace their
underdog status while mocking it. Fulfilling Lurie's demand for a coach
with emotional intelligence, Pederson fomented an inclusive locker-room
culture that empowered players to flaunt their personalities. Lurie knew
Pederson not only as a brilliant offensive mind but as someone who
taught Lurie's son, Julian, how to throw a football, and who, from being
pelted with batteries from impatient fans wishing McNabb to start as a
rookie in 1999, had the requisite toughness to coach in Philadelphia.

``Wherever he was,'' Lurie said of Pederson last week, ``he was always
the most genuine person in the room.''

Pederson also has a preternatural feel for optimizing his players. On
Saturday night, he told his team that he had formulated an assertive
game plan, with plenty of downfield throws and bold tactics.

Teams tend to shrink against New England, especially when leading. The
Eagles? No chance.

All season they have flouted conventional wisdom by
\href{https://www.nytimes3xbfgragh.onion/2018/02/02/sports/football/eagles-analytics-super-bowl-lii.html}{going
for it in counterintuitive situations}. Only one team, Green Bay, went
for it more often on fourth down. After Duron Harmon halted a promising
drive by intercepting Foles on a ball that caromed off Jeffery, and
after the Patriots proceeded to march 90 yards to draw within 15-12 with
about two minutes remaining in the first half, the Eagles regrouped and
found themselves at the New England 1 with 38 seconds left.

Instead of attempting a field goal, Philadelphia called its inverse:
\href{https://www.nytimes3xbfgragh.onion/2018/02/05/sports/foles-super-bowl.html?ribbon-ad-idx=3\&rref=sports\&module=Ribbon\&version=context\&region=Header\&action=click\&contentCollection=Sports\&pgtype=article}{a
play called Philly Special}. It resembles a version the Chicago Bears
have run, and the Eagles have been practicing it for the last month: a
direct snap to Clement, who pitched to Trey Burton --- a former college
quarterback who converted to tight end --- who tossed to Foles, who
became the first player to throw and catch a touchdown in a Super Bowl.

``That play we've been working on for the last couple of weeks, and just
needed the right time, right opportunity, and the guys executed it
brilliantly,'' Pederson said.

Burton said of Pederson: ``He's got a lot of guts.''

The Eagles' sideline turned into a mosh pit, with players and coaches
jumping around, as it did a couple hours later, when their only sack of
the game dislodged the ball from one of the greatest quarterbacks in
history.

Last week center Jason Kelce remarked that he had seen that the Chinese
Year of the Dog was coming up. ``So maybe the odds are in our favor,''
he quipped.

Indeed, they were. Despite losing their franchise quarterback, their
All-Pro left tackle (Jason Peters), their most versatile running back
(Darren Sproles) and their best special-teams player (Chris Maragos),
the Eagles thwarted the mighty Patriots in the final game of the season.

Outside the Eagles' locker room, a sign read: ``An individual can make a
difference. A team can make a miracle.'' And that is what it must have
felt like from Northern Liberties to Manayunk to Mantua, cathartic joy
that will not dissipate for days, weeks, months. They will scale
lightpoles and high-five strangers and chant ``Fly Eagles Fly'' until
their voice disappears because after 57 years the improbable has
happened.

The Eagles, finally, are Super Bowl champions.

Advertisement

\protect\hyperlink{after-bottom}{Continue reading the main story}

\hypertarget{site-index}{%
\subsection{Site Index}\label{site-index}}

\hypertarget{site-information-navigation}{%
\subsection{Site Information
Navigation}\label{site-information-navigation}}

\begin{itemize}
\tightlist
\item
  \href{https://help.nytimes3xbfgragh.onion/hc/en-us/articles/115014792127-Copyright-notice}{©~2020~The
  New York Times Company}
\end{itemize}

\begin{itemize}
\tightlist
\item
  \href{https://www.nytco.com/}{NYTCo}
\item
  \href{https://help.nytimes3xbfgragh.onion/hc/en-us/articles/115015385887-Contact-Us}{Contact
  Us}
\item
  \href{https://www.nytco.com/careers/}{Work with us}
\item
  \href{https://nytmediakit.com/}{Advertise}
\item
  \href{http://www.tbrandstudio.com/}{T Brand Studio}
\item
  \href{https://www.nytimes3xbfgragh.onion/privacy/cookie-policy\#how-do-i-manage-trackers}{Your
  Ad Choices}
\item
  \href{https://www.nytimes3xbfgragh.onion/privacy}{Privacy}
\item
  \href{https://help.nytimes3xbfgragh.onion/hc/en-us/articles/115014893428-Terms-of-service}{Terms
  of Service}
\item
  \href{https://help.nytimes3xbfgragh.onion/hc/en-us/articles/115014893968-Terms-of-sale}{Terms
  of Sale}
\item
  \href{https://spiderbites.nytimes3xbfgragh.onion}{Site Map}
\item
  \href{https://help.nytimes3xbfgragh.onion/hc/en-us}{Help}
\item
  \href{https://www.nytimes3xbfgragh.onion/subscription?campaignId=37WXW}{Subscriptions}
\end{itemize}
