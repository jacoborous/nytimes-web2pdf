Sections

SEARCH

\protect\hyperlink{site-content}{Skip to
content}\protect\hyperlink{site-index}{Skip to site index}

\href{https://www.nytimes3xbfgragh.onion/section/politics}{Politics}

\href{https://myaccount.nytimes3xbfgragh.onion/auth/login?response_type=cookie\&client_id=vi}{}

\href{https://www.nytimes3xbfgragh.onion/section/todayspaper}{Today's
Paper}

\href{/section/politics}{Politics}\textbar{}Kelly Says He's Willing to
Resign as Abuse Scandal Roils White House

\url{https://nyti.ms/2BOzIvY}

\begin{itemize}
\item
\item
\item
\item
\item
\item
\end{itemize}

Advertisement

\protect\hyperlink{after-top}{Continue reading the main story}

Supported by

\protect\hyperlink{after-sponsor}{Continue reading the main story}

\hypertarget{kelly-says-hes-willing-to-resign-as-abuse-scandal-roils-white-house}{%
\section{Kelly Says He's Willing to Resign as Abuse Scandal Roils White
House}\label{kelly-says-hes-willing-to-resign-as-abuse-scandal-roils-white-house}}

\includegraphics{https://static01.graylady3jvrrxbe.onion/images/2018/02/09/us/politics/porter-vid/porter-vid-videoSixteenByNine3000.jpg}

By \href{http://www.nytimes3xbfgragh.onion/by/maggie-haberman}{Maggie
Haberman},
\href{https://www.nytimes3xbfgragh.onion/by/julie-hirschfeld-davis}{Julie
Hirschfeld Davis} and
\href{http://www.nytimes3xbfgragh.onion/by/michael-s-schmidt}{Michael S.
Schmidt}

\begin{itemize}
\item
  Feb. 9, 2018
\item
  \begin{itemize}
  \item
  \item
  \item
  \item
  \item
  \item
  \end{itemize}
\end{itemize}

WASHINGTON --- John F. Kelly, the White House chief of staff, told
officials in the West Wing on Friday that he was willing to step down
over his handling of allegations of spousal abuse against Rob Porter,
the staff secretary who
\href{https://www.nytimes3xbfgragh.onion/2018/02/07/us/politics/rob-porter-resigns-abuse-white-house-staff-secretary.html}{resigned
in disgrace} this week over the accusations, according to two officials
aware of the discussions.

The officials emphasized that they did not consider a resignation
imminent, and that Mr. Kelly --- a retired
\href{https://www.nytimes3xbfgragh.onion/2018/02/08/us/politics/kelly-trump.html}{four-star
Marine general} who early in his tenure often used a threat of quitting
as a way to temper President Trump's behavior --- had made no formal
offer. In comments to reporters at the White House on Friday, Mr. Kelly
said he had not offered to resign.

But his suggestion in private that he would be willing to step down if
the president wanted him to reflected the degree to which the scandal
surrounding Mr. Porter has engulfed the White House, touching off a
bitter round of recriminations that could result in a shake-up at the
highest levels.

Two West Wing advisers and a third person painted a picture of a White
House staff rived and confused, with fingers pointed in all directions
and the president privately expressing dissatisfaction with Mr. Kelly.

Some complained that Donald F. McGahn II, the White House counsel, who
learned last January that Mr. Porter was concerned about potentially
damaging accusations from two ex-wives, had not been forthcoming enough
about what he knew. Others faulted Hope Hicks, the communications
director, who had been romantically involved with Mr. Porter, for
soliciting statements of support for him when the accusations became
public.

And many, including the president himself, have turned their ire on Mr.
Kelly for vouching for Mr. Porter's character and falsely asserting that
he had moved aggressively to oust him once his misdeeds were discovered.

For all the turmoil, Mr. Trump on Friday warmly praised Mr. Porter,
saying it was a ``tough time'' for his former aide and noting that Mr.
Porter had denied the accusations.

``We wish him well,'' Mr. Trump said of his former aide, who was accused
of physical and emotional abuse by two ex-wives. The president added,
``He also, as you probably know, says he is innocent, and I think you
have to remember that.''

``He worked very hard,'' Mr. Trump told reporters in the Oval Office
when asked for a comment about Mr. Porter. The president said he had
only ``recently'' learned of the allegations against his former aide and
was surprised.

``He did a very good job when he was in the White House, and we hope he
has a wonderful career, and he will have a great career ahead of him,''
Mr. Trump said. ``But it was very sad when we heard about it, and
certainly he's also very sad now.''

The glowing praise of a staff member accused of serial violence against
women was in line with the president's own denials of sexual impropriety
despite accusations from more than a dozen women and his habit of
accepting claims of innocence from men facing similar allegations. Among
them was Roy S. Moore, the former Republican Senate candidate in
Alabama, who is accused of molesting teenage girls.

But later in the day, the White House dealt far more aggressively with
another allegation of abuse. A spokesman confirmed that a second White
House staff member, David Sorensen, a speechwriter, had resigned over
allegations by his former wife that he had abused her during their
marriage. The spokesman said that officials confronted Mr. Sorenson on
Thursday night when they learned of the accusations and that he had
denied them.

Mr. Sorensen's resignation,
\href{https://www.washingtonpost.com/politics/second-white-house-official-departs-amids-abuse-allegations-which-he-denies/2018/02/09/72ba47e6-0d0d-11e8-8b0d-891602206fb7_story.html?hpid=hp_rhp-top-table-main_sorensen-8pm\%3Ahomepage\%2Fstory\&utm_term=.ddcd0363c04f}{first
reported by The Washington Post}, came as a new timeline emerged
indicating that top officials knew much earlier than previously
disclosed that Mr. Porter faced accusations of violence against women.

Shortly after Mr. Trump's inauguration, Mr. McGahn, the White House
counsel, first learned from Mr. Porter himself that there could be
allegations against him, according to two people briefed on the
situation. Mr. McGahn's knowledge of the accusations in January was
first
\href{https://www.washingtonpost.com/politics/top-white-house-officials-knew-of-abuse-allegations-against-top-aide-for-months/2018/02/08/2faddcf2-0ce9-11e8-95a5-c396801049ef_story.html?hpid=hp_hp-top-table-low_whitehouse-1045pm:homepage/story\&utm_term=.0e674d9a3fb7}{reported
by The Washington Post}.

Mr. Porter told him about the possible allegations because he was
concerned that what he characterized as false charges from aggrieved
women who were out to destroy him could derail his F.B.I. background
check, according to one of the two people briefed on the matter.

\includegraphics{https://static01.graylady3jvrrxbe.onion/images/2018/02/09/us/09dc-kelly/09dc-kelly-videoSixteenByNine3000-v3.jpg}

Six months later, the F.B.I. told Mr. McGahn that accusations of
domestic abuse had indeed surfaced in Mr. Porter's background check. Mr.
McGahn opted at that time to let the F.B.I. complete its investigation
into any incidents. Mr. Porter assured Mr. McGahn, another person
briefed on the matter said, that the accusations from the former wives
were lies.

The emerging timeline illustrates the degree to which Mr. Porter, a
clean-cut and ambitious former Rhodes scholar and Harvard Law
School-educated lawyer, concealed troublesome episodes from his past
that would normally be considered disqualifying for a senior White House
aide.

Those efforts appear to have succeeded for months, at least in part
because of the willingness of a virtually all-male staff in the top
echelons of the West Wing to believe a talented male colleague over
women they had never met.

Lawyers in the counsel's office believed that the bureau --- with its
vast investigative powers --- was best positioned to look into the
accusations, the two people briefed on the matter said, and that it was
not their job to investigate conduct that took place long before an
official began working in the administration.

That represents a sharp break with past practice, in which White House
counsels undertook elaborate vetting of senior advisers before they were
hired --- and looked into any serious allegations that surfaced
thereafter.

In November, the White House heard back from the F.B.I. Senior White
House officials, including Mr. Kelly, Joe Hagin, the deputy chief of
staff, and Mr. McGahn received word from the bureau that the allegations
were credible and that Mr. Porter was not likely to pass his background
check.

But while Mr. McGahn privately informed Mr. Porter and encouraged him to
consider moving on, according to one of the two people briefed, no
action was taken to immediately terminate him. Rather, Mr. McGahn
requested that the F.B.I. complete its investigation and come back to
the White House with a final recommendation, a process that could take
months.

On Friday, several White House aides described confusion among the staff
about why Mr. Kelly and others had initially rallied to defend Mr.
Porter, and some suggested that Mr. Kelly had tried to cover up what he
knew. Others insisted that while he was aware of the broad strokes of
accusations against Mr. Porter, and while he could have made more of an
effort to learn more, he trusted the staff secretary's denials.

Mr. Kelly began telling people at the White House on Thursday that he
had learned the details of Mr. Porter's situation only ``40 minutes
before he threw him out'' two days earlier, before pictures of Mr.
Porter's bruised ex-wife began circulating.

But on Wednesday afternoon, Sarah Huckabee Sanders, the White House
press secretary, and other aides had maintained to reporters that the
White House fully backed Mr. Porter, who they said was leaving of his
own accord. She issued a statement saying that both the president and
Mr. Kelly had ``full confidence'' in his performance.

In a meeting with senior staff members on Friday morning, Mr. Kelly
appeared to be trying to paint his handling of the matter in a more
favorable light. At the end of the session, according to people with
knowledge of his remarks, Mr. Kelly volunteered that he had something he
wanted to ``clarify.''

Mr. Kelly went on to say that he had learned of Mr. Porter's true
situation less than an hour before he removed him from his job. Two
people familiar with the comments said that most of the staff appeared
incredulous; one person said several people in the room knew that the
timeline Mr. Kelly had presented was false.

As the meeting broke up, Ms. Hicks loudly complained about what had
transpired, using an expletive, a person briefed on the meeting said.

The infighting unfolded amid signs of a brewing shake-up in the West
Wing.

The president has now sounded out several people as possible
replacements for Mr. Kelly. Those possible replacements include Mick
Mulvaney, the budget director; Representative Kevin McCarthy of
California; and Gary D. Cohn, Mr. Trump's top economic adviser. He has
also returned to a notion he has raised privately in the past, telling
people he would like Thomas J. Barrack Jr., a close friend and
confidant, to take the job.

Mr. Kelly is not ready to explicitly offer a resignation, according to a
person familiar with his thinking. But people close to Mr. Trump said
the president had begun the process of making the job so unpleasant for
Mr. Kelly that it might hasten his departure, the same sort of ritual
humiliation to which he subjected Reince Priebus, his first chief of
staff, before his departure in July.

Despite his warm words on Friday, two advisers said, Mr. Trump was livid
when he learned of the allegations against Mr. Porter, and referred to
his disgraced former aide in one phone call as ``bad garbage.'' He also
expressed his frustration with both Mr. Kelly and Ms. Hicks.

On Friday evening, the White House announced new posts for 31 officials,
including elevating to acting staff secretary Derek Lyons, another
Harvard Law School-educated lawyer, who formerly worked on Capitol Hill
and had been Mr. Porter's No. 2. Also included on the list was the
departure of Jim Carroll, a lawyer in the counsel's office who had only
recently been designated to be Mr. Kelly's deputy, but will now become
acting director of the Office of National Drug Control Policy.

Advertisement

\protect\hyperlink{after-bottom}{Continue reading the main story}

\hypertarget{site-index}{%
\subsection{Site Index}\label{site-index}}

\hypertarget{site-information-navigation}{%
\subsection{Site Information
Navigation}\label{site-information-navigation}}

\begin{itemize}
\tightlist
\item
  \href{https://help.nytimes3xbfgragh.onion/hc/en-us/articles/115014792127-Copyright-notice}{©~2020~The
  New York Times Company}
\end{itemize}

\begin{itemize}
\tightlist
\item
  \href{https://www.nytco.com/}{NYTCo}
\item
  \href{https://help.nytimes3xbfgragh.onion/hc/en-us/articles/115015385887-Contact-Us}{Contact
  Us}
\item
  \href{https://www.nytco.com/careers/}{Work with us}
\item
  \href{https://nytmediakit.com/}{Advertise}
\item
  \href{http://www.tbrandstudio.com/}{T Brand Studio}
\item
  \href{https://www.nytimes3xbfgragh.onion/privacy/cookie-policy\#how-do-i-manage-trackers}{Your
  Ad Choices}
\item
  \href{https://www.nytimes3xbfgragh.onion/privacy}{Privacy}
\item
  \href{https://help.nytimes3xbfgragh.onion/hc/en-us/articles/115014893428-Terms-of-service}{Terms
  of Service}
\item
  \href{https://help.nytimes3xbfgragh.onion/hc/en-us/articles/115014893968-Terms-of-sale}{Terms
  of Sale}
\item
  \href{https://spiderbites.nytimes3xbfgragh.onion}{Site Map}
\item
  \href{https://help.nytimes3xbfgragh.onion/hc/en-us}{Help}
\item
  \href{https://www.nytimes3xbfgragh.onion/subscription?campaignId=37WXW}{Subscriptions}
\end{itemize}
