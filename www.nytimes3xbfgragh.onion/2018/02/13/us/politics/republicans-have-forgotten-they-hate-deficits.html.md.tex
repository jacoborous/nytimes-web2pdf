Sections

SEARCH

\protect\hyperlink{site-content}{Skip to
content}\protect\hyperlink{site-index}{Skip to site index}

\href{https://www.nytimes3xbfgragh.onion/section/politics}{Politics}

\href{https://myaccount.nytimes3xbfgragh.onion/auth/login?response_type=cookie\&client_id=vi}{}

\href{https://www.nytimes3xbfgragh.onion/section/todayspaper}{Today's
Paper}

\href{/section/politics}{Politics}\textbar{}Republicans Have Forgotten
They Hate Deficits

\url{https://nyti.ms/2BWxwm2}

\begin{itemize}
\item
\item
\item
\item
\item
\item
\end{itemize}

Advertisement

\protect\hyperlink{after-top}{Continue reading the main story}

Supported by

\protect\hyperlink{after-sponsor}{Continue reading the main story}

News Analysis

\hypertarget{republicans-have-forgotten-they-hate-deficits}{%
\section{Republicans Have Forgotten They Hate
Deficits}\label{republicans-have-forgotten-they-hate-deficits}}

\includegraphics{https://static01.graylady3jvrrxbe.onion/images/2018/02/13/business/13DC-CONSERVATIVEECON1/merlin_133710765_b49bff91-6f13-4f1e-b3b4-493939222539-articleLarge.jpg?quality=75\&auto=webp\&disable=upscale}

By Jim Tankersley

\begin{itemize}
\item
  Feb. 13, 2018
\item
  \begin{itemize}
  \item
  \item
  \item
  \item
  \item
  \item
  \end{itemize}
\end{itemize}

WASHINGTON --- Since the 2008 financial crisis, conservative economists
and Republican politicians have claimed several policy errors were
holding back economic growth and contributing to a painfully slow
recovery, including lackluster wage growth. The government was spending
too much, it was borrowing too much and it was sowing ``uncertainty'' in
the business world with expiring tax cuts and other bitter policy
battles, they said.

If Republican leaders still believe that, they're not acting like it.

In a year of controlling power in Washington, President Trump and
Republicans in Congress have run up federal spending, approved
deficit-swelling tax cuts and presided over a marked increase in
``policy uncertainty'' in the economy. They still talk about the
importance of fiscal discipline, but they have yet to enforce it.

The
\href{https://www.whitehouse.gov/wp-content/uploads/2018/02/budget-fy2019.pdf}{\$4.4
trillion budget Mr. Trump released on Monday} spends as much over 10
years as any budget offered by President Barack Obama, whose policies
Republicans blamed for ballooning the size of the federal government and
hobbling the economy. It does not attempt to achieve balance at the end
of that time, despite optimistic
\href{https://www.whitehouse.gov/wp-content/uploads/2018/02/spec-fy2019.pdf}{economic
growth projections that far exceed} what most economists say is
possible.

Instead, it projects that deficits will grow \$7 trillion over the next
decade as the United States continues borrowing huge sums of money --- a
number that could double if the administration turns out to be
overestimating economic growth and if the \$3 trillion in spending cuts
the White House has floated do not materialize in Congress.

The promised \$3 trillion in nonmilitary spending cuts do not appear to
be high priorities for the president or Republican leaders. A key
proposal for self-imposed fiscal restraint, the so-called ``two penny
plan'' that would reduce nondefense discretionary spending by 2 percent
a year, is a holdover from last year's budget proposal. Americans have
heard almost nothing about it from the White House in the intervening
year. Mr. Trump has talked a bit more about his ``welfare reform''
ambitions to help cut federal costs but, again, has not mounted a
serious pressure campaign to urge Congress to enact them.

Congressional Republicans appear unlikely to even attempt to pass a
budget resolution that could allow them to push cuts of that sort
through the Senate without any Democratic votes, having reserved that
move last year not for spending cuts, but for tax cuts that the Joint
Committee on Taxation estimates will add more than \$1 trillion to the
national debt, even after accounting for increased economic growth.

The most energy the party put last year into actually reducing spending
and deficits was its attempted repeal of the Affordable Care Act, which
failed after three Republicans defected. Mr. Trump's budget assumes that
effort will now succeed, somehow. Party leaders have downplayed
suggestions floated by Speaker Paul D. Ryan of Wisconsin, among others,
that Congress could move this year to reduce future spending on safety
net programs such as Medicare and Social Security.

``Republicans will still tell you that runaway spending hurts the
economy,'' said Brian Riedl, a senior fellow at the conservative
Manhattan Institute who is a former chief economist for Senator Rob
Portman, Republican of Ohio. But the party, he said, ``isn't even
talking about Social Security and Medicare reform anymore. And that to
me is the sea change. There's no constituency for it, and Republicans
have essentially given up.''

While Mr. Trump has pushed to freeze or roll back regulations in
Washington, his election --- and the unpredictability of his
administration --- has coincided with a resurgence of uncertainty in the
United States.

Since Mr. Trump won election in November 2016, an
\href{http://www.policyuncertainty.com/us_monthly.html}{index of
economic policy uncertainty}, developed by economists from Stanford
University, the University of Chicago and Northwestern University, has
jumped 25 percent from its average level for the preceding four years.
Mr. Trump has regularly dangled the possibility of quitting the North
American Free Trade Agreement, for example, and his signature tax law
sets all of its individual cuts, and a key business provision, to expire
in a few years.

Democrats look at that evolution and see Republicans revealing a new set
of economic preferences for the Trump era: lower taxes, less regulation,
more federal spending --- including an infrastructure proposal that
could further swell the deficit --- and mounting piles of federal debt.

That combination looks less like the economics of the Tea Party, circa
2010, and more like the policies of President George W. Bush --- the
very policies that
\href{http://www.washingtonpost.com/wp-dyn/content/article/2006/05/10/AR2006051002040.html}{conservatives
blamed for their party losing its way}, and control of Congress, in
2006.

``On certainty, smaller deficits and less government spending, I do see
shifts,'' said Michael R. Strain, an economist at the conservative
American Enterprise Institute. ``But I don't think those shifts are as
significant as many are making them out to be. George W. Bush inherited
a budget surplus and left large deficits. He expanded Medicare without
paying for it. Dick Cheney said `deficits don't matter.' Reagan said
that the deficit is big enough to take care of itself. The G.O.P. has
not restrained federal spending when it has held the White House.''

The official line, from the White House to congressional leadership, is
that Republicans remain committed to spending restraint, but have been
forced by Democrats to cut deals that run up more spending. That is
their rationale for
\href{https://www.nytimes3xbfgragh.onion/2018/02/08/us/politics/congress-budget-deal-vote.html}{last
week's agreement} to lift Obama-era restraints on defense and nondefense
discretionary spending by \$300 billion, in order to secure the defense
spending increases that conservatives say are crucial to national
security. (Many also contend that the new tax law will generate enough
growth to offset any lost federal revenues, although no detailed
independent analysis supports that claim.)

\href{https://www.nytimes3xbfgragh.onion/interactive/2018/02/08/us/politics/budget-deficits-debt-bipartisan-spending-bill.html}{}

\includegraphics{https://static01.graylady3jvrrxbe.onion/images/2018/02/08/us/politics/budget-deficits-debt-bipartisan-spending-bill-1518122535885/budget-deficits-debt-bipartisan-spending-bill-1518122535885-articleLarge-v2.png}

\hypertarget{budget-deficits-are-projected-to-balloon-under-the-bipartisan-spending-deal}{%
\subsection{Budget Deficits Are Projected to Balloon Under the
Bipartisan Spending
Deal}\label{budget-deficits-are-projected-to-balloon-under-the-bipartisan-spending-deal}}

The two-year budget agreement reached by Senate leaders would contribute
hundreds of billions of dollars to federal deficits.

Some leading conservative economists say the party has not lost its core
convictions on spending and debt, which, they contend, remain solely a
problem of rising safety net spending in the years to come as the
population ages and the
\href{https://www.washingtonpost.com/posteverything/wp/2015/11/05/baby-boomers-are-whats-wrong-with-americas-economy/?utm_term=.bc489318a19c}{Baby
Boom generation} retires from the work force.

``I think their view all along has been that spending and debt are not
immediate threats to growth,'' said John H. Cochrane, an economist who
is a senior fellow at Stanford University's Hoover Institution. ``They
know that right now, a fundamental reform of entitlements won't happen.
So, they have avoided weekly chaos and gotten needed military spending
through by opening the spending bill, and they got an important
reduction in growth-distorting marginal corporate rates through by
accepting a bit more deficits. They know that can't be the end of the
story.''

Mr. Cochrane likened the economic threat of rising spending and debt to
living on an earthquake fault, poised to rumble at any moment. He said
those risks may be making markets ``jittery'', and that Republicans may
be running out of time to follow through on their convictions, but he
expressed hope that Republicans and Democrats might find a way to work
together to reduce future spending growth.

Republicans
\href{http://www.nytimes3xbfgragh.onion/2012/04/01/magazine/obama-vs-boehner-who-killed-the-debt-deal.html}{declined
to make such an agreement under Mr. Obama}, who was willing to reduce
Social Security spending but wanted to increase taxes as part of a deal.
They seem unlikely to make that gamble now under Mr. Trump, who, many
conservative economists note, promised in his campaign not to touch
Social Security or Medicare, and whose actions during last year's
Obamacare repeal debate left some Republican leaders feeling scarred.

``When push came to shove last year, he
\href{https://www.usatoday.com/story/news/politics/onpolitics/2017/06/13/trump-told-senators-obamacare-repeal-bill-he-once-celebrated-mean/102826114/}{described
entitlement reforms as `mean,'} and completely cut their legs out from
under them,'' said Douglas Holtz-Eakin, an economist who advised
President Bush and is now president of the conservative American Action
Forum. ``So where do they go from there, with any confidence?''

The president continues to send mixed signals. On the first page of his
budget released Monday, he warned that ``the current fiscal path is
unsustainable, and future generations deserve better.'' But late last
month, in his State of the Union address, deficits and debt
\href{https://www.huffingtonpost.com/entry/trump-deficits-debt-state-of-the-union_us_5a71d823e4b0be822ba25139}{were
not at the top of Mr. Trump's mind}. His was the first such speech in a
generation not to mention those issues at all.

Advertisement

\protect\hyperlink{after-bottom}{Continue reading the main story}

\hypertarget{site-index}{%
\subsection{Site Index}\label{site-index}}

\hypertarget{site-information-navigation}{%
\subsection{Site Information
Navigation}\label{site-information-navigation}}

\begin{itemize}
\tightlist
\item
  \href{https://help.nytimes3xbfgragh.onion/hc/en-us/articles/115014792127-Copyright-notice}{©~2020~The
  New York Times Company}
\end{itemize}

\begin{itemize}
\tightlist
\item
  \href{https://www.nytco.com/}{NYTCo}
\item
  \href{https://help.nytimes3xbfgragh.onion/hc/en-us/articles/115015385887-Contact-Us}{Contact
  Us}
\item
  \href{https://www.nytco.com/careers/}{Work with us}
\item
  \href{https://nytmediakit.com/}{Advertise}
\item
  \href{http://www.tbrandstudio.com/}{T Brand Studio}
\item
  \href{https://www.nytimes3xbfgragh.onion/privacy/cookie-policy\#how-do-i-manage-trackers}{Your
  Ad Choices}
\item
  \href{https://www.nytimes3xbfgragh.onion/privacy}{Privacy}
\item
  \href{https://help.nytimes3xbfgragh.onion/hc/en-us/articles/115014893428-Terms-of-service}{Terms
  of Service}
\item
  \href{https://help.nytimes3xbfgragh.onion/hc/en-us/articles/115014893968-Terms-of-sale}{Terms
  of Sale}
\item
  \href{https://spiderbites.nytimes3xbfgragh.onion}{Site Map}
\item
  \href{https://help.nytimes3xbfgragh.onion/hc/en-us}{Help}
\item
  \href{https://www.nytimes3xbfgragh.onion/subscription?campaignId=37WXW}{Subscriptions}
\end{itemize}
