Sections

SEARCH

\protect\hyperlink{site-content}{Skip to
content}\protect\hyperlink{site-index}{Skip to site index}

\href{https://www.nytimes3xbfgragh.onion/section/politics}{Politics}

\href{https://myaccount.nytimes3xbfgragh.onion/auth/login?response_type=cookie\&client_id=vi}{}

\href{https://www.nytimes3xbfgragh.onion/section/todayspaper}{Today's
Paper}

\href{/section/politics}{Politics}\textbar{}White House Seeks to Move On
From Abuse Scandal. But What Did It Learn?

\url{https://nyti.ms/2C881yw}

\begin{itemize}
\item
\item
\item
\item
\item
\end{itemize}

Advertisement

\protect\hyperlink{after-top}{Continue reading the main story}

Supported by

\protect\hyperlink{after-sponsor}{Continue reading the main story}

White House Memo

\hypertarget{white-house-seeks-to-move-on-from-abuse-scandal-but-what-did-it-learn}{%
\section{White House Seeks to Move On From Abuse Scandal. But What Did
It
Learn?}\label{white-house-seeks-to-move-on-from-abuse-scandal-but-what-did-it-learn}}

\includegraphics{https://static01.graylady3jvrrxbe.onion/images/2018/02/20/us/20dc-memo1/20dc-memo1-articleLarge.jpg?quality=75\&auto=webp\&disable=upscale}

By \href{http://www.nytimes3xbfgragh.onion/by/maggie-haberman}{Maggie
Haberman} and
\href{https://www.nytimes3xbfgragh.onion/by/julie-hirschfeld-davis}{Julie
Hirschfeld Davis}

\begin{itemize}
\item
  Feb. 19, 2018
\item
  \begin{itemize}
  \item
  \item
  \item
  \item
  \item
  \end{itemize}
\end{itemize}

On Day 10 of a scandal over spousal abuse that laid bare the internal
dysfunction and public relations miscalculations that have racked the
Trump White House, John F. Kelly, the chief of staff, issued a memo
widely seen as an effort to turn the page.

The five-page document ordered a review of security clearance procedures
that had allowed Rob Porter,
\href{https://www.nytimes3xbfgragh.onion/2018/02/07/us/politics/rob-porter-resigns-abuse-white-house-staff-secretary.html}{the
now-fired staff secretary}, to handle highly classified materials even
though the abuse allegations had prevented him from receiving a
permanent clearance.

But Mr. Kelly's memo was more than just an attempt to move on, one White
House official said. It was tantamount to a recognition that a system
for dealing with a problem like the one involving Mr. Porter did not
exist.

In a West Wing where senior officials have developed something of a
bunker mentality to keep the chaos at bay and survive each day, this
better-not-to-know approach allowed the Porter problem to fester and
raises questions about whether the White House is capable of creating a
system with greater accountability.

White House aides insist that the Kelly memo is not just a bid for
better headlines --- it would dramatically improve a clearance process
that had gotten out of hand, they said. But what Mr. Kelly has not
publicly proposed is investigating precisely how the unsettling episode
occurred.

There would be much to investigate. A few senior officials in the West
Wing had access to details of the allegations of physical and emotional
abuse made by Mr. Porter's ex-wives. A greater number of people had a
sense that something was amiss, but chose to avert their gaze instead of
asking questions.

The first hours after the issue was brought to light by
\href{http://www.dailymail.co.uk/news/article-5359731/Ex-wife-Rob-Porter-Trumps-secretary-tells-marriage.html}{an
article in The Daily Mail} were a confused blur, with officials like Mr.
Kelly at first endorsing Mr. Porter's resignation, but then agreeing to
issue statements in support of him. That disarray belied the belief of
some in the White House that someone --- Mr. Kelly --- was finally in
charge after six months of drama and infighting in 2017. In reality,
these officials now concede, no one truly was.

``They haven't figured out how the place operates, and apparently they
don't want to learn,'' said John Dean, a White House counsel under
President Richard M. Nixon. The Porter situation, he added, ``is a
manifestation of what happens when you have chaos.''

Mr. Kelly, a four-star Marine general, had been
\href{https://www.nytimes3xbfgragh.onion/2017/07/28/us/politics/john-kelly-chief-of-staff-donald-trump.html}{billed
as uniquely qualified} to bring order to Mr. Trump's world when he took
his post in July. Now he faces a morale crisis in the West Wing, where
aides describe a sense of betrayal by a chief of staff they no longer
trust after his claims that he had not fully known about Mr. Porter's
problems and had acted within minutes once he learned of them.

But above all, White House veterans say, President Trump is responsible
for the haphazard nature with which his operation has functioned, in
part because he knew and cared little for the rules and norms that
govern the executive branch. Unprepared to fill the ranks of a new
administration, Mr. Trump never tried to set a tone of discipline or
ethical rigor in his West Wing.

``A White House reflects the president,'' said William M. Daley, a chief
of staff under Mr. Obama. ``So he moves on to the next scandal, crisis,
attack, whatever, and no self-reflection. So why should anyone be
surprised that that sort of becomes the M.O., even when there's a belief
that there's institutional stuff one should consider.''

For 13 months, Mr. Trump's aides have approached their mercurial and
process-averse boss by putting their heads down and ignoring what they
could not control, avoiding information that could rattle their daily
balance.

But in the case of Mr. Porter, White House aides privately acknowledge,
some advisers appear to have chosen to try to protect a colleague who
was generally well liked and, more important, too competent to lose. Mr.
Trump himself often expressed his regard for Mr. Porter by describing
him as ``out of central casting.''

\includegraphics{https://static01.graylady3jvrrxbe.onion/images/2018/02/20/us/20dc-memo2/20dc-memo2-articleLarge.jpg?quality=75\&auto=webp\&disable=upscale}

In the hours after the allegations against Mr. Porter became public,
some senior officials in fact hesitated to force him out, in part
because they argued that if one top adviser could be felled by
accusations, they could all potentially be vulnerable. Mr. Porter denied
privately to his colleagues, as well as publicly, that he had ever
physically abused his wives.

What distinguished this crisis from the ones that preceded it was that
it was not localized to one official, such as Stephen K. Bannon, the
isolated chief strategist who
\href{https://www.nytimes3xbfgragh.onion/2017/08/18/us/politics/steve-bannon-trump-white-house.html}{left
in August}, or Reince Priebus, the often uninformed chief of staff who
departed a month before that. Mr. Porter's case had spread to so many
aspects of the White House that it ensnared a number of staff members.

While multiple officials in the West Wing were aware that he had been
unable to pass his background check, none appear to have made an effort
to ascertain precisely why and take action on it. Several former White
House counsels have said they would have made it their business to know
if an official at Mr. Porter's level was facing allegations as serious
as domestic abuse, but officials in the White House have privately
maintained that Donald F. McGahn II, the White House counsel, never did.

``They plainly abided a situation that was intolerable, and they
shouldn't have done it,'' said Robert F. Bauer, a White House counsel
under President Barack Obama.

C. Boyden Gray, who served as White House counsel under President George
Bush, said the problems might have been exacerbated by the existence of
the special counsel investigation into Russian election meddling and
potential Trump campaign ties to it.

Mr. McGahn and other senior officials may be reluctant to engage with
the F.B.I. or the Justice Department on security clearances or any other
matter, for fear of being drawn into the inquiry, he said. ``You just
didn't want to call the F.B.I. or call the D.O.J. on any subject, Russia
or not,'' Mr. Gray said of officials in the West Wing. ``That may have
contributed to the breakdown here.''

What's more,
\href{https://www.nytimes3xbfgragh.onion/2018/02/03/us/politics/trump-fbi-justice.html}{Mr.
Trump has been in open war with the F.B.I.} for months, a fact that some
West Wing aides believe contributed to the debacle.

The dismissal on Wednesday of George David Banks, who was the White
House point person on climate change, was seen as a particularly galling
move after 13 months in which nobody in the West Wing had seemed
particularly concerned about enforcing a standard for past conduct.

Mr. Banks had been told that he would not be granted a full security
clearance. In an interview, he said the reason was that he had smoked
marijuana a handful of times between 2010 and 2013, conduct that he said
he had self-reported to the F.B.I. nearly a year ago.

``It was unbelievable,'' Mr. Banks said of the decision to deny him a
clearance, which he learned about abruptly at the end of a workday, when
an official from Mr. McGahn's office and a human resources aide asked
for a meeting later that evening. After informing him of their decision
and allowing Mr. Banks to resign, they asked him to gather his things
and escorted him from the premises.

Inside the West Wing, officials said, Mr. Banks is seen as collateral
damage in a belated effort by Mr. Kelly and other top officials to show
they are cracking down on interim security clearances.

Yet former White House advisers from both parties have been mystified at
the ineptitude of Mr. Trump's team in dealing with the fallout from Mr.
Porter's situation.

Part of the problem, they argue, is a culture at the White House that
does not appear to prioritize getting at the truth.

``The fundamentals of Crisis Management 101 were certainly not observed
here, but more importantly, there does not seem to be a priority on
marshaling and disclosing the facts, whatever they may be,'' said Jack
Quinn, who served as White House counsel under Bill Clinton during the
Monica Lewinsky affair. ``They need to stop the bleeding, and I think
that they've got to understand that the bleeding is not going to stop
until the full truth is told.''

Advertisement

\protect\hyperlink{after-bottom}{Continue reading the main story}

\hypertarget{site-index}{%
\subsection{Site Index}\label{site-index}}

\hypertarget{site-information-navigation}{%
\subsection{Site Information
Navigation}\label{site-information-navigation}}

\begin{itemize}
\tightlist
\item
  \href{https://help.nytimes3xbfgragh.onion/hc/en-us/articles/115014792127-Copyright-notice}{©~2020~The
  New York Times Company}
\end{itemize}

\begin{itemize}
\tightlist
\item
  \href{https://www.nytco.com/}{NYTCo}
\item
  \href{https://help.nytimes3xbfgragh.onion/hc/en-us/articles/115015385887-Contact-Us}{Contact
  Us}
\item
  \href{https://www.nytco.com/careers/}{Work with us}
\item
  \href{https://nytmediakit.com/}{Advertise}
\item
  \href{http://www.tbrandstudio.com/}{T Brand Studio}
\item
  \href{https://www.nytimes3xbfgragh.onion/privacy/cookie-policy\#how-do-i-manage-trackers}{Your
  Ad Choices}
\item
  \href{https://www.nytimes3xbfgragh.onion/privacy}{Privacy}
\item
  \href{https://help.nytimes3xbfgragh.onion/hc/en-us/articles/115014893428-Terms-of-service}{Terms
  of Service}
\item
  \href{https://help.nytimes3xbfgragh.onion/hc/en-us/articles/115014893968-Terms-of-sale}{Terms
  of Sale}
\item
  \href{https://spiderbites.nytimes3xbfgragh.onion}{Site Map}
\item
  \href{https://help.nytimes3xbfgragh.onion/hc/en-us}{Help}
\item
  \href{https://www.nytimes3xbfgragh.onion/subscription?campaignId=37WXW}{Subscriptions}
\end{itemize}
