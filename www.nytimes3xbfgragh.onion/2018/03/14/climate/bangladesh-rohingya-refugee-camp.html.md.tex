Sections

SEARCH

\protect\hyperlink{site-content}{Skip to
content}\protect\hyperlink{site-index}{Skip to site index}

\href{/section/climate}{Climate}\textbar{}The Biggest Refugee Camp
Braces for Rain: `This Is Going to Be a Catastrophe'

\url{https://nyti.ms/2Dpk72k}

\begin{itemize}
\item
\item
\item
\item
\item
\item
\end{itemize}

\hypertarget{climate-and-environment}{%
\subsubsection{\texorpdfstring{\href{https://www.nytimes3xbfgragh.onion/section/climate?name=styln-climate\&region=TOP_BANNER\&block=storyline_menu_recirc\&action=click\&pgtype=Article\&impression_id=dd6269a0-f2b8-11ea-b60c-6f967e7f947d\&variant=undefined}{Climate
and
Environment}}{Climate and Environment}}\label{climate-and-environment}}

\begin{itemize}
\tightlist
\item
  \href{https://www.nytimes3xbfgragh.onion/interactive/2020/08/24/climate/racism-redlining-cities-global-warming.html?name=styln-climate\&region=TOP_BANNER\&block=storyline_menu_recirc\&action=click\&pgtype=Article\&impression_id=dd6290b0-f2b8-11ea-b60c-6f967e7f947d\&variant=undefined}{Environmental
  Racism}
\item
  \href{https://www.nytimes3xbfgragh.onion/interactive/2020/climate/trump-environment-rollbacks.html?name=styln-climate\&region=TOP_BANNER\&block=storyline_menu_recirc\&action=click\&pgtype=Article\&impression_id=dd6290b1-f2b8-11ea-b60c-6f967e7f947d\&variant=undefined}{Trump's
  Changes}
\item
  \href{https://www.nytimes3xbfgragh.onion/interactive/2020/04/19/climate/climate-crash-course-1.html?name=styln-climate\&region=TOP_BANNER\&block=storyline_menu_recirc\&action=click\&pgtype=Article\&impression_id=dd6290b2-f2b8-11ea-b60c-6f967e7f947d\&variant=undefined}{Climate
  101}
\item
  \href{https://www.nytimes3xbfgragh.onion/interactive/2018/08/30/climate/how-much-hotter-is-your-hometown.html?name=styln-climate\&region=TOP_BANNER\&block=storyline_menu_recirc\&action=click\&pgtype=Article\&impression_id=dd6290b3-f2b8-11ea-b60c-6f967e7f947d\&variant=undefined}{Is
  Your Hometown Hotter?}
\end{itemize}

\includegraphics{https://static01.graylady3jvrrxbe.onion/images/2018/03/12/climate/cli-bangladesh-topper-poster/cli-bangladesh-topper-poster-articleLarge.jpg?quality=75\&auto=webp\&disable=upscale}

\hypertarget{the-biggest-refugee-camp-braces-for-rain-this-is-going-to-be-a-catastrophe}{%
\section{The Biggest Refugee Camp Braces for Rain: `This Is Going to Be
a
Catastrophe'}\label{the-biggest-refugee-camp-braces-for-rain-this-is-going-to-be-a-catastrophe}}

More than half a million Rohingya refugees face looming disaster from
floods and landslides when the first storms of the monsoon season hit
their camp in Bangladesh.

With a population of nearly 600,000, the settlement near Cox's Bazar in
Bangladesh is the largest refugee camp in the world.Credit...United
Nations High Commissioner for Refugees

Supported by

\protect\hyperlink{after-sponsor}{Continue reading the main story}

By \href{http://www.nytimes3xbfgragh.onion/by/somini-sengupta}{Somini
Sengupta} and
\href{http://www.nytimes3xbfgragh.onion/by/henry-fountain}{Henry
Fountain}

\begin{itemize}
\item
  March 14, 2018
\item
  \begin{itemize}
  \item
  \item
  \item
  \item
  \item
  \item
  \end{itemize}
\end{itemize}

The world's largest refugee camp, a temporary home to more than half a
million people that sprawls precariously across barren hills in
southeastern Bangladesh, faces a looming disaster as early as April when
the first storms of the monsoon season hit, aid workers warn.

``It's going to be landslides, flash floods, inundation,'' said Tommy
Thompson, chief of emergency support and response for the World Food
Program. ``It's going to be a very, very challenging wet season. That's
if we don't have a cyclone.''

\includegraphics{https://static01.graylady3jvrrxbe.onion/images/2018/03/15/world/15cli-bangladesh-print/merlin_135190206_2db28e74-17cb-4d22-9693-cf90f27fb47e-articleLarge.jpg?quality=75\&auto=webp\&disable=upscale}

Nearly
\href{https://reliefweb.int/sites/reliefweb.int/files/resources/180225_weeklyiscg_sitrep_final.pdf}{600,000
Rohingya Muslim refugees live in the camp}, at Cox's Bazar, near the
southern tip of Bangladesh. Cyclones, which can occur from March to
July, would considerably worsen the situation beyond the dangers of
flooding and landslides.

The Rohingya camp --- known officially as the Kutupalong-Balukhali
settlements, and informally as the megacamp --- is the most urgent
example of the new calamities that come with the global refugee crisis:
a huge influx of desperate people fleeing war or persecution, only to
face natural disaster in an ecologically fragile area potentially made
more precarious by climate change.

``We can definitely see how this is going to be a catastrophe, no matter
what,'' said Mélody Braun, who studies risk reduction strategies at the
International Research Institute for Climate and Society at Columbia
University and is visiting the camp this week. ``There's really no
space. People are everywhere. Slopes are really high.''

Before the Rohingya started crossing into Bangladesh from Myanmar in
large numbers in the summer, fleeing attacks on their villages by the
army and allied mobs, the hills were dotted with forest.

Image

Bamboo poles, used for building basic shelters, were unloaded near the
settlement.Credit...Adriane Ohanesian for The New York Times

But then, in a matter of weeks, as refugees poured in by the tens of
thousands, trees were hacked away. Canals were dug. Bamboo-and-tarp
shacks went up. More trees were cut as refugees scrambled to find
firewood.

The hills, where elephants recently roamed, are now bare. Even the roots
have been pulled out, leaving nothing to hold the parched soil together
as rainwater washes downhill, potentially taking tents and people with
it and quickly inundating low-lying settlements. The United Nations says
\href{https://www.nytimes3xbfgragh.onion/2018/02/13/world/asia/rohingya-monsoons-myanmar-bangladesh.html}{100,000
refugees are at acute risk} from landslides and floods.

The early rains --- known in Bengali as kalboishakhi, which translates
loosely as the storms of an ``evil summer'' --- are a precursor to the
full-on monsoons. They strike when the soil is still dry and especially
susceptible to mudslides. The only warning of their approach is usually
hot winds that send the dry earth of summer swirling through the air.

``You have whirlwinds of dust,'' said Iffat Nawaz, a spokeswoman for
BRAC, an international relief agency that is based in Bangladesh.
``Suddenly it gets dark in middle of the day and it pours. We usually
welcome that. It's cooling. This year in the middle of the refugee
crisis, it's not something to look forward to.''

Image

A girl collected water from a borehole.Credit...Adriane Ohanesian for
The New York Times

Southeastern Bangladesh is already one of the wettest parts of a wet
country, with 12 feet of rain on average every year. A warming
atmosphere can hold more moisture and unleash more intense downpours,
and make wet places even wetter. That may already be happening in and
around Cox's Bazar. Total pre-monsoon rainfall in the region has
increased by about one inch every five years over the past five decades,
a 2014
\href{http://teacher.buet.ac.bd/akmsaifulislam/reports/Heavy_Rainfall_report.pdf}{study
by researchers from Bangladesh University of Engineering and Technology}
found.

When the rains come, latrines are likely to overflow, bringing the risk
of cholera and other waterborne diseases.

Image

The United Nations has hired refugees to clean drainage ditches in an
effort to keep them clear.Credit...Adriane Ohanesian for The New York
Times

In February, United Nations agencies began dispatching engineering crews
to clear blocked sewage canals at risk of overflowing in the rainy
season. They distributed compressed rice husk, an alternative to
firewood, but that met only a small fraction of the refugees' needs. The
government has yet to allow the United Nations refugee agency to
distribute gas stoves that would decrease the demand for firewood.

Bangladesh, one of the poorest, most densely populated countries in the
world, opened its borders to the Rohingya in August, when they began
arriving with tales of
\href{https://www.nytimes3xbfgragh.onion/2017/09/02/world/asia/rohingya-myanmar-bangladesh-refugees-massacre.html}{massacres
by the Myanmar military}. They crossed a swollen river to get to safety,
walking through muddy fields and sleeping under the open sky with their
children. The Bangladeshi government let them settle around an area
where there was already a relatively small Rohingya refugee camp.

A spokeswoman for the International Organization for Migration, which
manages the camp, said aid agencies were well aware of the natural
disaster risks, but that they struggled in the early months to provide
basic services and focused on immediate needs: water, food and shelter.

Image

A resident of the camp returned home with bundles of
firewood.Credit...Adriane Ohanesian for The New York Times

Bangladeshi and United Nations officials say they are preparing land
elsewhere to relocate roughly 100,000 refugees from the megacamp. In one
instance, the United Nations is leveling a hilly area allocated by the
government to relocate refugees. And officials are distributing more
tarp, bamboo and sandbags to refugees to shore up their tents before the
rains begin.

Could any of this have been prevented? Mr. Thompson said it would have
helped to set up a camp on flatter land and to prevent the clear-cutting
of the forest.

``We will be looking at decisions that were taken in the first six
months of this operation,'' he said by phone from Cox's Bazar. ``We
always learn from our previous experiences and get better. Having said
that, we find ourselves in circumstances we couldn't have imagined.''

Advertisement

\protect\hyperlink{after-bottom}{Continue reading the main story}

\hypertarget{site-index}{%
\subsection{Site Index}\label{site-index}}

\hypertarget{site-information-navigation}{%
\subsection{Site Information
Navigation}\label{site-information-navigation}}

\begin{itemize}
\tightlist
\item
  \href{https://help.nytimes3xbfgragh.onion/hc/en-us/articles/115014792127-Copyright-notice}{©~2020~The
  New York Times Company}
\end{itemize}

\begin{itemize}
\tightlist
\item
  \href{https://www.nytco.com/}{NYTCo}
\item
  \href{https://help.nytimes3xbfgragh.onion/hc/en-us/articles/115015385887-Contact-Us}{Contact
  Us}
\item
  \href{https://www.nytco.com/careers/}{Work with us}
\item
  \href{https://nytmediakit.com/}{Advertise}
\item
  \href{http://www.tbrandstudio.com/}{T Brand Studio}
\item
  \href{https://www.nytimes3xbfgragh.onion/privacy/cookie-policy\#how-do-i-manage-trackers}{Your
  Ad Choices}
\item
  \href{https://www.nytimes3xbfgragh.onion/privacy}{Privacy}
\item
  \href{https://help.nytimes3xbfgragh.onion/hc/en-us/articles/115014893428-Terms-of-service}{Terms
  of Service}
\item
  \href{https://help.nytimes3xbfgragh.onion/hc/en-us/articles/115014893968-Terms-of-sale}{Terms
  of Sale}
\item
  \href{https://spiderbites.nytimes3xbfgragh.onion}{Site Map}
\item
  \href{https://help.nytimes3xbfgragh.onion/hc/en-us}{Help}
\item
  \href{https://www.nytimes3xbfgragh.onion/subscription?campaignId=37WXW}{Subscriptions}
\end{itemize}
