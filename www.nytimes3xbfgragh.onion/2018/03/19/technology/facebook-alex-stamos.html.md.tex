Sections

SEARCH

\protect\hyperlink{site-content}{Skip to
content}\protect\hyperlink{site-index}{Skip to site index}

\href{https://www.nytimes3xbfgragh.onion/section/technology}{Technology}

\href{https://myaccount.nytimes3xbfgragh.onion/auth/login?response_type=cookie\&client_id=vi}{}

\href{https://www.nytimes3xbfgragh.onion/section/todayspaper}{Today's
Paper}

\href{/section/technology}{Technology}\textbar{}Facebook Exit Hints at
Dissent on Handling of Russian Trolls

\url{https://nyti.ms/2u3gNuk}

\begin{itemize}
\item
\item
\item
\item
\item
\item
\end{itemize}

Advertisement

\protect\hyperlink{after-top}{Continue reading the main story}

Supported by

\protect\hyperlink{after-sponsor}{Continue reading the main story}

\hypertarget{facebook-exit-hints-at-dissent-on-handling-of-russian-trolls}{%
\section{Facebook Exit Hints at Dissent on Handling of Russian
Trolls}\label{facebook-exit-hints-at-dissent-on-handling-of-russian-trolls}}

\includegraphics{https://static01.graylady3jvrrxbe.onion/images/2018/03/21/business/20STAMOS/merlin_135735063_e1713ceb-7155-48a9-bd7e-3b58798a2ac5-articleLarge.jpg?quality=75\&auto=webp\&disable=upscale}

By \href{https://www.nytimes3xbfgragh.onion/by/nicole-perlroth}{Nicole
Perlroth},
\href{https://www.nytimes3xbfgragh.onion/by/sheera-frenkel}{Sheera
Frenkel} and
\href{https://www.nytimes3xbfgragh.onion/by/scott-shane}{Scott Shane}

\begin{itemize}
\item
  March 19, 2018
\item
  \begin{itemize}
  \item
  \item
  \item
  \item
  \item
  \item
  \end{itemize}
\end{itemize}

As Facebook grapples with a backlash over its role in spreading
disinformation, an internal dispute over how to handle the threat and
the public outcry is resulting in the departure of a senior executive.

The impending exit of that executive --- Alex Stamos, Facebook's chief
information security officer --- reflects heightened leadership tension
at the top of the social network. Much of the internal disagreement is
rooted in how much Facebook should publicly share about how nation
states misused the platform and debate over organizational changes in
the run-up to the 2018 midterm elections, according to current and
former employees briefed on the matter.

Mr. Stamos, who plans to leave Facebook by August, had advocated more
disclosure around Russian interference of the platform and some
restructuring to better address the issues, but was met with resistance
by colleagues, said the current and former employees. In December, Mr.
Stamos's day-to-day responsibilities were reassigned to others, they
said.

Mr. Stamos said he would leave Facebook but was persuaded to stay
through August to oversee the transition of his responsibilities and
because executives thought his departure would look bad, the people
said. He has been overseeing the transfer of his security team to
Facebook's product and infrastructure divisions. His group, which once
had 120 people, now has three, the current and former employees said.

Mr. Stamos would be the first high-ranking employee to leave Facebook
since controversy over disinformation on its site. Company leaders ---
including Mark Zuckerberg, Facebook's chief executive, and Sheryl
Sandberg, the chief operating officer --- have struggled to address a
\href{https://www.nytimes3xbfgragh.onion/2018/03/18/us/cambridge-analytica-facebook-privacy-data.html}{growing
set of problems}, including Russian interference on the platform, the
rise of false news and the disclosure over the weekend that 50 million
of its user profiles had been harvested by Cambridge Analytica, a
voter-profiling company.

\begin{quote}
\emph{■}\href{https://www.nytimes3xbfgragh.onion/2018/03/17/us/politics/cambridge-analytica-trump-campaign.html?action=click\&module=Intentional\&pgtype=Article}{\emph{More
about how Cambridge Analytica collected Facebook data}}\emph{.}

\emph{■}
\href{https://www.nytimes3xbfgragh.onion/2018/03/18/us/cambridge-analytica-facebook-privacy-data.html?action=click\&module=Intentional\&pgtype=Article}{\emph{The
growing outcry on two continents over Facebook's role in misuse of
data.}} **
\end{quote}

The developments have taken a toll internally, said the seven people
briefed on the matter, who asked not to be identified because the
proceedings were confidential. Some of the company's executives are
weighing their own legacies and reputations as Facebook's image has
taken a beating. Several believe the company would have been better off
saying little about Russian interference and note that other companies,
such as Twitter, which have stayed relatively quiet on the issue, have
not had to deal with as much criticism.

One central tension at Facebook has been that of the legal and policy
teams versus the security team. The security team generally pushed for
more disclosure about how nation states had misused the site, but the
legal and policy teams have prioritized business imperatives, said the
people briefed on the matter.

``The people whose job is to protect the user always are fighting an
uphill battle against the people whose job is to make money for the
company,'' said Sandy Parakilas, who worked at Facebook enforcing
privacy and other rules until 2012 and now advises a nonprofit
organization called the
\href{https://www.nytimes3xbfgragh.onion/2018/02/04/technology/early-facebook-google-employees-fight-tech.html}{Center
for Humane Technology}, which is looking at the effect of technology on
people.

Mr. Stamos said in statement on Monday, ``These are really challenging
issues, and I've had some disagreements with all of my colleagues,
including other executives.'' On Twitter, he said he was ``still fully
engaged with my work at Facebook'' and acknowledged that his role has
changed, without addressing his future plans.

Facebook did not have a comment on the broader issues around Mr.
Stamos's departure.

Mr. Stamos joined Facebook from Yahoo in June 2015. He and other
Facebook executives, such as Ms. Sandberg, disagreed early on over how
proactive the social network should be in policing its own platform,
said the people briefed on the matter. In his statement, Mr. Stamos said
his relationship with Ms. Sandberg was ``productive.''

Mr. Stamos first put together a group of engineers to scour Facebook for
Russian activity in June 2016, the month the Democratic National
Committee announced it had been attacked by Russian hackers, the current
and former employees said.

By November 2016, the team had uncovered evidence that Russian
operatives had aggressively pushed DNC leaks and propaganda on Facebook.
That same month, Mr. Zuckerberg publicly dismissed the notion that fake
news influenced the 2016 election, calling it a ``pretty crazy idea.''

In the ensuing months, Facebook's security team found more Russian
disinformation and propaganda on its site, according to the current and
former employees. By the spring of 2017, deciding how much Russian
interference to disclose publicly became a major source of contention
within the company.

Mr. Stamos pushed to disclose as much as possible, while others
including Elliot Schrage, Facebook's vice president of communications
and policy, recommended not naming Russia without more ironclad
evidence, said the current and former employees.

A detailed memorandum Mr. Stamos wrote in early 2017 describing Russian
interference was scrubbed for mentions of Russia and winnowed into
a\href{https://fbnewsroomus.files.wordpress.com/2017/04/facebook-and-information-operations-v1.pdf}{blog
post} last April that outlined, in hypothetical terms, how Facebook
could be manipulated by a foreign adversary, they said. Russia was only
referenced in a vague footnote. That footnote acknowledged that
Facebook's findings did not contradict a declassified January 2017
report in which the director of national intelligence concluded Russia
had sought to undermine United States election, and Hillary Clinton in
particular.

Mr. Stamos said in his statement that ``we decided that the responsible
thing to do would be to make clear that our findings were consistent
with those released by the U.S. intelligence community, which clearly
connected the activity in their report to Russian state-sponsored
actors.''

But Facebook's decision to omit Russia backfired. Weeks later, a
\href{http://time.com/4783932/inside-russia-social-media-war-america/}{Time
magazine article} revealed that Russia had created fake accounts and
purchased fake ads to spread propaganda on the platform, allegations
that Facebook initially denied.

By last September, after Mr. Stamos's investigation had revealed further
Russian interference, Facebook was forced to reverse course. That month,
the company disclosed that beginning in June 2015, Russians had paid
Facebook \$100,000 to run roughly 3,000 divisive ads to show the
American electorate.

In response, lawmakers like Senator Mark Warner of Virginia, the top
Democrat on the intelligence committee, said that although Facebook's
revelation was a good first step, ``I'm disappointed it's taken 10
months of raising this issue before they've become much more
transparent.''

And the revelation also prompted more attention into how Russians had
manipulated the social network. Last October and November, Facebook was
grilled in front of lawmakers on Capitol Hill for Russian meddling on
its platform, along with executives from Twitter and YouTube.

The public reaction caused some at Facebook to recoil at revealing more,
said the current and former employees. Since the 2016 election, Facebook
has paid unusual attention to the reputations of Mr. Zuckerberg and Ms.
Sandberg, conducting polls to track how they are viewed by the public,
said Tavis McGinn, who was recruited to the company last April and
headed the executive reputation efforts through September 2017.

Mr. McGinn, who now heads Honest Data, which has done polling about
Facebook's reputation in different countries, said Facebook is ``caught
in a Catch-22.''

``Facebook cares so much about its image that the executives don't want
to come out and tell the whole truth when things go wrong,'' he said.
``But if they don't, it damages their image.''

Mr. McGinn said he left Facebook after becoming disillusioned with the
company's conduct.

By December 2017, Mr. Stamos, who reports to Facebook's general counsel,
proposed that he report directly to higher-ups. Facebook executives
rejected that proposal and instead reassigned Mr. Stamos's team,
splitting the security team between its product team, overseen by Guy
Rosen, and infrastructure team, overseen by Pedro Canahuati, according
to current and former employees.

Apart from managing a small team of engineers in San Francisco, Mr.
Stamos has largely been left as Facebook's security communicator. Last
month, he appeared as Facebook's representative at the Munich Security
Conference.

Over the weekend, after news broke that Cambridge Analytica had
harvested data on as many as 50 million Facebook users, Facebook's
communications team encouraged Mr. Stamos to tweet in defense of the
company, but only after it asked to approve Mr. Stamos's tweets,
according to two people briefed on the incident.

After the tweets set off a furious response, Mr. Stamos deleted them.

Roger B. McNamee, an early investor in Facebook who said he considered
himself a mentor to Mr. Zuckerberg, said the company was failing to face
the fundamental problems posed by the Russian meddling and other
manipulation of content.

``I told them, `Your business is based on trust, and you're losing
trust,''' said Mr. McNamee, a founder of the Center for Humane
Technology. ``They were treating it as a P.R. problem, when it's a
business problem. I couldn't believe these guys I once knew so well had
gotten so far off track.''

\begin{quote}
\emph{■}
\href{https://www.nytimes3xbfgragh.onion/2018/03/19/us/cambridge-analytica-alexander-nix.html?action=click\&module=Intentional\&pgtype=Article}{\emph{Cambridge
Analytica executives were caught on video offering to entrap clients'
opponents.}}

\emph{■}
\href{https://www.nytimes3xbfgragh.onion/2018/03/19/technology/facebook-cambridge-analytica-explained.html?action=click\&module=Intentional\&pgtype=Article}{\emph{How
Cambridge Analytica harvested Facebook data.}}
\end{quote}

Advertisement

\protect\hyperlink{after-bottom}{Continue reading the main story}

\hypertarget{site-index}{%
\subsection{Site Index}\label{site-index}}

\hypertarget{site-information-navigation}{%
\subsection{Site Information
Navigation}\label{site-information-navigation}}

\begin{itemize}
\tightlist
\item
  \href{https://help.nytimes3xbfgragh.onion/hc/en-us/articles/115014792127-Copyright-notice}{©~2020~The
  New York Times Company}
\end{itemize}

\begin{itemize}
\tightlist
\item
  \href{https://www.nytco.com/}{NYTCo}
\item
  \href{https://help.nytimes3xbfgragh.onion/hc/en-us/articles/115015385887-Contact-Us}{Contact
  Us}
\item
  \href{https://www.nytco.com/careers/}{Work with us}
\item
  \href{https://nytmediakit.com/}{Advertise}
\item
  \href{http://www.tbrandstudio.com/}{T Brand Studio}
\item
  \href{https://www.nytimes3xbfgragh.onion/privacy/cookie-policy\#how-do-i-manage-trackers}{Your
  Ad Choices}
\item
  \href{https://www.nytimes3xbfgragh.onion/privacy}{Privacy}
\item
  \href{https://help.nytimes3xbfgragh.onion/hc/en-us/articles/115014893428-Terms-of-service}{Terms
  of Service}
\item
  \href{https://help.nytimes3xbfgragh.onion/hc/en-us/articles/115014893968-Terms-of-sale}{Terms
  of Sale}
\item
  \href{https://spiderbites.nytimes3xbfgragh.onion}{Site Map}
\item
  \href{https://help.nytimes3xbfgragh.onion/hc/en-us}{Help}
\item
  \href{https://www.nytimes3xbfgragh.onion/subscription?campaignId=37WXW}{Subscriptions}
\end{itemize}
