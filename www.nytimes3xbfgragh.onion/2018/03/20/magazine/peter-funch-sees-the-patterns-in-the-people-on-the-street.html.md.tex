Sections

SEARCH

\protect\hyperlink{site-content}{Skip to
content}\protect\hyperlink{site-index}{Skip to site index}

\href{https://myaccount.nytimes3xbfgragh.onion/auth/login?response_type=cookie\&client_id=vi}{}

\href{https://www.nytimes3xbfgragh.onion/section/todayspaper}{Today's
Paper}

Peter Funch Sees the Patterns in the People on the Street

\url{https://nyti.ms/2GPv54I}

\begin{itemize}
\item
\item
\item
\item
\item
\end{itemize}

Advertisement

\protect\hyperlink{after-top}{Continue reading the main story}

Supported by

\protect\hyperlink{after-sponsor}{Continue reading the main story}

\href{/column/on-photography}{On Photography}

\hypertarget{peter-funch-sees-the-patterns-in-the-people-on-the-street}{%
\section{Peter Funch Sees the Patterns in the People on the
Street}\label{peter-funch-sees-the-patterns-in-the-people-on-the-street}}

\includegraphics{https://static01.graylady3jvrrxbe.onion/images/2018/03/25/magazine/25mag-onphoto1-diptych/25mag-onphoto1-diptych-articleLarge.jpg?quality=75\&auto=webp\&disable=upscale}

By \href{https://www.nytimes3xbfgragh.onion/by/teju-cole}{Teju Cole}

\begin{itemize}
\item
  March 20, 2018
\item
  \begin{itemize}
  \item
  \item
  \item
  \item
  \item
  \end{itemize}
\end{itemize}

He is unaware he's being photographed. The unposed portrait has been
made in bright sunshine on a busy street, and we can see other people,
blurred, behind him. The man is tanned, with a head of thinning white
hair and a short white goatee. His collared shirt is pale, striped and
open at the neck. He has rosebud lips and somewhat worried brows that
make him appear lost in thought or on the verge of making a decision.
Out of the flux of the street, a unique event has been preserved: this
man, this moment, this mien.

Now look at another portrait. It's the same man. Placed side by side
with the first portrait, it immediately raises new questions. The look
is almost the same: the tanned face, the small mouth, the dark, slightly
furrowed brows. With his narrowed eyes, he seems a bit more preoccupied.
His white goatee is fuller and more neatly shaped, giving him the
debonair look of a knight in a Renaissance painting. In this second
portrait, the man is all buttoned up, and he wears an ocher bow tie.
Behind him this time is a different crowd, and instead of the taxi seen
in the first picture, there is an armored truck.

It's not that hard to go out into the street and take a stranger's
picture. It is legal and, with the right equipment, technically simple.
But how do you arrive at two pictures of the same person, with almost
the same expression, on what seem to be different days? These
photographs were made by the Danish artist Peter Funch, and they are
part of a series of many such pairs. For nine years, from 2007 until
2016, Funch hung around Grand Central Terminal and watched commuters
during the morning rush between 8:30 and 9:30 a.m. Using a long-lensed
digital camera, he made countless portraits, an intriguing face here,
another one there, yet another over there. He began to notice
repetitions, the same people, the same faces, the same gestures, the
same clothes. Each person was in the self-enclosed reverie of getting
somewhere. The photos were all taken in May, June or July, in bright
summer sunshine. The resulting project, published last year in a
monograph titled ``42nd and Vanderbilt,'' is named for the street corner
on which Funch stationed himself. It contains dozens of pairs of
portraits (and a few in sequences of three), all of strangers.

Funch's book contains women and men, the old and the young, people of
all races and social classes. It looks like New York. One woman strides
by with her chin raised and eyes closed. She wears a green blouse in one
photo and a purple sweater in another. A man in a dark suit wears a
peach-colored shirt in one photo and a blue one in the other, and he
holds a cellphone in his right hand in both portraits. A woman crosses
her arms, twice; another smokes, twice; another smiles, twice. There are
two men walking side by side, and the same men walking side by side
again. Do they know each other? They seem not to. There are three young
women in medical scrubs, and surely they do know each other. One of them
checks her phone, twice; another glances downward, twice.

\includegraphics{https://static01.graylady3jvrrxbe.onion/images/2018/03/25/magazine/25mag-onphoto2-diptych/25mag-onphoto2-diptych-articleLarge-v2.jpg?quality=75\&auto=webp\&disable=upscale}

Funch's project is a feat of both patience and memory. But it is also a
record of the many individual rhythms that collectively make up city
life. ``42nd and Vanderbilt'' evokes ``Many Are Called,'' the series of
subway portraits made between 1938 and 1941 by Walker Evans. Evans sat
across from fellow passengers with a 35-millimeter Contax concealed in
his coat. He wasn't after repetition, but he did want to capture New
Yorkers unawares. ``The guard is down and the mask is off,'' he wrote.
``People's faces are in naked repose down in the subway.'' What he found
was a repetition across selves: when we are unguarded, all artifice is
gone and we begin to resemble one another.

This kind of photography is unquestionably invasive. After all, we ease
into our resting faces precisely at those moments when we don't expect
to be scrutinized. But in neither Evans's nor Funch's project is there
any sense of hostility toward their subjects. If the resulting
photographs are slightly discomfiting, they are also humane and
compassionate, because they make us see that even within the incessant
repetitions that constitute capitalist society, there is always a hidden
and highly personal pattern to our movements. What the Scottish poet
Thomas A. Clark wrote about walking in the country is applicable to city
commuters too:

Always, everywhere, people have

walked, veining the earth with

paths, visible and invisible, symmetri-

cal and meandering.

There are walks in which we tread in

the footsteps of others,

walks on which we strike out entirely

for ourselves.

\textbf{Many photographers,} since Evans's influential work, have been
tempted to try to see without being seen. They have come up with various
ways to record the poignant individuality of others against the backdrop
of public space, many arriving at the idea that transience is best
caught in literal moments of transit.

In the mid-1990s, John Schabel used a 500-millimeter lens to take
photographs of airplane passengers framed by the windows of the plane.
The work, ``Passengers,'' which wasn't published until 2011, captures an
eerie isolation in each frame, a grainy and pensive head behind the
glass. Also in the '90s, Luc Delahaye's series, ``L'Autre,'' revisited
Evans's approach and subject, but this time on the Paris Metro. Cropped,
the individual faces fill up the frame. The film director Chris Marker
also made a series of pictures of strangers on the Metro between 2008
and 2010, also called ``Passengers.'' Marker shot in wider views, in
color and with a digital camera, creating lo-fi images that were
poignant without being pretty.

Philip-Lorca diCorcia's ``Heads'' (2001) was made in Times Square with a
camera on a tripod and an elaborate strobe-light setup. The resulting
high-resolution portraits of strangers are poised between the cinematic
and the uncanny. One unhappy passer-by sued diCorcia for photographing
him without permission and for profiting from his image. It was an
important case: The court dismissed the suit, deciding that street
photographs count as a form of artistic expression that may not be
infringed.

The Dutch photographer Hans Eijkelboom is more interested in typologies
than in faces. His ``People of the Twenty-First Century'' (2014)
presents, in vast arrays, fashion choices in cities around the world:
fur hoods, trench coats, leopard prints, a certain kind of T-shirt or
hairstyle. And Peter Funch himself, in an earlier project, ``Babel
Tales'' (2006-11), created dense assemblages of repeated gestures or
outfits (people carrying yellow envelopes or black umbrellas, people
wearing red or posing for a photograph) by digitally layering a large
number of figures into a single street scene. When you encounter an
image from ``Babel Tales,'' it seems improbable and utterly mysterious.
How on earth was it made? Once you do know how, the mystery is solved.
The appeal of the work is largely in its technical achievement. But this
was the work that led Funch to ``42nd and Vanderbilt,'' a project that
tapped into some of the same energies, with less artfulness and more
art.

Image

``2012.06.27 08:33:09''; ``2012.06.08 08:25:58.''Credit...Peter Funch.
From V1 Gallery.

\textbf{The portraits in} ``42nd and Vanderbilt'' are labeled in
quasi-scientific manner with the date and time they were taken. The
first portrait of the tanned man with the white goatee, for instance, is
captioned ``2007.06.28 08:59:39.'' The people photographed most likely
could not say exactly where they were at that time or what they were
wearing or what sort of look was on their faces. The photographer,
having recorded these things, knows them all, while knowing very little
else about the person photographed, not even his or her name. The
caption for the second portrait of this pair reveals a particular
surprise. While most of the image pairings in the book are indeed taken
days or weeks apart, this one had a gap of more than five years between
the first image and the second: ``2012.07.03 08:54:01.'' The subject has
aged and is still himself.

Looking more closely at ``42nd and Vanderbilt,'' I noticed another kind
of repetition. Many of the dates recur, and a dozen of the portraits
that made it into the book were taken on just one day: 2012.06.27. A man
wearing a checked red shirt, a man wearing a checked blue shirt, a woman
with dark brown hair and folded arms, a man wearing a burgundy tie and
looking directly at the camera, a woman wearing a delicate necklace, and
so on, all seen within the same hour. Street photography, like
basketball, can be streaky, and June 27, 2012, seems to have been a
great day for Funch. His intuition led him into an unpredictable and
productive dance. About one in every five matchups in ``42nd and
Vanderbilt'' contains a photo taken on that day.

Photography can remember for us what we didn't think or were not able to
remember for ourselves. I was in New York on Funch's productive day, but
I wasn't at Grand Central. I know what I was thinking about that day
because, quite by chance, it was my birthday. I know my head was filled
with the thoughts that tend to preoccupy me each year on the anniversary
of my birth: the mystery of time, the habits of a self, how the face in
the mirror has changed, the meaning of our lives with others, how
beautiful it all is and how soon it will all be gone.

Advertisement

\protect\hyperlink{after-bottom}{Continue reading the main story}

\hypertarget{site-index}{%
\subsection{Site Index}\label{site-index}}

\hypertarget{site-information-navigation}{%
\subsection{Site Information
Navigation}\label{site-information-navigation}}

\begin{itemize}
\tightlist
\item
  \href{https://help.nytimes3xbfgragh.onion/hc/en-us/articles/115014792127-Copyright-notice}{©~2020~The
  New York Times Company}
\end{itemize}

\begin{itemize}
\tightlist
\item
  \href{https://www.nytco.com/}{NYTCo}
\item
  \href{https://help.nytimes3xbfgragh.onion/hc/en-us/articles/115015385887-Contact-Us}{Contact
  Us}
\item
  \href{https://www.nytco.com/careers/}{Work with us}
\item
  \href{https://nytmediakit.com/}{Advertise}
\item
  \href{http://www.tbrandstudio.com/}{T Brand Studio}
\item
  \href{https://www.nytimes3xbfgragh.onion/privacy/cookie-policy\#how-do-i-manage-trackers}{Your
  Ad Choices}
\item
  \href{https://www.nytimes3xbfgragh.onion/privacy}{Privacy}
\item
  \href{https://help.nytimes3xbfgragh.onion/hc/en-us/articles/115014893428-Terms-of-service}{Terms
  of Service}
\item
  \href{https://help.nytimes3xbfgragh.onion/hc/en-us/articles/115014893968-Terms-of-sale}{Terms
  of Sale}
\item
  \href{https://spiderbites.nytimes3xbfgragh.onion}{Site Map}
\item
  \href{https://help.nytimes3xbfgragh.onion/hc/en-us}{Help}
\item
  \href{https://www.nytimes3xbfgragh.onion/subscription?campaignId=37WXW}{Subscriptions}
\end{itemize}
