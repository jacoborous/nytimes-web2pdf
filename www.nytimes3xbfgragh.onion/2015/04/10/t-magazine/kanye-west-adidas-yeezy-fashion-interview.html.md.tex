Sections

SEARCH

\protect\hyperlink{site-content}{Skip to
content}\protect\hyperlink{site-index}{Skip to site index}

\href{https://myaccount.nytimes3xbfgragh.onion/auth/login?response_type=cookie\&client_id=vi}{}

\href{https://www.nytimes3xbfgragh.onion/section/todayspaper}{Today's
Paper}

The Agony and the Ecstasy of Kanye West

\begin{itemize}
\item
\item
\item
\item
\item
\item
\end{itemize}

Advertisement

\protect\hyperlink{after-top}{Continue reading the main story}

Supported by

\protect\hyperlink{after-sponsor}{Continue reading the main story}

\hypertarget{the-agony-and-the-ecstasy-of-kanye-west}{%
\section{The Agony and the Ecstasy of Kanye
West}\label{the-agony-and-the-ecstasy-of-kanye-west}}

By \href{http://www.nytimes3xbfgragh.onion/by/jon-caramanica}{Jon
Caramanica}

\begin{itemize}
\item
  April 10, 2015
\item
  \begin{itemize}
  \item
  \item
  \item
  \item
  \item
  \item
  \end{itemize}
\end{itemize}

\emph{For over a decade, the rap superstar has made music that pushes
boundaries, courts controversy and divides critics. Now, the man who has
compared himself to Jesus and Steve Jobs just wants to make clothes for
the masses.}

\includegraphics{https://static01.graylady3jvrrxbe.onion/images/2015/04/12/t-magazine/12culture-well-kanye-slide-0MPJ/12culture-well-kanye-slide-0MPJ-articleLarge.jpg?quality=75\&auto=webp\&disable=upscale}

On a Sunday afternoon in early February, Kanye West was in a makeshift
conference room at the downtown New York showroom of Adidas, mapping out
his vision for fashion, and everything else, too. It was three days
after the presentation of his first sportswear collection, Yeezy Season
1, at New York Fashion Week. Produced in collaboration with Adidas
Originals, Yeezy was the culmination of more than a decade of striving,
self-teaching, self-humbling and agitating for attention.

The Yeezy
\href{http://tmagazine.blogs.nytimes3xbfgragh.onion/2015/02/20/fall-winter-2015-new-york-fashion-week-top-ten/}{presentation}
--- where Jay Z, Beyoncé, Rihanna, Alexander Wang, West's wife, Kim
Kardashian, and their squalling daughter, North, sat in the front row
alongside Anna Wintour --- was not a traditional runway show. Instead,
West had staged a phalanx of 50 models, many of them selected from an
open call. The aesthetic of the unisex clothes borrowed from
contemporary ``athleisure'' wear and traditional military surplus,
distorting familiar silhouettes and distilling high-minded influences
into street-ready looks. Not all the feedback was positive, but West
seemed unfazed. ``We destroyed the first village, the fashion village,''
he told the group of 10 or so, graphic designers and members of his
creative team obliged to attend a Sunday lunchtime meeting.

Image

West displays his left wrist, tattooed the previous day with his
mother's birthday; vintage coat,
\href{http://www.thevintageshowroom.com}{thevintageshowroom.com}.Credit...Photograph
by Juergen Teller. Styled by Joe McKenna

West, who is partial to lofty rhetoric, is most at ease when
sermonizing, delivering extemporaneous speeches that are part Vince
Lombardi, part Tony Robbins, part Martin Luther King Jr. (``They
classify my motivational speeches as rants!'' he has said.) As the group
listened, rapt, he segued from his plans to teach feng shui and color
theory in schools, to having passed on what he says was a multimillion
dollar partnership with Apple, to --- in language both admiring and
profane --- the surpassing perfection of Kris Jenner's progeny. And then
he got to his point.

``It's literally like . . . I know this is really harsh, but it's like
Before Yeezy and After Yeezy,'' West said. ``This is the new Rome!'' He
was referring to his thunderous arrival in the fashion world, to his
oft-mocked bid not merely to design clothes but to build, in his words,
``the biggest apparel company in human history.'' But he could just as
easily have been talking about his own life and his recent attempts at
self-transformation: his dogged efforts to remake himself, to find a
comfortable balance between the self-proclaimed genius and provocateur
with the hair-trigger temper he's been, and the more moderate,
approachable, self-controlled designer-of-the-people he's trying so
strenuously to become --- all without losing his essential Kanye-ness.

IN THE CAR on the way home from the meeting, West took a call from
Kardashian, the high priestess of reality television and America's
leading entrepreneur of the self, whose own towering fame has combined
with West's to create a historic blizzard of celebrity. When the couple
appeared together on the cover of Vogue last year, the move was,
depending on your perspective, a stroke of PR genius or a naked plea for
highbrow validation.

``It went really good up at Adidas today,'' West told Kardashian, and
they chitchatted for a minute like any married couple. Then he paused,
obviously listening to her. It was clear from his face that he enjoys
this role --- collaborator, producer, Pygmalion: ``I like the black
latex, also with the black fur, and then maybe with tights and the Alaïa
high lace-ups,'' he said.

West is one of the true music superstars of the 2000s, the rare artist
respected as both a pop musician and experimenter, renowned as much for
his creative endeavors as for his tabloid exploits. He has remade
hip-hop's sonic palette three, maybe four times. His musical legacy is
peerless. And yet, as accomplished as he is, West has, for the past five
years, openly sought success and acceptance in the world of fashion.
It's a pursuit that many see as a quixotic fixation, and has often been
poignant to watch: West, ever-earnest and transparent about his
sartorial ambitions, has attempted to launch himself in a new realm
where his massive, inescapable celebrity does not necessarily confer any
significant advantage.

Of course, musicians have been crossing over into fashion for decades,
capitalizing on their cool to do a T-shirt line or maybe a capsule
collection. Some have even tried to build companies: Think Jay Z with
Rocawear or Sean Combs with Sean John. But West wants something
different --- a seat not just in the front row (though he does want
that) but at the creative table.

Image

Vintage coat and T-shirt. \textbf{Acne Studios} jeans. \textbf{Kanye
West x Adidas Originals Yeezy Season 1} boots.Credit...Photograph by
Juergen Teller. Styled by Joe McKenna

A couple of years ago, in 2013, West could be found inveighing against
the gatekeepers he perceived as impediments to his success: the
designers who wouldn't collaborate with him, the financiers who wouldn't
back him. He did this in interviews as well as on stage, from the
60-foot tall mountain that was the centerpiece of his Yeezus tour. ``I
would scream --- `Look at this mountain I just made! You don't think I
can make a T-shirt?' '' he told me. `` 'Look at everyone in the audience
--- we're selling \$300,000 worth of T-shirts every night!' ''

To West, his struggle was at root one against skepticism and prejudice.
Because while it might be argued that his celebrity allowed for a
line-skipping of sorts, he feels it was more frequently an obstacle (a
sentiment that is perhaps not surprising from the man who has likened
celebrities' fight for privacy to the civil rights movement). ``Fame is
often looked down upon in the design world, so it's actually been
something I had to overcome,'' he wrote on Twitter when Fern Mallis, the
creator of New York Fashion Week, told The New York Post that she was
``kind of over Kanye'': ``I mean, I'm not a fan of his music, and the
attitude and the agenda are not my style.''

In West's telling, he's had to howl because fashion people weren't
listening, and he needed their ear. ``I one hundred percent had to
scream,'' he said. ``I tried it every other way.''

In fact, he had: He'd previously made inroads via collaborations with
Louis Vuitton and Nike (limited-edition sneakers), as well as with
A.P.C. (two small collections that included jeans, T-shirts and
sweatshirts). He also, in
\href{http://runway.blogs.nytimes3xbfgragh.onion/2011/10/01/kanye-west-show-shrouded-in-silence-creates-a-scene/}{2011}
and 2012, presented two women's ready-to-wear collections that were
pilloried for their amateurism. But those were arguably ideas born of
the old Kanye, the one for whom luxury and exclusivity were the ultimate
goals. New Kanye wants everyone to have a taste of luxury, but without
the hefty price tag. He aspires to bring forward-thinking clothes to the
masses. Clearly fast fashion has been done --- and successfully, if not
always ethically --- by retailers like Zara and H\&M. But West wants to
design what might be called fast high fashion: clothes that are truly
avant-garde in their design, made from the finest materials, and that
would arrive with lightning-quick speed in stores where they could be
bought by the public at affordable prices. The Adidas deal is one step
--- his contract guarantees him a retail location, he said, and stores
have begun placing orders --- toward a future he's still working out.

West's overall ambition is to be to fashion what he is to music: a
mainstream innovator, a translator of tomorrow's ideas for today.
``Before the Internet, music was really expensive. People would use a
rack of CDs to show class, to show they had made it,'' West said at one
point. ``Right now, people use clothes to telegraph that. I want to
destroy that. The very thing that supposedly made me special --- the
jacket that no one could get, the direct communications with the
designers --- I want to give that to the world.'' Needless to say, there
are plenty of differences between the path to success in either field.
These days, a good song can travel a near-frictionless journey from
creation to consumption. It's harder to get from the fringes to the
center in fashion; a designer needs money, infrastructure and channels
of distribution for his or her work to get seen. Plus, it's a world
where exclusivity has cachet. When West says he wants everyone to have
access to high-end style, there are plenty who find the idea the very
antithesis of luxury.

Because the Yeezy collection is sportswear, there were no suits, no
tailored trousers or collared shirts. The looks shown at NYFW were a
streamlined, democratized version of what West (who has said, of his
personal style, ``I want to dress like a child as much as possible'')
usually wears. Lately, that's often been a velour sweatshirt by Haider
Ackermann (retail price: \$768), topped with a modified MA-1 bomber
jacket by Takahiro Miyashita (\$1,778). This is not, West clarifies, the
level of affordability he's striving for in the clothes he's making. He
claims that he's not wearing luxury for luxury's sake but rather as a
form of research. ``There's a transition,'' he says. ``I need to partake
in what's of value and of quality and soul in order to understand it, in
order to give it back.''

Image

Credit...Photograph by Juergen Teller. Styled by Joe McKenna

A COUPLE OF DAYS later, it was the middle of Fashion Week, and West was
in the lobby of the Mercer Hotel, killing time before the designer
Jeremy Scott's runway show, one of several front rows he'd enhance. It
was a far cry from 2009, when he and a few flamboyantly dressed friends
barnstormed their way through the Paris men's collections. ``That was
the beginning of the sit-in,'' he recalled, likening his quest for
fashion-world access to an act of social justice.

``I dreamed, since I was a little kid, of having my own store where I
could curate every shoe, sweatshirt and color,'' he said. ``I have
sketches of it. I cried over the idea of having my own store.''

Kanye West was born in 1977, and raised primarily in Chicago. His father
was a onetime Black Panther turned photojournalist; his mother was a
college professor. He grew up with creative pursuits and social politics
always on the agenda.

He draws a direct line between the sense of justice he was raised with
and his quest to do away with elitism in fashion. ``I'm not a celebrity,
I'm an activist,'' he says. ``The fact that when I see truth it's really
hard for me to sit back and just allow it to happen in front of me on my
clock makes me, a lot of times, a bad celebrity.''

Or maybe the best celebrity. West has always operated without a filter,
and has long been one of pop culture's great disrupters. He announced,
``George Bush doesn't care about black people'' into the camera on a
live Hurricane Katrina telethon and rushed the stage to interrupt Taylor
Swift's acceptance speech for Best Female Video at the 2009 MTV Video
Music Awards, saying, ``I'ma let you finish but Beyoncé had one of the
best videos of all time!'' In 2012, when the designer Hedi Slimane
reportedly said West could only attend his first runway show for Saint
Laurent and no others at Paris Fashion Week --- a fairly standard
request --- West took offense and said so, openly and repeatedly; the
two still don't speak. ``I'm not angry anymore,'' West said, ``but I had
to get my anger out.'' A few moments later, he pulls up a potential
cover image for a forthcoming single: It's a photo of the Saint Laurent
store in Chicago after it was robbed last year, its front window
shattered, the logo fractured.

Generally, though, the current-day West seems tempered --- at least
somewhat. Maybe this is owing to his newfound domesticity. He takes his
role as a husband and father seriously. ``I feel like now I have an
amazing wife, a supersmart child and the opportunity to create in two
major fields,'' he said. ``Before I had those outlets, my ego was all I
had.'' But he also speaks ``all the time'' to a doctor who specializes
in anger management therapy, a fortuitous byproduct of an altercation
with a paparazzo at Los Angeles International Airport. (He had two such
incidents; the second time he was court-ordered to anger management.)

He claims he's trying harder to let things go. When Beck beat out
Beyoncé for Best Album at the Grammys in February, West walked on stage
in a near-farcical echo of what he'd done to Taylor Swift, but then
thought better of it and returned to his seat. (He later
\href{http://www.nytimes3xbfgragh.onion/aponline/2015/02/26/us/ap-us-people-kanye-west.html}{apologized
to Beck} on Twitter.) And adversaries are being greeted with warmth,
which may actually be shrewdness. On Twitter, he invited Fern Mallis to
meet: ``If you wanna have a drink with me, book a table at the spotted
pig when I'm back in NY.'' More recently, after publicly chiding Bernard
Arnault, LVMH Chairman and C.E.O., for refusing to take a meeting with
him, West arranged a series of impromptu concerts through Arnault's
22-year-old son, Alexandre, and performed them at Fondation Louis
Vuitton in Paris. The elder Arnault attended the first concert, later
congratulating West backstage. (He got his meeting.)

West's newly mellowed self is also beginning to come through in his
music. Following the raw scrape of industrial noise-rap that was the
``Yeezus'' album, there was
``\href{https://www.youtube.com/watch?v=WibQR0tQ0P8}{Only One},'' a
tender number sung from the perspective of his late mother, Donda (who
died unexpectedly in 2007) and recorded with Paul McCartney. Then came
``\href{https://www.youtube.com/watch?v=kt0g4dWxEBo}{FourFiveSeconds},''
a stripped-down folk song with Rihanna and McCartney.

``I have this table in my new house,'' West said, offering a parable.
``They put this table in without asking. It was some weird nouveau riche
marble table, and I hated it. But it was literally so heavy that it took
a crane to move it. We would try to set up different things around it,
but it never really worked.

``I realized that table was my ego. No matter what you put around it,
under it, no matter who photographed it, the douchebaggery would always
come through.''

Image

\textbf{Off-White c/o Virgil Abloh} jacket,
\href{http://www.off---white.com}{off-\/-\/-white.com}. \textbf{Fear of
God L.A.} sweater, \href{http://www.fearofgodla.com}{fearofgodla.com}.
\textbf{Kanye West x Adidas Originals Yeezy Season 1} hooded sweatshirt.
Vintage T-shirt.Credit...Photograph by Juergen Teller. Styled by Joe
McKenna

WEST RESISTS CALLING himself a designer --- out of humility, maybe, or
to pre-empt criticism. But he has certainly invested time (and plenty of
his own money) to teach himself the business. In 2009, he and longtime
collaborator Virgil Abloh interned at Fendi to learn how a fashion house
operates. And like many established designers, West took counsel from
the late Louise Wilson, the Central Saint Martins professor who mentored
Alexander McQueen and Mary Katrantzou, and who died last year. His
poorly received ready-to-wear women's collections were paid for
completely out of pocket, which he says put him in debt. ``I gained
because I had the privilege to be educated,'' he now says. ``I had
enough of a value to be able to go into debt, and that was a blessing.
Some people don't even have the opportunity to be able to go into
debt.''

He also seems to be constantly looking for ways to improve. Just after
the Jeremy Scott show, he slipped into
\href{http://www.sweetwilliamltd.com/}{Sweet William}, a children's
store in NoLIta, to buy some stuffed animals for North. After selecting
a pig, an owl and a South African bat-eared fox, he turned his attention
to the clothes. He picked a tiny purple sweater that looked like Missoni
for the children of hippies. The store didn't have a size small enough
for North, but the sales clerk assured him it would shrink in the wash.
He then took a preschool-size denim jumpsuit off the rack and added it
to his haul. ``Who's this for?'' the clerk asked. ``It's for
reference,'' he said.

The day before the Yeezy presentation, in an overcrowded studio space in
NoHo, West looked like any other designer in the final throes of pulling
a show together: conferring with artist and collaborator Vanessa
Beecroft on how the models should stand, making a plethora of
last-minute tweaks, calling out for a Hennessy and Coke. At one end of
the space, a combination of models, downtown cool kids and Kardashians
were being photographed in various outfits. At the other, makeup artists
were trying out explosions of color around models' eyes, and
pattern-makers and seamstresses appeared to be building garments from
scratch.

West moved swiftly and decisively. He was presented with fully outfitted
models and he adjusted their looks, swapping out jackets or sweatshirts
or socks. From time to time he sneaked behind a rack of clothes and
tried on an ensemble himself, seeing how the clothes hung on his body.
``Too shiny,'' he said, when presented with a pair of Tyvek pants. ``I
want them to look matte.''

Some critics called the collection derivative, citing Raf Simons and
Helmut Lang as far-too-obvious references. But West didn't feel any
anxiety of influence himself: ``I would like to be influenced as much as
possible,'' he said. ``I don't care if you can see the influence in
something, as long as I made it better.'' And of course, though Helmut
sent bulletproof vests down the runway in the late '90s, Tupac Shakur
and other rappers were wearing bulletproof vests as fashion years before
that. Influence can have many tributaries, if you're willing to see
them.

Scorn is the lingua franca of the fashion world, and it's not surprising
that West has come in for so much of it. But scorn for West the designer
overlooks what has made West the tastemaker so exceptional in the world
of pop music. The shift in the hip-hop silhouette from baggy to slim,
the mainstreaming of Balenciaga and Givenchy for men, the rise of
athletic wear as high fashion: West's style legacy would have been
secure even if he'd never shown a collection of his own.

During New York Fashion Week, West met one of his idols, Ralph Lauren,
at the latter designer's show. Like West, Lauren began as an outsider
--- a Jewish kid from the Bronx who remade himself as the originator of
America's preppy fantasy. Early in West's music career, before he
discovered high fashion, Polo was his uniform.

In a widely circulated photo of the two men meeting, Lauren has placed
his hand gently on West's cheek. ``Do you know what he said when he did
that? `This is my son,' '' West said. ``And I was thinking, `I knew it!
I knew Ralph was my daddy!' ''

\includegraphics{https://static01.graylady3jvrrxbe.onion/images/2015/04/09/multimedia/tmag-kanyewest/tmag-kanyewest-videoSixteenByNine1050-v2.jpg}

Advertisement

\protect\hyperlink{after-bottom}{Continue reading the main story}

\hypertarget{site-index}{%
\subsection{Site Index}\label{site-index}}

\hypertarget{site-information-navigation}{%
\subsection{Site Information
Navigation}\label{site-information-navigation}}

\begin{itemize}
\tightlist
\item
  \href{https://help.nytimes3xbfgragh.onion/hc/en-us/articles/115014792127-Copyright-notice}{©~2020~The
  New York Times Company}
\end{itemize}

\begin{itemize}
\tightlist
\item
  \href{https://www.nytco.com/}{NYTCo}
\item
  \href{https://help.nytimes3xbfgragh.onion/hc/en-us/articles/115015385887-Contact-Us}{Contact
  Us}
\item
  \href{https://www.nytco.com/careers/}{Work with us}
\item
  \href{https://nytmediakit.com/}{Advertise}
\item
  \href{http://www.tbrandstudio.com/}{T Brand Studio}
\item
  \href{https://www.nytimes3xbfgragh.onion/privacy/cookie-policy\#how-do-i-manage-trackers}{Your
  Ad Choices}
\item
  \href{https://www.nytimes3xbfgragh.onion/privacy}{Privacy}
\item
  \href{https://help.nytimes3xbfgragh.onion/hc/en-us/articles/115014893428-Terms-of-service}{Terms
  of Service}
\item
  \href{https://help.nytimes3xbfgragh.onion/hc/en-us/articles/115014893968-Terms-of-sale}{Terms
  of Sale}
\item
  \href{https://spiderbites.nytimes3xbfgragh.onion}{Site Map}
\item
  \href{https://help.nytimes3xbfgragh.onion/hc/en-us}{Help}
\item
  \href{https://www.nytimes3xbfgragh.onion/subscription?campaignId=37WXW}{Subscriptions}
\end{itemize}
