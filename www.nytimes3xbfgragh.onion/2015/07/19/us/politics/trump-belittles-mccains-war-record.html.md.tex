Sections

SEARCH

\protect\hyperlink{site-content}{Skip to
content}\protect\hyperlink{site-index}{Skip to site index}

\href{https://www.nytimes3xbfgragh.onion/section/politics}{Politics}

\href{https://myaccount.nytimes3xbfgragh.onion/auth/login?response_type=cookie\&client_id=vi}{}

\href{https://www.nytimes3xbfgragh.onion/section/todayspaper}{Today's
Paper}

\href{/section/politics}{Politics}\textbar{}Donald Trump Says John
McCain Is No War Hero, Setting Off Another Storm

\url{https://nyti.ms/1e7oIt4}

\begin{itemize}
\item
\item
\item
\item
\item
\end{itemize}

Advertisement

\protect\hyperlink{after-top}{Continue reading the main story}

Supported by

\protect\hyperlink{after-sponsor}{Continue reading the main story}

\hypertarget{donald-trump-says-john-mccain-is-no-war-hero-setting-off-another-storm}{%
\section{Donald Trump Says John McCain Is No War Hero, Setting Off
Another
Storm}\label{donald-trump-says-john-mccain-is-no-war-hero-setting-off-another-storm}}

\includegraphics{https://static01.graylady3jvrrxbe.onion/images/2015/07/18/us/18WATCHING2-trump/18WATCHING2-trump-articleLarge.jpg?quality=75\&auto=webp\&disable=upscale}

By \href{http://www.nytimes3xbfgragh.onion/by/jonathan-martin}{Jonathan
Martin} and
\href{https://www.nytimes3xbfgragh.onion/by/alan-rappeport}{Alan
Rappeport}

\begin{itemize}
\item
  July 18, 2015
\item
  \begin{itemize}
  \item
  \item
  \item
  \item
  \item
  \end{itemize}
\end{itemize}

AMES, Iowa --- Donald J. Trump has made his name in politics with
provocative statements, but it was not until Saturday, after the
flamboyant businessman turned presidential candidate belittled Senator
John McCain's war record, that many Republicans concluded that silence
or equivocation about Mr. Trump's incendiary rhetoric was inadequate.

Mr. Trump upended a Republican presidential forum here, and the race
more broadly, by saying of the Arizona senator and former prisoner of
war: ``He's not a war hero. He's a war hero because he was captured. I
like people who weren't captured.''

Mr. McCain, a naval aviator, was shot down during the Vietnam War and
held prisoner for more than five years in Hanoi, refusing early release
even after being repeatedly beaten.

Mr. Trump and Mr. McCain have been engaged in a war of words over the
past week, since the Arizona senator said that Mr. Trump was riling up
``crazies'' in the party with the inflammatory remarks about illegal
immigrants from Mexico.

Yet Mr. Trump's comments on Saturday drew condemnation from his rivals
and senior officials in the party at a scale far greater than the
response to his portrayal of Mexican immigrants as rapists. The response
was an indication of the reverence many Republicans have for military
service and sacrifice. But it was also something more: their best
opening yet to marginalize Mr. Trump. ~

After weeks when many of them treaded lightly around Mr. Trump,~who once
again~Saturday~refused to rule out a third-party run, Republican leaders
seized the opportunity to unambiguously speak out against a candidate
they see as effectively hijacking their primaries.

Yet for all the outrage among party elites, some attendees at the
Christian conservative conference where Mr. Trump made his comments were
not nearly as offended, a reminder of the chasm between the Republican
power structure and its grass roots.

With his attack on Mr. McCain, Mr. Trump, whose caustic language about
immigration has lifted him in early polls, created a new, revealing
litmus test for how the Republican presidential hopefuls are handling
the bombastic real estate mogul.

Several of Mr. Trump's Republican opponents immediately denounced his
comments, and one said the remarks disqualified him from the presidency.

``Donald Trump owes every American veteran and in particular John McCain
an apology,'' said Rick Perry, the former Texas governor, upon taking
the stage. Mr. Perry argued that Mr. Trump's comment made him unfit to
be commander in chief.

Senator Lindsey Graham of South Carolina said that anybody serious about
being president would not be disrespectful of prisoners of war, and
predicted that the early nominating states would render an unmistakable
verdict on Mr. Trump's candidacy.

``Here's what I think they're going to say: `Donald Trump, you're
fired,'~'' Mr. Graham said to laughs and applause.

For Mr. Perry and Mr. Graham, both retired Air Force officers who have
struggled to get traction in the race, Mr. Trump's comments represented
an opportunity to highlight their own military service and demonstrate
to primary voters that they would not tolerate any impugning of a
veteran.

As telling was the difference between how Gov. Scott Walker of Wisconsin
and Senator Ted Cruz of Texas reacted to Mr. Trump. Both are running
aggressively in Iowa and pursuing the sort of conservative voters who
are now considering Mr. Trump.

Mr. Walker, who leads in early Iowa polls, had previously resisted
criticizing Mr. Trump. But in a sign of how quickly Mr. Trump's
provocation reshaped the expectations of how candidates should treat
him, Mr. Walker immediately changed course after Mr. Trump questioned
Mr. McCain's military record.

``I unequivocally denounce him,'' Mr. Walker said at a campaign stop in
Sioux City, Iowa.

Mr. Cruz, who poses a threat here on Mr. Walker's right, was more
cautious. He told reporters before his remarks here that Mr. McCain is
``an American hero,'' but added that he would not ``say something bad
about Donald Trump.''

Mr. Cruz's reluctance to confront Mr. Trump was perhaps best explained
by the reaction to Mr. Perry's denunciation: While many in the crowd
applauded, the ovation did not last long and nobody in the audience of
nearly 3,000 stood to show their approval.

``It was not important to me,'' said Rose Kendall, an attendee from
Burlington, Iowa, of Mr. Trump's comment on Mr. McCain. ``He said that
because John McCain talked him down.''

Many Democrats noted that there had been far less opprobrium for Mr.
Trump after he began his candidacy in June by saying of Mexican
immigrants: ``They're bringing drugs, they're bringing crime. They're
rapists and some, I assume, are good people.''

Republicans also treated the businessman more delicately in the 2012
campaign, when Mitt Romney, the party's nominee, sought and publicly
accepted Mr. Trump's endorsement even after the businessman had
questioned whether President Obama was born in the United States.

Speaking to reporters after his turn on stage, Mr. Trump tried to soften
the remarks, saying that any United States veteran who was a prisoner of
war was heroic. He also shifted his comments to assuage veterans, saying
that Mr. McCain had failed to address their needs.

``I'm with the veterans all the time,'' he said. ``I consider them
heroes.''

Asked about his own military draft status, Mr. Trump, 69, said that he
received medical deferments from the Vietnam War because of a bone spur
in his foot. Mr. Trump could not recall which foot was afflicted.

Yet Mr. Trump's awkward and ill-suited remarks about religion and
marriage here may have done more damage to his candidacy, at least with
Christian conservatives.

``I'm a religious person,'' Mr. Trump offered. ``I go to church. Do I do
things that are wrong? I guess so.''

Mr. Trump also struggled to answer if he had ever sought forgiveness
from God, before reluctantly acknowledging that he had not. ``If I do
something wrong, I try to do something right,'' he said. ``I don't bring
God into that picture.''

And Mr. Trump raised eyebrows with language rarely heard before an
evangelical audience --- saying ``damn'' and ``hell'' when discussing
education and the economy --- while also describing the taking of
communion in glib terms. ``When we go in church and I drink the little
wine, which is about the only wine I drink, and I eat the little cracker
--- I guess that's a form of asking forgiveness,'' Mr. Trump said.

If all that was not enough to roil the button-downed crowd, he also
described his three marriages in starkly frank terms, conceding that he
had difficulty finding a work-life balance.

``It was a work thing, it wasn't a bad thing,'' Mr. Trump said. ``It was
very hard for anybody to compete against the work.''

Despite his marital problems over the years, Mr. Trump said that he was
always available to his children and that he did his best to have dinner
with them on most nights even when his work was grueling. He worked
hard, he said, to instill good values and steer them away from drugs,
alcohol and cigarettes.

``I was actually a great father,'' Mr. Trump said. ``I was a better
father than I was a husband.''

It was these comments, not his attack on Mr. McCain, that prompted the
most muttering and unease in the audience.

``Well, I was turned off at the very start because I didn't like his
language,'' Becky Kruse, of Lovilia, Iowa, said of Mr. Trump, not
mentioning his comments about Mr. McCain. Ms. Kruse said she likes Mr.
Trump's hard line on immigration and came to the event considering him.
``I was not too impressed,'' she said, noting Mr. Trump's comment about
not seeking God's forgiveness. ``He sounds like he isn't really a
born-again Christian.''

Advertisement

\protect\hyperlink{after-bottom}{Continue reading the main story}

\hypertarget{site-index}{%
\subsection{Site Index}\label{site-index}}

\hypertarget{site-information-navigation}{%
\subsection{Site Information
Navigation}\label{site-information-navigation}}

\begin{itemize}
\tightlist
\item
  \href{https://help.nytimes3xbfgragh.onion/hc/en-us/articles/115014792127-Copyright-notice}{©~2020~The
  New York Times Company}
\end{itemize}

\begin{itemize}
\tightlist
\item
  \href{https://www.nytco.com/}{NYTCo}
\item
  \href{https://help.nytimes3xbfgragh.onion/hc/en-us/articles/115015385887-Contact-Us}{Contact
  Us}
\item
  \href{https://www.nytco.com/careers/}{Work with us}
\item
  \href{https://nytmediakit.com/}{Advertise}
\item
  \href{http://www.tbrandstudio.com/}{T Brand Studio}
\item
  \href{https://www.nytimes3xbfgragh.onion/privacy/cookie-policy\#how-do-i-manage-trackers}{Your
  Ad Choices}
\item
  \href{https://www.nytimes3xbfgragh.onion/privacy}{Privacy}
\item
  \href{https://help.nytimes3xbfgragh.onion/hc/en-us/articles/115014893428-Terms-of-service}{Terms
  of Service}
\item
  \href{https://help.nytimes3xbfgragh.onion/hc/en-us/articles/115014893968-Terms-of-sale}{Terms
  of Sale}
\item
  \href{https://spiderbites.nytimes3xbfgragh.onion}{Site Map}
\item
  \href{https://help.nytimes3xbfgragh.onion/hc/en-us}{Help}
\item
  \href{https://www.nytimes3xbfgragh.onion/subscription?campaignId=37WXW}{Subscriptions}
\end{itemize}
