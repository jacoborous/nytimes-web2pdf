Sections

SEARCH

\protect\hyperlink{site-content}{Skip to
content}\protect\hyperlink{site-index}{Skip to site index}

\href{https://www.nytimes3xbfgragh.onion/section/food}{Food}

\href{https://myaccount.nytimes3xbfgragh.onion/auth/login?response_type=cookie\&client_id=vi}{}

\href{https://www.nytimes3xbfgragh.onion/section/todayspaper}{Today's
Paper}

\href{/section/food}{Food}\textbar{}Taking Champagne Back to Its Roots

\begin{itemize}
\item
\item
\item
\item
\item
\end{itemize}

Advertisement

\protect\hyperlink{after-top}{Continue reading the main story}

Supported by

\protect\hyperlink{after-sponsor}{Continue reading the main story}

\href{/column/the-pour}{The Pour}

\hypertarget{taking-champagne-back-to-its-roots}{%
\section{Taking Champagne Back to Its
Roots}\label{taking-champagne-back-to-its-roots}}

\includegraphics{https://static01.graylady3jvrrxbe.onion/images/2008/11/05/dining/05pour_span.600.jpg?quality=75\&auto=webp\&disable=upscale}

By \href{https://www.nytimes3xbfgragh.onion/by/eric-asimov}{Eric Asimov}

\begin{itemize}
\item
  Nov. 4, 2008
\item
  \begin{itemize}
  \item
  \item
  \item
  \item
  \item
  \end{itemize}
\end{itemize}

WITH rough, work-thickened hands, unruly hair and a steady gaze, Anselme
Selosse looks the image of the French vigneron, a man more comfortable
tending vines and working in his cellar than he is in a New York
restaurant talking to sommeliers and wine writers.

But there he was last week, at Eleven Madison Park, leading a tasting of
his wines, speaking smoothly in French, gesturing with long arms that
seemed as if they would be a lot more comfortable sprung from the
confines of his rumpled blazer.

Mr. Selosse, 54, is not the usual emissary from Champagne, a smooth guy
in a suit, talking about product positioning, luxury brands and
lifestyles. To hear them tell it, Champagne pops into this world like a
genie from a lamp, ready to make magic.

But to Mr. Selosse, the magic occurs long before there is a wine. It
takes place deep underneath Champagne's chalky soil, where the roots of
the vines take hold of what Mr. Selosse calls the essence of the earth.

Jacques Selosse Champagne, named for Anselme's father, is not something
found at the corner wine shop. In fact, for five years, from 2002 to
2007, it wasn't sent to the United States at all, not after Mr. Selosse
severed ties with a previous American importer. But last year another
importer, the Rare Wine Company, made a deal with Mr. Selosse and began
to bring it in again --- though in minute quantities at high prices.

Suffice it to say that most of us probably can't afford Selosse
Champagne and may never drink it. Well, then, why should anybody care
about it, especially now when \$20 for a bottle of wine seems like a lot
of money, much less the \$250 you might pay for Selosse's
top-of-the-line Substance cuvée?

Because, as superb, striking and idiosyncratic as the Selosse Champagnes
can be, what Mr. Selosse represents is equally important, if not more
so. Yes, he and his wife, Corinne, had taken this rare trip to New York
to reintroduce their Champagnes to the wine trade, but what he had to
say about Champagne was possibly more meaningful than the wines
themselves.

The key word is wines. In almost every possible way, the corporate line
from Champagne is the antithesis of what consumers are taught about
every other important wine region in the world. Great wines, almost
everyone can agree, are distinctive. They ideally reflect their terroirs
and the conditions of their vintages. In short, as the rest of the wine
world preaches with varying degrees of honesty, great wines are made in
the vineyard.

But the dominant Champagne houses have divorced what's in the bottle
from what comes from the earth. Their story of Champagne, told through
decades of marketing, associates bubbles with elegance, luxury and
festivity, achieved through master blenders in the cellar. Champagne
does not celebrate the land and the vigneron, but the house and the
event. Too often, Champagne is a commodity, not a wine.

Mr. Selosse, by his example and his Champagnes, is intent on restoring
the ideas of vineyard, terroir and wine to the perception of Champagne.
He is not alone by any means. He is one of a growing number of Champagne
vignerons --- grape growers who also make the wine and bottle it
themselves --- who are intent on changing the nature of Champagne. Some
of the big houses make great Champagne, and not all of the small growers
are successful. But their influence has increased, and the big houses
are paying attention.

Any restaurant in New York with a decent wine list will have at least
one of these small Champagne houses among the big names.
Grower-producers like Larmandier-Bernier, Egly-Ouriet, Pierre Gimonnet,
Pierre Moncuit and Pierre Peters are making Champagnes that are
distinctive if not profound, reflecting the terroir in which the grapes
are born, and forcing people to rethink their ideas about Champagne. In
this company, no Champagne producer has been more influential or more
original than Mr. Selosse.

Not that Mr. Selosse heads any organized group. He leads more by
inspiration. He won't criticize his colleagues big or small, though he
was more impolitic as a younger man after he took over from his father
in 1980. His Champagnes are not adored unanimously, although you can
count me among the adorers. He has been criticized for making Champagnes
that are too oaky --- perhaps a fault once but no longer. That said, his
Champagnes --- his wines --- are distinctive, and distinctive wines will
always be at least somewhat divisive.

Mr. Selosse was trained in Burgundy, and his ideas about grape growing
are indeed Burgundian. He has likened himself to the Cistercian monks
who planted many of Burgundy's great vineyards in an effort to make the
most of their terroir. ``They were motivated by religion,'' Mr. Selosse
told me once. ``My religion is the vineyard.''

Mr. Selosse does not adhere to biodynamic viticulture, but he thinks of
the vineyard in biodynamic terms, seeing it as a harmonious eco-system
of plants, animals and micro-organisms. ``The greatest danger is man,
who can upset the balance,'' he said. His job, he said, is to observe
and guide with a gentle hand, but to stay out of the way.

``Essentially, we're of the countryside, and our goal is to give
expression to the countryside,'' he said. It's not an unusual thing to
hear from a vigneron, but revolutionary in Champagne, which strives for
a decidedly urban image.

Mr. Selosse is determined to emphasize what is singular in his wines,
rather than the Champagne norm of seeking house consistency year after
year. Yet he is not so Burgundian that he believes only in vintage
wines. Of the eight cuvées he poured at the New York tasting and at a
dinner later that evening, only one was a vintage wine, a 1999 blanc de
blancs extra brut. The others, including a floral, chalky rosé, a rich
yet energetic blanc de noirs and a beautifully subtle and textured extra
brut blanc de blancs called Version Originale, are all made from
multiple vintages.

Perhaps the most unusual of his Champagnes is Substance, made from a
single chardonnay vineyard in Avize. It uses a solera system, similar to
what is used to make sherry, in which successive vintages, back to 1987,
are blended. The result is an almost ethereal Champagne, with aromas of
flowers and seashells.

Rather than obscuring the terroir, Mr. Selosse asserts, the blending of
his solera Champagne emphasizes the qualities of the vineyard by
eliminating variables like weather.

``It takes all the different years --- the good, the bad, the wet, the
dry, the sunny --- and neutralizes the elements to bring out the
terroir,'' he said.

I asked him whether he would ever suggest this method to his friends in
Burgundy, where it would be looked on as heretical.

``No,'' he said. ``In Burgundy they already understand the terroir ---
it rises above the vintage.'' He looked thoughtful for a moment. ``Maybe
in Bordeaux,'' he said.

Advertisement

\protect\hyperlink{after-bottom}{Continue reading the main story}

\hypertarget{site-index}{%
\subsection{Site Index}\label{site-index}}

\hypertarget{site-information-navigation}{%
\subsection{Site Information
Navigation}\label{site-information-navigation}}

\begin{itemize}
\tightlist
\item
  \href{https://help.nytimes3xbfgragh.onion/hc/en-us/articles/115014792127-Copyright-notice}{©~2020~The
  New York Times Company}
\end{itemize}

\begin{itemize}
\tightlist
\item
  \href{https://www.nytco.com/}{NYTCo}
\item
  \href{https://help.nytimes3xbfgragh.onion/hc/en-us/articles/115015385887-Contact-Us}{Contact
  Us}
\item
  \href{https://www.nytco.com/careers/}{Work with us}
\item
  \href{https://nytmediakit.com/}{Advertise}
\item
  \href{http://www.tbrandstudio.com/}{T Brand Studio}
\item
  \href{https://www.nytimes3xbfgragh.onion/privacy/cookie-policy\#how-do-i-manage-trackers}{Your
  Ad Choices}
\item
  \href{https://www.nytimes3xbfgragh.onion/privacy}{Privacy}
\item
  \href{https://help.nytimes3xbfgragh.onion/hc/en-us/articles/115014893428-Terms-of-service}{Terms
  of Service}
\item
  \href{https://help.nytimes3xbfgragh.onion/hc/en-us/articles/115014893968-Terms-of-sale}{Terms
  of Sale}
\item
  \href{https://spiderbites.nytimes3xbfgragh.onion}{Site Map}
\item
  \href{https://help.nytimes3xbfgragh.onion/hc/en-us}{Help}
\item
  \href{https://www.nytimes3xbfgragh.onion/subscription?campaignId=37WXW}{Subscriptions}
\end{itemize}
