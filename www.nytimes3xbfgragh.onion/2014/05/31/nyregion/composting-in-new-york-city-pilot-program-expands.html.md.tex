Sections

SEARCH

\protect\hyperlink{site-content}{Skip to
content}\protect\hyperlink{site-index}{Skip to site index}

\href{https://www.nytimes3xbfgragh.onion/section/nyregion}{New York}

\href{https://myaccount.nytimes3xbfgragh.onion/auth/login?response_type=cookie\&client_id=vi}{}

\href{https://www.nytimes3xbfgragh.onion/section/todayspaper}{Today's
Paper}

\href{/section/nyregion}{New York}\textbar{}New York City Trying to Help
Households Turn Waste Into Compost

\url{https://nyti.ms/1pE7e8y}

\begin{itemize}
\item
\item
\item
\item
\item
\item
\end{itemize}

Advertisement

\protect\hyperlink{after-top}{Continue reading the main story}

Supported by

\protect\hyperlink{after-sponsor}{Continue reading the main story}

\hypertarget{new-york-city-trying-to-help-households-turn-waste-into-compost}{%
\section{New York City Trying to Help Households Turn Waste Into
Compost}\label{new-york-city-trying-to-help-households-turn-waste-into-compost}}

\includegraphics{https://static01.graylady3jvrrxbe.onion/images/2014/05/31/nyregion/31COMPOST/31COMPOST-articleLarge.jpg?quality=75\&auto=webp\&disable=upscale}

By \href{http://www.nytimes3xbfgragh.onion/by/vivian-yee}{Vivian Yee}

\begin{itemize}
\item
  May 30, 2014
\item
  \begin{itemize}
  \item
  \item
  \item
  \item
  \item
  \item
  \end{itemize}
\end{itemize}

For the first week or so, the brown bins hunkered on the streets like
alien capsules that had fallen to Earth: gingerly stepped around,
uneasily eyed, tenaciously ignored. One was pressed into service as a
wastebasket for empty liquor bottles. Another was unceremoniously tossed
in a street-corner trash can.

But many sat untouched outside the front steps of houses all over Bay
Ridge, Brooklyn, as though their new owners hoped that if they refused
to acknowledge the bins, they might simply disappear.

So began the latest phase of a plan to convert the people of New York
City into composters, collecting food scraps like vegetable peels,
chicken bones and even greasy pizza boxes and saving the pungent blend
for days at a time to be someday converted into renewable energy. A
voluntary city pilot program has already taken root in neighborhoods
like Windsor Terrace, Brooklyn, which has eagerly embraced it, and
Westerleigh, Staten Island, which was not quite as enthusiastic.

Now composting is coming to 70,000 more households in Brooklyn and
Queens, including most of Bay Ridge, where, on one recent evening before
the garbage was due for pickup, about one out of every three households
who had rolled their waste to the curb had included one of the brown
bins.

``It stayed outside my house for two weeks,'' confessed Nancy Gasparino,
57, as she walked her dog down a tree-lined block of tidy homes last
Monday, three weeks after sanitation workers first collected the bins.
``I know it's a good thing, I really do. I have four kids --- I want the
environment to be better for them. But I hesitated to start it.''

Ms. Gasparino worried it would stink. She worried ants in her kitchen
would swarm it. She worried that the cats and raccoons would rummage
through it. She did not look forward, she said, to serving as ``the
garbage police'' for her children at dinnertime.

But she noticed a bar code printed on the back of her bin and could not
quell the suspicion that the city maybe, who knows, just might be
keeping tabs on her and her composting compliance. ``Why do they put
that on there?'' she said doubtfully.

And so, Ms. Gasparino is now a full-fledged composter.

Despite instructions that came with each bin, many residents have called
the local councilman's office in alarm, demanding to know: Do I have to
do this?

The answer is no, not until the city makes the program mandatory, which
may not happen for several years. But the talk of the neighborhood has
gradually shifted to focus on where to buy biodegradable bags to line
kitchen compost buckets and how best to neutralize unpleasant odors:
Storing compost in the freezer works; Febreze vanilla air freshener, for
unknown reasons, makes it much worse.

For all the fears of a stench, however, few complaints were heard on
that score in an informal survey of composters in Bay Ridge and Windsor
Terrace, though one family in East New York who signed up for the
program said they remain shocked at how foul it could be.

``I didn't know it stinks like that,'' Donell Brant, 31, said. ``You've
got to hold your breath.'' His wife, Charlene Wynns, recommended
cinnamon-scented Febreze.

On Colonial Road, Joan Rochford, 60, opened her garbage bin up to show
off her trash to a visitor: She had one little plastic bag's worth, for
nearly a week. Everything else, she explained delightedly, had been
recycled or composted.

The city hopes that the roughly 3.2 million tons of waste that New
Yorkers send to landfills every year, at a cost of \$300 million, can be
diverted in large part to compost. Scraps from the pilot program
currently go to local and regional compost facilities. But a new
facility from the power company National Grid at Newtown Creek
Wastewater Treatment Plant, which is scheduled to break ground this
summer, will eventually turn methane derived from residential compost
into enough natural gas to heat as many as 5,200 homes in Brooklyn.

Still, between 10 percent and 20 percent of the compost currently
treated at the plant still goes to landfills in the form of a dried-out
byproduct known as ``sludge cakes,'' which are sent to old coal mines in
Pennsylvania.

\includegraphics{https://static01.graylady3jvrrxbe.onion/images/2014/05/30/nyregion/30COMPOSTweb2/30COMPOSTweb2-articleLarge.jpg?quality=75\&auto=webp\&disable=upscale}

Composting has become a fact of life in dozens of cities across the
country, including Austin, Tex., which sold \$267,000 of yard
compost-turned-soil amendment last year; San Francisco; and Seattle ---
leaving New York, whose sanitation officials long believed the city was
too densely and vertically populated for composting to flourish, to play
catch-up.

Composting advocates cite some startling numbers: According to the
Environmental Protection Agency and environmental groups, Americans
threw away at least 36 million tons of food in 2012. The waste is worth
about \$165 billion annually. Even 15 percent of what is discarded every
year could feed 25 million people, according to the Natural Resources
Defense Council.

``We love it!'' said Ms. Rochford, who is planning to include more room
for a compost bin when she renovates the kitchen. Her husband, who takes
pride in meticulously sorting the recycling, is equally enthusiastic.

She would find many friends in Windsor Terrace, where composting has
been all the rage since the city's pilot started there last fall.

Ronni Horowitz's husband never had much interest in her home composting.
``But it's, like, a quest now to see how little he can put in the
trash,'' said Ms. Horowitz, 53, who used to compost garden waste,
vegetables and fruits on her own. She laughed and said, ``Now I have to
train him to take the bin out.''

Ms. Horowitz keeps a compost bin in the classroom where she teaches
second grade, drives a Prius --- but walks to work, she was quick to
note --- and has even considered composting her dog's droppings. ``It
was killing me to wrap it in a plastic bag,'' she lamented.

Like others interviewed in Windsor Terrace, she praised the city bins,
which so far appear animalproof, and the wide range of waste they
accept, which includes yard waste, fruit and vegetable scraps, animal
bones, eggshells, dirty napkins and practically anything except plastic
foam, recyclables and bathroom waste.

Or, as Cooper Formant, 30, who was playing with his young children in a
yard, put it: ``Everything but diapers.''

The household-by-household composting campaign can be slow going,
especially when the households themselves are divided over composting's
merits.

When the Sanitation Department first delivered a brown bin to Evelyn
Orton and Tom Bura's home in Windsor Terrace, said Mr. Bura, 63, ``I
thought it was like somebody leaving an abandoned cat at your door ---
like, what am I supposed to do with this?''

Months later, he had become resigned to the process, though he was still
grumbling: ``It's sort of an added burden that we have to take on along
with the many things we take on to make things better for modern
society.''

Ms. Orton, 65, had quickly embraced composting. ``I keep yelling at
him,'' she said. ``He keeps throwing things in there he's not supposed
to.''

Holdouts remain, as they do for paper, metal and glass recycling, which
the city introduced in 1986; more than 51,000 summonses were issued for
the top five recycling violations last year.

``I didn't even understand it. It's just in my backyard, sitting there.
We have more garbage cans than we know what to do with,'' said Madeline,
a longtime Bay Ridge resident who declined to give her last name because
``I don't want anybody coming after me.''

``I know the earth and the trees and the air, we do need it for that,
but I think they should have a better system,'' she said. ``People are
busy and doing things.''

She paused, then offered: ``If you needed a little garbage can, it was
nice.''

Advertisement

\protect\hyperlink{after-bottom}{Continue reading the main story}

\hypertarget{site-index}{%
\subsection{Site Index}\label{site-index}}

\hypertarget{site-information-navigation}{%
\subsection{Site Information
Navigation}\label{site-information-navigation}}

\begin{itemize}
\tightlist
\item
  \href{https://help.nytimes3xbfgragh.onion/hc/en-us/articles/115014792127-Copyright-notice}{©~2020~The
  New York Times Company}
\end{itemize}

\begin{itemize}
\tightlist
\item
  \href{https://www.nytco.com/}{NYTCo}
\item
  \href{https://help.nytimes3xbfgragh.onion/hc/en-us/articles/115015385887-Contact-Us}{Contact
  Us}
\item
  \href{https://www.nytco.com/careers/}{Work with us}
\item
  \href{https://nytmediakit.com/}{Advertise}
\item
  \href{http://www.tbrandstudio.com/}{T Brand Studio}
\item
  \href{https://www.nytimes3xbfgragh.onion/privacy/cookie-policy\#how-do-i-manage-trackers}{Your
  Ad Choices}
\item
  \href{https://www.nytimes3xbfgragh.onion/privacy}{Privacy}
\item
  \href{https://help.nytimes3xbfgragh.onion/hc/en-us/articles/115014893428-Terms-of-service}{Terms
  of Service}
\item
  \href{https://help.nytimes3xbfgragh.onion/hc/en-us/articles/115014893968-Terms-of-sale}{Terms
  of Sale}
\item
  \href{https://spiderbites.nytimes3xbfgragh.onion}{Site Map}
\item
  \href{https://help.nytimes3xbfgragh.onion/hc/en-us}{Help}
\item
  \href{https://www.nytimes3xbfgragh.onion/subscription?campaignId=37WXW}{Subscriptions}
\end{itemize}
