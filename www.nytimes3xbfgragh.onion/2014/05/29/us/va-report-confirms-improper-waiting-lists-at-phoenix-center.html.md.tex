Sections

SEARCH

\protect\hyperlink{site-content}{Skip to
content}\protect\hyperlink{site-index}{Skip to site index}

\href{https://www.nytimes3xbfgragh.onion/section/us}{U.S.}

\href{https://myaccount.nytimes3xbfgragh.onion/auth/login?response_type=cookie\&client_id=vi}{}

\href{https://www.nytimes3xbfgragh.onion/section/todayspaper}{Today's
Paper}

\href{/section/us}{U.S.}\textbar{}Severe Report Finds V.A. Hid Waiting
Lists at Hospitals

\url{https://nyti.ms/TUmZy8}

\begin{itemize}
\item
\item
\item
\item
\item
\item
\end{itemize}

Advertisement

\protect\hyperlink{after-top}{Continue reading the main story}

Supported by

\protect\hyperlink{after-sponsor}{Continue reading the main story}

\hypertarget{severe-report-finds-va-hid-waiting-lists-at-hospitals}{%
\section{Severe Report Finds V.A. Hid Waiting Lists at
Hospitals}\label{severe-report-finds-va-hid-waiting-lists-at-hospitals}}

\includegraphics{https://static01.graylady3jvrrxbe.onion/images/2014/05/29/us/PHOENIX/PHOENIX-articleLarge.jpg?quality=75\&auto=webp\&disable=upscale}

By
\href{https://www.nytimes3xbfgragh.onion/by/richard-a-oppel-jr}{Richard
A. Oppel Jr.} and
\href{http://www.nytimes3xbfgragh.onion/by/michael-d-shear}{Michael D.
Shear}

\begin{itemize}
\item
  May 28, 2014
\item
  \begin{itemize}
  \item
  \item
  \item
  \item
  \item
  \item
  \end{itemize}
\end{itemize}

In the first confirmation that Department of Veterans Affairs
administrators manipulated medical waiting lists at one and possibly
more hospitals, the department's inspector general reported on Wednesday
that 1,700 patients at the veterans medical center in Phoenix were not
placed on the official waiting list for doctors' appointments and may
never have received care.

The scathing report by Richard J. Griffin, the acting inspector general,
validates allegations raised by whistle-blowers and others that Veterans
Affairs officials in Phoenix employed artifices to cloak long waiting
times for veterans seeking medical care. Mr. Griffin said the average
waiting time in Phoenix for initial primary care appointments, 115 days,
was nearly five times as long as what the hospital's administrators had
reported.

He suggested that the falsified data may have led to more favorable
performance reviews for hospital personnel, and he indicated that some
instances of potentially manipulated data had been turned over to the
Justice Department.

Mr. Griffin said that similar kinds of manipulation to hide long and
possibly growing waiting times were ``systemic throughout'' the
sprawling Veterans Affairs health care system, with its 150 medical
centers serving eight million veterans each year. The inspector
general's office is reviewing practices at 42 Veterans Affairs medical
facilities.

Mr. Griffin's report brought immediate political consequences. For the
first time since the controversy erupted last month, several Senate
Democrats, including Mark Udall of Colorado and John Walsh of Montana,
demanded that the secretary of veterans affairs, Eric Shinseki, step
down, joining Republican lawmakers who have been making that demand for
weeks.

Senator John McCain, Republican of Arizona, a former naval aviator who
was a prisoner of war during the Vietnam War and is now an influential
voice on veterans issues, also called on Wednesday for Mr. Shinseki to
resign. Along with several other leading Republican lawmakers who had
been withholding judgment, Mr. McCain asked the F.B.I. to investigate
the Phoenix hospital. Mr. Griffin previously said that he was working
with the Justice Department to examine whether criminal violations had
occurred there.

Mr. Shinseki, in a statement, called the findings ``reprehensible to
me'' and ordered the department to ``immediately triage each of the
1,700 veterans'' and give them timely care. The department suspended two
senior officials at the Phoenix medical center shortly after the
allegations of falsified waiting lists became public this month.

Jay Carney, the White House press secretary, said President Obama found
the report ``extremely troubling,'' but he did not indicate whether Mr.
Shinseki had lost the confidence of the White House.

Mr. Griffin's interim report --- the final version is expected by August
--- did not address the most explosive allegations made about the
Phoenix facility: that as many as 40 veterans who were never put on the
official list for doctors' appointments might have died while awaiting
care. He said determinations could be made only after examining autopsy
reports and other documents that were still being reviewed. He had
previously said that after reviewing 17 of those cases, he had found no
indication that any of those deaths were tied to delays.

But the rest of his report was sweeping in its indictment of the Phoenix
hospital, and contained sharp criticism of much of the rest of the
veterans health care bureaucracy.

``While our work is not complete, we have substantiated that significant
delays in access to care negatively impacted the quality of care at this
medical facility,'' Mr. Griffin said.

Irregularities in how the 1,700 veterans were handled, he added, mean
that ``these veterans may never obtain a requested or required clinical
appointment.''

\href{https://www.nytimes3xbfgragh.onion/interactive/2014/05/27/us/27veteranscallout.html}{}

\includegraphics{https://static01.graylady3jvrrxbe.onion/images/2014/05/27/us/27veteranscallout-1401225429685/27veteranscallout-1401225429685-videoLarge.jpg}

\hypertarget{share-your-experience-with-veterans-affairs-health-care}{%
\subsection{Share Your Experience With Veterans Affairs Health
Care}\label{share-your-experience-with-veterans-affairs-health-care}}

New York Times journalists would like to hear from veterans about their
experiences with health care at Department of Veterans Affairs hospitals
and clinics

Investigators from the inspector general's office reviewed a sample of
226 patients and found that they waited an average of 115 days for their
first primary care appointment at the Phoenix medical center, but their
average waiting time was reported to the national Veterans Affairs
office as being only 24 days.

The interim report did not dwell on the motivations for falsely
reporting waiting times, nor did it single out any employees or hospital
administrators by name.

But it stated that a ``direct consequence'' of the inappropriate waiting
lists was that the medical center's leadership ``significantly
understated the time new patients waited for their primary care
appointment'' in its performance appraisal accomplishments for the 2013
fiscal year, which was a factor considered for bonuses and salary
increases.

Mr. Griffin also suggested that his team may have already found some
indication of criminal wrongdoing. ``When sufficient credible evidence
is identified supporting a potential violation of criminal and/or civil
law, we have contacted and are coordinating our efforts with the
Department of Justice,'' he wrote.

He said in his report that his investigators had identified several
types of improper scheduling practices in Phoenix. They found multiple
waiting lists aside from the official electronic waiting list, and said
that ``these additional lists may be the basis for allegations of
creating `secret' wait lists'' that have been cited by whistle-blowers.

The allegations identified by investigators were not limited to waiting
lists. Mr. Griffin said his office had received ``numerous allegations
daily of mismanagement, inappropriate hiring decisions, sexual
harassment, and bullying behavior by mid- and senior-level managers at
this facility.''

Mr. Shinseki, a soft-spoken former four-star Army general and chief of
staff, has had support on Capitol Hill from some lawmakers partly
because of his long military career.

But the release of the inspector general's report increased the pressure
on him to step down, especially after some Senate Democrats broke with
others in the party late in the day to demand his removal.

Mr. Walsh, the Montana senator, said that the report ``confirms the
worst of the allegations against the V.A.,'' and that ``it's time to put
the partisanship aside and focus on what's right for our veterans.''

Representative Jeff Miller, the Florida Republican who is the chairman
of the House Veterans Affairs Committee, said the report ``confirmed
beyond a shadow of a doubt what was becoming more obvious by the day:
wait time schemes and data manipulation are systemic throughout V.A. and
are putting veterans at risk in Phoenix and across the country.''

Mr. Miller had previously held off on calling for Mr. Shinseki's
resignation, but he did so on Wednesday, saying that the former general
``appears completely oblivious to the severity of the health care
challenges facing the department.''

Mr. McCain said on CNN that he had intended to wait to comment on Mr.
Shinseki's future until further hearings were held on the issue. But
after hearing about the report, he decided to speak out.

``I think it's reached that point,'' he said. ``This keeps piling up.''

Advertisement

\protect\hyperlink{after-bottom}{Continue reading the main story}

\hypertarget{site-index}{%
\subsection{Site Index}\label{site-index}}

\hypertarget{site-information-navigation}{%
\subsection{Site Information
Navigation}\label{site-information-navigation}}

\begin{itemize}
\tightlist
\item
  \href{https://help.nytimes3xbfgragh.onion/hc/en-us/articles/115014792127-Copyright-notice}{©~2020~The
  New York Times Company}
\end{itemize}

\begin{itemize}
\tightlist
\item
  \href{https://www.nytco.com/}{NYTCo}
\item
  \href{https://help.nytimes3xbfgragh.onion/hc/en-us/articles/115015385887-Contact-Us}{Contact
  Us}
\item
  \href{https://www.nytco.com/careers/}{Work with us}
\item
  \href{https://nytmediakit.com/}{Advertise}
\item
  \href{http://www.tbrandstudio.com/}{T Brand Studio}
\item
  \href{https://www.nytimes3xbfgragh.onion/privacy/cookie-policy\#how-do-i-manage-trackers}{Your
  Ad Choices}
\item
  \href{https://www.nytimes3xbfgragh.onion/privacy}{Privacy}
\item
  \href{https://help.nytimes3xbfgragh.onion/hc/en-us/articles/115014893428-Terms-of-service}{Terms
  of Service}
\item
  \href{https://help.nytimes3xbfgragh.onion/hc/en-us/articles/115014893968-Terms-of-sale}{Terms
  of Sale}
\item
  \href{https://spiderbites.nytimes3xbfgragh.onion}{Site Map}
\item
  \href{https://help.nytimes3xbfgragh.onion/hc/en-us}{Help}
\item
  \href{https://www.nytimes3xbfgragh.onion/subscription?campaignId=37WXW}{Subscriptions}
\end{itemize}
