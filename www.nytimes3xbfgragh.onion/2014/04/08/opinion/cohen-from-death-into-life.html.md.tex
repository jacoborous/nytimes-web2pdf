Sections

SEARCH

\protect\hyperlink{site-content}{Skip to
content}\protect\hyperlink{site-index}{Skip to site index}

\href{https://myaccount.nytimes3xbfgragh.onion/auth/login?response_type=cookie\&client_id=vi}{}

\href{https://www.nytimes3xbfgragh.onion/section/todayspaper}{Today's
Paper}

\href{/section/opinion}{Opinion}\textbar{}From Death Into Life

\url{https://nyti.ms/1hT0HVt}

\begin{itemize}
\item
\item
\item
\item
\item
\item
\end{itemize}

Advertisement

\protect\hyperlink{after-top}{Continue reading the main story}

Supported by

\protect\hyperlink{after-sponsor}{Continue reading the main story}

\href{/section/opinion}{Opinion}

\hypertarget{from-death-into-life}{%
\section{From Death Into Life}\label{from-death-into-life}}

\href{https://www.nytimes3xbfgragh.onion/by/roger-cohen}{\includegraphics{https://static01.graylady3jvrrxbe.onion/images/2014/11/01/opinion/cohen-circular/cohen-circular-thumbLarge-v6.png}}

By \href{https://www.nytimes3xbfgragh.onion/by/roger-cohen}{Roger Cohen}

\begin{itemize}
\item
  April 7, 2014
\item
  \begin{itemize}
  \item
  \item
  \item
  \item
  \item
  \item
  \end{itemize}
\end{itemize}

On July 21, 1944, my uncle Bert Cohen of the 6th South African Armored
Division, 19th Field Ambulance, reached Italy's Monte Cassino, abandoned
by German forces a few weeks earlier after repeated Allied assaults.

He made an entry in his war diary: ``Poor Cassino, horror, wreck and
desolation unbelievable, roads smashed and pitted, mines, booby traps
and graves everywhere. Huge shell holes, craters filled with stagnant
slime, smashed buildings, hardly outlines remaining, a silent sight of
ghosts and shadows. Pictures should be taken of this monument to
mankind's worst moments and circulated through every schoolroom in the
world.''

This was the Memento Mori proposed by my uncle, then aged 25 and
recently arrived in a bloody continent called Europe. Those pictures
were not circulated; and the miracle and fragility of European peace is
too often forgotten.

In April of that year, Capt. Cohen, born in the last year of World War I
and now thrust into World War II, had crossed from Egypt to Italy,
sailing beneath searchlights that ``deftly flick their fingers across
the face of the sky.'' He wondered at the ``circumstances that should
bring me --- plain-routine, rut-living Bertie Cohen of Johannesburg ---
to be driving in a cumbersome truck through a rural part of southern
Italy.''

War is a gale. It scoops up routine lives and (when it does not end
them) scatters them here and there, never again to be reconstituted in
the same form. Whether my uncle would in any event have emigrated from
South Africa is impossible to know.

He left first for Chicago where he gained a master's degree in Dental
Science from Northwestern University in 1948, and ultimately for London
where, in 1960, he became the first Nuffield Research Professor of
Dental Science at the Royal College of Surgeons of England. An oral
pathologist of great distinction, he was above all a scientist of
wide-ranging interests, a man passionate about literature and art, a
stranger to the narrow specialization in vogue today. In 1982, he was
appointed C.B.E. (Commander of the Order of the British Empire), an
honor he accepted but never talked about. He wore his many
accomplishments lightly.

\includegraphics{https://static01.graylady3jvrrxbe.onion/images/2014/04/08/opinion/08cohen/08cohen-articleLarge.jpg?quality=75\&auto=webp\&disable=upscale}

I relate all this because my uncle died last month at the age of 95 and
I have since found my life consumed by his. Each of us is allotted one
life. Bert needed two. As the fragments from his diary suggest, he
wanted to be a writer.

Early short stories showed promise. He talked his way, as a teenager,
into becoming South African correspondent of the boxing magazine, ``The
Ring,'' and filed many a fine dispatch. His diary places him, more than
once, on the brink of giving up dentistry for a life of writing. It was
not to be. Reading of the road not taken, I understood better Bert's
passionate interest in my work. Childless, he was living through me what
he had wanted to do.

Now he lives in me. The living are the custodians of the souls of the
dead, those stealthy migrants. Love bequeaths this responsibility.

I might never have known him. On April 24, 1945, he was ordered into a
bend in the Penaro River where a Nazi column was trapped. The fighting
was brutal. An artillery battery pulverized the enclave. Wounded horses,
nostrils flared in gasping horror, bayed --- a terrible sound. In the
carnage ammunition exploded and tires burst. One dead German in
particular caught Bert's eye: a blond square-jawed young man, hair
flecked with blood and smoke, legs twisted grotesquely, abdomen ripped
open, coils of gut spilling through a ragged gash into the dust,
sightless blue eyes gazing at infinity.

Beside the corpse lay letters from the soldier's mother in Hamburg. She
talked about Der Angriff, the Allied bombardment of the city. Uncertain
what to do, Bert returned the letters to the dead man's pocket. That
single German corpse haunted my uncle. Bert dwelt on him as if this
death was his responsibility, or as if he, a Jew from South Africa,
might somehow have brought this handsome young man, Hitler's model
Aryan, back to the life denied him. Bert thought that he should have
kept the letters, perhaps to return them to a bereaved mother in
Hamburg.

This tantalizing image stayed with me. So did another. On Oct. 14, 1944,
near Florence, a small bird settled on Bert's shoulder. It remained
there for five days. This extraordinary encounter, caught in a
photograph on the banks of the Arno, caused Florentines to prostrate
themselves, name Bert ``Captain Uccellino'' (or ``Little Bird'') and
proclaim him a saint. He was far from that but he had about him
something magical.

Of that the days since his death have left no doubt. He is now that bird
on my shoulder, reminding me to take care with my spelling and be aware
that love alone redeems human affairs.

Advertisement

\protect\hyperlink{after-bottom}{Continue reading the main story}

\hypertarget{site-index}{%
\subsection{Site Index}\label{site-index}}

\hypertarget{site-information-navigation}{%
\subsection{Site Information
Navigation}\label{site-information-navigation}}

\begin{itemize}
\tightlist
\item
  \href{https://help.nytimes3xbfgragh.onion/hc/en-us/articles/115014792127-Copyright-notice}{©~2020~The
  New York Times Company}
\end{itemize}

\begin{itemize}
\tightlist
\item
  \href{https://www.nytco.com/}{NYTCo}
\item
  \href{https://help.nytimes3xbfgragh.onion/hc/en-us/articles/115015385887-Contact-Us}{Contact
  Us}
\item
  \href{https://www.nytco.com/careers/}{Work with us}
\item
  \href{https://nytmediakit.com/}{Advertise}
\item
  \href{http://www.tbrandstudio.com/}{T Brand Studio}
\item
  \href{https://www.nytimes3xbfgragh.onion/privacy/cookie-policy\#how-do-i-manage-trackers}{Your
  Ad Choices}
\item
  \href{https://www.nytimes3xbfgragh.onion/privacy}{Privacy}
\item
  \href{https://help.nytimes3xbfgragh.onion/hc/en-us/articles/115014893428-Terms-of-service}{Terms
  of Service}
\item
  \href{https://help.nytimes3xbfgragh.onion/hc/en-us/articles/115014893968-Terms-of-sale}{Terms
  of Sale}
\item
  \href{https://spiderbites.nytimes3xbfgragh.onion}{Site Map}
\item
  \href{https://help.nytimes3xbfgragh.onion/hc/en-us}{Help}
\item
  \href{https://www.nytimes3xbfgragh.onion/subscription?campaignId=37WXW}{Subscriptions}
\end{itemize}
