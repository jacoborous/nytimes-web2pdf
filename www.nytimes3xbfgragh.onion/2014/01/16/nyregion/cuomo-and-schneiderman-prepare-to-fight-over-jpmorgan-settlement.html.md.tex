Sections

SEARCH

\protect\hyperlink{site-content}{Skip to
content}\protect\hyperlink{site-index}{Skip to site index}

\href{https://www.nytimes3xbfgragh.onion/section/nyregion}{New York}

\href{https://myaccount.nytimes3xbfgragh.onion/auth/login?response_type=cookie\&client_id=vi}{}

\href{https://www.nytimes3xbfgragh.onion/section/todayspaper}{Today's
Paper}

\href{/section/nyregion}{New York}\textbar{}With \$613 Million at Stake,
an Albany Rivalry Is Said to Escalate

\url{https://nyti.ms/1ht2hf4}

\begin{itemize}
\item
\item
\item
\item
\item
\item
\end{itemize}

Advertisement

\protect\hyperlink{after-top}{Continue reading the main story}

Supported by

\protect\hyperlink{after-sponsor}{Continue reading the main story}

\hypertarget{with-613-million-at-stake-an-albany-rivalry-is-said-to-escalate}{%
\section{With \$613 Million at Stake, an Albany Rivalry Is Said to
Escalate}\label{with-613-million-at-stake-an-albany-rivalry-is-said-to-escalate}}

\includegraphics{https://static01.graylady3jvrrxbe.onion/images/2014/01/16/nyregion/Feud/Feud-articleLarge.jpg?quality=75\&auto=webp\&disable=upscale}

By \href{http://www.nytimes3xbfgragh.onion/by/susanne-craig}{Susanne
Craig}

\begin{itemize}
\item
  Jan. 16, 2014
\item
  \begin{itemize}
  \item
  \item
  \item
  \item
  \item
  \item
  \end{itemize}
\end{itemize}

ALBANY --- Gov. Andrew M. Cuomo has asked people if they think Eric T.
Schneiderman, the attorney general of New York State, wears eyeliner.

Mr. Schneiderman has told people that he believes Mr. Cuomo's
administration is Machiavellian and is out to undermine him.

A little backbiting by the officials and their aides, who occupy power
suites at opposite ends of the State Capitol's second floor, might be
chalked up to the kind of rivalry that is an unseemly but unsurprising
fact of life atop the state's political food chain. But this
relationship, as described in repetitive detail by many in New York
Democratic circles, has gone from bad to toxic.

``The two men are like oil and water,'' said one Democrat who knows both
of them well, ``and lately fire seems to have been added.''

Numerous people in the two camps were interviewed for this article. None
would allow their names to be used when describing the content of such
private and sensitive conversations.

Now, the fractiousness between Mr. Cuomo and Mr. Schneiderman is
spilling over into the running of state government. With the annual
state budget process about to begin, the two leaders are girding for
battle over how to spend \$613 million obtained by the attorney
general's office in a settlement of securities litigation with JPMorgan
Chase.

Mr. Schneiderman and the bank negotiated the terms so that he would be
given sole discretion over how to allocate the money. He has big plans
for it: preventing avoidable foreclosures for thousands of struggling
homeowners and expanding his office's efforts to fight financial fraud.

But Mr. Cuomo, who is up for re-election this year, wants the money
deposited in the state's general fund, where it would be used as the
governor and legislators see fit.

The governor's budget, to be announced on Tuesday, will outline how to
pay for various programs. While the JPMorgan settlement will not cover
any expenses in the current state budget, it could finance a number of
big-ticket items, including a year's worth of universal prekindergarten
and programs for middle-school students in New York City or future tax
cuts for businesses.

This is hardly the first time the two men have clashed.

Mr. Schneiderman was elected attorney general in 2010, succeeding Mr.
Cuomo, who was elected governor. But Mr. Cuomo did not support Mr.
Schneiderman's bid in the primary for attorney general.

And while Mr. Schneiderman is said to want Mr. Cuomo's job someday, Mr.
Cuomo has at times seemed to want to hold onto part of Mr.
Schneiderman's.

Not long after they both took office, the governor set up a new agency,
the Department of Financial Services, that conspicuously overlaps the
duties of the attorney general in its pursuit of financial wrongdoing,
and installing a loyal and aggressive Cuomo lieutenant to lead it.

Mr. Schneiderman has found ways to make a name for himself. In his three
years as the state's top law enforcement officer, he has won major
victories in health care and succeeded in closing a loophole that
allowed firearms to be sold at gun shows without background checks. In
2012 alone, he announced more than \$335 million in Medicaid fraud
recoveries.

On Wall Street, he was a central player in concluding the JPMorgan
litigation, recently brought a high-profile case against the asset
manager BlackRock, and has been forceful in challenging financial data
providers that release potentially market-moving information to favored
subscribers ahead of others.

``In just three years in office, Attorney General Schneiderman has
revitalized virtually every aspect of his office, making it a more
effective champion for fairness and equality for all New Yorkers,'' his
spokesman, Damien LaVera, said. ``We'll stack Attorney General
Schneiderman's track record against any of his predecessors any day.''

Still, Mr. Schneiderman has not earned the kind of hard-charging
``sheriff of Wall Street'' image that Mr. Cuomo and, before him, Eliot
Spitzer, gained in the job.

Now, Mr. Cuomo and his advisers, past and present, are increasingly
raising concerns with allies and certain lawmakers about Mr.
Schneiderman's tenure as attorney general.

Mr. Cuomo's bill of particulars against Mr. Schneiderman includes
several missteps, according to these people. Among them: Mr.
Schneiderman's \$613 million settlement with JPMorgan, one of the
largest such settlements in the state's history, should have been turned
over to the general fund, not kept under the attorney general's control.
Equally troubling, they argue, the enormous payment was structured to
allow the bank to use the entire amount as a tax deduction ---
significantly reducing its state tax bill.

In addition, the handoff of another securities case, against Bank of
America, which was investigated during Mr. Cuomo's tenure as attorney
general, was fumbled, these people say, when Mr. Schneiderman's office
failed to settle quickly enough. Private lawyers negotiated their own
settlement first, hampering the state's ability to obtain the strongest
result for itself.

If the interest of Mr. Cuomo and his team has strayed into chatter about
whether the attorney general's eyes show signs of cosmetic intervention,
Mr. Schneiderman has a simple, though little-known, explanation: People
told of his condition say he has glaucoma, for which he takes a
medication whose published side effects include increased eyelash
``thickness'' and ``darkness.''

Mr. Cuomo, 56, and Mr. Schneiderman, 59, are cut from decidedly
different cloth. Mr. Schneiderman is a liberal product of the Upper West
Side of Manhattan who is devoted to yoga. Mr. Cuomo, bred in Queens and
apprenticed by a three-term governor, is a workaholic given to tinkering
with muscle cars.

The two met in the early 1980s while working on the campaign for
governor of Mario M. Cuomo, according to longtime friends; Mr.
Schneiderman was a new Harvard Law graduate, Mr. Cuomo his father's
aide.

But their paths thereafter seldom crossed. Mr. Schneiderman spent years
in private legal practice, with a sideline in pro bono advocacy, before
winning election to the State Senate in 1998. Mr. Cuomo by then had
risen to federal housing secretary.

The two men's circles did overlap in one noteworthy way: Mr.
Schneiderman's ex-wife, Jennifer Cunningham, from whom he was divorced
in 1996, later became an adviser to Mr. Cuomo. Ms. Cunningham, a partner
at the political consulting firm SKDKnickerbocker, remains on good terms
with Mr. Schneiderman.

When Mr. Cuomo first ran for governor, in 2002, Mr. Schneiderman sided
with much of the state Democratic establishment in backing H. Carl
McCall, whose broad support drove Mr. Cuomo to quit the primary. But Mr.
Schneiderman did support Mr. Cuomo's comeback as a candidate for
attorney general in 2006.

Yet in 2010, Mr. Cuomo quietly aided the primary bid of Kathleen M.
Rice, one of Mr. Schneiderman's opponents in the Democratic primary for
attorney general.

Tensions between the two politicians surfaced almost immediately, when
Mr. Cuomo created the Department of Financial Services and put Benjamin
Lawsky, a longtime aide and former federal prosecutor, in charge.

Mr. Cuomo wanted Mr. Lawsky to be able to investigate violations of the
Martin Act, the sweeping New York securities law that has been a crucial
tool in pursuing financial-industry malfeasance. But the governor backed
down after business groups raised concerns.

Still, the decision to install Mr. Lawsky created a powerful rival.

``If you support the attorney general, you don't do that,'' said
Jonathan R. Macey, a Yale law professor who closely follows Mr.
Schneiderman's office.

Although it is unclear whether Mr. Lawsky is interested in running for
political office, among some Wall Street lawyers he is already being
called ``future attorney general.''

The relationship is further complicated by the fact that Mr.
Schneiderman inherited some of Mr. Cuomo's major cases, including the
one against Bank of America.

What remains to be seen in Albany is who will decide how the \$613
million from JPMorgan is spent. Mr. Cuomo's spokeswoman called talks
with Mr. Schneiderman's office ``productive''; a spokesman for Mr.
Schneiderman called them ``constructive.''

The money now sits in an interest-bearing account.

``It's a pretty big slush fund,'' one official close to the governor
said. ``The question now is whose slush fund is it going to be.''

Advertisement

\protect\hyperlink{after-bottom}{Continue reading the main story}

\hypertarget{site-index}{%
\subsection{Site Index}\label{site-index}}

\hypertarget{site-information-navigation}{%
\subsection{Site Information
Navigation}\label{site-information-navigation}}

\begin{itemize}
\tightlist
\item
  \href{https://help.nytimes3xbfgragh.onion/hc/en-us/articles/115014792127-Copyright-notice}{©~2020~The
  New York Times Company}
\end{itemize}

\begin{itemize}
\tightlist
\item
  \href{https://www.nytco.com/}{NYTCo}
\item
  \href{https://help.nytimes3xbfgragh.onion/hc/en-us/articles/115015385887-Contact-Us}{Contact
  Us}
\item
  \href{https://www.nytco.com/careers/}{Work with us}
\item
  \href{https://nytmediakit.com/}{Advertise}
\item
  \href{http://www.tbrandstudio.com/}{T Brand Studio}
\item
  \href{https://www.nytimes3xbfgragh.onion/privacy/cookie-policy\#how-do-i-manage-trackers}{Your
  Ad Choices}
\item
  \href{https://www.nytimes3xbfgragh.onion/privacy}{Privacy}
\item
  \href{https://help.nytimes3xbfgragh.onion/hc/en-us/articles/115014893428-Terms-of-service}{Terms
  of Service}
\item
  \href{https://help.nytimes3xbfgragh.onion/hc/en-us/articles/115014893968-Terms-of-sale}{Terms
  of Sale}
\item
  \href{https://spiderbites.nytimes3xbfgragh.onion}{Site Map}
\item
  \href{https://help.nytimes3xbfgragh.onion/hc/en-us}{Help}
\item
  \href{https://www.nytimes3xbfgragh.onion/subscription?campaignId=37WXW}{Subscriptions}
\end{itemize}
