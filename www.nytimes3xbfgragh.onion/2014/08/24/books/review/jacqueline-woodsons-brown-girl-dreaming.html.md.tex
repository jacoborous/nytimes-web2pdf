Sections

SEARCH

\protect\hyperlink{site-content}{Skip to
content}\protect\hyperlink{site-index}{Skip to site index}

\href{https://www.nytimes3xbfgragh.onion/section/books/review}{Book
Review}

\href{https://myaccount.nytimes3xbfgragh.onion/auth/login?response_type=cookie\&client_id=vi}{}

\href{https://www.nytimes3xbfgragh.onion/section/todayspaper}{Today's
Paper}

\href{/section/books/review}{Book Review}\textbar{}Where We Enter

\url{https://nyti.ms/1qxoLO8}

\begin{itemize}
\item
\item
\item
\item
\item
\end{itemize}

Advertisement

\protect\hyperlink{after-top}{Continue reading the main story}

Supported by

\protect\hyperlink{after-sponsor}{Continue reading the main story}

\href{/column/childrens-books}{Children's Books}

\hypertarget{where-we-enter}{%
\section{Where We Enter}\label{where-we-enter}}

\includegraphics{https://static01.graylady3jvrrxbe.onion/images/2014/08/24/books/review/24CHAMBERS/24CHAMBERS-articleLarge.jpg?quality=75\&auto=webp\&disable=upscale}

By
\href{https://www.nytimes3xbfgragh.onion/by/veronica-chambers}{Veronica
Chambers}

\begin{itemize}
\item
  Aug. 22, 2014
\item
  \begin{itemize}
  \item
  \item
  \item
  \item
  \item
  \end{itemize}
\end{itemize}

I was 14 years old when I first read Nikki Giovanni's masterly
collection of poetry, ``Cotton Candy on a Rainy Day.'' As with most
everything I read between the ages of 12 and 16, there was so much I
didn't understand. I was the first-­generation daughter of people who
came from a small country, a country so small that I had yet to meet
someone from there who could not connect the dots to my family in five
seconds flat. I didn't know a thing about Jim Crow, the American South,
soul food or classic rhythm and blues. Yet like most kids who love to
read, I understood the feeling behind the words, if not all of the
meaning of the words. So when Giovanni wrote:

\emph{We are consumed by people who sing}\\
\emph{the same old song stay:}\\
\emph{as sweet as you are}\\
\emph{in my corner}\\
\emph{Or perhaps just a little bit longer}\\
\emph{But whatever you do don't change baby baby don't change}

I didn't really know what old song she was referring to, but the rhythm
of her words drew me in. And because I was a teenage girl, I was fairly
confident I knew exactly what Giovanni meant when she wrote:

\emph{If loneliness were a grape}\\
\emph{the wine would be vintage}\\
\emph{If it were a wood}\\
\emph{the furniture would be mahogany}\\
\emph{But since it is life it is}\\
\emph{Cotton Candy}\\
\emph{on a rainy day}\\
\emph{The sweet soft essence}\\
\emph{of possibility}\\
\emph{Never quite maturing}

I thought of Nikki Giovanni and the teenage girl I was, almost
constantly, as I read Jacqueline Woodson's wonderful memoir in verse,
``Brown Girl Dreaming,'' because I suspect this book will be to a
generation of girls what Giovanni's book was to mine: a history lesson,
a mash note passed in class, a book to read burrowed underneath the bed
covers and a life raft during long car rides when you want to float far
from wherever you are, and wherever you're going, toward the person you
feel destined to be.

I will say first that the title seems to confine the book in too narrow
a box. I wondered if the author and publishers, by calling the book
``Brown Girl Dreaming,'' were limiting its audience or, at the very
least, the audience of girls who would pick it up right away. Why not
call it ``Home Girl Dreaming'' or ``Tall Girl Dreaming'' or even just
``Girl Dreaming''? I believe strongly in the words of that most expert
of brown girl writers, Lorraine Hansberry, who said, ``To create the
universal, you must pay very great attention to the specific.'' But I
worry that such a specific title might lead a reader --- especially a
teenage reader --- to miss what a big tent Woodson is pitching. Will
girls who aren't brown know, without prompting, that they too are
invited to this party?

\emph{We take our food out to her stoop just as the grown-ups}\\
\emph{start dancing merengue, the women lifting their long dresses}\\
\emph{to show off their fast-moving feet,}\\
\emph{the men clapping and yelling,}\\
\emph{Baila! Baila! until the living room floor disappears.}

You can read ``Brown Girl Dreaming'' in one sitting, but it is as rich a
spread as the potluck table at a family reunion. Sure, you can plow
through the pages, grabbing everything you can in one go, like piling a
plate high with fried chicken and ribs, potato salad and corn bread. And
yes, it's entirely possible to hold that plate with one hand while
balancing a bowl of gumbo and a cup of sweet tea with the other. But
since the food isn't going anywhere, you'll make out just as well, maybe
even a little better, if you pace yourself. If you know Woodson's work
(which includes ``Hush'' and ``This Is the Rope: A Story From the Great
Migration''), read for her life story first:

\emph{Good enough name for me, my father said}\\
\emph{the day I was born.}\\
\emph{Don't see why}\\
\emph{she can't have it, too.}

\emph{But the women said no.}\\
\emph{My mother first.}\\
\emph{Then each aunt, pulling my pink blanket back}\\
\emph{patting the crop of thick curls}\\
\emph{tugging at my new toes}\\
\emph{touching my cheeks.}

\emph{We won't have a girl named Jack, my mother said.}

For young readers in the process of discovering what Anna Julia Cooper
so beautifully called ``when and where I ­enter,'' there are poems
galore. Poems about sibling rivalry, poems about parents who don't take
no mess, poems about grown-ups who make a mess of things and, most
poignantly, poems about the friends who help see you through. Such as
this one, in ``Maria.''

\emph{Late August now}\\
\emph{home from Greenville and ready}\\
\emph{for what the last of the summer brings me.}\\
\emph{All the dreams this city holds}\\
\emph{right outside --- just step through the door and walk}\\
\emph{two doors down to where}\\
\emph{my new best friend, Maria, lives. Every morning,}\\
\emph{I call up to her window, Come outside}\\
\emph{or she rings our bell, Come outside.}\\
\emph{Her hair is crazily curling down past her back,}\\
\emph{the Spanish she speaks like a song}\\
\emph{I am learning to sing.}\\
\emph{Mi amiga, Maria.}\\
\emph{Maria, my friend.}

The short poems are a gift too and made me think of April when the
Academy of American Poets leads a nationwide celebration called Poem in
Your Pocket Day. There are plenty of candidates for poems you can keep
in your pocket in ``Brown Girl Dreaming.'' I especially loved the series
of numbered short poems, threaded throughout the book, called ``How to
Listen.'' This is No. 8:

\emph{Do you remember . . . ?}\\
\emph{someone's always asking and}\\
\emph{someone else, always does.}

In ``Possession,'' A. S. Byatt wrote about how we are transformed by the
act of memorizing poetry ``by heart . . . as though poems were stored in
the bloodstream.'' Jacqueline Woodson's writing can seem so spare, so
effortless, that it is easy to overlook the wonder and magic of her
words. The triumph of ``Brown Girl Dreaming'' is not just in how well
Woodson tells us the story of her life, but in how elegantly she writes
words that make us want to hold those carefully crafted poems close,
apply them to our lives, reach into the mirror she holds up and make the
words and the worlds she explores our own.

This is a book full of poems that cry out to be learned by heart. These
are poems that will, for years to come, be stored in our bloodstream.

Advertisement

\protect\hyperlink{after-bottom}{Continue reading the main story}

\hypertarget{site-index}{%
\subsection{Site Index}\label{site-index}}

\hypertarget{site-information-navigation}{%
\subsection{Site Information
Navigation}\label{site-information-navigation}}

\begin{itemize}
\tightlist
\item
  \href{https://help.nytimes3xbfgragh.onion/hc/en-us/articles/115014792127-Copyright-notice}{©~2020~The
  New York Times Company}
\end{itemize}

\begin{itemize}
\tightlist
\item
  \href{https://www.nytco.com/}{NYTCo}
\item
  \href{https://help.nytimes3xbfgragh.onion/hc/en-us/articles/115015385887-Contact-Us}{Contact
  Us}
\item
  \href{https://www.nytco.com/careers/}{Work with us}
\item
  \href{https://nytmediakit.com/}{Advertise}
\item
  \href{http://www.tbrandstudio.com/}{T Brand Studio}
\item
  \href{https://www.nytimes3xbfgragh.onion/privacy/cookie-policy\#how-do-i-manage-trackers}{Your
  Ad Choices}
\item
  \href{https://www.nytimes3xbfgragh.onion/privacy}{Privacy}
\item
  \href{https://help.nytimes3xbfgragh.onion/hc/en-us/articles/115014893428-Terms-of-service}{Terms
  of Service}
\item
  \href{https://help.nytimes3xbfgragh.onion/hc/en-us/articles/115014893968-Terms-of-sale}{Terms
  of Sale}
\item
  \href{https://spiderbites.nytimes3xbfgragh.onion}{Site Map}
\item
  \href{https://help.nytimes3xbfgragh.onion/hc/en-us}{Help}
\item
  \href{https://www.nytimes3xbfgragh.onion/subscription?campaignId=37WXW}{Subscriptions}
\end{itemize}
