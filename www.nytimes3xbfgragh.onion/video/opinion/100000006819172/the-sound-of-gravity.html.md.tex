Sections

SEARCH

\protect\hyperlink{site-content}{Skip to
content}\protect\hyperlink{site-index}{Skip to site index}

new video loaded: The Sound of Gravity

transcript

Back

bars

0:00/13:02

-13:02

transcript

\hypertarget{the-sound-of-gravity}{%
\subsection{The Sound of Gravity}\label{the-sound-of-gravity}}

\hypertarget{albert-einstein-had-a-theory-these-scientists-proved-it-a-century-later}{%
\paragraph{Albert Einstein had a theory. These scientists proved it a
century
later.}\label{albert-einstein-had-a-theory-these-scientists-proved-it-a-century-later}}

\begin{itemize}
\tightlist
\item
  Wherever you go, there are things in nature that you can see or things
  that you hear. {[}MUSIC PLAYING{]} Things that sort of grab you and
  put some question in your mind. For example, you see a rainbow. I
  mean, they're this wonderful mysterious thing. But why? Why does this
  happen? And we find out it's because of the way the different colors
  get separated inside of a water drop. Understanding what's going on in
  nature, that makes you more at unity with what is around you. That's
  what science is about. And sometimes, when you look at the world,
  you're going to learn something brand new. Something nobody has ever
  seen before. And that's what this whole story is about. {[}MUSIC
  PLAYING{]} So when I was a kid, I was sort of a lousy student. School
  was quite rigid and grades were a thing that some people made a big
  fuss over. I just couldn't see it. But I got mixed up with electronics
  real early in my life. There was a section of New York City, which,
  for nickels and dimes, I could buy all sorts of wonderful electronic
  junk. You could make almost anything you wanted. And so I learned most
  of the things that I learned in my life by building something and then
  trying to make it work. So I came to M.I.T., and they took me, which
  is sort of miraculous. I don't know why, but they took me. And I
  became a physicist, by default. Because it looked to me like physics
  was a place where you had a little more freedom. And that's where,
  suddenly, I got my education into what became permanently part of my
  life. And it was the Einstein Theory of General Relativity. Now
  general relativity is tricky. But the way I can best explain it to you
  is that it was a completely different conceptualization of how gravity
  operates. So for Einstein, the way he thought about gravity was it
  actually distorted space and time. Took straight lines in space and
  made them curved lines. It's a thing that actually changes the realm
  of nature. At the time, it was just unbelievable shock-worthy stuff.
  But it turns out he was absolutely right. You have to understand,
  Einstein had an intuition that was remarkable. He understood nature in
  a visceral way, almost like he had a direct pipeline to God, if you
  want to call it that. Then in 1916, he writes about something called
  gravitational waves. He recognizes that matter makes waves when it
  gets accelerated. And it travels at the velocity of light and
  stretches space in this dimension and compresses space in that
  dimension. And it goes through everything. It just --- nothing stops
  the gravitational waves. But then, he comes to the conclusion right at
  the end of that paper and says, look, this new thing I just invented,
  this is never going to amount to anything you can measure. It's just
  too tiny. It's hopeless. But the fact is that, in this case, Einstein
  was completely wrong. And that is really that whole story of LIGO.
  {[}TAPE REWINDING{]} Anyway, so let's get back to M.I.T. Eventually, I
  became a faculty member. And I was asked to teach a general relativity
  course to graduate students. And as the course wore on, the students
  asked me how you would make a device that can measure gravitational
  waves. So I remember, on Sunday night, and the lecture was on Tuesday,
  it suddenly dawned on me, maybe the right way to explain this is
  really simple. You put a mass over here, and you put another mass over
  there. Put a mirror on this. And then you take a light source and
  measure the time it takes the light to go from here to there and back
  again. And now what happens is that a gravitational wave comes down on
  this thing and change the space in between those two masses. And that
  would change the time it took the light to go back and forth. Very
  simple calculation. I said that, ``Eh! That's the way to do a
  gravitational wave detection. Now you understand.'' So I said, ``OK,
  I'd like to build a prototype.'' So one of us has to stand on the
  table and --- And it eventually became clear to me I had to have three
  masses so you could measure both the stretching of space as well as
  the shrinking of space. And then you could compare those two. I
  clearly had to use lasers. And then we had to measure the tiny little
  motions that are associated with the gravitational wave. It's about
  one-thousandth the size of a nucleus of an atom. Such a small fraction
  that nobody can conceive of something so small. And measure it ---
  that's crazy! But anyway, we built a prototype. And we started
  operating it. So the laser is here. The light comes out and goes
  streaming through a whole bunch of stuff. And the first thing, this is
  Rai's phase modulating crystals. But at the same time, there was no
  support for this crazy research. Because I wasn't saying we are going
  to see gravitational waves with it. I couldn't. I would have lied if I
  said we have the technology now. We had another factor of a million to
  go. People threw us out the door when we told them that. But the
  National Science Foundation started funding me at a reasonable level
  so I could start building bigger detectors and a bigger team. I first
  met Rai Weiss as a first-year graduate student at M.I.T. And when I
  talked to him about gravitational waves, I actually thought he was
  really just crazy, because it was such an impossible measurement to
  make. But if it worked, it was really going to open up a completely
  new window into the universe. So I mean who can resist that? {[}MUSIC
  PLAYING{]} So when I joined, I was part of the team that built the
  LIGO Gravitational Wave Observatory that was proposed by Rai Weiss in
  the early 1970s. He proposed that you have to use very long detectors.
  So we built these four-kilometer-long facilities in Washington and
  Louisiana that were large enough that it became kind of possible to
  think about detecting gravitational waves from them. But at the same
  time, there were no guarantees that we'd see anything. We knew we were
  charting unknown territory. So LIGO is a gravitational wave telescope.
  And because of Einstein's theory, we knew that out in the universe we
  have black holes. And black holes, in some sense, are the most
  gravitational object you could imagine. They have so much mass in a
  small volume that even light can't escape their gravitational pull.
  But by the `70s and `80s, we had the theory that black holes can exist
  in pairs, orbiting each other. And they get closer and closer, and
  they orbit faster and faster until eventually they collide. And in
  that process, they radiate these ripples of spacetime that travel to
  us. And we here on the Earth get rocked, ever so slightly, by the
  passing gravitational wave. And then, in the detector, those mirrors
  will move closer together or farther apart. And so our computer would
  see the signal, which is just a series of peaks and troughs that were
  growing in size and getting closer in time. So that's what we were
  trying to see. But let me just tell you, no one had ever seen two
  black holes collide. And it was seen as something that was a pretty
  long shot. So yeah, I had moments where I would really question, you
  know, will we see these gravitational waves in my lifetime? Because
  even if it occurs in nature, we weren't sure if the instrument would
  be sensitive enough. But here we were in September 2015, construction
  was coming close to completion, and we had to at some point say, ``OK,
  the improvements are going to wait, we're going to turn on the
  instrument.'' And then, overnight, the instrument registered something
  interesting. Whoa, what's that? What's happened? I go to the computer,
  and I look at the screen. And lo and behold, there is this incredible
  picture of the wave form. And it looked like exactly the thing that
  had been imagined by Einstein. Now I have to admit to you that I was
  still skeptical for a long, long time. Because especially with so many
  reputations and all that beautiful work that had now gone into this
  thing, we did not want to come and make a false detection. But
  eventually, it came down to the point where we began to believe it.
  Ladies and gentlemen, we have detected gravitational waves. We did it.
  Scientists have announced what may be among the greatest discoveries
  in the history of physics. Gravity waves predicted by Einstein, but
  never observed. My first thought was, too bad Einstein isn't alive
  anymore, I'd love to tell him about it. And what would he have said?
  My guess is that Einstein would have been tickled pink but also very
  skeptical and want to know exactly how the technology worked. That's
  my guess. And then, it only slowly dawned on us that we had made new
  science. Science which is really at the basic core of the fundamentals
  of what makes nature operate. The Royal Swedish Academy of Sciences
  has decided to award the 2017 Nobel Prize in Physics to Rainer Weiss
  for the observation of gravitational waves. Rai Weiss knew how to
  measure gravitational waves in 1972. And he's been going at it ever
  since. And so thanks to this work, suddenly, we have developed an
  entirely new sense for observing the universe. We are hearing gravity
  for the first time. And in some sense, we may be able to sort of hear
  the assembly of the first galaxies to try to understand how our
  universe came to be and was assembled. It's pretty amazing. I think
  the science of physics --- the natural philosophy of the universe ---
  belongs to everybody. Belongs to you, belongs to me, belongs to
  everybody. But, I mean, I am 87. And so this new science is something
  which is a challenge to the young people. That's the future. So now
  let me ask you a question. With gravitational waves, you have a new
  way to look at the universe. You can see all of what nature has in
  store. So now comes the question: What do you want to find out? It's
  going to be spectacular, I think.
\end{itemize}

\hypertarget{the-sound-of-gravity-1}{%
\section{The Sound of Gravity}\label{the-sound-of-gravity-1}}

By Sarah Klein and Tom Mason•April 28, 2020

\hypertarget{albert-einstein-had-a-theory-these-scientists-proved-it-a-century-later-1}{%
\subsection{Albert Einstein had a theory. These scientists proved it a
century
later.}\label{albert-einstein-had-a-theory-these-scientists-proved-it-a-century-later-1}}

\begin{itemize}
\item
\item
\item
\item
\end{itemize}

\begin{itemize}
\item
  \href{https://www.nytimes3xbfgragh.onion/video/opinion/100000007304962/all-cats-are-gray-in-the-dark.html?action=click\&module=video-series-bar\&region=header\&pgtype=Article\&playlistId=video/op-docs}{}

  \includegraphics{https://static01.graylady3jvrrxbe.onion/images/2020/09/03/opinion/opdoc-all-cats-are-gray-img/opdoc-all-cats-are-gray-img-square320-v2.jpg}

  17:08

  \hypertarget{all-cats-are-gray-in-the-dark}{%
  \subsubsection{All Cats Are Gray in the
  Dark}\label{all-cats-are-gray-in-the-dark}}
\item
  \href{https://www.nytimes3xbfgragh.onion/video/opinion/100000006831441/gods-from-space.html?action=click\&module=video-series-bar\&region=header\&pgtype=Article\&playlistId=video/op-docs}{}

  \includegraphics{https://static01.graylady3jvrrxbe.onion/images/2020/08/20/opinion/opdoc-gods-from-space-img/opdoc-gods-from-space-img-square320.jpg}

  5:22

  \hypertarget{gods-from-space}{%
  \subsubsection{Gods From Space}\label{gods-from-space}}
\item
  \href{https://www.nytimes3xbfgragh.onion/video/opinion/100000007249913/dying-in-your-mothers-arms.html?action=click\&module=video-series-bar\&region=header\&pgtype=Article\&playlistId=video/op-docs}{}

  \includegraphics{https://static01.graylady3jvrrxbe.onion/images/2020/08/13/opinion/opdoc-palliative-img-print/opdoc-palliative-img-square320.jpg}

  22:21

  \hypertarget{dying-in-your-mothers-arms}{%
  \subsubsection{Dying in Your Mother's
  Arms}\label{dying-in-your-mothers-arms}}
\item
  \href{https://www.nytimes3xbfgragh.onion/video/opinion/100000007247238/tears-teacher.html?action=click\&module=video-series-bar\&region=header\&pgtype=Article\&playlistId=video/op-docs}{}

  \includegraphics{https://static01.graylady3jvrrxbe.onion/images/2020/07/30/opinion/opdoc-tears-teacher-img-print/opdoc-tears-teacher-img-square320.jpg}

  10:54

  \hypertarget{tears-teacher}{%
  \subsubsection{Tears Teacher}\label{tears-teacher}}
\item
  \href{https://www.nytimes3xbfgragh.onion/video/opinion/100000007229285/the-lonely-goalkeeper.html?action=click\&module=video-series-bar\&region=header\&pgtype=Article\&playlistId=video/op-docs}{}

  \includegraphics{https://static01.graylady3jvrrxbe.onion/images/2020/07/25/opinion/25video/opdoc-lonely-goalkeeper-img-square320.jpg}

  4:01

  \hypertarget{the-lonely-goalkeeper}{%
  \subsubsection{The Lonely Goalkeeper}\label{the-lonely-goalkeeper}}
\item
  \href{https://www.nytimes3xbfgragh.onion/video/opinion/100000007080462/huntsville-station.html?action=click\&module=video-series-bar\&region=header\&pgtype=Article\&playlistId=video/op-docs}{}

  \includegraphics{https://static01.graylady3jvrrxbe.onion/images/2020/07/09/opinion/09a2_video/opdoc-huntsville-station-img-square320.jpg}

  11:17

  \hypertarget{huntsville-station}{%
  \subsubsection{Huntsville Station}\label{huntsville-station}}
\item
  \href{https://www.nytimes3xbfgragh.onion/video/opinion/100000007205588/the-torture-letters.html?action=click\&module=video-series-bar\&region=header\&pgtype=Article\&playlistId=video/op-docs}{}

  \includegraphics{https://static01.graylady3jvrrxbe.onion/images/2020/06/29/opinion/opdoc-torture-letters-img/opdoc-torture-letters-img-square320.jpg}

  12:51

  \hypertarget{the-torture-letters}{%
  \subsubsection{The Torture Letters}\label{the-torture-letters}}
\item
  \href{https://www.nytimes3xbfgragh.onion/video/opinion/100000007172575/forgiveness-day.html?action=click\&module=video-series-bar\&region=header\&pgtype=Article\&playlistId=video/op-docs}{}

  \includegraphics{https://static01.graylady3jvrrxbe.onion/images/2020/06/27/opinion/opdoc-forgiveness-day-img-alt/opdoc-forgiveness-day-img-alt-square320.jpg}

  15:11

  \hypertarget{forgiveness-day}{%
  \subsubsection{Forgiveness Day}\label{forgiveness-day}}
\item
  \href{https://www.nytimes3xbfgragh.onion/video/opinion/100000007133685/all-i-have-to-offer-you-is-me.html?action=click\&module=video-series-bar\&region=header\&pgtype=Article\&playlistId=video/op-docs}{}

  \includegraphics{https://static01.graylady3jvrrxbe.onion/images/2020/06/18/opinion/opdoc-all-i-have-to-offer-img/opdoc-all-i-have-to-offer-img-square320.jpg}

  13:53

  \hypertarget{all-i-have-to-offer-you-is-me}{%
  \subsubsection{All I Have to Offer You Is
  Me}\label{all-i-have-to-offer-you-is-me}}
\item
  \href{https://www.nytimes3xbfgragh.onion/video/opinion/100000007139379/how-to-be-alone.html?action=click\&module=video-series-bar\&region=header\&pgtype=Article\&playlistId=video/op-docs}{}

  \includegraphics{https://static01.graylady3jvrrxbe.onion/images/2020/05/23/opinion/opdoc-how-to-be-alone-img/opdoc-how-to-be-alone-img-square320-v2.jpg}

  6:11

  \hypertarget{how-to-be-alone}{%
  \subsubsection{How to Be Alone}\label{how-to-be-alone}}
\item
  \href{https://www.nytimes3xbfgragh.onion/video/opinion/100000006590759/container-greece-migrants.html?action=click\&module=video-series-bar\&region=header\&pgtype=Article\&playlistId=video/op-docs}{}

  \includegraphics{https://static01.graylady3jvrrxbe.onion/images/2020/05/17/opinion/opdoc-container-img/opdoc-container-img-square320.jpg}

  18:07

  \hypertarget{container}{%
  \subsubsection{Container}\label{container}}
\item
  \includegraphics{https://static01.graylady3jvrrxbe.onion/images/2020/04/30/opinion/30video/opdoc-gravity-waves-img-alt-square320.jpg}

  NOW PLAYING

  \hypertarget{the-sound-of-gravity-2}{%
  \subsubsection{The Sound of Gravity}\label{the-sound-of-gravity-2}}
\item
  \href{https://www.nytimes3xbfgragh.onion/video/opinion/100000007062353/coronavirus-messages-from-quarantine.html?action=click\&module=video-series-bar\&region=header\&pgtype=Article\&playlistId=video/op-docs}{}

  \includegraphics{https://static01.graylady3jvrrxbe.onion/images/2020/04/02/opinion/opdoc-red-zone-img-alt/opdoc-red-zone-img-alt-square320.jpg}

  7:36

  \hypertarget{messages-from-quarantine}{%
  \subsubsection{Messages From
  Quarantine}\label{messages-from-quarantine}}
\item
  \href{https://www.nytimes3xbfgragh.onion/video/opinion/100000007026836/hysterical-girl.html?action=click\&module=video-series-bar\&region=header\&pgtype=Article\&playlistId=video/op-docs}{}

  \includegraphics{https://static01.graylady3jvrrxbe.onion/images/2020/03/17/opinion/opdoc-hysterical-girl-img/opdoc-hysterical-girl-img-square320.jpg}

  13:25

  \hypertarget{hysterical-girl}{%
  \subsubsection{Hysterical Girl}\label{hysterical-girl}}
\item
  \href{https://www.nytimes3xbfgragh.onion/video/opinion/100000007013675/self-quarantined-for-the-holidays.html?action=click\&module=video-series-bar\&region=header\&pgtype=Article\&playlistId=video/op-docs}{}

  \includegraphics{https://static01.graylady3jvrrxbe.onion/images/2020/03/09/opinion/opdoc-coronavirus-img/opdoc-coronavirus-img-square320.jpg}

  11:32

  \hypertarget{self-quarantined-for-the-holidays}{%
  \subsubsection{Self-Quarantined for the
  Holidays}\label{self-quarantined-for-the-holidays}}
\item
  \href{https://www.nytimes3xbfgragh.onion/video/opinion/100000006948977/single-in-china.html?action=click\&module=video-series-bar\&region=header\&pgtype=Article\&playlistId=video/op-docs}{}

  \includegraphics{https://static01.graylady3jvrrxbe.onion/images/2020/02/06/opinion/single-in-china-img/single-in-china-img-square320.jpg}

  9:07

  \hypertarget{single-in-china}{%
  \subsubsection{Single in China}\label{single-in-china}}
\item
  \href{https://www.nytimes3xbfgragh.onion/video/opinion/100000006946174/now-is-the-time.html?action=click\&module=video-series-bar\&region=header\&pgtype=Article\&playlistId=video/op-docs}{}

  \includegraphics{https://static01.graylady3jvrrxbe.onion/images/2020/02/09/opinion/opdoc-now-is-the-time-img/opdoc-now-is-the-time-img-square320.jpg}

  16:15

  \hypertarget{now-is-the-time}{%
  \subsubsection{Now Is the Time}\label{now-is-the-time}}
\item
  \href{https://www.nytimes3xbfgragh.onion/video/opinion/100000006923399/betye-saar-taking-care-of-business.html?action=click\&module=video-series-bar\&region=header\&pgtype=Article\&playlistId=video/op-docs}{}

  \includegraphics{https://static01.graylady3jvrrxbe.onion/images/2020/01/24/opinion/opdoc-betye-saar-img-alt/opdoc-betye-saar-img-alt-square320.jpg}

  8:22

  \hypertarget{betye-saar-taking-care-of-business}{%
  \subsubsection{Betye Saar: Taking Care of
  Business}\label{betye-saar-taking-care-of-business}}
\item
  \href{https://www.nytimes3xbfgragh.onion/video/opinion/100000006831435/music-and-clowns.html?action=click\&module=video-series-bar\&region=header\&pgtype=Article\&playlistId=video/op-docs}{}

  \includegraphics{https://static01.graylady3jvrrxbe.onion/images/2020/01/09/opinion/opdoc-music-clowns-img/opdoc-music-clowns-img-square320.jpg}

  7:38

  \hypertarget{music-and-clowns}{%
  \subsubsection{Music and Clowns}\label{music-and-clowns}}
\item
  \href{https://www.nytimes3xbfgragh.onion/video/opinion/100000006865864/almost-famous-the-lost-astronaut.html?action=click\&module=video-series-bar\&region=header\&pgtype=Article\&playlistId=video/op-docs}{}

  \includegraphics{https://static01.graylady3jvrrxbe.onion/images/2019/12/23/opinion/opdoc-dwight-img/opdoc-dwight-img-square320.jpg}

  12:44

  \hypertarget{almost-famous-the-lost-astronaut}{%
  \subsubsection{Almost Famous: The Lost
  Astronaut}\label{almost-famous-the-lost-astronaut}}
\item
  \href{https://www.nytimes3xbfgragh.onion/video/opinion/100000006865876/almost-famous-the-other-fab-four.html?action=click\&module=video-series-bar\&region=header\&pgtype=Article\&playlistId=video/op-docs}{}

  \includegraphics{https://static01.graylady3jvrrxbe.onion/images/2019/12/16/opinion/opdoc-liverbirds-img/opdoc-liverbirds-img-square320.jpg}

  16:13

  \hypertarget{almost-famous-the-other-fab-four}{%
  \subsubsection{Almost Famous: The Other Fab
  Four}\label{almost-famous-the-other-fab-four}}
\item
  \href{https://www.nytimes3xbfgragh.onion/video/opinion/100000006865878/almost-famous-kim-i-am.html?action=click\&module=video-series-bar\&region=header\&pgtype=Article\&playlistId=video/op-docs}{}

  \includegraphics{https://static01.graylady3jvrrxbe.onion/images/2019/12/18/opinion/opdoc-kim-hill-img/opdoc-kim-hill-img-square320.jpg}

  13:47

  \hypertarget{almost-famous-kim-i-am}{%
  \subsubsection{Almost Famous: Kim I Am}\label{almost-famous-kim-i-am}}
\item
  \href{https://www.nytimes3xbfgragh.onion/video/opinion/100000006808736/the-church-forests-of-ethiopia.html?action=click\&module=video-series-bar\&region=header\&pgtype=Article\&playlistId=video/op-docs}{}

  \includegraphics{https://static01.graylady3jvrrxbe.onion/images/2019/12/05/opinion/3opdoc-church-forests-img-alt/3opdoc-church-forests-img-alt-square320-v3.jpg}

  9:10

  \hypertarget{the-church-forests-of-ethiopia}{%
  \subsubsection{The Church Forests of
  Ethiopia}\label{the-church-forests-of-ethiopia}}
\item
  \href{https://www.nytimes3xbfgragh.onion/video/opinion/100000006616228/mumbais-midnight-gardeners.html?action=click\&module=video-series-bar\&region=header\&pgtype=Article\&playlistId=video/op-docs}{}

  \includegraphics{https://static01.graylady3jvrrxbe.onion/images/2019/11/01/opinion/29opdoc-mumbai-gardners-img/29opdoc-mumbai-gardners-img-square320-v2.jpg}

  11:27

  \hypertarget{mumbais-midnight-gardeners}{%
  \subsubsection{Mumbai's Midnight
  Gardeners}\label{mumbais-midnight-gardeners}}
\end{itemize}

Recent episodes in Op-Docs

Op-Docs is the New York Times' award-winning series of short
documentaries by independent filmmakers. From emerging directors to
Oscar winners, Op-Docs brings you the very best nonfiction filmmaking
from around the world.

Op-Docs is the New York Times' award-winning series of short
documentaries by independent filmmakers. From emerging directors to
Oscar winners, Op-Docs brings you the very best nonfiction filmmaking
from around the world.

Show more videos from Op-Docs

\href{/video}{}

\href{/video/latest-video}{Latest Video}

\href{/video/hk-protest}{Hong Kong Protests}

\href{/video/2020-Elections}{2020 Elections}

\href{/video/Most-Viewed}{Most-Viewed}

\href{/video/investigations}{Visual Investigations}

\href{/video/on-the-ground}{The Dispatch}

\href{/video/diaryofasong}{Diary of a Song}

\href{/video/how-we-got-here}{How We Got Here}

\href{/video/magazine}{Magazine}

\href{/video/t-magazine}{T Magazine}

\href{/video/op-docs}{Op-Docs}

\href{/video/opinion}{Opinion}

Advertisement

\protect\hyperlink{after-bottom}{Continue reading the main story}

\hypertarget{site-index}{%
\subsection{Site Index}\label{site-index}}

\hypertarget{site-information-navigation}{%
\subsection{Site Information
Navigation}\label{site-information-navigation}}

\begin{itemize}
\tightlist
\item
  \href{https://help.nytimes3xbfgragh.onion/hc/en-us/articles/115014792127-Copyright-notice}{©~2020~The
  New York Times Company}
\end{itemize}

\begin{itemize}
\tightlist
\item
  \href{https://www.nytco.com/}{NYTCo}
\item
  \href{https://help.nytimes3xbfgragh.onion/hc/en-us/articles/115015385887-Contact-Us}{Contact
  Us}
\item
  \href{https://www.nytco.com/careers/}{Work with us}
\item
  \href{https://nytmediakit.com/}{Advertise}
\item
  \href{http://www.tbrandstudio.com/}{T Brand Studio}
\item
  \href{https://www.nytimes3xbfgragh.onion/privacy/cookie-policy\#how-do-i-manage-trackers}{Your
  Ad Choices}
\item
  \href{https://www.nytimes3xbfgragh.onion/privacy}{Privacy}
\item
  \href{https://help.nytimes3xbfgragh.onion/hc/en-us/articles/115014893428-Terms-of-service}{Terms
  of Service}
\item
  \href{https://help.nytimes3xbfgragh.onion/hc/en-us/articles/115014893968-Terms-of-sale}{Terms
  of Sale}
\item
  \href{https://spiderbites.nytimes3xbfgragh.onion}{Site Map}
\item
  \href{https://help.nytimes3xbfgragh.onion/hc/en-us}{Help}
\item
  \href{https://www.nytimes3xbfgragh.onion/subscription?campaignId=37WXW}{Subscriptions}
\end{itemize}
