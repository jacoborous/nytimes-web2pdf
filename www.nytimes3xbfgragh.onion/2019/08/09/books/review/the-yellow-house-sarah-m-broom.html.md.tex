Sections

SEARCH

\protect\hyperlink{site-content}{Skip to
content}\protect\hyperlink{site-index}{Skip to site index}

\href{https://www.nytimes3xbfgragh.onion/section/books/review}{Book
Review}

\href{https://myaccount.nytimes3xbfgragh.onion/auth/login?response_type=cookie\&client_id=vi}{}

\href{https://www.nytimes3xbfgragh.onion/section/todayspaper}{Today's
Paper}

\href{/section/books/review}{Book Review}\textbar{}After Hurricane
Katrina, How Do You Return Home When Home No Longer Exists?

\url{https://nyti.ms/2GWX1ES}

\begin{itemize}
\item
\item
\item
\item
\item
\item
\end{itemize}

Advertisement

\protect\hyperlink{after-top}{Continue reading the main story}

Supported by

\protect\hyperlink{after-sponsor}{Continue reading the main story}

Nonfiction

\hypertarget{after-hurricane-katrina-how-do-you-return-home-when-home-no-longer-exists}{%
\section{After Hurricane Katrina, How Do You Return Home When Home No
Longer
Exists?}\label{after-hurricane-katrina-how-do-you-return-home-when-home-no-longer-exists}}

\includegraphics{https://static01.graylady3jvrrxbe.onion/images/2019/08/11/books/review/11Flournoy-COVER_SUB04/merlin_158687685_ab2de7e2-2b57-4903-898b-7c67bbb5bd03-articleLarge.jpg?quality=75\&auto=webp\&disable=upscale}

Buy Book ▾

\begin{itemize}
\tightlist
\item
  \href{https://www.amazon.com/gp/search?index=books\&tag=NYTBSREV-20\&field-keywords=The+Yellow+House+Sarah+M.+Broom}{Amazon}
\item
  \href{https://du-gae-books-dot-nyt-du-prd.appspot.com/buy?title=The+Yellow+House\&author=Sarah+M.+Broom}{Apple
  Books}
\item
  \href{https://www.anrdoezrs.net/click-7990613-11819508?url=https\%3A\%2F\%2Fwww.barnesandnoble.com\%2Fw\%2F\%3Fean\%3D9780802125088}{Barnes
  and Noble}
\item
  \href{https://www.anrdoezrs.net/click-7990613-35140?url=https\%3A\%2F\%2Fwww.booksamillion.com\%2Fp\%2FThe\%2BYellow\%2BHouse\%2FSarah\%2BM.\%2BBroom\%2F9780802125088}{Books-A-Million}
\item
  \href{https://bookshop.org/a/3546/9780802125088}{Bookshop}
\item
  \href{https://www.indiebound.org/book/9780802125088?aff=NYT}{Indiebound}
\end{itemize}

When you purchase an independently reviewed book through our site, we
earn an affiliate commission.

By Angela Flournoy

\begin{itemize}
\item
  Aug. 9, 2019
\item
  \begin{itemize}
  \item
  \item
  \item
  \item
  \item
  \item
  \end{itemize}
\end{itemize}

\textbf{\textbf{THE YELLOW HOUSE}}\\
By Sarah M. Broom

\emph{{[}}
\href{https://www.nytimes3xbfgragh.onion/2019/11/22/books/review/best-books.html}{\emph{This
was selected as one of the Book Review's 10 best books of 2019. See the
full list}}\emph{. {]}}

When Sarah M. Broom was in high school, she and her mother briefly
attended a revivalist megachurch near their New Orleans home, the kind
where people got ``drunk on the Holy Spirit'' and burned with ``Holy
Ghost Fire.'' The youngest of 12 siblings, Broom spent her childhood on
the hunt for an adequate mode of self-expression. She took to the
practice of speaking in tongues. ``The only control was in letting go,''
she writes. ``When you gave yourself over to it, it came bubbling out
from you, this foreign language you did not need to study for, that was
specific to you and your tongue, and that you did not know you spoke ---
until you did.'' As an adult, Broom continued to seek the place and
language that felt most like home to her, and it wasn't until she
returned to New Orleans, with its particular cadence and history and
sins, that she found it.

\includegraphics{https://static01.graylady3jvrrxbe.onion/images/2019/08/11/books/review/11Flournoy/merlin_158286672_40d99bd6-b6bc-4c1a-b5b8-e4897a5d67f1-articleLarge.jpg?quality=75\&auto=webp\&disable=upscale}

This journey is one aspect of Broom's extraordinary, engrossing debut,
``The Yellow House,'' but Broom recognizes that she needs to find the
language to tell an even more expansive story. She pushes past the
baseline expectations of memoir as a genre to create an entertaining and
inventive amalgamation of literary forms. Part oral history, part urban
history, part celebration of a bygone way of life, ``The Yellow House''
is a full indictment of the greed, discrimination, indifference and poor
city planning that led her family's home to be wiped off the map. It is
an instantly essential text, examining the past, present and possible
future of the city of New Orleans, and of America writ large.

New Orleans East, where Broom grew up, is an area ``50 times the size of
the French Quarter,'' yet nowhere to be found on most tourist maps. Her
neighborhood, centered on the short end of a street cut off from the
rest of the city by a raging thoroughfare, is a familiar sort for many
black folks in this country: comprising the scraps of real estate whites
have passed over or fled. We witness the street through the eyes of
Broom's 11 older siblings, who saw it transform from integrated and
residential to segregated and ``light industrial'' over the years
beginning in the '60s. Broom leaves the street and New Orleans behind in
the late '90s.

The city demolished the house less than a year after Hurricane Katrina,
the only prior notice having been, in an act of civic absurdity, mailed
to its address. ``Remembering is a chair that it is hard to sit still
in,'' Broom writes. ``The Yellow House'' is a conscious act of abiding
in such memories in order to create a textual record where the physical
one no longer exists.

Image

Broom's siblings and their father outside their home at 4121 Wilson in
1977.Credit...via Sarah M. Broom

\emph{{[}}
\href{https://www.nytimes3xbfgragh.onion/2019/11/20/books/national-book-award-winners-susan-choi-sarah-broom.html}{\emph{"The
Yellow House" won the 2019 National Book Award for nonfiction. Read more
about the other award winners}}\emph{. {]}}

Broom is our guide, but not the sort who holds readers' hands,
uninterested as she is in tidy transitions between one type of writing
and another. The through line is her thought process, her frequent
questioning: ``When you come from a mythologized place, as I do, who are
you in that story?'' she asks while living for a year in the French
Quarter after a lifetime of merely shuttling through it for work. ``Why
do I sometimes feel that I do not have the right to the story of the
city I come from?'' she asks after signing the contract for this book
and embarking on the research to write it. One question posed in the
center of the book --- ``How to resurrect a house with words?'' ---
trembles beneath the surface of every page, like the ripple of a stone
dropped in water.

Broom searches for her own answers, undertaking what she calls
``investigations'' via archives and interviews and living. She claims
that her favorite place to be is ``on the verge of discovery,'' and
because she is skilled at making each inquiry feel urgent, this quickly
becomes the reader's favorite place as well.

Image

Broom in her living room mirror.Credit...via Sarah M. Broom

Similar to the writer Gayl Jones, who in works like the novel
``Corregidora'' uses her characters' dialogue to create a subtext of
knotted history, Broom allows us to infer what might lie in the silences
between the words her family members speak to her, during what must have
amounted to whole days' worth of recorded interviews. Here is Broom's
mother, Ivory Mae, remembering her own darker-skinned mother: ``She
wasn't black to me. She was my mama and my mama wasn't black. Looked to
me like they was trying to make my mama like the black people I didn't
like.''

The interviews also yield unforgettable scenes. As the waters rose
during the worst of Hurricane Katrina, Broom's older brother Carl, who
also goes by Rabbit, stood in an attic with a meat cleaver, a gun and
his two Pekingese dogs, Mindy and Tiger. Carl hacked his way out onto
the roof, and the three were eventually ferried to dry land. ``Mindy and
them wasn't on no leash,'' he recalls. ``I had some Adidas tennis on,
but they was so tight. I took the shoestrings off and made leashes.''

These days, the question of who should be allowed to tell a story,
whether fictional or fact-based, seems to hang in the air around many a
work of literature. That Broom is a New Orleans native will
automatically put some readers at ease, those who think authority is
inextricably linked to biography; but that would be selling Broom's
craftsmanship short. The true test of her worthiness is her empathy and
focused attention. She is a responsible historian, granting her subjects
the grace of multiple examinations over the years. Her brother Darryl,
drug addicted and desperate for money, frightens her as a teenager in
the '90s to the point that she doesn't recall looking him directly in
the eye. Years later we meet him again, the sobered-up head of a
delightfully mundane Arizona household, his only daughter named after
his wary, observant youngest sister.

Image

One question --- ``How to resurrect a house with words?'' --- trembles
beneath the surface of every page, like the ripple of a stone dropped in
water.Credit...

The person who sustains the most considered attention is Broom's mother,
Ivory Mae, the twice-widowed steward of the crumbling yellow house
itself. ``My voice is not a distinguished voice,'' Ivory insists, but
her words and actions buoy ``The Yellow House,'' holding up to the light
those moments Broom was too young or unwilling to witness firsthand. ``I
was a little pathetic at first,'' Ivory Mae admits of her early widowed
years, ``I needed to make myself know things.'' She sets to this task
with fervor, going to night school for her G.E.D. and a nursing
credential so that she can fill the role of breadwinner suddenly thrust
upon her. If Broom's arc in this memoir is that of coming of age and
consciousness, Ivory Mae's is of doggedly persevering as her
circumstances shift.

Ivory Mae tries mightily to keep the house in good condition, sewing
curtains and valances to hide the disrepair, but the house is a
``belligerent unyielding child'': Rats and lizards find their way
inside, linoleum peels prematurely and areas under the sink grow slick
with mold. ``This house not all that comfortable for other people,''
becomes Ivory's standard rejoinder when the kids try to host sleepovers.
This seesawing between stubborn pride in the home she bought herself and
``slow creeping'' shame for the poverty that prevents her from improving
it makes the nature of its demolition, without her consent, one of the
book's central tragedies.

Ivory's second husband, Simon Broom, died when their youngest daughter
was 6 months old. ``My father is six pictures,'' she writes, photos she
takes with her as she travels from Texas to Berkeley to as far as
Burundi in an effort to understand where and how she fits in the world.
A riveting, heartbreaking scene toward the book's ending dramatizes
Broom's attempts to find additional physical evidence of him in city
archives: ``Part of me was afraid to see him alive. \ldots{} In the
world of dead parents, logic fails.''

Image

Broom's sister Lynette with their father, Simon, outside the yellow
house in 1976.Credit...via Sarah M. Broom

Broom's deadpan humor comes through clearest in her descriptions of
herself. On the now-vanished supermarket she visited as a child: ``one
of my favorite places to act a fool.'' On the tenuous position of
authority granted to her by her siblings' children, some of whom are
older than she is: ``I am these people's Auntie even though I am still
peeing in the bed.'' These moments, coupled with the singular,
unvarnished voices of her family members, coalesce to form a lesson on
how to keep a necessarily heavy book feeling limber.

``Calling places by what they originally were, especially when the
landscape is marred, is one way to fight erasure,'' Broom writes. There
are black and Latino neighborhoods from Detroit to Los Angeles where
refusing to call a place by a new name is the last line of defense, but
what happened in New Orleans East is more than the result of housing
segregation, white divestment or the hyper-capitalist, winner-takes-all
land grab that we call gentrification. ``The Yellow House'' is among
other things a climate-change narrative, the book suited to these last
days for taking action to prevent rising sea levels and other dire
consequences of unfettered carbon emission.

Broom's siblings, living in places like Vacaville, Calif., and Ozark,
Ala., with no paths to come home, are part of the Katrina diaspora, and
as extreme weather becomes the new normal, other diasporas have
followed. The phrase ``the Water'' is the one she uses to refer to
Hurricane Katrina and the subsequent displacement, loss of life and
livelihood. One can imagine a wider array of people soon adopting this
language --- ``the Water'' becoming a shorthand for all that is lost
when nature defies the plans we've made for where and how we live.

Any book as kinetic and omnivorous as ``The Yellow House'' is bound to
succeed more on some fronts than on others. It begins at the
chronological beginning, with Broom tracing her mother's lineage, which
means the first section of the book is more removed and reportorial.
This doesn't seem like a liability until around Page 100, when Broom's
own voice and perspective vault the language into another dimension. But
even this choice feels rooted in Broom's aesthetic intentions. ``The
Yellow House'' is a book that triumphs much as a jazz parade does: by
coming loose when necessary, its parts sashaying independently down the
street, but righting itself just in the nick of time, and teaching you a
new way of enjoying it in the process.

Advertisement

\protect\hyperlink{after-bottom}{Continue reading the main story}

\hypertarget{site-index}{%
\subsection{Site Index}\label{site-index}}

\hypertarget{site-information-navigation}{%
\subsection{Site Information
Navigation}\label{site-information-navigation}}

\begin{itemize}
\tightlist
\item
  \href{https://help.nytimes3xbfgragh.onion/hc/en-us/articles/115014792127-Copyright-notice}{©~2020~The
  New York Times Company}
\end{itemize}

\begin{itemize}
\tightlist
\item
  \href{https://www.nytco.com/}{NYTCo}
\item
  \href{https://help.nytimes3xbfgragh.onion/hc/en-us/articles/115015385887-Contact-Us}{Contact
  Us}
\item
  \href{https://www.nytco.com/careers/}{Work with us}
\item
  \href{https://nytmediakit.com/}{Advertise}
\item
  \href{http://www.tbrandstudio.com/}{T Brand Studio}
\item
  \href{https://www.nytimes3xbfgragh.onion/privacy/cookie-policy\#how-do-i-manage-trackers}{Your
  Ad Choices}
\item
  \href{https://www.nytimes3xbfgragh.onion/privacy}{Privacy}
\item
  \href{https://help.nytimes3xbfgragh.onion/hc/en-us/articles/115014893428-Terms-of-service}{Terms
  of Service}
\item
  \href{https://help.nytimes3xbfgragh.onion/hc/en-us/articles/115014893968-Terms-of-sale}{Terms
  of Sale}
\item
  \href{https://spiderbites.nytimes3xbfgragh.onion}{Site Map}
\item
  \href{https://help.nytimes3xbfgragh.onion/hc/en-us}{Help}
\item
  \href{https://www.nytimes3xbfgragh.onion/subscription?campaignId=37WXW}{Subscriptions}
\end{itemize}
