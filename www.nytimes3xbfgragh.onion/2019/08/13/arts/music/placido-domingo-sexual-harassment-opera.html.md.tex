Sections

SEARCH

\protect\hyperlink{site-content}{Skip to
content}\protect\hyperlink{site-index}{Skip to site index}

\href{https://www.nytimes3xbfgragh.onion/section/arts/music}{Music}

\href{https://myaccount.nytimes3xbfgragh.onion/auth/login?response_type=cookie\&client_id=vi}{}

\href{https://www.nytimes3xbfgragh.onion/section/todayspaper}{Today's
Paper}

\href{/section/arts/music}{Music}\textbar{}Plácido Domingo, Opera Star,
to Be Investigated for Sexual Harassment

\url{https://nyti.ms/2H47n5U}

\begin{itemize}
\item
\item
\item
\item
\item
\item
\end{itemize}

Advertisement

\protect\hyperlink{after-top}{Continue reading the main story}

Supported by

\protect\hyperlink{after-sponsor}{Continue reading the main story}

\hypertarget{pluxe1cido-domingo-opera-star-to-be-investigated-for-sexual-harassment}{%
\section{Plácido Domingo, Opera Star, to Be Investigated for Sexual
Harassment}\label{pluxe1cido-domingo-opera-star-to-be-investigated-for-sexual-harassment}}

The Los Angeles Opera, which Mr. Domingo helped found, will launch an
inquiry into allegations reported by The Associated Press.

\includegraphics{https://static01.graylady3jvrrxbe.onion/images/2019/08/13/arts/13domingo1/merlin_159218454_87fb9682-eb8a-4f7c-814a-14b2e4ee61d8-articleLarge.jpg?quality=75\&auto=webp\&disable=upscale}

\href{https://www.nytimes3xbfgragh.onion/by/michael-cooper}{\includegraphics{https://static01.graylady3jvrrxbe.onion/images/2018/02/16/multimedia/author-michael-cooper/author-michael-cooper-thumbLarge.jpg}}\href{https://www.nytimes3xbfgragh.onion/by/alex-marshall}{\includegraphics{https://static01.graylady3jvrrxbe.onion/images/2018/09/10/multimedia/author-alex-marshall/author-alex-marshall-thumbLarge.png}}

By \href{https://www.nytimes3xbfgragh.onion/by/michael-cooper}{Michael
Cooper} and
\href{https://www.nytimes3xbfgragh.onion/by/alex-marshall}{Alex
Marshall}

\begin{itemize}
\item
  Aug. 13, 2019
\item
  \begin{itemize}
  \item
  \item
  \item
  \item
  \item
  \item
  \end{itemize}
\end{itemize}

\href{https://www.nytimes3xbfgragh.onion/es/2019/08/13/espanol/cultura/placido-domingo-acusaciones.html}{Leer
en español}

The opera star
\href{https://www.nytimes3xbfgragh.onion/2020/02/25/arts/music/placido-domingo-sexual-misconduct.html}{Plácido
Domingo} was placed under investigation on Tuesday by the Los Angeles
Opera, which he helped found and has led since 2003, after The
Associated Press reported that multiple women had accused him of sexual
harassment over the years.

The allegations shook the opera world --- where Mr. Domingo remains, at
78, an enormous force. The Philadelphia Orchestra withdrew its
invitation for Mr. Domingo to sing at its opening night gala next month,
and the San Francisco Opera canceled a concert with him in October. The
Metropolitan Opera, where Mr. Domingo is scheduled to star in Verdi's
``Macbeth'' next month opposite the soprano Anna Netrebko, said it would
await the results of the Los Angeles Opera's investigation ``before
making any final decisions about Mr. Domingo's ultimate future at the
Met.''

The Associated
Press\href{https://www.apnews.com/c2d51d690d004992b8cfba3bad827ae9}{reported}
the allegations of multiple women who said that Mr. Domingo had
pressured them into sexual relationships in a series of encounters
beginning in the late 1980s --- including seven women who said that they
felt their careers had been harmed after they rebuffed him. Mr. Domingo
said in a statement that he believed ``all of my interactions and
relationships were always welcomed and consensual.''

Mr. Domingo, who has been married for more than 50 years, said in the
statement that ``the allegations from these unnamed individuals dating
back as many as 30 years are deeply troubling, and as presented,
inaccurate,'' but added that ``it is painful to hear that I may have
upset anyone or made them feel uncomfortable --- no matter how long ago
and despite my best intentions.''

He said that ``the rules and standards by which we are --- and should be
--- measured against today are very different than they were in the
past'' and pledged to hold himself ``to the highest standards.''

Mr. Domingo occupies a unique position in the opera world. After
shooting to fame as a star tenor --- then reaching a far broader global
audience as one of the Three Tenors, alongside Luciano Pavarotti and
José Carreras --- he also became a conductor; founded the prestigious
young artist competition Operalia; and began adding managerial
positions, becoming the general director of Washington National Opera
and then the Los Angeles Opera, a position he still holds. He also
continues to have a prolific singing career in baritone roles.

These activities have made him one of the most influential figures in
opera.

The Los Angeles Opera said in a statement that it would engage an
outside counsel to investigate what it called ``the concerning
allegations'' against Mr. Domingo.

``Plácido Domingo has been a dynamic creative force in the life of L.A.
Opera and the artistic culture of Los Angeles for more than three
decades,'' the opera company said in a statement. ``Nevertheless, we are
committed to doing everything we can to foster a professional and
collaborative environment where all our employees and artists feel
equally comfortable, valued and respected.''

The A.P. spoke with eight singers and one dancer --- all but one of whom
were quoted anonymously --- who said that Mr. Domingo had used his power
to pursue them sexually. They described him as calling them repeatedly
and making dates, often under the guise of offering professional advice.
One accuser told the news agency that Mr. Domingo had stuck his hand
down her skirt, and three others said that he had forced wet kisses on
their lips, in a dressing room, a hotel room and at a lunch meeting.

One singer told the news agency that Mr. Domingo had repeatedly
propositioned her at the Los Angeles Opera in 1998, the year the company
announced that he would become its artistic director, eventually taking
her to his apartment for what she described as ``heavy petting'' and
``groping.''

``I was totally intimidated and felt like saying no to him would be
saying no to God,'' she was quoted as saying. ``How do you say no to
God?''

She told the news agency she eventually told him to stop calling her,
adding that she was never hired again in Los Angeles after Mr. Domingo
got power over casting.

Patricia Wulf, the only accuser named in the A.P. article, told The New
York Times in a telephone interview that her experiences with Mr.
Domingo also dated to 1998, when she appeared with Washington National
Opera, where he was then general director, in ``The Magic Flute.''

When Ms. Wulf left the stage during one rehearsal, she said, Mr. Domingo
was waiting for her in the wings. He came up to her, she recalled, and
said, ``Patricia, do you have to go home tonight?''

Ms. Wulf said she laughed, thinking it was a joke, but later the
situation repeated itself. ``I remember thinking, `That is not
acceptable,''' she said.

Mr. Domingo did not inappropriately touch her, Ms. Wulf said, but he
made her so uncomfortable that she hid in her dressing room and asked a
colleague to walk her to her car so she did not have to risk
encountering him. She said she did not report Mr. Domingo's behavior
because she felt that doing so would damage her career.

She said she kept saying no to Mr. Domingo until he stopped, about two
and a half years later.

In his statement, which was initially made to The A.P. and later sent to
The Times, Mr. Domingo said, ``People who know me or who have worked
with me know that I am not someone who would intentionally harm, offend,
or embarrass anyone.''

At the Salzburg Festival in Austria, where Mr. Domingo is scheduled to
sing in concert performances of Verdi's ``Luisa Miller'' this month,
Helga Rabl-Stadler, the festival's president, said that he would perform
as planned.

``I have known Plácido Domingo for more than 25 years,'' Ms.
Rabl-Stadler said in a statement. ``In addition to his artistic
competence, I was impressed from the very beginning by his appreciative
treatment of all festival employees. He knows every name, from the
concierge to the secretary; he never fails to thank anyone performing
even the smallest service for him. Had the accusations against him been
voiced inside the Festspielhaus in Salzburg, I am sure I would have
heard of it.''

Joseph Volpe, the general manager of the Met from 1990 to 2006, also
said that he had never heard any allegations against Mr. Domingo. ``I've
known Plácido since he made his debut at the Met in 1968, and there was
never, ever a complaint made against him about sexual harassment,'' Mr.
Volpe said in a telephone interview. ``He's such a gentleman, and so
caring about people.''

The Met, which fired its former music director, James Levine, in 2018
amid allegations of sexual misconduct --- and which just last week
\href{https://www.nytimes3xbfgragh.onion/2019/08/06/arts/music/james-levine-metropolitan-opera.html}{quietly
settled a rancorous lawsuit} he had filed against the company --- said
in a statement that it takes ``accusations of sexual harassment and
abuse of power with extreme seriousness.'' The statement noted that Mr.
Domingo, as a guest artist, had never been in a position to influence
casting decisions there.

Washington National Opera merged with the Kennedy Center in 2011, the
year Mr. Domingo departed. The company's current general director,
Timothy O'Leary, and Deborah F. Rutter, the president of the Kennedy
Center, said in a joint statement that the allegations against Mr.
Domingo had all predated the center's
\href{http://www.washingtonpost.com/wp-dyn/content/article/2011/01/20/AR2011012001924.html}{takeover
of the company.}

``The Kennedy Center did not receive any documented complaints about Mr.
Domingo's behavior prior to W.N.O.'s affiliation with the Kennedy
Center, and we have not received any since then,'' the statement said.

A spokesman for the Royal Opera House in London did not respond to
questions about whether Mr. Domingo would perform there next year, as
scheduled. The spokesman said in a statement that the company ``has not
been made aware of any accusations pertaining to Plácido Domingo's time
as a visiting artist or conductor. However, we have a zero-tolerance
policy toward harassment of any kind.''

The San Francisco Opera said in a statement that it ``places a great
priority on creating a safe and secure environment where everyone can
focus on their work and art, and in which colleagues are treated with
respect, dignity and collegiality.''

The company said that ticket holders would be able to exchange their
tickets to the canceled Domingo concert for other performances this
season, or request refunds.

Advertisement

\protect\hyperlink{after-bottom}{Continue reading the main story}

\hypertarget{site-index}{%
\subsection{Site Index}\label{site-index}}

\hypertarget{site-information-navigation}{%
\subsection{Site Information
Navigation}\label{site-information-navigation}}

\begin{itemize}
\tightlist
\item
  \href{https://help.nytimes3xbfgragh.onion/hc/en-us/articles/115014792127-Copyright-notice}{©~2020~The
  New York Times Company}
\end{itemize}

\begin{itemize}
\tightlist
\item
  \href{https://www.nytco.com/}{NYTCo}
\item
  \href{https://help.nytimes3xbfgragh.onion/hc/en-us/articles/115015385887-Contact-Us}{Contact
  Us}
\item
  \href{https://www.nytco.com/careers/}{Work with us}
\item
  \href{https://nytmediakit.com/}{Advertise}
\item
  \href{http://www.tbrandstudio.com/}{T Brand Studio}
\item
  \href{https://www.nytimes3xbfgragh.onion/privacy/cookie-policy\#how-do-i-manage-trackers}{Your
  Ad Choices}
\item
  \href{https://www.nytimes3xbfgragh.onion/privacy}{Privacy}
\item
  \href{https://help.nytimes3xbfgragh.onion/hc/en-us/articles/115014893428-Terms-of-service}{Terms
  of Service}
\item
  \href{https://help.nytimes3xbfgragh.onion/hc/en-us/articles/115014893968-Terms-of-sale}{Terms
  of Sale}
\item
  \href{https://spiderbites.nytimes3xbfgragh.onion}{Site Map}
\item
  \href{https://help.nytimes3xbfgragh.onion/hc/en-us}{Help}
\item
  \href{https://www.nytimes3xbfgragh.onion/subscription?campaignId=37WXW}{Subscriptions}
\end{itemize}
