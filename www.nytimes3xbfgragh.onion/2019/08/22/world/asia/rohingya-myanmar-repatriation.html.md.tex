Sections

SEARCH

\protect\hyperlink{site-content}{Skip to
content}\protect\hyperlink{site-index}{Skip to site index}

\href{https://www.nytimes3xbfgragh.onion/section/world/asia}{Asia
Pacific}

\href{https://myaccount.nytimes3xbfgragh.onion/auth/login?response_type=cookie\&client_id=vi}{}

\href{https://www.nytimes3xbfgragh.onion/section/todayspaper}{Today's
Paper}

\href{/section/world/asia}{Asia Pacific}\textbar{}Massacred at Home, in
Misery Abroad, 730,000 Rohingya Are Mired in Hopelessness

\url{https://nyti.ms/2P9Q4Hc}

\begin{itemize}
\item
\item
\item
\item
\item
\item
\end{itemize}

Advertisement

\protect\hyperlink{after-top}{Continue reading the main story}

Supported by

\protect\hyperlink{after-sponsor}{Continue reading the main story}

Promises Made

\hypertarget{massacred-at-home-in-misery-abroad-730000-rohingya-are-mired-in-hopelessness}{%
\section{Massacred at Home, in Misery Abroad, 730,000 Rohingya Are Mired
in
Hopelessness}\label{massacred-at-home-in-misery-abroad-730000-rohingya-are-mired-in-hopelessness}}

\includegraphics{https://static01.graylady3jvrrxbe.onion/images/2019/08/23/world/23rohingya-promises-p1/merlin_159548847_9b725503-6ea7-4382-bd15-4beab75a35e5-articleLarge.jpg?quality=75\&auto=webp\&disable=upscale}

By \href{https://www.nytimes3xbfgragh.onion/by/hannah-beech}{Hannah
Beech}

\begin{itemize}
\item
  Published Aug. 22, 2019Updated Jan. 23, 2020
\item
  \begin{itemize}
  \item
  \item
  \item
  \item
  \item
  \item
  \end{itemize}
\end{itemize}

\href{https://www.nytimes3xbfgragh.onion/es/2019/08/26/espanol/mundo/rohinya-exodo-banglades.html}{Leer
en español}

\href{https://www.nytimes3xbfgragh.onion/2019/08/25/world/asia/bengali-rohingya-promises-made.html}{বাংলায়
পড়তে ক্লিক করুন}

\emph{When things go wrong, those in power often promise to make it
right. But do they? In}
\emph{\href{https://www.nytimes3xbfgragh.onion/spotlight/promises-made?module=inline}{this
series}, The Times investigates to see if those promises were kept.}

NGA KHU YA, Myanmar --- Rusting behind barbed wire, rows of trailers at
a repatriation center sit empty and uninviting, evocative of a prison
awaiting its inmates.

In a deserted arrivals trailer, uniformed officers loiter at their
desks, expectant grins on their faces. Signs explain the steps involved
in welcoming
\href{https://www.nytimes3xbfgragh.onion/2020/01/23/world/asia/myanmar-rohingya-genocide.html}{Rohingya
Muslims} back to
\href{https://www.nytimes3xbfgragh.onion/2020/01/23/world/asia/myanmar-rohingya-genocide.html}{Myanmar}:
Stand here for photographs, go there for identity cards.

Men stand guard with security wands, as if this were an international
airport rather than an inhospitable holding pen in a desolate frontier.

What is so obviously missing at the Nga Khu Ya repatriation center are
the Rohingya themselves.

Ever since more than 730,000 Rohingya started fleeing to Bangladesh, two
years ago this Sunday, to escape a
\href{https://www.nytimes3xbfgragh.onion/2017/09/02/world/asia/rohingya-myanmar-bangladesh-refugees-massacre.html?action=click\&module=RelatedCoverage\&pgtype=Article\&region=Footer}{vicious
campaign of ethnic cleansing}, governments from both countries have
repeatedly vowed that a return of the Muslim minority to
\href{https://www.nytimes3xbfgragh.onion/2020/01/23/world/asia/myanmar-rohingya-genocide.html}{Myanmar}
was imminent.

But that promise has been broken, time and again.

The
\href{https://www.nytimes3xbfgragh.onion/2020/01/23/world/asia/myanmar-rohingya-genocide.html}{Rohingya}
have not returned by the hundreds of thousands, or even by the
thousands.

In fact, they have hardly come back at all.

After all the assurances that it was safe for them to return to Myanmar,
only a few dozen have done so.

\includegraphics{https://static01.graylady3jvrrxbe.onion/images/2019/08/22/world/00rohingya-promises-2new/merlin_159552150_fe1307b7-beea-40df-8d25-b04d9aa06f53-articleLarge.jpg?quality=75\&auto=webp\&disable=upscale}

The first batch of about 1,200 returnees was supposed to be sent home in
January 2018. That plan was delayed by the Bangladeshi government, after
an international outcry over the idea of returning traumatized victims
to the epicenter of one of the worst eruptions of ethnic cleansing in
this century.

After the two countries promised in April 2018 to proceed with safe,
voluntary and dignified repatriations, several new deadlines were set.
None were met.

Most recently, the Myanmar government said the repatriation of 3,450
Rohingya would begin on Thursday. That target, too, passed with no
movement across the border.

Maintaining the fiction that repatriations are about to occur is
politically useful for both sides.

Myanmar, which
\href{https://www.nytimes3xbfgragh.onion/2018/08/27/world/asia/myanmar-rohingya-genocide.html}{United
Nations officials say should be tried on genocide charges}over the
orchestrated killings that began on Aug. 25, 2017, is keen to prove it
is not a human rights pariah.

Bangladesh, struggling with overpopulation and poverty, wants to
reassure its citizens that scarce funds are not being diverted to
refugees.

But the charade at Nga Khu Ya, with its corroded buildings devoid of any
Rohingya presence, proves the lie in the repatriation commitment. The
place is so quiet that a dog snoozes at the main entrance, undisturbed.

Even the repatriation center's watchtowers are empty of soldiers. There
is no one to watch.

Image

The Rohingya who escaped to Bangladesh now live in squalid conditions in
the world's largest refugee camp.Credit...Adam Dean for The New York
Times

\textbf{What We Found}

\hypertarget{promise-repatriation-fail-to-deliver-repeat}{%
\subsection{Promise Repatriation. Fail to Deliver.
Repeat.}\label{promise-repatriation-fail-to-deliver-repeat}}

The lack of returnees on Thursday followed the same tragicomic script as
previous efforts to get the Rohingya home.

First, Myanmar unilaterally announced a date for repatriation, but
approved the return of only a tiny fraction of those eligible.

Bangladesh, the Muslim-majority nation where most of the Rohingya have
sought refuge, then said it supported the idea.

``I'm very positive,'' Foreign Affairs Minister A.K. Abdul Momen told
reporters in early August. ``I'm expecting that we can start this
month.''

But the Rohingya --- hundreds of thousands of whom are squeezed into
overflowing camps in Bangladesh --- balked, having received scant
consultation about their own futures. Not a single Rohingya boarded the
five buses and two trucks that were prepared on Thursday to transfer
them over the border to Myanmar.

International human rights groups stepped in to urge caution about
returning anyone, having interviewed Rohingya who were terrified, not
joyful, to learn that they were on the repatriation list.

On Thursday, Radhika Coomaraswamy, an expert with the United Nations
fact-finding mission on the Myanmar violence, said conditions were not
conducive for the return of Rohingya.

``We have been shown satellite imagery which shows the situation in
northern Rakhine, which is basically where all the villages have been
bulldozed, not a tree standing,'' she said at a news conference at the
United Nations headquarters in New York.

That left the Myanmar side with the perfect opportunity to declare
itself surprised that the Rohingya weren't coming back.

``I have no idea why repatriation has not happened yet,'' said U Win
Myint, a spokesman for the government in Rakhine State, which Myanmar's
Rohingya once called home. ``Everything is ready on our side.''

This scenario has played out before, with similarly hollow outcomes.

In November, Win Myat Aye, Myanmar's minister of social welfare, relief
and resettlement, told The New York Times that a round of repatriation
would begin in a couple days' time. Over a 15-day period, 2,165 people
would be processed through Nga Khu Ya repatriation camp, he promised.
Then, soon after, another 5,000, and so on.

``They can apply for citizenship,'' Mr. Win Myat Aye said. ``They can
live in the place where they're originally from. If there is no housing
there, they can live near where they're from.''

The government's own facts indicate this is a fantasy.

According to the Myanmar immigration authorities' figures, from May 2018
to May 2019, only 185 Rohingya were repatriated from Bangladesh. Even
that tiny number is inflated. Of those 185 people, 92 had been caught by
the authorities in Myanmar while trying to escape the country by boat.
Sixty-two others had just been released from jails in Myanmar.

Only 31 Rohingya --- of the nearly three-quarters of a million who left
Myanmar --- had returned ``of their own volition,'' according to the
government.

When pressed to account for such minuscule numbers, the Myanmar
authorities accuse Rohingya militants and Muslim charities operating in
the refugee camps in Bangladesh of dissuading people from going back.

``Muslim terrorists in the camps say that it is not safe to return, so
people don't dare,'' said U Soe Aung, the head of the General
Administration Department in Maungdaw, a township in Rakhine that was
once overwhelmingly Rohingya. ``Even though it's totally safe.''

Image

Rohingya refugees from Myanmar arriving at Dakhinpara, Bangladesh, in
September 2017.Credit...Adam Dean for The New York Times

\textbf{What We Found}

\hypertarget{home-is-missed-but-deeply-feared}{%
\subsection{Home Is Missed, but Deeply
Feared}\label{home-is-missed-but-deeply-feared}}

Assurances that Myanmar has laid out the welcome mat have come from none
other than Daw Aung San Suu Kyi, the de facto head of the civilian
government and a Nobel Peace Prize laureate.

``The state counselor already decided to receive back the people who
lived in Myanmar and left the country for some reason,'' said her social
welfare minister, Mr. Win Myat Aye, referring to Ms. Aung San Suu Kyi by
her formal title. ``There is no reason not to come back.''

But the Rohingya's dread about what might await them is understandable,
considering what
\href{https://www.nytimes3xbfgragh.onion/2017/09/02/world/asia/rohingya-myanmar-bangladesh-refugees-massacre.html}{drove
their flight} in the first place --- and what has happened, and not
happened, in Myanmar since the exodus.

After a band of Rohingya insurgents
\href{https://www.nytimes3xbfgragh.onion/2017/09/17/world/asia/myanmar-rohingya-militants.html?module=inline}{attacked
police posts} and an army encampment on Aug. 25, 2017, a
\href{https://www.nytimes3xbfgragh.onion/2017/10/11/world/asia/rohingya-myanmar-atrocities.html}{burst
of brutality} against the Muslim minority followed within hours: mass
executions, rape and the burning of hundreds of villages by security
forces. Buddhist mobs participated in the bloodletting.

Doctors Without Borders says that at least 6,700 Rohingya met violent
deaths in the month after the killings began.

While the Myanmar government defended its actions as ``clearance
operations'' targeting only militants, the large buildup of troops in
the weeks before the attack --- and the military helicopters that rained
down rockets on villagers in the days afterward --- suggest a highly
coordinated, long-planned campaign of ethnic cleansing that had been
waiting for the right catalyzing event.

The Rohingya who escaped to Bangladesh now live in
\href{https://www.nytimes3xbfgragh.onion/2017/09/29/world/asia/rohingya-refugees-myanmar-bangladesh.html}{a
teeming, squalid settlement} ---
\href{https://www.nytimes3xbfgragh.onion/2018/03/14/climate/bangladesh-rohingya-refugee-camp.html}{the
world's largest refugee encampment}.

Human trafficking is rife, with girls destined for brothels and men for
indentured servitude in Southeast Asia. When the monsoons descend on the
camps, sewage and mud mix into a disease-breeding brew. Landslides are
common, and Rohingya have even been killed by
\href{https://www.nytimes3xbfgragh.onion/2018/02/22/world/asia/elephant-tramples-rohingya-bangladesh.html}{rampaging
elephants}. There is little, if any, incentive to stay.

But despite these intolerable conditions, Myanmar looks worse to many
refugees, who are bewildered at the idea that they should return to a
country whose government has refused to admit that atrocities were
committed.

``How can we believe those who killed our nearest and dearest?'' said
Ramjan Ali, the sole survivor of a family that was massacred in the
village of Tula Toli.

Those Rohingya who stayed in northern Rakhine State after the killing
began are marooned in communities cut off from jobs, education and basic
services. Since June,
\href{https://www.nytimes3xbfgragh.onion/2019/07/02/world/asia/internet-shutdown-myanmar-rakhine.html}{the
region's mobile internet connection has been severed}.

Incarceration rates among Rohingya men are high, with many accused of
terrorist activity. Those released from jail are sometimes
\href{https://www.nytimes3xbfgragh.onion/2018/08/02/world/asia/myanmar-rohingya-rakhine.html}{paraded
as repatriated Rohingya}, even if they have never left Myanmar.

``I miss my home a lot,'' said Saiful Islam, a Rohingya leader in the
camps in Bangladesh. ``But I don't want to go back to a place where my
family could be killed.''

Image

A Rohingya Muslim reads from the Quran in one of the few undamaged
mosques in northern Rakhine State, in Ngan Chaung Village, Maungdaw
Township, in May.Credit...Adam Dean for The New York Times

\textbf{What We Found}

\hypertarget{building-military-bases-on-villages-ashes}{%
\subsection{Building Military Bases on Villages'
Ashes}\label{building-military-bases-on-villages-ashes}}

Any Rohingya who did return to Myanmar would find a transformed
landscape.

Drive across the salty marsh of
\href{https://www.nytimes3xbfgragh.onion/2018/08/02/world/asia/myanmar-rohingya-rakhine.html}{northern
Rakhine}, and the silence is overwhelming. About a million Rohingya once
lived in this area. Now most are gone, the occasional carcass of a
burned mosque or stand of charred palms the only evidence that they
existed.

The government has funneled money into infrastructure development in
Rakhine: new power stations, government buildings and, most of all,
military and border guard bases.

But many of those new facilities have been built on land emptied by
ethnic cleansing.

Analysis of satellite imagery by the International Cyber Policy Center
at the Australian Strategic Policy Institute found that nearly 60
Rohingya settlements were razed last year, well after the violence
peaked in 2017. Destruction of Rohingya villages continued into this
year, the study found.

Officials in Myanmar have never been clear about where, exactly,
returnees would live --- even as they showed off rows of prefabricated
houses supposedly built for repatriated families.

In a troubling precedent, about 120,000 Rohingya from central Rakhine
State who were targeted in a 2012 conflict have been confined to
internment camps for the past seven years. Their businesses have been
taken over by members of Myanmar's Buddhist majority, and most of their
homes have been destroyed.

As construction transforms Rakhine, bringing Buddhist pagodas to areas
where the Islamic call to prayer once resounded, the beneficiaries of
the building boom are companies run by cronies of the military, which
still dominates the government.

On Aug. 5, a United Nations fact-finding mission released a
\href{https://www.ohchr.org/Documents/HRBodies/HRCouncil/FFM-Myanmar/EconomicInterestsMyanmarMilitary/A_HRC_42_CRP_3.pdf}{report}
recommending targeted sanctions against these military-linked firms,
which it said had helped in ``re-engineering the region in a way that
erases evidence of Rohingya belonging to Myanmar.''

Image

Myanmar police officers shelter in the shade of a tree while Rakhine
villagers farm in the razed Rohingya quarter of Inn Din village,
Maungdaw Township, in May.Credit...Adam Dean for The New York Times

\textbf{What We Found}

\hypertarget{see-no-evil}{%
\subsection{See No Evil}\label{see-no-evil}}

The United Nations says no refugees should have to return to a place
where their safety and security is not assured. Doing so is called
refoulement, and it's against international law.

But Myanmar has done little to reassure the Rohingya that the conditions
that led to the mass killings have changed.

The country has steadfastly refused to admit that its security forces,
which engaged in widespread sexual violence and sprayed fleeing children
with gunfire, according to Rohingya testimony and investigations by
human rights groups, did anything wrong.

``Not a single innocent Muslim was killed,'' said Mr. Soe Aung, the
Maungdaw Township official.

Ms. Aung San Suu Kyi has declined to hold the military responsible for
the violence, even as United Nations-appointed investigators recommended
last year that commanders be
\href{https://news.un.org/en/story/2018/08/1017802}{investigated for
crimes against humanity}.

Despite the fact that Myanmar clearly is their home, most Rohingya are
officially considered illegal immigrants from Bangladesh.

And before any are accepted for repatriation, they must often come up
with evidence proving that they came from Myanmar. That's a tall order
for refugees who fled burning homes.

More controversially, those who wish to return must accept identity
cards that critics say will make their statelessness official.

Myanmar's government
\href{https://www.nytimes3xbfgragh.onion/2017/10/24/world/asia/myanmar-rohingya-ethnic-cleansing.html}{does
not even accept the name ``Rohingya.''} Instead, those who return are
issued documents that identify them as Bengali, implying they are
foreign interlopers from Bangladesh, not an ethnic group from Rakhine.

``We are Rohingya,'' whispered Abdul Kadir, an imam from a northern
Rakhine village who has been unable to flee, in broken English. ``No say
Rohingya in Myanmar. No say.''

```Rohingya' is not real,'' said Kyaw Kyaw Khine, the deputy head of
immigration at Nga Khu Ya repatriation camp. ``Why do foreigners use
this word?''

Image

A child in the Kutupalong Rohingya refugee camp in June.Credit...Adam
Dean for The New York Times

\textbf{What We Found}

\hypertarget{on-the-other-side-of-the-border}{%
\subsection{On the Other Side of the
Border}\label{on-the-other-side-of-the-border}}

The official narrative in Myanmar goes like this: The Rohingya burned
down their own homes to garner international sympathy, and to feast on
plentiful aid rations in Bangladesh provided by Muslim nations.

Myanmar officials also accuse Bangladeshi officials of dawdling, and
wonder if they're reluctant to let the Rohingya leave.

``Maybe they want people to stay there,'' said U Kyaw Sein, an
administrator at the Nga Khu Ya camp.

The truth couldn't be more different.

Bangladeshis have displayed tremendous hospitality to the Rohingya, who
poured over the border in the fastest inflow of refugees in a
generation. But the country's patience has worn thin.

The Bangladeshi authorities keep threatening to resettle the Rohingya to
an island that is little more than a cyclone-prone sandbar in the middle
of the Bay of Bengal.

Bangladesh does not consider the vast majority of Rohingya to be
refugees, lest that designation cement their right to live in exile
forever.

As a consequence, they have no legal right to study or work outside of
the camps. Muslim extremists stalk camp mosques, promising salvation
through militancy.

Hopelessness is the only plentiful commodity.

``Will my children live the rest of their lives here?'' asked Mr. Islam,
the Rohingya camp leader. ``Is this the only life I can give them?''

\textbf{The Takeaway}: No one wants the Rohingya, least of all their
homeland.

Image

The Nga Khu Ya repatriation center for Rohingya returning from
Bangladesh in northern Rakhine in July 2018.Credit...Adam Dean for The
New York Times

Advertisement

\protect\hyperlink{after-bottom}{Continue reading the main story}

\hypertarget{site-index}{%
\subsection{Site Index}\label{site-index}}

\hypertarget{site-information-navigation}{%
\subsection{Site Information
Navigation}\label{site-information-navigation}}

\begin{itemize}
\tightlist
\item
  \href{https://help.nytimes3xbfgragh.onion/hc/en-us/articles/115014792127-Copyright-notice}{©~2020~The
  New York Times Company}
\end{itemize}

\begin{itemize}
\tightlist
\item
  \href{https://www.nytco.com/}{NYTCo}
\item
  \href{https://help.nytimes3xbfgragh.onion/hc/en-us/articles/115015385887-Contact-Us}{Contact
  Us}
\item
  \href{https://www.nytco.com/careers/}{Work with us}
\item
  \href{https://nytmediakit.com/}{Advertise}
\item
  \href{http://www.tbrandstudio.com/}{T Brand Studio}
\item
  \href{https://www.nytimes3xbfgragh.onion/privacy/cookie-policy\#how-do-i-manage-trackers}{Your
  Ad Choices}
\item
  \href{https://www.nytimes3xbfgragh.onion/privacy}{Privacy}
\item
  \href{https://help.nytimes3xbfgragh.onion/hc/en-us/articles/115014893428-Terms-of-service}{Terms
  of Service}
\item
  \href{https://help.nytimes3xbfgragh.onion/hc/en-us/articles/115014893968-Terms-of-sale}{Terms
  of Sale}
\item
  \href{https://spiderbites.nytimes3xbfgragh.onion}{Site Map}
\item
  \href{https://help.nytimes3xbfgragh.onion/hc/en-us}{Help}
\item
  \href{https://www.nytimes3xbfgragh.onion/subscription?campaignId=37WXW}{Subscriptions}
\end{itemize}
