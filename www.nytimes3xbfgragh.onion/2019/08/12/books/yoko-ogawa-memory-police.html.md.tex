Sections

SEARCH

\protect\hyperlink{site-content}{Skip to
content}\protect\hyperlink{site-index}{Skip to site index}

\href{https://www.nytimes3xbfgragh.onion/section/books}{Books}

\href{https://myaccount.nytimes3xbfgragh.onion/auth/login?response_type=cookie\&client_id=vi}{}

\href{https://www.nytimes3xbfgragh.onion/section/todayspaper}{Today's
Paper}

\href{/section/books}{Books}\textbar{}Yoko Ogawa Conjures Spirits in
Hiding: `I Just Peeked Into Their World and Took Notes'

\url{https://nyti.ms/2GZDUd9}

\begin{itemize}
\item
\item
\item
\item
\item
\end{itemize}

Advertisement

\protect\hyperlink{after-top}{Continue reading the main story}

Supported by

\protect\hyperlink{after-sponsor}{Continue reading the main story}

Profile

\hypertarget{yoko-ogawa-conjures-spirits-in-hiding-i-just-peeked-into-their-world-and-took-notes}{%
\section{Yoko Ogawa Conjures Spirits in Hiding: `I Just Peeked Into
Their World and Took
Notes'}\label{yoko-ogawa-conjures-spirits-in-hiding-i-just-peeked-into-their-world-and-took-notes}}

\includegraphics{https://static01.graylady3jvrrxbe.onion/images/2019/08/13/books/review/13Ogawa1/13Ogawa1-articleLarge.jpg?quality=75\&auto=webp\&disable=upscale}

By \href{https://www.nytimes3xbfgragh.onion/by/motoko-rich}{Motoko Rich}

\begin{itemize}
\item
  Aug. 12, 2019
\item
  \begin{itemize}
  \item
  \item
  \item
  \item
  \item
  \end{itemize}
\end{itemize}

ASHIYA, Japan --- When Yoko Ogawa discovered ``The Diary of Anne Frank''
as a lonely teenager in Japan, she was so taken by it that she began to
keep a diary of her own, writing to Anne as if she were a cherished
friend.

To conjure the kind of physical captivity that Anne experienced, Ogawa
would crawl, notebook in hand, into a drawer or under a table draped
with a quilt.

``Anne's heart and mind were so rich,'' said Ogawa, now 57 and the
author of more than 40 novels and story collections. ``Her diary proved
that people can grow even in such a confined situation. And writing
could give people freedom.''

Decades later, Ogawa transmuted her imagining of Anne's world into ``The
Memory Police,'' a dystopian novel that is Ogawa's fifth book to be
translated into English and which goes on sale in the United States this
week. It takes place on a mysterious island where an authoritarian
government makes whole categories of objects or animals disappear
overnight, wiping them from the memories of citizens.

Those who retain their recall are outlaws who go into hiding. The
narrator, a novelist, shelters her editor --- who remembers everything
--- in a room reminiscent of the Dutch annex where Anne hid with her
family.

``I wanted to digest Anne's experience in my own way and then recompose
it into my work,'' said Ogawa during an interview in her home, located
in a suburb between Kobe and Osaka.

\emph{{[} ``The Memory Police'' was one of our most anticipated titles
of August.}
\emph{\href{https://www.nytimes3xbfgragh.onion/2019/07/31/books/new-august-books.html?module=inline}{See
the full list}. {]}}

Although ``The Memory Police'' was first released in Japan in 1994, the
novel is particularly resonant now, at a time of rising authoritarianism
across the globe. Throughout the book, citizens live under police
surveillance. Novels are burned. People are detained and interrogated
without explanation. Neighbors are taken away in the middle of the
night.

All the while, the citizens, cowed by fear, do nothing to stop the
disappearances. ``Regardless of what had happened, it was almost
certainly an unfortunate event,'' the narrator explains, ``and,
moreover, simply talking about it could put you in danger.''

\href{https://www.nytimes3xbfgragh.onion/2007/10/15/arts/15fair.html?ref=arts}{Anna
Stein}, Ogawa's English-language agent, said she and
\href{http://www.middlebury.edu/ls/contact/node/52591}{Stephen Snyder},
the writer's longtime English-language translator, selected ``The Memory
Police'' for translation in 2014. ``Of course the horror of it applies
now more than ever in our lifetime,'' said Stein.

\includegraphics{https://static01.graylady3jvrrxbe.onion/images/2019/08/16/books/review/13Ogawa2/13Ogawa2-articleLarge-v3.jpg?quality=75\&auto=webp\&disable=upscale}

In Japan, where history itself has been subject to revision --- and
those who bring up the country's wartime past can be denounced
\href{https://www.nytimes3xbfgragh.onion/2019/08/05/world/asia/japan-aichi-trienniale.html}{or
even censored} --- the novel's lament for erased memories could be read
as veiled criticism. Yet Ogawa did not intend to write a political
allegory, she said. ``I am just trying to depict each individual
character and how those characters are living in their current time.''

Lexy Bloom, a senior editor at Knopf Doubleday Publishing Group, said
she is drawn to Ogawa's skill with character and detail. ``In `The
Memory Police,' she is tackling these big themes, but it's also about
these small moments between people,'' said Bloom, who has long been a
fan of Ogawa's work, which was previously published in the United States
by Picador. ``And that's a hard thing to do effectively.''

\emph{{[} Read our}
\href{https://www.nytimes3xbfgragh.onion/2019/08/15/books/review/read-receipts-on-two-dystopian-novels-predict-the-surveillance-state.html}{\emph{review
of ``The Memory Police.''}} \emph{{]}}

None of the protagonists in ``The Memory Police'' are named. Few markers
identify the island, a trait shared by some of Ogawa's other novels. ``I
myself like to keep a certain distance from my native culture or
environment,'' said Ogawa, whose spacious two-story home in an affluent
neighborhood overlooking the sea has a Spanish tiled roof, wrought-iron
balconies and chairs upholstered in French floral tapestry.

Snyder, a professor of Japanese studies at Middlebury College, said
Ogawa's novels relate to Japanese culture in ``ancillary ways.'' Though
she raises socially relevant themes, he said, she is never doctrinaire.

``There is a naturalness to what she writes so it never feels forced,''
he said. ``Her narrative seems to be flowing from a source that's hard
to identify.''

Growing up, Ogawa wrote for herself. When she married a steel company
engineer, she quit her job as a medical university secretary --- a
common life step for many
\href{https://www.nytimes3xbfgragh.onion/2019/02/02/world/asia/japan-working-mothers.html}{women
of her generation}.

While her husband worked, she wrote. She didn't intentionally keep it
secret, she said, but her husband only learned about her writing when
her debut novel, ``The Breaking of the Butterfly,'' received a literary
prize.

``I wasn't telling anyone in a big voice, `I'm writing a novel,''' she
said. ``But I always thought, no matter how my life changes, I want to
have a life of writing. Whether I could make any money off it, I did not
know.''

Ogawa gave birth to a son, and when he was just a toddler, her novella
``Pregnancy Diary'' won the prestigious Akutagawa Prize for literature,
cementing her reputation in Japan.

She continued to write. ``I would change a diaper and then write a
sentence,'' she recalled. ``Then I'd make a meal and write a sentence.''

Image

``I see a rainbow that I have to climb over to move to the next scene,''
Yoko Ogawa said. ``That's how I write.''Credit...Chang W. Lee/The New
York Times

Sometimes, Ogawa mused about what it would be like to write without such
interruptions. ``But now that my son has grown, I feel like I was at my
happiest when I was writing while raising my child,'' she said. ``Now
that I can write as much as I want 24 hours a day, it's not as if I
produce any greater work now than I did in the past.''

She now writes at a craftsman desk in an airy room with a twin bed for
naps and a bookshelf that slides sideways to reveal another full shelf
behind it. A whole section is devoted to books about Anne Frank and the
Holocaust. She pulled out a copy of Anne's diary to show that she had
tagged virtually every page with a Post-it note.

On her desk, Ogawa keeps a beaver skull, an animal she admires for its
industry. It gives her inspiration, she said, echoing the sentiments of
the editor in ``The Memory Police,'' who says that touching objects that
have disappeared ``became a way of confirming that I was still whole.''

Ogawa achieved best seller status, and a film adaptation, with
``\href{https://www.nytimes3xbfgragh.onion/2009/03/01/books/review/Overbye-t.html}{The
Professor and the Housekeeper},'' another novel with memory as its
theme. Much lighter in tone than ``The Memory Police,'' it tells the
story of a single mother who takes a job as a cook and cleaner for a
mathematician who cannot remember anything new for more than 80 minutes.

Along with memory, another of Ogawa's preoccupations is the human
capacity for cruelty.
``\href{https://www.nytimes3xbfgragh.onion/2008/03/23/books/review/McCulloch-t.html}{The
Diving Pool},'' a collection of novellas published in English (including
the prizewinning ``Pregnancy Diary''), features characters who use
subterfuge to inflict pain on people close to them. In
``\href{https://www.nytimes3xbfgragh.onion/2010/05/16/books/review/McCulloch-t.html}{Hotel
Iris},'' a sadistic older widower engages a 17-year-old high school
dropout in increasingly brutal sexual trysts.

Ogawa said she does not write about cruel characters to damn them but
rather to explore what might drive someone to physical or emotional
violence. ``People try to hide it from others or try to cover it up,''
she said. ``But in the world of literature, you can reveal that nature,
and it's O.K. to do so.''

Given that she writes vividly about female bodies, and the violence that
men can do to them, some critics have dubbed her a feminist writer.

``In a lot of her work she is interested in women's roles in the family
and women's bodies,'' said
\href{https://otemaeuniversity.academia.edu/KathrynTanaka}{Kathryn
Tanaka}, an associate professor of cultural and historical studies at
Otemae University in Nishinomiya, Japan. ``You really can't separate
that from questions of feminism and untangle her from that gendered
private space that her texts inhabit.''

Ogawa resists the label, saying she considers herself an eavesdropper on
her characters. ``I just peeked into their world and took notes from
what they were doing,'' she said.

``I see a bridge from that item to the next scene, or I see a rainbow
that I have to climb over to move to the next scene,'' she said.
``That's how I write.''

Advertisement

\protect\hyperlink{after-bottom}{Continue reading the main story}

\hypertarget{site-index}{%
\subsection{Site Index}\label{site-index}}

\hypertarget{site-information-navigation}{%
\subsection{Site Information
Navigation}\label{site-information-navigation}}

\begin{itemize}
\tightlist
\item
  \href{https://help.nytimes3xbfgragh.onion/hc/en-us/articles/115014792127-Copyright-notice}{©~2020~The
  New York Times Company}
\end{itemize}

\begin{itemize}
\tightlist
\item
  \href{https://www.nytco.com/}{NYTCo}
\item
  \href{https://help.nytimes3xbfgragh.onion/hc/en-us/articles/115015385887-Contact-Us}{Contact
  Us}
\item
  \href{https://www.nytco.com/careers/}{Work with us}
\item
  \href{https://nytmediakit.com/}{Advertise}
\item
  \href{http://www.tbrandstudio.com/}{T Brand Studio}
\item
  \href{https://www.nytimes3xbfgragh.onion/privacy/cookie-policy\#how-do-i-manage-trackers}{Your
  Ad Choices}
\item
  \href{https://www.nytimes3xbfgragh.onion/privacy}{Privacy}
\item
  \href{https://help.nytimes3xbfgragh.onion/hc/en-us/articles/115014893428-Terms-of-service}{Terms
  of Service}
\item
  \href{https://help.nytimes3xbfgragh.onion/hc/en-us/articles/115014893968-Terms-of-sale}{Terms
  of Sale}
\item
  \href{https://spiderbites.nytimes3xbfgragh.onion}{Site Map}
\item
  \href{https://help.nytimes3xbfgragh.onion/hc/en-us}{Help}
\item
  \href{https://www.nytimes3xbfgragh.onion/subscription?campaignId=37WXW}{Subscriptions}
\end{itemize}
