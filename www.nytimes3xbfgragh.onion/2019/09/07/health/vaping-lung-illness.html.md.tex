Sections

SEARCH

\protect\hyperlink{site-content}{Skip to
content}\protect\hyperlink{site-index}{Skip to site index}

\href{https://www.nytimes3xbfgragh.onion/section/health}{Health}

\href{https://myaccount.nytimes3xbfgragh.onion/auth/login?response_type=cookie\&client_id=vi}{}

\href{https://www.nytimes3xbfgragh.onion/section/todayspaper}{Today's
Paper}

\href{/section/health}{Health}\textbar{}What You Need to Know About
Vaping-Related Lung Illness

\url{https://nyti.ms/2N0omKU}

\begin{itemize}
\item
\item
\item
\item
\item
\end{itemize}

Advertisement

\protect\hyperlink{after-top}{Continue reading the main story}

Supported by

\protect\hyperlink{after-sponsor}{Continue reading the main story}

\hypertarget{what-you-need-to-know-about-vaping-related-lung-illness}{%
\section{What You Need to Know About Vaping-Related Lung
Illness}\label{what-you-need-to-know-about-vaping-related-lung-illness}}

Coughing, fatigue and shortness of breath are warning signs for anyone
who has vaped within the last 90 days.

\includegraphics{https://static01.graylady3jvrrxbe.onion/images/2019/09/07/us/politics/00vaping-disease-explainer/merlin_160293528_5b0b0b27-3c5e-49cf-b7eb-bb1ddce4646d-articleLarge.jpg?quality=75\&auto=webp\&disable=upscale}

By \href{https://www.nytimes3xbfgragh.onion/by/matt-richtel}{Matt
Richtel} and
\href{https://www.nytimes3xbfgragh.onion/by/denise-grady}{Denise Grady}

\begin{itemize}
\item
  Published Sept. 7, 2019Updated Sept. 11, 2019
\item
  \begin{itemize}
  \item
  \item
  \item
  \item
  \item
  \end{itemize}
\end{itemize}

\href{https://www.nytimes3xbfgragh.onion/es/2019/09/10/espanol/ciencia-y-tecnologia/vapeo-riesgos-salud.html}{Leer
en español}

Hundreds of people across the country have been
\href{https://www.nytimes3xbfgragh.onion/2019/09/06/health/third-death-vaping-related-disease.html?action=click\&module=Well\&pgtype=Homepage\&section=Health}{sickened
by a severe lung illness} linked to vaping, and a handful have died,
according to public health officials. Many were otherwise healthy young
people, in their teens or early 20s. Investigators from numerous states
are working with the federal Centers for Disease Control and Prevention
and the Food and Drug Administration in an urgent effort to figure out
why.

Here's what we know so far.

\emph{\textbf{{[}Read more:}}
\textbf{\href{https://www.nytimes3xbfgragh.onion/2019/09/11/health/trump-vaping-flavored-ecigarettes.html}{\emph{The
Trump administration is weighing a ban on some flavored
e-cigarettes.}}\emph{{]}}}

\hypertarget{who-is-at-risk}{%
\subsection{Who is at risk?}\label{who-is-at-risk}}

Anyone who uses e-cigarettes or other vaping devices, whether to consume
nicotine or substances extracted from marijuana or hemp, may be at risk
because investigators have not determined whether a specific device or
type of vaping liquid is responsible.

The Food and Drug Administration is warning that there appears to be a
particular danger for people who vape THC, the psychoactive chemical in
marijuana. The F.D.A. said a significant subset of samples of vaping
fluid used by sick patients included THC and also contained a chemical
called vitamin E acetate.

The F.D.A. issued this statement: ``Because consumers cannot be sure
whether any THC vaping products may contain vitamin E acetate, consumers
are urged to avoid buying vaping products on the street, and to refrain
from using THC oil or modifying/adding any substances to products
purchased in stores.''

But some of the patients who have fallen severely ill said they did not
vape THC. In 53 cases of the illness in Illinois and Wisconsin, 17
percent of the patients said they had vaped only nicotine products,
according to
\href{https://www.nejm.org/doi/full/10.1056/NEJMoa1911614}{an article
published on Friday} in The New England Journal of Medicine.

The researchers who wrote the journal article cautioned, ``e-cigarette
aerosol is not harmless; it can expose users to substances known to have
adverse health effects, including ultrafine particles, heavy metals,
volatile organic compounds and other harmful ingredients.''

The health effects of some of those chemicals are not fully understood,
the researchers wrote, even though the products are already on the
market.

\hypertarget{what-are-the-symptoms}{%
\subsection{What are the symptoms?}\label{what-are-the-symptoms}}

The early symptoms include fatigue, nausea, vomiting, coughing and
fever, escalating to shortness of breath, which can become so extreme it
can prompt an emergency room visit or require hospitalization.

Some patients have needed supplementary oxygen, including a
\href{https://www.nejm.org/doi/full/10.1056/NEJMoa1911614}{ventilator in
as many}\href{https://www.nejm.org/doi/full/10.1056/NEJMoa1911614}{as a
third} of cases analyzed in The New England Journal of Medicine.

On lung scans, the illness looks like a bacterial or viral pneumonia
that has attacked the lungs, but no infection has been found in testing.

\hypertarget{whats-the-best-way-to-prevent-the-illness}{%
\subsection{What's the best way to prevent the
illness?}\label{whats-the-best-way-to-prevent-the-illness}}

Health officials say that the riskiest behavior is using vaping products
bought on the street instead of from a retailer, or those that have been
tampered with or mixed.

Mitch Zeller, director for the Center for Tobacco Products at the
F.D.A., said, ``If you're thinking of purchasing one of these products
off the street, out of the back of a car, out of a trunk, in an alley,
or if you're going to then go home and make modifications to the product
yourself using something that you purchased from some third party or got
from a friend, think twice.''

The C.D.C. and some state health officials have recommended that people
give up vaping of any type until the cause of the lung damage is
determined.

An \href{https://www.nejm.org/doi/full/10.1056/NEJMe1912032}{editorial
in The New England Journal of Medicine} stated bluntly that doctors
should discourage their patients from vaping.

E-cigarettes and other vaping devices were developed to help cigarette
smokers quit their dangerous habit by providing a way to satisfy their
nicotine addiction without inhaling the toxins the come from burning
tobacco. But many medical experts now think even smokers should think
twice about turning to e-cigarettes --- and anyone who does not smoke
should not vape.

``Adult smokers who are attempting to quit should consult with their
health care provider and use proven treatments,'' the authors of the
analysis in The New England Journal of Medicine wrote.

They added, ``Irrespective of these findings, e-cigarettes should never
be used by youths, young adults, pregnant women and adults who do not
currently use tobacco products.''

\hypertarget{what-should-i-do-if-i-think-i-have-the-lung-illness}{%
\subsection{What should I do if I think I have the lung
illness?}\label{what-should-i-do-if-i-think-i-have-the-lung-illness}}

The C.D.C. says: ``If you are concerned about your health or the health
of a loved one who is using an e-cigarette product, contact your health
care provider, or your local poison control center at 1-800-222-1222.''

Anyone who has shortness of breath that lasts more than a few hours or
becomes severe should seek medical attention quickly. It is a warning
that should not be ignored, doctors say.

\hypertarget{why-do-health-investigators-think-this-is-linked-to-vaping}{%
\subsection{Why do health investigators think this is linked to
vaping?}\label{why-do-health-investigators-think-this-is-linked-to-vaping}}

Health investigators believe the illnesses are linked to vaping for
several key reasons: The patients have vaped nicotine or marijuana
extracts, or both, and do not have an infection or other condition that
would explain the lung disease. Patients are now characterized as having
the illness only if they have reported vaping within 90 days. In many of
the reported cases, the patients had vaped much more recently.

\hypertarget{e-cigarettes-have-been-around-for-years-why-is-this-happening-now}{%
\subsection{E-cigarettes have been around for years. Why is this
happening
now?}\label{e-cigarettes-have-been-around-for-years-why-is-this-happening-now}}

There are several theories. One is that some dangerous chemical or
combination of chemicals has been introduced into the pipeline of vaping
products. Public health officials believe that when people vape this
noxious cocktail, it sets off a dangerous, even lethal, reaction inside
the lungs. These officials have said repeatedly that they do not yet
know which substance or device may be causing this reaction, and that is
the subject of their urgent investigation.

A second theory is that this syndrome is not, in fact, entirely new and
that some people had gotten sick previously, but that the condition had
not been recognized and identified as being linked to vaping. As vaping
has grown in popularity --- both with nicotine and marijuana --- more
cases may be showing up.

For the time being, though, public health officials seem to believe that
the first theory is more likely and they are hunting for a substance,
substances or process that might explain the surge in illnesses.

Advertisement

\protect\hyperlink{after-bottom}{Continue reading the main story}

\hypertarget{site-index}{%
\subsection{Site Index}\label{site-index}}

\hypertarget{site-information-navigation}{%
\subsection{Site Information
Navigation}\label{site-information-navigation}}

\begin{itemize}
\tightlist
\item
  \href{https://help.nytimes3xbfgragh.onion/hc/en-us/articles/115014792127-Copyright-notice}{©~2020~The
  New York Times Company}
\end{itemize}

\begin{itemize}
\tightlist
\item
  \href{https://www.nytco.com/}{NYTCo}
\item
  \href{https://help.nytimes3xbfgragh.onion/hc/en-us/articles/115015385887-Contact-Us}{Contact
  Us}
\item
  \href{https://www.nytco.com/careers/}{Work with us}
\item
  \href{https://nytmediakit.com/}{Advertise}
\item
  \href{http://www.tbrandstudio.com/}{T Brand Studio}
\item
  \href{https://www.nytimes3xbfgragh.onion/privacy/cookie-policy\#how-do-i-manage-trackers}{Your
  Ad Choices}
\item
  \href{https://www.nytimes3xbfgragh.onion/privacy}{Privacy}
\item
  \href{https://help.nytimes3xbfgragh.onion/hc/en-us/articles/115014893428-Terms-of-service}{Terms
  of Service}
\item
  \href{https://help.nytimes3xbfgragh.onion/hc/en-us/articles/115014893968-Terms-of-sale}{Terms
  of Sale}
\item
  \href{https://spiderbites.nytimes3xbfgragh.onion}{Site Map}
\item
  \href{https://help.nytimes3xbfgragh.onion/hc/en-us}{Help}
\item
  \href{https://www.nytimes3xbfgragh.onion/subscription?campaignId=37WXW}{Subscriptions}
\end{itemize}
