Sections

SEARCH

\protect\hyperlink{site-content}{Skip to
content}\protect\hyperlink{site-index}{Skip to site index}

\href{https://myaccount.nytimes3xbfgragh.onion/auth/login?response_type=cookie\&client_id=vi}{}

\href{https://www.nytimes3xbfgragh.onion/section/todayspaper}{Today's
Paper}

Eight Great Things to See at the London Design Festival

\url{https://nyti.ms/2V0ynca}

\begin{itemize}
\item
\item
\item
\item
\item
\end{itemize}

Advertisement

\protect\hyperlink{after-top}{Continue reading the main story}

Supported by

\protect\hyperlink{after-sponsor}{Continue reading the main story}

\hypertarget{eight-great-things-to-see-at-the-london-design-festival}{%
\section{Eight Great Things to See at the London Design
Festival}\label{eight-great-things-to-see-at-the-london-design-festival}}

A labyrinth in front of Westminster Cathedral, a Kengo Kuma bamboo
structure outside the V\&A, and much more.

By \href{https://www.nytimes3xbfgragh.onion/by/aimee-farrell}{Aimee
Farrell}

\begin{itemize}
\item
  Published Sept. 18, 2019Updated Sept. 25, 2019
\item
  \begin{itemize}
  \item
  \item
  \item
  \item
  \item
  \end{itemize}
\end{itemize}

The \href{https://www.londondesignfestival.com/}{London Design
Festival}, now in its 17th edition, takes over 11 districts of the
sprawling city with more than 300 events, including trade shows, product
releases, pop-up exhibitions and installations. This year's fair also
features an ambitious roster of special commissions and festival
projects with a strong international presence, exemplified by works like
Martino Gamper's towering
\href{https://www.londondesignfestival.com/event/disco-carbonara}{disco-inspired
facade} in the new King's Cross District and Camille Walala's jazzy
postmodern
\href{https://www.londondesignfestival.com/event/walala-lounge}{street
furniture in the West End.} Here are standouts from the British
capital's eight-day celebration of design.

\emph{{[}}\href{https://www.nytimes3xbfgragh.onion/newsletters/t-list?module=inline}{\emph{Sign
up here}} \emph{for the T List newsletter, a weekly roundup of what T
Magazine editors are noticing and coveting now.{]}}

\includegraphics{https://static01.graylady3jvrrxbe.onion/images/2019/09/17/t-magazine/16tmag-londondesign-slide-WEUX/16tmag-londondesign-slide-WEUX-articleLarge.jpg?quality=75\&auto=webp\&disable=upscale}

\hypertarget{bamboo-ring-weaving-into-lightness-by-kengo-kuma}{%
\subsection{``Bamboo Ring: Weaving Into Lightness'' by Kengo
Kuma}\label{bamboo-ring-weaving-into-lightness-by-kengo-kuma}}

It feels fitting that Kengo Kuma's
``\href{http://londondesignfestival.com/event/bamboo-zhu-ring-weaving-lightness}{Bamboo
Ring: Weaving Into Lightness}'' would be given pride of place inside the
courtyard of the Victoria \& Albert Museum in London. Kuma, an architect
based in Tokyo, designed the institution's first northern outpost, the
V\&A Dundee, a powerful pyramidical structure that sits on the River
Tay. More than a decade in the making, the Scottish site celebrates its
first anniversary this month. For the London Design Festival, Kuma
imagined a similarly striking sculpture in Japanese bamboo interwoven
with carbon fiber --- a cross between a nest and a supersize Slinky.

The piece taps into Kuma's recollections of playing in the bamboo forest
behind his childhood home in Kanagawa, Japan. ``It's a very important
memory for me --- the smell, the green color and the softness of the
forest floor,'' says the designer, whose meditative bamboo spiral
levitates above the garden's oval pool. Kuma's hope is that this
carbon-strengthened woven technique can be applied to permanent
structures in the future --- **** a naturalistic architectural vision
that draws closer with the completion of his timber lattice stadium for
the 2020 Olympics in Tokyo. ``Unlike steel, bamboo has a warmness and a
softness to it that people really crave,'' he says. ``Here, the light
bamboo structure is in complete harmony with the peaceful beauty of the
V\&A gardens.''

Image

Frith Kerr's installation at Fenton House.Credit...Oskar Proctor

\hypertarget{please-sit-at-fenton-house}{%
\subsection{``Please Sit,'' at Fenton
House}\label{please-sit-at-fenton-house}}

The **** furniture and exhibition designer Gitta Gschwendtner makes her
curatorial debut at
\href{https://www.nationaltrust.org.uk/fenton-house-and-garden}{Fenton
House} with the tellingly titled show
``\href{https://www.londondesignfestival.com/event/please-sit}{Please
Sit}.'' Gschwendtner invited five designers to take the 17th-century
merchant's house in Hampstead, North London, as their starting point and
create seating installations positioned throughout the National Trust
property. ``The brief was to create an intervention that made visitors
dwell longer and to unlock some of the stories of the house,'' she says.
Inhabited until 1952 by Lady Katherine Binning, a collector of ceramics,
antique furniture and needlepoint embroidery, it's a decorative treasure
trove on an unusually intimate scale. Each installation taps into the
eccentricities of Lady Binning's life: In the drawing room, Nina
Tolstrup of Studiomama created an abstract mahogany bench that echoes
the shapes of the house's harpsichord collection, and Frith Kerr of
Studio Frith turned the bedroom into a haven of kitsch, casting her
ceramic poodle collection as shiny orange pillows strewn across the bed.
**** ``The idea is to question taste,'' Gschwendtner says. ``They look
like pumpkins --- but they're crazy poodle-inspired puffballs.''
Gschwendtner's own Jacob's ladder-style chairs, which nod to Binning's
love of needlework, can be found in the garden, which includes a
400-year-old apple orchard.

Image

``Life Labyrinth'' (2019), by Anna Murray and Grace Winteringham of
Patternity, installed at Westminster Cathedral in London.Credit...Andy
Stagg

Image

An aerial view of ``Life Labyrinth.''Credit...Andy Stagg

\hypertarget{life-labyrinth-at-westminster-cathedral}{%
\subsection{``Life Labyrinth,'' at Westminster
Cathedral}\label{life-labyrinth-at-westminster-cathedral}}

It's been 10 years since Anna Murray and Grace Winteringham established
Patternity, a project-led design studio and online archive that
celebrates the all-encompassing power of pattern. The duo's third major
London Design Festival commission, titled
``\href{https://www.londondesignfestival.com/event/life-labyrinth}{Life
Labyrinth},'' is a monumental monochrome maze outside Westminster
Cathedral (not to be confused with the rather grander Westminster Abbey
down the street). ``It's a celebration of the architecture of one of the
most underappreciated buildings in London,'' Murray says of the
installation's graphic stripes, which mirror the cathedral's
Byzantine-style décor. Inspired by the long heritage of the labyrinth, a
geometric symbol that appears across religious denominations and
throughout the natural world, Murray and Winteringham believe that
walking through the maze will encourage self-reflection and insight ---
or just a moment to enjoy the miniature garden at the center. ``The
spiraling formation is meant to be meditative,'' Murray says. ``It
definitely feels significant to be creating a space for contemplation in
Westminster, which is the epicenter of political chaos at the moment.''

Image

Max Lamb's ``Urushi Chair'' (2019).Credit...Courtesy of Kate Anglestein
for Gallery FUMI

\hypertarget{urushi-wajima-at-gallery-fumi}{%
\subsection{``Urushi Wajima,'' at Gallery
Fumi}\label{urushi-wajima-at-gallery-fumi}}

The first time the London-based designer \href{http://maxlamb.org/}{Max
Lamb} dabbled in \emph{urushi} --- a centuries-old Japanese lacquerware
technique --- he was undiscerning of theintricacy of the process. ``Just
looking at pictures in my studio in London, I couldn't tell the
difference between a piece of plastic and a piece of lacquerware,'' he
admits. For a project in 2010, Lamb presented a stool as part of a group
show at the Japanese Embassy in London that was designed in his studio
then shipped to Japan to undergo the urushi technique. Now, he returns
to the craft with
``\href{https://www.londondesignfestival.com/event/urushi-wajima-max-lamb}{Urushi
Wajima},'' a forensically executed, heartfelt showcase of the many
skilled pairs of hands --- 23 to be precise --- involved in the making
of the cabinets, stools, tables, benches and Wajima-nuri bowls of Lamb's
design. ``It's a celebration of the city of Wajima and its community of
craftspeople,'' he says of the show, which also spotlights the makers'
own work. ``From the timber growers to the wood splitters to the top
coaters and polishers.''

It's an unusually hands-off approach for Lamb, who spent many weeks and
months over the course of six years in Wajima --- a fishing port in
northeastern Japan known for its morning market --- immersed in the
urushi community. ``I love its pureness,'' he says of the material.
``It's essentially created using sap from a tree so it's entirely
renewable since it can be grown and harvested.'' Unlike plastic, which
fades over time, urushi hardens and develops a deeper, darker and richer
patina. ``It creates such a juicy and luscious surface finish,'' Lamb
says. ``The sheer number of processes involved in the making of urushi
means you end up with the glossiest thing on the planet.''

Image

Emily Alston's ``Never Lost'' (2019), installed inside the courtyard of
citizenM in Shoreditch, East London.Credit...GBPhotos

\hypertarget{never-lost-at-citizenm}{%
\subsection{``Never Lost,'' at CitizenM}\label{never-lost-at-citizenm}}

``Childlike, but not childish,'' is how the British designer and artist
Emily Alston --- who works under the moniker
\href{http://www.emilyforgot.co.uk/}{Emily Forgot} --- describes her
joyful furniture and sculptural objects. ``I can't be too serious ---
even if I try.'' Alston typically wears her architectural influences on
her sleeve: Miniature wooden assemblages are modeled after the
otherworldly buildings of Ricardo Bofill and Richard England, while the
graphic rugs and furnishings she created for the Dutch department store
De Bijenkorf celebrate the clean-lined beauty of the Bauhaus. Now,
Alston realizes her color-blocked vision on a heightened scale with
``\href{https://www.londondesignfestival.com/event/never-lost-citizenm-shoreditch}{Never
Lost},'' a geometric maze inside the courtyard of citizenM in
Shoreditch, East London. This interconnecting series of rooms and
passageways --- hand painted in a punchy palette of Yves Klein blue,
mint green and tomato orange --- has dead-end spaces to encourage
deviation. ``Mazes are the most surreal structure I can think of,''
Alston says gleefully. ``Normally you try to avoid taking wrong turns,
but here they can lead to interesting discoveries.''

Image

Granby Workshop's recycled ceramic tableware.Credit...Courtesy of Granby
Workshop

\hypertarget{granby-workshop-in-kings-cross}{%
\subsection{Granby Workshop in King's
Cross}\label{granby-workshop-in-kings-cross}}

\href{https://www.londondesignfestival.com/event/granby-workshop-launch-worlds-first-ceramic-tableware-made-100-waste}{Granby
Workshop} in Liverpool is one of Britain's most inventive ceramics
studios. A community-led practice, it was initially a place to make
decorative features --- fireplaces, door handles and bathroom tiles ---
to furnish the rows of Victorian houses being regenerated by the
architectural collective Assemble, together with the Community Land
Trust. That project, which won the 2015 Turner Prize, remains underway:
The group just received permission to transform a derelict building into
a cafe and apartments, and its facade will be clad in Granby tiles.

Now the workshop has created a collection of ceramic tableware, forged
entirely from waste materials. Everything from discarded refractory
bricks and laboratory test tubes to tiles from their own kiln are ****
ground up coarsely and bound together with wastewater from
industrial-scale producers, many in the nearby ceramics center of
Stoke-on-Trent. ``When the ceramics factories hose down their machines,
we get the sludge,'' explains Lewis Jones of the Granby Workshop team.
``The irony is that although it's an incredibly pure and refined
material it's also quite inconsistent, so we have to get every batch
chemically analyzed in a lab and then adjust our ingredients
accordingly.'' This hardy-looking table service, which comes in six
shades from multihued to mushroom, goes on display at Coal Drops Yard in
London's Kings Cross as part of a 28-day Kickstarter campaign to raise
the funds for the project to move into full-fledged production. ``It's
like alchemy,'' Jones says of the complex, time-consuming and careful
process. ``After all, this is the world's first 100 percent recycled
ceramic tableware.''

Image

Bertjan Pot's installation ``The Incredible House of Cards'' (2019) for
``Hem at Play'' at the Redchurch Rooms in Shoreditch.Credit...Courtesy
of Hem

\hypertarget{the-incredible-house-of-cards-at-the-redchurch-rooms}{%
\subsection{``The Incredible House of Cards,'' at the Redchurch
Rooms}\label{the-incredible-house-of-cards-at-the-redchurch-rooms}}

``The house of cards's geometric shape has long connections to modernist
architecture --- it's such an iconic design,'' says Petrus Palmér, the
founder of the Stockholm furniture maker Hem. For
``\href{https://www.londondesignfestival.com/event/hem-play}{Hem at
Play},'' the company commissioned Bertjan Pot, a renegade Dutch
designer, to create **** a decorative accessory that draws on the house
of cards's modular form --- most memorably replicated by Ray and Charles
Eames in 1952 --- in durable, die-cut
\href{https://www.gfsmith.com/}{G.F Smith}paper. Hem has decked the
exterior of the Redchurch Room exhibition space in a graphic mural that
echoes Pot's pyramidical patterned creation, making it hard for visitors
to the \href{https://www.shoreditchdesigntriangle.com/}{Shoreditch
Design Triangle} to miss. Alongside the ``Incredible House of Cards''
will be the fruits of Hem's other recent collaborations, with the
ceramist John Booth and the designer Max Lamb --- all part of Palmér's
mission to supercharge what he sees as a lackluster middle market with
compellingly creative and affordable design.

Image

A mask by Lucia Massari for SEEDS.Credit...Courtesy of SEEDS

Image

A mask by Bethan Laura Wood for SEEDS.Credit...Courtesy of SEEDS

\hypertarget{masters-of-disguise-in-south-kensington}{%
\subsection{``Masters of Disguise,'' in South
Kensington}\label{masters-of-disguise-in-south-kensington}}

Masks are having a moment at this year's
\href{https://bromptondesigndistrict.com/}{Brompton Design District},
one of the most thoughtfully curated quarters of the festival. For
``Masters of Disguise,''
\href{https://www.nytimes3xbfgragh.onion/2018/04/30/t-magazine/design/nathalie-assi-seeds-gallery.html}{SEEDS},
an experimental design platform, adorns a South Kensington townhouse
with masks, furniture and decorative objects by 20 artists including
Martino Gamper, Bethan Laura Wood and Nathalie Du Pasquier. The results
are brilliantly, dizzyingly diverse. There's Wood's netted veil,
stitched with a strong-browed self-portrait; a preternaturally glossy
pink resin block from Sabine Marcelis; and a kitschy mirrored ode to
Murano by Lucia Massari. Nearby a palatial table created by the show's
curator, Marco Campardo, of M-L-XL, is set for dinner with the
sculptural glassware of Jochen Holz and the glass-specked ceramics of
his partner, Attua Aparicio. To delve into the identity-altering magic
of masks on a more monumental scale, head to the
\href{https://www.vam.ac.uk/collections/tapestry}{V\&A's tapestry
galler}y around the corner. In the exhibition
``\href{https://www.vam.ac.uk/event/rKRpwj5N/bras-coupe-ldf-installation-sept-2019}{Black
Masking Culture},'' the breathtaking hand-sewn Mardi Gras suits of the
artist Big Chief Demond Melancon of the Young Seminole Hunters, his ****
tribe in the Lower Ninth Ward of New Orleans are on display for the
first time outside the United States.

Advertisement

\protect\hyperlink{after-bottom}{Continue reading the main story}

\hypertarget{site-index}{%
\subsection{Site Index}\label{site-index}}

\hypertarget{site-information-navigation}{%
\subsection{Site Information
Navigation}\label{site-information-navigation}}

\begin{itemize}
\tightlist
\item
  \href{https://help.nytimes3xbfgragh.onion/hc/en-us/articles/115014792127-Copyright-notice}{©~2020~The
  New York Times Company}
\end{itemize}

\begin{itemize}
\tightlist
\item
  \href{https://www.nytco.com/}{NYTCo}
\item
  \href{https://help.nytimes3xbfgragh.onion/hc/en-us/articles/115015385887-Contact-Us}{Contact
  Us}
\item
  \href{https://www.nytco.com/careers/}{Work with us}
\item
  \href{https://nytmediakit.com/}{Advertise}
\item
  \href{http://www.tbrandstudio.com/}{T Brand Studio}
\item
  \href{https://www.nytimes3xbfgragh.onion/privacy/cookie-policy\#how-do-i-manage-trackers}{Your
  Ad Choices}
\item
  \href{https://www.nytimes3xbfgragh.onion/privacy}{Privacy}
\item
  \href{https://help.nytimes3xbfgragh.onion/hc/en-us/articles/115014893428-Terms-of-service}{Terms
  of Service}
\item
  \href{https://help.nytimes3xbfgragh.onion/hc/en-us/articles/115014893968-Terms-of-sale}{Terms
  of Sale}
\item
  \href{https://spiderbites.nytimes3xbfgragh.onion}{Site Map}
\item
  \href{https://help.nytimes3xbfgragh.onion/hc/en-us}{Help}
\item
  \href{https://www.nytimes3xbfgragh.onion/subscription?campaignId=37WXW}{Subscriptions}
\end{itemize}
