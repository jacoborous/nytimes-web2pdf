The New Generation of Chefs Pushing Japanese Food in Unexpected
Directions

\url{https://nyti.ms/2loL9U1}

\begin{itemize}
\item
\item
\item
\item
\item
\end{itemize}

\includegraphics{https://static01.graylady3jvrrxbe.onion/images/2019/09/02/t-magazine/02tmag-japanamerfood-slide-9150/02tmag-japanamerfood-slide-9150-articleLarge.jpg?quality=75\&auto=webp\&disable=upscale}

Sections

\protect\hyperlink{site-content}{Skip to
content}\protect\hyperlink{site-index}{Skip to site index}

Food Matters

\hypertarget{the-new-generation-of-chefs-pushing-japanese-food-in-unexpected-directions}{%
\section{The New Generation of Chefs Pushing Japanese Food in Unexpected
Directions}\label{the-new-generation-of-chefs-pushing-japanese-food-in-unexpected-directions}}

With their fresh, freewheeling interpretations, these restaurants are
challenging long-held ideas about what authenticity actually means.

A bounty of Japanese tempura --- (from top) kabocha squash, shrimp,
broccoli, eggplant, shiitake mushroom and shiso leaf --- with ikura
(salmon roe), unagi (barbecue eel) and sake (salmon)
sushi.Credit...Photo by Mari Maeda and Yuji Oboshi. Food styling by
Rebecca Jurkevich. Prop styling by Victoria Petro-Conroy

Supported by

\protect\hyperlink{after-sponsor}{Continue reading the main story}

By Ligaya Mishan

\begin{itemize}
\item
  Published Sept. 2, 2019Updated Sept. 19, 2019
\item
  \begin{itemize}
  \item
  \item
  \item
  \item
  \item
  \end{itemize}
\end{itemize}

IN 1906, THE YOKOHAMA-BORN scholar Okakura Kakuzo published
``\href{https://www.penguinrandomhouse.com/books/308458/the-book-of-tea-by-kakuzo-okakura/9780141191843/}{The
Book of Tea},'' a brief tract for Western readers on \emph{chanoyu}, the
centuries-old, highly ritualized Japanese tea ceremony. He argued that
the aestheticization of the humble act of drinking tea --- ``the
adoration of the beautiful among the sordid facts of everyday
existence'' --- must be understood as an ethos underlying an entire
culture, from its arts and literature to the ``delicate dishes'' of its
cuisine. His intent was to demystify, but his words had almost the
opposite effect, heightening the sense of opacity surrounding both the
Japanese approach to food and the island nation itself, which from the
early 17th century until 1853 was almost completely closed off from the
rest of the world.

More than 50 years after Kakuzo's treatise, the French literary theorist
Roland Barthes, in his 1970 monograph
``\href{https://us.macmillan.com/books/9780374522070}{Empire of
Signs},'' described Japanese cooking in even more esoteric terms,
arguing that it privileged the infinitesimal over Western abundance and
was practiced ``in a profound space which hierarchizes man, table and
universe.'' Today, Westerners remain in thrall to this vision of
\emph{washoku}, traditional Japanese cuisine, as forbiddingly precise,
each ingredient presented sparely and simply within the narrow window of
ripeness in which it has fulfilled its destiny, to reflect the
ephemerality of life. (Never mind that this philosophy has only ever
applied to \emph{kaiseki}, the most rarefied level of Japanese dining.)

\emph{{[}}\href{https://www.nytimes3xbfgragh.onion/newsletters/t-list?module=inline}{\emph{Sign
up here}} \emph{for the T List newsletter, a weekly roundup of what T
Magazine editors are noticing and coveting now.{]}}

So it's slightly disconcerting to find a bag of Safeway-brand jalapeño
Cheddar cheese bagels --- surely not representative of the beautiful in
any culture --- in the kitchen at
\href{https://www.hannyatou.com/}{Hannyatou}, a tiny sake bar in
\href{https://www.nytimes3xbfgragh.onion/2018/07/05/travel/what-to-do-in-seattle.html}{Seattle}
helmed by the chef Mutsuko Soma. Lumpy and craggy, the bagels are
treated as a serious ingredient: pulverized, then calibrated with salt
and koji (grains or legumes inoculated with spores of Aspergillus
oryzae, phylogenetically kin to the mold that turns coagulated milk into
blue cheese) and left to turn funky and fetid over weeks. Soma grew up
north of Tokyo and came to the United States at the age of 18. She is
one of several chefs outside Japan --- expatriates, immigrants and nisei
and sansei (second- and third-generation descendants of immigrants), as
well as gaijin (foreigners), drawn, often circuitously, to the cuisine
--- who have opened restaurants in the past few years that are pushing
Japanese food in unexpected, even counterintuitive directions.

Purists might dispute the idiosyncratic unfolding of kaiseki at the
haute \href{https://www.odo.nyc/}{Odo}, half-hidden like a speakeasy at
the back of a cocktail bar in the Flatiron district of Manhattan, where
Hiroki Odo has been known to forsake tempura in the \emph{agemono}
(fried) course in favor of a French croquette heavy with béchamel. There
will be quibbles over the dashi deployed at the Los Angeles breakfast
and lunch spot \href{https://konbila.com/}{Konbi}, since the chefs,
\href{https://www.nytimes3xbfgragh.onion/2019/01/28/dining/konbi-egg-salad-sandwich-instagram.html}{Akira
Akuto and Nick Montgomery}, leave the bonito shavings to steep and
simmer longer than usual, privileging deep, brooding flavor over
clarity. And downright bewilderment might greet the melting of Swiss
cocoa powder into curry at
\href{https://www.nytimes3xbfgragh.onion/2018/04/30/dining/austin-texas-japanese-restaurants.html}{Tatsu
Aikawa}'s cheekily named \href{https://domo-tatsuya.com/}{Domo
Alley-Gato} bar in Austin, Tex.

Yet however maverick or heretical on the surface, the work of these
chefs is rooted in Japanese technique. Soma treats those jalapeño
Cheddar cheese bagels as if they were soybeans en route to miso, and the
paste they become achieves the same desirable tang of salty-sweet
underground rot. Nor is there anything radical about these chefs'
attention to seasonality and place, tenets at the heart of washoku. It
just so happens that the place in question is not Japan but Paris, in
the 11th Arrondissement, where the American chefs Robert Compagnon and
Jessica Yang of \href{https://www.lerigmarole.com/}{Le Rigmarole} have
adopted Japanese yakitori as, Compagnon says, ``a framing mechanism for
whatever is in season'' --- tiny charred tomatoes with puckering skins,
leeks daubed with cod-roe mayonnaise --- and made variants on the
sour-spicy condiment yuzu kosho out of French citrus fruits as they come
in and out of harvest. In Brooklyn,
\href{https://www.nytimes3xbfgragh.onion/2018/10/31/magazine/how-to-cook-all-the-mushrooms.html}{Patch
Troffer}, an American chef of Japanese descent who last year took over
the kitchen at the farm-to-table institution
\href{http://marlowandsons.com/}{Marlow \& Sons}, supplants wasabi with
horseradish root grown in upstate New York. ``It's the food of the
displaced and the diaspora,'' Troffer says. ``What happens when you
don't have the right ingredients'' --- a lesson he learned from his
Japanese grandmother, who married a marine during the Korean War and
wound up in South Carolina, making dashi out of canned clams and writing
to \href{http://katagiri.com/}{Katagiri} grocery in New York to beg for
shipments of soy sauce and umeboshi.

Odo, a native of
\href{https://www.nytimes3xbfgragh.onion/2017/10/09/t-magazine/travel/new-japan-hotels.html}{Kyushu},
has had to adjust to the tastes and textures of American ingredients, as
well as the bias of the American palate toward more flagrant flavors.
The almost ascetic simplicity of classical kaiseki can be a cultural
barrier; diners here ``might feel like they're eating nothing,'' he
says. (His American-born sous chef, Brian Saito, translated for us.)
Foraged vegetables from Pennsylvania and upstate New York are delivered
to the restaurant once a week. On a recent afternoon in April, they
included ramps, whose garlicky punch would be considered too strong for
dishes intended to accompany the tea ceremony in Kyoto, where Odo
apprenticed in the cuisine. But ``this is New York kaiseki,'' he says,
so he commits to richness and pairs the ramps with wild Alaskan king
salmon, an oily fish that is marinated in bourbon --- instead of sake
--- chosen partly for aroma and partly for provenance: It's made nearby
at Brooklyn's \href{http://kingscountydistillery.com/}{Kings County
Distillery}.

For Aikawa, who at the age of 10 was whisked by his mother from Tokyo to
a rural Texan commune, food tells the story of immigration and the
meeting of cultures. ``When I go to a barbecue, I bring a tub of rice,''
he says. At \href{http://kemuri-tatsuya.com/}{Kemuri Tatsu-ya}, the
half-izakaya, half-barbecue spot he and the chef Takuya Matsumoto opened
in 2017, he categorizes brisket as lean or \emph{toro}, borrowing from
sushi vocabulary the designation of fatty tuna. ``I want to treat
brisket like sashimi --- put it on a pedestal,'' he says. His take on
Texas barbecue is straightforward (``out of respect''), but there's a
touch of miso in the sauce, and he anoints yakitori skewers of chicken
skin with garlic salt and lime to honor his Mexican neighbors.

Within this cohort, several chefs revel in the juxtaposition of Japanese
and Italian cuisine --- the latter long beloved in Japan, where it is
fondly called \emph{itameshi}, and where local chefs obsess over
perfecting Neapolitan pizza with kerchief-thin, pliant crusts and
cooking spaghetti to the exact second of al dente. Amid the parade of
yakitori at Le Rigmarole, Compagnon and Yang present pasta that shows a
clear debt to Italy while resembling no codified recipe; even their
noodle shapes and names --- cushioni, for ravioli that look like doll
pillows; faniciulle, from the Italian word for maidens, elaborately
folded like demure hoods --- are the chefs' inventions. At
\href{https://www.blackshipla.com/}{Blackship} in West Hollywood, which
opened last December, the New York-raised Keiichi Kurobe presses shiso
leaves into housemade noodles and garnishes dishes with them in lieu of
basil. And a few miles away, in the Palms neighborhood, the best-known
dish at
\href{https://www.nytimes3xbfgragh.onion/2018/05/16/t-magazine/food/female-chefs-rita-sodi-jody-williams-erika-nakamura.html}{Niki
Nakayama}'s \href{https://n-naka.com/}{n/naka} is the pasta that
materializes in the middle of her otherwise recognizably Japanese
kaiseki: Derived from a genre of food called
\href{https://www.nytimes3xbfgragh.onion/2008/03/26/dining/26japan.html}{\emph{yoshoku}}
--- dishes borrowed from the West and freely altered with local
ingredients to satisfy Japanese tastes --- her spaghetti is glossed with
mentaiko (pickled cod roe), as it might appear in Japan, then strewn
with petals of razor-cut abalone and black truffles.

\includegraphics{https://static01.graylady3jvrrxbe.onion/images/2019/09/02/t-magazine/02tmag-japanamerfood-slide-2CM3/02tmag-japanamerfood-slide-2CM3-articleLarge.jpg?quality=75\&auto=webp\&disable=upscale}

THESE DISHES CONFOUND Western notions of what Japanese food should be,
in part because diners who haven't grown up eating the cuisine often
encounter it in the limited binary framework of high and low: austere
sushi bars where the tab starts at three figures versus
\href{https://www.nytimes3xbfgragh.onion/2010/01/31/travel/31ramen.html}{quick-turnover
ramen shops}, with few options in between. In adopting ingredients and
techniques from other cultures, the new movement might even
uncomfortably recall the Asian-fusion trend that started in the late
'80s, which was spearheaded by chefs of European descent. But where
those chefs filtered Japanese cuisine through a Western perspective,
taking Japanese elements out of context and subsuming and bending them
to their will, today's chefs are doing the opposite --- viewing the West
and its culinary traditions through a Japanese lens. As the thinking on
diversity in America has evolved from the metaphor of a melting pot to a
mosaic, in which each piece keeps its integrity while enriching the
whole, the concept of fusion has become archaic, replaced by a more
organic understanding of how food changes when people immigrate and have
to adapt to the ingredients on hand.

By refuting rigid orthodoxy --- and some inchoate standard of
authenticity --- these chefs remind us that Japanese cuisine is not some
repository of edicts past but a lived and living tradition, as well as a
pastiche, one that has borrowed unapologetically from other cultures
throughout history, despite the country's long seclusion. Tempura, both
dish and word, was a gift from the Portuguese, whose language was
brought accidentally to Japan when, in 1543, three Portuguese sailors on
a Chinese ship made contact in southern Japan. Jesuit missionaries
followed, ultimately passing on a recipe for peixinhos da horta
(``little fish of the garden''): green beans dusted in flour and
deep-fried.

Curry arrived in the 19th century, during the Meiji era, from India via
the British Royal Navy, when the subcontinent was part of the Raj. It
was considered a Western dish and thus pricey, until the late 1950s,
when Japanese companies started selling instant curry that produced a
dish milder and sweeter than either its British or Indian counterpart.
Troffer modeled his curry after the best-selling S\&B brand but with a
lashing of heat; during the colder months, it's served at Marlow as it
often appears in Japan, with pork katsu, a cutlet gilded in panko.
Aikawa took his Texas version further afield, finding kinship to
Louisiana gumbo and Mexican mole as he wrangled more than two dozen
spices trying to strike the right balance, recalibrating by the gram in
batch after batch. He serves his curry straight or amped up into a near
chili, which is stuffed in a brioche bun and topped by a hot dog that's
been patted down with panko and deep-fried so it suggests a hard-shell
taco.

\hypertarget{the-concept-of-fusion-has-become-archaic-replaced-by-a-more-organic-understanding-of-how-food-changes-when-people-immigrate}{%
\subsection{The concept of fusion has become archaic, replaced by a more
organic understanding of how food changes when people
immigrate.}\label{the-concept-of-fusion-has-become-archaic-replaced-by-a-more-organic-understanding-of-how-food-changes-when-people-immigrate}}

Ramen, likewise, has no time-honored history. According to
\href{https://www.nytimes3xbfgragh.onion/2014/04/30/dining/an-indie-spirits-shop-a-history-book-on-ramen-real-baby-carrots-and-more.html}{George
Solt}'s
``\href{https://www.ucpress.edu/book/9780520282353/the-untold-history-of-ramen}{The
Untold History of Ramen}'' (2014), the dish is said to have first
appeared in 1910 in Tokyo, under the name shina soba (Chinese noodles);
almost vanished during World War II, when flour was strictly rationed
and street vendors were banned; and revived with imports of wheat under
the midcentury U.S. occupation --- when Americans hoped to keep the
population sated and therefore invulnerable to the promises of communism
--- to eventually flourish postwar as a hearty and cheap lunch. Of all
Japanese foods, it might be ``the most open, the most receptive to
change and experimentation,'' the American-born chef
\href{https://www.nytimes3xbfgragh.onion/2013/12/18/dining/a-life-of-noodles-comes-full-circle.html}{Ivan
Orkin} wrote (with Chris Ying) in the 2013 cookbook
``\href{https://www.penguinrandomhouse.com/books/222246/ivan-ramen-by-ivan-orkin-with-chris-ying-foreword-by-david-chang/9781607744467/}{Ivan
Ramen}.''

Japanese chefs must typically apprentice for years before they get the
opportunity to run their own kitchens, but Shigetoshi Nakamura won fame
for his ramen shop in Tokyo while still in his 20s. Earlier this decade,
he opened an \href{http://www.nakamuranyc.com/}{eponymous shop} on the
Lower East Side of Manhattan, and this year he converted the storefront
next door to \href{http://www.nakamuranyc.com/niche}{Niche}, focusing on
mazemen, a version of ramen that largely dispenses with broth. In homage
to the neighborhood's historic Jewish delis, Nakamura cold-smokes salmon
in-house and drapes it over noodles in a loose sauce of cod roe and
olive oil.

Even the California roll, often held up as an example of sacrilege, is
believed to have been invented by a Japanese immigrant chef in the late
1960s, who, finding himself in Los Angeles without a reliable supply of
bluefin tuna, swapped in an ingredient more plentiful on the West Coast,
one with its own richness and heft: avocado.

AS THE CONTOURS and definitions of Japanese food have expanded, many
chefs who are not of Japanese descent have also devoted themselves to
this contemporary, freewheeling style, further collapsing and
questioning the boundaries between the East and the West. Compagnon and
Yang see their Paris restaurant as an ideal compact between cultures
equally obsessed with mastery; as Compagnon says with a laugh, ``France
and Japan are the only two culinary cultures that respect each other
while looking down on everyone else.'' (Some of today's most lauded
French restaurants in Paris --- including
\href{https://www.les-enfants-rouges.fr/}{Les Enfants Rouge},
\href{http://www.clown-bar-paris.com/}{Clown Bar} and
\href{https://www.lafourchette.com/restaurant/abri/308115?cc=18174-54f}{Abri}
--- are run by chefs from Japan, whose compatriots back in their native
country are equally scrupulous in their devotion to French cuisine.)
Compagnon came to Japanese food by first studying Japanese language and
literature, as did Orkin, who grew up on New York's Long Island and
lived in Tokyo for years. These chefs are quick to acknowledge their
status as gaijin and students, not masters, of the dishes they've come
to love. Orkin and Ying's forthcoming book's title,
``\href{https://www.hmhbooks.com/shop/books/The-Gaijin-Cookbook/9781328954350}{The
Gaijin Cookbook: Japanese Recipes from a Chef, Father, Eater, and
Lifelong Outsider},'' addresses it directly, while Compagnon and Yang
demur from calling Le Rigmarole a Japanese restaurant, speaking instead
of a prevailing aesthetic and attention to technique.

\hypertarget{im-being-very-clear-to-myself-that-im-not-doing-japanese-food-says-chef-patch-troffer-i-want-to-explore-what-it-is-to-be-japanese-american}{%
\subsection{`I'm being very clear to myself that I'm not doing Japanese
food,' says chef Patch Troffer. `I want to explore what it is to be
Japanese-American.'}\label{im-being-very-clear-to-myself-that-im-not-doing-japanese-food-says-chef-patch-troffer-i-want-to-explore-what-it-is-to-be-japanese-american}}

For Troffer --- whose half-Japanese mother didn't cook Japanese food
often but always had nori and a pot of rice at the ready --- there is no
distance between East and West. ``I'm being very clear to myself that
I'm not doing Japanese food,'' he says. ``I want to explore what it is
to be Japanese-American.'' The results reflect an attunement to the full
range of possibility latent in each ingredient: He layers and deepens
flavors by using dashi instead of water, ``finding every little moment
where an ingredient can slip its way in and add something,'' he says.
Nevertheless, his grandmother was skeptical when he showed her a
photograph of his
\href{https://cooking.nytimes3xbfgragh.onion/recipes/1020172-classic-okonomiyaki-japanese-cabbage-and-pork-pancakes}{okonomiyaki},
which he calls a sour cabbage pancake on his menu in homage to how she
used to make it, with little more than shredded cabbage, soy sauce and
flour. A fried egg is laid over it, in a flop. ``She gave me the most
disapproving eyebrow,'' he says.

Yet in Japan, this would hardly be heretical. Freedom is built into the
very name of the dish; broken down into \emph{okonomi} and \emph{yaki},
it means whatever you want, thrown on the grill. And although the
okonomiyaki most commonly found throughout Japan originated in Osaka,
there are a number of regional variations, including the Hiroshima
style, in which the dish is built one strata at a time: first batter,
followed by cabbage, bean sprouts, pork and noodles and, finally, yes, a
fried egg, with the rest of the pancake shoveled over it and then
flipped so the egg lands on top.

Sometimes Japanese visitors to Odo's restaurant in Manhattan tell him
that they miss the milder flavors of traditional kaiseki. But the chef
remains firm in his mission. The strict etiquette and radical simplicity
of the formal meal are ``not very welcoming to Americans,'' he says,
which contradicts the Japanese principle of omotenashi, an elevated form
of hospitality in which the guest's happiness is the focus of all action
and thought. Even in Japan, kaiseki can intimidate diners, particularly
of the younger generation. To ameliorate this, Zaiyu Hasegawa, the chef
of \href{https://www.jimbochoden.com/en/}{Den}, a modern kaiseki spot
that opened in 2008 in Tokyo, begins each meal with monaka, an everyday
Japanese treat of adzuki bean paste smeared between mochi wafers. While
his filling is elevated, studded with foie gras and persimmon, its
appearance is not: The dish arrives at the table as the kind of sandwich
cookie sold at convenience stores, complete with a paper wrapper. Later
comes a salad with carrots cut into the emoji with hearts for eyes and a
box evoking Kentucky Fried Chicken that contains wings shucked of bone
and stuffed with sticky rice, nestled on a bed of straw.

The food is thrillingly irreverent, so at first you don't notice how
fastidious it is, how close to perfection. You laugh, and then you fall
silent, the quick visual delight giving way to depths of flavor and
something more elusive --- a consciousness of food as past and present,
at once memory and daily recurrence. The old ways meet the new --- not
in combat but in continuance.

Advertisement

\protect\hyperlink{after-bottom}{Continue reading the main story}

\hypertarget{site-index}{%
\subsection{Site Index}\label{site-index}}

\hypertarget{site-information-navigation}{%
\subsection{Site Information
Navigation}\label{site-information-navigation}}

\begin{itemize}
\tightlist
\item
  \href{https://help.nytimes3xbfgragh.onion/hc/en-us/articles/115014792127-Copyright-notice}{©~2020~The
  New York Times Company}
\end{itemize}

\begin{itemize}
\tightlist
\item
  \href{https://www.nytco.com/}{NYTCo}
\item
  \href{https://help.nytimes3xbfgragh.onion/hc/en-us/articles/115015385887-Contact-Us}{Contact
  Us}
\item
  \href{https://www.nytco.com/careers/}{Work with us}
\item
  \href{https://nytmediakit.com/}{Advertise}
\item
  \href{http://www.tbrandstudio.com/}{T Brand Studio}
\item
  \href{https://www.nytimes3xbfgragh.onion/privacy/cookie-policy\#how-do-i-manage-trackers}{Your
  Ad Choices}
\item
  \href{https://www.nytimes3xbfgragh.onion/privacy}{Privacy}
\item
  \href{https://help.nytimes3xbfgragh.onion/hc/en-us/articles/115014893428-Terms-of-service}{Terms
  of Service}
\item
  \href{https://help.nytimes3xbfgragh.onion/hc/en-us/articles/115014893968-Terms-of-sale}{Terms
  of Sale}
\item
  \href{https://spiderbites.nytimes3xbfgragh.onion}{Site Map}
\item
  \href{https://help.nytimes3xbfgragh.onion/hc/en-us}{Help}
\item
  \href{https://www.nytimes3xbfgragh.onion/subscription?campaignId=37WXW}{Subscriptions}
\end{itemize}
