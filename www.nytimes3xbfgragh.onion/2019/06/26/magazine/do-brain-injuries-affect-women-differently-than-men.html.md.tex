Sections

SEARCH

\protect\hyperlink{site-content}{Skip to
content}\protect\hyperlink{site-index}{Skip to site index}

\href{https://myaccount.nytimes3xbfgragh.onion/auth/login?response_type=cookie\&client_id=vi}{}

\href{https://www.nytimes3xbfgragh.onion/section/todayspaper}{Today's
Paper}

Do Brain Injuries Affect Women Differently?

\url{https://nyti.ms/2FvshKo}

\begin{itemize}
\item
\item
\item
\item
\item
\item
\end{itemize}

Advertisement

\protect\hyperlink{after-top}{Continue reading the main story}

Supported by

\protect\hyperlink{after-sponsor}{Continue reading the main story}

\href{/column/studies-show}{Studies Show}

\hypertarget{do-brain-injuries-affect-women-differently}{%
\section{Do Brain Injuries Affect Women
Differently?}\label{do-brain-injuries-affect-women-differently}}

\includegraphics{https://static01.graylady3jvrrxbe.onion/images/2019/06/30/magazine/30Studies/30Studies-articleLarge.jpg?quality=75\&auto=webp\&disable=upscale}

By Kim Tingley

\begin{itemize}
\item
  June 26, 2019
\item
  \begin{itemize}
  \item
  \item
  \item
  \item
  \item
  \item
  \end{itemize}
\end{itemize}

In 1994, the National Football League formed a Committee on Mild
Traumatic Brain Injury to study an alarming trend: Players were retiring
early because of what seemed to be concussion-related problems,
including persistent headaches, vertigo, cognitive impairment,
personality changes, fatigue and difficulty performing ordinary daily
activities. Around the same time, Eve Valera, then a Ph.D. student in
clinical psychology at the University of Illinois, began to volunteer in
a domestic-violence shelter and wondered how many of the women there
might be experiencing comparable post-concussive symptoms as a result of
head injuries inflicted by their partners.

When Valera could not find any published studies on brain trauma related
to such violence, she decided to conduct one herself, by interviewing
the women where she volunteered. She published the results in 2003 ---
two years before Bennet Omalu, then a pathologist at the University of
Pittsburgh, \href{https://www.ncbi.nlm.nih.gov/pubmed/15987548}{reported
the first known case in a deceased N.F.L. player of chronic traumatic
encephalopathy} (C.T.E.), a neurodegenerative disease characterized by
some of the same symptoms plaguing the retired players. Three-quarters
of the women, Valera found, had received at least one traumatic brain
injury (T.B.I.); half had sustained multiple mild traumatic brain
injuries.

In the U.S., the Centers for Disease Control and Prevention estimates
that one in three women over the age of 15 has experienced what it
categorizes as ``intimate partner violence.'' When Valera extends her
sample to the overall population, she gets estimates that as many as 31
million women might have had a T.B.I. and 21 million might have had
multiple mild ones. ``Using annual estimates of severe physical
violence,'' Valera notes in
\href{https://www.liebertpub.com/doi/abs/10.1089/neu.2018.5734}{a study
published last fall in the Journal of Neurotrauma}, ``1.6 million women
can be estimated to sustain repetitive T.B.I.s in comparison to the
total annual numbers of T.B.I.s reported for the military and N.F.L. at
18,000 and 281 respectively.''

Yet most of what scientists know about the potential prevalence and
consequences of mild traumatic brain injury has come from studying
contact sports, especially football --- so, mostly men and boys --- over
the past 15 years. It's a vivid illustration of a broad and pernicious
problem in medical research, which is that some groups of people get far
more attention than others --- often leading to important gaps in
medical understanding, even around conditions that the public regards as
``widely studied.''

For her study in the Journal of Neurotrauma, Valera, now a
neuroscientist at Harvard Medical School, imaged areas of white matter
thought to be involved in learning and memory in the brains of 20 women
subjected to partner violence. The brain injuries were associated with
what she believes reflects abnormalities in these brain regions. But she
says that the study had significant limitations because of meager
funding: a small sample size and no control group of women who were
assaulted by partners but did not have head trauma. Understanding the
effect of such changes over time would require expensive long-term
studies. Yet, the news that thousands of women might be dealing with
undiagnosed brain damage did not garner much attention: According to
Altmetric, which tracks the online activity generated by scientific
studies, Valera's findings were
\href{https://www.altmetric.com/details/43423449}{tweeted four times}.

\includegraphics{https://static01.graylady3jvrrxbe.onion/images/2019/06/30/magazine/30Studies_2/30Studies_2-articleLarge.jpg?quality=75\&auto=webp\&disable=upscale}

In contrast,
\href{https://www.liebertpub.com/doi/abs/10.1089/neu.2014.3822}{a 2015
study of football players' white matter}, conducted by researchers at
Boston University and published in the same journal, was
\href{https://www.altmetric.com/details/4308278}{tweeted 50 times} and
received more widespread notice. (``Are You Ready for Some Football
Brain Damage?'' a USA Today headline asked.) It compared the white
matter in areas of the brain of 20 former N.F.L. players who began
playing football before age 12 with that of 20 who were the same age and
started at or after age 12 and found many more abnormalities in the
brains of the younger group, suggesting for the first time that the age
a person is first exposed to football may influence his later
susceptibility to brain injuries. It, too, acknowledged the limits of
its sample (small and specific) and called for further research, much
more of which has now been done, including on youth players who never
reach elite levels. ``We've shown over and over that it isn't just
concussions,'' Ann McKee, who is the director of Boston University's
C.T.E. Center, told me. ``It's number of playing years.'' She adds:
``It's the lower-level hits, what we call subconcussions, that are
asymptomatic, that the player plays right through without even
recognizing that he's had an injury. Those are the low-level hits that
we've shown increase the risk and severity of C.T.E.''

The media has raised alarm about these findings --- so much so that
other experts worry that the media is overstating the absolute risk of
developing C.T.E. and understating the substantial health benefits that
team sports, including football, offer. In
\href{https://jnnp.bmj.com/content/88/6/462}{a 2017 editorial in The
Journal of Neurology, Neurosurgery and Psychiatry}, Alan Carson, a
professor of neuropsychiatry at the University of Edinburgh, points to
\href{https://www.ncbi.nlm.nih.gov/pmc/articles/PMC4098841/}{a 2012
study of 3,439 former N.F.L. players}. It found that they died from
neurodegenerative diseases at three times the rate of the general
population, but were half as likely to die of any other cause.

The trouble with comparing N.F.L. players with the general population,
however, is that people who go on to become elite athletes may be a
healthier cohort to begin with. Their superior health may lead them to
play football, rather than it being the case that playing football
improves their health. Last month,
\href{https://jamanetwork.com/journals/jamanetworkopen/fullarticle/2734063}{a
new study by researchers at Harvard sought to control for this bias by
comparing N.F.L. players with Major League Baseball players}. It found
that the football players had higher levels of mortality from all
causes, including cardiovascular and neurodegenerative diseases, than
the baseball players did, which could indicate that football itself was
detrimental.

All available evidence suggests that reducing exposure to tackle
football would reduce the incidence of C.T.E., which meets the criteria
of a public health concern, the authors of
\href{https://www.tandfonline.com/doi/abs/10.1080/10807039.2018.1456899}{a
paper last year in the journal Human and Ecological Risk Assessment}
wrote. It is hard to say how much of the lingering debate over the risks
of tackle football are a result of the N.F.L. becoming a major donor to
concussion research; in the past, the league has attempted to defund
researchers whose work shows that the accumulation of lesser hits may be
even more detrimental. ``In many ways, it's to their advantage if the
debate continues,'' Philip M. Rosoff, of the Trent Center for Bioethics,
Humanities \& History of Medicine at Duke University, told me. But the
paper also noted a ``large and growing disconnect'' between how public
health scientists read the data and how clinicians do: a pediatrician
whose young patients suffer from obesity, for example, may see football
as a risk worth taking.

But if these risks are important to understand and mitigate for the
million-plus boys playing tackle football --- clearly they are --- then
why have we not put equal resources into studying them in women, a
potentially vast number of whom could have been exposed to head trauma?
The implications could be profound. For example, researchers hope that
learning how C.T.E. works could help them diagnose and treat other
neurodegenerative diseases, like Alzheimer's, in which abnormal proteins
in the brain may appear decades before they eventually damage tissue and
lead to symptoms; unlike Alzheimer's, which has no known cause, C.T.E.
now appears to have a clear starting point in head trauma, which makes
it possible to study its progression over time. (Researchers are still
searching for a way to definitively test for both diseases in a living
brain.) This progression, however, may be different in men and women. In
fact, the little research on head injuries in female athletes and
service members suggests that their brains may be more susceptible to
trauma than men's are. Two-thirds of those who get Alzheimer's diagnoses
are women.

Unfortunately, they may never benefit from adequate research. Part of
the problem is that women hurt by intimate partners tend to hide that
fact, making them hard to identify and study. But the bigger issue is
that public outrage and advocacy play a major role in determining what
research gets funded. In the case of head trauma, almost all the
attention is going to football --- and so, by extension, to only one
gender.

Advertisement

\protect\hyperlink{after-bottom}{Continue reading the main story}

\hypertarget{site-index}{%
\subsection{Site Index}\label{site-index}}

\hypertarget{site-information-navigation}{%
\subsection{Site Information
Navigation}\label{site-information-navigation}}

\begin{itemize}
\tightlist
\item
  \href{https://help.nytimes3xbfgragh.onion/hc/en-us/articles/115014792127-Copyright-notice}{©~2020~The
  New York Times Company}
\end{itemize}

\begin{itemize}
\tightlist
\item
  \href{https://www.nytco.com/}{NYTCo}
\item
  \href{https://help.nytimes3xbfgragh.onion/hc/en-us/articles/115015385887-Contact-Us}{Contact
  Us}
\item
  \href{https://www.nytco.com/careers/}{Work with us}
\item
  \href{https://nytmediakit.com/}{Advertise}
\item
  \href{http://www.tbrandstudio.com/}{T Brand Studio}
\item
  \href{https://www.nytimes3xbfgragh.onion/privacy/cookie-policy\#how-do-i-manage-trackers}{Your
  Ad Choices}
\item
  \href{https://www.nytimes3xbfgragh.onion/privacy}{Privacy}
\item
  \href{https://help.nytimes3xbfgragh.onion/hc/en-us/articles/115014893428-Terms-of-service}{Terms
  of Service}
\item
  \href{https://help.nytimes3xbfgragh.onion/hc/en-us/articles/115014893968-Terms-of-sale}{Terms
  of Sale}
\item
  \href{https://spiderbites.nytimes3xbfgragh.onion}{Site Map}
\item
  \href{https://help.nytimes3xbfgragh.onion/hc/en-us}{Help}
\item
  \href{https://www.nytimes3xbfgragh.onion/subscription?campaignId=37WXW}{Subscriptions}
\end{itemize}
