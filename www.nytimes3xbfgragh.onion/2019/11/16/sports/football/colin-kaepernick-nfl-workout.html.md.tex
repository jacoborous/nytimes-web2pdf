Sections

SEARCH

\protect\hyperlink{site-content}{Skip to
content}\protect\hyperlink{site-index}{Skip to site index}

\href{https://www.nytimes3xbfgragh.onion/section/sports/football}{Pro
Football}

\href{https://myaccount.nytimes3xbfgragh.onion/auth/login?response_type=cookie\&client_id=vi}{}

\href{https://www.nytimes3xbfgragh.onion/section/todayspaper}{Today's
Paper}

\href{/section/sports/football}{Pro Football}\textbar{}Colin
Kaepernick's Workout Derailed by Dispute With N.F.L.

\url{https://nyti.ms/359xxxC}

\begin{itemize}
\item
\item
\item
\item
\item
\end{itemize}

Advertisement

\protect\hyperlink{after-top}{Continue reading the main story}

Supported by

\protect\hyperlink{after-sponsor}{Continue reading the main story}

\hypertarget{colin-kaepernicks-workout-derailed-by-dispute-with-nfl}{%
\section{Colin Kaepernick's Workout Derailed by Dispute With
N.F.L.}\label{colin-kaepernicks-workout-derailed-by-dispute-with-nfl}}

Kaepernick moved his tryout at the last moment amid a disagreement with
the N.F.L. over media access. Few team scouts followed.

\includegraphics{https://static01.graylady3jvrrxbe.onion/images/2019/11/16/us/16Kapworkout2/merlin_164521881_a5e9a193-4cf2-466b-8b81-2add2b37d5c5-articleLarge.jpg?quality=75\&auto=webp\&disable=upscale}

\href{https://www.nytimes3xbfgragh.onion/by/ken-belson}{\includegraphics{https://static01.graylady3jvrrxbe.onion/images/2018/02/16/multimedia/author-ken-belson/author-ken-belson-thumbLarge.jpg}}

By \href{https://www.nytimes3xbfgragh.onion/by/ken-belson}{Ken Belson}

\begin{itemize}
\item
  Nov. 16, 2019
\item
  \begin{itemize}
  \item
  \item
  \item
  \item
  \item
  \end{itemize}
\end{itemize}

RIVERDALE, Ga. --- Colin Kaepernick's long journey from Super Bowl
quarterback to N.F.L. exile to media machine made a pit stop at Charles
R. Drew High School south of Atlanta on Saturday.

There, in shorts and a black tank top, the former San Francisco 49ers
star zipped passes to four receivers running routes on the football
field. Eight N.F.L. scouts looked on, one of them later calling the
performance ``impressive.'' As the sun faded and the temperature
dropped, a couple hundred people stood along a chain-link fence behind
one end zone and cheered.

``Stay focused, man! We believe in you!'' one fan yelled.

The question these days is who to believe. Part N.F.L. tryout, part
public rally, part media circus, the impromptu workout was put together
in a few hours after a
\href{https://www.nytimes3xbfgragh.onion/2019/11/13/sports/football/colin-kaepernick-workout.html}{contentious
week of negotiations} between the quarterback and the league. It was the
latest twist in a showdown that has captivated the sports world since
Kaepernick began kneeling during the national anthem at the start of the
2016 season.

After signing autographs for fans, many wearing his 49ers jersey,
Kaepernick returned to the field and spoke to the media for the first
time in years.

``I've been prepared for three years, I've been denied for three years
and you all know why,'' Kaepernick said. ``I've been ready. I'm staying
ready.''

Then he piled into a van and drove off, leaving as many, if not more,
questions about his future as there were when he began the day. Some of
the scouts appeared impressed, but does any team want the attention that
is likely to come with signing Kaepernick? His arm looked strong, but at
this point in the season, which team needs him? And on and on.

As with much of the quarterback's recent interactions with the league,
Saturday's workout was the result of bad blood poured on bad blood. On
Tuesday, the league called Kaepernick's agent and gave him
\href{https://www.nytimes3xbfgragh.onion/2019/11/12/sports/football/colin-kaepernick-nfl-workout.html}{two
hours to accept} an offer for Kaepernick to work out at the Falcons'
training facility an hour north of Atlanta in front of all 32 teams ---
the same 32 teams that have declined to invite him to work out for the
past three seasons.

Commissioner Roger Goodell, egged on by Jay-Z, the music impresario who
is now advising the league on social justice issues,
\href{https://www.nytimes3xbfgragh.onion/2019/11/15/sports/football/kaepernick-jay-z.html}{offered
the olive branch} to Kaepernick so the teams could see for themselves
whether the 32-year-old quarterback still has the skills to play in the
N.F.L.

Nike, which works with both Kaepernick and the N.F.L., was all set to
run an advertisement featuring the quarterback. Unlike the ``Dream
Crazy'' commercial that ran
\href{https://mailtrack.io/trace/link/2475e9e4187c233e10656adb8279299471bfd1b1?notrack=1\&url=https\%3A\%2F\%2Fwww.nytimes3xbfgragh.onion\%2F2018\%2F09\%2F05\%2Fsports\%2Fcolin-kaepernick-nike.html\&userId=3621650\&signature=cd85d3a79202475f}{during
the opening telecast} of last season, the N.F.L. approved this one.

The ad did not run on Saturday afternoon as planned.

The distrust between the league and Kaepernick is profound and perhaps
irreversible. In 2017, Kaepernick accused the league of blackballing him
because of his decision to kneel during the national anthem. The two
sides settled their differences when the N.F.L. paid Kaepernick a
\href{https://www.nytimes3xbfgragh.onion/2019/03/21/sports/colin-kaepernick-nfl-settlement.html}{multimillion-dollar
settlement} in February.

The legal matter settled, teams were freed up to call him. Yet no teams
did.

Then, out of the blue, the league offered Kaepernick a one-time chance
to show his stuff.

``Something didn't smell right,'' Jeff Nalley, Kaepernick's agent, said
after the workout.

The two sides fought over whether Kaepernick would be provided a list of
which N.F.L. personnel would be present. They tussled over whether media
would be allowed to watch, and over whether Kaepernick could bring his
own film crew to the workout.

\includegraphics{https://static01.graylady3jvrrxbe.onion/images/2019/11/16/us/16kap-workout5/16kap-workout5-articleLarge.jpg?quality=75\&auto=webp\&disable=upscale}

The final straw was, as it is with so many things involving the N.F.L.,
a legal affair. Kaepernick and the four receivers signed standard
waivers that indemnified the league if they got injured. The N.F.L. sent
back a far longer form with a number of other restrictions. Kaepernick's
lawyers rejected what they called an ``unusual liability waiver'' as a
precondition.

At 2:30 p.m. Eastern time, with about two dozen scouts waiting at the
Falcons' facility, Kaepernick announced that the workout would be moved
to a high school an hour away. Many scouts threw up their arms and
headed straight to the airport. Dozens of reporters and cameramen drove
south to the high school field.

Kaepernick arrived around 4:10 p.m. and drove onto the field of the
outdoor stadium. The high school, just outside Atlanta, is about 60
miles southwest of the Falcons' facility.

He came out and stretched, joked with friends and hugged his former
teammate Eric Reid, who
\href{https://www.nytimes3xbfgragh.onion/2017/09/25/opinion/colin-kaepernick-football-protests.html}{knelt
with Kaepernick} when they played for the 49ers. Reid plays for the
Carolina Panthers.

While the vibe at the high school was friendly, with cheers and
encouragement, the crowd in front of the Falcons' facility earlier on
Saturday was mixed,
\href{https://www.nytimes3xbfgragh.onion/2019/11/15/sports/football/kaepernick-jay-z.html}{with
protesters} on both sides of the main entrance to the Falcons' practice
facility.

On one side, Jim McIntyre stood with a wooden sign that said ``Stand Up
For the Flag'' that he and his wife made the night before. McIntyre, who
lives in town, said he supported Kaepernick's right to free speech but
believed his form of protest was disrespectful to the flag, despite
Kaepernick's assertions to the contrary.

McIntyre said he stopped watching N.F.L. games when Kaepernick and other
players began kneeling because it made him uncomfortable. ``I really
wish the N.F.L. would have a policy to make players stand for the
anthem,'' he said.

On the other side of the entrance, Scott Brooks sat in a lawn chair
holding a handwritten sign that read ``I'm With Kap.'' Brooks drove two
hours from Tennessee to show his support. Wearing Kaepernick's red 49ers
jersey, he said that he agreed with the quarterback's goal of raising
awareness of police brutality against African-Americans.

``I hate to see it overshadowed'' by the controversy over his decision
to kneel, Brooks said. ``He lost his job for not even committing a
crime.''

Image

The crowd in front of the Falcons' facility on Saturday was of mixed
opinion, with protesters on both sides of the main entrance to the
Falcons' practice facility.Credit...Todd Kirkland/Associated Press

About two dozen alumni of the Kappa Alpha Psi fraternity also arrived to
support Kaepernick, who was a member in college. They said they admired
his willingness to kneel to shine a light on police brutality against
African-Americans.

Some drivers slowed down and honked support, though it was often unclear
who they were supporting.

Later at the high school, the only ambiguity was where Kaepernick will
go next. Nalley, his agent, said he did not expect many clubs to call.

``I hope so, but to be honest, I'm a little pessimistic because I've
talked to all 32 teams'' already, he said, and none have offered
Kaepernick a tryout.

Kaepernick added his own coda that suggests that Saturday's tryout is
only one chapter in an ongoing tug-of-war.

``I'm ready to go anywhere,'' he said. ``The ball's in their court.
We're ready to go.''

\emph{Kevin Draper contributed reporting from New York.}

Advertisement

\protect\hyperlink{after-bottom}{Continue reading the main story}

\hypertarget{site-index}{%
\subsection{Site Index}\label{site-index}}

\hypertarget{site-information-navigation}{%
\subsection{Site Information
Navigation}\label{site-information-navigation}}

\begin{itemize}
\tightlist
\item
  \href{https://help.nytimes3xbfgragh.onion/hc/en-us/articles/115014792127-Copyright-notice}{©~2020~The
  New York Times Company}
\end{itemize}

\begin{itemize}
\tightlist
\item
  \href{https://www.nytco.com/}{NYTCo}
\item
  \href{https://help.nytimes3xbfgragh.onion/hc/en-us/articles/115015385887-Contact-Us}{Contact
  Us}
\item
  \href{https://www.nytco.com/careers/}{Work with us}
\item
  \href{https://nytmediakit.com/}{Advertise}
\item
  \href{http://www.tbrandstudio.com/}{T Brand Studio}
\item
  \href{https://www.nytimes3xbfgragh.onion/privacy/cookie-policy\#how-do-i-manage-trackers}{Your
  Ad Choices}
\item
  \href{https://www.nytimes3xbfgragh.onion/privacy}{Privacy}
\item
  \href{https://help.nytimes3xbfgragh.onion/hc/en-us/articles/115014893428-Terms-of-service}{Terms
  of Service}
\item
  \href{https://help.nytimes3xbfgragh.onion/hc/en-us/articles/115014893968-Terms-of-sale}{Terms
  of Sale}
\item
  \href{https://spiderbites.nytimes3xbfgragh.onion}{Site Map}
\item
  \href{https://help.nytimes3xbfgragh.onion/hc/en-us}{Help}
\item
  \href{https://www.nytimes3xbfgragh.onion/subscription?campaignId=37WXW}{Subscriptions}
\end{itemize}
