Sections

SEARCH

\protect\hyperlink{site-content}{Skip to
content}\protect\hyperlink{site-index}{Skip to site index}

\href{https://myaccount.nytimes3xbfgragh.onion/auth/login?response_type=cookie\&client_id=vi}{}

\href{https://www.nytimes3xbfgragh.onion/section/todayspaper}{Today's
Paper}

\href{/section/opinion}{Opinion}\textbar{}We Need Walter Mercado's
Optimism

\url{https://nyti.ms/2oLHFx0}

\begin{itemize}
\item
\item
\item
\item
\item
\end{itemize}

Advertisement

\protect\hyperlink{after-top}{Continue reading the main story}

\href{/section/opinion}{Opinion}

Supported by

\protect\hyperlink{after-sponsor}{Continue reading the main story}

\hypertarget{we-need-walter-mercados-optimism}{%
\section{We Need Walter Mercado's
Optimism}\label{we-need-walter-mercados-optimism}}

The Puerto Rican astrologer's affirmations brought the Latinx community
together and made me, an awkward teenager, feel less alone.

By Isvett Verde

Ms. Verde is an editor with the Opinion section.

\begin{itemize}
\item
  Nov. 5, 2019
\item
  \begin{itemize}
  \item
  \item
  \item
  \item
  \item
  \end{itemize}
\end{itemize}

\includegraphics{https://static01.graylady3jvrrxbe.onion/images/2019/11/08/opinion/05verde/05verde-articleLarge.jpg?quality=75\&auto=webp\&disable=upscale}

\href{https://www.nytimes3xbfgragh.onion/es/2019/11/06/espanol/opinion/walter-mercado.html}{Leer
en español}

I was transported back to my childhood in Florida when news of the death
of
\href{https://www.nytimes3xbfgragh.onion/2019/11/03/world/americas/walter-mercado-dead.html}{Walter
Mercado}, a beloved Puerto Rican astrologer, swept across the internet
on Sunday. Suddenly I could hear my father's radio alarm clock cutting
into the morning silence and the walls of our bedrooms at full volume,
jolting the household awake at what felt like an ungodly hour.

I saw myself moving through my morning rituals --- brushing my teeth,
getting dressed for school, eating breakfast --- to the soundtrack of a
morning talk show that consisted mostly of two men arguing about Cuban
politics, with Mr. Mercado's horoscope predictions sprinkled between
segments.

I'd wait for him to make his way through the zodiac to my sign, Cancer.
``Your friends will multiply today, and Mars will ignite your house of
passion,'' he'd say. I'd pile into my mom's car with gusto, ready to
conquer the world. If you're Latinx and grew up between the 1970s and
'90s, Mr. Mercado was most likely a fixture in your home, too.

Surely there have been other Latin American astrologers, but none as
revered or fabulous. He defied categorization. ``He was our Oprah, Mr.
Rogers, Liberace and spiritual adviser all rolled into one,'' said
Cristina Costantini, a co-director of a forthcoming documentary about
Mr. Mercado. His was a gentle, decidedly positive brand of astrology. In
those days we had only ourselves or the universe to blame for our poor
judgment and broken electronics --- not Mercury retrograde.

Mr. Mercado was born on March 9, a Pisces. But did his spirit come to
earth in Ponce, Puerto Rico, or at sea on a ship that was making its way
to the island from Spain? Was he 87 or 88 years old when he died? Did he
just happen to fall into astrology because he was at the right place at
the right time, or was he born with \emph{the gift}? The details were
not important. What is certain is that this seemingly otherworldly,
mystical being emerged from an unlikely place: rural Puerto Rico.

On his \href{https://www.youtube.com/watch?v=gRmNxqZWl-Q}{call-in show}
on the Psychic Friends Network, he'd
\href{https://www.youtube.com/watch?v=kMW9BnjILpU\&t=111s}{prescribe
baths} with champagne, red wine, crystals and more to attract luck in
love. He'd urge viewers to ``saturate yourself in love,'' to celebrate
and accept yourself just as you are. It was a message I suspect hit home
for many an awkward Latinx teenager who, like me, was straddling two
worlds desperately trying to figure out how she fit in both. By the
'90s, our family was among the
\href{https://www.historymiami.org/exhibition/walter-mercado/}{estimated
120 million people a day} who'd sit in rapt attention in front of the
television when he'd shout out our sign while flapping his cape for
extra drama.

Mr. Mercado never identified as gay, but it was the first time we saw
someone who defied gender norms on TV. David Gonzalez, a fellow
Cuban-American writer and friend, remarked: ``I think our families
instinctively knew that sexuality is a spectrum. They just didn't want
their kids being the outliers.'' A unifying figure, so colorful and
flashy, he transcended all the rules the Latinx community ascribed to
gender.

We connected with him because his message of hope was a salve for those
struggling to find their footing in a foreign country that at times felt
inhospitable. Hearing tomorrow was going to be a better day, believe in
yourself and be strong no matter what life threw at you reverberated
deeply in our community. His death has stirred a collective nostalgia in
the Latinx psyche, a longing for the days when our grandmothers, who
lovingly watched him along with us, were still around. It feels like now
more than ever we need his optimism. We need his voice urging us to keep
our heads up, despite how grim things seem.

In the past few years, he had retreated to his home in Puerto Rico with
his dog, Runo, and wasn't as prominent a presence as he once was. But it
gave me comfort to see his name pop up on my Facebook feed right before
the summer solstice, urging me, a child of the moon, to control my
emotions and embrace change. He appears to have left
``\href{https://www.miamiherald.com/news/nation-world/world/americas/article236994414.html}{a
trove of horoscopes ready for future publication}'' to tide us over for
now. But I already miss him.

Isvett Verde
(\href{https://twitter.com/isvettverde?lang=en}{@isvettverde}) is an
editor with the Opinion section.

\emph{The Times is committed to publishing}
\href{https://www.nytimes3xbfgragh.onion/2019/01/31/opinion/letters/letters-to-editor-new-york-times-women.html}{\emph{a
diversity of letters}} \emph{to the editor. We'd like to hear what you
think about this or any of our articles. Here are some}
\href{https://help.nytimes3xbfgragh.onion/hc/en-us/articles/115014925288-How-to-submit-a-letter-to-the-editor}{\emph{tips}}\emph{.
And here's our email:}
\href{mailto:letters@NYTimes.com}{\emph{letters@NYTimes.com}}\emph{.}

\emph{Follow The New York Times Opinion section on}
\href{https://www.facebookcorewwwi.onion/nytopinion}{\emph{Facebook}}\emph{,}
\href{http://twitter.com/NYTOpinion}{\emph{Twitter (@NYTopinion)}}
\emph{and}
\href{https://www.instagram.com/nytopinion/}{\emph{Instagram}}\emph{.}

Advertisement

\protect\hyperlink{after-bottom}{Continue reading the main story}

\hypertarget{site-index}{%
\subsection{Site Index}\label{site-index}}

\hypertarget{site-information-navigation}{%
\subsection{Site Information
Navigation}\label{site-information-navigation}}

\begin{itemize}
\tightlist
\item
  \href{https://help.nytimes3xbfgragh.onion/hc/en-us/articles/115014792127-Copyright-notice}{©~2020~The
  New York Times Company}
\end{itemize}

\begin{itemize}
\tightlist
\item
  \href{https://www.nytco.com/}{NYTCo}
\item
  \href{https://help.nytimes3xbfgragh.onion/hc/en-us/articles/115015385887-Contact-Us}{Contact
  Us}
\item
  \href{https://www.nytco.com/careers/}{Work with us}
\item
  \href{https://nytmediakit.com/}{Advertise}
\item
  \href{http://www.tbrandstudio.com/}{T Brand Studio}
\item
  \href{https://www.nytimes3xbfgragh.onion/privacy/cookie-policy\#how-do-i-manage-trackers}{Your
  Ad Choices}
\item
  \href{https://www.nytimes3xbfgragh.onion/privacy}{Privacy}
\item
  \href{https://help.nytimes3xbfgragh.onion/hc/en-us/articles/115014893428-Terms-of-service}{Terms
  of Service}
\item
  \href{https://help.nytimes3xbfgragh.onion/hc/en-us/articles/115014893968-Terms-of-sale}{Terms
  of Sale}
\item
  \href{https://spiderbites.nytimes3xbfgragh.onion}{Site Map}
\item
  \href{https://help.nytimes3xbfgragh.onion/hc/en-us}{Help}
\item
  \href{https://www.nytimes3xbfgragh.onion/subscription?campaignId=37WXW}{Subscriptions}
\end{itemize}
