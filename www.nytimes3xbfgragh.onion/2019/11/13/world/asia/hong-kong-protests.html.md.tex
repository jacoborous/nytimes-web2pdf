Sections

SEARCH

\protect\hyperlink{site-content}{Skip to
content}\protect\hyperlink{site-index}{Skip to site index}

\href{https://www.nytimes3xbfgragh.onion/section/world/asia}{Asia
Pacific}

\href{https://myaccount.nytimes3xbfgragh.onion/auth/login?response_type=cookie\&client_id=vi}{}

\href{https://www.nytimes3xbfgragh.onion/section/todayspaper}{Today's
Paper}

\href{/section/world/asia}{Asia Pacific}\textbar{}Why Are People
Protesting in Hong Kong?

\url{https://nyti.ms/33KNji7}

\begin{itemize}
\item
\item
\item
\item
\item
\end{itemize}

Advertisement

\protect\hyperlink{after-top}{Continue reading the main story}

Supported by

\protect\hyperlink{after-sponsor}{Continue reading the main story}

\hypertarget{why-are-people-protesting-in-hong-kong}{%
\section{Why Are People Protesting in Hong
Kong?}\label{why-are-people-protesting-in-hong-kong}}

A movement that began peacefully six months ago has descended into
chaos.

\includegraphics{https://static01.graylady3jvrrxbe.onion/images/2019/11/13/world/13hk-explainer-1/merlin_163706775_2b6049b5-f6e4-4668-a78a-71483c2cd65e-articleLarge.jpg?quality=75\&auto=webp\&disable=upscale}

\href{https://www.nytimes3xbfgragh.onion/by/daniel-victor}{\includegraphics{https://static01.graylady3jvrrxbe.onion/images/2018/06/14/multimedia/author-daniel-victor/author-daniel-victor-thumbLarge.png}}

By \href{https://www.nytimes3xbfgragh.onion/by/daniel-victor}{Daniel
Victor}

\begin{itemize}
\item
  Published Nov. 13, 2019Updated May 18, 2020
\item
  \begin{itemize}
  \item
  \item
  \item
  \item
  \item
  \end{itemize}
\end{itemize}

\href{https://cn.nytimes3xbfgragh.onion/china/20191114/hong-kong-protests/}{阅读简体中文版}\href{https://cn.nytimes3xbfgragh.onion/china/20191114/hong-kong-protests/zh-hant/}{閱讀繁體中文版}

HONG KONG --- What started in June with
\href{https://www.nytimes3xbfgragh.onion/2019/06/09/world/asia/hong-kong-extradition-protest.html}{peaceful
rallies} in opposition to
\href{https://www.nytimes3xbfgragh.onion/2019/06/10/world/asia/hong-kong-extradition-bill.html}{contentious
legislation} has devolved into a steady stream of mayhem, with some
protesters embracing violent behaviors in response to
\href{https://www.nytimes3xbfgragh.onion/2019/09/22/world/hong-kong-police-protests.html}{brutal
police tactics}.

The protesters and the
\href{https://www.nytimes3xbfgragh.onion/2020/07/01/world/asia/hong-kong-purple-flag.html}{Hong
Kong} government, backed by Beijing, appear to have intractable
differences, and there have been few signs of either side backing down.

In the months since the demonstrations began, the unpopular bill, which
would have allowed extraditions to mainland China, has been withdrawn,
but the protesters' demands have expanded to include increased democracy
and an investigation of the police.

The standoff has taken on international importance. China has viewed the
protests as a challenge to its fervent nationalism, while democracy
supporters worldwide have
\href{https://www.nytimes3xbfgragh.onion/2019/11/03/world/asia/hong-kong-protesters-call-for-us-help-china-sees-a-conspiracy.html}{cheered
what they see as a poke in the eye} of the autocratic Chinese
government. It all comes amid a rancorous trade war between China and
the United States, and some international businesses,
\href{https://www.nytimes3xbfgragh.onion/2019/10/07/sports/basketball/nba-china-hong-kong.html}{including
the N.B.A.}, have found themselves stuck in a political mess they wanted
no part of.

Here's a guide to how we got here, and why violence on both sides has
escalated.

\subsubsection{}

\begin{itemize}
\tightlist
\item
  \protect\hyperlink{link-58d22028}{What is Hong Kong's relationship
  with China?}
\item
  \protect\hyperlink{link-334754bd}{What do protesters want?}
\item
  \protect\hyperlink{link-5173e263}{Why have the demonstrations turned
  violent?}
\item
  \protect\hyperlink{link-44447ff1}{How does it end?}
\end{itemize}

\hypertarget{what-is-hong-kongs-relationship-with-china}{%
\subsection{What is Hong Kong's relationship with
China?}\label{what-is-hong-kongs-relationship-with-china}}

\href{https://www.nytimes3xbfgragh.onion/2020/05/18/world/asia/hong-kong-protests-fight-legco.html}{Hong
Kong}, an international finance hub on China's southern coast with more
than seven million residents, was a British colony until 1997, when it
was handed back to China under a policy known as
\href{https://www.nytimes3xbfgragh.onion/2014/09/30/world/asia/the-hong-kong-protests-what-you-should-know.html?module=inline}{one
country, two systems}.

The policy made Hong Kong part of China but let it keep many liberties
denied to citizens on the mainland, including free speech, unrestricted
internet access and the right to free assembly. The territory has its
own laws, system of government and police force under a
mini-constitution known as the Basic Law.
\href{https://www.nytimes3xbfgragh.onion/2019/07/01/world/asia/hong-kong-china-handover.html?module=inline}{China
promised that this system would remain in place} until at least 2047.

But many Hong Kongers feel that Beijing is already
\href{https://sinosphere.blogs.nytimes3xbfgragh.onion/2014/06/11/beijings-white-paper-sets-off-a-firestorm-in-hong-kong/}{chipping
away at the city's autonomy}, and that the local government does its
bidding. The territory's top leader, the chief executive --- currently
Carrie Lam --- is appointed by a pro-Beijing committee.

\hypertarget{what-do-protesters-want}{%
\subsection{What do protesters want?}\label{what-do-protesters-want}}

Many people in the territory feel deep contempt for the Chinese
government, and hope to preserve their freedoms for as long as possible.
At its core, the movement is aimed at resisting encroachment from the
mainland --- but it has been complicated by rising violence.

At first, the movement was focused on a bill,
\href{https://www.nytimes3xbfgragh.onion/2019/09/04/world/asia/hong-kong-carrie-lam-protests.html}{since
scrapped}, that would have allowed people accused of crimes to be sent
to places with which Hong Kong had no extradition treaty ---
\href{https://www.nytimes3xbfgragh.onion/2019/06/07/world/asia/hong-kong-china-extradition-protest.html?searchResultPosition=8\&module=inline}{including
mainland China}, where the courts are controlled by the Communist Party.
Hundreds of thousands of people, fearing the bill would allow Beijing to
target dissidents in Hong Kong with phony charges, joined
\href{https://www.nytimes3xbfgragh.onion/2019/06/09/world/asia/hong-kong-extradition-protest.html}{a
peaceful march} to oppose the bill on June 9.

\href{https://www.nytimes3xbfgragh.onion/video/world/asia/100000006602584/hong-kong-police-protest-video-investigation.html?rref=collection\%2Fbyline\%2Fbarbara-marcolini\&action=click\&contentCollection=undefined\&region=stream\&module=stream_unit\&version=latest\&contentPlacement=9\&pgtype=collection}{On
June 12}, for the first time, the police used pepper spray, batons and
more than 150 canisters of tear gas to
\href{https://www.nytimes3xbfgragh.onion/2019/06/12/world/asia/hong-kong-extradition-protest.html?searchResultPosition=8\&module=inline}{disperse
thousands of protesters}, a small number of whom had thrown projectiles
at the police. Irate at the police response, protesters demanded an
independent investigation of the police force --- a demand leaders have
refused.

While
\href{https://www.nytimes3xbfgragh.onion/video/world/asia/100000006702862/hong-kong-protests-police-officers.html?rref=collection\%2Fbyline\%2Fbarbara-marcolini}{anger
at the police} has been a driving force, protesters have extended their
demands to include amnesty for arrested participants and direct
elections for all lawmakers and the chief executive. The movement is
largely leaderless, with decisions frequently made through
\href{https://www.nytimes3xbfgragh.onion/interactive/2019/06/28/world/asia/hong-kong-protests.html}{voting
in online forums}.

\href{https://www.nytimes3xbfgragh.onion/interactive/2019/world/asia/hong-kong-protests-arc.html}{}

\includegraphics{https://static01.graylady3jvrrxbe.onion/images/2019/10/01/world/asia/Sequence-06/Sequence-06-articleLarge.jpg}

\hypertarget{six-months-of-hong-kong-protests-how-did-we-get-here}{%
\subsection{Six Months of Hong Kong Protests. How Did We Get
Here?}\label{six-months-of-hong-kong-protests-how-did-we-get-here}}

The protests started as peaceful marches and rallies against an
unpopular bill. Then came dozens of rounds of tear gas and a government
that refused to back down.

\hypertarget{why-have-the-demonstrations-turned-violent}{%
\subsection{Why have the demonstrations turned
violent?}\label{why-have-the-demonstrations-turned-violent}}

A vast majority of participants have been nonviolent, staging strikes,
surrounding police stations,
\href{https://www.nytimes3xbfgragh.onion/2019/08/12/world/asia/hong-kong-airport-protest.html?module=inline}{shutting
down the airport} and forming huge marches. The city's creative class
has turned protest into
\href{https://www.nytimes3xbfgragh.onion/2019/10/11/world/asia/hong-kong-protest-art.html?module=inline}{art}
and
\href{https://www.nytimes3xbfgragh.onion/2019/09/12/world/asia/glory-to-hong-kong-anthem.html?module=inline}{song}.

But a minority of protesters has become increasingly destructive, hoping
to force the government's hand. Since only one of the protesters'
demands has been met --- the withdrawal of the extradition bill --- the
more violent participants felt peaceful rallies were ineffective.

Some protesters have thrown bricks and Molotov cocktails, and in one
case
\href{https://www.nytimes3xbfgragh.onion/2019/10/13/world/asia/hong-kong-protests-face-masks.html?module=inline}{stabbed
a police officer}. The police say that one homemade bomb
\href{https://www.nytimes3xbfgragh.onion/2019/10/14/world/asia/hong-kong-bomb-ied.html?module=inline}{was
detonated during a protest}. On several occasions, protesters have doled
out vigilante justice, beating people who were perceived to be against
their movement, including one man who was
\href{https://www.nytimes3xbfgragh.onion/2019/11/11/world/hong-kong-protests.html}{doused
with fluid and set on fire}. A firebrand pro-Beijing lawmaker, Junius
Ho, was
\href{https://www.nytimes3xbfgragh.onion/2019/11/05/world/asia/junius-ho-stabbed-hong-kong.html}{attacked
with a knife}. There has been considerable property damage to the train
system, which protesters have accused of supporting the police, and
businesses seen as pro-China.

Protesters say the violence is in defense of excessively violent police
tactics. Officers have liberally deployed tear gas in ways that defy
international standards, including firing canisters from a height and
using it in enclosed spaces. Several protesters have been
\href{https://www.nytimes3xbfgragh.onion/2019/11/10/world/asia/hong-kong-protests-general-strike.html}{shot
with live rounds}. Cannons shooting water laced with a corrosive blue
dye have become a routine presence at protests. Individual interactions
with the police that are captured on video, such as the use of pepper
spray against a pregnant woman and an officer on a motorcycle
\href{https://t.me/stuckwithyou/4553}{swerving into protesters}, have
infuriated demonstrators.

Passions were further inflamed in November when
\href{https://www.nytimes3xbfgragh.onion/2019/11/07/world/asia/hong-kong-protest-student-dies.html}{a
Hong Kong student died} after falling from a parking garage near
demonstrations, possibly the first death of the movement. The exact
circumstances surrounding his death remain unknown.

\href{https://www.nytimes3xbfgragh.onion/2019/11/11/world/asia/hong-kong-protests-shooting.html}{Standoffs
on university campuses} have resembled trench warfare, with officers
firing tear gas and protesters attacking the police with firebombs,
arrows and other projectiles.
\href{https://www.nytimes3xbfgragh.onion/2019/11/17/world/asia/hong-kong-protests-chinese-soldiers.html}{A
standoff at Hong Kong Polytechnic University in November} went on for
days, with protesters trapped as the police laid siege to the campus. A
police officer was hit in the leg with an arrow, while dozens of
protesters were injured or suffered from hypothermia after being hit by
a water cannon.

\hypertarget{how-does-it-end}{%
\subsection{How does it end?}\label{how-does-it-end}}

No one knows.

But despite some domestic propaganda showing
\href{https://www.nytimes3xbfgragh.onion/2019/08/19/world/asia/hong-kong-china-troops.html}{tanks
assembling across the border} in Beijing, it appears China is trying to
avoid a Tiananmen-style crackdown. Though the Chinese military has a
garrison in Hong Kong, the international business community would most
likely see a military intervention as the end of ``one country, two
systems,'' and an exodus of businesses could soon follow.

Still, China does not want to bend to the protesters, whom the state
news media have depicted as lawless, spoiled separatists. (Most
protesters say they are uninterested in independence.)

The Hong Kong government and the police have done little to calm
tensions, repeatedly denouncing the protesters while mostly defending
the police's conduct. And protesters have shown few signs of fatigue,
despite thousands of arrests.

Advertisement

\protect\hyperlink{after-bottom}{Continue reading the main story}

\hypertarget{site-index}{%
\subsection{Site Index}\label{site-index}}

\hypertarget{site-information-navigation}{%
\subsection{Site Information
Navigation}\label{site-information-navigation}}

\begin{itemize}
\tightlist
\item
  \href{https://help.nytimes3xbfgragh.onion/hc/en-us/articles/115014792127-Copyright-notice}{©~2020~The
  New York Times Company}
\end{itemize}

\begin{itemize}
\tightlist
\item
  \href{https://www.nytco.com/}{NYTCo}
\item
  \href{https://help.nytimes3xbfgragh.onion/hc/en-us/articles/115015385887-Contact-Us}{Contact
  Us}
\item
  \href{https://www.nytco.com/careers/}{Work with us}
\item
  \href{https://nytmediakit.com/}{Advertise}
\item
  \href{http://www.tbrandstudio.com/}{T Brand Studio}
\item
  \href{https://www.nytimes3xbfgragh.onion/privacy/cookie-policy\#how-do-i-manage-trackers}{Your
  Ad Choices}
\item
  \href{https://www.nytimes3xbfgragh.onion/privacy}{Privacy}
\item
  \href{https://help.nytimes3xbfgragh.onion/hc/en-us/articles/115014893428-Terms-of-service}{Terms
  of Service}
\item
  \href{https://help.nytimes3xbfgragh.onion/hc/en-us/articles/115014893968-Terms-of-sale}{Terms
  of Sale}
\item
  \href{https://spiderbites.nytimes3xbfgragh.onion}{Site Map}
\item
  \href{https://help.nytimes3xbfgragh.onion/hc/en-us}{Help}
\item
  \href{https://www.nytimes3xbfgragh.onion/subscription?campaignId=37WXW}{Subscriptions}
\end{itemize}
