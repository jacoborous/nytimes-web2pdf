Sections

SEARCH

\protect\hyperlink{site-content}{Skip to
content}\protect\hyperlink{site-index}{Skip to site index}

\href{https://www.nytimes3xbfgragh.onion/section/world/asia}{Asia
Pacific}

\href{https://myaccount.nytimes3xbfgragh.onion/auth/login?response_type=cookie\&client_id=vi}{}

\href{https://www.nytimes3xbfgragh.onion/section/todayspaper}{Today's
Paper}

\href{/section/world/asia}{Asia Pacific}\textbar{}Gotabaya Rajapaksa
Wins Sri Lanka Presidential Election

\url{https://nyti.ms/2Qpcl2E}

\begin{itemize}
\item
\item
\item
\item
\item
\end{itemize}

Advertisement

\protect\hyperlink{after-top}{Continue reading the main story}

Supported by

\protect\hyperlink{after-sponsor}{Continue reading the main story}

\hypertarget{gotabaya-rajapaksa-wins-sri-lanka-presidential-election}{%
\section{Gotabaya Rajapaksa Wins Sri Lanka Presidential
Election}\label{gotabaya-rajapaksa-wins-sri-lanka-presidential-election}}

Mr. Rajapaksa, a former defense chief and brother of an ex-president,
vowed to bring stability to a country still reeling from attacks on
Easter Sunday.

\includegraphics{https://static01.graylady3jvrrxbe.onion/images/2019/11/18/world/18srilanka/merlin_164484666_25b05d11-bfec-4faa-96fa-9a440a331d43-articleLarge.jpg?quality=75\&auto=webp\&disable=upscale}

By Dharisha Bastians and
\href{https://www.nytimes3xbfgragh.onion/by/kai-schultz}{Kai Schultz}

\begin{itemize}
\item
  Nov. 17, 2019
\item
  \begin{itemize}
  \item
  \item
  \item
  \item
  \item
  \end{itemize}
\end{itemize}

COLOMBO, Sri Lanka --- Gotabaya Rajapaksa declared victory on Sunday in
\href{https://www.nytimes3xbfgragh.onion/2019/11/16/world/asia/sri-lanka-election-rajapaksa.html?rref=collection\%2Fbyline\%2Fkai-schultz\&action=click\&contentCollection=undefined\&region=stream\&module=stream_unit\&version=latest\&contentPlacement=1\&pgtype=collection}{Sri
Lanka's presidential election}, signaling the return to power of a
divisive family credited for ending the country's long civil war through
brutal means.

Mr. Rajapaksa defeated his closest opponent, Sajith Premadasa, by about
10 percentage points, according to an
\href{https://twitter.com/AzzamAmeen/status/1196016908446752768}{official
tally} from Sri Lanka's election commission. His party expects him to be
sworn into office early this week.

``As we usher in a new journey for Sri Lanka, we must remember that all
Sri Lankans are part of this journey,'' Mr. Rajapaksa
\href{https://twitter.com/GotabayaR/status/1195949575166521346}{wrote on
Twitter} in his first remarks about the victory. ``Let us rejoice
peacefully, with dignity and discipline in the same manner in which we
campaigned.''

During the election, Mr. Rajapaksa, 70, a former wartime defense chief
nicknamed ``Terminator'' by his family, capitalized on public outrage at
the current government's
\href{https://www.nytimes3xbfgragh.onion/2019/04/22/world/asia/ntj-warning-sri-lanka-government.html?action=click\&module=RelatedLinks\&pgtype=Article}{mishandling
of intelligence reports} warning of terrorist attacks in Sri Lanka, a
lush island at the foot of India. In April, a Muslim militant group
claiming loyalty to the Islamic State
\href{https://www.nytimes3xbfgragh.onion/2019/04/21/world/asia/sri-lanka-bombings.html?rref=collection\%2Fbyline\%2Fkai-schultz\&module=inline}{killed
hundreds of people} in coordinated suicide bombings at churches and
hotels on Easter Sunday.

The attacks shattered a fragile postwar peace in Sri Lanka, where wounds
still fester from the war with separatist ethnic Tamils during which
thousands of people died. In 2009, Mr. Rajapaksa and his brother Mahinda
Rajapaksa, then Sri Lanka's president, ended that conflict, but they
stand accused of crimes against humanity, including directing the
bombings of civilian hospitals and torturing journalists.

During campaign speeches, Gotabaya Rajapaksa vowed to take a tough
stance on terrorism as president and to bring stability to Sri Lanka,
where a collapse in tourism after the bombings threw the economy into a
tailspin. Many Sri Lankans struggling to make ends meet support the
Rajapaksas in the hopes that they can revive the economy, which boomed
toward the end of their stretch in power.

Others worry that democracy and freedom of speech will be curtailed
under Mr. Rajapaksa, whose party, Sri Lanka Podujana Peramuna, is likely
to appoint Mahinda Rajapaksa as the new prime minister.

It is not the first time the former president has sought the position.
In October 2018, Sri Lanka's departing president, Maithripala Sirisena,
\href{https://www.nytimes3xbfgragh.onion/2018/10/26/world/asia/sri-lanka-political-crisis.html?module=inline}{abruptly
fired Prime Minister Ranil Wickremesinghe}, calling him inept and
corrupt, and then appointing Mahinda Rajapaksa to the position. Many
considered the move a coup, and by the time the power grab was ruled
illegal, two protesters had been killed.

But the Easter Sunday bombings and struggling economy were enough to
allow another Rajapaksa to rise. Asanga Welikala, the director of the
Edinburgh Center for Constitutional Law and an expert on Sri Lanka, said
that Gotabaya Rajapaksa won this election by leveraging the same
hard-line approach to national security and ``social discipline'' that
has propelled populists to power around the world.

``This is a mandate that rejects reform, democratization, civil freedom
and broad tolerance of pluralism,'' Mr. Welikala said of the vote.

Mahinda Rajapaksa's decade as president was known for tightly
centralized power and the spread of a strident Sinhalese Buddhist
nationalism that has inspired attacks against the country's large
minority communities. Hostility toward Muslims, in particular, has risen
since the Easter Sunday attacks, heightening fears of retribution
against innocent Sri Lankans.

And although the economy grew under Mahinda Rajapaksa, so did the
country's debt to China, whose influence in Sri Lankan affairs spiked
under the family's watch. A lopsided dependence on China for development
projects was a key part of
\href{https://www.nytimes3xbfgragh.onion/2015/01/09/world/asia/sri-lanka-election-president-mahinda-rajapaksa.html?module=inline}{Mahinda
Rajapaksa's startling election defeat in 2015}.

Since then, the country was forced to
\href{https://www.nytimes3xbfgragh.onion/2018/06/25/world/asia/china-sri-lanka-port.html?module=inline}{give
up a port complex to China} as the debt crisis battered the economy,
though analysts said Gotabaya Rajapaksa's government would most likely
tread carefully with China this time around.

``Gota will play the China card, but Beijing is now less inclined to
repeat the large financial investments it did five or 10 years ago due
to growing domestic opposition and international scrutiny,'' said
Constantino Xavier, a foreign policy fellow at Brookings India in New
Delhi.

Officials said turnout on Saturday, when 16 million eligible voters
chose among 35 candidates, was more than 80 percent, and that Mr.
Rajapaksa had won about 52 percent of the vote. The election was largely
peaceful, though some violence was reported, including an attack on
buses carrying Muslims to polling stations in northwest Sri Lanka.
(There were no reports of injuries.)

Gotabaya Rajapaksa's chief competitor, Mr. Premadasa, 52, whose father
was killed during the war by a Tamil Tiger rebel, did well with minority
Muslims and Tamils in conflict-torn northern and eastern provinces.

But Mr. Rajapaksa won big with a crucial voting bloc: Sinhalese
Buddhists who make up around 70 percent of the Sri Lankan population and
credit the Rajapaksa family for ending the war. Sunday morning, Mr.
Rajapaksa's supporters hugged and cheered outside his home on the
outskirts of Colombo, the capital.

In a statement conceding defeat, Mr. Premadasa, of the United National
Front, congratulated Mr. Rajapaksa, urging him to ``strengthen and
protect the democratic institutions and values that enabled his peaceful
election.''

Dharisha Bastians reported from Colombo, and Kai Schultz from New Delhi.

Advertisement

\protect\hyperlink{after-bottom}{Continue reading the main story}

\hypertarget{site-index}{%
\subsection{Site Index}\label{site-index}}

\hypertarget{site-information-navigation}{%
\subsection{Site Information
Navigation}\label{site-information-navigation}}

\begin{itemize}
\tightlist
\item
  \href{https://help.nytimes3xbfgragh.onion/hc/en-us/articles/115014792127-Copyright-notice}{©~2020~The
  New York Times Company}
\end{itemize}

\begin{itemize}
\tightlist
\item
  \href{https://www.nytco.com/}{NYTCo}
\item
  \href{https://help.nytimes3xbfgragh.onion/hc/en-us/articles/115015385887-Contact-Us}{Contact
  Us}
\item
  \href{https://www.nytco.com/careers/}{Work with us}
\item
  \href{https://nytmediakit.com/}{Advertise}
\item
  \href{http://www.tbrandstudio.com/}{T Brand Studio}
\item
  \href{https://www.nytimes3xbfgragh.onion/privacy/cookie-policy\#how-do-i-manage-trackers}{Your
  Ad Choices}
\item
  \href{https://www.nytimes3xbfgragh.onion/privacy}{Privacy}
\item
  \href{https://help.nytimes3xbfgragh.onion/hc/en-us/articles/115014893428-Terms-of-service}{Terms
  of Service}
\item
  \href{https://help.nytimes3xbfgragh.onion/hc/en-us/articles/115014893968-Terms-of-sale}{Terms
  of Sale}
\item
  \href{https://spiderbites.nytimes3xbfgragh.onion}{Site Map}
\item
  \href{https://help.nytimes3xbfgragh.onion/hc/en-us}{Help}
\item
  \href{https://www.nytimes3xbfgragh.onion/subscription?campaignId=37WXW}{Subscriptions}
\end{itemize}
