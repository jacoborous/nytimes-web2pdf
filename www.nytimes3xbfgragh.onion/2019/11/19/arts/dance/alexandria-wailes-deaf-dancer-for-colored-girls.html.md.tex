Sections

SEARCH

\protect\hyperlink{site-content}{Skip to
content}\protect\hyperlink{site-index}{Skip to site index}

\href{https://www.nytimes3xbfgragh.onion/section/arts/dance}{Dance}

\href{https://myaccount.nytimes3xbfgragh.onion/auth/login?response_type=cookie\&client_id=vi}{}

\href{https://www.nytimes3xbfgragh.onion/section/todayspaper}{Today's
Paper}

\href{/section/arts/dance}{Dance}\textbar{}My `Face Is My Voice': A Deaf
Dancer Lands Her Dream Role

\url{https://nyti.ms/32XxnYV}

\begin{itemize}
\item
\item
\item
\item
\item
\item
\end{itemize}

Advertisement

\protect\hyperlink{after-top}{Continue reading the main story}

Supported by

\protect\hyperlink{after-sponsor}{Continue reading the main story}

\hypertarget{my-face-is-my-voice-a-deaf-dancer-lands-her-dream-role}{%
\section{My `Face Is My Voice': A Deaf Dancer Lands Her Dream
Role}\label{my-face-is-my-voice-a-deaf-dancer-lands-her-dream-role}}

Alexandria Wailes deftly weaves choreography and American Sign Language
in Ntozake Shange's ``For Colored Girls.''

\includegraphics{https://static01.graylady3jvrrxbe.onion/images/2019/11/19/arts/19WAILES-A-SUB/merlin_164673279_af999187-189d-43b5-bb33-260735a3c7a8-articleLarge.jpg?quality=75\&auto=webp\&disable=upscale}

By \href{https://www.nytimes3xbfgragh.onion/by/gia-kourlas}{Gia Kourlas}

\begin{itemize}
\item
  Nov. 19, 2019
\item
  \begin{itemize}
  \item
  \item
  \item
  \item
  \item
  \item
  \end{itemize}
\end{itemize}

\href{http://www.alexandriawailes.com/}{Alexandria Wailes} has had a
cathartic, enlightening autumn. As the Lady in Purple in ``For Colored
Girls Who Have Considered Suicide/When the Rainbow Is Enuf,'' she
finally has a part that reflects her just the way she is: deaf, mixed
race and a dancer.

In
\href{https://www.nytimes3xbfgragh.onion/2019/10/09/arts/dance/for-colored-girls.html}{Ntozake
Shange's celebrated feminist choreopoem,} through Dec. 8 at the Public,
seven women of color, named after and dressed in different hues of the
rainbow, explore trauma and resilience through movement and text. Ms.
Wailes's performance is captivating for the ease in which she weaves
Camille A. Brown's choreography with American Sign Language.

``That was a challenge,'' Ms. Wailes said through an interpreter in an
interview. ``I didn't want it to become too movement-based so that the
language started to get lost.''

Dance runs deep in her body. Ms. Wailes, who became deaf after
contracting meningitis just after her first birthday, has danced nearly
all of her life. When she was around 3, a doctor suggested that she try
a class. It was a way, she said, ``to help me heal and deal with the
world.''

Even outside of the studio, dance has served her well. ``Dancing has
always been a way of communicating with people who didn't understand
me,'' she said. ``It was a way of breaking down barriers between
languages.''

``For Colored Girls'' features a series of monologues. In the Lady in
Purple's, Ms. Wailes narrates the story of a mixed-race dancer who
performs the role of an Egyptian goddess of love. The production's
director, Leah C. Gardiner, was impressed with Ms. Wailes's elegant
weaving of two visual languages, dance and A.S.L.

``I could request something,'' Ms. Gardiner said, and ``she would ask me
questions and then take that information and put that into her body and
translate that into A.S.L. in relationship to the text.''

Ms. Wailes worked with Onudeah Nicolarakis, who is credited as the
production's director of A.S.L., to focus on making sign and spoken
language work together, as well as imbuing the choreography with
expressiveness and nuance.

``She didn't just translate the words, she translated the experience and
emotion,'' Ms. Gardiner said. ``One of my favorite moments in the show
is when Alexandria turns upstage and is talking about how her hoop skirt
falls. She gestures at her bottom and pulling the skirt down, but she
does it kind of looking back. It's cheeky and expressive.''

The role was not originally for a deaf actress, but in casting, Ms.
Gardiner --- with Shange's approval --- wanted to broaden the idea of
what an African-American woman could be. She also had another ambition
in this production: To illuminate the idea of colorism, in which skin
tone --- whether lighter or darker --- can lead to favoritism and
discrimination within an ethnic group.

Ms. Wailes, who is half black, has a lighter complexion than the others
onstage. ``When Alexandria came in and auditioned there was, for me, the
excitement of, Wow, maybe my dream of exploring colorism can come
true,'' Ms. Gardiner said. ``And, oh my gosh, she's also she's deaf?
This is insane.''

``For Colored Girls'' had its premiere at the Public in 1976, the year
after Ms. Wailes was born. And that connection is meaningful to her.
``As a deaf woman of color who grew up dancing,'' she said she could see
herself in the role. As she put it, ``I felt like I needed to be doing
this show at this time.''

Recently Ms. Wailes spoke about bringing A.S.L. and movement together,
listening with her body and the freedom that dancing gives her. These
are edited excerpts from that conversation.

\includegraphics{https://static01.graylady3jvrrxbe.onion/images/2019/11/20/arts/19wailes-B/merlin_162314169_ff9abac2-6a1c-4346-9a56-fa48712022f8-articleLarge.jpg?quality=75\&auto=webp\&disable=upscale}

\textbf{How does dance break down barriers between languages?} ****

It is in the body when you learn to listen. And you learn to listen
differently as a dancer. Being deaf, we always use our eyes; it's so
critical for us to survive and to take in the world. So to bring in
dancing is just an automatic extension of my way of life as a deaf
person.

I was very lucky because in the last two years of high school, I
transferred to the Model Secondary School for the Deaf. It was during
the 1990s and they had a strong performing-arts program. I started to
meet other deaf dancers.

\textbf{What did that give you?}

More motivation and incentive to stay true to my path as an artist
considering I never had anyone to look up to. I had no role model. A
deaf woman of color? \emph{Dancing}?{[}Laughs{]} I had to go, O.K., if
that's not out there, I want to create it.

\textbf{How do you bring yourself to the character?}

This is me in my true element as an actor as a dancer. I have seven
siblings. We're all girls. There are seven women in the show, and I'm
the only deaf person in the cast \emph{and} in my family.

\textbf{Are you generally very expressive when you sign?}

Yes. I can turn it down and be less expressive. I've worked with {[}the
contemporary choreographer{]} Heidi Latsky for a few years. She had a
piece, ``Somewhere,'' inspired by different renditions of ``Over the
Rainbow.'' I told her that I wanted to challenge the notion of signing,
which often tends to look so very beautiful and pastoral and emotive and
expressive. I wanted to see signing used in an urban manner. I wanted it
to feel gritty, edgy and just bigger --- more like an attack in a
positive way, like in your face.

\textbf{How did she work with you?}

She worked with me on my expression. She said: ``Don't put it in your
face. It's not about putting on a show.'' I understood, but it was a
challenge because face is my voice. My expressions are my voice. But
over time, it was freeing, because I was focusing on bringing American
Sign Language and movement together.

\textbf{How do you find rhythm without hearing the words?}

Signing naturally has a lot of musicality within it. The challenge is
determining which signs best honor the length of the text, because in
Ntozake's text you often see the slashes or you see space or an
ampersand.

How do I embody this physical language on top of a dancing language on
top of trusting and working with the other actors who are speaking the
lines that Ntozake wrote? Because a big difference between speaking and
signing is I can keep signing, but you need to take a breath, right?
There is a difference in the way that breath is used in both languages.

\textbf{This production explores the notion of colorism. How has skin
tone affected you in your life?}

I didn't ask to look like this, do you know what I mean? {[}Laughs{]} So
I was really grateful for Leah. She just got me. She understood the
inherent challenges that I deal with in life because I'm always passing.
I'm not going to deny that that privilege is there.

But I've had to suppress who I am --- because the idea of who I am
doesn't match who I am inside. It's an interesting tension, and it's a
constant dynamic that I'm negotiating all the time.

\textbf{Do you know why you took to dance as a deaf person?}

Because it's another language. I learned sign language and dance around
the same time. Dance is a physical vocabulary and a way of
communicating. Sometimes it relies on sound --- but not always.

Advertisement

\protect\hyperlink{after-bottom}{Continue reading the main story}

\hypertarget{site-index}{%
\subsection{Site Index}\label{site-index}}

\hypertarget{site-information-navigation}{%
\subsection{Site Information
Navigation}\label{site-information-navigation}}

\begin{itemize}
\tightlist
\item
  \href{https://help.nytimes3xbfgragh.onion/hc/en-us/articles/115014792127-Copyright-notice}{©~2020~The
  New York Times Company}
\end{itemize}

\begin{itemize}
\tightlist
\item
  \href{https://www.nytco.com/}{NYTCo}
\item
  \href{https://help.nytimes3xbfgragh.onion/hc/en-us/articles/115015385887-Contact-Us}{Contact
  Us}
\item
  \href{https://www.nytco.com/careers/}{Work with us}
\item
  \href{https://nytmediakit.com/}{Advertise}
\item
  \href{http://www.tbrandstudio.com/}{T Brand Studio}
\item
  \href{https://www.nytimes3xbfgragh.onion/privacy/cookie-policy\#how-do-i-manage-trackers}{Your
  Ad Choices}
\item
  \href{https://www.nytimes3xbfgragh.onion/privacy}{Privacy}
\item
  \href{https://help.nytimes3xbfgragh.onion/hc/en-us/articles/115014893428-Terms-of-service}{Terms
  of Service}
\item
  \href{https://help.nytimes3xbfgragh.onion/hc/en-us/articles/115014893968-Terms-of-sale}{Terms
  of Sale}
\item
  \href{https://spiderbites.nytimes3xbfgragh.onion}{Site Map}
\item
  \href{https://help.nytimes3xbfgragh.onion/hc/en-us}{Help}
\item
  \href{https://www.nytimes3xbfgragh.onion/subscription?campaignId=37WXW}{Subscriptions}
\end{itemize}
