Sections

SEARCH

\protect\hyperlink{site-content}{Skip to
content}\protect\hyperlink{site-index}{Skip to site index}

\href{https://myaccount.nytimes3xbfgragh.onion/auth/login?response_type=cookie\&client_id=vi}{}

\href{https://www.nytimes3xbfgragh.onion/section/todayspaper}{Today's
Paper}

Letter of Recommendation: `Honesty Is Still in Style'

\url{https://nyti.ms/2TLLII3}

\begin{itemize}
\item
\item
\item
\item
\item
\end{itemize}

Advertisement

\protect\hyperlink{after-top}{Continue reading the main story}

Supported by

\protect\hyperlink{after-sponsor}{Continue reading the main story}

\href{/column/letter-of-recommendation}{Letter of Recommendation}

\hypertarget{letter-of-recommendation-honesty-is-still-in-style}{%
\section{Letter of Recommendation: `Honesty Is Still in
Style'}\label{letter-of-recommendation-honesty-is-still-in-style}}

\includegraphics{https://static01.graylady3jvrrxbe.onion/images/2019/03/11/magazine/0324Mag-LOR-1/0324Mag-LOR-1-articleLarge.jpg?quality=75\&auto=webp\&disable=upscale}

By Lauren Viera

\begin{itemize}
\item
  March 19, 2019
\item
  \begin{itemize}
  \item
  \item
  \item
  \item
  \item
  \end{itemize}
\end{itemize}

I can't remember when I first noticed the sign. It must have been
sometime in my mid-20s, after I moved to Logan Square in Chicago,
sinking my life's savings into a fixer-upper Victorian --- a consolation
prize after a major breakup. I would pass it when I cut through Humboldt
Park, driving to and from the bars where I'd hope to meet a rebound, or
speeding to work on rainy days. It didn't loom large on those drives,
exactly, but it was big enough to politely clear its throat and announce
its presence, at once a warning, a reminder and an adage: ``Honesty is
Still in Style.''

The sign was hand-painted, nailed to the side of a currency-exchange
building on the edge of the park. Its letters were hokey: bright red,
outlined in black, their kerning nonsensical. The words clung
self-consciously to a couple of whitewashed plywood boards, each
claiming its own line so that the dot of one line's i stuck to the
bottom of another line's n. The top lines read as if the painter was
tight on space, forced to cram letters together to make it all fit. But
toward the bottom, everything relaxed. The last word, Style, kind of
floated, confidently taking up more space than it needed. Even its y
hung low, breaking the sign's border.

The more often I passed the sign, the more I thought about it. For
better or worse, ``Honesty is still in style'' became my motto. In those
years, I was hellbent on finding a partner to share that ramshackle
Victorian with me, to fill its rooms with the life I thought I'd bought
with my nest egg (plus a pair of pre-Great Recession mortgages). I'd
meet someone; I'd be honest with them about what I thought I wanted; it
would end. When, eventually, I met my future husband on a blind date a
couple of blocks from the sign, honesty had cemented itself as a guiding
principle in my life. By then I was turning 30, the age at which OB-GYNs
begin to measure their responses to single female patients who express a
desire to possibly want kids someday. So on our third date, I was
honest: I told him I wasn't interested in dating anyone who didn't want
to ultimately marry and potentially have a baby. He concurred, we
high-fived and have been together ever since.

For whatever reason --- the conviction of its message, its lovably
rudimentary design --- that sign wove its way into my personal life, and
the lives of everyone I've encountered who remembers it, too. It's not
nostalgia. It's because the sign addressed a basic principle we seldom
talk about: this idea that humans are expected to communicate truthfully
with one another --- and that said connection is not only desirable,
it's also stylish, a trend of an emotional ideal. It's the word Still, I
think, that lends the phrase its power --- the idea that no matter how
life evolves, certain standards remain paramount. In hindsight, every
relationship I've pursued, every opportunity I've seized in the years
since I first saw it, has been heavy with honesty, sometimes to a fault.
Over a decade ago, real honesty laid the groundwork with my husband, and
I believe that same brutal frankness is what ultimately saved our
marriage when it waned. In conversations with my daughter, I emphasize
the value of being honest about how you feel, even though she's too
young to really get it. As I'm settling into life as a 40-year-old, I've
been making a point to be more honest with myself about everything, from
how many hours I work to how much sugar I consume, which has led to a
heightened self-awareness typically reserved for regulars at Esalen.

Eventually we moved out of the Victorian, and I didn't see the sign as
much. So last summer on a jog through the park, I tried to make it my
destination, but when I arrived at that familiar corner, the sign was
gone. It had come down with very little fanfare, and I was inexplicably
heartbroken. I started digging around in hopes of learning its
whereabouts, and found an Australian artist who, after a Chicago
sabbatical, produced a zine named for the sign. I talked to a local sign
painter who paid homage with a synonymous silk banner, hand-lettered in
a masonic-regalia style. Instagram led me to a local writer-illustrator
who had started a replica at some point. I stopped by a neighboring taco
shop and, as if investigating a missing person, flashed a grainy Google
Images photo to the owner, who offered a quiet reply from across the
counter: ``Yes, of course I remember it,'' he smiled. ``It means
everything.'' Finally, a few days later, I found Maria --- the woman who
originally curated the sign.

For 15 years, Maria and Jay Feldman owned the currency-exchange building
to which the sign was tacked, overlooking a small gravel parking lot. An
optimistic Maria had it installed to thwart off bad energy. And it
worked --- until it didn't. In 2015, a fraudulent check forced the
Feldmans to declare bankruptcy, causing them to lose everything: two
homes, five businesses, their cars, their credit. Jay, meanwhile, had
fallen victim to opiates. He had an affair. Divorce was discussed. When
it came time to decide whether the couple would stay together or split,
the sign became scripture. ``What had started off as an existential
question for the crooks,'' Maria told me, ``had weaved itself into our
personal lives.'' They're still married.

Advertisement

\protect\hyperlink{after-bottom}{Continue reading the main story}

\hypertarget{site-index}{%
\subsection{Site Index}\label{site-index}}

\hypertarget{site-information-navigation}{%
\subsection{Site Information
Navigation}\label{site-information-navigation}}

\begin{itemize}
\tightlist
\item
  \href{https://help.nytimes3xbfgragh.onion/hc/en-us/articles/115014792127-Copyright-notice}{©~2020~The
  New York Times Company}
\end{itemize}

\begin{itemize}
\tightlist
\item
  \href{https://www.nytco.com/}{NYTCo}
\item
  \href{https://help.nytimes3xbfgragh.onion/hc/en-us/articles/115015385887-Contact-Us}{Contact
  Us}
\item
  \href{https://www.nytco.com/careers/}{Work with us}
\item
  \href{https://nytmediakit.com/}{Advertise}
\item
  \href{http://www.tbrandstudio.com/}{T Brand Studio}
\item
  \href{https://www.nytimes3xbfgragh.onion/privacy/cookie-policy\#how-do-i-manage-trackers}{Your
  Ad Choices}
\item
  \href{https://www.nytimes3xbfgragh.onion/privacy}{Privacy}
\item
  \href{https://help.nytimes3xbfgragh.onion/hc/en-us/articles/115014893428-Terms-of-service}{Terms
  of Service}
\item
  \href{https://help.nytimes3xbfgragh.onion/hc/en-us/articles/115014893968-Terms-of-sale}{Terms
  of Sale}
\item
  \href{https://spiderbites.nytimes3xbfgragh.onion}{Site Map}
\item
  \href{https://help.nytimes3xbfgragh.onion/hc/en-us}{Help}
\item
  \href{https://www.nytimes3xbfgragh.onion/subscription?campaignId=37WXW}{Subscriptions}
\end{itemize}
