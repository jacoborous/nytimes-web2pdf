Sections

SEARCH

\protect\hyperlink{site-content}{Skip to
content}\protect\hyperlink{site-index}{Skip to site index}

\href{https://myaccount.nytimes3xbfgragh.onion/auth/login?response_type=cookie\&client_id=vi}{}

\href{https://www.nytimes3xbfgragh.onion/section/todayspaper}{Today's
Paper}

A Dating Show Made for the Age of Apps

\begin{itemize}
\item
\item
\item
\item
\item
\item
\end{itemize}

Advertisement

\protect\hyperlink{after-top}{Continue reading the main story}

Supported by

\protect\hyperlink{after-sponsor}{Continue reading the main story}

\href{/column/screenland}{Screenland}

\hypertarget{a-dating-show-made-for-the-age-of-apps}{%
\section{A Dating Show Made for the Age of
Apps}\label{a-dating-show-made-for-the-age-of-apps}}

\includegraphics{https://static01.graylady3jvrrxbe.onion/images/2019/03/17/magazine/17mag-talk-dating/17mag-talk-dating-articleLarge-v2.gif?quality=75\&auto=webp\&disable=upscale}

By Lauren Oyler

\begin{itemize}
\item
  March 14, 2019
\item
  \begin{itemize}
  \item
  \item
  \item
  \item
  \item
  \item
  \end{itemize}
\end{itemize}

Watching the Netflix show ``Dating Around'' is like sitting next to a
Tinder date at a bar: The possibility that something outrageous, sexy or
at least interesting will happen holds your attention long after it has
become clear that the people you're spying on are just as boring as you
are. The series is part of a naturalistic downshift in reality TV; it
features neither overt competition nor narrative arc. It simply follows
a person going on five blind dates over the course of a week, and then
choosing one person to go out with again. The five dates must know
they're being judged against one another, but the show avoids
acknowledging this, and the dater's deliberations are never shown. To
the extent that anyone on the show is looking for love, they're doing so
casually, nonaggressively, realistically. They're merely game --- not
playing one.

First dates are inherently dramatic, even when they're dull. The
atmospheric nerves --- choosing an outfit, worrying you've said
something dumb --- easily create enough tension to carry a 30-minute
television show. What's most revealing about ``Dating Around,'' though,
is the way it's structured. The lead dater wears the same outfit and
eats five different meals at the same restaurant. This allows the five
dates to be edited into one four-dimensional hyperdate. Rather than
showing each date in succession, episodes are organized into three
segments --- drinks, followed by dinner, then ``after hours,'' during
which daters may respectfully part ways or head onward to a bar --- with
all the dates interwoven so they all appear to have happened in a single
evening. It's as if Ashley has body-swapped with Kate on her bathroom
break, over and over and over again. All dating shows are contrived, but
the contrivances on ``Dating Around'' are not preposterous, designed to
shock or entertain --- in fact, they're depressingly familiar.

A minute-long sequence in the first episode epitomizes the show's
attitude toward romance. After dinner at a Thai restaurant in Brooklyn,
Luke, a motorcycle-riding real estate agent, asks Victoria, the clear
front-runner among his five options, if she wants to get out of there.
The scene cuts to B-roll footage of the New York City streets, and then
emerging from the restaurant are our couple --- Luke and now
\emph{Betty}, a divorced 30-year-old wearing a very short dress. Betty
points at the sky as though it's a dish she just whipped up with
whatever she had in the fridge. ``Look at that,'' she says. Luke stops
and complies. ``Oh, wow,'' he says, the opposite of awe-struck, his pose
a lazy Vanna White, forearm raised to present to her what she's already
presented to him. ``Full moon. Yeah.'' He moves back to where she's
standing so they may look up at it together. ``That's beautiful,'' he
says. ``So beautiful,'' she agrees.

Cut to a shot of the moon, looking like the moon. Now we hear Luke's
voice, slightly more upbeat --- ``This is a nice night!'' --- as a
squeakier woman's voice asks, ``Do you see the full moon?'' This voice
belongs to Tiffany, a third option. They're standing in front of the
same Thai restaurant. Soon they're making out --- her initiative --- and
as they walk away from the restaurant holding hands, Luke expresses
gratitude for the full moon. Cut back to Luke and Betty, who is calling
for a ``sexy dance'' on the same stretch of sidewalk where he was just
--- or would soon be? --- necking with another girl.

\includegraphics{https://static01.graylady3jvrrxbe.onion/images/2019/03/17/magazine/17mag-screenland-promo/17mag-screenland-promo-videoSixteenByNineJumbo1600.png}

The trick of the editing is not to highlight differences among the
daters but to suggest that on some level they're interchangeable. No
script is necessary because they rarely deviate from how things are
supposed to go. Tepid small talk about drink selection --- ``What is
this?'' ``Like, a margarita'' --- moves on to ``Where are you from?''
followed by a pause for menu consideration, then onto job talk and
canned flattery like ``How are you single?'' The blind dates eventually
converge on what feel like serious topics, though the same ones come up
almost every night of the week: past relationships, kids, priorities.
``I just want love,'' Betty says. ``Connection, chemistry, love.'' A
minute later, Tiffany explains the importance of the ``three C's'':
``compatibility, chemistry and connection.''

The vocabulary --- abstract nouns that fail to conjure the grand
concepts they're supposed to --- recalls nothing so much as dating-app
marketing, while the show's carousel-like form reproduces the experience
of using Tinder and the rest. Not only do the daters skew toward the
kinds of people you commonly see on the apps --- youngish, professional,
fluent with an iPhone --- but they're also eager to filter their options
with getting-to-know-you questionnaire material, the sort of information
that you want to find out at some point but that wouldn't necessarily
come up were you to meet by chance, say, at a friend's party.

The impulse to control or strategize romance isn't new --- red flags and
deal-breakers, and the analysis they inspire, abound in 1990s romantic
comedies, and courtship rituals predate humanity entirely. What seems
uniquely contemporary about ``Dating Around'' is the rote, bored way
people enact these norms, as if they have no choice --- or rather
because they have so much of it. \emph{Regard the moon: It's in a lot of
poems.} Its repeat cameo here is a way to signify romance, even where no
romance was present; whether its appearance was noticed naturally by the
daters or pointed out by the producers, it functions as a symbol of a
symbol, inspiring the young not-lovers to go through the motions.

The importance of compatibility reinforces the sense that love can be
found through a formula or a checklist; the idea is as seductive as
anyone on this show. When, during an ``after hours'' conversation, one
contestant uses the word ``swipe'' to refer to dating itself, without
having to explain the word's provenance, she reveals that dating has
become so process-oriented that it's practically indistinguishable from
the mechanisms that were meant to streamline it. Though dating apps may
improve many aspects of modern romance --- by making people safer and
more accessible --- their guardrails also seem to limit the
possibilities for it. The stakeslessness of ``Dating Around'' might be a
refreshing lack of pressure, but it might also reflect the disturbing
effects of the same phenomenon in real life.

Despite what tech companies would have us believe, people cannot be
optimized for one another; an overwhelming abundance of options
discourages the leaps of faith that can transform the terrible
uncertainty of dating into something great. Nothing is especially wrong
with this arrangement, but is anything right? The second episode,
featuring a divorced 36-year-old woman, ends with a shot of her walking
in SoHo, arms laden with shopping bags, fine with it all, catching the
eyes of strangers who pass: She hasn't called any of her dates back, but
maybe one of the next five will work out. Like keeping up with a decent
TV show, it's at least something to do.

Advertisement

\protect\hyperlink{after-bottom}{Continue reading the main story}

\hypertarget{site-index}{%
\subsection{Site Index}\label{site-index}}

\hypertarget{site-information-navigation}{%
\subsection{Site Information
Navigation}\label{site-information-navigation}}

\begin{itemize}
\tightlist
\item
  \href{https://help.nytimes3xbfgragh.onion/hc/en-us/articles/115014792127-Copyright-notice}{©~2020~The
  New York Times Company}
\end{itemize}

\begin{itemize}
\tightlist
\item
  \href{https://www.nytco.com/}{NYTCo}
\item
  \href{https://help.nytimes3xbfgragh.onion/hc/en-us/articles/115015385887-Contact-Us}{Contact
  Us}
\item
  \href{https://www.nytco.com/careers/}{Work with us}
\item
  \href{https://nytmediakit.com/}{Advertise}
\item
  \href{http://www.tbrandstudio.com/}{T Brand Studio}
\item
  \href{https://www.nytimes3xbfgragh.onion/privacy/cookie-policy\#how-do-i-manage-trackers}{Your
  Ad Choices}
\item
  \href{https://www.nytimes3xbfgragh.onion/privacy}{Privacy}
\item
  \href{https://help.nytimes3xbfgragh.onion/hc/en-us/articles/115014893428-Terms-of-service}{Terms
  of Service}
\item
  \href{https://help.nytimes3xbfgragh.onion/hc/en-us/articles/115014893968-Terms-of-sale}{Terms
  of Sale}
\item
  \href{https://spiderbites.nytimes3xbfgragh.onion}{Site Map}
\item
  \href{https://help.nytimes3xbfgragh.onion/hc/en-us}{Help}
\item
  \href{https://www.nytimes3xbfgragh.onion/subscription?campaignId=37WXW}{Subscriptions}
\end{itemize}
