Sections

SEARCH

\protect\hyperlink{site-content}{Skip to
content}\protect\hyperlink{site-index}{Skip to site index}

\href{https://myaccount.nytimes3xbfgragh.onion/auth/login?response_type=cookie\&client_id=vi}{}

\href{https://www.nytimes3xbfgragh.onion/section/todayspaper}{Today's
Paper}

I Detest the N.R.A. What Should I Do With My Gun?

\url{https://nyti.ms/2TVDz3K}

\begin{itemize}
\item
\item
\item
\item
\item
\item
\end{itemize}

Advertisement

\protect\hyperlink{after-top}{Continue reading the main story}

Supported by

\protect\hyperlink{after-sponsor}{Continue reading the main story}

\href{/column/the-ethicist}{The Ethicist}

\hypertarget{i-detest-the-nra-what-should-i-do-with-my-gun}{%
\section{I Detest the N.R.A. What Should I Do With My
Gun?}\label{i-detest-the-nra-what-should-i-do-with-my-gun}}

\includegraphics{https://static01.graylady3jvrrxbe.onion/images/2018/10/07/magazine/31mag-ethicist-image1/31mag-ethicist-image1-articleLarge-v16.jpg?quality=75\&auto=webp\&disable=upscale}

By Kwame Anthony Appiah

\begin{itemize}
\item
  March 26, 2019
\item
  \begin{itemize}
  \item
  \item
  \item
  \item
  \item
  \item
  \end{itemize}
\end{itemize}

\emph{As a gun owner who abhors the ``slippery slope'' philosophy of the
N.R.A., every new mass shooting sickens me. I would like to sell one of
my three weapons and give the proceeds to March for Our Lives or
Everytown for Gun Safety. Is it better to: 1) Sell it knowing the
\$750-\$1,000 would do some good; 2) Keep the gun knowing it won't be
used; or 3) Destroy it/surrender it to the police for disposal?} Name
Withheld

\textbf{By the ``slippery slope''} philosophy of the N.R.A., I assume
you mean its tendency to argue that any proposed gun regulation is a
step toward canceling the Second Amendment: to equate restriction with
abolition. I assume too that you think that some of these mass shootings
could have been stopped by laws that the N.R.A. opposes and that are
nevertheless consistent with the Constitution. The gun-control
organizations you mention share those beliefs; sending them a check
would express your support and enable you to join the community of
people trying to do something about gun violence.

But the case against (1) is that your check, which isn't going to make
the difference between success and failure for either of these
gun-control groups, scarcely changes the likelihood of future gun
deaths, while selling the gun marginally increases the likelihood that
it will end up being used in a crime. As for (2), you can't be
absolutely sure that a gun won't cause harm just because it's in your
house. For one thing, someone might break in and steal it; for another,
you might use it against yourself. Suicide, remember, accounts for a
majority of gun-related deaths. If you went for (3), you could render
the gun harmless by having it destroyed, but doing so wouldn't much
change the likelihood of future gun deaths, either.

One reason that I have misgivings about what's been called ``quandary
ethics'' --- ethics conceived of as solving puzzles like these --- is
that, as in this case, it can be close to impossible to calculate the
costs and benefits of the various outcomes you consider. A deeper
problem is that there are typically options you haven't considered. In
this case, you could wait for a gun-buyback program --- they've had
these recently in many cities --- and send the money (which would be
less than the market value of your weapon) to one of these
organizations, thus both supporting gun control and making sure that the
weapon won't be used for malign purposes.

For that matter, you could throw yourself into the work of one of these
organizations, which are going to succeed only if more of their
supporters aren't content with just sending them money. The influence of
the N.R.A. can't be reduced to the power of the purse; though its
political expenditures far exceed those of gun-control groups, its
finances are surprisingly precarious. And note that the labor sector,
say, hugely outspends it, while its political power has seemed to
diminish. Many political experts would say that the N.R.A.'s ability to
mobilize its millions of members has a lot to do with its efficacy. If
gun control matters to you, your support shouldn't be limited to your
checkbook.

\emph{I agreed to sponsor a cousin's immigration application, which
involved affirming I can and will provide the applicant with a financial
safety net for seven years. I'd met the cousin only once, but I count
his aunts among my favorite relatives and was told by his father that he
is financially independent, hardworking and eager to contribute to his
new homeland.}

\emph{With the process underway, my cousin's father came to visit, and
the two insisted on conveying their gratitude by hosting my family at a
brunch. Somehow the subject of Trump came up. Though my husband and I
did not vote for the president, we maintain relationships with friends
and family who did. Yet hearing the young man that we put on the road to
citizenship express his admiration for Trump and his anti-immigration
policies struck us as odd.}

\emph{I prodded ever so gently: At what point should the president close
the gate --- before or after your papers are finalized? Without a hint
of irony, he clarified that, of course, he was referring to the wrong
kind of immigrants.}

\emph{Hours later, I couldn't shake the notion that we'd made a terrible
mistake. Yes, my husband and I refuse to allow politics to define our
social circle, but could we remain neutral, knowing that we'd
inadvertently recruited a new supporter of Trump's views on immigration,
which we regard as racist --- or worse? And if we decide that we cannot,
should we break with our extended family by revoking our support of my
cousin's application? Or could we let the cousin know that we find his
views on immigration hypocritical and make our ongoing support of his
application contingent on his joining the 50 percent of Americans who
choose not to exercise their right to vote?} Name Withheld

\textbf{I'm not sure} you've found a great way to model social
tolerance. Your cousin sounds pretty blinkered in his views, I'll grant.
But adding one more bigot to the American population isn't going to make
a difference in our politics. And if he ends up, as is most likely,
living in one of our great, culturally plural metropolises, his views on
immigration may well evolve. Either way, it would be wrong to try to
blackmail him into not exercising the right to vote --- and ineffectual,
because such an agreement would be unenforceable.

Nor is it fair, at this point, to pull out from your affidavit of
support, unless you think that something you were told and that you
therefore attested to was substantially untrue. You do have every right
to tell him --- like anyone else who expresses political views you
consider odious --- why you think he's wrong. Given what you have done
for him, he owes it to you to pay attention to what you say. But if your
sponsoring your cousin was right, it didn't become wrong when you
learned about his opinions.

The point can be broadened. Policies you support in principle aren't
invalidated when their beneficiaries turn out to have vexing or perverse
views. We don't withdraw Social Security Disability Insurance from those
who think that its other beneficiaries are largely wastrels. We don't
deny Medicare coverage to those who are skeptical of the program. We
can't give up on public-health measures to reduce suicide simply because
suicide rates are highest in ``red state'' regions that aren't inclined
to back such measures. And a partisan who favors only the expanded
immigration of people inclined to support her party isn't interested in
immigration reform; she's interested in allies. Your cousin's views, as
I say, may evolve; perhaps yours will, too.

\emph{My stepdaughter-in-law confided in me that she is planning on
leaving my stepson. They have a young child. She also informed me that
she's gay. She asked me to be discreet with this information, but I feel
compelled to do something; I don't feel that it's fair for my stepson to
be hit out of the blue. Is there a way for me to encourage her to talk
to her husband, or should I say something directly to him? I'm fine with
keeping the information about her sexuality secret. It's more her plan
to dump him that is weighing on me. I'm concerned not only for his
well-being but also for how this secret could affect my relationship
with him if he finds out I knew and didn't say anything.} Name Withheld

\textbf{Let me invoke} a rule I've posited before: In normal
circumstances, when someone tells you something with the implicit
expectation that you won't pass it on, you shouldn't break that
confidence without first telling her. The confidence sharer has the
right to try to dissuade you and the right to pre-empt you by passing on
the information and managing the consequences herself. So tell your
stepdaughter-in-law what you're thinking. And what you're thinking is
correct: Your stepson has a right to know that she's planning to leave
him. She should come clean. And if she won't, you should. (A discussion
of her reasons should indeed remain between the couple.) You're also
correct to fear that your relationship with your stepson could be
damaged by your reticence: He would be within his rights to consider it
a betrayal.

Advertisement

\protect\hyperlink{after-bottom}{Continue reading the main story}

\hypertarget{site-index}{%
\subsection{Site Index}\label{site-index}}

\hypertarget{site-information-navigation}{%
\subsection{Site Information
Navigation}\label{site-information-navigation}}

\begin{itemize}
\tightlist
\item
  \href{https://help.nytimes3xbfgragh.onion/hc/en-us/articles/115014792127-Copyright-notice}{©~2020~The
  New York Times Company}
\end{itemize}

\begin{itemize}
\tightlist
\item
  \href{https://www.nytco.com/}{NYTCo}
\item
  \href{https://help.nytimes3xbfgragh.onion/hc/en-us/articles/115015385887-Contact-Us}{Contact
  Us}
\item
  \href{https://www.nytco.com/careers/}{Work with us}
\item
  \href{https://nytmediakit.com/}{Advertise}
\item
  \href{http://www.tbrandstudio.com/}{T Brand Studio}
\item
  \href{https://www.nytimes3xbfgragh.onion/privacy/cookie-policy\#how-do-i-manage-trackers}{Your
  Ad Choices}
\item
  \href{https://www.nytimes3xbfgragh.onion/privacy}{Privacy}
\item
  \href{https://help.nytimes3xbfgragh.onion/hc/en-us/articles/115014893428-Terms-of-service}{Terms
  of Service}
\item
  \href{https://help.nytimes3xbfgragh.onion/hc/en-us/articles/115014893968-Terms-of-sale}{Terms
  of Sale}
\item
  \href{https://spiderbites.nytimes3xbfgragh.onion}{Site Map}
\item
  \href{https://help.nytimes3xbfgragh.onion/hc/en-us}{Help}
\item
  \href{https://www.nytimes3xbfgragh.onion/subscription?campaignId=37WXW}{Subscriptions}
\end{itemize}
