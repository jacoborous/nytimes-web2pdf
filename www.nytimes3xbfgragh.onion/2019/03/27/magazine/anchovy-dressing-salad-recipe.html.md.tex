Sections

SEARCH

\protect\hyperlink{site-content}{Skip to
content}\protect\hyperlink{site-index}{Skip to site index}

\href{https://myaccount.nytimes3xbfgragh.onion/auth/login?response_type=cookie\&client_id=vi}{}

\href{https://www.nytimes3xbfgragh.onion/section/todayspaper}{Today's
Paper}

The Startlingly Flavorful Dressing That Will Boost More Than Just Your
Salads

\url{https://nyti.ms/2UTyF3W}

\begin{itemize}
\item
\item
\item
\item
\item
\end{itemize}

Advertisement

\protect\hyperlink{after-top}{Continue reading the main story}

Supported by

\protect\hyperlink{after-sponsor}{Continue reading the main story}

\href{/column/magazine-eat}{Eat}

\hypertarget{the-startlingly-flavorful-dressing-that-will-boost-more-than-just-your-salads}{%
\section{The Startlingly Flavorful Dressing That Will Boost More Than
Just Your
Salads}\label{the-startlingly-flavorful-dressing-that-will-boost-more-than-just-your-salads}}

\includegraphics{https://static01.graylady3jvrrxbe.onion/images/2019/03/31/magazine/31mag-eat-slideshow-slide-K0AE/31mag-eat-slideshow-slide-K0AE-articleLarge.png?quality=75\&auto=webp\&disable=upscale}

By Gabrielle Hamilton

\begin{itemize}
\item
  March 27, 2019
\item
  \begin{itemize}
  \item
  \item
  \item
  \item
  \item
  \end{itemize}
\end{itemize}

I have three refrigerators in my life: the one you can walk into at the
restaurant, with the interior safety-release door clasp and the
diamond-plate floor; and the double-door one at home that blooms
incandescent light like time-lapse daybreak when you open her. Then
there is the one at our weekend getaway place, sturdy but cramped, with
shelves that don't adjust so that you have to store the half-and-half
horizontally when the door cubbies are too full of various condiments.
Because we get away to that one only on a random
Tuesday-through-Wednesday, maybe twice a month, and we can never
remember if we've run out of Hellman's up there, those cubbies are
chockablock with condimental redundancies. But in each and every one of
these three fridges --- at any time, on any day of the week, no matter
the season or the hour --- there is a container of anchovy-garlic-chile
dressing on one shelf or another, within reach. It's usually in a
plastic pint container, labeled in blue tape with black Sharpie, and it
just says: GH Crack Sauce.

Like everybody else, I've often heard that chefs' home refrigerators are
famously empty. Because we work so hard, so long, so late, the mythology
suggests, we don't ever cook at home. Everybody seems to know, and
admire, that our home fridges are echo chambers but for a six-pack of
beer and a bottle of hot sauce. I notice I respond to this bit of rock
'n' roll chef hype the same way I do when I read about that supposedly
inviolable routine of accomplished and serious writers, the one in which
the writer wakes up and writes for four hours and then goes for a walk
to clear the mind and then after lunch reads in the afternoon and then
goes to the pub for a few drinks before dinner and then retires by the
fire before bed.

Which is: Oh, man. Must be nice!

But if you find yourself in the thick of, say, a brisk chef career, with
maybe a professional writing gig on the side, as well as the
not-negligible project of keeping a home, loaded up with the added
considerations of some compelling children in the household whom you
might like to see and know before they are gone, you probably have
managed to keep a few vegetables in the drawer and a carton of eggs and
a bin of cheeses and deli meats at least. I recommend adding a container
of this anchovy-garlic dressing to your fridge too, at all times. It's
nothing, really --- a jar of startlingly flavorful salad dressing
essentially --- but it does some improbable heavy lifting. And at
mealtime, no matter how loosely defined, it can make you feel less like
a barn animal at the trough and more like a civilized human at the
table. Even if you are standing in front of the fridge with the door
open rooting in the drawers at 2:30 in the morning.

It's so inexplicable, but when the ingredients of this sauce --- garlic,
anchovies, chile flakes, lemon juice, extra-virgin olive oil --- are
drizzled or drenched or dotted accordingly, the taste can differ
depending where and how it lands, kind of like a mood ring that passes
through clear opal green to dark charcoal and smoky black. When you eat
it straight, it can really punch you in the face. The intense burn of
fresh garlic --- and so shockingly much of it --- and the tight
sandpaper astringency of the lemon juice can make you think you've made
a mistake and gotten the proportions wrong. But by the time you spoon it
generously over cold steamed broccoli, or plain boiled cauliflower, it
is tamed. And when it's tossed with curly leafy mustard greens, it
evolves in a different and distinct way --- a little deeper, more
warmly, less sharp.

\includegraphics{https://static01.graylady3jvrrxbe.onion/images/2019/03/31/magazine/31mag-eat-slideshow-slide-9ABY/31mag-eat-slideshow-slide-9ABY-articleLarge.png?quality=75\&auto=webp\&disable=upscale}

I once dipped the left-behind crusts of the kids' pizza into it and
accidentally discovered that it should be mandatory as a drizzle over a
standard white pie. I've whisked in a healthy dollop of thick plain
yogurt and dressed cold boiled spinach and thought of it as a tasty
Western version of spinach \emph{goma ae} --- the Asian cold dish with
sesame paste. Poured over still-warm diced, boiled potatoes, then
showered with celery leaves, it makes a lunch. And if you blend in some
freshly grated, still-on-the-moist-side young Parmesan cheese along with
a sieved hard-boiled egg and a ton of black pepper, you could
legitimately pass it off as a Caesar salad, even if spooned over
escarole on toast.

I know the lonely rattle of hot sauce and beer bottles makes our
reputation a little more badass, and makes us look a little edgier, a
little skinnier in our torn tight black jeans, a little more lone wolf.
But I would argue that those same late hours and those same long shifts
and that unrelenting work would make you even more formidable if you
could pull off a little something civilized, and interesting to eat,
whenever you finally get home.

\textbf{Recipe:}
\href{https://cooking.nytimes3xbfgragh.onion/recipes/1020127-anchovy-garlic-dressing}{Anchovy-Garlic
Dressing}

Advertisement

\protect\hyperlink{after-bottom}{Continue reading the main story}

\hypertarget{site-index}{%
\subsection{Site Index}\label{site-index}}

\hypertarget{site-information-navigation}{%
\subsection{Site Information
Navigation}\label{site-information-navigation}}

\begin{itemize}
\tightlist
\item
  \href{https://help.nytimes3xbfgragh.onion/hc/en-us/articles/115014792127-Copyright-notice}{©~2020~The
  New York Times Company}
\end{itemize}

\begin{itemize}
\tightlist
\item
  \href{https://www.nytco.com/}{NYTCo}
\item
  \href{https://help.nytimes3xbfgragh.onion/hc/en-us/articles/115015385887-Contact-Us}{Contact
  Us}
\item
  \href{https://www.nytco.com/careers/}{Work with us}
\item
  \href{https://nytmediakit.com/}{Advertise}
\item
  \href{http://www.tbrandstudio.com/}{T Brand Studio}
\item
  \href{https://www.nytimes3xbfgragh.onion/privacy/cookie-policy\#how-do-i-manage-trackers}{Your
  Ad Choices}
\item
  \href{https://www.nytimes3xbfgragh.onion/privacy}{Privacy}
\item
  \href{https://help.nytimes3xbfgragh.onion/hc/en-us/articles/115014893428-Terms-of-service}{Terms
  of Service}
\item
  \href{https://help.nytimes3xbfgragh.onion/hc/en-us/articles/115014893968-Terms-of-sale}{Terms
  of Sale}
\item
  \href{https://spiderbites.nytimes3xbfgragh.onion}{Site Map}
\item
  \href{https://help.nytimes3xbfgragh.onion/hc/en-us}{Help}
\item
  \href{https://www.nytimes3xbfgragh.onion/subscription?campaignId=37WXW}{Subscriptions}
\end{itemize}
