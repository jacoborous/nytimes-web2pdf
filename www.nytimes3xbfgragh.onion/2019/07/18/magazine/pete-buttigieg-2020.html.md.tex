Pete Buttigieg Is Still Figuring This Out

\url{https://nyti.ms/2Lo9Pbb}

\begin{itemize}
\item
\item
\item
\item
\item
\item
\end{itemize}

\begin{itemize}
\item
  \href{https://www.nytimes3xbfgragh.onion/live/2020/09/07/us/trump-vs-biden?action=click\&pgtype=Article\&state=default\&region=TOP_BANNER\&context=storylines_menu}{Election
  Updates}
\item
  \href{https://www.nytimes3xbfgragh.onion/interactive/2020/us/elections/election-states-biden-trump.html?action=click\&pgtype=Article\&state=default\&region=TOP_BANNER\&context=storylines_menu}{Paths
  to 270}
\item
  \href{https://www.nytimes3xbfgragh.onion/interactive/2020/08/31/us/politics/vote-by-mail-deadlines.html?action=click\&pgtype=Article\&state=default\&region=TOP_BANNER\&context=storylines_menu}{Voting
  by Mail}
\item
  \href{https://www.nytimes3xbfgragh.onion/interactive/2019/us/elections/2020-presidential-election-calendar.html?action=click\&pgtype=Article\&state=default\&region=TOP_BANNER\&context=storylines_menu}{Key
  Dates}
\item
  \href{https://www.nytimes3xbfgragh.onion/newsletters/politics?action=click\&pgtype=Article\&state=default\&region=TOP_BANNER\&context=storylines_menu}{Politics
  Newsletter}
\end{itemize}

\includegraphics{https://static01.graylady3jvrrxbe.onion/images/2019/07/21/magazine/21mag-Buttigieg-image/21mag-Buttigieg-image-articleLarge-v2.jpg?quality=75\&auto=webp\&disable=upscale}

Sections

\protect\hyperlink{site-content}{Skip to
content}\protect\hyperlink{site-index}{Skip to site index}

Feature

\hypertarget{pete-buttigieg-is-still-figuring-this-out}{%
\section{Pete Buttigieg Is Still Figuring This
Out}\label{pete-buttigieg-is-still-figuring-this-out}}

A dark-horse presidential candidate tries to stay on track.

Credit...Angie Smith for The New York Times

Supported by

\protect\hyperlink{after-sponsor}{Continue reading the main story}

By \href{https://www.nytimes3xbfgragh.onion/by/mark-leibovich}{Mark
Leibovich}

\begin{itemize}
\item
  July 18, 2019
\item
  \begin{itemize}
  \item
  \item
  \item
  \item
  \item
  \item
  \end{itemize}
\end{itemize}

\textbf{I}t was a humid Sunday in June, a quiet afternoon that Pete
Buttigieg knew would not remain quiet. ``You know, there are always
going to be ups and downs,'' the 37-year-old mayor of South Bend, Ind.,
told me as he puttered through the kitchen of the century-old Victorian
home he shares with his husband of a year, Chasten. ``You can't just
have an uninterrupted meteoric rise,'' he said.

Buddy and Truman, the Twitter-certified
\href{https://twitter.com/firstdogsSB}{``First Dogs of South Bend,''}
were lounging on hardwood floors as Buttigieg poured coffee into a mug
and settled in at his dining-room table. In a few hours, he would be
speaking to a noisy town hall at Washington High School, on his city's
predominantly black west side, where he would be called upon --- shouted
upon --- to answer questions about what the cable networks had variously
called ``Mayor Pete's Crisis at Home'' and the ``Nightmare in South
Bend.'' A week before, a white South Bend Police officer,
\href{https://www.nytimes3xbfgragh.onion/2019/06/18/us/politics/buttigieg-south-bend-police-shooting.html}{Sgt.
Ryan O'Neill, shot} a 54-year-old African-American man, Eric Logan,
after the officer responded to a report of a suspicious person going
through cars in the parking lot of an apartment complex. O'Neill claimed
that Logan approached him with a knife, but his body camera was turned
off, so there was no footage to back up his account. Logan was later
pronounced dead at Memorial Hospital in South Bend.

The killing set off days of protest aimed at the local police, city
officials and Buttigieg, whose unlikely surge into the top tier of
Democratic presidential candidates had been blunted by ambivalence from
African-American voters, among whom he had been polling close to zero
nationally, even before the shooting. No shortage of pundits offered
theories on
\href{https://www.nytimes3xbfgragh.onion/2019/10/18/us/politics/buttigieg-laquan-mcdonald-lawyer.html}{Buttigieg's
``black problem,''} as the former chairwoman of the Congressional Black
Caucus, Marcia Fudge, called it in The Daily Beast after the incident.
They posited some combination of Buttigieg's lesser name recognition;
the reluctance of more socially conservative blacks to embrace an openly
gay candidate; and the perception that the Harvard- and Oxford-bred
sensation was just another privileged white politician in a hurry.

But Buttigieg has also had a fraught relationship with the black
community of South Bend for much of his eight years as mayor, especially
over matters of policing --- a fact that the national media, after
months of laudatory coverage of Buttigieg's mayoral successes, now began
to understand. Up to that point, Buttigieg had mostly confronted
race-related questions from a safe, aspirational remove. He was quizzed
at a Fox News town hall in New Hampshire a few weeks earlier (by a white
woman from Vermont) about \href{https://youtu.be/Y3sr0RtIfgI?t=338}{what
he would do to better reach voters of color}; the host, Chris Wallace,
cited a poll showing that less than 1 percent of nonwhite primary voters
supported him. ``It's a really important strategic but also ethical
question for our campaign,'' Buttigieg ruminated.

He has quoted the Rev. Martin Luther King Jr. (``Dr. King'') and shared
lofty-sounding ideas like his ``Douglass Plan'' (`to improve black
American prosperity''). He is diligent about promising his friendly
white crowds that he understands the urgency of civil rights as an
unrealized national goal. ``Racial inequality,'' he assures his
audiences, ``either will be solved in our lifetime or it will blow apart
the American project.''

\includegraphics{https://static01.graylady3jvrrxbe.onion/images/2019/07/21/magazine/21mag-Buttigieg-image-04/21mag-Buttigieg-image-04-articleLarge.jpg?quality=75\&auto=webp\&disable=upscale}

Buttigieg has a knack for reducing the intractable issues of American
life to some academic-sounding ``project,'' as if racial inequality were
just another puzzle for the smart kids at McKinsey --- where Buttigieg
worked as a consultant after college --- to solve. He is also deft about
acknowledging that this is exactly what he is doing: noting his own
privileged detachment as he is exercising it. ``There's a certain luxury
associated with being able to step back and be analytical about any of
this,'' Buttigieg told me. I had been checking in periodically with
Buttigieg through the spring, a period in which said ``meteoric rise''
would accelerate in earnest.
\href{https://www.nytimes3xbfgragh.onion/2019/04/24/magazine/pete-buttigieg-smart-harvard-rhodes-scholar-norwegian-language.html}{A
video clip of him speaking Norwegian} was bouncing across social media;
the novel concept of supporting ``a Maltese-American left-handed
Episcopalian gay war veteran mayor millennial,'' as he described
himself, was proving irresistible to a certain sector of the educated
white electorate. He seemed to be living
\href{https://www.youtube.com/watch?v=u7SHQSGesyM}{on late-night TV
couches} when he was not delighting photo-hungry, check-writing crowds.
They were not only learning how to pronounce ``BOOT-edge-edge'' but also
chanting the eye-chart name and wearing it on phonetic-spelling T-shirts
that I saw several people sporting at a small-donor fund-raiser in
Minneapolis in early May. ``I see Pete Buttigieg as more of a
healer-warrior, and there's an absence of vitriol with him,'' Dave
Dvorak, a physician I met there, told me. ``And maybe that's what we
need.''

\emph{{[}}\href{https://www.nytimes3xbfgragh.onion/2019/07/14/us/politics/pete-buttigieg-gay.html}{\emph{Read
about Buttigieg's life in the closet}}\emph{.{]}}

The luxury of Buttigieg's safe remove ended with the shooting of Eric
Logan. The mayor woke to the news before dawn on June 16 --- Father's
Day, the first since his own father, Joseph Buttigieg, a Maltese
immigrant, died in January. That scrambled Buttigieg's plan to take
Sunday off in New York with Chasten to celebrate their first wedding
anniversary, which also fell on that day. ``The first thing you hear is
that there was an officer-involved shooting, which is bad but not the
first time it's happened,'' Buttigieg recounted a week later. ``Then you
hear the guy's in surgery, then you realize, O.K., he may not live. Then
you hear the deceased is black and the cop is white. And you keep
getting bits of information, some of it accurate, some you've got to run
down. And it didn't take long to realize I needed to get home.''

Buttigieg made his way back to South Bend on Sunday, canceling a Monday
appearance at an L.G.B.T.Q. gala in Manhattan and fund-raising events in
California that Tuesday and Wednesday. He had planned a return to South
Carolina, where the state's Democratic Party was holding its convention
in Columbia, that Friday and Saturday. The weekend featured the World
Famous Fish Fry, a sweaty mob scene of a tradition hosted by
Representative Jim Clyburn of South Carolina, the House majority whip
and the highest-ranking African-American member of Congress, and
attended by nearly all of the Democratic presidential hopefuls. But the
fish fry conflicted with a hastily scheduled Justice for South Bend
march through downtown to honor Logan.

Buttigieg had very much wanted to be in South Carolina, the early-voting
state in which 60 percent of the party's primary electorate is
African-American --- the place, as Buttigieg put it, ``where most
Democratic candidates try to find their voice on race.'' Instead, South
Bend's black community was calling for the mayor to stay home and
listen. ``You got a plane to catch somewhere?'' one angry rally-goer
yelled at the grounded candidate, one of many who would taunt his higher
ambitions. Here was Buttigieg being laid low by the same riddle of race
relations in America that has stymied generations of well-meaning
progressive mayors that came before him.

``You can seek to do the right thing,'' Buttigieg said, ``and be
reasonably confident you made the less bad choice and get your ass
handed to you all the same.'' That was the nature of being a mayor, he
added --- a far more tactile and hands-on job than, say, being a member
of Congress, for whom running for president would not necessarily entail
more than missing a few floor votes. ``A lot of it is just being there
to absorb a lot of pain,'' Buttigieg said of his ultimate decision to
attend the rally in South Bend. ``It's not like Eric Logan's mother is
going to be happy about anything we come up with.''

``I've been here my whole life, and you all don't do a damn thing about
me or my son or none of these people out here,'' Logan's mother, Shirley
Newbill, told Buttigieg at the rally. Latisa McKinney, an
African-American woman from South Bend told me at the town hall there on
Sunday: ``Ain't nothing gonna change for us. Ain't nothing gonna change
for us after Mayor Pete's gone, either.'' It was maybe some small solace
to Buttigieg that even a frustrated constituent respected his brand,
taking care to address him by his folksy small-town moniker, ``Mayor
Pete.''

Image

``You can seek to do the right thing,'' Buttigieg said, ``and be
reasonably confident you made the less bad choice and get your ass
handed to you all the same.''Credit...Angie Smith for The New York Times

\textbf{Some of Buttigieg's} giddier supporters and profilers have
likened him to Barack Obama, not just in his appeal to a new generation
of political consumers but also in his intent to create a new way of
thinking and discussing politics. He is the next level of
anti-politician politician, quintessentially political but running
against what he sees as the counterproductive outrage that seems to have
taken hold in American politics, particularly in the Trump era. ``Our
response is going to be to model something completely different,''
Buttigieg told me.

And indeed, he possesses an Obama-like ability to wield cool detachment
--- impassioned and remote at the same time, calmly in a rush. Even his
execution of the necessary and grubby candidate activities, like
fund-raising, has an earnestly above-it-all air. ``Hey,'' he began a
blast email appeal to his supporters on the eve of the last Federal
Election Commission fund-raising deadline. ``You know that we don't
subscribe to inauthentic urgency here at Pete for America. That's not
why we're here. We are here to build trusted relationships.'' He then
hit up his ``trusted relationships'' for donations.

\href{https://www.nytimes3xbfgragh.onion/2019/06/12/magazine/democratic-primary-candidates-iowa-caucus.html}{\emph{{[}Read
about the 23 Democratic candidates running for president}}\emph{.{]}}

Like most presidential candidates, Buttigieg published a book on the eve
of his candidacy, part blueprint and part memoir of an ordinary and yet
extraordinary life. Unlike most candidates' books, ``Shortest Way Home''
is actually a decent read and even seems to have been written by the
candidate himself (he confirmed this). In it, Buttigieg describes his
rampage through the checkpoints of American high achievement. The son of
Notre Dame professors, he attended Harvard, Oxford on a Rhodes
scholarship, worked at McKinsey \& Company, served as an intelligence
officer in the United States Navy Reserves and was deployed in
Afghanistan while serving as mayor of his hometown, an office that he
was elected to in 2011 at age 29. Beyond the author himself, South Bend
is the unquestioned star of the book, the main instrument through which
the protagonist tells his coming-of-age story. He purchased his creaky
old home, built in 1905, and renovated it himself over two years in the
late '00s. It sits across West North Shore Drive from the swelling
waters of the St. Joseph River --- a river ``in a hurry to get
somewhere,'' as Buttigieg characterizes it, or projects upon it.

He describes how he first imagined leading an ``administration that ran
on business principles without abandoning its public character.''
Initially, he disdained the ceremonial tasks that filled a mayor's
schedule: the ribbon cuttings, holiday tributes and solemn remembrances.
``Shaped by my consulting background, I arrived in office wanting to get
concrete, measurable things done,'' Buttigieg writes. Eventually, he
would learn to embrace that part of the job, equating the simple act of
representing a city to a kind of moral position. ``The value was not in
the cleverness of what I had to say but simply the fact of my being
there,'' he writes. ``Introvert that I am, I even came to love a good
parade.''

No sitting mayor has ever been elected president; it's rare they even
seek the office at all, much less from a jurisdiction as little as South
Bend, the fourth-largest city in Indiana. Yet the smallness of the town
--- its flyover-country coordinates, familiar mostly via Notre Dame
football on TV --- lends it an allegorical credibility. ``The Bend''
could be anywhere, and that's the point. In the telling of its most
famous current resident, South Bend's story became an accessible,
replicable tale of a proud city that was in touch with its history and
confident enough in its future that its mayor was not promising to make
anything great again.

Betsy Hodges, a former mayor of Minneapolis and a friend and supporter
of Buttigieg, points out that the current president makes a particularly
rich foil for a small-town mayor's story. ``I think the Trump agenda and
Trump demeanor have increased our capacity to dehumanize one another,''
she told me. Social media, she added, has already accelerated this
tendency to become detached and alienated from our communities and
leaders. In this regard South Bend is small enough to model a civic
compact, dramatizing how politicians and people and places should relate
to one another. ``A mayor's main agenda is to never forget that a policy
is at its core about people,'' Hodges said.

This can work in both directions, naturally, and reality does tend to
assert itself in unpleasant ways, as inevitably as potholes. ``There is
tremendous accountability that goes with being a mayor,'' said Dan
Pfeiffer, a former campaign and White House aide to Barack Obama who is
unaffiliated with any 2020 candidate. ``Every turd tends to land on your
doorstep. And everyone knows where your doorstep is.''

\textbf{``This hurts,'' Buttigieg} told me at his home before heading
out to the town hall to discuss Eric Logan. ``This really hurts.'' He
seemed to be straining to convince me, acknowledging that he is not
always ``symptomatic'' in exhibiting emotion. He got mixed reviews from
theater-critic pundits who found his ``performance'' at previous
Logan-related events to be lacking on the
\href{https://video.wgbh.org/video/american-experience-clinton-and-crisis-the-oklahoma-city-bombing/}{Bill
Clinton scale of ``I feel your pain'' empathy-showing}. This is not a
new critique of Buttigieg, who has quite clearly contemplated the
subject. ``I think a lot of time when people are talking about what they
want to see you do emotionally, what they really are asking is that they
want you to make them feel a certain way.''

He looked momentarily excited, as if a small epiphany had just struck
him. ``A mayor is sometimes described as having a role of a pastor and a
commander in chief all in one,'' he told me. ``Pastors aren't always the
most emotional, although interestingly it's certainly an important part
of black tradition, and maybe that's part of why these things sometimes
read differently across cultures, right?''

Buttigieg offered his own disposition as being consistent with the aura
he wants to project. ``A big part of what makes this campaign work is an
ability to make people feel things they haven't felt in a while,'' he
told me. ``One of them is hope. Another one of them is calm.''

Image

Supporters and profilers have likened him to Barack Obama, not just in
his appeal to a new generation of political consumers but also in his
intent to create a new way of thinking and discussing
politics.Credit...Angie Smith for The New York Times

Neither quality was in evidence in the crowd at Washington High School.
``We're not running from this,'' Buttigieg insisted there. After about
45 minutes, the gathering had pretty much devolved: shouting and
cross-shouting and a few near-confrontations where it seemed as if
complete bedlam might ensue. No one was asking Mayor Pete to speak
Norwegian. ``Get back to South Carolina,'' a man sitting a few rows
behind me in the auditorium yelled at the mayor. Buttigieg took his
abuse with hands placed in a prayerlike repose over his lips, sitting
perfectly still, except for his shoulders, which rocked ever so
slightly.

I caught up with Buttigieg again a few days later in Miami, where he was
ensconced in a 17th-floor suite at the downtown Hilton. He would be
participating in the second night of the back-to-back Democratic
debates, to be held on Thursday; he came down on Monday, the day after
the town hall. ``We've had some supporter events,'' he explained, which
is usually candidate-speak for ``fund-raising.''

South Bend, he told me, was ``taking a moment to breathe and process
everything.'' This was convenient, because Buttigieg has many donors in
Miami and the June F.E.C. filing deadline was a few days away. His
fund-raising diligence would pay off a few days later when the campaign
announced that it had
\href{https://www.nytimes3xbfgragh.onion/2019/07/01/us/politics/pete-buttigieg-fundraising.html}{collected
\$24.8 million} from more than 230,000 donors for the three-month period
that ended in June. The ``Crisis at Home'' headlines would soon be
replaced with breathless assessments of ``Mayor Pete's Impressive
Haul.''

I asked him whether he had ever considered leaving his mayor's job to
focus on his run for president. ``I re-evaluate that constantly,'' he
told me, though not since the recent turmoil began; if anything, it's
more important than ever that he stay in his job and see the crisis
through at home.

``Yes, but you're in Miami,'' I pointed out: South Beach, not South
Bend.

``Yes,'' he acknowledged. ``But I'm in charge.''

Buttigieg seemed entirely at home here, surrounded by political
tourists, reporters and strategists, some of whom he had known since his
days as president of Harvard's Institute of Politics. Scores of campaign
aides and blow-dried TV people and candidates were cavorting in the
lobby. The place had a political-summer-camp feel reminiscent of a party
convention or caucus night at the bar of the Marriott-Des Moines. Beto
O'Rourke walked by on the way upstairs, nursing a venti cup of
something. ``I think Hick is staying here, too,'' Buttigieg said,
presumably referring to another fellow camper, the former Colorado
governor John Hickenlooper.

Amid the attention paid to Buttigieg's eclecticisms --- his frequent
literary references, his ability to speak eight languages, his classical
piano training and Radiohead fandom --- it's easy to overlook the fact
that he is, at heart, a fairly conventional political animal. Buttigieg
is steeped in campaign life, having worked for John Kerry in 2004 and
Obama in 2008, and he tends to talk, more than most candidates, like an
operative. In 2017, he ran unsuccessfully to be chairman of the
Democratic National Committee --- a position that is essentially that of
a glorified fund-raiser, talking head and political strategist rolled
into one. His early ambitions, his methodical climb up the
accomplishment ladder and his youthful attention to networking have more
in common with Bill Clinton than Obama.

This is relevant, Pfeiffer pointed out, because ``Mayor Pete's message
is basically a punditlike critique of politics.'' He talks a lot about
how Democrats must reach voters in the Midwest, the importance of
reaching faith-based citizens and how it's time for the country to
``change the channel'' from the tired horror show of our recent
political battles. Buttigieg's campaign has, to this point, been short
on policy details and heavy on ``laying out the values,'' as he often
says. In watching Buttigieg, the values are more about the vehicle: that
is, Mayor Pete himself. It's easy to overlook that the campaign has
largely been personality-based to this point --- much more about the
Buttigieg résumé, quirkiness and style than any ideological or policy
direction. But part of that style is self-conscious humility, the idea
that while the mayor might be a singular generational hope, at least
he's sheepish about it. Buttigieg has perfected the cultivated modesty
of the millennial striver.

I talked to Buttigieg for a final time on July 10, a Wednesday, by
phone. It had been nearly four weeks since the Logan shooting, and he
was about to reveal his oft-mentioned Douglass Plan. It involves
measures to ``dismantle a fundamentally racist criminal-justice system''
and ``directly attack the racial wealth gap, building wealth in black
communities.'' He told me that the Douglass Plan had been in the works
for months, though the Logan incident might have given its release more
urgency and attention. ``I am perhaps the white candidate who will be
asked most frequently about race,'' Buttigieg told me --- a curious
statement given that
\href{https://www.nytimes3xbfgragh.onion/2019/07/15/us/politics/biden-busing.html}{Joe
Biden seems to have spent much of the last month being questioned about
little else}.

Buttigieg told me that if he was not a politician, he might have been a
writer. ``If I was more creative, I would have been a novelist,'' he
told me. ``I can do the prose, but I just don't have the imagination
that it takes.''

He seems to very much enjoy the narrative journey of campaign life, with
its cinematic pace and stranger-than-fiction turns. Best of all is that
he gets to be the central player in his story. He used to partake of
politics as a spectator sport, as a detached observer. ``Well, it's not
so much a refuge from the day to day to be following national matters
anymore,'' he told me. ``If nothing else, being in the middle of this
has allowed me to shed a lot of the illusions of how it all works.''
How? ``Well,'' he said. ``I've discovered that a show like `Veep' is
more realistic than most Americans would care to imagine.''

Image

``I am perhaps the white candidate who will be asked most frequently
about race.''Credit...Angie Smith for The New York Times

Despite all the attention he has received, Buttigieg remains very much a
long shot in the race. He has in recent polls
\href{https://www.nytimes3xbfgragh.onion/interactive/2020/us/elections/democratic-polls.html}{dropped
solidly behind the top group of candidates}: Biden, Bernie Sanders,
Elizabeth Warren and Kamala Harris, the latter two of whom have
inherited the star status Buttigieg enjoyed for much of the spring. His
fund-raising ensures he will be around for a while, and his performance
in the Miami debate was generally well regarded --- especially his blunt
assessment of the ``mess'' he left behind in South Bend. But the joy
ride of his early campaign months now calls for a next turn.

In the coming weeks, Buttigieg said he would be releasing more detailed
policy plans. ``We've laid out the values, now we lay out the details,''
he said. That will be the next phase, if not the next act. I heard a
flurry of screeches and beeping over the phone in the background. The
mayor of South Bend was in a hurry, as ever, and announced that he had
to jump on another thing. He was in a car, in Washington for the day. He
was not sure where they were or were headed, exactly. ``I'm glimpsing at
some shiny buildings,'' he said.

\hypertarget{our-2020-election-guide}{%
\section{Our 2020 Election Guide}\label{our-2020-election-guide}}

Updated ~Sept. 7, 2020

\begin{center}\rule{0.5\linewidth}{\linethickness}\end{center}

\begin{itemize}
\item ~
  \hypertarget{the-latest}{%
  \subsection{The Latest}\label{the-latest}}

  \begin{itemize}
  \item
    The unofficial Labor Day kickoff to the fall presidential campaign
    centered on Pennsylvania and Wisconsin,
    \href{https://www.nytimes3xbfgragh.onion/2020/09/07/us/politics/wisconsin-biden-harris-trump-pence.html?action=click\&pgtype=Article\&state=default\&region=BELOW_MAIN_CONTENT\&context=storylines_guide}{two
    pivotal states for both President Trump and Joseph R. Biden Jr}.
  \end{itemize}
\item ~
  \hypertarget{how-to-win-270}{%
  \subsection{How to Win 270}\label{how-to-win-270}}

  \begin{itemize}
  \item
    Joe Biden and Donald Trump need 270 electoral votes to reach the
    White House. Try building
    \href{https://www.nytimes3xbfgragh.onion/interactive/2020/us/elections/election-states-biden-trump.html?action=click\&pgtype=Article\&state=default\&region=BELOW_MAIN_CONTENT\&context=storylines_guide}{your
    own coalition of battleground states}~to see potential outcomes.
  \end{itemize}
\item ~
  \hypertarget{voting-by-mail}{%
  \subsection{Voting by Mail}\label{voting-by-mail}}

  \begin{itemize}
  \item
    Will you have enough time to vote by mail in your state? Yes, but
    it's risky to procrastinate.
    \href{https://www.nytimes3xbfgragh.onion/interactive/2020/08/31/us/politics/vote-by-mail-deadlines.html?action=click\&pgtype=Article\&state=default\&region=BELOW_MAIN_CONTENT\&context=storylines_guide}{Check
    your state's deadline.}
  \item
    \href{https://www.nytimes3xbfgragh.onion/interactive/2020/us/elections/joe-biden.html?action=click\&pgtype=Article\&state=default\&region=BELOW_MAIN_CONTENT\&context=storylines_guide}{}

    \hypertarget{joe-biden}{%
    \section{Joe Biden}\label{joe-biden}}

    \hypertarget{democrat}{%
    \subsection{Democrat}\label{democrat}}

    \href{https://www.nytimes3xbfgragh.onion/interactive/2020/us/elections/donald-trump.html?action=click\&pgtype=Article\&state=default\&region=BELOW_MAIN_CONTENT\&context=storylines_guide}{}

    \hypertarget{donald-trump}{%
    \section{Donald Trump}\label{donald-trump}}

    \hypertarget{republican}{%
    \subsection{Republican}\label{republican}}
  \end{itemize}
\item
  \hypertarget{keep-up-with-our-coverage}{%
  \subsection{Keep Up With Our
  Coverage}\label{keep-up-with-our-coverage}}

  \begin{itemize}
  \item
    Get an
    \href{https://www.nytimes3xbfgragh.onion/newsletters/politics?action=click\&pgtype=Article\&state=default\&region=BELOW_MAIN_CONTENT\&context=storylines_guide}{email}~recapping
    the day's news
  \item
    Download our mobile app on
    \href{https://apps.apple.com/us/app/nytimes/id284862083?ls=1\&mat_click_id=5c79ae7455014fd1bd66b5610c05b8f2-20191112-16948\&referrer=mat_click_id\%3D5c79ae7455014fd1bd66b5610c05b8f2-20191112-16948\%26link_click_id\%3D722930677036718082}{iOS}~and
    \href{http://a.localytics.com/android?id=com.nytimes.android\&referrer=utm_source\%3Dother_nyt_mobile_web\%26utm_medium\%3DWeb\%2520page\%26utm_term\%3DGeneral\%2520Mobile\%2520Page\%26utm_campaign\%3DNYT\%2520Mobile\%2520General\%2520Page}{Android}~and
    turn on Breaking News and Politics alerts
  \end{itemize}
\end{itemize}

Advertisement

\protect\hyperlink{after-bottom}{Continue reading the main story}

\hypertarget{site-index}{%
\subsection{Site Index}\label{site-index}}

\hypertarget{site-information-navigation}{%
\subsection{Site Information
Navigation}\label{site-information-navigation}}

\begin{itemize}
\tightlist
\item
  \href{https://help.nytimes3xbfgragh.onion/hc/en-us/articles/115014792127-Copyright-notice}{©~2020~The
  New York Times Company}
\end{itemize}

\begin{itemize}
\tightlist
\item
  \href{https://www.nytco.com/}{NYTCo}
\item
  \href{https://help.nytimes3xbfgragh.onion/hc/en-us/articles/115015385887-Contact-Us}{Contact
  Us}
\item
  \href{https://www.nytco.com/careers/}{Work with us}
\item
  \href{https://nytmediakit.com/}{Advertise}
\item
  \href{http://www.tbrandstudio.com/}{T Brand Studio}
\item
  \href{https://www.nytimes3xbfgragh.onion/privacy/cookie-policy\#how-do-i-manage-trackers}{Your
  Ad Choices}
\item
  \href{https://www.nytimes3xbfgragh.onion/privacy}{Privacy}
\item
  \href{https://help.nytimes3xbfgragh.onion/hc/en-us/articles/115014893428-Terms-of-service}{Terms
  of Service}
\item
  \href{https://help.nytimes3xbfgragh.onion/hc/en-us/articles/115014893968-Terms-of-sale}{Terms
  of Sale}
\item
  \href{https://spiderbites.nytimes3xbfgragh.onion}{Site Map}
\item
  \href{https://help.nytimes3xbfgragh.onion/hc/en-us}{Help}
\item
  \href{https://www.nytimes3xbfgragh.onion/subscription?campaignId=37WXW}{Subscriptions}
\end{itemize}
