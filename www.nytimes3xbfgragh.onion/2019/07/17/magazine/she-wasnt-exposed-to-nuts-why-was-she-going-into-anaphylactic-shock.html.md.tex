Sections

SEARCH

\protect\hyperlink{site-content}{Skip to
content}\protect\hyperlink{site-index}{Skip to site index}

\href{https://myaccount.nytimes3xbfgragh.onion/auth/login?response_type=cookie\&client_id=vi}{}

\href{https://www.nytimes3xbfgragh.onion/section/todayspaper}{Today's
Paper}

She Wasn't Exposed to Nuts. Why Was She Going Into Anaphylactic Shock?

\begin{itemize}
\item
\item
\item
\item
\item
\item
\end{itemize}

Advertisement

\protect\hyperlink{after-top}{Continue reading the main story}

Supported by

\protect\hyperlink{after-sponsor}{Continue reading the main story}

\href{/column/diagnosis}{Diagnosis}

\hypertarget{she-wasnt-exposed-to-nuts-why-was-she-going-into-anaphylactic-shock}{%
\section{She Wasn't Exposed to Nuts. Why Was She Going Into Anaphylactic
Shock?}\label{she-wasnt-exposed-to-nuts-why-was-she-going-into-anaphylactic-shock}}

\includegraphics{https://static01.graylady3jvrrxbe.onion/images/2019/07/21/magazine/21mag-diagnosis-1/21mag-diagnosis-1-articleLarge.png?quality=75\&auto=webp\&disable=upscale}

By \href{https://www.nytimes3xbfgragh.onion/by/lisa-sanders-md}{Lisa
Sanders, M.D.}

\begin{itemize}
\item
  July 17, 2019
\item
  \begin{itemize}
  \item
  \item
  \item
  \item
  \item
  \item
  \end{itemize}
\end{itemize}

``Call 911,'' the 21-year-old woman gasped to her older sister. ``I'm
having an attack.'' The older sister looked over at the young woman. She
looked scared, a thin layer of sweat glistening on her pale face. The
older sister immediately picked up the phone and dialed. She'd heard
about her sister's strange attacks but had never seen one. Her younger
sister had come to her bedroom earlier that evening, saying that the
inside of her mouth was itchy and that she was worried this was an
allergic reaction. She didn't want to be alone, in case the two doses of
Benadryl she'd already taken weren't enough.

The young woman lay motionless on the bed, her heart pounding, too weak
to move. Her eyes were closed, and her breathing was fast and noisy.
Suddenly she sat up and gestured toward the trash can near the door. Her
sister brought it over, and she vomited. Then her eyes rolled back, and
she slumped onto the bed. ``Talk to me!'' the older sister shouted into
her ear.

There was a sound at the door, and two firefighters rushed into the
room. The young woman's blood pressure was too low to measure. An
emergency medical worker gave her epinephrine; another called for an
ambulance.

The young woman was awake by the time she got to the emergency room at
N.Y.U. hospital in Manhattan. She got another dose of epinephrine and
some IV fluids. The doctors were pretty sure that she'd had an
anaphylactic reaction. She had a deadly allergy to peanuts and all nuts.
It was so bad that just being around someone who'd eaten them hours
earlier could bring on symptoms. But the young woman was very careful
and knew she hadn't ingested anything with nuts.

\hypertarget{scary-reactions}{%
\subsection{\texorpdfstring{\textbf{Scary
Reactions}}{Scary Reactions}}\label{scary-reactions}}

This was the third allergy-like incident she'd had in the previous six
months. The first was the worst, because she was alone. In January,
while taking a shower, she felt suddenly lightheaded. She hurried out
and lay on the floor, conditioner from her hair pooling around her. Then
her heart started racing. Moments later, she felt herself heave, and she
turned her head to vomit. Soon after, cramping in her lower belly warned
her of impending diarrhea. She pulled herself into the bathroom and onto
the toilet. When the paroxysms passed, she lowered herself to the floor,
unable to do anything more.

When her strength returned, she called her parents and then went to see
her doctor. He examined her and took blood. Everything was normal. A
second episode happened in May. It was just like the first, except she
wasn't alone, and she wasn't naked. A friend called an ambulance. Those
doctors were baffled as well.

After that attack, she called her cardiologist. As a child, she
developed a rapid heartbeat caused by abnormal cardiac signaling. Her
cardiologist figured out that problem back then. Now he ordered an
echocardiogram and a Holter monitor to record her every heartbeat. Both
were normal. So were all the other tests he ordered.

After the third attack at her sister's, she went to her allergist. He
tested her for all kinds of non-nut allergies that could lead to
anaphylaxis, the most serious allergic response. He asked her about what
she'd eaten or been exposed to before each attack, but he couldn't
identify a trigger.

\includegraphics{https://static01.graylady3jvrrxbe.onion/images/2019/07/21/magazine/21mag-Diagnosis-2/21mag-Diagnosis-2-articleLarge.png?quality=75\&auto=webp\&disable=upscale}

\hypertarget{diagnostic-flash-mob}{%
\subsection{\texorpdfstring{\textbf{Diagnostic Flash
Mob}}{Diagnostic Flash Mob}}\label{diagnostic-flash-mob}}

That summer, a family friend put her in touch with Dr. Bradley Benson at
the University of Minnesota Medical Center-Fairview. Benson, a
specialist in both medicine and pediatrics, has a reputation for solving
tough cases. He agreed to see her and told her to write down everything
she remembered about the attacks and send it to him, along with all the
records she could gather.

At the university, Benson was surrounded by doctors --- many he'd
trained and then hired --- who were as fascinated by difficult diagnoses
as he was. When the documents arrived, he organized something he called
a diagnostic flash mob. As he had for the past decade, when facing a
tough diagnostic problem, he sent out a message inviting anyone who was
free to help him think through the case.

A week before the young woman was scheduled to visit, Benson presented
her case to his colleagues, his flash mob. One doctor pointed out that
no one had identified what had caused her allergic reaction --- why not?
Perhaps she was allergic to something rare, suggested another. Could she
have catamenial anaphylaxis, a reaction caused by a hormone of her
menstrual cycle? Or what if there was no specific trigger --- what if
the whole system was somehow in overdrive?

Or maybe this wasn't anaphylaxis, offered another. Maybe it's her heart
after all. Benson took notes throughout the discussion.

\hypertarget{possible-disorders}{%
\subsection{\texorpdfstring{\textbf{Possible
Disorders}}{Possible Disorders}}\label{possible-disorders}}

The following week, the young woman and her parents flew from New York
to see Benson. He spent more than an hour with her, going over these
episodes and her medical history. He examined her, then sent her to the
lab to get a few tests. When the results came back, Benson put together
a letter for her and her doctor outlining all the diagnostic
possibilities he'd considered and how likely he thought each was. Like
her allergist, he couldn't find a pattern to these attacks. It wasn't
food, or some environmental exposures or even her monthly hormonal
changes. And if there was no trigger to what sounded very much like an
allergic reaction, Benson's list of the likely possibilities got much
shorter. And at the top of that list were diseases of mast-cell
proliferation.

Allergic reactions are caused by a part of the immune system called mast
cells. When these cells encounter bacteria, or something they mistake
for an invader (i.e., the allergen), they release chemicals that cause
blood vessels to enlarge and tissues to swell and become more permeable
so that the fighter cells of the immune system can get in and destroy
the invader. At their mildest, these cells cause the swollen, runny nose
of allergic rhinitis. At the other end of the spectrum, they can cause
the hypotension and swollen, closed airway or gastrointestinal distress
of anaphylaxis.

There are unusual disorders caused by the development of too many mast
cells. The most common of these is called systemic mastocytosis, in
which many of those mast cells develop a mutation that makes them more
excitable and more easily triggered. Once set off, these overabundant,
hyperactive cells turn their biological weapons against the self,
causing a severe allergic reaction.

To test his theory, Benson explained to the patient, right after her
next attack she needed to get her blood tested for the two most
important chemicals in the mast-cell armamentarium: histamine and
tryptase. If she had a mast-cell disorder, these would be sky-high.

\hypertarget{testing-a-theory}{%
\subsection{\texorpdfstring{\textbf{Testing a
Theory}}{Testing a Theory}}\label{testing-a-theory}}

Three months later, she had one more episode and one more ambulance ride
to the E.R. A card in her wallet instructed the doctors about which
blood test to order. Both chemicals were elevated. Additional testing
revealed that she had the mutation found in systemic mastocytosis. This
patient, like most people with this disorder, has an indolent form. They
have too many mast cells, but the number of cells is stable. There are
other versions, less common and more dangerous, in which the mast cells
continue to proliferate, ultimately crowding out other types of blood
cells in the body.

To prevent these attacks, the patient's mast cells have to be
controlled. Twice a day, she takes an antihistamine and an antacid ---
medications that block histamine, the primary actor in allergies. And
once a day, she takes vitamin D to make her overexcitable mast cells
less likely to erupt into an allergic reaction. She recently added a new
medication --- a shot she gets every three weeks --- to block mast-cell
triggers. It's a lot of medicine for someone who had never taken
anything. But it's much better than the alternative. And she hasn't had
another attack since.

Advertisement

\protect\hyperlink{after-bottom}{Continue reading the main story}

\hypertarget{site-index}{%
\subsection{Site Index}\label{site-index}}

\hypertarget{site-information-navigation}{%
\subsection{Site Information
Navigation}\label{site-information-navigation}}

\begin{itemize}
\tightlist
\item
  \href{https://help.nytimes3xbfgragh.onion/hc/en-us/articles/115014792127-Copyright-notice}{©~2020~The
  New York Times Company}
\end{itemize}

\begin{itemize}
\tightlist
\item
  \href{https://www.nytco.com/}{NYTCo}
\item
  \href{https://help.nytimes3xbfgragh.onion/hc/en-us/articles/115015385887-Contact-Us}{Contact
  Us}
\item
  \href{https://www.nytco.com/careers/}{Work with us}
\item
  \href{https://nytmediakit.com/}{Advertise}
\item
  \href{http://www.tbrandstudio.com/}{T Brand Studio}
\item
  \href{https://www.nytimes3xbfgragh.onion/privacy/cookie-policy\#how-do-i-manage-trackers}{Your
  Ad Choices}
\item
  \href{https://www.nytimes3xbfgragh.onion/privacy}{Privacy}
\item
  \href{https://help.nytimes3xbfgragh.onion/hc/en-us/articles/115014893428-Terms-of-service}{Terms
  of Service}
\item
  \href{https://help.nytimes3xbfgragh.onion/hc/en-us/articles/115014893968-Terms-of-sale}{Terms
  of Sale}
\item
  \href{https://spiderbites.nytimes3xbfgragh.onion}{Site Map}
\item
  \href{https://help.nytimes3xbfgragh.onion/hc/en-us}{Help}
\item
  \href{https://www.nytimes3xbfgragh.onion/subscription?campaignId=37WXW}{Subscriptions}
\end{itemize}
