\href{/section/politics}{Politics}\textbar{}Pete Buttigieg's Life in the
Closet

\url{https://nyti.ms/30z9qGp}

\begin{itemize}
\item
\item
\item
\item
\item
\item
\end{itemize}

\begin{itemize}
\item
  \href{https://www.nytimes3xbfgragh.onion/live/2020/09/08/us/trump-vs-biden?action=click\&pgtype=Article\&state=default\&region=TOP_BANNER\&context=storylines_menu}{Election
  Updates}
\item
  \href{https://www.nytimes3xbfgragh.onion/interactive/2020/us/elections/election-states-biden-trump.html?action=click\&pgtype=Article\&state=default\&region=TOP_BANNER\&context=storylines_menu}{Paths
  to 270}
\item
  \href{https://www.nytimes3xbfgragh.onion/interactive/2020/08/31/us/politics/vote-by-mail-deadlines.html?action=click\&pgtype=Article\&state=default\&region=TOP_BANNER\&context=storylines_menu}{Voting
  by Mail}
\item
  \href{https://www.nytimes3xbfgragh.onion/interactive/2019/us/elections/2020-presidential-election-calendar.html?action=click\&pgtype=Article\&state=default\&region=TOP_BANNER\&context=storylines_menu}{Key
  Dates}
\item
  \href{https://www.nytimes3xbfgragh.onion/newsletters/politics?action=click\&pgtype=Article\&state=default\&region=TOP_BANNER\&context=storylines_menu}{Politics
  Newsletter}
\end{itemize}

\includegraphics{https://static01.graylady3jvrrxbe.onion/images/2019/07/14/us/politics/14buttigieg-closet-p1/00buttigieg-closet-articleLarge.jpg?quality=75\&auto=webp\&disable=upscale}

Sections

\protect\hyperlink{site-content}{Skip to
content}\protect\hyperlink{site-index}{Skip to site index}

THE LONG RUN

\hypertarget{pete-buttigiegs-life-in-the-closet}{%
\section{Pete Buttigieg's Life in the
Closet}\label{pete-buttigiegs-life-in-the-closet}}

And why it took him until he was 33 to come out.

Pete Buttigieg as a senior at Harvard in 2004.Credit...Harvard
University

Supported by

\protect\hyperlink{after-sponsor}{Continue reading the main story}

\href{https://www.nytimes3xbfgragh.onion/by/jeremy-w-peters}{\includegraphics{https://static01.graylady3jvrrxbe.onion/images/2018/11/06/multimedia/author-jeremy-w-peters/author-jeremy-w-peters-thumbLarge.png}}

By \href{https://www.nytimes3xbfgragh.onion/by/jeremy-w-peters}{Jeremy
W. Peters}

\begin{itemize}
\item
  July 14, 2019
\item
  \begin{itemize}
  \item
  \item
  \item
  \item
  \item
  \item
  \end{itemize}
\end{itemize}

The closet that Pete Buttigieg built for himself in the late 1990s and
2000s was a lot like the ones that other gay men of his age and ambition
hid inside. He dated women, deepened his voice and furtively looked at
MySpace and Friendster profiles of guys who had come out --- all while
wondering when it might be safe for him to do so too.

Chris Pappas, who was two years ahead of Mr. Buttigieg at Harvard and is
now a Democratic congressman from New Hampshire, said he arrived at
college ``pretty much convinced that I couldn't have a career or pursue
politics as an L.G.B.T. individual.'' Jonathan Darman, who was one class
ahead of Mr. Buttigieg, remembered how people often reacted to a
politician's coming out then: ``It wasn't a story of love but of
acknowledging illicit desire.'' And Amit Paley, who graduated in Mr.
Buttigieg's class, recalled that ``it was still a time where vocalizing
anti-gay sentiments was not only common, but I think pretty accepted.''

The thought that 15 years later someone they might have shared a dorm or
sat in a lecture hall with would become the first serious openly gay
candidate for president of the United States never crossed their minds.
But no one would have found the possibility more implausible than the
young man everyone on campus knew as Peter.

Mr. Buttigieg, now the mayor of South Bend, Ind., struggled for a decade
after leaving Harvard to overcome the fear that being gay was ``a career
death sentence,'' as he put it in his memoir.

Many in his generation and in his college class decided to come out as
young adults, whether they were confident they would be accepted or not,
and had their 20s to navigate being open about their identity --- a
process that helped make Americans more aware and accepting of their gay
friends, family members and co-workers. Instead, Mr. Buttigieg spent
those years trying to reconcile his private life with his aspirations
for a high-profile career in public service.

Attitudes toward gay rights changed immensely during that period, though
he acknowledges that he was not always able or willing to see what
broader social and legal shifts meant for him personally.

``Because I was wrestling with this, I'm not sure I fully processed the
idea that it related to me,'' he said in an interview.

More than most people his age --- even more than most of the ambitious
young men and women he competed against at Harvard --- he possessed a
remarkably strong drive for perfection. He went on to become a Rhodes
scholar, work on a presidential campaign, join the military and be
elected mayor all before he turned 30. After being deployed with the
Navy to Afghanistan in 2014, he said he realized he could die having
never been in love, and he resolved to change that. He finally came out
in 2015, when he was 33.

\hypertarget{someone-who-would-run-for-president}{%
\subsection{`Someone Who Would Run for
President'}\label{someone-who-would-run-for-president}}

He took a longer journey than his peers did, he has said, because of the
inner turmoil he experienced over whether in fact he wanted to be known
as the ``gay'' politician.

\includegraphics{https://static01.graylady3jvrrxbe.onion/images/2019/04/19/us/14buttigieg-rally3/14buttigieg-rally3-videoSixteenByNine3000.jpg}

His record of accomplishment during those years in the closet is
impossible to separate from the isolation and anxiety he felt as he
weighed the cost of telling his family, friends and constituents who he
really was. Pursuing so many goals had two outcomes, intentionally or
not: It distracted his busy brain from a reality he wasn't ready to
face, and provided him the armor of a life experience that would make
his sexual orientation just one of a litany of attributes.

``Peter struck me very early on, at 18 or 19, as someone who would run
for president regardless,'' said Randall Winston, a close friend of Mr.
Buttigieg's from college. Over beers and Chinese food, Mr. Winston said,
they spent late nights on campus talking about the right and wrong
reasons for getting into politics. ``If you want to be a political
leader, why?'' he recalled. ``Is it about yourself? Is it really about
the good of the nation? I think he was asking himself those questions
from the jump.''

Mr. Buttigieg said in the interview that if he had been interested in a
career other than politics, he would have found the decision to come out
much easier. ``The arts is one where you could have jumped in there in
the 2000s, and it would have been sort of incidental,'' he said.
``Whereas something like finance, it was getting there. And in politics
it would have been completely defining.''

Few experiences in his young adulthood were as formative in shaping his
identity as the hypercompetitive environment he encountered at Harvard.
Even liberal Cambridge, where meeting a gay student or professor would
have been fairly unremarkable, did not always nurture the sense of
confidence that he and many of his gay classmates felt they needed to be
themselves. At times their surroundings seemed to do just the opposite.

\includegraphics{https://static01.graylady3jvrrxbe.onion/images/2019/07/12/us/politics/00buttigieg-closet3/00buttigieg-closet3-articleLarge.jpg?quality=75\&auto=webp\&disable=upscale}

In interviews with a dozen of Mr. Buttigieg's friends and classmates,
people described a culture in which a mix of abundant ambition and
youthful insecurity made students carefully attuned to the way they
presented themselves to others.

Mr. Winston recalled the dual pressures of having high expectations for
yourself while also being aware --- sometimes realistically, sometimes
not --- that your classmates and professors had their own ideas about
who you were too.

``I don't want to say it's all artifice --- a lot of this is just common
to growing up,'' he said. But the culture at Harvard, he added, caused a
lot of students to think, ```O.K., I'm going to maintain this aura, this
impression I'm giving to others.'''

\hypertarget{a-life-his-teenage-self-wouldnt-believe}{%
\subsection{A Life His Teenage Self Wouldn't
Believe}\label{a-life-his-teenage-self-wouldnt-believe}}

Describing the insecurities he felt as a young man, Mr. Buttigieg has
said he sometimes marvels at how differently the world treats him today
compared with what he expected when he was too afraid to come out. On
the day he kicked off his presidential campaign, he said he had imagined
what he would say to his teenage self. ``To tell him that on that day he
announces his campaign for president, he'll do it with his husband
looking on,'' he said with a note of disbelief in his voice. ``Would he
believe me?''

Mr. Buttigieg took a long and fraught path from life as an undergraduate
who once had a girlfriend to a presidential candidate who travels the
country with his husband in tow. While he was still in the closet, the
country became a different place very quickly. And to understand Mr.
Buttigieg's journey is to understand the microgeneration in which he
came of age.

Image

A vigil for Matthew Shepard in 1998.Credit...Evan Agostini/Getty Images

When members of the Harvard class of 2004 were juniors in high school,
Matthew Shepard, a 21-year-old gay man from Wyoming, was bludgeoned,
tied to a fence post and left to die in a murder that shocked the
nation's conscience. By the time they shipped off to Cambridge, few
would have any gay friends --- at least ones who were open about it. And
the idea of a man marrying another man, or a woman marrying another
woman, seemed almost absurd. The closest thing gay men and lesbians had
to marriage was a civil union, which in 2000 was legal in exactly one
state: Vermont.

``Gay marriage was not this obvious liberal no-brainer,'' said Mr.
Darman, a journalist and historian who came out in his senior year of
college, 12 years before Mr. Buttigieg would. While Harvard was
certainly a liberal bubble, it was still in many ways very socially
conventional in the early 2000s, he said. ``In a lot of social settings
at Harvard in that period, the default assumption was that you were
straight. And that would not have been true even five years later.''

Friends and classmates remembered Mr. Buttigieg as thoughtful and
clearly on a trajectory that would bring him success of some kind, even
if it dawned on few of them that might mean the White House.

One thing no one seemed to peg him for was someone wrestling with being
gay. He was so discreet that many of his friends and classmates said in
interviews that they never would have guessed he was hiding anything
until he told them. He left the testosterone-fueled campus sex banter to
others. Hegel and de Tocqueville were more to his conversational tastes.

``His sexuality didn't present as a really big thing in his life,'' said
Joe Flood, a classmate. ``I think he always thought about himself
politically,'' he added, noting that Mr. Buttigieg would become active
in the university's Institute of Politics, an organization at the
Kennedy School of Government that hosted big-name politicians like
Senator Edward M. Kennedy and Howard Dean during their time in school.
``You don't end up there accidentally,'' Mr. Flood said.

By the beginning of his sophomore year, Mr. Buttigieg had been elected
to lead one of the institute's committees. When he was a junior, he was
\href{https://www.thecrimson.com/article/2002/12/6/iop-members-elect-new-board-the/}{elected}
as its president. His platform, in part, called for strengthening the
community of politically minded students by having gatherings to watch
``The West Wing.'' He wrote for The Crimson under the byline Peter P.M.
Buttigieg and covered subjects as disparate as
\href{https://www.thecrimson.com/article/2003/11/10/rock-the-vote-this-week-i/}{Dave
Matthews} and
\href{https://www.thecrimson.com/article/2003/9/29/frightenedand-fighting-fear-if-you-feel/}{Yeats}.
For fun, he and his friends sometimes indulged themselves with
less-than-puerile pursuits like a day trip to the birthplace of John
Adams.

There was a small, close-knit social circle of L.G.B.T.Q. students. But
they existed a world apart from Mr. Buttigieg's Harvard.

\href{https://www.nytimes3xbfgragh.onion/news-event/2020-election}{Election
2020 ›}

\hypertarget{live-updates}{%
\subsection{\texorpdfstring{\href{https://www.nytimes3xbfgragh.onion/live/2020/09/08/us/trump-vs-biden}{Live
Updates}}{Live Updates}}\label{live-updates}}

\href{https://www.nytimes3xbfgragh.onion/live/2020/09/08/us/trump-vs-biden\#a-top-house-democrat-calls-for-the-suspension-of-postmaster-general-louis-dejoy-over-campaign-finance-allegations}{}

Sept. 8, 2020, 10:00 a.m. ET

\href{https://www.nytimes3xbfgragh.onion/live/2020/09/08/us/trump-vs-biden\#a-top-house-democrat-calls-for-the-suspension-of-postmaster-general-louis-dejoy-over-campaign-finance-allegations}{A
top House Democrat calls for the suspension of Postmaster General Louis
DeJoy over campaign finance
allegations.}\href{https://www.nytimes3xbfgragh.onion/live/2020/09/08/us/trump-vs-biden\#in-a-closely-watched-new-hampshire-primary-democrats-will-pick-a-challenger-to-governor-sununu}{}

Sept. 8, 2020, 9:37 a.m. ET

\href{https://www.nytimes3xbfgragh.onion/live/2020/09/08/us/trump-vs-biden\#in-a-closely-watched-new-hampshire-primary-democrats-will-pick-a-challenger-to-governor-sununu}{In
a closely watched New Hampshire primary, Democrats will pick a
challenger to Governor
Sununu.}\href{https://www.nytimes3xbfgragh.onion/live/2020/09/08/us/trump-vs-biden\#pence-and-harris-vied-for-wisconsin-a-pivotal-state-in-the-2020-race}{}

Sept. 8, 2020, 8:40 a.m. ET

\href{https://www.nytimes3xbfgragh.onion/live/2020/09/08/us/trump-vs-biden\#pence-and-harris-vied-for-wisconsin-a-pivotal-state-in-the-2020-race}{Pence
and Harris vied for Wisconsin, a pivotal state in the 2020 race.}

``We were definitely on opposite ends of the gay spectrum --- he was
closeted and I was literally the campus drag queen, Miss Harvard 2002,''
said William Lee Adams, who graduated in Mr. Buttigieg's class and is
now a broadcaster at the BBC World Service in London. Mr. Adams started
coming out at age 12. Arriving at Harvard from his home in Georgia, he
said, was like ``fleeing the desert.'' The two were not friends, though
Mr. Adams did recall his classmate as ``sweet but rather serious.''

At the time, Mr. Adams said he was somewhat resentful of his peers who
kept their identities hidden, having been bullied at school while he was
growing up. Now, however, he is far more sympathetic because he better
understands how personal it is to come out. ``I felt a great sense of
freedom at Harvard that I had never felt before because I could be out
and not have food thrown at me,'' he said. ``Whereas Pete must have felt
trapped, like he was in a straitjacket.''

Mr. Flood, who wrote for The Crimson and knew Mr. Buttigieg as a friend,
said that someone who worked so hard and thought so intensely about his
future had to feel frustrated as he realized there was this immutable
aspect of his life he was helpless to change.

``It's like the one thing he couldn't control about who he was and how
he was going to present and how he was going to do all these things,''
he said.

Image

A couple celebrated after applying for a marriage license in Boston
after Massachusetts legalized same-sex unions in May 2004. Credit...Ruth
Fremson/ The New York Times

But when Mr. Buttigieg and his peers left college and started embarking
on their professional lives, the country was changing in significant
ways, jolting their sense of what it could mean to be openly gay and
have a high-profile career.

One of the biggest developments was right in Harvard's backyard. In
2004, Massachusetts became the first state where same-sex couples could
marry. Students flocked to Cambridge City Hall in the early-morning
hours on May 17 to watch the first couples wed at 12:01 a.m. --- the
earliest moment possible under the new law. Mr. Buttigieg remembers the
occasion but was not there. ``I don't remember feeling that connected to
it actually,'' he said.

Soon states from Iowa to Maine would start allowing same-sex couples to
marry. Then Congress would repeal the military's ``don't ask, don't
tell'' ban on serving openly as gay or lesbian. And the Supreme Court
would declare the rights of gay men and lesbians to have their
relationships recognized by the state, first in 2013 when it struck down
the Defense of Marriage Act in
\href{https://www.oyez.org/cases/2012/12-307}{United States v. Windsor},
and then again in the 2015 decision that guaranteed same-sex marriage as
a right protected by the Constitution in
\href{https://www.oyez.org/cases/2014/14-556}{Obergefell v. Hodges}.

In 2004, when Mr. Buttigieg's class graduated,
\href{https://www.pewforum.org/fact-sheet/changing-attitudes-on-gay-marriage/}{public
opinion polls} showed that roughly one-third of Americans favored
allowing same-sex couples to marry. A decade later it was more than half
the country and rising.

Image

Rainbow lights illuminated the White House in June 2015, after the
Supreme Court decided that same-sex marriage was a right protected by
the Constitution.Credit...Zach Gibson/The New York Times

Many closeted people found their plight more difficult during the early
years of social and legal change, as they wrestled with whether to
finally open up after years of trying to maintain an impression of
themselves that was false.

Mr. Paley, who was Mr. Buttigieg's college classmate, remembers sitting
in his dorm room in 2003 as a closeted junior and crying as he read
Justice Anthony M. Kennedy's opinion in the landmark case
\href{https://www.oyez.org/cases/2002/02-102}{Lawrence v. Texas}, which
struck down bans on intimacy between homosexuals on grounds that such
laws were an affront to their dignity. ``That helped me realize I can't
live my life this way,'' he said of hiding his sexual orientation. It
took Mr. Paley until the end of his senior year to fully come out, and
he now serves as chief executive of the
\href{https://www.thetrevorproject.org/}{Trevor Project}, an
organization that works to advance the rights of L.G.B.T.Q. youth.

Mr. Pappas, the congressman from New Hampshire, ran his first race for
state legislature in 2002 as an openly gay candidate and won. ``It's an
important facet of who I am,'' he said. ``And I think over time I
realized how powerful it was that I share that with more and more
people.''

He said he ran as an out candidate in that first race because he saw no
point in turning back after he came out in college. And after hearing
from people who told him how encouraging it was to see him as an openly
gay man in politics, Mr. Pappas realized he had made the right choice
regardless of the political implications. ``I don't think I fully
appreciated that at first,'' he said.

\hypertarget{being-gay-is-not-the-only-part}{%
\subsection{Being Gay Is `Not the Only
Part'}\label{being-gay-is-not-the-only-part}}

After he graduated, Mr. Buttigieg went to work for John Kerry's
presidential campaign in Arizona and quickly immersed himself in the
job. Mara Lee, who worked with him at the time and remains a friend,
remembered meeting her co-worker for the first time: ``Here's this guy
who's doing a million things at once. He has seven or eight TVs on to
monitor the local and national news. He's introducing himself to me ---
being genuine --- and having a conversation while typing.'' She
remembers two computer screens on his desk.

Once he came out, she said that being gay was never the first thing he
wanted people to see when they met him --- a veteran, Rhodes scholar,
polyglot who was first elected mayor of South Bend when he was 29.
``While it's an important part of who he is, it's not the only part,''
she said.

When he first ran for mayor in 2011 and won, he was closeted. A local
gay rights group did not initially endorse him in that race, opting
instead for a candidate with a more established track record on the
issues. Mr. Buttigieg endured some awkward moments, like signing a city
law banning discrimination based on sexual orientation in 2012. To not
think about how the law directly affected him, he acknowledged, ``took a
little compartmentalization.''

Image

Mr. Buttigieg's husband, Chasten, regularly accompanies him on the
campaign trail.Credit...Bridget Bennett for The New York Times

His employees and constituents saw an eligible bachelor in their young
mayor and wanted to set him up with their daughters. Some on his staff
even joked about his old light green Ford Taurus as a ``chick magnet.''
He did not bother to correct them.

When he did come out in the summer of 2015, the forum he chose was an
\href{https://www.southbendtribune.com/news/local/south-bend-mayor-why-coming-out-matters/article_4dce0d12-1415-11e5-83c0-739eebd623ee.html}{op-ed}
for The South Bend Tribune. ``It took years of struggle and growth for
me to recognize that it's just a fact of life, like having brown hair,
and part of who I am,'' he wrote.

He may have waited far longer than most young gay men today. But ever
the overachiever, he made record time in setting a new bar. In less than
four years he went from being single and closeted to being married and
out as a gay candidate for president.

\hypertarget{our-2020-election-guide}{%
\section{Our 2020 Election Guide}\label{our-2020-election-guide}}

Updated ~Sept. 8, 2020

\begin{center}\rule{0.5\linewidth}{\linethickness}\end{center}

\begin{itemize}
\item ~
  \hypertarget{the-latest}{%
  \subsection{The Latest}\label{the-latest}}

  \begin{itemize}
  \item
    The campaign
    \href{https://www.nytimes3xbfgragh.onion/live/2020/09/08/us/trump-vs-biden?action=click\&pgtype=Article\&state=default\&region=BELOW_MAIN_CONTENT\&context=storylines_guide}{shifts
    to a higher gear this week}, with President Trump set to visit
    Florida and North Carolina today and Joseph R. Biden heading to
    Michigan tomorrow.
  \end{itemize}
\item ~
  \hypertarget{how-to-win-270}{%
  \subsection{How to Win 270}\label{how-to-win-270}}

  \begin{itemize}
  \item
    Joe Biden and Donald Trump need 270 electoral votes to reach the
    White House. Try building
    \href{https://www.nytimes3xbfgragh.onion/interactive/2020/us/elections/election-states-biden-trump.html?action=click\&pgtype=Article\&state=default\&region=BELOW_MAIN_CONTENT\&context=storylines_guide}{your
    own coalition of battleground states}~to see potential outcomes.
  \end{itemize}
\item ~
  \hypertarget{voting-by-mail}{%
  \subsection{Voting by Mail}\label{voting-by-mail}}

  \begin{itemize}
  \item
    Will you have enough time to vote by mail in your state? Yes, but
    it's risky to procrastinate.
    \href{https://www.nytimes3xbfgragh.onion/interactive/2020/08/31/us/politics/vote-by-mail-deadlines.html?action=click\&pgtype=Article\&state=default\&region=BELOW_MAIN_CONTENT\&context=storylines_guide}{Check
    your state's deadline.}
  \item
    \href{https://www.nytimes3xbfgragh.onion/interactive/2020/us/elections/joe-biden.html?action=click\&pgtype=Article\&state=default\&region=BELOW_MAIN_CONTENT\&context=storylines_guide}{}

    \hypertarget{joe-biden}{%
    \section{Joe Biden}\label{joe-biden}}

    \hypertarget{democrat}{%
    \subsection{Democrat}\label{democrat}}

    \href{https://www.nytimes3xbfgragh.onion/interactive/2020/us/elections/donald-trump.html?action=click\&pgtype=Article\&state=default\&region=BELOW_MAIN_CONTENT\&context=storylines_guide}{}

    \hypertarget{donald-trump}{%
    \section{Donald Trump}\label{donald-trump}}

    \hypertarget{republican}{%
    \subsection{Republican}\label{republican}}
  \end{itemize}
\item
  \hypertarget{keep-up-with-our-coverage}{%
  \subsection{Keep Up With Our
  Coverage}\label{keep-up-with-our-coverage}}

  \begin{itemize}
  \item
    Get an
    \href{https://www.nytimes3xbfgragh.onion/newsletters/politics?action=click\&pgtype=Article\&state=default\&region=BELOW_MAIN_CONTENT\&context=storylines_guide}{email}~recapping
    the day's news
  \item
    Download our mobile app on
    \href{https://apps.apple.com/us/app/nytimes/id284862083?ls=1\&mat_click_id=5c79ae7455014fd1bd66b5610c05b8f2-20191112-16948\&referrer=mat_click_id\%3D5c79ae7455014fd1bd66b5610c05b8f2-20191112-16948\%26link_click_id\%3D722930677036718082}{iOS}~and
    \href{http://a.localytics.com/android?id=com.nytimes.android\&referrer=utm_source\%3Dother_nyt_mobile_web\%26utm_medium\%3DWeb\%2520page\%26utm_term\%3DGeneral\%2520Mobile\%2520Page\%26utm_campaign\%3DNYT\%2520Mobile\%2520General\%2520Page}{Android}~and
    turn on Breaking News and Politics alerts
  \end{itemize}
\end{itemize}

Advertisement

\protect\hyperlink{after-bottom}{Continue reading the main story}

\hypertarget{site-index}{%
\subsection{Site Index}\label{site-index}}

\hypertarget{site-information-navigation}{%
\subsection{Site Information
Navigation}\label{site-information-navigation}}

\begin{itemize}
\tightlist
\item
  \href{https://help.nytimes3xbfgragh.onion/hc/en-us/articles/115014792127-Copyright-notice}{©~2020~The
  New York Times Company}
\end{itemize}

\begin{itemize}
\tightlist
\item
  \href{https://www.nytco.com/}{NYTCo}
\item
  \href{https://help.nytimes3xbfgragh.onion/hc/en-us/articles/115015385887-Contact-Us}{Contact
  Us}
\item
  \href{https://www.nytco.com/careers/}{Work with us}
\item
  \href{https://nytmediakit.com/}{Advertise}
\item
  \href{http://www.tbrandstudio.com/}{T Brand Studio}
\item
  \href{https://www.nytimes3xbfgragh.onion/privacy/cookie-policy\#how-do-i-manage-trackers}{Your
  Ad Choices}
\item
  \href{https://www.nytimes3xbfgragh.onion/privacy}{Privacy}
\item
  \href{https://help.nytimes3xbfgragh.onion/hc/en-us/articles/115014893428-Terms-of-service}{Terms
  of Service}
\item
  \href{https://help.nytimes3xbfgragh.onion/hc/en-us/articles/115014893968-Terms-of-sale}{Terms
  of Sale}
\item
  \href{https://spiderbites.nytimes3xbfgragh.onion}{Site Map}
\item
  \href{https://help.nytimes3xbfgragh.onion/hc/en-us}{Help}
\item
  \href{https://www.nytimes3xbfgragh.onion/subscription?campaignId=37WXW}{Subscriptions}
\end{itemize}
