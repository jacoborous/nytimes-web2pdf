Sections

SEARCH

\protect\hyperlink{site-content}{Skip to
content}\protect\hyperlink{site-index}{Skip to site index}

\href{https://myaccount.nytimes3xbfgragh.onion/auth/login?response_type=cookie\&client_id=vi}{}

\href{https://www.nytimes3xbfgragh.onion/section/todayspaper}{Today's
Paper}

Letter of Recommendation: Piezo Mics

\url{https://nyti.ms/2GbRtWE}

\begin{itemize}
\item
\item
\item
\item
\item
\end{itemize}

Advertisement

\protect\hyperlink{after-top}{Continue reading the main story}

Supported by

\protect\hyperlink{after-sponsor}{Continue reading the main story}

\href{/column/letter-of-recommendation}{Letter of Recommendation}

\hypertarget{letter-of-recommendation-piezo-mics}{%
\section{Letter of Recommendation: Piezo
Mics}\label{letter-of-recommendation-piezo-mics}}

\includegraphics{https://static01.graylady3jvrrxbe.onion/images/2019/07/21/magazine/21Mag-LOR-1/21Mag-LOR-1-articleLarge.jpg?quality=75\&auto=webp\&disable=upscale}

By David Rees

\begin{itemize}
\item
  July 16, 2019
\item
  \begin{itemize}
  \item
  \item
  \item
  \item
  \item
  \end{itemize}
\end{itemize}

I have a confession to share about shame and experimentation in a
windowless room. The good news is it's not about sex; the bad news is
it's about noise music.

It began last winter, when my musically intimidating friend Corey
invited me to rock out. I agreed before remembering how bored and
embarrassed I was of my guitar playing. So I decided to make new sounds
instead. I started modifying old keyboards, rewiring them so they would
distort and glitch --- but I couldn't really play keyboards either. If I
was to maintain my dignity while jamming with Corey, I needed to get
more abstract --- so abstract that musical talent, or its absence,
wouldn't be a factor.

I began searching for old electronic children's toys. By manipulating
their voltages and then running them into an amplifier, I succeeded in
making them sound more ominous and demonic than intended. Corey was
impressed (``Bro, that sounds sick!''). I was relieved.

But even hacking keyboards and old toys comes with limitations,
circumscribed by the chips inside their circuit boards. You can make
interesting sounds --- especially if you incorporate effects pedals ---
but you're still building off the electronic guts you've inherited. I
wanted to increase my options and start playing with sounds made in the
real world. Perhaps a more disciplined musician would have picked up his
guitar again. Instead, I discovered piezo contact microphones, these
light, flat metal discs with a thin ceramic layer on top. They don't
pick up sound vibrations in the air; they're not for singing into.
Instead, they attach directly onto objects, converting surface
vibrations into electric current (and vice versa; smoke alarms use piezo
discs as speakers). Musicians install piezo mics in cigar-box guitars
for amplification; you'll also find pressure-sensitive piezo elements in
underwater microphones.

They look unassuming, but once they're plugged into an amplifier, piezo
discs become psychedelic microscopes for your ears, completely changing
your sense of sonic scale. I taped one to the bottom of a water bottle
on a hot afternoon and ran the signal through a reverb pedal; the ice
cubes banging around sounded like gongs from distant planets. Rubbing a
piezo mic against a felt cowboy hat sent me down a sound-dappled path of
contemplation, musing on the subtleties of surface texture and how
difficult it would be to play croquet on a felt cowboy hat if you were,
say, 10 molecules tall. My dumb guitar never led me to such insights.

Best of all, piezo mics are cheap --- probably one of the most
affordable technologies for completely transforming your appreciation of
our world. You can spend hundreds of dollars on a high-quality contact
mic specially designed for the subtle timbres of an orchestral string
instrument if that's what you want. If what you want, however, is to
drink some beers, plug in some pedals and freak out some friends by
turning an old Garfield paperback into a wailing orgy of dissonance, you
can be up and running for just a few dollars (excluding the cost of the
beers).

A common use of piezo mics among experimental musicians is in ``noise
boxes,'' ungainly contraptions in which household objects are mounted
around and amplified by piezo mics. I've built a few of these, including
an old cookie tin I outfitted with various springs and filled with
Ping-Pong balls. When I shake the tin and turn the volume up, it sounds
as if the springs and the Ping-Pong balls are role-playing the end of
the world. People build noise boxes with combs, wires, silverware,
rubber bands, fidget spinners, sandpaper, old saw blades --- tiny
orchestras of singing objects, monumentalized by the mics placed inside.
Indeed, my hobby has made trash nights newly enticing, as I wander the
neighborhood looking for garbage that might sound interesting. My wife
loves my hobby and never teases me about it.

Earlier this summer, I visited a different windowless room for a
performance by one of the grandmasters of piezo-mic mayhem: Justice
Yeldham, an Australian noise artist known for attaching contact mics to
large pieces of broken glass. Yeldham uses glass like a wind instrument,
smooshing his face against the pane as he blows, hums, bites and
otherwise imitates the world's least subtle peeping Tom. The signal runs
through a small metropolis of effects pedals that amplify and expand the
resonances and sputterings of his mouth against the glass. One moment
you hear John Coltrane playing a volcano, the next you hear a string
section being squeezed through a toothpaste tube. It's a high-stakes,
smeary embouchure that can end with Yeldham's face covered with blood.
(He ended the performance I saw by suddenly breaking his instrument over
his head, something Yo-Yo Ma has yet to do.) This may sound like a
gimmick --- G.G. Allin for grad students --- but
\href{http://dualplover.com/yeldham/}{Yeldham coaxes a truly amazing
variety of sounds from his shard}.

Realizing that Yeldham was \emph{playing} the broken glass --- that he
was bringing talent and discipline to bear on what would usually be
considered detritus --- helped me understand what it is about piezo mics
that excites me so much. They don't just change how I hear things. They
change how I see things: Every object is a potential musical instrument.
Every object is worth engaging with, however briefly, however loudly, as
you seek its potential to wow your friend in a windowless room.

Advertisement

\protect\hyperlink{after-bottom}{Continue reading the main story}

\hypertarget{site-index}{%
\subsection{Site Index}\label{site-index}}

\hypertarget{site-information-navigation}{%
\subsection{Site Information
Navigation}\label{site-information-navigation}}

\begin{itemize}
\tightlist
\item
  \href{https://help.nytimes3xbfgragh.onion/hc/en-us/articles/115014792127-Copyright-notice}{©~2020~The
  New York Times Company}
\end{itemize}

\begin{itemize}
\tightlist
\item
  \href{https://www.nytco.com/}{NYTCo}
\item
  \href{https://help.nytimes3xbfgragh.onion/hc/en-us/articles/115015385887-Contact-Us}{Contact
  Us}
\item
  \href{https://www.nytco.com/careers/}{Work with us}
\item
  \href{https://nytmediakit.com/}{Advertise}
\item
  \href{http://www.tbrandstudio.com/}{T Brand Studio}
\item
  \href{https://www.nytimes3xbfgragh.onion/privacy/cookie-policy\#how-do-i-manage-trackers}{Your
  Ad Choices}
\item
  \href{https://www.nytimes3xbfgragh.onion/privacy}{Privacy}
\item
  \href{https://help.nytimes3xbfgragh.onion/hc/en-us/articles/115014893428-Terms-of-service}{Terms
  of Service}
\item
  \href{https://help.nytimes3xbfgragh.onion/hc/en-us/articles/115014893968-Terms-of-sale}{Terms
  of Sale}
\item
  \href{https://spiderbites.nytimes3xbfgragh.onion}{Site Map}
\item
  \href{https://help.nytimes3xbfgragh.onion/hc/en-us}{Help}
\item
  \href{https://www.nytimes3xbfgragh.onion/subscription?campaignId=37WXW}{Subscriptions}
\end{itemize}
