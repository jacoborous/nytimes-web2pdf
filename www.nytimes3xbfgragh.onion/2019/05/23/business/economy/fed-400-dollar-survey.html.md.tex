Sections

SEARCH

\protect\hyperlink{site-content}{Skip to
content}\protect\hyperlink{site-index}{Skip to site index}

\href{https://www.nytimes3xbfgragh.onion/section/business/economy}{Economy}

\href{https://myaccount.nytimes3xbfgragh.onion/auth/login?response_type=cookie\&client_id=vi}{}

\href{https://www.nytimes3xbfgragh.onion/section/todayspaper}{Today's
Paper}

\href{/section/business/economy}{Economy}\textbar{}Many Adults Would
Struggle to Find \$400, the Fed Finds

\url{https://nyti.ms/2HTBlJh}

\begin{itemize}
\item
\item
\item
\item
\item
\end{itemize}

Advertisement

\protect\hyperlink{after-top}{Continue reading the main story}

Supported by

\protect\hyperlink{after-sponsor}{Continue reading the main story}

\hypertarget{many-adults-would-struggle-to-find-400-the-fed-finds}{%
\section{Many Adults Would Struggle to Find \$400, the Fed
Finds}\label{many-adults-would-struggle-to-find-400-the-fed-finds}}

\includegraphics{https://static01.graylady3jvrrxbe.onion/images/2019/05/24/business/24DC-FEDSURVEY-01-print/24DC-FEDSURVEY-01-articleLarge.jpg?quality=75\&auto=webp\&disable=upscale}

By \href{https://www.nytimes3xbfgragh.onion/by/jeanna-smialek}{Jeanna
Smialek}

\begin{itemize}
\item
  May 23, 2019
\item
  \begin{itemize}
  \item
  \item
  \item
  \item
  \item
  \end{itemize}
\end{itemize}

Four in 10 American adults wouldn't be able to cover an unexpected \$400
expense with cash, savings or a credit-card charge that could be quickly
paid off, a new Federal Reserve survey finds.

About 27 percent of people surveyed would need to borrow or sell
something to pay for such a bill, and 12 percent would not be able to
cover it at all, according to the Fed's
\href{https://www.federalreserve.gov/publications/files/2018-report-economic-well-being-us-households-201905.pdf}{2018
report on the economic well-being of households}, which was released
Thursday.

The share that could cover such an expense more easily has been climbing
steadily and now stands at 61 percent, up from just half when the Fed
started this annual survey in 2013. Still, the finding underlines the
fact that many Americans remain on the edge financially even as this
economic expansion is approaching record length and people have become
more optimistic.

Household finances over all have shown a marked improvement over the
life of this report, thanks in large part to an improving labor market
that has lifted wages and left more Americans with jobs. Three-quarters
of adults said they were ``doing O.K.'' or ``living comfortably'' when
asked about their economic well-being, up from 63 percent in 2013.

``We continue to see the growing U.S. economy supporting most American
families,'' Michelle W. Bowman, a Federal Reserve Board governor, said
in a statement.

Underlying disparities persist. Just 52 percent of rural residents said
their local economy was doing well, compared with 66 percent of city
dwellers. And while nearly seven in 10 white adults viewed their area's
economy as good or excellent, only six in 10 Hispanic adults and fewer
than half of black adults said the same thing.

But adults belonging to minority groups were more likely to say that
they were better off than their parents. About 64 percent of black
adults with at least a bachelor's degree reported doing better
financially than their parents had, a figure that fell to 58 percent for
white adults. The gap was even wider among the less educated: About 61
percent of black high school graduates said they were better off than
their parents, compared with 52 percent of whites with a similar
education.

Hispanic adults also reported progress at higher rates than their white
counterparts.

``This measure shows some evidence of narrowing racial disparities
across a generation,'' the report said. ``In addition, having a
bachelor's degree or more is generally associated with greater upward
economic mobility.''

Advertisement

\protect\hyperlink{after-bottom}{Continue reading the main story}

\hypertarget{site-index}{%
\subsection{Site Index}\label{site-index}}

\hypertarget{site-information-navigation}{%
\subsection{Site Information
Navigation}\label{site-information-navigation}}

\begin{itemize}
\tightlist
\item
  \href{https://help.nytimes3xbfgragh.onion/hc/en-us/articles/115014792127-Copyright-notice}{©~2020~The
  New York Times Company}
\end{itemize}

\begin{itemize}
\tightlist
\item
  \href{https://www.nytco.com/}{NYTCo}
\item
  \href{https://help.nytimes3xbfgragh.onion/hc/en-us/articles/115015385887-Contact-Us}{Contact
  Us}
\item
  \href{https://www.nytco.com/careers/}{Work with us}
\item
  \href{https://nytmediakit.com/}{Advertise}
\item
  \href{http://www.tbrandstudio.com/}{T Brand Studio}
\item
  \href{https://www.nytimes3xbfgragh.onion/privacy/cookie-policy\#how-do-i-manage-trackers}{Your
  Ad Choices}
\item
  \href{https://www.nytimes3xbfgragh.onion/privacy}{Privacy}
\item
  \href{https://help.nytimes3xbfgragh.onion/hc/en-us/articles/115014893428-Terms-of-service}{Terms
  of Service}
\item
  \href{https://help.nytimes3xbfgragh.onion/hc/en-us/articles/115014893968-Terms-of-sale}{Terms
  of Sale}
\item
  \href{https://spiderbites.nytimes3xbfgragh.onion}{Site Map}
\item
  \href{https://help.nytimes3xbfgragh.onion/hc/en-us}{Help}
\item
  \href{https://www.nytimes3xbfgragh.onion/subscription?campaignId=37WXW}{Subscriptions}
\end{itemize}
