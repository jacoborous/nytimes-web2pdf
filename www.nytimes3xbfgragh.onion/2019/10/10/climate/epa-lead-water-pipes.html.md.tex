Sections

SEARCH

\protect\hyperlink{site-content}{Skip to
content}\protect\hyperlink{site-index}{Skip to site index}

\href{https://www.nytimes3xbfgragh.onion/section/climate}{Climate}

\href{https://myaccount.nytimes3xbfgragh.onion/auth/login?response_type=cookie\&client_id=vi}{}

\href{https://www.nytimes3xbfgragh.onion/section/todayspaper}{Today's
Paper}

\href{/section/climate}{Climate}\textbar{}New E.P.A. Lead Standards
Would Slow Replacement of Dangerous Pipes

\url{https://nyti.ms/2B0mLwU}

\begin{itemize}
\item
\item
\item
\item
\item
\end{itemize}

\hypertarget{climate-and-environment}{%
\subsubsection{\texorpdfstring{\href{https://www.nytimes3xbfgragh.onion/section/climate?name=styln-climate\&region=TOP_BANNER\&block=storyline_menu_recirc\&action=click\&pgtype=Article\&impression_id=63ffa950-f2a0-11ea-b7f7-11b03ec68d1d\&variant=undefined}{Climate
and
Environment}}{Climate and Environment}}\label{climate-and-environment}}

\begin{itemize}
\tightlist
\item
  \href{https://www.nytimes3xbfgragh.onion/article/why-does-california-have-wildfires.html?name=styln-climate\&region=TOP_BANNER\&block=storyline_menu_recirc\&action=click\&pgtype=Article\&impression_id=63ffa951-f2a0-11ea-b7f7-11b03ec68d1d\&variant=undefined}{California
  Fires}
\item
  \href{https://www.nytimes3xbfgragh.onion/interactive/2020/climate/trump-environment-rollbacks.html?name=styln-climate\&region=TOP_BANNER\&block=storyline_menu_recirc\&action=click\&pgtype=Article\&impression_id=63ffd060-f2a0-11ea-b7f7-11b03ec68d1d\&variant=undefined}{Trump's
  Changes}
\item
  \href{https://www.nytimes3xbfgragh.onion/interactive/2020/04/19/climate/climate-crash-course-1.html?name=styln-climate\&region=TOP_BANNER\&block=storyline_menu_recirc\&action=click\&pgtype=Article\&impression_id=63ffd061-f2a0-11ea-b7f7-11b03ec68d1d\&variant=undefined}{Climate
  101}
\item
  \href{https://www.nytimes3xbfgragh.onion/interactive/2018/08/30/climate/how-much-hotter-is-your-hometown.html?name=styln-climate\&region=TOP_BANNER\&block=storyline_menu_recirc\&action=click\&pgtype=Article\&impression_id=63ffd062-f2a0-11ea-b7f7-11b03ec68d1d\&variant=undefined}{Is
  Your Hometown Hotter?}
\end{itemize}

Advertisement

\protect\hyperlink{after-top}{Continue reading the main story}

Supported by

\protect\hyperlink{after-sponsor}{Continue reading the main story}

\hypertarget{new-epa-lead-standards-would-slow-replacement-of-dangerous-pipes}{%
\section{New E.P.A. Lead Standards Would Slow Replacement of Dangerous
Pipes}\label{new-epa-lead-standards-would-slow-replacement-of-dangerous-pipes}}

\includegraphics{https://static01.graylady3jvrrxbe.onion/images/2019/10/10/climate/10CLI-LEAD1/10CLI-LEAD1-articleLarge.jpg?quality=75\&auto=webp\&disable=upscale}

\href{https://www.nytimes3xbfgragh.onion/by/coral-davenport}{\includegraphics{https://static01.graylady3jvrrxbe.onion/images/2018/10/03/multimedia/author-coral-davenport/author-coral-davenport-thumbLarge-v2.png}}

By \href{https://www.nytimes3xbfgragh.onion/by/coral-davenport}{Coral
Davenport}

\begin{itemize}
\item
  Oct. 10, 2019
\item
  \begin{itemize}
  \item
  \item
  \item
  \item
  \item
  \end{itemize}
\end{itemize}

WASHINGTON --- The Trump administration on Thursday proposed new
regulations on lead and copper in drinking water, updating a nearly
30-year-old rule that may have contributed to the
\href{https://www.nytimes3xbfgragh.onion/interactive/2016/01/21/us/flint-lead-water-timeline.html?module=inline}{lead-tainted
water crisis in Flint, Mich.}, that began in 2015.

The draft plan, announced by the Environmental Protection Agency
administrator, Andrew Wheeler, at a news conference in Green Bay, Wis.,
includes some provisions designed to strengthen oversight of lead in
drinking water. But it skips a pricey safety proposal advocated by
public health groups and water utilities: the immediate replacement of
six million lead pipes that connect homes to main water pipes. The
proposed new rule would also more than double the amount of time
allotted to replace lead pipes in water systems that contain high levels
of lead.

Mr. Wheeler framed the new regulations as a major step forward in
protecting water supplies.

``The water sector has known for years and years that the regulations
governing lead and copper in our water need to be improved, but
administration after administration has failed to get it done,'' Mr.
Wheeler said, noting that the standards were last updated in 1991. ``We
are delivering on the president's commitment that all Americans have
access to clean and safe drinking water.''

Although the new proposal would extend the timetable for replacing lead
pipes, it would include new requirements that schools and day care
centers be tested for lead, and, if elevated lead levels are found,
customers would have to be told within 24 hours, not the current
standard of 30 days. It would also require water utilities to conduct
inventories of their lead service pipes and publicly report their
locations.

Environmental activists said those moves forward would not make up for
the relaxation of standards in other areas.

\href{\%3Ca\%20href=\%22https://www.nytimes3xbfgragh.onion/section/climate?action=click\&pgtype=Article\&state=default\&region=MAIN_CONTENT_1\&context=storylines_keepup\%22\%3Ehttps://www.nytimes3xbfgragh.onion/section/climate?action=click\&pgtype=Article\&state=default\&region=MAIN_CONTENT_1\&context=storylines_keepup\%3C/a\%3E}{}

\hypertarget{climate-and-environment-}{%
\subsubsection{Climate and Environment
›}\label{climate-and-environment-}}

\hypertarget{keep-up-on-the-latest-climate-news}{%
\paragraph{Keep Up on the Latest Climate
News}\label{keep-up-on-the-latest-climate-news}}

Updated Sept. 6, 2020

Here's what you need to know this week:

\begin{itemize}
\item
  \begin{itemize}
  \tightlist
  \item
    Americans back
    \href{https://www.nytimes3xbfgragh.onion/2020/09/04/climate/flood-fire-building-restrictions.html?action=click\&pgtype=Article\&state=default\&region=MAIN_CONTENT_1\&context=storylines_keepup}{tough
    limits on building in fire and flood zones}, new research shows.
  \item
    California's wildfires are driving another crisis: More and more
    \href{https://www.nytimes3xbfgragh.onion/2020/09/02/climate/wildfires-insurance.html?action=click\&pgtype=Article\&state=default\&region=MAIN_CONTENT_1\&context=storylines_keepup}{homeowners
    can't get insurance}.
  \item
    The Trump administration has relaxed Obama-era rules limiting the
    release of
    \href{https://www.nytimes3xbfgragh.onion/2020/08/31/climate/trump-coal-plants.html?action=click\&pgtype=Article\&state=default\&region=MAIN_CONTENT_1\&context=storylines_keepup}{toxic
    waste from coal plants}.
  \end{itemize}
\end{itemize}

The slower timetable for the replacement of lead pipes is a ``huge
weakening change that will swallow up the few small improvements in the
proposal,'' wrote Erik D. Olson, an expert in drinking water policy at
the Natural Resources Defense Council, an advocacy group, in an email.

The new rule proposes changing a key element of the current rules, which
requires that a water system that is found to contain lead levels higher
than 15 parts per billion must replace 7 percent of its lead service
lines each year for as long as the lead levels exceed that measurement.
The new proposal would instead require water systems with those lead
levels to replace 3 percent of lead service lines each year.

\includegraphics{https://static01.graylady3jvrrxbe.onion/images/2019/10/10/climate/10CLI-LEAD2/merlin_161363940_9ff7cb3d-cd10-4f07-9c0d-d050ee311de6-articleLarge.jpg?quality=75\&auto=webp\&disable=upscale}

Mr. Olson's group estimated that the loosening of standards could extend
the length of time needed to replace dangerous lead water pipes from 13
years to 33 years.

``It means that another generation of American kids will be exposed to
dangerous levels of lead from their drinking water,'' he said.

President Trump has made the
\href{https://www.nytimes3xbfgragh.onion/interactive/2019/climate/trump-environment-rollbacks.html}{rollback
of environmental regulations} a hallmark of his administration, with
initiatives to weaken or erase dozens of E.P.A. regulations on climate
change, chemical pollution and water quality. At the same time, he has
also called attention to the concerns about lead in water that were
ignited by the discovery of high levels of lead and other contaminants
that poisoned Flint's drinking water for more than a year. He also
frequently emphasizes his desire to promote ``crystal-clear water.''

During a 2016 campaign stop in Flint, Mr. Trump said: ``It used to be,
cars were made in Flint and you couldn't drink the water in Mexico. Now,
the cars are made in Mexico and you cannot drink the water in Flint.''

``We shouldn't allow it to happen,'' he said.

Mr. Trump's first E.P.A. administrator, Scott Pruitt, announced that he
would prioritize removing lead from water, saying that the agency was
declaring ``war on lead.'' Mr. Pruitt stepped down amid a corruption
scandal in 2017.

A
\href{https://www.nytimes3xbfgragh.onion/2018/07/19/us/flint-water-crisis-epa.html}{2018
report from the E.P.A.'s Office of Inspector General} said management
weaknesses had hobbled the agency's response to the Flint crisis and
that federal officials should have taken stronger action to correct
repeated blunders by state regulators.

Mr. Pruitt's successor, Mr. Wheeler, announcing the plan, highlighted it
as part of Children's Health Month, which falls in October.

But for Flint advocates, it fell well short.

``We need urgent action and bold investments to rebuild America's water
infrastructure, not weakened policies that fail to protect the health
and safety of our citizens,'' said Representative Dan Kildee, Democrat
of Michigan, who was born and raised in Flint and represents it in
Congress. ``The Flint water crisis should have taught policymakers at
all levels of government that we must get serious about removing lead
from our water systems. Yet the president's policies have continuously
put special corporate interests ahead of public health.''

The draft plan will be open for public comment for 60 days, and Mr.
Wheeler said that he expected to complete the final plan next summer.

Advertisement

\protect\hyperlink{after-bottom}{Continue reading the main story}

\hypertarget{site-index}{%
\subsection{Site Index}\label{site-index}}

\hypertarget{site-information-navigation}{%
\subsection{Site Information
Navigation}\label{site-information-navigation}}

\begin{itemize}
\tightlist
\item
  \href{https://help.nytimes3xbfgragh.onion/hc/en-us/articles/115014792127-Copyright-notice}{©~2020~The
  New York Times Company}
\end{itemize}

\begin{itemize}
\tightlist
\item
  \href{https://www.nytco.com/}{NYTCo}
\item
  \href{https://help.nytimes3xbfgragh.onion/hc/en-us/articles/115015385887-Contact-Us}{Contact
  Us}
\item
  \href{https://www.nytco.com/careers/}{Work with us}
\item
  \href{https://nytmediakit.com/}{Advertise}
\item
  \href{http://www.tbrandstudio.com/}{T Brand Studio}
\item
  \href{https://www.nytimes3xbfgragh.onion/privacy/cookie-policy\#how-do-i-manage-trackers}{Your
  Ad Choices}
\item
  \href{https://www.nytimes3xbfgragh.onion/privacy}{Privacy}
\item
  \href{https://help.nytimes3xbfgragh.onion/hc/en-us/articles/115014893428-Terms-of-service}{Terms
  of Service}
\item
  \href{https://help.nytimes3xbfgragh.onion/hc/en-us/articles/115014893968-Terms-of-sale}{Terms
  of Sale}
\item
  \href{https://spiderbites.nytimes3xbfgragh.onion}{Site Map}
\item
  \href{https://help.nytimes3xbfgragh.onion/hc/en-us}{Help}
\item
  \href{https://www.nytimes3xbfgragh.onion/subscription?campaignId=37WXW}{Subscriptions}
\end{itemize}
