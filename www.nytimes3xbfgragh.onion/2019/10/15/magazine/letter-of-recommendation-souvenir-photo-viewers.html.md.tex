Sections

SEARCH

\protect\hyperlink{site-content}{Skip to
content}\protect\hyperlink{site-index}{Skip to site index}

\href{https://myaccount.nytimes3xbfgragh.onion/auth/login?response_type=cookie\&client_id=vi}{}

\href{https://www.nytimes3xbfgragh.onion/section/todayspaper}{Today's
Paper}

Letter of Recommendation: Souvenir Photo Viewers

\url{https://nyti.ms/2IOryWb}

\begin{itemize}
\item
\item
\item
\item
\item
\item
\end{itemize}

Advertisement

\protect\hyperlink{after-top}{Continue reading the main story}

Supported by

\protect\hyperlink{after-sponsor}{Continue reading the main story}

\href{/column/letter-of-recommendation}{Letter of Recommendation}

\hypertarget{letter-of-recommendation-souvenir-photo-viewers}{%
\section{Letter of Recommendation: Souvenir Photo
Viewers}\label{letter-of-recommendation-souvenir-photo-viewers}}

\includegraphics{https://static01.graylady3jvrrxbe.onion/images/2019/10/20/magazine/20mag-lor/20mag-lor-articleLarge.jpg?quality=75\&auto=webp\&disable=upscale}

By Kate Dwyer

\begin{itemize}
\item
  Oct. 15, 2019
\item
  \begin{itemize}
  \item
  \item
  \item
  \item
  \item
  \item
  \end{itemize}
\end{itemize}

This past spring, I spent two months subletting an apartment in
Amsterdam, where tourists not infrequently outnumber residents. The day
I arrived, I was struck by how many American accents I heard and how
many selfie sticks jutted out onto the narrow sidewalks near the canals
in the Nine Streets. Nothing made tourists more visible than their
fervent Instagramming and ever-present iPhones. Jogging in the
Vondelpark, I saw a tourist crash a bike while videoblogging. From a
hotel lobby on Keizersgracht, I watched someone climb onto a locked bike
near the canal, take a photo on it and then go about her afternoon. Ask
a local about the tourism in Amsterdam, and you're likely to receive an
eye roll and a comment about how it's ``out of control,'' as one cafe
owner told me during my first week there. Last year, the city attracted
almost 20 million visitors, but there are fewer than one million
residents. In December, locals became so frustrated by tourist
congestion that the city removed the ``I amsterdam'' sign near the
Rijksmuseum, which reportedly generated upward of 6,000 photographs per
day. Many of these photos very likely ended up on Instagram, where the
hashtag \#iamsterdam has been used 1.54 million times and counting.

Souvenir photos weren't always so easily reproduced. My grandmother's
nightstand displays a bouquet of key chains, each a truncated pyramid
with a hole on one end and a piece of flat, once-white plastic snapped
onto the other. They resemble loupes for examining gemstones, and most
are emblazoned with the name of a resort, in tacky metallic gold. If you
raise one of these souvenirs to the light and peer through the lens, you
will see a backlit 35-millimeter slide on the other end, its colors
reflected against each of the four sides like a kaleidoscope. My
grandmother has 16 of these key chains --- from the Loews Paradise
Island Hotel and Villas, Fernwood in the Poconos, the Bahamas Princess
Tower and other destinations bookable by a travel agent circa 1985.

Viewer key chains were popular at amusement parks, resorts and national
parks from the 1950s through the 1990s. In most cases, a photographer
would walk around, take your picture and hand you a ticket to exchange
for the photo later in the day, either as a print or a key chain.
Sometimes you'd need to find your photo on a wall, behind a counter. The
resulting souvenir is pure kitsch; its only purpose is to view a single
photo, so its clunkiness is both warranted and extra. In 2019, if you
carry one on your keys, it's a fashion statement conveying nostalgia and
sentimentality: a Hawaiian shirt of key chains. When you hold it up to
the light, you see colors that come into focus the closer you draw it to
your face, the image revealing itself slowly. Because your other eye is
closed, and the room around you is blocked by plastic siding, it is easy
to imagine that you are looking at the only image in the world. Even if
they sit in drawers, these key chains beg to be viewed, the way conch
shells ask to be held to the ear. Each one is a small mystery --- it's
impossible to tell which image is inside by looking at its plastic
armor. My mother says the experience is similar to rediscovering a
memory.

Souvenir photo viewers are antithetical to Instagram tourism because
they permit only one person to view an image at a time. The photos
themselves are largely unremarkable, as they weren't intended to prove
that a person \emph{mastered} a vacation --- by capturing the photo
intended to amass the most desirable number of ``likes'' --- but only
that he or she took one.

These days, travel photos are captured with the understanding that they
will be shared on social media in a feed of hundreds of other photos,
further calcifying your personal brand. For much of the 20th century,
the novelties of travel photography were not a photo's framing or the
caption but the photo itself, which commemorated the luxury of travel:
Tourism was romanticized, and because rolls of film were finite, every
photo was precious. Now, instead of displaying a photo from a place on
your desk, it matters \emph{which photos} from \emph{which places} are
displayed on your Instagram page, as a public record of where you've
been and how you want to be perceived. Last year, for this magazine,
Teju Cole observed that travelers often take photos of the same
landmarks from the same angles, making originality even harder to
achieve. But if these travel photos are for ourselves, to help us
remember where we've been, why should originality matter? Shouldn't the
photos resemble our experiences? What purpose should they serve besides
sparking a memory?

There is one souvenir photo in my grandmother's room that continues to
mystify me. In it, my mother is in her mid-20s, just a few years older
than I am now, walking on a path in the Bahamas with my
great-grandmother, who died before I was born. There is no context in
the image whatsoever, no clue about where they're going or where they're
coming from. Of course I've asked, and my mother doesn't remember, but
when I put the viewer up to my eye, it feels almost as if I do. I look
at the photo slowly --- not via a quick scroll --- taking a journey down
a tunnel toward what feels like an image projected on a movie screen. It
comes into focus slowly at first, then all at once, just the way it's
supposed to.

Advertisement

\protect\hyperlink{after-bottom}{Continue reading the main story}

\hypertarget{site-index}{%
\subsection{Site Index}\label{site-index}}

\hypertarget{site-information-navigation}{%
\subsection{Site Information
Navigation}\label{site-information-navigation}}

\begin{itemize}
\tightlist
\item
  \href{https://help.nytimes3xbfgragh.onion/hc/en-us/articles/115014792127-Copyright-notice}{©~2020~The
  New York Times Company}
\end{itemize}

\begin{itemize}
\tightlist
\item
  \href{https://www.nytco.com/}{NYTCo}
\item
  \href{https://help.nytimes3xbfgragh.onion/hc/en-us/articles/115015385887-Contact-Us}{Contact
  Us}
\item
  \href{https://www.nytco.com/careers/}{Work with us}
\item
  \href{https://nytmediakit.com/}{Advertise}
\item
  \href{http://www.tbrandstudio.com/}{T Brand Studio}
\item
  \href{https://www.nytimes3xbfgragh.onion/privacy/cookie-policy\#how-do-i-manage-trackers}{Your
  Ad Choices}
\item
  \href{https://www.nytimes3xbfgragh.onion/privacy}{Privacy}
\item
  \href{https://help.nytimes3xbfgragh.onion/hc/en-us/articles/115014893428-Terms-of-service}{Terms
  of Service}
\item
  \href{https://help.nytimes3xbfgragh.onion/hc/en-us/articles/115014893968-Terms-of-sale}{Terms
  of Sale}
\item
  \href{https://spiderbites.nytimes3xbfgragh.onion}{Site Map}
\item
  \href{https://help.nytimes3xbfgragh.onion/hc/en-us}{Help}
\item
  \href{https://www.nytimes3xbfgragh.onion/subscription?campaignId=37WXW}{Subscriptions}
\end{itemize}
