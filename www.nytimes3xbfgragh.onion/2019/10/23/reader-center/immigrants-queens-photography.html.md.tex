\href{/section/reader-center}{Reader Center}\textbar{}Finding Amado in
Queens, and Going With Him to Mexico

\url{https://nyti.ms/2BBj40L}

\begin{itemize}
\item
\item
\item
\item
\item
\item
\end{itemize}

\includegraphics{https://static01.graylady3jvrrxbe.onion/images/2019/10/24/insider/24insider-amado3/24insider-amado3-articleLarge.jpg?quality=75\&auto=webp\&disable=upscale}

Sections

\protect\hyperlink{site-content}{Skip to
content}\protect\hyperlink{site-index}{Skip to site index}

Times Insider

\hypertarget{finding-amado-in-queens-and-going-with-him-to-mexico}{%
\section{Finding Amado in Queens, and Going With Him to
Mexico}\label{finding-amado-in-queens-and-going-with-him-to-mexico}}

In New York, thousands of immigrants dwell in illegal basement
apartments. One man let Times journalists into his life.

Credit...Ryan Christopher Jones for The New York Times

Supported by

\protect\hyperlink{after-sponsor}{Continue reading the main story}

By \href{https://www.nytimes3xbfgragh.onion/by/john-otis}{John Otis}

\begin{itemize}
\item
  Oct. 23, 2019
\item
  \begin{itemize}
  \item
  \item
  \item
  \item
  \item
  \item
  \end{itemize}
\end{itemize}

\href{https://www.nytimes3xbfgragh.onion/series/times-insider}{\emph{Times
Insider}} \emph{explains who we are and what we do, and delivers
behind-the-scenes insights into how our journalism comes together.}

The greatest challenge to reporting on one of New York City's biggest
open secrets was not a dearth of subjects, it was their unwillingness to
talk.

Nikita Stewart, a Metro reporter, wanted to spotlight the multitudes of
people living in illegal basement apartments in Queens. The cellar
residences, often overcrowded and unsafe, are occupied primarily by
immigrants.

``People in Queens already know, but people outside Queens --- now you
know how people are living underground,'' Ms. Stewart said.

Because of their precarious legal status, many tenants refused to talk
or to be photographed.

Except for Amado, a 50-something man born in Mexico, in possession of a
green card, who has lived in the United States for 29 years. (The Times
identified him only by his first name.)

His forthcoming nature, coupled with the rapport he developed with the
photographer Ryan Christopher Jones, led to a pivot for the project.
Amado became the narrative spine, the human face, of a complex story
about illegal housing, immigration and sacrifice.

Amado's home is a dark, cramped basement quarters that he shares with
several others. They are not friends. Amado and his roommate know very
little about each other. Amado works at a restaurant six days a week and
sends most of his income to family in Coatzingo, Mexico, including his
wife and two stepdaughters.

His monotonous worker-bee routine, state of privation and lack of social
life made it difficult for Mr. Jones to create compelling visuals.

\includegraphics{https://static01.graylady3jvrrxbe.onion/images/2019/10/24/insider/24insider-amado2/merlin_152933322_9686085c-5997-4bf9-9a5d-dfd3a6aea440-articleLarge.jpg?quality=75\&auto=webp\&disable=upscale}

``How do you make an interesting photo of a guy in his underlit
apartment watching TV, talking to his wife via text on his phone?'' Mr.
Jones said.

A decision was made to broaden the story to include Amado's life on the
other side of the border. Mr. Jones would join Amado on one of his
biannual trips to Mexico and document a lesser-seen element of the
migrant experience.

``You see a lot of stories about immigration, but you never really see
what the sacrifice is, what these guys are giving up, and the lives that
are kind of left behind,'' said Jeffrey Furticella, the photo editor on
the project.

Within hours of arriving in his homeland, Amado's shift in posture,
countenance and energy was remarkable; he became ebullient, ditching his
baseball cap for a ranchero hat, Mr. Jones said.

``He really did come alive,'' Mr. Jones said.

Image

Amado with his wife and stepdaughter in Coatzingo, Mexico, where he
returns twice a year. There, a transformation takes place.Credit...Ryan
Christopher Jones for The New York Times

Image

In Mexico, he is a community man with a piece of land and
crops.Credit...Ryan Christopher Jones for The New York Times

Photographs captured Amado bonding with his family members; attending
parties with friends and loved ones; and devoting time to precious
leisure, swimming in a desert basin or watching television, which had
been purchased with Amado's remittances.

``It gives a much more nuanced view of how money and the economy work
within the immigrant culture,'' said Mr. Jones. ``People come to the
United States, they work, they send money back to their families, but we
actually never see the fruits of those labors back home.''

The project had been conceived as one that leaned heavily on
photographs. It's in Thursday's newspaper as a stand-alone section, and
was published in a visually innovative online format, something Ms.
Stewart had championed.

``We just want people to be able to have easier access to my stories and
feel more engaged with the people I'm writing about,'' she said.

And yet, the stark contrast between Amado's two realities surpassed
expectations.

``It's heartbreaking,'' Mr. Furticella said. ``It's a really sad,
complete portrait of his life.''

Ms. Stewart, meanwhile, pursued additional story leads. With great
effort, she tracked down others living in illegal, ramshackle
households. Those willing to speak became the subjects of vignettes in
the larger piece.

``It's important to write about how people survive in the city,'' Ms.
Stewart said. ``Everybody should have an opportunity to live here, and
unfortunately some people have to live in less-than-desirable places.''

The project's threads came together over a much longer time period than
anticipated, roughly nine months. Those delays came with benefits,
however, particularly where Amado and Mr. Jones were concerned.

``It took an investment of time and in the relationship to accumulate
the moments and make these contemplative photographs,'' Mr. Furticella
said. ``It's easy to reduce ideas like this piece to stereotypes and
common misconceptions, and I think that what's successful here and
what's taken us so long is making sure that we've shown Amado the
respect that he deserves.''

Image

Many people who live in basement apartments wouldn't speak due to their
their precarious legal status. Amado is in possession of a green card
and developed a rapport with the photographer for the project, Ryan
Christopher Jones.Credit...Ryan Christopher Jones for The New York Times

\begin{center}\rule{0.5\linewidth}{\linethickness}\end{center}

Follow the \href{https://twitter.com/readercenter}{@ReaderCenter} on
Twitter for more coverage highlighting your perspectives and experiences
and for insight into how we work.

Advertisement

\protect\hyperlink{after-bottom}{Continue reading the main story}

\hypertarget{site-index}{%
\subsection{Site Index}\label{site-index}}

\hypertarget{site-information-navigation}{%
\subsection{Site Information
Navigation}\label{site-information-navigation}}

\begin{itemize}
\tightlist
\item
  \href{https://help.nytimes3xbfgragh.onion/hc/en-us/articles/115014792127-Copyright-notice}{©~2020~The
  New York Times Company}
\end{itemize}

\begin{itemize}
\tightlist
\item
  \href{https://www.nytco.com/}{NYTCo}
\item
  \href{https://help.nytimes3xbfgragh.onion/hc/en-us/articles/115015385887-Contact-Us}{Contact
  Us}
\item
  \href{https://www.nytco.com/careers/}{Work with us}
\item
  \href{https://nytmediakit.com/}{Advertise}
\item
  \href{http://www.tbrandstudio.com/}{T Brand Studio}
\item
  \href{https://www.nytimes3xbfgragh.onion/privacy/cookie-policy\#how-do-i-manage-trackers}{Your
  Ad Choices}
\item
  \href{https://www.nytimes3xbfgragh.onion/privacy}{Privacy}
\item
  \href{https://help.nytimes3xbfgragh.onion/hc/en-us/articles/115014893428-Terms-of-service}{Terms
  of Service}
\item
  \href{https://help.nytimes3xbfgragh.onion/hc/en-us/articles/115014893968-Terms-of-sale}{Terms
  of Sale}
\item
  \href{https://spiderbites.nytimes3xbfgragh.onion}{Site Map}
\item
  \href{https://help.nytimes3xbfgragh.onion/hc/en-us}{Help}
\item
  \href{https://www.nytimes3xbfgragh.onion/subscription?campaignId=37WXW}{Subscriptions}
\end{itemize}
