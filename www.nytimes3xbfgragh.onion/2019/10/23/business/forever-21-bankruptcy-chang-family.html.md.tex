Sections

SEARCH

\protect\hyperlink{site-content}{Skip to
content}\protect\hyperlink{site-index}{Skip to site index}

\href{https://www.nytimes3xbfgragh.onion/section/business}{Business}

\href{https://myaccount.nytimes3xbfgragh.onion/auth/login?response_type=cookie\&client_id=vi}{}

\href{https://www.nytimes3xbfgragh.onion/section/todayspaper}{Today's
Paper}

\href{/section/business}{Business}\textbar{}One Family Built Forever 21,
and Fueled Its Collapse

\url{https://nyti.ms/31Gbuwv}

\begin{itemize}
\item
\item
\item
\item
\item
\item
\end{itemize}

Advertisement

\protect\hyperlink{after-top}{Continue reading the main story}

Supported by

\protect\hyperlink{after-sponsor}{Continue reading the main story}

\hypertarget{one-family-built-forever-21-and-fueled-its-collapse}{%
\section{One Family Built Forever 21, and Fueled Its
Collapse}\label{one-family-built-forever-21-and-fueled-its-collapse}}

\includegraphics{https://static01.graylady3jvrrxbe.onion/images/2019/10/18/business/18forever21-01/merlin_162717138_d7d472aa-d124-46e0-b194-205896595cbd-articleLarge.jpg?quality=75\&auto=webp\&disable=upscale}

By \href{https://www.nytimes3xbfgragh.onion/by/sapna-maheshwari}{Sapna
Maheshwari}

\begin{itemize}
\item
  Published Oct. 23, 2019Updated Oct. 24, 2019
\item
  \begin{itemize}
  \item
  \item
  \item
  \item
  \item
  \item
  \end{itemize}
\end{itemize}

\href{https://cn.nytimes3xbfgragh.onion/business/20191030/forever-21-bankruptcy-chang-family/}{阅读简体中文版}\href{https://cn.nytimes3xbfgragh.onion/business/20191030/forever-21-bankruptcy-chang-family/}{阅读简体中文版}\href{https://cn.nytimes3xbfgragh.onion/business/20191030/forever-21-bankruptcy-chang-family/zh-hant/}{閱讀繁體中文版}\href{https://cn.nytimes3xbfgragh.onion/business/20191030/forever-21-bankruptcy-chang-family/zh-hant/}{閱讀繁體中文版}

When Forever 21 filed for bankruptcy last month, the fast fashion chain
described its history in documents that read, at times, like a pitch for
a memoir or a Netflix special.

Photos of the company's husband and wife founders, Do Won and Jin Sook
Chang, and their two daughters appeared under headings like ``Forever
Striving: A Story of Grit, Determination, and Passion.'' The filing
emphasized the improbable success of the Changs, who immigrated to the
United States from South Korea in 1981 and built a multibillion-dollar
business from scratch.

There were references to the daughters' undergraduate degrees from ``Ivy
League universities'' --- both are top executives at the company --- and
summer breaks spent at Forever 21 stores. A definition of the American
dream, as explained by Investopedia.com, even appeared on one page.

The Changs were indeed a unique success story, and Forever 21 was far
from a run-of-the-mill family operation. At its peak, the retailer
brought in more than \$4 billion in annual sales and employed more than
43,000 people worldwide in hundreds of stores. Now it is leaving 40
countries and closing up to 199, or more than 30 percent, of its stores
in the United States as
\href{https://www.nytimes3xbfgragh.onion/2019/09/29/business/forever-21-bankruptcy.html}{part
of its bankruptcy,} and former employees and industry experts are
pointing to the Changs' insular management style as a significant reason
for the collapse. They cite disastrous real estate deals and the chain's
bungled merchandising strategy in recent years.

\includegraphics{https://static01.graylady3jvrrxbe.onion/images/2019/10/14/business/00forever21-02/merlin_161682021_4cbbd6a9-dfc4-46cc-9ad4-7a1f180ef23b-articleLarge.jpg?quality=75\&auto=webp\&disable=upscale}

``On the founder side, this hubris thing is pretty common, but it's
particularly deadly if you've been successful for a long time,'' said
Erik Gordon, a management expert at the University of Michigan Ross
School of Business. ``They didn't have a board of directors to give them
a reality check, they didn't have equity analysts to give them a reality
check.''

He added: ``You can live in your self-created bubble for a lot longer,
but then the bubble pops.''

The bankruptcy filing provides a rare glimpse inside a retailer that has
been intensely secretive and privately held for decades. Six former
employees, including three executives, also spoke to The New York Times
about their experiences at Forever 21 on the condition of anonymity,
citing nondisclosure agreements.

Forever 21's missteps, combined with industrywide changes in consumer
tastes and shopping habits, will have far-reaching effects for thousands
of people who work for the company, its vendors and malls. The chain
says it will still operate hundreds of stores, along with its website.
Through a spokeswoman, the Chang family declined to comment for this
article.

Forever 21 --- named because Mr. Chang considered 21 to be ``the most
enviable age'' --- was built on the idea of identifying apparel trends,
then working with vendors to bring those products to stores quickly at
cut-rate prices. From its early days, Mr. Chang, who is still the
company's chief executive, oversaw landlord and vendor relationships
while Mrs. Chang led design and merchandising.

Former employees say that the top floor of the company's Los Angeles
headquarters was viewed as Mr. Chang's world, where corporate strategy
unfolded and people kept quiet outside his office, while the bottom
floor was Mrs. Chang's domain of buyers and planners, who showed their
bags to security when leaving the building. Three former employees said
that, as recently as this year, Mr. Chang was personally signing off on
employee expenses and questioning executives about receipts for lunches
or Uber rides.

The couple's daughters eventually joined the executive ranks. The older
daughter, Linda Chang, is the executive vice president and has been
viewed as Mr. Chang's successor; her sister, Esther Chang, is vice
president of merchandising.

The Changs never took Forever 21 public, unlike their biggest
fast-fashion rivals, ``declining numerous opportunities that would
facilitate generational wealth,'' the filing said.

Their inner circle included another Korean-American couple: Alex Ok,
Forever 21's president and a former supplier, and his wife, SeongEun
Kim, who works in merchandising. Internally, some referred to Mrs. Chang
and Mrs. Ok as the ``Missuses,'' a powerful pair who directed the tens
of thousands of styles that landed in Forever 21's bustling stores. The
filing showed that the Chang family owned 99 percent of equity in the
chain, while Mr. Ok held 1 percent.

Image

The pop star Ariana Grande, right, has sued Forever 21, claiming that it
used a look-alike model of the singer in online ads, left.Credit...U.S.
District Court C.D. Calif., via Reuters

As the business expanded, the Changs struggled with their desire to hire
experienced executives and their distrust of outsiders, five of the
employees said. In recent years, they said, Forever 21 eagerly recruited
experts to overhaul parts of the business, then later ignored their
recommendations on everything from new technology to marketing.

Image

Former employees said company marketers wanted to work with Ms. Grande
in 2014, but were overruled by management.Credit...U.S. District Court
C.D. Calif., via Reuters

Some were reminded of that dynamic after the singer Ariana Grande
\href{https://www.nytimes3xbfgragh.onion/2019/09/03/arts/music/ariana-grande-forever-21.html}{filed
a lawsuit} against Forever 21 in September. The company's marketers had
urged it to partner with Ms. Grande for a holiday campaign in 2014,
according to two former employees, but management hired the rapper Iggy
Azalea instead. Now, Ms. Grande is far more popular, and Forever 21 is
\href{https://www.nytimes3xbfgragh.onion/2019/09/03/arts/music/ariana-grande-forever-21.html}{defending
itself} against claims that it used a look-alike model of the singer in
online ads.

The Changs' Christian faith played a role in the way they ran the
company. Forever 21's bright yellow shopping bags are stamped with
``John 3:16,'' a reference to a Bible verse. Mr. Chang has said the
verse ``shows us how much God loves us,'' and hoped others would learn
of that love. Former employees said Bibles were sometimes visible in
conference rooms and on Mr. Chang's desk. It was not unusual for
department leaders to have ties to the family or their church but no
experience working for another retailer, employees said.

Image

Forever 21's bright yellow shopping bags are stamped with ``John 3:16,''
a reference to a Bible verse.Credit...Haruka Sakaguchi for The New York
Times

``Every once in a while, when we hired someone who had been there, we'd
learn that they were never allowed to see the totality of the business
performance and they were only given reporting on their specific
sector,'' said Margaret Coblentz, a former e-commerce director at
Charlotte Russe. Rivals saw Forever 21 ``as both monolithic and
inscrutable,'' she added.

But Forever 21 made its biggest mistakes in real estate. In the years
before and after the recession, the company expanded aggressively and
decided to open huge flagship stores, setting up in cavernous spaces
once occupied by Mervyn's, the bankrupt department store, as well as
Borders, Sears and Saks. Its former head of real estate
\href{https://webcache.googleusercontent.com/search?q=cache:MSSRC9kqzfoJ:https://www.bloomberg.com/news/articles/2011-01-20/forever-21s-fast-and-loose-fashion-empire+\&cd=1\&hl=en\&ct=clnk\&gl=us}{told
Bloomberg Businessweek} in 2011 that ``having really big stores has
always been Mr. Chang's dream.''

The stores became hard to fill with new merchandise, then turn over,
however, and saddled Forever 21 with long leases just as technology was
beginning to wreak havoc on American malls. Seven of the leases at the
old Mervyn's stores were not set to expire until 2027 or 2028, which is
longer than a typical lease, according to internal documents obtained by
The Times.

In an interview conducted last month, when the company filed for
bankruptcy, Linda Chang acknowledged issues with the larger stores.
``Having to fill those boxes on top of having to deal with the
complexities of expanding internationally did stress our merchant
organization,'' she said.

She also cited shifts in mall traffic and the rise of e-commerce as
challenges, and said that the bankruptcy was ``a strategic move on our
part.''

Mr. Chang, who sought to sign each lease and design every store himself
even as the count soared past 500, was loath to close even
underperforming locations, and at times, would simply move a store to
another spot in the same mall, two former employees said.

``Forever 21's problem is not the malls --- it's that they didn't get
out of the malls earlier,'' Mr. Gordon, the management expert, said.
``If they want to point a finger, they need to stand in front of a
mirror and point it at themselves.''

Image

Mr. Chang sought to design every Forever 21 store himself, even when the
company was expanding aggressively, and was reluctant to close
underperforming locations.Credit...Lisa Baertlein/Reuters

The retailer also raced into expensive, massive new stores overseas
without local expertise, as it surged from seven international stores in
2005 to 262 a decade later. Two employees said that the chain often did
not understand local labor laws and made mistakes, like failing to
recognize that customers in some European countries shopped for winter
merchandise earlier in the year than American consumers. One employee
said the chain moved into Germany without realizing stores in the
country typically closed on Sundays. It didn't help that many of these
areas were familiar turf for H\&M, which is based in Sweden, and Zara,
whose owner is in Spain.

Forever 21 said in the filing that most of its international locations
were unprofitable as of 2015 and that its stores in Canada, Europe and
Asia were losing an average of \$10 million per month in the past year.
Overall, the annual occupancy cost of Forever 21's stores was \$450
million.

``They've gotten into categories and expression of fashion that are not
closely aligned with their fast-fashion customer's preferences,'' said
Mark A. Cohen, the director of retail studies at Columbia Business
School. ``They never built the intelligence into the business that would
have cautioned them from this real estate orgy and would have kept them
from the kind of exposure that they have now.''

Yet even as its errors abroad became clear, Mr. Chang and his real
estate counterparts bet on even more United States stores. An internal
playbook from 2015 described the retailer's plans for a new strip mall
chain called F21 Red that would target mothers under 35. Its \$1.80
camisoles and \$7.80 jeans were meant to swipe at the Irish retailer
Primark, which entered the United States that year.

The playbook showed that six stores were already open, and that Forever
21 planned to open 35 more that year, including in regular malls, which
was a surprise to the employees who had planned F21 Red. By 2017,
several new F21 Red stores were posting sales that were around 50
percent below company projections, internal sales reports show.

That year, Forever 21 also introduced a beauty chain, Riley Rose, that
was viewed as the company's next wave of growth and sought to capitalize
on the boom in Korean skin care products. It was created by Linda and
Esther Chang and called ``ground-breaking'' in the bankruptcy filing,
which grouped its sales with the slumping international division.

While former employees praised the sisters' work ethic, they said that
Riley Rose, which had 15 stores this year, was an expensive gamble in
high-priced malls and struggled to maintain vendor relationships. The
company told The Times last month that Riley Rose may end up as a store
within Forever 21 locations. It has filed to reject leases for nine
previously planned Riley Rose locations.

Mrs. Chang's side of the business was also making errors with the
sprawling store base. Merchandising was based on the previous year's
sales, and Forever 21 bought too little inventory in 2017, then too much
in 2018, the filing said. It also duplicated merchandise by designing
for ``styles'' like weekend or work looks, rather than categories like
tops or dresses.

Forever 21 had about 6,400 full-time employees and 26,400 part-time
employees when it filed, numbers that will likely shrink throughout the
bankruptcy process. Forever 21 said that it would change how it
merchandises, winnow its operations to the United States, Mexico and
Latin America, aim to increase e-commerce sales to more than just 16
percent of the business and take other cost-cutting measures. When it
filed, the company owed \$347 million to its vendors.

And the Chang family will be listening to new voices. Its board of
directors will grow from three members --- Mr. Chang, Linda Chang and
Mr. Ok --- to six, including Forever 21's former head of real estate, a
lawyer and the former chief executive of Things Remembered. It also said
that it had added several new managers in recent months, including a new
chief financial officer. Mr. Chang remains the chief executive.

``Forever 21 has basically been a one-trick pony,'' Mr. Cohen said.
``The founder and his wife did remarkably well until the business got
too big for them to continue to do remarkably well by themselves.''

Advertisement

\protect\hyperlink{after-bottom}{Continue reading the main story}

\hypertarget{site-index}{%
\subsection{Site Index}\label{site-index}}

\hypertarget{site-information-navigation}{%
\subsection{Site Information
Navigation}\label{site-information-navigation}}

\begin{itemize}
\tightlist
\item
  \href{https://help.nytimes3xbfgragh.onion/hc/en-us/articles/115014792127-Copyright-notice}{©~2020~The
  New York Times Company}
\end{itemize}

\begin{itemize}
\tightlist
\item
  \href{https://www.nytco.com/}{NYTCo}
\item
  \href{https://help.nytimes3xbfgragh.onion/hc/en-us/articles/115015385887-Contact-Us}{Contact
  Us}
\item
  \href{https://www.nytco.com/careers/}{Work with us}
\item
  \href{https://nytmediakit.com/}{Advertise}
\item
  \href{http://www.tbrandstudio.com/}{T Brand Studio}
\item
  \href{https://www.nytimes3xbfgragh.onion/privacy/cookie-policy\#how-do-i-manage-trackers}{Your
  Ad Choices}
\item
  \href{https://www.nytimes3xbfgragh.onion/privacy}{Privacy}
\item
  \href{https://help.nytimes3xbfgragh.onion/hc/en-us/articles/115014893428-Terms-of-service}{Terms
  of Service}
\item
  \href{https://help.nytimes3xbfgragh.onion/hc/en-us/articles/115014893968-Terms-of-sale}{Terms
  of Sale}
\item
  \href{https://spiderbites.nytimes3xbfgragh.onion}{Site Map}
\item
  \href{https://help.nytimes3xbfgragh.onion/hc/en-us}{Help}
\item
  \href{https://www.nytimes3xbfgragh.onion/subscription?campaignId=37WXW}{Subscriptions}
\end{itemize}
