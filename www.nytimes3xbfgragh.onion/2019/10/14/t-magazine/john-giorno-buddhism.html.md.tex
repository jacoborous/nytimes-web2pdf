Sections

SEARCH

\protect\hyperlink{site-content}{Skip to
content}\protect\hyperlink{site-index}{Skip to site index}

\href{https://myaccount.nytimes3xbfgragh.onion/auth/login?response_type=cookie\&client_id=vi}{}

\href{https://www.nytimes3xbfgragh.onion/section/todayspaper}{Today's
Paper}

John Giorno on His Most Precious Possession

\url{https://nyti.ms/2BaBuVX}

\begin{itemize}
\item
\item
\item
\item
\item
\end{itemize}

Advertisement

\protect\hyperlink{after-top}{Continue reading the main story}

Supported by

\protect\hyperlink{after-sponsor}{Continue reading the main story}

The Story of a Thing

\hypertarget{john-giorno-on-his-most-precious-possession}{%
\section{John Giorno on His Most Precious
Possession}\label{john-giorno-on-his-most-precious-possession}}

The artist, poet and activist, who died on Friday at 82, spoke to T last
week about a Buddhist statue he kept at his home in New York.

\includegraphics{https://static01.graylady3jvrrxbe.onion/images/2019/10/14/t-magazine/14tmag-giorno/14tmag-giorno-articleLarge.jpg?quality=75\&auto=webp\&disable=upscale}

As told to
\href{https://www.nytimes3xbfgragh.onion/by/emily-spivack}{Emily
Spivack}

\begin{itemize}
\item
  Oct. 14, 2019
\item
  \begin{itemize}
  \item
  \item
  \item
  \item
  \item
  \end{itemize}
\end{itemize}

\emph{In}
\href{https://www.nytimes3xbfgragh.onion/column/story-of-a-thing}{\emph{this
series}} \emph{for T, Emily Spivack, the author of
``}\href{http://wornstories.com/}{\emph{Worn Stories}}\emph{,''
interviews creative types about their most prized possessions. Here, the
artist, poet and activist John Giorno, who spoke to Spivack several days
before he}
\href{https://www.nytimes3xbfgragh.onion/2019/10/13/obituaries/john-giorno-dead.html?action=click\&module=Well\&pgtype=Homepage\&section=Obituaries}{\emph{died}}
\emph{on Friday at 82, describes a Buddhist statue that sits in a shrine
in his New York City home on the Bowery, where he had lived since 1962.}

At Columbia University in the 1950s, I took classes in philosophy and
art within Oriental Studies, a word you would not use today --- and that
was when I began studying Buddhism. In the mid-60s, one of my first LSD
trips was a bad one, and I realized it wasn't the drug but my mind. I
didn't know what to do, so I meditated, which meant I sat in the posture
of a statue I'd seen in Life magazine, closed my eyes, rested my mind
and the problems went away. By 1970, I decided to go to India, and it
was there that I met Dudjom Rinpoche, who would become my teacher in the
Nyingma tradition of Tibetan Buddhism for the next 60 years.

Image

The shrine that Giorno created in the writer William Burroughs's former
apartment, one floor below his own loft on the Bowery.Credit...Jacob
Pritchard

Image

The statue Giorno made 45 years ago in Kathmandu, Nepal, to which he
later added diamonds he inherited from his mother and
grandmother.Credit...Jacob Pritchard

In Tibetan Buddhism, we have different ways of representing the
{[}Buddhist master{]} Guru Rinpoche and his consort, Yeshe Tsogyal.
You'll see an image that represents the union of wisdom and emptiness,
or the union of subject and object. Generally, Tibetan Buddhists never
made statues in human form because that's too explicit. But about 45
years ago, I said, ``Why don't I have a statue made?'' Part of ****
meditation is visualization, and this was my visualization. I was in
Kathmandu, Nepal, and my teacher said he knew a great statue maker in a
small city in the Kathmandu Valley. The man was about 82, the last of
the great statue men, and he had never done anything like this before.
It took him two years to complete. When the statue was brought to New
York, it was consecrated by the great lama Dzogchen Rinpoche, the head
of the Nyingma lineage.

In the statue, Guru Rinpoche and Yeshe Tsogyal are in a union, a sexual
union. When you see it as three-dimensional, it becomes more explicit
than when you see it abstracted, flat on wood, in a painting. But it's
symbolic --- the male being the realization of emptiness and the female
being wisdom, and it's the union of that. In this specific form, the
piece represents the true nature of the mind.

I had been living in a loft upstairs from William Burroughs before he
moved to Kansas in '82. I was well into being a Buddhist, so when he
left, I invited lamas to give teachings in his space, and it became a
shrine room. I would do my meditation upstairs, but the bunker below
became a more formal space for teaching, and it's where the shrine has
always been. William would come to New York twice a year for readings or
events and stay in his space. He loved to be surrounded by this even
though he was not a Buddhist in any sense other than his profound
understanding of the empty nature of the mind.

When my mother died, about 15 years ago, I inherited two two-carat
diamonds --- one was from her engagement ring and the other had been my
grandmother's. What do I do with them? I'm a gay man. And I'm
Italian-American. Unconsciously, I thought of the tradition in Naples of
putting jewels in the crown of Madonna. So I put them in Guru Rinpoche's
crown. I thought, ``What a good resting place.''

\emph{This interview has been edited and condensed.}

An exhibition of John Giorno's work,
``\href{https://www.speronewestwater.com/exhibitions/john-giorno/installations}{Do
the Undone},'' is on view until Oct. 26 at Sperone Westwater gallery in
New York City.

Advertisement

\protect\hyperlink{after-bottom}{Continue reading the main story}

\hypertarget{site-index}{%
\subsection{Site Index}\label{site-index}}

\hypertarget{site-information-navigation}{%
\subsection{Site Information
Navigation}\label{site-information-navigation}}

\begin{itemize}
\tightlist
\item
  \href{https://help.nytimes3xbfgragh.onion/hc/en-us/articles/115014792127-Copyright-notice}{©~2020~The
  New York Times Company}
\end{itemize}

\begin{itemize}
\tightlist
\item
  \href{https://www.nytco.com/}{NYTCo}
\item
  \href{https://help.nytimes3xbfgragh.onion/hc/en-us/articles/115015385887-Contact-Us}{Contact
  Us}
\item
  \href{https://www.nytco.com/careers/}{Work with us}
\item
  \href{https://nytmediakit.com/}{Advertise}
\item
  \href{http://www.tbrandstudio.com/}{T Brand Studio}
\item
  \href{https://www.nytimes3xbfgragh.onion/privacy/cookie-policy\#how-do-i-manage-trackers}{Your
  Ad Choices}
\item
  \href{https://www.nytimes3xbfgragh.onion/privacy}{Privacy}
\item
  \href{https://help.nytimes3xbfgragh.onion/hc/en-us/articles/115014893428-Terms-of-service}{Terms
  of Service}
\item
  \href{https://help.nytimes3xbfgragh.onion/hc/en-us/articles/115014893968-Terms-of-sale}{Terms
  of Sale}
\item
  \href{https://spiderbites.nytimes3xbfgragh.onion}{Site Map}
\item
  \href{https://help.nytimes3xbfgragh.onion/hc/en-us}{Help}
\item
  \href{https://www.nytimes3xbfgragh.onion/subscription?campaignId=37WXW}{Subscriptions}
\end{itemize}
