Sections

SEARCH

\protect\hyperlink{site-content}{Skip to
content}\protect\hyperlink{site-index}{Skip to site index}

\href{https://www.nytimes3xbfgragh.onion/section/politics}{Politics}

\href{https://myaccount.nytimes3xbfgragh.onion/auth/login?response_type=cookie\&client_id=vi}{}

\href{https://www.nytimes3xbfgragh.onion/section/todayspaper}{Today's
Paper}

\href{/section/politics}{Politics}\textbar{}Congress Can Seek Trump's
Financial Records, Appeals Court Rules

\url{https://nyti.ms/33nKe7i}

\begin{itemize}
\item
\item
\item
\item
\item
\end{itemize}

Advertisement

\protect\hyperlink{after-top}{Continue reading the main story}

Supported by

\protect\hyperlink{after-sponsor}{Continue reading the main story}

\hypertarget{congress-can-seek-trumps-financial-records-appeals-court-rules}{%
\section{Congress Can Seek Trump's Financial Records, Appeals Court
Rules}\label{congress-can-seek-trumps-financial-records-appeals-court-rules}}

\includegraphics{https://static01.graylady3jvrrxbe.onion/images/2019/10/11/us/politics/11dc-mazars/merlin_162495219_aa0255ed-e1b3-4306-949a-c6b27d2b461f-articleLarge.jpg?quality=75\&auto=webp\&disable=upscale}

By \href{https://www.nytimes3xbfgragh.onion/by/charlie-savage}{Charlie
Savage}

\begin{itemize}
\item
  Published Oct. 11, 2019Updated July 9, 2020
\item
  \begin{itemize}
  \item
  \item
  \item
  \item
  \item
  \end{itemize}
\end{itemize}

WASHINGTON --- President Trump's accounting firm must comply with a
House committee's demands for eight years of his financial records, a
federal appeals court panel ruled on Friday in a major victory for House
Democrats in their struggle against his vow to stonewall ``all'' of
their oversight subpoenas.

In a 66-page ruling, the panel rejected Mr. Trump's argument that
Congress had no legitimate legislative authority to seek his business
records from the firm, Mazars USA, because the committee was trying to
determine whether he broke existing laws --- not weighing whether to
enact a new one.

``Having considered the weighty issues at stake in this case, we
conclude that the
\href{https://www.nytimes3xbfgragh.onion/2020/07/09/us/supreme-court-trump-tax-records.html}{subpoena
issued by the committee to Mazars} is valid and enforceable,'' wrote
Judge David S. Tatel of the United States Court of Appeals for the
District of Columbia.

Mr. Trump is virtually certain to appeal the ruling, either to the full
Court of Appeals or to the Supreme Court. But the decision --- affirming
an earlier ruling by a Federal District Court judge --- was the first
test at the appeals court level of the Trump legal team's sweeping
challenges to the constitutional authority of Congress to conduct
oversight of his activities.

Judge Tatel was joined by Judge Patricia A. Millett in the majority of
the three-judge ruling. Both were appointed by Democratic presidents.

Judge Neomi Rao, a former Trump administration official whom Mr. Trump
appointed to the bench in March, dissented, saying she would have
quashed the subpoena as exceeding the House's legislative powers.

Representative Elijah E. Cummings, the Democratic chairman of the
oversight committee, hailed the appeals court's decision. ``Today's
ruling is a fundamental and resounding victory for congressional
oversight, our constitutional system of checks and balances and the rule
of law,'' he said in a statement. ``For far too long, the president has
placed his personal interests over the interests of the American
people.''

Lawyers for Mr. Trump were reviewing the decision, said one, Jay
Sekulow. ``We continue to believe that this subpoena is not a legitimate
exercise of Congress's legislative authority,'' he said. Sarah E.
Sutton, a spokeswoman for the Justice Department, declined to comment on
the ruling.

The scope of Congress's power to compel the production of information
--- and the president's power to keep information secret --- has emerged
as a recurring battleground between House Democrats and Mr. Trump, whose
legal team has put forth novel legal arguments in carrying out his vow
to systematically defy House subpoenas.

This week, Mr. Trump's White House counsel, Pat A. Cipollone, sent a
letter to the House declaring that the administration
\href{https://www.nytimes3xbfgragh.onion/2019/10/08/us/politics/sondland-trump-ukraine-impeach.html}{would
not cooperate} with the House's impeachment inquiry, such as by
providing documents or permitting witnesses to testify.

The Trump legal team has separately argued that Mr. Trump's current and
former White House aides are absolutely immune from subpoenas for their
testimony --- meaning they would not even have to show up --- and that
Congress lacks legitimate legislative authority to scrutinize potential
wrongdoing in the executive branch.

The appeals court ruling on Friday centered on that third argument, and
it was in some respects already obsolete because the premise of the
argument was that the House was relying only on its routine legislative
and oversight authorities, rather than any extra investigative powers
that lawmakers gain when engaged in an impeachment inquiry.

But since the Mazars case started going through the courts, the House
Judiciary Committee and Speaker Nancy Pelosi have declared that the
chamber is conducting an impeachment inquiry. The Trump administration
has disputed that premise, since the full House has not voted for a
resolution approving such an investigation.

Friday's ruling did not address the question of whether an impeachment
inquiry is underway --- and, if so, whether that matters. The majority
on the panel ruled for the House without any need to invoke its
impeachment powers.

Specifically, the fight centered on whether the House Oversight and
Reform Committee had the authority to compel Mazars to turn over Mr.
Trump's financial records as part of its routine legislative and
oversight powers, even if the committee's purpose was in part to
determine whether Mr. Trump committed crimes.

The committee, led by Mr. Cummings, issued the subpoena after it came to
light that Mr. Trump had failed to list on his ethics disclosure forms a
debt that he owed and then repaid to his former lawyer and fixer,
Michael D. Cohen, for paying hush money to a pornographic actress just
before the 2016 election to keep quiet about an affair she claimed to
have had with Mr. Trump. He has denied the relationship.

Mr. Cohen also testified before Congress that Mr. Trump routinely
changed the value of his assets for different financial purposes, like
inflating their value for loan applications but deflating them for
taxes. (Mr. Cohen was separately convicted of lying to Congress in
earlier testimony.)

In making the request for documents and then issuing the subpoena, Mr.
Cummings said that Congress was trying to determine whether the
president had broken laws, but he also said that lawmakers were trying
to decide whether to update financial disclosure laws.

Mr. Trump's legal team, which sued Mazars to obtain a court order
blocking it from complying with the subpoena, argued that this meant the
committee was effectively trying to carry out a criminal investigation.
That, they asserted, was an executive branch job, not a legislative one,
and so the House's subpoena was invalid.

But Judge Tatel wrote that Supreme Court precedent granted broad
authority to Congress to issue subpoenas for information that could be
used to enact new laws --- even if it could also be useful as criminal
evidence --- and imposed no requirement on lawmakers to say that
legislation was their specific purpose. In any case, he also wrote, Mr.
Cummings had made reference to investigating whether new legislation was
necessary.

``We conclude that the public record reveals legitimate legislative
pursuits, not an impermissible law enforcement purpose, behind the
committee's subpoena,'' he wrote.

Judge Rao disagreed. Lawyers for the House, led by their counsel Douglas
Letter, had relied only on the House's normal oversight powers in
support of the subpoena, she noted, arguing that permitting lawmakers to
pursue such information without an impeachment inquiry would make it too
easy to harass the presidency.

``When the House chooses to investigate the president for alleged
violations of the laws and the Constitution, it must proceed through
impeachment, an exceptional and solemn exercise of judicial power
established as a separate check on public officials,'' she wrote.

But Judge Rao did not address the House Democrats' more recent
assertions that they are, in fact, already conducting an impeachment
inquiry --- nor the Trump administration's claims that there is no such
inquiry for legal purposes.

In a letter to Democrats celebrating the ruling, Ms. Pelosi wrote: ``The
president's actions threaten our national security, violate our
Constitution and undermine the integrity of our elections. No one is
above the law. The president will be held accountable.''

Advertisement

\protect\hyperlink{after-bottom}{Continue reading the main story}

\hypertarget{site-index}{%
\subsection{Site Index}\label{site-index}}

\hypertarget{site-information-navigation}{%
\subsection{Site Information
Navigation}\label{site-information-navigation}}

\begin{itemize}
\tightlist
\item
  \href{https://help.nytimes3xbfgragh.onion/hc/en-us/articles/115014792127-Copyright-notice}{©~2020~The
  New York Times Company}
\end{itemize}

\begin{itemize}
\tightlist
\item
  \href{https://www.nytco.com/}{NYTCo}
\item
  \href{https://help.nytimes3xbfgragh.onion/hc/en-us/articles/115015385887-Contact-Us}{Contact
  Us}
\item
  \href{https://www.nytco.com/careers/}{Work with us}
\item
  \href{https://nytmediakit.com/}{Advertise}
\item
  \href{http://www.tbrandstudio.com/}{T Brand Studio}
\item
  \href{https://www.nytimes3xbfgragh.onion/privacy/cookie-policy\#how-do-i-manage-trackers}{Your
  Ad Choices}
\item
  \href{https://www.nytimes3xbfgragh.onion/privacy}{Privacy}
\item
  \href{https://help.nytimes3xbfgragh.onion/hc/en-us/articles/115014893428-Terms-of-service}{Terms
  of Service}
\item
  \href{https://help.nytimes3xbfgragh.onion/hc/en-us/articles/115014893968-Terms-of-sale}{Terms
  of Sale}
\item
  \href{https://spiderbites.nytimes3xbfgragh.onion}{Site Map}
\item
  \href{https://help.nytimes3xbfgragh.onion/hc/en-us}{Help}
\item
  \href{https://www.nytimes3xbfgragh.onion/subscription?campaignId=37WXW}{Subscriptions}
\end{itemize}
