The China Connection: How One D.E.A. Agent Cracked a Global Fentanyl
Ring

\url{https://nyti.ms/31lx7BU}

\begin{itemize}
\item
\item
\item
\item
\item
\item
\end{itemize}

\includegraphics{https://static01.graylady3jvrrxbe.onion/images/2019/10/20/magazine/20mag-fentanyl-01/20mag-fentanyl-01-articleLarge-v6.jpg?quality=75\&auto=webp\&disable=upscale}

Sections

\protect\hyperlink{site-content}{Skip to
content}\protect\hyperlink{site-index}{Skip to site index}

\hypertarget{the-china-connection-how-one-dea-agent-cracked-a-global-fentanyl-ring}{%
\section{The China Connection: How One D.E.A. Agent Cracked a Global
Fentanyl
Ring}\label{the-china-connection-how-one-dea-agent-cracked-a-global-fentanyl-ring}}

Fentanyl is quickly becoming America's deadliest drug. But law
enforcement couldn't trace it to its source --- until one teenager
overdosed in North Dakota.

Credit...Photo illustration by Najeebah Al-Ghadban

Supported by

\protect\hyperlink{after-sponsor}{Continue reading the main story}

By Alex W. Palmer

\begin{itemize}
\item
  Published Oct. 16, 2019Updated Oct. 24, 2019
\item
  \begin{itemize}
  \item
  \item
  \item
  \item
  \item
  \item
  \end{itemize}
\end{itemize}

Around 3 a.m. on Saturday, Jan. 3, 2015, Laura and Jason Henke awoke
with a start at their home in Minot, N.D. Their dog was barking wildly.
At the door, in the early morning shadows, they found a police officer
and, behind him, a pastor. The officer asked to see Laura's ID to
confirm that he was at the correct address. Then he told them that their
18-year-old son, Bailey, was dead.

The officer didn't have many details. Bailey Henke was living in Grand
Forks, three hours east of his parents' home in Minot, and the police
there were working the case. The officer gave Laura the phone number for
a detective in Grand Forks. She called and wrote down what he said:
overdose, fentanyl. Laura had never heard of fentanyl; she wasn't even
sure how to spell it.

After a few minutes, the officer and the pastor left. A heavy snowstorm
had closed the roads, leaving Laura and Jason unable to reach Grand
Forks that night. They spent the dark hours sitting on the couch,
waiting for the storm to clear, moving in and out of spasms of
inconsolable crying. They mostly passed the time in silence. Their son
was dead. What was there to say?

Before that knock on the door, Laura was certain that she knew
everything about Bailey. She was the person he talked to when he had his
first crush, and when he started dating his first girlfriend; she knew
that he loved wearing Halloween costumes on random days throughout the
year because it reminded him of playing dress-up as a kid; she laughed
at the funny accents he practiced, at the dorky jokes only the two of
them shared. In high school, Bailey was beloved. His teachers teased him
about his ``clown car,'' because so many of the other students wanted to
pile in to join him for lunch break. He was the type of kid that
teachers remember, that they keep talking about for years.

When Bailey was a junior in high school, Laura caught him smoking pot in
the basement. She said he had to stop, and he was apologetic,
embarrassed, not defiant. She thought that was the end of it. Bailey
just learned to be more discreet. His drug habit became worse in the
fall of 2014, when he dropped out of community college after only a few
months of classes and moved in with one of his best friends, Kain
Schwandt, in Grand Forks. By the time they became roommates, Schwandt
was using heroin multiple times a day. Bailey told his friends that he
had tried heroin a few times over that summer. Living together, they
both used more and more, until they found something even stronger.

Schwandt's fentanyl connection was a friend of a friend, a local
teenager named Ryan Jensen. Schwandt experimented with fentanyl before
he began buying from Jensen, but it was in the form of a medicinal
patch, a legitimate pharmaceutical product diverted from its intended
use as a pain reliever. The powder Jensen sold was cheaper and more
potent, and a small amount lasted a long time. Some medicinal patches
held 100 micrograms and cost \$300-\$400. Ten milligrams of the powder
--- 100 times more than the patch --- cost \$10 and kept you high all
day. The danger, too, was significantly greater, but once Schwandt tried
the powder, he was hooked.

Jensen, the dealer, was quiet, introverted and brainy. He tried
explaining to Schwandt once how he bought fentanyl and where it came
from, but Schwandt wasn't interested. ``He said he got it on this
website, and mentioned Bitcoin,'' Schwandt told me. ``It's like he was
speaking Chinese.'' At first, Jensen only bought for himself; he wasn't
in it to make money, his friends told me. The allure of fentanyl was
that it didn't show up on standard drug screenings. Once word got out,
people started coming over to Jensen's house to get high with him. His
mother later confided to Schwandt's mother that she was just happy he
had friends.

Jensen had a system, according to those who used with him. He knew that
fentanyl was so potent that even a small dose could be deadly, so he
liked to be there to make sure nothing went wrong. When he sold it, it
was in carefully measured amounts. He gave it only to people he knew and
trusted. Schwandt was one of those people; after coming over enough
times with Schwandt, Bailey became one, too. The night Bailey died, just
a few months after he began using, Jensen broke one of his rules: After
they smoked together, he let Bailey leave with several doses.

Bailey's memorial service was held a week after his death. The Rev. Paul
Knight, the pastor at Hope Church, took the pulpit in front of more than
300 people, most of them friends and teachers of Bailey's, and told the
story of the prodigal son. He concluded with a line that he marked in
bolded capital letters in his notes: ``WHAT IF YOU JUST SAID, `YES, I
NEED TO COME BACK TO MY SENSES ... I NEED TO COME HOME.' '' It was part
admonition, part plea: If the fentanyl crisis remained unnoticed in the
rest of the country, in Grand Forks it was already bursting hideously
into view. The night that Bailey overdosed, another local teenager --- a
friend of Bailey's --- overdosed and survived. There were several other
overdoses that same evening. One of Bailey's friends suffered an
overdose later that week, in the same apartment where Bailey had died.
According to someone familiar with the incident, the friend had found
the rest of the fentanyl that had killed Bailey and tried it. When
paramedics arrived at the apartment, they had to walk over the
bloodstain from Bailey's death to help the girl.

More overdoses followed. In the span of a few weeks, Grand Forks had
more fentanyl overdoses than it experienced in previous decades. No one
knew where it was coming from or how all these kids had gotten access to
it so easily. Paramedics began to wonder if there would be any kids left
in town once the outbreak passed.

An investigation into Bailey's death was already underway by the time of
the funeral. The police had his wallet and phone, and they were
beginning to track the fatal dose higher and higher up the distribution
chain. The case quickly expanded beyond North Dakota, across state lines
and federal agencies and into Canada. Chris Myers, who was then the
first assistant United States attorney for the District of North Dakota,
stepped in to coordinate.

After the funeral service, Bailey's mother, Laura, approached Pastor
Knight and thanked him. ``This is a much bigger case than people
realize,'' she said as they embraced. ``Some good will come of this.''

\includegraphics{https://static01.graylady3jvrrxbe.onion/images/2019/10/20/magazine/20mag-fentanyl-04/20mag-fentanyl-04-articleLarge-v2.jpg?quality=75\&auto=webp\&disable=upscale}

\textbf{A year and} a half earlier, 2,000 miles southeast of Grand
Forks, a young Drug Enforcement Administration agent in West Palm Beach,
Fla., named Mike Buemi was deep into his own investigation. The target
was a drug ring that had been importing a product unrelated to fentanyl
called Molly. In D.E.A. parlance, Molly was known as a new psychoactive
substance, or N.P.S., a catchall term meant to encompass the growing
class of mostly synthetic drugs that looked and acted like traditional
drugs but that had been chemically modified just enough to avoid
scrutiny from law enforcement.

Buemi was in his early 30s, with a square jaw, close-cropped brown hair
and an easy smile. He started at the D.E.A. in 2012, after college
R.O.T.C. and seven years as an Army officer, including a tour in
Afghanistan. He had retained an officer's sense of leadership and a
no-nonsense approach to grinding out a problem, no matter how long it
took. ``You get up at 4:30 every morning and don't know when you're
getting off,'' he told me. ``Then you get deployed and don't see your
family for a year and a half. That teaches you a good work ethic.''
Service had also inculcated in Buemi a healthy disrespect for arbitrary
rules and regulations. ``If something's interesting to me, I want to get
into it,'' he says. ``I don't make policy, but I can solve things. When
I hit roadblocks, I want to figure it out and get around them.''

When he began working the Molly case, Buemi's ambition was to identify
the ultimate source of the drugs. He launched a virtual reconnaissance
mission, sleuthing through online ads and forum postings, many of which
linked back to a saleswoman in China who went by the name Li Li. Posing
as a prospective buyer, Buemi reached out to Li Li and, after a few
weeks, had learned enough to begin mapping out the network of American
distributors.

For Buemi, the China connection was hardly a surprise.
\href{https://www.uscc.gov/Research/fentanyl-china\%E2\%80\%99s-deadly-export-united-states}{According
to the State Department}, China has between 160,000 and 400,000 chemical
companies operating legally, illegally or somewhere in between --- an
expansive estimate that reflects both the vastness of the industry and
the scarcity of the information available. Some of these facilities
manufacture tons of chemicals every week, or more than a million pills
per day. In 2016, the industry made up 3 percent of China's national
economy, with over \$100 billion in profits annually. Most of these
companies are members of the vast pharmaceutical underclass, pumping out
huge quantities of inexpensive generic drugs and pharmaceutical
ingredients. It's a low-cost, low-profit business, but the barriers to
entry are minimal, and the market is immense: The basic pharmaceutical
ingredients that China produces are needed by more advanced drug
companies everywhere in the world --- including the United States ---
for synthesis into more complex and profitable medicines.

The agency responsible for overseeing production of drugs and detecting
malfeasance in China is understaffed and overwhelmed: As of 2017, there
were around 2,000 inspectors at the agency, and they conducted a total
of only 751 inspections that year, a minuscule figure compared with the
enormousness of the industry. In the United States, law enforcement and
prosecutors have the tools to react quickly to the rise of new copycat
drugs that could be used for illicit purposes. Under the Controlled
Substance Analogue Enforcement Act, passed in 1986, any new compound
that is ``substantially similar'' to an already banned, or scheduled,
drug can be treated as if it were chemically identical. But chemicals
banned in the United States often remain legal in China, where the
process for controlling chemicals is slow and cumbersome, especially for
substances like fentanyl that exist in the purgatory between legitimate
pharmaceuticals and illicit drugs.

The scale of the problem was enough to overwhelm entire agencies, much
less one investigator like Buemi. Yet his contact with Li Li offered a
starting point. Looking through the catalog of drugs on offer, Buemi saw
Molly --- but he also saw pills containing a mix of oxycodone and
something called acetyl fentanyl, dyed and pressed to look like
legitimate prescription pain pills. Buemi had a hunch that the same drug
ring responsible for moving Molly into South Florida might also be
importing the acetyl fentanyl pills. He placed an order for both.

Li Li took an interest in her new customer. Apparently hoping to groom
Buemi as an American distributor, she shared with him the name and
telephone number of a man in the Northeastern United States who
distributed the company's drugs across the country. ``Reach out to
him,'' Buemi recalls Li Li saying. ``He'll tell you everything about
us.''

On their first call, almost before Buemi could introduce himself, the
distributor leapt into a long, feverish recap of a lifetime of illicit
activities. All he wanted to do was talk about drugs. The distributor
said that he had been caught once by the D.E.A., but that he had learned
his lesson. Now he was more careful.

Buemi had by then received the acetyl fentanyl pills from Li Li: 50 of
them, five strips of 10 wrapped neatly in paper, tucked inside a
letter-size Canada Post envelope. He was doing his own research on
fentanyl, but still didn't really understand what it was or why it was
being shipped to the United States as a street drug. On the call, the
distributor brought it up unprompted.

``Have you ever heard of fentanyl?'' the distributor asked. He walked
Buemi through the basics: Just sprinkle fentanyl into heroin or a batch
of counterfeit prescription pills, and the potency --- and the street
value --- increases enormously. Unlike heroin, cocaine or marijuana, the
distributor explained, fentanyl didn't require farmland, crops,
sunshine, rain or field hands. It was made in a lab and was easy to
synthesize from a few ingredients. Cheap, highly addictive and easy to
move, fentanyl was a drug trafficker's dream. It was also deadly. The
distributor told Buemi that he had put some of a heroin-fentanyl mixture
into a friend's needle. Almost as soon as his friend injected, the man
overdosed and died. It was the friend's fault, the distributor added ---
he had been drunk and high when he took it.

Buemi knew that the danger of the drug would not deter traffickers ---
there was simply too much money to be made. A kilogram of fentanyl,
purchased for only a few thousand dollars, can be mixed with heroin and
made into a couple million dollars' worth of pills; by contrast, a
kilogram of undiluted heroin nets less than \$80,000 in profit.
\href{https://www.dea.gov/sites/default/files/2018-07/DIR-040-17_2017-NDTA.pdf}{A
Ziploc bag of fentanyl will pay for college, with money left over to buy
a house and car.}

Fentanyl was coming, Buemi realized, and no one was prepared for its
arrival. Hoping to raise the alarm, he took his case file to a local
prosecutor. But the prosecutor could see no way to pursue a case. She
couldn't indict anyone in China, and it wasn't clear that the exotic new
fentanyl analogues were illegal in the first place.

Buemi couldn't let it go. There was something odd about that package
he'd received, about the whole operation he'd uncovered. He had been
trying to buy pills from China --- why were they being shipped from
Canada?

Well, Buemi thought, this is going to be an interesting case.

Image

Shanghai's automated Yangshan Port is one of the busiest in the world.
Chinese parcel volume grew to 20.6 billion units in 2015 from 1.2
billion in 2007, further complicating law enforcement's ability to
identify an illicit shipment.Credit...Johannes Eisele/Agence
France-Presse --- Getty Images

\textbf{Fentanyl was first} synthesized in 1960 by Paul Janssen, a young
Belgian doctor and pharmacologist who started his own lab on the third
floor of the family's pharmaceutical business. Janssen, who eventually
became one of the most prolific drug inventors of all time, making
important contributions in allergy, botanical disorders, fungal disease,
gastroenterology, immunology, parasitology, psychiatry, veterinary
medicine and virology, had watched his 4-year-old sister die of an
agonizing and prolonged bout of tuberculosis. His ambition was to create
novel compounds, drugs that would change --- and save --- people's
lives.

Fentanyl represented a towering achievement. At the time, it was the
most potent opioid in the world, 100 to 200 times stronger than
morphine, as well as the fastest-acting and the easiest to dose safely.
The drug caught the attention of Johnson \& Johnson, which acquired
Janssen Pharmaceutica in 1962. It also drew interest from the United
States Army Chemical Corps, which recognized the drug's destructive
potential and began researching fentanyl's use as a chemical weapon or
incapacitating agent.

Fentanyl hit the European market in 1963, but the drug struggled to
catch on in the United States until researchers began to study the
application of high-dose fentanyl as an anesthetic for use during
open-heart surgery. In contrast to morphine, which had serious drawbacks
when used in major surgery, fentanyl produced a smooth, predictable
anesthesia with minimal side effects. Within a few years, high-dose
fentanyl was being used as the anesthetic in virtually every cardiac
surgery in the United States.

The commercial adoption of fentanyl came a decade later, when the
proliferation of novel delivery systems turned the drug into a mainstay
for patients suffering from acute pain. In 1984, researchers at the
University of Utah created a ``child friendly'' sweetened, red
lollipop-like product called Oralet; soon thereafter came sprays,
tablets, films and more. The most successful of these was a transdermal
patch called Duragesic, which allowed for the slow, controlled release
of fentanyl in patients with chronic pain. Here, fentanyl was truly
revolutionary: Unlike many other drugs, which needed to be injected
directly into a patient's bloodstream, fentanyl was so potent that
merely placing it on the skin was enough to relieve pain. By 2004,
Duragesic sales were more than \$2 billion worldwide.

Paul Janssen's son, Pablo Janssen, told me that he could recall one
occasion during his father's life in which the idea of fentanyl's latent
lethality came up. The two were watching a movie in which one of the
characters took fentanyl as a recreational drug and became addicted.
Father and son found the scene laughable. Fentanyl had been used safely
by millions of patients around the world for decades. The likelihood
that it would ever turn into a commonly used street drug seemed absurd.
How could such a thing ever happen?

Yet Dr. Robert Dripps, the chairman of the Department of Anesthesiology
at the University of Pennsylvania from 1943 to 1973 and a towering
figure in American medicine, initially argued strongly against the
drug's certification by the F.D.A. There was no need for most doctors to
be using such a powerful drug, Dripps said. He worried that fentanyl's
potency, and the near-instant high the drug produced, would lend itself
to abuse. The only barrier to fentanyl's widespread nonpharmaceutical
use was the relatively cumbersome synthesis process. Janssen's technique
for creating fentanyl was long, difficult and dangerous. But in the
decades after Janssen's initial discovery, a series of chemists managed
to streamline and simplify the production.

One anonymous innovator, claiming to be a French chemist by the name of
Siegfried, posted his findings online, in a brief, awkwardly worded
paper that was unlike anything that had come before it. His interest in
fentanyl was recreational, not academic. He had dug up an obscure
synthesis method, then gathered reactions from users for over a year.
``The feedback of the consumers was very good,'' he wrote. But the risk
of overdose, he noted, was extreme: The fentanyl produced by the method
``MUST be diluted, else there will be a lot of overdoses!'' (The true
identity of Siegfried has never been discovered, and no one has come
forward to claim credit for his work.)

The development of ever-simpler synthesis methods helped catalyze a
devastating fentanyl outbreak in the mid-2000s, when illicit fentanyl
flooded drug markets in the Midwest, claiming more than 1,000 lives. The
D.E.A. traced production to a single lab in Toluca, Mexico, where one of
the lab's operators told authorities that he'd bought the necessary
chemicals from a Chinese company. The outbreak followed a familiar
pattern, one established by a series of smaller outbreaks in the
preceding decades: A rogue operator produces illicit fentanyl in a
single underground lab; law enforcement finds the lab, catches the bad
guy and identifies the chemicals used so that they can be banned or
controlled. After the authorities shut down the Toluca lab in May 2006,
the overdoses soon stopped.

But China's rapidly expanding drug industry changed the equation. The
simplicity of the new organizational structure that Buemi had begun to
uncover --- dealer, shipper, buyer --- made it nearly impossible for law
enforcement to crack the networks. There were no foot soldiers to round
up and press for information, no face-to-face meetings to surveil. And
because the internet offered a bevy of manufacturers to choose from,
buyers in the United States could change sources constantly, making it
difficult to detect patterns in their purchases. ``If Mexican cartels
were the big-box stores of the drug market,'' says Dr. Bryce Pardo, a
drug-policy researcher at the RAND Corporation, ``the Chinese are
Amazon'' --- cheap, convenient and ubiquitous.

The Postal Service suddenly became perhaps the largest
drug-transportation network in the world, delivering fentanyl from China
straight to American homes. Catching an illicit shipment in transit was
nearly impossible. According to
\href{https://www.hsgac.senate.gov/imo/media/doc/Combatting\%20the\%20Opioid\%20Crisis\%20-\%20Exploiting\%20Vulnerabilities\%20in\%20International\%20Mail1.pdf}{a
2018 report} by the United States Senate Committee on Homeland Security
and Governmental Affairs Permanent Subcommittee on Investigations,
Chinese parcel volume increased to 20.6 billion units in 2015 from 1.2
billion in 2007. The United States Postal Service's international parcel
volume increased to nearly 500 million packages in 2017 --- far more
than FedEx, U.P.S. and DHL combined --- from about 150 million in 2013.
More than half of those arrived at just one facility in New York.

In the face of that torrential flow, the odds of catching a few hundred
grams of powder tucked in an envelope were minuscule. ``We went from
looking for a needle in a haystack to looking for a bacteria colony on a
needle in a haystack,'' Pardo told me. ``If you can move 10 grams from
China, and it's profitable, it's almost impossible to stop.''

Image

Jason Berry, currently imprisoned in Canada after trying to ship
fentanyl to Colorado.Credit...Photo illustration by Najeebah Al-Ghadban

\textbf{When he began} his investigation, which has never been fully
reported on, Buemi just wanted to figure out how an envelope of acetyl
fentanyl had made its way to South Florida. His first step was to
isolate the Canadian connection. Using the acetyl fentanyl seizure as
evidence, Buemi obtained email-search warrants for the addresses he knew
to be associated with Li Li. The company was using United States-based
servers, which allowed Buemi to pore through emails and identify
customers, dealers and distributors. One name kept popping up: Jason
Berry.

Buemi obtained an email-search warrant for addresses associated with
Berry and soon realized that he had stumbled onto a major criminal
organization. At its head appeared to be Berry, coordinating the
importation of huge quantities of drugs into the United States, via
Canada, from Chinese chemical companies. ``It was an astronomical amount
of drugs coming in from China,'' Buemi says --- fentanyl chief among
them.

Berry had already amassed a notable history with fentanyl. In April
2013, while out on probation from a previous drug conviction, he and an
accomplice were arrested at a Montreal U.P.S. store after trying to ship
a microwave and a toaster oven containing 10,000 pills of desmethyl
fentanyl to an address in Colorado. Raids on storage units associated
with Berry turned up hundreds of thousands of synthetic prescription
pills and 1,500 kilograms (about 3,300 pounds) of raw ingredients ---
enough to make three million pills. Canadian law enforcement also found
pill presses capable of producing thousands of pills an hour. By the
time Buemi located him, Berry was incarcerated at the Drummond
Institution, a medium-security facility in Quebec. Prison didn't seem to
have slowed him down.

Buemi told me that in April 2014 he tried to make contact with Berry,
posing, as he had with Li Li, as a prospective distributor. Berry
advertised his products with postings on forums that catered to
recreational drug users, research-chemical enthusiasts and pill addicts.
On one of these sites, called LookChem, which bills itself as ``the
world's leading B2B trading platform, especially good for looking for
chemicals from China,'' Buemi found an old email account of Berry's and
sent him a message asking for fentanyl. ``This email is burned,'' the
account replied. ``Contact me at
\href{mailto:undergroundportal777@gmail.com}{\nolinkurl{undergroundportal777@gmail.com}}.''
He said to call him Mountain.

On the new account, Mountain described the cornucopia of fentanyls he
had on offer, including fentanyl citrate, fentanyl HCL and acetyl
fentanyl pills he called boxy-100s: circular blue pills with ``oxy'' on
one side and ``100'' on the other, the same pills that Buemi had
received in the Canada Post envelope. What was Buemi looking for? ``I
want it all,'' Buemi said. ``Everything you've got.''

Mountain started vetting Buemi for a role in his organization. ``What
can you move?'' he asked. He talked endlessly about his search for the
perfect pill, one with the right mixture and a catchy brand name. One
potential product was called OD-100. ``You know, even though it says OD
on it, users like that --- police won't be on to it but users will,''
Mountain said. Buemi pushed Mountain to continue communicating via
email, rather than more secure messaging apps, amassing evidence to feed
to his email-search warrant.

As the weeks passed and their correspondence continued, Buemi began to
suspect that he wasn't talking to Berry at all. From the emails he was
seeing and his observations of the network's operators, it seemed as if
there was someone else involved at the prison. Buemi set up a second
buy, in September 2014, as a means of forcing the co-conspirator to
reveal himself. He wired the payment for the drugs to accounts Mountain
specified, and also alerted the Royal Canadian Mounted Police. Mounted
Police agents surveilled two people picking up the money at a bank and a
Western Union. One of them was associated with another inmate at the
Drummond Institution named Daniel Vivas Ceron. Vivas Ceron was serving a
long prison sentence for smuggling a kilogram of cocaine into Canada
from Panama and opening fire on four people after an argument at a bar.
Buemi obtained a search warrant on Vivas Ceron's Facebook account to
confirm that Mountain had been Vivas Ceron all along. As a Canadian
police official explained it to me, it isn't unusual for a new inmate
with a booming illicit business, like Berry, to enlist a longtime
prisoner with a cellphone and connections to the outside world, like
Vivas Ceron, to help handle operations. (Berry, through his lawyers,
denied any role in fentanyl distribution from prison.)

Buemi and Vivas Ceron continued talking regularly on email and Wickr, an
encrypted messaging service that allows users to set an expiration time
for their communications. Eventually, Buemi caught a glimpse of the
network's final and most important component: Zaron Bio-tech, a company
based in Shanghai and registered in Hong Kong. Zaron billed itself as a
food-additive manufacturer, but it was in fact responsible for a
majority of the fentanyl that Berry and Vivas Ceron were moving into the
United States. The two inmates were merely deputies of a global
operation --- whoever was behind Zaron was the real mastermind.

Buemi had seen what happened when United States agents tried to build a
case in China. Before Berry's arrest in April 2013, Buemi says,
investigators in Colorado had been corresponding with Zaron, sending
payment for drug shipments that Berry was then assembling and
distributing. When the investigators tried to make headway with Chinese
officials, however, they got nowhere. ``They were sending chemicals here
and saying, Hey, this isn't illegal, we have no issues here,'' Buemi
recalls. He knew that if he pressed, he would get the same reply. ``It
wasn't the right time with China,'' he says.

Image

Daniel Vivas Ceron, another Canadian prison inmate, sold fentanyl over
the internet while incarcerated.Credit...Photo illustration by Najeebah
Al-Ghadban

Despite all the evidence he had assembled, Buemi was still having
trouble convincing American prosecutors to take fentanyl seriously. ``No
one could wrap their heads around what we were doing'' --- undercover
deals, writing email-search warrants, Buemi says. ``They didn't know how
to prosecute this.''

The D.E.A. itself was missing the warning signs. The agency's 2014
National Drug Threat Assessment mentioned fentanyl only briefly, as a
subcategory of heroin. While there had been an uptick in fentanyl
seizures and overdoses across the country, the report stated, it was
``not as deadly as the 2005-07 outbreak'' caused by the lab in Toluca,
Mexico. ``I didn't realize the severity of it, how quickly it would
move,'' Derek Maltz, the former head of the agency's Special Operations
Division, told me. William Brownfield, the former head of international
narcotics and law enforcement affairs at the State Department, put it
more bluntly: ``We were blindsided.''

Buemi needed irrefutable proof that the drugs being shipped from China
were killing users, as well as a clear chain of possession stretching
from overdose victim all the way back to the Chinese source. Without
those pieces of evidence in place, the Chinese government would refuse
to cooperate. Then, in early January 2015, Buemi received a phone call:
A young man named Bailey Henke in Grand Forks, N.D., had died of a
fentanyl overdose. Buemi thought he'd found his case.

\textbf{The weather was} harsh and unwelcoming when Buemi arrived in
Fargo, but he felt newly hopeful. At their first meeting, Chris Myers,
the acting United States attorney, told Buemi what they had found: On
Jan. 1, 2015, Ryan Jensen received two packages in the mail containing
one gram of fentanyl and 12 grams of heroin. The following day, he
smoked some of the fentanyl with Bailey and two other friends at his
house. Bailey left after buying another share of Jensen's fentanyl
supply.

The police visited Jensen soon after Bailey's death and pressed him for
answers. He confessed to his drug dealing and gave investigators his
login information for Evolution, a site on the dark web, allowing them
to track the purchase back to its source: a vendor with the username
pdxblack, whom the police determined to be Brandon Corde Hubbard, a man
living in Portland, Ore. Hubbard was working out of his home while
directing drug shipments to a seemingly innocuous second address:
Northwest Oil Solutions in Woodland, Wash., where a co-conspirator
worked. Two months before Bailey died, Hubbard purchased about 750 grams
of fentanyl, with a street value of \$1.5 million. The dose that killed
Bailey, investigators believed, had come from that shipment. The police,
knowing that Hubbard was communicating with someone using the same Wickr
name as a suspect in Buemi's case, believed that Hubbard was a
lieutenant in a larger network, but they couldn't follow the trail any
farther.

``They had no idea how big of a case they had,'' Buemi told me.

Buemi told the team in North Dakota about Vivas Ceron, Berry and Zaron
Bio-tech --- and how their overdose case connected all the pieces, from
producer to distributor and user. The team in North Dakota immediately
understood the significance of the case and resolved to combine their
efforts. Their joint investigation would become known as Operation
Denial.

Over the next six months, the investigation unspooled in strange and
grotesque ways, an epidemic in miniature. The team coordinated with
Portland law enforcement to arrest Hubbard, but before they could, his
car was stolen. The car thief, thinking the baggie he found in the
vehicle was cocaine, shared the drug with three friends. It was
fentanyl, and all four overdosed. In March, the police arrested
Hubbard's girlfriend on charges of tampering with evidence related to
the ongoing investigation. They didn't realize that she had hidden a bag
of fentanyl inside her body. After being booked into jail, she gave a
portion of the drug to an inmate, who then shared it with others. Four
women overdosed, one fatally. (Hubbard pleaded guilty on three charges
and was sentenced to life; his girlfriend was sentenced to 11 years.)

Investigators also learned that a friend of Bailey's named Evan Poitra,
who died of a fentanyl overdose in July 2014, had been a victim of the
same distribution network that killed Bailey. They found overdoses going
back as early as January 2014 that they were able to connect directly to
Zaron Bio-tech. Six months after the initial meeting in North Dakota,
Buemi and another agent arrested Vivas Ceron when he was released from
prison, shutting down one node of the network. (Vivas Ceron pleaded
guilty to drug-trafficking and money-laundering charges and is awaiting
sentencing.)

Despite their efforts, the pace of the fentanyl epidemic in the United
States was accelerating. In
Ohio,\href{https://www.nytimes3xbfgragh.onion/2018/04/04/opinion/carfentanil-fentanyl-opioid-crisis.html}{fentanyl-related
deaths rose} to 1,155 in 2015 from 75 in 2012. United States border
agents reported seizing 32 kilograms of fentanyl in 2015; the next year,
they confiscated more than 270. Zaron was sending hundreds of kilograms
of the drug into the country, helping fuel a crisis that claimed lives
at a rate eclipsing even H.I.V., gun deaths and car crashes at their
most lethal.

The time was right, Buemi thought, to bring the case to the Chinese.
``Put the shoe on the other foot,'' Buemi says. ``They're halfway across
the world. If we're not sharing, they're not interested.'' People were
already dying by the thousands. He had nothing to lose.

\textbf{Just inside the} entrance to the D.E.A. Special Operations
Division headquarters in Virginia, a large bronze cutout of the globe is
mounted on the wall. Planted in the center are North and South America,
with Europe and Africa off to the right. China's eastern coast sneaks
in, just barely, on the far left of the map. The orientation isn't an
accident. For decades, combating Latin American cartels has been the
D.E.A.'s \emph{raison d'être.} The D.E.A. has at least 11 offices in
Mexico, seven in Central America and another 14 in South America. An
agent informed me that the only D.E.A. presence in China is at the
United States Embassy in Beijing, where there are a handful of agents
and a small support staff. (Another office, in the southern port city of
Guangzhou, was announced in 2017, and is expected to open soon.)

The agency knew how to function effectively in Latin America. As one
D.E.A. agent put it to me: ``You call up specialized units or guys
you're familiar with and say, Hey, there's a ship that just arrived, we
have intelligence that it may have drugs in it, would you take a look?
Of course, yeah. A couple hours later he calls you back and tells you
what he found.'' In Latin America, the daily work of a D.E.A. agent
isn't as fraught with diplomatic baggage. ``China,'' the agent added,
``is not like that.''

Though few countries treat drug offenses as seriously and harshly as
China, the relationship between the D.E.A. and Chinese law enforcement
has long been fractious. Differences in culture, language and mission
compounded a dearth of trust on both sides. Justin Schoeman, a former
agency attaché at the Embassy in Beijing, spent years learning to
navigate the divide. After arriving in Beijing as an assistant attaché
in 2011, he set about trying to forge personal relationships with his
Chinese counterparts, but found that building trust and camaraderie was
difficult --- Chinese agents were too formal for anything spontaneous,
like sharing an after-work beer, and they were reluctant to open up
about life outside the job. Schoeman found himself balancing his
American colleagues' perceptions of what could be done with the
political and diplomatic realities in Beijing. ``The situation in China
is: We need China's help,'' he told me. ``I had people ask, Why don't
you get in there and demand they do this? It just doesn't work that
way.''

The State Department found that it, too, had little room to maneuver.
Serious conversations about fentanyl began around 2013 and went nowhere.
The Chinese vehemently denied any role in the epidemic sweeping across
the United States. They pointed out that opioids were tightly controlled
in China, and that only a handful of firms were licensed to produce or
export medicinal fentanyl. Chinese officials, moreover, seemed to have
fresh memories of the opium scourge imposed on their country by Western
merchants; they had little sympathy for rich countries battling
drug-abuse scandals of their own making. Even while countless Chinese
companies were producing illicit powdered fentanyl for sale to American
customers, the Chinese maintained their stance, and the companies
continued to operate unimpeded. ``It was like talking to a stone wall,''
one former United States diplomat told me. The Chinese were unequivocal:
``We don't know what you're talking about,'' the United States diplomat
recalls being told. ``We have no fentanyl problem.''

Image

The process for controlling chemicals is slow and cumbersome in China,
especially for substances like fentanyl that exist in the purgatory
between legitimate pharmaceuticals and illicit drugs.Credit...Photo
illustration by Najeebah Al-Ghadban. Source photo: United States Drug
Enforcement Administration, via Associated Press.

Schoeman understood that it would be a challenge to persuade his Chinese
counterparts to pay attention to fentanyl. Just like law enforcement
officers in the United States, investigators with China's Ministry of
Public Security (M.P.S.) were evaluated for their accomplishments in
solving crimes and closing cases. China has drug abuse issues of its
own, especially involving meth, heroin and ketamine. Whatever toll
fentanyl was taking in the United States wasn't reflected in China,
where the drug's analogues and precursors weren't considered illegal.
``How would an investigator feel about being put on a U.S. case that
isn't even against the law?'' Schoeman says. They'd naturally wonder:
``Why am I wasting my time on this? I'm tripping over meth and heroin
cases.''

But in October 2015, the situation changed. That month, the Chinese
government said that it was adding 116 synthetic chemicals to its list
of controlled substances. A huge majority of these were new psychoactive
substances, but the list also included six fentanyl analogues. The
D.E.A. had asked for the analogues to be controlled, but part of the
impetus for the move seemed internal. In June 2015, customs officials in
the port city of Guangzhou reportedly seized nearly 50 kilograms of
fentanyl. Around the same time, officials seized another 70 kilograms
hidden in a container on its way to Mexico. Six Chinese agents became
ill after handling the drugs; one fell into a coma. Publicly, Chinese
officials remained reluctant to acknowledge that fentanyl was an issue.
But privately, they were ready to talk. ``We don't have a fentanyl
problem,'' one American diplomat was told by a Chinese counterpart. ``We
have a ketamine problem --- hint hint*.''*

The new regulations contained another unexpected clause, which Schoeman
referred to as the ``golden nugget'' of the measure. For the first time,
Chinese officials could consider damage done in another country as a
criterion for regulating a drug domestically. It was a seemingly minor
change in wording, but the result was groundbreaking: If American
investigators show definitively that a drug from China had killed an
American, Chinese officials would be empowered to act.

Reaching that level of proof would be difficult --- fentanyl
manufacturers had devised systems for keeping themselves in the shadows.
The shipment on its way to Mexico had been sent through five separate
freight forwarders, the nameless middlemen of international shipping,
who handle all the paperwork and logistics of export. Fentanyl
manufacturers used these companies to ship their product anonymously and
disguise its point of origin with layers of anonymity and dead-end
addresses. For freight forwarders, any shipment to the United States was
good business. They earned a healthy profit even while accepting no risk
--- they had no responsibility for checking the contents of the
shipments they were given, or ensuring that the information provided by
the sender was accurate or complete.

Still, the 116 newly listed chemicals, and the accompanying change in
drug-regulating power, gave Schoeman hope that there was an opening for
cooperation. He already saw the mounting casualties from fentanyl back
in the United States; every minor victory he achieved felt as if it was
too little, too late. When Buemi contacted him with news of a case in
North Dakota, Schoeman saw an opportunity.

\textbf{Schoeman's diplomatic} assistance in China allowed Buemi to
initiate undercover contact with Zaron Bio-tech. He sent an email to an
address he had identified through his search warrants, posing as a
friend of Jason Berry and referring to his moniker, Daniel Desnoyers.

``My friend Daniel told me about you so we can work together,'' Buemi
wrote.

``How's he doing?'' the Zaron account replied.

Buemi knew he was being vetted. ``Oh he's good, we talk sometimes but
not too much because he's in the garden,'' he wrote, using a euphemism
for prison. Then Buemi turned the conversation to fentanyl and used the
exchange to get a search warrant for that email account and any others
used by Zaron. He also began making small purchases of furanyl fentanyl
--- a popular fentanyl analogue not covered by the October 2015
regulations --- to establish himself with Zaron.

Buemi noticed that Zaron's shipments arrived from a Los Angeles address.
Coordinating through the D.E.A. Special Operations Division, he learned
that the California distributor, a man named Gary Resnik, was also
communicating with Zaron via Wickr. With agents in California planning
to take down Resnik, Buemi thought of a way to make himself
indispensable to Zaron. On March 14, 2016, he put in an order with
Resnik for 1,000 fentanyl pills. The next day, D.E.A. agents in
California
\href{https://www.justice.gov/usao-cdca/pr/long-beach-man-sentenced-over-26-years-prison-leading-counterfeit-opioid-scheme}{raided
a storage unit} and arrested Resnik and three other men, who were later
indicted on charges including possession of drugs for distribution.
(Resnik pleaded guilty and was sentenced with a co-conspirator to nearly
27 years.) At the storage unit, agents found bulk imports of acetyl
fentanyl from China --- imported as ``toys'' or ``children's clothes''
--- and a makeshift lab equipped with five pill presses capable of
making 40,000 pills, enough product for the operation to net more than
\$3.6 million a year in street sales alone. There was also, Buemi says,
a printout with a tracking number for his purchase and the note ``1k
pills,'' a final order that Resnik was never able to ship.

A few days later, Buemi reached out to Zaron. ``Where are my pills?'' he
asked.

``I don't know,'' the Zaron account replied. ``My guy isn't
responding.'' The arrest was a huge blow for Zaron --- with the Canadian
and the California branches gone, Zaron had lost two of its most
important distributors. Zaron promised to front Buemi several hundred
grams of furanyl fentanyl and U-47700, another powerful synthetic
opioid. Buemi countered with a better offer.

``Hey, I know how to make pills,'' he told Zaron. ``Just give me your
customer orders and I'll fill them.'' Zaron leapt at the offer, and
Buemi instantly became the company's most important lieutenant, with
access to information on all of its active customers across North
America.

But there was a problem. Unlike powdered fentanyl, which Buemi could
easily imitate using, say, pancake mix (a trick he had done once
before), pills were difficult to fake. Most fentanyl dealers, including
Zaron, advertised their pills as perfect replicas of legitimate,
brand-name opioid pills --- circular blue A 215s, square Mallinckrodts
--- but with fentanyl swapped in for the usual pharmaceutical
ingredients. If Buemi was going to keep up the ruse, he had to send out
placebo pills that would be indistinguishable from the real thing.

He told me he turned to the only organizations that he knew could make
flawless pills: the pharmaceutical companies whose products were being
spoofed. ``Drug dealers are using your stamp and killing people,'' he
told the companies. Would they help put a stop to it? The companies
agreed, and sent him thousands of lactose placebos, pressed and stamped
exactly like normal pills.

With the placebos ready, Buemi began his work as Zaron's chief
distributor. Small shipments arrived in a baggie inside a magazine
tucked in an envelope; larger shipments came ensconced in mylar
packaging inside a square box. In both cases, the return address was a
Chinese freight-forwarding company that had handled the shipment. Buemi
took the real drugs into evidence, packaged the harmless pills for
shipment and then sent them wherever Zaron asked.

Buemi was soon able to identify a host of drug-trafficking organizations
throughout the United States that were using Zaron as their source of
supply. There were organizations in Nebraska, Maryland, New Jersey,
South Carolina, Florida, California and Ohio, and more still. They
received fentanyl shipments at hundreds of addresses: car garages and
P.O. boxes, derelict apartments and expensive homes in gated suburban
communities. The network's reach was mind-boggling.

Buemi set about busting the dealers one by one, in a series of elaborate
digital impersonations. After he shipped the placebo pills to Zaron's
chief contact in Nebraska, Buemi coordinated with local agents and had
the buyer arrested. Then, with the buyer silenced, he took over his
Wickr account and contacted Zaron with another order to keep up the
ruse. Zaron then messaged Buemi on his own account. ``Hey man, I need
you to ship pills to Nebraska.''

``Ok,'' Buemi replied. ``What's the address?''

When Zaron messaged the Nebraska man about payment details, Buemi went
to Western Union, opened an account under the man's name and wired
himself the money, creating the paper trail he needed to convince Zaron
that the deal had gone through. Zaron never realized Buemi was posing as
both the sender and the recipient.

The shipment was similarly staged: fake pills, fake address, but with a
real tracking number to show Zaron that the package had been sent and
received. In fact, Buemi had the shipment scanned as if it had been
sent, held it for two days, and then scanned it again. All Zaron saw was
a confirmed arrival. ``I was making sure I wasn't breaking the law, but
I just had to create stuff,'' he told me. ``I didn't know if anyone else
had done this, but it had to happen, and it fit the scenario.'' In the
span of a few months, Buemi turned a prolific fentanyl kingpin,
responsible for hundreds of kilograms of drugs entering the United
States, into a trafficker of lactose pills.

Image

Zhang Jian, the Chinese national whose wares included counterfeit
clothes, sweeteners, sex toys and eventually fentanyl.Credit...Photo
illustration by Najeebah Al-Ghadban

\textbf{Though he used} the moniker Hong Kong Zaron, the man behind
Zaron was actually from Qingdao, a city in eastern China. His real name
was Zhang Jian. Buemi learned he was not a fentanyl manufacturer; he did
not personally operate any factories. Instead, Zhang was a logistician
and a trader. He took orders from customers, coordinated with
manufacturers and sent off the product --- a glorified salesman. ``His
job was: sell stuff, get money,'' Buemi says. Zhang's family was
involved, too. In China, picking up money from Western Union --- Zhang's
preferred method of payment --- requires presentation of a valid
national ID. By tracking who collected the illicit funds, officials saw
that Zhang Jian's wife, brother-in-law and parents were all helping him
retrieve and launder funds.

Zhang had not always been a drug trafficker. He previously sold clothes,
stevia, sex toys, anything on which he could turn a profit. It was only
after connecting with Berry around 2012 that Zhang ventured into drugs.
Sometimes his more innocuous products became a gateway for grooming a
new fentanyl distributor: The organization in Nebraska that Buemi
dismantled had been purchasing steroids from Zhang before transitioning
into fentanyl, and the organization in South Carolina started off buying
counterfeit clothing. In both China and the United States, the booming
fentanyl market was too lucrative for criminals to ignore.

Buemi labored to gain Zhang's trust. For more than a year, the two
talked almost every night. Buemi found him to be much like any good
salesperson: friendly, hospitable, always putting the customer first.
``He came off as a normal dude, not a trafficker,'' Buemi recalls. But
Zhang had no illusions about the business he was engaged in. Buemi and
Zhang had long discussions about finding the right recipe for the pills,
and the importance of avoiding overdoses among what Zhang believed to be
the pair's growing customer base. ``No need to kill all of our
customers,'' he told Buemi.

It was an exhausting double life: During the day Buemi wrote reports,
helped other D.E.A. agents work their cases and searched through the
text messages and emails of the new players he was identifying. After
work, about the same time that Zhang woke up in China, he resumed his
role as a drug lieutenant. Buemi wrote email and Wickr messages during
dinner with family. He often stayed up until 1 or 2 a.m., working his
cover. ``I had to make it happen,'' he says. ``You can't pause and say
you'll do it in a week. It's in motion.''

At the same time, cooperation with the Chinese Ministry of Public
Security was going smoothly. In June 2016, Buemi and a team of
investigators traveled to Beijing to meet with Schoeman and their
Chinese counterparts. The Chinese had opened a domestic case against
Zhang, and they were conducting surveillance, monitoring his mail and
poring over tax records. At the meeting, they provided a translated copy
of Zhang's communications from a Chinese server that Buemi had been
unable to access. They had also intercepted a kilogram of fentanyl being
shipped to Zhang. Buemi and Schoeman were impressed, both by the
enthusiasm of the Chinese agents and by the work they'd already put in.
``They were trying,'' Schoeman says. Whether the Chinese caught Zhang
for drug trafficking or tax evasion or customs fraud didn't matter to
the United States agents, so long as they stopped him.

Buemi returned to Florida and started working on the final piece of the
investigation. Now that he had Zhang's complete trust, he proposed their
biggest deal yet. He told Zhang that he had a contact in the Mexican
cartels. ``They want 100 kilos of furanyl fentanyl, shipped straight to
Mexico,'' he said. Zhang agreed and arranged to send the shipment from
Qingdao via container ship, at a price of \$3,000 per kilogram. Pressed
into pills, 100 kilograms of highly pure fentanyl would make 50 million
potentially fatal doses, with a total street value of hundreds of
millions or even billions of dollars. If the deal went well, Buemi told
Zhang, they could repeat the shipment every month.

The fake cartel deal offered Buemi a chance to make the evidence so
clear and irrefutable that the M.P.S. would have to step in. Though
Zhang preferred to communicate via text message, Buemi pushed for a
Skype call, with another D.E.A. agent playing the role of the cartel
connection. ``These are bad dudes,'' he told Zhang. ``If the money
doesn't go through, they'll kill me, and they'll find you and kill you
too.''

In August, as Buemi was about to finalize the shipment, Zhang suddenly
went silent. Buemi soon discovered why --- Zhang had been arrested in
China. The Chinese hadn't notified him, Schoeman or anyone at the D.E.A.
ahead of time. They must have nabbed him for that kilogram of fentanyl
they found in his mail, Buemi figured. But then, just a few weeks later,
Buemi learned that Zhang had been released. He was not being charged.
Buemi was told that there was an issue with the evidence, but nothing
more.

The joint investigation was over.

\textbf{On Oct. 17, 2017,} Rod Rosenstein, then the deputy attorney
general, announced that Zhang Jian and eight North American
co-conspirators, including Berry,
\href{https://www.justice.gov/opa/speech/deputy-attorney-general-rod-j-rosenstein-delivers-remarks-enforcement-actions-stop-deadly}{had
been indicted in North Dakota}. Zhang was charged in the death of four
people, including Bailey Henke, and the serious injury of five more.
Buemi's work allowed prosecutors to detail the extent of Zhang's
network: distributors in at least 11 states and paying customers
scattered across almost all 50. Between January 2013 and August 2016, he
sent ``many thousands'' of shipments to the United States, part of the
larger flood of Chinese-made fentanyl that officials there continued to
deny.

It was the first time that a Chinese national had been indicted on a
charge of fentanyl trafficking in the United States. Zhang and a second
Chinese national, Yan Xiaobing, were also the first fentanyl traffickers
designated by the Justice Department as Consolidated Priority
Organization Targets, or CPOTs, which denotes an individual with
``command and control'' over the most prolific international
drug-trafficking and money-laundering networks. They became two of some
200 total CPOTs ever designated by the department, joining organizations
like the Sinaloa Cartel and the Colombian guerrilla group FARC.

``We also received valuable investigative assistance from the Ministry
of Public Security of China,'' Rosenstein said. ``And we want to thank
them for their help.'' He did not mention Zhang's arrest in China and
subsequent release.

The Justice and State Departments had clashed over whether to announce
the indictments and how public to make them, according to officials
familiar with the discussions. Officials at the State Department argued
for a more cautious approach, but Justice Department officials thought a
big public statement demonstrated law enforcement's commitment to
stopping fentanyl traffickers. The department wanted to send a message
that the United States would not sit idly by.

Chris Myers, now promoted to United States attorney in North Dakota, was
initially reluctant to indict Zhang, according to agents involved in the
case. He didn't want just an empty name; he wanted a real person who
could be pictured and identified. But once Myers had been convinced, he
was emphatic. ``We've got names,'' Laura Henke, Bailey's mother, recalls
him saying. ``It's a nonextradition country. But one day they'll go on
vacation to Vegas or Mexico or New York, and we'll get them.''

The D.E.A. tried to warn its counterparts in China about the coming
announcement. Dan Baldwin, a former attaché in Beijing now serving as
head of a regional office of the Office of Global Enforcement, was
assigned the thankless task of massaging the news. The Chinese were
furious. Wei Xiaojun, the deputy director general of the
narcotics-control bureau at the M.P.S.,
\href{https://www.nytimes3xbfgragh.onion/2017/11/03/world/asia/china-opioid-fentanyl-trump.html}{told
reporters}, ``China regrets that the U.S. chose to unilaterally hold a
news conference to announce the hunt for these fugitives.'' The release
of the information would ``impact the ongoing joint investigation into
the case.'' Wei further claimed that ``based on the intelligence and
evidence shared'' with the M.P.S., there was no reason to conclude that
``a large portion'' of the illicit fentanyl in the United States had
come from China. He also said that China had noted the United States'
failure to thank China for its cooperation on fentanyl cases, despite
Rosenstein's remarks.

In April 2018, the Treasury Department
\href{https://home.treasury.gov/news/press-releases/sm0372}{announced
that it was sanctioning Zhang Jian}, his company and four of his family
members as Significant Foreign Narcotics Traffickers under the Kingpin
Act --- the first time an accused fentanyl trafficker had been
designated a kingpin.

Watching all of this unfold from Florida, Buemi was both proud and
perplexed. He thought everything had been going smoothly with China. The
cooperation was sincere, and the M.P.S. seemed eager to build its own
case. Zhang had been caught with fentanyl --- not a novel analogue, but
regular fentanyl. ``What happened?'' Buemi said.

Questions also remained about the extent of Zaron's business. In online
postings, Zaron boasted of a corporate history stretching back to 1991.
The company claimed to operate eight factories in China, Vietnam,
Thailand and Singapore, including a production facility in southern
China covering 100,000 square meters and operated by a staff of more
than 1,000 employees. From its headquarters in Hong Kong,
\href{http://www.tradeholding.net/default.cgi/action/viewcompanies/companyid/667101/}{Zaron's
advertisements said}, it had become a leader in ``the sector of food
additive'' across Asia, with its products sold in more than 30
countries. In a release from the Treasury Department, several of these
details were noted. No one I spoke to at the D.E.A., the Justice
Department or Homeland Security Investigations actually knew if the
claims were true; in any case, the investigation was now out of their
hands.

But the trail had not gone completely cold. A few weeks before Zhang's
kingpin designation, corporate ownership of Zaron Bio-tech was
transferred to a man named He Wenxiang at an address in Guangzhou,
China. Searching through Hong Kong corporate registration documents, I
saw that He was the longtime owner of a freight-forwarding company. I
also noticed that, in the five months around Zhang's indictment in North
Dakota, He Wenxiang had suddenly become director of three other
seemingly random and unrelated companies, including one, Qingdao Gold
Crown Fashion Jewelry Co., Limited, based in Zhang Jian's hometown.
These were just the clues that could be turned up by searching online
--- what else could be learned in China?

Image

Rod Rosenstein, then the deputy attorney general, announcing the
indictments of Zhang Jian on Oct. 17, 2017. Zhang was one of the first
Chinese nationals to be charged as a fentanyl kingpin in the United
States.Credit...Alex Wong/Getty Images

\textbf{Zaron Bio-tech's officially} registered address is on the 20th
floor of a run-down business tower in Wan Chai, a congested commercial
area on Hong Kong Island's north shore. When Zaron filed its last annual
return, in April 2016, the office belonged to Hongkong Keyray
Accountants. Visiting the office earlier this year, I saw a new sign,
for Rich Moral CPA Limited. A young woman in faded jeans and a red
varsity jacket answered the door. I asked if she knew about a company
called Zaron Bio-tech that was registered at this address. She told me
that dozens of companies were registered there; in fact, a list of all
the companies used to be posted on the wall just inside the door so that
employees could keep them straight. The list was gone, she said, but
people still come looking for a different company every month or so. It
was easy cash for Rich Moral CPA, charging other companies to use their
address. Whatever those businesses were actually up to, the woman told
me, ``we don't want to know.'' Her boss promptly came to the door and
told me to leave.

It was a fitting introduction to the bizarre shadow world of Hong Kong
shell companies and the secretarial firms enlisted to conceal them.
Until March 2018, secretarial firms were totally unregulated. They
didn't have to comply with laws combating money laundering or the
financing of terrorism. Creating a corporate entity in Hong Kong could
be done in a day from anywhere in the world, and hiring a ``corporate
secretary'' was as simple as searching online and sending in a few
anonymous forms. It is a volume business; companies appear and disappear
all the time. Zaron was just another faceless name in the sea of papers.

I visited 13 addresses tied to Zaron Bio-tech or He Wenxiang, all of
which turned out to be secretarial or accounting firms. The addresses
seemed to multiply like Russian nesting dolls --- every address led to
five more, scattered across Hong Kong and corresponding to gleaming
office complexes, rundown multistory markets and rickety apartment
buildings alike. The rabbit hole had no end.

Eight floors below Rich Moral CPA Limited was another address used by
Zaron. A kind woman in professional attire named Ms. Wang answered the
door, invited me into the conference room and poured a glass of water.
Keyray Accountants, she said, had asked her firm to complete Zaron's
annual review on their behalf in 2016. Keyray's name was still on the
filings, but her firm had done the actual paperwork, which mostly
involved writing ``Nil'' across several pages. For that, the firm
received 500 Hong Kong dollars --- about \$64 --- half of what Keyray
received from Zaron. When Zaron suddenly went dark and stopped
responding, she didn't think too much of it --- it happened all the
time.

Every Zaron address held its own surreal scene. Behind one door, I met a
tiny woman in a red cardigan, hidden behind stacks of paper in an office
crammed to the ceiling with Star Wars figurines and anime dolls. On one
wall was a framed and mounted list of 72 company names; on the other, a
giant black-and-white photograph of Manhattan. There was classical music
playing, and the bookshelves were filled with volumes of Shakespeare,
Sherlock Holmes and the Haruki Murakami novel ``1Q84,'' as well as
bottles of Macallan and Glenfiddich scotch. ``I can't help you,'' the
woman said in a voice barely above a whisper. ``There is no information
here.''

Behind another door, two old men packaged dense cubes of dehydrated
bird's nest for sale to restaurants. Oh, yes, one of the men said, there
was a guy who rented the office some years ago, and used it to register
over a thousand companies. Actually, he had never been there at all ---
his company was in mainland China, but he kept the Hong Kong address
because the city is less regulated and better for business. He moved out
four or five years ago, the man said, but the landlord still got
thousands of pieces of unwanted mail and countless phone calls. While he
spoke, his colleague was weighing the bird's-nest cubes. When the weight
was too low, he cut a small slit in the plastic packaging, sprayed water
in the nest to increase the size and tried again. A lot of people come
to visit, the first man continued. Just recently there had been a police
official, a tax officer, a gangster and an old lady whose money had been
stolen by one of the companies registered at the address. All of them
left disappointed.

Sometimes the secretarial firms just dispensed with the pretense up
front. At the address of Hong Kong VFON Business Technology Co.,
Limited, the firm that had transferred ownership of a company called
iGlory Technology to He Wenxiang, Zaron's new director, there was simply
a locked door with a blue screen displaying an endless scroll of company
names: Sunpzone Lighting Co., HK Williamsburg Musical Instrument Group,
Germany Imonce Fluid Technology, Great Textiles Ltd. and on and on. I
found 18,668 companies at a single address, but most of the secretarial
firms seemed to have no idea how many companies they actually
represented.

Image

On the left, an auto-parts store; on the right, an office building. Both
Qingdao addresses were claimed by Zaron Bio-tech.Credit...Roy Sen

It seems certain there was never any Zaron Bio-tech; no factories in
Vietnam, Thailand or Singapore; no employees; no facilities across
southern China. The search for Zhang himself was equally fruitless. The
office in Shanghai is occupied by another tenant; another address listed
by the company is a used-car-parts store. Locals said the storefront
actually belonged to an elderly couple who had been in business for
decades. Zhang was, as Buemi had discovered, an opportunist, not a
high-powered executive.

In early 2017, several months after his arrest and release, a second
company registered to Zhang, called Qingdao Zunyun, received Chinese
government approval to expand its operations. But that company's listed
address, in a dilapidated commercial tower in Qingdao, is now empty.
Zhang first rented the space, the landlord said, in 2015 or 2016. The
office, on the 10th floor, was the smallest in the whole building, just
10 square meters. Zhang paid up front in cash, 5,000 RMB (\$700) for six
months. He stayed one year. The landlord didn't ask for an ID card, and
there was no contract or other documentation. ``I just want the money,''
the landlord said.

Zhang arrived alone every morning, usually before 7 o'clock, went up to
his office and locked the door. The building manager didn't know what
Zhang's business was or what he did all day behind the locked door. He
only remembered Zhang because he was one of the few people who would say
hello when he passed by. He was polite and unremarkable, the manager
remembered. He seemed kind.

\textbf{I managed to} meet with one person linked to Zaron: the
company's new titular owner, He Wenxiang. He is a small, bubbly
middle-aged man with a fleshy face, short limbs, and deeply-etched laugh
lines. When my Chinese colleague, Susie Wu, and I arrived at He
Wenxiang's three-room office in Science City, a government-sponsored
business complex in Guangzhou, he greeted us warmly, invited us into a
small meeting room and prepared tea on an elaborate hot plate built into
the table. We discussed the United States-China trade war and its impact
on his freight-forwarding business. Almost everything he said started or
ended with a laugh, and as we spoke, he pressed his hands into the black
leather couch where he sat, swaying backward and forward and swinging
his knees together like an energetic child. We didn't initially mention
Zaron.

The story he told of his life was the story of modern China. He Wenxiang
was born in an impoverished village in southern China in 1976, the year
of Mao Zedong's death. He went to middle school, followed by technical
school, and then took a job at a supply-and-marketing company in his
home province. On May 18, 2001 --- he remembers the date precisely ---
he moved to Guangzhou. The city was booming thanks to China's economic
reform, and migrants like him were flooding in, eager for work and
wealth. In his hometown he had made 400 RMB (about \$58 today) per
month. As soon as he arrived in Guangzhou, he was making double that. In
May 2005, four years after leaving his village, He opened his own
company, Guangzhou Changda International Freight Co., Limited. Today his
company has a staff of 10, annual returns above 10 million RMB (nearly
\$1.5 million) and business relationships all around the world, shipping
anything and everything. He tried expanding his business portfolio
several times over the years, but every time he did, he told me, he
seemed to end up in trouble. In 2015 a trade company in Beijing
defrauded him, and he ended up in a lawsuit; in 2016 he invested 50,000
RMB in a company selling car speakers, but the company took his money
and disappeared; that same year, he invested in a medicine company that
said it had found a cure for burns, and after investing 100,000 RMB, He
Wenxiang was ripped off again, and ended up in another lawsuit.

I brought the documents detailing He Wenxiang's directorship of Zaron
and three other Hong Kong-registered companies. When I handed him the
forms and asked about the companies, he looked shocked --- he grabbed
the papers and began flipping wildly, searching out his signature on
every page. I examined his expression closely, hoping to pick up on
whether his dismay was genuine.

He ran into another room and returned with a thick folder of papers and
documents relating to his real company. He said that he had registered a
shell company in Hong Kong in 2012, solely for the purpose of moving
money more easily into and out of the mainland. In 2017, he had seen an
advertisement in a QQ group --- a Chinese messaging and social media
service --- asking if anyone had a company in Hong Kong that they were
willing to sell. He Wenxiang's shell company was no longer of use to
him, so he sold it for 10,000 RMB. He never asked why the person wanted
a company, or couldn't just open one themselves.

Maybe, he suggested, someone had used his information and signature from
that transaction to steal his identity and make him as a scapegoat for
Zaron and the three other firms he now controlled. He took out his
phone, called his accounting firm, and asked if they knew that he was
director of so many companies.

``I cannot help you,'' the woman on the phone said. ``You don't even
know how many companies you own? Take care of it yourself.''

No one else had asked him about this, He Wenxiang said --- not the
police, not the M.P.S. Now that he knew all these companies were
registered to his home address, he was thinking of moving. By the end of
the meeting, I had no idea what to believe. He was either an
extraordinary liar or extraordinarily unlucky. Both seemed equally
plausible. (Further attempts to reach He and confirm his story were
unsuccessful.)

The last person I met with in Hong Kong was Kenneth Leung, a tax adviser
and member of the Hong Kong Legislative Council who has helped lead
reforms of the city's financial shadow world. He said that as of 2018,
secretarial firms must check their clients' identities in order to
comply with standard anti-money laundering and antiterrorism practices,
but that's about it. Anyone can call themselves an accountant or
secretarial firm --- there are no professional regulations. Hiding
behind layers and layers of these firms, he said, ``is quite standard
practice.''

When I showed him the documents on He Wenxiang and Zaron, and asked what
he made of them, he said there was no way to figure out the truth of He
Wenxiang's story. I either believed him or I didn't.

I asked what he would do. Leung thought for a moment, then replied,
laughing, ``Don't trust anyone.''

\textbf{The investigation that} began the night of Bailey Henke's death
is continuing. There have been 32 individuals indicted and more than 70
arrests, and seizures of over 1,000 kilograms of drugs as well as assets
and currency worth \$1.5 million. Zhang Jian was found to be the source
of narcotics for eight other federal investigations across the country.
``It's good to see Bailey's name out there,'' Laura Henke told me. ``His
death made a difference.''

On May 1, China implemented class scheduling of all fentanyl analogues,
criminalizing the unregulated production of any chemical with a
structure similar to fentanyl's. Still, the Chinese government continues
to reject the idea that China has played any role in the fentanyl
crisis. Chinese officials have claimed that
\href{http://www.globaltimes.cn/content/1138921.shtml}{the United
States' problem is ``liberalism,''} because ``some people link drug
consumption with freedom, individuality and liberation.'' They have
insisted that ``not one gram of fentanyl has been found to flow through
illegal channels, and it is even less likely to enter the U.S.''; and,
``so far China has not received any samples of fentanyl-related
substances coming from the U.S. that have been confirmed as coming from
China.'' At least one of these statements was made by an official who
has privately told a United States diplomat of his desire to help stanch
the epidemic.

These regulations, however strictly enforced, are unlikely to halt
Chinese fentanyl exports. Logan Pauley and Michael Lohmuller of C4ADS, a
Washington research firm focused on transnational security issues, have
already found Chinese companies adapting to the ban by trafficking
uncontrolled fentanyl precursors to drug cartels in Mexico and other
countries for final synthesis and delivery. And India, the world's other
pharmaceutical and chemical giant, is a looming threat to step in and
fill any gaps created by China's new regulations. Last September,
intelligence officers in Indore, in west-central India, seized nearly 11
kilograms of fentanyl. A local businessman had been asked by contacts in
Mexico to make fentanyl; they gave him the formula and told him how to
make it.

Yet China remains the center of the global fentanyl economy. On Alibaba,
Facebook and other sites, Chinese companies openly advertise the drug
and its precursors and analogues, as well as their ability to deliver
orders while eluding United States customs. Some of these companies,
unconcerned with scrutiny or confident of official indifference, operate
from the same addresses and use the same phone numbers that Zaron
Bio-tech once used.

At the same time, Chinese fentanyl has taken on a newfound urgency and
notoriety as a centerpiece of the Trump administration's trade
negotiations. Robert Lighthizer, the United States trade representative,
has said that he hopes to include Chinese commitments to halt fentanyl
exports in a final trade agreement --- tariff rollbacks in exchange for
a fentanyl crackdown. For Buemi, it was a long-overdue correction. ``At
a strategic level, as an agent we can only do so much,'' he told me.
``We need our leadership talking to their leadership to stop these
chemicals from being manufactured.'' The Trump administration's
enthusiasm for the issue was welcome, even if, he said, it's ``obviously
a little too late.''

A few days after the summit meeting between President Trump and
President Xi Jinping in Buenos Aires last December, at which
\href{https://www.nytimes3xbfgragh.onion/2018/12/01/world/trump-xi-g20-merkel.html}{Trump
brought up fentanyl} as one of the first items of discussion, Global
Times, a hard-line nationalist tabloid operated by the Central Committee
of the Communist Party, published an editorial explaining the fentanyl
crisis to Chinese readers. In its blunt and jingoistic tone, the article
veered unusually close to the truth. It began by describing how
America's bottomless demand for fentanyl came about. ``If we analyze the
`public health emergency' in the United States, it is easy to find that
the root cause is the abuse of opioids,'' the editorial explained.
Doctors gave out too many painkillers, and Americans became addicted.
Now they had started using fentanyl, and it was killing them. Drug abuse
was an old problem for Americans, the article said; now they had just
encountered ``a new devil.''

The killing stops when the buying stops, the article noted --- there is
no market without demand. How much fentanyl has flooded into the United
States from China, into towns like Grand Forks and West Palm Beach,
``only Heaven knows.''

\textbf{Kyra Coffman, one} of Bailey's best friends, did not follow the
string of arrests and trials that resulted from his death. ``It didn't
change much,'' she told me. ``At the end of the day, he was still
gone.'' During the last months of Bailey's life, Coffman pleaded with
him to quit. He promised he would, again and again, only to relapse a
few days later. Coffman knew how much Bailey loved his mom, and how
close the two were. Coffman thought about telling her, hoping it might
force Bailey to get clean. ``He didn't want to disappoint her,'' Coffman
told me. ``It's why any kid hides things from their parents.''

The last person to see Bailey alive was his friend Tanner Gerszewski. It
was Bailey who first introduced him to fentanyl. Bailey had already been
using for a few months, but Gerszewski knew what fentanyl did to people.
Then one Friday night in fall 2014, he was at home relaxing after a bad
day at work, and Bailey brought some over. Gerszewski agreed to try it.
He put some of the powder on foil, lit it, and breathed in the smoke.
``Before I even blew out,'' Gerszewski told me, ``I said, `How much more
of this you got?' '' He spent every dollar he had on it.

Around 9 p.m. on Jan. 2, Bailey's friend and roommate Kain Schwandt
drove him over to Gerszewski's house, then left to run errands. Bailey
was still high from smoking fentanyl earlier in the day at Ryan Jensen's
house, but the two friends did more fentanyl together before settling
onto the couch to play an Ultimate Fighting video game on Xbox. After a
few minutes, Bailey's character on the screen stopped moving. Gerszewski
looked over and saw Bailey was nodding off, so Gerszewski nudged him.
``I figured he was tired,'' Gerszewski said. ``We'd both seen each other
so much worse.''

Bailey came to and kept playing. But his character went still again a
couple of minutes later. The third time Gerszewski nudged him, Bailey's
head fell back, and Gerszewski saw that his friend was turning white.

Death from a fentanyl overdose is death from suffocation. Overwhelmed by
opioids, the nervous system signals for blood pressure and respiration
to drop. The pupils constrict; blood oxygen levels plummet; the heart
goes into arrhythmia. Within a few minutes, cellular death begins, and
the patient is lost.

It was a scene from a nightmare --- Gerszewski too high to help but
sober enough to realize what was happening. He could barely talk, barely
use a phone. He called Schwandt. Schwandt could hear Gerszewski
panicking, breathing heavily and running around the apartment. It
sounded as if Gerszewski was tearing the place apart. Schwandt told him
to wake Bailey up. A minute later, Gerszewski called again. Schwandt
went to the apartment, shouted to call 911 and began CPR.

After Bailey died, he was found wearing a hemp necklace that Coffman had
made. She had stopped by his place that afternoon to find Bailey and
Schwandt laughing and joking, cracking each other up as they cleaned the
house. Before she left, Bailey told her that he loved her.

``Doing drugs didn't make him less of a good person,'' Coffman said.
``He just got a little lost.''

\begin{center}\rule{0.5\linewidth}{\linethickness}\end{center}

\textbf{Alex W. Palmer} is a writer based in Washington.
\href{https://www.nytimes3xbfgragh.onion/2018/04/03/magazine/the-case-of-hong-kongs-missing-booksellers.html}{He
last wrote for the magazine about the case of Hong Kong's missing
booksellers.} Additional reporting by \textbf{Susie Wu}.

Opening image: Photograph by Sarah Anne Ward for The New York Times.
Source photographs from Getty Images; Roy Sen; United States Drug
Enforcement Administration, via Associated Press; U.S. Attorney's
Office, Utah, via Associated Press; Drew Angerer/Getty Images; David L.
Ryan/The Boston Globe, via Getty Images; Ty Wright/The Washington Post,
via Getty Images; Laura Henke.

Advertisement

\protect\hyperlink{after-bottom}{Continue reading the main story}

\hypertarget{site-index}{%
\subsection{Site Index}\label{site-index}}

\hypertarget{site-information-navigation}{%
\subsection{Site Information
Navigation}\label{site-information-navigation}}

\begin{itemize}
\tightlist
\item
  \href{https://help.nytimes3xbfgragh.onion/hc/en-us/articles/115014792127-Copyright-notice}{©~2020~The
  New York Times Company}
\end{itemize}

\begin{itemize}
\tightlist
\item
  \href{https://www.nytco.com/}{NYTCo}
\item
  \href{https://help.nytimes3xbfgragh.onion/hc/en-us/articles/115015385887-Contact-Us}{Contact
  Us}
\item
  \href{https://www.nytco.com/careers/}{Work with us}
\item
  \href{https://nytmediakit.com/}{Advertise}
\item
  \href{http://www.tbrandstudio.com/}{T Brand Studio}
\item
  \href{https://www.nytimes3xbfgragh.onion/privacy/cookie-policy\#how-do-i-manage-trackers}{Your
  Ad Choices}
\item
  \href{https://www.nytimes3xbfgragh.onion/privacy}{Privacy}
\item
  \href{https://help.nytimes3xbfgragh.onion/hc/en-us/articles/115014893428-Terms-of-service}{Terms
  of Service}
\item
  \href{https://help.nytimes3xbfgragh.onion/hc/en-us/articles/115014893968-Terms-of-sale}{Terms
  of Sale}
\item
  \href{https://spiderbites.nytimes3xbfgragh.onion}{Site Map}
\item
  \href{https://help.nytimes3xbfgragh.onion/hc/en-us}{Help}
\item
  \href{https://www.nytimes3xbfgragh.onion/subscription?campaignId=37WXW}{Subscriptions}
\end{itemize}
