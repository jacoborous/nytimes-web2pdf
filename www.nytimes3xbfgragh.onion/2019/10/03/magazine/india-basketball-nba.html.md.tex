Sections

SEARCH

\protect\hyperlink{site-content}{Skip to
content}\protect\hyperlink{site-index}{Skip to site index}

Can the N.B.A. Find a Basketball Superstar in India?

\begin{itemize}
\item
\item
\item
\item
\item
\item
\end{itemize}

\includegraphics{https://static01.graylady3jvrrxbe.onion/images/2019/10/06/magazine/06mag-indiannba/06mag-indiannba-articleLarge.jpg?quality=75\&auto=webp\&disable=upscale}

Feature

\hypertarget{can-the-nba-find-a-basketball-superstar-in-india}{%
\section{Can the N.B.A. Find a Basketball Superstar in
India?}\label{can-the-nba-find-a-basketball-superstar-in-india}}

The league thinks India could be its next billion-dollar market. But
first it needs to convince young Indians to fall in love with the game.

Ankit Hooda (second from left), 20, and Navdeep Grewal (far right), 19,
members of the Ramjas College basketball team, training on a Saturday
morning.Credit...George Georgiou for The New York Times

Supported by

\protect\hyperlink{after-sponsor}{Continue reading the main story}

By Reid Forgrave

\begin{itemize}
\item
  Oct. 3, 2019
\item
  \begin{itemize}
  \item
  \item
  \item
  \item
  \item
  \item
  \end{itemize}
\end{itemize}

The soupy morning air in one of the world's most polluted cities had all
but cleared by the time Saturday morning's outdoor practice ended for
the men's basketball team at Ramjas College. If players had to wait out
New Delhi's smog, they would never practice, because there are few
indoor courts in India. Ankit Hooda, a 20-year-old student who is one of
the team's best players, lingered on the court and laughed as his friend
showed off a right ring finger that was dripping blood. Moments before,
his teammate, Navdeep Grewal, had sliced the finger on the metal rim
while dunking the ball. For Grewal, who is 19, playing this sport is a
joy. Grewal is the son of a military man, firmly ensconced in India's
middle class. Hooda, however, is not from the same class. To him,
basketball means much more.

Hooda is shorter than Grewal but a better shooter. The wallpaper on his
phone shows J.J. Redick, the New Orleans Pelicans' guard, one of the
best shooters in the N.B.A. Hooda is from a three-room house in Haryana,
a state in northern India. His father grows wheat and rice on a 15-acre
farm, where Hooda grew up a farmhand, feeding and milking cows, hitching
the bull to the plow to sow seeds and spread fertilizer. Farming was not
the life he wanted, though. By Indian standards, his family was not
poor, but he felt stuck, a city boy in a country life. The first time he
heard about basketball, when he was 14, it sounded like a way out.

Through his teens, Hooda woke up at 5 in the morning to train: skipping
rope, doing push-ups, running stairs. After school, he biked to the only
court around. He has found inspiration in the few Indian players who
have come close to reaching the N.B.A. ---
\href{https://www.nytimes3xbfgragh.onion/2015/07/30/world/asia/satnam-singh-india-nba-basketball.html}{Satnam
Singh}, the first Indian-born player selected in the N.B.A. draft, in
2015; \href{https://gleague.nba.com/player/amjyot-singh/}{Amjyot Singh},
who has played in the N.B.A.'s minor-league system; and
\href{https://in.nba.com/news/princepal-singh-latest-punjabi-star-make-basketball-headlines-jr-nba-academy-india/1pfhe26tjegbl1tc2p7udtphhx}{Princepal
Singh}, an 18-year-old near-seven-footer who is now considered India's
top N.B.A. prospect --- and he knows that if a player from India makes
it to the N.B.A., his sport would become a national phenomenon. But he
is realistic. Above all, he wants to ensure financial security and a
more suitable path for himself: a government job as a physics teacher.

``I wouldn't have gotten admission to this college if I was not playing
basketball,'' Hooda told me in Hindi through an interpreter. Many Indian
colleges have athlete quotas they have to fill, as part of the national
government's effort to improve the country's sports performances; that's
how Hooda got in. ``It's not that basketball will be a means to my life.
But basketball will remain a passion forever. It's a way of life for me.
There are so many courts being put up in various colleges and schools.
Once you put up the courts, people will say: `What is that he's playing?
Is that basketball?' ''

\includegraphics{https://static01.graylady3jvrrxbe.onion/images/2019/10/06/magazine/06mag-indiannba-02/06mag-indiannba-02-articleLarge.jpg?quality=75\&auto=webp\&disable=upscale}

The car-horn-heavy soundtrack of urban India played in the background.
Nearby, a street vendor carried a sizzling skillet of kebabs on his head
and a group of 50 students played cricket. Basketball has not yet fully
captured the Indian imagination --- Hooda guessed that only a tiny
percentage of the students in his college, which is part of the
University of Delhi system, are avid basketball players --- but it was
noteworthy that, after the men's team finished, a group of 10 or so
students walked onto the court and started shooting. One wore a Jordan
Brand Nike sweatshirt. ``Basketball is as American as you can get,''
Hooda told me. ``It's like when you go to America and ask to find
cricketers. That's how it is trying to find basketballers here.''

That is changing. While cricket is still the country's dominant sport
--- the 2019 International Cricket Council World Cup attracted 545
million Indian viewers, far exceeding the entire population of the
United States --- plenty see basketball as a faster-paced alternative,
fit for a country that's simultaneously urbanizing and getting younger.
The N.B.A. has invested accordingly: Its projects include a youth
basketball program that it claims has reached 10 million Indian
children; an N.B.A.-run basketball academy located outside New Delhi
that houses, educates and trains two dozen teenagers full time; and the
hosting of the N.B.A.'s first preseason games in India, when the
Sacramento Kings play the Indiana Pacers in Mumbai this month. At a
September rally in Houston for India's prime minister, Narendra Modi,
President Trump touted the upcoming games and asked,
``\href{https://www.youtube.com/watch?v=iXwnrB0OkrU}{Am I invited}?''
Afterward, the
\href{https://www.hindustantimes.com/india-news/am-i-invited-mr-prime-minister-trump-on-india-s-first-nba-game/story-15hW9MfyEAwyO1FtrpPT4M.html}{Hindustan
Times} referred to the N.B.A. as ``the latest U.S. export to India.''

Scenes like the one at Ramjas College are becoming more common
throughout India. In a farming village newly encroached upon by New
Delhi's sprawl, a teenage boy wakes up in his family's one-room
apartment before dawn and turns the television on to an N.B.A. game
broadcast in Hindi. In an upper-class development down the road,
promising teenage prospects train in that N.B.A.-funded academy, its
avowed goal to produce the first Indian N.B.A. player. Some 1,300 miles
south, in Bangalore, a teenage girl prepares to leave for the United
States as the second Indian woman to secure a Division I college
basketball scholarship. Down the road, hundreds of physical-education
teachers meet in an outdoor stadium and receive instructions in a
Basketball 101 class: ``The team with the ball is on offense. They try
to score points by shooting the ball through the hoop.'' In Mumbai,
thousands have gathered for years at outdoor courts on rainy summer
evenings for the Monsoon League, where the games go late into the night
in a partylike setting.

``I call India a raucous democracy --- its colors, its smells, its
spices,''
\href{https://www.nytimes3xbfgragh.onion/2019/10/03/sports/vivek-ranadive-sacramento-kings.html}{Vivek
Ranadivé, chairman of the Kings} and the first person of Indian descent
to own an N.B.A. franchise, told me. ``If there was a country that was a
symbol of basketball, to me India's that country. Basketball is color.
It's a spectacle. It's nonstop action. It's celebrity. It's music.''

Image

The N.B.A. Basketball School at Ramagya School, Noida.Credit...George
Georgiou for The New York Times

It is, Ranadivé believes, the perfect sport for India. ``I won't be
greedy and say we're going to replace cricket as the national pastime,''
he says. ``But I'll be greedy and say we'll become a strong No. 2.''

\emph{{[}}\href{https://www.nytimes3xbfgragh.onion/2019/10/03/sports/vivek-ranadive-sacramento-kings.html}{\emph{Read
a profile of Vivek Ranadivé}}\emph{.{]}}

\textbf{When Adam Silver,} the commissioner of the N.B.A., started
working full time for the league, he was a 30-year-old lawyer whose
first assignment was to write a position paper about the Dream Team. It
was 1992, and the American men's Olympic basketball team, whose stars
included Michael Jordan, Magic Johnson and Larry Bird, was about to
become a sensation at the Summer Games in Barcelona. At the time, the
N.B.A. was lukewarm about allowing its American stars to compete in the
Olympics, but Silver argued it was best not just for the N.B.A. but also
for the sport. As Silver ascended to the top of the N.B.A.'s hierarchy,
he noticed that two phenomena tended to account for how young
international players became interested in basketball. One was the Dream
Team. The other was Michael Jordan videos. The lessons he took from this
observation have become accepted wisdom: Expanding the N.B.A.'s global
reach is as much about selling individual stars as about selling the
sport itself.

In the nearly three decades since the Dream Team won the gold medal, the
N.B.A. has become an increasingly international league, with roughly a
quarter of its players coming from outside the United States, up from
less than 5 percent at around the time of the Barcelona Olympics. Last
season underscored just how global the N.B.A. has become. The Toronto
Raptors were the first team from outside the United States to win a
championship, and international players virtually swept the
end-of-season awards. The M.V.P. was Giannis Antetokounmpo, born in
Greece to Nigerian immigrants. The rookie of the year was Luka Doncic,
who comes from Slovenia and whose game blossomed in Spain. The defensive
player of the year was Rudy Gobert, from France, and the most improved
player was Pascal Siakam, from Cameroon. That same month, Rui Hachimura
became the first Japanese player selected in the first round of the
N.B.A. draft.

Benchmarks like these show how far the league has come from the days
when Silver's predecessor as commissioner, David Stern, shipped video
cassettes to Italy in the 1980s so recorded games could be broadcast a
week later. The international push has been decades in the making. In
1987, at the McDonald's Open in Milwaukee, the Soviet national team
played the Milwaukee Bucks and a professional Italian team. The next
year, the Atlanta Hawks, then owned by Ted Turner, toured the Soviet
Union. The first regular-season games to take place abroad followed in
1990, when the Phoenix Suns and Utah Jazz played each other twice in
Tokyo. The league now regularly puts on in-season games in cities like
London and Mexico City, and preseason games worldwide --- in China,
Spain, the Philippines, Brazil. The N.B.A.'s growth abroad is especially
striking when contrasted with that of the N.F.L., an older league that,
despite its own efforts to appeal to international audiences, still has
not gained a lasting foothold overseas. Major League Baseball has found
audiences in Latin America and parts of Asia, but its spread has not
been as swift and broad.

Image

Farooque Shaikh (far left, in back) coaching at the Mastan Y.M.C.A. in
the Nagpada district of Mumbai.Credit...George Georgiou for The New York
Times

Global expansion is one of Silver's key goals. He took his first big
foreign trip as N.B.A. commissioner shortly after getting the job in
2014 --- to India, with Ranadivé. Silver wants the portion of the
league's revenue that comes from international sources to rise from its
current 20 percent to 50 percent within 20 years. For that to happen,
the league must cultivate more global stars. As part of the effort, it
has so far established seven basketball training centers around the
world. The N.B.A. Academy India opened in 2017; the other academies are
in Australia, Mexico, Senegal and China, which has three. The sport's
explosive growth in China stokes --- and offers guidance for --- the
N.B.A.'s ambitions in India. Basketball was introduced to China in the
1890s. The country fielded an Olympic team in 1936, and in the 1980s
Stern met with officials from its state-run television network to
broadcast games there. By 1994, CCTV was broadcasting every game of the
N.B.A. Finals. But the sport didn't truly take off in China until 2002,
when the Houston Rockets used the No. 1 overall pick to draft Yao Ming,
a 7-foot-6 center from Shanghai. More than 200 million Chinese watched
Yao's first game. Basketball is now one of the most popular sports in
the country, and the N.B.A. says its operations there are valued at more
than \$4 billion.

\href{https://www.nytimes3xbfgragh.onion/2019/09/02/magazine/wnba-atlanta-dream.html}{\emph{{[}Read
about the struggles of the W.N.B.A.{]}}}

The N.B.A.'s recent focus on India seeks to bolster the basketball
infrastructure there, from grass roots to the highest level. With
India's population poised to surpass China's, its 1.3 billion people
represent a huge potential market for the N.B.A. But for an enterprise
that surpassed \$8 billion in revenue in the 2017-18 season while
basking in the glow of its association with progressive politics, the
N.B.A.'s India venture presents a challenge. For all its economic growth
and nouveau riches --- more billionaires reside in India than anywhere
other than the U.S., China and Germany, according to Forbes --- the
country is still impoverished. Its poverty rate, upward of 21 percent,
is one of the world's highest. The per capita gross domestic product in
2018 ranked 142nd out of 187 countries, according to the International
Monetary Fund, and the per capita income is roughly \$150 a month. But
the N.B.A. isn't in India on a charity mission. It wants to find the
first Indian basketball star.

There's something of a paradox in this endeavor to turn India into the
N.B.A.'s next frontier: The N.B.A. needs that single, Yao Ming-like
superstar to boost the sport's popularity, yet a widespread Indian
basketball culture is probably needed to create that superstar. The
N.B.A. is trying. India's Jr. N.B.A. program --- the league's
international youth program for children from 6 to 14 --- is by far the
world's largest. The N.B.A. has trained more than 10,000 Indian
physical-education teachers and staged more than 1,500 grass-roots
events in 34 Indian cities. N.B.A. All-Stars like Kevin Durant have
visited. Four Indian television networks broadcast more than 300 live
N.B.A. games a season, including 85 with Hindi commentary, and during
the 2017-18 season --- the first in which the N.B.A. Finals were
broadcast in both English and Hindi --- the games captured more than 91
million viewers. As the N.B.A. scours India for a unicorn-like breakout
player, it's also doing something else: spreading the N.B.A.'s vision of
itself as the embodiment of select American virtues. It's an
evangelistic impulse that has been a part of the sport since its
creation.

\textbf{High above Midtown} Manhattan, in the 52-story Olympic Tower,
7,000 miles from where the Ramjas College team practices on a dusty
outdoor court, Silver suddenly materialized in a conference room
overlooking St. Patrick's Cathedral one afternoon last December. His
white dress shirt partly untucked, he appeared quite caffeinated,
speeding through sentences like Russell Westbrook on a fast break.
Silver swept through an array of topics --- the phenom Zion Williamson,
then playing at Silver's alma mater, Duke University, on his way to
becoming the No. 1 pick in the draft; the success of small-market teams
in San Antonio and Oklahoma City --- before he got to the N.B.A.'s
ambitions in India and how the league sells itself as a global brand.

Image

Pranav Prince (front, center), 15, a 6-foot-6 player whom coaches
consider one of the most promising young Indian
prospects.Credit...George Georgiou for The New York Times

Basketball has already, Silver believes, become one of two truly global
sports, alongside soccer: ``I don't think it's an accident they are both
round balls,'' he said. ``One you use your feet, one you use your
hands.'' But there's more to its appeal than simplicity and minimal
equipment needs. Even though the N.B.A.'s international push was never
intended to serve as an extension of American soft power, it plays ---
and embraces --- that role. This comes up often when Silver travels to
places where the N.B.A. is seeking to expand.

Government officials abroad and N.B.A. business partners, Silver told
me, ``turn the conversation back to American values and to the history
of the league, the principles that underlie this league --- which to me
are often the things that we've been in the news for, in essence players
speaking out on important issues to them, including political issues.''
His subtext: The N.B.A. celebrates its identity as the most outwardly
politically progressive of the American sports leagues.

Basketball has been linked to spreading values and influence ever since
its beginning. James Naismith, who invented the game in 1891, was a
follower of the ``muscular Christianity'' movement, which held that
athleticism and discipline could lead not just to physical development
of the body but also to moral development of the soul. For him,
basketball --- not an explicitly Christian game, but one that eschewed
violence and encouraged teamwork and character development --- was an
evangelical tool. Within a few years of Naismith's inventing the game at
the International Y.M.C.A. Training School in Springfield, Mass.,
missionaries introduced it to China. Missionaries did the same in
France, Turkey and Brazil.

Part of the N.B.A.'s delicate balance overseas is the league's alignment
with progressive politics. This means more than vague memes of
flag-waving and apple pie. It includes LeBron James publicly condemning
police violence against black men; Steve Kerr, the Golden State Warriors
coach, advocating gun-control legislation; Gregg Popovich, the San
Antonio Spurs coach, criticizing President Trump; and Jason Collins
becoming the first openly gay athlete to play in one of the four major
American sports leagues --- and it also happens to be considered the
league in which the players and their union wield the most power in
relation to the teams. As far as Silver is concerned, as long as the
sport itself continues to be more important than the N.B.A. and its
players' progressive bent --- which might not be as eagerly received in
places like authoritarian China --- that's fine.

``These are human issues, and they transcend politics,'' Silver said.
``When people talk about LeBron's response to `Shut up and dribble' ''
--- the public reprimand of James by Fox News's Laura Ingraham after he
spoke out on social issues --- ``I say, `Good for LeBron.' But I also
point out that Bill Russell decades earlier was there at the Lincoln
Memorial during the `I Have a Dream' speech that Martin Luther King Jr.
gave. This has been part of the DNA of this league since its earliest
days.''

The N.B.A.'s touting its progressive bona fides can ring hollow at
times: For years it tolerated the racism of the Los Angeles Clippers'
owner, Donald Sterling, for example, before Silver, amid an outcry by
players and the public, pushed him out of the league. But the league has
leaned into its players' politics during the Trump era. There's an
obvious counterexample, one that Silver hesitates to acknowledge: the
N.F.L. The controversy that resulted from Colin Kaepernick's kneeling
during the national anthem in 2016 has been a political flash point ever
since. In his lawsuit against the N.F.L., recently settled, Kaepernick
claimed that the league had blackballed him. The N.B.A. has a
longstanding rule that players must stand for the anthem, but that was
mostly ignored by the media during the Kaepernick saga, even as the
N.F.L. faced intense criticism. Perhaps the politics of many of the
league's players provide cover for its own anthem policy. But the idea
of America evoked by the N.F.L. is different from its N.B.A. equivalent.
Whether this is partly because the N.B.A.'s gentler vision is more
palatable to an international community, or partly because the N.F.L.'s
endemic violence simply doesn't sell abroad, the N.F.L. has never taken
off internationally the way the N.B.A. has.

As I got ready to leave the Manhattan conference room, Silver stopped
me. ``I want to show you one thing before you leave,'' he said. He
dashed into his office, then returned with a marked-up recent copy of
The New York Times. ``I actually cut it out to send to my colleagues,''
Silver said, thrusting the folded-up newspaper at me. The front section
included a profile of the president of the Democratic Republic of Congo,
Joseph Kabila, who was stepping down after nearly two war-torn (and
corruption-tinged) decades of rule, in one of the poorest countries on
earth. Silver had marked the second paragraph in red marker: ``He enjoys
watching the N.B.A. and gets on with little sleep.''

\href{https://www.nytimes3xbfgragh.onion/interactive/2018/02/01/magazine/winter-olympics-luge-india.html}{\emph{{[}The
Lonely Mission of India's sole luger.{]}}}

\textbf{When sports are} exported, some things are lost in translation,
others gained. A decade ago, the N.B.A. hired Troy Justice, then the
athletic director at Cincinnati Christian University, as its first
employee in India. Justice moved to Mumbai and saw what he considered a
distinctly Indian style of basketball. The pace was frenetic almost to
the point of being out of control. Coaches taught players to grab a
defensive rebound, cup the ball against their arm, take a cricket hop
and toss the ball over the top for a teammate to chase it down. ``Hurry,
hurry!'' coaches would shout.

Image

Practice at Dribble Academy, a program created to teach neighborhood
kids basketball in the village of Gejha, in Noida.Credit...George
Georgiou for The New York Times

``Everything in India moves at a fast pace,'' Justice says. ``That's
their style of basketball --- find an opening and move forward.'' His
perspective evokes the national stereotypes that are said to be
reflected in the way some countries play soccer: The Germans are
mechanical, precise and highly coached; the Brazilians value flair and
freedom of expression; the Spaniards prefer intricate passing and
elegance. Indian basketball was a messy and frantic beauty. Justice
wanted to refine the game without erasing its essence.

The most refined version of Indian basketball could be found earlier
this year on the outskirts of New Delhi, past an elegant resort with a
Greg Norman-designed golf course and through the guarded gates of the
Jaypee Integrated Sports Complex. Inside the gym, a player touched a
coach's feet as a sign of respect. Practice resembled a developmental
session for collegebound American prospects. A group of players, each
standing nearly seven feet tall, practiced post-up moves. Guards worked
on transition offense, then ran off screens for catch-and-shoot
three-pointers. ``For India,'' said Marc Pullés, a Spaniard in charge of
basketball operations for N.B.A. India, ``this is like our Dream Team.''

On the third floor were bunks for 24 teenage boys. ``When they
interviewed me, they said, `You gotta get us an N.B.A. player in five
years,' '' said Scott Flemming, a former collegiate coach who now leads
N.B.A. Academy India. They were joking --- kind of. Producing an N.B.A.
player is ultimately how Flemming will be judged. He seemed to think he
was up to the job when he self-deprecatingly recalled an episode from
his days coaching for Mount Vernon Nazarene University, a small private
school in Ohio. Flemming came back from a tournament and told
colleagues, ``I saw a skinny six-four kid who looks pretty good'' --- a
14-year-old named LeBron James.

Flemming thinks Pranav Prince, who is 15, looks pretty good, too. He's
6-foot-6, with long, thin limbs suited for basketball. He's taking his
naturally smooth game --- refined moves when he dribbles past defenders,
a silky, high-arcing jump shot, surprising toughness --- and learning to
play point guard. Flemming sees him as a Division I prospect. The N.B.A.
has relationships with American college programs and sends Indian
prospects to American basketball camps and tournaments where a college
coach might take notice. For Prince, a spike in development, or a growth
spurt, could push him to a basketball future even beyond a Division I
program.

Image

Children watching and playing basketball at the Mastan Y.M.C.A. in the
Nagpada district of Mumbai.Credit...George Georgiou for The New York
Times

After practice, Prince --- wide-eyed, seemingly still in shock that he
had made it this far --- recounted his journey. He spoke in English, as
most at the program do. Prince comes from a small village, where his
father drove a cab and his mother was a police officer. For several
years he stayed with his grandmother; his parents had moved to a bigger
city in order to get his sister treatment for blood cancer. When his
grandmother learned Prince's friends were smoking cigarettes, she sent
him to live with his parents.

By the time Prince was 10, his father decided to channel his son's
energy into sports. They watched N.B.A. games together, and his father
told him, ``I need you to play in this league.'' They woke up at 5 a.m.
for the first of three daily practices. They studied Stephen Curry's
ballhandling on YouTube. Prince practiced shooting until well after
sunset. At 13, he tried out for the academy. He didn't make it. ``I
won't stop until you succeed,'' his father said. Last spring, Prince
made it. ``My father was crying,'' he said. ``He told me, `This is your
first step toward your goal.' ''

Prince knows that becoming the Indian Yao Ming is the longest of long
shots. Two players recently moved from the India academy to the N.B.A.
Global Academy in Australia, the league's hub for its top international
prospects. But the contrasting reality is, players get cut. For each of
these young players, the academy is an opportunity, but the N.B.A. ---
which pays seven scouts to fan out around the country to identify
basketball talents --- is not in this business for altruism. Prince
hopes basketball takes him to the N.B.A. Or maybe another professional
league; that might enable him to make a decent living and send some
money to his parents.

It's an ambition often expressed by Indian players. Harsh Dagar, who is
13, is another promising prospect, and his only goal is to help his
mother, widowed eight years ago when his father died of typhoid fever.
Maybe all the time Prince and Dagar have spent on the court will lead to
the United States. But if their basketball careers end here, they say
that the experience will have been worth it. They practice at 6 three
mornings a week before school, then at 3:15 six afternoons a week. They
work with tutors and learn to be on time --- five minutes early to
everything, combating ``IST,'' or ``Indian Stretchable Time.'' ``Without
basketball,'' Prince said, ``I'd be going toward the bad things.''

\textbf{Three goats wandered} near the Mastan Y.M.C.A. in Mumbai's
Nagpada neighborhood. The aroma of lamb shawarma floated over two dusty
basketball courts. A law student played alongside a man from the nearby
slum. In the corner, a half-dozen homeless laborers slept on blankets.
Ashok Anjara, a 43-year-old mining financier who splits time between
Mumbai, London and New York, explained to me how these inner-city courts
are a metaphor for an India that's transitioning from third-world
country to world power. ``This area has a reputation: the gangsters, the
underworld, the crime kingpins,'' Anjara said. ``Every Don Corleone in
India was brought up in this neighborhood. And in the middle of that
area, you find this patch of hope.''

Anjara calls these courts Mumbai's basketball incubator. Captains for
the Indian men's national team learned the game here. When a heated game
is underway, thousands of spectators crowd the concrete steps. The
original Monsoon League, with games that started in the rain at 11 p.m.,
used to be played here during August, before moving to another nearby
location --- the league was created so that neighborhood kids wouldn't
have to go without basketball for the entirety of the four-month monsoon
season. ``Basketball here has always been played with very high
enthusiasm, if not at a very high level,'' Anjara told me. ``But a
billion people? That's the biggest friend basketball in India can
have.'' What he meant was that there has to be a future N.B.A. talent
somewhere in this huge country.

Anjara motioned to the referee. His name was Farooque Shaikh. His family
was among the 10 million refugees who migrated to India from present-day
Bangladesh after the India-Pakistan war of 1971. They found shelter in a
one-room shanty a few blocks from the Mastan Y.M.C.A. Shaikh is 39 now,
and he still lives in the same shanty, No. 25. He went to school only
through fourth grade. He has a wife and six children, the oldest a
16-year-old daughter, the youngest a 3-year-old son. Every day for two
decades he has risen at 6 to wash cars. He has a stable of clients: 50
motorcycles and eight cars a day. He makes \$7 a month per car and \$4 a
month per motorcycle. He's done by 2 p.m., spends a few hours with his
family, then heads to the court, where he coaches and referees until
late at night.

``We live within our means,'' he said in Hindi through an interpreter.
``When your means are less, you need less. It could be better, but I'm
happy it's not worse. It's basketball that stopped me from falling into
a life of misery and crime. In the shanties around me there's drugs,
crime, alcohol. Everyone I grew up with, they have been ruined.
Basketball gave me a focus.''

It gave him something else too. The Mastan Y.M.C.A. pays him about \$100
a month, and that enables him to send his children to private school ---
an English-speaking school, the best around, he said, with evident
pride. As we walked toward his home next to a 25-story high-rise, he
pulled out his cellphone to show me a photo of his son Fardeen, who is
6, wearing a Mastan jersey. Shaikh speaks to the neighborhood kids about
staying on the right path. In a country that refers to teachers as
``gurus,'' Shaikh has moved up in status, from just another shanty kid
to a respected basketball guru.

Shaikh put his children to bed, and soon the lights on the court went
off. Anjara moonlights as a sports agent, India's only agent certified
by the International Basketball Federation. He shares a goal with the
N.B.A.: to get an Indian into the world's best basketball league and to
watch the sport take off at home. But as a Mumbaikar who learned to play
on these courts, he emphasizes that basketball's spread in India does
more than just increase revenues for an American-based business. The
world won't feel the impact from Shaikh's rise in social status brought
about by basketball. But Shaikh's experience suggests how the
establishment of a real Indian basketball culture can produce benefits
beyond the N.B.A.'s bottom line.

``The hope he's gotten from basketball --- that's inspirational,''
Anjara told me. ``When we were nothing, this game gave us something.
Without basketball, he would have been a nobody --- a nobody. Now all
the families around here have to talk to Farooque. Otherwise he would be
a nobody, living in a shanty, washing cars, and nobody would give a
damn.''

Advertisement

\protect\hyperlink{after-bottom}{Continue reading the main story}

\hypertarget{site-index}{%
\subsection{Site Index}\label{site-index}}

\hypertarget{site-information-navigation}{%
\subsection{Site Information
Navigation}\label{site-information-navigation}}

\begin{itemize}
\tightlist
\item
  \href{https://help.nytimes3xbfgragh.onion/hc/en-us/articles/115014792127-Copyright-notice}{©~2020~The
  New York Times Company}
\end{itemize}

\begin{itemize}
\tightlist
\item
  \href{https://www.nytco.com/}{NYTCo}
\item
  \href{https://help.nytimes3xbfgragh.onion/hc/en-us/articles/115015385887-Contact-Us}{Contact
  Us}
\item
  \href{https://www.nytco.com/careers/}{Work with us}
\item
  \href{https://nytmediakit.com/}{Advertise}
\item
  \href{http://www.tbrandstudio.com/}{T Brand Studio}
\item
  \href{https://www.nytimes3xbfgragh.onion/privacy/cookie-policy\#how-do-i-manage-trackers}{Your
  Ad Choices}
\item
  \href{https://www.nytimes3xbfgragh.onion/privacy}{Privacy}
\item
  \href{https://help.nytimes3xbfgragh.onion/hc/en-us/articles/115014893428-Terms-of-service}{Terms
  of Service}
\item
  \href{https://help.nytimes3xbfgragh.onion/hc/en-us/articles/115014893968-Terms-of-sale}{Terms
  of Sale}
\item
  \href{https://spiderbites.nytimes3xbfgragh.onion}{Site Map}
\item
  \href{https://help.nytimes3xbfgragh.onion/hc/en-us}{Help}
\item
  \href{https://www.nytimes3xbfgragh.onion/subscription?campaignId=37WXW}{Subscriptions}
\end{itemize}
