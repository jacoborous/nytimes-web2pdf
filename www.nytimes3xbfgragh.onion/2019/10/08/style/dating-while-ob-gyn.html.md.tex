Sections

SEARCH

\protect\hyperlink{site-content}{Skip to
content}\protect\hyperlink{site-index}{Skip to site index}

\href{https://www.nytimes3xbfgragh.onion/section/style}{Style}

\href{https://myaccount.nytimes3xbfgragh.onion/auth/login?response_type=cookie\&client_id=vi}{}

\href{https://www.nytimes3xbfgragh.onion/section/todayspaper}{Today's
Paper}

\href{/section/style}{Style}\textbar{}Dating While Ob-Gyn

\begin{itemize}
\item
\item
\item
\item
\item
\end{itemize}

Advertisement

\protect\hyperlink{after-top}{Continue reading the main story}

Supported by

\protect\hyperlink{after-sponsor}{Continue reading the main story}

The Cycle

\hypertarget{dating-while-ob-gyn}{%
\section{Dating While Ob-Gyn}\label{dating-while-ob-gyn}}

When a romantic prospect learns what I do for a living, things often get
interesting. And weird.

\includegraphics{https://static01.graylady3jvrrxbe.onion/images/2019/10/07/fashion/07Cycle-obdate/07Cycle-obdate-articleLarge.jpg?quality=75\&auto=webp\&disable=upscale}

By Jen Gunter

\begin{itemize}
\item
  Oct. 8, 2019
\item
  \begin{itemize}
  \item
  \item
  \item
  \item
  \item
  \end{itemize}
\end{itemize}

When you first meet someone and she tells you what she does for a
living, it's not uncommon to think about what that means for \emph{you}.

For example, when I met a man online who was in construction, I thought
of all my badly needed home repairs.

When I met someone who was a financial planner, I pondered my retirement
planning.

So, when I tell a romantic prospect that I'm an obstetrician and
gynecologist, I know exactly what they are thinking about.

Vaginas.

Possibly even my vagina.

I accept this as human nature (everyone, including non-dates, probably
thinks of vaginas when they meet an ob-gyn) and part of the gig. Then
when I tell a prospect that I recently published a book called ``The
Vagina Bible,'' I get one of two responses.

There are the men who respond with a respectful ``oh'' or ``that's
interesting,'' and then we move on to another subject.

Then there are the amateur gynecologists who inevitably want to discuss
two things: female ejaculation and G spots.

So I explain the facts --- that the
\href{https://onlinelibrary.wiley.com/doi/abs/10.1111/jsm.12799}{best
evidence} tells us that so-called female ejaculation is urine (and who
cares if you're having fun, sex is messy) and that
\href{https://www.nytimes3xbfgragh.onion/ask/answers/how-do-i-find-my-g-spot}{there
is no specific G spot}, but many women have clitoral tissue close to the
urethra that can be very sensitive when stimulated correctly.

These amateur gynecologists gyno-splain to me that I am mistaken. That
their personal mastery informs them that I must be wrong.

My response is to tell them that statistically --- especially with that
attitude --- they have likely induced
\href{https://www.nytimes3xbfgragh.onion/2018/12/21/style/jen-gunter-says-your-vagina-is-terrific.html}{more
fake orgasms} than real ones, so I am uninterested in their opinions.

It's as effective as blocking someone on Twitter.

After ending a seven-year relationship last year, I began dating again.
And as someone who spends a lot of time communicating online and
thinking about sex, both personally and professionally, it's fascinating
---~and often disheartening ---~to observe the role technology plays in
our modern mating rituals.

There is, of course, sexting. Take this real-life exchange between me
and a would-be suitor, whose name I have changed:

\emph{Hey Jen, this is} \emph{Chad} \emph{from {[}insert random dating
site{]}.}

\emph{Hey Chad, this is Jen. Thanks for reaching out!}

\emph{I can recommend a good espresso machine.}

\emph{Oh that's great. My coffee making skills, as you noticed in my
profile, suck.}

\emph{I like anal.}

\emph{Oooh, Chad. I like to know someone's last name, if they're as
divorced as they claim and where screen shots of this interaction might
end up before we slide from lattes to my anus. I commend you on your
bold move, as only about} \emph{13 percent} \emph{of women reported anal
sex as a part of their recent sexual repertoire. Being this forward must
be a time saver. Best of luck in your search. ;)}

Then there is the unsolicited penis photo.

This is something
\href{https://yougov.co.uk/topics/politics/articles-reports/2018/02/16/four-ten-female-millennials-been-sent-dick-pic}{41
percent of women ages 18 to 36 have received}. A
\href{https://www.tandfonline.com/doi/full/10.1080/00224499.2019.1639036}{recent
study} tells us this photographic exhibitionism from heterosexual men is
often an overture for a naked picture in reply --- a pictogram for,
``Take off your clothes and send me a photo.'' It is also associated
with narcissism and arousal at eliciting disgust and shock.

I don't think I receive more sexts or unsolicited photos than the
general population, though I may experience a more rapid escalation
there once prospects learn
\href{https://www.nytimes3xbfgragh.onion/column/the-cycle}{I write about
sex}.

So, how is dating as an ob-gyn different? A few ways.

First and foremost:
\href{https://www.nytimes3xbfgragh.onion/2019/08/13/style/sti-stigma-sexual-transmitted-infections.html}{Sexually
transmitted infections} are front of mind for me and anyone else in my
profession. We spend a lot of time thinking about S.T.I.s, and this
causes some apprehension on my part about the false intimacy of sexting.

Why? There is data that suggests that
\href{https://link.springer.com/article/10.1007\%2Fs10508-019-01497-w}{sexting
someone you are not in a romantic relationship with, a.k.a. a relative
stranger}, may be associated with
\href{https://guilfordjournals.com/doi/pdf/10.1521/aeap.2016.28.2.138}{riskier
sexual behaviors} like unprotected sex.

We ob-gyns see infertility from chlamydia that lurked silently in the
fallopian tubes. Syphilis acquired from a supposedly monogamous partner.
H.I.V. in a teenager. All things that are preventable with condoms,
dental dams, testing, medications and communication.

How, then, does a dating ob-gyn approach the S.T.I. Talk? First I ask my
partner about his sexual history, and we discuss S.T.I. testing and
whether it's time for one or both of us to get tested.

Knowing what I know, I recently received the human papilloma virus (HPV)
vaccine. I'm out of the
\href{https://www.nytimes3xbfgragh.onion/2019/04/30/well/when-is-hpv-a-problem.html}{recommended
age range} (9 to 45 years old), but there is no medical risk to getting
the vaccine at my age, 53, just the chance that it may not be covered by
my insurance.

While statistically I have likely been exposed to at least one or two
types of HPV over my lifetime, the Gardasil 9 vaccine covers nine types
of HPV. (Though it's unlikely I have contracted all nine, it's nice to
have all bases covered!)

I'm dating men close to my age --- overtures from 28-year-olds, while
flattering, are just not for me~--- and while overall rates of HPV
positivity drop with age, I'm a realist: When people are dating, they
are more likely to have recent sexual activity, which is a
\href{https://sti.bmj.com/content/78/3/215}{risk factor for being HPV
positive}.

I also won't date someone who doesn't support reproductive rights. (This
of course is not an ob-gyn-specific dating preference.) Because this is
a deal breaker for me, I believe in getting that information out front
and center. I'll ask potential dates specifically about that and other
issues that are important to me before we migrate from the dating
platform to text.

Finally, there is one thing about the way that I date that is likely
unique, even among ob-gyns. If all the predate hurdles have been
cleared, I give my full name, knowing a Google search will ensue. I know
an article I wrote for The New York Times will come up:
``\href{https://www.nytimes3xbfgragh.onion/2017/11/16/style/my-vagina-is-terrific-your-opinion-about-it-is-not.html}{My
Vagina Is Terrific. Your Opinion About It Is Not.}''

If the men inquire about this piece, I explain that my vagina \emph{is}
terrific. And any man should count himself lucky to get anywhere near
it.

That conviction took almost a lifetime to acquire. If I could give a
gift to every woman, it would be the knowledge that her body is
glorious, along with the confidence to tell anyone who says otherwise
--- whether on a date or not --- that he is not worthy.

Online dating can make it easy to change behavior, and even lower the
standards that matter to us. For example, I wouldn't approach five or
six men in rapid succession at a party and ask if they were interested
in getting to know me romantically, but online dating --- whether it's
the algorithm or the anonymity --- encourages that behavior.

In some ways it's like shopping online and ending up with a shirt you
really didn't need or even want.

I think what's missing in all this for me is a lack of courtship: the
process of getting to know someone without every communication being
translated into computer code.

I've met enough people after extended digital communications to know
that while superficially many are who they said they were (politics,
favorite food, coffee habits, etc.), emails, texts and a review of their
social media use provided only the illusion of knowledge. It misses the
essence of that person, which was almost always very different. The
longer I communicated exclusively online before meeting, the more
disappointing the first date.

So more and more I find myself pulling back and asking, ``What would I
do in real life?''

I'm not sure I have the answer, but I think I'm going to have to go old
school and pick up the telephone more often and try to remember how this
used to work before online dating sites gave us the illusion of
connection.

Dr. Jen Gunter is an obstetrician and gynecologist in California. She is
the author of the
``\href{https://www.penguinrandomhouse.ca/books/636190/the-vagina-bible-by-dr-jen-gunter/9780735277373}{The
Vagina Bible}'' and writes
\href{https://www.nytimes3xbfgragh.onion/column/the-cycle}{The Cycle}, a
column on women's health that appears regularly in Styles.

Advertisement

\protect\hyperlink{after-bottom}{Continue reading the main story}

\hypertarget{site-index}{%
\subsection{Site Index}\label{site-index}}

\hypertarget{site-information-navigation}{%
\subsection{Site Information
Navigation}\label{site-information-navigation}}

\begin{itemize}
\tightlist
\item
  \href{https://help.nytimes3xbfgragh.onion/hc/en-us/articles/115014792127-Copyright-notice}{©~2020~The
  New York Times Company}
\end{itemize}

\begin{itemize}
\tightlist
\item
  \href{https://www.nytco.com/}{NYTCo}
\item
  \href{https://help.nytimes3xbfgragh.onion/hc/en-us/articles/115015385887-Contact-Us}{Contact
  Us}
\item
  \href{https://www.nytco.com/careers/}{Work with us}
\item
  \href{https://nytmediakit.com/}{Advertise}
\item
  \href{http://www.tbrandstudio.com/}{T Brand Studio}
\item
  \href{https://www.nytimes3xbfgragh.onion/privacy/cookie-policy\#how-do-i-manage-trackers}{Your
  Ad Choices}
\item
  \href{https://www.nytimes3xbfgragh.onion/privacy}{Privacy}
\item
  \href{https://help.nytimes3xbfgragh.onion/hc/en-us/articles/115014893428-Terms-of-service}{Terms
  of Service}
\item
  \href{https://help.nytimes3xbfgragh.onion/hc/en-us/articles/115014893968-Terms-of-sale}{Terms
  of Sale}
\item
  \href{https://spiderbites.nytimes3xbfgragh.onion}{Site Map}
\item
  \href{https://help.nytimes3xbfgragh.onion/hc/en-us}{Help}
\item
  \href{https://www.nytimes3xbfgragh.onion/subscription?campaignId=37WXW}{Subscriptions}
\end{itemize}
