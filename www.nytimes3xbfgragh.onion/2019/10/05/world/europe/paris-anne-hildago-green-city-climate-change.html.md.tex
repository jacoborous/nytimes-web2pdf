Sections

SEARCH

\protect\hyperlink{site-content}{Skip to
content}\protect\hyperlink{site-index}{Skip to site index}

\href{https://www.nytimes3xbfgragh.onion/section/world/europe}{Europe}

\href{https://myaccount.nytimes3xbfgragh.onion/auth/login?response_type=cookie\&client_id=vi}{}

\href{https://www.nytimes3xbfgragh.onion/section/todayspaper}{Today's
Paper}

\href{/section/world/europe}{Europe}\textbar{}The Greening of Paris
Makes Its Mayor More Than a Few Enemies

\url{https://nyti.ms/2VdCems}

\begin{itemize}
\item
\item
\item
\item
\item
\end{itemize}

Advertisement

\protect\hyperlink{after-top}{Continue reading the main story}

Supported by

\protect\hyperlink{after-sponsor}{Continue reading the main story}

\hypertarget{the-greening-of-paris-makes-its-mayor-more-than-a-few-enemies}{%
\section{The Greening of Paris Makes Its Mayor More Than a Few
Enemies}\label{the-greening-of-paris-makes-its-mayor-more-than-a-few-enemies}}

\includegraphics{https://static01.graylady3jvrrxbe.onion/images/2019/10/06/world/06Paris-Hidalgo/merlin_161653869_7e2f61ae-2e7b-4c69-bd16-515cc7891d83-articleLarge.jpg?quality=75\&auto=webp\&disable=upscale}

By \href{https://www.nytimes3xbfgragh.onion/by/adam-nossiter}{Adam
Nossiter}

\begin{itemize}
\item
  Oct. 5, 2019
\item
  \begin{itemize}
  \item
  \item
  \item
  \item
  \item
  \end{itemize}
\end{itemize}

PARIS --- All over Paris, streets have been dug up and cut in two, and
old paving stones overturned to build dozens of miles of bike lanes. A
major urban highway has been closed to cars and turned over to
pedestrians. Paris is readying feverishly for an emergency that, in the
minds of the city and its mayor, is already here.

The scorching summer blasted Parisians off the streets and turned the
city into an eerie dystopia of what may lie ahead. The brutal heat
removed any doubt, if there was one, in Mayor Anne Hidalgo's mind:
Climate change has arrived.

Even before
\href{https://www.nytimes3xbfgragh.onion/2019/07/24/world/europe/record-temperatures-heatwave.html?searchResultPosition=5}{July's
record temperatures} --- 42 degrees Celsius, or 107.6 Fahrenheit --- Ms.
Hidalgo spent the last five years as mayor trying to transform the
millennial city into a greener version of itself.

Along the way, Ms. Hidalgo, who is up for re-election next year, has won
supporters for her forward thinking, and earned thousands of enemies for
the inevitable disruption her plans have created.

But she has positioned herself prominently among the mayors of the
world's premier capitals as an advocate for what she bills as a new, and
necessary, kind of urban landscape.

\includegraphics{https://static01.graylady3jvrrxbe.onion/images/2019/10/06/world/06Paris-Hidalgo2/merlin_161653881_873a69e3-d3d2-468b-964a-84ba61857856-articleLarge.jpg?quality=75\&auto=webp\&disable=upscale}

Image

All over the city, bike lanes have been established, including on some
of its most emblematic thoroughfares like the Rue de
Rivoli.Credit...Andrea Mantovani for The New York Times

Paris today is like a giant construction site. By the end of the summer
more than 8,000 projects, most private but all approved by the city,
were underway, with historic squares like Madeleine, Bastille and Nation
transformed to make them more friendly to pedestrians.

Ugly craters in the asphalt signal work on the electrical grid, the
urban heating system for big buildings, the subway and regional
transport, and of course the city's miles of protected bike lanes.

Visitors and residents alike can now glide for miles along the river
banks on a bike, protected from the cars by granite separators, or all
the way across the city, from Concorde to Bastille, beating the
gridlocked traffic. Ms. Hidalgo is aiming for more than 600 miles of
bike lanes by next year, up from around 400 when she started.

As monotonous green-gray metal separators, shielding the projects, have
sprouted across Paris, the inconveniences have been accompanied by an
angry chorus of groans from hapless motorists.

The environmental results are ambiguous at best. There were around five
days with elevated ozone levels, for instance, in 2014, the year Ms.
Hidalgo took over; in 2018 there were from 15 to 22, depending on which
part of the city you were in.

``There are fewer cars, but there is more congestion, and that can
affect pollution levels,'' said Paul Lecroart, an urban planning expert
at the Paris regional planning agency.

But the fights the Spanish-born mayor has already taken on and won
demonstrate that she has no intention of backing down.

``There's been a very violent reaction at times,'' Ms. Hidalgo said,
smiling slightly in an interview in her cavernous office at the
grandiose Hôtel de Ville, the City Hall.

``Part of it has to do with being a woman,'' said Ms. Hidalgo, the
daughter of working-class immigrants from Spain. ``And being a woman
that wants to reduce the number of cars meant that I upset lots of men.
Two-thirds of public transport users are women.''

The idea of Ms. Hidalgo, a former workplace inspector at the French
labor ministry who worked her way up the political hierarchy, is simple.
To help secure an uncertain climate future, Paris must project itself
back into the past --- a past with fewer automobiles.

``What we've undertaken is a whole program of adaptation, of putting
nature back in this city,'' she said. ``We're trying to build this
around the individual. But change is difficult.''

``We can't live as before,'' Ms. Hidalgo added. ``There's been an
acceleration in climate change.''

She has declared war on cars. Every cursing motorist who seeks to
navigate the obstacle course of Parisian streets wishes she had never
been elected. Don't mention her name to taxi drivers.

Image

The Champs-Élysées on a car-free day in September.Credit...Andrea
Mantovani for The New York Times

Image

A traffic jam around the Paris Opera during rush hour. Some blame the
mayor's public works for adding to congestion.Credit...Andrea Mantovani
for The New York Times

``She's a hysteric,'' said Hamza Hansal, who owns a fleet of 10 cabs,
pausing next to yet another city construction site at the Place des
Fêtes in the working-class 19th Arrondissement. ``Nothing but bicycle
lanes and construction sites. Total chaos. Such BS. Traffic jams 24/7.''

Other critics, particularly on the right, accuse the Socialist mayor of
crazed environmental experimentation at the expense of the city's
inhabitants.

``She's put us in debt, just to make the BoBos happy,'' said Mr. Hansal,
using French slang for the bourgeois bohemians, the environmentally
conscious middle class that is the mayor's base.

Such responses reflect a widespread impression that fighting climate
change has become an elitist concern. But people in wealthier suburbs
--- who own cars --- have also been up in arms. The French car owners'
association, furious, published Ms. Hidalgo's office number and invited
motorists to flood it with calls.

But the city itself has seen a steep drop in car ownership, from 60
percent of households in 2001, to 35 percent today. Paris, meanwhile,
has risen in the
\href{https://www.wired.com/story/most-bike-friendly-cities-2019-copenhagenize-design-index/}{list
of bike-friendly cities} to eighth place from 17th since 2015.

Image

Critics say Ms. Hidalgo has been less aggressive in addressing other
problems, like an unceasing invasion of mass tourism that threatens to
turn the city into a vast open-air theme park .Credit...Andrea Mantovani
for The New York Times

Image

``There's been a very violent reaction at times,'' to her changes, Ms.
Hidalgo said. ``Part of it has to do with being a woman,'' who make up
two-thirds of public transport users, she said.

Credit...Andrea Mantovani for The New York Times

As elections approach next year, Ms. Hidalgo, a relatively solid
Socialist in a party otherwise headed for extinction, holds a
substantial lead in polling over her nearest rival, Benjamin Griveaux,
the candidate of President Emmanuel Macron's political movement.

Around the Place des Fêtes in the 19th Arrondissement, there were few
neutral parties among Parisians shopping at the outdoor market. But the
mayor's supporters outnumbered the critics in a random sampling.

``On cars, she's pretty tough,'' said Darnaud Guilhem, a professional
gardener. ``But I think she's right. She's causing some teeth-gnashing.
But Paris, with all this traffic, has become nearly unlivable.''

``She's going in the right direction,'' he said. ``Pretty farsighted.''

One of the mayor's most contested and controversial steps was to shut
down parts of the 42-year-old highway along the Right Bank of the Seine
and turn it into a park.

The long riverside expanse is now jammed with partying youth on warm
nights, and is becoming, in terms of race and class, one of the most
integrated spots in the city.

``She had everybody against her on that,'' said Corinne Lepage, a former
French environment minister. ``It was a very big symbolic victory.''

``She's got a lot of courage,'' Ms. Lepage added. ``Now, nobody even
thinks about putting cars back on there. That was a real reconquest of
urban space.''

Now Ms. Hidalgo is planning ``urban forests,'' clumps of trees on the
river bank and in front of the some of the city's most iconic spots,
like the Opera Garnier, the Hôtel de Ville and the Gare de Lyon train
station.

A top official has written a policy note detailing a possible green
corridor up and down the city if she is elected to a second term.

Leo Fauconnet, an urban expert with the Paris region's planning agency,
gave Ms. Hidalgo credit. ``We've got a proactive policy, compared to
other cities in the world,'' he said.

But hard-core environmentalists, growing in political clout in France
with the surge of the Green party, are not persuaded.

``Environmental politics is about limiting the damage,'' said Jacques
Boutault, the Green mayor of Paris's central Second Arrondissement*.*
``You can't just allow the concrete to flow, then plant a few trees on
the pavement.''

Image

Renovations around the Samaritaine building. There have been 8,000
construction projects in the city, provoking criticism that Ms. Hidalgo
has been too friendly toward developers.Credit...Andrea Mantovani for
The New York Times

Image

One of the mayor's most contested steps was to shut down parts of a
42-year-old highway along the right bank of the Seine and turn it into a
park.Credit...Andrea Mantovani for The New York Times

Tougher critics wondered about City Hall's friendliness toward
developers. The mayor's office ``speaks with a double language,'' said
Antoine Picon, an architectural historian at Harvard. ``Green, yes, but
let's continue to encourage the densification of Paris, in what is
already one of the densest cities in the world.''

``The city has never been as up for sale as it is today,'' he said.
``The entire city is being transformed into a shopping mall.''

Indeed, the character of much of the city is changing in ways that Ms.
Hidalgo has been less forceful in addressing. An unceasing invasion of
mass tourism threatens to turn Paris into a vast open-air theme park for
the global affluent.

Working families are leaving central areas like the Second
Arrondissement, which has lost 10 percent of its population since 2015.
Airbnb has created neighborhoods of absentee landlords. The average
apartment price per square meter has topped \$11,000, making Paris now
the world's third-most expensive city.

Ms. Hidalgo said she was aware of those dangers, too, and was working to
mitigate them.

``Paris can't just be a city for winners,'' Ms. Hidalgo said. ``The role
of politicians is to regulate. And to stop this city from being one only
for the winners.''

Advertisement

\protect\hyperlink{after-bottom}{Continue reading the main story}

\hypertarget{site-index}{%
\subsection{Site Index}\label{site-index}}

\hypertarget{site-information-navigation}{%
\subsection{Site Information
Navigation}\label{site-information-navigation}}

\begin{itemize}
\tightlist
\item
  \href{https://help.nytimes3xbfgragh.onion/hc/en-us/articles/115014792127-Copyright-notice}{©~2020~The
  New York Times Company}
\end{itemize}

\begin{itemize}
\tightlist
\item
  \href{https://www.nytco.com/}{NYTCo}
\item
  \href{https://help.nytimes3xbfgragh.onion/hc/en-us/articles/115015385887-Contact-Us}{Contact
  Us}
\item
  \href{https://www.nytco.com/careers/}{Work with us}
\item
  \href{https://nytmediakit.com/}{Advertise}
\item
  \href{http://www.tbrandstudio.com/}{T Brand Studio}
\item
  \href{https://www.nytimes3xbfgragh.onion/privacy/cookie-policy\#how-do-i-manage-trackers}{Your
  Ad Choices}
\item
  \href{https://www.nytimes3xbfgragh.onion/privacy}{Privacy}
\item
  \href{https://help.nytimes3xbfgragh.onion/hc/en-us/articles/115014893428-Terms-of-service}{Terms
  of Service}
\item
  \href{https://help.nytimes3xbfgragh.onion/hc/en-us/articles/115014893968-Terms-of-sale}{Terms
  of Sale}
\item
  \href{https://spiderbites.nytimes3xbfgragh.onion}{Site Map}
\item
  \href{https://help.nytimes3xbfgragh.onion/hc/en-us}{Help}
\item
  \href{https://www.nytimes3xbfgragh.onion/subscription?campaignId=37WXW}{Subscriptions}
\end{itemize}
