Sections

SEARCH

\protect\hyperlink{site-content}{Skip to
content}\protect\hyperlink{site-index}{Skip to site index}

\href{https://www.nytimes3xbfgragh.onion/section/style/self-care/}{Self-Care}

\href{https://myaccount.nytimes3xbfgragh.onion/auth/login?response_type=cookie\&client_id=vi}{}

\href{https://www.nytimes3xbfgragh.onion/section/todayspaper}{Today's
Paper}

\href{/section/style/self-care/}{Self-Care}\textbar{}Is Palo Santo
Endangered?

\url{https://nyti.ms/34pSio7}

\begin{itemize}
\item
\item
\item
\item
\item
\end{itemize}

Advertisement

\protect\hyperlink{after-top}{Continue reading the main story}

Supported by

\protect\hyperlink{after-sponsor}{Continue reading the main story}

\hypertarget{is-palo-santo-endangered}{%
\section{Is Palo Santo Endangered?}\label{is-palo-santo-endangered}}

No, but, you should probably reconsider how you are using it.

\includegraphics{https://static01.graylady3jvrrxbe.onion/images/2019/12/19/style/shutterstock_1042011265/shutterstock_1042011265-videoSixteenByNineJumbo1600.jpg}

\href{https://www.nytimes3xbfgragh.onion/by/crystal-martin}{\includegraphics{https://static01.graylady3jvrrxbe.onion/images/2019/03/01/multimedia/author-crystal-martin/author-crystal-martin-thumbLarge.png}}

By \href{https://www.nytimes3xbfgragh.onion/by/crystal-martin}{Crystal
Martin}

\begin{itemize}
\item
  Dec. 16, 2019
\item
  \begin{itemize}
  \item
  \item
  \item
  \item
  \item
  \end{itemize}
\end{itemize}

\href{https://www.nytimes3xbfgragh.onion/es/2019/12/23/espanol/palo-santo-peligro-de-extincion.html}{Leer
en español}

Palo santo --- the aromatic wood that has been used for traditional
healing and in spiritual ceremonies in indigenous and mestizo Latin
American cultures for centuries --- has seen a growth in commercial
popularity alongside concerns about its conservation status.

Its popularity within the mainstream self-care community is driven by
its warm scent when burned as an incense and oblique promises to clear a
space of ``bad vibes.'' Memes portraying palo santo as being a couple
packs of incense away from extinction are prolific. Some even claim
there are only 250 trees left. The reality isn't quite so grim but is
certainly complicated.

\hypertarget{what-exactly-is-palo-santo}{%
\subsection{What, exactly, is palo
santo?}\label{what-exactly-is-palo-santo}}

The tree species being used in the wellness world is called bursera
graveolens, which grows all over the Americas including Mexico, Peru,
Costa Rica, Guatemala, Honduras, Galápagos Islands and plentifully in
mainland Ecuador. To get palo santo of the best quality, the wood must
be harvested in a certain way. ``They come to a ripe age at 50 to 70
years old. That's relatively short. Once the tree dies of natural
causes, it has to be left for a minimum of five to eight years for the
oils in the heartwood to mature enough to make quality incense,'' said
Jonathon Miller Weisberger an ethnobotanist and author of ``Rainforest
Medicine: Preserving Indigenous Science and Biodiversity in the Upper
Amazon..'' That fallen, aged wood is then processed into sticks used as
incense. Watch out for synthetic palo santo, where chemicals are used to
produce the signature palo santo scent. We don't know how much is out
there, but read the fine print on packaging. It will typically say
``synthetic.''

\hypertarget{is-palo-santo-endangered-1}{%
\subsection{Is palo santo
endangered?}\label{is-palo-santo-endangered-1}}

Palo santo is not endangered. This month, the International Union for
Conservation of Nature (IUCN) for the first time released a review of
bursera graveolens's conservation status and declared it ``of least
concern.''

\hypertarget{so-why-the-confusion}{%
\subsection{So why the confusion?}\label{so-why-the-confusion}}

A few factors are at play. The first: a case of mistaken identity. A
totally different species, bulnesia sarmientoi, is also commonly known
as ``palo santo'' and grows in the Gran Chaco region of South America.
Its dark, mahogany-lookalike wood is used for its essential oils and in
making products like furniture. And that tree is in fact threatened and
may be confused with bursera graveolens when people are researching the
topic online.

There's also the matter of regional versus international status. The
IUCN which has declared bursera graveolens of least concern takes global
populations of a species into consideration when making its assessment.
But national governments determine a plant's regional conservation
status, meaning a plant can be listed as endangered in one country and
not another. The online rumors may date back to 2005, when Peru listed
its palo santo as endangered.

\hypertarget{what-is-palo-santo-good-for}{%
\subsection{What is palo santo good
for?}\label{what-is-palo-santo-good-for}}

Palo santo has a sweet yet complex scent. Its essential oil is used in
consumer products, like shampoo, perfume and soap for fragrance. Palo
santo sticks are burned as a sacred tool in spiritual ceremonies like
smudging, which has varying purposes but is commonly said to cleanse
negative energy. These practices have their roots in indigenous
cultures, but palo santo is used in Catholic religious ceremonies in
Latin America, too.

\hypertarget{can-i-go-back-to-doing-my-palo-santo-thing-then}{%
\subsection{Can I go back to doing my palo santo thing
then?}\label{can-i-go-back-to-doing-my-palo-santo-thing-then}}

Not without some thought. Though palo santo isn't endangered, its
habitat, tropical dry forest is threatened. ``Dry tropical forest have
been decimated. Estimates are that only five to ten percent of dry
tropical forests are still intact around the world,'' said Susan
Leopold, Ph.D., and the executive director of United Plant Savers, a
conservation organization. She said that because dry tropical forests
have a dry period (unlike rain forests) they are hospitable to human
activity. People can go in and log or clear forests to use the land for
something else, like cattle ranching.

According to Mr. Miller Weisberger, the most abundant populations of
palo santo are in Ecuador, but other regions have small populations. If
you don't know where your palo santo is coming from, it could be from
one of these tiny populations where improper harvesting could erase that
specific, regional group of palo santo.

``Without a doubt buying palo santo is potentially jeopardizing and
people could be participating in the decimation of isolated rare
populations of palo santo,'' said Mr. Miller Weisberger. To complicate
matters further, what we know as bursera graveolens could actually be
multiple species. ``The isolated pockets throughout Central America and
on the Galápagos may be subspecies or even a different species that is
so reduced that suitable harvesting isn't possible,'' said Mr. Miller
Weisberger. This all means consumers must ensure the palo santo they're
buying is sustainably and ethically produced.

\includegraphics{https://static01.graylady3jvrrxbe.onion/images/2019/12/11/style/self-care/oakImage-1576101302510/oakImage-1576101302510-articleLarge.jpg?quality=75\&auto=webp\&disable=upscale}

\hypertarget{how-do-i-buy-palo-santo-sustainably}{%
\subsection{How do I buy palo santo
sustainably?}\label{how-do-i-buy-palo-santo-sustainably}}

By all means, don't stop buying palo santo. Experts like those at the
IUCN say that more demand combined with responsible cultivation and
harvesting could be good for the species and its habitat. Land that
might be razed to raise cattle would have higher economic value if
farmers can plant palo santo and sell it for a good price. Buy it from
small business not a huge corporate retailer. Look for a supplier that
is completely transparent and doing its own the legwork in sourcing palo
santo. Adriana Ayales is a rainforest herbalist who grew up in Costa
Rica and runs \href{https://animamundiherbals.com/}{Anima Mundi
Apothecary}. ``Look for companies where they themselves have gone to the
area where the trees are from, met the farmers, know their names, know
the area and regularly return to the area. There are a lot people who
are essentially middlemen of Latin American distributors who aren't
doing that kind of legwork,'' said Ms. Ayales.

\hypertarget{but-wait--is-using-palo-santo-cultural-appropriation}{%
\subsection{But wait --- is using palo santo cultural
appropriation?}\label{but-wait--is-using-palo-santo-cultural-appropriation}}

If you're using it in a quasi-spiritual way without proper knowledge or
training, yes, probably. Indigenous and Latin American people have
developed a cultural heritage around many different types of herbal
healing and spiritual ceremonies. While smudging has become popular,
it's very rarely done with the participation or consultation with those
groups. ``So when you scroll through Instagram and see a non-Native
person smudging with sage or palo santo and taking their artful picture
of that, they've probably purchased that item from a corporate source.
They're using our culture but removing our faces from the picture. It
forwards the narrative that we don't exist and that we're not experts in
our own fields and heritage. And that's harmful to us because it
perpetuates the extremely prevalent notion that we don't exist,'' said
\href{http://chelseyluger.com/}{Chelsey Luger,} founder of
\href{https://www.wellforculture.com/}{Well for Culture}, an indigenous
wellness initiative. You'll have to make the call, but at least consider
buying palo santo from small, local and indigenous-owned businesses and
do your research on the heritage of these spiritual practices.

Advertisement

\protect\hyperlink{after-bottom}{Continue reading the main story}

\hypertarget{site-index}{%
\subsection{Site Index}\label{site-index}}

\hypertarget{site-information-navigation}{%
\subsection{Site Information
Navigation}\label{site-information-navigation}}

\begin{itemize}
\tightlist
\item
  \href{https://help.nytimes3xbfgragh.onion/hc/en-us/articles/115014792127-Copyright-notice}{©~2020~The
  New York Times Company}
\end{itemize}

\begin{itemize}
\tightlist
\item
  \href{https://www.nytco.com/}{NYTCo}
\item
  \href{https://help.nytimes3xbfgragh.onion/hc/en-us/articles/115015385887-Contact-Us}{Contact
  Us}
\item
  \href{https://www.nytco.com/careers/}{Work with us}
\item
  \href{https://nytmediakit.com/}{Advertise}
\item
  \href{http://www.tbrandstudio.com/}{T Brand Studio}
\item
  \href{https://www.nytimes3xbfgragh.onion/privacy/cookie-policy\#how-do-i-manage-trackers}{Your
  Ad Choices}
\item
  \href{https://www.nytimes3xbfgragh.onion/privacy}{Privacy}
\item
  \href{https://help.nytimes3xbfgragh.onion/hc/en-us/articles/115014893428-Terms-of-service}{Terms
  of Service}
\item
  \href{https://help.nytimes3xbfgragh.onion/hc/en-us/articles/115014893968-Terms-of-sale}{Terms
  of Sale}
\item
  \href{https://spiderbites.nytimes3xbfgragh.onion}{Site Map}
\item
  \href{https://help.nytimes3xbfgragh.onion/hc/en-us}{Help}
\item
  \href{https://www.nytimes3xbfgragh.onion/subscription?campaignId=37WXW}{Subscriptions}
\end{itemize}
