Sections

SEARCH

\protect\hyperlink{site-content}{Skip to
content}\protect\hyperlink{site-index}{Skip to site index}

\href{https://www.nytimes3xbfgragh.onion/section/politics}{Politics}

\href{https://myaccount.nytimes3xbfgragh.onion/auth/login?response_type=cookie\&client_id=vi}{}

\href{https://www.nytimes3xbfgragh.onion/section/todayspaper}{Today's
Paper}

\href{/section/politics}{Politics}\textbar{}House Votes to Give the
Government the Power to Negotiate Drug Prices

\url{https://nyti.ms/38wDKq3}

\begin{itemize}
\item
\item
\item
\item
\item
\end{itemize}

Advertisement

\protect\hyperlink{after-top}{Continue reading the main story}

Supported by

\protect\hyperlink{after-sponsor}{Continue reading the main story}

\hypertarget{house-votes-to-give-the-government-the-power-to-negotiate-drug-prices}{%
\section{House Votes to Give the Government the Power to Negotiate Drug
Prices}\label{house-votes-to-give-the-government-the-power-to-negotiate-drug-prices}}

The high-profile measure would curb the price of prescription drugs and
extend more benefits to Medicare beneficiaries.

\includegraphics{https://static01.graylady3jvrrxbe.onion/images/2019/12/12/us/politics/12DC-DRUGS/merlin_164544093_dd0f09bf-ec70-4d04-804f-3c5f5a381ec8-articleLarge.jpg?quality=75\&auto=webp\&disable=upscale}

\href{https://www.nytimes3xbfgragh.onion/by/sheryl-gay-stolberg}{\includegraphics{https://static01.graylady3jvrrxbe.onion/images/2018/11/26/multimedia/author-sheryl-gay-stolberg/author-sheryl-gay-stolberg-thumbLarge.png}}

By
\href{https://www.nytimes3xbfgragh.onion/by/sheryl-gay-stolberg}{Sheryl
Gay Stolberg}

\begin{itemize}
\item
  Dec. 12, 2019
\item
  \begin{itemize}
  \item
  \item
  \item
  \item
  \item
  \end{itemize}
\end{itemize}

WASHINGTON --- The House, delivering on one of Democrats' central
campaign promises, passed ambitious legislation on Thursday to lower the
rising cost of prescription drugs by empowering the federal government
to negotiate prices with pharmaceutical manufacturers.

The bill, known as H.R. 3 --- a numerical designation that reflects its
position on Democrats' priority list --- would make significant changes
to the federal Medicare program, which provides health coverage to older
Americans. It passed largely on party lines, 230 to 192, and includes
provisions to create new vision, dental and hearing benefits, and caps
out-of-pocket drug costs for Medicare beneficiaries at \$2,000.

Lowering the cost of prescription drugs is a huge priority for voters
and politicians of both parties --- including President Trump, who has
made curbing the cost of prescription drugs a
\href{https://www.nytimes3xbfgragh.onion/2019/07/11/health/drug-prices-rebate-donald-trump.html}{central
theme of his 2020 re-election campaign}.

Though Mr. Trump
\href{https://www.whitehouse.gov/wp-content/uploads/2019/12/SAP_HR-3.pdf}{has
vowed to veto the Democratic bill}, its passage could pressure Senator
Mitch McConnell of Kentucky, the Republican leader, to take up
\href{https://www.grassley.senate.gov/news/news-releases/grassley-wyden-release-updated-prescription-drug-pricing-reduction-act-reach}{a
bipartisan drug-price measure} pending there, or press senators to act
on other bills. Pharmaceutical makers strongly oppose both the House
bill and the Senate bill, which was drafted in the Finance Committee.

At another moment, House passage might have also jump-started
negotiations with the White House, but that is unlikely given that
Democrats
\href{https://www.nytimes3xbfgragh.onion/2019/12/12/us/politics/trump-impeachment-judiciary.html}{are
on the verge of impeaching the president}.

The central --- and most contentious --- provision of the measure that
passed Thursday is its language enabling the Department of Health and
Human Services, which administers Medicare, to negotiate the price of up
to 250 commonly used drugs, including insulin. It would also require the
manufacturers to offer the agreed-on prices to private insurers, giving
it huge reach.

And it would require pharmaceutical manufacturers to pay rebates to
Medicare if the price of their drugs increased faster than inflation ---
another provision loathed by drug makers.

The Senate bill, sponsored by Senators Charles E. Grassley, Republican
of Iowa, and Ron Wyden, Democrat of Oregon, does not contain those
provisions. But like the House bill, it would cap out-of-pocket expenses
and require drug companies to pay rebates to Medicare if they raised
prices faster than inflation.

House Republicans countered the Democrats' bill with a more modest
measure, which drew support from eight Democrats when it came to a vote
on Thursday. That measure, an amalgam of other bills that have drawn
bipartisan support, would also cap out-of-pocket expenses and require
insurance companies to make information available about drug prices to
patients in doctors' offices before doctors prescribe them. But it
excludes the Medicare negotiation provision.

Mr. Trump campaigned in 2016 on allowing the government to negotiate
drug prices, but Republicans argued that giving that power to the
government would force pharmaceutical companies to eliminate research
and development, depriving the public of lifesaving medicines.

``For an issue as crucial as lowering the cost of prescription drugs for
Americans, partisanship should be set aside,'' Representative Kevin
McCarthy of California, the Republican leader, said as he made his
party's closing argument for his bill. He warned that the Democrats'
bill was ``opening the door to a government takeover of our prescription
drug market.''

Representative Greg Walden, Republican of Oregon, who managed the bill
for Republicans on the House floor, cited a Congressional Budget Office
estimate showing that the Democrats' measure would result in 40 fewer
drugs over the next two decades.

Democrats say their bill addresses research and development concerns by
allocating more than \$10 billion to the National Institutes of Health
for biomedical research, with the goal of advancing breakthrough cures.
The \href{https://www.cbo.gov/publication/55936}{Congressional Budget
Office has estimated} that the bill would save taxpayers \$5 billion
over a 10-year period.

Allowing Medicare to negotiate prices has been a long-sought goal of
Democrats; because the insurance program buys drugs in bulk, it can
effectively set the price for all insurers. It was a matter of intense
debate in 2003, when the Republican-led Congress
\href{https://www.nytimes3xbfgragh.onion/2003/06/27/us/house-and-senate-pass-measures-for-broad-overhaul-of-medicare.html}{passed
the bill creating Medicare Part D,}which allows Medicare to pay for the
cost of prescription drugs.

Since that time, the rising list prices of brand-name drugs, coupled
with insurance plans that have increasingly raised deductibles and asked
consumers to pay more out of pocket, have left many Americans facing
difficult choices about whether to forgo medicine in favor of other
necessities, including food.

That has created pressure on lawmakers to take action. Democrats
campaigned aggressively last year on their promise to lower drug prices,
which helped them regain the majority in the House. On Thursday morning,
Speaker Nancy Pelosi appeared on the Capitol steps with members of the
freshman class to drive home that message.

Although the bill passed on Thursday is unlikely to become law in
anything close to its present form, it will serve as a campaign document
for the Democrats, to show voters what their vision is on prescription
drugs and that they have the will to make a substantive change in the
system rather than tinker around the edges.

The bill, named the Elijah E. Cummings Lower Drug Costs Now Act, for the
recently deceased chairman of the House Oversight and Reform Committee,
passed with unanimous support from Democrats and the backing of two
Republicans, Representatives Brian Fitzpatrick of Pennsylvania and Jaime
Herrera Beutler of Washington.

Not all Democrats were happy about it. Representative Lloyd Doggett,
Democrat of Texas, said the bill did not go far enough. Although he
voted for the measure as ``as a statement about the importance" of
negotiating drug prices, Mr. Doggett said he sought a more expansive
measure that would extend health coverage to the roughly 30 million
Americans who lack it.

``This bill was developed to appeal to Trump on the theory that
President Trump would follow the advice of Candidate Trump, who called
for bidding, talked about the billions that could be saved,'' Mr.
Doggett said. ``As it became evident that was not going to work, I felt
that this bill didn't need to move left, but it did need to deal with
those who are left out.''

Hours after the measure passed, the Democratic Congressional Campaign
Committee, which works to elect House Democrats, announced that it was
running an advertising campaign on Facebook to use the bill's passage to
shore up Democrats and target Republicans.

Representative Cheri Bustos of Illinois, who heads the committee, said
the ads were aimed at showing consumers that Republicans ``will always
prioritize padding the pockets of their special interest backers over
the people they were elected to represent.''

Advertisement

\protect\hyperlink{after-bottom}{Continue reading the main story}

\hypertarget{site-index}{%
\subsection{Site Index}\label{site-index}}

\hypertarget{site-information-navigation}{%
\subsection{Site Information
Navigation}\label{site-information-navigation}}

\begin{itemize}
\tightlist
\item
  \href{https://help.nytimes3xbfgragh.onion/hc/en-us/articles/115014792127-Copyright-notice}{©~2020~The
  New York Times Company}
\end{itemize}

\begin{itemize}
\tightlist
\item
  \href{https://www.nytco.com/}{NYTCo}
\item
  \href{https://help.nytimes3xbfgragh.onion/hc/en-us/articles/115015385887-Contact-Us}{Contact
  Us}
\item
  \href{https://www.nytco.com/careers/}{Work with us}
\item
  \href{https://nytmediakit.com/}{Advertise}
\item
  \href{http://www.tbrandstudio.com/}{T Brand Studio}
\item
  \href{https://www.nytimes3xbfgragh.onion/privacy/cookie-policy\#how-do-i-manage-trackers}{Your
  Ad Choices}
\item
  \href{https://www.nytimes3xbfgragh.onion/privacy}{Privacy}
\item
  \href{https://help.nytimes3xbfgragh.onion/hc/en-us/articles/115014893428-Terms-of-service}{Terms
  of Service}
\item
  \href{https://help.nytimes3xbfgragh.onion/hc/en-us/articles/115014893968-Terms-of-sale}{Terms
  of Sale}
\item
  \href{https://spiderbites.nytimes3xbfgragh.onion}{Site Map}
\item
  \href{https://help.nytimes3xbfgragh.onion/hc/en-us}{Help}
\item
  \href{https://www.nytimes3xbfgragh.onion/subscription?campaignId=37WXW}{Subscriptions}
\end{itemize}
