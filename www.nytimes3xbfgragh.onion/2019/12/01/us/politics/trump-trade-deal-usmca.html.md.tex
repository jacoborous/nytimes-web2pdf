Sections

SEARCH

\protect\hyperlink{site-content}{Skip to
content}\protect\hyperlink{site-index}{Skip to site index}

\href{https://www.nytimes3xbfgragh.onion/section/politics}{Politics}

\href{https://myaccount.nytimes3xbfgragh.onion/auth/login?response_type=cookie\&client_id=vi}{}

\href{https://www.nytimes3xbfgragh.onion/section/todayspaper}{Today's
Paper}

\href{/section/politics}{Politics}\textbar{}Trump's Trade Deal Steals a
Page From Democrats' Playbook

\url{https://nyti.ms/2OXEdIQ}

\begin{itemize}
\item
\item
\item
\item
\item
\item
\end{itemize}

Advertisement

\protect\hyperlink{after-top}{Continue reading the main story}

Supported by

\protect\hyperlink{after-sponsor}{Continue reading the main story}

\hypertarget{trumps-trade-deal-steals-a-page-from-democrats-playbook}{%
\section{Trump's Trade Deal Steals a Page From Democrats'
Playbook}\label{trumps-trade-deal-steals-a-page-from-democrats-playbook}}

The president has made a trade agreement that caters to his opposition
--- and that's why it stands a chance of passing Congress.

\includegraphics{https://static01.graylady3jvrrxbe.onion/images/2019/11/27/business/00DC-USMCA1/merlin_162855642_b94f126f-6116-49d6-9249-4cdbe5aec2c0-articleLarge.jpg?quality=75\&auto=webp\&disable=upscale}

\href{https://www.nytimes3xbfgragh.onion/by/ana-swanson}{\includegraphics{https://static01.graylady3jvrrxbe.onion/images/2018/12/10/multimedia/author-ana-swanson/author-ana-swanson-thumbLarge.png}}\href{https://www.nytimes3xbfgragh.onion/by/emily-cochrane}{\includegraphics{https://static01.graylady3jvrrxbe.onion/images/2018/11/28/multimedia/author-emily-cochrane/author-emily-cochrane-thumbLarge-v3.png}}

By \href{https://www.nytimes3xbfgragh.onion/by/ana-swanson}{Ana Swanson}
and \href{https://www.nytimes3xbfgragh.onion/by/emily-cochrane}{Emily
Cochrane}

\begin{itemize}
\item
  Dec. 1, 2019
\item
  \begin{itemize}
  \item
  \item
  \item
  \item
  \item
  \item
  \end{itemize}
\end{itemize}

WASHINGTON --- House Democrats return to Washington on Monday facing a
difficult choice: Should they hand President Trump a victory in the
midst of a heated impeachment battle or walk away from one of the most
progressive trade pacts ever negotiated by either party?

The Trump administration
\href{https://www.nytimes3xbfgragh.onion/2018/11/30/world/americas/trump-trudeau-canada-mexico.html}{agreed
with Canada and Mexico on revisions} to the North American Free Trade
Agreement one year ago, but the deal still needs the approval of
Congress.
\href{https://www.nytimes3xbfgragh.onion/2019/12/10/us/politics/trump-aides-and-democrats-strike-deal-on-north-american-trade-pact.html}{A
handshake agreement with the administration} in the coming days would
give the Democratic caucus a tangible accomplishment on an issue that
has animated its base. It could also give Democrats a chance to lock in
long-sought policy changes to a trade pact they criticize as
prioritizing corporations over workers, laying the groundwork for future
trade agreements.

Those factors have coaxed Democrats to the table at an improbable
moment, when Washington is split by partisan fights and deeply divided
over an impeachment inquiry. After months of talks, including through
the Thanksgiving break, both sides say they're in the final phase of
negotiations. But Democrats insist the administration must make more
changes to the labor, environmental and other provisions before Speaker
Nancy Pelosi of California will bring legislation implementing the new
United States-Mexico-Canada Agreement to a vote.

``By any standard, what we've already negotiated is substantially better
than NAFTA,'' said Representative Richard E. Neal of Massachusetts, who
is heading the Democratic group negotiating with the administration.
``Labor enforcement, in my judgment, is the last hurdle.''

The deal presents a dilemma for Democrats because it contains measures
they have supported for years, from requiring more of a car's parts to
be made in North America to rolling back a special system of arbitration
for corporations and strengthening Mexican labor unions.

In borrowing from the Democrats' playbook, the revised pact reflects Mr.
Trump's populist trade approach --- one that has blurred party lines and
appealed to many of the blue-collar workers Democrats once counted among
their base. It also reflects a broader backlash to more traditional free
trade deals, which have been criticized for hollowing out American
manufacturing and eliminating jobs.

``Taken as a whole, it looks more like an agreement that would've been
negotiated under the Obama administration,'' said Senator Rob Portman,
Republican of Ohio and
\href{https://www.nytimes3xbfgragh.onion/2005/03/18/business/congressman-from-ohio-is-chosen-for-trade-post.html}{a
former trade representative} during the George W. Bush administration,
who supports the pact. ``There are some aspects to it that Democrats
have been calling for, for decades.''

In fact, it goes so far to the left of traditional Republican views on
trade that some congressional Republicans only grudgingly support it ---
or may vote against the final deal.

Senator Patrick J. Toomey of Pennsylvania, one of the most ardent
Republican critics of the deal, has called the pact ``a complete
departure from the free trade agreements we've pursued through our
history'' and urged fellow Republicans to vote it down.

``If we adopt this agreement, it will be the first time that I know of
in the history of the Republic that we will agree to a new trade
agreement that is designed to diminish trade,'' Mr. Toomey said at a
hearing in July, sitting next to a large red sign that said: ``NAFTA
\textgreater{} U.S.M.C.A.''

Still, most Republicans have supported the pact and urged rapid action.
If the deal is not approved soon, proponents fear it could become the
target of more frequent attacks by Democratic presidential candidates,
making it even more difficult for Democrats in Congress to vote for the
pact.

Mr. Trump has spent weeks accusing Ms. Pelosi of being ``grossly
incompetent'' and prioritizing impeachment over a trade deal that could
benefit workers. ``She's incapable of moving it,'' Mr. Trump said last
week, warning that a ``great trade deal for the farmers, manufacturers,
workers of all types, including unions'' could fall apart if the
Democrats don't take action.

\includegraphics{https://static01.graylady3jvrrxbe.onion/images/2019/11/27/business/00DC-USMCA5/merlin_164402424_d3fe0b93-3651-4277-b22c-ddbe2e16eefe-articleLarge.jpg?quality=75\&auto=webp\&disable=upscale}

While long demonized by Mr. Trump, Democrats and labor unions, NAFTA has
become critical to companies and consumers across North America, guiding
commerce around the continent for a quarter century. Entire industries
have grown up around the trade agreement, which allows goods like cars,
avocados and textiles to flow tariff free among Canada, Mexico and the
United States.

But Mr. Trump and other critics have blamed the deal for encouraging
companies to move their factories to Mexico. The president has routinely
called NAFTA the ``worst trade deal ever made'' and promised during his
campaign that he would rewrite it in America's favor --- or
\href{https://www.nytimes3xbfgragh.onion/2018/12/02/us/politics/trump-withdraw-nafta.html}{scrap
it} altogether.

The revised pact took
\href{https://www.nytimes3xbfgragh.onion/2018/09/30/us/politics/us-canada-nafta-deal-deadline.html}{over
a year of rancorous talks} to complete, resulting in a complex
2,082-page agreement covering a wide range of topics. While much of it
simply updates NAFTA for the 21st century, it also contains changes
intended to encourage manufacturing in the United States, including by
raising how much of a car must be made in North America to qualify for
zero tariffs.

The new agreement requires at least 70 percent of an automaker's steel
and aluminum to be bought in North America, which could help boost
United States metal production. And 40 to 45 percent of a car's content
must be made by workers earning an average wage of \$16 an hour. That
\$16 floor is an effort to force auto companies to either raise low
wages in Mexico or hire more workers in the United States and Canada, an
outcome Democrats have long supported.

It also rolls back
\href{https://www.nytimes3xbfgragh.onion/2017/10/16/us/politics/nafta-united-states-canada.html}{a
special system of arbitration for corporations} that the Democratic
presidential candidate Elizabeth Warren
\href{https://www.warren.senate.gov/newsroom/press-releases/warren-urges-us-trade-rep-to-remove-isds-provisions-during-next-round-of-nafta-negotiations}{has
criticized} as allowing companies to bypass the American legal system
and Trump administration officials describe as an incentive for
companies to send their factories abroad.

The pact also includes, at least on paper, provisions that aim to do
away with sham Mexican labor unions that have done little to help
workers by requiring every company in Mexico to seek worker approval of
collective bargaining agreements by secret ballot in the next four
years.

Some Democrats are skeptical that the Mexican government will allocate
the necessary funds to ensure that companies are complying with these
changes. But if the rules are enforced, Democrats say they may help stem
the flow of jobs to Mexico and put American workers on a more equal
footing.

Several sticking points remain, including a
\href{https://www.nytimes3xbfgragh.onion/2019/03/21/us/politics/nafta-drug-prices.html}{provision
that offers an advanced class of drugs} 10 years of protection from
cheaper alternatives, which Democratic lawmakers say would lock in high
drug prices.

Other Democratic proposals aim to add teeth to the pact's labor and
environmental provisions. Democrats want to reverse a change made by the
Trump administration that they say essentially guts NAFTA's enforcement
system. They are also arguing for additional resources that would allow
customs officials to inspect factories or stop goods at the border if
companies violate labor rules.

Mr. Neal told reporters late last month that he believed House Democrats
could soon work out their differences with Robert Lighthizer, Mr.
Trump's trade representative.

Image

``Labor enforcement, in my judgment, is the last hurdle,'' said
Representative Richard E. Neal of Massachusetts, who is heading the
Democratic group negotiating with the Trump administration over the
trade deal.Credit...J. Scott Applewhite/Associated Press

Ms. Pelosi, who has continued to suggest that she wants to ``get to
yes'' on the deal, responded to Mr. Trump's rebuke last week by saying
that she needed to see the administration's commitments in writing
before moving forward.

The agreement still has skeptics,
\href{https://www.nytimes3xbfgragh.onion/2019/05/29/us/politics/pelosi-trump-nafta-deal.html}{including
labor leaders} and others on the left.

``Unless Donald Trump agrees to add stronger labor and environmental
standards and enforcement, and secures progress on labor reforms in
Mexico, NAFTA job outsourcing will continue,'' said Lori Wallach, the
director of Public Citizen's Global Trade Watch. ``And the Big Pharma
giveaways Trump added must go: They make U.S.M.C.A. worse than NAFTA.''

But Democrats say that if the additional changes they are seeking get
made, the deal would be more progressive than the original NAFTA and the
Trans-Pacific Partnership --- both of which were negotiated by
Democratic administrations. Mr. Trump pulled the United States out of
the Trans-Pacific Partnership within days of taking office.

Jesús Seade, Mexico's chief negotiator for the United
States-Mexico-Canada Agreement, said many tweaks Democrats want are
``improvements.''

``If the amendments suggested are acceptable improvements, then there's
no reason we should not be shaking hands next week,'' he said on Friday,
after meeting with Canadian officials.

Some congressional Republicans, who generally oppose unions and believe
the deal's new rules could burden auto companies, have been taken aback
by how far the administration has gone to woo Democrats.

At a private lunch on June 11 at the Capitol, Republican senators
peppered Vice President Mike Pence with questions about why the
administration was not lobbying Democrats harder to back the deal. Mr.
Pence claimed that it already had the support of 80 Democrats, a high
number that caught some Republicans by surprise, according to a person
familiar with the meeting who spoke on condition of anonymity.

``What's in it for Pelosi?'' asked Senator Ben Sasse, Republican of
Nebraska.

Mr. Pence responded that the pact had the most aggressive labor and
automotive standards ever put in a trade agreement --- an admission for
some Republicans in the room that it was the worst trade agreement they
had been asked to support.

Image

Senator Patrick J. Toomey of Pennsylvania is one of the most ardent
Republican critics of the deal.Credit...Erin Schaff/The New York Times

Jennifer Hillman, a trade expert at the Council on Foreign Relations,
said many of Mr. Lighthizer and Mr. Trump's views on trade ``are
basically borrowing what Democrats have said for many, many years.''

``To the extent that Trump gained votes in the industrial Midwest, it
was by espousing Democratic trade ideas,'' she said.

Throughout the negotiations, Mr. Lighthizer has kept up a steady
dialogue with labor unions like the United Steelworkers and Democrats
like Ms. Pelosi, Mr. Neal and Senator Sherrod Brown of Ohio. At times,
Mr. Lighthizer appeared more at odds with congressional Republicans and
traditional allies like the Chamber of Commerce, who he said should give
up ``a little bit of the sugar'' that had sweetened trade agreements for
multinational corporations.

Image

Robert Lighthizer, the United States trade representative, has at times
appeared more at odds with congressional Republican than with
Democrats.Credit...Anna Moneymaker/The New York Times

``If you can get some labor unions on board, Democrats on board,
mainstream Republicans on board, I think you can get big numbers,'' Mr.
Lighthizer said in January 2018. ``If you do, that's going to change the
way all of us look at these kind of deals.''

Advertisement

\protect\hyperlink{after-bottom}{Continue reading the main story}

\hypertarget{site-index}{%
\subsection{Site Index}\label{site-index}}

\hypertarget{site-information-navigation}{%
\subsection{Site Information
Navigation}\label{site-information-navigation}}

\begin{itemize}
\tightlist
\item
  \href{https://help.nytimes3xbfgragh.onion/hc/en-us/articles/115014792127-Copyright-notice}{©~2020~The
  New York Times Company}
\end{itemize}

\begin{itemize}
\tightlist
\item
  \href{https://www.nytco.com/}{NYTCo}
\item
  \href{https://help.nytimes3xbfgragh.onion/hc/en-us/articles/115015385887-Contact-Us}{Contact
  Us}
\item
  \href{https://www.nytco.com/careers/}{Work with us}
\item
  \href{https://nytmediakit.com/}{Advertise}
\item
  \href{http://www.tbrandstudio.com/}{T Brand Studio}
\item
  \href{https://www.nytimes3xbfgragh.onion/privacy/cookie-policy\#how-do-i-manage-trackers}{Your
  Ad Choices}
\item
  \href{https://www.nytimes3xbfgragh.onion/privacy}{Privacy}
\item
  \href{https://help.nytimes3xbfgragh.onion/hc/en-us/articles/115014893428-Terms-of-service}{Terms
  of Service}
\item
  \href{https://help.nytimes3xbfgragh.onion/hc/en-us/articles/115014893968-Terms-of-sale}{Terms
  of Sale}
\item
  \href{https://spiderbites.nytimes3xbfgragh.onion}{Site Map}
\item
  \href{https://help.nytimes3xbfgragh.onion/hc/en-us}{Help}
\item
  \href{https://www.nytimes3xbfgragh.onion/subscription?campaignId=37WXW}{Subscriptions}
\end{itemize}
