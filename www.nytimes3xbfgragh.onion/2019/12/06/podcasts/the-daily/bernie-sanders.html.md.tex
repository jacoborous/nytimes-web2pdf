\href{/podcasts/the-daily}{The Daily}\textbar{}The Candidates: Bernie
Sanders

\url{https://nyti.ms/2Yojmmr}

\begin{itemize}
\item
\item
\item
\item
\item
\item
\end{itemize}

\begin{itemize}
\item
  \href{https://www.nytimes3xbfgragh.onion/live/2020/09/07/us/trump-vs-biden?action=click\&pgtype=Article\&state=default\&region=TOP_BANNER\&context=storylines_menu}{Election
  Updates}
\item
  \href{https://www.nytimes3xbfgragh.onion/interactive/2020/us/elections/election-states-biden-trump.html?action=click\&pgtype=Article\&state=default\&region=TOP_BANNER\&context=storylines_menu}{Paths
  to 270}
\item
  \href{https://www.nytimes3xbfgragh.onion/interactive/2020/08/31/us/politics/vote-by-mail-deadlines.html?action=click\&pgtype=Article\&state=default\&region=TOP_BANNER\&context=storylines_menu}{Voting
  by Mail}
\item
  \href{https://www.nytimes3xbfgragh.onion/interactive/2019/us/elections/2020-presidential-election-calendar.html?action=click\&pgtype=Article\&state=default\&region=TOP_BANNER\&context=storylines_menu}{Key
  Dates}
\item
  \href{https://www.nytimes3xbfgragh.onion/newsletters/politics?action=click\&pgtype=Article\&state=default\&region=TOP_BANNER\&context=storylines_menu}{Politics
  Newsletter}
\end{itemize}

\includegraphics{https://static01.graylady3jvrrxbe.onion/images/2019/12/06/podcasts/06sanders-03/merlin_164604849_49aca12c-7001-44a4-93ef-fe745e2f7269-articleLarge.jpg?quality=75\&auto=webp\&disable=upscale}

Sections

\protect\hyperlink{site-content}{Skip to
content}\protect\hyperlink{site-index}{Skip to site index}

\hypertarget{the-candidates-bernie-sanders}{%
\section{The Candidates: Bernie
Sanders}\label{the-candidates-bernie-sanders}}

We spoke with the Vermont senator about his journey from the fringes of
American politics to the forefront --- and the ideas that shaped him
along the way.

Credit...Joe Buglewicz for The New York Times

Supported by

\protect\hyperlink{after-sponsor}{Continue reading the main story}

Written by
\href{https://www.nytimes3xbfgragh.onion/by/lauren-jackson}{Lauren
Jackson}

\begin{itemize}
\item
  Published Dec. 6, 2019Updated Aug. 19, 2020
\item
  \begin{itemize}
  \item
  \item
  \item
  \item
  \item
  \item
  \end{itemize}
\end{itemize}

\textbf{``The Candidates: Bernie Sanders'' was hosted by Michael
Barbaro, produced by Alexandra Leigh Young and Jessica Cheung, and
edited by Paige Cowett and Lisa Tobin. Alexander Burns contributed
reporting.}

\emph{\textbf{Listen and subscribe to our podcast from your mobile
device:}}\\
\textbf{\href{https://itunes.apple.com/us/podcast/the-daily/id1200361736?mt=2}{\emph{Via
Apple Podcasts}}} \emph{\textbf{\textbar{}}}
\textbf{\href{https://play.radiopublic.com/88f7d8c3-7289-4dc6-b300-5ba71b43f5e5}{\emph{Via
RadioPublic}}} \emph{\textbf{\textbar{}}}
\textbf{\href{http://www.stitcher.com/podcast/the-new-york-times/the-daily-10}{\emph{Via
Stitcher}}}

In Part 2 of our series on pivotal moments in the lives of the 2020
Democratic presidential contenders, we spoke with
\href{https://www.nytimes3xbfgragh.onion/2020/01/17/podcasts/the-daily/bernie-sanders-latino.html}{Bernie
Sanders}, the Vermont senator and democratic socialist. Mr. Sanders
reflected on his education in politics and how he galvanized grass-roots
support to evolve from outraged outsider to mainstream candidate with
little shift in his message.

\includegraphics{https://static01.graylady3jvrrxbe.onion/images/2017/01/29/podcasts/the-daily-album-art/the-daily-album-art-articleInline-v2.jpg?quality=75\&auto=webp\&disable=upscale}

\hypertarget{listen-to-the-daily-the-candidates-bernie-sanders}{%
\subsubsection{Listen to `The Daily': The Candidates: Bernie
Sanders}\label{listen-to-the-daily-the-candidates-bernie-sanders}}

We spoke with the Vermont senator about his journey from the fringes of
American politics to the forefront --- and the ideas that shaped him
along the way.

transcript

Back to The Daily

bars

0:00/38:01

-38:01

transcript

\hypertarget{listen-to-the-daily-the-candidates-bernie-sanders-1}{%
\subsection{Listen to `The Daily': The Candidates: Bernie
Sanders}\label{listen-to-the-daily-the-candidates-bernie-sanders-1}}

\hypertarget{hosted-by-michael-barbaro-produced-by-alexandra-leigh-young-and-jessica-cheung-and-edited-by-lisa-tobin-and-paige-cowett}{%
\subsubsection{Hosted by Michael Barbaro, produced by Alexandra Leigh
Young and Jessica Cheung, and edited by Lisa Tobin and Paige
Cowett}\label{hosted-by-michael-barbaro-produced-by-alexandra-leigh-young-and-jessica-cheung-and-edited-by-lisa-tobin-and-paige-cowett}}

\hypertarget{we-spoke-with-the-vermont-senator-about-his-journey-from-the-fringes-of-american-politics-to-the-forefront--and-the-ideas-that-shaped-him-along-the-way}{%
\paragraph{We spoke with the Vermont senator about his journey from the
fringes of American politics to the forefront --- and the ideas that
shaped him along the
way.}\label{we-spoke-with-the-vermont-senator-about-his-journey-from-the-fringes-of-american-politics-to-the-forefront--and-the-ideas-that-shaped-him-along-the-way}}

\begin{itemize}
\item
  bernie sanders\\
  Hello.
\item
  michael barbaro\\
  Hello.
\item
  bernie sanders\\
  How are you?
\item
  michael barbaro\\
  Senator, Michael Barbaro.
\item
  bernie sanders\\
  Nice to see you, Michael.
\item
  michael barbaro\\
  Great pleasure.
\item
  jessica cheung\\
  Hi, Jessica Cheung.
\item
  bernie sanders\\
  Jessica. Nice to see you. Where would you like me?
\item
  michael barbaro\\
  Very nice to meet you, Michael Barbaro. So, Senator Sanders, my
  colleague Alex Burns told me that to understand your political career,
  and your presidential campaign today, we have to go back to the first
  time that you won elected office as mayor of Burlington in 1981. So
  that's ---
\item
  bernie sanders\\
  The New York Times got it right. Every once in a while.
\item
  michael barbaro\\
  So that's what I want to ask you about.
\item
  bernie sanders\\
  All right, there you go.
\item
  michael barbaro\\
  From The New York Times, I'm Michael Barbaro. This is ``The Daily.''
\item
  archived recording\\
  Burlington is the largest city in Vermont. Situated as it is on Lake
  Champlain with the Adirondack Mountains view, it's a lovely, lovely
  spot. We'd like you to meet its new mayor. Mayor Sanders got a lot of
  attention recently, not only with his ten vote victory, but mostly
  because he is a socialist.
\end{itemize}

michael barbaro

Part two in our series on pivotal moments in the lives of the top four
Democratic candidates for president. Today ---

\begin{itemize}
\tightlist
\item
  archived recording (bernie sanders)\\
  The people who are living in all of the Burlington housing authority
  developments, both the senior citizen development and the low income
  housing projects are going to be receiving the lowest cable television
  bills in the state of Vermont.
\end{itemize}

michael barbaro

Bernie Sanders.

\begin{itemize}
\tightlist
\item
  archived recording (bernie sanders)\\
  Ronald Reagan and his billionaire friends do not represent America,
  but we do. Lastly, I want to touch upon an issue that dear to my
  heart, and that is the issue of affordable health care. The people of
  Burlington voted overwhelmingly, overwhelmingly in support of Congress
  moving forward to establish a national health care system. I think
  that is exactly how this country is going to have to go on that issue.
\end{itemize}

michael barbaro

It's Friday, December 6.

Alex Burns, why this moment?

alex burns

Bernie Sanders is such an unusual character in American politics, as a
lifelong socialist and left wing activist who has endured for decades as
a major political figure, and who has become a leading presidential
candidate. And to understand how he got from, really, the fringes of
American politics to the absolute forefront, you have to go back to this
moment in the early 1980s where he becomes mayor of Burlington, where he
figures out how to take those ideas and actually win elections with
them, and then govern. This is a period I've been spending a lot of time
on in my own reporting because it's just such a vital formative
experience for Sanders. And so the story starts with the turn of the
1970s as Bernie Sanders arrives in Vermont with a whole lot of left wing
ideas, not a whole lot of local connections, and links up with a new
marginal political party called the Liberty Union.

bernie sanders

And that party had been formed around opposition to the war in Vietnam,
and in the fight for economic justice. It's a very small party in a very
small state.

michael barbaro

Bernie Sanders starts showing up to Liberty Union meetings, and the
party identifies him as the man they want to run for a U.S. Senate seat
in 1971.

bernie sanders

And it was a very interesting campaign, and so forth, and so on. I got
two percent of the vote.

michael barbaro

He loses that election, but then he has gotten the electoral bug.

bernie sanders

A year later, there was the general election. I ran for governor of the
state, I got one percent.

michael barbaro

He loses again.

bernie sanders

Then I ran for Senate again against Pat Leahy, as Leahy often reminds
me, and I got four percent.

michael barbaro

And again, he loses. I'm seeing a pattern here.

bernie sanders

Yes.

michael barbaro

It is loss, after loss, after loss. And while he's running and losing,
he has a series of odd jobs.

bernie sanders

I was doing some writing. I was banging nails, doing a little bit of
carpentry work.

michael barbaro

He also had a job putting together newsreels and educational film strips
about history for school kids.

bernie sanders

That's before video. For younger people, there was a thing called film
strips. I won't go through what they were, photographs and sound. And I
did most of the work myself, had a little bit of help, photography and
so forth. It was lot of fun, actually.

michael barbaro

He sells these films to schools in the region, and he also spends time
putting together a project that he's personally quite invested in and
proud of.

\begin{itemize}
\tightlist
\item
  archived recording\\
  If you are the average American who watches television 40 hours a
  week, you have probably heard of such important people as Kojak and
  Wonder Woman. Strangely enough, however, nobody has told you about
  Eugene Debs, one of the most important Americans of the 20th century.
\end{itemize}

michael barbaro

Which is a film about the life of the legendary American socialist
leader, Eugene Debs.

bernie sanders

Debs was a very great American. He was one of the original founders of
industrial unionism, socialist party candidate for president six times,
somebody I admired a whole lot.

\begin{itemize}
\tightlist
\item
  archived recording (eugene debs)\\
  The ruling class has always taught and trained you to believe it to be
  your patriotic duty to go to war, and to have yourself slaughtered at
  their command. But in all the history of the world, you the people
  have never had a voice in declaring war.
\end{itemize}

michael barbaro

So throughout the 1970s, he is this activist educator who is running
campaign after campaign, and losing every time. He's not really
developing a professional or political career for himself in Vermont.
But in the city of Burlington ---

bernie sanders

In 1980 or so, some friends of mine came up to me. And they said, you
know, there's going to be a mayor's election coming up in `81. And you
know what, we checked the records, and you did pretty well running as a
Liberty Union candidate. You got actually 12 percent of the vote in some
of the working class districts in Burlington.

michael barbaro

So two percent, or four percent, or six percent ---

bernie sanders

That was statewide. But in Burlington, we did better.

michael barbaro

You were doing better.

bernie sanders

Yeah. So we got a bunch of people together, and they said, O.K., we'll
do it.

\begin{itemize}
\tightlist
\item
  archived recording\\
  Bernie Sanders, a Brooklyn born self-described socialist running for
  mayor for the first time in 1981, running against a Democratic old
  guard that had run the city for a decade.
\end{itemize}

michael barbaro

When Sanders becomes a candidate for mayor, he is facing off against a
powerful Democratic establishment. Burlington at this point, for decades
has been essentially a one party town with a relatively conservative
Democratic ruling clique that has just had a hammerlock on city
politics. The incumbent mayor is not seen by anybody as vulnerable, to
the point that the Republicans don't even field a candidate against him.
He's also up against, just a culture of apathy when it comes to
municipal elections. People generally don't show up to vote for mayor,
or for other city offices. So Bernie Sanders and this kind of ragtag
group of academics and activists and intellectuals band together to try
to figure out how to crack a city election in a place where nobody like
them has ever won before.

bernie sanders

You would literally not believe if I told you how little we knew about
politics. At the end of the day, I mean, real politics. It's one thing
to run for statewide office knowing you're not going to win and get on
our radio show and talk about issues, which I could do, but the
nitty-gritty of politics.

michael barbaro

So as a newcomer to city politics, Bernie Sanders runs a different kind
of campaign from the campaigns he's run before. This isn't about 30,000
foot ideological issues, like when he was a protest candidate for the
Senate.

\begin{itemize}
\item
  archived recording (bernie sanders)\\
  Ruth, you're a volunteer worker at the old North end food co-op here
  in Burlington.
\item
  archived recording\\
  Right. We're on disability, social security.
\item
  archived recording (bernie sanders)\\
  You got cut from \$131 to ---
\item
  archived recording\\
  \$48.
\item
  archived recording (bernie sanders)\\
  And what was the justification for that? How do they expect you to
  live on the difference?
\item
  archived recording\\
  They don't care.
\end{itemize}

michael barbaro

This is a ground level campaign that's waged over really concrete
kitchen table issues that are relevant, he hopes, to a wide array of
constituencies in the city that feel like they've been left out by the
existing power structure.

bernie sanders

We had a lot of support in, for example, low income housing projects
from people were getting a raw deal from the city that ran the projects.
We had support from environmental groups. We had support from one group
in the south end of the city. There was going to be a major highway
going right through the neighborhood, and they said, we don't like that.

michael barbaro

So Sanders is gaining some real support in this race. He's not a trivial
candidate, but still, the powers that be in Burlington do not see him as
a threat to win this election. Let me talk about election night. What
was the story of that night for you?

bernie sanders

Well, when I walked in on election day, I was of two opinions. Number
one, that we would lose very heavily. And the newspapers, some guy, a
newspaper writer was covering it, he said, the odds of Sanders winning
are about 100 to one. That was literally what they wrote. So either we
were doing something magical, or we would lose overwhelmingly. What I
did not anticipate is that on election night, I think the results were,
we were ahead by 14 votes. And after the recount, 10 votes. That, I did
not expect.

\begin{itemize}
\tightlist
\item
  archived recording\\
  Many people in Burlington are still in a state of shock following that
  city's most stunning political upset in memory last night.
\end{itemize}

michael barbaro

The press reports from election night describe him as stunned, and then
elated that he wins. And he wins by the absolute narrowest of margins,
just ten votes.

alex burns

Ten votes.

\begin{itemize}
\tightlist
\item
  archived recording\\
  Bernard Sanders, one of the founders of the Liberty Union Party, and a
  consistent loser in previous quests for elective office, was now the
  big winner. Considered by many to be unelectable because of his
  so-called radical views, Mr. Sanders put together an unlikely
  coalition of supporters and edged the ten year incumbent Gordon
  Paquette.
\end{itemize}

michael barbaro

So your strategy had worked. When you take office, how did becoming an
elected official, the day to day reality of it, match your expectations
of the power of winning this office and being mayor?

bernie sanders

Well, we had a very unique experience.

alex burns

Bernie Sanders has pulled off an extraordinary feat. He has upended the
city establishment, he has become a socialist mayor in the United States
at the height of the Cold War. But what happens next is he runs into a
brick wall of political opposition. There is a body in Burlington, the
Board of Aldermen. We would call it a city council. There are 13
members, 11 of them are either Democrats or Republicans, but their party
label matters less than the fact that they are opposed to Bernie
Sanders. He comes in, the powerful Democrats and the powerful
Republicans both essentially say, he should not be the mayor of the
city, and he will not be the mayor of the city for very long. Because
we're going to make sure that he can't get anything done.

michael barbaro

People were trying to sabotage you?

bernie sanders

Trying to sabotage me, yes. They were trying to sabotage me. The first
thing they did was to fire my secretary.

michael barbaro

They have the power to fire your secretary?

bernie sanders

Yeah, they did.

michael barbaro

So they reject his secretary. They take it back pretty quickly, but the
damage to the relationship is kind of done. Not only do the board of
Aldermen mess with his ability to hire a secretary, they reject all of
his nominees for the top jobs in the city, city clerk, city treasurer,
city attorney.

bernie sanders

And they made me run the city for the first year with exactly the
administration of the guy I had beaten. You know, it's like ---

michael barbaro

You're being neutered.

bernie sanders

Yes. So it's like, you know, Donald Trump running his administration
with Barack Obama's appointees.

alex burns

So for really his first full year as mayor, he has a somewhat ornamental
role.

michael barbaro

How are how are you thinking about this challenge?

bernie sanders

Well, their attitude, what they had said, and one of their leaders said,
well, look. Bernie Sanders is a fluke. That was the word they used, and
they said ---

michael barbaro

Your brand of politics, everything about you, they thought was just a
fluke.

bernie sanders

Right, this is an accident that should never have happened. And we will
stonewall him in the first year, people will see that he can't
accomplish anything, then we'll go back to the old ways.

michael barbaro

They're going to drive you from office.

bernie sanders

Yeah. Well, it was a brutal year. So what we had to do was literally
form a parallel city government without any money. I mean, we didn't,
couldn't pay anybody, but we brought together a group of strong
supporters and we had them helping us working on legislation and ideas.
And we did everything that we could while we were being absolutely
opposed by the Democratic and Republicans on the board of Aldermen. So
we organized at the grass-roots level, we mobilize people. Our job was
to get people involved in the political process.

michael barbaro

How did you do that?

bernie sanders

Well, I'll tell you how we did it. Even before I took office, we had
meetings on issues that people were concerned about that had been
ignored for a very long time. We said, we believe in arts. You know, a
city has got to be vital and alive. What do we do about the arts? What
do we do about economic development? What do we do about women's rights?
So we ended up storming a council on Arts, a council on women, a council
on youth. We started what we called neighborhood planning associations,
which meant we gave local neighborhoods, each ward had a certain amount
of money and they spent it however they wanted. So we tried to
democratize it, and we brought people into the process. So it wasn't me
saying, we're going to do A, B and C. These were people who themselves
were now empowered.

michael barbaro

So you're finding a way to essentially circumvent these aldermen who
think you're a fluke, and think they can block you by literally tapping
into ---

bernie sanders

Right.

michael barbaro

More voters, more people. When did you know that this strategy was
actually working?

bernie sanders

Well, when hundreds of people would show up at city council meetings and
demand our agenda. We were fighting for an agenda, it was being blocked
by the city council. So people were upset about it, and here's the
interesting thing: We have elections for mayor then every two years, but
half the board of Aldermen comes up on the odd year.

michael barbaro

So in 1982, one year after Sanders becomes mayor, seven of the 13
members of the board of aldermen are up for re-election.

bernie sanders

Essentially, there was a referendum on my administration.

michael barbaro

These elections became a chance for Mayor Sanders to go directly to the
voters and ask them to replace these intransigent members of the board
of aldermen with people who are friendly to him and supportive of his
ideas.

bernie sanders

We ran candidates in almost every ward in the city. I probably have
never worked so hard in my life. I knocked on almost every door in the
city with the candidates that we were running with. And this is the
winter time in Vermont, so we're talking about 10 below zero, in zero
weather. And on election night, the turnout was phenomenal for a
non-mayor's race, it was just off the charts. In five --- if my memory
is correct, in the five wards that we ran in, we won outright three of
the wards, in all of the working class areas.

And here is the most exciting thing about all of this. If you go back to
the basement of City Hall and check the old records in Burlington, what
you'll find is that between 1979, that was the previous election before
I won, and two years later when I was running for re-election we doubled
voter turnout.

michael barbaro

He's right, voter turnout really did rise in Burlington when Sanders got
involved in city politics. And some of that is about him and his message
and his political organization. Some of it is just having contested
elections, elections where people file to run against the people who are
already sitting in public office. When you have two choices rather than
one, then yeah, more people show up to make a choice.

\begin{itemize}
\tightlist
\item
  archived recording (bernie sanders)\\
  Good evening. We're recording this on Friday, March 5th, and we've
  decided to get out of City Hall, get out of the office. And we're here
  on the first floor the Burlington Square Mall.
\end{itemize}

michael barbaro

And he continues to engage and attempt to inspire voters in this same
way, getting out in the community. He's a highly visible mayor.

\begin{itemize}
\tightlist
\item
  archived recording (bernie sanders)\\
  And I think what we'll do is have some informal discussions with
  Vermonters as they walk past us, and as we can grab them. And we'll
  see if we can get their views on some of the important issues of the
  day. But before we do ---
\end{itemize}

michael barbaro

He creates a local television show called ``Bernie Speaks With the
Community,'' where he is just out there and connected to your average
voter.

\begin{itemize}
\item
  archived recording (bernie sanders)\\
  Oops, oh, here we go.
\item
  archived recording\\
  Hi, Shannon.
\item
  archived recording (bernie sanders)\\
  Shannon, do you live in Burlington, Shannon?
\item
  archived recording\\
  Yeah.
\item
  archived recording (bernie sanders)\\
  O.K. So how are things going with you?
\item
  archived recording\\
  Pretty good. I was just wondering, my mother had this idea for an
  indoor/outdoor amusement park by the waterfront, and she wanted ---
  and I want to know if, is there anything going to be done about it?
\item
  archived recording (bernie sanders)\\
  Well, I can't say for sure that something will be done immediately. I
  think it is a good idea, and interestingly enough, your mother
  mentioned ---
\end{itemize}

michael barbaro

It's a highly unusual approach for a municipal politician, and
especially in a city where the mayor had not been that kind of man about
town previous.

\begin{itemize}
\item
  archived recording (bernie sanders)\\
  O.K. The next person that we've kidnapped here off the streets for a
  few words is Jodie Baggerly. Jodie, welcome.
\item
  archived recording (jodie baggerly)\\
  Well, thank you. One thing I want to appreciate, being a disabled
  person, is the little discount we get on our cable TVs, because I
  think it's a positive point to have educational programs to be able to
  watch and fill our minds at periods when we are unable to get out.
\item
  archived recording (bernie sanders)\\
  Let me just jump in and remind our viewers. What Jodie is talking
  about is, the city negotiated with the Mountain Cable Company ---
\end{itemize}

michael barbaro

So Senator Sanders, in brief, what are the lessons of this moment for
you?

bernie sanders

Lessons of this moment is that winning politics is grass-roots politics,
that winning politics is developing coalitions of working people, of low
income people, of women, of environmentalists. So coalition is, we do it
from the bottom on up, and we ended up in my years as mayor taking on
everybody.

michael barbaro

We'll be right back.

{[}music{]}

michael barbaro

So Alex, in Sanders's telling, in the face of total political opposition
and stonewalling, his solution is to essentially do what got him
political power in the first place, which is go to the people, talk to
the people, always the people.

alex burns

That's his political brand as mayor, much as it's his political brand
now. And in Burlington, it's an approach that really works for him. It
establishes him as a legitimate city executive with an independent power
base who cannot just be treated as an interloper in City Hall. It's also
the first of a couple stages in Mayor Sanders' campaign to reinvent
Burlington city politics. And if the first part of that is about really
engaging with city politics at the ground level, the next stage after
he's been mayor for a couple of years, is to look way beyond Burlington
and take on big national and international political issues, and connect
them back to the local level.

\begin{itemize}
\item
  archived recording\\
  If I were the president of the largest bank in Burlington, I'd be real
  nervous about you.
\item
  archived recording (bernie sanders)\\
  Well, they may be, they may be. But I think --- and they are. But I
  think what we've often talked about also, is that my powers as mayor
  are in many ways limited. And I have my visions as to what life should
  be in Vermont, in Burlington, and in the United States, but we are
  going to speak out, though, on national and international issues which
  affect the city of Burlington. For example, obviously, we're very
  concerned about Mr. Reagan's policies which are impacting
  devastatingly on low income and working people. But we know what our
  powers are within the city of Burlington.
\end{itemize}

michael barbaro

So Senator Sanders, during this period you start talking about national
issues. You start talking about President Reagan, his economic policy.
You start talking about foreign policy. You send letters to the leaders
of Japan expressing regret for the two bombs that were dropped on that
country by the U.S. What was your thinking? As you're building this
coalition locally, you start talking about issues beyond the borders of
Burlington. And what is your thinking ---

bernie sanders

Well, let's be clear. 90 plus percent of our energy was dealing with
local issues like reforming the police department and paving the
streets. We brought a minor league baseball team into Burlington,
Vermont. 95, 98 percent of our work was locally, doing what mayors are
supposed to do. But as part of empowering people, what we also believed
is it was important to think globally and act locally. So if we were
spending a whole lot of money in Washington under Reagan, investing in
military spending, or giving tax breaks to the rich, that impacted the
city of Burlington. We are mayors, we need money to help us with
housing. We need money to help us with roads and infrastructure, and yet
Washington is spending this money on the military, or they're busy
invading another country, or whatever they're doing. We should be
speaking up on those issues.

\begin{itemize}
\tightlist
\item
  archived recording (bernie sanders)\\
  The question is whether we use the incredible wealth and natural
  resources and intelligence of our society to create a decent standard
  of living, a decent life for all of our people in this country and
  abroad, or do we develop the greatest military machine for killing in
  the history of the world. That's what the choice is.
\end{itemize}

bernie sanders

This was in the middle of the Cold War, and we started a sister-city
program. I know the, some of the right wing media misinterprets this.
But what we did is, I took a group of about a dozen people from
Burlington to Yaroslavl in the old Soviet Union.

\begin{itemize}
\tightlist
\item
  archived recording\\
  {[}SINGING{]}
\end{itemize}

bernie sanders

We had hockey teams coming out, we had doctors coming in and out, we had
kids coming in and out. It really --- I love the idea of sister-city
programs, and it worked phenomenally well.

\begin{itemize}
\tightlist
\item
  archived recording\\
  {[}SINGING{]} This land is your land, this land is my land ---
\end{itemize}

bernie sanders

And it involved a lot of people. So the kids began to learn about
Russia, and I happened to believe then, and I believe now, that if we're
going to bring peace to the world, we need a lot of cultural exchanges,
we need a lot of youth exchanges. In fact, I recently proposed taking
one-tenth of one percent of the military budget and putting it into
cultural exchanges, which I think is a very good investment.

\begin{itemize}
\item
  archived recording\\
  {[}SINGING{]} Your land, this land is my land. From California to the
  New York islands. From the red wood forests ---
\item
  archived recording (bernie sanders)\\
  I had an experience this last summer, I was invited by the government
  of Nicaragua to attend the sixth anniversary of their revolution. And
  they must have had four or 500,000 people out there listening to
  speeches, and the horrible thought that I had really sunk my stomach,
  was that kids in my own city, young kids, working class kids, might be
  asked by this president to go to Nicaragua to kill and get killed. And
  it was a horrible thought.
\end{itemize}

michael barbaro

Some of these endeavors were relatively bold. At a certain point, you go
to Nicaragua. You end up meeting with the leader of the Sandinistas. And
I --- oh, no. I'm not worried about any ---

bernie sanders

No, I'm just ---

michael barbaro

Oh, you're worried about time?

bernie sanders

Yeah, we're running. Yes, we are running --- how are we doing on time?

michael barbaro

Five more minutes?

bernie sanders

I think we're probably going to have to end it right now.

michael barbaro

Oh. No, no, no. This is not a --- don't end it on this question if
that's the issue.

bernie sanders

Well, you know, the issue is ---

michael barbaro

Trust me, this is not --- all I was going to asking you was, how do
events like that connect to voters in Burlington? In your mind ---

bernie sanders

Good, absolutely. Very good question. Well, I'll tell you why ---

michael barbaro

How does meeting with an international leader like that ---

bernie sanders

I'll tell you why it does. Because I believe we have to empower people.
One of the things we did is, we said to people, speak out on national
and international issues. Yes, the mayor of the city of Burlington can't
determine the defense budget. But if we rally people all over the
country speaking out on these issues, then the members of Congress and
U.S. senators will hear that. So to answer your question, this is just
another mechanism that we had to say to people, you have a voice. Do you
think we should be spending more money on nuclear weapons? Vote on it.
Talk about it. So all of this has to do with empowering people to
understand that in a democracy, they can determine the future.

michael barbaro

Alex, what do you make of that?

alex burns

That is really the essence of the Bernie Sanders approach to politics,
that the most important thing a political leader can do is give voice to
people's deepest concerns and frustrations, and encourage people to give
voice to it themselves. And if there is a gap between what that
political leader is expressing or channeling, and what he can actually
accomplish, it's almost irrelevant, that the act of expression and
engagement is the most important thing. And what happens when you draw
people in on an issue like climate change, or the Reagan
administration's policies in Central America, is they become politically
activated in a way that then transforms politics closer to home at the
local level.

michael barbaro

And Sanders did eventually articulate what you're describing, but not
before he got frustrated with me and seemed to indicate he might end the
interview when I mentioned Nicaragua. What do you think that that's
about?

alex burns

Well first of all this is a really charged moment in his early career
and in American politics. There has been a revolution in Nicaragua.
Daniel Ortega and the Sandinistas are a left wing revolutionary movement
that overthrows a repressive regime. They are seen as dangerous by the
Reagan administration because they are so left wing, and the Republicans
in Washington prop up a brutal right wing militia to fight the
Sandinistas. Bernie Sanders is one of many Americans on the left who get
involved at that point in demonstrating in favor of the Sandinistas, or
against Ronald Reagan. Except Sanders takes it considerably further when
he actually goes to Nicaragua, shakes hands with Ortega himself. This is
a story that someone in Sanders's position probably ought to be able to
explain. Or at least you would think he would feel comfortable
explaining it. And what I find somewhat confounding as a reporter is how
much he resents even the prompt to go into his thinking at the time and
to reflect a little bit on some of the things about his support for the
Sandinistas that may not look as justifiable in retrospect. He doesn't
want to do that. My sense is that, at the heart of it for him, is this
sense that even asking the question is a kind of red baiting, that it
reflects the way the political establishment --- and he very much lumps
the media in with the political establishment, is out to get him. Much
as it was in Burlington, much as he believes it was in the 2016
campaign. This is a guy who in his early days as mayor was described by
a fellow elected official as representing the fungus of socialism. He is
somebody who is very, very sensitive to anything he perceives as the
charge that he is not just a populist, not just very liberal, but this
wildly outside the mainstream dangerous radical. And when you raise
Nicaragua, I do think that's the nerve that it hits.

michael barbaro

So that there's no misunderstanding, those who listen and ask about
Nicaragua, I want to give you a chance to make sure that there's no
confusion for any listener who's casually checking in.

bernie sanders

All right ---

michael barbaro

The question is, was there anything about Daniel Ortega that ---

bernie sanders

Let me just say this.

michael barbaro

You knew at the time that gave you pause.

bernie sanders

Well, what gave me pause was that the United States at that time, as you
may recall. I don't know, do you remember ---

michael barbaro

I was a student.

bernie sanders

Who the president was before Ortega? A dictator named Somoza who was a
dictator, a very bad guy supported by the United States of America. Then
Ortega came to power, the Sandinistas came to power, and the United
States intended to do what it had done in many instances. You are aware
the United States has a habit of overthrowing governments in Latin
America.

michael barbaro

Yes.

bernie sanders

I didn't think that was a good idea. Didn't think it was a good idea
then, and I didn't think it was a good idea now. So we worked against
American intervention. So we went there to say, as part of a national
movement, that the United States should not be involved.

michael barbaro

Right.

bernie sanders

About overthrowing small governments.

michael barbaro

And for the record, in `85, were you aware of any human rights issues or
abuses by Ortega?

bernie sanders

Well, we were aware that this was a very controversial moment, having
taken over from a dictatorship. We were also aware that the United
States at that time was supporting many governments in Latin America who
were much more brutal than Ortega was.

alex burns

What you hear there is such an evasiveness about assessing the Ortega
government on its own merits, that he really wants to talk about his
advocacy around Nicaragua exclusively as a repudiation of Reagan, and
not as an endorsement of what was going on there. And if you look at his
comments and activities at the time, that's not quite right. He was more
explicitly supportive of what the Sandinistas were doing then just going
there as a sort of anti-interventionist advocate. But in fairness to
Sanders, this was not a fringe position at the time. Support for the
Sandinistas has obviously not necessarily aged as well as Bernie Sanders
might have expected it to politically, and that's I think where you hear
his real discomfort talking about it in the context of this campaign.

michael barbaro

But in taking this trip and talking about it the way he does, he is
living his creed, essentially.

alex burns

Exactly. It is using all the levers of his power and public influence as
an elected official to weigh in on this subject that is about as
distant, literally, from Burlington as you can get.

{[}music{]}

michael barbaro

As this strategy is being deployed, you're winning your fourth term as
mayor. And you go on, successfully this time, to run for statewide
office. The House, the Senate, and then, of course, you run for
president 2016, now again in 2020. In each of these campaigns and each
of these moments, you're building larger and more powerful coalitions of
voters. And given that history and your success in doing that, what do
you think is the big lesson from this early phase? Why do you think it
is that when we went to Alex Burns and asked him this question, and he
said, you have to go back to `81, you have to go back to Burlington to
understand Senator Sanders and this campaign and this moment. Why is he
right, why do we have to go back to that race and that moment?

bernie sanders

Politics in America has been very much from the top on down. You still
read articles in The New York Times where wealthy donors gathered today
at a hotel to express concern about the Democratic candidates. Who cares
about wealthy donors? We have over 1.1 million Americans who have made
donations to our campaign. And they're not wealthy, these are working
class people. They're teachers, they're workers at Amazon, they're
workers at Walmart. What I believe then and what I believe now, the way
you transform society is from the bottom on up. You talk about issues
that are relevant to working people, issues that are relevant to low
income people, issues that are relevant to young people, and you grow
the voter base. So what I pointed out to you, maybe the most important
thing that we did, is from 1979 to 1983, we doubled, doubled the number
of people voting. And what we're trying to do in this campaign right now
---

michael barbaro

Is parallel.

bernie sanders

Is to significantly increase the voter turnout by talking to people who
don't vote. In Burlington, what happened is low income and working class
people saw that government could work for them. And they said, oh my
God, I never knew that. My kids now have a program, we have an after
school program they didn't have. We have a child care center we didn't
have. Our streets are getting paved, snow is getting moved. I didn't
know that --- we're going to go out, we're going to support Bernie, and
we're going to support the candidates that he wants for Board of
Aldermen. And now what we have to do in this country, which has one of
the lowest voter turnout rates of any major country on earth, is to
reach out to those working people, reach out to those young people. And
when they start participating in the political process, that is the
political revolution.

michael barbaro

If you become president, the question will be, sure, you talked about
national and international issues when you were mayor, but there was no
real expectation that you could change the course of events as a single
mayor of a town in Burlington. If you become president, that expectation
will be real and urgent and present. So are we to understand that if you
run into political obstacles as president, that your strategy will be to
call upon the expanded electorate that you have created and turned that
into a political force that you would then, all of what you did in
Burlington with the Board of Aldermen?

bernie sanders

Yes. Well, it's not an if. That is exactly what is going to happen. When
we talk about the need to join the rest of the industrialized world and
guarantee health care to every person through a Medicare for all,
single-payer program, the only way that's going to happen is when
millions of people stand up and take on the insurance companies and the
drug companies. When we talk about transforming our energy system to
save the planet from the devastation, absolute devastation of the global
crisis regarding climate change, the only way that happens is when
millions of people stand up to take on the fossil fuel industry. So on
all of the issues we are talking about, that's what the political
revolution is about. It's saying that we're going to mobilize millions
of people to stand up for an agenda that works for working families. And
when they do that, there will be no stopping them. We will be able to
create a government and an economy that works for all, not just the one
percent.

michael barbaro

Alex, what do you think of that theory?

alex burns

Well, the coalition that he's trying to build as a presidential
candidate is such an echo of what he accomplished in Burlington. And his
theory is, they are going to show up for him in a way they wouldn't show
up for any other candidate because he is speaking to their concerns
directly. And Sanders has plenty of reason to expect that might really
be the case. He has charted this remarkable ascent as a national
political figure on the strength of this coalition and mainstreamed a
set of socialist and quasi-socialist ideas that were seen as really
outside the mainstream when he started campaigning on them decades ago,
into the center of one of the country's two major political parties. The
question for someone like Bernie Sanders is, does that theory work in a
general election on a national scale? Can you really bring in that many
new people into the political process where there already are a whole
lot of people who vote, and who have pretty vested interests in the
context of an American election? Can you speak in the way he does to the
concerns of working class people without alienating millions of people
who already vote for the Democratic Party, and don't necessarily share
that world view? And could you, as the president, use that exact same
playbook, that exact same coalition to master Washington and break a
Republican Senate in the same way that he transformed Burlington and
broke a city council?

{[}music{]}

michael barbaro

Alex, thank you.

alex burns

Thank you.

bernie sanders

Thank you very much.

michael barbaro

I wish you the best. Thank you very much for your time.

{[}music{]}

michael barbaro

``The Daily'' is made by Theo Balcomb, Andy Mills, Lisa Tobin, Rachel
Quester, Lynsea Garrison, Annie Brown, Clare Toeniskoetter, Paige
Cowett, Michael Simon Johnson, Brad Fisher, Larissa Anderson, Wendy
Dorr, Chris Wood, Jessica Cheung, Alexandra Leigh Young, Jonathan Wolfe,
Lisa Chow, Eric Krupke, Marc Georges, Luke Vander Ploeg, Adizah Eghan,
Kelly Prime, Julia Longoria, Sindhu Gnanasambandan, Jazmín Aguilera,
M.J. Davis Lin, Austin Mitchell, Sayre Quevedo, Monika Evstatieva, Neena
Pathak, Dave Shaw and Dan Powell. Our theme music is by Jim Brunberg and
Ben Landsverk of Wonderly. Special thanks to Sam Dolnick, Mikayla
Bouchard, Stella Tan, Lauren Jackson and Julia Simon.

{[}music{]}

That's it for ``The Daily.'' I'm Michael Barbaro. See you on Monday.

\includegraphics{https://static01.graylady3jvrrxbe.onion/images/2019/09/06/us/00sandershealthcare/merlin_158252829_740a0b53-d9ba-42b9-8d0b-8ca28a32c9b1-articleLarge.jpg?quality=75\&auto=webp\&disable=upscale}

\hypertarget{four-key-moments-from-our-interview-with-the-senator}{%
\subsection{Four key moments from our interview with the
senator}\label{four-key-moments-from-our-interview-with-the-senator}}

\hypertarget{his-arrival-in-vermont-and-early-involvement-with-the-liberty-union}{%
\subsubsection{His arrival in Vermont and early involvement with the
Liberty
Union}\label{his-arrival-in-vermont-and-early-involvement-with-the-liberty-union}}

Although Mr. Sanders is best-known for his association with Vermont, the
schoolhouse for his early career in politics, he spent his childhood in
a rent-controlled apartment in the Flatbush neighborhood of Brooklyn.
``Sensitivity to class was embedded in me then quite deeply,''
\href{https://www.nytimes3xbfgragh.onion/2007/01/21/magazine/21Sanders.t.html}{Mr.
Sanders has said of this time}.

The factors that took him from Brooklyn to Vermont reflect an insistence
on revolution that has colored his career.

Born to Polish immigrants and raised in precarious financial
circumstances, Mr. Sanders had early inclinations toward socialism that
were cemented with the
\href{https://www.nytimes3xbfgragh.onion/2019/09/09/us/politics/bernie-sanders-health-care.html}{death
of his mother}. Caring for her during his college years exposed him
intimately to gaps in the American health care system.

Mr. Sanders went in search of a place that could nurture his nascent
political ideology, visiting a
\href{https://www.nytimes3xbfgragh.onion/2016/02/06/us/politics/bernie-sanders-kibbutz.html}{socialist
kibbutz in Israel} and ultimately landing with the peaceniks of rural
Vermont.

``I was doing some writing. I was banging nails, doing a little bit of
carpentry work,'' Mr. Sanders said of this time. He freelanced for an
alternative newspaper, The Vermont Freeman,
\href{https://nytimes3xbfgragh.onion/2015/07/04/us/politics/bernie-sanderss-revolutionary-roots-were-nurtured-in-60s-vermont.html}{writing
articles} like ``The Revolution Is Life Versus Death'' while making film
strips about a socialist he admired: Eugene Debs. Mr. Debs was the
``Socialist Party candidate for president six times,'' Mr. Sanders
noted. ``You know, somebody I admired a whole lot.''

Mr. Sanders and Mr. Debs have something in common: resilience in the
face of political failure. In Vermont, Mr. Sanders became involved with
a fringe political party, the Liberty Union, that sought to champion
industrial nationalization and opposition to the Vietnam War. Mr.
Sanders ran as a Liberty Union candidate in four state elections,
receiving less than 5 percent of the vote each time.

His promise to address wealth inequality and his condemnation of
billionaires began to resonate with working-class people upstate. Soon,
he had his eyes fixed on the 1981 race for mayor of Burlington, the
state's largest city.

Image

Mr. Sanders was called a ``fluke'' when he was elected mayor of
Burlington in 1981.~Credit...Donna Light/Associated Press

\hypertarget{winning-his-first-election--by-10-votes}{%
\subsubsection{Winning his first election --- by 10
votes}\label{winning-his-first-election--by-10-votes}}

``You would literally not believe if I told you how little we knew about
politics,'' Mr. Sanders said of his first race for mayor.

``I mean real politics,'' he said. ``It's one thing to run for statewide
office knowing you're not going to win and get on a radio show and talk
about issues, which I could do. But the nitty-gritty of politics, you
know.''

Mr. Sanders's strategy was to mobilize grass-roots support in the
working-class districts of Burlington --- specifically people in
``low-income housing projects where people were getting a raw deal from
the city,'' he said.

In doing so, Mr. Sanders generated a higher turnout than most mayoral
races commanded. After a recount, Mr. Sanders won by 10 votes, beating a
10-year incumbent and roiling establishment politicians in the city.

``Lessons of this moment is that winning politics is grass-roots
politics.'' Mr. Sanders said ``that winning politics is developing
coalitions of working people, of low-income people, of women, of
environmentalists.''

Image

Mr. Sanders taking the oath of office to become Burlington's
mayor.Credit...Donna Light/Associated Press

\hypertarget{a-parallel-city-government}{%
\subsubsection{A parallel city
government}\label{a-parallel-city-government}}

Mr. Sanders faced the limits of his political outrage during his first
term as mayor, which became an education in coalition building. He was
viewed by the Board of Aldermen, Burlington's version of a city council,
as ``an accident that should never have happened,'' he said.

``Bernie Sanders is a fluke,'' he said. ``That was the word they used.''

Mr. Sanders had to figure out how to accomplish his agenda despite
opposition from Democrats and Republicans. After the board fired his
secretary, Mr. Sanders got the message that his appointees would not be
welcome in the city government. ``It was a brutal year,'' he said. ``So
what we had to do was literally form a parallel city government.''

He gathered volunteers to staff his informal team of unpaid appointees.
They started ``neighborhood planning associations,'' allocating city
funds to neighborhood councils to spend at their discretion. In doing
so, they cultivated a widespread sense of antagonism toward the board.

By knocking doors in a freezing Burlington winter, Mr. Sanders nearly
doubled voter turnout in the board election the coming year. Turnout
``was just off the charts,'' he said.

Unseating board members in working-class districts gave him the support
his agenda needed, enabling his rise as a major political figure in the
state.

Image

Mr. Sanders speaking to a voter during a 1986 campaign stop when he was
running for governor.Credit...Toby Talbot/Associated Press

\hypertarget{going-national}{%
\subsubsection{Going national}\label{going-national}}

Mr. Sanders began to connect his structural grievances with national
politics to his constituency~--- working to convince local voters that
the actions of far-off politicians in far-off places should matter to
them.

In doing so, he managed to fix Burlington's pot holes and plow the
streets while also
\href{https://www.nytimes3xbfgragh.onion/2007/01/21/magazine/21Sanders.t.html}{establishing
relations} with the Sandinistas in Nicaragua. His rejection of American
intervention in Latin America resulted in his controversial support for
the Nicaraguan leftist leader Daniel Ortega.

Asked about this stance, Mr. Sanders said he believed at the time that
the United States should not be involved with ``overthrowing small
governments.''

``We were aware that this was a very controversial moment,'' he added.
``We were also aware that the United States at that time was supporting
many governments in Latin America who were much more brutal than Ortega
was.''

Mr. Sanders says that his campaign against intervention was relevant to
Burlington. ``If we were spending a whole lot of money in Washington
under Reagan --- investing in military spending or we're giving tax
breaks to the rich --- that impacted the city of Burlington,'' he said.

Today, his insistence that the global affects the local still forms the
bedrock of his presidential platform --- one that is built on
overhauling health care, tax policy and the national budget. He says
that if Washington is ``spending this money on the military or they're
busy invading another country or whatever they're doing, we should be
speaking up on those issues.''

``All of this,'' he said, ``has to do with empowering people to
understand that in a democracy, they can determine the future.''

\textbf{On today's episode:}

\begin{itemize}
\item
  Bernie Sanders, the Vermont senator and candidate for the 2020
  Democratic presidential nomination.
\item
  \href{https://www.nytimes3xbfgragh.onion/by/alexander-burns}{Alexander
  Burns}, who covers national politics for The New York Times.
\end{itemize}

\includegraphics{https://static01.graylady3jvrrxbe.onion/images/2019/03/18/us/politics/xxvid-bernie-2020/xxvid-bernie-2020-videoSixteenByNine3000-v4.jpg}

\textbf{Background reading:}

\begin{itemize}
\item
  Mr. Sanders has staked his presidential campaign, and much of his
  political legacy, on transforming health care in America. His mother's
  illness and a trip he made to study the Canadian system
  \href{https://www.nytimes3xbfgragh.onion/2019/09/09/us/politics/bernie-sanders-health-care.html}{help
  explain why}.
\item
  We asked 21 candidates the same 18 questions.
  \href{https://www.nytimes3xbfgragh.onion/interactive/2019/us/politics/bernie-sanders-2020-campaign.html}{Hear
  Mr. Sanders's answers}.
\end{itemize}

\emph{Tune in, and tell us what you think. Email us at}
\href{mailto:thedaily@NYTimes.com}{\emph{thedaily@NYTimes.com}}\emph{.
Follow Michael Barbaro on Twitter:}
\href{https://twitter.com/mikiebarb}{\emph{@mikiebarb}}\emph{. And if
you're interested in advertising with ``The Daily,'' write to us at}
\href{mailto:thedaily-ads@NYTimes.com}{\emph{thedaily-ads@NYTimes.com}}\emph{.}

``The Daily'' is made by Theo Balcomb, Andy Mills, Lisa Tobin, Rachel
Quester, Lynsea Garrison, Annie Brown, Clare Toeniskoetter, Paige
Cowett, Michael Simon Johnson, Brad Fisher, Larissa Anderson, Wendy
Dorr, Chris Wood, Jessica Cheung, Alexandra Leigh Young, Jonathan Wolfe,
Lisa Chow, Eric Krupke, Marc Georges, Luke Vander Ploeg, Adizah Eghan,
Kelly Prime, Julia Longoria, Sindhu Gnanasambandan, Jazmín Aguilera,
M.J. Davis Lin, Dan Powell, Austin Mitchell, Sayre Quevedo, Monika
Evstatieva and Neena Pathak. Our theme music is by Jim Brunberg and Ben
Landsverk of Wonderly. Special thanks to Sam Dolnick, Mikayla Bouchard,
Stella Tan, Julia Simon and Lauren Jackson.

\hypertarget{our-2020-election-guide}{%
\section{Our 2020 Election Guide}\label{our-2020-election-guide}}

Updated ~Sept. 7, 2020

\begin{center}\rule{0.5\linewidth}{\linethickness}\end{center}

\begin{itemize}
\item ~
  \hypertarget{the-latest}{%
  \subsection{The Latest}\label{the-latest}}

  \begin{itemize}
  \item
    The unofficial Labor Day kickoff to the fall presidential campaign
    centered on Pennsylvania and Wisconsin,
    \href{https://www.nytimes3xbfgragh.onion/2020/09/07/us/politics/wisconsin-biden-harris-trump-pence.html?action=click\&pgtype=Article\&state=default\&region=BELOW_MAIN_CONTENT\&context=storylines_guide}{two
    pivotal states for both President Trump and Joseph R. Biden Jr}.
  \end{itemize}
\item ~
  \hypertarget{how-to-win-270}{%
  \subsection{How to Win 270}\label{how-to-win-270}}

  \begin{itemize}
  \item
    Joe Biden and Donald Trump need 270 electoral votes to reach the
    White House. Try building
    \href{https://www.nytimes3xbfgragh.onion/interactive/2020/us/elections/election-states-biden-trump.html?action=click\&pgtype=Article\&state=default\&region=BELOW_MAIN_CONTENT\&context=storylines_guide}{your
    own coalition of battleground states}~to see potential outcomes.
  \end{itemize}
\item ~
  \hypertarget{voting-by-mail}{%
  \subsection{Voting by Mail}\label{voting-by-mail}}

  \begin{itemize}
  \item
    Will you have enough time to vote by mail in your state? Yes, but
    it's risky to procrastinate.
    \href{https://www.nytimes3xbfgragh.onion/interactive/2020/08/31/us/politics/vote-by-mail-deadlines.html?action=click\&pgtype=Article\&state=default\&region=BELOW_MAIN_CONTENT\&context=storylines_guide}{Check
    your state's deadline.}
  \item
    \href{https://www.nytimes3xbfgragh.onion/interactive/2020/us/elections/joe-biden.html?action=click\&pgtype=Article\&state=default\&region=BELOW_MAIN_CONTENT\&context=storylines_guide}{}

    \hypertarget{joe-biden}{%
    \section{Joe Biden}\label{joe-biden}}

    \hypertarget{democrat}{%
    \subsection{Democrat}\label{democrat}}

    \href{https://www.nytimes3xbfgragh.onion/interactive/2020/us/elections/donald-trump.html?action=click\&pgtype=Article\&state=default\&region=BELOW_MAIN_CONTENT\&context=storylines_guide}{}

    \hypertarget{donald-trump}{%
    \section{Donald Trump}\label{donald-trump}}

    \hypertarget{republican}{%
    \subsection{Republican}\label{republican}}
  \end{itemize}
\item
  \hypertarget{keep-up-with-our-coverage}{%
  \subsection{Keep Up With Our
  Coverage}\label{keep-up-with-our-coverage}}

  \begin{itemize}
  \item
    Get an
    \href{https://www.nytimes3xbfgragh.onion/newsletters/politics?action=click\&pgtype=Article\&state=default\&region=BELOW_MAIN_CONTENT\&context=storylines_guide}{email}~recapping
    the day's news
  \item
    Download our mobile app on
    \href{https://apps.apple.com/us/app/nytimes/id284862083?ls=1\&mat_click_id=5c79ae7455014fd1bd66b5610c05b8f2-20191112-16948\&referrer=mat_click_id\%3D5c79ae7455014fd1bd66b5610c05b8f2-20191112-16948\%26link_click_id\%3D722930677036718082}{iOS}~and
    \href{http://a.localytics.com/android?id=com.nytimes.android\&referrer=utm_source\%3Dother_nyt_mobile_web\%26utm_medium\%3DWeb\%2520page\%26utm_term\%3DGeneral\%2520Mobile\%2520Page\%26utm_campaign\%3DNYT\%2520Mobile\%2520General\%2520Page}{Android}~and
    turn on Breaking News and Politics alerts
  \end{itemize}
\end{itemize}

Advertisement

\protect\hyperlink{after-bottom}{Continue reading the main story}

\hypertarget{site-index}{%
\subsection{Site Index}\label{site-index}}

\hypertarget{site-information-navigation}{%
\subsection{Site Information
Navigation}\label{site-information-navigation}}

\begin{itemize}
\tightlist
\item
  \href{https://help.nytimes3xbfgragh.onion/hc/en-us/articles/115014792127-Copyright-notice}{©~2020~The
  New York Times Company}
\end{itemize}

\begin{itemize}
\tightlist
\item
  \href{https://www.nytco.com/}{NYTCo}
\item
  \href{https://help.nytimes3xbfgragh.onion/hc/en-us/articles/115015385887-Contact-Us}{Contact
  Us}
\item
  \href{https://www.nytco.com/careers/}{Work with us}
\item
  \href{https://nytmediakit.com/}{Advertise}
\item
  \href{http://www.tbrandstudio.com/}{T Brand Studio}
\item
  \href{https://www.nytimes3xbfgragh.onion/privacy/cookie-policy\#how-do-i-manage-trackers}{Your
  Ad Choices}
\item
  \href{https://www.nytimes3xbfgragh.onion/privacy}{Privacy}
\item
  \href{https://help.nytimes3xbfgragh.onion/hc/en-us/articles/115014893428-Terms-of-service}{Terms
  of Service}
\item
  \href{https://help.nytimes3xbfgragh.onion/hc/en-us/articles/115014893968-Terms-of-sale}{Terms
  of Sale}
\item
  \href{https://spiderbites.nytimes3xbfgragh.onion}{Site Map}
\item
  \href{https://help.nytimes3xbfgragh.onion/hc/en-us}{Help}
\item
  \href{https://www.nytimes3xbfgragh.onion/subscription?campaignId=37WXW}{Subscriptions}
\end{itemize}
