Sections

SEARCH

\protect\hyperlink{site-content}{Skip to
content}\protect\hyperlink{site-index}{Skip to site index}

\href{https://myaccount.nytimes3xbfgragh.onion/auth/login?response_type=cookie\&client_id=vi}{}

\href{https://www.nytimes3xbfgragh.onion/section/todayspaper}{Today's
Paper}

How Today's Queer Artists Are Revising History

\url{https://nyti.ms/2r3S3BE}

\begin{itemize}
\item
\item
\item
\item
\item
\item
\end{itemize}

Advertisement

\protect\hyperlink{after-top}{Continue reading the main story}

Supported by

\protect\hyperlink{after-sponsor}{Continue reading the main story}

\includegraphics{https://static01.graylady3jvrrxbe.onion/images/2019/12/04/t-magazine/04tmag-gay-03/04tmag-gay-03-superJumbo.png}

\hypertarget{how-todays-queer-artists-are-revising-history}{%
\section{How Today's Queer Artists Are Revising
History}\label{how-todays-queer-artists-are-revising-history}}

By revisiting and refuting the cultural history of the West, this group
is using time as its primary medium, looking backward to inform a
different kind of gay future.

By \href{https://www.nytimes3xbfgragh.onion/by/jesse-green}{Jesse Green}

Dec. 4, 2019

\begin{center}\rule{0.5\linewidth}{\linethickness}\end{center}

\hypertarget{1-habitations-in-history}{%
\subsubsection{1. Habitations in
History}\label{1-habitations-in-history}}

Two women and their baby stare straight at us from the American past.
Jessie Evans-Whinery, as she is identified in the caption, looks a lot
like Elvis, with the tousled pompadour and watchful eyes of a rebel in
repose. Her dowdier wife, Edith Evans-Whinery, leans toward her in
affection and probably exhaustion; she's clearly the homemaker in this
homesteading family in Pie Town, N.M., circa 1940. It's she who holds
the dirty baby with the quizzical expression and a finger in his ear. Is
he confused?

At any rate, I was. Were there lesbians living openly like this, I
wondered, in hardscrabble rural outposts like Pie Town some 80 years
ago? The photographs I was looking at certainly made it seem so, because
the Evans-Whinerys are not the only female couple depicted. There are
also Fae and Doris Caudill, with their two children, ``eating dinner in
their dugout,'' as the caption tells us. Like Edith, Doris looks weary
but fulfilled in her apron, passing the biscuits. And why not? Fae,
despite that name, looks like the winner at a butch realness contest.

Elsewhere in the series, an unnamed couple --- a female couple, once you
zoom in --- walks down the dirt road of Main Street together, a toddler
alongside and a babe in arms. Likewise posing outdoors are the Norrises,
sunbaked founders of the town: Jean in a cowboy hat, Virginia in straw
with a modest ribbon. Their neighbor Ann Hesse looks like the Marlboro
man's sister. Nell Leathers shoots at hawks.

But it's the ``swing your partner square dance'' that startled me into
understanding. What look at first like boys --- in flat-front pants and
white shirts, their hands superior to their partners' as they do-si-do
--- are upon closer inspection girls, their hair shaved close on the
sides but otherwise possessed of soft chins and slight busts.

\includegraphics{https://static01.graylady3jvrrxbe.onion/images/2019/12/05/t-magazine/05tmag-revisionists-slide-TUME/05tmag-revisionists-slide-TUME-articleLarge.jpg?quality=75\&auto=webp\&disable=upscale}

Now you fully realize that, despite a few boy babies, there are no men
in Pie Town. There are only women, living their lives, singly, in
couples, in families, in community. What heaven, what haven, what mirage
is this?

It's Photoshop.

The original images, taken by the American photojournalist Russell Lee
for the Farm Security Administration in 1940, depicted Pie Town as it
actually was, its traditional gender roles and pairings strictly
observed. According to Lee's caption, Jessie and Edith Evans-Whinery
were, in reality, just Jack Whinery and his unnamed wife. The Caudills
weren't Fae and Doris but Faro and his anonymous missus. Dozens of other
photographs have been manipulated as well, pushing existing women closer
together to suggest coupledom, or creating wives where husbands had been
by removing Adam's apples, lantern jaws and five-o'clock shadows.

``My Pie Town,'' as the series of altered photographs is called, is the
work of the American artist
\href{https://www.debbiegrossman.com/}{Debbie Grossman}: an attempt, she
has said, ``to make the history I wish was real.'' She describes the
task as ``an intimate act'' at the pixel level --- a phrase suggesting
both the caressing of sculpture and the twiddling of sex: ``Manipulating
the touch of one woman's hand on another's shoulder is a way for me to
access and merge my desire with figures which would have otherwise
remained frozen in time.''

I, too, am frozen in time when I look at Grossman's photographs,
especially in conjunction with Lee's. Together they suggest not an
invented false history but a secret real one, as if the queerness had
always been there, a kind of digital potential waiting to be released.
Part of what freezes me is the calm beauty of the result; even Nell
Leathers with her rifle is relaxed, her principal concern being
chicken-stealing hawks. (There are no homophobes in Pie Town.) Another
part is the simplicity of the intervention that produced such a
disproportionately huge, and in some ways violent, result. For if ``My
Pie Town'' is an act of repair, it also, at times, verges on vengeance,
as each man is forcibly, digitally resexed. (In one case, a husband is
erased entirely.) The women, too, are resexed in a way: That is, they
are given their sex back. As lesbians, they somehow look so much
happier, like they're getting some on the regular.

Image

Accompanying this article are works by artists who are reimagining queer
history --- and art history, as well --- by creating gay, lesbian and
trans narratives that look backward, reinserting their lives and stories
into the past. Some, like the artist Martine Gutierrez and the
multidisciplinary duo McDermott \& McGough, are exploring how previous
eras might have looked and functioned had L.G.B.T.Q. people been more
fully recognized, while others, like the photographers Catherine Opie
and John Dugdale, are referencing bygone artistic codes and methods (for
Opie, Dutch Golden Age portraiture; for Dugdale, 19th-century cyanotype
processing, as shown here in 1999's ``I Could Not See to See'') to
assert a visual legacy and a sense of dignity.Credit...John Dugdale, ``I
Could Not See to See,'' 1999, Morton Street, N.Y.C., courtesy of the
artist

The porousness of time in Grossman's work --- the way the year 2010,
when she completed the series, infiltrates and ultimately reforms 1940
--- in some ways mirrors that of all art, which is necessarily
retrospective the moment it is made. But it is especially reflective of
recent queer art, so much of which seems to start in the past, not just
end up there. In fiction, drama, movies, painting and photography, gay
men and lesbians, and others on the queer spectrum, have been building
habitations in history, whether near or distant, sometimes in search of
a more congenial domesticity and sometimes establishing more aggressive
encampments.

What are they doing there? And why now? The canonical works of queer art
--- Radclyffe Hall's novel ``The Well of Loneliness'' in 1928, for
instance, or
\href{https://www.nytimes3xbfgragh.onion/2016/03/04/t-magazine/my-10-favorite-books-larry-kramer.html}{Larry
Kramer}'s ``Faggots,'' 50 years later --- have most often been
contemporary, set or styled in the artist's own day. Not without plenty
of exceptions, of course:
\href{https://www.nytimes3xbfgragh.onion/1987/12/19/obituaries/marguerite-yourcenar-writer-and-scholar-dies.html}{Marguerite
Yourcenar}'s ``Memoirs of Hadrian'' takes place about 18 centuries
before its 1951 publication;
\href{https://www.nytimes3xbfgragh.onion/topic/person/virginia-woolf}{Virginia
Woolf}'s ``Orlando: A Biography'' (1928) imagines an Elizabethan poet
who changes sex and sexual orientation as fast as a clown fish over the
ensuing 300 years. And the painter Paul Cadmus, with his ancient egg
tempera technique, might as well have been working in the early
Renaissance instead of the 20th century, no matter that what he produced
with his yolks and fine brush strokes was gay beefcake bordering on
pornography. For that matter, the same could be said of Michelangelo.

But on the whole, queer art, which fully emerged from the closet in the
1960s and 1970s --- around the same time people in great numbers did ---
has mostly concerned itself with its own moment, as if to say, ``Here I
am.'' That approach continues because, after all, each new
microgeneration of gay people born to straight parents in a straight
world must create itself and its aesthetics from scratch. For better or
worse, coming-out novels thus remain an evergreen genre, a kind of gay
bar exam every queer writer takes. (I should know; I
\href{https://www.barnesandnoble.com/w/o-beautiful-jesse-green/1000128265?ean=9780345374707}{wrote
one}.) And each subsequent chapter in the emergence of gay art has
announced itself as urgently new. Works about H.I.V. and AIDS in the
1980s and 1990s had the immediacy of reportage; from the 2000s onward,
narratives about marriage and parenting, as well as the backlash against
gay domesticity, have felt like real-time anthropology. If that line of
contemporaneous comment has been petering out, at least as a phenomenon
of privilege, it may be because there is nowhere left to take it.

Yet with
\href{https://lens.blogs.nytimes3xbfgragh.onion/2013/06/13/examining-identity-one-gender-at-a-time/}{works
like} ``My Pie Town,'' another approach has been emerging in tandem. You
can see it in the British writer
\href{https://www.nytimes3xbfgragh.onion/2015/10/16/t-magazine/my-10-favorite-books-sarah-waters.html}{Sarah
Waters}'s novels; the French filmmaker Céline Sciamma's new movie,
``Portrait of a Lady on Fire''; the American playwright Matthew Lopez's
two-part drama,
``\href{https://www.nytimes3xbfgragh.onion/2019/10/25/theater/the-inheritance-broadway.html}{The
Inheritance}'' (which opened on Broadway last month); and in art by
\href{https://www.nytimes3xbfgragh.onion/2018/06/18/t-magazine/glenn-ligon-adrian-piper-art.html}{Glenn
Ligon},
\href{https://www.nytimes3xbfgragh.onion/2019/10/02/t-magazine/catherine-opie.html}{Catherine
Opie} and
\href{https://www.nytimes3xbfgragh.onion/2019/09/16/t-magazine/peter-mcgough.html}{McDermott
\& McGough}, to name just a sample from various disciplines. The
watchcry for these works isn't so much ``Here I am'' as ``There we
were.'' More trenchantly, they sometimes ask how the two ideas are, or
aren't, related. What is the queer past for?

Image

Debbie Grossman's ``Jessie Evans-Whinery, Homesteader, With Her Wife
Edith Evans-Whinery and Their Baby'' (2010), from her ``My Pie Town''
series.Credit...Inkjet print, the Metropolitan Museum of Art, purchase,
Charina Foundation Inc., gift © Debbie Grossman, courtesy of Julie Saul
Projects, N.Y.

\hypertarget{2-publishable-but-worth-it}{%
\subsubsection{2. ``Publishable, but Worth
It?''}\label{2-publishable-but-worth-it}}

A history lesson is one answer. Though we may think of young queers as
knowledgeable and engaged, their knowledge is often ahistoric, their
engagement thin. I am not so troubled by gay men in their 20s who look
at me blankly if I mention
\href{https://www.nytimes3xbfgragh.onion/topic/person/judy-garland}{Judy
Garland}; every demographic deserves its own diva. But few seem to know
who Kramer is, either. And the idea that homosexuality began in 1969 is
disturbingly, if understandably, prevalent among people born in the
decades after. Some members of the cast of ``The Inheritance,'' Lopez's
six-and-a-half-hour doubleheader about the AIDS generation and its
antecedents and successors, are so young that the Broadway production
--- after a world premiere in London last year --- asked that
battle-scarred survivor
\href{https://www.nytimes3xbfgragh.onion/2018/06/26/t-magazine/edmund-white-queer-writing-book-influence.html}{Edmund
White} to attend a rehearsal and answer questions as if to provide proof
that the gay past exists.

White's own books are mostly set in the present, or the perpetual
present of autobiography; only now are they becoming history, some four
decades after he began publishing them. But other queer artists head for
history right from the start, as if deputized to repair a pothole there.
It's a large pothole: Lovers, heroes, victories and villains, not to
mention the simple pleasure of enjoyable clichés, have all disappeared
into it. To rescue them is to back-form the dignity that comes with
having a long story to share and the art that goes with it. It's no
accident that the past-mining operation that has always existed as a
subgenre of queer art is intensifying now, just as the history of other
disenfranchised groups is finding wider audiences. (I'd estimate that
I've seen more plays by black playwrights in the past two years than in
the previous five.) But if the truth of some marginalized Americans is
finally beginning to be dignified in museums and taught in schools, it's
a spotty revolution; gays and lesbians, let alone transgender people,
still lack the cultural affirmation that gets you a chapter in the
curriculum and a building on the National Mall. Queer artists have had
to find workarounds, in part by expanding the idea that tradition is
always something handed down like a gift from the powerful or, like an
heirloom, from parent to child.

It's also the case that in today's so-called cancel culture, purely
aesthetic gestures are suspect. Even wit must put on the drag of
politics. Framing their goals in civic terms --- the granting of full
citizenship in a democracy of ideas --- queer artists who might once
have painted merely gorgeous portraits, or goosed the genre of domestic
comedy, more often demonstrate their bona fides by shaping their art as
a form of moral instruction, focusing on the responsibility to honor our
past so we can credibly reshape our present.

Image

Wardell Milan's ``Sunday, Sitting on the Bank of Butterfly Meadow''
(2013).Credit...Digital C-print photograph, courtesy of the artist

Lopez, who was born in 1977, does that implicitly in ``The
Inheritance,'' acknowledging a debt not only to
\href{https://www.nytimes3xbfgragh.onion/topic/person/tony-kushner}{Tony
Kushner}'s
``\href{https://www.nytimes3xbfgragh.onion/2018/03/07/theater/tony-kushner-angels-in-america-broadway.html}{Angels
in America}'' --- a similarly big-boned lyric bombshell, written from
within the epidemic in the late 1980s and early 1990s --- but to E.M.
Forster's ``Howards End,'' published in 1910. The closeted Forster is
even a character in ``The Inheritance,'' looking with awe and no little
concern at his hotheaded progeny --- a sampling of contemporary gay men
in a muddle --- as they make the same blunders his parallel ``Howards
End'' characters made then: ignoring history, trying to manhandle the
future, acting from selfishness and self-delusion.

Of course, Forster's characters weren't queer in the modern sense. They
couldn't be; ``Howards End'' was published just 15 years after
\href{https://www.nytimes3xbfgragh.onion/topic/person/oscar-wilde}{Oscar
Wilde} was convicted of ``gross indecency'' and sentenced to two years'
hard labor in prison. (The British law that criminalized him was not
fully repealed until 2004.) Writers in that time had to work the inverse
of Grossman's ``My Pie Town'' magic; in Forster's published novels, it
was homosexuality that was erased, or transformed into heterosexuality
by an internal Photoshop. The same-sex singles and couples of his
imagination, some based on actual gay men he knew, were replaced with
straight ones --- and yet gay people ever since have been able to scent
them out. After all, discerning queerness in ambiguous texts is part of
our adaptive armamentarium, honed on
\href{https://www.nytimes3xbfgragh.onion/2017/05/15/t-magazine/william-friedkin-marcel-proust.html}{Proust},
\href{https://www.nytimes3xbfgragh.onion/topic/person/emily-dickinson}{Dickinson}
and
\href{https://www.nytimes3xbfgragh.onion/topic/person/william-shakespeare}{Shakespeare}.

Forster did dare one uncloseted novel, which becomes the subject of an
argument in ``The Inheritance.'' Written a few years after ``Howards
End'' --- and revised over the decades but left unpublished at the
author's instruction until after his death in 1970 --- ``Maurice'' is
shocking, not so much for its forthright description of gay love as for
its happy ending. (Maurice, a stockbroker, and Alec, a gamekeeper,
decide to live more or less openly as a couple, despite their class
differences and the danger involved.) But what is likely a happy ending
for them was not one for Forster; the plaintive note he attached to the
manuscript, which remained in a drawer, asked: ``Publishable, but worth
it?''

\includegraphics{https://static01.graylady3jvrrxbe.onion/images/2019/12/06/t-magazine/06tmag-revisionsts-bts/06tmag-revisionsts-bts-superJumbo.jpg}

He was not concerned about a blow to his finances but a blow to his
reputation. Though he outlived Stonewall, he could not have foreseen
that, far from a pariah, he would become a revered gay eminence in a
play 50 years later. Like
\href{https://www.nytimes3xbfgragh.onion/topic/person/matthew-shepard}{Matthew
Shepard}, the gay martyr mourned in
``\href{https://www.nytimes3xbfgragh.onion/2009/08/04/theater/04theater.html}{The
Laramie Project}'' in 2000, and Charlotte von Mahlsdorf, the transgender
subject of Doug Wright's ``I Am My Own Wife,'' which won a Pulitzer
Prize in 2004, Forster has proved to be grist for the queer-hero mill,
which locates in the gay equivalent of the biblical past likely
patriarchs and matriarchs for a community that lacks them. True, he
might have rejected the designation bestowed on him in ``The
Inheritance,'' while also clucking over the physiques of the cast and
their frequent balletic sex arias. But surely he would be flattered to
find them wrangling over, if also rewriting, his bequest to them. Though
one character all but calls him a coward, saying he could have made a
bigger difference, even saved lives, if he'd not been closeted, others,
obviously speaking for Lopez, get the last word. ``You are essential to
our story,'' says one of the young men. ``There's still so much we don't
know,'' says another.

If the men are making Forster an honorary member of their not-so-merry
queer band, that's an equivocal gift; history in ``The Inheritance'' is
a one-way street pointing only toward the present. The play's thesis
question, voiced by the main character, is ``What does it mean now'' ---
in light of the past --- ``to be a gay man?'' As vivid as ``The
Inheritance'' is, the question, tinged with youthful self-absorption, is
beginning to bore me, just as I grow old enough to forget how much it
once consumed me. In any case, I find myself more compelled by art, like
``My Pie Town,'' in which history moves in the opposite direction and
asks the opposite question: What --- in light of today --- might being
gay have meant in the past?

As such, I keep returning to two recent works that address the question
directly, both of them monumental and a little bit nuts. One is Kramer's
``The American People,'' a two-volume novel whose 775-page first volume
was published in 2015. (The second volume --- another 880 pages --- will
be released in January.) ``The American People'' is nothing less than a
biography of America retold as an almost exclusively gay invention. That
its chief gay inventors --- among them
\href{https://www.nytimes3xbfgragh.onion/topic/person/george-washington}{George
Washington},
\href{https://www.nytimes3xbfgragh.onion/topic/person/alexander-hamilton}{Alexander
Hamilton} and
\href{https://www.nytimes3xbfgragh.onion/topic/person/abraham-lincoln}{Abraham
Lincoln} --- are typically rendered by history as heterosexual is part
of why Kramer grudgingly calls the work a novel, even though he believes
every word of it. Fantastical as his story may seem, it has the urgency
(and chaotic despotism) of truth --- if not always the particular truth,
then at least a general one. To imagine that everything Kramer argues is
mad is to make an even madder argument: That no one of such consequence
as these men could ever be gay.

Image

April Dawn Alison's ``Untitled,'' no date.Credit...San Franciscio Museum
of Modern Art, gift of Andrew Masullo, courtesy of SFMOMA and MACK

``The American People'' is thus a kind of raid and kidnapping, marching
into the country's past to out its closeted celebrities and bringing
them back to be traded for ransom. The only recent work visionary enough
to complement and counterweight Kramer's is
``\href{https://www.nytimes3xbfgragh.onion/2016/09/18/theater/taylor-mac-24-decade-history-of-popular-music.html}{A
24-Decade History of Popular Music},'' by
\href{https://www.nytimes3xbfgragh.onion/2019/04/02/magazine/taylor-mac-gary-broadway.html}{Taylor
Mac}, though it is more of a pride parade than a sortie. Each hour of
the marathon piece of performance art examines one decade between 1776
and 2016, reframing its popular music --- from ``Yankee Doodle'' in the
first decade to
\href{https://www.nytimes3xbfgragh.onion/topic/person/lauryn-hill}{Lauryn
Hill}'s ``Everything Is Everything'' in the penultimate one --- with
liberating, pansexual fervor and glee. First performed as a 24-hour
whole at St. Ann's Warehouse in Brooklyn in 2016, the show is a
brilliant example of low-tech, high-impact theatricality. Mac, onstage
throughout and accompanied by a 24-piece orchestra that shrinks over
time to nothing, serves as both shaman and chanteuse, introducing each
number, singing the hell out of it and explaining its secret messages.
In between songs, he changes clothes --- an entire Mardi Gras of
dizzying costumes has been concocted from trash by the designer known as
Machine Dazzle --- and goads the audience into all sorts of
participatory acts. (I not only donned drag during one decade but threw
Ping-Pong balls, as instructed, at a passing temperance parade.) Like a
Seder, the show seems designed to exhaust or intoxicate its audience
into absorbing Mac's radical rereading of history, while also keeping
everyone awake long enough to melt a crowd into a community.

But the production's mega-camp style should not distract us from
appreciating a text that is demonically clever and brilliantly
researched. (Who knew there was an 18th-century hit by the Scottish poet
Robert Burns titled ``Nine Inch Will Please a Lady''?) What Mac is able
to demonstrate in 246 songs is that queerness of all sorts --- including
the spirit of resistance that often arises from it --- has been part of
our DNA from the beginning; it's packed into Western cultural
expression. Sometimes it's just waiting for an artist to go rummaging in
the attic to find it.

Image

Lyle Ashton Harris's ``Billie \#26'' (2002).Credit...Unique polaroid,
courtesy of Salon 94, New York and David Castillo Gallery, Miami

\hypertarget{3-conversations-with-the-past}{%
\subsubsection{3. Conversations With the
Past}\label{3-conversations-with-the-past}}

Thinking about these contrasting works, I began to picture a divided
highway being built by queer artists. On one side, some were updating
the present with imports from the past, while across the guardrail,
others were backfilling the past with exports from the present. Among
the backfillers is \href{https://newdramatists.org/donja-r-love}{Donja
R. Love}, whose plays ``Sugar in Our Wounds'' and ``Fireflies'' were
both staged off Broadway last year. The two are part of a trilogy called
``The Love* Plays,'' the asterisk indicating queer love, erased from
black history. Love works to repair that erasure in daring ways.
``Sugar'' explores the relationship between two enslaved men whose
romance is conducted under the supportive gaze of an elder called Aunt
Mama. ``Fireflies,'' about a couple clearly based on
\href{https://www.nytimes3xbfgragh.onion/topic/person/martin-luther-king-jr}{Martin
Luther King Jr}. and
\href{https://www.nytimes3xbfgragh.onion/topic/person/coretta-scott-king}{Coretta
Scott King}, goes even further, imagining the latter character not only
as the author of her husband's great speeches but also as a lesbian.
``It's not in the realm of the unbelievable that the figures `Fireflies'
was based on had a private life that didn't always reflect their public
life,'' Love says.

It's nice to think so, but Love, smartly, does not allow his plays to
become past-life wish fulfillments. Both ``Sugar'' and ``Fireflies'' are
more tragic than not. So is ``In the Middle,'' the upcoming conclusion
to his trilogy, about a black mother whose queer son is killed by the
police. The mother's remaining family adapts what Love describes as an
old African ritual of uplifting to assuage her modern grief and guilt.
In that, the playwright jumps the guardrail, which turns out not to be
very high. Artists cross it regularly. Love sees his plays not as
one-way monologues but as ``conversations with the past.''

Still, the different methods have different implications. In
\href{https://www.nytimes3xbfgragh.onion/interactive/2019/04/10/t-magazine/jeremy-o-harris-ye.html}{Jeremy
O. Harris'}s
``\href{https://www.nytimes3xbfgragh.onion/2019/09/11/theater/slave-play-broadway-jeremy-harris.html}{Slave
Play},'' on Broadway now, three interracial couples attend a weeklong
``antebellum sexual performance therapy'' retreat. As part of the
prescribed role play, the sole gay couple takes on characters that
reverse the expected plantation power dynamic: Gary, who is black,
dominates Dustin, who is white. But they can't keep a straight face, as
their real-life conflict keeps infecting their fantasy. The past in
``Slave Play'' is not there to be repaired; it is explicitly a tool for
repairing the present, and (as the results of the therapy bear out) an
imperfect one at that.

But an artist isn't a repair person, and ``Slave Play'' is doubtful on
whether the larger interracial union of America can ever be fixed as
long as white people fail to acknowledge the obliterating shadow of
racism we all live under. What playwrights like Love and Harris are
doing in that sense is staging an intervention: not just writing
blackness into history but, in a kind of double axel, blackness into
gayness into history. When a culture has difficulty acknowledging even
one form of marginality, two or more forms may result in total
invisibility, the resistance coming from every direction. And so it's no
surprise that queer artists look backward to see how their multiple
identities --- you could add gender, class, disability and others to the
list --- first began to merge or diverge. We all want proof that people
like us have existed before, and if it can't be discovered it must be
invented.

Image

McDermott \& McGough's ``Serviced Apartment, 1965'' (2017), from their
``Hollywood (Homosexual) Hopeful'' series.Credit...Oil on canvas,
courtesy of McDermott \& McGough

Sometimes that means creating a personal time machine. The artist Glenn
Ligon is now known primarily for his works exploring racial identity: In
his 1993 ``Runaways'' series, he slips descriptions of himself as a
missing person, solicited from friends, into re-creations of
19th-century advertisements seeking the return of escaped slaves. ``RAN
AWAY, Glenn, a black male, 5'8'', very short hair cut, nearly completely
shaved, stocky build, 155-165 lbs.,'' one of them begins. The text, set
in period typography, is illustrated with a drawing of a man on the run
with his bindle, suggesting at the same time an artist on the run from
the past.

But Ligon, whose early work in the 1980s included appropriations of gay
pornography, has also played with time as a way of commenting on more
complicated questions of identity. The source material for a 2000 show
called ``Coloring'' came from a series of workshops in Minneapolis in
1999 at which he asked children to fill in pictures of black cultural
heroes found in coloring books from the 1960s and 1970s. He then
reproduced their hilariously impertinent scrawls in paint and oil stick
on canvas, turning
\href{https://www.nytimes3xbfgragh.onion/topic/person/malcolm-x}{Malcolm
X}, for instance,
\href{https://www.whitney.org/WatchAndListen/728}{into} a blue-lidded,
fuchsia-lipped drag queen. The point is not to impute counterfactual
gayness on notably heterosexual historical figures but to suggest that
identity is created by context --- and can just as easily be dissolved
by it, too. That duality is no doubt part of what makes the past so
attractive to queer artists. They can use it to swing either way.

Among photographers, naturally, the contrast is especially clear.
Representing the importers is Catherine Opie, some of whose photographic
portraits appropriate painterly tropes of former centuries to valorize
subjects --- many of them her lesbian and trans friends --- who too
often are marked as beyond the bounds of culture. To be honest, they are
often self-marked, or at any rate marked by choice, with elaborately
transgressive tattoos and piercings. In one astonishing piece,
``Self-Portrait/Cutting'' (1993), Opie faces away from the camera so we
can see a drawing made of wounds on her back. As if fashioned by a child
with a razor, two stick-figure women in triangle skirts hold hands
beneath a puffy cloud while a startled house beside them bleeds. This
image is placed in tension with an emerald green 18th-century-style
backdrop, swagged to bursting with damask fruit that lends a
gorgeousness, and an imprimatur of mainstream probity, to the otherwise
discomfiting scene. Here and in so many other portraits, Opie's demand
of the past is that it serve the present and in doing so salve it.

Then there's \href{http://johndugdalestudio.com/}{John Dugdale}, whose
photographs, many of them tinted the eerie blue of the 19th-century
cyanotype process, wash away all marks of modernity. It was in the
1990s, after Dugdale lost most of his eyesight from an AIDS-related
illness, that he reconfigured his technique, acquiring a large-format
antique camera and experimenting with older chemical methods and older
styles of portraiture. Men in nature, classically posed nudes, male
couples locked in athletic embraces, even an ornately semi-draped hunk
caressing a lamb: These are images that could have passed as pornography
for Forster.

Image

TM Davy's ``Kalup, Reclining Nude'' (2008).Credit...Oil on canvas,
private collection, courtesy of the artist and Van Doren Waxter, N.Y.

Probably not for us, though. Is it odd that the most stirring --- O.K.,
hot --- depictions of sex in the queer art I've been exploring have come
not from photographs but from literary and dramatic works? (I am a
theater critic, but still.) One is in the 2015 play ``Indecent,'' in
which
\href{https://www.nytimes3xbfgragh.onion/topic/person/paula-vogel}{Paula
Vogel} refracts the history of an earlier work through a series of
modern lenses. Her inspiration, a 1906 Yiddish melodrama by Sholem Asch
called ``God of Vengeance,'' became notorious when it was shut down for
obscenity after its English-language Broadway debut in 1923. This was
not because the respectable businessman at the center of its story runs
a brothel in the basement of his proper Jewish home. It was because his
virginal daughter falls in love with one of his prostitutes and, in the
play's most beautiful scene, kisses her passionately in the rain.

Though the scene was apparently cut from the 1923 production, we see it
repeatedly, from different angles, in the director Rebecca Taichman's
staging of ``Indecent''; from none is it merely pretty or chaste. (The
women's nightgowns get drenched.) In its heat no less than its beauty,
it remains one of the most surprising images of eros I've seen onstage.
The surprise isn't just that we have accessed it, like a radio signal
from the past, but that a male, presumably heterosexual playwright in
1906 had the daring to write it down. And how could he have preserved
such a moment if it didn't exist? Vogel's text insists that it did, even
hinting that the past's version of queerness, before the arrival of
identity politics, was not perhaps an entirely bad thing. If we are now
freer to express our experience, we are no longer free from having to
categorize it --- and in categorizing, to limit its meaning. Only in a
society that had no name for them could these women be so unbound.

And yet, outside of ``Maurice,'' such plots rarely end happily.
(``Indecent'' is framed as a Holocaust story; Hall's ``The Well of
Loneliness'' concludes with its lesbian protagonist praying piteously
for ``the right to our existence.'') In ``Portrait of a Lady on Fire,''
which won the Best Screenplay Award at the Cannes Film Festival in May,
a young woman in Brittany in the late 18th century falls in love with
the woman painting her portrait. This is dazzlingly done, and a major
addition to our undocumented history, but only enhances its inevitable
sadness, as the purpose of the portrait is to confirm the girl's
marriageability to an Italian man she's never met and doesn't want to.
We last see her crying for what seems like an eternity.

Image

Catherine Opie's ``Rocco'' (2012), from her ``Portraits and Landscapes''
series.Credit...Pigment print © Catherine Opie, courtesy of Regen
Projects, Los Angeles and Lehmann Maupin, New York, Hong Kong and Seoul

Image

Martine Gutierrez's ``Neo-Indeo, Legendary Cakchiquel, p20 From
Indigenous Woman'' (2018).Credit...C-print mounted on Sintra, edition of
eight © Martine Gutierrez, image courtesy of the artist and Ryan Lee
Gallery, New York

\hypertarget{4-the-world-as-it-might-have-been}{%
\subsubsection{4. The World As It Might Have
Been}\label{4-the-world-as-it-might-have-been}}

The title of Peter McGough's memoir, published this fall, is
``\href{https://www.penguinrandomhouse.com/books/561291/ive-seen-the-future-and-im-not-going-by-peter-mcgough/}{I've
Seen the Future and I'm Not Going}.'' That pretty much sums up the
philosophy under which he and David McDermott, his longtime collaborator
and former partner, have made art since 1980 under the moniker McDermott
\& McGough. They not only turned their lives into a performance piece
based in the past --- wearing detachable starched collars and
high-button shoes too fey even for a visit to Howards End --- but also
made their artwork an expression of their born-in-the-wrong-century
lives. Among their most famous pieces is a yellow canvas covered with a
thesaurus of quaint (and not-so-quaint) epithets for gay men,
decoratively arranged in varying old-timey typefaces like a jaunty
printer's sample. It's called ``A Friend of Dorothy, 1947,'' though it
was made 40 years later; they often sign their work with a long-bygone
date to ``save it from obscurity'' --- obscurity being the present.

In that pursuit, the art, it seems, has been more successful than the
life; McGough, who almost died of complications of AIDS after he refused
modern medicine, now calls their dream of living as Edwardian gentlemen
in 1980s Manhattan ``a failed utopia.'' Perhaps that's inevitable when
you see your world as a ``time death trap,'' as McDermott once put it,
and seek to escape it with reproductions of antique wallpaper and a hand
pump in lieu of running water. Yet there's a contradiction: The work the
men consider their masterpiece --- and when I saw it, I thought so, too
--- is ``The Oscar Wilde Temple,'' versions of which appeared in New
York in 2017 at a Greenwich Village church and in London at Studio
Voltaire a year later. ``It is probably our greatest work,'' McGough
says, ``because it's not about us.'' A religious environment in which
the religion is queerness, complete with stations of Wilde's cross and
portraits of other gay martyrs, it did what religious art so often and
so confusingly does: glorifies a past that produced such a figure but
also desecrated him.

In a similar way, ``The Inheritance'' is a Forster temple. It's only the
play's nightmare narcissist character who denies him, by denying
``Howards End'': ``I can't identify with it at all.'' This is the myopic
plaint of the time refugee, unable to credit the past because he isn't
in it. What the most powerful works of queer historical imagination seem
to be doing instead is positing a vision of the world as it might have
been had gayness been a more central if not yet accepted part of it. And
by doing so, stoutly acknowledging that, in reality, things are neither
better nor worse but both.

Which is not to criticize those works that, almost by default, suggest
unimpeded progress; if we are able to enjoy the sad narrative of
``Fireflies'' in the theater or sigh over ``The Scapegoat, 1863,'' a
1986 McDermott \& McGough painting depicting a sacrificial male nude on
a sofa, we are already better off than their subjects. They are visible
to us as they could not be to themselves. Instead, I want to suggest
that we look at such art as a warning against defeatism and the
appropriation of pathos. One model for that would be ``The Colored
Museum,'' the playwright and director George C. Wolfe's satire of black
cultural representation; his ``exhibits'' --- a series of
vaudeville-like sketches of black life and clichés, with titles like
``The Last Mama-on-the-Couch Play'' --- ripped holes in older portraits
that have never quite healed. As the critic Frank Rich put it in
\href{https://www.nytimes3xbfgragh.onion/1986/11/03/arts/stage-colored-museum-satire-by-george-c-wolfe.html}{his
New York Times review} of the 1986 Public Theater premiere, Wolfe was
asking how black men and women could ``at once honor and escape the
legacy of suffering that is the baggage of their past.''

Implicit in the two halves of ``honor and escape'' is a prohibition
against mawkish brooding and artless earnestness; and I'd add that
looking to the past to justify a sense of personal grievance may not be
warranted for today's queer artists either. That vein is tapped out, at
least for those white gay cisgender men who have emerged from the
disasters of the closet and AIDS into lives of great privilege. With our
husbands and children legally attached --- I'm lucky enough to have both
--- perhaps a different approach is in order. While watching transgender
and other marginalized artists rise and face their own historical
reckoning, I'm concerned less about my own past than their future, and
how we will lend ourselves to it.

Image

The cover of
\href{https://www.nytimes3xbfgragh.onion/issue/t-magazine/2019/11/21/ts-dec-8-holiday-issue}{T's
Dec. 8 Holiday issue}. Left: \textbf{Lou Dallas} dress, \$315, and
sleeve, \$290, \href{https://www.loudallas.com/}{loudallas.com}. Vintage
earrings, \$3,000, Broken English, (212) 219-1264. Right: \textbf{Ann
Demeulemeester} shirt, price on request,
\href{https://www.anndemeulemeester.com/}{anndemeulemeester.com}.
Stylist's own collar. Shot at Blairsden house in New
Jersey.Credit...Photo by James Hawkinson. Styled by Jay Massacret

This doesn't mean there is no joy to be had in more traditional queer
art; indeed, joy is the best thing left in it. And joy is what I feel
reading Sarah Waters's novels, starting with ``Tipping the Velvet'' in
1998. Most of them take place in the lesbian past; ``tipping the
velvet'' was slang for cunnilingus in Victorian England. They involve
clever women learning to create themselves without obvious models in
hostile environments, yet they never feel maudlin; rather, in their
urbanity, as well as in their momentum and subversion of genre, they
read as if written by Dickens after a year spent majoring in gender
studies at Vassar. They recreate real worlds erased by time (``Tipping
the Velvet'' begins in a theatrical demimonde featuring celebrated male
impersonators) but also revel in the deliciousness of imagining them as
fantasy. What would a pub for lesbians, like the one that features in
that novel, have felt like if it existed back then? Somehow Waters makes
the Boy in the Boat, as her pub is called, seem more real --- with its
treacherous staircase, sand on the floorboards, ``trousered toms'' at
the billiard table and ``square-chinned'' Mrs. Swindles behind the bar
--- than it ever could if not invented.

Creating lives that ought to have existed is basically the novelist's
brief, whatever the specific subject or genre. But the pressure exerted
by contemporary politics to make queer lives visible sometimes
interferes with art-making, which at its best observes no external
agendas. Waters has avoided that problem, not by retreating from the
present but by locating the thing in the past she loves and making it
live again. ``I'm fascinated by how quickly social and cultural
landscapes change,'' she says, ``by how behaviors that at one point seem
appropriate and acceptable --- or wildly inappropriate and unacceptable
--- can, within a few years, have been turned on their heads.'' She
looks to history precisely to see that it is not our story: ``because
then we realize that if the past can change, so can the present.''

Perhaps that's why ``Tipping the Velvet'' insists on the happy ending
denied to the heroine of ``The Well of Loneliness.'' So in its way does
``The Sparsholt Affair,'' a strange and beautiful novel by
\href{https://www.nytimes3xbfgragh.onion/2015/10/09/t-magazine/my-10-favorite-books-alan-hollinghurst.html}{Alan
Hollinghurst}, published in 2017. Like most of Hollinghurst's three
decades of fiction, ``The Sparsholt Affair'' is obsessed with history.
(``Contemporary life doesn't have the things I find most interesting,''
he told a reporter in 2018. ``Secrecy, concealment, danger.'') What
begins in 1940 as a story of a closeted hunk at Oxford, and erupts 26
years later as a gay version of the 1960s British sex-and-politics
Profumo scandal, eventually skitters through five more decades. During
that time, the disgraced man's son, Johnny, learns to live the life his
father couldn't, marrying a man and becoming a father himself. When we
leave him in the 2010s, he is learning to adapt to yet another new age,
this one featuring gayness triumphant in the form of dancing, designer
drugs and Grindr.

If ecstatic at times, Hollinghurst's characters are not especially
heroic, but it's still instructive to see how easily they change or are
overridden by time. It's no accident that Johnny starts out
professionally as an art restorer and ends up a successful painter: His
minute study and improvement of old canvases lays the groundwork for a
future his father dared not dream. By chronicling both men's lives in
patient prose, Hollinghurst stitches Forster's wound --- and not just
Forster's. Repairing a narrow corner of the past, he and the other
artists I've been thinking about are repairing us. They are taking gay
history out of the drawer where it has lain untouched, awaiting
usefulness. They are answering the question of ``Maurice''
affirmatively: It is publishable. It is worth it.

Models in the first photo: Krow Kian at Heroes Model Management, Jeffrey
Prada at Nöni, Otto Zinsou at the Claw Models and Carmen Amare at Muse
Management. Models on the cover: Prada and Kian. Hair by Tomi Kono at
Julian Watson Agency using R\&Co. Grooming by Frankie Boyd at Streeters.
Set design by Randall Peacock. Food styling by Young Gun Lee. Casting by
Noah Shelley at Streeters. Production by Hen's Tooth. Manicure: Yuko
Tsuchihashi using Chanel Le Vernis. Floral arrangements and greens:
Lauren Messelian. Lighting design: David Diesing. Digital tech: Travis
Drennen. Photo assistants: Will Englehardt, Kevin Vast, Jason Acton and
Alex Cohen. Hair assistants: Chika Nishiyama and Mayumi Maeda. Grooming
assistant: Jeff Santiago. Set assistants: Todd Knopke, Peter Davis and
Genevieve Ward. Food stylist's assistant: Mariko Makino. Tailor: Curie
Choi. Stylist's assistant: Sho Tatsuishi.

Advertisement

\protect\hyperlink{after-bottom}{Continue reading the main story}

\hypertarget{site-index}{%
\subsection{Site Index}\label{site-index}}

\hypertarget{site-information-navigation}{%
\subsection{Site Information
Navigation}\label{site-information-navigation}}

\begin{itemize}
\tightlist
\item
  \href{https://help.nytimes3xbfgragh.onion/hc/en-us/articles/115014792127-Copyright-notice}{©~2020~The
  New York Times Company}
\end{itemize}

\begin{itemize}
\tightlist
\item
  \href{https://www.nytco.com/}{NYTCo}
\item
  \href{https://help.nytimes3xbfgragh.onion/hc/en-us/articles/115015385887-Contact-Us}{Contact
  Us}
\item
  \href{https://www.nytco.com/careers/}{Work with us}
\item
  \href{https://nytmediakit.com/}{Advertise}
\item
  \href{http://www.tbrandstudio.com/}{T Brand Studio}
\item
  \href{https://www.nytimes3xbfgragh.onion/privacy/cookie-policy\#how-do-i-manage-trackers}{Your
  Ad Choices}
\item
  \href{https://www.nytimes3xbfgragh.onion/privacy}{Privacy}
\item
  \href{https://help.nytimes3xbfgragh.onion/hc/en-us/articles/115014893428-Terms-of-service}{Terms
  of Service}
\item
  \href{https://help.nytimes3xbfgragh.onion/hc/en-us/articles/115014893968-Terms-of-sale}{Terms
  of Sale}
\item
  \href{https://spiderbites.nytimes3xbfgragh.onion}{Site Map}
\item
  \href{https://help.nytimes3xbfgragh.onion/hc/en-us}{Help}
\item
  \href{https://www.nytimes3xbfgragh.onion/subscription?campaignId=37WXW}{Subscriptions}
\end{itemize}
