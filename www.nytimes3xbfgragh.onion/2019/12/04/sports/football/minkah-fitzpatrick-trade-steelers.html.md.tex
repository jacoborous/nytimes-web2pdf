Sections

SEARCH

\protect\hyperlink{site-content}{Skip to
content}\protect\hyperlink{site-index}{Skip to site index}

\href{https://www.nytimes3xbfgragh.onion/section/sports/football}{Pro
Football}

\href{https://myaccount.nytimes3xbfgragh.onion/auth/login?response_type=cookie\&client_id=vi}{}

\href{https://www.nytimes3xbfgragh.onion/section/todayspaper}{Today's
Paper}

\href{/section/sports/football}{Pro Football}\textbar{}The Best N.F.L.
Trade of the Season? Look to Pittsburgh

\url{https://nyti.ms/2rRHdyw}

\begin{itemize}
\item
\item
\item
\item
\item
\item
\end{itemize}

Advertisement

\protect\hyperlink{after-top}{Continue reading the main story}

Supported by

\protect\hyperlink{after-sponsor}{Continue reading the main story}

\hypertarget{the-best-nfl-trade-of-the-season-look-to-pittsburgh}{%
\section{The Best N.F.L. Trade of the Season? Look to
Pittsburgh}\label{the-best-nfl-trade-of-the-season-look-to-pittsburgh}}

After forcing his way out of Miami, Minkah Fitzpatrick has powered the
Steelers' surprising surge into contention.

\includegraphics{https://static01.graylady3jvrrxbe.onion/images/2019/12/04/sports/04minkah1/04minkah1-articleLarge-v2.jpg?quality=75\&auto=webp\&disable=upscale}

\href{https://www.nytimes3xbfgragh.onion/by/ben-shpigel}{\includegraphics{https://static01.graylady3jvrrxbe.onion/images/2018/02/20/multimedia/author-ben-shpigel/author-ben-shpigel-thumbLarge.jpg}}

By \href{https://www.nytimes3xbfgragh.onion/by/ben-shpigel}{Ben Shpigel}

\begin{itemize}
\item
  Published Dec. 4, 2019Updated Dec. 9, 2019
\item
  \begin{itemize}
  \item
  \item
  \item
  \item
  \item
  \item
  \end{itemize}
\end{itemize}

PITTSBURGH --- Minkah Fitzpatrick has always known that playing well
gives him options. When he received dozens of scholarship offers in high
school, he elected to play at Alabama because it was where he would
develop most quickly. Then after his junior season, when the Crimson
Tide won the second national title of his tenure and he was honored as
college football's best defensive player, Fitzpatrick entered the N.F.L.
draft because he didn't think he had anything more to prove.

``I know how I am, I know what I can do and how I carry myself,''
Fitzpatrick said in an interview. ``I'm always going to bet on myself.''

These were, he said recently, strictly ``business decisions,'' made to
augment his brand, potential value and earning power in the N.F.L. The
principles steering those choices guided him again after he got there.
In mid-September, about 16 months after the Dolphins selected him
\href{https://www.pro-football-reference.com/players/F/FitzMi00.htm}{No.
11 over all}, as the first defensive back taken, Fitzpatrick requested a
trade because he didn't think Miami's new coaching staff was maximizing
his abilities.

Fitzpatrick committed to leaving but, unlike unhappy peers in the
N.B.A., could not choose where he'd be going. That destination turned
out to be Pittsburgh, which surrendered a bounty of draft choices that
included its first- and fifth-round picks in 2020. His range, steadiness
and
\href{https://www.pro-football-reference.com/players/F/FitzMi00/gamelog/2019/}{turnover-hoarding
aptitude} at free safety have powered the Steelers' resurgence to such
an extent that their fans might very well consider petitioning to rename
the city after him.

Fitzburgh, anyone?

The Steelers (7-5) have vaulted into the A.F.C. playoff chase by winning
seven of their 10 games, and six of the last seven heading into Sunday's
game at Arizona, since acquiring Fitzpatrick, who has accounted for
eight of the team's league-leading 28 takeaways over that span --- five
interceptions, two fumble recoveries and a forced fumble **** --- while
scoring two touchdowns. Or as many as Odell Beckham Jr., Alvin Kamara
and Greg Olsen.

The Steelers' deal for Fitzpatrick was an isolated occurrence of the
player-driven movement that's common in the N.B.A. but has yet to
infiltrate the N.F.L. Really, how often are there trades involving
players of his caliber, barely 23 years old and under a contract
favorable to his team through 2021?

All professional teams act on their own best interests, signing and
cutting players without regret, but the N.F.L.'s system is uniquely
crafted to smother player power --- with a hard salary cap, a franchise
tag, a rookie wage scale. Amid that landscape a few standouts have tried
seizing control of their careers.

\href{https://www.sbnation.com/nfl/2019/10/4/20799103/amy-trask-raiders-women-nfl-leaders-interview}{Amy
Trask, the former chief executive officer of the Raiders who worked
nearly three decades in the N.F.L.}, hesitated to label this a trend,
acknowledging the different circumstances that affect an organization's
decision. She also speculated that the next iteration of the collective
bargaining agreement, which expires after the 2020 season, could also
affect players' empowerment. But she thought it notable that on either
side of Fitzpatrick's departure, two other stars pried their way out in
such proximity to one another.

On the eve of the season, the Pro Bowl defensive end Jadeveon Clowney
refused to end a holdout in Houston over signing a franchise tender that
would prevent him from becoming a free agent, and so the
\href{https://www.nytimes3xbfgragh.onion/2019/09/01/sports/football/jadeveon-clowney-trade-seahawks.html}{Texans
shipped him to Seattle}. Six weeks later, Jalen Ramsey, one of the
league's best cornerbacks, was sent to the Rams after forcing his way
out of Jacksonville.

``Clearly, this is something one would never see with people that don't
have that sort of level of gravitas within their organization, if you
will,'' Trask, now an analyst for CBS, said in a telephone interview.

Fitzpatrick declined to explain why he requested a trade from Miami,
where team officials, he said, tried persuading him to reconsider. But
he indicated that he was not pleased that the Dolphins wanted to move
him around the defensive backfield from week to week instead of
anchoring him at one position. By Week 2, when Fitzpatrick played the
majority of his defensive snaps at his desired spot, free safety,
forcing and recovering a fumble against New England, the rift was beyond
repair.

``A lot of people just think that they're stuck in the situation that
they're in, and that isn't the case, you know what I'm saying?''
Fitzpatrick said in a recent interview at the Steelers' practice
facility. ``Some people are. I was fortunate enough to be in a situation
where I can speak my mind and share my opinion and also hear their
opinion.''

He added: ``I didn't want to leave --- I wanted to work things out and
communicate and find a way to move forward --- but it just didn't
happen, couldn't happen. We had different opinions.''

At the time, the Dolphins,
\href{https://www.nytimes3xbfgragh.onion/2019/10/13/sports/football/redskins-dolphins.html}{after
detonating their roster to amass draft picks and gain financial
flexibility}, were 0-2. So were the Steelers, who had far more talent
than Miami but had just lost two critical starters to season-ending
injuries:
\href{https://www.nytimes3xbfgragh.onion/2019/09/16/sports/football/ben-roethlisberger-drew-brees-injury.html}{quarterback
Ben Roethlisberger} (elbow) and safety Sean Davis (shoulder). On Sept.
16, hours after announcing that Roethlisberger needed surgery,
Pittsburgh took a series of calculated risks.

\includegraphics{https://static01.graylady3jvrrxbe.onion/images/2019/12/04/sports/04minkah2/merlin_160855566_27a2b455-9294-4ea6-ac71-58d7755de9b7-articleLarge.jpg?quality=75\&auto=webp\&disable=upscale}

Not since 1966 had the Steelers parted with a first-round draft
selection, and if they were going to do so for Fitzpatrick, they had to
believe that the team, led by a backup quarterback, would finish well
enough that its pick would fall late in the round, or at least not in
the top 10. Otherwise, following a poor season and with their
quarterback situation in flux, they would be squandering a prime
opportunity to draft Roethlisberger's successor.

``We end up with our No. 1 being 20th or 25th again, we probably made a
good trade,'' defensive coordinator Keith Butler said. ``I thought we
made a good trade anyway, regardless of where we land. I'm glad we got
him.''

Coach Mike Tomlin added in an interview, ``It really wasn't a hard
decision on our part.''

Tomlin scouted Fitzpatrick twice before the 2018 draft --- just in case,
Tomlin figured --- even though he knew he would be gone when Pittsburgh
picked at No. 28. He liked Fitzpatrick's awareness and his ball skills,
communication style and personality. Tomlin said he also valued the
versatility that allowed Fitzpatrick to practice at six positions in
Miami, though he intends to keep him at free safety, at least for now.

Before Fitzpatrick had practiced with the Steelers, Tomlin announced
that he would start that week at San Francisco. The team accommodated
Fitzpatrick by simplifying his assignments, giving him a menu of
defenses it expected to run so he could relate them to those from his
previous schemes. Every week the Steelers have layered on more
responsibilities, and Fitzpatrick has conquered them all.

``For a guy who's only been on the defense for a few weeks still,''
outside linebacker T.J. Watt said, ``he's communicating like he's been
in this defense his whole life.''

The Steelers have a
\href{https://steelerswire.usatoday.com/2019/07/13/ranking-the-7-best-defensive-backs-in-steelers-history/}{rich
history of outstanding defensive backs}, from Mel Blount to Rod Woodson,
Mike Wagner to Donnie Shell. But they haven't had a player with
Fitzpatrick's acumen and athleticism since Troy Polamalu, curly locks
flowing from beneath his helmet, last patrolled the secondary in 2014.
Butler said it was too early, and unfair, to compare Fitzpatrick, in his
second season, to Polamalu,
\href{https://www.nytimes3xbfgragh.onion/2005/01/20/sports/football/polamalu-dances-with-fire-just-not-in-the-end-zone.html}{a
former defensive player of the year who's most likely headed to the Pro
Football Hall of Fame}.

``But I wouldn't think that would be too far down the line,'' Butler,
who joined the Steelers' coaching staff in 2003, the same year they
drafted Polamalu, said in an interview. ``He has a lot of the same
instincts and vision of the field like Troy. We haven't had it for a
while.''

Fitzpatrick refines that intuition during the week. When studying film,
he visualizes what he would do if the quarterback looked this way, if
the receiver ran this route. He wants the information to be so ingrained
that he can just react.

That's what he did on Nov. 3 against Indianapolis, when after reading
the formation and deciphering tight end Jack Doyle's route, he noticed
quarterback Brian Hoyer looking left as his shoulders opened to the
right. Before Hoyer released the ball Fitzpatrick had darted toward
Doyle, bounding in front to intercept the pass, returning it 96 yards
for a touchdown.

``If the ball's in the air,'' cornerback Mike Hilton said of
Fitzpatrick's anticipation, ``it could be ours.''

That proves true even when Fitzpatrick is not in coverage. Late in the
fourth quarter against the Rams on Nov. 10, Fitzpatrick sensed that
quarterback Jared Goff would be throwing in cornerback Joe Haden's
direction, so he played over the top, waiting for the ball to come. It
did. Diving, Haden deflected it --- right to Fitzpatrick, trailing
behind,
\href{https://twitter.com/NFL/status/1193694265152401409?ref_src=twsrc\%5Etfw\%7Ctwcamp\%5Etweetembed\%7Ctwterm\%5E1193694265152401409\&ref_url=https\%3A\%2F\%2Fwww.theringer.com\%2Fnfl\%2F2019\%2F11\%2F10\%2F20958753\%2Fpittsburgh-steelers-los-angeles-rams-minkah-fitzpatrick-tj-watt}{for
the victory-securing interception}. On the Fox broadcast, Daryl Johnston
relayed a conversation with Goff, who had emphasized the importance of
locating Fitzpatrick after every snap.

``You can't be afraid to make decisions,'' Butler said, ``and he's
not.''

At his locker, Fitzpatrick explained why. His best quality as a player,
he said, is his instinct, and not once since leaving Miami nearly three
months ago has he regretted asking to be dealt. He has never grappled
with how he would be perceived in South Florida, where his former
teammates are toiling through a 3-9 season.

He still has friends there, but not as many as on his new team. Since
the trade, the Steelers rank second in opposing completion percentage
and passer rating, according to Pro Football Reference, and have charged
into contention. \href{http://www.tankathon.com/nfl}{According to
tankathon.com}, that first-round pick heading to Miami projects at No.
22.

Fitzpatrick changed his fortunes for the better. His team's, too.

Advertisement

\protect\hyperlink{after-bottom}{Continue reading the main story}

\hypertarget{site-index}{%
\subsection{Site Index}\label{site-index}}

\hypertarget{site-information-navigation}{%
\subsection{Site Information
Navigation}\label{site-information-navigation}}

\begin{itemize}
\tightlist
\item
  \href{https://help.nytimes3xbfgragh.onion/hc/en-us/articles/115014792127-Copyright-notice}{©~2020~The
  New York Times Company}
\end{itemize}

\begin{itemize}
\tightlist
\item
  \href{https://www.nytco.com/}{NYTCo}
\item
  \href{https://help.nytimes3xbfgragh.onion/hc/en-us/articles/115015385887-Contact-Us}{Contact
  Us}
\item
  \href{https://www.nytco.com/careers/}{Work with us}
\item
  \href{https://nytmediakit.com/}{Advertise}
\item
  \href{http://www.tbrandstudio.com/}{T Brand Studio}
\item
  \href{https://www.nytimes3xbfgragh.onion/privacy/cookie-policy\#how-do-i-manage-trackers}{Your
  Ad Choices}
\item
  \href{https://www.nytimes3xbfgragh.onion/privacy}{Privacy}
\item
  \href{https://help.nytimes3xbfgragh.onion/hc/en-us/articles/115014893428-Terms-of-service}{Terms
  of Service}
\item
  \href{https://help.nytimes3xbfgragh.onion/hc/en-us/articles/115014893968-Terms-of-sale}{Terms
  of Sale}
\item
  \href{https://spiderbites.nytimes3xbfgragh.onion}{Site Map}
\item
  \href{https://help.nytimes3xbfgragh.onion/hc/en-us}{Help}
\item
  \href{https://www.nytimes3xbfgragh.onion/subscription?campaignId=37WXW}{Subscriptions}
\end{itemize}
