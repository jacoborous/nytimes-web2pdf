Sections

SEARCH

\protect\hyperlink{site-content}{Skip to
content}\protect\hyperlink{site-index}{Skip to site index}

\href{https://myaccount.nytimes3xbfgragh.onion/auth/login?response_type=cookie\&client_id=vi}{}

\href{https://www.nytimes3xbfgragh.onion/section/todayspaper}{Today's
Paper}

The Key to a Perfect Italian-American Sauce

\url{https://nyti.ms/2PFHtsS}

\begin{itemize}
\item
\item
\item
\item
\item
\item
\end{itemize}

Advertisement

\protect\hyperlink{after-top}{Continue reading the main story}

Supported by

\protect\hyperlink{after-sponsor}{Continue reading the main story}

\href{/column/magazine-eat}{Eat}

\hypertarget{the-key-to-a-perfect-italian-american-sauce}{%
\section{The Key to a Perfect Italian-American
Sauce}\label{the-key-to-a-perfect-italian-american-sauce}}

\includegraphics{https://static01.graylady3jvrrxbe.onion/images/2019/12/15/magazine/15mag-eat-image/15mag-eat-image-articleLarge.jpg?quality=75\&auto=webp\&disable=upscale}

By \href{https://www.nytimes3xbfgragh.onion/by/sam-sifton}{Sam Sifton}

\begin{itemize}
\item
  Dec. 11, 2019
\item
  \begin{itemize}
  \item
  \item
  \item
  \item
  \item
  \item
  \end{itemize}
\end{itemize}

Pimientos is what they're called in dictionaries, in seed catalogs. But
in Brooklyn, they're known as cherry peppers, just as they are in most
places where Italian-Americans gather and eat. Small and pumpkin-shaped,
like miniature bell peppers, they can be found in deli cases stuffed
with cubes of provolone and wisps of prosciutto, on antipasto platters
at weddings and funerals or suspended in vinegar in big jars at the sub
shop, hot and sweet at once, with a zing of sour acidity that makes them
an ideal topping for a sandwich of cured meats and salty cheese.
Sometimes I pair pickled hot cherry peppers with fried eggplant and
mozzarella on a sesame hero with a swipe of mayonnaise. I like them with
chunks of Parmesan, with raw clams, with coins of \emph{soppressata}.

I like pickled cherry peppers most, however, when they're used in sauce
and ladled onto meat: spicy and fragrant and slightly syrupy, what the
Italians call \emph{agrodolce}. I've had versions of the sauce with the
pan-roasted pork chops at Bamonte's in Brooklyn, and with the
oven-roasted stuffed ones at
\href{https://www.nytimes3xbfgragh.onion/2019/09/25/style/patsys-a-rat-pack-redoubt-turns-75.html}{Patsy's}
in Midtown Manhattan. A similar sauce accompanies the grilled veal at
Rao's in East Harlem. You can find it at Dominick's on Arthur Avenue in
the Bronx, and at DeLuca's on Staten Island. Steak with ``vinegar
peppers'' at the Italian steakhouse with the red-checked tablecloths, at
the pizzeria on the boardwalk, at your mother-in-law's on a Sunday
afternoon with family? You've had this sauce!

At
\href{https://www.nytimes3xbfgragh.onion/2013/06/05/dining/reviews/restaurant-review-carbone-in-manhattan.html}{Carbone},
on Thompson Street in Manhattan, the chefs Mario Carbone and Rich
Torrisi use pickled cherry peppers in many, many more recipes than you
might imagine. They're in the glaze on the restaurant's pork ribs, shiny
and piquant, with just a hint of fire. They're in the creamy sauce for
the fried calamari. They're in the \emph{mignonette} for the oysters
too. Carbone and I stood in the restaurant's kitchen during a lunch rush
eating all the variations, talking butter and brine. Earlier, Torrisi
told me about a hot-smoked rib-eye steak he cooked and served with a
sauce made of ham stock and the pickle juice from the cherry peppers.
``The smoky fat really likes the B\&G fire,'' he said, referring to the
brand of cherry peppers he and Carbone have been cooking with since
their early days at Torrisi Italian Specialties in Little Italy. (B\&G
isn't an Italian specialty; the company was named for the Bloch and
Guggenheimer families, pickle merchants in New York since 1889.)

Carbone and Torrisi are serious, technical chefs who execute
complicated, exacting recipes at their restaurants and charge big money
for the experience of eating them. They clued me in to a beautiful
secret, though: Cherry-pepper sauce is easy to make, and forgiving too.
You can set it up quickly, with sautéed garlic and diced peppers, a
splash of wine. You can make it slowly, with whole peppers and a lot of
the pickling juice cooked down into gloss. You can serve it with pork,
with veal, with steak: grilled, roasted or seared in a pan.

The Italians have a saying about how to play particular pieces of music,
and how to prepare certain dishes as well. ``\emph{A piacere},'' they
say, meaning ``as you like it,'' or literally, ``at pleasure.''
Cherry-pepper sauce with a chop is absolutely an \emph{a piacere}
situation: You can make the sauce, and the protein that accompanies it,
entirely as you like. For myself, I liked it first as a Bamonte's-style
pan sauce with diced peppers and later as a plusher, long-cooked version
that I found terrific with pork and brilliant with veal --- seeded
peppers and stock and brine, with a little butter thrown in at the end
for body and shine. That is the recipe that follows here, but you can
regard it as merely a suggestion, a place to begin. It may leave you
feeling as if you've opened your own little Rao's, a bootleg Carbone.
That is a good feeling, as it happens.

``It's a flavor that's purely Italian-American,'' Carbone told me when
we were standing in his kitchen, slurping some cherry-pepper sauce off
plastic spoons. He laughed and signaled for some more calamari, and half
a dozen oysters. ``You won't find it in Italy --- no way.''

\textbf{Recipe:}
\href{https://cooking.nytimes3xbfgragh.onion/recipes/1020700-veal-chops-in-cherry-pepper-sauce}{Veal
Chops in Cherry Pepper Sauce} \textbar{}
\href{https://cooking.nytimes3xbfgragh.onion/recipes/1020701-pork-chops-in-cherry-pepper-sauce}{Pork
Chops in Cherry-Pepper Sauce}

Advertisement

\protect\hyperlink{after-bottom}{Continue reading the main story}

\hypertarget{site-index}{%
\subsection{Site Index}\label{site-index}}

\hypertarget{site-information-navigation}{%
\subsection{Site Information
Navigation}\label{site-information-navigation}}

\begin{itemize}
\tightlist
\item
  \href{https://help.nytimes3xbfgragh.onion/hc/en-us/articles/115014792127-Copyright-notice}{©~2020~The
  New York Times Company}
\end{itemize}

\begin{itemize}
\tightlist
\item
  \href{https://www.nytco.com/}{NYTCo}
\item
  \href{https://help.nytimes3xbfgragh.onion/hc/en-us/articles/115015385887-Contact-Us}{Contact
  Us}
\item
  \href{https://www.nytco.com/careers/}{Work with us}
\item
  \href{https://nytmediakit.com/}{Advertise}
\item
  \href{http://www.tbrandstudio.com/}{T Brand Studio}
\item
  \href{https://www.nytimes3xbfgragh.onion/privacy/cookie-policy\#how-do-i-manage-trackers}{Your
  Ad Choices}
\item
  \href{https://www.nytimes3xbfgragh.onion/privacy}{Privacy}
\item
  \href{https://help.nytimes3xbfgragh.onion/hc/en-us/articles/115014893428-Terms-of-service}{Terms
  of Service}
\item
  \href{https://help.nytimes3xbfgragh.onion/hc/en-us/articles/115014893968-Terms-of-sale}{Terms
  of Sale}
\item
  \href{https://spiderbites.nytimes3xbfgragh.onion}{Site Map}
\item
  \href{https://help.nytimes3xbfgragh.onion/hc/en-us}{Help}
\item
  \href{https://www.nytimes3xbfgragh.onion/subscription?campaignId=37WXW}{Subscriptions}
\end{itemize}
