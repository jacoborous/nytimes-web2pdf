Sections

SEARCH

\protect\hyperlink{site-content}{Skip to
content}\protect\hyperlink{site-index}{Skip to site index}

\href{https://myaccount.nytimes3xbfgragh.onion/auth/login?response_type=cookie\&client_id=vi}{}

\href{https://www.nytimes3xbfgragh.onion/section/todayspaper}{Today's
Paper}

A Secret Ingredient Makes This Chef's Galbijjim Perfect. Just Don't Tell
Mom.

\url{https://nyti.ms/2HeK6kf}

\begin{itemize}
\item
\item
\item
\item
\item
\end{itemize}

Advertisement

\protect\hyperlink{after-top}{Continue reading the main story}

Supported by

\protect\hyperlink{after-sponsor}{Continue reading the main story}

\href{/column/magazine-eat}{Eat}

\hypertarget{a-secret-ingredient-makes-this-chefs-galbijjim-perfect-just-dont-tell-mom}{%
\section{A Secret Ingredient Makes This Chef's Galbijjim Perfect. Just
Don't Tell
Mom.}\label{a-secret-ingredient-makes-this-chefs-galbijjim-perfect-just-dont-tell-mom}}

\includegraphics{https://static01.graylady3jvrrxbe.onion/images/2019/01/20/magazine/20mag-eat-slide-A0VW/20mag-eat-slide-A0VW-articleLarge.png?quality=75\&auto=webp\&disable=upscale}

By \href{https://www.nytimes3xbfgragh.onion/by/sam-sifton}{Sam Sifton}

\begin{itemize}
\item
  Jan. 16, 2019
\item
  \begin{itemize}
  \item
  \item
  \item
  \item
  \item
  \end{itemize}
\end{itemize}

No one makes better \emph{galbijjim} than your mother, at least if she's
Korean and she's following the usual script about this fragrant stew of
short ribs and root vegetables. She boils the meat briefly, drains it
off, then cooks it slowly in a pot filled with puréed Asian pear and
mirin and soy sauce and brown sugar, garlic and onion, a lot of root
vegetables. Everything bubbles away softly until the dish comes together
in glossy, soft-shredded perfection. She serves the \emph{galbijjim}
with rice and kimchi. It tastes of home and innocence, though your
friend may raise an eyebrow. Her mom's version is better.

\emph{Galbijjim} takes a lot of time, and short ribs are expensive, so
she makes the dish only a few times a year --- for birthdays, for
holidays. You make the dish yourself from her instructions, and it
doesn't taste right. ``You need to add rice syrup,'' she might say in a
whisper, revealing a secret. ``You should have used apple juice. Where
are the chestnuts?''

My mother is as Korean as Marge Simpson. But I've cooked or eaten
versions of \emph{galbijjim} made by plenty of Korean mothers' children:
Emily Kim, the online cooking star known as Maangchi; the Los Angeles
chef and restaurateur Roy Choi; Hooni Kim, of Danji and Hanjan in New
York; Esther Choi of the Brooklyn restaurant Mokbar. The closest I've
come to feeling part of something larger than myself, as if I were now
cooking a recipe and eating a dish I could pass along to others like a
gift, was when I started eating and making the \emph{galbijjim} that the
chef Peter Cho serves at Han Oak, the restaurant he owns with his wife,
Sun Young Park, in Portland, Ore.

I first had that \emph{galbijjim} at a long communal table at Han Oak
last fall, eating with my own family, a family-style meal, as Cho came
out of the kitchen to bounce his son Frankie on his knee. It was
revelatory: Each flavor was distinct and deeply burnished, as if
assembled rather than stewed. Which turned out to be true, as it
happens. For the restaurant, Cho smokes the short ribs before using them
in the \emph{galbijjim}; roasts his vegetables; adds fried rice cakes to
sop up the jammy sauce; and swirls kale in at the end to add a bright
note.

\includegraphics{https://static01.graylady3jvrrxbe.onion/images/2019/01/20/magazine/20mag-eat-slide-DN7B/20mag-eat-slide-DN7B-articleLarge.png?quality=75\&auto=webp\&disable=upscale}

``It's less soupy than my mom's,'' he told me. He said he built his
recipe out of memories of hers, though, along with a desire to ape the
cooking of Korean barbecue restaurants in Los Angeles and New York and a
chef's need to balance textures and flavors, to make a dish larger than
its components. ``There are formulas to this stuff,'' he said. ``The
fresh greens against the rich meat, for example. You do it long enough,
and it begins to make sense.''

With Cho's help, I set out to make his \emph{galbijjim} at home, for a
family rather than a restaurant service, without access to a smoker or a
deep-fryer, or to the quince paste he uses to sweeten the braise in
place of honey. At his suggestion, I didn't boil the meat but roasted it
above a collection of root vegetables under aggressive sprays of kosher
salt and black pepper, on sheet pans so that everything could be spread
out wide. All that surface area led to a lot of caramelization, to a
depth of flavor that held through as the stew matured. Then, for the
braise, I combined Asian pears and red onion, chicken stock, honey, soy
sauce, rice-wine vinegar and plenty of red-pepper flakes, cooked it all
down, then blitzed the mixture in a food processor to make a smooth and
flavorful sauce. The roasted meat went into this concoction for more
than an hour, bubbling along until it was tender and starting to flake
from the bone.

The root vegetables followed. And then the rice cakes, hungry for sauce.
And Cho was right about the greens, which I added at the very end.
Barely softened by the heat, they served to offset the deep, unctuous
quality of the short ribs, the chewiness of the rice cakes. They
provided a perfect equilibrium.

Of course Cho's recipe had a secret ingredient. He is an exacting,
technical chef, who trained under April Bloomfield at the Spotted Pig in
New York and ran the kitchen at the Breslin for her before moving home
to Portland to be closer to family and finding fame at Han Oak. His
cooking is precise and beautiful; he serves his poached cod with charred
scallions and mushroom broth, with chanterelles, below dots of
laboriously made herb oil. He has lots of secrets. Here is one of them.
To the braising liquid for the short ribs, he told me early on, you want
to add a can of Coke. He offered the advice to do that the way a mother
might, with a shrug. It was not wrong. ``I'm not sure you'll even taste
it,'' Cho said, accurately. ``But it's kind of got to be in there.''

\textbf{Recipe:}
\href{https://cooking.nytimes3xbfgragh.onion/recipes/1019918-han-oak-galbijjim}{Han
Oak Galbijjim}

Advertisement

\protect\hyperlink{after-bottom}{Continue reading the main story}

\hypertarget{site-index}{%
\subsection{Site Index}\label{site-index}}

\hypertarget{site-information-navigation}{%
\subsection{Site Information
Navigation}\label{site-information-navigation}}

\begin{itemize}
\tightlist
\item
  \href{https://help.nytimes3xbfgragh.onion/hc/en-us/articles/115014792127-Copyright-notice}{©~2020~The
  New York Times Company}
\end{itemize}

\begin{itemize}
\tightlist
\item
  \href{https://www.nytco.com/}{NYTCo}
\item
  \href{https://help.nytimes3xbfgragh.onion/hc/en-us/articles/115015385887-Contact-Us}{Contact
  Us}
\item
  \href{https://www.nytco.com/careers/}{Work with us}
\item
  \href{https://nytmediakit.com/}{Advertise}
\item
  \href{http://www.tbrandstudio.com/}{T Brand Studio}
\item
  \href{https://www.nytimes3xbfgragh.onion/privacy/cookie-policy\#how-do-i-manage-trackers}{Your
  Ad Choices}
\item
  \href{https://www.nytimes3xbfgragh.onion/privacy}{Privacy}
\item
  \href{https://help.nytimes3xbfgragh.onion/hc/en-us/articles/115014893428-Terms-of-service}{Terms
  of Service}
\item
  \href{https://help.nytimes3xbfgragh.onion/hc/en-us/articles/115014893968-Terms-of-sale}{Terms
  of Sale}
\item
  \href{https://spiderbites.nytimes3xbfgragh.onion}{Site Map}
\item
  \href{https://help.nytimes3xbfgragh.onion/hc/en-us}{Help}
\item
  \href{https://www.nytimes3xbfgragh.onion/subscription?campaignId=37WXW}{Subscriptions}
\end{itemize}
