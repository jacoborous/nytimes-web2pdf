Sections

SEARCH

\protect\hyperlink{site-content}{Skip to
content}\protect\hyperlink{site-index}{Skip to site index}

\href{https://myaccount.nytimes3xbfgragh.onion/auth/login?response_type=cookie\&client_id=vi}{}

\href{https://www.nytimes3xbfgragh.onion/section/todayspaper}{Today's
Paper}

How Secrecy Fuels Facebook Paranoia

\url{https://nyti.ms/2HfE6HZ}

\begin{itemize}
\item
\item
\item
\item
\item
\end{itemize}

Advertisement

\protect\hyperlink{after-top}{Continue reading the main story}

Supported by

\protect\hyperlink{after-sponsor}{Continue reading the main story}

\href{/column/on-technology}{On Technology}

\hypertarget{how-secrecy-fuels-facebook-paranoia}{%
\section{How Secrecy Fuels Facebook
Paranoia}\label{how-secrecy-fuels-facebook-paranoia}}

\includegraphics{https://static01.graylady3jvrrxbe.onion/images/2019/01/20/magazine/20OnTech_Opener/20OnTech_Opener-articleLarge.png?quality=75\&auto=webp\&disable=upscale}

By \href{https://www.nytimes3xbfgragh.onion/by/john-herrman}{John
Herrman}

\begin{itemize}
\item
  Jan. 16, 2019
\item
  \begin{itemize}
  \item
  \item
  \item
  \item
  \item
  \end{itemize}
\end{itemize}

In December of last year, the Senate Intelligence Committee
\href{https://www.intelligence.senate.gov/press/new-reports-shed-light-internet-research-agency\%E2\%80\%99s-social-media-tactics}{released
two reports} it had commissioned concerning Russia's efforts to
influence the 2016 election. The outline of the interference effort has
always been easy to make out, but key questions about its significance
--- Could Russia really affect voter sentiment by posting? Did trolls
really lower voter turnout in key states? --- are still, in large part,
matters for speculation. And the committee's findings did little to
change that.

On the subject of Facebook, for example, the reports analyzed new
internal data on the campaign efforts of the Internet Research Agency,
Russia's ``troll farm,'' and offered some examples of its work. We know
that there were more than 76 million ``engagements'' with content from
the Russian group --- likes, comments, shares and clicks --- but it is
not currently possible to know how many came from real people. ``Since
Facebook did not provide data about any sock-puppet accounts involved in
the distribution of the content or the existence of `fake Likes' from
these accounts,'' one report says, ``we are operating under the
assumption that this engagement was from real people and that this
content was pushed into the News Feeds of their Friends as well.''

\includegraphics{https://static01.graylady3jvrrxbe.onion/images/2019/01/20/magazine/20OnTech_Secondary/20OnTech_Secondary-articleLarge.png?quality=75\&auto=webp\&disable=upscale}

Those are two enormous leaps. We do not currently know for sure whether
these engagements were carried out by humans or bots or a mixture of the
two, so the researchers were left to assume the worst. It's worth
scanning other recent Facebook narratives for similar assumptions. Did
Facebook cause the \emph{gilets jaunes} protests in France? Is the
movement ``a beast born almost entirely from Facebook,'' as
\href{https://www.buzzfeednews.com/article/ryanhatesthis/france-paris-yellow-jackets-facebook}{a
recent BuzzFeed piece} suggested? Or are the
\href{https://popula.com/2018/12/19/five-notes-on-the-yellow-vest-movement/}{protesters
just using Facebook} as they sustain a long tradition of civil unrest in
France?

The latter theory is obscured by the basic difficulty of discerning why
an individual feels a certain way, much less a diverse nation of
millions. The former, however, feels as though it has been hidden from
us, because Facebook should, in theory, be able to shed light on this
question. Its users live in a state of full surveillance, with them and
everything around them subject to near-total tracking. The company is at
least capable of knowing how a piece of content found its way from one
user to thousands or how a \emph{gilets jaunes} group functions on the
social platform. Far more than any outside researchers, Facebook is
capable of answering questions about the Internet Research Agency in
2016: For example, how many of the accounts that interacted with the
group's posts went on to interact with other political content? Was the
audience for these posts even real in the first place or part of the
operation itself? (Both reports criticized tech companies for doing the
bare minimum to assist in the committee's efforts.)

The biggest internet platforms are businesses built on asymmetric
information. They know far more about their advertising, labor and
commerce marketplaces than do any of the parties participating in them.
We can guess, but can't know, why we were shown a friend's Facebook post
about a divorce, instead of another's about a child's birth. We can
theorize, but won't be told, why YouTube thinks we want to see a
right-wing polemic about Islam in Europe after watching a video about
travel destinations in France. Everything that takes place within the
platform kingdoms is enabled by systems we're told must be kept private
in order to function. We're living in worlds governed by trade secrets.
No wonder they're making us all paranoid.

\textbf{The original sin} of our current tech hegemons is that in order
to work, the rest of the world can't know how they work. How does Google
trawl the web and produce search results? That's a secret that helped it
become the dominant search engine and that sustains its business model.
How does Facebook choose what comes next in your News Feed? How do
YouTube recommendations work? Too much transparency, the defense goes,
would expose them to reverse-engineering or abuse by bad actors.

The technologist Anil Dash recently described some of the internet's
biggest platform businesses as
\href{https://medium.com/humane-tech/tech-and-the-fake-market-tactic-8bd386e3d382}{``fake
markets.''} These are businesses that purport to be marketplaces, making
money by connecting parties --- people who want rides with drivers,
advertisers with eyeballs --- but are not actually markets in the strict
sense of the word. They're centrally and often assertively managed and
manipulated. Some hardly resemble markets at all. Dash singled out Uber:
In that ``market,'' drivers don't set prices, consumers don't actually
have much choice and resources are allocated by trade-secret algorithms.
Our ignorance of how such things work is easier to ignore when a
platform is establishing itself and sharing the benefits of its growth.
This dynamic only starts to bother us after a platform wins, when there
are fewer alternatives or none at all.

When platforms become entrenched and harder for users to leave, the
secrets they keep are reflected back to them as resentments. On
Facebook, where every user's experience is a mystery to all others ---
and where real-world concepts like privacy, obscurity and serendipity
have been recreated on the terms of an advertising platform --- users
understandably imagine that anything could be happening around them:
that their peers are being indoctrinated, tricked, sheltered or misled
on a host of issues. These gaps in knowledge are magnified once a
platform becomes involved in real-world events. Twitter is broadly
understood to be a catalytic political force, but this understanding is
usually half hunch --- its relationship to the Arab Spring is still in
dispute; its usefulness to Donald Trump will never be fully understood;
and Twitter seems to be in no hurry to help figure it out. Similarly: Is
YouTube reflecting a rise in reactionary politics, enabling it or
creating it? These are already impossible questions to answer in full;
that everything is unfolding inside of a closely guarded attention
marketplace makes them difficult to even approach.

So, did Facebook cause the gilets jaunes uprising in France? Maybe,
interesting theory; meanwhile, though, I don't even know why it's
recommending I friend someone I've never heard of. Did Facebook swing
the 2016 election? Could have, as far as we know; anyway, I can't even
guess why Instagram started showing me a bunch of photos of a certain
breed of dog or why it's suddenly serving me ads for meal kits. I know
how these things make me feel, but Facebook knows how they made me
behave --- knowledge it won't soon share.

It appears to be the tendency of the press, and of our imaginations in
general, to extend these theories in a particular direction. After years
of Facebook's telling us how good it is at connecting people and
influencing their decisions, it is tempting to say something like: Yes,
O.K., then wouldn't it have been easy to use these same tools to
persuade people to vote for Donald Trump? Facebook has equivocated on
this question in a telling way. It's helpful to imagine what it would
have to say to successfully combat claims Russia used its platform to
swing the election: Sure, this many users saw propaganda but only for a
moment; and besides, this content didn't seem to affect their behavior
in any way at all; sure, this amount of money was spent on ads, but
those ads don't appear to have done anything; yes, these Instagram
accounts had that many followers, but more than half of them were bots
themselves; O.K., 76 million people were exposed to this content in some
way or another, but they mostly glossed over it, like spam.

They've inched tellingly in this direction but going all the way would
involve unflattering disclosure --- the sort that would need to be
legally compelled. Mainly, it would be tantamount to admitting that the
systems we're not allowed to know about --- and the metrics we aren't
allowed to see --- might not be quite as valuable, or as worthy of trade
secrecy, as Facebook needs us to think they are. While it's true that
perceptions of the tech industry have shifted, they aren't necessarily
closer to reality. These companies mythologized their own omniscience
when it was a boon to their business. Versions of these myths persist,
but they're no longer under their creator's control --- and they're
starting to bite back.

Advertisement

\protect\hyperlink{after-bottom}{Continue reading the main story}

\hypertarget{site-index}{%
\subsection{Site Index}\label{site-index}}

\hypertarget{site-information-navigation}{%
\subsection{Site Information
Navigation}\label{site-information-navigation}}

\begin{itemize}
\tightlist
\item
  \href{https://help.nytimes3xbfgragh.onion/hc/en-us/articles/115014792127-Copyright-notice}{©~2020~The
  New York Times Company}
\end{itemize}

\begin{itemize}
\tightlist
\item
  \href{https://www.nytco.com/}{NYTCo}
\item
  \href{https://help.nytimes3xbfgragh.onion/hc/en-us/articles/115015385887-Contact-Us}{Contact
  Us}
\item
  \href{https://www.nytco.com/careers/}{Work with us}
\item
  \href{https://nytmediakit.com/}{Advertise}
\item
  \href{http://www.tbrandstudio.com/}{T Brand Studio}
\item
  \href{https://www.nytimes3xbfgragh.onion/privacy/cookie-policy\#how-do-i-manage-trackers}{Your
  Ad Choices}
\item
  \href{https://www.nytimes3xbfgragh.onion/privacy}{Privacy}
\item
  \href{https://help.nytimes3xbfgragh.onion/hc/en-us/articles/115014893428-Terms-of-service}{Terms
  of Service}
\item
  \href{https://help.nytimes3xbfgragh.onion/hc/en-us/articles/115014893968-Terms-of-sale}{Terms
  of Sale}
\item
  \href{https://spiderbites.nytimes3xbfgragh.onion}{Site Map}
\item
  \href{https://help.nytimes3xbfgragh.onion/hc/en-us}{Help}
\item
  \href{https://www.nytimes3xbfgragh.onion/subscription?campaignId=37WXW}{Subscriptions}
\end{itemize}
