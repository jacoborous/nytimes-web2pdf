Harry Reid Has a Few Words for Washington

\url{https://nyti.ms/2RpFiwi}

\begin{itemize}
\item
\item
\item
\item
\item
\item
\end{itemize}

\includegraphics{https://static01.graylady3jvrrxbe.onion/images/2019/01/06/magazine/06mag-reid-image1/06mag-reid-image1-articleLarge-v4.jpg?quality=75\&auto=webp\&disable=upscale}

Sections

\protect\hyperlink{site-content}{Skip to
content}\protect\hyperlink{site-index}{Skip to site index}

Feature

\hypertarget{harry-reid-has-a-few-words-for-washington}{%
\section{Harry Reid Has a Few Words for
Washington}\label{harry-reid-has-a-few-words-for-washington}}

The former Senate majority leader on President Trump and Senator Chuck
Schumer, and on why he doesn't regret ending the filibuster for judicial
appointments.

Harry Reid in Nevada in December.Credit...Dan Winters for The New York
Times

Supported by

\protect\hyperlink{after-sponsor}{Continue reading the main story}

By \href{https://www.nytimes3xbfgragh.onion/by/mark-leibovich}{Mark
Leibovich}

\begin{itemize}
\item
  Jan. 2, 2019
\item
  \begin{itemize}
  \item
  \item
  \item
  \item
  \item
  \item
  \end{itemize}
\end{itemize}

**E**arly on the afternoon of Dec. 11, about an hour after an Oval
Office meeting between President Trump, the Senate minority leader Chuck
Schumer and the incoming House speaker Nancy Pelosi devolved on live TV
into a shouting match --- a ``tinkle contest with a skunk,'' in Pelosi's
postgame grandiloquence --- I pulled up to a McMansion in a gated
community outside Las Vegas. I presented my ID and pre-issued bar-code
pass to a security guard. Another guard emerged from a sedan in the
driveway, instructed me to leave my rental car across the street and
pointed me to the front door.

``I put this out here because I knew you were coming,'' Harry Reid, the
former Senate leader, said, pointing to a large gold menorah on his
desk. It was not clear whether Reid had someone buy the menorah
especially for my visit or just keeps one lying around in case some
reporter of (nominal) Jewish identity happens to drop by around
Hanukkah. (Reid's wife, Landra, was raised in a Jewish household in Los
Angeles before she and Reid converted to Mormonism together, after they
married.) Either way, Reid seemed both amused and pleased with himself,
as if he could see that I was not quite sure how to receive this
odd-duck gesture. During his time in office, he always got a kick out of
embracing the awkward panders of political life, even if --- especially
if --- they mocked the refinements of smoother politicians than him.

Reid, who is 79, does not have long to live. I hate to be so abrupt
about this, but Reid probably would not mind. In May, he went in for a
colonoscopy, the results of which caused concern among his doctors. This
led to an M.R.I. that turned up a lesion on Reid's pancreas: cancer.
Reid's subdued and slightly cold manner, and aggressive anticharisma,
have always made him an admirably blunt assessor of situations,
including, now, his own: ``As soon as you discover you have something on
your pancreas, you're dead.''

I had planned to visit Reid, who had not granted an interview since his
cancer diagnosis, in November, but he put me off, saying he felt too
weak. People close to him were saying that he had months left, if not
weeks. Valedictories were planned, and lifetime awards were bestowed.
Efforts were underway to rename the Las Vegas airport in his honor,
preferably before his own time of departure. Reid refuses to believe
that this honor will ever happen. ``When I practiced law, I did a lot of
personal-injury work, and I never spent one penny until that check was
cashed,'' he explained to me.

When I went to see him in December, he was confined to a desk near the
front door of the house, unable to move without the aid of a walker that
rested behind him. Still, he looked better than I thought he would. The
last time I saw Reid, during the 2016 presidential campaign, he was
wearing dark glasses and was still bruised from a freakish
exercise-session mishap in early 2015, when an elastic band apparently
snapped and propelled him into some cabinets, breaking ribs and bones in
his face and blinding him in his right eye. The visible damage from this
incident had abated at last. Wearing a tan sweater over a dress shirt,
he looked about how he did a decade ago: roughly his current age, in
other words.

Reid's health, even before the cancer diagnosis, was a factor in opting
not to seek re-election for a sixth Senate term in 2016. Over the last
few months, he has had chemotherapy and two back surgeries and has
suffered a range of other ordeals, some related to the accident, for
which Trump delighted in mocking him. ``I think he should go back and
start working out again with his rubber workout pieces,'' Trump said in
an interview with The Washington Post in September 2016.

In fairness, Reid had dismissed Trump as a ``spoiled brat,'' a ``con
man'' and a ``human leech.'' As Senate majority leader, Reid was
essential to passing President Barack Obama's legislative agenda, but
his dead-eyed realism and morose tones always hung in contrast to the
hope-and-change intoxications of those years. His den is adorned with a
bright painted portrait of the Rev. Martin Luther King Jr. --- one of
his heroes, whose view that ``the arc of the moral universe is long, but
it bends toward justice'' was often echoed by Obama. But Reid himself
always seemed more predisposed to believing that the arc of the universe
bent toward an ornery brawl.

Reid once called the Federal Reserve chairman Alan Greenspan a
``political hack,'' Justice Clarence Thomas ``an embarrassment'' and
President George W. Bush a ``loser'' (for which he later apologized) and
a ``liar'' (for which he did not). In 2016, he dismissed Trump as ``a
big fat guy'' who ``didn't win many fights.'' Reid himself was more than
ready to fight, and fight dirty: ``I was always willing to do things
that others were not willing to do,'' he told me.

During the 2012 presidential campaign, he claimed, with no proof, that
Mitt Romney had not paid any taxes over the past decade. Romney released
tax returns showing that he did. After the election, Reid told CNN by
way of self-justification, ``Romney didn't win, did he?'' Reid took
rightful criticism over this. Still, in retrospect, there's something
almost quaint about the outrage over the episode; Trump routinely
surpasses Reid's unscrupulousness with a few tweets before breakfast.

Leaving Washington on the eve of Trump's takeover, Reid insisted that he
was happy to be escaping. Maybe, he allowed, it would have been
different if Hillary Clinton had won. But ``with this, no,'' he told New
York magazine at the time. ``I'm not going to miss it.''

And yet, two years later, it was easy to sense him pining for not just
the political action but also the particular political action of Trump's
Washington. ``No one would enjoy the fight with Trump like Harry Reid
would,'' said Senator Claire McCaskill, the Missouri Democrat who lost
her re-election race in November. The president ``is an inherently weak
man,'' she said. ``Harry would smell the weakness and say, `Damn the
consequences.' ''

In some ways, Washington, under Trump, has devolved into the feral state
that Reid, in his misanthropic heart, always knew it could become under
the right conditions. Politicians are always claiming to be eternal
optimists; Reid is no optimist. ``I figure, if you're pessimistic,
you're never disappointed,'' he told me.

\textbf{Reid has decided} to live out his last years in Henderson, a
fast-growing and transient Las Vegas suburb. His house is in the upscale
Anthem neighborhood: a fortified village of beige dwellings of various
sizes and otherwise indistinguishable appearances. There is a Witness
Protection Program vibe to the place, accentuated by the security
detail.

Reid attended high school in Henderson, hitchhiking 45 miles each way
from his hometown, Searchlight: a drive-through smudge of a town between
Las Vegas and Needles, Calif., which, in his youth, boasted at least a
half-dozen brothels and not a single church. His acidic outlook was
informed by his childhood, during which he endured extreme poverty and
dysfunction and substance abuse in his family. He took up boxing in high
school and put himself through George Washington University Law School
by working as a Capitol Police officer. Back in Nevada, he was schooled
in the piranha bowl of Las Vegas politics. This education included a
stint as Nevada's gaming chairman in the 1970s, which placed him in the
cross hairs of the Las Vegas mob. (Some of the plot of the film
``Casino'' was based loosely on Reid's experiences.) There were numerous
threats to his life and at least one actual attempt (a bomb discovered
under the hood of his family car).

The former F.B.I. director James Comey, after he was fired by Trump,
compared Trump to the head of a mafia family, with its codes of silence
and loyalty, its fear-based leadership style and fealty to a single
godfather. ``It's not about anything else except the boss,'' Comey said
in a recent interview at the 92nd Street Y in New York. Others have
drawn the same parallel, and I asked Reid if, given his unusually
relevant professional experience in this area, it rang true. Reid
expelled a quick and dismissive chuckle. ``Organized crime is a
business,'' he told me, ``and they are really good with what they do.
But they are better off when things are predictable. In my opinion, they
do not do well with chaos. And that's what we have going with Trump.''

Still, Reid added: ``Trump is an interesting person. He is not immoral
but is amoral. Amoral is when you shoot someone in the head, it doesn't
make a difference. No conscience.'' There was a hint of grudging respect
in Reid's tone, which he seemed to catch and correct. ``I think he is
without question the worst president we've ever had,'' he said. ``We've
had some bad ones, and there's not even a close second to him.'' He
added: ``He'll lie. He'll cheat. You can't reason with him.'' Once more,
a hint of wonder crept into his voice, as if he was describing a rogue
beast on the loose in a jungle that Reid knows well.

The Trump era and Reid's illness have occasioned an inevitable
reconsideration of Reid's legacy and all its contradictions. The
Affordable Care Act, which Reid managed to navigate past the
oppositional tactics of his persistent nemesis, the Republican Senate
leader (and now majority leader), Mitch McConnell, has so far withstood
McConnell and Trump's efforts to dismantle the legislation. Reid was
also prescient in urging the Obama administration and congressional
Republicans to go public about the investigation into Russian meddling
in the 2016 election; the letter that Republican leaders agreed to
co-sign weeks after they were briefed on the investigation did not
identify Russia by name. ``They did nothing --- or nothing that I'm
aware of,'' Reid said.

But McConnell's and Trump's own most substantial accomplishment to date,
the appointment to the federal bench of an unprecedented number of
conservative judges, including two Supreme Court justices who might well
end up hearing a challenge to the Affordable Care Act, was made vastly
easier by Reid's decision, in 2013, to get rid of the filibuster for
judicial appointments. Reid remains unrepentant about this. ``They can
say what they want,'' he told me. ``We had over 100 judges that we
couldn't get approved, so I had no choice. Either Obama's presidency
would be a joke or Obama's presidency would be one of fruition.''

Still, a certain nostalgia for the Senate leader has set in among
Democrats, even those who had their disagreements with him. McCaskill
was critical of Reid during their tenure together and did not back him
for caucus leader in 2014. There are two major components of a Senate
leader's job, she said. ``One is to make the trains run on time and
getting things done that his caucus believes in,'' McCaskill told me.
``But the trains need to be bright and shiny while they're running,''
she added, referring to the communication and messaging part of the job
that she said Reid was less well suited to.

McCaskill told Reid at the time that she did not plan to vote for him
and explained her reasons to him. He replied that she was the only one
of his nonsupporters who had the nerve to tell him directly. ``Oh, no,
why would I?'' Reid told me when I asked him if he felt betrayed. ``And
I won, didn't I?''

Reid's successor is Chuck Schumer, his former caucus deputy who
engineered much of the Senate Democrats' communications and campaign
strategy during Reid's tenure. They had been close during Reid's 12
years as Democratic leader, Reid serving as the arid desert yin to
Schumer's bombastic Brooklyn yang. When we spoke, Reid told me he did
not wish to be seen as second-guessing Schumer. ``My personal feeling
should have nothing to do with it,'' he said. But clearly Reid has more
than a few of those personal feelings. He has told confidants that he
felt Schumer was too eager to assume his job before Reid was ready to
leave. Reid has also criticized, privately, Schumer's instinct for
accommodation with both McConnell and Trump.

In our conversation, Reid seemed incapable of not constantly reminding
me that he did not wish to talk about Schumer, as if this itself was
something he wanted me to emphasize. ``I do not call Schumer,'' he told
me. Then: ``I call him once in a while --- not weekly. Let's say monthly
I may call him.'' This sounded straightforward enough until he added:
``I talk to Nancy often. I love Nancy Pelosi. We did so many good
things, and we still talk about that.'' And just the day before, he
said, he called Richard Durbin, the Illinois Democrat who, along with
Schumer, was Reid's top lieutenant in the Senate and is now Schumer's
Democratic whip. ``We came to the House together in 1982,'' Reid said of
Durbin. ``We had wonderful conversations.'' (Schumer declined to be
interviewed; his spokesman said in a statement that Schumer and Reid
``have different styles but they complemented each other well. They are
still good friends and talk regularly.'')

In fairness, there's little that any Democratic leader can do at a time
when the opposing party controls the presidency and both houses of
Congress, as Republicans did until this month. Durbin told me that he
has sat with Schumer and Trump together at the White House. ``They are
discussing things at a New York level that most of us on the outside
don't understand,'' Durbin said. ``With Chuck, it's his grandfather who
had some business with Trump's father or some darned thing. It's a
totally different ballgame.''

I asked Durbin whether this approach had yielded any results. ``The
obvious answer,'' he conceded, ``is it hasn't worked very well so far.''

**D**avid Krone, Reid's former chief of staff, is of the view that
leaving Washington saved Reid's life. ``He wouldn't be alive today if he
had pancreatic cancer and he was still the Senate leader,'' he told me.
``He would not have made it.'' Still, Krone said, ``I think he misses
it, definitely.''

When he was in Washington, Reid used to spend an inordinate amount of
time on the Senate floor. ``I was always afraid that I would miss
something,'' Reid used to say and told me again in Nevada. In
retirement, he said, ``For me to sit here and say I don't follow
politics --- you wouldn't believe me, O.K.?''

On the Friday afternoon before Christmas, just hours before the
government shut down over Trump's demands for more funding for a border
wall, I called Reid to see how closely he was following this latest
brinkmanship. ``Landra and I have been watching the news; we have it on
now,'' Reid told me. The shutdown, he allowed, was ``interesting.'' Reid
takes an anthropological interest in the changes that Trump has wrought
on his old institution. ``You can't legislate when you have a chief
executive who's weird, for lack of a better description,'' he told me.
He said he could never understand how his former Senate colleague Jeff
Sessions allowed himself to be so abused and humiliated by the
president. ``Why in the hell didn't Sessions leave?'' he said. ``Same
with Kelly,'' referring to the departing chief of staff, John Kelly.
``I'd say, `Go screw yourself.' I could not look my children in the
eye.''

I asked him if he could identify at all with Trump's dark worldview. ``I
disagree that Trump is a pessimist,'' Reid said, as if to allow him that
mantle would be paying him an undeserved compliment. ``I think he's a
person who is oblivious to the real world.''

One of Reid's assets as a leader, when he was in office, was his
willingness to feed the egos of his colleagues before his own; he was
happy to yield credit, attention and TV appearances. Yet when I visited
Reid in Nevada, I detected a whiff of, if not neediness per se, maybe a
need to remind me that he has not been forgotten. He told me that he
received a lovely call that morning from Barbara Boxer, the former
Democratic senator from California. He gets calls from his former
colleagues all the time, he said, and they tell Reid he is missed. He
had a final conversation with John McCain over the summer, just before
McCain died, punctuated with ``I love you''s.

Reading Reid can be difficult. Is he playing a game or working an angle
or even laughing at a private joke he just told himself? When speaking
of his final goodbye with McCain, he broke into a strange little grin,
his lips pressed upward as if he could have been stifling either
amusement or tears. It occurred to me that Reid, typically as self-aware
as he is unsentimental, could have been engaged in a gentle playacting
of how two old Senate combatants of a fast-vanishing era are supposed to
say goodbye to each other for posterity.

Reid seemed to recognize my puzzlement and shrugged. ``As has been
written since I left,'' he told me, ``I was kind of a strange guy.''

Advertisement

\protect\hyperlink{after-bottom}{Continue reading the main story}

\hypertarget{site-index}{%
\subsection{Site Index}\label{site-index}}

\hypertarget{site-information-navigation}{%
\subsection{Site Information
Navigation}\label{site-information-navigation}}

\begin{itemize}
\tightlist
\item
  \href{https://help.nytimes3xbfgragh.onion/hc/en-us/articles/115014792127-Copyright-notice}{©~2020~The
  New York Times Company}
\end{itemize}

\begin{itemize}
\tightlist
\item
  \href{https://www.nytco.com/}{NYTCo}
\item
  \href{https://help.nytimes3xbfgragh.onion/hc/en-us/articles/115015385887-Contact-Us}{Contact
  Us}
\item
  \href{https://www.nytco.com/careers/}{Work with us}
\item
  \href{https://nytmediakit.com/}{Advertise}
\item
  \href{http://www.tbrandstudio.com/}{T Brand Studio}
\item
  \href{https://www.nytimes3xbfgragh.onion/privacy/cookie-policy\#how-do-i-manage-trackers}{Your
  Ad Choices}
\item
  \href{https://www.nytimes3xbfgragh.onion/privacy}{Privacy}
\item
  \href{https://help.nytimes3xbfgragh.onion/hc/en-us/articles/115014893428-Terms-of-service}{Terms
  of Service}
\item
  \href{https://help.nytimes3xbfgragh.onion/hc/en-us/articles/115014893968-Terms-of-sale}{Terms
  of Sale}
\item
  \href{https://spiderbites.nytimes3xbfgragh.onion}{Site Map}
\item
  \href{https://help.nytimes3xbfgragh.onion/hc/en-us}{Help}
\item
  \href{https://www.nytimes3xbfgragh.onion/subscription?campaignId=37WXW}{Subscriptions}
\end{itemize}
