Sections

SEARCH

\protect\hyperlink{site-content}{Skip to
content}\protect\hyperlink{site-index}{Skip to site index}

\href{https://myaccount.nytimes3xbfgragh.onion/auth/login?response_type=cookie\&client_id=vi}{}

\href{https://www.nytimes3xbfgragh.onion/section/todayspaper}{Today's
Paper}

He Thought He Just Had the Flu at First. Then His Heart Could Barely
Pump.

\url{https://nyti.ms/2GSE9cs}

\begin{itemize}
\item
\item
\item
\item
\item
\item
\end{itemize}

Advertisement

\protect\hyperlink{after-top}{Continue reading the main story}

Supported by

\protect\hyperlink{after-sponsor}{Continue reading the main story}

\href{/column/diagnosis}{Diagnosis}

\hypertarget{he-thought-he-just-had-the-flu-at-first-then-his-heart-could-barely-pump}{%
\section{He Thought He Just Had the Flu at First. Then His Heart Could
Barely
Pump.}\label{he-thought-he-just-had-the-flu-at-first-then-his-heart-could-barely-pump}}

\includegraphics{https://static01.graylady3jvrrxbe.onion/images/2019/01/06/magazine/06Mag-Diagnosis-1/06Mag-Diagnosis-1-articleLarge.png?quality=75\&auto=webp\&disable=upscale}

By \href{https://www.nytimes3xbfgragh.onion/by/lisa-sanders-md}{Lisa
Sanders, M.D.}

\begin{itemize}
\item
  Jan. 2, 2019
\item
  \begin{itemize}
  \item
  \item
  \item
  \item
  \item
  \item
  \end{itemize}
\end{itemize}

They weren't looking for a diagnosis, the middle-aged woman explained.
Her husband \emph{had} a diagnosis. They just wanted help figuring out
why, even with all the treatments he was getting, he wasn't getting
better.

Until a year and a half earlier, her 54-year-old husband had been
perfectly healthy. Never missed a day of work, never took so much as an
aspirin. Then he got what he thought was the flu. But even after the
fever and congestion went away, the terrible body aches remained. He
coughed constantly and felt so tired that just walking to the mailbox
would leave him panting for air and shaking with fatigue.

Still, he went back to work. He enjoyed his job driving a locomotive for
a manufacturing plant in rural Connecticut. Besides, his wife told me,
he was a guy who needed to be busy. And he stayed busy until, one
morning a couple of weeks after starting back at the job, he was driving
to work and suddenly found himself rumbling over the shoulder, drifting
toward the strip of grass and the woods beyond. All he remembered was
that one minute he was on the road, and the next he wasn't. For the
first time since he initially got sick, he was worried.

\hypertarget{draining-liters-of-fluids}{%
\subsection{\texorpdfstring{\textbf{Draining Liters of
Fluids}}{Draining Liters of Fluids}}\label{draining-liters-of-fluids}}

He went to an urgent-care center. A chest X-ray showed fluid surrounding
his lungs. And his EKG was abnormal. The nurse who saw the man was
concerned. She ordered an antibiotic to treat a possible pneumonia and
referred him to a cardiologist.

That was the first of many doctor's appointments but the last one he
went to alone. His wife was worried that her husband, a quiet man who
seemed to be wasting away in front of her eyes, wouldn't ask the
questions that needed to be asked.

The cardiologist ordered a second chest X-ray, which showed even more
fluid in the sack surrounding his lungs --- so much that it was hard for
the man to take a deep breath. The cardiologist sent him to have the
sack drained of more than a liter of a clear yellow fluid. It made the
patient feel better, but it didn't last; within days the shortness of
breath returned.

Two weeks later, when the man saw a pulmonologist, that doctor referred
him to have another liter of the same yellow fluid drained from his
lungs. The man's abdomen began to swell with even more fluid. Where was
it all coming from? No one could tell him. After the fluid reaccumulated
just as quickly, his doctors sent the man to the hospital.

There, an echocardiogram --- an ultrasound of the heart --- showed that
he now had fluid in the pericardium, the sack surrounding his heart. The
heart could barely pump. He was rushed to the operating room, and a hole
was cut into the pericardium to allow the fluid to drain and give his
heart room enough to beat normally. A retinue of subspecialists searched
for an explanation for this flood of fluids. His heart was strong, he
was told. His lungs were fine. His liver was fine. There was no
infection, and no cancer.

\hypertarget{an-autoimmue-disorder--or-two}{%
\subsection{\texorpdfstring{\textbf{An Autoimmue Disorder --- or
Two?}}{An Autoimmue Disorder --- or Two?}}\label{an-autoimmue-disorder--or-two}}

Finally, a rheumatologist found an answer. The patient tested positive
for something called Sjogren's syndrome. In this autoimmune disorder,
white blood cells attack the organs that make the fluids needed to
lubricate the body. Those with Sjogren's often have dry eyes because
they don't make enough tears, or a dry mouth because they don't make
enough saliva. They have problems with dry skin, as well as with their
joints and G.I. tract.

\includegraphics{https://static01.graylady3jvrrxbe.onion/images/2019/01/06/magazine/06Mag-Diagnosis-2/06Mag-Diagnosis-2-articleLarge.png?quality=75\&auto=webp\&disable=upscale}

The patient and his doctors were thrilled to finally have a diagnosis.
Still, most with Sjogren's don't need treatment for the disease, only
for the annoying symptoms of dryness and discomfort it causes. It was
strange that the syndrome would, in this patient, produce this
extraordinary fluid overload. The rheumatologist theorized that a second
disease called undifferentiated connective tissue disorder (U.C.T.D.)
might be involved. The patient was immediately started on two
immune-suppressing medications.

He continued to need to have his lungs and abdomen drained of 10 to 20
liters of fluids every couple of weeks. More drugs were added. But when,
after months of treatment, he was no better, his wife insisted they get
a second opinion. He was referred to a rheumatologist in New York. That
doctor suggested still another immune-suppressing agent.

\hypertarget{when-a-remedy-fails}{%
\subsection{\texorpdfstring{\textbf{When a Remedy
Fails}}{When a Remedy Fails}}\label{when-a-remedy-fails}}

The patient was on four of these medications when I first met him this
summer. Despite the medications, he continued to have liters of fluid
drained from his belly and around his lungs.

After hearing about their terrible journey, I examined the patient
carefully, trying to find some clue to what could be going on. His arms
were thin and wiry, just bone and sinewy muscle; the overlying skin hung
loosely, reflecting significant muscle loss. In contrast, his abdomen
was huge --- the belly of two Santas. The skin there was drum-tight. His
neck, like his arms, was thin, and the veins on each side were hugely
distended with blood.

Once he had dressed and I was able to gather my thoughts, I told the
couple that only the heart could cause such a huge buildup of fluids.
No, the man emphatically said: My cardiologist assured me that all the
tests show that my heart is strong. I told them I'd pore through the
thick folder of carefully organized records they had collected and come
up with a plan.

I didn't believe it was his autoimmune disease causing all this. Even
though he had Sjogren's and possibly U.C.T.D., he was being treated. And
when a remedy fails, you must consider the possibility that what it's
treating is not the cause of the problem, and ask: What else could this
be?

\hypertarget{trust-but-verify}{%
\subsection{\texorpdfstring{\textbf{Trust, but
Verify}}{Trust, but Verify}}\label{trust-but-verify}}

So I dug and thought and came up with a list of rarities that could
cause these symptoms. I put the question to my friend and mentor at
Yale, Andre Sofair, an internist on the faculty of the program where I
had trained and now taught. His answer was familiar --- surely this was
the heart. I told him what the patient told me, that his heart had been
tested and was in the clear. Andre was surprised but turned his mind to
thinking of other causes. He added a couple more items to my list.

I sent the patient for more tests, and when they came up with nothing, I
thought back to Andre's first instinct. Was it his heart? In such a
setting, I think of the Russian proverb Ronald Reagan used during his
negotiations with Mikhail Gorbachev on the treaty regulating
intermediate-range missiles: One must trust, but also verify.

When the heart muscle has been damaged by, say, a heart attack, it
doesn't pump as well, and fluids can back up. We call that congestive
heart failure, and it was one possibility. But the bulging neck veins I
saw suggested another, rarer possibility: constrictive pericarditis. In
this disorder, the pericardium is injured --- usually by a viral
infection --- and as it heals, it shrinks. Stuck in this shrunken
jacket, the heart can only pump a fraction of the blood needed by the
body. Could the virus that caused the flulike symptoms at the start of
this illness have attacked his pericardium?

\hypertarget{back-to-the-heart}{%
\subsection{\texorpdfstring{\textbf{Back to the
Heart}}{Back to the Heart}}\label{back-to-the-heart}}

I sent the patient for another echocardiogram. It showed a heart pumping
hard but constrained inside a shrunken, thickened pericardium, unable to
process the normal measure of blood. I spoke with his rheumatologist,
who stopped all the immune-suppressing agents, and I sent him to John
Elefteriades, a well-respected heart surgeon at Yale. Elefteriades cut
away the damaged sack. Once he made the initial incision down the length
of the scarred pericardium, the blood flow through the heart more than
doubled.

The man's recovery from this surgery has been remarkably fast. He had a
foot-long incision down the middle of his chest, but within two weeks of
the operation he was home and walking around. Three weeks later he was
back at work --- just in time for him and his wife to prepare for a
holiday season they had worried they would never see together.

Advertisement

\protect\hyperlink{after-bottom}{Continue reading the main story}

\hypertarget{site-index}{%
\subsection{Site Index}\label{site-index}}

\hypertarget{site-information-navigation}{%
\subsection{Site Information
Navigation}\label{site-information-navigation}}

\begin{itemize}
\tightlist
\item
  \href{https://help.nytimes3xbfgragh.onion/hc/en-us/articles/115014792127-Copyright-notice}{©~2020~The
  New York Times Company}
\end{itemize}

\begin{itemize}
\tightlist
\item
  \href{https://www.nytco.com/}{NYTCo}
\item
  \href{https://help.nytimes3xbfgragh.onion/hc/en-us/articles/115015385887-Contact-Us}{Contact
  Us}
\item
  \href{https://www.nytco.com/careers/}{Work with us}
\item
  \href{https://nytmediakit.com/}{Advertise}
\item
  \href{http://www.tbrandstudio.com/}{T Brand Studio}
\item
  \href{https://www.nytimes3xbfgragh.onion/privacy/cookie-policy\#how-do-i-manage-trackers}{Your
  Ad Choices}
\item
  \href{https://www.nytimes3xbfgragh.onion/privacy}{Privacy}
\item
  \href{https://help.nytimes3xbfgragh.onion/hc/en-us/articles/115014893428-Terms-of-service}{Terms
  of Service}
\item
  \href{https://help.nytimes3xbfgragh.onion/hc/en-us/articles/115014893968-Terms-of-sale}{Terms
  of Sale}
\item
  \href{https://spiderbites.nytimes3xbfgragh.onion}{Site Map}
\item
  \href{https://help.nytimes3xbfgragh.onion/hc/en-us}{Help}
\item
  \href{https://www.nytimes3xbfgragh.onion/subscription?campaignId=37WXW}{Subscriptions}
\end{itemize}
