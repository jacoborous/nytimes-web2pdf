Sections

SEARCH

\protect\hyperlink{site-content}{Skip to
content}\protect\hyperlink{site-index}{Skip to site index}

\href{https://www.nytimes3xbfgragh.onion/section/science}{Science}

\href{https://myaccount.nytimes3xbfgragh.onion/auth/login?response_type=cookie\&client_id=vi}{}

\href{https://www.nytimes3xbfgragh.onion/section/todayspaper}{Today's
Paper}

\href{/section/science}{Science}\textbar{}Watch Scientists Brew Their
Own Lava

\url{https://nyti.ms/2RE31sO}

\begin{itemize}
\item
\item
\item
\item
\item
\item
\end{itemize}

Advertisement

\protect\hyperlink{after-top}{Continue reading the main story}

Supported by

\protect\hyperlink{after-sponsor}{Continue reading the main story}

sciencetake

\hypertarget{watch-scientists-brew-their-own-lava}{%
\section{Watch Scientists Brew Their Own
Lava}\label{watch-scientists-brew-their-own-lava}}

\includegraphics{https://static01.graylady3jvrrxbe.onion/images/2019/01/09/science/08SCI-TAKE-promo/explosion-side-videoSixteenByNine3000.jpg}

By
\href{https://www.nytimes3xbfgragh.onion/by/nicholas-st-fleur}{Nicholas
St. Fleur}

\begin{itemize}
\item
  Jan. 8, 2019
\item
  \begin{itemize}
  \item
  \item
  \item
  \item
  \item
  \item
  \end{itemize}
\end{itemize}

\href{https://www.nytimes3xbfgragh.onion/es/2019/01/12/volcanes-explosiones-lava/}{Leer
en español}

When lava meets water, the results are often explosive.

Last year, lava from Hawaii's Kilauea volcano flowed into the ocean,
creating bombs of molten rock that were flung into the sky and then
smashed into a nearby tourist boat,
\href{https://www.nytimes3xbfgragh.onion/2018/07/17/us/lava-bomb-boat-video-hawaii.html}{injuring
23 people}.

In 2010, a glacier-covered volcano in Iceland called Eyjafjallajökull
erupted and spewed a plume of ash 30,000 feet into the air, causing
\href{https://www.nytimes3xbfgragh.onion/2010/04/16/world/europe/16ash.html?rref=collection\%2Ftimestopic\%2FEyjafjallajokull\%20Volcano\&action=click\&contentCollection=timestopics\&region=stream\&module=stream_unit\&version=latest\&contentPlacement=20\&pgtype=collection}{hundreds
of flights in Europe to be grounded}.

Scientists want to better understand these violent reactions to help
prepare communities near volcanoes and bodies of water or groundwater.
But doing so at active sites can be impractical. Instead, a team of
researchers recently brewed their own backyard lava.

``We are not just crazy people mixing and seeing what happens,'' said
\href{https://www.acsu.buffalo.edu/~ingomark/}{Ingo Sonder}, a
volcanologist at the University at Buffalo. ``We are scientists and we
want to quantify, and we do have an idea of what we are doing here.''

First, Dr. Sonder and his colleagues got black chunks of ancient
solidified lava, called basalt, from a quarry in Texas. They poured
about 120 pounds of basalt into a crucible inside a furnace. Over four
hours, with a few occasional stirs, the furnace heated the rocks to
about 2,400 degrees Fahrenheit, until the basalt became a bubbling
molten mix.

\textbf{\emph{{[}}\href{http://on.fb.me/1paTQ1h}{\emph{Like the Science
Times page on Facebook.}}} ****** \emph{\textbar{} Sign up for the}
\textbf{\href{http://nyti.ms/1MbHaRU}{\emph{Science Times
newsletter.}}\emph{{]}}}

Donning silver thermal suits to protect against the intense heat and
radiation, the researchers then poured 10 gallons of glowing goop into a
series of insulated steel boxes. The containers varied in size: Some
were long and flat, while others were tall and narrow, like a chimney,
in order to replicate the types of magma columns found in nature.

The walls of the steel containers had injectors designed to spray
pressurized water into the piping hot lava. After each contraption was
moved a safe distance from the shed that housed the furnace, the
scientists initiated a countdown on their computers.

Often, the instant the water hit the lava, it exploded, sending a
blazing blob about six feet into the air, and a few lava bombs as high
as 15 feet. But other times nothing happened, so the researchers used a
remotely controlled sledgehammer to knock the container and trigger the
blast.

The team used high-speed cameras to catch video of the explosions and
published the first of their results, still preliminary, last month in
the journal
\href{https://agupubs.onlinelibrary.wiley.com/doi/10.1029/2018JB015682}{JGR-Solid
Earth}. They plan to continue brewing lava, carrying out the experiment
with differently shaped containers and varying amounts of water.

``If we have a better idea of what conditions cause these violent
lava-water explosions,'' said
\href{http://www.buffalo.edu/news/experts/greg-valentine-faculty-expert-volcanoes.html}{Greg
Valentine}, a volcanologist at the University at Buffalo and an author
on the paper, ``then we can do a better job of kind of warning people
and mitigating the hazard.''

Advertisement

\protect\hyperlink{after-bottom}{Continue reading the main story}

\hypertarget{site-index}{%
\subsection{Site Index}\label{site-index}}

\hypertarget{site-information-navigation}{%
\subsection{Site Information
Navigation}\label{site-information-navigation}}

\begin{itemize}
\tightlist
\item
  \href{https://help.nytimes3xbfgragh.onion/hc/en-us/articles/115014792127-Copyright-notice}{©~2020~The
  New York Times Company}
\end{itemize}

\begin{itemize}
\tightlist
\item
  \href{https://www.nytco.com/}{NYTCo}
\item
  \href{https://help.nytimes3xbfgragh.onion/hc/en-us/articles/115015385887-Contact-Us}{Contact
  Us}
\item
  \href{https://www.nytco.com/careers/}{Work with us}
\item
  \href{https://nytmediakit.com/}{Advertise}
\item
  \href{http://www.tbrandstudio.com/}{T Brand Studio}
\item
  \href{https://www.nytimes3xbfgragh.onion/privacy/cookie-policy\#how-do-i-manage-trackers}{Your
  Ad Choices}
\item
  \href{https://www.nytimes3xbfgragh.onion/privacy}{Privacy}
\item
  \href{https://help.nytimes3xbfgragh.onion/hc/en-us/articles/115014893428-Terms-of-service}{Terms
  of Service}
\item
  \href{https://help.nytimes3xbfgragh.onion/hc/en-us/articles/115014893968-Terms-of-sale}{Terms
  of Sale}
\item
  \href{https://spiderbites.nytimes3xbfgragh.onion}{Site Map}
\item
  \href{https://help.nytimes3xbfgragh.onion/hc/en-us}{Help}
\item
  \href{https://www.nytimes3xbfgragh.onion/subscription?campaignId=37WXW}{Subscriptions}
\end{itemize}
