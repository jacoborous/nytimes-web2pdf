Sections

SEARCH

\protect\hyperlink{site-content}{Skip to
content}\protect\hyperlink{site-index}{Skip to site index}

\href{https://www.nytimes3xbfgragh.onion/section/business}{Business}

\href{https://myaccount.nytimes3xbfgragh.onion/auth/login?response_type=cookie\&client_id=vi}{}

\href{https://www.nytimes3xbfgragh.onion/section/todayspaper}{Today's
Paper}

\href{/section/business}{Business}\textbar{}It's Official: The Trump Tax
Cuts Didn't Pay for Themselves in Year One

\url{https://nyti.ms/2RGkXDc}

\begin{itemize}
\item
\item
\item
\item
\item
\end{itemize}

Advertisement

\protect\hyperlink{after-top}{Continue reading the main story}

Supported by

\protect\hyperlink{after-sponsor}{Continue reading the main story}

News analysis

\hypertarget{its-official-the-trump-tax-cuts-didnt-pay-for-themselves-in-year-one}{%
\section{It's Official: The Trump Tax Cuts Didn't Pay for Themselves in
Year
One}\label{its-official-the-trump-tax-cuts-didnt-pay-for-themselves-in-year-one}}

Federal tax revenues declined in 2018 while economic growth accelerated,
undercutting the Trump administration's insistence that the \$1.5
trillion tax package would pay for itself.

\href{https://www.nytimes3xbfgragh.onion/by/jim-tankersley}{\includegraphics{https://static01.graylady3jvrrxbe.onion/images/2018/10/19/multimedia/author-jim-tankersley/author-jim-tankersley-thumbLarge.png}}

By \href{https://www.nytimes3xbfgragh.onion/by/jim-tankersley}{Jim
Tankersley}

\begin{itemize}
\item
  Jan. 11, 2019
\item
  \begin{itemize}
  \item
  \item
  \item
  \item
  \item
  \end{itemize}
\end{itemize}

It's time to put to rest any notion that President Trump's signature tax
cuts
\href{https://www.breitbart.com/economy/2019/01/09/deficitroseandsodidtaxes/}{are
paying for themselves}. Anyone who says otherwise is lying with numbers.

A year after the \$1.5 trillion tax-cut package took effect, economic
growth has accelerated, just as Republicans promised it would when
pushing the law through Congress. Growth appears likely to hit 3 percent
for 2018, after adjusting for inflation, which is a full percentage
point higher than the Congressional Budget Office forecast for the year
in 2017. Not all of that increase is attributable to the tax cuts, but
some of it is.

That's good news for Republicans' longstanding claim that cutting taxes
would provide such an economic bump that additional tax revenue would
flow in to make up for what was lost through lower tax rates.

But the bad news is that hasn't happened. The additional tax revenue has
yet to show up, even with stronger growth.

Data released this week by the budget office provides the first complete
picture of federal revenues for the 2018 calendar year, when the tax
cuts were in full effect. (The government's 2018 fiscal year included
three months from the end of 2017, when most of the tax cuts were not in
effect.)

In the inaugural year of the tax cuts --- with economic growth
accelerating and the jobless rate falling to an 18-year low --- federal
revenues from corporate, payroll and personal income taxes actually **
fell.

That's true whether you adjust revenues and growth for inflation --- or
not.

After adjusting, it looks even worse. Revenues fell by 2.7 percent ---
or \$83 billion --- from 2017. Contrast that with the last time economic
growth approached 3 percent, back in 2015. The economy grew by 2.9
percent after adjusting for inflation that year --- and tax revenues
grew by 7 percent.

The historical contrast makes the drop-off look even steeper. Typically,
economists expect stronger growth to generate more revenue. People earn
more money, corporations generate higher profits and they all pay taxes
on it.

The way most economists ``score'' a tax proposal is to ask how it would
change revenue levels compared to what you would expect the government
to collect if the tax cut had not passed --- what economists call a
``baseline.''

In the summer of 2017, for example, the budget office projected that the
economy would grow by 2 percent in the 2018 fiscal year, and that
personal, corporate and payroll taxes would add up to \$3.24 trillion.
Then the tax cuts passed, growth accelerated and, for the 2018 fiscal
year, tax revenues fell \$183 billion --- or 5.6 percent --- short of
that projection.

Republicans, particularly in the Trump administration, sold the tax law
on claims that it would pay for itself --- even when economists outside
the administration, like the congressional Joint Committee on Taxation,
released models contradicting them. As corporate tax receipts fell
significantly last year, some Republicans began to insist that, in fact,
the bill was paying for itself, because total tax revenues were very
slightly up.

The 2018 figures contradict that argument, too.

The uncomfortable truth for the bill's supporters is that the tax cuts
are substantially contributing to a widening federal budget deficit,
which now appears on track to top \$1 trillion this year. If growth
fades in the coming years --- as many economists believe it will --- the
cuts could exacerbate the deficit even more.

The best-case scenario for proponents is that the cuts spur a sustained
increase in productivity and growth, which in turn produces increasingly
higher revenues several years down the road --- enough to reduce the
``cost'' of the bill to the budget deficit.

The 2018 results are, oddly enough, what a lot of economists predicted
would happen with Mr. Trump's cuts, including ones who generally favor
tax cuts. Total federal revenues in 2018 came in roughly where the Tax
Foundation, a Washington think tank that typically projects large growth
boosts from tax cuts, had forecast --- which is to say, well below the
budget office's baseline.

Just because the new law helped to increase economic growth, said Kyle
Pomerleau, an economist with the Tax Foundation, ``it doesn't mean that
it is going to pay for itself.'' Mr. Pomerleau said additional growth
from the law ``will continue to be modest over the next couple of
years.''

``That will offset some of the initial cost,'' he continued, ``but it
will still be nowhere near enough to make the tax cut self-financing.''

In December 2017, as Republicans sped the tax cuts through Congress, the
Tax Foundation released a projection that the cuts would add about \$450
billion to federal deficits over 10 years, after accounting for the
additional economic growth it would spur. The group has since redone the
analysis, with what Mr. Pomerleau called improvements to its
methodology. It now predicts deficits will increase by \$900 billion ---
double its original forecast.

Advertisement

\protect\hyperlink{after-bottom}{Continue reading the main story}

\hypertarget{site-index}{%
\subsection{Site Index}\label{site-index}}

\hypertarget{site-information-navigation}{%
\subsection{Site Information
Navigation}\label{site-information-navigation}}

\begin{itemize}
\tightlist
\item
  \href{https://help.nytimes3xbfgragh.onion/hc/en-us/articles/115014792127-Copyright-notice}{©~2020~The
  New York Times Company}
\end{itemize}

\begin{itemize}
\tightlist
\item
  \href{https://www.nytco.com/}{NYTCo}
\item
  \href{https://help.nytimes3xbfgragh.onion/hc/en-us/articles/115015385887-Contact-Us}{Contact
  Us}
\item
  \href{https://www.nytco.com/careers/}{Work with us}
\item
  \href{https://nytmediakit.com/}{Advertise}
\item
  \href{http://www.tbrandstudio.com/}{T Brand Studio}
\item
  \href{https://www.nytimes3xbfgragh.onion/privacy/cookie-policy\#how-do-i-manage-trackers}{Your
  Ad Choices}
\item
  \href{https://www.nytimes3xbfgragh.onion/privacy}{Privacy}
\item
  \href{https://help.nytimes3xbfgragh.onion/hc/en-us/articles/115014893428-Terms-of-service}{Terms
  of Service}
\item
  \href{https://help.nytimes3xbfgragh.onion/hc/en-us/articles/115014893968-Terms-of-sale}{Terms
  of Sale}
\item
  \href{https://spiderbites.nytimes3xbfgragh.onion}{Site Map}
\item
  \href{https://help.nytimes3xbfgragh.onion/hc/en-us}{Help}
\item
  \href{https://www.nytimes3xbfgragh.onion/subscription?campaignId=37WXW}{Subscriptions}
\end{itemize}
