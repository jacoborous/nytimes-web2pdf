Sections

SEARCH

\protect\hyperlink{site-content}{Skip to
content}\protect\hyperlink{site-index}{Skip to site index}

\href{https://www.nytimes3xbfgragh.onion/section/politics}{Politics}

\href{https://myaccount.nytimes3xbfgragh.onion/auth/login?response_type=cookie\&client_id=vi}{}

\href{https://www.nytimes3xbfgragh.onion/section/todayspaper}{Today's
Paper}

\href{/section/politics}{Politics}\textbar{}Elizabeth Warren's Higher
Education Plan: Cancel Student Debt and Eliminate Tuition

\url{https://nyti.ms/2GuaqmF}

\begin{itemize}
\item
\item
\item
\item
\item
\item
\end{itemize}

\begin{itemize}
\item
  \href{https://www.nytimes3xbfgragh.onion/live/2020/09/07/us/trump-vs-biden?action=click\&pgtype=Article\&state=default\&region=TOP_BANNER\&context=storylines_menu}{Election
  Updates}
\item
  \href{https://www.nytimes3xbfgragh.onion/interactive/2020/us/elections/election-states-biden-trump.html?action=click\&pgtype=Article\&state=default\&region=TOP_BANNER\&context=storylines_menu}{Paths
  to 270}
\item
  \href{https://www.nytimes3xbfgragh.onion/interactive/2020/08/31/us/politics/vote-by-mail-deadlines.html?action=click\&pgtype=Article\&state=default\&region=TOP_BANNER\&context=storylines_menu}{Voting
  by Mail}
\item
  \href{https://www.nytimes3xbfgragh.onion/interactive/2019/us/elections/2020-presidential-election-calendar.html?action=click\&pgtype=Article\&state=default\&region=TOP_BANNER\&context=storylines_menu}{Key
  Dates}
\item
  \href{https://www.nytimes3xbfgragh.onion/newsletters/politics?action=click\&pgtype=Article\&state=default\&region=TOP_BANNER\&context=storylines_menu}{Politics
  Newsletter}
\end{itemize}

Advertisement

\protect\hyperlink{after-top}{Continue reading the main story}

Supported by

\protect\hyperlink{after-sponsor}{Continue reading the main story}

\hypertarget{elizabeth-warrens-higher-education-plan-cancel-student-debt-and-eliminate-tuition}{%
\section{Elizabeth Warren's Higher Education Plan: Cancel Student Debt
and Eliminate
Tuition}\label{elizabeth-warrens-higher-education-plan-cancel-student-debt-and-eliminate-tuition}}

\includegraphics{https://static01.graylady3jvrrxbe.onion/images/2019/04/22/us/22warren1/22warren1-articleLarge.jpg?quality=75\&auto=webp\&disable=upscale}

By \href{https://www.nytimes3xbfgragh.onion/by/astead-w-herndon}{Astead
W. Herndon}

\begin{itemize}
\item
  April 22, 2019
\item
  \begin{itemize}
  \item
  \item
  \item
  \item
  \item
  \item
  \end{itemize}
\end{itemize}

Senator Elizabeth Warren of Massachusetts, who has structured her
presidential campaign around a steady unveiling of disruptive policy
ideas, on Monday proposed her biggest one yet: a \$1.25 trillion plan to
reshape higher education by canceling most student loan debt and
eliminating tuition at every public college.

Ms. Warren's sweeping plan has several planks. She would pay for it with
revenue generated by her proposed
\href{https://www.nytimes3xbfgragh.onion/2019/01/24/us/politics/wealth-tax-democrats.html}{increase
in taxes for America's most wealthy families} and corporations, which
the campaign estimates to be \$2.75 trillion over 10 years. In addition
to eliminating undergraduate tuition at public colleges and
universities, she would expand federal grants to help students with
nontuition expenses and create a \$50 billion fund to support
historically black colleges and universities.

She would eliminate up to \$50,000 in student loan debt for every person
with a household income of less than \$100,000; borrowers who make
between \$100,000 and \$250,000 would have a portion of their debt
forgiven.

``This touches people's lives,'' Ms. Warren said in an interview. ``This
is a chance to talk about what's broken and how we fix it. This is the
American dream.''

\emph{{[}}\href{https://www.nytimes3xbfgragh.onion/newsletters/politics?smid=rd?action=click\&module=Intentional\&pgtype=Article}{\emph{Sign
up for our politics newsletter}} \emph{and join the conversation around
the 2020 presidential race.{]}}

Ms. Warren has already offered proposals to provide
\href{https://www.nytimes3xbfgragh.onion/2019/02/19/us/politics/elizabeth-warren-child-care.html}{universal
child care} and
\href{https://www.theroot.com/sen-elizabeth-warren-breaks-down-america-s-ugly-histor-1829439424}{expand
affordable housing}, paid for by her new taxes on the wealthy. She has
called for the
\href{https://www.nytimes3xbfgragh.onion/2019/03/18/us/politics/elizabeth-warren-town-hall-electoral-college.html}{elimination
of the Electoral College}, and promised to expand the role of the
federal government in reining in unfettered capitalism, with separate
proposals on
\href{https://www.nytimes3xbfgragh.onion/2019/03/08/us/politics/elizabeth-warren-amazon.html}{breaking
up technological giants such} as Amazon and Facebook, new regulations
aimed at
\href{https://www.nytimes3xbfgragh.onion/2019/04/15/us/politics/elizabeth-warren-public-lands.html}{protecting
public lands}, and reversing consolidation in the agriculture sector.

The furious pace of the announcements is partly intended to help Ms.
Warren stand out in the Democratic field, which features more than a
dozen candidates unique in ideology and identity. She is also hoping to
create a progressive legacy that goes beyond electoral politics, pushing
the Democratic Party to the left during a pivotal moment.

A Warren aide familiar with her higher education proposal, noting that
the vast majority of student loan debt is held by the federal
government, said the government would simply cancel the eligible debt on
its books. Doing so would affect more than 42 million Americans and
eliminate all student loan debt for more than 75 percent of borrowers.

For debt held by private companies, the government would ``work with the
borrower and the owner of the debt'' with the goal of elimination, Ms.
Warren said in a Medium post.

``The enormous student debt burden weighing down our economy isn't the
result of laziness or irresponsibility,'' Ms. Warren wrote. ``It's the
result of a government that has consistently put the interests of the
wealthy and well-connected over the interests of working families.''

As student loan debt continues to rise, the issue has forced itself into
the Democratic presidential primary, with several candidates promising
to expand access to federal grants or eliminate tuition at public or
community colleges. Senator Bernie Sanders of Vermont was credited with
\href{https://www.nytimes3xbfgragh.onion/2017/02/03/education/edlife/bernie-sanders-on-free-tuition-campaign.html}{thrusting
the issue of free college onto the national stage} during his 2016
presidential run, but Ms. Warren has pushed the idea further, calling
for the government to cancel debt in addition to expanding college
affordability.

In statements provided by the campaign Monday, scholars and education
advocates said Ms. Warren's policy would improve the financial futures
of a debt-burdened generation of young people, and help reduce the
racial wealth gap between white people and racial minorities, who have
been disproportionately burdened by student loans.

``Going to college shouldn't result in a lifetime sentence of student
debt, but that is exactly what is happening and it's only getting
worse,'' said Randi Weingarten, president of the American Federation of
Teachers, the powerful union whose endorsement is often essential for
Democrats. ``Senator Warren's plan would release Americans from their
debt sentence so they can live their lives, care for their families and
have a fair shot at the American dream.''

Ms. Warren's policy would also prohibit colleges from considering
citizenship status and criminal history in admission decisions, cut off
federal money from for-profit colleges, and require an ``annual equity
audit'' for public colleges that would identify ``shortfalls in
enrollment and completion rates for lower-income students and students
of color.''

The idea has been dismissed by some more moderate candidates seeking the
Democratic nomination for president, like
\href{https://www.nytimes3xbfgragh.onion/2019/04/22/us/politics/amy-klobuchar-2020-president.html}{Senator
Amy Klobuchar of Minnesota} and Mayor Pete Buttigieg of South Bend, Ind.

``Americans who have a college degree earn more than Americans who
don't,'' Mr. Buttigieg said while addressing
\href{https://www.politico.com/story/2019/04/03/pete-buttigieg-northeastern-university-1254963}{college
students in Boston}. ``As a progressive, I have a hard time getting my
head around the idea of a majority who earn less because they didn't go
to college subsidizing a minority who earn more because they did.''

Education experts from conservative and libertarian think tanks seemed
to echo that idea in criticizing Ms. Warren's proposal.

Beth Akers, a senior fellow at the Manhattan Institute, said that many
on the left seem ``insistent on the notion'' that people with college
degrees need ``a bailout.''

``It's hard for me to stomach the idea of billing the masses --- about
two thirds of whom don't benefit from the earnings power afforded by a
college degree --- so that college graduates can enjoy the fruits of
their education without the hindrance of having to pay for it,'' she
said Monday.

Neal P. McCluskey, the director of the Center for Educational Freedom at
the Cato Institute, agreed, saying that people who go to college often
do so at least in part because they hope to increase their earning
potential.

``It is unfair that they should not have to repay the taxpayers who had
no choice but to give them that money, on the terms the borrowers
voluntarily agreed to,'' he said.

While Ms. Warren's proposals have made her popular among left-leaning
intellectuals, it remains to be seen if that will translate into support
from the broader Democratic electorate. She
\href{https://www.nytimes3xbfgragh.onion/2019/03/31/us/politics/elizabeth-warren-fundraising.html}{languished
in early fund-raising} after pledging to sustain her presidential run
through grass-roots donations instead of the high-dollar fund-raisers
that many of her Democratic rivals have enjoyed.

She has sought to break out through a strategy of endurance, traveling
to states atypical for campaigning this early in the primary process,
including Mississippi, Tennessee, Utah and Colorado, and through her
near-constant stream of detailed proposals.

When asked recently if she could keep up the frenzied pace of policy
announcements, Ms. Warren laughed.

``I love this,'' she responded.

\hypertarget{our-2020-election-guide}{%
\section{Our 2020 Election Guide}\label{our-2020-election-guide}}

Updated ~Sept. 7, 2020

\begin{center}\rule{0.5\linewidth}{\linethickness}\end{center}

\begin{itemize}
\item ~
  \hypertarget{the-latest}{%
  \subsection{The Latest}\label{the-latest}}

  \begin{itemize}
  \item
    The unofficial Labor Day kickoff to the fall presidential campaign
    centered on Pennsylvania and Wisconsin,
    \href{https://www.nytimes3xbfgragh.onion/2020/09/07/us/politics/wisconsin-biden-harris-trump-pence.html?action=click\&pgtype=Article\&state=default\&region=BELOW_MAIN_CONTENT\&context=storylines_guide}{two
    pivotal states for both President Trump and Joseph R. Biden Jr}.
  \end{itemize}
\item ~
  \hypertarget{how-to-win-270}{%
  \subsection{How to Win 270}\label{how-to-win-270}}

  \begin{itemize}
  \item
    Joe Biden and Donald Trump need 270 electoral votes to reach the
    White House. Try building
    \href{https://www.nytimes3xbfgragh.onion/interactive/2020/us/elections/election-states-biden-trump.html?action=click\&pgtype=Article\&state=default\&region=BELOW_MAIN_CONTENT\&context=storylines_guide}{your
    own coalition of battleground states}~to see potential outcomes.
  \end{itemize}
\item ~
  \hypertarget{voting-by-mail}{%
  \subsection{Voting by Mail}\label{voting-by-mail}}

  \begin{itemize}
  \item
    Will you have enough time to vote by mail in your state? Yes, but
    it's risky to procrastinate.
    \href{https://www.nytimes3xbfgragh.onion/interactive/2020/08/31/us/politics/vote-by-mail-deadlines.html?action=click\&pgtype=Article\&state=default\&region=BELOW_MAIN_CONTENT\&context=storylines_guide}{Check
    your state's deadline.}
  \item
    \href{https://www.nytimes3xbfgragh.onion/interactive/2020/us/elections/joe-biden.html?action=click\&pgtype=Article\&state=default\&region=BELOW_MAIN_CONTENT\&context=storylines_guide}{}

    \hypertarget{joe-biden}{%
    \section{Joe Biden}\label{joe-biden}}

    \hypertarget{democrat}{%
    \subsection{Democrat}\label{democrat}}

    \href{https://www.nytimes3xbfgragh.onion/interactive/2020/us/elections/donald-trump.html?action=click\&pgtype=Article\&state=default\&region=BELOW_MAIN_CONTENT\&context=storylines_guide}{}

    \hypertarget{donald-trump}{%
    \section{Donald Trump}\label{donald-trump}}

    \hypertarget{republican}{%
    \subsection{Republican}\label{republican}}
  \end{itemize}
\item
  \hypertarget{keep-up-with-our-coverage}{%
  \subsection{Keep Up With Our
  Coverage}\label{keep-up-with-our-coverage}}

  \begin{itemize}
  \item
    Get an
    \href{https://www.nytimes3xbfgragh.onion/newsletters/politics?action=click\&pgtype=Article\&state=default\&region=BELOW_MAIN_CONTENT\&context=storylines_guide}{email}~recapping
    the day's news
  \item
    Download our mobile app on
    \href{https://apps.apple.com/us/app/nytimes/id284862083?ls=1\&mat_click_id=5c79ae7455014fd1bd66b5610c05b8f2-20191112-16948\&referrer=mat_click_id\%3D5c79ae7455014fd1bd66b5610c05b8f2-20191112-16948\%26link_click_id\%3D722930677036718082}{iOS}~and
    \href{http://a.localytics.com/android?id=com.nytimes.android\&referrer=utm_source\%3Dother_nyt_mobile_web\%26utm_medium\%3DWeb\%2520page\%26utm_term\%3DGeneral\%2520Mobile\%2520Page\%26utm_campaign\%3DNYT\%2520Mobile\%2520General\%2520Page}{Android}~and
    turn on Breaking News and Politics alerts
  \end{itemize}
\end{itemize}

Advertisement

\protect\hyperlink{after-bottom}{Continue reading the main story}

\hypertarget{site-index}{%
\subsection{Site Index}\label{site-index}}

\hypertarget{site-information-navigation}{%
\subsection{Site Information
Navigation}\label{site-information-navigation}}

\begin{itemize}
\tightlist
\item
  \href{https://help.nytimes3xbfgragh.onion/hc/en-us/articles/115014792127-Copyright-notice}{©~2020~The
  New York Times Company}
\end{itemize}

\begin{itemize}
\tightlist
\item
  \href{https://www.nytco.com/}{NYTCo}
\item
  \href{https://help.nytimes3xbfgragh.onion/hc/en-us/articles/115015385887-Contact-Us}{Contact
  Us}
\item
  \href{https://www.nytco.com/careers/}{Work with us}
\item
  \href{https://nytmediakit.com/}{Advertise}
\item
  \href{http://www.tbrandstudio.com/}{T Brand Studio}
\item
  \href{https://www.nytimes3xbfgragh.onion/privacy/cookie-policy\#how-do-i-manage-trackers}{Your
  Ad Choices}
\item
  \href{https://www.nytimes3xbfgragh.onion/privacy}{Privacy}
\item
  \href{https://help.nytimes3xbfgragh.onion/hc/en-us/articles/115014893428-Terms-of-service}{Terms
  of Service}
\item
  \href{https://help.nytimes3xbfgragh.onion/hc/en-us/articles/115014893968-Terms-of-sale}{Terms
  of Sale}
\item
  \href{https://spiderbites.nytimes3xbfgragh.onion}{Site Map}
\item
  \href{https://help.nytimes3xbfgragh.onion/hc/en-us}{Help}
\item
  \href{https://www.nytimes3xbfgragh.onion/subscription?campaignId=37WXW}{Subscriptions}
\end{itemize}
