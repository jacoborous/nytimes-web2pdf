`I Want What My Male Colleague Has, and That Will Cost a Few Million
Dollars'

\url{https://nyti.ms/2VTcmf0}

\begin{itemize}
\item
\item
\item
\item
\item
\item
\end{itemize}

\includegraphics{https://static01.graylady3jvrrxbe.onion/images/2019/04/21/magazine/21mag-Salk-image1/21mag-Salk-image1-articleLarge.jpg?quality=75\&auto=webp\&disable=upscale}

Sections

\protect\hyperlink{site-content}{Skip to
content}\protect\hyperlink{site-index}{Skip to site index}

\hypertarget{i-want-what-my-male-colleague-has-and-that-will-cost-a-few-million-dollars}{%
\section{`I Want What My Male Colleague Has, and That Will Cost a Few
Million
Dollars'}\label{i-want-what-my-male-colleague-has-and-that-will-cost-a-few-million-dollars}}

Women at the Salk Institute say they faced a culture of marginalization
and hostility. The numbers from other elite scientific institutions
suggest they're not alone.

Beverly Emerson, Ph.D., who worked at the Salk Institute from 1986 to
2017.Credit...Holly Andres for The New York Times

Supported by

\protect\hyperlink{after-sponsor}{Continue reading the main story}

By Mallory Pickett

\begin{itemize}
\item
  April 18, 2019
\item
  \begin{itemize}
  \item
  \item
  \item
  \item
  \item
  \item
  \end{itemize}
\end{itemize}

Northern San Diego County is a scientific mecca, home to some of the
world's leading biotech companies, renowned research institutions and a
world-class university. But the Salk Institute for Biological Research,
perched on a cliff above the Pacific Ocean in La Jolla, is distinguished
even among its neighbors. Jonas Salk founded the institution in 1963 as
a kind of second legacy, after the millions of lives saved with his
polio vaccine. He envisioned it as a place where scientists would work
in open, collaborative laboratories, free from university bureaucracies:
They would be professors, supervising graduate students and postdocs,
but with no teaching requirements. He recruited 10 of the top men in
biology to join him, including Francis Crick, newly famous for
discovering, with James Watson, DNA's double helix. In a 1960 letter,
Watson called the idea ``Jonas's utopia.''

By 2017, the biochemist Beverly Emerson had worked in this utopia for 31
years. She was, at the time, onto an exciting idea --- a novel approach
to understanding tumor growth --- but her 66th birthday was coming up,
and with it her contract with Salk would expire. To renew it, the
Institute required that she have enough grant money to cover half her
salary. She didn't.

Emerson had pinned her hopes on a new funding initiative she was
developing with the Salk's president, Elizabeth Blackburn. So when she
went to meet with Blackburn that fall, she thought it might be about
their progress. Instead, she found Blackburn flanked by the Salk's chief
finance and science officers. ``You and I have had long careers,''
Emerson remembers Blackburn saying. She knew it was over.

Emerson broke the news to her lab employees and turned to the work of
shutting down experiments. On her final day, she took one last look
around; she had spent 40 years going to a lab almost every day, and
couldn't imagine a life without one. The Salk made no announcement of
her departure. It was Kathy Jones, another professor, who sent around an
email letting colleagues know Emerson was leaving, and thanking her for
her years of service.

Emerson suspected funding wasn't the only reason her tenure at the Salk
ended so unceremoniously. Two months before that final meeting, she and
Jones and Vicki Lundblad --- three of the four women among the
Institute's 32 full professors --- filed state gender-discrimination
lawsuits against the Salk, claiming that an ``old boys' club'' of senior
faculty restricted their access to funds, laboratory resources and
influence. Select male faculty, they said, effectively ran the Institute
and were showered with private donations, while women were forced to
fire essential staff and were shut out from power.

At the time, women made up just 16 percent of the Salk's senior faculty
and 32 percent of assistant professors --- a striking statistic, given
that the biological sciences are one of the only scientific fields in
which women earn more than half the doctoral degrees. This state of
affairs wasn't unique to the Salk: Women make up a similar share of
senior faculty at similar research institutions, and just 28 percent of
tenured biology professors at elite public universities.

Jones started at the Institute on July 1, 1986, when she was 31. She was
one of two new hires for the Salk's regulatory biology department. The
other was Emerson, who was still en route, driving cross-country with
her mother, writing grant applications by the light of motel signs. On
her first day, as Jones made her way through the Salk's campus to meet
with the president, she was dazzled by its grandeur. Two teak-accented
buildings enclosed a marble courtyard, amid orange trees, terraced pools
and a grand ocean view. But as she waited outside the president's
office, she says she had an encounter she would later think of often. A
woman who introduced herself as Carolyn Stinson, the Institute's
director of program analysis, shook Jones's hand and then blurted out a
warning: ``You need to understand that at the Salk Institute, women do
not get the same pay as men,'' Jones remembers her saying, ``and they do
not get anywhere near the same resources.''

``I didn't know her from anybody,'' Jones recalls. ``She just walked up
and said that.'' (Stinson could not be reached for comment.) But Jones
didn't give the warning much weight; her first thought, she told me,
was, Well, it's not going to be like this in 10 years. Her hiring itself
felt like proof that the world had changed; any inequities would fade
away as out-of-touch older scientists retired or died off.

It had been two years since Suzanne Bourgeois, the wife of one of the
Institute's founders, started the regulatory biology department, which
was located in the basement of one of the two Salk buildings. The only
other professor in it was Pamela Mellon. Emerson and Jones didn't take
much notice that their all-woman department was cloistered in the
basement, or that senior female faculty had relatively small labs. But
Mellon, who was slightly older and further along in her career, noticed,
and was concerned. The senior women, she recalls, ``were secondary. They
weren't consulted. They weren't invited into joint grants. They weren't
in the leadership.'' She left the Salk for U.C. San Diego in 1991. ``I
could see even then that I was not going to be in the leadership,'' she
says. ``That was not going to happen because of my gender.''

\href{https://www.nytimes3xbfgragh.onion/2017/10/18/magazine/when-the-revolution-came-for-amy-cuddy.html}{\emph{{[}Read
about public attacks on the work of social psychologist Amy Cuddy.{]}}}

Jones and Emerson pitied the older women. But they imagined their own
exceptional scientific work, and the changing culture around them, would
ensure their bright futures.

Early on, Jones worked to decipher how H.I.V. uses a specific protein to
duplicate itself. By 1997, she felt sure she was on the verge of
figuring it out. ``I remember driving down the freeway, and thinking to
myself, Don't get in an accident, because then no one will know the
answer to this!'' she recalls. When she gave birth to her second child,
she took her work with her to the hospital. The resulting paper,
published in Cell in 1998, put her name on the map.

By then, Bourgeois had retired, leaving Emerson and Jones alone in a
department of two. They wanted to search for new hires; the
administration, they say, suggested they instead shut down their
department and join a different group. Inder Verma, a lauded cancer
biologist, told them he'd decided to allow them to be part of his group,
housed in the lab next door --- a merger that would give him authority
over their space. He'd asked if any other department would take them,
they remember him saying, but ``nobody wants you.'' (Verma declined to
be interviewed, but said he had no recollection of this conversation.)
``Not the most attractive invitation,'' Jones recalls. They said no.

After that decision, the women say, the Salk reconfigured the building.
Verma's department took over space on their floor. ``We were used to
hearing insults toward women,'' Emerson says, ``but this really showed
their power --- and that we had no power.''

\includegraphics{https://static01.graylady3jvrrxbe.onion/images/2019/04/21/magazine/21mag-Salk-image2/21mag-Salk-image2-articleLarge.jpg?quality=75\&auto=webp\&disable=upscale}

\textbf{It was a conflict} over 200 square feet of laboratory space that
once led some of the country's leading universities to focus on the
progress --- or lack thereof --- of women in science.

On paper, Nancy Hopkins had a charmed career. As an undergraduate at
Radcliffe College, she was handpicked by James Watson, of Watson and
Crick, for mentorship. In 1973, she was hired as an assistant professor
at M.I.T.'s Center for Cancer Research, and was quickly promoted to
receive tenure. But in the early '90s, when she began a new set of
genetics experiments using zebra fish and requested an extra 200 square
feet for aquariums, the faculty member in charge of facilities refused.

Hopkins, then in her early 50s, had a feeling her lab was already
smaller than those of her male peers. For a year, she fought for more
space --- a battle that culminated with her measuring the entire
department. ``I literally got a tape measure and measured every lab in
the building,'' she told me. She found that her space was 500 square
feet smaller than the average male junior professor's, and from 1,500 to
4,500 square feet smaller than her male fellow full professors'. The
person in charge refused to look at her data. ``Here I was measuring the
goddamn laboratories,'' she says, ``and I brought him the measurements,
and he wouldn't even look at them.''

She began to notice that other women seemed just as invisible as she
felt, while men whose science didn't seem any better were lauded as
geniuses. But it was years before she named this problem in her mind;
she thought of discrimination as something that ended when women were
allowed to hold jobs like hers. ``It's hard to believe anyone could be
that naïve, but I was,'' she says. At the start of her career, her
science seemed to speak for itself, and her talents were encouraged by
great men in her field. Now that she considered herself their equal, she
was confronted with all the ways they seemed to disagree.

The question of whether women could, or should, be a part of the
scientific profession was as old as the profession itself. According to
Londa Schiebinger, a historian at Stanford, it was during the
Enlightenment era in Europe, as science was transformed into a
profession and universities and academies formed, that a fork in the
road was reached, and women were formally excluded from Western science.
Earlier, they'd been among its practitioners, but by 1911, Marie Curie
was denied entry to the French national academy despite having shared a
Nobel Prize. (The same year, she became the first person ever to win a
second.) After a long debate, the academy concluded that it should
``respect the immutable tradition against the election of women,'' so as
not to ``break the unity of this elite body.''

\href{https://www.nytimes3xbfgragh.onion/2019/02/13/magazine/women-coding-computer-programming.html}{\emph{{[}Read
about the secret history of women in coding.{]}}}

Later in the 20th century, as Hopkins and other women were allowed into
the ``elite body'' of academia for the first time, they found that
cultural and structural barriers remained. Frustrated, Hopkins drafted a
letter to Charles Vest, the president of M.I.T., calling for action. It
was eventually signed, and added to, by 16 of the 17 tenured female
faculty members in the School of Science (there were 194 tenured men),
and became the basis for
\href{http://web.mit.edu/fnl/women/women.html}{a 1999 M.I.T. report},
written by Hopkins and other science faculty, documenting how women in
the sciences had access to fewer resources than their male counterparts.
Almost none of the junior female faculty, the report found, believed
``that gender bias will impede their careers.'' It was after receiving
tenure that ``many senior women faculty begin to feel marginalized,
including those who felt well supported as junior faculty. They sense
that they and their male colleagues may not be treated equally after
all.''

Hopkins's report catalyzed significant change at M.I.T. Vest convened
meetings with eight other universities, all of which committed to
support female faculty, using policies --- collecting equity and
inclusion data, providing support for faculty with children ---
recommended in the report. Hopkins submitted a two-page grant
application to support her zebra-fish research, and was offered \$10
million --- two million more than she needed. ``Once you raise money
that easily, that fast, guess what? Your research gets easier!'' she
says. ``I got elected to the National Academy. I became, like, a real
person.''

But a few men in her department, she says, never spoke to her again. In
2005, she attended an invitation-only conference in Cambridge, where
Lawrence Summers, Harvard's president at the time, wondered aloud
whether the scarcity of female scientists at elite universities might be
a function of ``intrinsic aptitude.'' Hopkins immediately closed her
computer and left the room. ``I think what's so painful about it is that
we're scientists and we're supposed to be interested in the truth,'' she
says. ``If it's true, O.K., it's true. But then show me the data. And
this is a topic on which you cannot show me the data.'' For Summers to
ask the question ``was not science,'' she says. ``It was not
interesting.''

Image

Kathy JonesCredit...Alejandro Tamayo/The San Diego Union-Tribune

Image

Vicki LundbladCredit...Alejandro Tamayo/The San Diego Union-Tribune

\textbf{The women in} the Salk's regulatory biology group had not
believed that Inder Verma's patronage was their only option. But when
they began, with the Institute's permission, recruiting for their
department --- inviting prospective hires to give talks on their
research --- Jones says most of the senior male faculty would not
attend, making it impossible to get approval for a final decision. ``Are
you tired of your seminar series yet?'' Jones says Verma asked her,
after yet another interviewee had come and gone without his input.
(Verma says he does not recall this conversation.)

If anyone typified the male ``rock star'' scientists said to have held
sway over the Salk, it was Verma. As of 2015, he was the Institute's
highest-paid scientist --- and one of its most famous, known for
developing the technique of retroviral gene therapy for cancer
treatment. His reputation was important for fund-raising. Among
biologists, the Salk was sometimes known as ``Inder's Institute.''

Some current and former Salk employees identified Wylie Vale, Ron Evans,
Stephen Heinemann and Rusty Gage as the men who, along with Verma,
seemed to enjoy extraordinary resources and status (though only Verma
was mentioned in the lawsuits). These men, titans in their fields, spoke
often at faculty retreats, and on milestone birthdays would reign over
symposia in their honor. Parties would be advertised via posters
photoshopped to depict them as larger-than-life personas: Gage's head on
a muscled body beneath the text: ``The Ultimate Gage Fighter''; Vale as
a mafia don with a cigar hanging from his lips. Each researcher at the
Institute is responsible for funding his or her own work, and grants and
donations seemed especially easy for these men to access. As Salk
presidents came and went, it became clear to Jones and Emerson where the
real power lay.

When Richard Murphy was brought on as president in 2000, he established,
partly at the urging of senior female faculty, a committee to explore
the status of women at the Salk and produce a report similar to
M.I.T.'s. It concluded that while salaries and promotion rates were
roughly equitable, the representation of women on the faculty was as at
many similar institutions well below what it should be, compared with
the applicant pool, and that women were being promoted more slowly than
their male peers. The committee recommended several policy changes,
including actively recruiting more women, and in 2004, the Salk hired
Vicki Lundblad, a well-known biologist who studied telomeres.

Murphy left the Salk in 2007 and was succeeded by William Brody, a
radiologist who had been president of Johns Hopkins. Soon after he was
hired, Emerson attended a faculty retreat where, she says, Verma asked
if the new president wanted to say anything to the assembled faculty.
Brody began by noting that many friends had asked why he took the job,
as the Salk had a reputation for short-tenured presidents. His plan, he
said, was simple: ``I'm going to let the faculty run the Institute.''

``I knew what that meant, and my hopes were sunk,'' Emerson recalls.
``The faculty does not mean the whole faculty.'' (Brody denies that this
happened.)

The Institute's new recruit, Lundblad, soon found herself hemmed in too.
Karl Willert, a U.C.S.D. biologist and old friend, remembers how excited
Lundblad had been to be in California, working at Salk. But he watched
that enthusiasm slowly erode. ``She became really upset about how things
were run over there,'' he says. ``Everyone knows that Salk could get
awarded 10- or a hundred-million-dollar donations or grants. I think she
oftentimes felt like she wasn't included in that and was sort of on the
sidelines.''

In 2014, according to Lundblad's complaint, Brody called her into his
office and told her that a group of senior faculty had decided she was
``in a downward spiral,'' and that ``the field has passed her by.'' This
felt shockingly out of step with what she was hearing from her peers.
Reviewers of an N.I.H. grant proposal a year earlier had described her
as a ``key player'' and ``established leader'' in her field. (Through a
lawyer, Brody denied this account, saying that any discussions he had
about productivity were ``only in terms of facts and metrics, not
judgment.'')

A few months later, Lundblad applied to hire a lab worker who was fully
funded by a fellowship; her request was denied. She eventually got the
decision reversed --- but later that year, it happened again. Jones and
Emerson were also having trouble staffing their labs. Emerson was told
she was underfunded and asked to fire three of her five workers. In
2015, after Jones and one of her postdocs had spent five years
identifying what seemed like a potential cancer treatment, she was
pushed to fire the postdoc.

``It was like being in `Lord of the Flies,' but with this incredible
Dilbert overlay,'' Lundblad says. In 2015, one of Jones's former
postdocs came to visit and was shocked by the state of the lab: ``It
looks like you're almost extinct,'' she remarked.

In 2014, Emerson was elected faculty chairwoman, only the second woman
ever to hold the position. From this position of ostensible power, she
asked the external relations office to diversify its matchmaking between
scientists and donors. In a meeting, when she solicited input from the
faculty on fund-raising, Brody, she says, told her to stop --- she was
just ``confusing'' the development office. According to her complaint,
he reviewed a plan she presented on including more women in leadership
and responded that ``boys run committees and boys choose boys.''
(Through his lawyer, Brody denies these comments as well.)

Emerson found herself thinking back on the older women whom she and
Jones once pitied. ``Thirty-one years later, we are the ones that we saw
through those young eyes,'' she says. ``Nothing's changed. Even after a
successful career.''

Image

Dr. Jonas Salk, developer of the polio vaccine and founder of the Salk
Institute.Credit...Bettmann/Getty Images

\textbf{Elizabeth Blackburn,} a Nobel Laureate in physiology/medicine
for her discovery of telomerase, had for years spoken out about issues
facing female scientists at all levels. So when it was announced in 2015
that Brody was retiring and Blackburn would be the next Salk president
--- Emerson had sat on the hiring committee that helped select her ---
the women were sure this promised relief. Lundblad had done her postdoc
in Blackburn's molecular biology lab at Berkeley, and Emerson and Jones
had known her for years. ``All of us thought, Liz is coming --- thank
God this is over,'' says Jones.

Blackburn, they say, said she would work with them to secure funding.
She initiated a strategic planning effort meant to lay the groundwork
for the Salk's next 50 years. The faculty was split into committees; the
mandate for Emerson and Lundblad's group included assessing the state of
faculty culture and diversity.

Working with four colleagues, they found that the Institute had hired
3.75 men for every woman since 2010, and that there was a
gender-specific skew in lab size: senior women had three of the five
smallest spaces on campus, despite raising twice as much N.I.H. funding
per employee as men. ``When I saw that report, I just felt physically
ill,'' Jones says. Lundblad says, ``It's not that they haven't hired
women, but they have failed to retain them, they've failed to hire
enough, failed to promote. Salk has not provided a work environment that
allows women faculty to flourish.''

Emerson, in her complaint, says she expected the full findings to be
presented to the board. Blackburn, in a statement, says that while she
considered presenting the reports to trustees, their volume and
early-draft nature meant they were summarized and discussed in later
meetings: ``Somewhere along the line,'' Emerson says, ``she probably
learned how the place is run and had to run it that way.''

In the spring of 2017, Jones and Lundblad both spoke at the annual
faculty retreat, a sumptuous affair in Borrego Springs where faculty
members present new research to their colleagues. It was only the second
time in 14 years that Lundblad had presented anything, and Jones's third
time in 30. Afterward, Jones says, one of the junior faculty
congratulated her, saying it was one of the best talks he'd heard at
Salk. ``It was very bittersweet,'' Jones says. ``I felt like, Oh, I
missed this. I missed this excitement, this potential for collaboration
with colleagues.'' The two women drove home to San Diego together,
wondering what to do next.

When Lundblad got home, she looked at her husband and said, ``I think
---'' He understood the look on her face and finished her sentence:
``It's time.'' That summer, Jones and Lundblad both filed
gender-discrimination lawsuits against the Salk. Emerson filed a week
later.

\textbf{Salk responded} swiftly to the first suits, with a statement in
July 2017 that said Jones and Lundblad's performances had ``long
remained within the bottom quartile of {[}their{]} peers,'' that in the
past 10 years they had ``failed to publish a single paper in any of the
most respected scientific publications (Cell, Nature and Science)'' and
that they had fallen short of the median 29 papers published per faculty
member over the past decade.

This statement prompted scientists across the country to rally to the
women's defense. To many, the use of Cell, Nature and Science as
benchmarks of excellence was particularly galling. Some well-known
scientists boycott those ``prestige'' journals to protest their
perceived clubbiness and inflated status. Carol Greider, the telomere
biologist who shared the Nobel Prize with Blackburn,
\href{https://twitter.com/CWGreider/status/886975788599320576}{defended
Lundblad on Twitter}, saying, ``Contrary to Salk press release character
smear, Vicki Lundblad is outstanding, creative LEADER in Telomere
field.'' Soon after, she drafted
\href{https://science.sciencemag.org/content/357/6356/1105}{a letter}
titled ``Not Just Salk,'' signed by 37 prominent researchers, including
two other Nobel Laureates, and published in the pages of Science. ``The
next generation of scientists is watching,'' it said, ``and many are
choosing not to pursue a career in science, where they feel they will
not have support.''

Bias and discrimination can be hard to quantify, especially with a
sample size of just three women, and the Salk mounted a case against the
women's claims. In a statement through an external communications firm,
the Institute stressed that faculty are responsible for bringing in
their own funding, and these women didn't supply enough to support
larger labs. It said that the plaintiffs' salaries were on par with
their peers, and that they were supplied with institutional funding to
bridge gaps in resources. It also disputed the lab-size analysis, which
was
\href{https://www.sciencemag.org/news/2017/08/leaked-documents-expose-long-standing-gender-tensions-salk-institute}{leaked
to Science in 2017}, describing it as an unfinalized draft ``prepared in
large part by the now-plaintiffs.''

Former Salk employees generally agree the Institute is ``clubby'' and
run by a few powerful men. But it's a question, for some of the men, at
least, whether gender is the most important criterion for entry. ``To be
perfectly honest, I didn't see it affecting women more than men,'' a
former junior faculty member --- a man who worked there for nine years
--- told me. (He requested anonymity to preserve working relationships
at the Salk.) ``It was just the way the Salk worked. There were a few
people who had very good connections with the Salk office, and then
other people were not put out in front of donors.'' He, too, was forced
to fire staff, he said, even though he knew one of his colleagues, who
had the same amount of funding and many more employees, was not.

``There was a general suspicion that a small number of people were
benefiting a lot from certain donors or funding agencies,'' another male
former faculty member said. ``I never felt discriminated against. I just
felt like I wasn't on the gravy train.''

Emerson and the other women agree that the Salk hierarchy is inherently
unfair, but say it's not an accident that senior women are consistently
found at the bottom. All the female former faculty members and research
staff interviewed were in agreement on this. ``It was especially unfair
for women,'' Mellon says. ``There's no question in my mind. We were made
to feel on shaky ground to be there as scientists in the first place.''
Holly Ingraham, now a professor of pharmacology at U.C. San Francisco,
was a postdoc at the Salk when Emerson and Jones were hired. Years
later, when she returned to visit, she was shocked to see their labs.
``They were in the worst space that Salk had to offer,'' she recalls.
``So how did that decision get made?'' When she heard news of the
lawsuit, she wrote to Emerson that while ``similar problems exist
everywhere,'' key figures at the Salk had ``instilled a culture that
consistently stacked the odds against women.''

Blackburn stepped down from the Salk's presidency in January 2018, five
months after the suits were filed, and returned to her research at
U.C.S.F. She is bound by a nondisparagement clause, but she did tell me
that ``the way we lose the talents of women because of all these things
that go on in careers, it's just extraordinary. It's a very bad, bad
thing to lose all that training and talent.'' She added: ``Often these
situations which go on in a woman's career --- workplace situations ---
they don't seem big. But I heard someone say a marvelous thing in this
context: `A ton of feathers still weighs a ton.' ''

Doctoral programs have churned out female Ph.D.'s for decades --- women
have earned the majority of biosciences Ph.D.'s since 2009 --- but they
\href{https://www.nature.com/news/inequality-quantified-mind-the-gender-gap-1.12550}{remain
underrepresented} on the faculties of top American universities.
Hopkins, who continues to spend much of her time speaking about gender
equity, says she's stunned to see how few women wield real influence in
science. ``I think it's hard for us even to recognize how little power
women have at the top, where the big money resides, where the real power
resides,'' she says.

Those who do make it into prestigious positions sometimes find that
efforts at equity don't include them. ``I know a lot of men who
sincerely promote gender-equality opportunities for women, but all their
efforts are devoted toward younger women,'' Emerson says --- because
it's less costly. ``But I want what my male colleague has, and that will
cost a few million dollars.''

Science is, in theory, a meritocracy, in which results should speak for
themselves and buoy their authors to the top. The problem is that nobody
knows the best way to measure the merits of a scientist's work.
Data-driven metrics have become increasingly popular: The ``h-index,''
for instance, was created in 2005 as a measure of an author's overall
number of publications and how often that work is cited. But citation
practices vary widely even within disciplines, and citing yourself can
artificially drive your number up. The h-index often becomes a measure
of quantity, not quality.

By some metrics, formal institutional decisions such as hiring at the
junior levels, time to promotion and salaries are at or are approaching
equity. Some reports have found grant and publication-review processes
are largely gender neutral. But something is still holding women back,
or driving them out.

The number of women drops off at every increasing level of rank in
academic science, but in biology, an especially large drop comes between
earning Ph.D.'s and applying for tenure-track positions --- a
competitive bottleneck that frequently coincides with people starting
families. Yet a growing body of research indicates that those women who
do apply face barriers to inclusion. A
\href{https://www.pnas.org/content/109/41/16474}{famous 2012
experiment}, in which equivalent résumés with male and female names were
evaluated for lab manager jobs, found that the man was rated as more
qualified and offered a higher salary. A
\href{https://hbr.org/2017/03/research-junior-female-scientists-arent-getting-the-credit-they-deserve}{2017
Harvard Business Review analysis} found that female postdocs were
systematically undervalued for their work, taking a year longer to
receive an N.I.H. grant than men with equivalent publication histories.
And
\href{https://www.nap.edu/catalog/24994/sexual-harassment-of-women-climate-culture-and-consequences-in-academic}{a
2018 report} from the National Academies of Sciences, Engineering and
Medicine found that up to 40 percent of female science students had been
sexually harassed by faculty or staff.

``We are going into a setting that we did not create, we did not
define,'' says Shirley Malcom, who directs education and human resources
programs at the American Association for the Advancement of Science.
``You can't make the assumption that the same structures are going to be
accommodating and supportive of different players.''

Image

Beverly Emerson, Ph.D., in the Knight Cancer Institute research lab,
Oregon Health \& Science University.Credit...Holly Andres for The New
York Times

\textbf{Soon after her} dismissal, in early 2018, sitting on her couch
at home with the latest issues of Science and Nature on the coffee
table, Beverly Emerson told me she was not done with science. She showed
me the pipettes, painted with her initials, that she'd brought home from
her lab. Later, a position at Oregon Health \& Science University would
let her continue to work on her projects. But she won't be getting all
the benefits of an elder stateswoman. Emeritus professors often stick
around institutions for decades; some former researchers still drop into
the Salk for lunch.

Jones and Lundblad settled their suits in August 2018, about a year
after filing. The details of the settlement are confidential, and both
women are forbidden to make any further comments to the media. (All
interviews for this story took place before the settlement.) But they
both retained their employment at the Salk, and are expected to stay
there for the foreseeable future. Emerson continued her suit alone for
several more months, but in November 2018 she also settled. The Salk
admitted no liability in connection with any of the settlements. Emerson
hoped, she'd said, that the lawsuit would lead to better outcomes for
the current generation of women. There are more now: Clodagh O'Shea was
promoted to full professor in February 2018, the first woman to achieve
that rank since Emerson in 1999. And the institute has hired others:
Susan Kaech from Yale, and Kay Tye from M.I.T.

Nearly everyone who has worked there --- including the women who sued it
--- say the Salk Institute has lived up to its founders' promise in many
ways; the lack of bureaucracy facilitates real scientific breakthroughs.
But the lawsuit's claims suggest the lack of formal governance also made
a push for equity impossible, and let a power structure formed in the
'60s survive to the present day.

On April 21, 2018, Dan Lewis, the chairman of the Salk's board, sent a
notice to Salk employees announcing that Inder Verma had been placed on
administrative leave pending an investigation of allegations of sexual
assault. ``Dr. Verma has made extraordinary contributions to scientific
research,'' he wrote, but ``Salk will not condone any findings of
inappropriate conduct in the workplace, regardless of one's stature or
influence.'' The stories published soon after suggested otherwise. As
\href{https://www.sciencemag.org/news/2018/04/famed-cancer-biologist-allegedly-sexually-harassed-women-decades}{reported
by Meredith Wadman in Science that month}, eight women, including
Emerson and Mellon, accused Verma of sexual assault in a variety of
professional situations: pinching the buttocks of a woman who had come
to interview for a faculty position; forcibly kissing a young scientist
in his lab in 1992; grabbing and kissing Emerson in the Salk library in
2001. The incidents spanned four decades, from 1976 to 2016. In the late
1980s, Mellon reported to the Salk's human resources department that
Verma had once grabbed her breasts at a faculty party, and didn't let go
until she kicked him in the shins. She was told she needed counseling,
and, she told me, was asked not to share her claim with anyone later. In
a statement, Verma wrote that he had ``never inappropriately touched,
nor have I made any sexually charged comments, to anyone affiliated with
the Salk Institute,'' and has since produced a statement of support
signed by 56 former students and postdocs.

In the same email announcing that the Salk would take these alleged
claims seriously, Lewis announced that Rusty Gage, long one of the
Institute's stars, would be the next president. Verma resigned two
months later.

Anila Madiraju came to the Salk in 2015 as a postdoc, after completing
her Ph.D. in cellular and molecular physiology at Yale. Since she was 13
--- when she would tag along with her father, a cancer biologist, to his
lab --- her dream has been a career in academic science, and the Salk
seemed like an institution that embodied scientific freedom and
creativity. She says she has been treated well there, and her current
adviser is one of many excellent male mentors she has had in her career
so far.

And yet at every stage in her career, the future has seemed less
certain. ``It's hard already,'' she says, ``even before you throw gender
into the mix.'' She's watched successful students fail to land
tenure-track positions, or be hired as assistant professors only to
struggle with funding. She'll be applying for jobs in a year or two, but
worries about a lack of opportunity, even considering alternate careers.
When Science reported on the claims against Verma, she read the story
three times.

The younger generation at Salk, she says, seems committed to a fair
playing field, and ready to work to create one. People are talking more
about issues like discrimination and harassment. The Institute has
created an Office of Equity and Inclusion, and Madiraju says she has
noticed an effort to highlight the achievements of women staff and
professors. ``It makes you feel like, I do have a place here,'' she
says, ``and that they are making it clear that the way science is being
done --- that the face of science --- is changing.''

Of course, in 1986, when Jones assured herself that things at the Salk
would change with the departure of the old guard, Inder Verma was 39,
only eight years older than she. While a 66-year-old Emerson was
dismantling her life's work, Verma gave three hourlong lectures on the
occasion of his 70th birthday.

Reading through the lawsuits and reports about the Salk was
heartbreaking, Madiraju says. What shocked her most was the names of the
women involved, names she had known for years. Her father knew Emerson
and Lundblad's work, and at his recommendation she had read their papers
during her early years as a scientist. ``You're grasping wherever you
can for role models,'' she says, and these women's work was considered
groundbreaking. ``What is the definition of a good scientist if people
like Beverly Emerson and Lundblad are not good scientists in the eyes of
Salk?''

\begin{center}\rule{0.5\linewidth}{\linethickness}\end{center}

Mallory Pickett is a journalist in Los Angeles who writes about science,
the environment and technology.
\href{https://www.nytimes3xbfgragh.onion/2017/05/04/magazine/why-hollywoods-most-thrilling-scenes-are-now-orchestrated-thousands-of-miles-away.html}{She
last wrote for the magazine about the visual-effects industry.}

Advertisement

\protect\hyperlink{after-bottom}{Continue reading the main story}

\hypertarget{site-index}{%
\subsection{Site Index}\label{site-index}}

\hypertarget{site-information-navigation}{%
\subsection{Site Information
Navigation}\label{site-information-navigation}}

\begin{itemize}
\tightlist
\item
  \href{https://help.nytimes3xbfgragh.onion/hc/en-us/articles/115014792127-Copyright-notice}{©~2020~The
  New York Times Company}
\end{itemize}

\begin{itemize}
\tightlist
\item
  \href{https://www.nytco.com/}{NYTCo}
\item
  \href{https://help.nytimes3xbfgragh.onion/hc/en-us/articles/115015385887-Contact-Us}{Contact
  Us}
\item
  \href{https://www.nytco.com/careers/}{Work with us}
\item
  \href{https://nytmediakit.com/}{Advertise}
\item
  \href{http://www.tbrandstudio.com/}{T Brand Studio}
\item
  \href{https://www.nytimes3xbfgragh.onion/privacy/cookie-policy\#how-do-i-manage-trackers}{Your
  Ad Choices}
\item
  \href{https://www.nytimes3xbfgragh.onion/privacy}{Privacy}
\item
  \href{https://help.nytimes3xbfgragh.onion/hc/en-us/articles/115014893428-Terms-of-service}{Terms
  of Service}
\item
  \href{https://help.nytimes3xbfgragh.onion/hc/en-us/articles/115014893968-Terms-of-sale}{Terms
  of Sale}
\item
  \href{https://spiderbites.nytimes3xbfgragh.onion}{Site Map}
\item
  \href{https://help.nytimes3xbfgragh.onion/hc/en-us}{Help}
\item
  \href{https://www.nytimes3xbfgragh.onion/subscription?campaignId=37WXW}{Subscriptions}
\end{itemize}
