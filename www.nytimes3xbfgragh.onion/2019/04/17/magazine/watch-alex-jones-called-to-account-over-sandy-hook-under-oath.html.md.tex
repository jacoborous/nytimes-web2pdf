Sections

SEARCH

\protect\hyperlink{site-content}{Skip to
content}\protect\hyperlink{site-index}{Skip to site index}

\href{https://myaccount.nytimes3xbfgragh.onion/auth/login?response_type=cookie\&client_id=vi}{}

\href{https://www.nytimes3xbfgragh.onion/section/todayspaper}{Today's
Paper}

Alex Jones Under Oath Is an Antidote to a `Post-Truth' Age

\url{https://nyti.ms/2V9mZNJ}

\begin{itemize}
\item
\item
\item
\item
\item
\item
\end{itemize}

Advertisement

\protect\hyperlink{after-top}{Continue reading the main story}

Supported by

\protect\hyperlink{after-sponsor}{Continue reading the main story}

\href{/column/screenland}{Screenland}

\hypertarget{alex-jones-under-oath-is-an-antidote-to-a-post-truth-age}{%
\section{Alex Jones Under Oath Is an Antidote to a `Post-Truth'
Age}\label{alex-jones-under-oath-is-an-antidote-to-a-post-truth-age}}

\includegraphics{https://static01.graylady3jvrrxbe.onion/images/2019/04/21/magazine/21mag-screenland/21mag-screenland-articleLarge-v2.gif?quality=75\&auto=webp\&disable=upscale}

By Charles Homans

\begin{itemize}
\item
  April 17, 2019
\item
  \begin{itemize}
  \item
  \item
  \item
  \item
  \item
  \item
  \end{itemize}
\end{itemize}

Five days after Attorney General William Barr released his
expectation-deflating summary of the investigation into the Trump
campaign's suspected Russia connections, a pair of videos appeared on
YouTube, labeled ``Alex Jones / Sandy Hook Video Deposition.'' The
hundreds of thousands of views those videos have accumulated attest to
their appeal as a \#Resistance consolation prize: Maybe it's not a
habitually lying president, but at least someone is getting called to
account, under oath, for his role in the post-truthification of American
public life.

Jones, the Trump-endorsed proprietor of the conspiracy-mongering
\href{https://www.nytimes3xbfgragh.onion/2019/12/09/us/politics/owen-shroyer-infowars.html}{Infowars}
media empire, is being
\href{https://www.nytimes3xbfgragh.onion/2019/02/07/us/politics/alex-jones-sandy-hook.html}{sued
for defamation by 10 families} of children who were murdered at Sandy
Hook Elementary School in 2012. That mass shooting, Jones maintained
until recently, was a hoax, perpetrated with the connivance of the
victims' parents --- many of whom have found themselves harassed,
threatened and in some cases hounded from their homes by believers in
this conspiracy theory.

Some ambivalence is probably in order about the practice of publicly
posting deposition videos. But in the particular case of Alex Jones ---
who swam happily in YouTube's abyssal depths before being mostly banned
for hate-speech-policy violations last August --- you have to at least
appreciate the karmic elegance of it. The deposition is the sort of
thing you could imagine him experiencing in a particularly unpleasant
dream. In \href{https://www.youtube.com/watch?v=I7siWJ86g40}{the video},
he sits at a table, much as he sits behind the desk on his flagship
``The Alex Jones Show,'' but he is not in charge of the production.
Instead, he is compelled to answer
\href{https://www.youtube.com/watch?v=XES-AydpIoc}{the questions} of a
young attorney named Mark Bankston (offscreen and unseen), who over the
course of more than three hours meticulously deconstructs the world that
Jones has conjured for his audience.

\includegraphics{https://static01.graylady3jvrrxbe.onion/images/2019/04/21/magazine/21mag-jones-promo/db02475e967a4be4b9c287a280069288-videoSixteenByNineJumbo1600.png}

Bankston seems less interested in the ``whys'' of Jones's universe ---
the ultimately unsolvable riddle of how fully Jones believes what he
says on the air --- than he is in the ``hows'': the way information gets
chopped and screwed on Infowars, distended and looped and played back
into the public discourse. In a way, it's an examination of the whole
unstable architecture of influence in today's politics. When a
particularly cancerous meme surfaces in Trump's Twitter feed, or when
white supremacists suddenly materialize en masse in the streets of a
college town, the operative question now always seems to be: Where the
hell did \emph{that} come from?

At one point, Bankston dials in on an April 2017 broadcast in which
Jones and his colleague Rob Dew discuss the police inspection of the
Sandy Hook school grounds after the shooting. ``They're finding people
in the back woods that are dressed up in SWAT gear,'' Dew says. Jones,
in the video, agrees.

Bankston, in the deposition, reads this aloud and asks Jones: ``That's
not true, is it?''

``I saw it on the national news,'' Jones says.

``You saw somebody in SWAT gear in the woods?''

``Black and camouflage --- the police arrested him, they said there was
a SWAT drill in the area?'' he offers hopefully.

``No, Mr. Jones, I'm asking you: Did you see a video of a man in SWAT
gear being arrested?''

Jones, like most conspiracy theorists, presents himself as a close
reader of reality, scrutinizing the gaps in the official narrative that
reveal the big lie. But when that close reading is itself subjected to a
close reading, you realize that Jones's appeal comes not from his
attention to details but from the velocity with which he blows past them
--- the way he hurtles through an asteroid belt of informational debris
on his way to explicating the galaxy-scale perfidy of his villains. The
law, and Bankston, do the opposite. They bore patiently inward, toward
particularity.

Jones pauses, stares off-camera, blinks. ``I saw the helicopter, talking
about it, they said they later arrested the man.''

``So when you told your audience he was dressed up in SWAT gear, that's
just something you made up, isn't it? There's nobody dressed up in SWAT
gear.''

``I do remember that being on the news,'' Jones says.

``What being on the news?''

``The helicopter and the man behind the school. And the report of the
guy in the SWAT gear and the police saying they arrested him, and later
they said they didn't ---''

``Yeah, it's two reporters with cameras! There's reports about it.
There's no man in SWAT gear in that video, is there? That's just
something you made up.''

``Nope, I didn't make it up,'' Jones insists, defiantly but also a
little plaintively.

\textbf{Writing in} The Daily Beast two years ago, the conservative
commentator Matt Lewis
\href{https://www.thedailybeast.com/alex-jones-youre-a-real-sicko}{placed
Jones in the lineage of right-wing radio talkers} like Rush Limbaugh,
who had effectively reverse-engineered politics from pro-wrestling-style
confrontational entertainment. The problem is that ``politics is
inherently different,'' Lewis wrote. ``The stakes are higher. And since
ideas have consequences, our words can have grave consequences.''

This was a sly turn of phrase. ``Ideas have consequences'' has been a
conservative battle cry since 1948, when the acerbic anti-modernist
Richard M. Weaver
\href{https://www.press.uchicago.edu/ucp/books/book/chicago/I/bo17116688.html}{published
a book-length polemic} by that title bemoaning the decline of universal
truth. ``On the verbal level, we see `fact' substituted for `truth,' ''
Weaver wrote. ``With what pathetic trust does he'' --- the ``average
man'' --- ``recite his facts! He has been told that knowledge is power,
and knowledge consists of a great many small things.''

As with most adages, the use of ``ideas have consequences'' has become
dumber with time. In the mouths of figures like Limbaugh and Dinesh
D'Souza, it calcified into a sort of pretentious playground taunt: You
liberals have facts, but we have \emph{ideas}! Alex Jones surely would
have appalled Weaver, but he represents the logical, self-parodying
extreme of this rhetorical pose, filling in the outlines of an
increasingly conservative-traditionalist worldview with ridiculous
particulars about demons reincarnated as Clintons. Last week, Candace
Owens, the video blogger and recent Infowars on-air presence, was called
by House Republicans to testify in a hearing on white supremacy, where
she insisted that the so-called Southern strategy --- the Nixon-era
Republican Party's wooing of white Southern conservatives --- ``never
happened.'' It is not really possible anymore to say where Jones's
universe ends and mainstream conservatism begins.

What's commonly lamented as post-truth politics is, if we're being
precise about it, really post-fact politics: not the death of a higher
truth (belief in which has proved robust enough) but of that ``great
many small things.'' Small things like whether or not there was a man in
SWAT gear in those woods. They might not matter in politics anymore, but
they do in a courtroom. ``We have a right in this country to question
things,'' Jones protests at one point in the deposition. To which
Bankston replies: ``I'm not saying what you didn't and did have a right
to do. I'm just asking you what you did.''

Advertisement

\protect\hyperlink{after-bottom}{Continue reading the main story}

\hypertarget{site-index}{%
\subsection{Site Index}\label{site-index}}

\hypertarget{site-information-navigation}{%
\subsection{Site Information
Navigation}\label{site-information-navigation}}

\begin{itemize}
\tightlist
\item
  \href{https://help.nytimes3xbfgragh.onion/hc/en-us/articles/115014792127-Copyright-notice}{©~2020~The
  New York Times Company}
\end{itemize}

\begin{itemize}
\tightlist
\item
  \href{https://www.nytco.com/}{NYTCo}
\item
  \href{https://help.nytimes3xbfgragh.onion/hc/en-us/articles/115015385887-Contact-Us}{Contact
  Us}
\item
  \href{https://www.nytco.com/careers/}{Work with us}
\item
  \href{https://nytmediakit.com/}{Advertise}
\item
  \href{http://www.tbrandstudio.com/}{T Brand Studio}
\item
  \href{https://www.nytimes3xbfgragh.onion/privacy/cookie-policy\#how-do-i-manage-trackers}{Your
  Ad Choices}
\item
  \href{https://www.nytimes3xbfgragh.onion/privacy}{Privacy}
\item
  \href{https://help.nytimes3xbfgragh.onion/hc/en-us/articles/115014893428-Terms-of-service}{Terms
  of Service}
\item
  \href{https://help.nytimes3xbfgragh.onion/hc/en-us/articles/115014893968-Terms-of-sale}{Terms
  of Sale}
\item
  \href{https://spiderbites.nytimes3xbfgragh.onion}{Site Map}
\item
  \href{https://help.nytimes3xbfgragh.onion/hc/en-us}{Help}
\item
  \href{https://www.nytimes3xbfgragh.onion/subscription?campaignId=37WXW}{Subscriptions}
\end{itemize}
