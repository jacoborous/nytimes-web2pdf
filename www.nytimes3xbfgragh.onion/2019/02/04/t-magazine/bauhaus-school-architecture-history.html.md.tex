How Bauhaus Redefined What Design Could Do for Society

\url{https://nyti.ms/2DRFqwT}

\begin{itemize}
\item
\item
\item
\item
\item
\item
\end{itemize}

\includegraphics{https://static01.graylady3jvrrxbe.onion/images/2019/02/17/t-magazine/17tmag-bauhaus-slide-L89K/17tmag-bauhaus-slide-L89K-articleLarge.jpg?quality=75\&auto=webp\&disable=upscale}

Sections

\protect\hyperlink{site-content}{Skip to
content}\protect\hyperlink{site-index}{Skip to site index}

\hypertarget{how-bauhaus-redefined-what-design-could-do-for-society}{%
\section{How Bauhaus Redefined What Design Could Do for
Society}\label{how-bauhaus-redefined-what-design-could-do-for-society}}

A century after its founding, the German school of art and architecture
remains one of the most transcendent --- and frustrating --- movements
of the Modernist age.

The Bauhaus building designed in 1925 by Walter Gropius in Dessau,
Germany, was the school's second headquarters.Credit...Photograph by
Fabrice Fouillet. Walter Gropius, ``Bauhaus School Dessau'' © 2019 ARS,
NY/VG Bild-Kunst, Bonn

Supported by

\protect\hyperlink{after-sponsor}{Continue reading the main story}

By \href{https://www.nytimes3xbfgragh.onion/by/nikil-saval}{Nikil Saval}

\begin{itemize}
\item
  Feb. 4, 2019
\item
  \begin{itemize}
  \item
  \item
  \item
  \item
  \item
  \item
  \end{itemize}
\end{itemize}

ON APRIL 11, 1933, the architect
\href{https://www.nytimes3xbfgragh.onion/topic/person/ludwig-mies-van-der-rohe}{Ludwig
Mies van der Rohe} stepped off the tram in the Steglitz neighborhood in
southwest
\href{https://www.nytimes3xbfgragh.onion/2018/08/23/t-magazine/berlin-guide.html}{Berlin},
crossed a bridge and found that his place of work had been surrounded by
the police.
\href{https://www.nytimes3xbfgragh.onion/2016/08/14/travel/bauhaus-germany-art-design.html}{The
Bauhaus}, where he taught and served as the director, had occupied an
old telephone factory building there since 1932. The school first opened
in Weimar in 1919, as a place for uniting craftsmanship with the arts in
the service of architecture; over time, it changed, becoming more about
uniting art with industrial techniques. Once Mies took over the
directorship in 1930, it became almost purely a school for architecture.

But this instability, even vagueness, of purpose helped propagate its
influence. In just over a decade, it had become a byword for modernity
in design, a symbol of a progressive age across the world, from
\href{https://www.nytimes3xbfgragh.onion/interactive/2016/11/03/travel/what-to-do-36-hours-manhattan-new-york-city.html}{New
York} to
\href{https://www.nytimes3xbfgragh.onion/2018/11/22/travel/budget-travel-kolkata-india.html?login=email\&auth=login-email}{Calcutta}.
The Nazis perceived the Bauhaus to be, along with atonal music and
Expressionist painting, yet another specimen of the globe-spanning
Jewish Bolshevik conspiracy they sought to eliminate. They weren't wrong
to intuit a basic radicalism at the heart of the Bauhaus project:
Uniting all of its multiple tendencies and impulses was an attempt to
put art and architecture to use as social regeneration for the world's
working classes. As National Socialism steadily took power across the
country, the school became itinerant, always in search of a safe home.
It traveled from Weimar, where it lay not far from where the
constitution of the first German Republic had been drawn up, to
industrial Dessau, where it left its most enduring architectural
presence, before ending up in the capital, where its time would be
fleeting, with no physical testament to its having ever been there. By
that point, the Bauhaus was on its third director, Mies; political
developments ensured that he was to be the last.

\emph{{[}}\href{https://www.nytimes3xbfgragh.onion/newsletters/t-list?module=inline}{\emph{Sign
up here}} \emph{for the T List newsletter, a weekly roundup of what T
Magazine editors are noticing and coveting now.{]}}

The local government in Dessau, among the first municipalities in
Germany to be won by the National Socialists in 1931, had voted to close
the Bauhaus, which was a state-funded school, in 1932. Mies reopened it
as a private institution in Berlin later that year, but it only lasted
one semester. The Nazis in Dessau sought, according to one fascist
editorial, nothing less than ``the disappearance from German soil of one
of the most prominent places of Jewish-Marxist `art' manifestation,''
and they were not going to relent. With Hitler now chancellor of
Germany, the Dessau public prosecutor called for a search of the
school's new Berlin headquarters. The police found materials that were
deemed to be subversive, making it subject to closure. Three months of
fruitless attempts by Mies and others to forestall this inevitable
conclusion followed; they tried various ways to accommodate themselves
to the Nazis and preserve the Bauhaus as a private art school. But in
the end, the Dessau authorities used a new Nazi law to declare that
``support for and action on behalf of the Bauhaus, which presented
itself as a Bolshevist cell,'' amounted to a political crime. In July
1933, Mies and other Bauhaus masters gathered together at the studio of
the interior designer Lilly Reich in Berlin. Mies discussed the
financial and political situation of the school and proposed that it
should be closed. The proposal was met with unanimous agreement, and the
Bauhaus was dissolved.

\includegraphics{https://static01.graylady3jvrrxbe.onion/images/2019/02/17/t-magazine/17tmag-bauhaus-slide-PBY6/17tmag-bauhaus-slide-PBY6-articleLarge.jpg?quality=75\&auto=webp\&disable=upscale}

THIS, THE FORMAL end to the Bauhaus as a school, only precipitated the
birth of the Bauhaus as an enduring myth, with its various iterations
created and carried on by its former students and teachers, who began to
flee Germany, arriving on the shores and at the borders of other nations
as refugees. What might plausibly have been only a minor episode in the
history of Modernism became a recurring one, translated into different
languages and geographies and contexts and economies: a movement whose
aesthetic was inextricable from the fact of its diaspora. In retrospect,
the Bauhaus invested a particular concept, ``design,'' with such a
quantity of meaning that it overwhelmed the word. Governments across the
globe were experimenting with forms of planning, from the city block to
the factory floor to the entire economy itself. In that context, the
Bauhaus was an idea that could accompany that process --- could give
aesthetic, architectural and spiritual weight to the revival of society
through design.

Naturally, everyone had their own version of what this looked like. Over
time, the exodus took the Bauhaus to
\href{https://www.nytimes3xbfgragh.onion/2019/01/16/travel/five-places-to-visit-in-london.html}{London},
New York,
\href{https://www.nytimes3xbfgragh.onion/2018/10/18/travel/what-to-do-in-chicago.html}{Chicago},
\href{https://www.nytimes3xbfgragh.onion/interactive/2015/12/30/travel/what-to-do-in-36-hours-in-tel-aviv.html}{Tel
Aviv}. Walter Gropius, the principal founder, made his way to
Massachusetts and became a longtime professor at
\href{https://www.nytimes3xbfgragh.onion/topic/organization/harvard-university}{Harvard};
Hannes Meyer, the second director and an avowed Marxist, followed his
political ideals to the Soviet Union. After the war, some stayed abroad
in their newly adopted homes; others returned to one or another side of
a newly divided Germany, each part of which would refashion its own
Bauhaus. The New Bauhaus was founded in Chicago in 1937 (now known as
the \href{https://id.iit.edu/the-new-bauhaus/}{Institute of Design at
the Illinois Institute of Technology}), and another ``new'' Bauhaus was
founded in the West German city of Ulm in the 1950s (the Ulm School of
Design). The politics of the Cold War constricted and hardened the
available meanings of the Bauhaus. West Germany adopted the Bauhaus as a
symbol of democracy, East Germany much later as a symbol of progress.
For left-wing members of the '68 student revolts, Bauhaus was
stultifying conformity; for the right-wing American novelist and writer
\href{https://www.nytimes3xbfgragh.onion/2018/05/15/obituaries/tom-wolfe-pyrotechnic-nonfiction-writer-and-novelist-dies-at-88.html}{Tom
Wolfe}, author of the 1981 polemic
``\href{https://us.macmillan.com/books/9780312429140}{From Bauhaus to
Our House},'' it was the same. Everyone had founded or refounded or kept
in their memory their own Bauhaus, each smaller than the original. It
was among the oldest stories of exile: Remember Aeneas, the refugee,
who, on the wayward trail to Italy, finds that Helenus, a son of Priam,
is married to the widowed wife of his brother Hector, and that they have
built for themselves a Troy in miniature.

I went to Germany in September of last year to visit the remaining sites
of the Bauhaus in advance of the 100th anniversary of its opening, but
it was impossible not to think of its closing and the trajectories of
the school's refugees. ``You've picked an interesting time to come to
Germany,'' a friend told me when I arrived in Berlin. Just days before,
thousands of neo-Nazis had
\href{https://www.nytimes3xbfgragh.onion/2018/08/30/world/europe/germany-neo-nazi-protests-chemnitz.html}{marched
in Chemnitz}, in eastern Germany, and, surrounding an enormous statue of
Karl Marx, who had once fled Germany for political reasons himself, had
declared their hostility to immigration and refugees. In Bitterfeld,
where I was transferring trains from Dessau to Weimar --- toward the end
of the German Democratic Republic, it was the world's most polluted city
--- a drunk man seated himself next to me and repeatedly asked me where
I came from and why I came, denouncing
\href{https://www.nytimes3xbfgragh.onion/2018/12/07/world/europe/angela-merkel-germany.html}{Angela
Merkel}'s refugee policy. Inspecting the collections of the
\href{https://www.uni-weimar.de/en/university/start/}{Bauhaus-Universität
Weimar}, the archivist reminded me that the Bauhaus was forced to leave
the city because of the rise of the political right --- ``just like
today!'' she cried, with gallows cheeriness. Last October, Merkel's
Christian Democratic Union party posted disastrous results in regional
elections in Bavaria and Hesse, and
\href{https://www.nytimes3xbfgragh.onion/2018/10/29/world/europe/angela-merkel-germany.html?module=inline}{Merkel
announced} that she would step down as party leader and would not seek
re-election as chancellor in 2021. Even sections of the political left
were proposing more restrictions on asylum for refugees.

Image

Inside the Bauhaus building in Dessau, where the school was housed from
1925 to 1932.Credit...Photograph by Fabrice Fouillet. Walter Gropius,
``Bauhaus School Dessau.'' © 2019 ARS, NY/VG Bild-Kunst, Bonn

Germany is beset by anniversaries, many of them celebrating unhappy or
ambiguous events, which it nonetheless feels duty bound to observe. But
the founding of the Bauhaus happens to be one of the few good ones, and
the country is marking the centenary with a flurry of building and
activity. Two new museums are set to open this year, in Dessau and
Weimar; the \href{https://www.bauhaus.de/en/}{Bauhaus-Archiv} in Berlin
is adding a new building to be completed in 2022; and all the various
sites, from small housing projects to the central, monumental building
in Dessau, are being outfitted with new displays. A full calendar of
events is taking place in each city throughout the year. Television
shows are in the works: During my visit, at the site of the school's
first headquarters in Weimar, a German production company was filming a
six-part dramatic mini-series with the working title ``Bauhaus: The New
Era.'' The floors were strewn with piles of camera equipment and period
newspapers, and I waited for a mustachioed, cream-suited, fictional
Gropius to finish conducting business in the real Gropius's original
office before I could make my visit. The subject of the show is typical
of a newer approach to the Bauhaus: The main protagonist is a historical
figure, Dörte Helm, a painter who entered the school in 1919 and found
both freedom and constriction in the heady atmosphere. In the show, Helm
has an affair with Gropius (there was a rumor that he had an affair with
a student in Weimar, but it was never proven), and she protests the
unequal treatment of women at the school. For years, the roster of
Bauhaus luminaries --- such as Gropius, Mies,
\href{https://www.nytimes3xbfgragh.onion/search?query=Kandinsky\%252C+Wassily}{Wassily
Kandinsky} and
\href{https://www.nytimes3xbfgragh.onion/search?query=Klee\%252C+Paul}{Paul
Klee} --- was seen as exclusively male; recently, the contributions (as
well as marginalization) of its brilliant women designers --- such as
Gunta Stölzl and
\href{https://www.nytimes3xbfgragh.onion/2018/10/08/arts/tate-modern-anni-albers-retrospective.html}{Anni
Albers} in textiles; Lotte Stam-Beese in architecture; and Ré Soupault
in fashion design, photography and journalism --- have been the subject
of continuing scholarship. ``Blaupause'' (``Blueprint''), a
well-received novel by Theresia Enzensberger about a female student at
the Bauhaus who wants to be an architect, is coming out in English this
year.

While ``Bauhaus'' became shorthand for functionalist architecture, an
identikit style of angular, boxy white buildings and ribbon windows,
there were, in fact, many different Bauhauses that existed during the
school's short life span, and even more so in its afterlife. What made
for its vitality was the sheer number of movements for which the Bauhaus
provided temporary shelter: Expressionism, functionalism and --- as the
Nazis correctly surmised --- Communism. Many came to the Bauhaus because
they wanted to refound the world, from the pot in which you brewed your
tea to the painting you hung on your wall to the housing complex that
you lived in and the street that you walked on. Only a few buildings
emerged from the brains on campus. More common were the designs for
typefaces, furniture, flatware: a planned revolution in the texture and
feel and look of everyday life. Gropius would speak of a ``new unity,''
first of craft and fine art, later of art and technology, the ultimate
aim being the building as a \emph{Gesamtkunstwerk}, or total work of art
(he would later call this ``total architecture''). It was a school that
was also --- unusual for Germany --- a campus: a place where students
and teachers came to live. It was meant to embody the life that its
teachers and students were also expected to make available to the world.

Image

The former Dessau Employment Office building, designed by Gropius and
completed in 1929. Renovated in 2003, it currently houses the city's
Office of Public Safety.Credit...Photograph by Fabrice Fouillet. Walter
Gropius, ``Arbeitsamt.'' © 2019 ARS, NY/VG Bild-Kunst, Bonn

The sensuous, humane, world-changing spirit of the Bauhaus was captured
by the American poet, essayist and former architecture student
\href{https://www.nytimes3xbfgragh.onion/2002/06/18/arts/june-jordan-65-poet-and-political-activist.html}{June
Jordan} in her book
``\href{https://www.nytimes3xbfgragh.onion/1981/08/09/books/opinions-and-poems.html}{Civil
Wars}'' (1981). Recalling her early interest in architecture in New York
City, she remembered thinking while paging through design books at the
public library that:

\begin{quote}
If I could make things as simple, as necessary, and as wonderful as a
spoon of Bauhaus design, then I could be sure, in a deep way, of doing
some good, of changing, for instance, the kitchen where I grew up,
baffled by the archaeological layers of aimless, wrong-year calendars,
and high-gloss, clashing wall colors, and four cans of paprika and
endlessly, dysfunctional clutter/material of no morale, of clear,
degenerating morass and mire, of slum, of resignation.
\end{quote}

One hundred years on, the Bauhaus will once again be exhumed, today amid
conditions around the world that echo those of its birth and collapse: a
decade of economic crisis; hundreds of millions poorly or not at all
housed; plutocracy unchecked; the far-right rampant; endless war, now
often conducted under the aegis of the United States; swelling attacks
on migrants and refugees; the appearance everywhere of walls, camps,
barbed wire; the sudden resurgence in the public sphere of Nazi salutes
and swastikas.

In Berlin, where the Bauhaus ended, it also re-emerges in one of the
worst housing crises since the end of World War II, where lately prices
have been rising faster than in any other city in the world. ``The
hardest hit, as everywhere, are those who have no choice,'' wrote the
philosopher Theodor Adorno, reflecting broadly on the idea of the home,
in 1944. ``They live, if not in slums, in bungalows that tomorrow may be
leaf-huts, trailers, cars, camps, or the open air.'' The Bauhaus emerged
to forestall just such a dire situation, only to be defeated by it. But
what remains of its sparse record, its ennobled settings for the
underprivileged, demonstrates the possibility for art and architecture
to not only serve as a balm for a turbulent history but also as an
alternative to it.

Image

The Gropius-designed Monument to the March Dead, constructed in Weimar
in 1922 to commemorate the workers who lost their lives resisting the
Kapp Putsch, a failed attempt to overthrow the Weimar Republic.
Destroyed by the Nazis in 1936, it was rebuilt 10 years
later.Credit...Photograph by Fabrice Fouillet. Walter Gropius,
``Monument to March Dead Weimar'' © 2019 ARS, NY/VG Bild-Kunst, Bonn

TO FOLLOW THE TRAIL of the Bauhaus is in some sense to invite
disappointment. Not much was built that the Bauhaus could call its own.
The central building, finished in sleepy Dessau in 1926, has been
lovingly restored and is its one gleaming masterpiece: an asymmetrical
complex separated into functions, only comprehensible in its use rather
than at a single glance, with brightly colored beams and accent walls.
The Masters' Houses, where Gropius and his colleagues lived, give less
to the imagination. Both Gropius's and the photographer
\href{https://www.nytimes3xbfgragh.onion/2016/05/27/arts/design/moholy-nagy-future-present-vision-and-precision-in-a-fluid-braid.html}{Laszlo
Moholy-Nagy}'s homes were bombed during the war and were reconstructed
in a more streamlined form in 2014 by the architecture firm
\href{https://bfm.berlin/}{Bruno Fioretti Marquez}; the Kandinsky/Klee
House, which survived the war, is being renovated. In fact, all of these
buildings were designed by Gropius's office, not by students or teachers
in the Bauhaus. But part of the disappointment also comes from the fact
that some of the school's best achievements are, while thoughtful, not
immediately beautiful. They bear the impress of a collective setting out
to solve fundamentally social, rather than formal, problems.

The Bauhaus was founded in 1919, when the already renowned architect
Walter Gropius took over the Grand Duccal Academy of Art and the School
of Applied Arts in Weimar, rechristened the combined institution the
Bauhaus and turned it into a force for artistic and architectural
Modernism, bringing together the visual artists Paul Klee,
\href{https://www.nytimes3xbfgragh.onion/2017/12/14/arts/design/josef-albers-mexico-guggenheim-museum-homage-to-the-square-mesoamerica.html}{Josef
Albers}, Wassily Kandinsky and
\href{https://www.nytimes3xbfgragh.onion/2011/07/22/arts/design/lyonel-feininger-show-at-the-whitney-review.html}{Lyonel
Feininger}; the textile artists Gunta Stölzl and Anni Albers; and the
painter and theater designer
\href{https://www.nytimes3xbfgragh.onion/1984/01/22/arts/dance-bauhaus-design-by-oskar-schlemmer.html}{Oskar
Schlemmer}. Hidden in the folds of this fairly straightforward history
is an enormous variety of activities that went on under the name
Bauhaus: controversies and internal dissent, party going and occult
happenings, affairs and fights and struggles simply to maintain the
school's financial existence in one of the most tumultuous periods in
the history of Germany. Gropius, a man of military build and significant
reserve --- his ex-wife,
\href{https://www.nytimes3xbfgragh.onion/1964/12/12/archives/alma-m-werfel-widow-of-writer-she-was-also-married-to-mahler-and.html}{Alma
Mahler}, wrote to him, ``Your beautiful male hardness is a wall around
you'' --- emphasized collective work but preserved medieval hierarchies:
In the manner of a guild, there were various ranks of ``masters.'' When
Gropius introduced an architecture course led by Hannes Meyer in 1927,
women were steered away from taking it, and overall, they were largely
segregated in the textile classes.

Much of the early controversy around the Bauhaus centered on Johannes
Itten, the first teacher of the school's innovative, multidisciplinary
preliminary course (Vorkurs). A follower of Mazdaznan, a religion with
roots in
\href{https://www.nytimes3xbfgragh.onion/2006/09/06/us/06faith.html}{Zoroastrianism},
he shaved his head, dressed in robes and practiced strict vegetarianism.
(Alma Mahler, a composer who, before Gropius, was married to the
composer Gustav Mahler, and after to the writer Franz Werfel, expressed
in her 1958 memoir her horror at the ``obligatory diet of uncooked mush
in garlic'' that Itten insisted be served on campus and noted that she
found ``Bauhaus disciples recognizable at a distance, by the garlic
smell.'') Itten began classes with gymnastics and breathing exercises
before moving on to elemental discussions of the nature of materials,
the contrasts between them and aspects of color theory --- all in order
to reground students in new perceptions of the basics of making art and
objects. He held classes at the Tempelherrenhaus, an 18th-century
neo-Gothic folly, where he could scandalize the bourgeoisie of Weimar en
plein air. Eventually seen as too spiritual and craft-oriented for the
early Bauhaus, Itten was essentially forced to depart by Gropius. He was
replaced in 1923 by Moholy-Nagy, who had a far more traditional
pedagogical approach --- though Itten's influence on the curriculum
persisted for several years.

Image

The interior of the 1929 Employment Office in Dessau.Credit...Photograph
by Fabrice Fouillet. Walter Gropius, ``Arbeitsamt'' Interior, © 2019
ARS, NY/VG Bild-Kunst, Bonn

On the other side of Ittenism lay the Bauhaus's communism, another
insolubility. Early histories of the Bauhaus, filtered through West
Germany, where the first Bauhaus archive was founded, and the United
States, where several of the key instructors lived after the school
closed down, minimized the influence of socialism on the school. At the
start of the Bauhaus, Gropius's own sympathies were often unstated and
unclear. In 1920, he had designed an exceptional monument to the
striking workers who had resisted a putsch attempt to end the German
republic in Weimar: a snaking concrete thunderbolt rising up in the
middle of the cemetery among the grave sites of Thuringia's most
hallowed bourgeois families. His appeals to unite craft and fine art
echoed the British socialist William Morris, who blamed capitalism for
the degradation of the decorative arts, among other societal ills. But
he was otherwise unaffiliated, and having moved to the United States in
1937, he did his best after World War II to accommodate himself to the
Cold War norm.

The mandarin Mies, who had also produced a monument to the left --- his
was to the assassinated Communists Karl Liebknecht and Rosa Luxemburg
--- made no secret of his hostility to Communists at the Bauhaus,
expelling many of them, as he programmatically tried to ensure that the
Bauhaus became strictly an architecture school, focused on producing
work of high quality for the upper classes. Both Mies's and Gropius's
leftist memorials were eventually destroyed by the Nazis, but in 1933,
out of some combination of ego, opportunism and survival instinct, the
two of them seriously competed for Hitler's first big architectural
commission, a new building for the national bank, which brings to mind
the
\href{https://www.nytimes3xbfgragh.onion/2018/12/14/books/review/bertolt-brecht-collected-poems.html}{Bertolt
Brecht} aphorism ``Robbing a bank's no crime compared to owning one.''

Historians have especially sought to separate the Bauhaus from politics
by denigrating the contributions of its least-understood director,
Hannes Meyer, a committed Communist who led the Bauhaus from 1928 to
1930. Where Gropius had attempted to move the school toward a closer
union with German industry, with the aim of making products for a
general market, as well as adopting a broadly formalist approach to
architecture, Meyer, though also working with industry, was more
explicitly political in his aims. ``\emph{Volksbedarf statt
Luxusbedarf}'' (``The needs of the people instead of the need of
luxury'') became his slogan and that of the students who followed him.
He reorganized the curriculum, emphasizing the importance of building as
a social, rather than formal, phenomenon. In April 1919, Gropius had
published the founding manifesto along with the basic program for the
school: ``The ultimate aim of all artistic activity is building!'' he
wrote. ``The ultimate, if distant, aim of the Bauhaus is the unified
work of art.'' In 1929, writing in the Bauhaus journal, Meyer
consciously revised the statement, in poetic form, no less: ``thus the
ultimate aim of all Bauhaus work / the summation of all life-forming
forces / to the harmonious arrangement of our society.'' Later, Gropius
would adopt similar language, calling for the planner and designer to
take up ``the civilized life of man in all its major aspects,'' a form
of dirigisme that could only be satisfied under comprehensive central
planning.

Image

One of three 1926 Masters' Houses in Dessau originally designed by
Gropius as residences for Bauhaus instructors. This one, for Laszlo
Moholy-Nagy, was destroyed in World War II and reinterpreted in 2014 by
the Berlin firm Bruno Fioretti Marquez.Credit...Photograph by Fabrice
Fouillet. A New Master's House in Dessau by Bruno Fioretti Marquez

Despite his short tenure, Meyer has a nearly equal claim with Gropius to
what remains of the Bauhaus's built record, including the one Bauhaus
building in the region around Berlin: a trade-union school in the De
Chirico-dreary suburb of Bernau. This building, not a standard stop on
the Bauhaus tour, is one of the school's most extraordinary achievements
and a monument to its educational mission. Except that here, the
education was designed for ordinary workers, and the spirit of the
Bauhaus was meant to infuse the everyday lives of trade unionists.

Unlike Gropius's Bauhaus building, which rises from a flattened
landscape and makes a show of a long glass curtain-walled facade, the
ADGB Trade Union School is built into a hillside, its various rooms and
functions disaggregated into a series of connected buildings of
diminishing height: more a complex than a single structure,
incomprehensible at any particular moment or angle. Its most bravura
feature is a glass-walled corridor that descends the slope on the
school's northwest side, breaking off on the right into dormitories as
you walk down. Even on a gray afternoon, floods of light pour in through
the windows, and the landscape runs right up to meet you on the other
side of the curtain wall. There is also a subtle use of materials and
color: The low, dark-wood-slatted roof is threaded with thin red beams,
and the floor is gray cement, at once brightening and subduing the
atmosphere. Like the corridor, designed to encourage fraternizing in
between courses, the other collective spaces are simultaneously
light-filled, airy and monumental-feeling. In the extraordinary dining
hall, thick exposed concrete columns and beams lift and segment the
roof, which is lined with glass brick to filter in light.

Outside the bungalows for the ADGB Trade Union schoolteachers, a small
plaque with a relief sculpture records a tribute to Hermann Duncker, one
of the founders of the German Communist Party, with the slogan
``\emph{Jeder kann alles lernen}'' (``Everyone can learn everything'').
After 1933, the Nazis turned the ADGB into a training school for the SS.

Image

A staircase in the 1911 building designed by the Belgian architect Henry
van de Velde that was the site of the original Bauhaus. Today, it is the
Bauhaus-Universität Weimar.Credit...Photograph by Fabrice Fouillet.
Henry Van de Velde, ``Ducal Art School'' © 2019 ARS, NY/SABAM, Brussels

THE HISTORY OF the Bauhaus is therefore also a history of its
controversies, false starts and failures: Directors failed to maintain
order, politics overran the school, women were consistently
subordinated. It is also a history in which design as a social concern
gave way to design as the styling of consumer goods. But it is also a
history of other schools, with which it was contemporary and to which it
gave birth. The poet Rabindranath Tagore's
\href{https://www.nytimes3xbfgragh.onion/2013/02/03/travel/where-a-poets-vision-lives-on-in-india.html}{Visva-Bharati
University}, founded in Santiniketan in rural West Bengal, India, in
1921, bears comparison with the Bauhaus. (Tagore, who visited the school
on a trip to Europe that year, also helped organize a 1922 exhibition in
Calcutta featuring artists from the Bauhaus and the Indian avant-garde.)
So, too, does
\href{https://www.nytimes3xbfgragh.onion/2015/12/18/arts/design/the-short-life-and-long-legacy-of-black-mountain-college.html}{Black
Mountain College} in Asheville, N.C., founded in 1933, the year the
Bauhaus closed, where Josef and Anni Albers taught.
\href{https://www.nytimes3xbfgragh.onion/1994/12/14/obituaries/max-bill-85-painter-sculptor-and-architect-in-austere-style.html}{Max
Bill}, a former Bauhaus student, co-founded the
\href{https://www.nytimes3xbfgragh.onion/1964/02/04/archives/ulm-school-carries-on-bauhaus-aims.html}{Ulm
School of Design} in 1953 in West Germany, which collaborated early on
with the German manufacturing company Braun, whose Dieter Rams-designed
products directly influenced
\href{https://www.nytimes3xbfgragh.onion/1998/02/05/garden/at-home-with-jonathan-ive-making-computers-cute-enough-to-wear.html}{Jony
Ive}, the chief designer of Apple. Which brings us back via a commodius
vicus to design as the styling of consumer goods.

Why did things end up there? After all, the Bauhaus began as a protest
against the thoughtless direction of industrialization, the harm it did
to mind and spirit. ``Only work which is the product of an inner
compulsion can have spiritual meaning,'' Gropius wrote in 1923.
``Mechanized work is lifeless, proper only to the lifeless machine ...
The solution depends on a change in the individual's attitude toward his
work.'' But Gropius was also intent on partnering with German industry
to market Bauhaus products; under Meyer's directorship, the Bauhaus
actually became profitable through its commercial partnerships.
Nonetheless, he encouraged the internal agitation of the increasing
number of Communist students --- even the Bauhaus journal took on a
communist bent --- and his activities (not to mention the prospect of a
financially independent Bauhaus) began to be viewed with alarm by the
Dessau authorities. In July 1930, the mayor dismissed him, and Mies was
anointed as his successor. In an alternately lugubrious, self-pitying
and sarcastic open letter, Meyer accused the city of ``attempting to rid
the Bauhaus, so heavily infected by me, of the spirit of Marxism'':

\begin{quote}
Morality, propriety, manners, and order are now to return once more hand
in hand with the Muses. As my successor you have had Mies van der Rohe
prescribed for you by Gropius and not --- according to the statutes ---
on the advice of the Masters. My colleague poor fellow, is no doubt
expected to take his pickax and demolish my work in blissful
commemoration of the Moholyan past of the Bauhaus. It looks as if this
wicked materialism is to be fought with the sharpest weapons and hence
the very life beaten out of the innocent white Bauhaus box. ... I see
through it all. I understand nothing.
\end{quote}

In 1930, Meyer arrived in Moscow along with several of his students. He
became involved in several Soviet projects, including commissions
related to the country's first five-year plan, but he found that
scarcity of materials, plus the neo-Classical taste of
\href{https://www.nytimes3xbfgragh.onion/topic/person/joseph-stalin}{Joseph
Stalin}, stymied many of his efforts. Like countless others, Meyer and
his foreign associates came under suspicion in the era of Stalin's
purges, and in 1936, he emigrated to Switzerland. Two years later, he
moved to Mexico, where he was appointed the director of a short-lived
institute for urban planning. He remained there for 10 years, working
mostly in public service, before returning to Switzerland, where he died
in 1954.

Image

A building by the architect Hans Scharoun in the 1934 Siemensstadt
Housing Estate in Berlin, which was partially designed by
Gropius.Credit...Photograph by Fabrice Fouillet. Walter Gropius,
``Siemensstadt Housing, Berlin'' © 2019 ARS, NY/VG Bild-Kunst, Bonn

The triumph, after the war, of the Bauhaus as a style and a brand were
almost inversely proportional to its failure as a social program.
Bauhaus furniture and objects became marketable, Bauhaus architecture a
cuboid product available to anyone. Bauhaus became one more form of
enabling the growth of consumer society, with its microgradations of
taste corresponding to class and status. One of the high (or low) points
was the
\href{https://www.moma.org/calendar/exhibitions/3251?}{exhibition of a
model house} by the Bauhaus alumnus
\href{https://www.nytimes3xbfgragh.onion/1981/07/02/nyregion/marcel-breuer-79-dies-architect-and-designer.html}{Marcel
Breuer} in the sculpture garden of the Museum of Modern Art in 1949: An
early blockbuster exhibition in MoMA's history, it also betrayed the
spirit of the school by showing a house that was far too expensive for
most working-class Americans. (John D. Rockefeller Jr. bought the actual
house that was exhibited and used it as a guesthouse on his estate in
Pocantico Hills, N.Y.) This was a trend observable even to Bauhaus
contemporaries. In a 1930 essay titled ``Ten Years of the Bauhaus,'' the
Hungarian art theorist Erno Kallai, who edited the Bauhaus journal under
Meyer, laconically telegraphed the standardization of form at the
expense of content: ``Tubular steel armchair frames: Bauhaus style. Lamp
with nickel-coated body and a disk of opaque glass as lampshade: Bauhaus
style. Wallpaper patterned in cubes: Bauhaus style. No painting on the
wall: Bauhaus style.''

One of the fears that attends the centenary is that this Bauhaus brand
will overwhelm any similar attempt to revive the radical spirit of the
school at its founding. As the Bauhaus's influence spread across the
world as the pre-eminent global design concept of the postwar era, its
history and goals became increasingly watered down, a way to sell
gift-shop-ready objects and promote cultural tourism instead of using
design to improve the lives of working-class people. This contrast has
created a complicated legacy. A loose grouping of intellectuals and
architectural theorists, which goes under the name Projekt Bauhaus,
intends to take the problem head-on. Last September, I met with Anh-Linh
Ngo, one of its members and the editor in chief of
\href{https://www.archplus.net/home/}{Arch+ magazine}, in eastern Mitte,
a now tony section of former East Berlin with many examples of spruced
up Plattenbau, prefab concrete housing construction. Ngo told me the
group wants to ``look at the legacy of the Bauhaus from an outside,
critical perspective.'' In honor of the original Bauhaus's critical
spirit, they intend --- while everyone in Berlin, Dessau and Weimar is
celebrating the centenary --- to conduct a ceremonial ``burial'' of the
Bauhaus, culminating in a musical requiem directed by Schorsch Kamerun
at the Volksbühne theater in Berlin in June.

``We need to bury this kind of undead figure --- this kind of zombie ---
to put certain aspects of the Bauhaus to rest in order to deal with our
own problems,'' Ngo told me. Those problems are much in evidence in
Germany, he said, with the rise once more of the far right. He pointed
to their attempts to gain hegemony in urban spaces using techniques
pioneered by the student movement. There are reconstruction projects
taking place all over Germany that focus on aspects of pre-20th century
German heritage (especially churches), which are primarily initiatives
of the far right. Perhaps thinking of the attacks on refugees that had
taken place in Chemnitz just a week before we spoke, he suggested
quietly that, rather than resurrect the Bauhaus once again, it was
``more important to think about the mutual obligations we have toward
each other.''

TO SEE THE SITES of the Bauhaus firsthand today is in many ways to
glimpse the failure of its collective wisdom: Buildings that weren't
destroyed or commandeered by the Nazis were left to slowly decay after
the war, and the ones that still function mostly do so as tourist
landmarks. And yet, especially with the works that still operate as they
were originally intended, it is possible to glimpse the image of the
future that the Bauhaus evoked for its students, teachers and
contemporaries.

Unlike Weimar, with its overwhelming German classical and Romantic
heritage --- the erstwhile home of Goethe and Liszt --- the eerie,
moribund town of Dessau is overwhelmed by the legacy of the Bauhaus. The
capital of one of the country's numerous provincial princely states from
the 16th to the early 19th century, it was almost completely destroyed
by Allied bombing in World War II, which targeted the city because of
its role in airplane manufacturing. During the onslaught, the main
Bauhaus building was somehow spared, if damaged. During the German
Democratic Republic, it again became an industrial center and played
host to an impressive variety (if that is the term) of Plattenbau, still
visible across the skyline. The average age of the residents is 50.

In Dessau, the complex known as the Laubenganghäuser, and usually
translated with dogged literalness as the ``Houses With Balcony
Access,'' reflects the humane principles of reproducible workers'
housing. Designed under Hannes Meyer, these, like the ADGB Trade Union
School, were true Bauhaus buildings, conceived and executed under the
collective imprimatur of the Bauhaus. Three-story brick buildings, with
apartments linked by long-running balconies --- a precursor to the
``streets in the sky'' of later British social housing complexes --- the
Laubenganghäuser cram a number of amenities into small spaces. Kitchen
cabinets are hidden behind sliding doors; bright shades of maroon and
mauve enliven the otherwise incredibly tight quarters, which give off an
impression of openness and space. A current resident recently told
Berlin's \href{https://www.monopol-magazin.de/}{Monopol magazine} that
because there aren't ``sterile corridors in the building,'' residents
``spend a lot of time outdoors, so neighbors often come into contact
with each other. It feels more like a community than an apartment
building, and many people have become friends.'' Here was a vision of
the Bauhaus's potential beyond consumer society, beyond the rule of
markets and private property --- one in which collective provision
defeats private greed, and in which strangers are made to feel welcome
as members of a group. Had history not intervened, there might have been
more of them in Dessau: nearly anonymous testaments to the ideals of the
old, fractious, continuously fascinating school, present only by
implication, as its students and teachers wanted, in everyday life.

Production: Monika Bergmann at Picture Worx

Advertisement

\protect\hyperlink{after-bottom}{Continue reading the main story}

\hypertarget{site-index}{%
\subsection{Site Index}\label{site-index}}

\hypertarget{site-information-navigation}{%
\subsection{Site Information
Navigation}\label{site-information-navigation}}

\begin{itemize}
\tightlist
\item
  \href{https://help.nytimes3xbfgragh.onion/hc/en-us/articles/115014792127-Copyright-notice}{©~2020~The
  New York Times Company}
\end{itemize}

\begin{itemize}
\tightlist
\item
  \href{https://www.nytco.com/}{NYTCo}
\item
  \href{https://help.nytimes3xbfgragh.onion/hc/en-us/articles/115015385887-Contact-Us}{Contact
  Us}
\item
  \href{https://www.nytco.com/careers/}{Work with us}
\item
  \href{https://nytmediakit.com/}{Advertise}
\item
  \href{http://www.tbrandstudio.com/}{T Brand Studio}
\item
  \href{https://www.nytimes3xbfgragh.onion/privacy/cookie-policy\#how-do-i-manage-trackers}{Your
  Ad Choices}
\item
  \href{https://www.nytimes3xbfgragh.onion/privacy}{Privacy}
\item
  \href{https://help.nytimes3xbfgragh.onion/hc/en-us/articles/115014893428-Terms-of-service}{Terms
  of Service}
\item
  \href{https://help.nytimes3xbfgragh.onion/hc/en-us/articles/115014893968-Terms-of-sale}{Terms
  of Sale}
\item
  \href{https://spiderbites.nytimes3xbfgragh.onion}{Site Map}
\item
  \href{https://help.nytimes3xbfgragh.onion/hc/en-us}{Help}
\item
  \href{https://www.nytimes3xbfgragh.onion/subscription?campaignId=37WXW}{Subscriptions}
\end{itemize}
