Sections

SEARCH

\protect\hyperlink{site-content}{Skip to
content}\protect\hyperlink{site-index}{Skip to site index}

\href{https://myaccount.nytimes3xbfgragh.onion/auth/login?response_type=cookie\&client_id=vi}{}

\href{https://www.nytimes3xbfgragh.onion/section/todayspaper}{Today's
Paper}

The Historical Portraits One Artist Looks at Every Day

\url{https://nyti.ms/2Ed8Tl9}

\begin{itemize}
\item
\item
\item
\item
\item
\end{itemize}

Advertisement

\protect\hyperlink{after-top}{Continue reading the main story}

Supported by

\protect\hyperlink{after-sponsor}{Continue reading the main story}

The Story of a Thing

\hypertarget{the-historical-portraits-one-artist-looks-at-every-day}{%
\section{The Historical Portraits One Artist Looks at Every
Day}\label{the-historical-portraits-one-artist-looks-at-every-day}}

Martha Rosler says pictures of Harriet Tubman and Frederick Douglass act
as a daily reminder of history --- and the power of positive imagery.

\includegraphics{https://static01.graylady3jvrrxbe.onion/images/2019/02/12/t-magazine/oakImage-1550002303205/oakImage-1550002303205-articleLarge.jpg?quality=75\&auto=webp\&disable=upscale}

As told to
\href{https://www.nytimes3xbfgragh.onion/by/emily-spivack}{Emily
Spivack}

\begin{itemize}
\item
  Feb. 14, 2019
\item
  \begin{itemize}
  \item
  \item
  \item
  \item
  \item
  \end{itemize}
\end{itemize}

\emph{In}
\href{https://www.nytimes3xbfgragh.onion/column/story-of-a-thing}{\emph{this
series}} \emph{for T, Emily Spivack, the author of
``}\href{http://wornstories.com/}{\emph{Worn Stories}}\emph{,''
interviews creative types about their most prized possessions. Martha
Rosler, an artist whose work advocates for social justice through
everyday objects and imagery, explains the significance of two portraits
that hang in her home studio.
``}\href{https://thejewishmuseum.org/index.php/exhibitions/martha-rosler-irrespective}{\emph{Irrespective}}\emph{,''
an exhibition spanning five decades of Rosler's career, is on view at
the Jewish Museum now through March 3, 2019.}

These portraits of Harriet Tubman and Frederick Douglass were in my
son's room when he was a small child. When he moved away, they became my
wall companions. They have been in my studio in Greenpoint since the
'80s. I've never discussed it with my son. He recognizes that they
represent a kind of transmission between me and him, but also him back
to me, that these are part of our family.

I would never \emph{not} have them in my studio. They bear a certain
sentimental value, but they're also a constant reminder of the depth of
history and the power of affirming images.

Image

The artist Martha Rosler in her Brooklyn home studio.Credit...Matthew
Novak

What I admire about Tubman and Douglass is their commitment to action,
to participation and to inclusiveness in the broadest way. My son was
born in 1967 and my generation took very seriously educating our
children in what an inclusive society would mean without preaching. I
wanted my son to learn about the narratives of struggle, mythology and
history through the ethos of a generation.

We have a primitive response to a face looking at us. And these are two
people's faces looking out at me with serious, composed expressions.
These portraits look at me and I at them every day. I never don't see
them.

As you probably remember, Harriet Tubman was going to be on the
20-dollar bill. And it looks like that's been squashed by our president.
So here she is offering her rebuke and her steadfastness, looking out
from history saying, I'm still here, you'll be gone. Of course, I'll be
gone, too.

\emph{This interview has been edited and condensed.}

Advertisement

\protect\hyperlink{after-bottom}{Continue reading the main story}

\hypertarget{site-index}{%
\subsection{Site Index}\label{site-index}}

\hypertarget{site-information-navigation}{%
\subsection{Site Information
Navigation}\label{site-information-navigation}}

\begin{itemize}
\tightlist
\item
  \href{https://help.nytimes3xbfgragh.onion/hc/en-us/articles/115014792127-Copyright-notice}{©~2020~The
  New York Times Company}
\end{itemize}

\begin{itemize}
\tightlist
\item
  \href{https://www.nytco.com/}{NYTCo}
\item
  \href{https://help.nytimes3xbfgragh.onion/hc/en-us/articles/115015385887-Contact-Us}{Contact
  Us}
\item
  \href{https://www.nytco.com/careers/}{Work with us}
\item
  \href{https://nytmediakit.com/}{Advertise}
\item
  \href{http://www.tbrandstudio.com/}{T Brand Studio}
\item
  \href{https://www.nytimes3xbfgragh.onion/privacy/cookie-policy\#how-do-i-manage-trackers}{Your
  Ad Choices}
\item
  \href{https://www.nytimes3xbfgragh.onion/privacy}{Privacy}
\item
  \href{https://help.nytimes3xbfgragh.onion/hc/en-us/articles/115014893428-Terms-of-service}{Terms
  of Service}
\item
  \href{https://help.nytimes3xbfgragh.onion/hc/en-us/articles/115014893968-Terms-of-sale}{Terms
  of Sale}
\item
  \href{https://spiderbites.nytimes3xbfgragh.onion}{Site Map}
\item
  \href{https://help.nytimes3xbfgragh.onion/hc/en-us}{Help}
\item
  \href{https://www.nytimes3xbfgragh.onion/subscription?campaignId=37WXW}{Subscriptions}
\end{itemize}
