Sections

SEARCH

\protect\hyperlink{site-content}{Skip to
content}\protect\hyperlink{site-index}{Skip to site index}

\href{https://www.nytimes3xbfgragh.onion/section/movies}{Movies}

\href{https://myaccount.nytimes3xbfgragh.onion/auth/login?response_type=cookie\&client_id=vi}{}

\href{https://www.nytimes3xbfgragh.onion/section/todayspaper}{Today's
Paper}

\href{/section/movies}{Movies}\textbar{}Madea Honored the Strong Black
Women I Grew Up With, but It's Time to Move On

\url{https://nyti.ms/2EDYkI7}

\begin{itemize}
\item
\item
\item
\item
\item
\end{itemize}

Advertisement

\protect\hyperlink{after-top}{Continue reading the main story}

Supported by

\protect\hyperlink{after-sponsor}{Continue reading the main story}

\hypertarget{madea-honored-the-strong-black-women-i-grew-up-with-but-its-time-to-move-on}{%
\section{Madea Honored the Strong Black Women I Grew Up With, but It's
Time to Move
On}\label{madea-honored-the-strong-black-women-i-grew-up-with-but-its-time-to-move-on}}

\includegraphics{https://static01.graylady3jvrrxbe.onion/images/2019/03/03/arts/03tylerperry-essay1/03tylerperry-essay1-articleLarge.jpg?quality=75\&auto=webp\&disable=upscale}

By Tyler Perry

\begin{itemize}
\item
  Feb. 28, 2019
\item
  \begin{itemize}
  \item
  \item
  \item
  \item
  \item
  \end{itemize}
\end{itemize}

\emph{After nearly 20 years,}
\href{https://www.nytimes3xbfgragh.onion/topic/person/tyler-perry}{\emph{Tyler
Perry}} \emph{is making his final appearances as Madea onscreen, in ``A
Madea Family Funeral,'' due March 1, and onstage in ``Madea's Farewell
Tour'' running through May.}

When I was a kid, my mother, Maxine, would take me into the projects in
New Orleans for her regular Friday night card game. This was no bridge
game with a table of proper ladies --- this was an apartment full of
sharp-witted, drinking, cussing, smoking, strong, irreverent but caring
women. I can clearly remember sitting on the concrete floor, choking
from the smoke and covering my ears from the loud laughter and blasting
blues music. Every now and again I would listen to these women talk
about the woes in their lives and relationships, but no sooner had
sadness entered the room than one of them would make a joke and the
laughter would start all over again.

Looking back on this now, I realize that I was in a master class and my
6-year-old brain was soaking it all in. It was in those moments that I
learned that laughter can stand arm in arm with agony.

\includegraphics{https://static01.graylady3jvrrxbe.onion/images/2019/03/03/arts/03tylerperry-essay2/03tylerperry-essay2-articleLarge.jpg?quality=75\&auto=webp\&disable=upscale}

Later when my mother and I were home, and my father would fly off into
his usual fits of abusive rage, I quickly used my newfound tactics on my
mother. After he left the room, I would walk in and imitate her and
those card-playing women, and eventually she'd laugh. That laugh was
medicine for my young soul.

Now I look back on a lot of my early writing and I feel that I was
subconsciously talking to my mother. I was carrying her and those
beautiful, powerful, strong black women in my spirit. And much to the
dismay of my critics, I'm pretty sure that this is how and why I started
putting very serious subject matter right alongside humor in my work. I
knew that my audience was full of women like the ones that I loved so
much growing up.

To that end, when I first did the tough-talking, truth-telling Madea at
the Regal Theater in Chicago almost 20 years ago, I had no idea that
imitating my mother would end up bringing joy to millions of people
around the world. I thought I was just an actor donning a costume to
entertain in a live comedic play. Most of the show was full of
over-the-top jokes that brought lots of laughter, but around the last 30
minutes something happened. Madea got the opportunity to riff about pain
and heartache, forgiveness, and God and faith. When I got to those life
lessons, something happened. There was utter silence in the theaters,
and it became so clear to me that the audience was hanging on her every
word.

I understood very early on that this mostly blue-collar African-American
audience was feeling inspired. They were getting answers to a lot of
what was going on in our community that no one was talking about. I was
blown away that somehow this ridiculous-looking 6-foot-6 guy in a dress
had found a way to do for this audience the same thing that I had done
for my mother. I could lift them with humor and use that laughter as an
anesthetic and talk about really deep, sensitive issues that were
destroying so many of us --- things like rape and molestation and the
inability to forgive.

This was further confirmed when I started receiving messages like
``Madea did in two hours what my family hasn't been able to do in 12
years --- convince my sister to finally leave an abusive relationship.''
I understood then how important it was to continue in the rawness of
what I had created. To continue speaking a language that was a shorthand
my audience and I understood. To continue to break all the traditional
rules of storytelling.

Image

Perry in his first big-screen turn as Madea, in the 2005 ``Diary of a
Mad Black Woman.''Credit...Alfeo Dixon/Lionsgate

If these strong, nurturing black women from my childhood knew that their
stories and strengths had inspired me to pay homage to them and that
this loving tribute had moved, helped and lifted people all around the
world, they would be so proud.

For these reasons, I wrestled for awhile with the question of the right
time to end the character. But then I thought --- this is the year I'm
turning 50. There's much more I want to do, so many more stories I want
to tell and more roles I want to play. I was thrilled to be Colin Powell
in ``Vice'' and Tanner Bolt in ``Gone Girl*,''* and I'm looking forward
to more opportunities like those. But that old broad has been good to
me, so who knows --- maybe one day I'll tell the story of Madea in the
'70s and hire a real actress to play the role. But the time of playing
her has come to an end.

I have to say it's very evident that the audience is not ready to say
goodbye to her either: 25,000 to 30,000 people still show up every week
to the Madea Farewell Play tour. And as always, the last half-hour of
the show is the only time I don't feel ridiculous in that dress, because
I'm doing what the character was created to do.

I recently got a call from Oprah telling me that the girls at her school
in South Africa were talking about the lessons Madea taught them in my
movies and plays, lessons like self-esteem, respecting yourself, and
staying strong. I'll admit that my resolve to let Madea go wavered
again, but even with those wonderful words of inspiration, I feel that
it's time.

It has always and will always be my hope that something this character
has done or said has made someone's life better. And that some little
child who is sitting on the floor with their mother in pain can play one
of these movies or shows and they both can smile together. She's been
that for me and I hope she continues to be that for the world.

Advertisement

\protect\hyperlink{after-bottom}{Continue reading the main story}

\hypertarget{site-index}{%
\subsection{Site Index}\label{site-index}}

\hypertarget{site-information-navigation}{%
\subsection{Site Information
Navigation}\label{site-information-navigation}}

\begin{itemize}
\tightlist
\item
  \href{https://help.nytimes3xbfgragh.onion/hc/en-us/articles/115014792127-Copyright-notice}{©~2020~The
  New York Times Company}
\end{itemize}

\begin{itemize}
\tightlist
\item
  \href{https://www.nytco.com/}{NYTCo}
\item
  \href{https://help.nytimes3xbfgragh.onion/hc/en-us/articles/115015385887-Contact-Us}{Contact
  Us}
\item
  \href{https://www.nytco.com/careers/}{Work with us}
\item
  \href{https://nytmediakit.com/}{Advertise}
\item
  \href{http://www.tbrandstudio.com/}{T Brand Studio}
\item
  \href{https://www.nytimes3xbfgragh.onion/privacy/cookie-policy\#how-do-i-manage-trackers}{Your
  Ad Choices}
\item
  \href{https://www.nytimes3xbfgragh.onion/privacy}{Privacy}
\item
  \href{https://help.nytimes3xbfgragh.onion/hc/en-us/articles/115014893428-Terms-of-service}{Terms
  of Service}
\item
  \href{https://help.nytimes3xbfgragh.onion/hc/en-us/articles/115014893968-Terms-of-sale}{Terms
  of Sale}
\item
  \href{https://spiderbites.nytimes3xbfgragh.onion}{Site Map}
\item
  \href{https://help.nytimes3xbfgragh.onion/hc/en-us}{Help}
\item
  \href{https://www.nytimes3xbfgragh.onion/subscription?campaignId=37WXW}{Subscriptions}
\end{itemize}
