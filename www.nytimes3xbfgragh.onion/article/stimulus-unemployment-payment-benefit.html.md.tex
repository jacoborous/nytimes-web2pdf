Sections

SEARCH

\protect\hyperlink{site-content}{Skip to
content}\protect\hyperlink{site-index}{Skip to site index}

\href{https://www.nytimes3xbfgragh.onion/section/business/economy}{Economy}

\href{https://myaccount.nytimes3xbfgragh.onion/auth/login?response_type=cookie\&client_id=vi}{}

\href{https://www.nytimes3xbfgragh.onion/section/todayspaper}{Today's
Paper}

\href{/section/business/economy}{Economy}\textbar{}\$300 Unemployment
Benefit: Who Will Get It and When?

\url{https://nyti.ms/2ExgQ7m}

\begin{itemize}
\item
\item
\item
\item
\item
\end{itemize}

\hypertarget{the-coronavirus-outbreak}{%
\subsubsection{\texorpdfstring{\href{https://www.nytimes3xbfgragh.onion/news-event/coronavirus?name=styln-coronavirus-markets\&region=TOP_BANNER\&block=storyline_menu_recirc\&action=click\&pgtype=Article\&impression_id=411548b0-f4c5-11ea-b924-e5b48aa72b34\&variant=undefined}{The
Coronavirus
Outbreak}}{The Coronavirus Outbreak}}\label{the-coronavirus-outbreak}}

\begin{itemize}
\tightlist
\item
  live\href{https://www.nytimes3xbfgragh.onion/2020/09/11/world/covid-19-coronavirus.html?name=styln-coronavirus-markets\&region=TOP_BANNER\&block=storyline_menu_recirc\&action=click\&pgtype=Article\&impression_id=411548b1-f4c5-11ea-b924-e5b48aa72b34\&variant=undefined}{Latest
  Updates}
\item
  \href{https://www.nytimes3xbfgragh.onion/interactive/2020/us/coronavirus-us-cases.html?name=styln-coronavirus-markets\&region=TOP_BANNER\&block=storyline_menu_recirc\&action=click\&pgtype=Article\&impression_id=411548b2-f4c5-11ea-b924-e5b48aa72b34\&variant=undefined}{Maps
  and Cases}
\item
  \href{https://www.nytimes3xbfgragh.onion/interactive/2020/science/coronavirus-vaccine-tracker.html?name=styln-coronavirus-markets\&region=TOP_BANNER\&block=storyline_menu_recirc\&action=click\&pgtype=Article\&impression_id=411548b3-f4c5-11ea-b924-e5b48aa72b34\&variant=undefined}{Vaccine
  Tracker}
\item
  \href{https://www.nytimes3xbfgragh.onion/2020/09/10/us/politics/fda-coronavirus-vaccine.html?name=styln-coronavirus-markets\&region=TOP_BANNER\&block=storyline_menu_recirc\&action=click\&pgtype=Article\&impression_id=411548b4-f4c5-11ea-b924-e5b48aa72b34\&variant=undefined}{F.D.A.
  Regulators' Self-Defense}
\item
  \href{https://www.nytimes3xbfgragh.onion/2020/09/09/upshot/coronavirus-surprise-test-fees.html?name=styln-coronavirus-markets\&region=TOP_BANNER\&block=storyline_menu_recirc\&action=click\&pgtype=Article\&impression_id=411548b5-f4c5-11ea-b924-e5b48aa72b34\&variant=undefined}{Surprise
  Test Fees}
\end{itemize}

Advertisement

\protect\hyperlink{after-top}{Continue reading the main story}

Supported by

\protect\hyperlink{after-sponsor}{Continue reading the main story}

\hypertarget{300-unemployment-benefit-who-will-get-it-and-when}{%
\section{\$300 Unemployment Benefit: Who Will Get It and
When?}\label{300-unemployment-benefit-who-will-get-it-and-when}}

Forty-seven states have so far signed on to President Trump's stopgap
program to get more money to the jobless. Here's how it works.

\includegraphics{https://static01.graylady3jvrrxbe.onion/images/2020/08/21/business/00virus-supplement-1/00virus-supplement-1-articleLarge.jpg?quality=75\&auto=webp\&disable=upscale}

\href{https://www.nytimes3xbfgragh.onion/by/patricia-cohen}{\includegraphics{https://static01.graylady3jvrrxbe.onion/images/2018/02/16/multimedia/author-patricia-cohen/author-patricia-cohen-thumbLarge.jpg}}\href{https://www.nytimes3xbfgragh.onion/by/tiffany-hsu}{\includegraphics{https://static01.graylady3jvrrxbe.onion/images/2018/12/06/multimedia/author-tiffany-hsu/author-tiffany-hsu-thumbLarge.png}}

By \href{https://www.nytimes3xbfgragh.onion/by/patricia-cohen}{Patricia
Cohen} and
\href{https://www.nytimes3xbfgragh.onion/by/tiffany-hsu}{Tiffany Hsu}

\begin{itemize}
\item
  Sept. 10, 2020
\item
  \begin{itemize}
  \item
  \item
  \item
  \item
  \item
  \end{itemize}
\end{itemize}

In early August, President Trump declared a plan to deliver
\href{https://www.nytimes3xbfgragh.onion/2020/08/08/us/politics/trump-stimulus-bill-coronavirus.html}{\$400
in extra weekly benefits} to tens of millions of
\href{https://www.nytimes3xbfgragh.onion/2020/08/27/business/economy/unemployment-claims.html}{unemployed}
Americans --- a short-term fix meant to replace the
\href{https://www.nytimes3xbfgragh.onion/2020/07/29/business/economy/unemployment-benefits-coronavirus.html}{\$600-a-week}emergency
federal supplement that expired in July.

What is now clear is that the federal supplement is \$300 a week, not
\$400. And few states have started paying out.

Here is what we know.

\hypertarget{most-unemployed-workers-will-get-an-extra-300-a-week}{%
\subsection{Most unemployed workers will get an extra \$300 a
week.}\label{most-unemployed-workers-will-get-an-extra-300-a-week}}

The Federal Emergency Management Agency, which normally provides
disaster relief, will provide \$300 per recipient. An additional \$100
was supposed to be supplied by states, but most are struggling to meet
other expenses. Tax
\href{https://www.nytimes3xbfgragh.onion/2020/08/14/business/economy/state-local-budget-pain.html}{revenues
have been sinking} at the same time that costs --- like precautions to
curb the spread of the coronavirus --- have soared. Ultimately the
administration said the states' basic benefit payments could be counted
toward their \$100 share.

So far, only three states,
\href{https://www.courier-journal.com/story/news/local/2020/08/21/kentucky-unemployment-benefits-feds-approve-400-weekly-boost/3407444001/}{Kentucky},
\href{https://apnews.com/c74b1d3f46341434e61f19b4c824aaf2}{Montana} and
\href{https://wvmetronews.com/2020/08/28/w-va-approved-for-federal-enhanced-unemployment-benefit-but-questions-arise-over-how-far-that-goes/}{West
Virginia}, have decided to supply the extra \$100. Vermont's plan to
bring the total payment to \$400
\href{https://labor.vermont.gov/press-release/press-release-vermont-secures-federal-funding-increased-unemployment-benefits-through}{is
awaiting approval} from the state's legislature. Kansas also has said it
plans to supply the extra \$100.

\hypertarget{jobless-workers-with-small-unemployment-benefits-will-not-get-the-supplement}{%
\subsection{Jobless workers with small unemployment benefits will not
get the
supplement.}\label{jobless-workers-with-small-unemployment-benefits-will-not-get-the-supplement}}

Only people who qualify for at least \$100 per week in unemployment
benefits --- either through the regular state program or a federal
pandemic assistance program --- are eligible for the extra federal
funds.

In Colorado, for example, roughly 28,000 people, or about 6 percent
currently receiving unemployment pay, will not receive the new benefit,
said Cher Haavind, deputy executive director of the state Department of
Labor.

\includegraphics{https://static01.graylady3jvrrxbe.onion/images/2020/08/21/business/00virus-supplement-2/merlin_175122579_060172a3-fd68-45ad-9e38-77ef08bb9433-articleLarge.jpg?quality=75\&auto=webp\&disable=upscale}

\hypertarget{47-states-have-signed-on-so-far}{%
\subsection{47 states have signed on so
far.}\label{47-states-have-signed-on-so-far}}

As of Wednesday,
\href{https://www.fema.gov/fact-sheet/lost-wages-assistance-approved-states}{funds
had been approved} for 47 states:

\begin{quote}
Alabama\\
Alaska\\
Arizona\\
Arkansas\\
California\\
Colorado\\
Connecticut\\
Delaware\\
Florida\\
Georgia\\
Hawaii\\
Idaho\\
Illinois\\
Indiana\\
Iowa\\
Kansas\\
Kentucky\\
Louisiana\\
Maine\\
Maryland\\
Massachusetts\\
Michigan\\
Minnesota\\
Mississippi\\
Missouri\\
Montana\\
New Hampshire\\
New Jersey\\
New Mexico\\
New York\\
North Carolina\\
North Dakota\\
Ohio\\
Oklahoma\\
Oregon\\
Pennsylvania\\
Rhode Island\\
South Carolina\\
Tennessee\\
Texas\\
Utah\\
Vermont\\
Virginia\\
Washington\\
West Virginia\\
Wisconsin\\
Wyoming
\end{quote}

One state has declined to take part.
\href{https://www.nytimes3xbfgragh.onion/2020/08/16/us/elections/south-dakota-governor-turns-down-extra-unemployment-funding-saying-the-state-doesnt-need-it.html}{South
Dakota's governor, Kristi Noem, announced} that her state would forgo
the federal funds, saying they were not needed because South Dakota had
recovered 80 percent of its job losses.

That leaves two states that have not been approved: Nebraska and Nevada.
Both states say they have applied.

States have until Sept. 10 to apply for the funds.

\hypertarget{payments-could-still-be-weeks-away}{%
\subsection{Payments could still be weeks
away.}\label{payments-could-still-be-weeks-away}}

Each state is supposed to administer the new supplement, just as it
processes regular state unemployment insurance and federal pandemic
jobless benefits.

In the spring, when state unemployment systems were overwhelmed with
claims, there were delays of weeks or even months because computer
systems had to be updated and reprogrammed, and staff members trained.

Now states must again work out how to process a new program while they
keep existing benefits flowing.
\href{https://www.nytimes3xbfgragh.onion/2020/08/20/business/economy/unemployment-claims.html}{New
claims} for state jobless benefits unexpectedly jumped in the most
recent weekly report to 1.1 million.

On a conference call with reporters on Thursday, John P. Pallasch,
assistant secretary for employment and training at the Labor Department,
said it could take some states up to six weeks to figure out how to get
a program up and running.

On Aug. 17, Arizona became the first state to start paying out. By
Wednesday, five additional states --- Louisiana, Missouri, Montana,
Tennessee and Texas --- had started paying out benefits, according to
the Labor Department.

Most states, however, said it could take until mid-September or later to
reprogram computer systems and take other steps to get the money to
recipients. Some states don't expect to send out funds until early
October.

\hypertarget{the-extra-benefit-is-likely-to-run-out-in-september}{%
\subsection{The extra benefit is likely to run out in
September.}\label{the-extra-benefit-is-likely-to-run-out-in-september}}

To finance the program without a congressional appropriation, Mr. Trump
set it up to draw from federal disaster funds --- a limited pool --- and
the administration said that no more than \$44 billion would be spent.

According to estimates from FEMA and the Labor Department, that sum will
cover four or five weeks of payments. The funds are supposed to be
retroactive to Aug. 1, so recipients might be paid only through early
September.

Keith Turi, a FEMA official, said on the call on Thursday that the
initial approvals were for three weeks. ``We'll add additional weeks
from there as needed,'' he said.

\hypertarget{congress-is-at-an-impasse-on-longer-term-support}{%
\subsection{Congress is at an impasse on longer-term
support.}\label{congress-is-at-an-impasse-on-longer-term-support}}

Mr. Trump acted after Democrats and Republicans were unable to work out
a deal on supplemental benefits before the August congressional recess.
Democrats have steadfastly supported restarting the \$600 weekly booster
that ended last month. Republicans have pushed for a smaller supplement
--- initially proposing \$200 a week, arguing that bigger sums
discourage people from returning to work.

Studies by economists across the political spectrum have concluded that
the additional benefits have not deterred job seekers. The
\href{https://bfi.uchicago.edu/working-paper/2020-112/}{latest,} by the
Becker Friedman Institute for Research in Economics at the University of
Chicago, found that despite anecdotal reports of people turning down
jobs, ``very few workers would not have returned to work'' if given the
opportunity. For most, the temporary nature of the supplement, the
difficulty of finding another job, and concerns about career setbacks
and permanently lower wages outweigh the short-term financial gain. And
workers who reject job offers are no longer eligible for unemployment
benefits.

Nearly 30 million people are receiving some form of jobless benefits. At
the end of June, there were
\href{https://www.bls.gov/news.release/pdf/jolts.pdf}{roughly 5.9
million job openings}.

Economists say the emergency federal checks this year have kept the
economy functioning, fueling spending that has supported restaurants,
retailers and other businesses. The \$600-a-week supplement injected
roughly \$70 billion a month into the economy between April and July,
almost 5 percent of total household income.

Nelson D. Schwartz contributed reporting.

Advertisement

\protect\hyperlink{after-bottom}{Continue reading the main story}

\hypertarget{site-index}{%
\subsection{Site Index}\label{site-index}}

\hypertarget{site-information-navigation}{%
\subsection{Site Information
Navigation}\label{site-information-navigation}}

\begin{itemize}
\tightlist
\item
  \href{https://help.nytimes3xbfgragh.onion/hc/en-us/articles/115014792127-Copyright-notice}{©~2020~The
  New York Times Company}
\end{itemize}

\begin{itemize}
\tightlist
\item
  \href{https://www.nytco.com/}{NYTCo}
\item
  \href{https://help.nytimes3xbfgragh.onion/hc/en-us/articles/115015385887-Contact-Us}{Contact
  Us}
\item
  \href{https://www.nytco.com/careers/}{Work with us}
\item
  \href{https://nytmediakit.com/}{Advertise}
\item
  \href{http://www.tbrandstudio.com/}{T Brand Studio}
\item
  \href{https://www.nytimes3xbfgragh.onion/privacy/cookie-policy\#how-do-i-manage-trackers}{Your
  Ad Choices}
\item
  \href{https://www.nytimes3xbfgragh.onion/privacy}{Privacy}
\item
  \href{https://help.nytimes3xbfgragh.onion/hc/en-us/articles/115014893428-Terms-of-service}{Terms
  of Service}
\item
  \href{https://help.nytimes3xbfgragh.onion/hc/en-us/articles/115014893968-Terms-of-sale}{Terms
  of Sale}
\item
  \href{https://spiderbites.nytimes3xbfgragh.onion}{Site Map}
\item
  \href{https://help.nytimes3xbfgragh.onion/hc/en-us}{Help}
\item
  \href{https://www.nytimes3xbfgragh.onion/subscription?campaignId=37WXW}{Subscriptions}
\end{itemize}
