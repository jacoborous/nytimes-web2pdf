Sections

SEARCH

\protect\hyperlink{site-content}{Skip to
content}\protect\hyperlink{site-index}{Skip to site index}

\href{https://www.nytimes3xbfgragh.onion/section/arts}{Arts}

\href{https://myaccount.nytimes3xbfgragh.onion/auth/login?response_type=cookie\&client_id=vi}{}

\href{https://www.nytimes3xbfgragh.onion/section/todayspaper}{Today's
Paper}

\href{/section/arts}{Arts}\textbar{}Review/Pop; Laurie Anderson Returns

\url{https://nyti.ms/29mNKsc}

\begin{itemize}
\item
\item
\item
\item
\item
\end{itemize}

Advertisement

\protect\hyperlink{after-top}{Continue reading the main story}

Supported by

\protect\hyperlink{after-sponsor}{Continue reading the main story}

\hypertarget{reviewpop-laurie-anderson-returns}{%
\section{Review/Pop; Laurie Anderson
Returns}\label{reviewpop-laurie-anderson-returns}}

By John Rockwell

\begin{itemize}
\item
  July 23, 1988
\item
  \begin{itemize}
  \item
  \item
  \item
  \item
  \item
  \end{itemize}
\end{itemize}

\includegraphics{https://s1.graylady3jvrrxbe.onion/timesmachine/pages/1/1988/07/23/054088_360W.png?quality=75\&auto=webp\&disable=upscale}

See the article in its original context from\\
July 23, 1988, Section 1, Page
11\href{https://store.nytimes3xbfgragh.onion/collections/new-york-times-page-reprints?utm_source=nytimes\&utm_medium=article-page\&utm_campaign=reprints}{Buy
Reprints}

\href{http://timesmachine.nytimes3xbfgragh.onion/timesmachine/1988/07/23/054088.html}{View
on timesmachine}

TimesMachine is an exclusive benefit for home delivery and digital
subscribers.

About the Archive

This is a digitized version of an article from The Times's print
archive, before the start of online publication in 1996. To preserve
these articles as they originally appeared, The Times does not alter,
edit or update them.

Occasionally the digitization process introduces transcription errors or
other problems; we are continuing to work to improve these archived
versions.

Laurie Anderson has been lying low of late, so her triumphant return to
New York concertizing Wednesday night at Alice Tully Hall as part of
Lincoln Center's Serious Fun festival was a heartwarming affair.

It was heartwarming because artists, and especially ''serious'' artists
who have dallied with the ''fun'' of the commercial music industry,
might seem especially prone to burnout. When Ms. Anderson emerged on the
downtown performance-art scene in the mid-1970's, her act seemed
mesmerizingly fresh; she didn't so much emerge on the scene as create
it. But after her pop-music success and big tours and movies and videos,
her grave comic monologues and funny voices and little speech-songs and
props and film clips had begun to seem just a little coy and
predictable.

On Wednesday, Ms. Anderson had changed little in her basic modus
operandi. She showed various videos from the last two years. But all of
her live material was new - and it was excellent. Furthermore, the very
act of withholding herself from the public while she made a new record
gave her a rest from us and us a rest from her.

For Ms. Anderson, ''making a record'' is an even more complex process
than it is for a high-tech rock band. Her art involves the realization
of the potential of complicated, miniaturized, ever-evolving electronic
equipment. And for an artist who uses video technology in her act and
also makes videos and films that relate to her records, any recording
project inevitably entails visual extensions as well.

One appealing aspect of Wednesday's program was that it seemed less
rigidly plotted than some of her big tours, and more like an assortment
of notions and sketches strung together into a sequence. The feeling of
surprise and experimentation was back, even though anything Ms. Anderson
does requires precise coordination with unseen technicians. Everything
seemed to come off nearly without a hitch.

The two partial innovations in her work concerned her singing and her
accompaniment. Her singing seemed new because there was so much of it,
and it sounded strong compared with her mostly spoken story-songs of the
past or her mousy vocalizations of a couple of years ago.

The accompaniment consisted of herself alone, bereft of a quasi-rock
band. Perhaps she will re-enlist a band when she embarks on a proper
promotional tour for the new album. But it was fascinating and somehow
ingratiating to see her alone on stage, the way she used to be back at
the Kitchen and other haunts, mustering up an extraordinary range of
vocal and instrumental textures all by herself.

She does this, with richer musical results than ever, through the clever
and sophisticated assemblage and programming of ''midi'' equipment - the
standardized ''musical instrument digital interface'' system that allows
all manner of musical and visual effects to be linked and controlled
from a single work-station.

But any technology is heartless unless there's artistry behind it. Ms.
Anderson's new material seemed poetically rich and musically compelling.
There was a wonderful new song about America and the West, and another
strong feminist statement revolving around the color red. Nearly all the
funny monologues got laughs in the right places, too.

If Ms. Anderson plays her cards right, she may be able to achieve that
delicate double success that so many crossover artists are aiming for
these days. If she's lucky, she'll deepen her art at the same time that
she broadens her appeal. It would be difficult to think of anyone more
likely and deserving.

Ms. Anderson will appear at Tully Hall again next Saturday.

Advertisement

\protect\hyperlink{after-bottom}{Continue reading the main story}

\hypertarget{site-index}{%
\subsection{Site Index}\label{site-index}}

\hypertarget{site-information-navigation}{%
\subsection{Site Information
Navigation}\label{site-information-navigation}}

\begin{itemize}
\tightlist
\item
  \href{https://help.nytimes3xbfgragh.onion/hc/en-us/articles/115014792127-Copyright-notice}{©~2020~The
  New York Times Company}
\end{itemize}

\begin{itemize}
\tightlist
\item
  \href{https://www.nytco.com/}{NYTCo}
\item
  \href{https://help.nytimes3xbfgragh.onion/hc/en-us/articles/115015385887-Contact-Us}{Contact
  Us}
\item
  \href{https://www.nytco.com/careers/}{Work with us}
\item
  \href{https://nytmediakit.com/}{Advertise}
\item
  \href{http://www.tbrandstudio.com/}{T Brand Studio}
\item
  \href{https://www.nytimes3xbfgragh.onion/privacy/cookie-policy\#how-do-i-manage-trackers}{Your
  Ad Choices}
\item
  \href{https://www.nytimes3xbfgragh.onion/privacy}{Privacy}
\item
  \href{https://help.nytimes3xbfgragh.onion/hc/en-us/articles/115014893428-Terms-of-service}{Terms
  of Service}
\item
  \href{https://help.nytimes3xbfgragh.onion/hc/en-us/articles/115014893968-Terms-of-sale}{Terms
  of Sale}
\item
  \href{https://spiderbites.nytimes3xbfgragh.onion}{Site Map}
\item
  \href{https://help.nytimes3xbfgragh.onion/hc/en-us}{Help}
\item
  \href{https://www.nytimes3xbfgragh.onion/subscription?campaignId=37WXW}{Subscriptions}
\end{itemize}
