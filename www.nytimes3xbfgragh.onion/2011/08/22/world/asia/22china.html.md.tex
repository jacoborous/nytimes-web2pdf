Sections

SEARCH

\protect\hyperlink{site-content}{Skip to
content}\protect\hyperlink{site-index}{Skip to site index}

\href{https://www.nytimes3xbfgragh.onion/section/world/asia}{Asia
Pacific}

\href{https://myaccount.nytimes3xbfgragh.onion/auth/login?response_type=cookie\&client_id=vi}{}

\href{https://www.nytimes3xbfgragh.onion/section/todayspaper}{Today's
Paper}

\href{/section/world/asia}{Asia Pacific}\textbar{}China and U.S. Choose
Safe Site for Biden Visit

\begin{itemize}
\item
\item
\item
\item
\item
\end{itemize}

Advertisement

\protect\hyperlink{after-top}{Continue reading the main story}

Supported by

\protect\hyperlink{after-sponsor}{Continue reading the main story}

\hypertarget{china-and-us-choose-safe-site-for-biden-visit}{%
\section{China and U.S. Choose Safe Site for Biden
Visit}\label{china-and-us-choose-safe-site-for-biden-visit}}

\includegraphics{https://static01.graylady3jvrrxbe.onion/images/2011/08/22/world/asia/CHINA/CHINA-articleLarge.jpg?quality=75\&auto=webp\&disable=upscale}

By \href{https://www.nytimes3xbfgragh.onion/by/edward-wong}{Edward Wong}

\begin{itemize}
\item
  Aug. 21, 2011
\item
  \begin{itemize}
  \item
  \item
  \item
  \item
  \item
  \end{itemize}
\end{itemize}

DUJIANGYAN, China --- The choice of sites for visits by foreign leaders
is always the product of careful deliberations, but those calculations
were particularly evident on Sunday when Vice President Joseph R. Biden
Jr. stopped at a school here.

Chinese and American officials knew that whichever school was selected
in Dujiangyan, in the western province of Sichuan, it had to reflect the
resilience of residents in recovering from the devastating earthquake of
May 12, 2008, which left more than 86,000 dead or missing.

But Chinese officials were no doubt careful to steer Mr. Biden away from
any school that would evoke memories of the thousands of students killed
when classrooms collapsed, spurring angry, grieving parents to press the
government for investigations into what they suspected was shoddy
construction. The government
\href{http://www.nytimes3xbfgragh.onion/2008/07/24/world/asia/24quake.html}{bought
the silence} of many parents with compensation money or
\href{http://www.nytimes3xbfgragh.onion/2008/06/04/world/asia/04china.html}{detained
those who were especially troublesome}.

And so on Sunday afternoon, Mr. Biden's entourage drove up to
Qingchengshan High School, set against a backdrop of green mountains
wreathed in mist. The original buildings had been badly damaged during
the earthquake, but no students had died, and the school was not
considered one that had particularly suffered from ``tofu
construction,'' now a catch phrase among Chinese for anything poorly
built.

That meant that the school was a good fit for a photo opportunity.
Moreover, the N.B.A. had helped build four basketball courts here after
the earthquake, and the United States Agency for International
Development and Cisco Systems had equipped classrooms with electronic
whiteboards. Mr. Biden and Vice President Xi Jinping, presumed to be the
next leader of China, chatted with dozens of students in white uniforms
who were playing basketball on the outdoor courts.

Just hours earlier, Mr. Biden had brought up the issue of human rights
during a speech to 400 people at Sichuan University. ``Maybe the biggest
difference in our respective approaches are our approaches to what we
refer to as human rights,'' he said.

``I recognize that many of you in this auditorium see our advocacy of
human rights as at best an intrusion and at worst an assault on your
sovereignty,'' he added. But for Americans, he said ``there is a
significant moral component to our advocacy.''

Image

Credit...The New York Times

Mr. Biden's words, though coming from a man of power, do not resonate as
much across China as the memory of the victims of the collapsed schools
--- an official estimate said 5,335 students had died --- and of the
suppression by the government of the voices of the grieving parents and
their demands for justice. Critics say the crackdown by the government
was one of the worst abuses of human rights in recent years.

And parents in China are increasingly asking whether a child's right to
a safe life should be considered a human right, universal to all and
somehow guaranteed by the government. Since the collapsed schools first
galvanized that sentiment, scare after scare has followed. In the fall
of 2008, news spread of a tainted milk scandal --- six infants had died
and 300,000 had fallen ill --- and of the government scramble to buy off
or detain angry parents. In the spring of 2010, a series of unrelated
school stabbings prompted questions about the ability of officials to
ensure security at schools. This year, food safety is a bigger issue
than ever before.

``But of course we're angry,'' Tian Wenyao, the mother of a 12-year-old
boy who died in the collapse of
\href{http://www.nytimes3xbfgragh.onion/2008/05/15/world/asia/15morgue.html}{Xinjian
Primary School} here in Dujiangyan, said in June 2008. ``Who wouldn't be
angry? In the morning, my child said to me, `Mama, I'm going to school.'
In the evening, he turned up a corpse.''

Ms. Tian was among a group of mothers who had peacefully protested in
front of government offices only to have the police break up the
demonstration and drag away some parents.

The most famous human rights case this year, the three-month detention
of the artist Ai Weiwei, also has ties to the school collapses. One of
Mr. Ai's most political projects in recent years was to direct a team of
volunteers in trying to compile a list of names of students who had died
in the collapses. He created an exhibition with children's backpacks to
symbolize the deaths, and he traveled in 2009 to Sichuan to lend support
to Tan Zuoren, an advocate of the parents who was being tried for
inciting subversion. Mr. Ai was beaten by police officers, and Mr. Tan
was sentenced to three years in prison.

On Sunday, sticking to diplomacy, Mr. Biden did not mention the grieving
parents in any public manner when meeting with Mr. Xi. The two first
spoke together in a large gathering outside the Qingchengshan school,
then walked over to the basketball players. Mr. Biden tried making a
shot six times before finally succeeding on the seventh. He and Mr. Xi
signed a basketball; Mr. Biden scrawled ``U.S.A.''

This was not Xinjian Primary School, nor was it
\href{http://www.nytimes3xbfgragh.onion/2008/05/13/world/asia/13scene.html?ref=asia}{Juyuan
Middle School}, in a suburb of Dujiangyan, where 900 students had died
in a cascade of concrete. But the gym teacher here, He Zhengzhong, still
shook his head when asked about the earthquake. ``We were out here in
prefabricated buildings for 10 months,'' he said.

Inside a classroom, Mr. Biden and Mr. Xi sat in front of a semicircle of
31 students studying English and took questions. Mr. Biden talked about
the dedication of his wife, Jill, to teaching (she has worked at
community colleges), and he urged one girl to go into that profession.
Mr. Xi sat and listened with little expression. But he became animated
toward the end of the session, speaking to the children and smiling when
he paraphrased a famous line from Mao. ``The young people are like the
sun in the morning,'' Mr. Xi said. ``The world belongs to you.''

Advertisement

\protect\hyperlink{after-bottom}{Continue reading the main story}

\hypertarget{site-index}{%
\subsection{Site Index}\label{site-index}}

\hypertarget{site-information-navigation}{%
\subsection{Site Information
Navigation}\label{site-information-navigation}}

\begin{itemize}
\tightlist
\item
  \href{https://help.nytimes3xbfgragh.onion/hc/en-us/articles/115014792127-Copyright-notice}{©~2020~The
  New York Times Company}
\end{itemize}

\begin{itemize}
\tightlist
\item
  \href{https://www.nytco.com/}{NYTCo}
\item
  \href{https://help.nytimes3xbfgragh.onion/hc/en-us/articles/115015385887-Contact-Us}{Contact
  Us}
\item
  \href{https://www.nytco.com/careers/}{Work with us}
\item
  \href{https://nytmediakit.com/}{Advertise}
\item
  \href{http://www.tbrandstudio.com/}{T Brand Studio}
\item
  \href{https://www.nytimes3xbfgragh.onion/privacy/cookie-policy\#how-do-i-manage-trackers}{Your
  Ad Choices}
\item
  \href{https://www.nytimes3xbfgragh.onion/privacy}{Privacy}
\item
  \href{https://help.nytimes3xbfgragh.onion/hc/en-us/articles/115014893428-Terms-of-service}{Terms
  of Service}
\item
  \href{https://help.nytimes3xbfgragh.onion/hc/en-us/articles/115014893968-Terms-of-sale}{Terms
  of Sale}
\item
  \href{https://spiderbites.nytimes3xbfgragh.onion}{Site Map}
\item
  \href{https://help.nytimes3xbfgragh.onion/hc/en-us}{Help}
\item
  \href{https://www.nytimes3xbfgragh.onion/subscription?campaignId=37WXW}{Subscriptions}
\end{itemize}
