Sections

SEARCH

\protect\hyperlink{site-content}{Skip to
content}\protect\hyperlink{site-index}{Skip to site index}

\href{https://www.nytimes3xbfgragh.onion/section/world/asia}{Asia
Pacific}

\href{https://myaccount.nytimes3xbfgragh.onion/auth/login?response_type=cookie\&client_id=vi}{}

\href{https://www.nytimes3xbfgragh.onion/section/todayspaper}{Today's
Paper}

\href{/section/world/asia}{Asia Pacific}\textbar{}As U.S. Looks to Asia,
It Sees China Everywhere

\begin{itemize}
\item
\item
\item
\item
\item
\end{itemize}

Advertisement

\protect\hyperlink{after-top}{Continue reading the main story}

Supported by

\protect\hyperlink{after-sponsor}{Continue reading the main story}

News Analysis

\hypertarget{as-us-looks-to-asia-it-sees-china-everywhere}{%
\section{As U.S. Looks to Asia, It Sees China
Everywhere}\label{as-us-looks-to-asia-it-sees-china-everywhere}}

\includegraphics{https://static01.graylady3jvrrxbe.onion/images/2011/11/16/world/ASIA/ASIA-articleLarge.jpg?quality=75\&auto=webp\&disable=upscale}

By \href{https://www.nytimes3xbfgragh.onion/by/ian-johnson}{Ian Johnson}
and \href{https://www.nytimes3xbfgragh.onion/by/jackie-calmes}{Jackie
Calmes}

\begin{itemize}
\item
  Nov. 15, 2011
\item
  \begin{itemize}
  \item
  \item
  \item
  \item
  \item
  \end{itemize}
\end{itemize}

The last time the remote Australian city of Darwin played a significant
role in American military planning was during the early days of World
War II, when Gen. Douglas MacArthur used the port as the base for his
campaign to reclaim the Pacific from the Japanese.

So it was with considerable symbolism that President Obama arrived on
Wednesday in Canberra, Australia's capital, for a trip that will include
an announcement that the United States plans to use Darwin as a new
center of operations in Asia as it seeks to reassert itself in the
region and grapple with China's rise.

The United States is taking some first steps --- bold in rhetoric, still
mostly modest in practice --- to prove to its Asian allies that it
intends to remain a crucial military and economic power in the region as
the wars in Iraq and Afghanistan draw to a close. The new Australian
base, coming after decades in which the Pentagon has been slowly but
steadily reducing its troop presence in Asia, puts American planes and
ships closer to trading corridors in the South China Sea, where some
traditional American allies worry that China is trying to flex its
military muscle.

Over the past year and a half, China has moved to assert territorial
claims in the resource-rich but hotly contested waters near the
Philippines and Vietnam. Many of the region's smaller countries have
asked Washington to re-engage in the region as a counterweight.

``The U.S. needs to show the Chinese that they still have the power to
overwhelm them, that they still can prevail if something really wrong
happens,'' said Huang Jing, a foreign affairs analyst and visiting
professor at the National University of Singapore. ``It's a hedging
policy.''

For the United States, the more muscular approach toward China has
far-reaching implications, not just geopolitically but also
economically. With Republicans at home calling for punitive measures
against China for its currency and trade practices, Mr. Obama wants to
appear strong in pressing Beijing. He made headway on an ambitious
American plan to create a Pacific free trade zone, known as the
Trans-Pacific Partnership, that, for now, would not include China.

For the Pentagon, which faces sweeping budget cuts in Congress, shifting
its focus toward Asia provides a strong argument against cutting back
its naval presence in the Pacific --- something that Defense Secretary
Leon E. Panetta explicitly ruled out in a recent visit to the region. He
and Secretary of State Hillary Rodham Clinton have been prime proponents
of the emphasis on Asia, with Mrs. Clinton shoring up old alliances,
like those with Japan and South Korea, and cultivating new partners,
like India and Indonesia.

Inside the White House, that emphasis has been reinforced by the
president's national security adviser, Thomas E. Donilon, who has argued
that the United States needs to ``rebalance'' its strategic emphasis,
from the combat theaters in Iraq and Afghanistan toward Asia, where he
contends that Washington has put too few resources in recent years,
because of its preoccupation with the two wars.

China has become the largest trading partner with most of the countries
in the region, undercutting American economic influence. It also is
projecting military power more broadly than at any other time in modern
history. Its true military budget is not made public, but experts say it
has at least tripled over the past decade, allowing China to strengthen
a relatively weak maritime presence by building more modern ships that
can operate with greater range and arming its first aircraft carrier. It
has shown off what appears to be new stealth aircraft and has bought
advanced weapons from Russia.

United States military spending remains many times larger than analysts'
projections of China's real military budget, but much of that has been
sucked into the Afghan and Iraq conflicts. Further, the Obama
administration has committed to cutting \$400 billion over 10 years, and
budget battles may result in further cuts.

The American situation widens the opening for a more assertive China.

Earlier last year, Chinese officials warned administration officials
visiting Beijing that China would not tolerate any interference in the
region. This year, Chinese ships or planes began taking more forceful
action. Officials in the Philippines say Chinese forces entered
Philippine waters or airspace six times, including once when a Chinese
frigate fired in the direction of a Philippine fishing boat. Vietnam has
reported that Chinese ships cut the cables of two exploration ships
carrying out seismic surveys.

On Tuesday, Philippine officials said China had recently protested their
plans to explore waters less than 50 miles offshore from the
Philippines, saying the waters fall under its territorial jurisdiction.

\includegraphics{https://static01.graylady3jvrrxbe.onion/images/2011/11/16/world/16obama_337/16obama_337-jumbo.jpg?quality=75\&auto=webp\&disable=upscale}

The United States began pushing back last year. A quadrennial Pentagon
review identified several countries in the region as strategic partners.
The United States also began to restore bilateral ties with Myanmar
(formerly Burma) and to promote ties with Indonesia.

Most dramatically, at a regional meeting in Hanoi in the summer of 2010,
Mrs. Clinton emphatically argued that the United States had a vital
interest in maintaining open and peaceful sea lanes in the South China
Sea. She called for all disputes to be settled in international forums.
China's foreign minister stormed out.

Administration officials have hewed to Mrs. Clinton's line. ``The South
China Sea is a very important maritime common for the entire region''
but also for the United States, Adm. Robert F. Willard, commander of the
United States Pacific Command, told reporters traveling with Mr. Obama.
The navigation lanes account for \$5.3 trillion in bilateral annual
trade, of which \$1.2 trillion is American, he said.

Obama administration officials say its stronger role is not just because
of American interests. Benjamin Rhodes, deputy national security adviser
for strategic communications, said Mr. Obama was focusing on
``responding to both the extraordinary interest we have in the region,
but also a demand, an interest from the nations of the region for the
United States to play a role.''

As a sign of this, Mr. Obama will join the leaders of 16 other nations
for the sixth East Asia Summit meeting in Bali this week, the first time
an American president has participated in the forum.

The move is part of a broader strategy to re-embrace multilateralism. In
recent years, Washington had come to view Asian regional groups as
limiting its ability to act, while China embraced regional partnerships
before its rise to regional superpower. Now, those roles appear to have
switched. The United States has `` turned the multilateral tables on
China,'' said Carlyle A. Thayer, a professor of international relations
at the Australian Defense Force Academy.

But multilateralism has taken on an aggressive tinge, some analysts
contend. ``Beneath the surface they're becoming an arena for subtle but,
for the region, quite unnerving power plays and influence games between
the U.S. and China,'' said one analyst in Washington, Michael Green of
the Center for Strategic and International Studies.

The more robust American position is proving difficult for many in China
to accept.

Global Times, a subsidiary of the Communist Party's flagship newspaper,
People's Daily, wrote Tuesday that the United States was trying to
``form a gang'' against China's territorial claims on the South China
Sea.

Many Chinese have grown angry over the American moves in the region,
which are frequently reported and heavily criticized in the
state-controlled press.

``The United States is trying to use the small nations as marionettes,''
said Ge Fen, a Hangzhou-based television producer. ``It's trying to hide
behind them to encircle China.''

But many more sober voices are also present.

``If the Chinese government is clever, it would do well to think about
the reason why the U.S. is suddenly so popular in the region,'' said Shi
Yinhong, director of the Center on American Studies at the Renmin
University in Beijing. ``Is it because China has not been good enough
when it comes to diplomacy with its neighboring countries?''

There are some signs that China may be adjusting its policies to answer
such criticism. Over the past few months, it has shown a renewed
willingness to strike more cooperative deals with its neighbors. Last
week, it announced that it would join its southeastern neighbors in
combating pirates on the lower Mekong River. In July, China set
guidelines for implementing a ``declaration on conduct'' with Southeast
Asian nations over the resolution of disputes in the South China Sea.

``We're back in a cautiously optimistic position,'' Professor Thayer
said.

Advertisement

\protect\hyperlink{after-bottom}{Continue reading the main story}

\hypertarget{site-index}{%
\subsection{Site Index}\label{site-index}}

\hypertarget{site-information-navigation}{%
\subsection{Site Information
Navigation}\label{site-information-navigation}}

\begin{itemize}
\tightlist
\item
  \href{https://help.nytimes3xbfgragh.onion/hc/en-us/articles/115014792127-Copyright-notice}{©~2020~The
  New York Times Company}
\end{itemize}

\begin{itemize}
\tightlist
\item
  \href{https://www.nytco.com/}{NYTCo}
\item
  \href{https://help.nytimes3xbfgragh.onion/hc/en-us/articles/115015385887-Contact-Us}{Contact
  Us}
\item
  \href{https://www.nytco.com/careers/}{Work with us}
\item
  \href{https://nytmediakit.com/}{Advertise}
\item
  \href{http://www.tbrandstudio.com/}{T Brand Studio}
\item
  \href{https://www.nytimes3xbfgragh.onion/privacy/cookie-policy\#how-do-i-manage-trackers}{Your
  Ad Choices}
\item
  \href{https://www.nytimes3xbfgragh.onion/privacy}{Privacy}
\item
  \href{https://help.nytimes3xbfgragh.onion/hc/en-us/articles/115014893428-Terms-of-service}{Terms
  of Service}
\item
  \href{https://help.nytimes3xbfgragh.onion/hc/en-us/articles/115014893968-Terms-of-sale}{Terms
  of Sale}
\item
  \href{https://spiderbites.nytimes3xbfgragh.onion}{Site Map}
\item
  \href{https://help.nytimes3xbfgragh.onion/hc/en-us}{Help}
\item
  \href{https://www.nytimes3xbfgragh.onion/subscription?campaignId=37WXW}{Subscriptions}
\end{itemize}
