Sections

SEARCH

\protect\hyperlink{site-content}{Skip to
content}\protect\hyperlink{site-index}{Skip to site index}

\href{https://www.nytimes3xbfgragh.onion/section/arts}{Arts}

\href{https://myaccount.nytimes3xbfgragh.onion/auth/login?response_type=cookie\&client_id=vi}{}

\href{https://www.nytimes3xbfgragh.onion/section/todayspaper}{Today's
Paper}

\href{/section/arts}{Arts}\textbar{}Gerhard Richter, Grand Master of Our
Time

\begin{itemize}
\item
\item
\item
\item
\item
\end{itemize}

Advertisement

\protect\hyperlink{after-top}{Continue reading the main story}

Supported by

\protect\hyperlink{after-sponsor}{Continue reading the main story}

Art

\hypertarget{gerhard-richter-grand-master-of-our-time}{%
\section{Gerhard Richter, Grand Master of Our
Time}\label{gerhard-richter-grand-master-of-our-time}}

By \href{https://www.nytimes3xbfgragh.onion/by/souren-melikian}{Souren
Melikian}

\begin{itemize}
\item
  Oct. 28, 2011
\item
  \begin{itemize}
  \item
  \item
  \item
  \item
  \item
  \end{itemize}
\end{itemize}

LONDON --- At age 79, Gerhard Richter is the towering figure of
Contemporary art on the international scene. ``Panorama,'' a
retrospective at Tate Modern, is a landmark that will be remembered long
after the show closes on Jan. 8.

The German artist, born in Dresden in 1932, is not only supremely
important for the admirable paintings from his abstractionist phase.

His work as a whole yields a clue to the astonishingly disparate
character of Contemporary art.

From its very beginnings, Mr. Richter's oeuvre reveals a concern that is
at the heart of Contemporary art, even if it never explicitly
acknowledged.

While artists active in the early 20th century concentrated on painterly
achievement, those who took center stage after World War II focused on
the modes of perception. In the 1960s, Mr. Richter used his skills in
photography to translate into oil painting the visual effects that can
be achieved with a camera.

``Curtain III,'' a large composition executed in 1965, catches the
rhythmic impression made by fabric folds. Other pictures reflecting the
impact of photography are about the emotions triggered by things
remembered. ``Bombers,'' painted in 1963 in shades of black and white,
looks like a still from a wartime movie.

\includegraphics{https://static01.graylady3jvrrxbe.onion/images/2011/10/29/arts/29iht-melikian29D/29iht-melikian29D-jumbo.jpg?quality=75\&auto=webp\&disable=upscale}

This would have powerfully resonated with the German generation that
lived in fear of British and U.S. aircraft dropping clusters of bombs at
the end World War II.

Early on, Mr. Richter invented a pictorial technique that may be called
Photographic Impressionism. In ``Woman With Child,'' painted in 1965,
the blur is reminiscent of the haziness of Degas pastels.

Switching at will to nonfigural art, Mr. Richter dwelled on the impact
that the intensity of color on one hand and optical effects on the other
have on our perception.

In 1966, the artist neatly reproduced 192 colored squares in a gigantic
color chart, while ``Grey Streaks,'' painted in 1968, resembles Op Art
of the 1960s --- the blurred streaks seem to be vibrating.

These experiments in perception left their imprint over pictures that do
not resemble a photograph. The composition of ``Townscape, Paris,''
painted in shades of black, gray and white, owes something to aerial
photography, but the crisscross brushstrokes reduce what might have been
a figural picture to near abstraction. It explodes with Expressionist
energy.

Image

"Abstract Painting, 1992." The oil on canvas could depict the wooden
poles of some gate in a flooded area rising above a mirror-like watery
surface that reflects them. The coloristic nuances and the subtlety of
the brush strokes are stunning.Credit...Gerhard Richter

With amazing versatility, Mr. Richter produced seascapes and cloud
studies in another vein derived from photography which reveal his
familiarity with the great masters of the past. ``Seascape (Cloudy),''
painted in 1969, calls to mind the seascapes of the 17th-century Dutch
artist Jan van de Cappelle, with his predilection for immense cloudy
skies seen in pale light.

A year later, Mr. Richter painted ``Clouds,'' a sequence illustrating
three stages in the slow disintegration of a cloud into the deep grayish
blue around it. Here, the German artist may have remembered the cloud
studies of the English school, from late 18th-century watercolorists
like John Cozens to Richard Bonington and Constable.

Remarkably for an artist who loved monochromatic painting, Mr. Richter
also indulged in magnificent polychromy. ``Study for Clouds (Abstract)''
does not really suggest clouds.

It looks like some kind of Abstractionist apotheosis perceived in a
lyrical frame of mind without any material subject. One is faintly
reminded of the skies in some of Murillo and El Greco's religious
pictures.

Clearly, the gems of the past were vividly present in the contemporary
master's mind. This is demonstrated by works such as ``Annunciation
After Titian'' done in 1973 in delicate, pastel-like hues. The blur
again conjures memories of Degas drawing pastel ballet scenes.

Image

"Seascape (Cloudy)," painted in 1969, calls to mind the seascapes of the
17th-century Dutch artist Jan van de Cappelle, with his predilection for
immense cloudy skies seen in pale light.Credit...Gerhard Richter

``Chinon,'' a panoramic view painted on a monumental scale in 1987 is
perhaps the most striking tribute paid by Mr. Richter to much admired
predecessors. The view is accurately observed, as in the Dutch
tradition.

It conveys a sense of immensity like the 17th-century landscapes of
Phillips de Koninck and shares with these the long horizontal
composition under a huge sky.

By contrast, the color scheme in dainty browns and greens has a delicate
Barbizon school touch.

The photography-derived blur recurs at intervals throughout the
painter's career.

When portraying a young woman in two different postures, perhaps based
on snapshots taken at short intervals, Mr. Richter may have been looking
back to his beginnings 20 years earlier. In ``Confrontation 1,'' the
sitter seen head and shoulders, three-quarters to the right, turns her
face to look at the viewer with a faint laugh on her lips in the
timeless manner of European portraiture. But the brushwork and the hazy
monochrome gray are pure Richter vintage.

Amazingly, the figural portraits of the young woman and the Chinon
panorama were executed at a time when Mr. Richter was also painting vast
abstract compositions. By 1987, he was on the point of producing his
greatest masterpieces.

Image

The composition of "Townscape, Paris" owes something to aerial
photography, and it explodes with Expressionist energy.Credit...Gerhard
Richter

``Abstract painting,'' which dates from that year, bears witness to the
artist's supreme mastery at using intense color in bold combinations.

As is so often the case with Mr. Richter's work, these are a throwback
to earlier schools --- the association of carmine red, acid yellow and a
bit of blue in several shades, is the color scheme already used between
1905 and 1907 in their landscapes by followers of the Fauve movement
such as Maurice de Vlaminck or André Derain.

As with some Old Masters, Mr. Richter's greatest works, all in his
Abstractionist manner, were executed in his later period starting around
1990.

That year, Mr. Richter completed two unforgettable compositions,
``Forest (3)'' and ``Forest (4).'' One, on loan from a private
collection not otherwise identified, is a sensational revelation. The
other, from the Fisher Collection in San Francisco, is better known to
specialists, but hardly to the wider public. Their titles
notwithstanding, both reminded me of glistening light reflections when
water projected by blustery showers runs down large panes of glass on a
rainy winter night in New York. Both follow compositional rules ---
whether by design or by spontaneous inspiration in the course of
execution, as Mr. Richter insists he does.

Around that time, Mr. Richter began producing what remains to this day
the most extraordinary succession of abstract pictures in Western art.
Their unequaled diversity is perhaps accounted for by the fact that,
somehow, nearly all appear to relate to the material world. None is a
gratuitous accumulation of color splashes or a haphazard distribution of
lines.

Image

"Forest (4)" follows compositional rules, whether by design or by
spontaneous inspiration in the course of execution, as the artist
Gerhard Richter insists he does.Credit...Gerhard Richter

Two works dating from 1992, which are merely titled ``Abstract
Painting,'' are poles apart. One, in oil on aluminum, is faintly
reminiscent of sliding doors in a Japanese mansion, with their small
translucent square panels. The other, in oil on canvas, could depict the
wooden poles of some gate in a flooded area rising above a mirror-like
watery surface that reflects them. The coloristic nuances and the
subtlety of the brush strokes are stunning.

Mr. Richter is equally at ease when painting monumental sizes and small
formats.

A series of exquisite abstract panels done in 1999 in oil on a support
called Aludibond all measure 50 x 72 centimeters, or 20 x 28 inches.

One, only identified by its archival registration number, CR:858-5,
looks like a study of reflections on the still waters of a pond. Another
(registration number CR: 858:7) could be inspired by the vertical planks
of some barrier painted in purplish red with silvery touches .

As if to confirm his endless versatility, Mr. Richter has continued all
along to paint in a figural manner.

``Reader,'' which portrays a blond girl seen sideways and dates from
1995, tells us that the painter spent time gazing at Georges de la
Tour's characters seen in the chiaroscuro light cast by candles. ``S.
with Child,'' a title used for several pictures of a mother
breastfeeding her baby, sends back memories of Berthe Morisot, the
Impressionist artist.

This aptitude at using constantly, and with equal virtuosity, the
pictorial language of figural and abstract art remains unique in our
time.

So is Mr. Richter's dexterity at wielding the brush, a rarity these
days. It makes the work of some of his acclaimed contemporaries look
futile and oddly amateurish.

Advertisement

\protect\hyperlink{after-bottom}{Continue reading the main story}

\hypertarget{site-index}{%
\subsection{Site Index}\label{site-index}}

\hypertarget{site-information-navigation}{%
\subsection{Site Information
Navigation}\label{site-information-navigation}}

\begin{itemize}
\tightlist
\item
  \href{https://help.nytimes3xbfgragh.onion/hc/en-us/articles/115014792127-Copyright-notice}{©~2020~The
  New York Times Company}
\end{itemize}

\begin{itemize}
\tightlist
\item
  \href{https://www.nytco.com/}{NYTCo}
\item
  \href{https://help.nytimes3xbfgragh.onion/hc/en-us/articles/115015385887-Contact-Us}{Contact
  Us}
\item
  \href{https://www.nytco.com/careers/}{Work with us}
\item
  \href{https://nytmediakit.com/}{Advertise}
\item
  \href{http://www.tbrandstudio.com/}{T Brand Studio}
\item
  \href{https://www.nytimes3xbfgragh.onion/privacy/cookie-policy\#how-do-i-manage-trackers}{Your
  Ad Choices}
\item
  \href{https://www.nytimes3xbfgragh.onion/privacy}{Privacy}
\item
  \href{https://help.nytimes3xbfgragh.onion/hc/en-us/articles/115014893428-Terms-of-service}{Terms
  of Service}
\item
  \href{https://help.nytimes3xbfgragh.onion/hc/en-us/articles/115014893968-Terms-of-sale}{Terms
  of Sale}
\item
  \href{https://spiderbites.nytimes3xbfgragh.onion}{Site Map}
\item
  \href{https://help.nytimes3xbfgragh.onion/hc/en-us}{Help}
\item
  \href{https://www.nytimes3xbfgragh.onion/subscription?campaignId=37WXW}{Subscriptions}
\end{itemize}
