Sections

SEARCH

\protect\hyperlink{site-content}{Skip to
content}\protect\hyperlink{site-index}{Skip to site index}

\href{https://www.nytimes3xbfgragh.onion/es/section/ciencia-y-tecnologia}{Ciencia
y Tecnología}

\href{https://myaccount.nytimes3xbfgragh.onion/auth/login?response_type=cookie\&client_id=vi}{}

\href{https://www.nytimes3xbfgragh.onion/section/todayspaper}{Today's
Paper}

\href{/es/section/ciencia-y-tecnologia}{Ciencia y
Tecnología}\textbar{}¿Tengo síntomas de COVID-19?

\url{https://nyti.ms/3a1k1Qm}

\begin{itemize}
\item
\item
\item
\item
\item
\end{itemize}

\hypertarget{tengo-suxedntomas-de-covid-19}{%
\section{¿Tengo síntomas de
COVID-19?}\label{tengo-suxedntomas-de-covid-19}}

Por \href{https://www.nytimes3xbfgragh.onion/by/tara-parker-pope}{Tara
Parker-Pope} y
\href{https://www.nytimes3xbfgragh.onion/by/mika-grondahl}{Mika
Gröndahl}Aug. 6, 2020

\href{https://www.nytimes3xbfgragh.onion/interactive/2020/08/05/well/covid-19-symptoms.html}{Read
in English}

\begin{itemize}
\item
\item
\item
\item
\end{itemize}

Congestión o secreción nasal

Tos con flema

Opresión en el pecho

Falta de aire al moverse

Náusea y vómitos

Diarrea y dolor abdominal

Escalofríos y dolor corporal

Congestión o secreción nasal

Tos con flema

Opresión en el pecho

Falta de aire al moverse

Náusea y vómitos

Escalofríos y dolor corporal

Diarrea y dolor abdominal

Congestión o secreción nasal

Tos con flema

Opresión en el pecho

Falta de aire al moverse

Náusea y vómitos

Diarrea y dolor abdominal

Escalofríos y dolor corporal

Dolor de cabeza

Mareo o pensamiento confuso

Fiebre

Malestar ocular

Pérdida del gusto o el olfato

Inflamación de garganta

Congestión o secreción nasal

Dificultad respiratoria en reposo

Tos seca

Palpitaciones, dolor de pecho

Tos con flema

Coágulos

Opresión en el pecho

Dolor muscular y articular agudo

Falta de aire al moverse

Salpullido

Náusea y vómitos

Ampollas en los dedos

Escalofríos y dolor corporal

Diarrea y dolor abdominal

Fatiga

Dolor de cabeza

Mareo o pensamiento confuso

Malestar ocular

Fiebre

Congestión o secreción nasal

Pérdida del gusto o el olfato

Inflamación de garganta

Tos seca

Tos con flema

Dificultad respiratoria en reposo

Opresión en el pecho

Falta de aire al moverse

Palpitaciones, dolor de pecho

Náusea y vómitos

Salpullido

Diarrea y dolor abdominal

Ampollas en los dedos

Dolor muscular y articular agudo

Escalofríos y dolor corporal

Fatiga

Coágulos

Dolor de cabeza

Mareo o pensamiento confuso

Malestar ocular

Fiebre

Congestión o secreción nasal

Pérdida del gusto o el olfato

Inflamación de garganta

Tos seca

Tos con flema

Dificultad respiratoria en reposo

Opresión en el pecho

Falta de aire al moverse

Palpitaciones, dolor de pecho

Náusea y vómitos

Salpullido

Diarrea y dolor abdominal

Dolor muscular y articular agudo

Ampollas en los dedos

Escalofríos y dolor corporal

Fatiga

Coágulos

Dolor de cabeza

Mareo o pensamiento confuso

Malestar ocular

Fiebre

Congestión o secreción nasal

Pérdida del gusto o el olfato

Inflamación de garganta

Tos seca

Tos con flema

Dificultad respiratoria en reposo

Opresión en el pecho

Falta de aire al moverse

Palpitaciones, dolor de pecho

Náusea y vómitos

Salpullido

Diarrea y dolor abdominal

Dolor muscular y articular agudo

Ampollas en los dedos

Escalofríos y dolor corporal

Fatiga

Coágulos

Dolor de cabeza

Mareo o pensamiento confuso

Malestar ocular

Fiebre

Congestión o secreción nasal

Pérdida del gusto o el olfato

Inflamación de garganta

Tos seca

Tos con flema

Dificultad respiratoria en reposo

Opresión en el pecho

Falta de aire al moverse

Palpitaciones, dolor de pecho

Náusea y vómitos

Salpullido

Diarrea y dolor abdominal

Ampollas en los dedos

Dolor muscular y articular agudo

Escalofríos y dolor corporal

Fatiga

Coágulos

Dolor de cabeza

Mareo o pensamiento confuso

Fiebre

Malestar ocular

Pérdida del gusto o el olfato

Inflamación de garganta

Congestión o secreción nasal

Dificultad respiratoria en reposo

Tos seca

Palpitaciones, dolor de pecho

Tos con flema

Coágulos

Opresión en el pecho

Dolor muscular y articular agudo

Falta de aire al moverse

Salpullido

Náusea y vómitos

Ampollas en los dedos

Escalofríos y dolor corporal

Diarrea y dolor abdominal

Fatiga

Dolor de cabeza

Mareo o pensamiento confuso

Fiebre

Malestar ocular

Pérdida del gusto o el olfato

Inflamación de garganta

Congestión o secreción nasal

Dificultad respiratoria en reposo

Tos seca

Palpitaciones, dolor de pecho

Tos con flema

Coágulos

Opresión en el pecho

Dolor muscular y articular agudo

Falta de aire al moverse

Salpullido

Náusea y vómitos

Ampollas en los dedos

Escalofríos y dolor corporal

Diarrea y dolor abdominal

Fatiga

Dolor de cabeza

Mareo o pensamiento confuso

Malestar ocular

Fiebre

Congestión o secreción nasal

Pérdida del gusto o el olfato

Inflamación de garganta

Tos seca

Tos con flema

Dificultad respiratoria en reposo

Opresión en el pecho

Falta de aire al moverse

Palpitaciones, dolor de pecho

Náusea y vómitos

Salpullido

Diarrea y dolor abdominal

Ampollas en los dedos

Dolor muscular y articular agudo

Escalofríos y dolor corporal

Fatiga

Coágulos

Dolor de cabeza

Mareo o pensamiento confuso

Malestar ocular

Fiebre

Congestión o secreción nasal

Pérdida del gusto o el olfato

Inflamación de garganta

Tos seca

Tos con flema

Dificultad respiratoria en reposo

Opresión en el pecho

Falta de aire al moverse

Palpitaciones, dolor de pecho

Náusea y vómitos

Salpullido

Diarrea y dolor abdominal

Dolor muscular y articular agudo

Ampollas en los dedos

Escalofríos y dolor corporal

Fatiga

Coágulos

La anosmia, o \textbf{pérdida del sentido del olfato}, es un síntoma
diferenciador. En un estudio, el 87 por ciento de los pacientes
perdieron el olfato y el 56 por ciento reportó pérdida del gusto.

Al principio la \textbf{secreción o congestión nasal} no se consideraba
un síntoma pero ahora los CDC dicen que la congestión y el goteo
postnasal son signos comunes**.**

Una tos seca y persistente es a menudo uno de los primeros síntomas de
la COVID-19, pero algunos pacientes presentan \textbf{tos con flema} o
\textbf{dolor de garganta}.

La anosmia, o \textbf{pérdida del sentido del olfato}, es un síntoma
diferenciador. En un estudio, el 87 por ciento de los pacientes
perdieron el olfato y el 56 por ciento reportó pérdida del gusto.

Al principio la \textbf{secreción o congestión nasal} no se consideraba
un síntoma pero ahora los CDC dicen que la congestión y el goteo
postnasal son signos comunes**.**

Una tos seca y persistente es a menudo uno de los primeros síntomas de
la COVID-19, pero algunos pacientes presentan \textbf{tos con flema} o
\textbf{dolor de garganta}.

La anosmia, o \textbf{pérdida del sentido del olfato}, es un síntoma
diferenciador. En un estudio, el 87 por ciento de los pacientes
perdieron el olfato y el 56 por ciento reportó pérdida del gusto.

Al principio la \textbf{secreción o congestión nasal} no se consideraba
un síntoma pero ahora los CDC dicen que la congestión y el goteo
postnasal son signos comunes**.**

Una tos seca y persistente es a menudo uno de los primeros síntomas de
la COVID-19, pero algunos pacientes presentan \textbf{tos con flema} o
\textbf{dolor de garganta}.

El \textbf{mareo} y la confusión mental son más comunes en pacientes
mayores.

Aunque la \textbf{fiebre} y el \textbf{dolor de cabeza} son comunes,
muchos pacientes ---hasta 60 por ciento en un estudio--- jamás
desarrollan fiebre.

Los síntomas oculares, entre ellos el \textbf{dolor}, \textbf{comezón},
\textbf{lagrimeo} y \textbf{enrojecimiento} de los ojos pueden darse
hasta en el 25 por ciento de los pacientes.

Aunque la \textbf{fiebre} y el \textbf{dolor de cabeza} son comunes,
muchos pacientes ---hasta 60 por ciento en un estudio--- jamás
desarrollan fiebre.

El \textbf{mareo} y la confusión mental son más comunes en pacientes
mayores.

Los síntomas oculares, entre ellos el \textbf{dolor}, \textbf{comezón},
\textbf{lagrimeo} y \textbf{enrojecimiento} de los ojos pueden darse
hasta en el 25 por ciento de los pacientes.

El \textbf{mareo} y la confusión mental son más comunes en pacientes
mayores.

Aunque la \textbf{fiebre} y el \textbf{dolor de cabeza} son comunes,
muchos pacientes ---hasta 60 por ciento en un estudio--- jamás
desarrollan fiebre.

Los síntomas oculares, entre ellos el \textbf{dolor}, \textbf{comezón},
\textbf{lagrimeo} y \textbf{enrojecimiento} de los ojos pueden darse
hasta en el 25 por ciento de los pacientes.

Es muy común sentirse \textbf{falto de aliento} y con \textbf{opresión
en el pecho} \emph{al realizar actividades.} Si estos síntomas persisten
o si la dificultad para respirar avanza, busca a un médico.

El dolor de pecho, las \textbf{palpitaciones} y los \textbf{problemas
cardíacos} son menos comunes y pueden ser causados por la inflamación,
la baja presión arterial o la falta de oxígeno.

Es posible que los pacientes hospitalizados sufran problemas renales que
pueden causar \textbf{hinchazón en las piernas} y \textbf{sangrado
urinario}.

La \textbf{diarrea}, el \textbf{dolor abdominal}, la \textbf{náusea} y
el \textbf{vómito} son comunes. Algunos pacientes solo presentan
problemas gastrointestinales como único síntoma.

Es muy común sentirse \textbf{falto de aliento} y con \textbf{opresión
en el pecho} \emph{al realizar actividades.} Si estos síntomas persisten
o si la dificultad para respirar avanza, busca a un médico.

El dolor de pecho, las \textbf{palpitaciones} y los \textbf{problemas
cardíacos} son menos comunes y pueden ser causados por la inflamación,
la baja presión arterial o la falta de oxígeno.

Es posible que los pacientes hospitalizados sufran problemas renales que
pueden causar \textbf{hinchazón en las piernas} y \textbf{sangrado
urinario}.

La \textbf{diarrea}, el \textbf{dolor abdominal}, la \textbf{náusea} y
el \textbf{vómito} son comunes. Algunos pacientes solo presentan
problemas gastrointestinales como único síntoma.

Es muy común sentirse \textbf{falto de aliento} y con \textbf{opresión
en el pecho} \emph{al realizar actividades.} Si estos síntomas persisten
o si la dificultad para respirar avanza, busca a un médico.

El dolor de pecho, las \textbf{palpitaciones} y los \textbf{problemas
cardíacos} son menos comunes y pueden ser causados por la inflamación,
la baja presión arterial o la falta de oxígeno.

La \textbf{diarrea}, el \textbf{dolor abdominal}, la \textbf{náusea} y
el \textbf{vómito} son comunes. Algunos pacientes solo presentan
problemas gastrointestinales como único síntoma.

Es posible que los pacientes hospitalizados sufran problemas renales que
pueden causar \textbf{hinchazón en las piernas} y \textbf{sangrado
urinario}.

Una respuesta inmunitaria puede causar \textbf{erupciones} en el cuerpo.
Los coágulos pueden ocasionar lesiones rojas y dolorosas en dedos de
manos y pies.

Muchos pacientes reportan \textbf{dolor muscular}, \textbf{escalofríos}
y \textbf{fatiga}.

Una respuesta inmunitaria puede causar \textbf{erupciones} en el cuerpo.
Los coágulos pueden ocasionar lesiones rojas y dolorosas en dedos de
manos y pies.

Muchos pacientes reportan \textbf{dolor muscular}, \textbf{escalofríos}
y \textbf{fatiga}.

Una respuesta inmunitaria puede causar \textbf{erupciones} en el cuerpo.
Los coágulos pueden ocasionar lesiones rojas y dolorosas en dedos de
manos y pies.

Muchos pacientes reportan \textbf{dolor muscular}, \textbf{escalofríos}
y \textbf{fatiga}.

Los \textbf{coágulos} y las \textbf{embolias} son infrecuentes pero
pueden ser mortales.

Los \textbf{coágulos} y las \textbf{embolias} son infrecuentes pero
pueden ser mortales.

Los \textbf{coágulos} y las \textbf{embolias} son infrecuentes pero
pueden ser mortales.

Fiebre prolongada (5 o + días)

Letargo, irritabilidad, confusión

Inapetencia o debilidad para comer o beber

Palidez, manchas y/o amoratamiento en niños de piel clara

Dificultad para respirar o respiración rápida

Taquicardia o dolor de pecho

Dolor abdominal severo, diarrea o vómito

Micción reducida

Fiebre prolongada (5 o + días)

Letargo, irritabilidad, confusión

Inapetencia o debilidad para comer o beber

Palidez, manchas y/o amoratamiento en niños de piel clara

Dificultad para respirar o respiración rápida

Taquicardia o dolor de pecho

Dolor abdominal severo, diarrea o vómito

Micción reducida

Fiebre prolongada (5 o + días)

Letargo, irritabilidad, confusión

Inapetencia o debilidad para comer o beber

Palidez, manchas y/o amoratamiento en niños de piel clara

Dificultad para respirar o respiración rápida

Taquicardia o dolor de pecho

Dolor abdominal severo, diarrea o vómito

Micción reducida

\begin{itemize}
\item
\item
\item
\item
\item
\item
\item
\item
\item
\item
\item
\end{itemize}

Cada tos, dolor de cabeza o estornudo por estos días te hace dudar:
¿podría ser la COVID-19? Los expertos médicos observan que la COVID-19
es \textbf{una enfermedad multiorgánica} que puede afectar el cuerpo de
la cabeza a los dedos de los pies y todo lo que hay entre un extremo y
otro. Esta guía te ayudará a comprender los síntomas.

Estos cuatro síntomas sonmuy comunes entre los pacientes de covid. A
diferencia de los de la gripe, que suelen aparecer rápidamente, los
síntomas de la COVID-19 demoran varios días en surgir.

Muchos pacientes también reportancomúnmenteuno o más de estos síntomas.
Algunos pacientes solo tienen una enfermedad leve, pero otros empiezan a
sentirse muy mal, experimentan incomodidad constante y empeoramiento de
los síntomas.

La COVID-19 no es solo una enfermedad respiratoria, y puede manifestarse
de varias formas inusuales. Estos síntomas son raros, o menos comunes,
pero también pueden ser signos de la COVID-19.

La \textbf{nariz} es
\href{https://www.nature.com/articles/s41591-020-0868-6}{la zona cero}
de la COVID-19 porque ahí abunda
\href{https://www.nytimes3xbfgragh.onion/es/2020/06/03/espanol/misterios-coronavirus.html}{un
receptor llamado ACE2}, que el virus utiliza para apoderarse de nuestras
células. Una vez que el virus se reproduce dentro de la nariz y se
propaga al tracto respiratorio y hacia los pulmones, los pacientes
empiezan a desarrollar varios síntomas respiratorios.

La jaqueca aguda es común, pero los problemas neurológicos más serios
suelen ser menos habituales. Los síntomas leves incluyen mareo o
aturdimiento. Entre los síntomas que requieren atención urgente se
encuentran la confusión, la incapacidad de despertar, el movimiento
descoordinado o los signos de una embolia como la parálisis o
insensibilidad facial, o la dificultad para hablar.

Algunos pacientes desarrollan neumonía por COVID-19 cuando el virus
ataca los pulmones. En ocasiones los niveles de oxígeno caen de manera
tan lenta que los pacientes no se dan cuenta. La respiración corta y
rápida o la falta severa de aire, particularmente al estar en reposo,
son signos de que se requiere atención médica urgente.

El virus puede expresarse de maneras muy inusuales en todo el cuerpo. Se
han reportado erupciones raras de la piel (desiguales, lisas, con o sin
comezón). En ocasiones poco frecuentes el virus causa inflamación o daña
los músculos de las piernas, los hombros o la espalda de manera muy
dolorosa.

El virus también parece aferrarse al interior de los vasos sanguíneos y,
en ocasiones poco frecuentes, causa coágulos que pueden ser mortales al
trasladarse a los pulmones, corazón o cerebro. En casos extremadamente
raros los coágulos pueden entorpecer el flujo sanguíneo hacia las
extremidades y se requiere amputación. Los pacientes de gravedad que
ameritan hospitalización pueden recibir medicamentos anticoagulantes a
modo de tratamiento o prevención.

Por lo general la COVID-19 es leve en el caso de los niños. En ocasiones
raras, puede causar una respuesta inflamatoria grave. Busca atención de
emergencias si un niño muestra algunos de estos síntomas u otros signos
preocupantes.

Si presentas algún síntoma que pueda ser de COVID-19, los médicos
recomiendan aislarse hasta que consigas realizarte una prueba. La
mayoría de los pacientes logran recuperarse en unas semanas.

Es buena idea
\href{https://www.nytimes3xbfgragh.onion/es/2020/04/29/espanol/estilos-de-vida/oximetro-para-que-sirve.html}{monitorear
los niveles de oxígeno en casa} con la ayuda de un oxímetro de pulso.
Presta atención a los síntomas en los primeros 5 a 10 días de la
enfermedad, cuando los niveles de oxígeno pueden bajar a niveles
peligrosos.

Busca atención médica si tienes dificultad para respirar, presentas
algún síntoma relacionado o tu condición empeora.

Desde un estornudo o tos que se siente como una alergia a dolores
corporales severos y fatiga paralizante, los síntomas del coronavirus
son impredecibles de la cabeza a los pies. Lee más sobre
\href{https://www.nytimes3xbfgragh.onion/2020/08/05/well/live/coronavirus-covid-symptoms.html}{los
muchos síntomas de la COVID-19}.

Notas y fuentes: Este gráfico se basa el trabajo inédito del Dr. Mark A.
Perazella, de la facultad de medicina de la Universidad de Yale; Dr.
Kenar D. Jhaveri, Escuela Zucker de Medicina de Hofstra/Northwell; y el
Dr. Hassan Izzedine, Clínica Internacional Parc Monceau. También toma
como referencia trabajos publicados en
\href{https://onlinelibrary.wiley.com/doi/full/10.1111/acem.14009}{Academic
Emergency Medicine} (Mayo 2020);
\href{https://www.acpjournals.org/doi/pdf/10.7326/M20-2428}{Annals of
Internal Medicine} (Mayo 2020);
\href{https://pubmed.ncbi.nlm.nih.gov/32492712/}{Blood} (Julio 2020);
\href{https://onlinelibrary.wiley.com/doi/10.1111/bjd.19327}{British
Journal of Dermatology} (Junio 2020);
\href{https://www.ncbi.nlm.nih.gov/pmc/articles/PMC7194555/}{Gastroenterology}
(Abril 2020);
\href{https://www.hopkinsguides.com/hopkins/view/Johns_Hopkins_ABX_Guide/540747/all/Coronavirus_COVID_19__SARS_CoV_2_}{Hopkins
Guides} (Julio 2020);
\href{https://jamanetwork.com/journals/jama/fullarticle/2761044}{JAMA}
(Febrero 2020);
\href{https://jamanetwork.com/journals/jamacardiology/fullarticle/2763845}{JAMA
Cardiology} (Marzo 2020);
\href{https://jamanetwork.com/journals/jamadermatology/fullarticle/2767775}{JAMA
Dermatology} (Junio 2020); JAMA Neurology
(\href{https://jamanetwork.com/journals/jamaneurology/fullarticle/2764549}{Abril
2020} y
\href{https://jamanetwork.com/journals/jamaneurology/fullarticle/2768098}{Julio
2020});
\href{https://www.jaad.org/article/S0190-9622(20)32126-5/fulltext}{Journal
of the American Academy of Dermatology} (Julio 2020);
\href{https://www.ncbi.nlm.nih.gov/pmc/articles/PMC7088708/}{Journal of
General Internal Medicine} (May 2020); The Lancet
(\href{https://www.thelancet.com/journals/lancet/article/PIIS0140-6736(20)30937-5/fulltext}{Abril}
y
\href{https://www.thelancet.com/journals/laneur/article/PIIS1474-4422(20)30221-0/fulltext}{Julio}
2020); \href{https://www.nature.com/articles/s41591-020-0868-6}{Nature
Medicine} (Abril 2020);
\href{https://www.nejm.org/doi/pdf/10.1056/NEJMoa2002032}{New England
Journal of Medicine} (Febrero 2020);
\href{https://www.nejm.org/doi/full/10.1056/NEJMc2008597}{NEJM} (Junio
2020);
\href{https://www.sciencedirect.com/science/article/pii/S016164202030405X?via\%3Dihub}{Ophthalmology}
(Julio 2020);
\href{https://www.reviewofoptometry.com/news/article/ocular-covid19-symptoms-more-common-than-thought}{Review
of Optometry} (Mayo 2020);
\href{https://www.ahajournals.org/doi/pdf/10.1161/STROKEAHA.120.030335}{Stroke}
(Julio 2020); y los reportes y las pautas de los
\href{https://www.cdc.gov/coronavirus/2019-ncov/symptoms-testing/symptoms.html}{Centros
para el Control y Prevención de Enfermedades}, el
\href{https://coronavirus.health.ny.gov/childhood-inflammatory-disease-related-covid-19?gclid=Cj0KCQjwyJn5BRDrARIsADZ9ykHf-lzC09eZFv1lbtLJ0n8TgUM3xN0jt_cBakctwMU7lV__mJyMC3waAmcxEALw_wcB}{Departamento
de Salud de Nueva York} y la
\href{https://www.who.int/docs/default-source/coronaviruse/who-china-joint-mission-on-covid-19-final-report.pdf}{Organización
Mundial de la Salud}.

Trabajo adicional de Josh Williams y Lalena Fisher.

\begin{itemize}
\item
\item
\item
\item
\end{itemize}

Advertisement

\protect\hyperlink{after-bottom}{Continue reading the main story}

\hypertarget{site-index}{%
\subsection{Site Index}\label{site-index}}

\hypertarget{site-information-navigation}{%
\subsection{Site Information
Navigation}\label{site-information-navigation}}

\begin{itemize}
\tightlist
\item
  \href{https://help.nytimes3xbfgragh.onion/hc/en-us/articles/115014792127-Copyright-notice}{©~2020~The
  New York Times Company}
\end{itemize}

\begin{itemize}
\tightlist
\item
  \href{https://www.nytco.com/}{NYTCo}
\item
  \href{https://help.nytimes3xbfgragh.onion/hc/en-us/articles/115015385887-Contact-Us}{Contact
  Us}
\item
  \href{https://www.nytco.com/careers/}{Work with us}
\item
  \href{https://nytmediakit.com/}{Advertise}
\item
  \href{http://www.tbrandstudio.com/}{T Brand Studio}
\item
  \href{https://www.nytimes3xbfgragh.onion/privacy/cookie-policy\#how-do-i-manage-trackers}{Your
  Ad Choices}
\item
  \href{https://www.nytimes3xbfgragh.onion/privacy}{Privacy}
\item
  \href{https://help.nytimes3xbfgragh.onion/hc/en-us/articles/115014893428-Terms-of-service}{Terms
  of Service}
\item
  \href{https://help.nytimes3xbfgragh.onion/hc/en-us/articles/115014893968-Terms-of-sale}{Terms
  of Sale}
\item
  \href{https://spiderbites.nytimes3xbfgragh.onion}{Site Map}
\item
  \href{https://help.nytimes3xbfgragh.onion/hc/en-us}{Help}
\item
  \href{https://www.nytimes3xbfgragh.onion/subscription?campaignId=37WXW}{Subscriptions}
\end{itemize}
