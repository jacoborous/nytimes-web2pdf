Sections

SEARCH

\protect\hyperlink{site-content}{Skip to
content}\protect\hyperlink{site-index}{Skip to site index}

\href{https://www.nytimes3xbfgragh.onion/es/}{en Español}

\href{https://myaccount.nytimes3xbfgragh.onion/auth/login?response_type=cookie\&client_id=vi}{}

\href{https://www.nytimes3xbfgragh.onion/section/todayspaper}{Today's
Paper}

\href{/es/}{en Español}\textbar{}Muchos latinos no pudieron quedarse en
casa. Ahora se están enfermando

\url{https://nyti.ms/31iLrid}

\begin{itemize}
\item
\item
\item
\item
\item
\item
\end{itemize}

\hypertarget{el-brote-de-coronavirus}{%
\subsubsection{\texorpdfstring{\href{https://www.nytimes3xbfgragh.onion/es/spotlight/coronavirus?name=styln-coronavirus-es\&region=TOP_BANNER\&block=storyline_menu_recirc\&action=click\&pgtype=Article\&impression_id=e2b81a60-f2b5-11ea-9507-e581488c52dd\&variant=undefined}{El
brote de
coronavirus}}{El brote de coronavirus}}\label{el-brote-de-coronavirus}}

\begin{itemize}
\tightlist
\item
  \href{https://www.nytimes3xbfgragh.onion/es/interactive/2020/08/06/espanol/ciencia-y-tecnologia/tengo-covid-19-sintomas.html?name=styln-coronavirus-es\&region=TOP_BANNER\&block=storyline_menu_recirc\&action=click\&pgtype=Article\&impression_id=e2b81a61-f2b5-11ea-9507-e581488c52dd\&variant=undefined}{Síntomas}
\item
  \href{https://www.nytimes3xbfgragh.onion/es/2020/09/02/espanol/ciencia-y-tecnologia/vacunas-experimentales-coronavirus.html?name=styln-coronavirus-es\&region=TOP_BANNER\&block=storyline_menu_recirc\&action=click\&pgtype=Article\&impression_id=e2b81a62-f2b5-11ea-9507-e581488c52dd\&variant=undefined}{Vacunas
  experimentales}
\item
  \href{https://www.nytimes3xbfgragh.onion/es/2020/08/31/espanol/mundo/rebrote-espana.html?name=styln-coronavirus-es\&region=TOP_BANNER\&block=storyline_menu_recirc\&action=click\&pgtype=Article\&impression_id=e2b84170-f2b5-11ea-9507-e581488c52dd\&variant=undefined}{Rebrote
  en España}
\item
  \href{https://www.nytimes3xbfgragh.onion/es/2020/09/02/espanol/negocios/desalojos-trump.html?name=styln-coronavirus-es\&region=TOP_BANNER\&block=storyline_menu_recirc\&action=click\&pgtype=Article\&impression_id=e2b84171-f2b5-11ea-9507-e581488c52dd\&variant=undefined}{Moratoria
  a los desalojos}
\item
  \href{https://www.nytimes3xbfgragh.onion/es/2020/08/26/espanol/ciencia-y-tecnologia/coronavirus-afecta-hombres.html?name=styln-coronavirus-es\&region=TOP_BANNER\&block=storyline_menu_recirc\&action=click\&pgtype=Article\&impression_id=e2b84172-f2b5-11ea-9507-e581488c52dd\&variant=undefined}{El
  impacto en los hombres}
\end{itemize}

Advertisement

\protect\hyperlink{after-top}{Continue reading the main story}

Supported by

\protect\hyperlink{after-sponsor}{Continue reading the main story}

Estados Unidos

\hypertarget{muchos-latinos-no-pudieron-quedarse-en-casa-ahora-se-estuxe1n-enfermando}{%
\section{Muchos latinos no pudieron quedarse en casa. Ahora se están
enfermando}\label{muchos-latinos-no-pudieron-quedarse-en-casa-ahora-se-estuxe1n-enfermando}}

Las tasas de infección de coronavirus en las comunidades latinas han
crecido rápidamente en todo Estados Unidos.

\includegraphics{https://static01.graylady3jvrrxbe.onion/images/2020/06/22/us/26virus-latinos-ES/merlin_173647548_873e3640-1730-4a38-8d47-ee8409a3aa98-articleLarge.jpg?quality=75\&auto=webp\&disable=upscale}

Por \href{https://www.nytimes3xbfgragh.onion/by/shawn-hubler}{Shawn
Hubler},
\href{https://www.nytimes3xbfgragh.onion/by/thomas-fuller}{Thomas
Fuller},
\href{https://www.nytimes3xbfgragh.onion/by/anjali-singhvi}{Anjali
Singhvi} y Juliette Love

\begin{itemize}
\item
  Publicado 26 de junio de 2020Actualizado 14 de julio de 2020
\item
  \begin{itemize}
  \item
  \item
  \item
  \item
  \item
  \item
  \end{itemize}
\end{itemize}

\href{https://www.nytimes3xbfgragh.onion/2020/06/26/us/corona-virus-latinos.html}{Read
in English}

\href{https://www.nytimes3xbfgragh.onion/newsletters/el-times}{Regístrate
para recibir nuestro boletín} con lo mejor de The New York Times.

\begin{center}\rule{0.5\linewidth}{\linethickness}\end{center}

DINUBA, California --- Cuando el coronavirus se propagó por primera vez
en los campos y las plantas procesadoras de alimentos en el valle
central de California, el jefe de Graciela Ramírez anunció que los
trabajadores de línea que temían a la infección iban a poder quedarse en
casa sin paga.

Como operadora de maquinaria en Ruiz Foods, el mayor fabricante de
burritos congelados de Estados Unidos, Ramírez siguió acudiendo a su
trabajo para asegurarse de no perder su remuneración de 750 dólares por
semana.

``Tengo necesidades'', dijo en español Ramírez, de 40 años y madre de
cuatro hijos. ``Mi comida, mi renta, mis cuentas''.

Pronto, sus compañeros de trabajo comenzaron a enfermarse, y cuando
Ramírez empezó a tener congestión y fatiga y
\href{https://www.nytimes3xbfgragh.onion/2020/03/22/health/coronavirus-symptoms-smell-taste.html}{perdió
la capacidad de oler} la diferencia entre el arroz en su estufa y la
sopa de fideo en su plato, su prueba también resultó positiva.

{[}\emph{Consulta el}
\textbf{\href{https://www.nytimes3xbfgragh.onion/es/interactive/2020/espanol/mundo/coronavirus-en-estados-unidos.html}{\emph{mapa
de coronavirus en Estados
Unidos}}}\href{https://www.nytimes3xbfgragh.onion/es/interactive/2020/espanol/mundo/coronavirus-en-estados-unidos.html}{**}\emph{con
los datos detallados de casos y muertes{]}}

Fue una variación de lo que se ha convertido en un tema demográfico
sombrío, y no solo en California. Las infecciones entre los latinos han
superado con creces a las del resto del país, evidencia de la
composición de la fuerza de trabajo esencial de la nación a medida que
la epidemia estadounidense ha aumentado una vez más en las últimas
semanas.

Los latinos en Estados Unidos no son un monolito cultural y no hay
evidencia de que ningún grupo étnico sea inherentemente más vulnerable
al virus que otros. Pero en las últimas dos semanas, los condados de
todo el país donde al menos una cuarta parte de la población es latina
han registrado un incremento del 32 por ciento en nuevos casos,
comparados con un aumento del 15 por ciento en todos los demás condados,
muestra un análisis del Times.

Donde los brotes han sido peores en los últimos 14 días

Condados con

Pocos latinos

Muchos latinos

4\%

7\%

12\%

25\%

20 casos nuevos

por cada 10.000 personas

15

10

5

0

Sur

Oeste

Medio oeste

Noreste

Where outbreaks have been

worst in the last 14 days

Condados con

Pocos latinos

Muchos latinos

4\%

7\%

12\%

25\%

20 casos nuevos

por cada 10.000 personas

15

10

5

0

Sur

Oeste

15 casos nuevos

por cada 10.000 personas

10

5

0

Medio oeste

Noreste

Nota: Los condados han sido separados en cinco grupos de aproximadamente
la misma población basándose en la proporción de latinos que viven en
ellos, según la encuesta American Community realizada durante cinco años
y publicada en 2018. La información sobre el número de casos proviene de
una base de datos del
\href{https://www.nytimes3xbfgragh.onion/interactive/2020/us/coronavirus-us-cases.html}{Times}.

El análisis confirma los amplios recuentos nacionales de los Centros
para el Control y la Prevención de Enfermedades, que muestran que los
latinos constituyen el 34 por ciento de los casos en todo Estados
Unidos, una proporción muy superior al 18 por ciento que el grupo
representa en la población total.

También subraya un cambio desde el inicio del brote, en especial en
áreas fuera de las ciudades, como el condado de Tulare, California, que
inicialmente había evitado en gran medida los picos debilitantes de
infecciones como los que se habían visto en Nueva York, Nueva Orleans,
Chicago y otras áreas metropolitanas importantes.

La disparidad es particularmente marcada en estados poblados como
\href{https://www.cdph.ca.gov/Programs/CID/DCDC/Pages/COVID-19/Race-Ethnicity.aspx}{California},
Florida y Texas. Pero también ha aparecido en otras zonas del país. En
Carolina del Norte, los latinos suman el 10 por ciento de la población,
pero el\href{https://covid19.ncdhhs.gov/dashboard/cases}{46 por ciento
de las infecciones.} En Wisconsin, son el 7 por ciento de la población y
el
\href{https://madison.com/wsj/news/local/health-med-fit/black-latino-covid-19-disparities-bring-attention-to-broad-health-inequities/article_4b5f367c-6b0a-58db-a711-046c4008b66d.html\#tracking-source=home-top-story-1}{33
por ciento de los casos}. En el condado de Yakima, Washington, el lugar
del peor brote del estado, la mitad de los residentes son latinos. En el
condado de Santa Cruz, que tiene la tasa más alta de casos en Arizona,
la proporción hispana de la población es del 84 por ciento.

Los datos detallados de coronavirus desglosados por origen étnico son
incompletos en muchos lugares, lo que dificulta saber por qué los
latinos se han infectado en tasas más altas. Los condados con una alta
proporción de latinos también tienden a presentar atributos que han
hecho a los condados
\href{http://jedkolko.com/2020/06/21/the-changing-geography-of-covid19/}{vulnerables}
al aumento reciente: hogares abarrotados, poblaciones más jóvenes y
clima más cálido que empuja a las personas al interior, dijo Jed Kolko,
investigador y economista jefe de Indeed.com, un sitio web de búsquedas
de empleo. El rastreo de contactos en algunas áreas también han asociado
los picos de infección con
\href{https://www.sacbee.com/news/coronavirus/article243376046.html}{grandes
reuniones familiares.}

Cómo han cambiado con el tiempo los nuevos casos

Condados con

Pocos latinos

Muchos latinos

4\%

7\%

12\%

25\%

3 casos nuevos

por cada 10.000 personas

Los casos crecieron este mes

en condados con muchos latinos.

2

1

1 de abril

1 de mayo

1 de junio

24 de junio

1 de marzo

Cómo han cambiado con el tiempo los nuevos casos

Condados con

Pocos latinos

Muchos latinos

4\%

7\%

12\%

25\%

3 casos nuevos

por cada 10.000

personas

Los casos que han crecido este

mes en condados

con muchos latinos.

2

1

1 de marzo

1 de abril

1 de mayo

1 de junio

NOTA: Los números de casos se agregan utilizando un promedio móvil de
siete días.

Pero el inexorable aumento de infecciones entre los latinos desde la
Pascua
---\href{https://www.nytimes3xbfgragh.onion/es/2020/06/24/espanol/america-latina/coronavirus-mexico-brasil-peru-chile-uruguay.html}{tanto
aquí como en los países latinoamericanos}--- ha alarmado a los
funcionarios de salud y las organizaciones latinas, que están pidiendo
pruebas más específicas, una recolección de datos más completa y mejores
protecciones en el lugar de trabajo a medida que se reabre la economía.

Y se ha convertido en un punto caliente en los estados que votan
generalmente por el Partido Republicano (llamados estados rojos en
inglés), donde las infecciones
\href{https://www.nytimes3xbfgragh.onion/interactive/2020/us/coronavirus-us-cases.html?action=click\&module=Top\%20Stories\&pgtype=Homepage\#states}{también
están al alza}. Líderes latinos demócratas y dirigentes de derechos
civiles
\href{https://www.tampabay.com/news/health/2020/06/19/ron-desantis-blames-florida-farmworkers-for-covid-aid-groups-say-testing-help-came-late/}{exigieron
un pedido de disculpas} esta semana al gobernador republicano de
Florida, Ron DeSantis, quien atribuyó el fuerte aumento de las pruebas
positivas de coronavirus en su estado a
\href{https://www.nytimes3xbfgragh.onion/2020/06/18/us/florida-coronavirus-immokalee-farmworkers.html}{``trabajadores
agrícolas abrumadoramente hispanos''}. Los críticos de DeSantis dicen
que su gobierno está convirtiendo en chivo expiatorio a los trabajadores
inmigrantes después de ignorar los pedidos de más pruebas y protección
para ellos.

En California, donde los latinos representan el 39 por ciento de la
población y
\href{https://www.cdph.ca.gov/Programs/CID/DCDC/Pages/COVID-19/Race-Ethnicity.aspx}{cerca
del 57 por ciento de los casos nuevos,} los picos han sido
particularmente confusos. El estado fue el primero del país en quedarse
en casa, y los datos de geolocalización de los teléfonos celulares
indican que sus residentes estuvieron entre los más comprometidos a
limitar su movimiento y, con ello, la propagación de la enfermedad.

Las tasas de infección se han mantenido relativamente bajas en los
vecindarios acaudalados, incluido los ocupados por los latinos ricos del
estado. Pero el confinamiento en casa nunca ocurrió para muchas familias
latinas cuyos miembros trabajan en industrias que no cerraron, lo que
los hace especialmente vulnerables al virus.

Durante el confinamiento, millones de latinos mantuvieron en
funcionamiento una economía escueta: en las mesas de corte de las
plantas procesadoras de alimentos, como agricultores, encargados de
hospitales, preparadores de alimentos, trabajadores de supermercado y en
muchos otros trabajos considerados esenciales. Y llevaron el virus a
casas a menudo hacinadas, lo que agravó la propagación.

``Este era un punto totalmente ciego'', dijo Alicia Fernández, profesora
de medicina en la Universidad de California, en San Francisco, que se
especializa en la salud de latinos e inmigrantes. ``Se necesita hacer
mucho, mucho más en cuestión de protección laboral en el espacio de
trabajo''.

Ahora el virus acecha a los latinos del sur al norte de California. El
condado de Imperial, una región agrícola predominantemente latina al
este de San Diego, tiene la tasa más alta de infección del estado: el
doble de la tasa de Los Ángeles y más alta que la del golpeado estado de
Nueva York. En San Francisco, los latinos constituyen el 15 por ciento
de la población pero representan la mitad de los casos de coronavirus.

Muchas calles de San Francisco estaban casi desiertas durante el
confinamiento. Pero la imagen era distinta en los hogares latinos del
área de la bahía, donde continuaba la rutina diaria de los
desplazamientos a lugares de trabajo distantes.

``Quedarse en casa es un lujo'', dijo Kirsten Bibbins-Domingo,
vicedecana de salud poblacional y equidad de acceso a la salud en la
Escuela de Medicina de San Francisco en la Universidad de California.
``En las partes más adineradas de la ciudad, la gente se quedó en casa
desde antes y por más tiempo, porque se requiere de recursos. No todas
las comunidades se pueden dar el lujo de hacer eso''.

Los investigadores dicen que una de las ilustraciones más crudas de cómo
el virus penetró en la comunidad latina viene de un estudio en Mission,
un distrito de la ciudad,
\href{https://www.ucsf.edu/news/2020/05/417356/initial-results-mission-district-covid-19-testing-announced}{dirigido
por la Universidad de California en San Francisco}.

En conjunto con organizaciones latinas locales, los investigadores
evaluaron a casi 4000 voluntarios para detectar el coronavirus en un
área de alrededor de cuatro cuadras por seis cuadras.

Aproximadamente un número igual de latinos (40 por ciento) y personas
blancas no latinas (41 por ciento) fueron evaluados en el estudio. Pero
casi todos los que estaban infectados eran latinos; menos del uno por
ciento fueron personas blancas no latinas.

Image

El sábado la gente se relajaba en un parque del distrito Mission de San
Francisco.Credit...Brian L. Frank para The New York Times

Image

Una tienda en el distrito Mission de San Francisco ofrece
cubrebocas.Credit...Brian L. Frank para The New York Times

Al inicio de la pandemia, los latinos no parecían ser más vulnerables
que otros. De hecho, los latinos de California estaban subrepresentados
en los datos de principios de abril y representaba alrededor del 30 por
ciento de los casos. Sin embargo, a medida que las órdenes de quedarse
en casa se afianzaron, las tasas de infección entre los latinos
aumentaron en comparación con otros grupos.

Los puntos críticos surgieron en áreas con grandes poblaciones latinas,
como el barrio de Fruitvale en Oakland.

El virus se propagó por el condado de Kings, un área agrícola en el
valle central con numerosas plantas de procesamiento de carne; ahora
tiene la segunda tasa más alta de infección en el estado. El condado de
Tulare, cuya población es 64 por ciento hispana, subió al cuarto lugar
entre los condados.

``Estamos viendo una concentración de impactos de la COVID-19 en
industrias que son mayoritariamente latinas: procesamiento de alimentos
en interiores, empaques, procesamiento de carne'', dijo Phoebe Seaton,
codirectora ejecutiva y cofundadora del Consejo de Liderazgo para la
Justicia y la Responsabilidad, una organización de derechos civiles con
sede en Fresno, California, que aboga por más protecciones contra el
virus en los lugares de trabajo.

El empleador de Ramírez, Ruiz Foods, tiene su sede en Dinuba, en el
condado de Tulare. La empresa familiar fue fundada por Fred Ruiz, uno de
los filántropos latinos más conocidos del estado y miembro del grupo de
trabajo pandémico para la recuperación económica del gobernador Gavin
Newsom. Ahora emplea a 1500 personas en California y a otras 2300 en
plantas en Texas y Carolina del Sur, que fabrican los burritos El
Monterey y alrededor de otras 200 variedades de comida mexicana
congelada.

\includegraphics{https://static01.graylady3jvrrxbe.onion/images/2020/06/23/us/26VIRUS-LATINOS-02/merlin_172253373_9b024b89-5997-4fc3-a839-e6d066f659cc-articleLarge.jpg?quality=75\&auto=webp\&disable=upscale}

Rachel P. Cullen, la directora ejecutiva de la compañía, dijo que, como
muchas empresas, la respuesta inicial de Ruiz fue dar a los empleados la
opción de trabajar en casa y tomarse días de vacaciones o tiempo no
remunerado si sus trabajos no podían hacerse de forma remota.

Después de la Pascua, sin embargo, el recuento de casos en el condado de
Tulare se disparó, y la compañía tomó medidas agresivas para abordar el
virus. Las pruebas fueron obligatorias para todos los empleados, dijo
Cullen en un comunicado, y rápidamente ``se aumentó el distanciamiento
físico, el uso obligatorio de cubrebocas, las barreras flexibles, el
control de síntomas y la toma de temperatura, la limitación de
visitantes y las restricciones de viaje''.

Ningún empleado ha muerto por la COVID-19, dijo Cullen, pero la planta
de Dinuba se convirtió en un punto crítico, y dos trabajadores fueron
hospitalizados. Dijo que 331 empleados se han recuperado de la COVID-19
desde abril, y que cerca de 15 tienen infecciones activas.

Ramírez sospecha que pescó el virus en el comedor de la compañía, donde
las mesas ahora están acordonadas para imponer el distanciamiento
social. En la línea de producción donde ella trabaja con otro centenar
de personas, dijo, se colocaron láminas de plástico para separar a los
empleados y botellas de desinfectante de manos en cada pasillo. Eso no
era el caso antes de abril, dijo.

Aún así, ella no culpa a su empleador. ``Muchos de nosotros no creíamos
en la COVID al principio'', dijo en español. ``Yo no creí, porque no he
mirado a nadie que lo tuviera hasta que me dio a mí''.

Una semana más tarde de que llegaran los resultados de su prueba, su
hija de 20 años, Cynthia Orozco, también dio positivo. Debido a que su
hija también cuida a los dos hijos más pequeños de Ramírez, de dos y
diez años, el doctor le dijo a Ramírez que asumiera que ellos también
tenían el virus.

Nadie de la familia ha requerido hospitalización, pero cuando se corrió
la voz entre los amigos y parientes en California, Nevada y México, ella
supo que su situación no era tan inusual como había pensado.

Un colega de trabajo de su esposo en una industria láctea en Visalia se
infectó. Igual que un primo en Bakersfield que era empleado en una
Dollar Store. En Modesto, un primo en el negocio de la construcción
tenía la COVID-19 y estaba preocupado por su equipo en San Francisco.

``Somos los que estamos afuera, en la fuerza laboral'', dijo Orozco,
estudiante de ingeniería civil en la Universidad Estatal de California,
Fresno, quien agregó que el virus también le había costado varias
semanas de trabajo.

Orozco dijo que ella y su madre aún no se han hecho las pruebas de
seguimiento. Pero el último fin de semana, tras meses desoladores de
distanciamiento de sus seres queridos, se pusieron cubrebocas y se
fueron a una gran fiesta familiar al aire libre.

``Todos usaron desinfectante para manos, y pusieron nombres en sus vasos
para que nadie accidentalmente tomara la bebida del otro'', dijo. ``Y
todos nada más nos saludamos con un puño al aire en vez de abrazarnos''.

Shawn Hubler es corresponsal en California con sede en Sacramento. Antes
de unirse al Times en 2020, pasó casi dos décadas cubriendo el estado
para Los Angeles Times como reportera itinerante, columnista y escritora
de revista. Compartió tres premios Pulitzer con el equipo Metro del
periódico. \href{https://twitter.com/ShawnHubler}{@ShawnHubler}

Thomas Fuller es el jefe del buró en San Francisco. Ha pasado las dos
últimas décadas en puestos en el extranjero para el Times y el
International Herald Tribune en Europa y, más recientemente, en el
sudeste asiático.
\href{https://twitter.com/thomasfullerNYT}{@thomasfullerNYT} •
\href{https://www.facebookcorewwwi.onion/thomas.fuller.9889}{Facebook}

Anjali Singhvi es editora de gráficos. Es arquitecta y tiene una
maestría en planificación urbana de la Universidad de Columbia.
\href{https://twitter.com/singhvianjali}{@singhvianjali}

Advertisement

\protect\hyperlink{after-bottom}{Continue reading the main story}

\hypertarget{site-index}{%
\subsection{Site Index}\label{site-index}}

\hypertarget{site-information-navigation}{%
\subsection{Site Information
Navigation}\label{site-information-navigation}}

\begin{itemize}
\tightlist
\item
  \href{https://help.nytimes3xbfgragh.onion/hc/en-us/articles/115014792127-Copyright-notice}{©~2020~The
  New York Times Company}
\end{itemize}

\begin{itemize}
\tightlist
\item
  \href{https://www.nytco.com/}{NYTCo}
\item
  \href{https://help.nytimes3xbfgragh.onion/hc/en-us/articles/115015385887-Contact-Us}{Contact
  Us}
\item
  \href{https://www.nytco.com/careers/}{Work with us}
\item
  \href{https://nytmediakit.com/}{Advertise}
\item
  \href{http://www.tbrandstudio.com/}{T Brand Studio}
\item
  \href{https://www.nytimes3xbfgragh.onion/privacy/cookie-policy\#how-do-i-manage-trackers}{Your
  Ad Choices}
\item
  \href{https://www.nytimes3xbfgragh.onion/privacy}{Privacy}
\item
  \href{https://help.nytimes3xbfgragh.onion/hc/en-us/articles/115014893428-Terms-of-service}{Terms
  of Service}
\item
  \href{https://help.nytimes3xbfgragh.onion/hc/en-us/articles/115014893968-Terms-of-sale}{Terms
  of Sale}
\item
  \href{https://spiderbites.nytimes3xbfgragh.onion}{Site Map}
\item
  \href{https://help.nytimes3xbfgragh.onion/hc/en-us}{Help}
\item
  \href{https://www.nytimes3xbfgragh.onion/subscription?campaignId=37WXW}{Subscriptions}
\end{itemize}
