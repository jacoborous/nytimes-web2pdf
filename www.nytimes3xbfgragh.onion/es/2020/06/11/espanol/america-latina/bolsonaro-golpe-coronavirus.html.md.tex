Sections

SEARCH

\protect\hyperlink{site-content}{Skip to
content}\protect\hyperlink{site-index}{Skip to site index}

\href{https://www.nytimes3xbfgragh.onion/es/section/america-latina}{América
Latina}

\href{https://myaccount.nytimes3xbfgragh.onion/auth/login?response_type=cookie\&client_id=vi}{}

\href{https://www.nytimes3xbfgragh.onion/section/todayspaper}{Today's
Paper}

\href{/es/section/america-latina}{América Latina}\textbar{}La amenaza
militar surge en Brasil al tiempo que las muertes por el coronavirus
aumentan

\url{https://nyti.ms/2XPT11K}

\begin{itemize}
\item
\item
\item
\item
\item
\end{itemize}

\hypertarget{el-brote-de-coronavirus}{%
\subsubsection{\texorpdfstring{\href{https://www.nytimes3xbfgragh.onion/es/spotlight/coronavirus?name=styln-coronavirus-es\&region=TOP_BANNER\&block=storyline_menu_recirc\&action=click\&pgtype=Article\&impression_id=81886f60-f4bd-11ea-b05b-4ba005ac198e\&variant=undefined}{El
brote de
coronavirus}}{El brote de coronavirus}}\label{el-brote-de-coronavirus}}

\begin{itemize}
\tightlist
\item
  \href{https://www.nytimes3xbfgragh.onion/es/interactive/2020/espanol/mundo/coronavirus-en-estados-unidos.html?name=styln-coronavirus-es\&region=TOP_BANNER\&block=storyline_menu_recirc\&action=click\&pgtype=Article\&impression_id=81889670-f4bd-11ea-b05b-4ba005ac198e\&variant=undefined}{Casos
  en Estados Unidos}
\item
  \href{https://www.nytimes3xbfgragh.onion/es/interactive/2020/08/06/espanol/ciencia-y-tecnologia/tengo-covid-19-sintomas.html?name=styln-coronavirus-es\&region=TOP_BANNER\&block=storyline_menu_recirc\&action=click\&pgtype=Article\&impression_id=81889671-f4bd-11ea-b05b-4ba005ac198e\&variant=undefined}{Síntomas}
\item
  \href{https://www.nytimes3xbfgragh.onion/es/2020/09/11/espanol/ciencia-y-tecnologia/cerebro-coronavirus.html?name=styln-coronavirus-es\&region=TOP_BANNER\&block=storyline_menu_recirc\&action=click\&pgtype=Article\&impression_id=81889672-f4bd-11ea-b05b-4ba005ac198e\&variant=undefined}{El
  efecto en el cerebro}
\item
  \href{https://www.nytimes3xbfgragh.onion/es/2020/09/09/espanol/ciencia-y-tecnologia/salud-mental-coronavirus.html?name=styln-coronavirus-es\&region=TOP_BANNER\&block=storyline_menu_recirc\&action=click\&pgtype=Article\&impression_id=81889673-f4bd-11ea-b05b-4ba005ac198e\&variant=undefined}{Convalecencia
  prolongada}
\item
  \href{https://www.nytimes3xbfgragh.onion/es/2020/09/08/espanol/ciencia-y-tecnologia/dentistas-covid-dientes.html?name=styln-coronavirus-es\&region=TOP_BANNER\&block=storyline_menu_recirc\&action=click\&pgtype=Article\&impression_id=81889674-f4bd-11ea-b05b-4ba005ac198e\&variant=undefined}{La
  salud dental y el virus}
\end{itemize}

Advertisement

\protect\hyperlink{after-top}{Continue reading the main story}

Supported by

\protect\hyperlink{after-sponsor}{Continue reading the main story}

SUDAMÉRICA

\hypertarget{la-amenaza-militar-surge-en-brasil-al-tiempo-que-las-muertes-por-el-coronavirus-aumentan}{%
\section{La amenaza militar surge en Brasil al tiempo que las muertes
por el coronavirus
aumentan}\label{la-amenaza-militar-surge-en-brasil-al-tiempo-que-las-muertes-por-el-coronavirus-aumentan}}

Mientras el país se tambalea por la peor crisis que ha experimentado en
décadas, el presidente Bolsonaro y sus aliados manejan la posibilidad de
una intervención militar para proteger su control sobre el poder.

\includegraphics{https://static01.graylady3jvrrxbe.onion/images/2020/06/08/world/americas/10Brasil-ES-1/09brazil-top-articleLarge-v3.jpg?quality=75\&auto=webp\&disable=upscale}

Por \href{https://www.nytimes3xbfgragh.onion/by/simon-romero}{Simon
Romero}, Letícia Casado y Manuela Andreoni

\begin{itemize}
\item
  11 de junio de 2020
\item
  \begin{itemize}
  \item
  \item
  \item
  \item
  \item
  \end{itemize}
\end{itemize}

\href{https://www.nytimes3xbfgragh.onion/2020/06/10/world/americas/bolsonaro-coup-coronavirus-brazil.html}{Read
in English}

\href{https://www.nytimes3xbfgragh.onion/newsletters/el-times}{Regístrate
para recibir nuestro boletín} con lo mejor de The New York Times.

\begin{center}\rule{0.5\linewidth}{\linethickness}\end{center}

Las amenazas se arremolinan alrededor del presidente de Brasil: cada día
que pasa,
\href{https://www.nytimes3xbfgragh.onion/interactive/2020/world/americas/brazil-coronavirus-cases.html}{la
cantidad de muertes a causa del coronavirus} es la más alta del mundo.
Los inversionistas están huyendo del país. El mandatario, sus hijos y
sus aliados están bajo investigación. Incluso se podría anular su
elección.

La crisis se ha intensificado tanto que algunas de las figuras militares
más poderosas de Brasil están mandando advertencias de inestabilidad, lo
cual produce escalofríos pues podrían asumir el poder y desmantelar la
democracia más grande de Latinoamérica.

Sin embargo, lejos de denunciar la idea, el círculo cercano del
presidente del país, Jair Bolsonaro, parece clamar por la intervención
del ejército en la refriega. De hecho, uno de los hijos del presidente,
un congresista que ha alabado la dictadura militar que padeció Brasil,
mencionó que era inevitable un quiebre institucional similar.

``La opinión ya no gira en torno a si ocurrirá, sino cuándo sucederá'',
le dijo hace poco el hijo del presidente, Eduardo Bolsonaro, a un
prominente bloguero brasileño, al advertir que sucedería una inminente
``ruptura'' en el sistema democrático de Brasil.

\includegraphics{https://static01.graylady3jvrrxbe.onion/images/2020/06/09/world/10Brasil-ES-2/merlin_173361519_01ab39c5-7600-4c79-85ee-4b75c1b42c7b-articleLarge.jpg?quality=75\&auto=webp\&disable=upscale}

El conflicto delinea un arco ominoso para Brasil, un país que se sacudió
de encima el yugo del gobierno militar en la década de 1980 para luego
construir una democracia boyante. Dos décadas después, Brasil llegó a
representar la energía y la promesa del mundo en vías de desarrollo, con
una economía floreciente y el derecho a albergar la Copa del Mundo y las
Olimpiadas.

Desde entonces, la economía se ha tambaleado, los escándalos de
corrupción han derrumbado o entrampado a muchos de sus líderes y una
batalla con juicio político expulsó a un poderoso gobierno de izquierda.

Jair Bolsonaro, un capitán retirado de las fuerzas armadas, ingresó en
ese tumulto: celebró el pasado militar de Brasil y prometió restaurar el
orden. Sin embargo, ha padecido críticas despiadadas por restarle
importancia al virus, sabotear las medidas de aislamiento y presidir con
toda displicencia uno de los saldos de víctimas más altos del mundo, al
decir:
``\href{https://www.nytimes3xbfgragh.onion/2020/06/04/world/middleeast/coronavirus-egypt-america-africa-asia.html}{Lamentamos
todas las muertes, pero es el destino de todos}''.

A Bolsonaro, su familia y quienes lo respaldan también se les está
señalando de cargos de abuso de poder, corrupción y propagación ilegal
de desinformación. No obstante, casi la mitad de su gabinete está
conformado por personalidades del ámbito militar y, ahora, quienes lo
critican aseguran que confía en la amenaza de una intervención militar
para mantener a raya los desafíos a su presidencia.

En mayo, un general retirado del gabinete de Bolsonaro, Augusto Heleno,
el asesor de seguridad institucional, sacudió al país cuando advirtió de
las ``consecuencias impredecibles para la estabilidad nacional'',
después de que el Supremo Tribunal Federal aprobó investigar a la gente
que respalda a Bolsonaro.

Image

Augusto Heleno, uno de los generales en el gabinete de Bolsonaro,
advirtió sobre ``consecuencias impredecibles para la estabilidad
nacional'' a medida que avanzaba una investigación sobre
Bolsonaro.Credit...Evaristo Sa/Agence France-Presse --- Getty Images

Otro general, el ministro de Defensa, no tardó en apoyar la provocación,
mientras que Bolsonaro también atacó, al sugerir que la policía ignorase
las
``\href{https://www.nytimes3xbfgragh.onion/2020/05/29/world/americas/brazil-bolsonaro-supreme-court.html}{órdenes
absurdas}'' de la corte.

``El objetivo es desestabilizar el país, justo durante una pandemia'',
dijo Sérgio Moro, el exministro de Justicia, quien rompió con Bolsonaro
en abril, para referirse a las amenazas de una intervención militar.
Aunque considera improbable una acción militar, agregó: ``Es reprobable.
El país no necesita vivir con este tipo de amenazas''.

Según líderes políticos y analistas, es poco probable que ocurra una
intervención militar. Sin embargo, esa posibilidad acecha a las
instituciones democráticas de la nación, las cuales están
\href{https://www.nytimes3xbfgragh.onion/2020/05/01/world/americas/brazil-bolsonaro-coronavirus-crisis.html}{investigando
a Bolsonaro y su familia en múltiples frentes}.

Dos hijos del presidente están bajo investigación por
\href{https://www.nytimes3xbfgragh.onion/2020/05/29/world/americas/brazil-bolsonaro-supreme-court.html}{el
tipo de campañas de desinformación y difamación} que ayudaron a la
elección de su padre en 2018 y, a finales del mes pasado, la Policía
Federal allanó varias propiedades ligadas con aliados influyentes de
Bolsonaro. El Tribunal Superior Electoral, la instancia que supervisa
las elecciones, tiene la autoridad para usar pruebas de esa
investigación con el fin de anular la elección y quitar del cargo a
Bolsonaro.

Asimismo, se está investigando a dos de sus hijos por cargos de
corrupción y hace poco el Supremo Tribunal Federal autorizó investigar a
Bolsonaro, pues se le acusa de haber intentado reemplazar al jefe de la
Policía Federal
\href{https://www.nytimes3xbfgragh.onion/2020/04/24/world/americas/brazil-bolsonaro-moro.html}{para
proteger a su familia y sus amigos}.

Incluso hay una amenaza legal en torno a la forma en que el mandatario
ha manejado la pandemia: el lunes, un juez del Supremo Tribunal Federal
le ordenó al gobierno que
\href{https://www.nytimes3xbfgragh.onion/es/2020/06/08/espanol/america-latina/brasil-cifras-coronavirus.html}{dejara
de ocultar datos sobre el creciente número de muertos de Brasil}.

Las amenazas de una intervención militar han incitado una reacción
negativa generalizada, incluso de algunos altos cargos de las fuerzas
armadas. Y el general Heleno, el asesor de seguridad institucional,
señaló que no apoyaba un golpe de Estado y aseveró que se le
malinterpretó.

Image

Brasil se libró del dominio militar en la década de 1980 y construyó una
próspera democracia en las décadas que siguieron.Credit...Ueslei
Marcelino/Reuters

Sin embargo, funcionarios militares y civiles del gobierno de Bolsonaro
---así como aliados del presidente en el Congreso, iglesias evangélicas
y asociaciones militares--- aseguran que la maniobra tiene como objetivo
impedir que las instituciones judiciales y legislativas de Brasil
destituyan al presidente.

Silas Malafaia, un
\href{https://www.nytimes3xbfgragh.onion/2011/11/26/world/americas/silas-malafaia-tv-evangelist-rises-in-brazils-culture-wars.html}{televangelista}
de derecha cercano a Bolsonaro, insistió en que el presidente no le
había comentado sobre ningún plan de intervención militar. No obstante,
señaló que las fuerzas armadas tenían el derecho de evitar que los
tribunales se sobrepasen o incluso destituyan al mandatario.

``Eso no es un golpe de Estado'', aseguró Malafaia. ``Es infundir orden
donde hay desorden''.

En general, los funcionarios a favor de Bolsonaro que han lanzado esas
amenazas no se refieren a la forma recurrente en la que se han
\href{https://www.nytimes3xbfgragh.onion/2011/10/21/world/americas/an-apology-for-a-guatemalan-coup-57-years-later.html}{llevado
a cabo} los golpes de Estado en Latinoamérica, con fuerzas armadas que
derrocan a un líder civil para instalar a uno de los suyos.

Más bien, parecen estar instando a un fenómeno similar al ocurrido en
Perú en 1992, cuando Alberto Fujimori, el líder de la derecha, usó a las
fuerzas armadas para disolver el Congreso, reorganizar el poder judicial
y perseguir a sus oponentes políticos.

Bolsonaro, quien todavía tiene el apoyo de un 30 por ciento de los
brasileños, ya se refiere a sí mismo como la personificación de la
cultura militar de Brasil y retrata a las fuerzas armadas como
administradoras éticas y eficientes.

Image

El general Emílio Garrastazu Médici, a la izquierda, después de haber
sido proclamado nuevo presidente de Brasil por orden militar en
1969.Credit...Associated Press

Las fuerzas armadas de Brasil ya ejercen una influencia excepcional en
su gobierno. Diez de los veintidós ministros del gabinete son figuras
militares, entre ellos altos generales retirados. El gobierno ha
\href{https://www.poder360.com.br/analise/os-2-897-militares-no-governo-e-a-falta-de-quadros-entre-os-aliados/}{nombrado}
a casi otros 2900 miembros activos del ejército para puestos
administrativos.

La influencia de las fuerzas armadas de Brasil se exhibió cuando los
líderes del Congreso los eximieron en su mayoría de una revisión de las
pensiones en 2019, lo que permitió a los miembros del ejército evitar
los recortes más profundos de beneficios que sufrieron otras partes de
la sociedad.

La respuesta a la pandemia de Bolsonaro mostró el creciente perfil de
los militares en su gobierno, así como los riesgos para los líderes de
las fuerzas armadas cuando los brasileños comienzan a atribuirles la
culpa a medida que la situación empeora.

Al inicio de la crisis y tomando como base los logros del sector
sanitario brasileño para combatir epidemias anteriores, el Ministerio de
Salud instó a que se tomaran medidas de distanciamiento social con el
fin de detener la propagación del virus.

Incluso Bolsonaro parecía de acuerdo con la estrategia, cuando disuadió
a sus seguidores de asistir a mítines callejeros. Luego cambió su
postura de forma abrupta y se le vio chocar puños con simpatizantes
afuera de su residencia.

Image

Partidarios de Bolsonaro salieron a las calles para mostrar su apoyo
mientras enfrentaba investigaciones.Credit...Victor Moriyama para The
New York Times

Bolsonaro también le otorgó a otro general el liderazgo de la respuesta
en contra de la pandemia: Walter Souza Braga Netto, su jefe de Gabinete.

El ministro de Salud fue reemplazado después de que lo hicieron a un
lado y se rehusó a expandir el uso de la hidroxicloroquina, un fármaco
para combatir la malaria que promocionó Bolsonaro a pesar de que faltan
pruebas de su efectividad en contra del virus. Su sucesor renunció tan
solo unas semanas después, y lo reemplazó un general del ejército,
Eduardo Pazuello.

Un exfuncionario del Ministerio de Salud señaló que los cambios abruptos
habían creado una sensación de caos dentro de la institución, que dieron
como resultado semanas de disfunción y parálisis en el momento más
crítico: cuando el país debía combatir la propagación descontrolada del
virus.

Por su parte, Luiz Henrique Mandetta, el ministro de Salud al inicio de
la pandemia, dijo que Bolsonaro apreciaba la estabilidad económica por
sobre las prioridades de salud, prefiriendo una figura militar al mando
del ministerio.

``Él necesitaba a alguien como un general o un coronel que viera al
ministerio como un peldaño, una forma de obtener un ascenso por
valentía'', dijo Mandetta.

Ahora Brasil tiene
\href{https://www.nytimes3xbfgragh.onion/interactive/2020/world/coronavirus-maps.html}{más
de 700.000 casos} confirmados de coronavirus, el
\href{https://www.nytimes3xbfgragh.onion/interactive/2020/world/coronavirus-maps.html}{segundo}
lugar tan solo detrás de Estados Unidos. Hasta el martes, al menos
37.000 personas habían muerto a causa del virus, con un conteo de
defunciones que a menudo llega a más de mil al día.

La agitación en Brasil está provocando que los inversionistas busquen a
toda prisa la salida. La fuga de capitales está
\href{https://economia.uol.com.br/colunas/jose-paulo-kupfer/2020/05/27/fuga-de-capitais-se-acentua-e-alerta-para-falta-de-confianca-no-brasil.htm}{alcanzando}
niveles que no se habían visto desde la década de 1990. El Banco Mundial
espera que este año la economía se contraiga un 8 por ciento. La
producción automovilística, que fue un próspero pilar de la economía, se
ha
\href{https://www.terra.com.br/parceiros/guia-do-carro/industria-automobilistica-tem-o-pior-resultado-desde-1957,2607f811c62d5d3abc66af24c1cb08712gk0gjq1.html}{desplomado}
a su nivel más bajo desde la década de 1950.

Image

Las manifestaciones contra Bolsonaro se han extendido a pesar de la
cuarentena.Credit...Victor Moriyama para The New York Times

Carlos Fico, historiador de la Universidad Federal de Río de Janeiro que
estudia al ejército brasileño, dijo que el poder creciente de las
fuerzas armadas corría el riesgo de revelar su incompetencia en áreas
cruciales.

``Creen que las cosas ocurren haciendo declaraciones grandilocuentes,
como ocurre en el terreno militar, donde se da una orden y los de rango
inferior la obedecen'', dijo Fico.

Sin embargo, Fico agregó que como el ejército ahora lidera la respuesta
frente a la pandemia, ``corre el riesgo de que la sociedad lo culpe de
lo que suceda después''.

Los principales aliados de Bolsonaro insisten en que las fuerzas armadas
no tienen planes de un golpe. ``Ninguno de los generales de cuatro
estrellas está a favor de una intervención militar'', dijo Sostenes
Cavalcante, un diputado de derecha.

Pero Cavalcante también argumentó que algo se debe hacer para frenar el
poder de la Corte Suprema. Sostuvo que la charla de un golpe de Estado
por parte del hijo de Bolsonaro era simplemente una forma de presionar
al poder judicial.

``Se podría interpretar eso ya que la Corte Suprema ha sobrepasado su
autoridad'', dijo Cavalcante.

Al mismo tiempo, algunos funcionarios del gobierno de Bolsonaro están
examinando activamente los escenarios en los cuales los militares
podrían intervenir. Un oficial militar en el gobierno que no fue
autorizado a dar declaraciones dijo que una intervención permanecía
fuera del radar por ahora, pero que ciertos movimientos del poder
judicial, como ordenar una búsqueda en el palacio presidencial de
Bolsonaro como parte de una investigación, podrían cambiar eso.

Image

Oficiales brasileños en una ceremonia militar para conmemorar el
aniversario de la dictadura que comenzó el 31 de marzo de
1964Credit...Ueslei Marcelino/Reuters

Del mismo modo, agregó el oficial, cualquier posible anulación de la
elección de 2018 por un juez también se consideraría inaceptable, porque
eliminaría no solo a Bolsonaro, sino también a su compañero de fórmula y
vicepresidente, Hamilton Mourão, un general retirado.

Mourão ha afirmado repetidamente que no se está considerando ningún tipo
de golpe militar. Pero incluso el debate sobre la intervención militar
está generando preocupación sobre la resiliencia de las instituciones
democráticas de Brasil y el regreso de la inestabilidad política
crónica, con una constante intromisión militar.

Fernando Henrique Cardoso, un expresidente civil que fue exiliado
durante la dictadura militar, dijo que no creía que un golpe fuera
inminente. Pero que le preocupaba que las tácticas de intimidación de
Bolsonaro pudieran intensificarse.

``¿Cómo mueren las democracias? No necesitas un golpe militar'',
\href{https://brasil.elpais.com/brasil/2020-05-31/fhc-quem-vai-ser-responsabilizado-pelos-erros-do-governo-queiram-ou-nao-serao-os-militares.html}{dijo}
Cardoso, de 88 años, quien ya instó a Bolsonaro a renunciar. ``El propio
presidente puede buscar poderes extraordinarios, y puede tomarlos''.

Simon Romero es un corresponsal nacional con base en Albuquerque, que
cubre inmigración y otros asuntos. Antes fue el jefe del buró en Brasil
y en Caracas, Venezuela, y reportó sobre la industria de la energía
global desde Houston.
\href{https://twitter.com/viaSimonRomero}{@viaSimonRomero}

Advertisement

\protect\hyperlink{after-bottom}{Continue reading the main story}

\hypertarget{site-index}{%
\subsection{Site Index}\label{site-index}}

\hypertarget{site-information-navigation}{%
\subsection{Site Information
Navigation}\label{site-information-navigation}}

\begin{itemize}
\tightlist
\item
  \href{https://help.nytimes3xbfgragh.onion/hc/en-us/articles/115014792127-Copyright-notice}{©~2020~The
  New York Times Company}
\end{itemize}

\begin{itemize}
\tightlist
\item
  \href{https://www.nytco.com/}{NYTCo}
\item
  \href{https://help.nytimes3xbfgragh.onion/hc/en-us/articles/115015385887-Contact-Us}{Contact
  Us}
\item
  \href{https://www.nytco.com/careers/}{Work with us}
\item
  \href{https://nytmediakit.com/}{Advertise}
\item
  \href{http://www.tbrandstudio.com/}{T Brand Studio}
\item
  \href{https://www.nytimes3xbfgragh.onion/privacy/cookie-policy\#how-do-i-manage-trackers}{Your
  Ad Choices}
\item
  \href{https://www.nytimes3xbfgragh.onion/privacy}{Privacy}
\item
  \href{https://help.nytimes3xbfgragh.onion/hc/en-us/articles/115014893428-Terms-of-service}{Terms
  of Service}
\item
  \href{https://help.nytimes3xbfgragh.onion/hc/en-us/articles/115014893968-Terms-of-sale}{Terms
  of Sale}
\item
  \href{https://spiderbites.nytimes3xbfgragh.onion}{Site Map}
\item
  \href{https://help.nytimes3xbfgragh.onion/hc/en-us}{Help}
\item
  \href{https://www.nytimes3xbfgragh.onion/subscription?campaignId=37WXW}{Subscriptions}
\end{itemize}
