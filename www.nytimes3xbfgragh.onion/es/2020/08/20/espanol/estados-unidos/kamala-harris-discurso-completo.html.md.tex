Sections

SEARCH

\protect\hyperlink{site-content}{Skip to
content}\protect\hyperlink{site-index}{Skip to site index}

\href{https://www.nytimes3xbfgragh.onion/es/section/estados-unidos}{Estados
Unidos}

\href{https://myaccount.nytimes3xbfgragh.onion/auth/login?response_type=cookie\&client_id=vi}{}

\href{https://www.nytimes3xbfgragh.onion/section/todayspaper}{Today's
Paper}

\href{/es/section/estados-unidos}{Estados Unidos}\textbar{}Kamala Harris
habla de su madre, la inclusión y el liderazgo en su discurso de
aceptación

\url{https://nyti.ms/3ggx4z5}

\begin{itemize}
\item
\item
\item
\item
\item
\item
\end{itemize}

\begin{itemize}
\item
  \href{https://www.nytimes3xbfgragh.onion/es/2020/09/07/espanol/estados-unidos/trump-biden-encuestas-elecciones.html?action=click\&pgtype=Article\&state=default\&region=TOP_BANNER\&context=storylines_menu}{Lo
  más reciente}
\item
  \href{https://www.nytimes3xbfgragh.onion/es/2020/08/31/espanol/estados-unidos/donald-trump.html?action=click\&pgtype=Article\&state=default\&region=TOP_BANNER\&context=storylines_menu}{Entrevista
  con Donald Trump}
\item
  \href{https://www.nytimes3xbfgragh.onion/es/2020/08/21/espanol/estados-unidos/plataforma-democratas-espanol.html?action=click\&pgtype=Article\&state=default\&region=TOP_BANNER\&context=storylines_menu}{La
  plataforma demócrata}
\item
  \href{https://www.nytimes3xbfgragh.onion/es/article/voto-por-correo.html?action=click\&pgtype=Article\&state=default\&region=TOP_BANNER\&context=storylines_menu}{Voto
  por correo}
\item
  \href{https://www.nytimes3xbfgragh.onion/es/2020/09/11/espanol/doug-emhoff-esposo-kamala-harris.html?action=click\&pgtype=Article\&state=default\&region=TOP_BANNER\&context=storylines_menu}{¿Quién
  es Doug Emhoff?}
\end{itemize}

Advertisement

\protect\hyperlink{after-top}{Continue reading the main story}

Supported by

\protect\hyperlink{after-sponsor}{Continue reading the main story}

Elecciones 2020

\hypertarget{kamala-harris-habla-de-su-madre-la-inclusiuxf3n-y-el-liderazgo-en-su-discurso-de-aceptaciuxf3n}{%
\section{Kamala Harris habla de su madre, la inclusión y el liderazgo en
su discurso de
aceptación}\label{kamala-harris-habla-de-su-madre-la-inclusiuxf3n-y-el-liderazgo-en-su-discurso-de-aceptaciuxf3n}}

Al aceptar la nominación de su partido a la vicepresidencia en la
tercera noche de la Convención Nacional Demócrata, Harris dijo que ``no
hay vacuna contra el racismo'' y ofreció su visión de un país más
incluyente.

\includegraphics{https://static01.graylady3jvrrxbe.onion/images/2020/08/19/us/politics/19breakouts-harris-speech/19breakouts-harris-speech-videoSixteenByNine3000.jpg}

\href{https://www.nytimes3xbfgragh.onion/by/matt-stevens}{\includegraphics{https://static01.graylady3jvrrxbe.onion/images/2019/04/03/multimedia/author-matt-stevens/author-matt-stevens-thumbLarge.png}}

Por \href{https://www.nytimes3xbfgragh.onion/by/matt-stevens}{Matt
Stevens}

\begin{itemize}
\item
  20 de agosto de 2020
\item
  \begin{itemize}
  \item
  \item
  \item
  \item
  \item
  \item
  \end{itemize}
\end{itemize}

\href{https://www.nytimes3xbfgragh.onion/2020/08/19/us/politics/kamala-harris-dnc-speech.html}{Read
in English}

\href{https://www.nytimes3xbfgragh.onion/newsletters/el-times}{Regístrate
para recibir nuestro boletín} con lo mejor de The New York Times.

\begin{center}\rule{0.5\linewidth}{\linethickness}\end{center}

La senadora
\href{https://www.nytimes3xbfgragh.onion/interactive/2020/us/elections/kamala-harris.html}{Kamala
Harris} de California,
\href{https://www.nytimes3xbfgragh.onion/es/2020/08/11/espanol/estados-unidos/kamala-harris-joe-biden-vicepresidenta.html}{la
primera mujer de color en ser nominada para vicepresidenta por uno de
los principales partidos}, se volvió a presentar ante Estados Unidos en
un discurso de aceptación en la Convención Nacional Demócrata el
miércoles por la noche. Durante el mensaje relató la historia de su
crianza como una joven biracial, hija de inmigrantes y esbozó su visión
de un país más incluyente.

En un discurso de unos 17 minutos, Harris, cuya madre era de la India y
cuyo padre es de Jamaica, condujo a los votantes por los primeros días
de su vida en la zona de la bahía de California. Habló del viaje de su
madre a Estados Unidos, la relación de sus padres y su posterior
separación, y dijo haber sido criada gran parte de su vida por una madre
soltera que le enseñó a poner a ``la familia primero''.

``Oh, cómo desearía que estuviera aquí esta noche'', dijo Harris de su
madre. ``Pero sé que me está mirando desde arriba. Sigo pensando en esa
mujer india de 25 años ---con su metro cincuenta de estatura--- quien me
dio a luz en el hospital Kaiser en Oakland, California. Ese día,
probablemente no imaginó que yo estaría ahora de pie ante ustedes
pronunciando estas palabras: acepto su nominación para vicepresidenta de
Estados Unidos de América''.

Harris, que fue fiscala durante mucho tiempo, recordó sus victorias en
los tribunales al luchar a favor de las sobrevivientes de agresiones
sexuales y en contra de organizaciones criminales y grandes bancos. El
recuento era parte de un argumento más amplio en favor de un Estados
Unidos que otorgue a todos sus ciudadanos igualdad de oportunidades,
igualdad de trato y justicia ante la ley.

``No hay vacuna contra el racismo'', dijo. ``Tenemos que hacer el
trabajo''.

Harris dijo que estaba comprometida con los valores que su madre le
había enseñado y con una ``visión de nuestra nación como una comunidad
de amor, donde todos son bienvenidos, sin importar cómo nos vemos, de
dónde venimos o a quién amamos''.

Dijo que imaginaba Estados Unidos como ``un país en el que puede que no
estemos de acuerdo en todos los detalles, pero en el que nos une la
creencia fundamental de que cada ser humano tiene un valor infinito'' y
también un lugar ``en el que nos cuidamos unos a otros'' y ``nos
levantamos y caemos como uno solo''.

Harris también dedicó mucho tiempo a defender a su compañero de fórmula,
\href{https://www.nytimes3xbfgragh.onion/es/interactive/2020/espanol/estados-unidos/joe-biden-elecciones.html}{Joe
Biden}. Argumentó que Biden era un padre cariñoso y presente para sus
hijos y un líder talentoso que había luchado por las mujeres y la
atención a la salud y contra los grupos de presión por las armas. Y
trató repetidamente de contrastar el carácter de Biden con el del
presidente Donald Trump, de quien dijo ``convierte nuestras tragedias en
armas políticas''.

``Estamos en un punto de inflexión'', dijo. ``El caos constante nos deja
a la deriva. La incompetencia nos hace sentir miedo. La insensibilidad
nos hace sentir solos. Es mucho''.

``Debemos elegir un presidente que traiga algo diferente, algo mejor y
que haga el trabajo importante'', añadió. ``Un presidente que nos reúna
a todos ---negros, blancos, latinos, asiáticos, indígenas--- para lograr
el futuro que queremos colectivamente. Debemos elegir a Joe Biden''.

\emph{Aquí está la traducción al español de la transcripción íntegra de
las palabras de Kamala Harris tal como fueron preparadas para su
discurso.}

\textbf{Kamala Harris:} Saludos, Estados Unidos.

Es un verdadero honor hablar con ustedes.

Que esté aquí esta noche es testimonio de la dedicación de generaciones
antes que yo. Mujeres y hombres que creían firmemente en la promesa de
igualdad, libertad y justicia para todos.

Esta semana marca el centenario de la aprobación de la decimonovena
enmienda. Y celebramos a las mujeres que lucharon por ese derecho.

Sin embargo, a muchas de las mujeres negras que ayudaron a asegurar esa
victoria se les prohibió votar mucho después de su ratificación.

Pero no se dejaron intimidar.

Sin fanfarrias ni reconocimiento, se organizaron, testificaron, se
manifestaron, marcharon y lucharon, no solo por su voto, sino por un
lugar en la mesa. Esas mujeres y las generaciones que siguieron
trabajando para hacer que la democracia y las oportunidades fueran
reales en la vida de todos los que vinimos después.

Allanaron el camino para el liderazgo pionero de Barack Obama y Hillary
Clinton.

Y esas mujeres nos inspiraron a tomar la posta y seguir luchando.

Mujeres como Mary Church Terrell y Mary McCleod Bethune. Fannie Lou
Hamer y Diane Nash. Constance Baker Motley y Shirley Chisholm.

A menudo no nos enseñan sus historias. Pero como estadounidenses, todos
nos levantamos sobre sus hombros.

Hay otra mujer, cuyo nombre no se conoce, cuya historia no se comparte.
Otra mujer sobre cuyos hombros me levanto. Y esa es mi madre, Shyamala
Gopalan Harris.

Vino aquí desde India a los 19 años para perseguir su sueño de curar el
cáncer. En la Universidad de California Berkeley conoció a mi padre,
Donald Harris, quien había venido de Jamaica para estudiar economía.

Se enamoraron de la manera más estadounidense, cuando marchaban juntos
por la justicia en el movimiento de derechos civiles en la década de
1960.

En las calles de Oakland y Berkeley, vi desde mi cochecito de bebé cómo
la gente se metía en lo que el gran John Lews llamaba ``buenos
problemas''.

Cuando tenía cinco años, mis padres se separaron y mi madre nos crió
principalmente por su cuenta. Como tantas otras madres, trabajaba las 24
horas del día para que funcionara, empacaba almuerzos antes de que
despertáramos y pagaba las facturas después de acostarnos. Nos ayudaba
con la tarea en la mesa de la cocina y nos llevaba a la iglesia para
practicar con el coro.

Ella hizo que pareciera fácil, aunque sé que nunca lo fue.

Mi madre nos inculcó a mi hermana Maya y a mí los valores que marcarían
el curso de nuestras vidas.

Nos crió para ser mujeres negras fuertes y orgullosas. Y nos crió para
conocer y estar orgullosas de nuestra herencia india.

Nos enseñó a poner a la familia primero: la familia en la que naciste y
la familia que eliges.

Familia es mi esposo Doug, a quien conocí en una cita a ciegas
organizada por mi mejor amiga. Familia son nuestros hermosos hijos, Cole
y Ella, quienes, como acaban de escuchar, me llaman \emph{Momala}.
Familia es mi hermana. Familia es mi mejor amiga, mis sobrinas y mis
ahijados. Familia son mis tíos, mis tías y mis \emph{chithis}. Familia
es la señora Shelton, mi segunda madre, quien vivía a dos puertas y
ayudó a criarme. Familia es mi amada Alpha Kappa Alpha\ldots{} nuestros
Nueve Divinos\ldots{} y mis hermanos y hermanas de las facultades y
universidades históricamente negras. Familia son los amigos a los que
recurrí cuando mi madre ---la persona más importante en mi vida---
falleció de cáncer.

E incluso cuando nos enseñó a mantener a nuestra familia en el centro de
nuestro mundo, también nos empujó a ver el mundo más allá de nosotras
mismas.

Nos enseñó a ser conscientes y compasivas con las dificultades de todas
las personas. A creer que el servicio público es una causa justa y que
la lucha por la justicia es una responsabilidad compartida.

Eso me llevó a convertirme en abogada, en fiscala de distrito, en
fiscala general y en senadora de Estados Unidos.

Y en cada paso del camino, me han guiado las palabras que pronuncié
desde la primera vez que me paré en un tribunal: Kamala Harris, en
representación del pueblo.

He luchado por los niños y las supervivientes de agresión sexual. He
luchado contra bandas transnacionales. Me enfrenté a los bancos más
grandes y ayudé a derribar una de las mayores universidades con fines de
lucro.

Reconozco a un depredador cuando lo veo.

Mi madre me enseñó que el servicio a los demás da un propósito y un
significado a la vida. Y, oh, cómo desearía que estuviera aquí esta
noche, pero sé que me está mirando desde arriba. Sigo pensando en esa
mujer india de 25 años ---con su metro cincuenta de estatura--- quien me
dio a luz en el hospital Kaiser en Oakland, California.

Ese día, probablemente no imaginó que yo estaría ahora de pie ante
ustedes pronunciando estas palabras: acepto su nominación para
vicepresidenta de Estados Unidos de América.

Acepto, comprometida con los valores que ella me enseñó. Con la palabra
que me enseña a guiarme por lo que creo y no por lo que veo. Y con una
visión transmitida a través de generaciones de estadounidenses, una que
Joe Biden comparte. Una visión de nuestra nación como una comunidad de
amor, donde todos son bienvenidos, sin importar cómo nos vemos, de dónde
venimos o a quién amamos.

Un país en el que puede que no estemos de acuerdo en todos los detalles,
pero en el que nos une la creencia fundamental de que cada ser humano
tiene un valor infinito y es merecedor de compasión, dignidad y respeto.

Un país en el que nos cuidamos unos a otros, en el que nos levantamos y
caemos como uno solo, en el que afrontamos nuestros retos y celebramos
nuestros triunfos, juntos.

Hoy\ldots{} ese país se siente distante.

El fracaso de Donald Trump en el liderazgo ha costado vidas y empleos.

Si eres un padre que tiene dificultades con el aprendizaje a distancia
de tu hijo, o eres un profesor que lucha al otro lado de esa pantalla,
sabes que lo que estamos haciendo ahora mismo no está funcionando.

Y somos una nación que está de duelo. Llorando la pérdida de vidas, la
pérdida de trabajos, la pérdida de oportunidades, la pérdida de
normalidad y, sí, la pérdida de certezas.

Y aunque este virus nos afecta a todos, seamos honestos: no es un
agresor que haga daño de forma equitativa. Negros, latinos e indígenas
están sufriendo y muriendo de forma desproporcionada.

Esto no es una coincidencia. Es efecto del racismo estructural.

De las desigualdades en la educación y la tecnología, la atención de la
salud y la vivienda, la seguridad laboral y el transporte.

De la injusticia en la atención de la salud reproductiva y materna. Del
uso excesivo de la fuerza por parte de la policía. Y de nuestro sistema
de justicia penal más amplio.

Este virus no tiene ojos, pero sabe exáctamente cómo nos vemos y cómo
nos tratamos unos a otros.

Y seamos claros: no hay vacuna contra el racismo. Tenemos que hacer el
trabajo.

Por George Floyd. Por Breonna Taylor. Por las vidas de muchos otros
nombres que quedan por pronunciar. Por nuestros hijos. Por todos
nosotros.

Tenemos que hacer el trabajo para cumplir esa promesa de justicia
igualitaria bajo la ley. Porque ninguno de nosotros es libre\ldots{}
hasta que todos seamos libres\ldots{}

Estamos en un punto de inflexión.

El caos constante nos deja a la deriva. La incompetencia nos hace sentir
miedo. La insensibilidad nos hace sentir solos.

Es mucho.

Y esta es la cuestión: podemos hacerlo mejor y merecemos mucho más.

Debemos elegir un presidente que traiga algo diferente, algo mejor y que
haga el trabajo importante. Un presidente que nos reúna a todos
---negros, blancos, latinos, asiáticos, indígenas--- para lograr el
futuro que queremos colectivamente.

Debemos elegir a Joe Biden.

Conocí a Joe como vicepresidente. Conocí a Joe en la campaña electoral.
Pero primero conocí a Joe como el padre de mi amigo.

El hijo de Joe, Beau, y yo fuimos fiscales generales de nuestros
estados, Delaware y California. Durante la Gran Recesión, hablábamos por
teléfono casi a diario, trabajando juntos por recuperar para los
propietarios de las casas los miles de millones de dólares de los
grandes bancos que embargaban los hogares de la gente.

Y Beau y yo hablábamos de su familia.

De cómo, como padre soltero, Joe pasaba cuatro horas diarias en el tren
de Wilmington a Washington. Beau y Hunter desayunaban todas las mañanas
con su papá. Se iban a dormir todas las noches con el sonido de la voz
de él leyendo cuentos para dormir. Y mientras soportaban una pérdida
indescriptible, estos dos niños pequeños siempre supieron que eran
profunda e incondicionalmente amados.

Y lo que también me conmovió de Joe es el trabajo que hizo, mientras iba
y venía. Este es el líder que escribió la Ley de Violencia contra las
Mujeres, y promulgó la Prohibición de las Armas de Asalto; quien, como
vicepresidente, implementó la Ley de Recuperación, que ayudó a nuestro
país a salir de la Gran Recesión. Defendió la Ley de Cuidado de Salud a
Bajo Precio, protegiendo a millones de estadounidenses con condiciones
preexistentes; pasó décadas promoviendo los valores estadounidenses en
todo el mundo, defendiendo a nuestros aliados y enfrentándose a nuestros
adversarios.

Ahora mismo, tenemos un presidente que convierte nuestras tragedias en
armas políticas.

Joe será un presidente que convierte nuestros desafíos en propósitos.

Joe nos unirá para construir una economía que no deje a nadie atrás.
Donde un trabajo bien pagado es el suelo, no el techo.

Joe nos unirá para terminar con esta pandemia y asegurarse de que
estamos bien preparados para la próxima.

Joe nos unirá para enfrentar y desmantelar la injusticia racial,
fomentando el trabajo de generaciones.

Joe y yo creemos que podemos construir esa comunidad de amor, una que
sea fuerte y decente, justa y amable. Una en la que todos podamos
reconocernos a nosotros mismos.

Esa es la visión por la que lucharon nuestros padres y abuelos. La
visión que hizo posible mi propia vida. La visión que hace que la
promesa estadounidense ---con todas sus complejidades e
imperfecciones--- sea una promesa por la que vale la pena luchar.

No se equivoquen, el camino a seguir no será fácil. Vamos a tropezar.
Puede que nos quedemos cortos. Pero les prometo que actuaremos con
valentía y afrontaremos nuestros retos con honestidad. Diremos la
verdad. Y actuaremos con la misma fe en ustedes que la que pedimos que
pongan en nosotros.

Creemos que nuestro país ---todos nosotros--- se mantendrá unido por un
futuro mejor. Ya lo estamos haciendo.

Lo vemos en los médicos, las enfermeras, los trabajadores de la salud a
domicilio y los trabajadores de primera línea que arriesgan sus vidas
para salvar a gente que nunca habían conocido.

Lo vemos en los maestros y los camioneros, los trabajadores de las
fábricas y los agricultores, los trabajadores postales y los
trabajadores electorales, todos arriesgando su propia seguridad para
ayudarnos a superar esta pandemia.

Y lo vemos en muchos de ustedes que están trabajando, no solo para
ayudarnos a superar nuestras crisis actuales, sino también para llegar a
un lugar mejor.

Algo está sucediendo en todo el país.

No se trata de Joe o de mí.

Se trata de ustedes.

Se trata de nosotros. Gente de todas las edades, colores y credos que se
están, sí, tomando las calles, y también persuadiendo a nuestros
familiares, reuniendo a nuestros amigos, organizando nuestros
vecindarios y acudiendo a votar.

Y hemos demostrado que, cuando votamos, ampliamos el acceso a la
atención médica, ampliamos el acceso a las urnas y aseguramos que más
familias trabajadoras puedan vivir dignamente.

Estoy tan inspirada por una nueva generación de liderazgo. Nos están
empujando a comprender los ideales de nuestra nación, empujándonos a
vivir los valores que compartimos: decencia y equidad, justicia y amor.

Ustedes son los patriotas que nos recuerdan que amar a nuestro país es
luchar por los ideales de nuestro país.

En estas elecciones tenemos la oportunidad de cambiar el curso de la
historia. Todos estamos en esta lucha. Tú, yo y Joe\ldots{} juntos.

Qué gran responsabilidad. Qué increíble privilegio.

Así que, luchemos con convicción. Luchemos con esperanza. Luchemos con
confianza en nosotros mismos, y con un compromiso con los demás, con el
Estados Unidos que sabemos que es posible, el Estados Unidos que amamos.

Dentro de unos años, este momento habrá pasado. Y nuestros hijos y
nuestros nietos nos mirarán a los ojos y nos preguntarán: ``¿Dónde
estabas cuando había tanto en juego?''.

Nos preguntarán: ``¿Cómo fue?''.

Y les contaremos. Les diremos no solo cómo nos sentimos.

Les diremos lo que hicimos.

Gracias. Que Dios los bendiga. Y que Dios bendiga a Estados Unidos de
América.

Isabella Grullón Paz colaboró con la reportería.

Matt Stevens es un reportero político radicado en Nueva York.
Anteriormente cubrió las noticias de última hora en el buró Express del
Times. \href{https://twitter.com/ByMattStevens}{@ByMattStevens}

\begin{center}\rule{0.5\linewidth}{\linethickness}\end{center}

Advertisement

\protect\hyperlink{after-bottom}{Continue reading the main story}

\hypertarget{site-index}{%
\subsection{Site Index}\label{site-index}}

\hypertarget{site-information-navigation}{%
\subsection{Site Information
Navigation}\label{site-information-navigation}}

\begin{itemize}
\tightlist
\item
  \href{https://help.nytimes3xbfgragh.onion/hc/en-us/articles/115014792127-Copyright-notice}{©~2020~The
  New York Times Company}
\end{itemize}

\begin{itemize}
\tightlist
\item
  \href{https://www.nytco.com/}{NYTCo}
\item
  \href{https://help.nytimes3xbfgragh.onion/hc/en-us/articles/115015385887-Contact-Us}{Contact
  Us}
\item
  \href{https://www.nytco.com/careers/}{Work with us}
\item
  \href{https://nytmediakit.com/}{Advertise}
\item
  \href{http://www.tbrandstudio.com/}{T Brand Studio}
\item
  \href{https://www.nytimes3xbfgragh.onion/privacy/cookie-policy\#how-do-i-manage-trackers}{Your
  Ad Choices}
\item
  \href{https://www.nytimes3xbfgragh.onion/privacy}{Privacy}
\item
  \href{https://help.nytimes3xbfgragh.onion/hc/en-us/articles/115014893428-Terms-of-service}{Terms
  of Service}
\item
  \href{https://help.nytimes3xbfgragh.onion/hc/en-us/articles/115014893968-Terms-of-sale}{Terms
  of Sale}
\item
  \href{https://spiderbites.nytimes3xbfgragh.onion}{Site Map}
\item
  \href{https://help.nytimes3xbfgragh.onion/hc/en-us}{Help}
\item
  \href{https://www.nytimes3xbfgragh.onion/subscription?campaignId=37WXW}{Subscriptions}
\end{itemize}
