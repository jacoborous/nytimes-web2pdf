Sections

SEARCH

\protect\hyperlink{site-content}{Skip to
content}\protect\hyperlink{site-index}{Skip to site index}

\href{https://www.nytimes3xbfgragh.onion/es/section/estados-unidos}{Estados
Unidos}

\href{https://myaccount.nytimes3xbfgragh.onion/auth/login?response_type=cookie\&client_id=vi}{}

\href{https://www.nytimes3xbfgragh.onion/section/todayspaper}{Today's
Paper}

\href{/es/section/estados-unidos}{Estados Unidos}\textbar{}Tiroteos en
Kenosha: en búsqueda del sospechoso

\url{https://nyti.ms/3b6y6g6}

\begin{itemize}
\item
\item
\item
\item
\item
\item
\end{itemize}

Advertisement

\protect\hyperlink{after-top}{Continue reading the main story}

Supported by

\protect\hyperlink{after-sponsor}{Continue reading the main story}

\hypertarget{tiroteos-en-kenosha-en-buxfasqueda-del-sospechoso}{%
\section{Tiroteos en Kenosha: en búsqueda del
sospechoso}\label{tiroteos-en-kenosha-en-buxfasqueda-del-sospechoso}}

Los videos parecen mostrar a un adolescente que dispara a tres personas
durante las protestas en Wisconsin. Rastreamos sus movimientos de esa
noche.

\includegraphics{https://static01.graylady3jvrrxbe.onion/images/2020/08/26/autossell/27vid-vi-kenosha-ES-1/vis-4-articleLarge.jpg?quality=75\&auto=webp\&disable=upscale}

Por \href{https://www.nytimes3xbfgragh.onion/by/haley-willis}{Haley
Willis}, \href{https://www.nytimes3xbfgragh.onion/by/muyi-xiao}{Muyi
Xiao},
\href{http://nytimes3xbfgragh.onion/by/christiaan-triebert}{Christiaan
Triebert},
\href{https://www.nytimes3xbfgragh.onion/by/christoph-koettl}{Christoph
Koettl}, Stella Cooper, David Botti,
\href{https://www.nytimes3xbfgragh.onion/by/john-ismay}{John Ismay} y
\href{https://www.nytimes3xbfgragh.onion/by/ainara-tiefenthaler}{Ainara
Tiefenthäler}

\begin{itemize}
\item
  27 de agosto de 2020
\item
  \begin{itemize}
  \item
  \item
  \item
  \item
  \item
  \item
  \end{itemize}
\end{itemize}

\href{https://www.nytimes3xbfgragh.onion/2020/08/27/us/kyle-rittenhouse-kenosha-shooting-video.html}{Read
in English}

\href{https://www.nytimes3xbfgragh.onion/newsletters/el-times}{Regístrate
para recibir nuestro boletín} con lo mejor de The New York Times.

\begin{center}\rule{0.5\linewidth}{\linethickness}\end{center}

Un adolescente que caminaba entre los manifestantes en Kenosha,
Wisconsin, con un rifle semiautomático estilo militar fue
\href{https://www.nytimes3xbfgragh.onion/2020/08/26/us/kenosha-shooting-protests-jacob-blake.html?action=click\&module=Top\%20Stories\&pgtype=Homepage}{arrestado
y enfrenta cargos} de homicidio intencional en primer grado en relación
con los tiroteos que dejaron dos personas muertas la noche del martes.

Kyle Rittenhouse, de 17 años, residente del estado de Illinois, aparece
en varios videos filmados por testigos y manifestantes que registraban
los eventos durante la noche en que las protestas pacíficas se
convirtieron en un caos en el que manifestantes, civiles armados y otros
se enfrentaron entre sí y con la policía en las calles oscuras.

La unidad de Investigaciones Visuales de The New York Times analizó
horas de registros en video para rastrear los movimientos de Rittenhouse
antes, durante y después de los tiroteos.

\hypertarget{quiuxe9n-es-kyle-rittenhouse}{%
\subsection{¿Quién es Kyle
Rittenhouse?}\label{quiuxe9n-es-kyle-rittenhouse}}

Rittenhouse fue arrestado temprano el miércoles en su ciudad de Antioch,
Illinois, que está a unos 30 minutos al suroeste de las
\href{https://www.nytimes3xbfgragh.onion/2020/08/27/us/kenosha-shooting-protests.html}{protestas
en Kenosha}, justo cruzando el límite estatal con Wisconsin.

Diversas publicaciones en sus redes sociales proclaman su apoyo por
causas a favor de la policía como el movimiento Blue Lives Matter (Las
vidas azules importan) y Humanize the Badge (Humanicemos las placas),
una organización sin fines de lucro a la que apoyó con una campaña de
recaudación de fondos en Facebook con motivo de su cumpleaños número 16.

Sus publicaciones en línea también sugieren predilección por las armas
de fuego: hay videos que lo muestran en práctica de tiro, posando con
pistolas y armando un fusil.

Pero aún se están empezando a conocer muchos detalles tanto de sus
antecedentes como de sus motivaciones para asistir a las protestas de
Kenosha con un rifle estilo militar semiautomático.

\hypertarget{antes-de-los-tiroteos}{%
\subsection{Antes de los tiroteos}\label{antes-de-los-tiroteos}}

Unas dos horas antes del primer tiroteo, el productor de una
\href{https://www.facebookcorewwwi.onion/kristantharris/videos/10164052138640646/?t=2}{transmisión
en vivo} entrevista a Rittenhouse en una concesionaria automotriz en
Kenosha.

Rittenhouse está ahí al mismo tiempo que otros hombres armados. Algunos
de ellos están ubicados sobre el techo del edificio mirando hacia el
estacionamiento donde un día antes se habían incendiado algunos
vehículos.

En un breve intercambio durante la transmisión en vivo se identifica
como ``Kyle''.

\includegraphics{https://static01.graylady3jvrrxbe.onion/images/2020/08/26/autossell/vis-2/vis-2-videoSixteenByNineJumbo1600.jpg}

En
\href{https://twitter.com/RichieMcGinniss/status/1298657958205820928?s=20}{otra
entrevista}, Rittenhouse habla con Richie McGinniss, un editor de video
de Daily Caller, un sitio conservador de noticias y opiniones.

Rittenhouse dice estar ahí para proteger el negocio. Dice que es su
trabajo, aunque no hay indicios de que se le haya solicitado cuidar el
local.

Más tarde le dice a otro videoreportero que alguien en una multitud
cercana lo ha rociado de gas pimienta cuando cuidaba la propiedad.

En la mayor parte de los videos previos a los tiroteos revisados por el
Times Rittenhouse se encuentra en esta zona. También ofrece asistencia
médica a los manifestantes.

Unos 15 minutos antes del primer tiroteo, agentes de policía pasan en
auto frente a Rittenhouse y los otros civiles que aseguran estar
protegiendo la concesionaria automotriz y les ofrecen agua en
agradecimiento.

Rittenhouse se acerca a pie a una patrulla policial con su rifle y
conversa con los policías.

Al final abandona el lugar y la policía le impide volver. Seis minutos
más tarde, se ve en video que Rittenhouse escapa de un grupo de personas
desconocidas y se refugia en el estacionamiento de otro negocio
automotriz a unas cuadras de ahí.

\hypertarget{primer-tiroteo}{%
\subsection{Primer tiroteo}\label{primer-tiroteo}}

Cuando Rittenhouse es perseguido por el grupo, un tirador desconocido
dispara al aire, aunque no se sabe por qué. El destello del cañón del
arma aparece en los videos que se grabaron en el lugar.

Rittenhouse gira hacia el sonido del disparo mientras que otra persona
se precipita hacia él desde la misma dirección. Rittenhouse dispara
entonces cuatro veces y parece darle al hombre en la cabeza.

\includegraphics{https://static01.graylady3jvrrxbe.onion/images/2020/08/26/video/27vid-vi-kenosha-ES-2/26vid-kenosha-muzzle3-articleLarge.jpg?quality=75\&auto=webp\&disable=upscale}

\hypertarget{segundo-tiroteo}{%
\subsection{Segundo tiroteo}\label{segundo-tiroteo}}

Rittenhouse parece hacer una llamada telefónica y escapa del lugar de
los hechos. Varias personas lo persiguen, algunas de ellas gritando:
``¡Ese es el atacante!''.

Rittenhouse corre, se tropieza y cae. Dispara cuatro veces hacia tres
personas que se dirigen corriendo hacia él. Una persona parece recibir
un balazo en el pecho y cae al suelo. Otra, que lleva un arma de mano,
recibe un disparo en el brazo y se va corriendo.

Los disparos de Rittenhouse se mezclan con el sonido de al menos otros
16 disparos que resuenan durante este tiempo.

\includegraphics{https://static01.graylady3jvrrxbe.onion/images/2020/09/26/video/26vid-kenosha-running/26vid-kenosha-running-videoSixteenByNineJumbo1600.jpg}

\hypertarget{respuesta-de-la-policuxeda}{%
\subsection{Respuesta de la policía}\label{respuesta-de-la-policuxeda}}

Mientras sucede el tiroteo, unos vehículos de la policía que se
encuentran a una manzana de distancia permanecen sin moverse.

Rittenhouse camina con las manos en alto hacia los vehículos de la
policía. Los testigos informan a los oficiales que se trata de alguien
que acaba de dispararle a unas personas.

La policía, de camino a atender a las víctimas, pasa junto a él sin
detenerse.

\includegraphics{https://static01.graylady3jvrrxbe.onion/images/2020/08/26/autossell/viz-4-dif/viz-4-dif-videoSixteenByNineJumbo1600.jpg}

Después de los tiroteos, funcionarios locales anunciaron un toque de
queda a partir de las 7:00 p.m. que continuará hasta el domingo. Y el
gobernador de Wisconsin, Tony Evers, dijo que enviaría cientos de
elementos adicionales de la Guardia Nacional a Kenosha.

Katie G. Nelson, Robin Stein, Evan Hill y Julie Bosman colaboraron con
reporteo. Whitney Hurst colaboró en la producción.

Haley Willis es reportera de la unidad de Investigaciones Visuales del
equipo de video de The New York Times.
\href{https://twitter.com/heytherehaIey}{@heytherehaIey}

Muyi Xiao es productora de video de Investigaciones Visuales, el equipo
de combina informes tradicionales con análisis forense digital avanzado.
\href{https://twitter.com/muyixiao}{@muyixiao}

Christiaan Triebert es periodista del equipo de Investigaciones
Visuales. \href{https://twitter.com/trbrtc}{@trbrtc}

Christoph Koettl es periodista visual de investigación, se ha
especializado en investigación geoespacial y de \emph{open-source}. Es
experto en conflictos armados, derechos humanos y verificación de redes
sociales. \href{https://twitter.com/ckoettl}{@ckoettl}

John Ismay es un escritor que cubre conflictos armados para The New York
Times Magazine. Vive en Washington y es un oficial retirado de
eliminación de artefactos explosivos de las Fuerzas Armadas de Estados
Unidos. \href{https://twitter.com/johnismay}{@johnismay}

Ainara Tiefenthäler es videoperiodista. Cubre noticias de último
momento, Europa, extremismo político y asuntos relacionados con las
mujeres y la comunidad LGBT. Se unió al Times en 2015.
\href{https://twitter.com/tiefenthaeler}{@tiefenthaeler}

\begin{center}\rule{0.5\linewidth}{\linethickness}\end{center}

Advertisement

\protect\hyperlink{after-bottom}{Continue reading the main story}

\hypertarget{site-index}{%
\subsection{Site Index}\label{site-index}}

\hypertarget{site-information-navigation}{%
\subsection{Site Information
Navigation}\label{site-information-navigation}}

\begin{itemize}
\tightlist
\item
  \href{https://help.nytimes3xbfgragh.onion/hc/en-us/articles/115014792127-Copyright-notice}{©~2020~The
  New York Times Company}
\end{itemize}

\begin{itemize}
\tightlist
\item
  \href{https://www.nytco.com/}{NYTCo}
\item
  \href{https://help.nytimes3xbfgragh.onion/hc/en-us/articles/115015385887-Contact-Us}{Contact
  Us}
\item
  \href{https://www.nytco.com/careers/}{Work with us}
\item
  \href{https://nytmediakit.com/}{Advertise}
\item
  \href{http://www.tbrandstudio.com/}{T Brand Studio}
\item
  \href{https://www.nytimes3xbfgragh.onion/privacy/cookie-policy\#how-do-i-manage-trackers}{Your
  Ad Choices}
\item
  \href{https://www.nytimes3xbfgragh.onion/privacy}{Privacy}
\item
  \href{https://help.nytimes3xbfgragh.onion/hc/en-us/articles/115014893428-Terms-of-service}{Terms
  of Service}
\item
  \href{https://help.nytimes3xbfgragh.onion/hc/en-us/articles/115014893968-Terms-of-sale}{Terms
  of Sale}
\item
  \href{https://spiderbites.nytimes3xbfgragh.onion}{Site Map}
\item
  \href{https://help.nytimes3xbfgragh.onion/hc/en-us}{Help}
\item
  \href{https://www.nytimes3xbfgragh.onion/subscription?campaignId=37WXW}{Subscriptions}
\end{itemize}
