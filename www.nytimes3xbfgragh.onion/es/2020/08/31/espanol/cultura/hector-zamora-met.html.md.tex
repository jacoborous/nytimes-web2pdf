Sections

SEARCH

\protect\hyperlink{site-content}{Skip to
content}\protect\hyperlink{site-index}{Skip to site index}

\href{https://www.nytimes3xbfgragh.onion/es/section/cultura}{Cultura}

\href{https://myaccount.nytimes3xbfgragh.onion/auth/login?response_type=cookie\&client_id=vi}{}

\href{https://www.nytimes3xbfgragh.onion/section/todayspaper}{Today's
Paper}

\href{/es/section/cultura}{Cultura}\textbar{}¿Qué es eso en la terraza
del Met? Un enorme y hermoso muro

\url{https://nyti.ms/3jAidS7}

\begin{itemize}
\item
\item
\item
\item
\item
\item
\end{itemize}

Advertisement

\protect\hyperlink{after-top}{Continue reading the main story}

Supported by

\protect\hyperlink{after-sponsor}{Continue reading the main story}

\hypertarget{quuxe9-es-eso-en-la-terraza-del-met-un-enorme-y-hermoso-muro}{%
\section{¿Qué es eso en la terraza del Met? Un enorme y hermoso
muro}\label{quuxe9-es-eso-en-la-terraza-del-met-un-enorme-y-hermoso-muro}}

``Lattice Detour'', del artista mexicano Héctor Zamora, es un monumento
a la apertura frente al encierro, que contrapone la ligereza a la
pesadez, la transitoriedad a la permanencia. También está cargado de
significados políticos.

\includegraphics{https://static01.graylady3jvrrxbe.onion/images/2020/08/28/arts/31Zamora-ES-00/merlin_176111385_7442fe83-b241-4443-a2f1-894e5cbef827-articleLarge.jpg?quality=75\&auto=webp\&disable=upscale}

\href{https://www.nytimes3xbfgragh.onion/by/holland-cotter}{\includegraphics{https://static01.graylady3jvrrxbe.onion/images/2018/02/16/multimedia/author-holland-cotter/author-holland-cotter-thumbLarge.jpg}}

Por \href{https://www.nytimes3xbfgragh.onion/by/holland-cotter}{Holland
Cotter}

\begin{itemize}
\item
  31 de agosto de 2020
\item
  \begin{itemize}
  \item
  \item
  \item
  \item
  \item
  \item
  \end{itemize}
\end{itemize}

\href{https://www.nytimes3xbfgragh.onion/2020/08/27/arts/design/met-roof-hector-zamora-wall.html}{Read
in English}

\href{https://www.nytimes3xbfgragh.onion/newsletters/el-times}{Regístrate
para recibir nuestro boletín} con lo mejor de The New York Times.

\begin{center}\rule{0.5\linewidth}{\linethickness}\end{center}

NUEVA YORK --- El Museo Metropolitano de Arte (Met), una fortaleza de
piedra gris, acero y vidrio que contrasta con Central Park, está
diseñado para dejar fuera casi todo lo que representa el parque. Está
aislado del clima y las estaciones, además de cualquier tipo de cambio
natural. La única excepción en el museo es la terraza Cantor. Dado que
está expuesta a los elementos, la lluvia y el sol la bañan todo el año.

Y aunque el resto del museo ha estado tan oscuro y quieto como una tumba
desde el comienzo del cierre por la pandemia, la terraza se ha llenado
de vida. Las semillas, transportadas por el viento, brotaron en su
pavimento. Los patos salvajes
\href{https://www.instagram.com/p/CDB3hhylj_Y/}{anidaron y criaron una
familia en una caja para sembrar}. En julio, el trabajo en una
instalación escultórica del artista Héctor Zamora, de Ciudad de México,
que dejaron a medio terminar en marzo, volvió a ponerse en marcha a
tiempo para la reapertura al público en general del Met el 29 de agosto.
(Sus miembros entraron el 27 y 28 de agosto).

El proyecto de Zamora, ``Lattice Detour'', el octavo de una serie de
comisiones anuales para la terraza, es perfectamente adecuado para su
momento y lugar. Organizado por Iria Candela, la curadora de arte
latinoamericano del museo, es un monumento a la apertura frente al
encierro, que contrapone la ligereza a la pesadez, la transitoriedad a
la permanencia. También es una imagen cargada de significado político
sobre lo que un muro ---y específicamente el muro fronterizo planeado
entre Estados Unidos y México, y aclamado como algo ``hermoso'' por el
actual presidente estadounidense--- debería ser y hacer.

\includegraphics{https://static01.graylady3jvrrxbe.onion/images/2020/08/27/arts/31Zamora-ES-01/merlin_176111379_68180d6b-bbb3-4881-8303-d95eaf841e87-articleLarge.jpg?quality=75\&auto=webp\&disable=upscale}

Cuando entras a la terraza desde el ascensor, la pieza parece ser lo
opuesto a la apertura y la luminosidad. El muro, una estructura curva de
ladrillos de terracota, con más de 30 metros de largo y tres metros de
alto, parece tener una superficie sólida y estar colocado perversamente
para ocultar una vista espectacular del parque y del horizonte de
Manhattan. Da la impresión de que, para tener una vista al aire libre,
hay que sortear este obstáculo prohibitivo.

Sin embargo, a medida que te acercas, la superficie muestra
paulatinamente su transparencia inesperada. Resulta que los ladrillos
son huecos y forman una malla porosa. Conforme te mueves a lo largo de
la pared, la textura calada se va haciendo evidente muy poco a poco.
Cuando encaras la pared, adquieres una vista completa de la ciudad y el
parque que se encuentran más allá, aunque están filtrados (y pixelados)
por esta. Además, si regresas en el transcurso de un día, verás que la
pared proyecta patrones cambiantes de sombra y luz de forma más
dramática en las primeras horas de la mañana y la tarde (y sin duda
también en las noches de luna llena).

Image

A medida que te acercas, la pared se revela de maneras
inesperadas.Credit...Hiroko Masuike/The New York Times

Al mismo tiempo, un muro es, por tradición, una barrera construida a
propósito, una que en este caso permite vislumbrar lo que hay del otro
lado, pero que no se puede atravesar. En su forma más agresiva, un muro
es un instrumento de separación y exclusión, destinado a mantenernos
alejados de una otredad despreciada y temida, una dinámica demasiado
familiar para los estadounidenses que se encuentran en ambos lados de
nuestra frontera sur.

Zamora, cuyo debut en solitario en Nueva York llegó con este encargo, ha
hecho del comentario político a través de la arquitectura un elemento
central de su obra. En 2004, construyó una estructura temporal de acero
y madera en lo alto del exterior del Museo de Arte Carrillo Gil en
Ciudad de México y vivió en la adición anexa durante semanas,
aprovechando las líneas eléctricas del museo para tener servicio de luz.
La pieza,
``\href{https://lsd.com.mx/artwork/paracaidista-av-revolucion-1608-bis/}{Paracaidista,
Av. Revolución 1608 bis}'' se refería tanto a los refugios ilegales
erigidos por los ocupantes rurales en los límites de la ciudad como a la
inclusión de la ahora comercializable presencia del ``foráneo'' en el
mundo del arte convencional.

Image

El cielo es el límite en ``Lattice Detour.''Credit...Hiroko Masuike/The
New York Times

En 2009, instaló una obra llamada ``Delirio atópico'' en dos rascacielos
modernistas casi idénticos en una calle del centro de Bogotá, Colombia.
Uno de ellos albergaba a inquilinos de lujo; el otro estaba en ruinas.
Llenó un departamento de cada edificio con
\href{https://www.oneart.org/galleries/hector-zamora-delirio-atopico-atopic-delirium}{racimos
de plátanos maduros}, tantos que la fruta parecía salir, como un tumor,
desde las ventanas, y comenzó a pudrirse a los pocos días.

Los plátanos fueron un recordatorio de la historia colonialista pasada y
presente de Colombia, específicamente la llamada masacre de las
bananeras de 1928, cuando, aparentemente bajo la presión del gobierno
estadounidense, las tropas colombianas mataron a tiros a los empleados
en huelga de la United Fruit Company, de propiedad estadounidense. El
legado político persistió y, en fecha tan reciente como 2007, Chiquita,
la multinacional sucesora de United Fruit, fue multada con 25 millones
de dólares por haber pagado dinero de protección a un grupo paramilitar
de derecha de ese país en la década de 1990.

Y en un performance de 2014, ``O abuso da história'', Zamora hizo que se
arrojaran cientos de palmeras en macetas desde las ventanas superiores
del Hospital Matarazzo de São Paulo (Brasil), un recurso urbano otrora
vital que había quedado abandonado desde 1993 y que ahora es el
emplazamiento de un proyectado hotel de lujo. Los árboles fueron dejados
donde cayeron en los patios desiertos del hospital. En poco tiempo,
varios comenzaron a echar raíces, lo que sugería que, a pesar de su
abuso a lo largo de la historia de la humanidad, la naturaleza manda, o
puede, como lo hizo durante un tiempo en la azotea del Met. (En
preparación para la reapertura del museo, los trabajadores limpiaron la
vegetación silvestre del techo, y los guardabosques
\href{https://www.instagram.com/p/CD_y4_vl9V1/}{trasladaron los patos} a
un nuevo hogar).

Image

Los ladrillos usados para la instalación de Zamora fueron hechos y
traídos de México.Credit...Hiroko Masuike/The New York Times

Los plátanos y las palmeras se han convertido, por supuesto, en imágenes
cliché de la vida ``tropical'', y Zamora hace pleno uso de sus
implicaciones exotizantes, como lo hace con el material de construcción
de ``Lattice Detour''. Los ladrillos de arcilla cocida utilizados son de
un tipo popular en todo el hemisferio sur. Llamados ``celosía'' en
español, son moldeados de un material fácilmente disponible, básicamente
la tierra de cualquier lugar. Su oquedad hace que sean fáciles de
transportar y arreglar, y les da propiedades térmicas útiles.

El hecho de que los ladrillos usados para la pieza del Met se hayan
fabricado y traído desde México ---transportados en camión a través de
la frontera y llevados a Nueva York--- añade una dimensión tópica al
muro de Zamora. También lo hace el hecho de que, al construirla, ha
dispuesto los ladrillos con originalidad. Lo normal es apilarlos en
posición vertical, con sus extremos abiertos invisibles, para formar
columnas verticales cerradas. Pero, en la pieza del Met, están
dispuestos horizontalmente, por lo que su oquedad, y el diseño
geométrico que revela, se vuelve funcional de una manera diferente,
práctica pero también estética, ornamental.

Image

``Lattice Detour'' proporciona su propia vista del
atardecer.Credit...Hiroko Masuike/The New York Times

Además, el arte se torna histórico. La curva del muro, y su juego de
transparencia y volumen, recuerda otra escultura mural previa: el
``\href{https://www.tate.org.uk/art/artists/richard-serra-1923/lost-art-richard-serra}{Arco
inclinado}'' de Richard Serra (1981). La pieza de Serra también era
curva y autónoma, pero completamente sólida, por lo que interrumpía la
vista. Con 3,6 metros de altura y fundida en acero autopatinable oscuro,
dividía la plaza que se encuentra afuera del Edificio Federal en el Bajo
Manhattan. Los oficinistas que cruzaban el espacio diariamente se
opusieron desde el principio a la obra, a su masa intrusiva que alteraba
los caminos y a lo que algunos veían como fealdad absoluta. En 1989,
después de acaloradas batallas legales, retiraron el ``Arco inclinado''.

El encargo que el Met asignó a Zamora sirve como homenaje y crítica de
``Arco inclinado''. Al hacerlo, reafirma la idea de que el arte público
y la política deben ser ---y simplemente \emph{son}--- inseparables.
Además, sugiere que, de maneras que los actuales líderes de Estados
Unidos no pueden siquiera empezar a imaginar, un muro puede expandir y
profundizar nuestro amor por un mundo que ninguna política de agresión o
de protección puede mantener fuera.

\begin{center}\rule{0.5\linewidth}{\linethickness}\end{center}

\textbf{El encargo para el jardín del techo del Met: `Lattice Detour',
de Héctor Zamora''}

Hasta el 7 de diciembre en el Museo Metropolitano de Arte, que reabrió
sus puertas el 29 de agosto. Visita
\href{https://www.metmuseum.org/}{metmuseum.org} para conocer el
panorama general de los protocolos de seguridad y la información sobre
los boletos de entrada.

\begin{center}\rule{0.5\linewidth}{\linethickness}\end{center}

Holland Cotter es el codirector de crítica de arte. Escribe sobre una
amplia gama de arte, viejo y nuevo, y ha hecho extensos viajes a África
y China. En 2009 fue galardonado con el Premio Pulitzer de Crítica.

Advertisement

\protect\hyperlink{after-bottom}{Continue reading the main story}

\hypertarget{site-index}{%
\subsection{Site Index}\label{site-index}}

\hypertarget{site-information-navigation}{%
\subsection{Site Information
Navigation}\label{site-information-navigation}}

\begin{itemize}
\tightlist
\item
  \href{https://help.nytimes3xbfgragh.onion/hc/en-us/articles/115014792127-Copyright-notice}{©~2020~The
  New York Times Company}
\end{itemize}

\begin{itemize}
\tightlist
\item
  \href{https://www.nytco.com/}{NYTCo}
\item
  \href{https://help.nytimes3xbfgragh.onion/hc/en-us/articles/115015385887-Contact-Us}{Contact
  Us}
\item
  \href{https://www.nytco.com/careers/}{Work with us}
\item
  \href{https://nytmediakit.com/}{Advertise}
\item
  \href{http://www.tbrandstudio.com/}{T Brand Studio}
\item
  \href{https://www.nytimes3xbfgragh.onion/privacy/cookie-policy\#how-do-i-manage-trackers}{Your
  Ad Choices}
\item
  \href{https://www.nytimes3xbfgragh.onion/privacy}{Privacy}
\item
  \href{https://help.nytimes3xbfgragh.onion/hc/en-us/articles/115014893428-Terms-of-service}{Terms
  of Service}
\item
  \href{https://help.nytimes3xbfgragh.onion/hc/en-us/articles/115014893968-Terms-of-sale}{Terms
  of Sale}
\item
  \href{https://spiderbites.nytimes3xbfgragh.onion}{Site Map}
\item
  \href{https://help.nytimes3xbfgragh.onion/hc/en-us}{Help}
\item
  \href{https://www.nytimes3xbfgragh.onion/subscription?campaignId=37WXW}{Subscriptions}
\end{itemize}
