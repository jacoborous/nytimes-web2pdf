Sections

SEARCH

\protect\hyperlink{site-content}{Skip to
content}\protect\hyperlink{site-index}{Skip to site index}

\href{https://www.nytimes3xbfgragh.onion/es/section/ciencia-y-tecnologia}{Ciencia
y Tecnología}

\href{https://myaccount.nytimes3xbfgragh.onion/auth/login?response_type=cookie\&client_id=vi}{}

\href{https://www.nytimes3xbfgragh.onion/section/todayspaper}{Today's
Paper}

\href{/es/section/ciencia-y-tecnologia}{Ciencia y
Tecnología}\textbar{}¿Y si las mascarillas pudieran ayudar a
inmunizarnos contra el coronavirus?

\url{https://nyti.ms/3haaNTP}

\begin{itemize}
\item
\item
\item
\item
\item
\item
\end{itemize}

\hypertarget{el-brote-de-coronavirus}{%
\subsubsection{\texorpdfstring{\href{https://www.nytimes3xbfgragh.onion/es/spotlight/coronavirus?name=styln-coronavirus-es\&region=TOP_BANNER\&block=storyline_menu_recirc\&action=click\&pgtype=Article\&impression_id=cf6dbb70-f52d-11ea-bb28-3d3dfc9ceefd\&variant=undefined}{El
brote de
coronavirus}}{El brote de coronavirus}}\label{el-brote-de-coronavirus}}

\begin{itemize}
\tightlist
\item
  \href{https://www.nytimes3xbfgragh.onion/es/interactive/2020/espanol/mundo/coronavirus-en-estados-unidos.html?name=styln-coronavirus-es\&region=TOP_BANNER\&block=storyline_menu_recirc\&action=click\&pgtype=Article\&impression_id=cf6de280-f52d-11ea-bb28-3d3dfc9ceefd\&variant=undefined}{Casos
  en Estados Unidos}
\item
  \href{https://www.nytimes3xbfgragh.onion/es/interactive/2020/08/06/espanol/ciencia-y-tecnologia/tengo-covid-19-sintomas.html?name=styln-coronavirus-es\&region=TOP_BANNER\&block=storyline_menu_recirc\&action=click\&pgtype=Article\&impression_id=cf6de281-f52d-11ea-bb28-3d3dfc9ceefd\&variant=undefined}{Síntomas}
\item
  \href{https://www.nytimes3xbfgragh.onion/es/2020/09/11/espanol/ciencia-y-tecnologia/cerebro-coronavirus.html?name=styln-coronavirus-es\&region=TOP_BANNER\&block=storyline_menu_recirc\&action=click\&pgtype=Article\&impression_id=cf6de282-f52d-11ea-bb28-3d3dfc9ceefd\&variant=undefined}{El
  efecto en el cerebro}
\item
  \href{https://www.nytimes3xbfgragh.onion/es/2020/09/09/espanol/ciencia-y-tecnologia/salud-mental-coronavirus.html?name=styln-coronavirus-es\&region=TOP_BANNER\&block=storyline_menu_recirc\&action=click\&pgtype=Article\&impression_id=cf6de283-f52d-11ea-bb28-3d3dfc9ceefd\&variant=undefined}{Convalecencia
  prolongada}
\item
  \href{https://www.nytimes3xbfgragh.onion/es/2020/09/08/espanol/ciencia-y-tecnologia/dentistas-covid-dientes.html?name=styln-coronavirus-es\&region=TOP_BANNER\&block=storyline_menu_recirc\&action=click\&pgtype=Article\&impression_id=cf6e0990-f52d-11ea-bb28-3d3dfc9ceefd\&variant=undefined}{La
  salud dental y el virus}
\end{itemize}

Advertisement

\protect\hyperlink{after-top}{Continue reading the main story}

Supported by

\protect\hyperlink{after-sponsor}{Continue reading the main story}

\hypertarget{y-si-las-mascarillas-pudieran-ayudar-a-inmunizarnos-contra-el-coronavirus}{%
\section{¿Y si las mascarillas pudieran ayudar a inmunizarnos contra el
coronavirus?}\label{y-si-las-mascarillas-pudieran-ayudar-a-inmunizarnos-contra-el-coronavirus}}

Los científicos plantean una idea provocadora, que no ha sido
comprobada: que las mascarillas exponen al usuario a la cantidad justa
de virus para provocar una respuesta inmunitaria protectora.

\includegraphics{https://static01.graylady3jvrrxbe.onion/images/2020/09/08/science/09sci-VIRUS-MASK-ES/merlin_175295325_ea704379-e120-4ed6-9d9f-c17d0b2dc640-articleLarge.jpg?quality=75\&auto=webp\&disable=upscale}

\href{https://www.nytimes3xbfgragh.onion/by/katherine-j--wu}{\includegraphics{https://static01.graylady3jvrrxbe.onion/images/2020/08/11/reader-center/author-katherine-j-wu/author-katherine-j-wu-thumbLarge.png}}

Por
\href{https://www.nytimes3xbfgragh.onion/by/katherine-j--wu}{Katherine
J. Wu}

\begin{itemize}
\item
  9 de septiembre de 2020
\item
  \begin{itemize}
  \item
  \item
  \item
  \item
  \item
  \item
  \end{itemize}
\end{itemize}

\href{https://www.nytimes3xbfgragh.onion/2020/09/08/health/covid-masks-immunity.html}{Read
in
English}\href{https://www.nytimes3xbfgragh.onion/2020/09/08/health/covid-masks-immunity.html}{Read
in English}

\href{https://www.nytimes3xbfgragh.onion/newsletters/el-times}{Regístrate
para recibir nuestro boletín} con lo mejor de The New York Times.

\begin{center}\rule{0.5\linewidth}{\linethickness}\end{center}

Mientras el mundo espera la llegada de una vacuna segura y eficaz contra
el coronavirus, un equipo de investigadores ha presentado una nueva
teoría provocadora: que las
\href{https://www.nytimes3xbfgragh.onion/es/interactive/2020/espanol/coronavirus-mejores-mascarillas.html}{mascarillas}
podrían ayudar a inmunizar, de manera burda o rústica, a algunas
personas contra el virus.

Esa idea, que aún no ha sido comprobada, fue descrita
en\href{https://www.nejm.org/doi/full/10.1056/NEJMp2026913}{un
comentario científico} publicado el martes en el New England Journal of
Medicine y está inspirada en el antiguo concepto de variolación, que
consiste en la exposición deliberada a un patógeno con el fin de generar
una respuesta inmunitaria protectora. Esa práctica arriesgada fue
probada contra la viruela hasta que finalmente cayó en desuso, pero
allanó el camino para el surgimiento de las vacunas modernas.

Exponerse al virus, mientras se usa un cubrebocas, no sustituye a una
vacuna auténtica. Pero los datos de animales infectados con el
coronavirus, así como la información obtenida en los estudios de otras
enfermedades, sugieren que las mascarillas, al reducir la cantidad de
virus que se encuentran en las vías respiratorias de una persona,
podrían reducir las posibilidades de que el usuario se enferme. Los
investigadores sostienen que si una pequeña cantidad de patógenos se
filtra, podría hacer que
\href{https://www.nytimes3xbfgragh.onion/es/interactive/2020/03/13/science/coronavirus-celulas-sintomas.html}{el
cuerpo produzca células inmunes}que recuerden el virus y se queden para
combatirlo.

``Pueden tener este virus pero permanecer asintomáticos'', dijo Monica
Gandhi, médica especializada en enfermedades infecciosas de la
Universidad de California en San Francisco, y una de las autoras del
comentario. ``Entonces, si se pueden aumentar las tasas de infección
asintomática con los cubrebocas, tal vez eso se convierta en una forma
de aplicar la variolación en la población''.

Eso no significa que las personas deban ponerse una mascarilla para
inocularse intencionalmente con el virus. ``Esa no es la recomendación
en absoluto'', dijo Gandhi. ``Tampoco lo son las fiestas de varicela'',
agregó, refiriéndose
a\href{https://www.nytimes3xbfgragh.onion/es/2020/07/13/espanol/mundo/fiesta-covid.html}{las
reuniones sociales} que mezclan a las personas sanas con las enfermas.

La teoría no puede ser probada directamente sin ensayos clínicos que
comparen los resultados de las personas que usan cubrebocas en presencia
del coronavirus, con quienes no los usan, una configuración experimental
que, además, resultaría poco ética. Y aunque los expertos externos están
intrigados por la teoría, se mostraron reacios a aceptarla sin más datos
y aconsejaron que se interprete de manera cuidadosa.

``Parece un salto'', dijo Saskia Popescu, epidemióloga de enfermedades
infecciosas con sede en Arizona que no participó en el comentario.
``Pero no tenemos mucho para respaldarlo''.

Tomada de manera incorrecta, la idea podría hacer que los enmascarados
adoptaran una
\href{https://www.nytimes3xbfgragh.onion/es/2020/03/24/espanol/ciencia-y-tecnologia/coronavirus-panico.html}{falsa
sensación de complacencia}, poniéndolos en mayor riesgo que antes, o que
tal vez se reforzara la noción incorrecta de que las cubiertas faciales
son completamente inútiles contra el coronavirus, porque no pueden
lograr que el portador sea impermeable a las infecciones.

``Queremos que la gente continúe con todas las estrategias de
prevención'', dijo Popescu. Eso significa mantenerse alerta para evitar
las multitudes, procurar el distanciamiento físico y la higiene de las
manos, comportamientos que se superponen en sus efectos, pero que no
pueden reemplazarse entre sí.

La teoría de la variolación del coronavirus se basa en dos suposiciones
que son difíciles de probar:
que\href{https://link.springer.com/article/10.1007/s11606-020-06067-8}{las
dosis más bajas del virus provocan una enfermedad menos grave} y que las
infecciones leves o asintomáticas pueden estimular la protección a largo
plazo contra episodios posteriores de la enfermedad. Aunque otros
patógenos ofrecen algún precedente para ambos conceptos, la evidencia
del coronavirus sigue siendo escasa, en parte porque los científicos
solo han tenido la oportunidad de estudiar el virus durante unos meses.

Los experimentos en hámsteres han sugerido una conexión entre la dosis y
la enfermedad. A principios de este año, un equipo de investigadores en
China descubrió que los hámsteres alojados detrás de una barrera hecha
de mascarillas quirúrgicas tenían menos probabilidades de infectarse con
el coronavirus. Y los que contrajeron el
virus\href{https://academic.oup.com/cid/advance-article/doi/10.1093/cid/ciaa644/5848814}{se
enfermaron menos} que otros animales que no tenían máscaras de
protección.

Algunas observaciones en humanos parecen respaldar esta tendencia. En
entornos abarrotados donde las mascarillas se utilizan ampliamente,
pareciera
que\href{https://www.nytimes3xbfgragh.onion/2020/07/14/health/coronavirus-hair-salon-masks.html}{las
tasas de infección caen en picada}. Y aunque las cubiertas faciales no
son capaces de bloquear todas las partículas de virus entrantes para
todas las personas, parecen estar relacionadas con una enfermedad de
menor intensidad. Los investigadores han descubierto brotes ---en gran
parte silenciosos y asintomáticos--- en lugares que
van\href{https://www.nytimes3xbfgragh.onion/2020/07/27/health/coronavirus-mask-protection.html}{desde
cruceros hasta plantas de procesamiento de alimentos}, todos abarrotados
y donde la mayoría de las personas usaban cubrebocas.

Se han recopilado datos que relacionan la dosis con los síntomas de
otros microbios que atacan las vías respiratorias
humanas,\href{https://www.thelancet.com/journals/lanres/article/PIIS2213-2600(20)30323-4/fulltext}{incluidos
los virus de la influenza y las bacterias que causan la tuberculosis}.

A pesar de décadas de investigación, la mecánica de la transmisión aérea
sigue siendo en gran medida ``una caja negra'', dijo Jyothi Rengarajan,
experta en vacunas y enfermedades infecciosas de la Universidad de Emory
que no participó en el comentario.

Eso se debe, en parte, a que es difícil precisar la dosis infecciosa
necesaria para enfermar a una persona, dijo Rengarajan. Incluso si los
investigadores finalmente establecen una dosis promedio, el resultado
variará de una persona a otra, puesto que factores como la genética, el
estado inmunológico de una persona y la arquitectura de sus conductos
nasales pueden influir en la cantidad de virus que pueden colonizar el
tracto respiratorio.

Y eso confirmaría la segunda mitad de la teoría de variolación, que las
mascarillas permiten la entrada de virus en cantidad suficiente para
preparar el sistema inmunológico, podría ser aún más complicado.
Aunque\href{https://www.nytimes3xbfgragh.onion/2020/08/16/health/coronavirus-immunity-antibodies.html}{varios
estudios recientes} han señalado la posibilidad de que los casos leves
de COVID-19 puedan provocar una fuerte respuesta inmune al coronavirus,
no se puede probar una protección duradera hasta que los investigadores
recopilen datos sobre las infecciones durante meses o años después de
que se hayan superado.

En general, la teoría ``tiene algunos méritos'', dijo Angela Rasmussen,
viróloga de la Universidad de Columbia que no participó en el
comentario. ``Pero sigo siendo bastante escéptica''.

Rasmussen dice que es importante recordar que las vacunas son
intrínsecamente menos peligrosas que las infecciones reales, por lo que
prácticas como la variolación (a veces llamada inoculación) finalmente
se volvieron obsoletas. Antes de que se descubrieran las vacunas, los
médicos se las arreglaban frotando trozos de costras de viruela o pus en
la piel de las personas sanas. Las infecciones resultantes generalmente
eran menos graves que los casos de viruela que se contagiaban de la
manera típica, pero ``definitivamente la gente contrajo viruela y murió
por variolación'', dijo Rasmussen. Y la variolación, a diferencia de las
vacunas, puede hacer que las personas sean contagiosas.

Gandhi reconoció estas limitaciones y señaló que la teoría no debe
interpretarse como algo más que eso: una teoría. Aun así, dijo: ``¿Por
qué no aumentar la posibilidad de no enfermarse y tener algo de
inmunidad mientras esperamos la vacuna?''.

Katherine J. Wu es una reportera que cubre ciencia y salud. Tiene un
doctorado en microbiología e inmunobiología de la Universidad de
Harvard.\href{https://twitter.com/KatherineJWu}{@KatherineJWu}

Advertisement

\protect\hyperlink{after-bottom}{Continue reading the main story}

\hypertarget{site-index}{%
\subsection{Site Index}\label{site-index}}

\hypertarget{site-information-navigation}{%
\subsection{Site Information
Navigation}\label{site-information-navigation}}

\begin{itemize}
\tightlist
\item
  \href{https://help.nytimes3xbfgragh.onion/hc/en-us/articles/115014792127-Copyright-notice}{©~2020~The
  New York Times Company}
\end{itemize}

\begin{itemize}
\tightlist
\item
  \href{https://www.nytco.com/}{NYTCo}
\item
  \href{https://help.nytimes3xbfgragh.onion/hc/en-us/articles/115015385887-Contact-Us}{Contact
  Us}
\item
  \href{https://www.nytco.com/careers/}{Work with us}
\item
  \href{https://nytmediakit.com/}{Advertise}
\item
  \href{http://www.tbrandstudio.com/}{T Brand Studio}
\item
  \href{https://www.nytimes3xbfgragh.onion/privacy/cookie-policy\#how-do-i-manage-trackers}{Your
  Ad Choices}
\item
  \href{https://www.nytimes3xbfgragh.onion/privacy}{Privacy}
\item
  \href{https://help.nytimes3xbfgragh.onion/hc/en-us/articles/115014893428-Terms-of-service}{Terms
  of Service}
\item
  \href{https://help.nytimes3xbfgragh.onion/hc/en-us/articles/115014893968-Terms-of-sale}{Terms
  of Sale}
\item
  \href{https://spiderbites.nytimes3xbfgragh.onion}{Site Map}
\item
  \href{https://help.nytimes3xbfgragh.onion/hc/en-us}{Help}
\item
  \href{https://www.nytimes3xbfgragh.onion/subscription?campaignId=37WXW}{Subscriptions}
\end{itemize}
