Sections

SEARCH

\protect\hyperlink{site-content}{Skip to
content}\protect\hyperlink{site-index}{Skip to site index}

\href{https://www.nytimes3xbfgragh.onion/es/section/opinion}{Opinión}

\href{https://myaccount.nytimes3xbfgragh.onion/auth/login?response_type=cookie\&client_id=vi}{}

\href{https://www.nytimes3xbfgragh.onion/section/todayspaper}{Today's
Paper}

\href{/es/section/opinion}{Opinión}\textbar{}La doble moral de nuestros
gobiernos

\url{https://nyti.ms/31WkEZg}

\begin{itemize}
\item
\item
\item
\item
\item
\end{itemize}

Advertisement

\protect\hyperlink{after-top}{Continue reading the main story}

\href{/es/section/opinion}{Opinión}

Supported by

\protect\hyperlink{after-sponsor}{Continue reading the main story}

Comentario

\hypertarget{la-doble-moral-de-nuestros-gobiernos}{%
\section{La doble moral de nuestros
gobiernos}\label{la-doble-moral-de-nuestros-gobiernos}}

Si eres de Latinoamérica quizás reconozcas una serie de dobles
estándares de tu gobierno, por ejemplo: se persiguen actos de corrupción
del pasado pero se justifican los de sus aliados.

\includegraphics{https://static01.graylady3jvrrxbe.onion/images/2020/09/03/multimedia/03Fonseca-ES/merlin_176464098_0310dee3-7f81-458e-b02b-4b12ddeb9859-articleLarge.jpg?quality=75\&auto=webp\&disable=upscale}

Por Diego Fonseca

Es escritor y periodista.

\begin{itemize}
\item
  3 de septiembre de 2020
\item
  \begin{itemize}
  \item
  \item
  \item
  \item
  \item
  \end{itemize}
\end{itemize}

\href{https://www.nytimes3xbfgragh.onion/newsletters/el-times}{Regístrate
para recibir nuestro boletín} con lo mejor de The New York Times.

\begin{center}\rule{0.5\linewidth}{\linethickness}\end{center}

La explicación fue excusa.
\href{https://latinus.us/2020/08/21/es-distinto-menos-dinero-aportaciones-amlo-video-pio-amlo-david-leon/}{Un
medio} publicó unos videos de Pío López Obrador, hermano del presidente
de México, recibiendo dinero en efectivo de un operador político. Todos
esperamos los pasos siguientes: escándalo, mea culpa, renuncias,
investigaciones. Por ahora, no pasó nada.

Pío no se excusó ni pío y Andrés Manuel López Obrador puso paños fríos
con velocidad de apagaincendios entrenado: que el dinero era menos que
en sonados casos de corrupción ---como si los principios se midieran por
cantidad de billetes--- y que las bolsas de papel con dinero en efectivo
no eran lo que todos creían que eran sino
\href{https://www.animalpolitico.com/2020/08/amlo-respuesta-video-pio-lopez-obrador-david-leon-dinero/}{contribuciones
populares} para financiar a su movimiento. ``La Revolución mexicana se
financió con la cooperación del pueblo'',
\href{https://twitter.com/Reforma/status/1296907427053334528}{comparó}.

El gobierno de AMLO creó un escudo de excusas para el extraño
comportamiento de su hermano. En sus Mañaneras, el presidente de México
ha mencionado sin cesar un
\href{https://www.nytimes3xbfgragh.onion/es/2020/08/21/espanol/opinion/emilio-lozoya-amlo.html}{video
con maletas de dinero sucio} como ejemplo de la corrupción ``del
pasado''. Ahora, dijo que la difusión de las imágenes de su hermano era
una reacción de sus opositores por las investigaciones de la justicia
sobre exfuncionarios del gobierno de su predecesor, Enrique Peña Nieto.
Como si asumiera que la política constituye un intercambio público de
prontuarios para ver quién más sucio.

Los videos no tienen estatuto jurídico, pero sí ético y político: la
doble moral es el trago de la casa. AMLO ha optado por establecer que la
opacidad ajena siempre es corrupción, pero la propia solo puede ser
financiamiento legítimo. No parece entender que llegó al gobierno con la
bandera de la transparencia y la honestidad mientras dinero
aparentemente no auditado engrasaba los mecanismos de su partido. Eligió
poner en la balanza un argumento de pesos ---nuestras bolsas de pan con
\emph{cash}, sus maletas de dinero electoral--- cuando era de esencias:
opacidad es opacidad, no importa si es tuya o mía.

Tras la difusión de los videos de Pío López Obrador, el gobierno de
México trató zanjar el asunto como suele hacer, con una declaración
definitiva. Por un lado, en su
\href{https://www.forbes.com.mx/politica-he-cumplido-95-de-mis-100-compromisos-presume-amlo-en-segundo-informe-de-gobierno/}{segundo
informe de gobierno} ---su balance de dos años de gestión--- AMLO
aseguró que la corrupción acabó con la Cuarta Transformación; todo lo
malo es hijo del pasado. ``Este gobierno no será recordado por
corrupto'', dijo. ``Nuestro principal legado será purificar la vida
pública de México y estamos avanzando''. Por el otro, dio un paso propio
de las revanchas autocráticas:
\href{https://www.lavanguardia.com/politica/20200828/483139613276/acusan-al-gobierno-de-mexico-de-estigmatizar-a-ong-que-critican-al-tren-maya.html}{inculpó
a diversas organizaciones} periodísticas de recibir financiamiento
internacional para investigar proyectos de su gobierno con fines
críticos.

En corto: defiendo a los míos porque son menos malos que los demás y
estigmatizo a quienes me cuestionan como enemigos de la causa. Otra vez,
doble moral.

Construir listas negras y sembrar descrédito en los que piensan distinto
y hacen su trabajo de contralor es una carta regular de proyectos
autoritarios, incluso elegidos por el voto. La ultraderecha como la
izquierda más insustancial crean enemigos y alimentan conspiraciones
mientras justifican los malos pasos de sus propios miembros, de Pío a
\href{https://cnnespanol.cnn.com/2019/12/20/tres-datos-que-debes-saber-sobre-manuel-bartlett-diaz-director-de-la-cfe/}{Manuel
Bartlett}, funcionario de AMLO a quien investigaciones periodísticas han
señalado de posibles casos de corrupción. Ninguno fue mencionado en el
informe presidencial.

El doble discurso es particularmente severo cuando sus promotores se
presentan como salvadores morales.

Hay un subtexto interesante entre los dirigentes que se dicen
progresistas y se aprovechan de su paso por el Estado diciendo que sus
malos actos no son corrupción, malversación o mala gestión sino justicia
revolucionaria. Como si el afán redistributivo incluyese llenar los
bolsillos de la militancia por los servicios prestados. Todos tienen un
relato redentor cuando sus actos en la función pública presentan
resultados que dañan a los menos privilegiados. Robar para la causa
---así sea dinero público--- es legítimo.

Un periodista argentino, kirchnerista él,
\href{https://www.tiempoar.com.ar/nota/y-si-hablamos-de-corrupcion-en-serio}{llegó
a justificar} ese tipo de corrupción como un movimiento de equilibrio
político: los partidos progresistas, decía su tesis, arrancan tan atrás
en términos de financiamiento respecto de las organizaciones
conservadoras que deben aceptar fondos de todo tipo para equiparar las
posibilidades de batalla contra los partidos del \emph{establishment} y
hacer visible la verdad revelada de las masas.

La doble moral de los cruzados es peor que la baja moral de los
corruptos porque se presentan como probos. La nueva política que acabará
con las castas aprovechadoras. Sus malos actos, por lo tanto, frustran
una de las últimas esperanzas de sociedades olvidadas. No tienen margen:
si se suponen salvadores, \emph{deben} ser mejores. Deben ser escrutados
en profundidad y sujetos a estándares mayores porque ellos solos
elevaron la barrera. Los demás podían pretender ser honestos; ellos no
tienen más opción que serlo. Sin embargo, no toleran que les señalen su
falta de integridad.

Si usted es latinoamericano, cuanto digo no le resultará extraño. He
aquí una lista del doble estándar donde, probablemente, hallarás a tu
gobierno: agravian a organizaciones que reciben financiamiento legal,
pero defienden recibir dinero en efectivo en reuniones mal iluminadas.
Postulan la democracia plebiscitaria, pero quien decide es el líder. Se
asumen abanderados del progresismo y sus naciones retroceden. Hablan de
justicia, cooptan jueces. Prometen países de mayorías inclusivas y
ensanchan la pobreza. Levantan la bandera de la transformación: dejan
detrás un desastre que obligará a mayores esfuerzos para regresar al
punto de partida. ¿Igualitarios?
\href{https://politica.expansion.mx/voces/2020/05/13/columnainvitada-misoginia-en-la-4t}{Excluyentes}.
¿Interesados en defender a los pobres? Solo mientras obedezcan a su
clientelismo.

Tampoco suelen ser los luchadores contra los oligopolios y las élites
que suponemos: su plan a menudo es reemplazar un bloque hegemónico con
un nuevo, pero suyo. Consideren este comportamiento como una concepción
de la política que supone la captura del Estado como una eterna batalla
de facciones entre probos y malos, y deja a los ciudadanos como
espectadores.

Un modo perverso de hacer política: no defienden a las mayorías; apenas
justifican el asalto al Estado burgués. Los líderes creerán que, para
conseguir resultados transformadores, pueden doblar algunas leyes y
pasar por alto varias normas. Y como ellos salvarán a los excluidos, ese
fin justifica cualquier medio. Se llame Cuarta Transformación,
kirchnerismo,
\href{https://www.nytimes3xbfgragh.onion/es/2020/08/23/espanol/opinion/colombia-alvaro-uribe.html}{uribismo}
o chavismo.

Es un problema doble, porque si señalamos sus errores, no hay
tolerancia. No son más papistas que un papa: son una nueva Inquisición
de moral flexible para los suyos y ferrosa para el resto.

Diego Fonseca es colaborador regular de The New York Times y director
del Institute for Socratic Dialogue de Barcelona. \emph{Voyeur}, su
nuevo libro de perfiles, se publicará pronto en España.

Advertisement

\protect\hyperlink{after-bottom}{Continue reading the main story}

\hypertarget{site-index}{%
\subsection{Site Index}\label{site-index}}

\hypertarget{site-information-navigation}{%
\subsection{Site Information
Navigation}\label{site-information-navigation}}

\begin{itemize}
\tightlist
\item
  \href{https://help.nytimes3xbfgragh.onion/hc/en-us/articles/115014792127-Copyright-notice}{©~2020~The
  New York Times Company}
\end{itemize}

\begin{itemize}
\tightlist
\item
  \href{https://www.nytco.com/}{NYTCo}
\item
  \href{https://help.nytimes3xbfgragh.onion/hc/en-us/articles/115015385887-Contact-Us}{Contact
  Us}
\item
  \href{https://www.nytco.com/careers/}{Work with us}
\item
  \href{https://nytmediakit.com/}{Advertise}
\item
  \href{http://www.tbrandstudio.com/}{T Brand Studio}
\item
  \href{https://www.nytimes3xbfgragh.onion/privacy/cookie-policy\#how-do-i-manage-trackers}{Your
  Ad Choices}
\item
  \href{https://www.nytimes3xbfgragh.onion/privacy}{Privacy}
\item
  \href{https://help.nytimes3xbfgragh.onion/hc/en-us/articles/115014893428-Terms-of-service}{Terms
  of Service}
\item
  \href{https://help.nytimes3xbfgragh.onion/hc/en-us/articles/115014893968-Terms-of-sale}{Terms
  of Sale}
\item
  \href{https://spiderbites.nytimes3xbfgragh.onion}{Site Map}
\item
  \href{https://help.nytimes3xbfgragh.onion/hc/en-us}{Help}
\item
  \href{https://www.nytimes3xbfgragh.onion/subscription?campaignId=37WXW}{Subscriptions}
\end{itemize}
