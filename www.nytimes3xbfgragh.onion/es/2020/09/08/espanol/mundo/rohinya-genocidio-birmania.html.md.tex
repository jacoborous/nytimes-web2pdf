Sections

SEARCH

\protect\hyperlink{site-content}{Skip to
content}\protect\hyperlink{site-index}{Skip to site index}

\href{/es/section/mundo}{Mundo}\textbar{}`Maten a todos los que vean':
dos soldados birmanos hablan por primera vez sobre la matanza de
rohinyás

\url{https://nyti.ms/339ysyP}

\begin{itemize}
\item
\item
\item
\item
\item
\item
\end{itemize}

\includegraphics{https://static01.graylady3jvrrxbe.onion/images/2020/09/08/world/08rohingya-graves-ES-00/merlin_176684058_c9fb54cc-a16a-42b6-b331-4eaed1c08263-articleLarge.jpg?quality=75\&auto=webp\&disable=upscale}

\hypertarget{maten-a-todos-los-que-vean-dos-soldados-birmanos-hablan-por-primera-vez-sobre-la-matanza-de-rohinyuxe1s}{%
\section{`Maten a todos los que vean': dos soldados birmanos hablan por
primera vez sobre la matanza de
rohinyás}\label{maten-a-todos-los-que-vean-dos-soldados-birmanos-hablan-por-primera-vez-sobre-la-matanza-de-rohinyuxe1s}}

El testimonio en video de dos miembros del ejército apoya las
acusaciones generalizadas de que el ejército de Birmania trató de
erradicar a la minoría étnica en una campaña genocida.

Los restos de una escuela rohinyá en el estado de Rakhine en el oeste de
Birmania el año pasado.Credit...Adam Dean para The New York Times

Supported by

\protect\hyperlink{after-sponsor}{Continue reading the main story}

Por \href{https://www.nytimes3xbfgragh.onion/by/hannah-beech}{Hannah
Beech}, Saw Nang y
\href{https://www.nytimes3xbfgragh.onion/by/marlise-simons}{Marlise
Simons}

\begin{itemize}
\item
  8 de septiembre de 2020
\item
  \begin{itemize}
  \item
  \item
  \item
  \item
  \item
  \item
  \end{itemize}
\end{itemize}

\href{https://www.nytimes3xbfgragh.onion/2020/09/08/world/asia/myanmar-rohingya-genocide.html}{Read
in English}

\href{https://www.nytimes3xbfgragh.onion/newsletters/el-times}{Regístrate
para recibir nuestro boletín} con lo mejor de The New York Times.

\begin{center}\rule{0.5\linewidth}{\linethickness}\end{center}

Los dos soldados confiesan sus crímenes de forma monótona. Solo unos
cuantos parpadeos delatan su emoción: ejecuciones, entierros en masa,
destrucción de pueblos y violaciones.

La orden de su comandante en agosto de 2017 fue clara, dijo el soldado
Myo Win Tun en un testimonio en video: ``Disparen a todo lo que vean y a
todo lo que escuchen''.

Dijo que obedeció, participó en la masacre de 30 musulmanes rohinyás y
los enterró en una fosa común cerca de una torre celular y una base
militar.

En un municipio vecino, el soldado Zaw Naing Tun dijo que alrededor de
la misma época, él y sus camaradas de otro batallón siguieron una
directiva casi idéntica de su superior: `Maten a todos los que vean,
sean niños o adultos'.

\includegraphics{https://static01.graylady3jvrrxbe.onion/images/2020/09/08/world/08rohingya-graves-ES-01/merlin_176717454_db6f0804-1f8c-44b4-a8d7-5f33bdcf53ce-articleLarge.jpg?quality=75\&auto=webp\&disable=upscale}

``Acabamos con unas 20 aldeas'', dijo el soldado Zaw Naing Tun, y agregó
que él también arrojó los cuerpos en una fosa común.

El testimonio en video de los dos soldados se compartió con los fiscales
internacionales y es la primera vez que miembros del Tatmadaw ---como se
conoce al ejército de Birmania--- confiesan abiertamente haber
participado en lo que los funcionarios de las Naciones Unidas dicen que
fue una campaña genocida contra la minoría musulmana rohinyá del país.

Image

Refugiados rohinyás de Birmania en Bangladés en agosto de 2017. Muchos
habían huído del área de Taung Bazar, donde el soldado Myo Win Tun ha
confesado haber participado en atrocidades.Credit...Adam Dean para The
New York Times

El 7 de septiembre, los dos hombres, que huyeron de Birmania el mes
pasado, fueron llevados a La Haya, donde la Corte Penal Internacional ha
abierto un caso que examina si los líderes del Tatmadaw cometieron
crímenes a gran escala contra los rohinyás.

Las atrocidades descritas por los dos hombres se hacen eco de las
pruebas de graves abusos a los derechos humanos que se han recogido
entre los más de un millón de rohinyás que ahora se refugian en el
vecino Bangladés. Lo que distingue su testimonio es que proviene de los
perpetradores, no de las víctimas.

``Este es un instante monumental para los rohinyás y el pueblo de
Birmania en su continua lucha por la justicia'', dijo Matthew Smith,
director ejecutivo de Fortify Rights, un organismo de vigilancia de
derechos humanos. ``Estos hombres podrían ser los primeros perpetradores
de Birmania juzgados en la Corte Penal Internacional, y los primeros
testigos internos bajo custodia del tribunal''.

The New York Times no pudo confirmar independientemente que los dos
soldados cometieron los crímenes que confesaron. Pero los detalles de
sus relatos se ajustan a las descripciones proporcionadas por decenas de
testigos y observadores, incluidos los refugiados rohinyás, los
residentes de Rakhine, los soldados del Tatmadaw y los políticos
locales.

Y múltiples aldeanos confirmaron de manera independiente la ubicación de
las fosas comunes que los soldados proporcionaron en su testimonio,
pruebas que serán aprovechadas en las investigaciones de la Corte Penal
Internacional y otros procedimientos judiciales. El gobierno de Birmania
ha negado repetidamente la existencia de esos sitios en toda la región.

Los crímenes que, según los soldados, fueron perpetrados por sus
batallones de infantería y otras fuerzas de seguridad ---unos 150
civiles muertos y docenas de aldeas destruidas--- son solo una parte de
la larga campaña de Birmania en contra de los rohinyás. Y retratan una
operación concertada y calculada para exterminar a un solo grupo étnico
minoritario, la cuestión que está en el centro de los actuales casos de
genocidio.

Las masacres de los rohinyás que culminaron en 2017 fueron catalizadoras
de una de las huidas más rápidas de refugiados del mundo. En cuestión de
semanas,
\href{https://www.nytimes3xbfgragh.onion/es/2019/08/26/espanol/mundo/rohinya-exodo-banglades.html}{tres
cuartos de millón de apátridas} fueron desarraigados de sus hogares en
el estado occidental de Rakhine, en Birmania, mientras las fuerzas de
seguridad atacaban sus aldeas con rifles, machetes y lanzallamas.

Image

Refugiados rohinyás en un campamento cerca de Amtali, Bangladés, en
agosto de 2017Credit...Adam Dean para The New York Times

Los ancianos fueron decapitados y las chicas jóvenes violadas y les
arrancaron los pañuelos de la cabeza para vendarles los ojos, dijeron
los testigos y los sobrevivientes. Médicos sin Fronteras calculó que por
lo menos
\href{https://www.nytimes3xbfgragh.onion/2017/12/14/world/asia/myanmar-rohingya-deaths.html}{6700
rohinyás}, entre ellos 730 niños, sufrieron muertes violentas desde
fines de agosto hasta fines de septiembre de 2017. Aproximadamente 200
asentamientos rohinyás fueron completamente arrasados entre 2017 y 2019,
según las Naciones Unidas.

En un
\href{https://www.ohchr.org/Documents/HRBodies/HRCouncil/FFM-Myanmar/20190916/A_HRC_42_CRP.5.pdf}{informe}
publicado el año pasado, una misión de investigación del Consejo de
Derechos Humanos de las Naciones Unidas dijo que ``existe un grave
riesgo de que se produzcan o se repitan actos genocidas y de que
Birmania no cumpla su obligación de prevenir el genocidio, de investigar
el genocidio y de promulgar leyes eficaces que lo tipifiquen como delito
y lo castiguen''.

El gobierno de Myanmar ha negado cualquier campaña orquestada contra los
rohinyás. El pasado diciembre,
\href{https://www.nytimes3xbfgragh.onion/2019/12/10/world/asia/aung-san-suu-kyi-myanmar-genocide-hague.html}{Daw
Aung San Suu Kyi}, líder civil de la nación, defendió a Birmania de los
cargos de genocidio en otro caso, éste en la Corte Internacional de
Justicia de La Haya. Ganadora del premio Nobel de la Paz, Aung San Suu
Kyi ha visto su legado empañado por su apoyo a los militares y su
negativa a condenar claramente la persecución de los rohinyás.

Solo unos pocos soldados del Tatmadaw han sido castigados, con breves
períodos de prisión, por actos que los militares dicen que fueron
eventos aislados en un par de aldeas.

Aunque los rohinyás son del estado de Rakhine en Birmania,
\href{https://www.nytimes3xbfgragh.onion/2017/10/24/world/asia/myanmar-rohingya-ethnic-cleansing.html}{el
gobierno del país afirma que son intrusos extranjeros}. Los funcionarios
birmanos han sugerido que los rohinyás quemaron sus propias aldeas para
obtener simpatía internacional.

Los relatos de los dos soldados destrozan esa narrativa oficial.

No está claro qué pasará con los dos hombres, que no están bajo arresto
pero que fueron efectivamente puestos bajo custodia de la Corte Penal
Internacional el lunes. Podrían prestar testimonio en los procedimientos
judiciales y ser puestos en protección de testigos. Podrían ser
juzgados. La oficina del fiscal del tribunal se negó a comentar
públicamente sobre un caso en curso, pero dos personas al tanto de las
investigaciones dijeron que los hombres ya habían sido interrogados
ampliamente por funcionarios del tribunal en las últimas semanas.

Image

Refugiados rohinyás en septiembre de 2017 después de cruzar a Bangladés.
Las aldeas arden en el fondo.Credit...Adam Dean para The New York Times

La Corte Penal Internacional normalmente enjuicia a figuras de alto
nivel acusadas de delitos graves como el genocidio o los crímenes de
lesa humanidad, no a soldados rasos.

Payam Akhavan, un abogado canadiense que representa a Bangladés en una
demanda contra Birmania ante la Corte Penal Internacional, no quiso
hacer comentarios sobre la identidad de los dos hombres. Sin embargo,
pidió que se rindieran cuentas para evitar nuevas atrocidades contra los
600.000 rohinyás que permanecen en Birmania.

``La impunidad no es una opción'', dijo Akhavan. ``Algo de justicia es
mejor que nada de justicia''.

Los relatos de los soldados también añadirán peso al caso separado en la
Corte Internacional de Justicia, donde se acusa a Birmania de intentar
``destruir a los rohinyás como grupo, en su totalidad, mediante el uso
de
\href{https://www.nytimes3xbfgragh.onion/2017/09/02/world/asia/rohingya-myanmar-bangladesh-refugees-massacre.html?action=click\&module=RelatedCoverage\&pgtype=Article\&region=Footer}{asesinatos
en masa, violaciones y otras formas de violencia sexual}, así como la
destrucción sistemática por el fuego de sus aldeas''.

Ese caso fue presentado el año pasado por Gambia en nombre de la
Organización para la Cooperación Islámica, integrada por 57 naciones. La
semana pasada, los Países Bajos y Canadá anunciaron que prestarían apoyo
jurídico a la labor de responsabilizar a Birmania por el genocidio,
calificándolo de asunto ``de interés para toda la humanidad''.

Image

Un campamento de refugiados cerca de Bazar de Cox, Bangladés, en
noviembre de 2017. Más de un millón de rohinyás se han refugiado en
Bangladés.Credit...Adam Dean para The New York Times

En agosto de 2017, los batallones de infantería ligera números 353 y 565
llevaron a cabo ``operaciones de despeje'' en las áreas donde los
hombres dijeron que lo hicieron, los municipios de Buthidaung y
Maungdaw. Los comandantes que el soldado Myo Win Tun dijo que ordenaron
eliminar a los rohinyás ---el coronel Than Htike, el capitán Tun Tun y
el sargento Aung San Oo--- estaban operando ahí en ese momento, según
sus compañeros.

Hay una torre celular cerca de la base del Batallón de Infantería Ligera
552, en las afueras de la ciudad de Taung Bazar, cerca de donde el
soldado Myo Win Tun dijo que había ayudado a cavar una fosa común. La
base es muy conocida en la zona porque, junto con dos docenas de puestos
de guardia fronterizos, fue atacada por
\href{https://www.nytimes3xbfgragh.onion/es/2017/09/28/espanol/como-la-violencia-en-birmania-radicalizo-a-una-nueva-generacion-rohinya.html}{insurgentes
rohinyás} el 25 de agosto de 2017, lo que impulsó las brutales
operaciones militares contra los civiles rohinyás.

Los refugiados rohinyás que vivían en un pueblo adyacente al campamento
552 dijeron reconocer al soldado Myo Win Tun. Describieron en detalle la
ubicación de dos fosas comunes en esa zona. Los residentes que aún están
en la región, que hablaron con el Times, también dijeron conocer los
sitios de entierro masivo cerca del campamento militar.

NORTE

Base del Batallón de

Infantería Ligera 552

Ubicación de la fosa común

confirmada por los aldeanos

Torre celular

La aldea de Thin Ga Net

Ubicación de otra fosa común

confirmada por los aldeanos

NORTE

Base del Batallón de

Infantería Ligera 552

Ubicación de la fosa común

confirmada por los aldeanos

Torre celular

La aldea de Thin Ga Net

Ubicación de otra fosa común

confirmada por los aldeanos

NORTE

Base del Batallón de

Infantería Ligera 552

Ubicación de la fosa común

confirmada por los aldeanos

Torre celular

La aldea de Thin Ga Net

Ubicación de otra fosa común

confirmada por los aldeanos

Ubicación de la fosa común

confirmada por los aldeanos

La aldea de Thin Ga Net

Base del Batallón de

Infantería Ligera 552

Torre celular

NORTE

Ubicación de otra fosa común

confirmada por los aldeanos

Por Jin Wu/The New York Times·Imagen satelital por Maxar Technologies,
tomada el 25 de septiembre de 2017.

Basha Miya, quien ahora es un refugiado en Bangladés, dijo que su abuela
fue enterrada en una de las fosas comunes de la base, junto con al menos
otras 16 personas del pueblo vecino de Thin Ga Net, conocido en la
lengua rohinyá como Phirkhali.

``Cuando la recuerdo, solo lloro'', dijo. ``Me siento mal por no haber
podido darle un funeral apropiado''.

Después de que los soldados arrojaron los cuerpos en dos tumbas a
orillas de los canales, trajeron excavadoras para cubrir los cadáveres,
según testigos presenciales. El soldado Myo Win Tun dijo que él y otros
enterraron a ocho mujeres, siete niños y 15 hombres en una tumba.

La aldea de Thin Ga Net fue borrada del mapa por el fuego. Hoy, solo un
par de depósitos de agua insinúan que una vez estuvo allí una bulliciosa
aldea rohinyá.

\hypertarget{la-aldea-de-thin-ga-net}{%
\subsubsection{La aldea de Thin Ga Net}\label{la-aldea-de-thin-ga-net}}

\includegraphics{https://static01.graylady3jvrrxbe.onion/packages/flash/multimedia/ICONS/transparent.png}

\includegraphics{https://static01.graylady3jvrrxbe.onion/newsgraphics/2020/09/08/rohingya-2020/assets/images/thin-ga-net-23may2017-2000.jpg}

23 de mayo de 2017

Aldeas rohinyás quemadas

Ubicación de la fosa común

confirmada por los aldeanos

Base del Batallón de

Infantería Ligera 552

Aldeas rohinyás quemadas

Ubicación de la fosa común

confirmada por los aldeanos

Base del Batallón de

Infantería Ligera 552

Aldeas

rohinyás

quemadas

Ubicación de la fosa común

confirmada por los aldeanos

Base del Batallón de

Infantería Ligera 552

Aldeas

rohinyás

quemadas

Ubicación de la fosa común

confirmada por los aldeanos

Base del Batallón de

Infantería Ligera 552

25 de septiembre de 2017

Por Jin Wu/The New York Times·Imagen satelital por Maxar Technologies

Mientras merodeaban por las aldeas alrededor de Taung Bazar, el soldado
Myo Win Tun, de 33 años, parece haber perdido la cuenta de cuántos
rohinyá mataron él y su batallón. ¿Fueron 60 o 70? ¿Quizás más?

``Disparamos indiscriminadamente a todo el mundo'', dijo en un
testimonio en video. ``Disparamos a los hombres musulmanes en la frente
y pateamos los cuerpos al agujero''.

Él también violó a una mujer, dijo.

El soldado Zaw Naing Tun, un exmonje budista, admitió que pasa por una
confusión similar, ya que el asesinato de unos 80 rohinyás se extendió
de horas a días. El soldado dijo que él y otros miembros de su batallón
irrumpieron en 20 aldeas del municipio de Maungdaw Township, incluyendo
Doe Tan, Ngan Chaung, Kyet Yoe Pyin, Zin Paing Nyar y U Shey Kya.

Algunas de estas aldeas fueron quemadas hasta los cimientos. Bashir
Ahmed dijo que los batallones del Tatmadaw entraron en su aldea natal,
Zin Paing Nyar, temprano en la mañana del 26 de agosto de 2017.

\hypertarget{la-aldea-de-zin-paing-nyar}{%
\subsubsection{La aldea de Zin Paing
Nyar}\label{la-aldea-de-zin-paing-nyar}}

\includegraphics{https://static01.graylady3jvrrxbe.onion/packages/flash/multimedia/ICONS/transparent.png}

\includegraphics{https://static01.graylady3jvrrxbe.onion/newsgraphics/2020/09/08/rohingya-2020/assets/images/cropped-zin-paing-nyar-15feb2017-2000.jpg}

15 de febrero de 2017

Aldeas

rohinyás

quemadas

Aldeas

rohinyás

quemadas

Aldeas

rohinyás

quemadas

Aldeas

rohinyás

quemadas

26 de noviembre de 2017

Imagen satelital por Maxar Technologies

\hypertarget{la-aldea-de-doe-tan}{%
\subsubsection{La aldea de Doe Tan}\label{la-aldea-de-doe-tan}}

\includegraphics{https://static01.graylady3jvrrxbe.onion/packages/flash/multimedia/ICONS/transparent.png}

\includegraphics{https://static01.graylady3jvrrxbe.onion/newsgraphics/2020/09/08/rohingya-2020/assets/images/doe-tan-23may2017-2000.jpg}

23 de mayo de 2017

Aldeas

rohinyás

quemadas

Aldeas

rohinyás

quemadas

Aldeas

rohinyás

quemadas

Aldeas

rohinyás

quemadas

9 de enero de 2018

Imagen satelital por Maxar Technologies

``Abrían fuego cuando encontraban a alguien delante de ellos'', dijo.
``Quemaron nuestras casas. No queda nada''.

Más de 30 residentes fueron asesinados en Zin Paing Nyar, según los
testimonios de los sobrevivientes.

El soldado Zaw Naing Tun, de 30 años, dijo que él y otros cuatro
miembros de su batallón mataron a tiros a siete rohinyás en Zin Paing
Nyar. Capturaron a diez hombres desarmados, los ataron con cuerdas, los
mataron y los enterraron en una fosa común al norte de la aldea, dijo en
el testimonio en video.

Hay algunas discrepancias entre el recuento de los soldados y el de los
aldeanos rohinyás. El soldado Myo Win Tun describió la torre celular
como al este de la base 552, cuando en realidad está al suroeste.

Image

Un musulmán rohinyá lee el Corán en una de las pocas mezquitas intactas
del pueblo de Ngan Chaung en el norte del estado de Rakhine el año
pasado.Credit...Adam Dean para The New York Times

Pero la mayoría de los otros detalles están corroborados por
declaraciones de testigos y sobrevivientes. En la aldea de Ngan Chaung,
parte de la cual se salvó de la destrucción, cinco o seis soldados del
Batallón 353 de Infantería Ligera llegaron una tarde a finales de agosto
de 2017 y eligieron a cinco mujeres para violarlas, dijo un residente
que aún vive en la aldea. Él y otros residentes relataron que los
esposos de las mujeres fueron asesinados después.

El soldado Zaw Naing Tun dijo que no cometió violencia sexual porque su
rango era demasiado bajo como para participar. En cambio, se mantuvo de
centinela mientras otros violaban a las mujeres rohinyás.

Los dos soldados que admitieron haber matado rohinyás son miembros de
minorías étnicas en un país donde la persecución de tales grupos está
institucionalizada.

A principios de este año, el par terminó bajo custodia del ejército de
Arakan Army, una milicia étnica de Rakhine que actualmente combate al
Tatmadaw y que grabó sus confesiones en video. Ambos hombres dijeron que
desertaron del Tatmadaw.

La deserción es un problema particular en las zonas de conflicto de
minorías étnicas, dijeron militares con información privilegiada. Se
cree que unos 60 soldados del Batallón de Infantería Ligera 565 han
desertado.

``Fui discriminado racialmente'', dijo en su testimonio en video el
soldado Myo Win Tun, miembro de la etnia shanni, en un raro estallido
sentimental.

Más tarde, describió, en voz baja, cómo su oficial al mando, el coronel
Than Htike, había instruido al batallón que ``exterminara'' a los
rohinyás.

``Estuve involucrado en el asesinato de 30 hombres, mujeres y niños
musulmanes inocentes, enterrados en una fosa común'', dijo, mientas
miraba estoicamente a la cámara.

Image

Los restos de una mezquita en Sabal Khone, un pueblo rohinyá
arrasado.Credit...Adam Dean para The New York Times

Hannah Beech ha sido la jefa de la corresponsalía para el sureste
asiático desde 2017, con sede en Bangkok. Antes de trabajar en The New
York Times, fue reportera de la revista Time durante veinte años, donde
reportó desde Shanghái, Pekín, Bangkok y Hong Kong.
\href{https://twitter.com/hkbeech}{@hkbeech}

\begin{center}\rule{0.5\linewidth}{\linethickness}\end{center}

Advertisement

\protect\hyperlink{after-bottom}{Continue reading the main story}

\hypertarget{site-index}{%
\subsection{Site Index}\label{site-index}}

\hypertarget{site-information-navigation}{%
\subsection{Site Information
Navigation}\label{site-information-navigation}}

\begin{itemize}
\tightlist
\item
  \href{https://help.nytimes3xbfgragh.onion/hc/en-us/articles/115014792127-Copyright-notice}{©~2020~The
  New York Times Company}
\end{itemize}

\begin{itemize}
\tightlist
\item
  \href{https://www.nytco.com/}{NYTCo}
\item
  \href{https://help.nytimes3xbfgragh.onion/hc/en-us/articles/115015385887-Contact-Us}{Contact
  Us}
\item
  \href{https://www.nytco.com/careers/}{Work with us}
\item
  \href{https://nytmediakit.com/}{Advertise}
\item
  \href{http://www.tbrandstudio.com/}{T Brand Studio}
\item
  \href{https://www.nytimes3xbfgragh.onion/privacy/cookie-policy\#how-do-i-manage-trackers}{Your
  Ad Choices}
\item
  \href{https://www.nytimes3xbfgragh.onion/privacy}{Privacy}
\item
  \href{https://help.nytimes3xbfgragh.onion/hc/en-us/articles/115014893428-Terms-of-service}{Terms
  of Service}
\item
  \href{https://help.nytimes3xbfgragh.onion/hc/en-us/articles/115014893968-Terms-of-sale}{Terms
  of Sale}
\item
  \href{https://spiderbites.nytimes3xbfgragh.onion}{Site Map}
\item
  \href{https://help.nytimes3xbfgragh.onion/hc/en-us}{Help}
\item
  \href{https://www.nytimes3xbfgragh.onion/subscription?campaignId=37WXW}{Subscriptions}
\end{itemize}
