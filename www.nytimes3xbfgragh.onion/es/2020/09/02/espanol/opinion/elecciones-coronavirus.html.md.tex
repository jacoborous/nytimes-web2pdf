Sections

SEARCH

\protect\hyperlink{site-content}{Skip to
content}\protect\hyperlink{site-index}{Skip to site index}

\href{https://www.nytimes3xbfgragh.onion/es/section/opinion}{Opinión}

\href{https://myaccount.nytimes3xbfgragh.onion/auth/login?response_type=cookie\&client_id=vi}{}

\href{https://www.nytimes3xbfgragh.onion/section/todayspaper}{Today's
Paper}

\href{/es/section/opinion}{Opinión}\textbar{}Lo que América Latina puede
aprender de las elecciones durante la pandemia

\url{https://nyti.ms/2YZGGZ5}

\begin{itemize}
\item
\item
\item
\item
\item
\end{itemize}

Advertisement

\protect\hyperlink{after-top}{Continue reading the main story}

\href{/es/section/opinion}{Opinión}

Supported by

\protect\hyperlink{after-sponsor}{Continue reading the main story}

Comentario

\hypertarget{lo-que-amuxe9rica-latina-puede-aprender-de-las-elecciones-durante-la-pandemia}{%
\section{Lo que América Latina puede aprender de las elecciones durante
la
pandemia}\label{lo-que-amuxe9rica-latina-puede-aprender-de-las-elecciones-durante-la-pandemia}}

Este es el momento para modernizar nuestros sistemas electorales.
Debemos introducir mecanismos especiales de votación que no erosionen la
confianza en la integridad de los comicios.

\includegraphics{https://static01.graylady3jvrrxbe.onion/images/2020/09/02/multimedia/02Casas-ES/merlin_174266055_c9045fb5-e91c-4d32-8fac-8733836285b3-articleLarge.jpg?quality=75\&auto=webp\&disable=upscale}

Por Kevin Casas Z.

Es secretario general de IDEA Internacional.

\begin{itemize}
\item
  2 de septiembre de 2020
\item
  \begin{itemize}
  \item
  \item
  \item
  \item
  \item
  \end{itemize}
\end{itemize}

\href{https://www.nytimes3xbfgragh.onion/newsletters/el-times}{Regístrate
para recibir nuestro boletín} con lo mejor de The New York Times.

\begin{center}\rule{0.5\linewidth}{\linethickness}\end{center}

ESTOCOLMO --- La pandemia de la COVID-19 ha trastocado el panorama
electoral en América Latina. Al menos
\href{https://www.idea.int/es/news-media/news/es/elecciones-y-covid-19-lecciones-de-am\%C3\%A9rica-latina}{12
elecciones programadas} en la región han sido pospuestas, incluyendo los
comicios presidenciales en Bolivia y República Dominicana ---finalmente
celebrados el 5 de julio--- y el plebiscito constitucional en Chile.

América Latina no está sola: de acuerdo con los datos del Instituto
Internacional para la Democracia y Asistencia Electoral (IDEA
Internacional), más de 70 países y jurisdicciones subnacionales en el
mundo han decidido postergar eventos electorales de todo tipo y más de
50 los han llevado a cabo en condiciones de pandemia.

Pero en la región, que ha batallado por décadas para dar credibilidad a
sus elecciones, debemos estar más atentos que nunca y preparar los
procesos electorales para enfrentar los desafíos planteados por la
crisis del coronavirus.

En los próximos 25 meses habrá nueve comicios presidenciales en América
Latina. Muchas de estas elecciones se llevarán a cabo en un contexto aún
definido por la emergencia de salud. Preservar la capacidad de las
democracias latinoamericanas para celebrar elecciones exitosas es vital:
las elecciones son con frecuencia la única válvula de escape para
sistemas políticos sometidos a las extraordinarias presiones derivadas
de una crisis sanitaria y económica sin precedentes. Prestar atención a
las lecciones que arroja el acervo de experiencia electoral acumulado en
los últimos meses es urgente.

La primera de esas lecciones es la importancia del consenso político
para sustentar las decisiones que se tomen sobre el calendario y los
procedimientos electorales. La decisión de celebrar o posponer
elecciones en medio de una pandemia involucra desde consideraciones de
salud pública, hasta la potencial lesión a la legitimidad derivada de
una baja participación electoral, pasando por el
\href{https://www.idea.int/sites/default/files/publications/managing-elections-during-covid-19-pandemic.pdf}{riesgo
de crisis constitucionales}. Tampoco son ligeras las determinaciones
sobre la habilitación de procedimientos alternativos para emitir el
sufragio y las precauciones sanitarias que rodeen la votación. Cada una
de estas decisiones es una fuente potencial de fricción política y de
esfuerzos para socavar la legitimidad del resultado electoral.

Por ello es esencial que todas estas decisiones se sustenten en acuerdos
políticos amplios. Fue así en el caso de la exitosa elección
parlamentaria de abril en
\href{https://www.france24.com/es/20200415-corea-sur-elecciones-moon-jaein-coronavirus}{Corea
del Sur}. Pero también hay errores de los que debemos aprender. En
\href{https://www.elmundo.es/internacional/2020/07/12/5f0b616521efa0ef7b8b469c.html}{Polonia},
la aprobación atropellada y controversial de reformas para llevar a cabo
la elección presidencial en mayo mediante voto postal generó un forcejeo
político que llevó a la posposición de los comicios, finalmente
celebrados en junio, mediante votación presencial y postal.

En nuestra región, el reciente caso dominicano se acerca a la
experiencia coreana y contrasta con la azarosa ruta seguida para definir
una nueva fecha para la elección en Bolivia (recientemente fijada para
el
\href{https://cnnespanol.cnn.com/2020/08/13/el-tse-de-bolivia-anuncia-el-18-de-octubre-como-fecha-definitiva-de-las-elecciones-generales/}{18
de octubre}), sugerente de la fragilidad de los acuerdos sobre las
reglas del juego electoral en ese país.

La segunda lección tiene que ver con la necesidad de habilitar diversas
modalidades para emitir el sufragio. El método tradicional de aglomerar
a millones de electores, miembros de mesa, representantes partidarios y
observadores en centros de votación a lo largo de pocas horas presenta
obvios riesgos para la salud pública en medio de una pandemia. Sin
embargo, desde el sufragio anticipado y postal hasta las incipientes
modalidades del voto en línea, la adopción de mecanismos especiales de
emisión del voto por parte de las legislaciones latinoamericanas marcha
muy por detrás del grado de desarrollo de otros aspectos de los procesos
electorales en la región. No hay un solo país de América Latina que hoy
contemple en su legislación la posibilidad del voto postal para los
residentes en el país o de extender por varios días la jornada
electoral.

La renuencia para adoptar esos mecanismos tiene relación con la
extendida desconfianza que afecta, con pocas excepciones, a la
institucionalidad electoral en la región.

De acuerdo con cifras de Latinobarómetro, solo
\href{https://www.latinobarometro.org/latContents.jsp}{28 por ciento} de
los latinoamericanos tenía alguna o mucha confianza en la institución
electoral de su país en 2018, una caída de 23 puntos desde 2006. En
América Latina la introducción de mecanismos especiales para votar es
vista menos en función de la participación electoral y más a través del
lente de los peligros que implique para la integridad de los comicios.
Pero este es el momento para adoptar los pasos necesarios para
modernizar nuestros sistemas electorales.

La pandemia ha hecho impostergable generar garantías sólidas de que los
mecanismos especiales de emisión del voto no pondrán en riesgo la
transparencia e integridad de los resultados. Porque lo cierto es que la
ausencia de esas modalidades se paga en participación electoral. En la
República Dominicana, el
\href{https://www.elpais.cr/2020/07/08/alta-abstencion-en-comicios-de-la-republica-dominicana/}{abstencionismo}
en la reciente elección presidencial aumentó 14 puntos con respecto a la
anterior. En Corea del Sur, en cambio, la elección de abril resultó en
el más alto nivel de participación electoral en casi tres décadas, en
parte por la habilitación del voto postal y de jornadas electorales
adicionales, durante las que votó más de una cuarta parte del
electorado. Nada de eso fue improvisado: ya estaba previsto por la
legislación coreana desde mucho antes de la elección.

Todo esto conduce a un tercer punto: los recursos. La pandemia obliga a
adoptar medidas que reduzcan los riesgos de contagio, que van desde la
disponibilidad de mascarillas y demás materiales de protección, hasta la
apertura de más centros de votación y la extensión de la jornada
electoral. En particular, prolongar la jornada electoral ---que permite
controlar el flujo de votantes a las urnas--- luce hoy como una opción
viable para casi cualquier país de América Latina. Si se quieren
elecciones adecuadas hay que estar dispuesto a dar mayores recursos
financieros y humanos a las autoridades electorales.

Pero nada de esto garantiza el éxito, pues hay un cuarto factor,
crucial: al igual que tantas otras cosas, las elecciones exitosas
dependen, en última instancia, del control de la pandemia.

Celebrar elecciones en cuarentena es imposible. Aún más, la evidencia
muestra que el momento de la curva de contagios en que se encuentre un
país
\href{https://www.idea.int/news-media/news/elections-pandemic-lessons-asia}{impacta
decisivamente la participación electoral}. El aumento de esta en Corea
del Sur debe mucho al hecho de que la elección se celebró cuando el
número de contagios tenía más de un mes de haber sido estabilizado. Por
el contrario, elecciones celebradas en medio de brotes crecientes, por
ejemplo los comicios parlamentarios de Irán en febrero o las elecciones
municipales de Francia en marzo, experimentaron
\href{https://www.france24.com/es/20200315-participacion-primera-vuelta-elecciones-municipales-francia}{caídas
aparatosas en la afluencia de votantes}.

Estos son apenas algunos aprendizajes que los países latinoamericanos,
confrontados con un ciclo electoral en condiciones complejas, harían
bien en tomar en cuenta. La celebración de elecciones periódicas, libres
y transparentes es el área más exitosa del desarrollo democrático de
América Latina durante las últimas cuatro décadas. Eso es un motivo de
esperanza, pero no lo debe ser de complacencia: proteger ese avance
nunca ha sido más urgente.

Kevin Casas Z. es politólogo y fue vicepresidente de Costa Rica. Desde
2019 es secretario general de IDEA Internacional.

Advertisement

\protect\hyperlink{after-bottom}{Continue reading the main story}

\hypertarget{site-index}{%
\subsection{Site Index}\label{site-index}}

\hypertarget{site-information-navigation}{%
\subsection{Site Information
Navigation}\label{site-information-navigation}}

\begin{itemize}
\tightlist
\item
  \href{https://help.nytimes3xbfgragh.onion/hc/en-us/articles/115014792127-Copyright-notice}{©~2020~The
  New York Times Company}
\end{itemize}

\begin{itemize}
\tightlist
\item
  \href{https://www.nytco.com/}{NYTCo}
\item
  \href{https://help.nytimes3xbfgragh.onion/hc/en-us/articles/115015385887-Contact-Us}{Contact
  Us}
\item
  \href{https://www.nytco.com/careers/}{Work with us}
\item
  \href{https://nytmediakit.com/}{Advertise}
\item
  \href{http://www.tbrandstudio.com/}{T Brand Studio}
\item
  \href{https://www.nytimes3xbfgragh.onion/privacy/cookie-policy\#how-do-i-manage-trackers}{Your
  Ad Choices}
\item
  \href{https://www.nytimes3xbfgragh.onion/privacy}{Privacy}
\item
  \href{https://help.nytimes3xbfgragh.onion/hc/en-us/articles/115014893428-Terms-of-service}{Terms
  of Service}
\item
  \href{https://help.nytimes3xbfgragh.onion/hc/en-us/articles/115014893968-Terms-of-sale}{Terms
  of Sale}
\item
  \href{https://spiderbites.nytimes3xbfgragh.onion}{Site Map}
\item
  \href{https://help.nytimes3xbfgragh.onion/hc/en-us}{Help}
\item
  \href{https://www.nytimes3xbfgragh.onion/subscription?campaignId=37WXW}{Subscriptions}
\end{itemize}
