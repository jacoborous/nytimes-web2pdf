Sections

SEARCH

\protect\hyperlink{site-content}{Skip to
content}\protect\hyperlink{site-index}{Skip to site index}

\href{https://www.nytimes3xbfgragh.onion/es/section/negocios}{Negocios}

\href{https://myaccount.nytimes3xbfgragh.onion/auth/login?response_type=cookie\&client_id=vi}{}

\href{https://www.nytimes3xbfgragh.onion/section/todayspaper}{Today's
Paper}

\href{/es/section/negocios}{Negocios}\textbar{}Trump suspendió los
desalojos: todo lo que debes saber

\url{https://nyti.ms/3jGWDLO}

\begin{itemize}
\item
\item
\item
\item
\item
\item
\end{itemize}

\hypertarget{el-brote-de-coronavirus}{%
\subsubsection{\texorpdfstring{\href{https://www.nytimes3xbfgragh.onion/es/spotlight/coronavirus?name=styln-coronavirus-es\&region=TOP_BANNER\&block=storyline_menu_recirc\&action=click\&pgtype=Article\&impression_id=3017e560-f2bb-11ea-8c2a-475020e0db7d\&variant=undefined}{El
brote de
coronavirus}}{El brote de coronavirus}}\label{el-brote-de-coronavirus}}

\begin{itemize}
\tightlist
\item
  \href{https://www.nytimes3xbfgragh.onion/es/interactive/2020/08/06/espanol/ciencia-y-tecnologia/tengo-covid-19-sintomas.html?name=styln-coronavirus-es\&region=TOP_BANNER\&block=storyline_menu_recirc\&action=click\&pgtype=Article\&impression_id=3017e561-f2bb-11ea-8c2a-475020e0db7d\&variant=undefined}{Síntomas}
\item
  \href{https://www.nytimes3xbfgragh.onion/es/2020/09/02/espanol/ciencia-y-tecnologia/vacunas-experimentales-coronavirus.html?name=styln-coronavirus-es\&region=TOP_BANNER\&block=storyline_menu_recirc\&action=click\&pgtype=Article\&impression_id=3017e562-f2bb-11ea-8c2a-475020e0db7d\&variant=undefined}{Vacunas
  experimentales}
\item
  \href{https://www.nytimes3xbfgragh.onion/es/2020/08/31/espanol/mundo/rebrote-espana.html?name=styln-coronavirus-es\&region=TOP_BANNER\&block=storyline_menu_recirc\&action=click\&pgtype=Article\&impression_id=30180c70-f2bb-11ea-8c2a-475020e0db7d\&variant=undefined}{Rebrote
  en España}
\item
  \href{https://www.nytimes3xbfgragh.onion/es/2020/09/02/espanol/negocios/desalojos-trump.html?name=styln-coronavirus-es\&region=TOP_BANNER\&block=storyline_menu_recirc\&action=click\&pgtype=Article\&impression_id=30180c71-f2bb-11ea-8c2a-475020e0db7d\&variant=undefined}{Moratoria
  a los desalojos}
\item
  \href{https://www.nytimes3xbfgragh.onion/es/2020/08/26/espanol/ciencia-y-tecnologia/coronavirus-afecta-hombres.html?name=styln-coronavirus-es\&region=TOP_BANNER\&block=storyline_menu_recirc\&action=click\&pgtype=Article\&impression_id=30180c72-f2bb-11ea-8c2a-475020e0db7d\&variant=undefined}{El
  impacto en los hombres}
\end{itemize}

Advertisement

\protect\hyperlink{after-top}{Continue reading the main story}

Supported by

\protect\hyperlink{after-sponsor}{Continue reading the main story}

\hypertarget{trump-suspendiuxf3-los-desalojos-todo-lo-que-debes-saber}{%
\section{Trump suspendió los desalojos: todo lo que debes
saber}\label{trump-suspendiuxf3-los-desalojos-todo-lo-que-debes-saber}}

Una orden del gobierno estadounidense podría permitir que muchos
inquilinos que no pueden pagar sus arriendos eviten el desalojo hasta el
31 de diciembre. Aquí respondemos a las preguntas de los arrendatarios.

\includegraphics{https://static01.graylady3jvrrxbe.onion/images/2020/09/02/business/02virus-evictionfaq-ES-00/merlin_133423170_63034e6f-846e-497e-b581-c7558039fe20-articleLarge.jpg?quality=75\&auto=webp\&disable=upscale}

\href{https://www.nytimes3xbfgragh.onion/by/ron-lieber}{\includegraphics{https://static01.graylady3jvrrxbe.onion/images/2018/10/22/multimedia/author-ron-lieber/author-ron-lieber-thumbLarge.png}}

Por \href{https://www.nytimes3xbfgragh.onion/by/ron-lieber}{Ron Lieber}

\begin{itemize}
\item
  2 de septiembre de 2020
\item
  \begin{itemize}
  \item
  \item
  \item
  \item
  \item
  \item
  \end{itemize}
\end{itemize}

\href{https://www.nytimes3xbfgragh.onion/2020/09/02/your-money/eviction-moratorium-covid.html}{Read
in English}

\href{https://www.nytimes3xbfgragh.onion/newsletters/el-times}{Regístrate
para recibir nuestro boletín} con lo mejor de The New York Times.

\begin{center}\rule{0.5\linewidth}{\linethickness}\end{center}

El gobierno de Donald Trump ha anunciado una
\href{https://s3.amazonaws.com/public-inspection.federalregister.gov/2020-19654.pdf}{orden}
para suspender la posibilidad de desalojo de millones de inquilinos que
han pasado penurias económicas debido a la pandemia del coronavirus. Los
\href{https://www.nytimes3xbfgragh.onion/2020/09/01/business/eviction-moratorium-order.html}{Centros
para el Control y la Prevención de Enfermedades} dijeron que la orden
---que por ley pueden tomar--- era una medida de emergencia.

Aquí respondemos a las preguntas que los inquilinos pueden tener sobre
la orden, que es más amplia que la moratoria (ya expirada) que esta
primavera formó parte del paquete de alivio del virus. Añadiremos nuevos
elementos a esta lista a medida que conozcamos más detalles. Por favor,
envía tus preguntas por correo electrónico a
\href{mailto:hubforhelp@NYTimes.com}{\nolinkurl{hubforhelp@NYTimes.com}}
en inglés o a
\href{mailto:comentarios@NYTimes.com}{\nolinkurl{comentarios@NYTimes.com}}
en español.

\textbf{¿Quién puede beneficiarse de la moratoria?}

Debes cumplir con cinco condiciones.

\begin{itemize}
\item
  Tienes que haber hecho tu ``mejor esfuerzo'' para obtener cualquiera y
  todas las formas de ayuda de alquiler que otorga el gobierno.
\item
  No puedes ``anticipar'' que vas a ganar más de 99.000 dólares en 2020
  o 198.000 si estás casado y presentas una declaración de impuestos
  conjunta. Si no calificas de esta manera, aún podrías ser elegible si
  no necesitas declarar ningún ingreso al gobierno federal en 2019 o si
  recibiste un cheque de estímulo a principios de este año.
\item
  Debes estar pasando por una pérdida ``sustancial'' de ingresos
  familiares, un despido o haber pagado de tu bolsillo gastos médicos
  ``extraordinarios'' (que la orden define como cualquier gasto no
  reembolsado que pueda exceder el 7,5 por ciento de tu ingreso bruto
  ajustado este año).
\item
  Debes estar haciendo todo lo posible para hacer pagos parciales
  ``oportunos'' que se acerquen lo más posible al monto total adeudado
  según ``las circunstancias lo permitan'', teniendo en cuenta otros
  gastos no discrecionales.
\item
  El desalojo ``probablemente'' ocasionaría que te quedes sin casa o
  tengas que mudarte a un lugar que es más caro o donde podrías
  enfermarte por estar cerca de otros.
\end{itemize}

\textbf{Muchos de estos aspectos son bastante subjetivos. Si es una
decisión difícil, ¿quién decide?}

Los propietarios que no están de acuerdo con las autoevaluaciones de los
inquilinos podrían intentar desalojar a quienes no pagan y retarlos a
que se defiendan legalmente. Entonces, podría ser un juez del tribunal
de vivienda quien decida si un inquilino tiene derecho a la moratoria o
si el propietario puede, de hecho, desalojar.

\textbf{¿Cómo le demuestro al propietario que soy elegible?}

La orden de los CDC hace referencia a una declaración que los inquilinos
deben redactar y luego proporciona un ejemplo hacia el final
\href{https://s3.amazonaws.com/public-inspection.federalregister.gov/2020-19654.pdf}{del
documento}.

\textbf{El formulario de declaración no establece que debo demostrarle
mis dificultades al propietario. ¿Debo adjuntar estados de mis cuentas
bancarias u otros documentos?}

No, no tiene que adjuntar eso en la declaración, al menos no al
principio. La forma en que está redactada la orden significa que no es
necesario especificar esos detalles en su declaración, dijo Emily
Benfer, profesora visitante de derecho en la Universidad de Wake Forest.
Sin embargo, según los altos funcionarios gubernamentales que ayudaron a
redactar la orden, si el arrendador cuestiona su evaluación inicial,
debe proporcionar detalles ``razonables'' para probar su elegibilidad.

\textbf{¿Quién debe hacer una declaración?}

La orden dice que cada adulto que está en el contrato de arrendamiento
debe redactar y firmar su propia declaración.

\textbf{Tengo un compañero de piso. ¿Qué pasa si uno de nosotros está
bajo el tope de ingresos y el otro no?}

La orden no menciona directamente a los compañeros de departamento, pero
los funcionarios aclararon que el límite de ingresos era de 99.000
dólares para quienes cohabitan la vivienda. Si una sola persona no puede
pagar en su totalidad, los detalles de qué debe pagar cada quien
dependerán de los términos del contrato de arrendamiento, de cualquier
acuerdo escrito que exista entre usted y su compañero de departamento y
las leyes estatales o locales aplicables.

\textbf{Estoy en una situación bastante mala. ¿Puedo exagerar un poco la
verdad?}

No deberías. La orden señala que la declaración es ``un testimonio
jurado, lo que significa que puedes ser procesado, ir a la cárcel o
pagar una multa si mientes, engañas u omites información importante''.

\textbf{¿Qué hago con las declaraciones una vez que están hechas?}

Envíalas por correo electrónico o mándaselas al propietario de manera
que puedas obtener pruebas de que las recibió. Así no habrá dudas de que
cumpliste con lo que debías hacer. Asegúrate de quedarte con una copia.

\textbf{¿Y luego qué?}

Sigue pagando todo lo que puedas. De lo contrario, te arriesgas a no
pasar la prueba de la elegibilidad, que dice que debes tratar de hacer
pagos parciales en la medida de tus posibilidades.

\textbf{¿El propietario puede desahuciarme por otras razones que no sean
la falta de pago?}

Sí. Todas las reglas usuales sobre comportamiento criminal o
interrupciones o destrucción de la propiedad aún se aplican. Y es
posible que el propietario busque con ahínco alguna otra razón para
iniciar el proceso de desalojo, por lo que es prudente seguir todos los
términos del contrato de arrendamiento, así como cualquier otra norma de
construcción o propiedad.

Amy Woolard, abogada y coordinadora de políticas del Centro de Justicia
y Asistencia Legal en Charlottesville, Virginia, advirtió sobre un tema
que ella y sus colegas ven frecuentemente en los casos de desalojo: se
trata de las personas que no están en el contrato de arrendamiento y,
sin embargo, viven en la propiedad. Esto podría ser un problema si
tienes invitados como, por ejemplo, un miembro de la familia que ha sido
desalojado de otro lugar.

\textbf{¿Se acumularán intereses o penalidades?}

La orden no prohíbe a los propietarios cobrar honorarios, penalidades o
intereses. Tampoco pone ninguna restricción en cuanto a lo que pueden
llegar. Revisa tu contrato de arrendamiento para ver si hay alguna
disposición sobre cómo puede funcionar esto.

\textbf{¿Tendré que pagar todo lo que debo de una sola vez en enero?}

Puede que sí. La orden menciona específicamente esta posibilidad. Y el
Consejo Nacional de Viviendas de Arriendo, un grupo comercial que agrupa
a los propietarios que poseen propiedades unifamiliares, dijo en un
comunicado el miércoles que ``una vez que expire la moratoria, los
inquilinos deberán pagar el alquiler atrasado durante varios meses''.

\textbf{¿La orden suspende los desalojos que ya están en proceso?}

Sí, según los funcionarios gubernamentales.

\textbf{¿La orden se aplica a todos los propietarios y a todos los
inquilinos de residencias del país?}

No. Aparte de los topes de ingresos, es posible que se apliquen las
reglas locales. Si estás en un estado, territorio o área tribal que
\href{https://evictionlab.org/covid-eviction-policies/}{ya tiene}una
moratoria que proporciona el mismo o mejor nivel de protección, entonces
esa acción más local tomará su lugar. Las jurisdicciones locales también
siguen siendo libres de imponer restricciones más fuertes que el orden
federal. La
\href{https://www.gov.ca.gov/2020/08/31/governor-newsom-signs-statewide-covid-19-tenant-and-landlord-protection-legislation/}{moratoria}
de California se extiende hasta finales de enero, por ejemplo.

La moratoria federal no se aplica en Samoa Estadounidense, aunque lo
hará si reporta sus primeros casos de coronavirus.

\textbf{Ahora mismo vivo en un motel. ¿La orden se aplica a esas
propiedades?}

No. La orden excluye específicamente a los hoteles y moteles.

\textbf{¿Qué pasa con los alquileres en Airbnb y otras propiedades
similares?}

La orden excluye cualquier ``casa de huéspedes alquilada a un huésped
temporal o inquilino de temporada según lo definido por las leyes del
estado, territorio, tribu o jurisdicción local''.

\textbf{¿Y si, de todos modos, mi casero me envía una notificación de
desalojo?}

Busca un abogado. Puedes buscar una oficina de asistencia legal gratuita
o de bajo costo cerca de ti a través del
\href{https://www.lsc.gov/what-legal-aid/find-legal-aid}{mapa} de la
Corporación de Servicios Legales. Just Shelter, un grupo de defensa de
inquilinos, también ofrece
\href{https://justshelter.org/community-resources/}{información} sobre
organizaciones locales que pueden ayudar a los arrendatarios.

\textbf{¿Especifica la orden la magnitud de las penalidades a las que
pueden estar sujetos los propietarios?}

Sí. Un propietario individual puede estar sujeto a una multa de hasta
100.000 dólares si no se produce ninguna muerte (digamos, de alguien que
se enferma después del desalojo) a causa de la violación, o un año de
cárcel o ambas cosas. Si se produce un fallecimiento, la multa se eleva
a una cifra que no supera los 250.000 dólares. Si se trata de una
organización, las multas van desde los 200.000 hasta los 500.000
dólares.

\textbf{¿La orden es legal?}

La Casa Blanca y los Centros para el Control y la Prevención de
Enfermedades así lo creen. Es posible que los grupos de la industria de
propietarios u otros demanden para detenerla, en ese caso la decisión le
corresponderá a los tribunales.

¿Algunos jueces de vivienda locales podrían ignorar la orden?

Los abogados dicen que no les sorprendería que eso suceda en las
jurisdicciones más pequeñas. ``Entonces dependería del inquilino reunir
los recursos necesarios para intentar presentar una demanda en un
tribunal federal o buscar una orden judicial de otra autoridad en el
sistema judicial de su estado'', dijo Rebecca Maurer, una abogada de
Cleveland.

\textbf{¿Cuándo entra en vigor la orden y cuánto tiempo dura?}

Entra en vigor en cuanto se publica en el Registro Federal. La orden
dice que eso sucederá el 4 de septiembre. La medida se aplicará hasta el
31 de diciembre, y es posible que se prorrogue.

\textbf{Estoy confundido por las diversas órdenes locales, estatales y
federales. ¿Esta es la última?}

Tal vez no. El Congreso podría aprobar un nuevo paquete de ayuda que
remplazaría a esta orden.

Ron Lieber ha sido el columnista de
\href{https://www.nytimes3xbfgragh.onion/column/your-money}{\emph{Your
Money}} ** desde 2008 y es el autor del libro
\href{http://ronlieber.com/books/what-to-pay-for-college/}{\emph{The
Price You Pay for College}}, que está por salir.
\href{https://twitter.com/ronlieber}{@ronlieber} •
\href{https://www.facebookcorewwwi.onion/facebookcorewwwi.onion/ronlieberauthor}{Facebook}

\begin{center}\rule{0.5\linewidth}{\linethickness}\end{center}

Advertisement

\protect\hyperlink{after-bottom}{Continue reading the main story}

\hypertarget{site-index}{%
\subsection{Site Index}\label{site-index}}

\hypertarget{site-information-navigation}{%
\subsection{Site Information
Navigation}\label{site-information-navigation}}

\begin{itemize}
\tightlist
\item
  \href{https://help.nytimes3xbfgragh.onion/hc/en-us/articles/115014792127-Copyright-notice}{©~2020~The
  New York Times Company}
\end{itemize}

\begin{itemize}
\tightlist
\item
  \href{https://www.nytco.com/}{NYTCo}
\item
  \href{https://help.nytimes3xbfgragh.onion/hc/en-us/articles/115015385887-Contact-Us}{Contact
  Us}
\item
  \href{https://www.nytco.com/careers/}{Work with us}
\item
  \href{https://nytmediakit.com/}{Advertise}
\item
  \href{http://www.tbrandstudio.com/}{T Brand Studio}
\item
  \href{https://www.nytimes3xbfgragh.onion/privacy/cookie-policy\#how-do-i-manage-trackers}{Your
  Ad Choices}
\item
  \href{https://www.nytimes3xbfgragh.onion/privacy}{Privacy}
\item
  \href{https://help.nytimes3xbfgragh.onion/hc/en-us/articles/115014893428-Terms-of-service}{Terms
  of Service}
\item
  \href{https://help.nytimes3xbfgragh.onion/hc/en-us/articles/115014893968-Terms-of-sale}{Terms
  of Sale}
\item
  \href{https://spiderbites.nytimes3xbfgragh.onion}{Site Map}
\item
  \href{https://help.nytimes3xbfgragh.onion/hc/en-us}{Help}
\item
  \href{https://www.nytimes3xbfgragh.onion/subscription?campaignId=37WXW}{Subscriptions}
\end{itemize}
