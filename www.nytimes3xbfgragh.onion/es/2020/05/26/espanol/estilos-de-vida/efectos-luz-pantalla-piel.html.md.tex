Sections

SEARCH

\protect\hyperlink{site-content}{Skip to
content}\protect\hyperlink{site-index}{Skip to site index}

\href{https://www.nytimes3xbfgragh.onion/es/section/estilos-de-vida}{Estilos
de Vida}

\href{https://myaccount.nytimes3xbfgragh.onion/auth/login?response_type=cookie\&client_id=vi}{}

\href{https://www.nytimes3xbfgragh.onion/section/todayspaper}{Today's
Paper}

\href{/es/section/estilos-de-vida}{Estilos de Vida}\textbar{}¿Sabes qué
le hacen a tu piel esas horas frente a la pantalla?

\url{https://nyti.ms/3gnXekp}

\begin{itemize}
\item
\item
\item
\item
\item
\end{itemize}

Advertisement

\protect\hyperlink{after-top}{Continue reading the main story}

Supported by

\protect\hyperlink{after-sponsor}{Continue reading the main story}

\hypertarget{sabes-quuxe9-le-hacen-a-tu-piel-esas-horas-frente-a-la-pantalla}{%
\section{¿Sabes qué le hacen a tu piel esas horas frente a la
pantalla?}\label{sabes-quuxe9-le-hacen-a-tu-piel-esas-horas-frente-a-la-pantalla}}

Preguntamos a los expertos cuáles son los riesgos de las luces dentro de
casa y cómo podemos protegernos.

\includegraphics{https://static01.graylady3jvrrxbe.onion/images/2020/05/19/fashion/19SKIN-BLUELIGHT/19SKIN-BLUELIGHT-articleLarge.jpg?quality=75\&auto=webp\&disable=upscale}

\href{https://www.nytimes3xbfgragh.onion/by/crystal-martin}{\includegraphics{https://static01.graylady3jvrrxbe.onion/images/2019/03/01/multimedia/author-crystal-martin/author-crystal-martin-thumbLarge.png}}

Por \href{https://www.nytimes3xbfgragh.onion/by/crystal-martin}{Crystal
Martin}

\begin{itemize}
\item
  26 de mayo de 2020
\item
  \begin{itemize}
  \item
  \item
  \item
  \item
  \item
  \end{itemize}
\end{itemize}

\href{https://www.nytimes3xbfgragh.onion/2020/05/20/style/skin-damage-blue-light-what-is-all-of-that-screen-time-doing-to-your-skin.html}{Read
in English}

\href{https://www.nytimes3xbfgragh.onion/newsletters/el-times}{Regístrate
para recibir nuestro boletín} con lo mejor de The New York Times.

\begin{center}\rule{0.5\linewidth}{\linethickness}\end{center}

Quizá últimamente has escuchado más sobre los peligros de la luz azul,
porque es más probable que estemos en casa y en línea. Nuestras
computadoras portátiles, celulares, tabletas, televisores e incluso los
focos de luz LED son fuentes de luz azul. Y ahora que estamos pegados a
nuestros dispositivos, ¿nos estamos empapando de esa luz? ¿Deberíamos
estar más preocupados por el daño que sufre nuestra piel?

Esto es lo que sabemos: en comparación con los peligros bien
comprendidos de la luz ultravioleta (el envejecimiento de la piel y el
cáncer), la ciencia no conoce claramente los efectos que las fuentes de
luz azul en interiores tienen sobre la piel. Pueden provocar
hiperpigmentación y envejecimiento prematuro, pero el resto ---cuál es
la dosis problemática, por ejemplo--- ya se debatía desde antes de que
quedáramos encerrados en casa.

Consultamos a expertos en luz azul y dermatología para ayudar a explicar
los verdaderos riesgos.

\hypertarget{quuxe9-es-la-luz-azul}{%
\subsection{¿Qué es la luz azul?}\label{quuxe9-es-la-luz-azul}}

Cuando pensamos en los efectos nocivos de la luz, generalmente estamos
pensando en la luz ultravioleta (UV), que es invisible. Pero podemos ver
la luz azul. Quizá la percibas como una luz blanca con un tono frío
(como la de los focos LED), o tal vez no veas el color azul para nada
porque tus fuentes de luz en interiores están emitiendo longitudes de
onda variables que se combinan para crear los colores que percibes.

Aunque los efectos en la piel de la luz azul aún no se entienden del
todo, la luz es una preocupación importante de salud debido a sus otros
riesgos. ``La luz azul daña la retina y reduce la liberación de
melatonina, así que interrumpe tu ciclo de sueño'', dijo Michelle Henry,
dermatóloga neoyorquina.

Desde luego, la proximidad es un factor cuando hablamos de los daños.
``Recibirás menos luz azul proveniente de tu televisión que de tu
computadora porque está más lejos'', dijo Henry. ``Y más luz azul de tu
celular que de tu computadora porque tu celular está muy cerca de tu
rostro''.

\hypertarget{cuxf3mo-se-dauxf1a-mi-piel-debido-a-la-luz-azul}{%
\subsection{¿Cómo se daña mi piel debido a la luz
azul?}\label{cuxf3mo-se-dauxf1a-mi-piel-debido-a-la-luz-azul}}

Aunque la luz ultravioleta daña directamente el ADN de las células, la
luz azul destruye el colágeno mediante el estrés oxidativo. Un químico
en la piel llamado flavina absorbe la luz azul. La reacción que tiene
lugar durante la absorción produce moléculas inestables de oxígeno
(radicales libres) que dañan la piel.

``Entran y básicamente hacen agujeros en tu colágeno'', explicó Henry.

La exposición a la luz azul es más problemática para la tez oscura. En
\href{https://www.sciencedirect.com/science/article/pii/S0022202X15349307}{un
estudio de 2010 publicado en The Journal of Investigative Dermatology},
se mostró que causa hiperpigmentación en las pieles de tono medio a
oscuro, mientras que deja la tez más clara relativamente intacta.

La comunidad médica clasifica el color de la piel con base en la manera
en que reacciona a la luz UV. El tipo 1 es el color más claro con la
mayor sensibilidad UV. ``Sería una tez como la de Nicole Kidman y Conan
O'Brien'', dijo Mathew M. Avram, director del Centro de Cosmética y
Dermatología Láser del Hospital General de Massachusetts en Boston. La
escala llega al tipo 6, que es la tez más oscura y con menos
probabilidades de quemarse.

En el estudio de 2010, la piel tipo 2 fue expuesta a la luz azul pero no
desarrolló pigmentación. La piel de color oscureció, y ese
oscurecimiento continuó durante un par de semanas.

``Hay algo en la pigmentación de los tipos 4, 5 y 6 que reacciona de
manera distinta que en pacientes con piel clara'', dijo Avram. ``Debería
haber más estudios a gran escala que analicen este problema porque la
pigmentación es una de las preocupaciones más grandes que tienen los
pacientes y el síntoma cuyo tratamiento genera menos satisfacción''.

\hypertarget{pero-no-se-usa-la-luz-azul-para-tratar-el-acnuxe9}{%
\subsection{¿Pero no se usa la luz azul para tratar el
acné?}\label{pero-no-se-usa-la-luz-azul-para-tratar-el-acnuxe9}}

Sí, las lámparas de luz azul se usan como tratamiento para el acné y
para las lesiones precancerosas. ``Daña la piel, pero, por otro lado,
puede ayudar a tratar el acné'', dijo Avram. ``También puede mejorar tu
estado de ánimo y tu memoria. Así que no es tan fácil como decir que es
`buena' o `mala'''.

\hypertarget{cuxf3mo-puedo-evitar-el-dauxf1o-en-la-piel}{%
\subsection{¿Cómo puedo evitar el daño en la
piel?}\label{cuxf3mo-puedo-evitar-el-dauxf1o-en-la-piel}}

La intervención más sencilla es limitar la cantidad de luz azul que
emiten tus dispositivos. Los productos de Apple tienen un ``modo
nocturno'' que genera un tono más cálido en las pantallas. Cambia tus
focos LED estándar por versiones que emitan menos luz azul.

Los protectores solares minerales con óxido de hierro son el estándar de
referencia en la protección de luz azul. Se ha demostrado que el óxido
de hierro
\href{https://www.ncbi.nlm.nih.gov/pmc/articles/PMC6718061/}{protege más
de la luz visible} que el óxido de zinc y el dióxido de titanio.

``Un buen truco es comprar bloqueadores solares con color, que
generalmente tienen óxido de hierro'', dijo Henry. El bloqueador
compacto
\href{https://skinbetter.com/products/sunbetter-tone-smart-spf-68-sunscreen-compact/}{Sunbetter
Tone Smart SPF 68, con un precio de 55 dólares, de Skinbetter Science},
es uno de esos filtros solares minerales. La fórmula incluye óxido de
zinc, dióxido de titanio y óxido de hierro, y se absorbe de manera
uniforme, incluso en la piel morena.

Los antioxidantes tópicos ayudan a controlar los radicales libres que
genera la luz azul, pero, como dijimos, aún no hay un consenso
científico.

``No puedo recomendar los antioxidantes desde un punto de vista
exclusivamente científico'', dijo Alexander Wolf, profesor adjunto
sénior de la Universidad de Medicina de Tokio y experto en la manera en
que la luz y el estrés oxidativo pueden provocar envejecimiento
prematuro. ``Pero, en efecto, hay muchos experimentos que muestran que
los antioxidantes funcionan bien en las células cultivadas. La vitamina
C entra directamente a las células, y si provocas daño oxidativo en las
células, la vitamina C o algún antioxidante definitivamente ayuda''.

``Pero una placa de Petri con células no es lo mismo que la piel'',
agregó Wolf.

Siempre y cuando tengas claro que no se ha demostrado que los
antioxidantes protegen de la luz azul, pero quizá lo hagan, son un buen
sustituto para el bloqueador solar si te parece extraño estar en casa
con la cara llena de minerales. Es probable que los antioxidantes
también minimicen el daño de la luz LED azul utilizada en casa para
tratar el acné. (Un bloqueador solar mineral bloquearía la luz azul y
frenaría su acción eliminadora de bacterias).

En cuanto a los antioxidantes, la vitamina C es una buena alternativa
porque la molécula en realidad es tan pequeña que puede penetrar la
piel. El suero
\href{https://gethyperskin.com/products/hyper-clear}{Hyper Skin Hyper
Clear Brightening Clearing Vitamin C}, de 36 dólares, contiene un 15 por
ciento de vitamina C combinada con vitamina E, y ambos ingredientes
aumentan el potencial del otro para combatir los radicales libres.

El debate en torno a la luz azul ha dado origen a nuevas líneas de
productos como Goodhabit. Su suero
\href{https://goodhabitskin.com/products/glow-potion-oil-serum}{Rescue
Me Glow Potion Oil}, de 80 dólares, combina proteínas marinas con
exopolisacáridos, es decir, polímeros secretados por microorganismos que
generan una barrera protectora en la piel. Los polímeros actúan como un
bloqueador solar que bloquea la luz azul (en vez de neutralizar los
radicales libres como un antioxidante).

Aunque el ácido alfa lipoico no se anuncia por su cualidad de protección
de la luz azul, Wolf ha estudiado su efecto en el estrés oxidativo (en
la piel de ratones) y cree que su uso es prometedor en la piel humana.

``Funciona de manera distinta a un antioxidante'', comentó. ``Activa las
defensas naturales de las células de la piel, haciendo que la célula
perciba el estrés oxidativo. La célula activa sus propios mecanismos de
defensa. Creo que esa es una manera mucho más elegante de defenderse''.

La crema humectante
\href{https://www.perriconemd.com/products/face-finishing-firming-moisturizer-51090023}{Perricone
MD High Potency Classics: Face Finishing \& Firming}, de 69 dólares,
contiene vitamina C y ácido alfa lipoico.

Un dato importante que a menudo se excluye del debate en torno a la luz
azul: el sol es, por mucho, nuestra fuente más abundante de luz azul.

``El brillo no es algo que perciba muy bien la vista humana, porque la
pupila se ajusta'', dijo Wolf. ``Quizá creas que tu tableta o tu celular
son brillantes, pero la cantidad de luz que llega a tu piel es muy
débil, especialmente en comparación con la luz solar''.

En conclusión, tu exposición a la luz azul quizá sea mucho menor en
comparación con tu vida antes de la pandemia, por la sencilla razón de
que pasas más tiempo en interiores.

\begin{center}\rule{0.5\linewidth}{\linethickness}\end{center}

Advertisement

\protect\hyperlink{after-bottom}{Continue reading the main story}

\hypertarget{site-index}{%
\subsection{Site Index}\label{site-index}}

\hypertarget{site-information-navigation}{%
\subsection{Site Information
Navigation}\label{site-information-navigation}}

\begin{itemize}
\tightlist
\item
  \href{https://help.nytimes3xbfgragh.onion/hc/en-us/articles/115014792127-Copyright-notice}{©~2020~The
  New York Times Company}
\end{itemize}

\begin{itemize}
\tightlist
\item
  \href{https://www.nytco.com/}{NYTCo}
\item
  \href{https://help.nytimes3xbfgragh.onion/hc/en-us/articles/115015385887-Contact-Us}{Contact
  Us}
\item
  \href{https://www.nytco.com/careers/}{Work with us}
\item
  \href{https://nytmediakit.com/}{Advertise}
\item
  \href{http://www.tbrandstudio.com/}{T Brand Studio}
\item
  \href{https://www.nytimes3xbfgragh.onion/privacy/cookie-policy\#how-do-i-manage-trackers}{Your
  Ad Choices}
\item
  \href{https://www.nytimes3xbfgragh.onion/privacy}{Privacy}
\item
  \href{https://help.nytimes3xbfgragh.onion/hc/en-us/articles/115014893428-Terms-of-service}{Terms
  of Service}
\item
  \href{https://help.nytimes3xbfgragh.onion/hc/en-us/articles/115014893968-Terms-of-sale}{Terms
  of Sale}
\item
  \href{https://spiderbites.nytimes3xbfgragh.onion}{Site Map}
\item
  \href{https://help.nytimes3xbfgragh.onion/hc/en-us}{Help}
\item
  \href{https://www.nytimes3xbfgragh.onion/subscription?campaignId=37WXW}{Subscriptions}
\end{itemize}
