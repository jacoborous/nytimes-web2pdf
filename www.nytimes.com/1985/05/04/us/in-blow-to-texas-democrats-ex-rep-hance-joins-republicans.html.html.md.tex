Sections

SEARCH

\protect\hyperlink{site-content}{Skip to
content}\protect\hyperlink{site-index}{Skip to site index}

\href{https://www.nytimes.com/section/us}{U.S.}

\href{https://myaccount.nytimes.com/auth/login?response_type=cookie\&client_id=vi}{}

\href{https://www.nytimes.com/section/todayspaper}{Today's Paper}

\href{/section/us}{U.S.}\textbar{}IN BLOW TO TEXAS DEMOCRATS, EX-REP.
HANCE JOINS REPUBLICANS

\url{https://nyti.ms/29BG1mP}

\begin{itemize}
\item
\item
\item
\item
\item
\end{itemize}

Advertisement

\protect\hyperlink{after-top}{Continue reading the main story}

Supported by

\protect\hyperlink{after-sponsor}{Continue reading the main story}

\hypertarget{in-blow-to-texas-democrats-ex-rep-hance-joins-republicans}{%
\section{IN BLOW TO TEXAS DEMOCRATS, EX-REP. HANCE JOINS
REPUBLICANS}\label{in-blow-to-texas-democrats-ex-rep-hance-joins-republicans}}

By Wayne King, Special To the New York Times

\begin{itemize}
\item
  May 4, 1985
\item
  \begin{itemize}
  \item
  \item
  \item
  \item
  \item
  \end{itemize}
\end{itemize}

\includegraphics{https://s1.nyt.com/timesmachine/pages/1/1985/05/04/215636_360W.png?quality=75\&auto=webp\&disable=upscale}

See the article in its original context from\\
May 4, 1985, Section 1, Page
8\href{https://store.nytimes.com/collections/new-york-times-page-reprints?utm_source=nytimes\&utm_medium=article-page\&utm_campaign=reprints}{Buy
Reprints}

\href{http://timesmachine.nytimes.com/timesmachine/1985/05/04/215636.html}{View
on timesmachine}

TimesMachine is an exclusive benefit for home delivery and digital
subscribers.

About the Archive

This is a digitized version of an article from The Times's print
archive, before the start of online publication in 1996. To preserve
these articles as they originally appeared, The Times does not alter,
edit or update them.

Occasionally the digitization process introduces transcription errors or
other problems; we are continuing to work to improve these archived
versions.

The Texas Democratic Party, already reeling from a serious defeat in
1984 and continuing defections in its ranks, suffered another blow today
with the announcement by Kent Hance, a popular former Representative,
that he too would rather be a Republican.

Mr. Hance, who ran unsuccessfully last year for the Democratic Senate
nomination, said in Washington today, ''I'm a Republican.''

He was joined at the announcement by Phil Gramm, another former Democrat
who was elected to the Senate last year as a Republican. Mr. Gramm
carried the state with 59 percent of the vote against a liberal
Democrat, State Senator Lloyd Doggett, in the strongest Republican
showing in more than a century.

Judges Join Republicans

In the wake of Mr. Gramm's victory, three Democratic judges switched to
the Republican Party, ignoring pleas by Gov. Mark White, a Democrat, not
to do so.

Mr. Gramm has repeatedly invited Mr. Hance to join the Republicans. As
Democratic members of the House, both had been prominent supporters of
President Reagan's economic policies, which they helped to draft.

Democrats were upset by the announcement, with some seeing it as an
example of the problems facing the party nationally.

George Christian, a Democratic political consultant in Austin who was a
special assistant to President Johnson, said in a telephone interview
that the switch was ''a blow to the Democratic Party, no question about
it.''

''It's symptomatic of the problem the Democrats are having holding onto
their conservative wing in Texas and elsewhere,'' he said. ''It's
serious business.''

Mr. Hance did not say what office he plans to seek as a Republican, but
even as a Democrat he was mentioned as a possible candidate for state
attorney general. That office is now held by a Democrat, Jim Mattox, who
was tried and acquitted in March on a charge of bribery.

Seen as Gubernatorial Candidate

Mr. Hance is also being touted as a possible challenger to Governor
White, now in his first term. Both Mr. Mattox's and Governor White's
terms expire next year.

As rumors of the switch began to circulate through the state Thursday,
Republican leaders were delighted, Democrats alarmed.

Mr. Christian foresaw a potential ''ripple effect'' that might prompt
other Democrats to switch parties. ''Certainly it will have some impact
in West Texas,'' where Mr. Hance's Congressional seat was, and possibly
even in some adjoining states, Mr. Christian said. ''The feeling abroad
among conservative Democrats is if they don't switch they'll get beat.''

The spokesman for the state Republican Party, Byron Nelson, said the
Hance switch ''could be the biggest event in the Republicanizing of the
state of Texas.''

According to some polls, Democrats are now a minority in Texas for the
first time in more than a century, with more voters identifying
themselves as either independents or Republicans than as Democrats.

'Party of the Future'

In Washington, the Republican national chairman, Frank K. Fahrenkopf
Jr., encouraged further defections. ''Kent Hance's switch today clearly
shows that the door of the Republican Party is open to the Democrats who
know our party is the party of the future,'' he said.

Angry Democrats attempted damage control. Representative Jim Wright of
Texas, the House majority leader who was instrumental in helping Mr.
Hance gain a seat on the Ways and Means Committee as a Washington
newcomer, said in Washington that he had not supported Mr. Hance in his
bid for the Senate last year because ''he is more responsible than any
other single member of Congress for the huge deficit our nation is
struggling with today.''

After his narrow loss last year to Mr. Doggett in a runoff for the
Democratic Senate nomination, Mr. Hance said, ''What we Democrats are
doing is nominating liberals and they're being defeated by
conservatives.''

Advertisement

\protect\hyperlink{after-bottom}{Continue reading the main story}

\hypertarget{site-index}{%
\subsection{Site Index}\label{site-index}}

\hypertarget{site-information-navigation}{%
\subsection{Site Information
Navigation}\label{site-information-navigation}}

\begin{itemize}
\tightlist
\item
  \href{https://help.nytimes.com/hc/en-us/articles/115014792127-Copyright-notice}{©~2020~The
  New York Times Company}
\end{itemize}

\begin{itemize}
\tightlist
\item
  \href{https://www.nytco.com/}{NYTCo}
\item
  \href{https://help.nytimes.com/hc/en-us/articles/115015385887-Contact-Us}{Contact
  Us}
\item
  \href{https://www.nytco.com/careers/}{Work with us}
\item
  \href{https://nytmediakit.com/}{Advertise}
\item
  \href{http://www.tbrandstudio.com/}{T Brand Studio}
\item
  \href{https://www.nytimes.com/privacy/cookie-policy\#how-do-i-manage-trackers}{Your
  Ad Choices}
\item
  \href{https://www.nytimes.com/privacy}{Privacy}
\item
  \href{https://help.nytimes.com/hc/en-us/articles/115014893428-Terms-of-service}{Terms
  of Service}
\item
  \href{https://help.nytimes.com/hc/en-us/articles/115014893968-Terms-of-sale}{Terms
  of Sale}
\item
  \href{https://spiderbites.nytimes.com}{Site Map}
\item
  \href{https://help.nytimes.com/hc/en-us}{Help}
\item
  \href{https://www.nytimes.com/subscription?campaignId=37WXW}{Subscriptions}
\end{itemize}
