Sections

SEARCH

\protect\hyperlink{site-content}{Skip to
content}\protect\hyperlink{site-index}{Skip to site index}

\href{https://www.nytimes.com/section/your-money}{Your Money}

\href{https://myaccount.nytimes.com/auth/login?response_type=cookie\&client_id=vi}{}

\href{https://www.nytimes.com/section/todayspaper}{Today's Paper}

\href{/section/your-money}{Your Money}\textbar{}Didn't Get Your Stimulus
Payment Yet? Here's What to Do

\url{https://nyti.ms/2yyQ7Vp}

\begin{itemize}
\item
\item
\item
\item
\item
\end{itemize}

\href{https://www.nytimes.com/news-event/coronavirus?action=click\&pgtype=Article\&state=default\&region=TOP_BANNER\&context=storylines_menu}{The
Coronavirus Outbreak}

\begin{itemize}
\tightlist
\item
  live\href{https://www.nytimes.com/2020/08/01/world/coronavirus-covid-19.html?action=click\&pgtype=Article\&state=default\&region=TOP_BANNER\&context=storylines_menu}{Latest
  Updates}
\item
  \href{https://www.nytimes.com/interactive/2020/us/coronavirus-us-cases.html?action=click\&pgtype=Article\&state=default\&region=TOP_BANNER\&context=storylines_menu}{Maps
  and Cases}
\item
  \href{https://www.nytimes.com/interactive/2020/science/coronavirus-vaccine-tracker.html?action=click\&pgtype=Article\&state=default\&region=TOP_BANNER\&context=storylines_menu}{Vaccine
  Tracker}
\item
  \href{https://www.nytimes.com/interactive/2020/07/29/us/schools-reopening-coronavirus.html?action=click\&pgtype=Article\&state=default\&region=TOP_BANNER\&context=storylines_menu}{What
  School May Look Like}
\item
  \href{https://www.nytimes.com/live/2020/07/31/business/stock-market-today-coronavirus?action=click\&pgtype=Article\&state=default\&region=TOP_BANNER\&context=storylines_menu}{Economy}
\end{itemize}

Advertisement

\protect\hyperlink{after-top}{Continue reading the main story}

Supported by

\protect\hyperlink{after-sponsor}{Continue reading the main story}

\hypertarget{didnt-get-your-stimulus-payment-yet-heres-what-to-do}{%
\section{Didn't Get Your Stimulus Payment Yet? Here's What to
Do}\label{didnt-get-your-stimulus-payment-yet-heres-what-to-do}}

Tens of millions of people have already received their payments, but if
you haven't, these are the things you should check on.

\includegraphics{https://static01.nyt.com/images/2020/05/05/business/05money/merlin_172039485_a2953f2f-ca8e-4e55-8e93-d454033160cd-articleLarge.jpg?quality=75\&auto=webp\&disable=upscale}

\href{https://www.nytimes.com/by/ron-lieber}{\includegraphics{https://static01.nyt.com/images/2018/10/22/multimedia/author-ron-lieber/author-ron-lieber-thumbLarge.png}}

By \href{https://www.nytimes.com/by/ron-lieber}{Ron Lieber}

\begin{itemize}
\item
  July 17, 2020
\item
  \begin{itemize}
  \item
  \item
  \item
  \item
  \item
  \end{itemize}
\end{itemize}

It's been weeks since people started getting
\href{https://www.nytimes.com/article/coronavirus-stimulus-package-questions-answers.html}{coronavirus
relief payments}. You've checked and rechecked your eligibility, just to
be sure.

But still, no \$1,200
\href{https://www.nytimes.com/2020/07/17/us/politics/mnuchin-congress-stimulus.html}{stimulus
payment} has arrived in your bank account or mailbox. Perhaps \$3,400 is
riding on this for you, your spouse and your two children, for whom
you're supposed to get \$500 each.

Tens of millions of people have already received their payments, but
many others are still waiting or wondering. There are a lot of reasons
you could be among them, even if the government has
\href{https://www.nytimes.com/2020/04/01/business/coronavirus-stimulus-social-security.html}{removed
some of the hurdles} it initially set up.

So what do you do if yours hasn't arrived?

\hypertarget{try-the-irs-tool-again}{%
\subsection{Try the I.R.S. tool again.}\label{try-the-irs-tool-again}}

A couple of weeks ago, the I.R.S. introduced its
\href{https://www.irs.gov/coronavirus/get-my-payment}{Get My Payment}
tool to help people figure out when and how their money might be
arriving. The unveiling didn't go so well: Many users did not realize
how picky the site was about, say, entering an address that precisely
matched the one on their most recent tax return.

Also, there were lots of confusing messages indicating that there was no
information available at all. Things have improved some since then, and
the I.R.S. is updating the information once each day, usually in the
middle of the night.

\hypertarget{latest-updates-global-coronavirus-outbreak}{%
\section{\texorpdfstring{\href{https://www.nytimes.com/2020/08/01/world/coronavirus-covid-19.html?action=click\&pgtype=Article\&state=default\&region=MAIN_CONTENT_1\&context=storylines_live_updates}{Latest
Updates: Global Coronavirus
Outbreak}}{Latest Updates: Global Coronavirus Outbreak}}\label{latest-updates-global-coronavirus-outbreak}}

Updated 2020-08-02T10:04:29.623Z

\begin{itemize}
\tightlist
\item
  \href{https://www.nytimes.com/2020/08/01/world/coronavirus-covid-19.html?action=click\&pgtype=Article\&state=default\&region=MAIN_CONTENT_1\&context=storylines_live_updates\#link-34047410}{The
  U.S. reels as July cases more than double the total of any other
  month.}
\item
  \href{https://www.nytimes.com/2020/08/01/world/coronavirus-covid-19.html?action=click\&pgtype=Article\&state=default\&region=MAIN_CONTENT_1\&context=storylines_live_updates\#link-780ec966}{Top
  U.S. officials work to break an impasse over the federal jobless
  benefit.}
\item
  \href{https://www.nytimes.com/2020/08/01/world/coronavirus-covid-19.html?action=click\&pgtype=Article\&state=default\&region=MAIN_CONTENT_1\&context=storylines_live_updates\#link-2bc8948}{Its
  outbreak untamed, Melbourne goes into even greater lockdown.}
\end{itemize}

\href{https://www.nytimes.com/2020/08/01/world/coronavirus-covid-19.html?action=click\&pgtype=Article\&state=default\&region=MAIN_CONTENT_1\&context=storylines_live_updates}{See
more updates}

More live coverage:
\href{https://www.nytimes.com/live/2020/07/31/business/stock-market-today-coronavirus?action=click\&pgtype=Article\&state=default\&region=MAIN_CONTENT_1\&context=storylines_live_updates}{Markets}

You may need information from recent tax returns at the ready to use the
tool, and it doesn't work for recipients of Supplemental Security Income
and Veterans Affairs benefits.

\hypertarget{make-sure-youve-filed-the-right-paperwork}{%
\subsection{Make sure you've filed the right
paperwork.}\label{make-sure-youve-filed-the-right-paperwork}}

People who don't usually file a tax return should give the I.R.S. an
assist.

If you haven't had to file a return because your gross income did not
exceed \$12,200 (\$24,400 for married couples), you still qualify for a
payment. But if you're not a recipient of S.S.I. or V.A. benefits, you
should fill out
\href{https://www.irs.gov/coronavirus/non-filers-enter-payment-info-here}{a
special form for non-filers}.

The government is also crosschecking all the Social Security and V.A.
databases and issuing payments to those recipients for whom it does have
bank account or similar information, but that process can add time.

But May 5 is an important deadline: S.S.I. and V.A. beneficiaries who
didn't have to file a tax return in 2018 or 2019 and have children 16 or
under should register online with the I.R.S. non-filer tool to get the
\$500 per child payment
\href{https://www.irs.gov/newsroom/va-ssi-recipients-with-eligible-children-need-to-act-by-tuesday-may-5-to-quickly-add-money-to-their-automatic-economic-impact-payment-plus-500-push-continues}{more
quickly}.

\hypertarget{dont-necessarily-panic-if-the-payment-went-to-a-strange-account}{%
\subsection{Don't (necessarily) panic if the payment went to a strange
account.}\label{dont-necessarily-panic-if-the-payment-went-to-a-strange-account}}

One known quagmire: If you filed taxes in 2018 or 2019 with the help of
a third-party company, you may have taken advantage of something called
a refund anticipation loan. The company may have set you up with a
temporary account to process the loan and give you access to that money.

The bank information the I.R.S. has for you may be for that account,
which may be closed at this point. That means that when the I.R.S. tries
to deposit the stimulus money there, the process will break down.

At that point, the I.R.S. is supposed to send a paper check to the
address on the most recent tax return, or one on file with the U.S.
Postal Service.

You should be able to track this whole messy process using the Get My
Payment service, but it could take several more weeks to get your
payment.

\href{https://www.nytimes.com/news-event/coronavirus?action=click\&pgtype=Article\&state=default\&region=MAIN_CONTENT_3\&context=storylines_faq}{}

\hypertarget{the-coronavirus-outbreak-}{%
\subsubsection{The Coronavirus Outbreak
›}\label{the-coronavirus-outbreak-}}

\hypertarget{frequently-asked-questions}{%
\paragraph{Frequently Asked
Questions}\label{frequently-asked-questions}}

Updated July 27, 2020

\begin{itemize}
\item ~
  \hypertarget{should-i-refinance-my-mortgage}{%
  \paragraph{Should I refinance my
  mortgage?}\label{should-i-refinance-my-mortgage}}

  \begin{itemize}
  \tightlist
  \item
    \href{https://www.nytimes.com/article/coronavirus-money-unemployment.html?action=click\&pgtype=Article\&state=default\&region=MAIN_CONTENT_3\&context=storylines_faq}{It
    could be a good idea,} because mortgage rates have
    \href{https://www.nytimes.com/2020/07/16/business/mortgage-rates-below-3-percent.html?action=click\&pgtype=Article\&state=default\&region=MAIN_CONTENT_3\&context=storylines_faq}{never
    been lower.} Refinancing requests have pushed mortgage applications
    to some of the highest levels since 2008, so be prepared to get in
    line. But defaults are also up, so if you're thinking about buying a
    home, be aware that some lenders have tightened their standards.
  \end{itemize}
\item ~
  \hypertarget{what-is-school-going-to-look-like-in-september}{%
  \paragraph{What is school going to look like in
  September?}\label{what-is-school-going-to-look-like-in-september}}

  \begin{itemize}
  \tightlist
  \item
    It is unlikely that many schools will return to a normal schedule
    this fall, requiring the grind of
    \href{https://www.nytimes.com/2020/06/05/us/coronavirus-education-lost-learning.html?action=click\&pgtype=Article\&state=default\&region=MAIN_CONTENT_3\&context=storylines_faq}{online
    learning},
    \href{https://www.nytimes.com/2020/05/29/us/coronavirus-child-care-centers.html?action=click\&pgtype=Article\&state=default\&region=MAIN_CONTENT_3\&context=storylines_faq}{makeshift
    child care} and
    \href{https://www.nytimes.com/2020/06/03/business/economy/coronavirus-working-women.html?action=click\&pgtype=Article\&state=default\&region=MAIN_CONTENT_3\&context=storylines_faq}{stunted
    workdays} to continue. California's two largest public school
    districts --- Los Angeles and San Diego --- said on July 13, that
    \href{https://www.nytimes.com/2020/07/13/us/lausd-san-diego-school-reopening.html?action=click\&pgtype=Article\&state=default\&region=MAIN_CONTENT_3\&context=storylines_faq}{instruction
    will be remote-only in the fall}, citing concerns that surging
    coronavirus infections in their areas pose too dire a risk for
    students and teachers. Together, the two districts enroll some
    825,000 students. They are the largest in the country so far to
    abandon plans for even a partial physical return to classrooms when
    they reopen in August. For other districts, the solution won't be an
    all-or-nothing approach.
    \href{https://bioethics.jhu.edu/research-and-outreach/projects/eschool-initiative/school-policy-tracker/}{Many
    systems}, including the nation's largest, New York City, are
    devising
    \href{https://www.nytimes.com/2020/06/26/us/coronavirus-schools-reopen-fall.html?action=click\&pgtype=Article\&state=default\&region=MAIN_CONTENT_3\&context=storylines_faq}{hybrid
    plans} that involve spending some days in classrooms and other days
    online. There's no national policy on this yet, so check with your
    municipal school system regularly to see what is happening in your
    community.
  \end{itemize}
\item ~
  \hypertarget{is-the-coronavirus-airborne}{%
  \paragraph{Is the coronavirus
  airborne?}\label{is-the-coronavirus-airborne}}

  \begin{itemize}
  \tightlist
  \item
    The coronavirus
    \href{https://www.nytimes.com/2020/07/04/health/239-experts-with-one-big-claim-the-coronavirus-is-airborne.html?action=click\&pgtype=Article\&state=default\&region=MAIN_CONTENT_3\&context=storylines_faq}{can
    stay aloft for hours in tiny droplets in stagnant air}, infecting
    people as they inhale, mounting scientific evidence suggests. This
    risk is highest in crowded indoor spaces with poor ventilation, and
    may help explain super-spreading events reported in meatpacking
    plants, churches and restaurants.
    \href{https://www.nytimes.com/2020/07/06/health/coronavirus-airborne-aerosols.html?action=click\&pgtype=Article\&state=default\&region=MAIN_CONTENT_3\&context=storylines_faq}{It's
    unclear how often the virus is spread} via these tiny droplets, or
    aerosols, compared with larger droplets that are expelled when a
    sick person coughs or sneezes, or transmitted through contact with
    contaminated surfaces, said Linsey Marr, an aerosol expert at
    Virginia Tech. Aerosols are released even when a person without
    symptoms exhales, talks or sings, according to Dr. Marr and more
    than 200 other experts, who
    \href{https://academic.oup.com/cid/article/doi/10.1093/cid/ciaa939/5867798}{have
    outlined the evidence in an open letter to the World Health
    Organization}.
  \end{itemize}
\item ~
  \hypertarget{what-are-the-symptoms-of-coronavirus}{%
  \paragraph{What are the symptoms of
  coronavirus?}\label{what-are-the-symptoms-of-coronavirus}}

  \begin{itemize}
  \tightlist
  \item
    Common symptoms
    \href{https://www.nytimes.com/article/symptoms-coronavirus.html?action=click\&pgtype=Article\&state=default\&region=MAIN_CONTENT_3\&context=storylines_faq}{include
    fever, a dry cough, fatigue and difficulty breathing or shortness of
    breath.} Some of these symptoms overlap with those of the flu,
    making detection difficult, but runny noses and stuffy sinuses are
    less common.
    \href{https://www.nytimes.com/2020/04/27/health/coronavirus-symptoms-cdc.html?action=click\&pgtype=Article\&state=default\&region=MAIN_CONTENT_3\&context=storylines_faq}{The
    C.D.C. has also} added chills, muscle pain, sore throat, headache
    and a new loss of the sense of taste or smell as symptoms to look
    out for. Most people fall ill five to seven days after exposure, but
    symptoms may appear in as few as two days or as many as 14 days.
  \end{itemize}
\item ~
  \hypertarget{does-asymptomatic-transmission-of-covid-19-happen}{%
  \paragraph{Does asymptomatic transmission of Covid-19
  happen?}\label{does-asymptomatic-transmission-of-covid-19-happen}}

  \begin{itemize}
  \tightlist
  \item
    So far, the evidence seems to show it does. A widely cited
    \href{https://www.nature.com/articles/s41591-020-0869-5}{paper}
    published in April suggests that people are most infectious about
    two days before the onset of coronavirus symptoms and estimated that
    44 percent of new infections were a result of transmission from
    people who were not yet showing symptoms. Recently, a top expert at
    the World Health Organization stated that transmission of the
    coronavirus by people who did not have symptoms was ``very rare,''
    \href{https://www.nytimes.com/2020/06/09/world/coronavirus-updates.html?action=click\&pgtype=Article\&state=default\&region=MAIN_CONTENT_3\&context=storylines_faq\#link-1f302e21}{but
    she later walked back that statement.}
  \end{itemize}
\end{itemize}

\hypertarget{but-worry-about-fraud-if-this-happens}{%
\subsection{But worry about fraud if this
happens.}\label{but-worry-about-fraud-if-this-happens}}

A lot of money is flowing right now, so people will indeed try to steal
it. The I.R.S. knows this, so 15 days after it issues your payment, it
is supposed to send confirmation letters to the most recent address it
has on file for you.

That letter should explain exactly how the I.R.S. made the payment. If
you haven't received the money yet, that's the time to worry about
whether someone else took it. The letter will contain contact
information for the I.R.S. if you need help.

\hypertarget{speed-up-delivery-this-way}{%
\subsection{Speed up delivery this
way.}\label{speed-up-delivery-this-way}}

For any number of reasons, the I.R.S. may not have up-to-date
information --- or any at all --- about your address or bank account.
For instance, plenty of people don't trust the I.R.S. with their
checking account information for direct deposits or payments. Instead,
they pay tax bills with paper checks and collect refunds that way, too.

If you're in that category but are willing to change your approach, you
may be able to get your payment more quickly. If the government hasn't
already started the process of sending you a paper check, it may still
be possible to enter your checking account information via the
\href{https://www.irs.gov/coronavirus/get-my-payment}{Get My Payment
tool} to get your money more quickly.

\hypertarget{check-your-eligibility-again}{%
\subsection{Check your eligibility
again.}\label{check-your-eligibility-again}}

People with higher incomes
\href{https://www.irs.gov/coronavirus/economic-impact-payment-information-center\#eligibility}{might
not get a payment}. The \$1,200 payment decreases until it stops
altogether for a single person earning \$99,000 or a married couple who
have no dependent children, file their taxes jointly and earn \$198,000.
And if someone else claimed you as a dependent, you don't get a check.

\hypertarget{check-with-the-irs-for-more-information}{%
\subsection{Check with the I.R.S. for more
information.}\label{check-with-the-irs-for-more-information}}

The agency got off to a pretty slow start in explaining how things would
work, but it has now answered 38 questions in its
\href{https://www.irs.gov/coronavirus/economic-impact-payment-information-center}{F.A.Q.}
It has also published a
\href{https://www.irs.gov/newsroom/how-to-use-the-tools-on-irsgov-to-get-your-economic-impact-payment}{chart}
to help you figure out what, if any, additional information you may need
to hand over to receive a payment or get one more quickly.

Advertisement

\protect\hyperlink{after-bottom}{Continue reading the main story}

\hypertarget{site-index}{%
\subsection{Site Index}\label{site-index}}

\hypertarget{site-information-navigation}{%
\subsection{Site Information
Navigation}\label{site-information-navigation}}

\begin{itemize}
\tightlist
\item
  \href{https://help.nytimes.com/hc/en-us/articles/115014792127-Copyright-notice}{©~2020~The
  New York Times Company}
\end{itemize}

\begin{itemize}
\tightlist
\item
  \href{https://www.nytco.com/}{NYTCo}
\item
  \href{https://help.nytimes.com/hc/en-us/articles/115015385887-Contact-Us}{Contact
  Us}
\item
  \href{https://www.nytco.com/careers/}{Work with us}
\item
  \href{https://nytmediakit.com/}{Advertise}
\item
  \href{http://www.tbrandstudio.com/}{T Brand Studio}
\item
  \href{https://www.nytimes.com/privacy/cookie-policy\#how-do-i-manage-trackers}{Your
  Ad Choices}
\item
  \href{https://www.nytimes.com/privacy}{Privacy}
\item
  \href{https://help.nytimes.com/hc/en-us/articles/115014893428-Terms-of-service}{Terms
  of Service}
\item
  \href{https://help.nytimes.com/hc/en-us/articles/115014893968-Terms-of-sale}{Terms
  of Sale}
\item
  \href{https://spiderbites.nytimes.com}{Site Map}
\item
  \href{https://help.nytimes.com/hc/en-us}{Help}
\item
  \href{https://www.nytimes.com/subscription?campaignId=37WXW}{Subscriptions}
\end{itemize}
