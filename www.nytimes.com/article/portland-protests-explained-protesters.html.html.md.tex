Sections

SEARCH

\protect\hyperlink{site-content}{Skip to
content}\protect\hyperlink{site-index}{Skip to site index}

\href{https://www.nytimes.com/section/us}{U.S.}

\href{https://myaccount.nytimes.com/auth/login?response_type=cookie\&client_id=vi}{}

\href{https://www.nytimes.com/section/todayspaper}{Today's Paper}

\href{/section/us}{U.S.}\textbar{}What Do Portland Protesters Want, and
How Have the Police Responded?

\url{https://nyti.ms/3jDm1Tl}

\begin{itemize}
\item
\item
\item
\item
\item
\end{itemize}

\href{https://www.nytimes.com/news-event/george-floyd-protests-minneapolis-new-york-los-angeles?action=click\&pgtype=Article\&state=default\&region=TOP_BANNER\&context=storylines_menu}{Race
and America}

\begin{itemize}
\tightlist
\item
  \href{https://www.nytimes.com/2020/07/26/us/protests-portland-seattle-trump.html?action=click\&pgtype=Article\&state=default\&region=TOP_BANNER\&context=storylines_menu}{Protesters
  Return to Other Cities}
\item
  \href{https://www.nytimes.com/2020/07/24/us/portland-oregon-protests-white-race.html?action=click\&pgtype=Article\&state=default\&region=TOP_BANNER\&context=storylines_menu}{Portland
  at the Center}
\item
  \href{https://www.nytimes.com/2020/07/23/podcasts/the-daily/portland-protests.html?action=click\&pgtype=Article\&state=default\&region=TOP_BANNER\&context=storylines_menu}{Podcast:
  Showdown in Portland}
\item
  \href{https://www.nytimes.com/interactive/2020/07/16/us/black-lives-matter-protests-louisville-breonna-taylor.html?action=click\&pgtype=Article\&state=default\&region=TOP_BANNER\&context=storylines_menu}{45
  Days in Louisville}
\end{itemize}

Advertisement

\protect\hyperlink{after-top}{Continue reading the main story}

Supported by

\protect\hyperlink{after-sponsor}{Continue reading the main story}

\hypertarget{what-do-portland-protesters-want-and-how-have-the-police-responded}{%
\section{What Do Portland Protesters Want, and How Have the Police
Responded?}\label{what-do-portland-protesters-want-and-how-have-the-police-responded}}

Eight weeks after the death of George Floyd, here's a look at why
longstanding protests in the city have recently intensified.

\includegraphics{https://static01.nyt.com/images/2020/07/24/us/24PORTLAND-EXPLAINER/merlin_174891849_c8c140fb-54ad-42e6-bf2e-7a4ff3ecdd1b-articleLarge.jpg?quality=75\&auto=webp\&disable=upscale}

By Giulia McDonnell Nieto del Rio

\begin{itemize}
\item
  July 31, 2020
\item
  \begin{itemize}
  \item
  \item
  \item
  \item
  \item
  \end{itemize}
\end{itemize}

When a video showing George Floyd's death in police custody spread
across social media, cities and towns nationwide soon erupted in
protests against systemic racism and police brutality. But while
protests in many places subsided after a few weeks,
\href{https://www.nytimes.com/2020/07/28/us/portland-protests-fact-check.html}{Portland},
Ore., has been holding demonstrations every night since May 29.

\href{https://www.nytimes.com/2020/07/17/us/portland-protests.html}{The
arrival of federal forces in the city} this month --- and concerns they
were
\href{https://www.nytimes.com/2020/07/17/us/portland-protests.html}{exceeding
their authority} and violating protesters' rights --- drew the ire of
local officials and reinvigorated nightly demonstrations. With renewed
force, marchers have spray-painted the walls of
\href{https://www.nytimes.com/2020/07/22/us/portland-protests-courthouse.html}{the
U.S. District Court building}, demanding that federal agents go home.
Groups of mothers have banded together, locking arms and
\href{https://www.nytimes.com/2020/07/19/us/portland-protests.html}{chanting}:
``Feds stay clear. Moms are here.''

Early in the protests,
\href{https://www.wweek.com/news/2020/05/30/video-portland-protesters-smash-windows-and-set-fires-in-multnomah-county-justice-center/}{protesters
broke into the Multnomah County Justice Center} and set some of the
offices on fire, and the Portland police have reported cases of looting.
More recently, demonstrators have thrown rocks and bottles at federal
officers. But many have protested peacefully, and Gov. Kate Brown has
called the presence of federal agents an ``abuse of power.''

President Trump has called the demonstrators ``anarchists'' who ``hate''
the country, and Chad F. Wolf, the acting secretary of homeland
security, has
\href{https://www.nytimes.com/2020/07/21/us/politics/homeland-security-portland-oregon.html}{blamed
Oregon officials} for the unrest.

\includegraphics{https://static01.nyt.com/images/2020/07/17/autossell/portland-v1-2/portland-v1-2-videoSixteenByNineJumbo1600.jpg}

\hypertarget{what-are-the-protesters-demanding}{%
\subsection{What are the protesters
demanding?}\label{what-are-the-protesters-demanding}}

What started out as a movement for police accountability and racial
justice has morphed into a complex mobilization. The protesters' goals
now include defunding the police, addressing income inequality and
pushing federal agents out of the city.

In Portland, which is
\href{https://www.nytimes.com/2020/07/24/us/portland-oregon-protests-white-race.html}{one
of America's whitest cities} and has a racist history, protesters have
maintained a public call for change that has subsided elsewhere in the
country.

\href{https://www.nytimes.com/2020/07/24/us/portland-oregon-protests-white-race.html}{Experts
say} the protests bring together a coalition of racial justice
proponents and anti-fascist advocates, who have long been active in
Portland. The groups share some intersecting grievances and common
goals, such as cutting police budgets and installing more civilian
oversight of the police.

Signs such as ``White Silence=Violence'' and ``Black Lives Matter'' are
widespread, and calls at the demonstrations to address racial inequities
persist. One woman held a sign that said: ``My Black Child is Watching!
\#BLM She Will Know Her Life Matters.''

Demonstrators have also expressed increasing frustration with the
federal presence and the Trump administration.

``What is making more people come to the street every night now is the
brutalization that's happening to regular community members at the hands
of Portland police and these federal agents,'' Jo Ann Hardesty, a city
commissioner, said at a news conference.

\hypertarget{how-has-the-city-responded}{%
\subsection{How has the city
responded?}\label{how-has-the-city-responded}}

Street protests began four days after the death of Mr. Floyd in
Minneapolis. As the demonstrations continued and officers used tear gas
to disperse crowds, public outrage against aggressive police tactics
increased and calls to defund the police escalated.

On June 8, after more than a week of large-scale demonstrations
involving thousands of marchers, the chief of the Portland Police Bureau
\href{https://www.nytimes.com/2020/06/08/us/george-floyd-protests.html}{stepped
down}, saying new leadership was needed to rebuild public trust. Shortly
after, a federal judge upheld restrictions on tear gas put in place by
Mayor Ted Wheeler, barring the use of the chemical agent except when
life or safety was at risk.

The City Council also passed a budget that would cut \$15 million from
the police in the upcoming fiscal year, a demand sought by protesters.

\includegraphics{https://static01.nyt.com/images/2020/07/24/us/00PORTLAND-EXPLAINER-may/merlin_172980756_13bbd693-8268-4c35-89be-35d4d6712b23-articleLarge.jpg?quality=75\&auto=webp\&disable=upscale}

\hypertarget{why-have-the-protests-continued-this-long}{%
\subsection{Why have the protests continued this
long?}\label{why-have-the-protests-continued-this-long}}

By late June, the size of protests had diminished significantly. Rose
City Justice, a major mobilizing force in Portland, announced plans to
pull back on organizing efforts. Nightly marches, numbering in the
hundreds, became more decentralized.

But after federal agents, including some from the Department of Homeland
Security, arrived in July, reports soon emerged that they had forcefully
\href{https://www.nytimes.com/2020/07/20/us/politics/portland-federal-agents-trump.html}{pulled
people into unmarked vehicles}, injured protesters, and deployed tear
gas. Mayor Wheeler, who called the situation ``an attack on our
democracy,'' was
\href{https://www.nytimes.com/2020/07/23/us/portland-protest-tear-gas-mayor.html}{tear-gassed
with a group of protesters} outside the federal courthouse.

By the time the federal agents arrived, city leaders said, the situation
on the streets had de-escalated. But outrage at the Trump
administration's deployment reinvigorated the daily rallies.

\hypertarget{which-federal-law-enforcement-agencies-are-involved}{%
\subsection{Which federal law enforcement agencies are
involved?}\label{which-federal-law-enforcement-agencies-are-involved}}

\href{https://www.nytimes.com/2020/07/21/us/politics/homeland-security-portland-oregon.html}{The
federal agents present in Portland} include personnel from the U.S.
Marshals and tactical agents from Customs and Border Protection and
Immigration and Customs Enforcement, in addition to the Federal
Protective Service, which was already stationed to protect federal
property in Portland.

Some of the agents are from a group known as BORTAC, the Border Patrol's
equivalent of a SWAT team, which typically investigates drug smuggling
organizations.

\hypertarget{what-is-motivating-the-movement-in-portland}{%
\subsection{What is motivating the movement in
Portland?}\label{what-is-motivating-the-movement-in-portland}}

Oregon has a history of white supremacy. A law passed in 1844 said that
any Black person would be ``whipped twice a year until he or she shall
quit the territory'' and leaders also later banned Black people from
entering the territory.

Some protesters say the state's deeply racist history is still reflected
in Portland's structures. One protester, Reginald Liggins, who is Black,
\href{https://www.nytimes.com/2020/07/24/us/portland-oregon-protests-white-race.html}{told
The New York Times} that he began commuting by bus after being pulled
over multiple times by the Portland police without reason. Liza
Lopetrone, a veterinary nurse who is white and joined the Wall of Moms
protest this week, said she wanted to bring the state's white
supremacist legacy to light.

Others were not moved to participate until federal agents entered the
city. Christopher J. David, a Navy veteran who was filmed being
\href{https://www.nytimes.com/2020/07/20/us/portland-protests-navy-christopher-david.html}{beaten
with a baton} by federal officers, had not followed the protests until
U.S. agents were deployed. He came to the protests to ask officers about
their use of violent tactics against protesters, which he said
conflicted with their oath to uphold the Constitution.

Reporting was contributed by Mike Baker, Thomas Fuller, John Ismay,
Zolan Kanno-Youngs and Sergio Olmos

Advertisement

\protect\hyperlink{after-bottom}{Continue reading the main story}

\hypertarget{site-index}{%
\subsection{Site Index}\label{site-index}}

\hypertarget{site-information-navigation}{%
\subsection{Site Information
Navigation}\label{site-information-navigation}}

\begin{itemize}
\tightlist
\item
  \href{https://help.nytimes.com/hc/en-us/articles/115014792127-Copyright-notice}{©~2020~The
  New York Times Company}
\end{itemize}

\begin{itemize}
\tightlist
\item
  \href{https://www.nytco.com/}{NYTCo}
\item
  \href{https://help.nytimes.com/hc/en-us/articles/115015385887-Contact-Us}{Contact
  Us}
\item
  \href{https://www.nytco.com/careers/}{Work with us}
\item
  \href{https://nytmediakit.com/}{Advertise}
\item
  \href{http://www.tbrandstudio.com/}{T Brand Studio}
\item
  \href{https://www.nytimes.com/privacy/cookie-policy\#how-do-i-manage-trackers}{Your
  Ad Choices}
\item
  \href{https://www.nytimes.com/privacy}{Privacy}
\item
  \href{https://help.nytimes.com/hc/en-us/articles/115014893428-Terms-of-service}{Terms
  of Service}
\item
  \href{https://help.nytimes.com/hc/en-us/articles/115014893968-Terms-of-sale}{Terms
  of Sale}
\item
  \href{https://spiderbites.nytimes.com}{Site Map}
\item
  \href{https://help.nytimes.com/hc/en-us}{Help}
\item
  \href{https://www.nytimes.com/subscription?campaignId=37WXW}{Subscriptions}
\end{itemize}
