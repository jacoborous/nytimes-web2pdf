Sections

SEARCH

\protect\hyperlink{site-content}{Skip to
content}\protect\hyperlink{site-index}{Skip to site index}

\href{https://www.nytimes.com/section/world/europe}{Europe}

\href{https://myaccount.nytimes.com/auth/login?response_type=cookie\&client_id=vi}{}

\href{https://www.nytimes.com/section/todayspaper}{Today's Paper}

\href{/section/world/europe}{Europe}\textbar{}With Michael Flynn Gone,
Russia Sees a Different Trump

\url{https://nyti.ms/2lfFIEU}

\begin{itemize}
\item
\item
\item
\item
\item
\end{itemize}

Advertisement

\protect\hyperlink{after-top}{Continue reading the main story}

Supported by

\protect\hyperlink{after-sponsor}{Continue reading the main story}

\hypertarget{with-michael-flynn-gone-russia-sees-a-different-trump}{%
\section{With Michael Flynn Gone, Russia Sees a Different
Trump}\label{with-michael-flynn-gone-russia-sees-a-different-trump}}

\includegraphics{https://static01.nyt.com/images/2017/02/15/world/15Russia/15Russia-articleLarge.jpg?quality=75\&auto=webp\&disable=upscale}

By \href{http://www.nytimes.com/by/neil-macfarquhar}{Neil MacFarquhar}

\begin{itemize}
\item
  Feb. 14, 2017
\item
  \begin{itemize}
  \item
  \item
  \item
  \item
  \item
  \end{itemize}
\end{itemize}

MOSCOW --- The champagne toasts that some Russian officials quaffed just
a few short months ago to celebrate the victory of Donald J. Trump have
gone a bit flat.

Euphoria was already starting to cede to caution before
\href{https://www.nytimes.com/2017/02/13/us/politics/donald-trump-national-security-adviser-michael-flynn.html}{Michael
T. Flynn}, President Trump's national security adviser and a perceived
friend of Russia, resigned. That cemented the uneasy mood.

The departure of Mr. Flynn on Monday over his contacts with the Russian
ambassador to Washington was the latest in a series of mixed signals
from Mr. Trump and his advisers on a host of issues important to Russia,
particularly the lifting of economic sanctions.

Now, many prominent political figures are wondering whether hopes for
change were premature, and whether Moscow will inevitably remain
Washington's main boogeyman. On Tuesday, the Pentagon was
confrontational, accusing Moscow of secretly deploying a cruise missile
system that
\href{https://www.nytimes.com/2017/02/14/world/europe/russia-cruise-missile-arms-control-treaty.html}{violates
a 1987 treaty} on intermediate-range missiles based on land.

Vladimir R. Soloviev, the host of a noisy Sunday night talk show on
state-run television viewed as reflecting Kremlin policy, this week
issued one of the most negative public assessments yet of Mr. Trump.
``Don't be charmed by Trump,'' he said in a message he addressed to all
politicians and experts. ``Don't think that Trump is a pro-Russian
politician. Don't hope that Trump, in the interests of Russia, will in
any way go against the basic, rooted interests of America.''

How things have changed since November, when the Russian Parliament
greeted Mr. Trump's election with a round of applause and a prominent
political leader --- albeit one famous for his antics --- toasted the
victory with champagne on national television. In January, Mr. Trump
garnered more mentions than President Vladimir V. Putin in the Russian
news media, knocking the Russian leader from the top spot for the first
time since 2011.

Only one man, Mr. Putin, really sets Russia's foreign policy course,
however. And he was never publicly celebratory, although his animosity
toward Hillary Clinton, whom he blamed for the
\href{http://www.nytimes.com/2012/05/07/world/europe/at-moscow-rally-arrests-and-violence.html}{angry
demonstrations} that greeted his return to the presidency in 2012, was
well known.

In recent years, Mr. Putin's main foreign policy goal has been to
resurrect the time when the United States and the Soviet Union, as the
two great nuclear superpowers, were the main arbitrators of the global
order. Lacking the might of the Soviet Union, Mr. Putin has tried to
punch above his weight by shocking the world with unexpected tactics
like
\href{https://www.nytimes.com/2014/03/19/world/europe/ukraine.html}{seizing
Crimea}, destabilizing Ukraine and deploying his military in Syria to
shore up President Bashar al-Assad.

President Barack Obama responded by referring to Russia as a
\href{https://www.nytimes.com/2014/03/26/world/europe/hague-summit-focuses-on-preventing-trafficking-of-nuclear-materials.html}{declining
regional power}. The two men had a poisonous personal relationship.

Mr. Trump seemed to presage a different era with all the praise he
heaped on Russia and Mr. Putin. He described him as a strong, smart
leader and said that Moscow seemed to be blamed for everything. And he
called for better relations with Moscow to fight the Islamic State and
other terrorist groups, echoing a longstanding Putin pitch.

Some voices in Moscow cautioned that Mrs. Clinton, as a calmer hand on
the tiller, would be the kind of predictable leader that the Kremlin
preferred, albeit a hostile one. Now, there is a sense that the Kremlin
might be unsettled by the president of a far more powerful country
deploying Mr. Putin's favorite tactic: unpredictability.

``Trump will be tamed and act more presidential, eventually, but he also
has a penchant for unpredictability that works against the Kremlin,''
said Konstantin von Eggert, a political commentator for
\href{https://www.theguardian.com/cities/2015/jun/09/tv-rain-russia-only-independent-television-channel}{TV
Rain}, Russia's only independent channel. ``This creates a situation in
which a stronger player with the same style of unpredictability as a
strategy comes on the stage. Putin did not anticipate that.''

There has been a certain amount of policy whiplash on issues important
to Russia. First, Mr. Trump said that NATO was obsolete, then that it
had America's solid backing. He seemed to indicate he would lift
economic sanctions imposed over the Ukraine crisis, and appointed as
secretary of state Rex W. Tillerson, who as head of Exxon Mobil cut
enormous oil deals with Russia and spoke out publicly against sanctions.

Then the new United States ambassador to the United Nations,
\href{https://www.nytimes.com/2017/01/27/world/americas/nikki-haley-united-nations.html}{Nikki
R. Haley}, sharply criticized Russia over Ukraine, suggesting that
sanctions were hinged to a peace deal there. Mr. Tillerson echoed that
line.

Finally, Mr. Trump started to mix geopolitical apples and oranges,
crossing issues in a way that Moscow deplores. He said maybe sanctions
could be lifted in exchange for a better deal on nuclear arms. The Trump
administration seemed to want the Kremlin to distance itself from Iran,
its ally in Syria, and from China.

``There is a cautious feeling about how Trump and his advisers
designated the possible ways of improving relations with Russia,'' said
Vladimir Frolov, an international affairs analyst. ``This has frightened
the Kremlin because it does not correspond to Russia's interests.''

Articles have just begun to appear in the Russian news media questioning
the need for improved ties with Washington.

Sergei A. Karaganov, a prominent political scientist perceived as close
to the Kremlin, wrote that Russia's foreign policy was a success and
that it should stay the course. He did not even mention Mr. Trump.
Fyodor Lukyanov, another establishment voice, wrote that Moscow risked
alienating a host of new important friends if it drew too close to
Washington.

On Monday, one of the first op-ed articles depicting Mr. Trump as
erratic appeared in Moskovsky Komsomolets, a popular tabloid. Mr. Trump
provoked an immediate constitutional crisis, the piece said, so who
could guarantee that his policy toward Russia would be consistent?

Of course, Mr. Trump still attracts defenders.

Margarita Simonyan, the head of satellite channel RT, the international
propaganda arm of the Kremlin, said that Western elites hate Mr. Trump
because he considers Russia a normal country. ``Anybody who says aloud
that Russia is normal is either an idiot or a provocateur or both,'' she
wrote on her blog.

The idea that Mr. Flynn was forced to resign over contacts with the
Russian ambassador, Sergey I. Kislyak, fed the suspicion that relations
with Moscow were the main target and that Russophobia was again stalking
Washington. Accusations that Russia interfered in the American elections
have generally been dismissed on these grounds.

Since Mr. Trump's victory there has also been a quiet drumbeat in
Moscow, where conspiracy theories are never far below the surface, that
the American establishment would overthrow him.

``Either Trump has not found the necessary independence and has been
driven into a corner,'' wrote Konstantin Kosachev, the head of the
international affairs committee in the upper house of Parliament. ``Or
Russophobia has permeated the new administration from top to bottom.''

Alexei Pushkov, another lawmaker, said on Twitter that after Mr. Flynn,
Mr. Trump himself might be the next target.

Dmitry S. Peskov, the spokesman for Mr. Putin, declined to comment on
Tuesday about the resignation, calling it an internal American affair.
Just last Friday, in an evident attempt to help Mr. Flynn, Mr. Peskov
had denied that the American official and the Russian ambassador had
discussed sanctions. In resigning, Mr. Flynn conceded that they had.

Mr. Peskov called it premature to predict the course of Russian-American
relations.

The first face-to-face meeting between two senior officials could come
this Thursday when Mr. Tillerson might meet with his Russian
counterpart, Sergey V. Lavrov, on the sidelines of a meeting of foreign
ministers from the G-20 countries in Bonn, Germany.

Mr. Trump and Mr. Putin spoke by telephone in late January, but no
meeting is anticipated before this summer.

The Kremlin is expected to spend the coming months trying to tamp down
the exaggerated public expectations already focused on that first
summit, said Mr. von Eggert, the political commentator.

``In these circumstances,'' he said, ``I think what remains for the
Kremlin is to sit and wait.''

Advertisement

\protect\hyperlink{after-bottom}{Continue reading the main story}

\hypertarget{site-index}{%
\subsection{Site Index}\label{site-index}}

\hypertarget{site-information-navigation}{%
\subsection{Site Information
Navigation}\label{site-information-navigation}}

\begin{itemize}
\tightlist
\item
  \href{https://help.nytimes.com/hc/en-us/articles/115014792127-Copyright-notice}{©~2020~The
  New York Times Company}
\end{itemize}

\begin{itemize}
\tightlist
\item
  \href{https://www.nytco.com/}{NYTCo}
\item
  \href{https://help.nytimes.com/hc/en-us/articles/115015385887-Contact-Us}{Contact
  Us}
\item
  \href{https://www.nytco.com/careers/}{Work with us}
\item
  \href{https://nytmediakit.com/}{Advertise}
\item
  \href{http://www.tbrandstudio.com/}{T Brand Studio}
\item
  \href{https://www.nytimes.com/privacy/cookie-policy\#how-do-i-manage-trackers}{Your
  Ad Choices}
\item
  \href{https://www.nytimes.com/privacy}{Privacy}
\item
  \href{https://help.nytimes.com/hc/en-us/articles/115014893428-Terms-of-service}{Terms
  of Service}
\item
  \href{https://help.nytimes.com/hc/en-us/articles/115014893968-Terms-of-sale}{Terms
  of Sale}
\item
  \href{https://spiderbites.nytimes.com}{Site Map}
\item
  \href{https://help.nytimes.com/hc/en-us}{Help}
\item
  \href{https://www.nytimes.com/subscription?campaignId=37WXW}{Subscriptions}
\end{itemize}
