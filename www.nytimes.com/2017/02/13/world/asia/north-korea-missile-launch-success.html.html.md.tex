Sections

SEARCH

\protect\hyperlink{site-content}{Skip to
content}\protect\hyperlink{site-index}{Skip to site index}

\href{https://www.nytimes.com/section/world/asia}{Asia Pacific}

\href{https://myaccount.nytimes.com/auth/login?response_type=cookie\&client_id=vi}{}

\href{https://www.nytimes.com/section/todayspaper}{Today's Paper}

\href{/section/world/asia}{Asia Pacific}\textbar{}North Korea Claims
Progress on Long-Range Goal With Missile Test

\url{https://nyti.ms/2kALk9S}

\begin{itemize}
\item
\item
\item
\item
\item
\end{itemize}

Advertisement

\protect\hyperlink{after-top}{Continue reading the main story}

Supported by

\protect\hyperlink{after-sponsor}{Continue reading the main story}

\hypertarget{north-korea-claims-progress-on-long-range-goal-with-missile-test}{%
\section{North Korea Claims Progress on Long-Range Goal With Missile
Test}\label{north-korea-claims-progress-on-long-range-goal-with-missile-test}}

\includegraphics{https://static01.nyt.com/images/2017/02/14/us/14KOREA-1/14KOREA-1-articleInline.jpg?quality=75\&auto=webp\&disable=upscale}

By \href{http://www.nytimes.com/by/choe-sang-hun}{Choe Sang-Hun} and
\href{http://www.nytimes.com/by/david-e-sanger}{David E. Sanger}

\begin{itemize}
\item
  Feb. 13, 2017
\item
  \begin{itemize}
  \item
  \item
  \item
  \item
  \item
  \end{itemize}
\end{itemize}

SEOUL, South Korea --- North Korea claimed on Monday that it had
successfully tested a new type of nuclear-capable missile, one that uses
a solid-fuel technology that American experts say will make it easier
for the country to hide its arsenal underground and roll its missiles
out for quick launch.

The test took place on Sunday (Saturday evening in the United States)
and was dramatic enough that aides to President Trump and Prime Minister
Shinzo Abe of Japan interrupted their dinner at the Mar-a-Lago resort in
Florida to bring them early reports of the launch.

Initially there was concern that North Korea's leader, Kim Jong-un, had
made good on his threat to test an intercontinental ballistic missile,
which one day may be able to reach the United States. Before dinner was
over, it was clear that was not the case.

North Korea's news service, KCNA, announced that the launch involved a
new missile, called the Pukguksong-2, which appears to be based on the
design of a submarine-launched missile it tested last year.

The Sunday test went only about 310 miles, falling harmlessly into the
sea after following a high-arc trajectory that took it briefly into
space. That is well short of the estimated real range of the missile, of
700 or 800 miles.

But the importance of the launch was not the missile's range --- though
it could reach much of Japan --- but in how hard it would be for the
United States, Japan or South Korea to have warning of a launch in a
real conflict.

The launch of older rockets provides warning time because the loading of
liquid fuel takes hours, and can usually be spotted by satellites.

Solid-fuel rockets like the new Pukguksong-2, if the North Korean
description is accurate, could provide little advance warning time. They
can be stored on mobile launchers, rolled out and prepared for launch in
minutes. The North said the test was conducted from a self-propelled
mobile launcher.

``All of these factors would make it much harder to find and
pre-emptively destroy the Pukguksong-2,'' John Schilling, a missile
expert, wrote on Monday on \href{http://38north.org/}{38 North}, an
online publication that specializes in North Korea.

For Mr. Trump, the new weapon complicates the problem of countering
North Korea's missile and nuclear program. It would be far harder for
Mr. Trump to threaten to strike North Korean launch sites if the
country's mountainous terrain is hiding scores of mobile missiles in
tunnels.

KCNA said that Mr. Kim inspected the test, which seemed to suggest that
he may have been present at the launch site.

``He expressed great satisfaction over the possession of another
powerful nuclear attack means, which adds to the tremendous might of the
country,'' the news agency said, using its typically boastful tone.

After previous tests, officials representing Washington and Beijing
agreed to issue statements condemning them. It remains to be seen what
posture the Trump administration will take --- and how China will react.

On Monday, the United Nations Security Council said it ``strongly
condemns'' the missile test, while the American ambassador, Nikki R.
Haley, warned in a statement that the Trump administration would seek to
hold Pyongyang accountable ``not with our words but with our actions.''
She did not elaborate and declined to speak with reporters.

North Korea has had a spotty record in test-launching the model known as
Musudan, which had been the North's only known intermediate-range
ballistic missile until the Pukguksong-2 was unveiled on Sunday. Its
last Musudan test, in October,
\href{https://www.nytimes.com/2016/10/20/world/asia/north-korea-musudan-missile-failure.html}{ended
in failure}.

``Now our rocket industry has radically turned into high-thrust
solid-fuel-powered engine from liquid-fuel rocket engine and rapidly
developed into a development- and creation-oriented industry, not just
copying samples,'' Mr. Kim was quoted as saying.

North Korea said the new missile was based on the solid-fuel,
submarine-launched ballistic missile, or SLBM. After several failed
attempts, the North said in August that it had
\href{https://www.nytimes.com/2016/08/25/world/asia/north-korea-kim-jong-un-missile-test.html}{successfully
launched the SLBM}, claiming that the continental United States, as well
as American military bases in the Pacific, were now within the range of
its missiles, an assertion that military experts questioned.

Analysts and defense officials in the region said that North Korea was
still years away from achieving the ability that Mr. Kim claimed. The
country still does not have submarines large and advanced enough to
travel long distances to attack distant targets across the Pacific
without being detected, they said.

But the North's tests of SLBMs and the Pukguksong-2 demonstrated the
advances the secretive country had made in its efforts to enhance the
range and stealth of its missiles, South Korean military officials said.
On Monday, North Korea said it launched its Pukguksong-2 at a sharp
angle to keep it from landing too close to Japan, indicating that it
could have flown further than 310 miles if it had launched it at a
normal angle.

Although North Korea has never fired an intercontinental ballistic
missile across the Pacific, it has boasted of successfully testing
crucial technologies in the past year. In March, it reported
\href{https://www.nytimes.com/2016/03/25/world/asia/north-korea-solid-fuel-rocket-engine.html}{the
successful ground test} of a newly designed solid fuel rocket engine. A
month later, it reported
\href{https://www.nytimes.com/2016/04/10/world/asia/north-korea-says-it-successfully-tested-missile-engine.html}{a
successful ground test} of a new intercontinental ballistic missile
engine.

Mr. Kim reminded the region of his missile threats
\href{https://www.nytimes.com/2017/01/01/world/asia/north-korea-intercontinental-ballistic-missile-test-kim-jong-un.html}{during
his New Year's Day speech}, in which he claimed that his country was in
a ``final stage'' of preparing to conduct its first test of an
intercontinental ballistic missile. North Korea later said it could
flight-test one ``anytime, anywhere.''

When he visited South Korea this month on
\href{https://www.nytimes.com/2017/02/05/us/politics/jim-mattis-south-korea-japan.html}{his
first official trip abroad}, Jim Mattis, the United States defense
secretary, emphasized the importance of
\href{https://www.nytimes.com/2017/02/02/world/asia/james-mattis-us-korea-thaad.html}{deploying
an advanced missile defense system}, known as Thaad, in the country this
year to counter the North Korean threat.

Tom Karako, a proliferation expert at the
\href{https://www.csis.org/}{Center for Strategic and International
Studies}, a policy institute in Washington, said that President Barack
Obama's strategy of ``strategic patience has failed'' and that it was
time for the new Trump administration to take a different approach.

``This weekend's launch is part of a larger pattern of aggressive
testing that confirms the North's intent to produce more capable, more
lethal and more survivable systems,'' he said.

Advertisement

\protect\hyperlink{after-bottom}{Continue reading the main story}

\hypertarget{site-index}{%
\subsection{Site Index}\label{site-index}}

\hypertarget{site-information-navigation}{%
\subsection{Site Information
Navigation}\label{site-information-navigation}}

\begin{itemize}
\tightlist
\item
  \href{https://help.nytimes.com/hc/en-us/articles/115014792127-Copyright-notice}{©~2020~The
  New York Times Company}
\end{itemize}

\begin{itemize}
\tightlist
\item
  \href{https://www.nytco.com/}{NYTCo}
\item
  \href{https://help.nytimes.com/hc/en-us/articles/115015385887-Contact-Us}{Contact
  Us}
\item
  \href{https://www.nytco.com/careers/}{Work with us}
\item
  \href{https://nytmediakit.com/}{Advertise}
\item
  \href{http://www.tbrandstudio.com/}{T Brand Studio}
\item
  \href{https://www.nytimes.com/privacy/cookie-policy\#how-do-i-manage-trackers}{Your
  Ad Choices}
\item
  \href{https://www.nytimes.com/privacy}{Privacy}
\item
  \href{https://help.nytimes.com/hc/en-us/articles/115014893428-Terms-of-service}{Terms
  of Service}
\item
  \href{https://help.nytimes.com/hc/en-us/articles/115014893968-Terms-of-sale}{Terms
  of Sale}
\item
  \href{https://spiderbites.nytimes.com}{Site Map}
\item
  \href{https://help.nytimes.com/hc/en-us}{Help}
\item
  \href{https://www.nytimes.com/subscription?campaignId=37WXW}{Subscriptions}
\end{itemize}
