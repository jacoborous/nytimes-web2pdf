Sections

SEARCH

\protect\hyperlink{site-content}{Skip to
content}\protect\hyperlink{site-index}{Skip to site index}

\href{https://www.nytimes.com/section/technology/personaltech}{Personal
Tech}

\href{https://myaccount.nytimes.com/auth/login?response_type=cookie\&client_id=vi}{}

\href{https://www.nytimes.com/section/todayspaper}{Today's Paper}

\href{/section/technology/personaltech}{Personal
Tech}\textbar{}Automatic Update Headaches

\url{https://nyti.ms/2lDM6nB}

\begin{itemize}
\item
\item
\item
\item
\item
\end{itemize}

Advertisement

\protect\hyperlink{after-top}{Continue reading the main story}

Supported by

\protect\hyperlink{after-sponsor}{Continue reading the main story}

\href{/column/tech-tip}{Tech Tip}

\hypertarget{automatic-update-headaches}{%
\section{Automatic Update Headaches}\label{automatic-update-headaches}}

By \href{http://www.nytimes.com/by/j-d-biersdorfer}{J. D. Biersdorfer}

\begin{itemize}
\item
  Feb. 24, 2017
\item
  \begin{itemize}
  \item
  \item
  \item
  \item
  \item
  \end{itemize}
\end{itemize}

\textbf{Q.} \emph{Microsoft seems to update Windows 10 constantly and it
results in computer freezes and other issues for me. If I wanted to
switch, is there a real advantage to Apple software as opposed to
Microsoft?}

\textbf{A.} System software updates --- and upgrade problems --- are not
a Windows-only issue. Apple has rolled out system upgrades that
\href{http://www.gottabemobile.com/common-os-x-el-capitan-problems-fixes/}{have
caused disruptions for Mac users when certain hardware and software
stopped working reliably}, sending upset users to
\href{https://discussions.apple.com/community/mac_os}{online support
forums} and
\href{http://osxdaily.com/2016/09/24/troubleshooting-macos-sierra-problems/}{tech-help
sites}. Subsequent system updates often fix problems, but outdated
driver software and programs incompatible with the new system software
can hinder any platform.

Image

Like Windows, the Mac operating system can automatically download
software updates, but it gives you a few options on when to install
them.Credit...The New York Times

Some users have accused Microsoft of
\href{http://www.computerworld.com/article/3030564/microsoft-windows/microsoft-uses-the-force-you-will-upgrade-to-windows-10.html}{being
overly aggressive} in pushing out Windows 10 system upgrades and
software updates to computers, especially when the
\href{https://www.thurrott.com/windows/windows-10/81659/microsoft-delivers-yet-another-broken-windows-10-update}{updates
caused problems} --- like
\href{http://www.digitaltrends.com/computing/microsoft-kb3206632-fixes-network-connectivity-problem/}{knocking
out internet connections}. Apple's
\href{https://support.apple.com/en-us/HT201475}{settings for its Mac
operating system} include an option to automatically download major
system upgrades, but asks first before installing them. You can also
stop updates from downloading automatically. (The
\href{https://support.apple.com/en-us/HT201541}{settings that control
Mac OS and app updates} can be found under the Apple menu in the App
Store area.)

Beyond \href{https://support.apple.com/en-us/HT204087}{transferring your
files from a PC to a Mac}, switching operating systems is a big step.
Along with the new hardware, you will have to get Mac versions of the
programs installed on your hard drive. Web-based apps and services
usually work the same, but you may find quirks between the platforms.

Companies update their systems to keep on top of security and
performance issues, introduce new features, and phase out aging code.
Regardless of the operating system you use, keeping it up-to-date
generally keeps your data safer, but maintaining a regular backup of
your
\href{https://support.microsoft.com/en-us/help/17143/windows-10-back-up-your-files}{Windows}
or \href{https://support.apple.com/en-us/HT201250}{Mac} system in case
an update goes wrong can provide a little peace of mind as well.

Advertisement

\protect\hyperlink{after-bottom}{Continue reading the main story}

\hypertarget{site-index}{%
\subsection{Site Index}\label{site-index}}

\hypertarget{site-information-navigation}{%
\subsection{Site Information
Navigation}\label{site-information-navigation}}

\begin{itemize}
\tightlist
\item
  \href{https://help.nytimes.com/hc/en-us/articles/115014792127-Copyright-notice}{©~2020~The
  New York Times Company}
\end{itemize}

\begin{itemize}
\tightlist
\item
  \href{https://www.nytco.com/}{NYTCo}
\item
  \href{https://help.nytimes.com/hc/en-us/articles/115015385887-Contact-Us}{Contact
  Us}
\item
  \href{https://www.nytco.com/careers/}{Work with us}
\item
  \href{https://nytmediakit.com/}{Advertise}
\item
  \href{http://www.tbrandstudio.com/}{T Brand Studio}
\item
  \href{https://www.nytimes.com/privacy/cookie-policy\#how-do-i-manage-trackers}{Your
  Ad Choices}
\item
  \href{https://www.nytimes.com/privacy}{Privacy}
\item
  \href{https://help.nytimes.com/hc/en-us/articles/115014893428-Terms-of-service}{Terms
  of Service}
\item
  \href{https://help.nytimes.com/hc/en-us/articles/115014893968-Terms-of-sale}{Terms
  of Sale}
\item
  \href{https://spiderbites.nytimes.com}{Site Map}
\item
  \href{https://help.nytimes.com/hc/en-us}{Help}
\item
  \href{https://www.nytimes.com/subscription?campaignId=37WXW}{Subscriptions}
\end{itemize}
