Sections

SEARCH

\protect\hyperlink{site-content}{Skip to
content}\protect\hyperlink{site-index}{Skip to site index}

\href{https://www.nytimes.com/section/politics}{Politics}

\href{https://myaccount.nytimes.com/auth/login?response_type=cookie\&client_id=vi}{}

\href{https://www.nytimes.com/section/todayspaper}{Today's Paper}

\href{/section/politics}{Politics}\textbar{}Turmoil at the National
Security Council, From the Top Down

\url{https://nyti.ms/2l4OBBM}

\begin{itemize}
\item
\item
\item
\item
\item
\item
\end{itemize}

Advertisement

\protect\hyperlink{after-top}{Continue reading the main story}

Supported by

\protect\hyperlink{after-sponsor}{Continue reading the main story}

\hypertarget{turmoil-at-the-national-security-council-from-the-top-down}{%
\section{Turmoil at the National Security Council, From the Top
Down}\label{turmoil-at-the-national-security-council-from-the-top-down}}

\includegraphics{https://static01.nyt.com/images/2017/02/13/us/13nsc-web01/13nsc-web01-articleLarge.jpg?quality=75\&auto=webp\&disable=upscale}

By \href{http://www.nytimes.com/by/david-e-sanger}{David E. Sanger},
\href{http://www.nytimes.com/by/eric-schmitt}{Eric Schmitt} and
\href{http://www.nytimes.com/by/peter-baker}{Peter Baker}

\begin{itemize}
\item
  Feb. 12, 2017
\item
  \begin{itemize}
  \item
  \item
  \item
  \item
  \item
  \item
  \end{itemize}
\end{itemize}

\emph{Over the weekend, The Times published this investigation into the
agency that Michael T. Flynn was leading before he stepped down. Read
our}
\href{https://www.nytimes.com/2017/02/13/us/politics/donald-trump-national-security-adviser-michael-flynn.html}{\emph{updated
article}} \emph{on Mr. Flynn's resignation.}

WASHINGTON --- These are chaotic and anxious days inside the National
Security Council, the traditional center of management for a president's
dealings with an uncertain world.

Three weeks into the Trump administration, council staff members get up
in the morning, read President Trump's Twitter posts and struggle to
make policy to fit them. Most are kept in the dark about what Mr. Trump
tells foreign leaders in his phone calls. Some staff members have turned
to encrypted communications to talk with their colleagues, after hearing
that Mr. Trump's top advisers are considering an ``insider threat''
program that could result in monitoring cellphones and emails for leaks.

The national security adviser, Michael T. Flynn, has hunkered down since
investigators began looking into what, exactly,
\href{https://www.nytimes.com/2017/02/09/us/flynn-is-said-to-have-talked-to-russians-about-sanctions-before-trump-took-office.html}{he
told the Russian ambassador to the United States} about the lifting of
sanctions imposed in the last days of the Obama administration, and
whether he misled Vice President Mike Pence about those conversations.
His survival in the job may hang in the balance.

Although Mr. Trump suggested to reporters aboard Air Force One on Friday
that he was unaware of the latest questions swirling around Mr. Flynn's
dealings with Russia, aides said over the weekend in Florida --- where
Mr. Flynn accompanied the president and Japan's prime minister, Shinzo
Abe --- that Mr. Trump was closely monitoring the reaction to Mr.
Flynn's conversations. There are transcripts of a conversation in at
least one phone call, recorded by American intelligence agencies that
wiretap foreign diplomats, which may determine Mr. Flynn's future.

\href{https://www.nytimes.com/2017/02/11/us/politics/stephen-miller-donald-trump-adviser.html}{Stephen
Miller}, the White House senior policy adviser, was circumspect on
Sunday about Mr. Flynn's future. Mr. Miller said on NBC's ``Meet the
Press'' that possibly misleading the vice president on communications
with Russia was ``a sensitive matter.'' Asked if Mr. Trump still had
confidence in Mr. Flynn, Mr. Miller responded, ``That's a question for
the president.''

This account of life inside the council --- offices made up of several
hundred career civil servants who advise the president on
counterterrorism, foreign policy, nuclear deterrence and other issues of
war and peace --- is based on conversations with more than two dozen
current and former council staff members and others throughout the
government. All spoke on the condition that they not be quoted by name
for fear of reprisals.

``It's so far a very dysfunctional N.S.C.,'' Representative Adam B.
Schiff of California, the senior Democrat on the House Intelligence
Committee, said in a telephone interview.

In a telephone conversation on Sunday afternoon, K. T. McFarland, the
deputy national security adviser, said that early meetings of the
council were brisker, tighter and more decisive than in the past, but
she acknowledged that career officials were on edge. ``Not only is this
a new administration, but it is a different party, and Donald Trump was
elected by people who wanted the status quo thrown out,'' said Ms.
McFarland, a veteran of the Reagan administration who most recently
worked for Fox News. ``I think it would be a mistake if we didn't have
consternation about the changes --- most of the cabinet haven't even
been in government before.''

There is always a shakedown period for any new National Security
Council, whose staff is drawn from the State Department, the Pentagon
and other agencies and is largely housed opposite the White House in the
Eisenhower Executive Office Building.

President Barack Obama replaced his first national security adviser,
\href{https://thecaucus.blogs.nytimes.com/2010/10/08/donilon-to-replace-jones-as-national-security-adviser/}{Gen.
James Jones}, a four-star former supreme allied commander in Europe,
after concluding that the general was a bad fit for the administration.
The first years of President George W. Bush's council were defined by
clashes among experienced bureaucratic infighters --- Dick Cheney,
Donald Rumsfeld and Colin Powell among them --- and by decisions that
often took place outside official channels.

But what is happening under the Trump White House is different,
officials say, and not just because of Mr. Trump's Twitter foreign
policy. (Two officials said that at one recent meeting, there was talk
of feeding suggested Twitter posts to the president so the council's
staff would have greater influence.)

A number of staff members who did not want to work for Mr. Trump have
returned to their regular agencies, leaving a larger-than-usual hole in
the experienced bureaucracy. Many of those who remain, who see
themselves as apolitical civil servants, have been disturbed by displays
of overt partisanship. At an all-hands meeting about two weeks into the
new administration, Ms. McFarland told the group it needed to ``make
America great again,'' numerous staff members who were there said.

New Trump appointees are carrying coffee mugs with that Trump campaign
slogan into meetings with foreign counterparts, one staff member said.

\includegraphics{https://static01.nyt.com/images/2017/01/19/us/19nsc1/19nsc1-videoSixteenByNine3000-v3.jpg}

Nervous staff members recently met late at night at a bar a few blocks
from the White House and talked about purging their social media
accounts of any suggestion of anti-Trump sentiments.

Mr. Trump's council staff draws heavily from the military --- often
people who had ties to Mr. Flynn when he served as a senior military
intelligence officer and then
\href{https://www.nytimes.com/2016/12/03/us/politics/in-national-security-adviser-michael-flynn-experience-meets-a-prickly-past.html}{as
the director of the Defense Intelligence Agency} before he was forced
out of the job. Many of the first ideas that have been floated have
involved military, rather than diplomatic, initiatives.

\includegraphics{https://static01.nyt.com/images/2017/02/13/us/13nsc-web02/13nsc-web02-articleLarge.jpg?quality=75\&auto=webp\&disable=upscale}

Last week, Defense Secretary Jim Mattis was exploring whether the Navy
could intercept and board an Iranian ship to look for contraband weapons
possibly headed to Houthi fighters in Yemen. The potential interdiction
seemed in keeping with recent instructions from Mr. Trump, reinforced in
meetings with Mr. Mattis and Secretary of State Rex W. Tillerson, to
crack down on Iran's support of terrorism.

But the ship was in international waters in the Arabian Sea, according
to two officials. Mr. Mattis ultimately decided to set the operation
aside, at least for now. White House officials said that was because
news of the impending operation leaked, a threat to security that has
helped fuel the move for the insider threat program. But others doubt
whether there was enough basis in international law, and wondered what
would happen if, in the early days of an administration that has already
seen one
\href{https://www.nytimes.com/2017/02/01/world/middleeast/donald-trump-yemen-commando-raid-questions.html}{botched
military action in Yemen}, American forces were suddenly in a firefight
with the Iranian Navy.

Ms. McFarland often draws on her television experience to make clear to
officials that they need to make their points in council meetings
quickly, and she signals when to wrap up, several participants said.

And while Mr. Obama liked policy option papers that were three to six
single-spaced pages, council staff members are now being told to keep
papers to a single page, with lots of graphics and maps.

``The president likes maps,'' one official said.

Paper flow, the lifeblood of the bureaucracy, has been erratic. A senior
Pentagon official saw a draft executive order on prisoner treatment only
through unofficial rumors and news media leaks. He called the White
House to find out if it was real and said he had concerns but was not
sure if he was authorized to make suggestions.

Officials said that the absence of an orderly flow of council documents,
ultimately the responsibility of Mr. Flynn, explained why Mr. Mattis and
\href{https://www.nytimes.com/2017/01/11/us/politics/mike-pompeo-cia-trump-nominee.html}{Mike
Pompeo, the director of the C.I.A.}, never saw a number of Mr. Trump's
executive orders before they were issued. One order had to be amended
after it was made public, to reassure Mr. Pompeo that he had a regular
seat on the council.

White House officials say that was a blunder, and that the process of
reviewing executive orders has been straightened out by Reince Priebus,
the White House chief of staff.

Image

Stephen K. Bannon, center, Mr. Trump's top strategist, who was made a
member of the National Security Council two weeks ago.Credit...Stephen
Crowley/The New York Times

Still, Mr. Flynn presents additional complications beyond his
conversations with the Russian ambassador. His aides say he is insecure
about whether his unfettered access to Mr. Trump during the campaign is
being scaled back and about a shadow council created by Stephen K.
Bannon, Mr. Trump's top strategist,
\href{https://www.nytimes.com/2017/01/29/us/stephen-bannon-donald-trump-national-security-council.html}{who
was invited to attend meetings} of the ``principals committee'' of the
council two weeks ago. For his part, Mr. Bannon sees the United States
as headed toward an inevitable confrontation with two adversaries ---
China and Iran.

Mr. Flynn finds himself in a continuing conflict with the intelligence
agencies, whose work on Russia and other issues he has dismissed as
subpar and politically biased. Last week, in an incident
\href{http://www.politico.com/story/2017/02/mike-flynn-nsa-aide-trump-234923}{first
reported by Politico}, one of Mr. Flynn's top deputies, Robin Townley,
was denied the high-level security clearance he needed before he could
take up his job on the council as the senior director for Africa.

It was not clear what in Mr. Townley's past disqualified him, and in
every administration some officials are denied clearances. But some saw
the intelligence community striking back.

Two people with direct access to the White House leadership said Mr.
Flynn was surprised to learn that the State Department and Congress play
a pivotal role in foreign arms sales and technology transfers. So it was
a rude discovery that Mr. Trump could not simply order the Pentagon to
send more weapons to Saudi Arabia --- which is clamoring to have an
Obama administration ban on the sale of cluster bombs and
precision-guided weapons lifted --- or to deliver bigger weapons
packages to the United Arab Emirates.

Several staff members said that Mr. Flynn, who was a career Army
officer, was not familiar with how to call up the National Guard in an
emergency --- for, say, a natural disaster like Hurricane Katrina or the
detonation of a dirty bomb in an American city.

At the all-hands meeting, Mr. Flynn talked about the importance of a
balanced work life, taking care of family, and using the time at the
council to gain experience that would help staff members in other parts
of the government. At one point, the crowd was asked for a show of hands
of how many expected to be working at the White House in a year.

Mr. Flynn turned to Ms. McFarland and, in what seemed to be a
self-deprecating joke, said, ``I wonder if we'll be here a year from
now?''

Advertisement

\protect\hyperlink{after-bottom}{Continue reading the main story}

\hypertarget{site-index}{%
\subsection{Site Index}\label{site-index}}

\hypertarget{site-information-navigation}{%
\subsection{Site Information
Navigation}\label{site-information-navigation}}

\begin{itemize}
\tightlist
\item
  \href{https://help.nytimes.com/hc/en-us/articles/115014792127-Copyright-notice}{©~2020~The
  New York Times Company}
\end{itemize}

\begin{itemize}
\tightlist
\item
  \href{https://www.nytco.com/}{NYTCo}
\item
  \href{https://help.nytimes.com/hc/en-us/articles/115015385887-Contact-Us}{Contact
  Us}
\item
  \href{https://www.nytco.com/careers/}{Work with us}
\item
  \href{https://nytmediakit.com/}{Advertise}
\item
  \href{http://www.tbrandstudio.com/}{T Brand Studio}
\item
  \href{https://www.nytimes.com/privacy/cookie-policy\#how-do-i-manage-trackers}{Your
  Ad Choices}
\item
  \href{https://www.nytimes.com/privacy}{Privacy}
\item
  \href{https://help.nytimes.com/hc/en-us/articles/115014893428-Terms-of-service}{Terms
  of Service}
\item
  \href{https://help.nytimes.com/hc/en-us/articles/115014893968-Terms-of-sale}{Terms
  of Sale}
\item
  \href{https://spiderbites.nytimes.com}{Site Map}
\item
  \href{https://help.nytimes.com/hc/en-us}{Help}
\item
  \href{https://www.nytimes.com/subscription?campaignId=37WXW}{Subscriptions}
\end{itemize}
