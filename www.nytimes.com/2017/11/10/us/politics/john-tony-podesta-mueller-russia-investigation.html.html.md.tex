Sections

SEARCH

\protect\hyperlink{site-content}{Skip to
content}\protect\hyperlink{site-index}{Skip to site index}

\href{https://www.nytimes.com/section/politics}{Politics}

\href{https://myaccount.nytimes.com/auth/login?response_type=cookie\&client_id=vi}{}

\href{https://www.nytimes.com/section/todayspaper}{Today's Paper}

\href{/section/politics}{Politics}\textbar{}Russia Scandal Befalls Two
Brothers: John and Tony Podesta

\url{https://nyti.ms/2ji8jKW}

\begin{itemize}
\item
\item
\item
\item
\item
\end{itemize}

Advertisement

\protect\hyperlink{after-top}{Continue reading the main story}

Supported by

\protect\hyperlink{after-sponsor}{Continue reading the main story}

\hypertarget{russia-scandal-befalls-two-brothers-john-and-tony-podesta}{%
\section{Russia Scandal Befalls Two Brothers: John and Tony
Podesta}\label{russia-scandal-befalls-two-brothers-john-and-tony-podesta}}

\includegraphics{https://static01.nyt.com/images/2017/11/11/us/11dc-Podesta-tony/03dc-Podesta-tony-articleLarge.jpg?quality=75\&auto=webp\&disable=upscale}

By \href{https://www.nytimes.com/by/kenneth-p-vogel}{Kenneth P. Vogel}

\begin{itemize}
\item
  Nov. 10, 2017
\item
  \begin{itemize}
  \item
  \item
  \item
  \item
  \item
  \end{itemize}
\end{itemize}

WASHINGTON --- One is a rail-thin liberal idealist who spent his career
in government, on campaigns and at think tanks. The other is an
overweight pragmatist who made a fortune lobbying for all manner of
liberal boogeymen.

And now, in a twist with Shakespearean undertones, the two influential
Washington brothers have found themselves on opposite sides of the
scandals over Russian interference in the 2016 election.

John D. Podesta, perhaps more than anyone except Hillary Clinton, was a
victim of the
\href{https://www.nytimes.com/2016/12/13/us/politics/russia-hack-election-dnc.html}{Russian
cyberassault} on allies of her presidential campaign, of which he was
the chairman. Emails stolen from his personal Gmail account were
dribbled out by WikiLeaks last fall, revealing the embarrassing rifts
roiling the campaign and Washington's Democratic establishment. He has
pushed hard for an aggressive investigation of Russia's role in the
election.

Mr. Podesta's older brother, Tony Podesta, has been ensnared in the
investigation by Robert S. Mueller III, the special counsel, into
Russia's meddling in the race and whether it involved any associates of
President Trump. The efforts by his firm, the Podesta Group, to win
support for the agenda of Viktor F. Yanukovych, the Russia-aligned
former president of Ukraine, were cited in an indictment handed down
last month against two former campaign aides to Mr. Trump, Paul Manafort
and Rick Gates, who arranged the Ukrainian lobbying work.

Neither Tony Podesta nor anyone at the Podesta Group has been publicly
charged in the case. But Mr. Mueller has
\href{https://www.nytimes.com/2017/09/21/us/politics/law-firm-faces-questions-for-ukraine-work-with-manafort.html}{subpoenaed
the firm and its employees} for documents and testimony related to their
work and interviewed roughly half a dozen people about Tony Podesta's
involvement. And as the Podesta Group has withered under the scrutiny,
its employees were informed in a tearful meeting on Thursday that they
may stop receiving paychecks after next week, according to people in
attendance.

For some Democrats, there is little question over who deserves the most
sympathy.

``Sure, the Clinton campaign made some mistakes, but that's not like
representing a dictatorship,'' said James Carville, who has known John
Podesta for years through Democratic political circles. Mr. Carville
said he was less acquainted with Tony Podesta, but added that ``by
reputation, you can't put them in the same book. John is one of the
straightest guys I know. Everybody says, `Can you believe that those
guys are brothers?'''

Mr. Trump, for his part, went after both brothers last month, lumping
them together as
\href{https://twitter.com/realDonaldTrump/status/925364408364171265}{emblematic
of ``the swamp}'' of Washington special interests against which he
campaigned.

In fact, the Podesta brothers, who were raised in a modest two-flat on
\href{http://www.chicagomag.com/Chicago-Magazine/Felsenthal-Files/December-2013/How-the-Podesta-Brothers-Rose-From-the-39th-Ward-to-the-White-House/}{Chicago's
northwest side} by a Greek-American mother and an Italian-American
father, represent very different strains of a Washington establishment
now under attack from left and right.

John Podesta, 68, served as a top White House aide to the last two
Democratic presidents, Bill Clinton and Barack Obama, and was believed
\href{https://www.politico.com/story/2016/10/podesta-tops-clintons-short-list-for-chief-of-staff-230366}{to
be in line as White House chief of staff} if Mrs. Clinton had won the
presidency. In between campaign and White House stints, John Podesta
helped to create and run some of the leading institutions on the
American left, including the Center for American Progress think tank,
and provided policy and political advice to generations of Democratic
politicians and operatives.

Tony Podesta, 74, built one of the highest-grossing lobbying firms in
Washington, signing clients across industries and ideologies ---
including defense contractors like Lockheed Martin, banks like Wells
Fargo, drug makers like Mylan and foreign regimes like the government of
the former Egyptian dictator Hosni Mubarak. Tony Podesta is known for
his fund-raising --- in 2016, he donated or raised nearly \$900,000 for
Mrs. Clinton's campaign and the Democratic Party --- as well as for a
lavish lifestyle.

Had Mrs. Clinton prevailed, as official Washington expected, it most
likely would have cemented the legacies of both brothers --- John
Podesta as a key confidant to presidents and one of the most important
Democratic strategists of the past century, and Tony Podesta as the
go-to lobbyist for Democratic administrations. Instead, the Podestas
have been drawn into the vortex of investigations and conspiracy
theories that have enveloped Washington in the Trump era.

Tony Podesta
\href{https://www.nytimes.com/2017/10/30/us/politics/tony-podesta-resignation-lobbying.html?_r=0}{stepped
down from his firm} hours after it was obliquely referenced in the
indictments of Mr. Manafort and Mr. Gates, though he had been in talks
about leaving for months. It is unclear what will happen to his
investment in the Podesta Group, which has been hemorrhaging clients and
employees, both because of the mounting scrutiny and because businesses
are looking for lobbyists with connections to the Trump administration,
according to interviews and lobbying filings. Some firm partners are
starting a new firm next month called Cogent Strategies, in which Tony
Podesta will have no stake.

Last week's indictment did not name the Podesta Group or another firm
with which it worked on the Ukraine account, Mercury Public Affairs.
Instead, the indictment referred to them as ``two Washington, D.C.,
firms'' that were recruited by Mr. Manafort and Mr. Gates.

While both firms disclosed the work to Congress under less-rigorous
domestic lobbying rules, they did not initially register under the
Foreign Agents Registration Act. The indictment alleges that was done
intentionally --- by routing the money through a Brussels-based
nonprofit --- ``to minimize public disclosure of their lobbying
campaign'' that ``was under the ultimate direction'' of Mr. Yanukovych,
his party and his government.

Tony Podesta and his lawyers are working to navigate the issues raised
by the Mueller investigation and accompanying attacks from the right.
They have already
\href{https://www.documentcloud.org/documents/4164203-Tony-Podesta-Cease-amp-Desist-to-Tucker-Carlson.html}{demanded
a retraction} from the conservative news media on claims that
\href{https://www.realclearpolitics.com/video/2017/10/25/tucker_carlson_source_podesta_brothers_and_manafort_not_trump_central_figures_in_mueller_probe.html}{both
Podestas} worked with Mr. Manafort to advance Russian interests.

\includegraphics{https://static01.nyt.com/images/2017/11/03/us/politics/03dc-podesta2-john/merlin_112352714_b7a508cb-8f98-4542-849d-adc92809f6fc-articleLarge.jpg?quality=75\&auto=webp\&disable=upscale}

John Podesta, for his part, has dedicated his time since the election to
trying to expose the connections between Mr. Trump, his associates and
Russia.

After the postelection
\href{https://www.nytimes.com/2017/01/11/us/politics/donald-trump-russia-intelligence.html}{publication
of a dossier} by a former British spy into those connections --- which
included some salacious claims --- John Podesta met with Glenn Simpson,
the co-founder of the firm that commissioned the opposition research, to
compare notes on Russia's involvement, according to an associate of Mr.
Podesta.

During the general election season, the firm's research was
\href{https://www.nytimes.com/2017/10/24/us/politics/clinton-dnc-russia-dossier.html}{funded}by
the Clinton campaign and the Democratic National Committee, though John
Podesta has told congressional investigators that
\href{http://www.cnn.com/2017/10/26/politics/john-podesta-debbie-wasserman-schultz-trump-dossier/index.html}{he
had no knowledge of those payments}. The associate said the meeting came
as Mr. Simpson was considering whether, and how, his firm could continue
its Russia-related Trump research. A spokeswoman for Mr. Simpson's firm,
Fusion GPS, declined to comment.

John Podesta has helped raise millions of dollars from major donors with
whom he has
\href{https://www.politico.com/story/2016/11/john-podesta-donor-foundation-230571}{personal
and financial relationships} --- including the San Francisco mortgage
billionaire Herbert Sandler --- for nonprofits fighting the Trump
administration. With funding from Mr. Sandler, he helped begin a group
called Democracy Forward that is suing the Trump team on a number of
fronts, including at least
\href{https://democracyforward.org/wp-content/uploads/2017/09/DOJ-FOIA-Complaint.pdf}{one
lawsuit} intended partly to reveal whether Mr. Trump's aides tried to
influence the Russia investigation.

John Podesta ``could have gone into a bunker and vanished, but instead
he dedicated himself to fighting this,'' said Faiz Shakir, who worked
for him at the Center for American Progress and serves with him on the
board of Democracy Forward. ``He is very driven by exacting some measure
of justice for the unscrupulous form of politics that was utilized
against him and his side,'' said Mr. Shakir, who considers Mr. Podesta a
mentor.

Both brothers declined interview requests through their representatives.
But John Podesta provided a written response to a question about whether
he saw it as tragic --- or at least ironic --- that the Russia
investigation for which he has been advocating has ensnared his brother.

``The only tragedy is that Donald Trump is president and got there with
the Russians' help,'' he said. ``That's a tragedy for the American
people.''

For both John and Tony Podesta, the connections with Democratic politics
began at an early age. In 1970, they worked together on the Rev. Joseph
D. Duffey's antiwar Senate campaign in Connecticut, for which Tony
Podesta served as a top official, and his younger brother --- as well as
Bill and Hillary Clinton --- were volunteers.

Over the next 23 years, the brothers' paths intertwined. They wove in
and out of campaigns and car-pooled together to Georgetown University's
law school, from which they both earned degrees in 1976. Tony Podesta
joined the United States attorney's office in Washington, while John
Podesta went to work at the Justice Department.

After a few years, the brothers left those jobs --- John Podesta joining
the Democratic staff of the Senate Judiciary Committee, and Tony Podesta
becoming the founding president of the liberal advocacy group People for
the American Way.

In 1987, the brothers teamed up to create a lobbying and public
relations firm called Podesta Associates, which was seen as having a
liberal bearing, representing a mix of public interest groups and media
companies and associations. But after Mr. Clinton won the presidency in
1992 --- a campaign for which both Podestas worked --- the brothers'
paths diverged once and for all. John Podesta took a job in the White
House and divested his stake in the firm, which Tony Podesta pointed in
a less ideological direction.

By President Obama's second year in office, domestic lobbying revenues
for the Podesta Group had risen to more than \$29 million, buoyed in
part by the perception of access to the new administration.

The brothers remained close, even as their lifestyles took different
paths.

Tony Podesta moved into a 7,000-square-foot house in Washington's
exclusive Kalorama neighborhood, which he had bought for \$3.9 million
in addition to his homes in Australia and Venice. After three years of
renovations, he turned it into a fund-raising hot spot with a
\href{http://washingtonlife.com/2015/06/05/inside-homes-private-viewing/}{modern
art collection} and a wine cellar with
\href{http://www.washingtonpost.com/wp-dyn/content/article/2009/08/23/AR2009082302381_4.html}{thousands
of bottles}. His divorce from his second wife, 26 years his junior, was
widely chronicled in Washington.

John Podesta, who remains married to the mother of his three children,
lives in a relatively modest house near American University in Northwest
Washington, where his clan makes a habit of getting together for home
cooked pasta.

In the end, said Lanny Davis, who has known both Podestas since the
Duffey campaign, the brothers have much in common.

``Tony Podesta may have clients that are way to the right of him in
politics,'' he said, ``but neither John nor Tony Podesta, no matter who
their clients are, have withdrawn one inch from being liberal Democrats
with progressive values.''

Advertisement

\protect\hyperlink{after-bottom}{Continue reading the main story}

\hypertarget{site-index}{%
\subsection{Site Index}\label{site-index}}

\hypertarget{site-information-navigation}{%
\subsection{Site Information
Navigation}\label{site-information-navigation}}

\begin{itemize}
\tightlist
\item
  \href{https://help.nytimes.com/hc/en-us/articles/115014792127-Copyright-notice}{©~2020~The
  New York Times Company}
\end{itemize}

\begin{itemize}
\tightlist
\item
  \href{https://www.nytco.com/}{NYTCo}
\item
  \href{https://help.nytimes.com/hc/en-us/articles/115015385887-Contact-Us}{Contact
  Us}
\item
  \href{https://www.nytco.com/careers/}{Work with us}
\item
  \href{https://nytmediakit.com/}{Advertise}
\item
  \href{http://www.tbrandstudio.com/}{T Brand Studio}
\item
  \href{https://www.nytimes.com/privacy/cookie-policy\#how-do-i-manage-trackers}{Your
  Ad Choices}
\item
  \href{https://www.nytimes.com/privacy}{Privacy}
\item
  \href{https://help.nytimes.com/hc/en-us/articles/115014893428-Terms-of-service}{Terms
  of Service}
\item
  \href{https://help.nytimes.com/hc/en-us/articles/115014893968-Terms-of-sale}{Terms
  of Sale}
\item
  \href{https://spiderbites.nytimes.com}{Site Map}
\item
  \href{https://help.nytimes.com/hc/en-us}{Help}
\item
  \href{https://www.nytimes.com/subscription?campaignId=37WXW}{Subscriptions}
\end{itemize}
