Sections

SEARCH

\protect\hyperlink{site-content}{Skip to
content}\protect\hyperlink{site-index}{Skip to site index}

\href{https://www.nytimes.com/section/us}{U.S.}

\href{https://myaccount.nytimes.com/auth/login?response_type=cookie\&client_id=vi}{}

\href{https://www.nytimes.com/section/todayspaper}{Today's Paper}

\href{/section/us}{U.S.}\textbar{}Outdoor Jail, a Vestige of Joe
Arpaio's Tenure, Is Closing

\url{https://nyti.ms/2oAyQEl}

\begin{itemize}
\item
\item
\item
\item
\item
\end{itemize}

Advertisement

\protect\hyperlink{after-top}{Continue reading the main story}

Supported by

\protect\hyperlink{after-sponsor}{Continue reading the main story}

\hypertarget{outdoor-jail-a-vestige-of-joe-arpaios-tenure-is-closing}{%
\section{Outdoor Jail, a Vestige of Joe Arpaio's Tenure, Is
Closing}\label{outdoor-jail-a-vestige-of-joe-arpaios-tenure-is-closing}}

\includegraphics{https://static01.nyt.com/images/2017/04/05/us/tent-one/tent-one-articleLarge.jpg?quality=75\&auto=webp\&disable=upscale}

By \href{http://www.nytimes.com/by/fernanda-santos}{Fernanda Santos}

\begin{itemize}
\item
  April 4, 2017
\item
  \begin{itemize}
  \item
  \item
  \item
  \item
  \item
  \end{itemize}
\end{itemize}

PHOENIX --- Tent City, the outdoor jail that stood as the last remaining
symbol of Joe Arpaio's long, turbulent tenure as sheriff of Maricopa
County, will close in the coming weeks, Mr. Arpaio's successor, Sheriff
Paul Penzone, said on Tuesday.

``Starting today, the circus ends, and the tents come down,'' Sheriff
Penzone said.

The jail, where inmates wore striped jumpsuits and pink underwear and
slept in 70 surplus Korean War tents, became an effective and telegenic
publicity tool for Mr. Arpaio. His unforgiving tough-on-crime stance and
his pursuit of illegal immigrants propelled him to re-election five
times, but also thrust him into lawsuits and controversy.

The facility opened in 1993 under the pretense that it would save money
while turning the desert's broiling summer into an element of
punishment. In the end, it did neither, Sheriff Penzone said. Tent City
never held more than 1,700 prisoners, and in recent years, it housed no
more than 800. But the cost of operating the jail did not change
significantly as its population declined; the same number of guards were
needed to patrol its seven-acre campus.

Inmates said they liked being outdoors, despite the heat, the meatless
meals served twice a day, the pink underwear and the spectacle that they
became under Mr. Arpaio, who rarely turned down a reporter's request to
visit the jail. Guards were the ones who suffered, Sheriff Penzone said,
having to wear bulletproof vests and work long hours outside in the heat
and the rain.

``There is no empirical evidence that shows that this facility in any
way deters crime,'' Sheriff Penzone said. The ``misperception,'' he
said, ``is no longer a story.''

A neon sign that Mr. Arpaio ordered installed high above the jail
flashed ``Vacancy,'' at once a statement of fact and, during his tenure,
a perverse taunt.

Mr. Arpaio's name was not mentioned Tuesday, but it was clear that the
closing was intended to topple another piece of his legacy. Mr. Penzone
and Grant Woods, chairman of the committee assembled to study the jail's
effectiveness, repeatedly spoke of the false premise that sustained the
county's commitment to Tent City --- it cost \$8.5 million a year to
operate --- and the stain that it brought to the state's image.

``The days of Arizona being a place where people are humiliated or
abused or ridiculed for the self-aggrandizing of others are over,'' said
Mr. Woods, a former attorney general for Arizona. ``We're moving on.''

In an interview, Mr. Arpaio dismissed Sheriff Penzone's criticism and
said Tent City was ``going to go down in history as one of the greatest
incarceration programs in our country.''

President Trump ``has been cracking down on illegal immigrants, and more
and more people will be coming into our jails, so we'll see them crowded
again,'' he said. Then he offered a suggestion: ``I hope Trump will put
the tents on the border for all the illegals that are caught there.''

In September, while Mr. Arpaio was still sheriff, county supervisors
floated the idea of shutting down the tents to help offset some of the
\$50 million in legal fees for his defense on a yearslong racial
profiling case. He refused, offering instead to save money by forgoing
raises for his deputies and guards.

Only convicted criminals are currently serving time in the tents, for
crimes that do not warrant sentences of more than a year: drug
possession, domestic violence, car theft. The pink underwear and socks
they wear were a point of pride for Mr. Arpaio, who said that if the
underwear was pink, no man would want to steal it. (The jail holds
women, too.)

The jail served two meatless meals a day; inmates referred to the
\href{https://www.facebook.com/ThisIsFernanda/videos/vb.656347072/10153670451937073/?type=2\&theater}{food
as slop} and were required to eat while watching the Food Channel in the
cafeteria, in the only brick-and-mortar building in the complex. Mr.
Arpaio once called it a
``\href{https://www.youtube.com/watch?v=1fj3mRGQ0ow}{concentration
camp}.''

On Monday, Mr. Penzone said Tent City ``goes against everything I stand
for.''

He convened a citizens' group during his first weeks in office, and its
recommendation to close Tent City was unanimous. Mr. Woods said that
during its investigation, the group's most surprising finding was that
inmates wanted to keep the jail open, asserting that it was better to
stay outdoors than to be confined to a six-foot-by-eight-foot cell, he
said.

``What does that tell you?'' Mr. Woods said. ``It tells you that this
negative energy that we've gotten since 1993, that we're so tough on
prisoners in Maricopa County, this is how we treat them, that it was
false.''

Sheriff Penzone said Tent City would close in 45 to 60 days, saving the
county about \$4.5 million a year. Prisoners will be sent to other jails
in the county.

Advertisement

\protect\hyperlink{after-bottom}{Continue reading the main story}

\hypertarget{site-index}{%
\subsection{Site Index}\label{site-index}}

\hypertarget{site-information-navigation}{%
\subsection{Site Information
Navigation}\label{site-information-navigation}}

\begin{itemize}
\tightlist
\item
  \href{https://help.nytimes.com/hc/en-us/articles/115014792127-Copyright-notice}{©~2020~The
  New York Times Company}
\end{itemize}

\begin{itemize}
\tightlist
\item
  \href{https://www.nytco.com/}{NYTCo}
\item
  \href{https://help.nytimes.com/hc/en-us/articles/115015385887-Contact-Us}{Contact
  Us}
\item
  \href{https://www.nytco.com/careers/}{Work with us}
\item
  \href{https://nytmediakit.com/}{Advertise}
\item
  \href{http://www.tbrandstudio.com/}{T Brand Studio}
\item
  \href{https://www.nytimes.com/privacy/cookie-policy\#how-do-i-manage-trackers}{Your
  Ad Choices}
\item
  \href{https://www.nytimes.com/privacy}{Privacy}
\item
  \href{https://help.nytimes.com/hc/en-us/articles/115014893428-Terms-of-service}{Terms
  of Service}
\item
  \href{https://help.nytimes.com/hc/en-us/articles/115014893968-Terms-of-sale}{Terms
  of Sale}
\item
  \href{https://spiderbites.nytimes.com}{Site Map}
\item
  \href{https://help.nytimes.com/hc/en-us}{Help}
\item
  \href{https://www.nytimes.com/subscription?campaignId=37WXW}{Subscriptions}
\end{itemize}
