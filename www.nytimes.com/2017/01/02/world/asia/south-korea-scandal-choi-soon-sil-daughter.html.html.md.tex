Sections

SEARCH

\protect\hyperlink{site-content}{Skip to
content}\protect\hyperlink{site-index}{Skip to site index}

\href{https://www.nytimes.com/section/world/asia}{Asia Pacific}

\href{https://myaccount.nytimes.com/auth/login?response_type=cookie\&client_id=vi}{}

\href{https://www.nytimes.com/section/todayspaper}{Today's Paper}

\href{/section/world/asia}{Asia Pacific}\textbar{}Daughter of Key Figure
in South Korean Scandal Is Detained in Denmark

\url{https://nyti.ms/2iW5KJ6}

\begin{itemize}
\item
\item
\item
\item
\item
\end{itemize}

Advertisement

\protect\hyperlink{after-top}{Continue reading the main story}

Supported by

\protect\hyperlink{after-sponsor}{Continue reading the main story}

\hypertarget{daughter-of-key-figure-in-south-korean-scandal-is-detained-in-denmark}{%
\section{Daughter of Key Figure in South Korean Scandal Is Detained in
Denmark}\label{daughter-of-key-figure-in-south-korean-scandal-is-detained-in-denmark}}

\includegraphics{https://static01.nyt.com/images/2017/01/03/world/03CHOI-1/03CHOI-1-articleInline.jpg?quality=75\&auto=webp\&disable=upscale}

By \href{http://www.nytimes.com/by/choe-sang-hun}{Choe Sang-Hun}

\begin{itemize}
\item
  Jan. 2, 2017
\item
  \begin{itemize}
  \item
  \item
  \item
  \item
  \item
  \end{itemize}
\end{itemize}

SEOUL, South Korea --- One of the most-wanted figures in the
influence-peddling scandal that led to President Park Geun-hye's
\href{http://www.nytimes.com/2016/12/09/world/asia/south-korea-president-park-geun-hye-impeached.html}{parliamentary
impeachment last month} has been detained by the Danish police, South
Korean officials said on Monday.

The suspect, Chung Yoo-ra --- the daughter of Choi Soon-sil, the
longtime confidante of Ms. Park who is at the center of the scandal ---
has been living in hiding in Europe, ignoring repeated calls from South
Korean investigators to return home to face criminal charges.

Her mother, Ms. Choi, has been indicted on charges of using her
influence with Ms. Park to extort tens of millions of dollars from big
businesses. Ms. Park herself has been identified as a criminal
accomplice in Ms. Choi's alleged extortion racket and is
\href{http://www.nytimes.com/2016/12/22/world/asia/south-korea-president-park-impeachment.html?_r=0}{now
on trial at the Constitutional Court}, which will decide whether she
should be formally removed from office.

But Ms. Chung, 20, has been elusive. Last month, a special prosecutor
asked the authorities in Germany, where Ms. Chung, a former member of
South Korea's national equestrian team, had been training, to detain and
extradite her. South Korean officials had also asked Interpol to look
for her and threatened to invalidate her passport.

Ms. Chung was holed up with her infant son, a nanny and two male
guardians in a house in the northern Danish city of Aalborg when the
local authorities found her on Sunday night, said the South Korean news
channel JTBC, which was at the scene when she was apprehended. It said
its reporters first alerted the authorities to her whereabouts.

``We will request her emergency extradition, working with the special
prosecutor's office,'' Lee Chul-sung, the chief of the National Police
Agency, said at a news conference in Seoul, the South Korean capital, on
Monday.

The Danish police said on Monday that they had arrested Ms. Chung after
receiving a tip from a South Korean journalist Sunday afternoon and
confirmation from the international law enforcement agency Interpol that
an international warrant for her arrest had been registered last
Tuesday.

Ms. Chung confirmed that she was aware that the South Korean authorities
wanted to question her, the police said, and added that she was in
Denmark for horse racing.

The police said that Ms. Chung had not been charged with a crime in
Denmark, although prosecutors said they had asked the court to keep Ms.
Chung in custody while the question of possible extradition was
clarified. South Korea and Denmark have an extradition treaty.

Ms. Chung was accused of illegally enrolling in Ewha Womans University
in Seoul in 2015, using her mother's political connections to force the
elite school to accept her despite poor qualifications. An inquiry by
the Education Ministry revealed that the school had admitted her at the
expense of other candidates with better credentials. Her enrollment was
revoked in November.

The special prosecutor is also
\href{http://www.nytimes.com/2016/12/31/world/asia/south-korea-samsung-merger-moon-hyung-pyo.html}{looking
into accusations} that Ms. Choi used millions of dollars from Samsung,
South Korea's largest conglomerate, to finance her daughter's equestrian
career and a luxurious lifestyle in Germany.

Ms. Chung has become a lightning rod for public anger as lurid
accusations have emerged about her lifestyle. The accusations of illegal
enrollment against her have been particularly inflammatory in South
Korea, where students cram for years to prepare for intensely
competitive college entrance exams. Ewha students were the first to take
to the streets to protest the Choi family's reported influence peddling.

After Ms. Chung lost a gold medal to a rival in a domestic equestrian
competition in 2013, her family was accused of using its connections to
open a government audit of the local equestrian association.

Two officials at the Ministry of Culture, Sports and Tourism who looked
into the scandal lost their jobs for not siding with Ms. Chung's family,
a former culture minister said. Ms. Park called the officials ``bad
men,'' according to the former minister, Yoo Jin-ryong.

Ms. Chung's family also used its influence to get Ewha to give her good
grades even though she hardly attended classes, Education Ministry
officials said.

On Tuesday, the special prosecutor asked a court to issue an arrest
warrant for an Ewha professor who was accused of ordering his teaching
assistants to complete assignments for Ms. Chung.

``You've got nothing but your parents to blame for your lack of
resources,'' Ms. Chung wrote in a Facebook post in 2014. The post was
shared widely in South Korea, where widening economic inequality has
fueled increasing public anger.

Advertisement

\protect\hyperlink{after-bottom}{Continue reading the main story}

\hypertarget{site-index}{%
\subsection{Site Index}\label{site-index}}

\hypertarget{site-information-navigation}{%
\subsection{Site Information
Navigation}\label{site-information-navigation}}

\begin{itemize}
\tightlist
\item
  \href{https://help.nytimes.com/hc/en-us/articles/115014792127-Copyright-notice}{©~2020~The
  New York Times Company}
\end{itemize}

\begin{itemize}
\tightlist
\item
  \href{https://www.nytco.com/}{NYTCo}
\item
  \href{https://help.nytimes.com/hc/en-us/articles/115015385887-Contact-Us}{Contact
  Us}
\item
  \href{https://www.nytco.com/careers/}{Work with us}
\item
  \href{https://nytmediakit.com/}{Advertise}
\item
  \href{http://www.tbrandstudio.com/}{T Brand Studio}
\item
  \href{https://www.nytimes.com/privacy/cookie-policy\#how-do-i-manage-trackers}{Your
  Ad Choices}
\item
  \href{https://www.nytimes.com/privacy}{Privacy}
\item
  \href{https://help.nytimes.com/hc/en-us/articles/115014893428-Terms-of-service}{Terms
  of Service}
\item
  \href{https://help.nytimes.com/hc/en-us/articles/115014893968-Terms-of-sale}{Terms
  of Sale}
\item
  \href{https://spiderbites.nytimes.com}{Site Map}
\item
  \href{https://help.nytimes.com/hc/en-us}{Help}
\item
  \href{https://www.nytimes.com/subscription?campaignId=37WXW}{Subscriptions}
\end{itemize}
