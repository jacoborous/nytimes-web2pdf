Sections

SEARCH

\protect\hyperlink{site-content}{Skip to
content}\protect\hyperlink{site-index}{Skip to site index}

\href{https://www.nytimes.com/section/world/europe}{Europe}

\href{https://myaccount.nytimes.com/auth/login?response_type=cookie\&client_id=vi}{}

\href{https://www.nytimes.com/section/todayspaper}{Today's Paper}

\href{/section/world/europe}{Europe}\textbar{}Russians Charged With
Treason Worked in Office Linked to Election Hacking

\url{https://nyti.ms/2jFJaoO}

\begin{itemize}
\item
\item
\item
\item
\item
\end{itemize}

Advertisement

\protect\hyperlink{after-top}{Continue reading the main story}

Supported by

\protect\hyperlink{after-sponsor}{Continue reading the main story}

\hypertarget{russians-charged-with-treason-worked-in-office-linked-to-election-hacking}{%
\section{Russians Charged With Treason Worked in Office Linked to
Election
Hacking}\label{russians-charged-with-treason-worked-in-office-linked-to-election-hacking}}

By \href{http://www.nytimes.com/by/scott-shane}{Scott Shane},
\href{http://www.nytimes.com/by/david-e-sanger}{David E. Sanger} and
\href{http://www.nytimes.com/by/andrew-e-kramer}{Andrew E. Kramer}

\begin{itemize}
\item
  Jan. 27, 2017
\item
  \begin{itemize}
  \item
  \item
  \item
  \item
  \item
  \end{itemize}
\end{itemize}

WASHINGTON --- Ever since American intelligence agencies accused Russia
of trying to influence the American election, there have been questions
about the proof they had to support the accusation.

But the news from Moscow may explain how the agencies could be so
certain that it was the Russians who hacked the email of Hillary
Clinton's campaign and the Democratic National Committee. Two Russian
intelligence officers who worked on cyberoperations and a Russian
computer security expert have been arrested and charged with treason for
providing information to the United States, according to multiple
Russian news reports.

As in most espionage cases, the details made public so far are
incomplete, and some rumors in Moscow suggest that those arrested may be
scapegoats in an internal power struggle over the hacking. Russian media
reports link the charges to the disclosure of the Russian role in
attacking state election boards, including the scanning of voter rolls
in Arizona and Illinois, and do not mention the parallel attacks on the
D.N.C. and the email of John Podesta, Mrs. Clinton's campaign chairman.

But one current and one former United States official, speaking about
the classified recruitments on condition of anonymity, confirmed that
human sources in Russia did play a crucial role in proving who was
responsible for the hacking.

The former official said the agencies were initially reluctant to
disclose their certainty about the Russian role for fear of setting off
a mole hunt in Moscow.

The public disclosure of the arrests, and the severity of the treason
charge, come at a delicate moment for President Trump.

He has been loath to accept the intelligence agencies' conclusion that
Russia tried to help him win, which he sees as part of an effort to
delegitimize his election.

The Russian role will loom over the conversation with Mr. Putin that Mr.
Trump is scheduled to have on Saturday since it was the Russian
president who James R. Clapper Jr., the former director of national
intelligence, told Congress ordered the hacking and leaking.

One topic of the phone conversation is likely to be the sanctions that
the Obama administration imposed on Russia, including ones that were
imposed in December in retaliation for the election hacking.

For months, Mr. Trump rejected the finding that Russia was behind the
hacking, accusing the intelligence agencies of incompetence and
political bias. After a classified briefing in New York a month ago, he
grudgingly accepted that Russia had a role, while playing down the
hacking by noting that China and other countries also hacked the United
States.

Steven L. Hall, a former C.I.A. head of Russian operations, said it was
``very tempting and certainly reasonable'' to connect the arrests to the
American intelligence findings.

But he added a cautionary note: ``The rule of law doesn't apply in
Russia, and they manipulate the law to do whatever they want to do. So
what they call treason may not be what we call treason.''

Mark Galeotti, a Russia expert at the Institute of International
Relations in Prague, noted that the intelligence agencies' report on the
election attack found with ``high confidence'' that Russia had carried
out the election attack, which involved fake news stories and propaganda
as well as the hacks and leaks.

``It was always pretty obvious that they had more than just the computer
evidence,'' Mr. Galeotti said. ``The arrests are a big deal.''

The arrests, according to reports by the Russian newspaper Kommersant
and Novaya Gazeta, among others, were made in early December and
amounted to a purge of the cyberwing of the F.S.B., the main Russian
intelligence and security agency.

Those arrested by the agency's internal affairs bureau included Sergei
Mikhailov, a deputy director of the Center for Information Security, the
agency's computer security arm, and Ruslan Stoyanov, a senior researcher
at a prominent Russian computer security company, Kaspersky Lab.

A nationalist publication, Tsargrad, and RBC, a respected business
newspaper, identified on Friday a third suspect, Dmitry Dokuchayev.

Described as a former hacker who used the online pseudonym Forb, Mr.
Dokuchayev had agreed to work for the F.S.B. to avoid prosecution for
credit card fraud, a rampant crime in Russia.

RBC also reported an alternative theory about the counterintelligence
investigation, saying it may have begun after a hacking group, Shaltai
Boltai, or Humpty Dumpty, stole the emails of a senior Russian official
a year ago. By this account, the investigation of email theft led to Mr.
Dokuchayev.

Both Novaya Gazeta, an outlet for the liberal opposition, and the
hard-line nationalist Tsargrad reported that the F.S.B. added a
theatrical touch to the arrest of Mr. Mikhailov.

Agents arresting the suspected spy placed a bag over his head in the
midst of a congress of senior intelligence agency officers in Moscow and
led him from the room, the two publications reported.

``The arrest was certainly colorful,'' Tsargrad's report said.
``Mikhailov was led from the congress of F.S.B. colleagues with a bag on
his head.''

The virtually simultaneous appearance of at least four prominent news
reports on the arrests, citing numerous anonymous sources, suggested
that the normally opaque Russian government wanted the information out,
though it was unclear why.

A prominent Russian criminal defense lawyer on Friday confirmed that the
authorities in Moscow were prosecuting at least one computer security
expert for treason.

The confirmation by the Russian lawyer, Ivan Pavlov, in written answers
to questions from The New York Times, came the closest so far to a
formal acknowledgment of the arrests.

Mr. Pavlov declined to identify his client or elaborate on the reason
for the indictment for ``betraying the state,'' punishable by up to 20
years in a penal colony.

The report in Novaya Gazeta said the F.S.B. began the internal
investigation after news media reports that a United States
cybersecurity company, ThreatConnect, had linked the election hacking to
a Siberian server company.

That company, King Servers, was otherwise used largely for criminal and
marginal computer activities, such as distributing pornography and
counterfeit goods, by the admission of its owner.

The report said the investigation led to Mr. Mikhailov, a senior officer
involved in tracking criminal computer activity in Russia.

The hints suggested that the Russian government may be signaling that it
might, however indirectly through a treason trial, reveal details of
election hacking, which would have the potential to damage Mr. Trump's
administration.

But there is another explanation, if something of a counterintuitive
one: Documenting a Russian role in the electoral hacks could also serve
Moscow's foreign policy interests by underscoring the extent and power
of the Kremlin's reach in the world.

Cyberattacks, mixed with information warfare, have proven a vital tool
for the Kremlin, used in Europe and the Baltics before the attack on the
United States election. And now, there is evidence of new meddling in
France and Germany, both of which have major elections this year.

The Russian Foreign Ministry has denied any role in the hacking.

ThreatConnect, the cybersecurity company that released the report about
King Servers, said its analysis was based on information published by
the F.B.I.

ThreatConnect declined to comment after the arrests in Moscow.

Advertisement

\protect\hyperlink{after-bottom}{Continue reading the main story}

\hypertarget{site-index}{%
\subsection{Site Index}\label{site-index}}

\hypertarget{site-information-navigation}{%
\subsection{Site Information
Navigation}\label{site-information-navigation}}

\begin{itemize}
\tightlist
\item
  \href{https://help.nytimes.com/hc/en-us/articles/115014792127-Copyright-notice}{©~2020~The
  New York Times Company}
\end{itemize}

\begin{itemize}
\tightlist
\item
  \href{https://www.nytco.com/}{NYTCo}
\item
  \href{https://help.nytimes.com/hc/en-us/articles/115015385887-Contact-Us}{Contact
  Us}
\item
  \href{https://www.nytco.com/careers/}{Work with us}
\item
  \href{https://nytmediakit.com/}{Advertise}
\item
  \href{http://www.tbrandstudio.com/}{T Brand Studio}
\item
  \href{https://www.nytimes.com/privacy/cookie-policy\#how-do-i-manage-trackers}{Your
  Ad Choices}
\item
  \href{https://www.nytimes.com/privacy}{Privacy}
\item
  \href{https://help.nytimes.com/hc/en-us/articles/115014893428-Terms-of-service}{Terms
  of Service}
\item
  \href{https://help.nytimes.com/hc/en-us/articles/115014893968-Terms-of-sale}{Terms
  of Sale}
\item
  \href{https://spiderbites.nytimes.com}{Site Map}
\item
  \href{https://help.nytimes.com/hc/en-us}{Help}
\item
  \href{https://www.nytimes.com/subscription?campaignId=37WXW}{Subscriptions}
\end{itemize}
