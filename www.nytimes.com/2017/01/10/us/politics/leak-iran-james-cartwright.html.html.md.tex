Sections

SEARCH

\protect\hyperlink{site-content}{Skip to
content}\protect\hyperlink{site-index}{Skip to site index}

\href{https://www.nytimes.com/section/politics}{Politics}

\href{https://myaccount.nytimes.com/auth/login?response_type=cookie\&client_id=vi}{}

\href{https://www.nytimes.com/section/todayspaper}{Today's Paper}

\href{/section/politics}{Politics}\textbar{}Prosecutors Seek 2 Years in
Jail for James Cartwright in Leak Case

\url{https://nyti.ms/2jsR9oS}

\begin{itemize}
\item
\item
\item
\item
\item
\end{itemize}

Advertisement

\protect\hyperlink{after-top}{Continue reading the main story}

Supported by

\protect\hyperlink{after-sponsor}{Continue reading the main story}

\hypertarget{prosecutors-seek-2-years-in-jail-for-james-cartwright-in-leak-case}{%
\section{Prosecutors Seek 2 Years in Jail for James Cartwright in Leak
Case}\label{prosecutors-seek-2-years-in-jail-for-james-cartwright-in-leak-case}}

\includegraphics{https://static01.nyt.com/images/2017/01/18/us/18cartwright/11cartwright-articleLarge.jpg?quality=75\&auto=webp\&disable=upscale}

By \href{http://www.nytimes.com/by/charlie-savage}{Charlie Savage}

\begin{itemize}
\item
  Jan. 10, 2017
\item
  \begin{itemize}
  \item
  \item
  \item
  \item
  \item
  \end{itemize}
\end{itemize}

WASHINGTON --- Federal prosecutors
\href{https://www.documentcloud.org/documents/3260025-Cartwright-Government-Sentencing.html}{asked
a judge} on Tuesday to sentence James E. Cartwright, a retired Marine
Corps general and former vice chairman of the Joint Chiefs of Staff, to
two years in prison for lying to F.B.I. agents about his discussions
with reporters about Iran's nuclear program.

The Justice Department's request to Judge Richard J. Leon, of Federal
District Court for the District of Columbia, was significantly harsher
than what prosecutors had agreed was the normal sentencing guideline
range --- a year of probation to six months in prison --- when General
Cartwright
\href{https://www.nytimes.com/2016/10/18/us/marine-general-james-cartwright-leak-fbi.html}{pleaded
guilty to that charge} in October to settle a four-year leak
investigation.

Although the general was convicted only on a charge of lying to
investigators as part of his plea deal, prosecutors argued that his case
should be seen as a leak case and that a two-year sentence would serve
as a deterrent by showing ``that disclosing such information to persons
not authorized to receive it has severe consequences.''

Lawyers for General Cartwright
\href{https://www.documentcloud.org/documents/3259744-Cartwright-Letters-of-Support.html}{told
the judge} their client should not go to prison. They said he had
already suffered grievous damage to his reputation and significant lost
income and should be sentenced to a year of probation and 600 hours of
community service.

``The enormous consequences to General Cartwright of this prosecution
and his very public fall from grace already have been more than
sufficient to warn others to be truthful in speaking with federal
investigators,'' they wrote.

Judge Leon has set a sentencing hearing date for Jan. 31.

The leak investigation into General Cartwright, who left government in
2011, began in June 2012 after David E. Sanger, a reporter for The New
York Times, published a book,
\href{http://www.nytimes.com/2012/06/06/books/confront-and-conceal-by-david-sanger.html}{Confront
and Conceal}, and
\href{http://www.nytimes.com/2012/06/01/world/middleeast/obama-ordered-wave-of-cyberattacks-against-iran.html}{a
related article} in The Times that provided details about Operation
Olympic Games, an American-Israeli covert effort to sabotage Iranian
nuclear centrifuges with the so-called Stuxnet computer virus.

According to the
\href{https://www.documentcloud.org/documents/3260025-Cartwright-Government-Sentencing.html}{government's
sentencing memo}, F.B.I. agents came to focus on General Cartwright as a
possible source for Mr. Sanger's reporting, as well as for a
\href{http://www.newsweek.com/obamas-dangerous-game-iran-65711}{February
2012 Newsweek article} by Daniel Klaidman that also discussed
cyberattacks against Iran. But when agents interviewed the retired
general about the book and articles on Nov. 2, 2013, he lied about his
discussions with the journalists.

The memo said agents then showed General Cartwright email exchanges
between him and the two reporters that contradicted his account, and as
he read them, ``his speech became slurred and he subsequently slumped
over in his chair and lost consciousness.'' The general was taken to a
hospital, and when the interview resumed three days later, he admitted
discussing classified information with them, it said.

General Cartwright and his lawyers have argued that he talked with the
reporters for the purpose of shaping stories they had already reported
out and preventing publication of more damaging information. Prosecutors
expressed doubt, saying he did not articulate that explanation when the
F.B.I. talked to him in November 2013.

Still, the defense submitted a letter from Mr. Sanger in which he noted
that the existence of the Stuxnet virus had been publicly known since
2010, and he said he had already talked to ``many sources in the United
States, Europe and Israel'' before his discussions with General
Cartwright, who he said had expressed concerns about the revelation of
certain secrets that influenced him when deciding what to withhold from
publication.

The defense also submitted a sealed letter from Mr. Klaidman and
numerous letters from lawmakers and current and former executive branch
officials who urged leniency in light of the general's career of public
service.

Advertisement

\protect\hyperlink{after-bottom}{Continue reading the main story}

\hypertarget{site-index}{%
\subsection{Site Index}\label{site-index}}

\hypertarget{site-information-navigation}{%
\subsection{Site Information
Navigation}\label{site-information-navigation}}

\begin{itemize}
\tightlist
\item
  \href{https://help.nytimes.com/hc/en-us/articles/115014792127-Copyright-notice}{©~2020~The
  New York Times Company}
\end{itemize}

\begin{itemize}
\tightlist
\item
  \href{https://www.nytco.com/}{NYTCo}
\item
  \href{https://help.nytimes.com/hc/en-us/articles/115015385887-Contact-Us}{Contact
  Us}
\item
  \href{https://www.nytco.com/careers/}{Work with us}
\item
  \href{https://nytmediakit.com/}{Advertise}
\item
  \href{http://www.tbrandstudio.com/}{T Brand Studio}
\item
  \href{https://www.nytimes.com/privacy/cookie-policy\#how-do-i-manage-trackers}{Your
  Ad Choices}
\item
  \href{https://www.nytimes.com/privacy}{Privacy}
\item
  \href{https://help.nytimes.com/hc/en-us/articles/115014893428-Terms-of-service}{Terms
  of Service}
\item
  \href{https://help.nytimes.com/hc/en-us/articles/115014893968-Terms-of-sale}{Terms
  of Sale}
\item
  \href{https://spiderbites.nytimes.com}{Site Map}
\item
  \href{https://help.nytimes.com/hc/en-us}{Help}
\item
  \href{https://www.nytimes.com/subscription?campaignId=37WXW}{Subscriptions}
\end{itemize}
