Sections

SEARCH

\protect\hyperlink{site-content}{Skip to
content}\protect\hyperlink{site-index}{Skip to site index}

\href{https://www.nytimes.com/section/us}{U.S.}

\href{https://myaccount.nytimes.com/auth/login?response_type=cookie\&client_id=vi}{}

\href{https://www.nytimes.com/section/todayspaper}{Today's Paper}

\href{/section/us}{U.S.}\textbar{}Trump Tells Congress to Repeal and
Replace Health Care Law `Very Quickly'

\url{https://nyti.ms/2jrlO5S}

\begin{itemize}
\item
\item
\item
\item
\item
\end{itemize}

Advertisement

\protect\hyperlink{after-top}{Continue reading the main story}

Supported by

\protect\hyperlink{after-sponsor}{Continue reading the main story}

\hypertarget{trump-tells-congress-to-repeal-and-replace-health-care-law-very-quickly}{%
\section{Trump Tells Congress to Repeal and Replace Health Care Law
`Very
Quickly'}\label{trump-tells-congress-to-repeal-and-replace-health-care-law-very-quickly}}

\includegraphics{https://static01.nyt.com/images/2017/01/11/us/11trump/11trump-articleInline.jpg?quality=75\&auto=webp\&disable=upscale}

By \href{http://www.nytimes.com/by/maggie-haberman}{Maggie Haberman} and
\href{https://www.nytimes.com/by/robert-pear}{Robert Pear}

\begin{itemize}
\item
  Jan. 10, 2017
\item
  \begin{itemize}
  \item
  \item
  \item
  \item
  \item
  \end{itemize}
\end{itemize}

President-elect Donald J. Trump demanded on Tuesday that Congress
immediately repeal the Affordable Care Act and pass another health law
quickly. His remarks put Republicans in the nearly impossible position
of having only weeks to replace a health law that took nearly two years
to pass.

``We have to get to business,'' Mr. Trump told The New York Times in a
telephone interview. ``Obamacare has been a catastrophic event.''

Mr. Trump appeared to be unclear both about the timing of already
scheduled votes in Congress and about the difficulty of his demand --- a
repeal vote ``probably some time next week'' and a replacement ``very
quickly or simultaneously, very shortly thereafter.''

But he was clear on one point: Plans by congressional Republicans to
repeal the health law now, then take years to create and implement a
replacement law are unacceptable to the incoming president.

Republican leaders have made the repeal of President Obama's signature
domestic achievement a top priority. They hope that the Senate will vote
on Thursday and the House will vote on Friday to approve parliamentary
language created to protect repeal legislation from a filibuster in the
Senate.

\href{https://www.nytimes.com/interactive/2016/12/03/us/politics/why-it-will-be-hard-to-repeal-obamacare.html}{}

\includegraphics{https://static01.nyt.com/images/2016/12/03/us/politics/why-it-will-be-hard-to-repeal-obamacare-1480740532639/why-it-will-be-hard-to-repeal-obamacare-1480740532639-thumbLarge.jpg}

\hypertarget{how-republicans-can-repeal-obamacare-piece-by-piece}{%
\subsection{How Republicans Can Repeal Obamacare Piece by
Piece}\label{how-republicans-can-repeal-obamacare-piece-by-piece}}

Peeling away pieces of the law could lead to market chaos.

The House speaker, Paul D. Ryan of Wisconsin, who consults often with
Mr. Trump, set out a similar timetable on Tuesday, saying that a bill to
repeal the health care law would include some legislation to replace
aspects of it, though Republicans have yet to agree on the details of
their alternative.

``It is our goal to bring it all together concurrently,'' Mr. Ryan said.

But those ambitions will be difficult to achieve and will almost
certainly require Democratic cooperation. Until now, Republicans could
vote to repeal Mr. Obama's health law with no fear that they would have
to live with the political consequences of scuttling a law that provides
health care for 20 million Americans and protects millions more from
discrimination for pre-existing medical conditions, ends lifetime caps
on insurance coverage and allows children to remain on their parents'
insurance policies until age 26.

With complete control of Washington, what comes next in health policy
will belong to the Republican Party. For several days, congressional
Republicans of diverse political views --- moderates and conservatives
alike --- have been saying they are nervous about repealing the law
without any clear path forward. Five Senate Republicans have pressed to
delay the deadline for committees to produce repeal legislation until
March, and several House Republicans are also demanding that the pace
slow down.

``In an ideal situation, we would repeal and replace Obamacare
simultaneously, but we need to make sure that we have at least a
detailed framework that tells the American people what direction we're
headed,'' said one of those five Republicans, Senator Susan Collins of
Maine.

As it stands, the budget resolution that will fast-track that vote gives
Senate and House committees until Jan. 27 to write legislation that
would repeal major provisions of the health care law. But the schedule
for action on that legislation, its effective date and the timetable for
phasing in a new system of health insurance coverage are all unresolved
questions.

\includegraphics{https://static01.nyt.com/images/2017/01/11/us/11health2/11health2-articleLarge.jpg?quality=75\&auto=webp\&disable=upscale}

Even the Jan. 27 deadline is not enforceable or particularly meaningful,
Senate aides said, indicating that Congress could follow any timetable
its leaders might prescribe.

That uncertainty apparently persuaded Mr. Trump to leap into the fray.
Not only did he try to steel Republican spines, but he threatened
Democrats who might stand in his way, saying he would campaign against
them, especially in states that he won in November.

``It may not get approved the first time, and it may not get approved
the second time, but the Democrats who will try not to approve it'' will
be at risk, he said, warning that ``they have 10 people coming up'' for
re-election in 2018. That alluded to Democratic senators in states he
won.

``I won some of those states by numbers that nobody has seen. I will be
out there campaigning,'' he said.

He described the health law as a catastrophe. ``I feel that repeal and
replace have to be together, for, very simply, I think that the
Democrats should want to fix Obamacare,'' he said. ``They cannot live
with it, and they have to go together.''

After meeting on Tuesday with House Republicans, Mr. Ryan took a similar
tone, calling the campaign to repeal the health law ``a rescue mission
to save families who are getting caught up in the death spiral that has
become Obamacare.''

Aides to Mr. Ryan said the effort to dismantle the Affordable Care Act
would include not only the main bill that would be protected from a
filibuster in the Senate, but also legislation that would not enjoy such
protections. That legislation would take Democratic cooperation to be
passed because Senate Republicans are eight votes short of a
filibuster-proof majority.

Congressional Democrats say that the Affordable Care Act, far from being
in a ``death spiral,'' is one of the best health laws since the creation
of Medicare and Medicaid in 1965. And the Obama administration reported
on Tuesday that more than 11.5 million people nationwide had signed up
for health insurance or been automatically re-enrolled under the
Affordable Care Act as of Dec. 24, 2016, an increase of nearly 300,000
from this time last year.

Of that total, officials said, more than 8.7 million people came in
through HealthCare.gov, the online federal marketplace, and 2.8 million
were enrolled in states using their own marketplace platforms.

``Today's data show that this market is not merely stable, it is
actually on track for growth,'' Aviva Aron-Dine, a senior counselor to
Sylvia Mathews Burwell, the secretary of health and human services, said
in a conference call with reporters. ``Today we can officially proclaim
these death spiral claims dead.''

\href{https://www.nytimes.com/interactive/2014/us/07aca-callout.html}{}

\includegraphics{https://static01.nyt.com/images/2017/05/23/science/23SCI-NUMBERwebSUB/23SCI-NUMBERwebSUB-square640.jpg}

\hypertarget{new-hampshire-residents-did-you-sign-up-for-insurance-through-the-affordable-care-act}{%
\subsection{New Hampshire Residents: Did You Sign up for Insurance
Through the Affordable Care
Act?}\label{new-hampshire-residents-did-you-sign-up-for-insurance-through-the-affordable-care-act}}

The Times would like to hear from Americans who are signing up for
insurance under the Affordable Care Act.

The fourth annual open enrollment period started on Nov. 1 and ends on
Jan. 31, 11 days after Mr. Trump's inauguration.

The enrollment numbers have some Republicans nervous. ``The fear is that
the strategy is repeal and delay, and then hope for the best, when we
should be planning for the worst,'' said Representative Charlie Dent of
Pennsylvania, a chairman of the moderate Republican caucus known as the
Tuesday Group.

Republican leaders tried on Tuesday to ease such concerns. But they may
be making promises that will be difficult to keep.

``Let me be clear,'' Representative Cathy McMorris Rodgers of
Washington, the chairwoman of the House Republican Conference, told
reporters. ``No one who has coverage because of Obamacare today will
lose that coverage. We're providing relief. We aren't going to pull the
rug out from anyone.''

The Obama administration also provided new information to Congress on
Tuesday about one of the most unpopular provisions of the health care
law, which imposes tax penalties on people who go without insurance and
do not qualify for an exemption from the requirement to have coverage.

The commissioner of the Internal Revenue Service, John A. Koskinen,
reported that 6.5 million taxpayers were subject to penalties last year.
The penalties totaled \$3 billion, he said. The average payment was
about \$470.

Under another section of the health law, low- and moderate-income people
can obtain subsidies, in the form of tax credits, to help pay for
insurance bought through a public marketplace. In 2016, Mr. Koskinen
said, 5.3 million taxpayers claimed \$19.2 billion in premium tax
credits, for an average of about \$3,620.

Advertisement

\protect\hyperlink{after-bottom}{Continue reading the main story}

\hypertarget{site-index}{%
\subsection{Site Index}\label{site-index}}

\hypertarget{site-information-navigation}{%
\subsection{Site Information
Navigation}\label{site-information-navigation}}

\begin{itemize}
\tightlist
\item
  \href{https://help.nytimes.com/hc/en-us/articles/115014792127-Copyright-notice}{©~2020~The
  New York Times Company}
\end{itemize}

\begin{itemize}
\tightlist
\item
  \href{https://www.nytco.com/}{NYTCo}
\item
  \href{https://help.nytimes.com/hc/en-us/articles/115015385887-Contact-Us}{Contact
  Us}
\item
  \href{https://www.nytco.com/careers/}{Work with us}
\item
  \href{https://nytmediakit.com/}{Advertise}
\item
  \href{http://www.tbrandstudio.com/}{T Brand Studio}
\item
  \href{https://www.nytimes.com/privacy/cookie-policy\#how-do-i-manage-trackers}{Your
  Ad Choices}
\item
  \href{https://www.nytimes.com/privacy}{Privacy}
\item
  \href{https://help.nytimes.com/hc/en-us/articles/115014893428-Terms-of-service}{Terms
  of Service}
\item
  \href{https://help.nytimes.com/hc/en-us/articles/115014893968-Terms-of-sale}{Terms
  of Sale}
\item
  \href{https://spiderbites.nytimes.com}{Site Map}
\item
  \href{https://help.nytimes.com/hc/en-us}{Help}
\item
  \href{https://www.nytimes.com/subscription?campaignId=37WXW}{Subscriptions}
\end{itemize}
