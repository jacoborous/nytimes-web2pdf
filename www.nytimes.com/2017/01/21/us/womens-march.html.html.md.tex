Sections

SEARCH

\protect\hyperlink{site-content}{Skip to
content}\protect\hyperlink{site-index}{Skip to site index}

\href{https://www.nytimes.com/section/us}{U.S.}

\href{https://myaccount.nytimes.com/auth/login?response_type=cookie\&client_id=vi}{}

\href{https://www.nytimes.com/section/todayspaper}{Today's Paper}

\href{/section/us}{U.S.}\textbar{}Women's March Highlights as Huge
Crowds Protest Trump: `We're Not Going Away'

\href{https://nyti.ms/2kbJ0pb}{https://nyti.ms/2kbJ0pb}

\begin{itemize}
\item
\item
\item
\item
\item
\item
\end{itemize}

Advertisement

\protect\hyperlink{after-top}{Continue reading the main story}

Supported by

\protect\hyperlink{after-sponsor}{Continue reading the main story}

\hypertarget{womens-march-highlights-as-huge-crowds-protest-trump-were-not-going-away}{%
\section{Women's March Highlights as Huge Crowds Protest Trump: `We're
Not Going
Away'}\label{womens-march-highlights-as-huge-crowds-protest-trump-were-not-going-away}}

\includegraphics{https://static01.nyt.com/images/2017/01/22/us/22MARCH18/22MARCH18-articleInline.jpg?quality=75\&auto=webp\&disable=upscale}

By \href{http://www.nytimes.com/by/anemona-hartocollis}{Anemona
Hartocollis} and
\href{http://www.nytimes.com/by/yamiche-alcindor}{Yamiche Alcindor}

\begin{itemize}
\item
  Jan. 21, 2017
\item
  \begin{itemize}
  \item
  \item
  \item
  \item
  \item
  \item
  \end{itemize}
\end{itemize}

• Hundreds of thousands of women gathered in Washington on Saturday in a
kind of counterinauguration after President Trump
\href{https://www.nytimes.com/2017/01/20/us/politics/trump-inauguration-day.html}{took
office on Friday}. A \href{https://www.womensmarch.com/speakers/}{range
of speakers and performers} cutting across generational lines rallied
near the Capitol before marchers made their way toward the White House.

• They were joined by crowds in cities across the country: In Chicago,
the size of a rally so quickly outgrew early estimates that the march
that was to follow was canceled for safety. In Manhattan, Fifth Avenue
became a river of pink hats, while in downtown Los Angeles, even before
the gathering crowd stretched itself out to march, it was more than a
quarter mile deep on several streets.

• Begun as \href{https://www.facebook.com/events/2169332969958991/}{a
Facebook post} just after the election, the march is the start of what
organizers hope could be a sustained campaign of protest in a polarized
America, unifying demonstrators around issues like reproductive rights,
immigration and civil rights. The movement has also
\href{https://www.nytimes.com/2017/01/09/us/womens-march-on-washington-opens-contentious-dialogues-about-race.html?_r=0}{encountered}
\href{https://www.nytimes.com/2017/01/18/us/womens-march-abortion.html}{divisions}.

• The Times had journalists covering the marches in Washington; New
York; Boston; Atlanta; Denver; Los Angeles; Phoenix; St. Paul; and Key
West, Fla. Check out
\href{https://twitter.com/NYTNational/lists/women-s-march}{what they
posted on Twitter} and what
\href{https://www.nytimes.com/interactive/2017/01/21/us/questions-march-live.html}{readers
asked of them} live. See
\href{https://www.nytimes.com/interactive/2017/01/21/world/womens-march-pictures.html}{photos
from marches around the world}, too. (All times listed below are local.)

\includegraphics{https://static01.nyt.com/images/2017/01/22/us/22march9/22march9-articleInline.jpg?quality=75\&auto=webp\&disable=upscale}

\hypertarget{here-are-some-highlights-from-the-rally-in-washington}{%
\subsection{Here are some highlights from the rally in
Washington:}\label{here-are-some-highlights-from-the-rally-in-washington}}

(Or
\href{https://www.nytimes.com/video/us/politics/100000004879811/washington-march-live-video.html}{watch
video of the whole event}.)

• The singer and actress \textbf{Janelle Monae} highlighted the issue of
police violence, leading the crowd in a chant of ``Sandra Bland! Say her
name!'', a reference to
\href{https://www.nytimes.com/2016/04/13/us/panel-calls-for-major-changes-at-texas-jail-that-held-sandra-bland.html}{the
high-profile case} where a black woman died in police custody in Texas
after being arrested in 2015.

She then brought the microphone to each of the women in
\href{http://www.cnn.com/2016/07/26/politics/mothers-movement-dnc-hillary-clinton/}{``Mothers
of the Movement''} who had joined her onstage. One by one, they joined
in the chant, each inserting the name of her child who had died at the
hands of the police.

• The actress \textbf{Ashley Judd} delivered an uninhibited speech that
ended with her referencing how Mr. Trump bragged, in
\href{https://www.nytimes.com/2016/10/08/us/donald-trump-tape-transcript.html}{a
2005 recording}, that he could use his celebrity status to force himself
on women, even groping their private parts.

They ``ain't for grabbing,'' she said. ``They are for birthing new
generations of filthy, vulgar, nasty, proud, Christian, Muslim,
Buddhist, Sikh, you name it, for new generations of nasty women.''

• \textbf{Gloria Steinem}, the feminist icon of the 1960s and 1970s,
told the women in the group to get to know one another more personally.

``Make sure you introduce yourselves to each other and decide what we're
going to do tomorrow, and tomorrow and tomorrow,'' she said. ``We're
never turning back!''

Image

Gloria Steinem spoke to thousands of people gathering at the Women's
March in Washington.Credit...Ruth Fremson/The New York Times

• ``It's been a heart-rending time to be both a woman and an immigrant
in this country,'' said the actress and activist \textbf{America
Ferrera}.``But the president is not America. His cabinet is not America.
Congress is not America. We are America! And we are here to stay.''

• After getting to the crowd to repeat a number to call Congress, the
filmmaker \textbf{Michael Moore} urged people to run for office:

``This is not the time for shy people! Shy people, you have two hours to
get over it.''

• The actress \textbf{Scarlett Johansson} told a story about how she had
visited a Planned Parenthood clinic in New York City after starting her
acting career, and how a doctor there had treated her with compassion,
``no judgment, no questions asked.''

``I feel that in the face of this current political climate, it is vital
that we all make it our mission to get really, really personal,'' she
said.

``President Trump, I did not vote for you,'' she continued. ``I want to
be able to support you. But first I ask that you support me. Support my
sister. Support my mother. Support my best friend and all of our
girlfriends.''

Otherwise, Ms. Johansson said, her own daughter, ``may potentially not
have the right to make choices for her body and her future that your
daughter Ivanka has been privileged to have.''

\emph{\_\_\_\_\_}

\hypertarget{whats-up-with-those-pussyhats-ive-heard-about}{%
\subsection{What's up with those ``pussyhats'' I've heard
about?}\label{whats-up-with-those-pussyhats-ive-heard-about}}

Image

A sea of pink hats on march participants in Washington on
Saturday.Credit...Ruth Fremson/The New York Times

Many participants believed that Mr. Trump expressed misogynistic views
during the presidential campaign, with remarks about
\href{https://www.nytimes.com/2016/11/16/business/media/megyn-kellys-cautionary-tale-of-crossing-donald-j-trump.html}{Megyn
Kelly},
\href{https://www.nytimes.com/politics/first-draft/2015/09/10/donald-trumps-uncomplimentary-comments-about-carly-fiorina/}{Carly
Fiorina} and
\href{https://www.nytimes.com/2016/10/21/us/politics/hillary-clinton-women.html}{Hillary
Clinton}. After the 2005
\href{https://www.nytimes.com/2016/10/08/us/donald-trump-tape-transcript.html}{recording}
surfaced, several women came forward
\href{https://www.nytimes.com/2016/10/21/us/politics/donald-trump-women.html}{to
accuse Mr. Trump of inappropriate sexual conduct}. He dismissed the
recording as ``locker room banter'' and assailed
\href{https://www.nytimes.com/2016/10/15/us/politics/donald-trump-campaign.html}{his
accusers}.

In a sly allusion to the crude remarks Mr. Trump made in the recording,
many marchers, men and women alike, wore pink ``pussyhats,'' complete
with cat ears. The hats are described on
\href{https://www.pussyhatproject.com/}{pussyhatproject.com} as a way to
``make a unique collective visual statement which will help activists be
better heard.''

\emph{\_\_\_\_\_}

\hypertarget{mr-trump-seemed-to-go-out-of-his-way-to-ignore-the-march}{%
\subsection{Mr. Trump seemed to go out of his way to ignore the
march}\label{mr-trump-seemed-to-go-out-of-his-way-to-ignore-the-march}}

Just after 10 a.m., Mr. Trump and his family headed in the opposite
direction of the march in Washington for the National Prayer Service, an
inaugural tradition, at the National Cathedral. When he spoke at C.I.A.
headquarters in Langley, Va., in the midafternoon, he told his audience
that they were his ``No. 1 stop'' on his first full day in office,
because they were ``really special amazing people.''

He also
\href{https://www.nytimes.com/2017/01/21/us/politics/trump-white-house-briefing-inauguration-crowd-size.html}{ruminated
about how big the attendance} had been at his inaugural speech, but he
did not mention the large crowds of the women's march, where
demonstrators were challenging his administration on a number of
policies, or even that the march was taking place as he was speaking.

\emph{\_\_\_\_\_}

\hypertarget{hillary-clinton-tweets-her-support}{%
\subsection{Hillary Clinton tweets her
support}\label{hillary-clinton-tweets-her-support}}

Mrs. Clinton was not expected to attend the march in Washington,
\href{https://www.nytimes.com/2017/01/20/us/politics/hillary-clinton-donald-trump-inauguration.html}{The
Times reported on Friday}, but her Twitter account sent a midmorning
note anyway.

\emph{\_\_\_\_\_}

\hypertarget{elizabeth-warren-me-im-here-to-fight-back}{%
\subsection{Elizabeth Warren: `Me, I'm here to fight
back'}\label{elizabeth-warren-me-im-here-to-fight-back}}

In a speech in Boston, Ms. Warren, a Democratic senator from
Massachusetts, said fundamental freedoms, like abortion rights and gay
marriage, could be at stake under Mr. Trump's Supreme Court.

``We can whimper, we can whine or we can fight back,'' she said, as
demonstrators in pink hats waved American flags. ``Me, I'm here to fight
back.''

``We believe in science,'' Ms. Warren said, adding, ``we know that
climate change is real.'' A police officer patrolling the rally pumped
his fists in agreement.

``We also believe that immigration makes this country a stronger
country,'' Ms. Warren said. ``We will not build a stupid wall and we
will not tear millions of families apart.''

``You know, I could do this all day,'' she added, to laughs and cheers.
``But we gotta march.''

\emph{Jess Bidgood}

\emph{\_\_\_\_\_}

Image

Crowds gathered outside the National Center for Civil and Human Rights
for the start of the march in Atlanta on Saturday.Credit...Kevin D.
Liles for The New York Times

\hypertarget{john-lewis-dont-let-anybody-anybody-turn-you-around}{%
\subsection{John Lewis: `Don't let anybody, anybody, turn you
around'}\label{john-lewis-dont-let-anybody-anybody-turn-you-around}}

\textbf{Notable Signs:} ``Bend toward justice,'' evokeing the work of
the Rev. Dr. Martin Luther King Jr.

``I'm ready to march again,'' said Mr. Lewis, a Democratic
representative of Georgia, who chaired the Student Nonviolent
Coordinating Committee in the 1960s. ``I've come here to say to you:
Don't let anybody, anybody, turn you around.''

Citing the demonstrations across the country, Mr. Lewis urged marchers,
who flowed onto the street running near the Center for Civil and Human
Rights, to ``use this unity to organize'' future political efforts.

''The next election, we must get out and vote like we never, ever voted
before,'' said Mr. Lewis, who was embroiled in a public clash with Mr.
Trump recently.

\emph{Alan Blinder}

\emph{\_\_\_\_\_}

\hypertarget{everyone-wants-to-know-how-many-people-turned-out}{%
\subsection{Everyone wants to know: How many people turned
out?}\label{everyone-wants-to-know-how-many-people-turned-out}}

The crowds appeared to be huge in most places, with marchers in
Washington, New York City and Chicago seeming to stretch to the
horizons. Police departments, at times, decline to provide crowd
estimates, and
\href{https://www.nytimes.com/interactive/2017/01/18/us/politics/How-Crowds-at-the-Capitol-Have-Been-Counted.html}{crowds
are notoriously hard to estimate}, even with a good satellite image. But
some official and unofficial estimates have given a sense of the
turnout.

Attendance in \textbf{New York City} was more than 400,000, according to
Mayor Bill de Blasio's office. The \textbf{St. Paul} police issued an
official crowd count of 50,000 to 60,000 people. Attendance in
\textbf{Boston} was 175,000, according to Nicole Caravella, a
spokeswoman for Mayor Martin J. Walsh. The \textbf{Atlanta} Police
Department estimated about 60,000 people attended a rally there. The
Department of Public Safety in \textbf{Phoenix} estimated that some
20,000 marched, while in \textbf{Key West, Fla.}, a town of 25,000,
police said more than 2,000 people marched.

Organizers in \textbf{Chicago} estimated the crowd there at 250,000, the
\href{http://www.chicagotribune.com/news/ct-womens-march-chicago-0122-20170121-story.html}{Chicago
Tribune} said. The Office of Emergency Management and Communications
there said late on Saturday morning that Grant Park, the sprawling area
where the rally-goers had gathered, had been filled to capacity. Though
the official march was canceled, many still chose to walk through
downtown holding protest signs.

Although the mayor's office in \textbf{Washington} and organizers
declined to provide an estimate of the size of the flagship march, The
Associated Press reported that the District of Columbia's homeland
security director, Christopher Geldart, said it was safe to say the
crowd at the march there was more than the 500,000 that organizers told
city officials to expect.

``The crowd was so heavy, we didn't know which way to go,'' said Sabitha
Pillai-Friedman, a psychotherapist who traveled to Washington from
Philadelphia with her 17-year-old child, Sanji, and a friend, Pallavi
Sreedhar. ``We were squeezed, touching.''

(March organizers offered a worldwide
\href{https://www.womensmarch.com/sisters}{tally} for the 673 ``sister''
marches, but when asked, could not provide an explanation of how the
tally had been calculated.)

\emph{\_\_\_\_\_}

\hypertarget{heres-a-rundown-of-scenes-across-the-country-first-up-location-washington-time-443-pm}{%
\subsection{Here's a rundown of scenes across the country. First up?
Location: Washington. Time: 4:43
p.m.}\label{heres-a-rundown-of-scenes-across-the-country-first-up-location-washington-time-443-pm}}

\textbf{Overheard Chant:} ``Yes we can'' as people walked past the White
House.

As the sun set downtown, protesters made their way to the White House
and assembled in small groups in a park just across from the building's
entrance. There in an area surrounded by temporary gates, people walked
single file through one open entrance and one by one laid protest signs
across gates set up for inauguration several hundred feet away from the
White House.

While the temporary gates made walking up to the building impossible,
people stood shaking their heads in frustration.

Fontella Garraway, a 50-year-old retired Army veteran who drove three
and half hours from her home in Rocky Mount, N.C., sat on a bench
staring at the White House with a pin that read ``girl power.''

``Even looking at the White House, it's like I hope he's looking out
here at us,'' she said of Mr. Trump. ``I hope it's penetrating to him
that we mean business and we are serious.''

Moments later she lay a handwritten sign that read ``Love trumps hate;
Hear our voice,'' on the a fence facing the White House.

''That's his inauguration gift,'' she said.

\emph{Yamiche Alcindor}

\emph{\_\_\_\_\_}

Image

Women's March participants in Phoenix on Saturday carried signs and
American flags along the procession's route.Credit...Caitlin O'Hara for
The New York Times

\hypertarget{location-phoenix-time-101-pm}{%
\subsection{Location: Phoenix. Time: 1:01
p.m.}\label{location-phoenix-time-101-pm}}

\textbf{Notable Chant:} ``Tell me what America looks like! This is what
America looks like.''

\textbf{Notable Sash:} ``65,855,610 votes for a woman,'' worn by Sara
Powell, 61, of Phoenix, and nine of her friends.

\textbf{Overheard:} ``My arms are tired. This is a good workout,'' said
Rima Borgogni, 50, owner of a Pilates studio in Sedona, Ariz., after
holding a sign throughout the mile-long march.

Ellen Ferreira and her friends felt as if they were fighting for some of
the same things they used to. They are mostly retired and many of them
are veterans of past protests, including the March on Washington for
Jobs and Freedom in 1963, when Dr. Martin Luther King Jr. delivered his
``I Have a Dream'' speech.

``For our right to choose,'' said Piya Jacob, 70, a retired elementary
school principal.

``For equality,'' said Mary Helsaple, 67, an artist.

``For healing justice,'' said Gretchen Vorbeck, 72, who runs a nonprofit
that buys grocery gift cards for public schoolteachers.

Carol Decker, 70, a retired magazine publisher, jumped in and said,
``We're mad as hell and we're not going to take it anymore.''

\emph{Fernanda Santos}

\emph{\_\_\_\_\_}

\hypertarget{location-washington-time-310-pm}{%
\subsection{Location: Washington. Time: 3:10
p.m.}\label{location-washington-time-310-pm}}

\textbf{Notable Chants:} ``We are the popular vote!''

\textbf{Notable Shirt:} A blue shirt with ``Make Sexism Wrong Again'' in
the same style as ``Make America Great Again'' campaign shirts.

Just off 15th street, a block north of the parade's official end point,
a large flatbed float with big ``TRUMP'' letters arched along the back
parked itself in the middle of the street, drawing the ire of the
thousands of marchers, who berated the float with chants of ``Shame!''
and ``We are the popular vote!''

Police officers formed a barricade around the float with more than a
half-dozen sidecar motorcycles. The six or so men and one woman on the
float all took pictures of the protesters.

Yet some of those who chanted to chase the float away weren't surprised
at its appearance at their march.

``I mean the inauguration was yesterday,'' said Chrissy Fiore, 39, of
Washington, though she said it was ``crazy that they made it down here
and that now they're getting police escorted out.''

Officers wouldn't let reporters approach those on the float or those
driving it, but a magnet on the side said ``Trump Unity Bridge.''

As the float headed east to move away from the parade, Sheriff David A.
Clarke Jr. of Milwaukee County, a Trump supporter, was seen walking
along the sidewalk, taking in the scene but remaining silent. He did not
respond to a reporter's question about his opinion of the march or
protest.

\emph{Nick Corasaniti}

\emph{\_\_\_\_\_}

Image

A brass band played for demonstrators at the Women's March in New York
City.Credit...Jessica Bal for The New York Times

\hypertarget{location-new-york-city-time-1-pm}{%
\subsection{Location: New York City. Time: 1
p.m.}\label{location-new-york-city-time-1-pm}}

\textbf{Chant:} ``Don't take away our ACA'' and ``Who's the boss? We
are!''

\textbf{Notable Signs:} ``Show us your taxes;'' ``you can't comb over
sexism;'' ``1459 days;'' and ``build a bridge not a wall.''

\textbf{Overheard:} One woman speaking at the rally told the story of
having an abortion when she was young, making the minimum wage and could
not support a child. She said she was fighting for equal pay ``not just
for white women.''

At the rally in Mr. Trump's hometown near Trump World Tower, elected
officials and celebrities assailed the president. Signs in the crowd
mocked his bouffant hair and the size of his hands. The actress Whoopi
Goldberg said it would be the first of many protests against the
president.

``This is how people ended the war in Vietnam,'' Ms. Goldberg told the
cheering crowd.

Grace Huezo, 20, a student at Hunter College, marched with her twin
sister holding a ``Nasty Woman'' sign. She said she was there to defend
women's rights after she was appalled by Mr. Trump's comments about
grabbing women.

``We're here saying, no, people do not have permission to grab women
without our permission,'' she said.

She said she was buoyed by the huge turnout and the camaraderie.

``I'm hopeful to see so many people that are not giving up and they're
keeping their spirit,'' she said. ``We're all just going to stick
together over the next four years.''

\emph{Emma G. Fitzsimmons}

\emph{\_\_\_\_\_}

\hypertarget{location-denver-time-1022-am}{%
\subsection{Location: Denver. Time: 10:22
a.m.}\label{location-denver-time-1022-am}}

\textbf{Popular chant:} ``March! March! March!''

\textbf{Notable signs:} ``I won't stop til it rains glass;'' ``You can't
comb over misogyny'' (accompanied by a drawing of Mr. Trump's hair);
``Flunk the Electoral College.''

\textbf{Overheard:} ``I got to bring my high school punk rock out,''
said Emily Hastings, 39, a woman from Denver wearing a black ``eat the
rich'' T-shirt and carrying a ``Don't tread on women'' sign. ``Punk rock
is all about resistance.''

The march began in a park at the center of the city with a group singing
``You've got a friend.'' Marchers blanketed the park nestled between the
gold-domed state capitol and city hall, hauling strollers, wearing pink
hats and often hugging and kissing.

\emph{Julie Turkewitz}

\emph{\_\_\_\_\_}

\hypertarget{location-st-paul-time-1055-am}{%
\subsection{Location: St. Paul. Time: 10:55
a.m.}\label{location-st-paul-time-1055-am}}

\textbf{Notable Sign:} ``Make America Compassionate Again,'' and ``I
Love You''

Thousands of demonstrators gathered on a drizzly morning clad in rain
boots, ponchos and pink knit ``pussyhats'' to march to the Capitol.

``What Trump has said is so based on exclusion and winning and being
right versus taking care of everyone,'' said Hilary James, 27, a
musician from Minneapolis. ``Even if he doesn't listen to us, I feel
it's important to not sit back.''

\emph{Christina Capecchi}

\emph{\_\_\_\_\_}

\hypertarget{location-boston-time-1025-am}{%
\subsection{Location: Boston. Time: 10:25
a.m.}\label{location-boston-time-1025-am}}

\textbf{Notable Sign:} Make America Think Again

Gloria Cole, 66, had turned the protest into a family affair, traveling
here with her wife, her daughter, her daughter's boyfriend, and her
brother and sister-in-law.

``I drew a line, it's like, I'm an old woman --- I'm not that old, I'm
66 --- I have to stand up for equal rights for everyone, for human
rights,'' Ms. Cole said. ``We're here, and we're not going away.''

Aili Shaw, 14, held a white sign that read, ``Our arms are tired from
holding these signs since the 1920s.''

Ms. Shaw had traveled here, by train and car, with friends from her home
in Coventry, R.I.

``Women don't have the rights they should,'' she said.

\emph{Jess Bidgood}

\emph{\_\_\_\_\_}

\hypertarget{location-washington-time-1030-am}{%
\subsection{Location: Washington. Time: 10:30
a.m.}\label{location-washington-time-1030-am}}

\textbf{Popular Chant:} ``Thank You.'' Women were chanting this to the
organizers of the march as they kicked off the day's events.

\textbf{Notable Clothing:} At the corner of C and Third Southwest, many
women (and some men) were wearing cat-eared ``pussyhats'' of all shades
of pink. Organizers wanted to knit as many as one million hats for this
event.

People were also getting creative with the signs they carry. Alan and
Alison Lewis drove in from Astoria with their 20-month-old, Grace.

``You shouldn't have to have a relationship to a woman to stand up for
women,'' Mr. Lewis said. ``Equality and justice is enough of a reason to
be here.''

\emph{Katie Rogers}

\emph{\_\_\_\_\_}

\hypertarget{and-now-meet-a-family-with-three-generations-of-marchers}{%
\subsection{And now meet a family with three generations of
marchers}\label{and-now-meet-a-family-with-three-generations-of-marchers}}

\textbf{Who She Is:} Jessica Coleman, 56, of Stone Mountain, Ga.

\textbf{Backstory:} A black retired teacher who used to show her
daughter documentaries about black history and march with her daughter
and church members during Martin Luther King holiday weekends.

``I wanted them to know you can be a smart, intelligent black person.
You don't have to sag your pants and follow certain things that became
media culture. I wanted them to know that people marched, bled and died
for us to be able to vote and be able to go to college and have certain
jobs.''

``You can really lose your sense of self if you don't know where you've
come from and you don't have a vision of where you want to go.''

Image

Jessica Coleman, far right, at home in Washington on Friday with (left
to right) Margaret Beddoe, Nicole Babwar and Amber Coleman, and Amber
Coleman's children, including Garvey Mortley, center in the gray shirt
with black and gold lettering on it. All of them planned to attend the
Women's March on Saturday.Credit...Ty Wright for The New York Times

\_\_\_\_\_

\textbf{Who She Is:} Amber Coleman-Mortley, 34, of Bethesda, Md., is Ms.
Coleman's daughter

\textbf{Backstory:} Works at a nonprofit focused on civic education

``On the evening of election night, after saying, `Hey, I'm going to
vote and I'm going to get my friends to vote,' I sat on the sofa bawling
trying to figure out what to say to my daughters the next morning
because they went to bed certain that Hillary was going to win.''

``Marching is my way of putting my money where my mouth is as far as
being an active citizen.''

``I want my daughters to have agency and have control over their bodies
and feel comfortable in the country that they are in so this is my way
of saying, `Hey everybody, I agree with all the people who are out here
for different reasons and we don't agree with what is happening right
now and we are taking a stand.'''

\_\_\_\_\_

\textbf{Who She Is:} Garvey Mortley, 8, of Bethesda, Md., is Ms.
Coleman's granddaughter

\textbf{Backstory:} Third grader; was named after Marcus Garvey

``I think it's good to share the moment with them and help protest
Donald Trump because we want to stand up for our rights because a long
time ago, lots of women could not vote. Now we can vote and protest
people and stuff.''

``If he affects the world in a bad way --- like I have lots of friends
from different countries and he could make them all move away \ldots{}
that makes me mad because all the people from the civil rights movement
had a hard time trying to put us together. It's like a puzzle, all these
people came together, piece by piece, and now Donald Trump is coming
over and just breaking those puzzle pieces.''

\includegraphics{https://static01.nyt.com/images/2017/01/22/us/22womena-march-videostill/22womena-march-videostill-videoSixteenByNine3000.jpg}

Advertisement

\protect\hyperlink{after-bottom}{Continue reading the main story}

\hypertarget{site-index}{%
\subsection{Site Index}\label{site-index}}

\hypertarget{site-information-navigation}{%
\subsection{Site Information
Navigation}\label{site-information-navigation}}

\begin{itemize}
\tightlist
\item
  \href{https://help.nytimes.com/hc/en-us/articles/115014792127-Copyright-notice}{©~2020~The
  New York Times Company}
\end{itemize}

\begin{itemize}
\tightlist
\item
  \href{https://www.nytco.com/}{NYTCo}
\item
  \href{https://help.nytimes.com/hc/en-us/articles/115015385887-Contact-Us}{Contact
  Us}
\item
  \href{https://www.nytco.com/careers/}{Work with us}
\item
  \href{https://nytmediakit.com/}{Advertise}
\item
  \href{http://www.tbrandstudio.com/}{T Brand Studio}
\item
  \href{https://www.nytimes.com/privacy/cookie-policy\#how-do-i-manage-trackers}{Your
  Ad Choices}
\item
  \href{https://www.nytimes.com/privacy}{Privacy}
\item
  \href{https://help.nytimes.com/hc/en-us/articles/115014893428-Terms-of-service}{Terms
  of Service}
\item
  \href{https://help.nytimes.com/hc/en-us/articles/115014893968-Terms-of-sale}{Terms
  of Sale}
\item
  \href{https://spiderbites.nytimes.com}{Site Map}
\item
  \href{https://help.nytimes.com/hc/en-us}{Help}
\item
  \href{https://www.nytimes.com/subscription?campaignId=37WXW}{Subscriptions}
\end{itemize}
