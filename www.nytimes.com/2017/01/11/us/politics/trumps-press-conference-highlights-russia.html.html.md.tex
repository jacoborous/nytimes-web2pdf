Sections

SEARCH

\protect\hyperlink{site-content}{Skip to
content}\protect\hyperlink{site-index}{Skip to site index}

\href{https://www.nytimes.com/section/politics}{Politics}

\href{https://myaccount.nytimes.com/auth/login?response_type=cookie\&client_id=vi}{}

\href{https://www.nytimes.com/section/todayspaper}{Today's Paper}

\href{/section/politics}{Politics}\textbar{}Donald Trump Concedes
Russia's Interference in Election

\url{https://nyti.ms/2ikPF27}

\begin{itemize}
\item
\item
\item
\item
\item
\end{itemize}

Advertisement

\protect\hyperlink{after-top}{Continue reading the main story}

Supported by

\protect\hyperlink{after-sponsor}{Continue reading the main story}

\hypertarget{donald-trump-concedes-russias-interference-in-election}{%
\section{Donald Trump Concedes Russia's Interference in
Election}\label{donald-trump-concedes-russias-interference-in-election}}

\includegraphics{https://static01.nyt.com/images/2017/01/12/us/12trump-web01/12trump-web01-articleInline.jpg?quality=75\&auto=webp\&disable=upscale}

By \href{https://www.nytimes.com/by/julie-hirschfeld-davis}{Julie
Hirschfeld Davis} and
\href{http://www.nytimes.com/by/maggie-haberman}{Maggie Haberman}

\begin{itemize}
\item
  Jan. 11, 2017
\item
  \begin{itemize}
  \item
  \item
  \item
  \item
  \item
  \end{itemize}
\end{itemize}

President-elect Donald J. Trump on Wednesday conceded for the first time
that Russia had carried out cyberattacks against the two major political
parties during the presidential election, but he angrily rejected
unsubstantiated reports that Moscow had gathered compromising personal
and financial information about him that could be used for extortion.

In a chaotic news conference in the lobby of Trump Tower in Manhattan
nine days before he is to be sworn in as the nation's 45th president,
Mr. Trump compared United States intelligence officials to Nazis,
sidestepped repeated questions about whether he or anyone in his
presidential campaign had had contact with Russia during the campaign,
and lashed out at the news media and political opponents, arguing that
they were out to get him.

``As far as hacking, I think it was Russia,'' Mr. Trump said, his first
comments accepting the conclusions of United States intelligence
officials that Moscow had interfered in the election to help him win.
But the president-elect expressed little outrage about that breach and
seemed to cast doubt on Russia's role moments after acknowledging it,
asserting that ``it could have been others also.''

He also quoted a Kremlin denial Tuesday night of reports that it had
gathered damaging information to compromise Mr. Trump. ``They said it
totally never happened,'' Mr. Trump said of President Vladimir V. Putin
of Russia and his government. ``I respected the fact that he said
that.''

The news conference displayed the showmanship, combativeness and
sensitivity to criticism that Mr. Trump exhibited throughout the 2016
presidential campaign and underscored his reflex to rebut any criticism
or question about his conduct. In his maligning of the nation's
intelligence agencies, journalists and Hillary Clinton, the
president-elect indicated that he would conduct himself the same way in
the White House.

Using the same boastful tone that characterized his campaign rallies,
Mr. Trump asserted that his victory in November had vindicated his view
that he should not release his tax returns, an issue that he said only
the news media cared about, not the public.

\includegraphics{https://static01.nyt.com/images/2017/01/11/us/trump-cnn/trump-cnn-videoSixteenByNine3000.jpg}

``I won,'' he said. ``I don't think they care at all.'' In a
\href{http://www.people-press.org/2017/01/10/negative-views-of-trumps-transition-amid-concerns-about-conflicts-tax-returns/}{Pew
Research Center poll} this month, 60 percent of respondents said Mr.
Trump should release his returns, although just 38 percent of Republican
respondents said he should.

Some moments bordered on bizarre for the next president of the United
States. Mr. Trump spoke of his awareness as a businessman that there
were hidden cameras in hotel rooms in Moscow and other foreign capitals.
He called himself ``very much of a germaphobe,'' apparently in an effort
to discredit unsubstantiated claims about sex videos with Mr. Trump and
prostitutes in a Russian hotel. ``Does anyone really believe that
story?'' he said, calling it ``phony stuff'' that ``never happened.''

At one point, Mr. Trump got into a confrontation with a correspondent
for CNN, which was among the first to report on the allegations, saying
to him, ``You are fake news.'' Moments later, though, Mr. Trump called
on another CNN correspondent.

A person who identified himself as a correspondent for RT, the Russian
English-language news organization that American intelligence agencies
deem a Russian propaganda tool, shouted repeatedly in vain attempts to
draw Mr. Trump's attention.

Mr. Trump voiced only faint concern about what United States
intelligence officials said was a campaign by Mr. Putin to meddle in
American democracy. He reserved his sharpest condemnation for American
intelligence officials who he said had failed to keep secret the
accusations that could be damaging to him.

On Wednesday, the director of national intelligence, James R. Clapper
Jr., said he had spoken with Mr. Trump that evening and expressed his
``profound dismay'' over the leaks of unsubstantiated information. He
said he had emphasized that this information was ``not a U.S.
intelligence community product'' and that the intelligence agencies had
not determined that it was reliable. He said he did not believe that the
leaks had come from the intelligence agencies.

The president-elect, asked at the news conference whether he believed
that Mr. Putin had directed the hacking effort to help him win the
presidency, said, ``If Putin likes Donald Trump, I consider that an
asset, not a liability, because we have a horrible relationship with
Russia.''

\includegraphics{https://static01.nyt.com/images/2017/01/11/us/12TRUMP4-hp/12TRUMP4-hp-videoSixteenByNineJumbo1600-v3.jpg}

``He shouldn't be doing it,'' Mr. Trump said later of the Russian
president. ``He won't be doing it. Russia will have much greater respect
for our country when I'm leading than when other people have led it.''

Of the intelligence officials who will soon serve him, Mr. Trump said:
``I think it was disgraceful --- disgraceful that the intelligence
agencies allowed any information that turned out to be so false and fake
out. That's something that Nazi Germany would have done, and did do.''

He did not address whether the sanctions President Obama imposed on
Moscow for the cyberattacks should stay or be strengthened as some
Republicans have urged, especially as the scope of the hacking has
become clearer.

The hourlong news conference --- Mr. Trump's first in nearly six months
--- touched not only on reports of espionage and attempted blackmail,
but also on
\href{https://www.nytimes.com/2017/01/11/us/politics/trump-organization-business-conflicts.html?hp\&action=click\&pgtype=Homepage\&clickSource=story-heading\&module=a-lede-package-region\&region=top-news\&WT.nav=top-news}{potential
conflicts of interest} with Mr. Trump's vast business empire and
questions about domestic policy.

The glut of pent-up questions for the president-elect gave him an
advantage in navigating the exchange; he interrupted inquiries about
Russia's hacking to introduce a lawyer, Sheri L. Dillon, who spoke at
length about how Mr. Trump would organize his business affairs and
explain why he was not divesting from his global business empire.
``President-elect Trump should not be expected to destroy the company he
built,'' Ms. Dillon said.

Mr. Trump offered glimpses of his plans for his first days in office,
including pledging to choose a Supreme Court nominee within two weeks of
Inauguration Day to succeed Justice Antonin Scalia and to invite
journalists to watch a series of ``signings'' at the White House, an
apparent allusion to the several executive orders he has promised to
sign to roll back major pieces of Mr. Obama's agenda.

Calling himself ``the greatest job-producer that God ever created,'' Mr.
Trump pledged to continue leaning on American companies to keep jobs in
the United States. He took particular aim at the pharmaceutical
industry, which he said ``has been disastrous'' and had been ``getting
away with murder'' on drug pricing. Taking on a powerful lobby that
Republicans have long defended, Mr. Trump said he wanted the federal
government to use its purchasing power to negotiate drug prices for
\href{http://topics.nytimes.com/top/news/health/diseasesconditionsandhealthtopics/medicare/index.html?inline=nyt-classifier}{Medicare}
and
\href{http://topics.nytimes.com/top/news/health/diseasesconditionsandhealthtopics/medicaid/index.html?inline=nyt-classifier}{Medicaid}
--- a proposal long favored by Democrats.

\includegraphics{https://static01.nyt.com/images/2017/01/12/us/12trump-web02/12trump-web02-articleInline.jpg?quality=75\&auto=webp\&disable=upscale}

But he broke starkly with Democrats over the Affordable Care Act as he
repeated a promise to submit a plan to repeal and replace the law
``essentially simultaneously,'' as soon as Representative Tom Price, his
choice to be secretary of health and human services, is confirmed.

``Obamacare is the Democrats' problem,'' Mr. Trump said Wednesday. ``We
could sit back and let them hang with it. We are doing the Democrats a
great service.''

He also insisted, despite repeated denials by Mexican officials, that
Mexico would pay to build a wall on the southern border of the United
States to block foreigners from entering illegally. Mr. Trump said Vice
President-elect Mike Pence was working with federal agencies to begin
construction quickly, and asserted that Mexico would ultimately
reimburse the cost through a tax or other payment.

Mexico's president, Enrique Peña Nieto, reiterated Wednesday that his
country would not pay for the wall, but said it would invest in more
border security.

In front of Mr. Trump was a table stacked with manila folders that he
said contained paperwork for a portion of the companies being put into a
trust to be controlled and run by his eldest sons, Eric and Donald Jr.,
and a trustee.

They stood to his side along with his daughter Ivanka Trump, who also
announced on Wednesday that she would sever ties with the Trump
Organization and her own company.

Closing the news conference, Mr. Trump even got in a veiled plug for his
former reality show, ``The Apprentice'' --- he remains an executive
producer of the current version, ``Celebrity Apprentice'' --- by saying
that if his sons did not manage his empire well while he served as
president, he would tell them, ``You're fired.''

Advertisement

\protect\hyperlink{after-bottom}{Continue reading the main story}

\hypertarget{site-index}{%
\subsection{Site Index}\label{site-index}}

\hypertarget{site-information-navigation}{%
\subsection{Site Information
Navigation}\label{site-information-navigation}}

\begin{itemize}
\tightlist
\item
  \href{https://help.nytimes.com/hc/en-us/articles/115014792127-Copyright-notice}{©~2020~The
  New York Times Company}
\end{itemize}

\begin{itemize}
\tightlist
\item
  \href{https://www.nytco.com/}{NYTCo}
\item
  \href{https://help.nytimes.com/hc/en-us/articles/115015385887-Contact-Us}{Contact
  Us}
\item
  \href{https://www.nytco.com/careers/}{Work with us}
\item
  \href{https://nytmediakit.com/}{Advertise}
\item
  \href{http://www.tbrandstudio.com/}{T Brand Studio}
\item
  \href{https://www.nytimes.com/privacy/cookie-policy\#how-do-i-manage-trackers}{Your
  Ad Choices}
\item
  \href{https://www.nytimes.com/privacy}{Privacy}
\item
  \href{https://help.nytimes.com/hc/en-us/articles/115014893428-Terms-of-service}{Terms
  of Service}
\item
  \href{https://help.nytimes.com/hc/en-us/articles/115014893968-Terms-of-sale}{Terms
  of Sale}
\item
  \href{https://spiderbites.nytimes.com}{Site Map}
\item
  \href{https://help.nytimes.com/hc/en-us}{Help}
\item
  \href{https://www.nytimes.com/subscription?campaignId=37WXW}{Subscriptions}
\end{itemize}
