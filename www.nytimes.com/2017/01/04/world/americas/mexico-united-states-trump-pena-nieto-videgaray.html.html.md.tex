Sections

SEARCH

\protect\hyperlink{site-content}{Skip to
content}\protect\hyperlink{site-index}{Skip to site index}

\href{https://www.nytimes.com/section/world/americas}{Americas}

\href{https://myaccount.nytimes.com/auth/login?response_type=cookie\&client_id=vi}{}

\href{https://www.nytimes.com/section/todayspaper}{Today's Paper}

\href{/section/world/americas}{Americas}\textbar{}Mexico's New Foreign
Minister Was Advocate of Visit by Donald Trump

\url{https://nyti.ms/2j6nqSp}

\begin{itemize}
\item
\item
\item
\item
\item
\end{itemize}

Advertisement

\protect\hyperlink{after-top}{Continue reading the main story}

Supported by

\protect\hyperlink{after-sponsor}{Continue reading the main story}

\hypertarget{mexicos-new-foreign-minister-was-advocate-of-visit-by-donald-trump}{%
\section{Mexico's New Foreign Minister Was Advocate of Visit by Donald
Trump}\label{mexicos-new-foreign-minister-was-advocate-of-visit-by-donald-trump}}

\includegraphics{https://static01.nyt.com/images/2017/01/05/us/05MEXICO/05MEXICO-articleLarge.jpg?quality=75\&auto=webp\&disable=upscale}

By \href{http://www.nytimes.com/by/kirk-semple}{Kirk Semple}

\begin{itemize}
\item
  Jan. 4, 2017
\item
  \begin{itemize}
  \item
  \item
  \item
  \item
  \item
  \end{itemize}
\end{itemize}

MEXICO CITY --- President Enrique Peña Nieto of
\href{https://www.nytimes.com/topic/destination/mexico?8qa}{Mexico}, in
a move to improve his government's footing as it prepares for the
presidency of Donald J. Trump, on Wednesday named as his foreign
minister the man who
\href{https://www.nytimes.com/2016/09/08/world/americas/mexico-finance-minister-luis-videgaray-resigns.html?_r=0}{resigned
his previous cabinet post} after championing Mr. Trump's enormously
unpopular visit to Mexico City last year.

The appointment was a remarkable political resurrection for the man,
Luis Videgaray, a longtime confidant of Mr. Peña Nieto's. He replaces
Claudia Ruiz Massieu, who had been the foreign minister since 2015 and
reportedly opposed Mr. Trump's August visit.

Mr. Peña Nieto, in announcing the change during a news conference in
Mexico City, said he had ordered Mr. Videgaray, who stepped down as
Mexico's finance minister in September after Mr. Trump's trip, ``to
accelerate dialogue and contacts so that from Day 1 of the new
administration, the basis of a constructive working relationship can be
established.''

The Mexican government, not to mention the nation's business community,
has been bracing for a Trump administration and the possible impact it
could have on all facets of the complex relationship between the United
States and Mexico. Mr. Trump has promised to renegotiate or revoke the
\href{https://www.nytimes.com/2017/01/04/world/americas/mexico-donald-trump-nafta.html?hp\&action=click\&pgtype=Homepage\&clickSource=story-heading\&module=first-column-region\&region=top-news\&WT.nav=top-news}{North
American Free Trade Agreement} and to get tougher on illegal immigration
by building a wall along the Mexican border and stepping up
deportations.

In the two months since the presidential election, Ms. Ruiz Massieu has
shuttled between Mexico City and the United States, meeting with
American government officials to emphasize the importance of the trade
agreement, and she prepared her diplomatic corps in the United States to
respond to Mr. Trump's immigration threats.

Mr. Peña Nieto, who was flanked at the news conference by both Mr.
Videgaray and Ms. Ruiz Massieu, highlighted Mr. Videgaray's experience
in the realm of finance and economics, particularly his work in relation
to the G-20 nations, as ``foundations of this new mission.''

Raul Benitez Manaut, a professor of international relations at the
National Autonomous University of Mexico, called the cabinet shuffle ``a
completely pragmatic decision given the new political circumstances in
the United States.''

``It is clear the dialogue has shifted from security issues to commerce,
which is new since this used to be a conflict-free issue in the past,''
he continued.

Until the fallout from the Trump visit, Mr. Videgaray, who is a former
investment banker and holds a Ph.D. in economics from the Massachusetts
Institute of Technology, had been a power player in the administration
of Mr. Peña Nieto and was considered a possible contender for the 2018
presidential election. He also coordinated Mr. Peña Nieto's 2012
presidential campaign. As finance minister, he was instrumental in Mr.
Peña Nieto's efforts to open the nation's oil industry, a state-run
monopoly since the 1930s, and overhaul the telecommunications and energy
industries. Critically, Mr. Videgaray also sided with Mr. Peña Nieto in
championing the idea of the Trump visit.

Mr. Peña Nieto staunchly defended the visit, saying he had sent
invitations to both of the leading presidential candidates to come to
discuss bilateral issues. Mr. Trump cast his trip as an effort to reach
out to a country he had alienated during his campaign.

But hours after leaving Mexico, Mr. Trump delivered a combative speech
in Phoenix that struck many of the anti-immigrant themes that defined
his candidacy. The episode was widely criticized by the Mexican public
and many politicians, and it embarrassed Mr. Peña Nieto, who watched his
already abysmal approval ratings sink even lower after the visit. On
Sept. 7, after a week of blistering criticism, Mr. Peña Nieto announced
the resignation of Mr. Videgaray. Though he did not give a reason for
the departure, it was widely viewed as an effort by the president to try
to put the Trump visit in the past.

Yet in the light of the American presidential election results, the
decision to invite Mr. Trump suddenly looked a little different, and
rumors almost immediately started circulating among politicians and
political observers that Mr. Peña Nieto might bushwhack a path back to
his cabinet for his friend Mr. Videgaray.

There was no immediate comment from the Trump transition team about the
appointment, but a day after Mr. Videgaray's resignation in September,
the candidate said on Twitter: ``Mexico has lost a brilliant finance
minister and wonderful man.''

Advertisement

\protect\hyperlink{after-bottom}{Continue reading the main story}

\hypertarget{site-index}{%
\subsection{Site Index}\label{site-index}}

\hypertarget{site-information-navigation}{%
\subsection{Site Information
Navigation}\label{site-information-navigation}}

\begin{itemize}
\tightlist
\item
  \href{https://help.nytimes.com/hc/en-us/articles/115014792127-Copyright-notice}{©~2020~The
  New York Times Company}
\end{itemize}

\begin{itemize}
\tightlist
\item
  \href{https://www.nytco.com/}{NYTCo}
\item
  \href{https://help.nytimes.com/hc/en-us/articles/115015385887-Contact-Us}{Contact
  Us}
\item
  \href{https://www.nytco.com/careers/}{Work with us}
\item
  \href{https://nytmediakit.com/}{Advertise}
\item
  \href{http://www.tbrandstudio.com/}{T Brand Studio}
\item
  \href{https://www.nytimes.com/privacy/cookie-policy\#how-do-i-manage-trackers}{Your
  Ad Choices}
\item
  \href{https://www.nytimes.com/privacy}{Privacy}
\item
  \href{https://help.nytimes.com/hc/en-us/articles/115014893428-Terms-of-service}{Terms
  of Service}
\item
  \href{https://help.nytimes.com/hc/en-us/articles/115014893968-Terms-of-sale}{Terms
  of Sale}
\item
  \href{https://spiderbites.nytimes.com}{Site Map}
\item
  \href{https://help.nytimes.com/hc/en-us}{Help}
\item
  \href{https://www.nytimes.com/subscription?campaignId=37WXW}{Subscriptions}
\end{itemize}
