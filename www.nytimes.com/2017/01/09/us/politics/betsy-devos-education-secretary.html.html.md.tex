Sections

SEARCH

\protect\hyperlink{site-content}{Skip to
content}\protect\hyperlink{site-index}{Skip to site index}

\href{https://www.nytimes.com/section/politics}{Politics}

\href{https://myaccount.nytimes.com/auth/login?response_type=cookie\&client_id=vi}{}

\href{https://www.nytimes.com/section/todayspaper}{Today's Paper}

\href{/section/politics}{Politics}\textbar{}Betsy DeVos, Trump's
Education Pick, Plays Hardball With Her Wealth

\url{https://nyti.ms/2jmXrWZ}

\begin{itemize}
\item
\item
\item
\item
\item
\end{itemize}

Advertisement

\protect\hyperlink{after-top}{Continue reading the main story}

Supported by

\protect\hyperlink{after-sponsor}{Continue reading the main story}

\hypertarget{betsy-devos-trumps-education-pick-plays-hardball-with-her-wealth}{%
\section{Betsy DeVos, Trump's Education Pick, Plays Hardball With Her
Wealth}\label{betsy-devos-trumps-education-pick-plays-hardball-with-her-wealth}}

\includegraphics{https://static01.nyt.com/images/2017/01/03/business/00devos1/00devos1-articleLarge.jpg?quality=75\&auto=webp\&disable=upscale}

By \href{https://www.nytimes.com/by/noam-scheiber}{Noam Scheiber}

\begin{itemize}
\item
  Jan. 9, 2017
\item
  \begin{itemize}
  \item
  \item
  \item
  \item
  \item
  \end{itemize}
\end{itemize}

After Tom Casperson, a Republican state senator from Michigan's Upper
Peninsula, began running for Congress in 2016, he assumed the family of
Betsy DeVos, President-elect Donald J. Trump's nominee to be education
secretary, would not oppose him.

The DeVoses, a dominant force in Michigan politics for decades with a
fortune in the billions, had contributed to one of Mr. Casperson's
earlier campaigns. But a week before his primary, family members sent
\$24,000 to one of his opponents, then poured \$125,000 into a ``super
PAC,'' Concerned Taxpayers of America, that ran ads attacking him.

The reason, an intermediary told Mr. Casperson: his support from
organized labor.

``Deceitful, dishonest and cowardly,'' was how Mr. Casperson's campaign
described the ads, complaining that the groups running them ``won't say
who they are or where their money is coming from.'' On Primary Day, Mr.
Casperson went down to defeat.

In announcing his intention to nominate Ms. DeVos, Mr. Trump described
her as ``a brilliant and passionate education advocate.'' Even critics
characterized her as a dedicated, if misguided, activist for school
reform. But that description understates both the breadth of Ms. DeVos's
political interests and the influence she wields as part of her powerful
family. More than anyone else who has joined the incoming Trump
administration, she represents the combination of wealth, free-market
ideology and political hardball associated with a better-known family of
billionaires: Charles and David Koch.

``They have this moralized sense of the free market that leads to this
total program to turn back the ideas of the New Deal, the welfare
state,'' Kim Phillips-Fein, a historian who has written extensively
about the conservative movement, said, describing the DeVoses.

Ms. DeVos declined to be interviewed for this article.

Like the Kochs, the DeVoses are generous supporters of think tanks that
evangelize for unrestrained capitalism, like Michigan's Acton Institute,
and that rail against unions and back privatizing public services, like
the Mackinac Center.

They have also funded national groups dedicated to cutting back the role
of government, including the National Center for Policy Analysis (which
has pushed for Social Security privatization and against environmental
regulation) and the Institute for Justice (which challenges regulations
in court and defends school vouchers). Both organizations have also
received money from the Koch family.

Indeed, the DeVoses' education activism, which favors alternatives to
traditional public schools, appears to derive from the same free-market
views that inform their suspicion of government. And perhaps more than
other right-wing billionaires, the DeVoses couple their seeding of
ideological causes with an aggressive brand of political spending. Half
a dozen or more extended family members frequently coordinate
contributions to maximize their impact.

In the 2016 cycle alone,
\href{http://mcfn.org/node/6043/devos-family-made-14-million-in-political-contributions-in-the-last-2-years-alone}{according
to the Michigan Campaign Finance Network}, the family spent roughly \$14
million on political contributions to state and national candidates,
parties, PACs and super PACs.

All of this would make Ms. DeVos --- whose confirmation hearing has been
delayed until next week amid mounting pressure that her government
ethics review be completed beforehand --- very different from past
education secretaries.

\href{https://www.nytimes.com/interactive/2016/12/05/us/politics/trump-cabinet-insiders-outsiders-millionaires.html}{}

\includegraphics{https://static01.nyt.com/images/2016/12/02/us/politics/trump-cabinet-insiders-outsiders-millionaires-1480717606838/trump-cabinet-insiders-outsiders-millionaires-1480717606838-thumbLarge-v2.png}

\hypertarget{outsiders-insiders-and-multimillionaires-in-trumps-cabinet}{%
\subsection{Outsiders, Insiders and Multimillionaires in Trump's
Cabinet}\label{outsiders-insiders-and-multimillionaires-in-trumps-cabinet}}

President-elect Donald J. Trump's cabinet and top staff are shaping up
to be a mix of wealthy Washington outsiders, Republican insiders and
former military officers who have been critical of the Obama
administration.

``She is the most emblematic kind of oligarchic figure you can put in a
cabinet position,'' said Jeffrey Winters, a political scientist at
Northwestern University who studies economic elites. ``What she and the
Kochs have in common is the unbridled use of wealth power to achieve
whatever political goals they have.''

\hypertarget{birth-of-a-power-couple}{%
\subsection{Birth of a Power Couple}\label{birth-of-a-power-couple}}

Ms. DeVos, 59, grew up in Holland, Mich., the daughter of a conservative
auto parts magnate who was an early funder of the Family Research
Council, a conservative Christian group. When she married Dick DeVos in
1979, it was akin to a merger between two royal houses of western
Michigan.

Her husband's father, Richard Sr., co-founder of the multilevel
marketing company Amway, was an active member of the Christian Reformed
Church that preached a mix of social conservatism and self-reliance. He
once told the church's official magazine that Chicago's poor dwelled in
slums because that was ``the way they choose to live,'' according to
\href{https://www.washingtonpost.com/archive/politics/1981/03/14/selling-free-enterprise/951e73a4-c888-48f9-8726-ddc31d15b471/?utm_term=.ba4f34df73b5}{a
Washington Post story} from the 1980s.

A fan of Rolls-Royces and pinkie rings, Richard Sr. wrote books with
titles like ``Ten Powerful Phrases for Positive People.''

A similar air hung over his business. Amway sales representatives, which
the company calls ``independent business owners,'' make money both by
selling the company's products --- everything from perfume to toilet
bowl cleaner --- and by recruiting other sales representatives.

The Federal Trade Commission once investigated the company for running a
pyramid scheme before concluding that it had misled potential recruits
about how much they could expect to earn.

The flip side of the family's proselytizing for capitalism, according to
Professor Phillips-Fein, has been an effort to dismantle much ``that
would counterbalance the power of economic elites.''

Amway funded a nationwide ad campaign in the early 1980s, protesting
high taxes and regulations. Not long after, the company pleaded guilty
to cheating the Canadian government out of more than \$20 million in
revenue.

The family had a more winning public face in Dick DeVos, who combined
the practiced empathy of a pitchman with the entitlement of an heir,
spending over \$30 million on an unsuccessful run for governor of
Michigan in 2006. The Detroit Free Press described him that year as the
wealthiest man to seek office in the state's modern history.

Betsy DeVos, who served as chairwoman of the Michigan Republican Party
for most of the decade between 1996 and 2005, has often played the role
of strategist in the relationship. She was a key adviser in her
husband's run for governor and publicly brooded that he had been too
gentlemanly in his first debate against the incumbent.

``He's very good with people, a retail politician who looks you in the
eye, shakes your hand, listens to what you say,'' said Randy
Richardville, a former Republican leader of the Michigan Senate,
describing the couple's strengths. ``I would never underestimate Betsy
DeVos in a knife fight.''

\includegraphics{https://static01.nyt.com/images/2017/01/03/business/00devos3/00devos3-articleInline.jpg?quality=75\&auto=webp\&disable=upscale}

Ms. DeVos has sometimes lacked her husband's finesse, once famously
blaming many of the state's economic woes on ``high wages.'' She has won
detractors, by their account, by
\href{https://www.nytimes.com/2016/11/23/us/politics/betsy-devos-trumps-education-pick-has-steered-money-from-public-schools.html?_r=0}{browbeating
legislators into voting her way}.

``Betsy DeVos was like my 4-year-old granddaughter at the time,'' said
Mike Pumford, a former Republican state representative who once clashed
with her. ``They were both sweet ladies as long as they kept hearing the
word `yes.' They turned into spoiled little brats when they were told
`no.'''

But Ms. DeVos has often made up for what she lacks in tact through sheer
force of will.

Mr. Richardville said he and Ms. DeVos disagreed over term limits, which
she supported as party chairwoman and he opposed: ``I said, `I don't
think you should be setting policy. You should be supporting those of us
who do make policy.' But she never backed down.''

While Dick and Betsy DeVos appear to practice a more tolerant form of
Christianity than their parents --- Ms. DeVos has spoken out against
anti-gay bigotry --- as recently as the early 2000s they funded some
groups like Focus on the Family, a large ministry that helps set the
political agenda for conservative evangelicals. They have also backed
groups that promote conservative values to students and Christian
education, including one
\href{https://www.crcna.org/news-and-views/worldwide-christian-schools-changes-its-name}{with
ties} to the Christian Reformed Church.

Their economic views are strikingly similar to the elder Mr. DeVos's.

According to federal disclosures, Amway, which Dick DeVos ran between
1993 and 2002, has lobbied frequently over the last 20 years to reduce
or repeal
\href{http://www.taxjusticeblog.org/archive/2015/03/the_three_fundamental_why_reas.php\#.WHPb-lMrJaR}{the
estate tax}. Only the top 0.2 percent wealthiest estates paid the tax in
2015.

The company has also opposed crackdowns on tax shelters.

Ms. DeVos has been an outspoken defender of unlimited contributions
known as soft money, which she described in a 1997 editorial as
``hard-earned American dollars that Big Brother has yet to find a way to
control.''

After Congress later passed a major campaign finance reform bill, a
nonprofit that Ms. DeVos helped to create and fund masterminded the
strategy that produced Citizens United, the 2010 Supreme Court decision
laying the groundwork for super PACs funded by corporations, unions and
individuals to raise and spend unlimited amounts in elections.

And then there are the family's efforts to rein in the labor movement.

Through their contributions to think tanks like the Mackinac Center, as
well as Mr. DeVos's direct prodding of Republican legislators, the
family played a key role in helping pass Michigan's so-called
right-to-work legislation in 2012. The legislation largely ended the
requirement that workers pay fees to unions as a condition of
employment.

Unions in the state bled members in 2014, the first full year the
measure was in effect.

Allies say the DeVoses fight for their beliefs. ``Betsy and Dick see
themselves as principled conservatives,'' said Frederick Hess of the
American Enterprise Institute. ``It kind of seems healthy and admirable
to give resources to folks who are going to fight for causes you believe
in.''

But the fights can appear to be as much about consolidating power as
ideology. Unions were arguably the family's most formidable political
opponent in Michigan, one of labor's traditional strongholds.

\hypertarget{changes-in-michigan}{%
\subsection{Changes in Michigan}\label{changes-in-michigan}}

The DeVos family's roots as education activists date back at least to
when Richard DeVos Sr. was running Amway and an institute based at the
company's headquarters trained teachers to inject free-market principles
into their curriculum.

According to an interview Ms. DeVos gave to
\href{http://www.philanthropyroundtable.org/topic/excellence_in_philanthropy/interview_with_betsy_devos}{Philanthropy
magazine}, she and her husband became interested in education causes
when they began visiting a Christian school that served low-income
children in Grand Rapids in the 1980s.

``If we could choose the right school for our kids'' --- by which she
appeared to mean primarily private schools --- ``it only seemed fair
that they could do the same for theirs,'' she told the magazine.

The family spent millions of dollars on a ballot proposal in 2000 asking
if Michigan should legalize vouchers, in which students can use taxpayer
money to attend private schools.

Many critics, like the education historian Diane Ravitch, argue that the
point of vouchers is to destroy public education and teachers' unions.
The group Americans United for Separation of Church and State
\href{http://www.au.org/church-state/september-2010-church-state/featured/sneak-attack}{has
documented} how conservative Christians have long supported vouchers,
which could fund religious schools.

After voters objected by more than a two-to-one ratio, Dick DeVos gave a
speech at the Heritage Foundation saying such efforts would have to
shift to state legislatures, where groups backed by deep-pocketed donors
could offer ``a political consequence for opposition, and political
reward for support of education reform issues.''

It is not unusual for the wealthy --- who devote nearly 50 percent of
their philanthropic dollars to education, according to the group
Wealth-X --- to spend aggressively in the political realm to impose
their preferred reforms.

Even by these standards, however, the DeVoses stand out for the amount
of money they spend trying to advance their goals through politics
rather than philanthropy, such as research into reforms or subsidizing
schools.

As Sarah Reckhow, an expert on education philanthropy at Michigan State
University, put it: ``The DeVoses are like: `No, we know what we want.
We don't need to have all this window dressing.'''

Ms. DeVos has led two nonprofits that have spent millions of dollars
electing governors and legislators sympathetic to school vouchers around
the country.

Matt Frendewey, a spokesman for one of the groups, said the efforts had
frequently been bipartisan, and that the amount of money they had spent
has been dwarfed by contributions from teachers' unions opposed to
reform. Yet in Michigan, at least, the family's political strategy has
not been subtle.

After he defied Ms. DeVos on a key charter school vote, Mr. Pumford, the
former Republican legislator, survived an effort by the Great Lakes
Education Project, a nonprofit the DeVoses bankrolled, to defeat him in
his 2002 primary.

\href{https://www.nytimes.com/interactive/2016/us/politics/donald-trump-administration.html}{}

\includegraphics{https://static01.nyt.com/images/2016/11/11/us/politics/donald-trump-administration-1478905372015/donald-trump-administration-1478905372015-square640.jpg}

\hypertarget{donald-trumps-cabinet-is-complete-heres-the-full-list}{%
\subsection{Donald Trump's Cabinet Is Complete. Here's the Full
List.}\label{donald-trumps-cabinet-is-complete-heres-the-full-list}}

A list of appointees and nominees for top posts in the new
administration.

But shortly after, the House speaker told him the Education Committee
chairmanship he coveted would not be forthcoming. ``I said, `Why?''' Mr.
Pumford recalled. ``He said: `You know why. The DeVoses will walk away
from us.''' Mr. Pumford added: ``She told me that was going to happen.''

(Rick Johnson, the House speaker, said he did not recall the
conversation but also that he had not promised Mr. Pumford the
chairmanship and would not have explained his reasons for withholding
it.)

Over time, the Great Lakes Education Project helped elect Republican
majorities sympathetic to the DeVoses' agenda. But the DeVoses'
lobbyists and operatives also discovered less messy ways to advance
legislation.

Late one night of their last workweek in 2015, the Michigan House and
Senate were about to approve some uncontroversial changes to campaign
finance law, when the bill abruptly grew by more than 40 pages.

After the legislators discovered what they had voted for, many said they
were horrified.

Tucked away in the new pages was a provision that would have made it
much harder for local bodies like school boards to raise money through
property tax increases.

``Michigan schools will likely suffer the brunt of the impact because
the vast majority rely on periodic voter approval of local operating
levy renewals for property taxes,'' the ratings agency Moody's wrote of
the measure the following month.

``I was fooled into voting for something I opposed,'' said Dave Pagel, a
Republican representative. ``I consider it the worst vote I've made.''

The chief culprits, according to Mr. Pagel and others at the state
Capitol when the bill passed, were lobbyists closely tied to the
DeVoses.

Tony Daunt, a spokesman for the Michigan Freedom Fund, a nonprofit
headed by the DeVoses' longtime political aide, and whose political
spending arm they have funded generously, said the group was ``part of
the discussion process with people in the legislature'' about the
proposal and ``had consistently expressed support for the policy.''

The law was later blocked by a federal judge, but the group has vowed to
try again.

\hypertarget{radical-suspicions}{%
\subsection{Radical Suspicions}\label{radical-suspicions}}

Ms. DeVos's advocates see in these fights the toughness to take on
entrenched opponents of expanding reforms like charter schools and
vouchers.

In promoting Ms. DeVos in
\href{https://www.washingtonpost.com/opinions/mitt-romney-trump-has-made-a-smart-choice-for-education-secretary/2017/01/06/627550e0-d421-11e6-9cb0-54ab630851e8_story.html?utm_term=.84da2c87052a}{The
Washington Post}, Mitt Romney, the Republican Party's 2012 presidential
nominee, emphasized that her wealth gave her the independence to be
``someone who isn't financially biased shaping education.'' He added,
``DeVos doesn't need the job now, nor will she be looking for an
education job later.''

But critics see someone with an unmistakable agenda. ``The signs are
there that she will do something radical,'' said Jack Jennings, a former
general counsel for the House education committee. ``Trump wouldn't have
appointed this woman for this position if he didn't intend something
radical.''

Advertisement

\protect\hyperlink{after-bottom}{Continue reading the main story}

\hypertarget{site-index}{%
\subsection{Site Index}\label{site-index}}

\hypertarget{site-information-navigation}{%
\subsection{Site Information
Navigation}\label{site-information-navigation}}

\begin{itemize}
\tightlist
\item
  \href{https://help.nytimes.com/hc/en-us/articles/115014792127-Copyright-notice}{©~2020~The
  New York Times Company}
\end{itemize}

\begin{itemize}
\tightlist
\item
  \href{https://www.nytco.com/}{NYTCo}
\item
  \href{https://help.nytimes.com/hc/en-us/articles/115015385887-Contact-Us}{Contact
  Us}
\item
  \href{https://www.nytco.com/careers/}{Work with us}
\item
  \href{https://nytmediakit.com/}{Advertise}
\item
  \href{http://www.tbrandstudio.com/}{T Brand Studio}
\item
  \href{https://www.nytimes.com/privacy/cookie-policy\#how-do-i-manage-trackers}{Your
  Ad Choices}
\item
  \href{https://www.nytimes.com/privacy}{Privacy}
\item
  \href{https://help.nytimes.com/hc/en-us/articles/115014893428-Terms-of-service}{Terms
  of Service}
\item
  \href{https://help.nytimes.com/hc/en-us/articles/115014893968-Terms-of-sale}{Terms
  of Sale}
\item
  \href{https://spiderbites.nytimes.com}{Site Map}
\item
  \href{https://help.nytimes.com/hc/en-us}{Help}
\item
  \href{https://www.nytimes.com/subscription?campaignId=37WXW}{Subscriptions}
\end{itemize}
