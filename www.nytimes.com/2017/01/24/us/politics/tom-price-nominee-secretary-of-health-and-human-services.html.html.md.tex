Sections

SEARCH

\protect\hyperlink{site-content}{Skip to
content}\protect\hyperlink{site-index}{Skip to site index}

\href{https://www.nytimes.com/section/politics}{Politics}

\href{https://myaccount.nytimes.com/auth/login?response_type=cookie\&client_id=vi}{}

\href{https://www.nytimes.com/section/todayspaper}{Today's Paper}

\href{/section/politics}{Politics}\textbar{}Tom Price's Heated Hearing
Is Unlikely to Derail His Nomination

\url{https://nyti.ms/2kqz0IC}

\begin{itemize}
\item
\item
\item
\item
\item
\end{itemize}

Advertisement

\protect\hyperlink{after-top}{Continue reading the main story}

Supported by

\protect\hyperlink{after-sponsor}{Continue reading the main story}

\hypertarget{tom-prices-heated-hearing-is-unlikely-to-derail-his-nomination}{%
\section{Tom Price's Heated Hearing Is Unlikely to Derail His
Nomination}\label{tom-prices-heated-hearing-is-unlikely-to-derail-his-nomination}}

\includegraphics{https://static01.nyt.com/images/2017/01/25/us/25price-vid/25price-vid-videoSixteenByNine3000.jpg}

By \href{https://www.nytimes.com/by/robert-pear}{Robert Pear}

\begin{itemize}
\item
  Jan. 24, 2017
\item
  \begin{itemize}
  \item
  \item
  \item
  \item
  \item
  \end{itemize}
\end{itemize}

WASHINGTON --- In a heated confirmation hearing that focused on ethical
issues, President Trump's nominee for secretary of health and human
services, Representative Tom Price, defended his trading of medical and
pharmaceutical stocks on Tuesday, saying, ``Everything that I did was
ethical, aboveboard, legal and transparent.''

Democrats accused Mr. Price of a potential conflict of interest at a
hearing of the Senate Finance Committee, saying he held more than
\$100,000 in stock in companies that could have benefited from
legislation he promoted. Mr. Price, a Georgia Republican, denied any
wrongdoing.

He also avoided being pinned down on the future of the Affordable Care
Act, Medicare and Medicaid, programs that insure more than 100 million
people.

While Mr. Price faced vigorous questioning from Democrats, he and other
cabinet nominees have benefited from solid support among Republicans,
who hold a 52-48 majority in the Senate.

But some nominees had bipartisan support. By a vote of 96 to 4, the
Senate approved the nomination of Gov. Nikki R. Haley of South Carolina
to be ambassador to the United Nations.

Senator Benjamin L. Cardin of Maryland, the senior Democrat on the
Foreign Relations Committee, praised Ms. Haley for ``moral clarity,''
noting that she had declared that the Russians were guilty of war crimes
in bombing the Syrian city of Aleppo --- an assertion that Rex W.
Tillerson, Mr. Trump's choice for secretary of state, declined to make.

Senate committees endorsed other nominations on Tuesday: Elaine L. Chao
for transportation secretary, Wilbur L. Ross for commerce secretary and
Ben Carson for secretary of housing and urban development.

Linda McMahon, selected by Mr. Trump to lead the Small Business
Administration, also sailed through her confirmation hearing. She was
introduced and endorsed by the senators from her home state,
Connecticut: Richard Blumenthal and Christopher S. Murphy, both
Democrats.

The endorsements were notable because Ms. McMahon ran for the Senate and
was defeated by Mr. Blumenthal in 2010 and Mr. Murphy in 2012.

\href{https://www.nytimes.com/interactive/2016/us/politics/donald-trump-administration.html}{}

\includegraphics{https://static01.nyt.com/images/2016/11/11/us/politics/donald-trump-administration-1478905372015/donald-trump-administration-1478905372015-square640.jpg}

\hypertarget{donald-trumps-cabinet-is-complete-heres-the-full-list}{%
\subsection{Donald Trump's Cabinet Is Complete. Here's the Full
List.}\label{donald-trumps-cabinet-is-complete-heres-the-full-list}}

A list of appointees and nominees for top posts in the new
administration.

Even as some nominees advanced, Republicans were becoming frustrated and
angry. They said Democrats were delaying and prolonging the reviews,
raising procedural objections without changing the likely outcome.

Senator John Cornyn of Texas, the No. 2 Senate Republican, accused the
Democrats of ``obstruction and foot-dragging'' and said they were
``attempting to litigate or re-litigate the election.'' Senator John
Thune, Republican of South Dakota, said: ``The president won an
election. He deserves to get his people in place.''

The hard feelings felt to some like payback. Democrats noted that
Republicans had stalled many of President Barack Obama's picks and
refused to act on his nomination of Judge Merrick B. Garland to the
Supreme Court.

In the Senate Committee on Energy and Natural Resources, Democrats held
up confirmation votes for two nominees: Representative Ryan Zinke of
Montana for interior secretary and former Gov. Rick Perry of Texas for
energy secretary. The Democrats said they had unanswered questions about
important research programs at the Energy Department.

Democrats on the Finance Committee put Mr. Price through three and a
half hours of grueling interrogation. But they appeared unlikely to
block his confirmation unless other damaging information comes to light.

``We've always known where the votes are,'' said Senator Ron Wyden of
Oregon, the senior Democrat on the committee. The panel could vote next
week on whether to recommend confirmation, with its Republican majority
strongly supporting Mr. Price.

Mr. Price was asked about an executive order
\href{https://www.nytimes.com/2017/01/20/us/politics/trump-executive-order-obamacare.html}{issued
by Mr. Trump} on Friday that tells federal officials to provide relief
from costs, penalties and regulatory burdens imposed on consumers,
insurers and health care providers by the Affordable Care Act.

Mr. Wyden asked Mr. Price if he would promise that no one would be worse
off and no one would lose coverage as a result of the order. Mr. Price
declined to provide an explicit assurance. He promised instead to work
with Congress to ``make certain that we have the highest-quality health
care and that every single American has access to affordable coverage.''

Senator Maria Cantwell, Democrat of Washington, said the administration
appeared to be planning a ``war on Medicaid.'' Mr. Price rejected that
description, saying he wanted to give states more control over Medicaid,
which covers more than 70 million low-income people.

Senator Dean Heller, Republican of Nevada, asked Mr. Price for a
commitment that any replacement for the Affordable Care Act would allow
states to continue the expansion of Medicaid eligibility already
approved by Nevada and 30 other states.

Mr. Price deferred to Congress, saying, ``This is a policy question that
needs to be worked out through both the House and the Senate.'' He said
he wanted to ensure that people ``who are currently covered through
Medicaid expansion either retain that coverage or in some way have
coverage through a different vehicle.''

In any event, he said, ``every single individual ought to be able to
have access to coverage.''

He added that many people had Medicaid coverage but could not find
doctors to treat them.

``One out of three physicians who ought to be able to see Medicaid
patients in this nation do not take any Medicaid patients'' because of
low reimbursement rates, burdensome regulations or the ``hassle
factor,'' Mr. Price said.

Mr. Price
\href{https://www.nytimes.com/2016/11/28/us/politics/tom-price-secretary-health-and-human-services.html}{has
led efforts} to repeal the Affordable Care Act. He refused on Tuesday to
say whether he would fight to preserve the expanded coverage of
prescription drugs provided to Medicare beneficiaries under the law.
Rather, he said, ``it is imperative we provide the greatest amount of
opportunity for individual seniors to be able to gain access to the
drugs that they need.''

That answer did not satisfy Senator Bill Nelson, Democrat of Florida,
which is home to more than four million Medicare beneficiaries. ``If I
gave them that answer,'' Mr. Nelson said, ``I would get run out of the
room with a group of senior citizens.''

Mr. Price also declined to say how he would carry out Mr. Trump's
promise to drive down prescription drug prices. Mr. Trump has said he
will do so by
\href{https://www.nytimes.com/2017/01/23/health/the-fight-trump-faces-over-drug-prices.html}{negotiating
with drug manufacturers} and by requiring them to bid for government
business.

A background investigation of Mr. Price found that he had understated
the value of his investments in an Australian pharmaceutical company and
claimed income tax deductions that he could not substantiate. The
findings emerged from a review of his tax returns and other official
documents by Finance Committee staff members from both parties.

In a questionnaire in December, the staff said, Mr. Price understated
the value of 400,613 shares of the Australian company, Innate
Immunotherapeutics, that he purchased in August through ``a private
placement offering.''

``It's hard to see how this can be anything but a conflict of interest
and an abuse of his position,'' Mr. Wyden said. But Mr. Price said that
he did not have any ``nonpublic information'' about the company. And the
committee chairman, Senator Orrin G. Hatch, Republican of Utah, said the
Democrats' attacks on Mr. Price's ethics were ``specious and
distorted.''

The committee staff added that Mr. Price had taken ``improper deductions
on his 2016 tax returns'' for the depreciation of land associated with
condominiums he owns in Washington and Nashville. He and his wife, both
physicians, also claimed ``miscellaneous employment deductions totaling
\$19,034'' for expenses in 2013, 2014 and 2015, the staff said. But
``proper documentation could not be located,'' so his returns will be
amended.

Advertisement

\protect\hyperlink{after-bottom}{Continue reading the main story}

\hypertarget{site-index}{%
\subsection{Site Index}\label{site-index}}

\hypertarget{site-information-navigation}{%
\subsection{Site Information
Navigation}\label{site-information-navigation}}

\begin{itemize}
\tightlist
\item
  \href{https://help.nytimes.com/hc/en-us/articles/115014792127-Copyright-notice}{©~2020~The
  New York Times Company}
\end{itemize}

\begin{itemize}
\tightlist
\item
  \href{https://www.nytco.com/}{NYTCo}
\item
  \href{https://help.nytimes.com/hc/en-us/articles/115015385887-Contact-Us}{Contact
  Us}
\item
  \href{https://www.nytco.com/careers/}{Work with us}
\item
  \href{https://nytmediakit.com/}{Advertise}
\item
  \href{http://www.tbrandstudio.com/}{T Brand Studio}
\item
  \href{https://www.nytimes.com/privacy/cookie-policy\#how-do-i-manage-trackers}{Your
  Ad Choices}
\item
  \href{https://www.nytimes.com/privacy}{Privacy}
\item
  \href{https://help.nytimes.com/hc/en-us/articles/115014893428-Terms-of-service}{Terms
  of Service}
\item
  \href{https://help.nytimes.com/hc/en-us/articles/115014893968-Terms-of-sale}{Terms
  of Sale}
\item
  \href{https://spiderbites.nytimes.com}{Site Map}
\item
  \href{https://help.nytimes.com/hc/en-us}{Help}
\item
  \href{https://www.nytimes.com/subscription?campaignId=37WXW}{Subscriptions}
\end{itemize}
