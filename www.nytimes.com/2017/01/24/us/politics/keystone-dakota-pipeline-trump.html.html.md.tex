Sections

SEARCH

\protect\hyperlink{site-content}{Skip to
content}\protect\hyperlink{site-index}{Skip to site index}

\href{https://www.nytimes.com/section/politics}{Politics}

\href{https://myaccount.nytimes.com/auth/login?response_type=cookie\&client_id=vi}{}

\href{https://www.nytimes.com/section/todayspaper}{Today's Paper}

\href{/section/politics}{Politics}\textbar{}Trump Revives Keystone
Pipeline Rejected by Obama

\url{https://nyti.ms/2kpePL5}

\begin{itemize}
\item
\item
\item
\item
\item
\item
\end{itemize}

Advertisement

\protect\hyperlink{after-top}{Continue reading the main story}

Supported by

\protect\hyperlink{after-sponsor}{Continue reading the main story}

\hypertarget{trump-revives-keystone-pipeline-rejected-by-obama}{%
\section{Trump Revives Keystone Pipeline Rejected by
Obama}\label{trump-revives-keystone-pipeline-rejected-by-obama}}

\includegraphics{https://static01.nyt.com/images/2017/01/25/us/25trump-1/25trump-1-videoSixteenByNineJumbo1600.jpg}

By \href{http://www.nytimes.com/by/peter-baker}{Peter Baker} and
\href{https://www.nytimes.com/by/coral-davenport}{Coral Davenport}

\begin{itemize}
\item
  Jan. 24, 2017
\item
  \begin{itemize}
  \item
  \item
  \item
  \item
  \item
  \item
  \end{itemize}
\end{itemize}

WASHINGTON ---
\href{https://www.nytimes.com/topic/person/donald-trump}{President
Trump} sharply changed the federal government's approach to the
environment on Tuesday as he cleared the way for two major oil pipelines
that had been blocked, and set in motion a plan to curb regulations that
slow other building projects.

In his latest moves to dismantle the legacy of his predecessor, Mr.
Trump resurrected the
\href{https://www.nytimes.com/topic/subject/keystone-xl-pipeline?inline=nyt-classifier}{Keystone
XL} pipeline that had
\href{https://www.nytimes.com/2015/11/07/us/obama-expected-to-reject-construction-of-keystone-xl-oil-pipeline.html}{stirred
years of debate}, and expedited another
\href{https://www.nytimes.com/2016/12/05/us/veterans-north-dakota-standing-rock.html}{pipeline
in the Dakotas} that had become a major flash point for Native
Americans. He also signed a directive ordering an end to protracted
environmental reviews.

``I am, to a large extent, an environmentalist, I believe in it,'' Mr.
Trump said during a meeting with auto industry executives. ``But it's
out of control, and we're going to make it a very short process. And
we're going to either give you your permits, or we're not going to give
you your permits. But you're going to know very quickly. And generally
speaking, we're going to be giving you your permits.''

The decisions expanded an effort to unravel much of the policy structure
left by former President
\href{https://www.nytimes.com/topic/person/barack-obama?inline=nyt-per}{Barack
Obama}, who made fighting
\href{https://www.nytimes.com/topic/subject/global-warming-climate-change?inline=nyt-classifier\%5C}{climate
change} a central priority. Just a day earlier, Mr. Trump formally
\href{https://www.nytimes.com/2017/01/23/us/politics/tpp-trump-trade-nafta.html?hp\&action=click\&pgtype=Homepage\&clickSource=story-heading\&module=first-column-region\&region=top-news\&WT.nav=top-news}{abandoned
the Trans-Pacific Partnership}, an ambitious 12-nation trade pact
negotiated by Mr. Obama.

In his opening days in office, Mr. Trump has also modified or reversed
Mr. Obama's policies on
\href{https://www.nytimes.com/2017/01/20/us/politics/trump-executive-order-obamacare.html}{health
care},
\href{https://www.nytimes.com/2017/01/23/world/trump-ban-foreign-aid-abortions.html}{abortion}and
housing while ordering a freeze of any pending regulations left behind
by the former administration.

The pipelines were more about symbol than substance but generated
enormous passion on both sides of the debate. Mr. Obama rejected the
proposed Keystone pipeline in 2015, arguing that it would undercut
American leadership in curbing the reliance on carbon energy. The
\href{https://www.nytimes.com/2016/12/04/us/federal-officials-to-explore-different-route-for-dakota-pipeline.html}{Army
sidetracked} the Dakota Access pipeline in North Dakota last month in
the waning days of the Obama administration.

Environmental activists quickly denounced Mr. Trump's decisions.
``Donald Trump has been in office for four days, and he's already
proving to be the dangerous threat to our climate we feared he would
be,'' said Michael Brune, the executive director of the Sierra Club.

\href{https://www.nytimes.com/interactive/2016/11/23/us/dakota-access-pipeline-protest-map.html}{}

\includegraphics{https://static01.nyt.com/images/2016/11/23/us/dakota-access-pipeline-1479877294784/dakota-access-pipeline-1479877294784-thumbLarge.jpg}

\hypertarget{the-conflicts-along-1172-miles-of-the-dakota-access-pipeline}{%
\subsection{The Conflicts Along 1,172 Miles of the Dakota Access
Pipeline}\label{the-conflicts-along-1172-miles-of-the-dakota-access-pipeline}}

A detailed map showing the Dakota Access Pipeline, the site of months of
clashes near the Standing Rock Sioux Reservation in North Dakota.

Mr. Trump made clear on the campaign trail that he saw Mr. Obama's
environmental policies as a threat to the economy and dismissed climate
change as a hoax perpetrated by China. Myron Ebell, a climate change
denier who headed Mr. Trump's Environmental Protection Agency transition
team, has drafted a 50-page blueprint for how he could eliminate Mr.
Obama's climate change policies. ``It is designed to implement all of
the president's campaign trail promises --- every single one,'' Mr.
Ebell said this week in an interview.

Mr. Trump's biggest target may be emission rules that would force the
closing of hundreds of coal-fired power plants meant to be replaced by
wind and solar power. But they are caught up in court battles that could
run for months or years.

By contrast, he could more quickly soften Mr. Obama's rules requiring
tougher vehicle emission standards. Mr. Trump met on Tuesday with
executives of major American automakers, who complained that before
leaving office, Mr. Obama finalized an ambitious E.P.A. rule requiring
that vehicles average 54.5 miles per gallon by 2026. Mr. Trump said he
would help with burdensome regulations, but offered no specifics.

Mr. Trump could lift a moratorium instituted last year by Mr. Obama on
new coal mining leases on public lands. As soon as next month, the
Republican-led Congress may pass legislation undoing Mr. Obama's
regulations on the practice of mountaintop-removal coal mining and on
leaks of planet-warming methane emissions from oil and gas drilling
rigs.

In the meantime, the Keystone and Dakota pipelines provided Mr. Trump
with visible ways to demonstrate action. As proposed by TransCanada, an
Alberta firm, Keystone would carry 800,000 barrels a day from the
Canadian
\href{https://www.nytimes.com/topic/subject/oil-sands?inline=nyt-classifier}{oil
sands} to the Gulf Coast. Republicans and some Democrats said that it
would create jobs and expand energy resources, while environmentalists
said it would encourage a form of oil extraction that produces more
gases that warm the planet than normal petroleum.

Studies showed that the pipeline would not have a momentous effect on
jobs or the environment, but both sides made it into a symbolic test
case. The State Department estimated that Keystone would support 42,000
temporary jobs for two years --- about 3,900 of them in construction and
the rest through indirect support, like food service --- but only 35
permanent jobs. Similarly, the government concluded that Keystone's
carbon emissions would equal less than 1 percent of the total greenhouse
gas emissions in the United States.

``Keystone has never been a significant issue from an environmental
point of view in substance, only in symbol,'' said David L. Goldwyn, an
energy market analyst and a former head of the State Department's energy
bureau in the Obama administration.

But it was a symbol Mr. Trump found important enough to seize on early
in his presidency. He signed an executive memorandum inviting
TransCanada ``to promptly resubmit its application to the Department of
State for a presidential permit'' for the pipeline, although the
document did not guarantee approval.

The president told reporters he would ``renegotiate some of the terms''
--- including possibly an insistence that the pipeline be built with
American steel --- but left little doubt that he wanted it approved.
``We'll see if we can get that pipeline built,'' he said. ``A lot of
jobs.''

In a statement, TransCanada accepted his invitation to seek permission
again. ``We are currently preparing the application and intend to do
so,'' the company said, vowing that it would create jobs and still
protect waterways and other sensitive resources.

The Dakota Access pipeline in North Dakota became the focus of protests
when the Standing Rock Sioux Tribe objected to its construction less
than a mile from its reservation. The tribe and its allies won victory
last month when the Army Corps of Engineers announced that it would look
for alternative routes for the \$3.7 billion pipeline instead of
allowing it to be drilled under a dammed section of the Missouri River.

Mr. Trump signed an executive memorandum directing the Army ``to review
and approve in an expedited manner'' the pipeline, ``to the extent
permitted by law and as warranted.'' In his session with reporters, he
added, ``Again, subject to terms and conditions to be negotiated by
us.''

Mr. Trump owned stock in Energy Transfer Partners, the company that is
building the Dakota Access pipeline, according to his most recent filing
with the Federal Election Commission. Last month, a spokesman for Mr.
Trump said he sold all of his stock in June, but there is no way of
verifying that sale, and Mr. Trump has not provided documentation of it.

Critics vowed to keep resisting the projects. Jan Hasselman, a lawyer
for Earthjustice, an environmental law group representing the tribe,
said Mr. Trump was discarding the findings of a review. ``They're just
ignoring the problems that the government has already found,'' he said,
``and that is the kind of thing that courts need to review very
closely.''

In Canada, the government of Prime Minister Justin Trudeau welcomed Mr.
Trump's decision. ``We have been supportive of this since the day we
were sworn into government,'' Jim Carr, the natural resources minister,
told reporters. Mr. Carr said the American reversal will lead ``to a
deepening of the relationship across the border.''

In addition to the Keystone and Dakota directives, Mr. Trump signed
three others intended to ease the way for businesses and to promote
American manufacturing. One instructed the Commerce Department to
develop a plan to ensure that future pipelines built in the United
States be constructed of American-made materials.

Another was aimed at streamlining what he called ``the incredibly
cumbersome, long, horrible permitting process and reducing regulatory
burdens for domestic manufacturing.'' The last directive was intended to
expedite environmental reviews for ``high-priority infrastructure
projects'' like highways and bridges.

Some news reports on Tuesday said that the E.P.A. and other departments
had issued orders forbidding employees from issuing news releases or
posting on social media. But longtime officials in multiple agencies
said the guidance was similar to that of when Mr. Obama took office
eight years ago.

Advertisement

\protect\hyperlink{after-bottom}{Continue reading the main story}

\hypertarget{site-index}{%
\subsection{Site Index}\label{site-index}}

\hypertarget{site-information-navigation}{%
\subsection{Site Information
Navigation}\label{site-information-navigation}}

\begin{itemize}
\tightlist
\item
  \href{https://help.nytimes.com/hc/en-us/articles/115014792127-Copyright-notice}{©~2020~The
  New York Times Company}
\end{itemize}

\begin{itemize}
\tightlist
\item
  \href{https://www.nytco.com/}{NYTCo}
\item
  \href{https://help.nytimes.com/hc/en-us/articles/115015385887-Contact-Us}{Contact
  Us}
\item
  \href{https://www.nytco.com/careers/}{Work with us}
\item
  \href{https://nytmediakit.com/}{Advertise}
\item
  \href{http://www.tbrandstudio.com/}{T Brand Studio}
\item
  \href{https://www.nytimes.com/privacy/cookie-policy\#how-do-i-manage-trackers}{Your
  Ad Choices}
\item
  \href{https://www.nytimes.com/privacy}{Privacy}
\item
  \href{https://help.nytimes.com/hc/en-us/articles/115014893428-Terms-of-service}{Terms
  of Service}
\item
  \href{https://help.nytimes.com/hc/en-us/articles/115014893968-Terms-of-sale}{Terms
  of Sale}
\item
  \href{https://spiderbites.nytimes.com}{Site Map}
\item
  \href{https://help.nytimes.com/hc/en-us}{Help}
\item
  \href{https://www.nytimes.com/subscription?campaignId=37WXW}{Subscriptions}
\end{itemize}
