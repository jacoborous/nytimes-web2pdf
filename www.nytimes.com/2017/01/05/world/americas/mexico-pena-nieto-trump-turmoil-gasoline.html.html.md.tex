Sections

SEARCH

\protect\hyperlink{site-content}{Skip to
content}\protect\hyperlink{site-index}{Skip to site index}

\href{https://www.nytimes.com/section/world/americas}{Americas}

\href{https://myaccount.nytimes.com/auth/login?response_type=cookie\&client_id=vi}{}

\href{https://www.nytimes.com/section/todayspaper}{Today's Paper}

\href{/section/world/americas}{Americas}\textbar{}For Mexican Leaders, a
Turbulent Start to the New Year

\url{https://nyti.ms/2hYjmG0}

\begin{itemize}
\item
\item
\item
\item
\item
\end{itemize}

Advertisement

\protect\hyperlink{after-top}{Continue reading the main story}

Supported by

\protect\hyperlink{after-sponsor}{Continue reading the main story}

\hypertarget{for-mexican-leaders-a-turbulent-start-to-the-new-year}{%
\section{For Mexican Leaders, a Turbulent Start to the New
Year}\label{for-mexican-leaders-a-turbulent-start-to-the-new-year}}

\includegraphics{https://static01.nyt.com/images/2017/01/07/world/06MEXICO-1/06MEXICO-1-articleLarge.jpg?quality=75\&auto=webp\&disable=upscale}

By \href{https://www.nytimes.com/by/elisabeth-malkin}{Elisabeth Malkin}

\begin{itemize}
\item
  Jan. 5, 2017
\item
  \begin{itemize}
  \item
  \item
  \item
  \item
  \item
  \end{itemize}
\end{itemize}

MEXICO CITY --- Six days into the new year,
\href{https://www.nytimes.com/topic/destination/mexico?8qa}{Mexico}
already has little to be happy about.

This week a jump in gasoline prices unleashed widespread protests that
spiraled into looting. The country received an ominous warning that
President-elect Donald J. Trump's protectionist rhetoric could have
concrete effects when Ford Motor canceled a \$1.6 billion investment.
The peso fell to its lowest level ever.

The new turmoil promises to make this year even more difficult for
President Enrique Peña Nieto, whose approval ratings have already
plunged below 25 percent.

He returned from a golf vacation on Wednesday and appealed for unity as
images ricocheted across social media of people carting away televisions
from Walmarts and stealing snack foods from stalled delivery trucks.

Protests continued on Thursday, as demonstrators blocked highways and
gas stations. Scattered looting continued, and marches are planned for
this weekend to demand a reversal of the price increases. The
president's explanation that the gasoline increase of almost 20 percent
was necessary to maintain economic stability did nothing to calm the
outrage. ``Even in good times, it is a problematic decision'' to raise
gasoline prices, Vidal Romero, a political analyst at the Autonomous
Technological Institute of Mexico, said. ``And this is a very bad
moment.''

Uncertainty has roiled Mexico as
\href{https://www.nytimes.com/2016/11/10/world/americas/mexico-donald-trump-peso.html?action=click\&contentCollection=Americas\&module=RelatedCoverage\&region=Marginalia\&pgtype=article}{the
government waits to see} how far Mr. Trump will go to keep his campaign
promises to renegotiate or tear up the North American Free Trade
Agreement, deport Mexican migrants and build a border wall.

On Tuesday, Ford announced that it was canceling its planned investment
to build a small-car plant in the state of San Luis Potosí. Although
falling sales of small cars may have had more to do with Ford's decision
than Mr. Trump's criticism on Twitter, the president-elect promised that
the Ford episode was just ``the beginning.''

He followed up with a broadside at General Motors for building the
Chevrolet Cruze hatchback in Mexico, although only 4,500 of them were
exported to the United States last year. On Thursday, he trained his
Twitter fire
\href{https://twitter.com/realDonaldTrump/status/817071792711942145}{on
Toyota}, saying ``No Way'' to the company's plan to build a Corolla
factory in Mexico and warned: ``Build plant in U.S. or pay big border
tax.''

In response to the Ford announcement, the peso sank to a record low,
prompting the central bank to intervene in markets on Thursday. The
peso's recovery proved short-lived after Mr. Trump took aim at Toyota.
According\href{http://pressroom.toyota.com/releases/toyota+manufacturing+billion+mexico+corolla.htm}{to
Toyota}, the new plant --- in the central state of Guanajuato, not Baja
California as Mr. Trump asserted --- would shift Corolla production from
a Canadian factory, which would then switch to producing midsize cars.

Since last summer, the Mexican government has struggled to respond to
Mr. Trump's rise. It even hosted him for a visit, prompting a furious
response from across Mexico's political spectrum. Luis Videgaray --- Mr.
Peña Nieto's finance minister at the time, who championed the visit ---
resigned. But on Wednesday, the president brought Mr. Videgaray back
into the cabinet
\href{https://www.nytimes.com/2017/01/04/world/americas/mexico-united-states-trump-pena-nieto-videgaray.html?rref=collection\%2Ftimestopic\%2FMexico\&action=click\&contentCollection=world\&region=stream\&module=stream_unit\&version=latest\&contentPlacement=1\&pgtype=collection}{as
foreign minister} in the hope that his presence would smooth relations
with the incoming Trump administration.

At the brief ceremony to announce Mr. Videgaray's return, Mr. Peña Nieto
seemed to address the upheaval caused by the gas prices as an
afterthought. ``I repeat, it hasn't been easy to take this measure,''
Mr. Peña Nieto said. ``But it is with a sense of responsibility to
safeguard the stability of our economy.''

Talk of economic sobriety sits poorly with Mexicans, disgusted by a
series of political scandals. ``This is a government with a terrible
record of corruption,'' Mr. Romero said. ``State, federal --- everything
smells of corruption.''

The gas-price increase was approved last year by Congress as part of an
austerity budget designed to insulate Mexico from the market
uncertainties of Mr. Trump's rise. The government plans to let prices
--- which have long been controlled and subsidized --- float by the end
of the year. This should lead to competition, and eventually lower
prices. ``The incredible thing is that the government didn't expect the
reaction,'' said Ignacio Marván, a political analyst at CIDE, a Mexico
City university.

Truck and taxi drivers have blocked highways since Sunday. Outbreaks of
looting escalated into a wave on Wednesday. A Mexico City police officer
was killed as he tried to stop looters.

``They didn't take the measure of people's anger,'' said Graco Ramírez,
the governor of the central state of Morelos and a member of the
left-wing opposition. ``Everything is going to be more expensive.''

Advertisement

\protect\hyperlink{after-bottom}{Continue reading the main story}

\hypertarget{site-index}{%
\subsection{Site Index}\label{site-index}}

\hypertarget{site-information-navigation}{%
\subsection{Site Information
Navigation}\label{site-information-navigation}}

\begin{itemize}
\tightlist
\item
  \href{https://help.nytimes.com/hc/en-us/articles/115014792127-Copyright-notice}{©~2020~The
  New York Times Company}
\end{itemize}

\begin{itemize}
\tightlist
\item
  \href{https://www.nytco.com/}{NYTCo}
\item
  \href{https://help.nytimes.com/hc/en-us/articles/115015385887-Contact-Us}{Contact
  Us}
\item
  \href{https://www.nytco.com/careers/}{Work with us}
\item
  \href{https://nytmediakit.com/}{Advertise}
\item
  \href{http://www.tbrandstudio.com/}{T Brand Studio}
\item
  \href{https://www.nytimes.com/privacy/cookie-policy\#how-do-i-manage-trackers}{Your
  Ad Choices}
\item
  \href{https://www.nytimes.com/privacy}{Privacy}
\item
  \href{https://help.nytimes.com/hc/en-us/articles/115014893428-Terms-of-service}{Terms
  of Service}
\item
  \href{https://help.nytimes.com/hc/en-us/articles/115014893968-Terms-of-sale}{Terms
  of Sale}
\item
  \href{https://spiderbites.nytimes.com}{Site Map}
\item
  \href{https://help.nytimes.com/hc/en-us}{Help}
\item
  \href{https://www.nytimes.com/subscription?campaignId=37WXW}{Subscriptions}
\end{itemize}
