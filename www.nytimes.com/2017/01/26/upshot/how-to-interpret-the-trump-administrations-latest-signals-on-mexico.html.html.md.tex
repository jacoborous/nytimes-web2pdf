Sections

SEARCH

\protect\hyperlink{site-content}{Skip to
content}\protect\hyperlink{site-index}{Skip to site index}

\href{https://myaccount.nytimes.com/auth/login?response_type=cookie\&client_id=vi}{}

\href{https://www.nytimes.com/section/todayspaper}{Today's Paper}

\href{/section/upshot}{The Upshot}\textbar{}How to Interpret the Trump
Administration's Latest Signals on Mexico

\url{https://nyti.ms/2k8Xt8N}

\begin{itemize}
\item
\item
\item
\item
\item
\item
\end{itemize}

Advertisement

\protect\hyperlink{after-top}{Continue reading the main story}

Supported by

\protect\hyperlink{after-sponsor}{Continue reading the main story}

Upshot

Tax Policy

\hypertarget{how-to-interpret-the-trump-administrations-latest-signals-on-mexico}{%
\section{How to Interpret the Trump Administration's Latest Signals on
Mexico}\label{how-to-interpret-the-trump-administrations-latest-signals-on-mexico}}

\includegraphics{https://static01.nyt.com/images/2017/01/27/upshot/27up-tax/27up-tax-articleInline.jpg?quality=75\&auto=webp\&disable=upscale}

By \href{http://www.nytimes.com/by/neil-irwin}{Neil Irwin}

\begin{itemize}
\item
  Jan. 26, 2017
\item
  \begin{itemize}
  \item
  \item
  \item
  \item
  \item
  \item
  \end{itemize}
\end{itemize}

The White House floated an idea on Thursday afternoon that, in initial
reports, sounded like a major tariff on Mexican imports --- something
that would have gone a long way toward unwinding one of the United
States' deepest economic relationships.

The reality of what Sean Spicer, the press secretary, suggested is a lot
less dramatic. But it sends important signals about how people in the
Trump administration are thinking about overhauling the tax code --- and
how they're thinking about claiming victory on some of the president's
audacious campaign promises. It is a sign of just how fluid things are
in this moment when so much of American public policy around taxes,
trade and diplomacy is in flux.

Mr. Spicer suggested a way the administration could accomplish President
Trump's goal of building a border wall paid for by Mexico. A 20 percent
tax on imports from Mexico would do the trick, Mr. Spicer said.

That might sound as if Mr. Spicer was proposing that the United States
slap a new tariff meant to punish Mexican exporters. Such a move would
result in higher prices for American consumers, create profound
challenges for industries with supply chains that span the border, and
possibly prompt the collapse of the North American Free Trade Agreement.

But you get a different picture when you put Mr. Spicer's words into the
context of the rapidly evolving debate in Washington around overhauling
corporate taxation.

He was pointing out that in an overhaul of taxes that House Republicans
are considering, imports from all countries would be taxed at 20 percent
while American exports would be tax free. It's called
\href{https://www.nytimes.com/2017/01/07/upshot/the-major-potential-impact-of-a-corporate-tax-overhaul.html}{border
adjustment,} and it would make the United States corporate tax code more
closely resemble the value-added tax that is commonplace in other
countries.

House Republicans see the policy as a way to reshape the tax code to
give businesses less incentive to move operations overseas while also
generating revenue they can use to reduce tax rates.

Opponents of the plan, which include major retailers, are skeptical.
Among the risks: It could drive up consumer prices for all sorts of
imported goods, from German cars to Mexican avocados, if the dollar does
not rise as much as economists predict. And the policy may violate World
Trade Organization rules, which could tangle it up in legal proceedings.

But that Mr. Spicer was floating that plan as a way to fulfill Mr.
Trump's Mexican wall promises is interesting on two levels.

First, less than two weeks ago, the then-president-elect threw cold
water on the House plan. ``Anytime I hear border adjustment, I don't
love it,'' Mr. Trump told
\href{http://www.wsj.com/articles/trump-warns-on-house-republican-tax-plan-1484613766}{The
Wall Street Journal}. ``Because usually it means we're going to get
adjusted into a bad deal.''

On Thursday, Mr. Spicer was explicitly suggesting that a border tax
could be used to pay for a border wall. Referring to the tax plan, he
said, ``This is something that we've been in close contact with both
houses in moving forward.''

The border adjustment strategy has plenty of enemies, and there's no
certainty that it will become part of a tax overhaul bill. But the
latest tea leaves suggest the administration is more open to it than it
may have seemed.

The second lesson from the incident is that the Trump administration
looks inclined to be flexible in finding ways to satisfy campaign
promises without doing major damage to the economy or international
relations.

Thursday was one of the roughest days for relations between the United
States and Mexico in some time, with the cancellation of a planned visit
by President Enrique Peña Nieto and tough talk from Mexico City, which
adamantly refuses to pay for an expansion of a border wall.

But Mr. Spicer's comments, which he later said were meant more to offer
an example than a concrete policy proposal, suggest that the
administration will look for creative ways to proclaim victory on
Trumpian promises. In other words, he will proclaim that Mexico has paid
for the wall as promised --- even if the Mexican government never
literally cuts a check to pay for new concrete.

Advocates of the border adjustment tax have been fond of it because it
would produce enough revenue to allow a deep reduction in tax rates. But
money is fungible. So if the president can claim political victory by
stating that the revenue from Mexican imports is going to pay for the
wall, no one is going to stop him.

It is a messy time for the making of economic policy. The Trump campaign
was notoriously light on policy detail, and the Trump administration
still has many key vacancies in economic policy jobs. Nominees for
Treasury secretary, commerce secretary and U.S. trade representative
have not yet been confirmed, and key jobs on the Council of Economic
Advisers and most undersecretary and assistant secretary jobs remain
unfilled.

So the gaps are still being filled in on what the Trump administration
economic policy will really mean in practice. The way to read the latest
Mexico comments is as one more hint.

Advertisement

\protect\hyperlink{after-bottom}{Continue reading the main story}

\hypertarget{site-index}{%
\subsection{Site Index}\label{site-index}}

\hypertarget{site-information-navigation}{%
\subsection{Site Information
Navigation}\label{site-information-navigation}}

\begin{itemize}
\tightlist
\item
  \href{https://help.nytimes.com/hc/en-us/articles/115014792127-Copyright-notice}{©~2020~The
  New York Times Company}
\end{itemize}

\begin{itemize}
\tightlist
\item
  \href{https://www.nytco.com/}{NYTCo}
\item
  \href{https://help.nytimes.com/hc/en-us/articles/115015385887-Contact-Us}{Contact
  Us}
\item
  \href{https://www.nytco.com/careers/}{Work with us}
\item
  \href{https://nytmediakit.com/}{Advertise}
\item
  \href{http://www.tbrandstudio.com/}{T Brand Studio}
\item
  \href{https://www.nytimes.com/privacy/cookie-policy\#how-do-i-manage-trackers}{Your
  Ad Choices}
\item
  \href{https://www.nytimes.com/privacy}{Privacy}
\item
  \href{https://help.nytimes.com/hc/en-us/articles/115014893428-Terms-of-service}{Terms
  of Service}
\item
  \href{https://help.nytimes.com/hc/en-us/articles/115014893968-Terms-of-sale}{Terms
  of Sale}
\item
  \href{https://spiderbites.nytimes.com}{Site Map}
\item
  \href{https://help.nytimes.com/hc/en-us}{Help}
\item
  \href{https://www.nytimes.com/subscription?campaignId=37WXW}{Subscriptions}
\end{itemize}
