Sections

SEARCH

\protect\hyperlink{site-content}{Skip to
content}\protect\hyperlink{site-index}{Skip to site index}

\href{https://www.nytimes.com/section/opinion/sunday}{Sunday Review}

\href{https://myaccount.nytimes.com/auth/login?response_type=cookie\&client_id=vi}{}

\href{https://www.nytimes.com/section/todayspaper}{Today's Paper}

\href{/section/opinion/sunday}{Sunday Review}\textbar{}How Gymnastics
Culture Breeds Sexual Abuse

\url{https://nyti.ms/2oEbkT1}

\begin{itemize}
\item
\item
\item
\item
\item
\end{itemize}

Advertisement

\protect\hyperlink{after-top}{Continue reading the main story}

Supported by

\protect\hyperlink{after-sponsor}{Continue reading the main story}

\href{/section/opinion}{Opinion}

Opinion

\hypertarget{how-gymnastics-culture-breeds-sexual-abuse}{%
\section{How Gymnastics Culture Breeds Sexual
Abuse}\label{how-gymnastics-culture-breeds-sexual-abuse}}

By Jennifer Sey

\begin{itemize}
\item
  March 30, 2017
\item
  \begin{itemize}
  \item
  \item
  \item
  \item
  \item
  \end{itemize}
\end{itemize}

\includegraphics{https://static01.nyt.com/images/2017/04/02/opinion/02sey/02sey-articleInline.jpg?quality=75\&auto=webp\&disable=upscale}

Jamie Dantzscher told the members of the Senate Judiciary Committee on
Tuesday that, starting when she was 12 years old, a man who was supposed
to be looking out for her well-being did just the opposite: He sexually
abused her. She was an elite gymnast. He was the team doctor.

Ms. Dantzscher, who was a member of the bronze-medal-winning women's
Olympic gymnastics team in 2000, was speaking at a hearing on a bill
with potential to check the culture of a sport in which young girls are
too often victimized, by requiring that the adults who work with them
report suspected sexual abuse.

Her testimony, which was at times delivered through tears, left me
feeling a familiar sense of dread. As a former elite gymnast and the
1986 national champion, I understand all too well the dynamics that have
been brought to light by the recent onslaught of public allegations of
sexual misconduct committed against young athletes.

Women's gymnastics is a sport in which the athletes are very young and
barely clothed, and many of the coaches are male. It is a sport in which
screaming insults at children is considered an accepted motivational
technique, in which competing with severe injuries is the norm, in which
discouraging athletes from eating is common practice and in which abuse,
broadly defined, is standard.

This is well known within the sport, and now the even more sinister side
of the world of gymnastics is getting attention.

Larry Nassar, the former team doctor for USA Gymnastics, faces multiple
sexual assault and pornography charges involving at least seven
gymnasts.

``Dr. Nassar abused me at the U.S. national training center in Texas,''
Ms. Dantzscher, who's now 34,
\href{http://www.ocregister.com/articles/gymnastics-747832-abuse-sexual.html}{said}
at Tuesday's hearing. ``He abused me in California at meets and all over
the world. Many times the abuse took place in my own room and my own
bed. Worse, he abused me in my hotel room in Sydney at the Olympic
Games.''

She said in a February interview with ``60 Minutes'' that, under the
guise of treating her back pain and other injuries, he would insert his
hand into her vagina. It's a procedure that Dr. Nassar's attorney
maintains is a standard osteopathic treatment. In an interview with
Sports Illustrated, a spokeswoman for the American Osteopathic
Association
\href{https://www.si.com/more-sports/2017/03/03/scorecard-protect-and-serve-lawrence-nassar-assault}{disagreed}.

Dr. Nassar has pleaded not guilty to all of the charges against him; USA
Gymnastics has denied any wrongdoing in the matter and emphasized that
it reported him to the F.B.I.

The problems within gymnastics culture are much bigger than the
allegations against this doctor. An investigative report by
\href{http://www.indystar.com/story/news/2017/03/16/indianapolis-star-indystar-usa-gymnastics-steve-penny-child-sexual-abuse/99270916/}{IndyStar}has
revealed that between 1996 and 2006, USA Gymnastics failed to
immediately ban some of the 54 coaches who had sexual abuse convictions.
(In a March 3 statement,
\href{https://www.usagym.org/pages/post.html?PostID=19818}{USA
Gymnastics said} that of the 54 coaches whose sexual abuse complaint
files were in the court documents obtained by IndyStar, it had banned
37, and ``48 of the matters involved law enforcement.'')

But in a 2015
\href{https://www.documentcloud.org/documents/2940062-Penny-No-Duty-for-Third-Party-to-Report.html}{deposition},
Steve Penny, then the C.E.O. of the gymnastics organization, suggested
that it was not obligated to make such reports. ``To the best of my
knowledge, there's no duty to report if you are --- if you are a third
party to some allegation,'' he said.

At Tuesday's hearing, Rick Adams, chief of Paralympic sports for the
United States Olympic Committee, said, ``The athletes have spoken very
clearly to what is a flawed culture where the brand and the sport and
the results are given a higher priority than the health and well-being
of the athletes.''

He's right. And I know this environment well. When I was training, I
blackened my eyes when I fell on my head on the beam after fasting for
three days before a competition. ``I don't coach fat gymnasts'' was a
common refrain from coaches antagonizing me about my weight. I competed
on an injured ankle swollen to the size of a baseball. At one point, I
required monthly cortisone injections to limp through my floor routine.

After I broke my femur at the 1985 world championships, I had the cast
removed early under pressure from my coaches so that I could train for
the next national championships. I competed and won, but not without
breaking the opposite ankle in the process.

The message I got was that if you couldn't take it, you were weak. If
you complained, you didn't deserve to be on the team. In fact, if you
perceived it as abuse, rather than just plain old tough coaching, you
were delusional.

I wasn't the victim of sexual misconduct. But the consequences of the
culture that allowed the kind of treatment I endured can't be
overstated. In such an environment, you learn to focus only on
achievement and to disregard your own sense of right and wrong, along
with your own well-being. Because of this, I can understand how young
gymnasts might be confused about whether and how to speak up for
themselves when they've been mistreated.

But there's no excuse for adults to turn a blind eye to sexual
misconduct.

That's why the new bill --- which would require amateur-athletics
governing bodies and those who work at their facilities to report
sex-abuse allegations to local or federal law enforcement, or a
child-welfare agency designated by the Justice Department --- is so
important.

While the attention of lawmakers and Mr. Penny's March 16
\href{https://www.nytimes.com/2017/03/16/sports/steve-penny-resigns-as-usa-gymnastics-president.html}{resignation}
are encouraging signs of improvement, they are just the beginning. To
dramatically shift the culture that has allowed abuse to go unchecked,
wholesale change in leadership is required. That includes the board of
directors and other key leadership positions at USA Gymnastics.

In addition, the organization should more stringently mandate education
programs for coaches and athletes, covering topics like what is
acceptable touching and what is not. When it comes to suspected sexual
assault, reporting protocol must be well outlined and adhered to, and
the consequence of noncompliance should be loss of membership.

The strength and discipline of our gymnasts shouldn't cause us to forget
that most of them are children for a majority of their careers. The
coaches, officials and other adults charged with harnessing their
talents must also stand up for their well-being.

I wish I'd had someone to stand up for me.

Advertisement

\protect\hyperlink{after-bottom}{Continue reading the main story}

\hypertarget{site-index}{%
\subsection{Site Index}\label{site-index}}

\hypertarget{site-information-navigation}{%
\subsection{Site Information
Navigation}\label{site-information-navigation}}

\begin{itemize}
\tightlist
\item
  \href{https://help.nytimes.com/hc/en-us/articles/115014792127-Copyright-notice}{©~2020~The
  New York Times Company}
\end{itemize}

\begin{itemize}
\tightlist
\item
  \href{https://www.nytco.com/}{NYTCo}
\item
  \href{https://help.nytimes.com/hc/en-us/articles/115015385887-Contact-Us}{Contact
  Us}
\item
  \href{https://www.nytco.com/careers/}{Work with us}
\item
  \href{https://nytmediakit.com/}{Advertise}
\item
  \href{http://www.tbrandstudio.com/}{T Brand Studio}
\item
  \href{https://www.nytimes.com/privacy/cookie-policy\#how-do-i-manage-trackers}{Your
  Ad Choices}
\item
  \href{https://www.nytimes.com/privacy}{Privacy}
\item
  \href{https://help.nytimes.com/hc/en-us/articles/115014893428-Terms-of-service}{Terms
  of Service}
\item
  \href{https://help.nytimes.com/hc/en-us/articles/115014893968-Terms-of-sale}{Terms
  of Sale}
\item
  \href{https://spiderbites.nytimes.com}{Site Map}
\item
  \href{https://help.nytimes.com/hc/en-us}{Help}
\item
  \href{https://www.nytimes.com/subscription?campaignId=37WXW}{Subscriptions}
\end{itemize}
