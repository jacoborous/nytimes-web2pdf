Sections

SEARCH

\protect\hyperlink{site-content}{Skip to
content}\protect\hyperlink{site-index}{Skip to site index}

\href{https://www.nytimes.com/section/world/europe}{Europe}

\href{https://myaccount.nytimes.com/auth/login?response_type=cookie\&client_id=vi}{}

\href{https://www.nytimes.com/section/todayspaper}{Today's Paper}

\href{/section/world/europe}{Europe}\textbar{}Geert Wilders Falls Short
in Election, as Wary Dutch Scatter Their Votes

\url{https://nyti.ms/2m29Khb}

\begin{itemize}
\item
\item
\item
\item
\item
\end{itemize}

Advertisement

\protect\hyperlink{after-top}{Continue reading the main story}

Supported by

\protect\hyperlink{after-sponsor}{Continue reading the main story}

\hypertarget{geert-wilders-falls-short-in-election-as-wary-dutch-scatter-their-votes}{%
\section{Geert Wilders Falls Short in Election, as Wary Dutch Scatter
Their
Votes}\label{geert-wilders-falls-short-in-election-as-wary-dutch-scatter-their-votes}}

\includegraphics{https://static01.nyt.com/images/2017/03/13/world/00wilders-video/00wilders-video-videoSixteenByNine3000.jpg}

By \href{https://www.nytimes.com/by/alissa-j-rubin}{Alissa J. Rubin}

\begin{itemize}
\item
  March 15, 2017
\item
  \begin{itemize}
  \item
  \item
  \item
  \item
  \item
  \end{itemize}
\end{itemize}

THE HAGUE --- The far-right politician
\href{https://www.nytimes.com/2017/02/27/world/europe/geert-wilders-reclusive-provocateur-rises-before-dutch-vote.html?action=click\&contentCollection=Europe\&module=RelatedCoverage\&region=Marginalia\&pgtype=article}{Geert
Wilders} fell short of expectations in Dutch elections on Wednesday,
gaining seats but failing to persuade a decisive portion of voters to
back his extreme positions on
\href{https://www.nytimes.com/2017/03/13/world/europe/netherlands-election-muslims.html?action=click\&contentCollection=Europe\&module=RelatedCoverage\&region=Marginalia\&pgtype=article}{barring
Muslim immigrants} and jettisoning the European Union, according to
official results and exit polls.

The results were immediately cheered by pro-European politicians who
hoped that they could help stall some of the momentum of the populist,
anti-European Union and anti-Muslim forces Mr. Wilders has come to
symbolize, and which have threatened to fracture the bloc.

Voters,
\href{https://www.nytimes.com/2017/03/15/world/europe/dutch-elections.html}{who
turned out in record numbers}, nonetheless rewarded right and
center-right parties that had co-opted parts of his hard-line message,
including that of the incumbent prime minister, Mark Rutte. Some parties
that challenged the establishment from the left made significant gains.

The Dutch vote was
\href{https://www.nytimes.com/2017/03/15/world/europe/dutch-vote-watched-across-europe-with-a-finger-in-the-wind.html?hp\&action=click\&pgtype=Homepage\&clickSource=story-heading\&module=second-column-region\&region=top-news\&WT.nav=top-news}{closely
watched as a harbinger} of potential trends in a year of important
European elections, including in France in just weeks, and later in
Germany and possibly Italy. Many of the Dutch parties that prevailed
favor the European Union --- a rare glimmer of hope at a time when
populist forces have created an existential crisis for the bloc and
Britain prepares for its withdrawal, or ``Brexit.''

``The Netherlands, after Brexit, after the American elections, said
`Whoa' to the wrong kind of populism,'' Mr. Rutte told a wildly
enthusiastic crowd, excited that his party, the People's Party for
Freedom and Democracy, had come in first among the parties and lost
fewer seats than it had feared.

``Today was a celebration of democracy, we saw rows of people queuing to
cast their vote, all over the Netherlands --- how long has it been since
we've seen that?'' Mr. Rutte said.

\href{https://www.nytimes.com/interactive/2016/05/22/world/europe/europe-right-wing-austria-hungary.html}{}

\includegraphics{https://static01.nyt.com/images/2016/05/22/world/europe/europe-right-wing-austria-hungary-1463897749837/europe-right-wing-austria-hungary-1463897749837-thumbLarge-v5.png}

\hypertarget{how-far-is-europe-swinging-to-the-right}{%
\subsection{How Far Is Europe Swinging to the
Right?}\label{how-far-is-europe-swinging-to-the-right}}

Right-wing parties have been achieving electoral success in a growing
number of nations.

Alexander Pechtold, the leader of Democrats 66, which appeared to have
won the most votes of any left-leaning party, struck a similar note
underscoring the vote as a victory against a populist extremist.

``During this election campaign, the whole world was watching us,'' Mr.
Pechtold said. ``They were looking at Europe to see if this continent
would follow the call of the populists, but it has now become clear that
call stopped here in the Netherlands.''

According to an unofficial tally compiled by the Dutch Broadcasting
Foundation, the country's public broadcaster, the People's Party for
Freedom and Democracy was likely to capture 33 of the 150 seats in
Parliament --- a loss of seven seats, but still far more than any other
party.

Mr. Wilders's Party for Freedom was expected to finish second, with 20
seats (an increase of eight); and the right-leaning Christian Democratic
Appeal and the left-leaning Democrats 66 were tied for third, with 19
each, the broadcaster reported.

In the Netherlands, the results betrayed a lingering distrust of turning
over the reins of power to the far right, even as its message dominated
the campaign and was likely to influence policies in the new government.

Yet there are limits to how much the Netherlands, one of Europe's most
socially liberal countries, will be a reliable predictor for Europe's
other important elections this year, including next month's presidential
elections in France.

Mark Bovens, a political scientist at Utrecht University, noted that Mr.
Wilders and other right-wing parties, despite their gains, did not
drastically cross traditional thresholds.

\includegraphics{https://static01.nyt.com/images/2017/03/16/world/16Netherlands02sub/16Netherlands02sub-articleLarge.jpg?quality=75\&auto=webp\&disable=upscale}

``The nationalist parties have won seats, compared to 2012 --- Wilders's
party has gained seats, as has a new party, the Forum for Democracy ---
but their electorate is stable, it has not grown,'' Mr. Bovens said.

Mr. Bovens pointed out that an earlier populist movement led by the
right-wing politician Pim Fortuyn had won 26 seats in 2002, and that Mr.
Wilders's won 24 seats in 2010. If Mr. Wilders's party rises to 20
seats, as the early returns seemed to indicate, it will still be lower
than the previous high-water marks.

``And some of the traditional parties have moved in a more nationalistic
direction, taking a bit of wind out of his sails,'' he said. ``You see
the same strategy in Germany.''

The German governing coalition led by Chancellor Angela Merkel, which is
facing a stiff election challenge of its own this year, was clearly
buoyed by the Dutch result, its foreign ministry sending a
\href{https://twitter.com/GermanyDiplo/status/842115682095517696}{warmly
enthusiastic message via Twitter}.

``Large majority of Dutch voters have rejected anti-European populists.
That's good news. We need you for a strong \#Europe!'' it read.

In the Netherlands's extremely fractured system of proportional
representation --- 28 parties ran and 13 are likely to have positions in
the 150-seat lower house of Parliament --- the results were, not
atypically, something of a dog's breakfast.

Mr. Rutte's party lost seats, even as it came out on top, and will need
to join forces with several others in order to wield power. Virtually
all parties said they would not work with Mr. Wilders in a coalition ---
so toxic he remains --- though his positions are likely to infuse
parliamentary debate.

Image

Supporters of the Green Party reacted in The Hague on
Wednesday.Credit...Robin Van Lonkhuijsen/Agence France-Presse --- Getty
Images

``Rutte has not seen the last of me yet!'' Mr. Wilders
\href{https://twitter.com/geertwilderspvv/status/842113131442765826}{wrote
on Twitter}, and indeed his anti-immigrant message, which dominated much
of the campaign, was not likely to go away.

It came into particularly sharp relief on the eve of the election, when
Turkey's foreign minister sought to enter the Netherlands to rally
support among Turks in Rotterdam for a referendum to increase the power
of the Turkish president, Recep Tayyip Erdogan. Dutch officials refused
him landing rights.

Mr. Wilders, who has seemed to relish being called the ``Dutch Donald
Trump,'' has been so extreme that some appear to have thought twice
about supporting him.

He has called for banning the Quran because he compares it to Hitler's
work ``Mein Kampf,'' which the Netherlands banned, and for closing
mosques and Islamic cultural centers and schools.

Election turnout was high, with polling places seeing a steady stream of
voters from early morning until the polls closed at 9 p.m. Of the 12.9
million Dutch citizens eligible to cast ballots, more than 80 percent
voted.

Some polling places ran out of ballots and called for additional ones to
be delivered. There were so many candidates listed that the ballots were
as voluminous as bath towels and had to be folded many times over to fit
into the ballot box.

The percentage of the vote that a party receives translates into the
number of seats it will get in Parliament. If a party gets 10 percent of
the total votes, it gets 10 percent of seats in the 150-seat Parliament,
given to its first 15 candidates listed on the ballot.

\href{https://www.nytimes.com/interactive/2016/05/22/world/europe/europe-right-wing-austria-hungary.html}{}

\includegraphics{https://static01.nyt.com/images/2016/05/22/world/europe/europe-right-wing-austria-hungary-1463897749837/europe-right-wing-austria-hungary-1463897749837-thumbLarge-v5.png}

\hypertarget{how-far-is-europe-swinging-to-the-right-1}{%
\subsection{How Far Is Europe Swinging to the
Right?}\label{how-far-is-europe-swinging-to-the-right-1}}

Right-wing parties have been achieving electoral success in a growing
number of nations.

The election was a success for the left-leaning Green Party, led by
30-year-old Jesse Klaver, a relative political newcomer, whose
leadership at least tripled the party's seats, making it the fifth-place
finisher and potentially a part of the government.

Mr. Klaver ran specifically on an anti-populist platform and worked hard
to turn out first-time voters.

``In these elections there was an overwhelming attention from the
foreign press, which is understandable because Brexit happened and Trump
was elected, and because France, Germany and maybe Italy will be holding
elections,'' Mr. Klaver said. ``They asked us: Will populism break
through in the Netherlands?''

The crowd shouted: ``No.''

``That is the answer that we have for the whole of Europe: Populism did
not break through,'' Mr. Klaver said.

Another striking development was the first-time election of three
members of Denk (Think), a new party formed by two former Labor Party
members of Turkish background. (The third member elected to Parliament,
Farid Azarkan, is Moroccan-Dutch.) It will be the only ethnic party in
the Dutch Parliament and is a reminder that Turks are the largest
immigrant community in the Netherlands. There are roughly 400,000
first-, second-, or third-generation Turkish immigrants in the nation.

The big loser was the center-left Labor Party, which was expected to
drop from being the second largest party in Parliament, with 38 seats
and a position as Mr. Rutte's coalition partner. The party was expected
to win only nine seats.

In past elections the impact of extremist right-leaning parties has been
largely blunted by a political system that for more than a century has
resulted in governance by coalition.

This year's election may give the Netherlands its most fragmented
government in history. Some political analysts believe it could take
weeks or months to form a government and that the governing coalition
will be fragile.

In Belgium, which has a similar political system as the Netherlands, it
famously took nearly a year and a half after inconclusive elections in
June 2010 to form a government.

Advertisement

\protect\hyperlink{after-bottom}{Continue reading the main story}

\hypertarget{site-index}{%
\subsection{Site Index}\label{site-index}}

\hypertarget{site-information-navigation}{%
\subsection{Site Information
Navigation}\label{site-information-navigation}}

\begin{itemize}
\tightlist
\item
  \href{https://help.nytimes.com/hc/en-us/articles/115014792127-Copyright-notice}{©~2020~The
  New York Times Company}
\end{itemize}

\begin{itemize}
\tightlist
\item
  \href{https://www.nytco.com/}{NYTCo}
\item
  \href{https://help.nytimes.com/hc/en-us/articles/115015385887-Contact-Us}{Contact
  Us}
\item
  \href{https://www.nytco.com/careers/}{Work with us}
\item
  \href{https://nytmediakit.com/}{Advertise}
\item
  \href{http://www.tbrandstudio.com/}{T Brand Studio}
\item
  \href{https://www.nytimes.com/privacy/cookie-policy\#how-do-i-manage-trackers}{Your
  Ad Choices}
\item
  \href{https://www.nytimes.com/privacy}{Privacy}
\item
  \href{https://help.nytimes.com/hc/en-us/articles/115014893428-Terms-of-service}{Terms
  of Service}
\item
  \href{https://help.nytimes.com/hc/en-us/articles/115014893968-Terms-of-sale}{Terms
  of Sale}
\item
  \href{https://spiderbites.nytimes.com}{Site Map}
\item
  \href{https://help.nytimes.com/hc/en-us}{Help}
\item
  \href{https://www.nytimes.com/subscription?campaignId=37WXW}{Subscriptions}
\end{itemize}
