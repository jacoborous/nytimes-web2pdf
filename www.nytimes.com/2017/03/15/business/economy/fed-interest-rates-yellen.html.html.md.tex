Sections

SEARCH

\protect\hyperlink{site-content}{Skip to
content}\protect\hyperlink{site-index}{Skip to site index}

\href{https://www.nytimes.com/section/business/economy}{Economy}

\href{https://myaccount.nytimes.com/auth/login?response_type=cookie\&client_id=vi}{}

\href{https://www.nytimes.com/section/todayspaper}{Today's Paper}

\href{/section/business/economy}{Economy}\textbar{}Fed Raises Interest
Rates for Third Time Since Financial Crisis

\url{https://nyti.ms/2mIo1g0}

\begin{itemize}
\item
\item
\item
\item
\item
\end{itemize}

Advertisement

\protect\hyperlink{after-top}{Continue reading the main story}

Supported by

\protect\hyperlink{after-sponsor}{Continue reading the main story}

\hypertarget{fed-raises-interest-rates-for-third-time-since-financial-crisis}{%
\section{Fed Raises Interest Rates for Third Time Since Financial
Crisis}\label{fed-raises-interest-rates-for-third-time-since-financial-crisis}}

\includegraphics{https://static01.nyt.com/images/2017/03/16/business/FED-RATE1/FED-RATE1-videoSixteenByNine3000-v3.jpg}

By \href{http://www.nytimes.com/by/binyamin-appelbaum}{Binyamin
Appelbaum}

\begin{itemize}
\item
  March 15, 2017
\item
  \begin{itemize}
  \item
  \item
  \item
  \item
  \item
  \end{itemize}
\end{itemize}

The Federal Reserve, which raised its benchmark rate on Wednesday for
the second time in three months, this time to a range between 0.75
percent and 1 percent, is finally moving toward the end of its
nine-year-old economic stimulus campaign, which began in the depths of
the financial crisis.

But Janet L. Yellen, the Fed's chairwoman, said at a news conference
after the decision was announced that the Fed did not share the optimism
of stock market investors and some business executives that economic
growth is gaining speed. It still plans to move slowly because the
economy continues to grow slowly. She suggested that the Fed would have
plenty of time to adjust its plans should President Trump and Congress
cut taxes or spend massively on infrastructure.

Her announcement was full of confidence. But it certainly was not
ebullient. ``The data have not notably strengthened,'' Ms. Yellen told
reporters. ``We haven't changed the outlook. We think we're moving on
the same course we've been on.''

\href{https://www.federalreserve.gov/monetarypolicy/files/monetary20170315a1.pdf}{The
Fed said that the United States economy continued to chug along},
expanding at a ``moderate pace.'' Employers are hiring, consumers are
spending and businesses --- the laggards in recent months --- are
starting to plow a little more money into their operations, too.

The Fed's sobriety did not appear to make much of an impression on
investors. The stock market's heady march that began after Mr. Trump's
election continued apace. The Standard \& Poor's 500-stock index rose
0.84 percent to close at 2,385.26 Wednesday, moving up sharply after the
announcement. Some said the Fed was still a long way from doing anything
that might hurt.

``The first four to eight rate hikes are the low-hanging fruit,'' said
Deron McCoy, the chief investment officer at SEIA, a Los Angeles firm.
``The real test will be whether the economy can withstand positive real
rates. And that still seems to be a 2019 topic.''

Some analysts said the Fed will want to see an impact from its actions.
``Policy makers hike rates to tighten financial conditions,'' said Ellen
Zentner, the chief United States economist at Morgan Stanley. ``If this
easing of financial conditions on the back of today's hike are
sustained, that would tell policy makers they need to do more.''

\href{https://www.nytimes.com/interactive/2017/03/15/business/federal-reserve-interest-rates.html}{}

\includegraphics{https://static01.nyt.com/images/2017/03/14/business/fed-rasies-rates-third-time-1489526885558/fed-rasies-rates-third-time-1489526885558-articleLarge-v2.png}

\hypertarget{why-the-fed-raised-rates}{%
\subsection{Why the Fed Raised Rates}\label{why-the-fed-raised-rates}}

The Federal Reserve raised interest rates for the third consecutive
quarter.

Ms. Zentner said she expected the Fed to raise rates again at its June
meeting. The Fed's policy-making committee next meets on May 2 and 3.

She noted that the Fed's longer-term outlook is less clear. Ms. Yellen's
term as Fed chairwoman ends in February, and Mr. Trump could then
replace her.

The Fed, charged with maximizing employment and moderating inflation, is
close to achieving both goals. The
\href{https://www.nytimes.com/2017/03/10/business/economy/february-unemployment-jobs-report.html}{unemployment
rate fell} to 4.7 percent in February, consistent with the normal churn
of people moving among jobs. And after several years of concern that
prices were not rising fast enough, inflation is reviving. The Fed's
preferred measure
\href{https://www.bea.gov/newsreleases/national/pi/2017/pdf/pi0117.pdf}{rose
1.9 percent over the 12 months ending in January}, close to its 2
percent annual target.

``The basis for today's decision is simply our assessment of the
progress of the economy,'' Ms. Yellen said at the postmeeting news
conference. ``And it's been doing nicely.''

The Fed, which had made more inflation a central objective, said on
Wednesday that it was now focused on stabilizing inflation. Ms. Yellen
took the opportunity to note that inflation may now rise a bit above 2
percent, just as it has been below 2 percent the last few years. ``It's
a reminder 2 percent is not a ceiling on inflation,'' she said. ``It's a
target.''

The Fed's increased confidence was reflected in a new round of policy
forecasts it also published Wednesday. An increased number of Fed
officials are expecting to raise rates at least twice more this year.
Only three of the 17 officials who submitted forecasts expect the
central bank to move more slowly. There was a similar coalescing around
tighter policy for the following two years, marking the first time in
recent years that the Fed's quarterly economic forecasts have shifted
toward a prediction of tighter monetary policy.

This is the third time the Fed has raised rates since the financial
crisis. The first hike came at the end of 2015 and the second almost
exactly one year later. This time the Fed waited just three months. The
benchmark rate remains below 1 percent, a very low level.

People with credit card debt are likely to see an immediate increase of
about a quarter percentage point in their interest rates. The effect on
longer-term loans is less direct, but the average rate on a 30-year
mortgage rose by half a percentage point over the last year.

The nation's largest borrower, the federal government, will also feel
the pinch of higher rates.
\href{https://www.cbo.gov/publication/51841}{The Congressional Budget
Office expects} federal interest payments, measured as a share of the
economy, to double over the next decade.

Savers are unlikely to benefit immediately. Banks tend to raise interest
rates on loans more quickly than they raise rates on deposits. Last
week,
\href{https://www.fdic.gov/regulations/resources/rates/index.html}{the
average rate on a six-month certificate of deposit was 0.14 percent}.
Last year at this time: 0.13 percent.

The Fed's move to raise rates puts it on course for a slow-motion
collision with President Trump, who has repeatedly promised to increase
economic growth through policies including cuts in taxation and
regulation and more spending on infrastructure and defense.

Fed officials have emphasized that the economy is already growing at
roughly its maximum sustainable pace; faster growth would therefore lead
to faster increases in interest rates.

Some economists and liberal activists argue that the Fed is raising
rates too quickly. Narayana Kocherlakota, an economist at the University
of Rochester and a former member of the Fed's policy-making committee,
noted that strong economic growth continued to pull people into the job
market while wage growth remained relatively weak. That suggests, he
said, that the economy has not yet returned to full employment.

``We should be seeing faster wage growth with this level of employment
growth if we were close to full employment,'' Mr. Kocherlakota said on
Twitter before the Fed's decision.

Mr. Kocherlakota's successor as president of the Federal Reserve Bank of
Minneapolis, Neel Kashkari, cast the sole vote against raising rates on
Wednesday.

The Fed's assessment of economic conditions remained quite measured. The
economy expanded by just 1.6 percent in 2016, and there is little sign
of an acceleration during the first quarter. Fed officials continue to
forecast a Goldilocks economy, with the unemployment rate remaining at
4.5 percent and inflation around 2 percent for the next three years.

Ms. Yellen played down surveys showing a sharp rise in the optimism of
consumers and business executives since the presidential election,
noting there is little evidence that such surveys predict spending
decisions.

She said that Fed officials spoke regularly to business leaders, and
that many were undoubtedly in ``a much more optimistic frame of mind.''
But she added that many of those executives have adopted a wait-and-see
attitude --- just like the Fed itself.

Advertisement

\protect\hyperlink{after-bottom}{Continue reading the main story}

\hypertarget{site-index}{%
\subsection{Site Index}\label{site-index}}

\hypertarget{site-information-navigation}{%
\subsection{Site Information
Navigation}\label{site-information-navigation}}

\begin{itemize}
\tightlist
\item
  \href{https://help.nytimes.com/hc/en-us/articles/115014792127-Copyright-notice}{©~2020~The
  New York Times Company}
\end{itemize}

\begin{itemize}
\tightlist
\item
  \href{https://www.nytco.com/}{NYTCo}
\item
  \href{https://help.nytimes.com/hc/en-us/articles/115015385887-Contact-Us}{Contact
  Us}
\item
  \href{https://www.nytco.com/careers/}{Work with us}
\item
  \href{https://nytmediakit.com/}{Advertise}
\item
  \href{http://www.tbrandstudio.com/}{T Brand Studio}
\item
  \href{https://www.nytimes.com/privacy/cookie-policy\#how-do-i-manage-trackers}{Your
  Ad Choices}
\item
  \href{https://www.nytimes.com/privacy}{Privacy}
\item
  \href{https://help.nytimes.com/hc/en-us/articles/115014893428-Terms-of-service}{Terms
  of Service}
\item
  \href{https://help.nytimes.com/hc/en-us/articles/115014893968-Terms-of-sale}{Terms
  of Sale}
\item
  \href{https://spiderbites.nytimes.com}{Site Map}
\item
  \href{https://help.nytimes.com/hc/en-us}{Help}
\item
  \href{https://www.nytimes.com/subscription?campaignId=37WXW}{Subscriptions}
\end{itemize}
