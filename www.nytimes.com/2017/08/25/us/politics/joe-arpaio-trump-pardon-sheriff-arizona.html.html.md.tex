Sections

SEARCH

\protect\hyperlink{site-content}{Skip to
content}\protect\hyperlink{site-index}{Skip to site index}

\href{https://www.nytimes.com/section/politics}{Politics}

\href{https://myaccount.nytimes.com/auth/login?response_type=cookie\&client_id=vi}{}

\href{https://www.nytimes.com/section/todayspaper}{Today's Paper}

\href{/section/politics}{Politics}\textbar{}Trump Pardons Joe Arpaio,
Who Became Face of Crackdown on Illegal Immigration

\url{https://nyti.ms/2vwEQx7}

\begin{itemize}
\item
\item
\item
\item
\item
\item
\end{itemize}

Advertisement

\protect\hyperlink{after-top}{Continue reading the main story}

Supported by

\protect\hyperlink{after-sponsor}{Continue reading the main story}

\hypertarget{trump-pardons-joe-arpaio-who-became-face-of-crackdown-on-illegal-immigration}{%
\section{Trump Pardons Joe Arpaio, Who Became Face of Crackdown on
Illegal
Immigration}\label{trump-pardons-joe-arpaio-who-became-face-of-crackdown-on-illegal-immigration}}

\includegraphics{https://static01.nyt.com/images/2017/08/26/us/00dc-arpaio-hfo/00dc-arpaio-hfo-articleInline.jpg?quality=75\&auto=webp\&disable=upscale}

By \href{https://www.nytimes.com/by/julie-hirschfeld-davis}{Julie
Hirschfeld Davis} and
\href{http://www.nytimes.com/by/maggie-haberman}{Maggie Haberman}

\begin{itemize}
\item
  Aug. 25, 2017
\item
  \begin{itemize}
  \item
  \item
  \item
  \item
  \item
  \item
  \end{itemize}
\end{itemize}

WASHINGTON --- President Trump on Friday pardoned Joe Arpaio, the former
Arizona sheriff whose aggressive efforts to hunt down and detain
undocumented immigrants made him a national symbol of the divisive
politics of immigration and earned him a
\href{https://www.nytimes.com/2017/07/31/us/sheriff-joe-arpaio-convicted-arizona.html}{criminal
contempt conviction}.

In a two-paragraph statement, the White House said that Mr. Arpaio gave
``years of admirable service to our nation'' and called him a ``worthy
candidate for a presidential pardon.''

Mr. Trump called Mr. Arpaio ``an American patriot'' in a tweet later
Friday. ``He kept Arizona safe!'' the president said.

\begin{quote}
I am pleased to inform you that I have just granted a full Pardon to 85
year old American patriot Sheriff Joe Arpaio. He kept Arizona safe!

--- Donald J. Trump (@realDonaldTrump)
\href{https://twitter.com/realDonaldTrump/status/901263061511794688?ref_src=twsrc\%5Etfw}{August
26, 2017}
\end{quote}

In his own tweets, Mr. Arpaio thanked Mr. Trump and called his
conviction ``a political witch hunt by holdovers in the Obama justice
department.'' He also pointed his supporters to a website that was
accepting donations to help him pay off his legal fees.

Mr. Trump, who made cracking down on illegal immigration a signature
campaign issue and had pressed for local officials to do more to assist
federal authorities in rounding up undocumented people, had been openly
flirting with the idea of pardoning Mr. Arpaio.

``I won't do it tonight because I don't want to cause any controversy,''
the
\href{https://www.nytimes.com/2017/08/22/us/politics/trump-rally-arizona.html}{president
said Tuesday night at a campaign-style rally} in Phoenix, after asking,
``Was Sheriff Joe convicted for doing his job?''

``I'll make a prediction: I think he's going to be just fine,'' Mr.
Trump said.

\href{https://www.nytimes.com/2017/05/23/us/joe-arpaio-arizona.html}{Mr.
Arpaio}, 85, served for 24 years as sheriff of Maricopa County --- which
includes Phoenix --- building a national reputation for
\href{https://www.nytimes.com/2017/04/04/us/arpaio-tent-city-maricopa-sheriff-penzone.html}{harsh
conditions in his county jail}, and for his campaign against
undocumented immigrants.

Mr. Arpaio had touted himself as ``America's toughest sheriff,'' making
inmates wear pink underwear and serving jail food that at least some
prisoners called inedible. He was also at the forefront of the so-called
birther movement that aimed to investigate President Barack Obama's
birth certificate.

The criminal conviction grew out of a lawsuit filed a decade ago
charging that the sheriff's office regularly violated the rights of
Latinos, stopping people based on racial profiling, detaining them based
solely on the suspicion that they were in the country illegally and
turning them over to the immigration authorities.

A federal district judge hearing the case ordered Mr. Arpaio in 2011 to
stop detaining people based solely on suspicion of their immigration
status, when there was no evidence that a state law had been broken. But
the sheriff insisted that his tactics were legal and that he would
continue employing them.

He was convicted last month of criminal contempt of court for defying
the order, a misdemeanor punishable by up to six months in jail.

The pardon was swiftly condemned on Twitter by Democrats in Congress as
``\href{https://twitter.com/repmarkpocan/status/901234877873442816}{outrageous
and completely unacceptable}'' and a
``\href{https://twitter.com/RepKClark/status/901236788735467521}{disgrace}.''

Its timing also raised eyebrows, coming on the eve of Hurricane Harvey,
a
\href{https://www.nytimes.com/2017/08/25/us/hurricane-harvey.html}{Category
4} storm, barreling down on coastal Texas. Senator Chuck Schumer,
Democrat of New York and the minority leader, accused Mr. Trump of
``using the cover of the storm'' to pardon Mr. Arpaio and to issue a
formal
\href{https://www.nytimes.com/2017/08/25/us/politics/trump-mattis-transgender-ban.html?hp\&action=click\&pgtype=Homepage\&clickSource=story-heading\&module=first-column-region\&region=top-news\&WT.nav=top-news}{ban
on transgender} people from joining the military. (The ban also gives
the secretary of defense wide latitude to decide whether currently
serving transgender troops should remain in the military.)

``The only reason to do these right now is to use the cover of Hurricane
Harvey to avoid scrutiny,'' Mr. Schumer said in a series of tweets late
Friday. ``So sad, so weak.''

Mr. Trump's supporters hailed the pardon as a sign the president was
keeping his word on his campaign pledge to crack down on illegal
immigration.

Kelli Ward, a former Arizona state senator who is challenging Senator
Jeff Flake in a Republican primary for his seat in 2018, called Mr.
Arpaio ``a patriot who did the job the Feds refused to do.'' Mr. Trump
has endorsed Ms. Ward's candidacy.

Meanwhile, Senator John McCain, also an Arizona Republican, denounced
the pardon of Mr. Arpaio.

``No one is above the law,'' he said, ``and the individuals entrusted
with the privilege of being sworn law officers should always seek to be
beyond reproach in their commitment to fairly enforcing the laws they
swore to uphold.''

The discussion about pardoning Mr. Arpaio had begun weeks ago, while Mr.
Trump's chief strategist, Stephen K. Bannon, was still in the
administration, according to two people briefed on the matter.

But the decision to make the announcement during a national news
blackout related to the impending hurricane was not accidental. Some in
the Trump administration had cautioned against it as too controversial,
and had urged waiting, if it were going to be done.

Mr. Bannon had favored the move, as had Mr. Trump's policy adviser,
Stephen Miller, a former adviser to Jeff Sessions, the attorney general
and a former senator for whom Mr. Miller served as press secretary.

Mr. Sessions and Mr. Miller share a hard-line view on curtailing
immigration levels, and Mr. Arpaio had become a national avatar for Mr.
Trump, who had a good relationship with the sheriff during the 2016
presidential campaign. Mr. Trump had once told Mr. Arpaio that he would
try to help him if he could down the road.

But that was before Mr. Trump was closing in on Hillary Clinton in the
presidential race. Still, he was fond of Mr. Arpaio, and was sold on the
pardon as a way of pleasing his political base. Additionally, Mr. Miller
fought hard for the pardon, according to a senior administration
official.

Advertisement

\protect\hyperlink{after-bottom}{Continue reading the main story}

\hypertarget{site-index}{%
\subsection{Site Index}\label{site-index}}

\hypertarget{site-information-navigation}{%
\subsection{Site Information
Navigation}\label{site-information-navigation}}

\begin{itemize}
\tightlist
\item
  \href{https://help.nytimes.com/hc/en-us/articles/115014792127-Copyright-notice}{©~2020~The
  New York Times Company}
\end{itemize}

\begin{itemize}
\tightlist
\item
  \href{https://www.nytco.com/}{NYTCo}
\item
  \href{https://help.nytimes.com/hc/en-us/articles/115015385887-Contact-Us}{Contact
  Us}
\item
  \href{https://www.nytco.com/careers/}{Work with us}
\item
  \href{https://nytmediakit.com/}{Advertise}
\item
  \href{http://www.tbrandstudio.com/}{T Brand Studio}
\item
  \href{https://www.nytimes.com/privacy/cookie-policy\#how-do-i-manage-trackers}{Your
  Ad Choices}
\item
  \href{https://www.nytimes.com/privacy}{Privacy}
\item
  \href{https://help.nytimes.com/hc/en-us/articles/115014893428-Terms-of-service}{Terms
  of Service}
\item
  \href{https://help.nytimes.com/hc/en-us/articles/115014893968-Terms-of-sale}{Terms
  of Sale}
\item
  \href{https://spiderbites.nytimes.com}{Site Map}
\item
  \href{https://help.nytimes.com/hc/en-us}{Help}
\item
  \href{https://www.nytimes.com/subscription?campaignId=37WXW}{Subscriptions}
\end{itemize}
