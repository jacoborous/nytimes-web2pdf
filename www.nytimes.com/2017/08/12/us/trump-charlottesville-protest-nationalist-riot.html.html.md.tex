Sections

SEARCH

\protect\hyperlink{site-content}{Skip to
content}\protect\hyperlink{site-index}{Skip to site index}

\href{https://www.nytimes.com/section/politics}{Politics}

\href{https://myaccount.nytimes.com/auth/login?response_type=cookie\&client_id=vi}{}

\href{https://www.nytimes.com/section/todayspaper}{Today's Paper}

\href{/section/politics}{Politics}\textbar{}Trump Is Criticized for Not
Calling Out White Supremacists

\url{https://nyti.ms/2vtmBvI}

\begin{itemize}
\item
\item
\item
\item
\item
\end{itemize}

Advertisement

\protect\hyperlink{after-top}{Continue reading the main story}

Supported by

\protect\hyperlink{after-sponsor}{Continue reading the main story}

\hypertarget{trump-is-criticized-for-not-calling-out-white-supremacists}{%
\section{Trump Is Criticized for Not Calling Out White
Supremacists}\label{trump-is-criticized-for-not-calling-out-white-supremacists}}

\includegraphics{https://static01.nyt.com/images/2017/08/14/us/13dc-trump-sub1/13dc-trump-sub1-videoSixteenByNineJumbo1600.jpg}

By \href{https://www.nytimes.com/by/glenn-thrush}{Glenn Thrush} and
\href{http://www.nytimes.com/by/maggie-haberman}{Maggie Haberman}

\begin{itemize}
\item
  Aug. 12, 2017
\item
  \begin{itemize}
  \item
  \item
  \item
  \item
  \item
  \end{itemize}
\end{itemize}

BRIDGEWATER, N.J. --- President Trump is rarely reluctant to express his
opinion, but he is often seized by caution when addressing the violence
and vitriol of white nationalists, neo-Nazis and alt-right activists,
some of whom are his supporters.

After days of genially bombastic interactions with the news media on
North Korea and the shortcomings of congressional Republicans, Mr. Trump
on Saturday condemned the bloody protests in Charlottesville, Va., in
what critics in both parties saw as muted, equivocal terms.

During a brief and uncomfortable address to reporters at his golf resort
in Bedminster, N.J., he called for an end to the violence. But he was
the only national political figure to spread blame for the ``hatred,
bigotry and violence'' that resulted in the death of one person to
``many sides.''

For the most part, Republican leaders and other allies have kept quiet
over several months about Mr. Trump's outbursts and angry Twitter posts.
But recently they have stopped averting their gazes and on Saturday a
handful criticized his reaction to Charlottesville as insufficient.

``Mr. President --- we must call evil by its name,''
\href{https://twitter.com/SenCoryGardner/status/896472477844385792}{tweeted}
Senator Cory Gardner, Republican from Colorado, who oversees the
National Republican Senatorial Committee, the campaign arm of the Senate
Republicans.

``These were white supremacists and this was domestic terrorism,'' he
added, a description several of his colleagues used.

Mike Huckabee, the former Arkansas governor and the father of the White
House press secretary, Sarah Huckabee Sanders, did not dispute Mr.
Trump's comments directly, but he called the behavior of white
nationalists in Charlottesville ``evil.''

Democrats have suggested that Mr. Trump is simply unwilling to alienate
the segment of his white electoral base that embraces bigotry. The
president has forcefully rejected any suggestion he harbors any racial
or ethnic animosities, and points to his son-in-law, Jared Kushner, an
observant Jew, and his daughter Ivanka, who converted to the faith, as
proof of his inclusiveness.

In one Twitter post on Saturday, Mr. Trump nodded to that inclusiveness.

``We must remember this truth: No matter our color, creed, religion or
political party, we are ALL AMERICANS FIRST,''
\href{https://twitter.com/realDonaldTrump/status/896481262776360960}{the
president wrote}, a statement that had echoes of his campaign slogan,
America First.

But like several other statements Mr. Trump made on Saturday, the tweet
made no mention that the violence in Charlottesville was initiated by
white supremacists brandishing anti-Semitic placards, Confederate battle
flags, torches and a few Trump campaign signs.

Mr. Trump, the product of a well-to-do, predominantly white Queens
enclave who in 1989 paid for a full-page ad in The New York Times
calling for the death penalty for
\href{https://www.nytimes.com/2016/10/18/opinion/why-trump-doubled-down-on-the-central-park-five.html}{five
black teenagers} convicted but later exonerated of raping a white woman
in Central Park, flirted with racial controversy during the 2016
campaign. He repeatedly expressed outrage that anyone could suggest he
was prejudiced.

When he retweeted white supremacists' accounts, he brushed aside
questions about them. When he was asked about the support he had been
given by David Duke, a former Ku Klux Klan leader, he chafed, insisting
he didn't know Mr. Duke.

Finally, at a news conference in South Carolina, Mr. Trump said ``I
disavow'' when pressed on Mr. Duke. He later described Mr. Duke as a
``bad person.''

When his social media director, Dan Scavino, posted an image on Mr.
Trump's Twitter feed with a Star of David near Hillary Clinton's head,
with money raining down, Mr. Trump rejected widespread criticism of the
image as anti-Semitic. And after years of
\href{https://twitter.com/i/moments/776795610817007616?lang=en}{questioning
President Barack Obama's citizenship}, he blamed others for raising the
issue in the first place.

In an interview that aired in September 2016, Mr. Trump said ``I am the
least racist person that you have ever met,'' a statement he repeated at
a White House news conference in February.

In Bedminster on Saturday, Mr. Trump said he and his team were ``closely
following the terrible events unfolding in Charlottesville, Va.,'' then
tried to portray the violence there as a chronic, bipartisan plague.
``It's been going on for a long time in our country,'' he said. ``It's
not Donald Trump, it's not Barack Obama.''

Mr. Trump did not single out the marchers, who included the white
supremacist Richard Spencer and Mr. Duke, for their ideology.

While Democrats and some Republicans faulted Mr. Trump for being too
vague, Mr. Duke was among the few Trump critics who thought the
president had gone too far.

``I would recommend you take a good look in the mirror \& remember it
was White Americans who put you in the presidency, not radical
leftists,''
\href{https://twitter.com/DrDavidDuke/status/896431991821926401}{he
wrote on Twitter}, shortly after the president spoke.

The Department of Justice announced late Saturday that it was opening a
civil-rights investigation into ``the circumstances of the deadly
vehicular incident,'' to be conducted by the F.B.I., the United States
attorney for the Western District of Virginia, and the department's
Civil Rights Division.

``The violence and deaths in Charlottesville strike at the heart of
American law and justice,'' Attorney General Jeff Sessions said in a
statement. ``When such actions arise from racial bigotry and hatred,
they betray our core values and cannot be tolerated.''

The president remained silent on the violence for most of the morning
even as House Speaker Paul D. Ryan, Mr. Trump's wife, Melania, and
dozens of other public figures condemned the march.

Mrs. Trump,
\href{https://twitter.com/FLOTUS/status/896409989568507906}{using her
official Twitter account}, wrote, ``Our country encourages freedom of
speech, but let's communicate w/o hate in our hearts. No good comes from
violence. \#Charlottesville.''

Mr. Ryan was
\href{https://twitter.com/SpeakerRyan/status/896400866361704449}{even
more explicit}. ``The views fueling the spectacle in Charlottesville are
repugnant. Let it only serve to unite Americans against this kind of
vile bigotry,'' he wrote on Twitter at noon, around the time that Gov.
Terry McAuliffe declared a state of emergency in the city.

Advertisement

\protect\hyperlink{after-bottom}{Continue reading the main story}

\hypertarget{site-index}{%
\subsection{Site Index}\label{site-index}}

\hypertarget{site-information-navigation}{%
\subsection{Site Information
Navigation}\label{site-information-navigation}}

\begin{itemize}
\tightlist
\item
  \href{https://help.nytimes.com/hc/en-us/articles/115014792127-Copyright-notice}{©~2020~The
  New York Times Company}
\end{itemize}

\begin{itemize}
\tightlist
\item
  \href{https://www.nytco.com/}{NYTCo}
\item
  \href{https://help.nytimes.com/hc/en-us/articles/115015385887-Contact-Us}{Contact
  Us}
\item
  \href{https://www.nytco.com/careers/}{Work with us}
\item
  \href{https://nytmediakit.com/}{Advertise}
\item
  \href{http://www.tbrandstudio.com/}{T Brand Studio}
\item
  \href{https://www.nytimes.com/privacy/cookie-policy\#how-do-i-manage-trackers}{Your
  Ad Choices}
\item
  \href{https://www.nytimes.com/privacy}{Privacy}
\item
  \href{https://help.nytimes.com/hc/en-us/articles/115014893428-Terms-of-service}{Terms
  of Service}
\item
  \href{https://help.nytimes.com/hc/en-us/articles/115014893968-Terms-of-sale}{Terms
  of Sale}
\item
  \href{https://spiderbites.nytimes.com}{Site Map}
\item
  \href{https://help.nytimes.com/hc/en-us}{Help}
\item
  \href{https://www.nytimes.com/subscription?campaignId=37WXW}{Subscriptions}
\end{itemize}
