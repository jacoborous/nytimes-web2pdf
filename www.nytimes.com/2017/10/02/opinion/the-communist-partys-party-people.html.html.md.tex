Sections

SEARCH

\protect\hyperlink{site-content}{Skip to
content}\protect\hyperlink{site-index}{Skip to site index}

\href{https://myaccount.nytimes.com/auth/login?response_type=cookie\&client_id=vi}{}

\href{https://www.nytimes.com/section/todayspaper}{Today's Paper}

\href{/section/opinion}{Opinion}\textbar{}The Communist Party's Party
People

\href{https://nyti.ms/2yD8eDU}{https://nyti.ms/2yD8eDU}

\begin{itemize}
\item
\item
\item
\item
\item
\item
\end{itemize}

Advertisement

\protect\hyperlink{after-top}{Continue reading the main story}

Supported by

\protect\hyperlink{after-sponsor}{Continue reading the main story}

\href{/section/opinion}{Opinion}

\href{/column/red-century}{Red Century}

\hypertarget{the-communist-partys-party-people}{%
\section{The Communist Party's Party
People}\label{the-communist-partys-party-people}}

By \href{http://www.nytimes.com/by/alessandra-stanley}{Alessandra
Stanley}

\begin{itemize}
\item
  Oct. 2, 2017
\item
  \begin{itemize}
  \item
  \item
  \item
  \item
  \item
  \item
  \end{itemize}
\end{itemize}

\includegraphics{https://static01.nyt.com/images/2017/10/02/opinion/02stanley1web/02stanley1web-articleInline.jpg?quality=75\&auto=webp\&disable=upscale}

There was no better time or place to be a Communist than in San
Francisco in the spring of 1945.

The world was poised to tip in a new direction and the start was the
creation of the United Nations in San Francisco. As the delegates began
taking their seats, the Red Army was battling to take Berlin. History
seemed to be bending toward Moscow. Yet even conservatives held out hope
that the United Nations could forge a lasting global peace. Expectations
were so high that one columnist called the conference ``the most
important human gathering since the Last Supper.''

That event would be a far more crucial turning point than any of the
participants could anticipate. The Cold War began in San Francisco as
soon as the fighting in Europe was over.

On April 25, the day the
\href{https://learning.blogs.nytimes.com/2012/04/25/april-25-1945-conference-to-form-un-meets-as-allied-forces-near-victory-over-nazis/?mcubz=0}{conference
opened}, American and Russian armies met at the River Elbe. Red Army
soldiers hoisted the Soviet flag over the Reichstag on May 2, and
Germany surrendered on May 7.

The Americans commandeered the best suites in the Fairmont Hotel, but
the St. Francis, home to the Soviet delegation, was the hottest ticket
in town. In all four hotel ballrooms, there were vodka-soaked parties
for the Russians, many of them hosted by an attractive San Francisco
heiress who so loved the Socialist cause she took the local K.G.B.
station chief as her lover.

The Russians were America's feted allies, so Vyacheslav Molotov,
Stalin's stolid deputy --- an apparatchik so bland that Lenin once
called him a ``filing clerk'' --- was the man of the hour, lionized like
a movie star and hounded for autographs. Even Hedda Hopper, the gossip
columnist who went on to browbeat blacklisted Hollywood stars, fell
under his spell, pronouncing Molotov ``charming'' and likening him in
her column to Teddy Roosevelt (probably because they both wore
pince-nez).

The conference was ``Grand Hotel'' on the bay. Everyone who mattered,
then or later, darted through the revolving doors: besides Molotov,
Winston Churchill's delegate, the future British prime minister Anthony
Eden, and President Harry Truman's secretary of state Edward R.
Stettinius, rubbed shoulders with show business royalty like Rita
Hayworth, Lana Turner, Jack Benny, Paul Robeson and Orson Welles.

The recently widowed Eleanor Roosevelt, later a key author of the United
Nations' Declaration of Human Rights, made small talk with politicians
and policy makers like Nelson Rockefeller, Adlai Stevenson, Averell
Harriman and John Foster Dulles. The philosopher Isaiah Berlin was there
and so was Carmen Miranda.

The secretary general of the conference, the man in charge of getting
all these notables position papers and hotel rooms, was a respected
veteran of Roosevelt's New Deal administrations by the name of Alger
Hiss.

Image

Delegates to the United Nations meeting in San Francisco in April,
1945.Credit...Associated Press

To provide the perspective of the ordinary fighting man, the Hearst
newspapers sent a 27-year-old naval officer recently back from the
Pacific named John F. Kennedy. Kennedy's dispatches were somewhat
cheeky, and so was he: At one formal dance, the young reporter cut in on
Anthony Eden.

As Charles Bohlen, known as ``Chip,'' who later became America's
ambassador to Moscow, put it in his memoir: ``San Francisco was so
hospitable that those attending the conference pursued recreation as
vigorously as work.''

The left in-crowd went all out. Jessica Mitford, the Communist writer
and British upper-class rebel, lived in San Francisco. A friend of
Mitford's, Claud Cockburn, was covering the event for Britain's
Communist newspaper, The Daily Worker.
\href{http://www.nytimes.com/1977/04/17/archives/memoirs-of-a-notsodutiful-daughter-they-were-an-unlikely-couple-in.html?mcubz=0\&_r=0}{Over
drinks one evening}, Mitford deeded him her one-sixth share of a
Scottish island --- a family inheritance --- as a gift to the British
Communist Party.

Another fixture of the party scene was Mitford's friend Louise Bransten,
a Bay Area hostess who spent part of her fortune on the Communist cause
and helped organize parties for the Russians at the St. Francis. Rich,
charming and divorced, Bransten was quite the catch. (The future New
York senator Jacob Javits, attending the conference as an observer, was
set up on a blind date with her.)

Two of Bransten's friends from Berkeley who would soon play a pivotal
role in her life were reunited at one of those parties:
\href{http://www.nytimes.com/1985/07/11/us/haakon-chevalier-83-author-and-translator.html?mcubz=0}{Haakon
Chevalier}, a dashing literature professor who served as an interpreter
at the conference, and George Eltenton, a British scientist working for
Shell.

Though they did not know it, the party was over. On May 12,
\href{https://www.cvce.eu/content/publication/1997/10/13/b62aff77-24ff-40af-a730-344a9b428cc8/publishable_en.pdf}{Churchill
sent Truman a telegram} about his concerns over Soviet actions: ``An
iron curtain is drawn down upon their front.'' It was his first recorded
use of the phrase he later made famous. Before the year was out, the
future director of the C.I.A., Allen Dulles, was also using it.

In the flush of victory, amid the celebrations of the birth of the
United Nations, few yet felt the chill --- but the Cold War had begun.
The F.B.I. had Bransten and her friends under surveillance.

Red baiters in Washington were needlessly paranoid for a reason: Not
every American Communist was a spy, but some were. While many innocent
people were groundlessly blacklisted and disgraced under McCarthyism, a
few who worked for the U.S.S.R. got away with it.

Louise Bransten was a little of both.

Bransten was famous for her parties and fund-raisers during the war. One
of her frequent guests was Grigori Kheifets, a vice consul at the
Russian consulate in San Francisco. Kheifets also happened to be her
lover --- and the K.G.B. station chief. Bransten helped him cultivate
Chevalier and Eltenton.

Bransten's social circle also included the physicist J. Robert
Oppenheimer, who was doing government research at the Berkeley Radiation
Laboratory. To get to Oppenheimer, Kheifets set his sights on Eltenton,
who had worked at a research institute in Russia in the 1930s and never
lost faith in the revolution, even as friends and colleagues vanished
during Stalin's purges. (His wife, Dorothea, wrote a memoir of their
stay, ``Laughter in Leningrad,'' which, for perhaps obvious reasons, was
published privately.)

In the fall of 1942, the Red Army was facing desperate odds: Leningrad
was still under siege and the grinding battle for Stalingrad had only
just begun. Leftists wanted to help the Soviets and feared that the
American government was holding back.

At least, that was the explanation Eltenton gave investigators for why
he agreed to ask Oppenheimer to give Russia atomic secrets. To do so, he
had turned to Chevalier, who shared Eltenton's political views and was
one of Oppenheimer's closest friends. Over martinis in Berkeley,
Chevalier told Oppenheimer that Eltenton had a way to slip top secret
research into Russian hands without detection.

Oppenheimer, who was soon to leave for Los Alamos, indignantly refused
to cooperate. The request was dropped. Kheifets and his confederates
moved on to other prey.

By the time Bransten, Chevalier and Eltenton were toasting the future
the St. Francis hotel in 1945, their wartime espionage effort seemed a
thing of the past. So it might have remained if Oppenheimer had not
eventually reported Chevalier's overture, albeit in hedged, conflicting
versions --- which he subsequently disavowed when interviewed by the
F.B.I. in 1946.

By then, however, the investigators' worst fears of K.G.B. infiltration
were confirmed. Starting in August and September 1945, a stream of
defectors, Russian and American, informed the F.B.I. about the moles in
Washington and the spies at Los Alamos. Oppenheimer eventually lost his
security clearance, after further investigations in 1954, because of
what became known as
``t\href{http://www.plosin.com/beatbegins/archive/Chevalier.htm}{he
Chevalier incident}.''

Chevalier lost his teaching post at Berkeley and moved to Paris, writing
books and translating works by André Malraux and Louis Aragon. In 1947,
Eltenton returned to England to work at a Shell facility there. The
F.B.I. wanted MI5 to pursue Eltenton, but back home, class snobbery
worked in his favor. The head of MI5, Sir Percy Sillitoe, responded that
their Cambridge-educated compatriot ``made a very good impression.''
Eltenton's boss at Shell dismissed the American accusations as ``stuff
and nonsense.''

Bransten was subpoenaed to appear before the House Un-American
Activities Committee in 1948, but she refused to answer questions,
citing the Fifth Amendment. Instead, she distributed a prepared a
statement that said in part: ``I believe in one world and agree with
Franklin Roosevelt that world peace must be based on cooperation between
the United States and the Soviet Union within the United Nations.''

Bransten was charged with contempt of Congress in 1949, but she got
lucky:
\href{http://www.nytimes.com/1988/04/28/obituaries/burnita-s-matthews-dies-at-93-first-woman-on-us-trial-courts.html?mcubz=0}{Judge
Burnita S. Matthews}, the first woman named to the Federal District
Court, ruled in her favor. One newspaper called Bransten the
``\href{https://www.newspapers.com/newspage/174460864/}{Red-Handed
Heiress}'' and her reputation became so radioactive that even the
liberal Republican Javits had to account to Congress for his fleeting
association with her.

In 1948, Hiss was denounced before the House Un-American Activities
Committee as a spy. After an investigation by a congressional
subcommittee into whether he'd committed perjury in denying the charges,
he was tried twice, and was eventually convicted in 1950. By the end of
1950, the chairman of the subcommittee that had undermined Hiss's
testimony was a senator; by 1953, he was vice president. His name was
Richard Nixon.

The San Francisco conference must have seemed like a glittering triumph
for American Communists. Instead, it was their last glimmer.

Advertisement

\protect\hyperlink{after-bottom}{Continue reading the main story}

\hypertarget{site-index}{%
\subsection{Site Index}\label{site-index}}

\hypertarget{site-information-navigation}{%
\subsection{Site Information
Navigation}\label{site-information-navigation}}

\begin{itemize}
\tightlist
\item
  \href{https://help.nytimes.com/hc/en-us/articles/115014792127-Copyright-notice}{©~2020~The
  New York Times Company}
\end{itemize}

\begin{itemize}
\tightlist
\item
  \href{https://www.nytco.com/}{NYTCo}
\item
  \href{https://help.nytimes.com/hc/en-us/articles/115015385887-Contact-Us}{Contact
  Us}
\item
  \href{https://www.nytco.com/careers/}{Work with us}
\item
  \href{https://nytmediakit.com/}{Advertise}
\item
  \href{http://www.tbrandstudio.com/}{T Brand Studio}
\item
  \href{https://www.nytimes.com/privacy/cookie-policy\#how-do-i-manage-trackers}{Your
  Ad Choices}
\item
  \href{https://www.nytimes.com/privacy}{Privacy}
\item
  \href{https://help.nytimes.com/hc/en-us/articles/115014893428-Terms-of-service}{Terms
  of Service}
\item
  \href{https://help.nytimes.com/hc/en-us/articles/115014893968-Terms-of-sale}{Terms
  of Sale}
\item
  \href{https://spiderbites.nytimes.com}{Site Map}
\item
  \href{https://help.nytimes.com/hc/en-us}{Help}
\item
  \href{https://www.nytimes.com/subscription?campaignId=37WXW}{Subscriptions}
\end{itemize}
