Sections

SEARCH

\protect\hyperlink{site-content}{Skip to
content}\protect\hyperlink{site-index}{Skip to site index}

\href{https://www.nytimes.com/section/politics}{Politics}

\href{https://myaccount.nytimes.com/auth/login?response_type=cookie\&client_id=vi}{}

\href{https://www.nytimes.com/section/todayspaper}{Today's Paper}

\href{/section/politics}{Politics}\textbar{}Medal of Honor Goes to
Vietnam Medic Who Ran Through `Hell on Earth'

\url{https://nyti.ms/2ueh7RD}

\begin{itemize}
\item
\item
\item
\item
\item
\end{itemize}

Advertisement

\protect\hyperlink{after-top}{Continue reading the main story}

Supported by

\protect\hyperlink{after-sponsor}{Continue reading the main story}

\hypertarget{medal-of-honor-goes-to-vietnam-medic-who-ran-through-hell-on-earth}{%
\section{Medal of Honor Goes to Vietnam Medic Who Ran Through `Hell on
Earth'}\label{medal-of-honor-goes-to-vietnam-medic-who-ran-through-hell-on-earth}}

\includegraphics{https://static01.nyt.com/images/2017/08/01/us/01dc-medal-01/merlin-to-scoop-125444462-636467-articleInline.jpg?quality=75\&auto=webp\&disable=upscale}

By \href{https://www.nytimes.com/by/nicholas-fandos}{Nicholas Fandos}

\begin{itemize}
\item
  July 31, 2017
\item
  \begin{itemize}
  \item
  \item
  \item
  \item
  \item
  \end{itemize}
\end{itemize}

WASHINGTON --- It was clear almost as soon as the May 1969 assault began
that the orders --- to attack Nui Yon Hill, near the city of Tam Kỳ,
Vietnam --- were flawed. The North Vietnamese and the Vietcong had the
American position surrounded on three sides.

Into their gunfire stepped --- or, rather, ran --- James C. McCloughan,
a 23-year-old private first class Army medic who had been drafted just a
year earlier from his hometown, South Haven, Mich., to join the
fighting.

Zigzagging through enemy bullets again and again, Mr. McCloughan is
credited with saving the lives of 10 members of his company over the
next two days of battle. When shrapnel from a rocket-propelled grenade
and later small-arms fire tore into his head and arm, he refused to
follow a superior's urging to evacuate for medical help and carried on.

``He knew me enough to know that I wasn't going,'' Mr. McCloughan told
The Associated Press recently.

On Monday, 48 years after the battle, Mr. McCloughan, 71, stood in the
East Room of the White House to be recognized for that persistence with
the country's highest military decoration, the Medal of Honor. Ten men
from his company, including five whose lives he helped save, looked on.

President Trump, in his first time awarding the decoration, said Mr.
McCloughan had made it through ``hell on earth'' and earned a ``place
among legends.''

``He would not yield. He would not rest. He would not stop. And he would
not flinch in the face of sure death and definite danger,'' Mr. Trump
said. ``Though he was thousands of miles from home, it was as if the
strength and pride of our whole nation was beating inside Jim's heart.''

For Mr. Trump, the sober ceremony turned out to be a respite from the
latest episode in the
\href{https://www.nytimes.com/2017/07/29/us/politics/trump-presidency-setbacks.html}{insistent
chaos} that has consumed his administration in recent weeks. This time,
it was
\href{https://www.nytimes.com/2017/07/31/us/politics/anthony-scaramucci-white-house.html?hp\&action=click\&pgtype=Homepage\&clickSource=story-heading\&module=first-column-region\&region=top-news\&WT.nav=top-news}{the
sacking of Anthony Scaramucci} from his position as communications
director, just days after he had been brought on, that set the White
House abuzz as 250 guests, including top military officials and members
of Mr. Trump's cabinet, mingled before the ceremony.

For Mr. McCloughan, standing nearby as Mr. Trump recounted his story,
the award had been long delayed and not exactly expected. After he
returned from Vietnam in 1970, he accepted a deferred job offer from
South Haven High School and more or less picked up civilian life where
he left off. For four decades, he taught sociology and psychology and
coached football, baseball and wrestling, before retiring in 2008.

Lt. Randall J. Clark, Mr. McCloughan's onetime platoon leader, had
inquired after the battle about awarding the medic the Distinguished
Service Cross, but he was given a bronze star for valor instead. It was
not until 2009, when Mr. McCloughan's uncle secured him a meeting with
Representative Fred Upton, Republican of Michigan, that the topic was
revisited. With the help of Mr. Clark, Mr. Upton and Michigan's
congressional delegation, the case eventually got the attention of
Ashton B. Carter, President Barack Obama's defense secretary, who
recommended last year that Mr. McCloughan be awarded the Medal of Honor.

But because the award must be given within five years of the actions it
recognizes, Congress had to vote to grant special permission. By the
time it had done so, the Obama administration ran out of time to invite
Mr. McCloughan to the White House.

Mr. McCloughan, who was known as Doc to members of his platoon, was
drafted into the Army shortly after he finished college, in 1968. By
March 1969, he was assigned to a base in South Vietnam and two months
later found himself part of a dwindling company tasked with securing a
transportation route near Tam Kỳ.

When they flew by helicopter into the area, the Americans almost
immediately came under fire. Though they did not realize it yet, more
than 2,000 North Vietnamese and Vietcong soldiers had their position
surrounded. With his company under fire and in retreat, Mr. McCloughan
hoisted his first downed soldier onto his shoulders and evacuated him to
safety.

By late afternoon, he again sprinted forward, weaponless, to pull two
more stranded men back to a trench (likely
\href{https://www.google.com/url?q=https\%3A\%2F\%2Fwww.army.mil\%2Farticle\%2F191325\%2Fcombat_medic_to_receive_medal_of_honor_for_intrepid_actions_in_vietnam\&sa=D\&sntz=1\&usg=AFQjCNHiZL3UKXF2Bg27FWeQsU99UC3fiQ}{dug
by French troops} in the 1950s). This time, he was hit. Mr. McCloughan
looked down and saw he was covered with blood. Shrapnel from a
rocket-propelled grenade had hit his head and body.

Over the two-day battle, Mr. McCloughan voluntarily risked his life
seven more times, according to an account of the battle released by the
White House, returning fire and, in one case, taking out an enemy
rocket-propelled grenade position. When a superior ordered him to get in
a medevac helicopter alongside one of the men he was treating, Mr.
McCloughan refused.

``As Jim now says,'' Mr. Trump said, ```I would have rather died on the
battlefield than know that men died because they did not have a
medic.'''

By May 15, when the fighting subsided, Mr. McCloughan had been without
food, water or sleep for two days. He collapsed from dehydration after
loading medevac helicopters.

Hours before that, as he evacuated another man badly wounded in the
stomach, Mr. McCloughan turned his thoughts to home and proposed a deal
with God, Mr. Trump said.

``He asked God,'' Mr. Trump said. ```If you get me out of this hell on
earth so I can tell my dad I love him, I will be the best coach and the
best father you could ever ask for.'''

The president said both sides had held up their end of the bargain.

Advertisement

\protect\hyperlink{after-bottom}{Continue reading the main story}

\hypertarget{site-index}{%
\subsection{Site Index}\label{site-index}}

\hypertarget{site-information-navigation}{%
\subsection{Site Information
Navigation}\label{site-information-navigation}}

\begin{itemize}
\tightlist
\item
  \href{https://help.nytimes.com/hc/en-us/articles/115014792127-Copyright-notice}{©~2020~The
  New York Times Company}
\end{itemize}

\begin{itemize}
\tightlist
\item
  \href{https://www.nytco.com/}{NYTCo}
\item
  \href{https://help.nytimes.com/hc/en-us/articles/115015385887-Contact-Us}{Contact
  Us}
\item
  \href{https://www.nytco.com/careers/}{Work with us}
\item
  \href{https://nytmediakit.com/}{Advertise}
\item
  \href{http://www.tbrandstudio.com/}{T Brand Studio}
\item
  \href{https://www.nytimes.com/privacy/cookie-policy\#how-do-i-manage-trackers}{Your
  Ad Choices}
\item
  \href{https://www.nytimes.com/privacy}{Privacy}
\item
  \href{https://help.nytimes.com/hc/en-us/articles/115014893428-Terms-of-service}{Terms
  of Service}
\item
  \href{https://help.nytimes.com/hc/en-us/articles/115014893968-Terms-of-sale}{Terms
  of Sale}
\item
  \href{https://spiderbites.nytimes.com}{Site Map}
\item
  \href{https://help.nytimes.com/hc/en-us}{Help}
\item
  \href{https://www.nytimes.com/subscription?campaignId=37WXW}{Subscriptions}
\end{itemize}
