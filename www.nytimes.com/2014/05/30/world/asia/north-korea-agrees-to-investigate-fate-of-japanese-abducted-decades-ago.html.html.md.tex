Sections

SEARCH

\protect\hyperlink{site-content}{Skip to
content}\protect\hyperlink{site-index}{Skip to site index}

\href{https://www.nytimes.com/section/world/asia}{Asia Pacific}

\href{https://myaccount.nytimes.com/auth/login?response_type=cookie\&client_id=vi}{}

\href{https://www.nytimes.com/section/todayspaper}{Today's Paper}

\href{/section/world/asia}{Asia Pacific}\textbar{}North Korea Will
Investigate Fate of Abducted Japanese

\url{https://nyti.ms/1hCAis0}

\begin{itemize}
\item
\item
\item
\item
\item
\end{itemize}

Advertisement

\protect\hyperlink{after-top}{Continue reading the main story}

Supported by

\protect\hyperlink{after-sponsor}{Continue reading the main story}

\hypertarget{north-korea-will-investigate-fate-of-abducted-japanese}{%
\section{North Korea Will Investigate Fate of Abducted
Japanese}\label{north-korea-will-investigate-fate-of-abducted-japanese}}

By \href{http://www.nytimes.com/by/martin-fackler}{Martin Fackler}

\begin{itemize}
\item
  May 29, 2014
\item
  \begin{itemize}
  \item
  \item
  \item
  \item
  \item
  \end{itemize}
\end{itemize}

SEOUL, South Korea --- North Korea has agreed to open a new
investigation into the fate of Japanese citizens abducted by its agents
during the Cold War, the two countries said Thursday, signaling a
possible diplomatic breakthrough in an emotional issue that has divided
Japan and the North.

At talks held in Stockholm, North Korean negotiators agreed to Japanese
requests to investigate what happened to more than a dozen Japanese
believed to have been kidnapped by the isolated Stalinist regime decades
ago, reversing the North's earlier insistence that the issue had been
settled.

The top Japanese government spokesman, Chief Cabinet Secretary Yoshihide
Suga, said that in return, Japan would start lifting sanctions that it
had imposed on the North over the abduction issue. Those include a ban
on travel between the two countries, on the transfer of money, and also
on visits by North Korean ships to Japanese ports, he said.

``We expect this to yield concrete results in quickly resolving problems
involving Japanese, including the return of any surviving abductees,''
Mr. Suga told reporters.

The deal could lead to a resolution of a problem that had driven Japan
to cut off virtually all ties with North Korea ever since the North
admitted in 2002 that it had kidnapped Japanese citizens, and returned
five of them alive.

The Japanese public was outraged by the revelations, and by the vague
and often puzzling accounts that the North Korean government gave of the
fate of several other abductees, who it said had died. Most of them were
snatched by North Korean agents in the 1970s and 1980s as they relaxed
on the beach or walked home from school. Their fate had been a mystery
until the North's sudden admission.

Japan has been pressing North Korea ever since to produce a fuller
account of what happened to the other abductees, amid unconfirmed
reports that some had been seen alive even after 2002 in the North, one
of the world's most closed and secretive countries. North Korean
diplomats had rejected those demands, saying it had disclosed all the
information on them that it had.

The North's willingness to reverse that stance may signal a new desire
by the dictator Kim Jong-un to open his impoverished nation ever more
slightly to the outside world, either to bolster its decrepit economy or
to reduce its dependence on China, its main trading partner.

For Japan, the possible breakthrough is a rare diplomatic success for
Prime Minister Shinzo Abe, a conservative who has presided over a
souring of ties with other neighbors, China and South Korea.

``The complete resolution of the abductee issue is one of the top
priorities of the Abe administration,'' Mr. Abe said in announcing the
deal. ``Our mission is not over until all the families of abductees can
once again hold their children in their arms.''

As part of Thursday's deal, Mr. Suga said, North Korea agreed to set up
a committee to conduct an internal investigation into what happened to
the abductees. The committee will also examine the fate of other
Japanese in the North, including those who accompanied their Korean
spouses to the country in the 1950s, and search for the remains of
Japanese who died there in the chaotic final days of World War II.

Mr. Suga also said that North Korea had agreed to return any surviving
abductees that it found. Though it is unclear if any could still be
alive after so many years, and after the North had already declared them
all to be dead, the statement reflected the hopes of Japanese families
to be reunited with their missing loved ones.

Advertisement

\protect\hyperlink{after-bottom}{Continue reading the main story}

\hypertarget{site-index}{%
\subsection{Site Index}\label{site-index}}

\hypertarget{site-information-navigation}{%
\subsection{Site Information
Navigation}\label{site-information-navigation}}

\begin{itemize}
\tightlist
\item
  \href{https://help.nytimes.com/hc/en-us/articles/115014792127-Copyright-notice}{©~2020~The
  New York Times Company}
\end{itemize}

\begin{itemize}
\tightlist
\item
  \href{https://www.nytco.com/}{NYTCo}
\item
  \href{https://help.nytimes.com/hc/en-us/articles/115015385887-Contact-Us}{Contact
  Us}
\item
  \href{https://www.nytco.com/careers/}{Work with us}
\item
  \href{https://nytmediakit.com/}{Advertise}
\item
  \href{http://www.tbrandstudio.com/}{T Brand Studio}
\item
  \href{https://www.nytimes.com/privacy/cookie-policy\#how-do-i-manage-trackers}{Your
  Ad Choices}
\item
  \href{https://www.nytimes.com/privacy}{Privacy}
\item
  \href{https://help.nytimes.com/hc/en-us/articles/115014893428-Terms-of-service}{Terms
  of Service}
\item
  \href{https://help.nytimes.com/hc/en-us/articles/115014893968-Terms-of-sale}{Terms
  of Sale}
\item
  \href{https://spiderbites.nytimes.com}{Site Map}
\item
  \href{https://help.nytimes.com/hc/en-us}{Help}
\item
  \href{https://www.nytimes.com/subscription?campaignId=37WXW}{Subscriptions}
\end{itemize}
