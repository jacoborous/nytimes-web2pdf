Sections

SEARCH

\protect\hyperlink{site-content}{Skip to
content}\protect\hyperlink{site-index}{Skip to site index}

\href{https://www.nytimes.com/section/us}{U.S.}

\href{https://myaccount.nytimes.com/auth/login?response_type=cookie\&client_id=vi}{}

\href{https://www.nytimes.com/section/todayspaper}{Today's Paper}

\href{/section/us}{U.S.}\textbar{}In Protests From Midwest to Both
Coasts, Fury Boils Over

\url{https://nyti.ms/1yaaMFm}

\begin{itemize}
\item
\item
\item
\item
\item
\end{itemize}

Advertisement

\protect\hyperlink{after-top}{Continue reading the main story}

Supported by

\protect\hyperlink{after-sponsor}{Continue reading the main story}

\hypertarget{in-protests-from-midwest-to-both-coasts-fury-boils-over}{%
\section{In Protests From Midwest to Both Coasts, Fury Boils
Over}\label{in-protests-from-midwest-to-both-coasts-fury-boils-over}}

\includegraphics{https://static01.nyt.com/images/2014/11/26/us/26ferguson-promo-still/26ferguson-promo-still-videoSixteenByNine3000.jpg}

By \href{http://www.nytimes.com/by/john-eligon}{John Eligon} and
\href{http://www.nytimes.com/by/manny-fernandez}{Manny Fernandez}

\begin{itemize}
\item
  Nov. 24, 2014
\item
  \begin{itemize}
  \item
  \item
  \item
  \item
  \item
  \end{itemize}
\end{itemize}

FERGUSON, Mo. --- Months of anger and frustration, in the end, led only
to more anger and frustration.

Shops were looted and burned on Ferguson's main street. There were smoke
bombs, tear gas, thrown rocks and random gunshots. In Ferguson, the
aftermath of the shooting death of Michael Brown was almost as bitter
and hollow as his killing itself.

Brien Redmon, 31, stood in the cold watching a burning police car and
sporadic looting after the announcement that there would be no
indictments for Mr. Brown's death at 18.

``This is not about vandalizing,'' he said. ``This is about fighting a
police organization that doesn't care about the lives they serve.''

Thomas Perry, 30, was equally bitter. ``I support my people who are out
there doing it,'' he said. ``For years they've been taking from us. We
don't care.''

The situation seemed to worsen as the night wore on, with fires and
looting mostly limited to certain areas, but seemingly on the edge of
spinning out of control. Officials said firefighters and police officers
had been shot at during the evening.

\href{https://www.nytimes.com/interactive/2014/11/25/us/evidence-released-in-michael-brown-case.html}{}

\includegraphics{https://static01.nyt.com/images/2014/11/25/us/evidence-released-in-michael-brown-case-1416904064502/evidence-released-in-michael-brown-case-1416904064502-articleLarge.jpg}

\hypertarget{documents-released-in-the-ferguson-case}{%
\subsection{Documents Released in the Ferguson
Case}\label{documents-released-in-the-ferguson-case}}

Documents and evidence presented to the grand jury that was deciding
whether to indict Officer Darren Wilson in the shooting of Michael
Brown.

Thousands of people took to the streets in cities across the country ---
from Los Angeles to Atlanta to New York --- to protest the grand jury's
decision, and in most places the demonstrations were peaceful.

In New York City, a rowdy group of hundreds of protesters made its way
up Seventh Avenue through Times Square, halting traffic as police
officers raced on foot to keep up. ``No justice, no peace,'' the group
yelled as cars honked and tourists snapped photos from the sidewalks.

``Everybody is frustrated,'' said Hugh Jackson, 28, who just moved to
New York from Atlanta and wore an American-flag-print bandanna over his
mouth as he passed Carnegie Hall. Referring to a young black man killed
a few days ago in Brooklyn, Mr. Jackson added that ``you're kind of numb
to it at a certain point. It's so systematic.''

In Philadelphia, a large but orderly crowd gathered downtown, singing,
playing drums and chanting, ``Justice for Mike Brown.''

In South Los Angeles, a crowd of protesters chanted, ``From Ferguson to
L.A., these killer cops have got to pay,'' while about half a dozen
police officers stood nearby. By 7:30 p.m., the crowd that gathered in a
South Los Angeles park had dwindled to about 70 people. Chanting had
given way to somber speeches.

``We're not here to socialize. We're here to demand justice,'' said
Melina Abdullah, a professor and chairwoman of the Pan-African studies
department at California State University, Los Angeles.

But in Ferguson, the destruction that erupted in fits and starts after
the announcement was part of a scene of seething anger, frustration and
grief that ebbed and flowed all day before the announcement and after
it.

About 200 people stood in the cold in front of the Ferguson Police
Department, listening on radios as the St. Louis County prosecuting
attorney, Robert P. McCulloch, read his statement on Monday, reality
dawning that they were not going to hear what they wanted.

During Mr. McCulloch's announcement, Mr. Brown's mother, Lesley
McSpadden, and stepfather, Louis Head, stepped up onto a platform where
protest leaders were standing.

``Defend himself from what!'' Ms. McSpadden yelled, when Mr. McCulloch
spoke of Officer Darren Wilson, the officer who shot Mr. Brown,
defending himself.

She bowed her head and tears started streaming down her cheeks.

``Everybody wants me to be calm,'' she said, her eyes covered with
sunglasses. ``You know what them bullets did to my son!'' ``They still
don't care!'' she yelled. ``They never going to care!'' Ms. McSpadden
then sank her head into her husband's chest and bounced as she wept
vigorously.

Mr. Head then turned and began to yell.

``Burn this down!'' he repeatedly shouted, inserting an expletive.

\includegraphics{https://static01.nyt.com/images/2014/11/24/multimedia/ferguson-jury-decision/ferguson-jury-decision-videoSixteenByNine1050.jpg}

The crowd then began to roar. Some rushed toward the fence near where
the police were lined up. Representatives for the family helped them
down off the platform and ushered them away, through the crowd. Officers
in riot helmets and shields came out. Soon came the smoke bombs, the
random sounds of bullets, the chaos that was almost as predictable as
the verdict everyone expected.

The scene in downtown Ferguson near the police station grew increasingly
unruly after a group of protesters tried to overturn a St. Louis County
police car that was parked just off the road. As the police arrived,
protesters threw rocks and broke the windows of at least two police
cars. The police responded with tear gas, its strong odor permeating the
frigid night air.

Nearby, the sound of glass breaking could be heard. El Palenque, a
Mexican restaurant near the Ferguson Police Department, had broken
windows. Gunshots could also be heard outside the station.

Protesters ran down South Florissant Road, out of sight of the police,
and broke windows at several businesses, including a Beauty World store
that they looted. Bursts of apparent gunfire were heard repeatedly.

The looting was a remarkable change in tone after what had been a mostly
somber response to the news that Officer Wilson would not be charged.
Officers initially stayed behind a skirmish line outside the Ferguson
police station, and many demonstrators stewed peacefully in the street
for roughly an hour.

But slowly, tension built and people began running north away from the
police. Officers did not initially pursue them, and the first widespread
looting occurred at that point.

The police eventually followed, warning over a loudspeaker that anyone
who did not disperse would be arrested. Cars sped off in all directions
as people --- peaceful protesters and looters alike --- darted through
the street.

Closing in, the authorities warned over a bullhorn that the assembly was
no longer lawful.

There were numerous stretches of this city late Monday night where all
remained calm. Stores had ``I Love Ferguson'' signs in the windows. The
red bows and holiday lights wrapped around the light poles downtown were
still perfectly intact.

But there were pockets that felt like a city under siege.

A Little Caesars Pizza shop was in flames. There were shattered windows
at a UMB Bank branch. Thick smoke poured from the busted front entrance
of a Walgreens pharmacy. Men stepped in but quickly stepped out,
complaining that it was too difficult to see anything because of the
smoke. The sound of gunfire occasionally rang out in the distance, and
the acidic smell and aftertaste of tear gas filled the air. One man
exited the Walgreens store and jokingly asked aloud if anyone wanted
cigarettes.

At the intersection of North Florissant Road and Hereford Avenue ---
``Ferguson, a city since 1894,'' reads the sign at the corner ---
firefighters worked on putting out the Little Caesars blaze, but there
were no police or fire officials at Walgreens. The fire inside continued
to burn. Spectators drove up to the store, as did news crews. All the
while, the pharmacy's high-pitched security bell echoed, the soundtrack
of the evening's drama.

``Not often you get to see anarchy, huh?'' one man taking pictures
outside Walgreens said.

The Brown family, before and after the announcement, was talking about
systemic change, not violence, but like most in the crowd, it spent the
evening battling emotions, not always winning.

As day turned to night on Monday and the prosecutor's announcement got
closer, the frustration had swelled on the streets in the heart of
Ferguson, setting the stage for an outburst that had been months in the
making.

\href{https://www.nytimes.com/interactive/2014/11/09/us/10ferguson-michael-brown-shooting-grand-jury-darren-wilson.html}{}

\includegraphics{https://static01.nyt.com/images/2014/08/12/us/JP-STLOUIS3/JP-STLOUIS3-videoLarge-v2.jpg}

\hypertarget{tracking-the-events-in-the-wake-of-michael-browns-shooting}{%
\subsection{Tracking the Events in the Wake of Michael Brown's
Shooting}\label{tracking-the-events-in-the-wake-of-michael-browns-shooting}}

Updates on the events in Ferguson, Mo., following the shooting of
Michael Brown, an unarmed teenager, by a police officer on Aug. 9.

Many demonstrators, long resigned to the notion that they would not get
the outcome they wanted, seemed to respond spontaneously, from raw
emotion. How they would express their outrage, and how far law
enforcement would allow them to go, came with no easy answers. What was
certain was that people felt they were part of something larger.

``I only saw this stuff in school,'' said Courtney Ford, 30, an educator
who is black and who lives in St. Louis. He left work to observe the
protesters holding court across the street from the Ferguson Police
Department.

``The Selma marches, and Martin Luther King, and the civil rights
activists,'' he continued. ``But now, this is life. This is history. I'm
just out here really as a witness.''

By about 5 p.m., hours before the announcement of the grand jury's
decision, crowds started gathering across from the station, their energy
rising as the night went on.

A small crowd of roughly 50 protesters chanted, sang and shouted, all
while watched by dozens of reporters, photographers and other
spectators. The protesters were a postmodern, post-Ferguson amalgam of
political views, ages, ethnicities. A few men in black-and-white clergy
collars stood next to a circle of young black men and women who rapped
their anti-police slogans.

``No justice,'' a man with a bullhorn yelled.

``No peace,'' the crowd replied.

``No racist ---,'' he continued.

``Police,'' they answered.

A middle-aged white woman wove through the crowd, yelling, ``We need to
shut this down across America!'' and handing out fliers.

The woman, Jessie Davis, was a supporter of the Revolutionary Communist
Party and came here from Chicago.

Someone held a protest sign made out of a Cheerios box. The pungent
scent of marijuana filled the air at one point. A few parents brought
their children. One protester hurried through the crowd, saying into a
walkie-talkie, ``What's your location?'' A young African-American man
--- in one of the ghoulish Guy Fawkes masks that have become a staple of
the protests --- was approached and asked what he hoped to accomplish.
He said nothing, shrugged and walked away.

Shortly after 5:30 p.m., a few protesters walked across the street and
began shaking the metal barricades that blocked the entrance to the
police station's parking lot. One of them wore a Guy Fawkes mask.
Another wore a bandanna across his face. They pushed and pulled the
barricades until several sections came unlocked. One of the protesters
stretched his arms in the victory pose: The barricade was down, and the
protesters across the street cheered.

Advertisement

\protect\hyperlink{after-bottom}{Continue reading the main story}

\hypertarget{site-index}{%
\subsection{Site Index}\label{site-index}}

\hypertarget{site-information-navigation}{%
\subsection{Site Information
Navigation}\label{site-information-navigation}}

\begin{itemize}
\tightlist
\item
  \href{https://help.nytimes.com/hc/en-us/articles/115014792127-Copyright-notice}{©~2020~The
  New York Times Company}
\end{itemize}

\begin{itemize}
\tightlist
\item
  \href{https://www.nytco.com/}{NYTCo}
\item
  \href{https://help.nytimes.com/hc/en-us/articles/115015385887-Contact-Us}{Contact
  Us}
\item
  \href{https://www.nytco.com/careers/}{Work with us}
\item
  \href{https://nytmediakit.com/}{Advertise}
\item
  \href{http://www.tbrandstudio.com/}{T Brand Studio}
\item
  \href{https://www.nytimes.com/privacy/cookie-policy\#how-do-i-manage-trackers}{Your
  Ad Choices}
\item
  \href{https://www.nytimes.com/privacy}{Privacy}
\item
  \href{https://help.nytimes.com/hc/en-us/articles/115014893428-Terms-of-service}{Terms
  of Service}
\item
  \href{https://help.nytimes.com/hc/en-us/articles/115014893968-Terms-of-sale}{Terms
  of Sale}
\item
  \href{https://spiderbites.nytimes.com}{Site Map}
\item
  \href{https://help.nytimes.com/hc/en-us}{Help}
\item
  \href{https://www.nytimes.com/subscription?campaignId=37WXW}{Subscriptions}
\end{itemize}
