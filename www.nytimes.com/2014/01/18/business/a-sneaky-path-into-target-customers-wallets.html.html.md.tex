Sections

SEARCH

\protect\hyperlink{site-content}{Skip to
content}\protect\hyperlink{site-index}{Skip to site index}

\href{https://www.nytimes.com/section/business}{Business}

\href{https://myaccount.nytimes.com/auth/login?response_type=cookie\&client_id=vi}{}

\href{https://www.nytimes.com/section/todayspaper}{Today's Paper}

\href{/section/business}{Business}\textbar{}A Sneaky Path Into Target
Customers' Wallets

\url{https://nyti.ms/1dEqI8E}

\begin{itemize}
\item
\item
\item
\item
\item
\item
\end{itemize}

Advertisement

\protect\hyperlink{after-top}{Continue reading the main story}

Supported by

\protect\hyperlink{after-sponsor}{Continue reading the main story}

\hypertarget{a-sneaky-path-into-target-customers-wallets}{%
\section{A Sneaky Path Into Target Customers'
Wallets}\label{a-sneaky-path-into-target-customers-wallets}}

\includegraphics{https://static01.nyt.com/images/2014/01/18/business/18target_375/18target_375-articleLarge.jpg?quality=75\&auto=webp\&disable=upscale}

By \href{http://www.nytimes.com/by/elizabeth-a-harris}{Elizabeth A.
Harris}, \href{http://www.nytimes.com/by/nicole-perlroth}{Nicole
Perlroth}, \href{http://www.nytimes.com/by/nathaniel-popper}{Nathaniel
Popper} and Hilary Stout

\begin{itemize}
\item
  Jan. 17, 2014
\item
  \begin{itemize}
  \item
  \item
  \item
  \item
  \item
  \item
  \end{itemize}
\end{itemize}

It was, in essence, a cybercriminal's dream.

For months, an amorphous group of Eastern European hackers had been
poking around the networks of major American retailers, searching for
loose portals that would take them deep into corporate systems.

In early November, before the holiday shopping season began, the hackers
found what they had been looking for --- a wide path into Target and
beyond.

Entering through a digital gateway, the criminals discovered that
Target's systems were astonishingly open --- lacking the virtual walls
and motion detectors found in secure networks like many banks'. Without
those safeguards, the thieves moved swiftly into the company's computer
servers containing Target's customer data and to the crown jewel: the
in-store systems where consumers swipe their credit and debit cards and
enter their PINs.

For weeks, the invasion went undetected; the malware installed by
hackers escaped whatever antivirus protections Target had. Shoppers
flooded Target stores over Thanksgiving weekend and into the following
weeks of holiday deals, unwittingly sending millions of bits of their
data into the corners of cyberspace controlled by a band of
sophisticated thieves.

Target had no clue until the Secret Service alerted the company about
two weeks before Christmas. Investigators who had been tracking these
criminals overseas and monitoring suspicious credit activity spotted in
December one common thread: charges and payments made at Target.

At least one major bank noticed a similar pattern. On Dec. 12, JPMorgan
Chase alerted some credit card companies that fraudulent charges were
showing up on cards used at Target, people involved in the conversation
said.

An examination by The New York Times into the enormous data theft,
including interviews with people knowledgeable about the investigation,
cybersecurity and credit experts and consumers shows that Target's
system was particularly vulnerable to attack. It was remarkably open,
experts say, which enabled hackers to wander from system to system,
scooping up batches of information.

Investigators have been piecing together the timetable of the attack and
continue to monitor the potential for additional fraud, especially since
experts say that batches of stolen credit card data have yet to be
dumped on the black market. The theft involved confidential credit and
debit card data of as many as 40 million Target customers, and personal
information, such as phone numbers and addresses, of as many as 70
million more.

With Secret Service agents in Minneapolis investigating the extent of
the fraud, Javelin Strategy \& Research, a consulting firm, estimates
the total damage to banks and retailers could exceed \$18 billion.
Consumers could be liable for more than \$4 billion in uncovered losses
and other costs. Investigators also say they believe that the invasive
hack at Target was part of a broader campaign aimed at least half a
dozen major retailers. So far, one other retailer, Neiman Marcus, has
said that its system was breached at the in-store level, not through
online shopping, and people with knowledge of the investigations have
been reluctant to discuss whether the two are related.

Image

Mandiant's founder and chief executive, Kevin Mandia.Credit...Jacquelyn
Martin/Associated Press

Investigators have seen some malicious software similar to that
installed at Target in recent years, but they described the design of
this malware on point-of-sale systems as particularly wily. The coding
was written in a way that was adaptive and persistent.

\textbf{Grabbing Data}

Once installed, the hackers' malware snatched customers' data ---
directly off the card's magnetic strips of credit and debit cards ---
that is normally sent for processing to banks and credit card companies.
The stolen data was then lifted and stored on an infected server inside
Target, awaiting an order from the criminals. The coding was easily
manipulated so that it could receive instructions from its handlers in
real-time, changing at their command.

Four miles from Target's headquarters in Minneapolis and more than a
week before the public learned of the data breach, Patrycia Miller
looked at the bill for the American Express account she and her husband
used in their dog day care business.

The usual charges appeared, including some from Target, where they shop
a couple of times a week. But a few stood out --- a membership fee to
Match.com and a \$1,291.58 plane ticket on South African Airways from
Lagos, Nigeria, to Johannesburg and Nairobi, Kenya.

She asked her husband what he was up to.

Puzzled, Mr. Miller assured her he had not signed up for an online
dating service and had not booked an African flight --- ``Not for that
price,'' he said.

American Express swiftly credited their account and issued new cards.

But it wasn't until Target confirmed the breach on Dec. 19 that the
Millers learned what had happened.

Gregg Steinhafel, Target's chief executive, declined to be interviewed
for this article, and requests for interviews with other company
officials involved in the theft investigation were denied. On Friday
evening, Mr. Steinhafel released a statement, saying: ``When the breach
was confirmed, I was devastated. I resolved in that moment to get to the
bottom of it, and my top priority since then has been our guests. We've
worked for 51 years to build a real relationship with them, and I am
determined to do whatever it takes to secure their trust.''

Mr. Steinhafel said in an interview with CNBC earlier this week that he
first learned of the data break-in when he received a phone call at home
on Dec. 15, a Sunday morning, as he was drinking coffee with his wife.
Secret Service and Justice Department officials had already met with
Target employees a few days earlier to notify them of their suspicions.

By then, credit and debit cards were showing up on the black market, and
shoppers like the Millers were seeing unauthorized charges on their
bills.

Image

A screen indicates that a buyer is entering a PIN.Credit...Joe
Raedle/Getty Images

It was not the first time criminals had managed to get inside a store's
point-of-sale systems at their registers. Nearly a decade ago, Albert
Gonzalez, one of the most prolific cybercriminals in American history,
was stealing credit card data from T. J. Maxx and Marshalls clothing
chains in much the same way.

But recently, criminals' techniques have evolved. At the Federal Bureau
of Investigation, a former official said there had been instances where
criminals had managed to physically implant malicious code into
point-of-sale systems on the factory floor. In most cases, however,
criminals installed the malware remotely after breaking into an
organization through other means.

This time, the code the criminals instructed Target's registers to send
customer data back to the infected Target server once every hour, on the
hour, and to cover its own tracks. After siphoning the data back to the
infected server, the malicious code immediately deleted the file where
it had been stored, so there was no memory of it, according to iSight
Partners, a security firm currently working with the Secret Service to
investigate the attacks.

The malware, known as a memory scraper, has been coined ``Kaptoxa''
after a word in its code --- Kaptoxa is Russian slang for ``potato'' and
is often used by underground criminals to refer to credit cards. Its
developers ensured the code would evade regular antivirus products ---
even a month after Target's breach was made public most antivirus
products still fail to catch it. To avoid setting off any alarms, the
criminals waited six days after moving the data from the infected server
to a web server that was itself infected with malware, and from there to
a server in Russia that served as a proxy to mask the criminals' true
whereabouts, according to Aviv Raff, the chief technology officer at
Seculert, a security company headquartered in Israel that has been
investigating the malware used on Target's systems.

Within two weeks, criminals had taken 11 gigabytes worth of Target's
customer data: less than the amount of memory on Apple's iPad Mini, but
enough to contain 40 million payment card records, encrypted PINs and 70
million records containing Target customers' information.

Shortly after, company executives flocked to headquarters and onto
conference call lines to begin coordinating the response.

\textbf{The Search Begins}

Forensics experts were brought in from Verizon, led by Bryan Sartin, and
from Mandiant, a computer security firm that responds to breaches,
extortion attacks and economic espionage campaigns. (Mandiant has since
announced it is being bought by FireEye.) They began digging through
Target's firewall logs, web traffic logs and emails, looking for digital
fingerprints and trying to determine how the criminals got in, what they
took, and how to stop the bleeding.

Investigators went about plugging Target's security holes, wiping
malware from the company's point-of-sales systems and changing
passwords. It was important to do everything at once.

It is a process that Kevin Mandia, the founder of Mandiant, has
described as akin to excising a malignancy: ``If you only remove the
cancer in your leg, but you have it in your arm, you might as well have
not had the operation in your leg,'' he said in an interview before the
Target breach.

Likewise, if Target missed one back door or one compromised password,
the criminals could come right back in.

Others in the company started planning just how, and when, to disclose
the news to the public. Then, they set about trying to determine the
impact of the breach, so they could notify affected customers, determine
liability and get ahead of the news cycle.

They wouldn't get so lucky.

On the morning of Dec. 18, voice messages started popping up on Target's
public affairs line from Brian Krebs, a prominent security blogger. Mr.
Krebs, 41, who specializes in cybercrime, was asking about a big data
breach.

In underground criminal forums, criminals had been bragging that they
had obtained a huge, very fresh batch of cards. And banks were dealing
with a spike of fraudulent purchases.

Mr. Krebs said in an interview that one contact at a large bank he would
not name said he had visited one of the more reliable underground credit
card sites --- a site called Rescator --- and bought a large batch of
cards.

The common point of purchase was Target, and all the purchases had been
made between Thanksgiving and mid-December. After further investigation,
Mr. Krebs began leaving messages with the company for comment.

Officials say the company's plan was always to go public quickly. By the
time Mr. Krebs's story was posted, a news release had already been
written and the portion of Target's website devoted to the breach was
already being built. The company decided not to immediately make a
public comment or issue a news release. Instead, they waited until the
website was ready and everyone who would be answering questions, either
at call centers or for the media, would have the same answers on hand. A
team of people worked all night to have the response ready.

On Dec. 19, the team on the front lines of the response arrived at
headquarters before the local Starbucks had opened. Before the sun was
up, the release was sent out.

\emph{A Deluge of Anger}

Customers jammed the company's website and phone lines and continue to
be angered by the violation of their privacy. On Target's Facebook page,
shoppers keep leaving furious messages.

Image

Target's president and chief, Gregg Steinhafel.Credit...Keith
Bedford/Reuters

``I am broke because someone used all my money to go on their shopping
spree,'' Shannon Smith wrote. Another customer, Melissa Milligan Gunter,
wrote: ``Dear Target, thanks for making me (and so many others) have to
go through and change everything that I use my debit and credit cards
for because you can't keep your customer's information private.''

Nearly 70 lawsuits have already been filed against Target, many of them
seeking class-action status. Credit card companies and banks have
replaced many customers' cards and accounts in the wake of the breach,
but warn that people should still vigilantly scrutinize their statements
and account charges.

In Minneapolis, hundreds of Target employees --- from the legal,
technology, finance and consumer and public relations departments ---
continue to be involved in the company's response, working out of the
32nd floor of the corporate headquarters. Earlier this month, when a
polar vortex plunged the city into temperatures below zero for several
days, the company suspended its dress code, and senior executives
gathered around the boardroom table to address the crisis in the
sweatshirts of their college alma maters.

Down the hall, packs of other employees colonized nearby rooms,
rearranging movable desks and rolling chairs. Several television screens
played multiple news networks. Surfaces were littered with extension
cords, chargers, newspapers, cups of coffee and soda.

Outside the corporation, attorneys general in several states are also
investigating Target's data breach, along with federal authorities who
would not comment publicly on the status of the investigation.

But it appears that the hackers left a few clues behind that may aid
investigators. One was a small word embedded in the code: Rescator.
Despite the sophistication of the malware, this was, by several
accounts, a rookie mistake. The name was left there when the criminals
were debugging their code.

It was the same name of the underground carding site, Rescator.la, where
a bank official had first purchased a large number of cards before
tipping off Mr. Krebs, he said.

Mr. Krebs scoured the Web for clues to Rescator's identity. In a deleted
comment from August 2011, he noted that Rescator introduced himself as
``Hel,'' one of the three founders of a defunct hacker forum called
darklife.ws. Mr. Krebs posted some of the information he learned about
aliases that may be related to Rescator, tracing one of them to Odessa,
Ukraine.

But investigators have not publicly pinpointed the location of the
criminals' nerve center, suggesting instead that the hackers tend to
move around, gather, disband and regroup.

But they are monitoring the shadowy chat forums and other netherworlds
where snippets of information about fake credit cards surfaces and is
shared for sale on the black market, where the stolen data promises rich
returns.

``We're expecting this to be a major contributor, if not the primary
driver of card fraud for the next 12 months,'' said Alphonse R. Pascual,
of Javelin Strategy \& Research. ``Those cards will continue to have
value for quite a while. These cards will still be available for
purchase a year from now.''

Advertisement

\protect\hyperlink{after-bottom}{Continue reading the main story}

\hypertarget{site-index}{%
\subsection{Site Index}\label{site-index}}

\hypertarget{site-information-navigation}{%
\subsection{Site Information
Navigation}\label{site-information-navigation}}

\begin{itemize}
\tightlist
\item
  \href{https://help.nytimes.com/hc/en-us/articles/115014792127-Copyright-notice}{©~2020~The
  New York Times Company}
\end{itemize}

\begin{itemize}
\tightlist
\item
  \href{https://www.nytco.com/}{NYTCo}
\item
  \href{https://help.nytimes.com/hc/en-us/articles/115015385887-Contact-Us}{Contact
  Us}
\item
  \href{https://www.nytco.com/careers/}{Work with us}
\item
  \href{https://nytmediakit.com/}{Advertise}
\item
  \href{http://www.tbrandstudio.com/}{T Brand Studio}
\item
  \href{https://www.nytimes.com/privacy/cookie-policy\#how-do-i-manage-trackers}{Your
  Ad Choices}
\item
  \href{https://www.nytimes.com/privacy}{Privacy}
\item
  \href{https://help.nytimes.com/hc/en-us/articles/115014893428-Terms-of-service}{Terms
  of Service}
\item
  \href{https://help.nytimes.com/hc/en-us/articles/115014893968-Terms-of-sale}{Terms
  of Sale}
\item
  \href{https://spiderbites.nytimes.com}{Site Map}
\item
  \href{https://help.nytimes.com/hc/en-us}{Help}
\item
  \href{https://www.nytimes.com/subscription?campaignId=37WXW}{Subscriptions}
\end{itemize}
