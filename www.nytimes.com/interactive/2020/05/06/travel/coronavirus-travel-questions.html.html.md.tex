Sections

SEARCH

\protect\hyperlink{site-content}{Skip to
content}\protect\hyperlink{site-index}{Skip to site index}

\hypertarget{comments}{%
\subsection{\texorpdfstring{\protect\hyperlink{commentsContainer}{Comments}}{Comments}}\label{comments}}

\href{}{The Future of Travel}\href{}{Skip to Comments}

The comments section is closed. To submit a letter to the editor for
publication, write to
\href{mailto:letters@nytimes.com}{\nolinkurl{letters@nytimes.com}}.

\hypertarget{the-future-of-travel}{%
\section{The Future of Travel}\label{the-future-of-travel}}

By Elaine Glusac,
\href{https://www.nytimes.com/by/tariro-mzezewa}{Tariro Mzezewa} and
Sarah FirsheinMay 6, 2020

\begin{itemize}
\item
\item
\item
\item
\item
  \emph{140}
\end{itemize}

Perhaps no industry has been as hard hit by the pandemic as tourism. As
restrictions on companies and travelers ease, what will the new world
look like?

\hypertarget{the-future-of-travel-1}{%
\section{The Future of Travel}\label{the-future-of-travel-1}}

How the industry will change after the pandemic.

By every measure, the coronavirus pandemic has decimated the travel
industry.

The images of the world's shutdown are
\href{https://www.nytimes.com/2020/02/27/world/europe/milan-coronavirus.html}{eerie},
the numbers are staggering. Approximately 100 million travel sector
jobs, according to
\href{https://wttc.org/News-Article/WTTC-now-estimates-over-100-million-jobs-losses-in-the-Travel-\&-Tourism-sector-and-alerts-G20-countries-to-the-scale-of-the-crisis}{one
global estimate}, have been eliminated or will be. Passenger traffic on
U.S. airlines is
\href{https://www.nytimes.com/2020/04/14/business/coronavirus-airlines-bailout-treasury-department.html}{down
95 percent} compared to last year, while international passenger
revenues are expected to decrease by more than \$300 billion. Domestic
hotel occupancy rates fell off a cliff and now
\href{https://str.com/press-release/str-us-hotel-results-week-ending-25-april}{hover
around 25 percent}.

Regions and countries are
\href{https://www.nytimes.com/2020/05/04/world/coronavirus-news.html\#link-383b3a65}{beginning
to open up}, but the outbreak will undoubtedly change how we think, act
and travel, at least in the short term.

``The pandemic is going to fade slowly, with aftereffects, a lot of
which will be psychological,'' said Frank Farley, a Temple University
psychology professor and the former president of the American
Psychological Association. ``There's so much uncertainty the average
folk might want to know everything about travel,'' he said. ``What's the
escape hatch? What are the safety issues?''

Yet the desire to travel will not go away:
\href{https://skift.com/2020/04/21/a-third-of-americans-want-to-travel-again-shortly-after-pandemic-is-contained-skift-researchs-latest-travel-tracker/}{In
a recent survey by Skift Research}, the research arm of the travel trade
publication, one-third of Americans said they hope to travel within
three months after restrictions are lifted.

To learn how the landscape might change, we talked to dozens of experts,
from academics to tour operators to airport architects. Across the
board, they highlighted issues of privacy and cleanliness and the
push-pull of people wanting to see the world while also wanting to stay
safe. Here, answers to 14 of the most pressing questions about travel's
future.

\emph{Expand All} \emph{Close All}

\textbf{Airlines}

\hypertarget{can-airlines-keep-people-apart-and-make-a-profit}{%
\subsection{Can airlines keep people apart and make a
profit?}\label{can-airlines-keep-people-apart-and-make-a-profit}}

+

-

Solving social distancing in airplanes --- currently attempted by
leaving middle seats open --- and returning to profitability seem at
odds without a medical solution to Covid-19. Nonetheless, expect
airlines to dangle cheap fares to get people in the air.

``There will be smoking-hot deals,'' said R.W. Mann, an industry analyst
and consultant. ``It will happen on the leisure side first, but on the
corporate side is where airlines make their money. They travel more
frequently and pay higher fares. Right now, they are very risk averse.''

Some of those executives appear to be turning to private charters, even
as public carriers mandate that passengers wear face masks and increase
the frequency and effectiveness of their cleaning protocols, including
filling the aircraft with a germ-killing fog before cleaning crews wipe
down surfaces. As it has long done with international flights,
\href{https://news.delta.com/coronavirus-update-aircraft-fogging-enhances-customer-safety}{Delta
Air Lines} is doing it nightly on domestic aircraft.

The CARES Act is helping prop up the airline industry with payroll
support until October, when many airline executives are openly
discussing layoffs. The government requirement that airlines continue
servicing airports they did before March 1 has kept some of the smaller
destinations on the route map, though service is often rare.

``Small airports are biting their nails right now,'' said Joe
Schwieterman, a transportation expert and professor at DePaul University
in Chicago.

Testing would go a long way in reassuring the public, of course, but so
far only one airline, Dubai-based Emirates, has offered virus tests to a
limited number of passengers. Groups including pilots unions have called
for temperature checks. The Transportation Security Administration
hasn't moved on the idea, but Air Canada plans to begin taking
temperature readings at check-in this month, and Paine Field just north
of Seattle recently installed a thermal camera that reads passengers'
temperatures before they enter security.

Passengers, beware: Low fares won't last. Assuming the virus puzzle is
solved, many expect a robust recovery in 2022.

--- ELAINE GLUSAC

\textbf{Airports}

\hypertarget{could-check-in-actually-get-better}{%
\subsection{Could check-in actually get
better?}\label{could-check-in-actually-get-better}}

+

-

Health screening, space-per-passenger ratios and a redesign of passenger
flow are likely to change in the wake of Covid-19.

Puerto Rico's Luis Muñoz Marín International Airport provides a window
into the future of airport screenings. Its new thermal-imaging cameras
screen arriving passengers, triggering an alarm when a temperature of
100.3 or higher is registered. Feverish passengers are taken aside for
evaluation.

After 9/11, many domestic airports adopted a militaristic appearance
with barriers and beefed-up security checkpoints. But abroad, airports
like
\href{https://www.nytimes.com/2019/12/02/travel/Singapore-Changi-Airport.html}{Singapore's
Changi} expanded to engage fliers who were required to spend more time
there.

``Brand-new airports will be akin to that model,'' said Ty Osbaugh, an
architect at the global firm \href{https://www.gensler.com/}{Gensler},
which has built terminals at New York City's Kennedy International
Airport and South Korea's Incheon, among others. ``Space gets you the
ability to deal with a pandemic in a new way. You're not jammed into a
facility.''

Airports hemmed in by roadways may expand vertically, he said. With the
arrival of autonomous vehicles, parking garages may be repurposed as
check-in and screening centers ``in order to use every empty space.''

Space will be vital to ensure passengers aren't in crowded security
lines. Cellphone location data may cue your arrival to an airport, which
can then check you in curbside and move you on to a security tunnel in
which passengers continue moving --- sci-fi style --- as they are
screened by T.S.A. and health authorities.

Gate space will be expanded and robots may load carry-ons, discouraging
jockeying for overhead bin space.

``The 9/11 response was very, very un-passenger-focused,'' Mr. Osbaugh
said. ``This time,'' he said, ``I think we can make a much better
passenger experience from curb to gate.''

--- ELAINE GLUSAC

\textbf{Cruises}

\hypertarget{will-people-get-back-on-the-boats}{%
\subsection{Will people get back on the
boats?}\label{will-people-get-back-on-the-boats}}

+

-

Ships turned away from port after port, passengers quarantined in
cabins, emergency workers in hazmat suits: Few travel sectors have taken
a harder hit than cruises, now largely halted per
\href{https://www.federalregister.gov/documents/2020/04/15/2020-07930/no-sail-order-and-suspension-of-further-embarkation-notice-of-modification-and-extension-and-other}{no-sail
orders} issued by the Centers for Disease Control and Prevention.
Carnival Corporation, the biggest company, has said it could resume
sailings Aug. 1.

Analysts believe large companies like Carnival and Royal Caribbean
Cruises have the financial endurance to wait out a recovery until 2021.
But discounted fares and flexible cancellation policies will only go so
far to reassure future passengers.

``The real challenge will be reducing perceived risk of actually getting
on a ship, and this will require changes in operational practices,''
Robert Kwortnik, an associate professor in the hotel school at Cornell
University, wrote in an email.

Among new practices, he listed passenger health screenings and
contingency plans for when infection occurs. In its order, the C.D.C.
directed the industry to take more responsibility for managing outbreaks
onboard, including plans for laboratory testing of samples, disinfection
protocols and providing personal protective equipment.

Genting Cruise Lines, the Hong Kong-based company that owns Crystal
Cruises and several other lines, has already issued
\href{http://gentingcruiselines.com/media/1267/20200408-genting-cruise-lines-announces-enhanced-preventive-measures-setting-new-standards-for-the-fleet-and-the-cruise-industry.pdf}{new
standards}, including banning self-service buffets, requiring
temperature checks at embarkation and disembarkation, twice-daily
temperature checks for crew members, and masks for housekeepers and food
servers. It will also require a doctor's note for passengers 70 and
over, indicating they are fit to travel.

Ships, too, may be deployed differently, said Ross Klein, a sociologist
at Memorial University in Newfoundland and a cruise-industry expert who,
since 2002, has run
\href{http://www.cruisejunkie.com/}{CruiseJunkie.com}. He foresees ships
being stationed at islands in the Caribbean, rather than traveling port
to port. ``If there's illness on board, you can walk off and fly home,''
he said of this hotel-like model. ``While at sea, you're captive.''

--- ELAINE GLUSAC

\textbf{destinations}

\hypertarget{where-will-travelers-go-first}{%
\subsection{Where will travelers go
first?}\label{where-will-travelers-go-first}}

+

-

With some states loosening travel restrictions, ``people want to get out
--- just within their own control,'' said Tori Barnes, the executive
vice president for public affairs and policy for the U.S. Travel
Association.

So expect a boom in road trips --- and travel companies and agencies
trying to benefit.

Visit California, the state's tourism office, is planning an in-state
campaign encouraging Californians to get in their cars and support local
businesses and destinations. South Dakota expects to see road trippers
looking for wide-open outdoor spaces, and hopes state parks, restaurants
and R.V. companies will reap the rewards.

``Road trips are a huge opportunity for California to help jump start
the economy,'' said Caroline Beteta, Visit California's president and
chief executive. ``That sense of freedom with personal controls will be
ideal for people who want security.''

International travel will take much longer to bounce back.

Countries will reopen at various times --- some have already begun to do
so --- and that staggered reopening might be confusing to travelers.
Getting permission to visit a country will likely be more tedious,
requiring more documentation and more rigorous health checks. A lack of
clarity over who is in charge where will dissuade many would-be
travelers, especially at the beginning of the recovery.

``There won't be a rule banning you from traveling from Oklahoma to
Albuquerque, for example, but there probably will be one banning you
from traveling from Paris to New York,'' said Stewart Verdery, chief
executive of the lobbying firm Monument Advocacy and former assistant
secretary for the Department of Homeland Security. ``But who will make
decisions about what the new rules are? Most countries set their own
policies, with a little bit of input from international organizations.
Are we going to be as tough on health as we have been on security and
illegal immigration?''

--- TARIRO MZEZEWA

\textbf{Family Travel}

\hypertarget{whats-important-to-families-now}{%
\subsection{What's important to families
now?}\label{whats-important-to-families-now}}

+

-

With
\href{https://www.nytimes.com/2020/04/30/business/economy/coronavirus-unemployment-claims.html}{roughly
one in five American workers} out of a job and perhaps some
belt-tightening among those who are still employed, affordability will
become even more important for family travel.

Price-consciousness among families is nothing new. According to the
\href{https://familytravel.org/}{Family Travel Association}, an industry
nonprofit, for the past five years parents have consistently named
affordability the biggest challenge when traveling with their children.
Family vacations post-pandemic are likely to become shorter --- the
fewer the nights, the lower the cost of the trip.

``The things that parents are naturally cautious about are going to be
addressed in spades,'' said Rainer Jenss, the president and founder of
the \href{https://familytravel.org/}{Family Travel Association}.
``You're going to see a lot of travel industry suppliers offer more
flexibility, more discounts and better rates.''

\href{https://www.virtuoso.com/}{Virtuoso}, a network of luxury travel
agencies, had predicted that Hawaii, Italy, Orlando, Costa Rica and
England would be the year's top family destinations. Now it's domestic
options that can be reached by car, from the national parks to cities
like Charleston, S.C.

``We've all been forced to be one unit because of social distancing.
That closeness is going to continue through travel, regardless of where
we go,'' said Misty Belles, Virtuoso's managing director of global
public relations.

To mollify anxious parents, hotel brands like
\href{http://clubmed.com/}{Club Med} --- a leader in family travel, with
all-inclusive rates and
\href{https://www.clubmed.us/l/all-inclusive-resorts-for-families\#AmazingFamily}{multi-generational
programming} --- will lift the veil on health and disinfection. Its new
protocols, from strengthened deep-cleaning to temperature checks for
entry into kids' clubs, will feature prominently in its post-pandemic
marketing.

``Health and safety will be top of travelers' minds; it will change how
families choose their destinations and it will change how travel
companies operate,'' said Carolyne Doyon, the president and chief
executive of Club Med North America and the Caribbean.

--- SARAH FIRSHEIN

\textbf{Home sharing}

\hypertarget{will-social-distancing-kill-home-sharing}{%
\subsection{Will social distancing kill home
sharing?}\label{will-social-distancing-kill-home-sharing}}

+

-

The future of home sharing depends largely on whether travelers see
rentals as private, often cheaper, alternatives to hotels, or a source
of exposure to strangers' germs. The vagaries of cancellation policies
among rentals will also have an impact.

While hotels offered generous cancellation policies as travel
restrictions set in, the home-sharing platforms took varying approaches.
Before the pandemic, their policies let hosts set their own rules.
Airbnb overrode them, offering refunds to travelers who otherwise might
have been penalized. VRBO, its main competitor, took the side of the
homeowner, though it
\href{https://help.vrbo.com/articles/What-can-I-do-if-my-reservation-is-affected-by-the-Coronavirus?_ga=2.40507106.1577521592.1588085811-204612396.1588085811}{encouraged}
refunds and future credits.

Airbnb hosts may dream of defecting to another platform to their own
detriment.

``In the U.S., the majority of short-term rentals are booked via
Airbnb,'' said Scott Shatford, the chief executive of
\href{https://www.airdna.co/}{Air}\href{https://www.airdna.co/}{DNA},
which measures global home rentals. ``They control too much demand for
there to be a credible alternative to listing on Airbnb.''

Listings on Airbnb will soon indicate whether hosts are practicing
stringent new cleaning guidelines, including a minimum 24-hour waiting
period between bookings. A new category of listings will indicate no
guest has occupied a rental 72 hours before arrival.

Other services are burnishing their custodial practices.
\href{https://www.vacasa.com/}{Vacasa}, which is based in Portland,
Ore., and manages 26,000 rental homes globally, is working on a cleaning
badge that will appear on a listing, attesting to newly raised
standards.

Beyond hygiene, home-sharing companies are championing privacy. ``Travel
will be less urban,'' said Brian Chesky, the chief executive of Airbnb,
noting that the fastest-growing segment of the service's guests travel
less than 50 miles from home. ``Hotels are in cities; they need a
minimum market size. Airbnb is in 100,000 cities and towns.'' His
company laid off a quarter of its work force on Tuesday, about 1,900
people.

Meanwhile, listings are growing as more people look for extra income.
``Over the next six months, we expect more supply, and demand won't
catch up,'' Mr. Shatford said. ``We will see cheaper prices and bigger
discounts on monthly rentals.''

--- ELAINE GLUSAC

\textbf{hospitality workers}

\hypertarget{will-workers-win-concessions-or-do-companies-have-the-upper-hand}{%
\subsection{Will workers win concessions, or do companies have the upper
hand?}\label{will-workers-win-concessions-or-do-companies-have-the-upper-hand}}

+

-

The coronavirus outbreak made it clear that many of the housekeepers,
concierges, bartenders and other people staffing the low-wage,
high-turnover travel sector were undervalued by the companies they
worked for.

``Some of the biggest things that hospitality workers are now thinking
hard about are health care benefits, which many people lost entirely,
and standards like protective gear that wasn't provided to them,'' said
D. Taylor, international president of Unite Here, the hotel and
restaurant workers' union.

Going forward, Mr. Taylor said, workers will demand more from their
employers, and he expects that local and federal authorities will
establish new health and safety regulations for hospitality companies to
follow.

``Workers used to think, `Does this employer pay well?' Mr. Taylor said.
``Now it will be, `How are they on cleanliness? Do they have good health
care? Is this a safe environment? Are they committed to having the kind
of safety gear I might need?'''

Travelers could play a role, by working only with companies and
operators that prioritize facility cleanliness and employee well-being
and training.

Mr. Taylor said that he wouldn't be surprised if travelers start rating
hygiene and cleanliness as they typically rate food or the view from a
hotel room. A cleanliness certificate, like the Leadership in Energy and
Environmental Design for green buildings, could also become the new
norm.

But according to the
\href{https://www.ahla.com/covid-19s-impact-hotel-industry}{American
Hotel and Lodging Association}, a staggering 4 million hospitality jobs
in the United States have been lost --- out of a total of 8 million ---
since the onset of the coronavirus outbreak.

Those numbers might mean that employees desperate to return to work,
will do so, even in unsafe conditions.

--- TARIRO MZEZEWA

\textbf{Hotels}

\hypertarget{is-cleaning-the-new-amenity}{%
\subsection{Is cleaning the new
amenity?}\label{is-cleaning-the-new-amenity}}

+

-

When travel restrictions lift and hotels reopen, travelers can expect to
see housekeeping front and center in hotels.

Experts foresee more touchless check-in via apps and expressions of
hygiene that go beyond the paper wrap over the toilet seat.

``Transparency and tangible cues will give consumers more comfort,''
said Donna Quadri-Felitti, director of the hospitality management school
at Pennsylvania State University, adding that housekeeping logs may
become public. ``Robotic cleaning is still novel, but there's a lot of
talk about automation.''

Where hotel lobbies once aimed for warmth, expect a cold but gleaming
scene, with custodians frequently circulating with disinfectant. Pens
and other knickknacks likely to be touched by other guests will be
replaced with sanitizing wipes. Major hotel companies are experimenting
with electrostatic spraying to disinfect interiors, and ultraviolet
light to sanitize room keys.

Hospitality will be faceless, and encourage social distancing. Marriott
plans to offer contact-free room service through its cellphone app.
Hilton rooms will have a seal on the doors, indicating they haven't been
entered since they were last cleaned.

--- ELAINE GLUSAC

\textbf{Loyalty Programs}

\hypertarget{where-do-frequent-fliers-fit-in}{%
\subsection{Where do frequent fliers fit
in?}\label{where-do-frequent-fliers-fit-in}}

+

-

Even now, travel companies are working hard to keep loyalty program
members happy --- and, at the earliest signs of recovery, to get them on
the road and into the skies.

``What we're seeing is the loyalty programs extending people's frequent
flier status and also not expiring their miles,'' said Gary Leff, the
founder of the airline loyalty blog
\href{https://viewfromthewing.com/}{View from the Wing}. ``No company
wants to burn bridges with customers right now.''

In recent weeks, many companies --- including Marriott, United Airlines,
American Airlines and Hilton --- have made it easier to qualify for
elite or higher status.

Mr. Leff added that as countries and states gradually open up, airlines
will need to entice customers to fly, so expect an abundance of deals
geared toward frequent fliers, which will include seats for less mileage
and easier upgrades.

The good deals won't last forever, though. As promotions encouraging
travel pile up, planes, hotels and resorts will begin to fill up.

For those travelers with points to spare, using them first before paying
for seats or to book hotel rooms certainly makes sense, experts say.

``The beauty of miles, in these times of uncertainty, is that cold hard
cash in your account is definitely optimal,'' said Brian Kelly, the
founder of The Points Guy, a website devoted to loyalty programs. ``If
you have miles, using those to travel will allow you to save your
cash.''

--- TARIRO MZEZEWA

\textbf{Parks}

\hypertarget{will-people-gravitate-to-nature}{%
\subsection{Will people gravitate to
nature?}\label{will-people-gravitate-to-nature}}

+

-

According to an ongoing
\href{https://www.destinationanalysts.com/insights-updates/}{survey} of
travelers by Destination Analysts, a tourism research and marketing
firm, more than half of American travelers say they plan to avoid
crowded destinations when they resume traveling.

That bodes well for parks, even if allowing travelers back in requires
social-distancing modifications to close popular trails and overlooks,
with an emphasis on enforcement.

Before South Carolina's state parks closed on March 28, traffic in some
was as high as the record-setting numbers for the 2017 solar eclipse as
people sought a respite from quarantine. The parks reopened May 1 with
the help of law enforcement to manage the crowds.

At the onset of the pandemic, many national parks were also attracting
record numbers. Visitation to Arches National Park in Utah, for example,
was up 40 percent this past February versus February 2019. Today, many
major parks, including Grand Canyon National Park, remain closed as the
park service evaluates reopening on a park-by-park basis.

``We're going to have to create new norms of how to behave around one
another in national parks to create space,'' said Will Shafroth, the
president and chief executive of the National Park Foundation, the
nonprofit organization devoted to the national parks, adding that the
boardwalk around the Old Faithful geyser in Yellowstone National Park
may be one-way-only when it reopens.

The expected surge is an opportunity for lesser-known parks, he
suggested. Half of all
\href{https://www.nps.gov/orgs/1207/2019-visitation-numbers.htm}{visits}
to the 419 units in the system in 2019 were at just 27 parks.

--- ELAINE GLUSAC

\textbf{Personal space}

\hypertarget{how-private-can-you-get}{%
\subsection{How private can you get?}\label{how-private-can-you-get}}

+

-

With overcrowding now viewed as a health risk, personal space and
cleanliness will become paramount.

``One thing that's loud and clear from our clients: Any short-term
travel needs to be private,'' said Jack Ezon, the founder and managing
partner of \href{https://www.embarkbeyond.com/}{Embark Beyond,} a luxury
travel agency. ``Finding a `hermetically sealed' option seems to be the
most responsible solution.''

At the luxury end, that means increased demand in villas and luxury
hotel brands like \href{https://www.aman.com/}{Aman,} known for its
remote locations and stand-alone accommodations, and
\href{https://www.rosewoodhotels.com/en/default}{Rosewood}, where many
properties have residences with private entrances or elevators.

``While in the past privacy could be viewed as a nonessential privilege,
today it is considered a key element to sustaining personal safety and
security,'' said Radha Arora, president of Rosewood Hotels \& Resorts.

With hygiene playing an increasingly important role in travel, companies
along the entire price spectrum will double-down on efforts to create
privacy and health ``bubbles.'' A new health and safety certification
program from the
\href{https://www.prtourism.com/dnn/Default.aspx}{Puerto Rico Tourism
Company} includes wellness checkpoints and social-distancing guidelines.
Hilton will push (and likely expand) its
\href{https://www.hilton.com/en/corporate/coronavirus/}{Digital Key}
program, which allows guests to check into their rooms without
interacting with anyone, as part of its
\href{https://www.hilton.com/en/corporate/coronavirus/}{new global
cleanliness program.} And
\href{https://www.wyndhamhotels.com/wyndham}{Wyndham Hotels \& Resorts},
which is already seeing increased demand for its properties with
exterior corridors (common among the company's economy brands like Super
8 and Days Inn) from travelers preferring a direct car-to-room
connection, will promote social distancing in public spaces in more than
6,000 hotels.

The public may be leery of large commercial planes and newfound airport
hassles --- like the full-body disinfection machine undergoing trials at
Hong Kong International Airport. Companies like
\href{https://www.jsx.com/}{JSX}, which provides hop-on, short-haul jet
service out of private terminals may benefit. JSX currently operates on
the West Coast and post-pandemic protocols include contactless check-in
and 20-person limits on 30-seat planes. **

--- SARAH FIRSHEIN

\textbf{Sustainability}

\hypertarget{is-the-green-wave-over}{%
\subsection{Is the green wave over?}\label{is-the-green-wave-over}}

+

-

Will planetary health be as urgent to travelers focused on preserving
personal health? In a germophobic world, will single-use plastics make a
comeback?

``The work on reduction of plastic is going to take a back seat to the
larger quest for the health and security of travelers,'' said Megan
Epler Wood, the managing director of the
\href{https://www.johnson.cornell.edu/center-for-sustainable-global-enterprise/programs/sustainable-tourism-asset-management-program-stamp/}{Sustainable
Tourism Asset Management Program} at Cornell University. ``But there are
plenty of reasons to find hope.''

Ms. Epler Wood points to efforts by tourism organizations and
destinations to plan for a return to travel that addresses overtourism.
The United Nations World Travel Organization released
\href{https://webunwto.s3.eu-west-1.amazonaws.com/s3fs-public/2020-04/COVID19_Recommendations_English_1.pdf}{recommendations}
for recovery that call for citizens' platforms for local feedback and
tourism councils that coordinate public and private sector involvement.

She also called the present situation an opportunity ``for our global
figures to see if we can build a better set of sustainable procedures.''

Social distancing may naturally ease overtourism, and the global
shutdown is
\href{https://www.nytimes.com/2020/04/30/climate/global-emissions-decline.html}{poised
to drop carbon emissions} by 8 percent in 2020. But while reduced
traffic may be good for many places, tourists are sorely missed in
others. In parts of Africa, for example, safaris and park admission fees
fund conservation. Without those sources of revenue,
\href{https://www.nytimes.com/2020/04/08/science/coronavirus-poaching-rhinos.html}{poaching}
has been up in South Africa and Botswana.

``These places have an experience economy that supports protection of
the wilderness,'' said Gregory Miller, the executive director of the
\href{https://www.responsibletravel.org/}{Center for Responsible
Travel}\href{https://www.responsibletravel.org/}{.} ``We need to restore
economies built on experience, not extraction. Otherwise, you have
poaching, slash and burn and the taking of resources.''

One possible upside of the pandemic is the awareness of how spending
locally helps communities.

``We are all becoming familiar with the idea of helping small businesses
through Covid-19,'' said Jonathon Day, an associate professor focused on
sustainable tourism at Purdue University. ``If we carry it into the
future to places we travel, thinking about whether the money will stay
in the community, that's something we can take from this experience.''

--- ELAINE GLUSAC

\textbf{Tour Operators}

\hypertarget{will-travelers-sign-up}{%
\subsection{Will travelers sign up?}\label{will-travelers-sign-up}}

+

-

The logistical ease of group tours comes with a trade-off: traveling
with strangers.

``I certainly appreciate the paradox: There is safety in numbers, there
is risk in numbers,'' said Jennifer Tombaugh, the president of
\href{https://www.tauck.com/}{Tauck}, a high-end tour and cruise
company. One solution, Ms. Tombaugh said, will be smaller groups with
lower guest-to-guide ratios --- a trend that had already been predicted
to rise, pre-pandemic, by the \href{https://ustoa.com/}{United States
Tour Operators Association}.

Debra Asberry, the founder and president of
\href{https://www.women-traveling.com/}{Women Traveling Together}, which
runs affordably priced small-group tours for women over 50, expects the
national parks trips to rebound first, **** just as they did after 9/11.

``It really saved us in 2002, and we think the same thing's going to
happen here: We'll have a much heavier emphasis on domestic tourists,
especially into the first half of 2021,'' Ms. Asberry said.

After being cooped up for months, tour-goers may gravitate toward
wellness experiences. ``If 2020 proves to be a year we spend a lot of
time indoors, 2021 will be about getting outdoors and getting active,
with tours centered around things like cycling, trekking and
mindfulness,'' said James Thornton, chief executive of
\href{https://www.intrepidtravel.com/}{Intrepid Travel,} which runs
tours on all seven continents.

And overtourism, an industrywide concern, has renewed importance. ``Ten
years ago, people wanted crowded markets and big, well-known cities,''
said Bruce Poon Tip, founder of \href{https://www.gadventures.com/}{G
Adventures}, a community-tourism-focused tour company whose eight-day
trips range from \$650 to \$3,200 per person. ``Now there's a real push
for tours in Antarctica, the Galápagos, Mongolia and Tibet --- all wide,
open spaces.''

Exactly when and where touring resumes will depend upon several factors,
including travel advisories and consumer confidence, particularly about
developing countries with insufficient medical care.

``We want to make sure that we can do this in a way that allows guests
to be present and soak in all they've desired to experience,'' Ms.
Tombaugh said.

--- SARAH FIRSHEIN

\textbf{travel Insurance}

\hypertarget{will-anyone-actually-buy-it}{%
\subsection{Will anyone actually buy
it?}\label{will-anyone-actually-buy-it}}

+

-

Many travelers were infuriated to learn that their travel insurance was
worthless during this pandemic. That aggravation underscored one key
truth: Buyer beware.

``Travel insurance is typically not a good economic deal; it's usually
way too expensive and filled with caveats and exclusions,'' said J.
Robert Hunter, the director of insurance for the
\href{https://consumerfed.org/}{Consumer Federation of America}, a
consumer-advocacy nonprofit.

If insurance companies are to rebuild consumer confidence --- especially
with airlines and hotels changing policies to make it easier to cancel
--- they will have little choice but to demystify the fine print.
Customers will demand it.

``There's going to be a lot more focus, rather than ticking a box and
moving on,'' said Anna Gladman, the chief executive of
\href{https://www.nibtravelinsurance.com.au/}{nib Travel,} which owns
\href{https://www.worldnomads.com/usa/}{World Nomads,} a travel
insurance company. ``People are going to be concerned about catching
this, so they'll want to know more about their products.''

Travelers will be more likely to build plans from a ``menu'' (say,
emergency evacuation and trip interruption) rather than relying on
pre-bundled packages. And the demand for Cancel For Any Reason coverage
--- a costly add-on that partially reimburses policyholders when they
cancel trips for any reason --- is likely to increase.

But clarified information may make shopping less frustrating.

``We're going to see more insurance carriers explicitly acknowledge
pandemics in their policies, either clearly covering them or excluding
them, in order to avoid mismanaged consumer expectations later on,''
said Jennifer Fitzgerald, the co-founder and chief executive of the
online insurance marketplace
\href{https://www.policygenius.com/}{Policygenius}.

Megan Moncrief, the chief marketing officer at
\href{https://www.squaremouth.com/}{Squaremouth}, a travel insurance
comparison website, said the industry will likely follow the precedent
set by 9/11, which forced terrorism coverage into insurance policies.

``The biggest pivot in the industry is going to be more policies with
pandemic coverage for things like C.D.C. alerts, travel advisories and
stay-at-home orders,'' Ms. Moncrief said.

--- SARAH FIRSHEIN

Read 140 Comments

\begin{itemize}
\item
\item
\item
\item
\end{itemize}

Advertisement

\protect\hyperlink{after-bottom}{Continue reading the main story}

\hypertarget{site-index}{%
\subsection{Site Index}\label{site-index}}

\hypertarget{site-information-navigation}{%
\subsection{Site Information
Navigation}\label{site-information-navigation}}

\begin{itemize}
\tightlist
\item
  \href{https://help.nytimes.com/hc/en-us/articles/115014792127-Copyright-notice}{©~2020~The
  New York Times Company}
\end{itemize}

\begin{itemize}
\tightlist
\item
  \href{https://www.nytco.com/}{NYTCo}
\item
  \href{https://help.nytimes.com/hc/en-us/articles/115015385887-Contact-Us}{Contact
  Us}
\item
  \href{https://www.nytco.com/careers/}{Work with us}
\item
  \href{https://nytmediakit.com/}{Advertise}
\item
  \href{http://www.tbrandstudio.com/}{T Brand Studio}
\item
  \href{https://www.nytimes.com/privacy/cookie-policy\#how-do-i-manage-trackers}{Your
  Ad Choices}
\item
  \href{https://www.nytimes.com/privacy}{Privacy}
\item
  \href{https://help.nytimes.com/hc/en-us/articles/115014893428-Terms-of-service}{Terms
  of Service}
\item
  \href{https://help.nytimes.com/hc/en-us/articles/115014893968-Terms-of-sale}{Terms
  of Sale}
\item
  \href{https://spiderbites.nytimes.com}{Site Map}
\item
  \href{https://help.nytimes.com/hc/en-us}{Help}
\item
  \href{https://www.nytimes.com/subscription?campaignId=37WXW}{Subscriptions}
\end{itemize}
