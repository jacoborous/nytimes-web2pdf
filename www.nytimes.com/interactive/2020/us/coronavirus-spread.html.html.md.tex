Sections

SEARCH

\protect\hyperlink{site-content}{Skip to
content}\protect\hyperlink{site-index}{Skip to site index}

\hypertarget{comments}{%
\subsection{\texorpdfstring{\protect\hyperlink{commentsContainer}{Comments}}{Comments}}\label{comments}}

\href{}{How the Virus Won}\href{}{Skip to Comments}

The comments section is closed. To submit a letter to the editor for
publication, write to
\href{mailto:letters@nytimes.com}{\nolinkurl{letters@nytimes.com}}.

\hypertarget{how-the-virus-won}{%
\section{How the Virus Won}\label{how-the-virus-won}}

By \href{https://www.nytimes.com/by/derek-watkins}{Derek Watkins},
\href{https://www.nytimes.com/by/josh-holder}{Josh Holder},
\href{https://www.nytimes.com/by/james-glanz}{James Glanz},
\href{https://www.nytimes.com/by/weiyi-cai}{Weiyi Cai},
\href{https://www.nytimes.com/by/benedict-carey}{Benedict Carey} and
\href{https://www.nytimes.com/by/jeremy-white}{Jeremy White}June 25,
2020

\begin{itemize}
\item
\item
\item
\item
\item
  \emph{819}
\end{itemize}

Invisible outbreaks sprang up everywhere. The United States ignored the
warning signs. We analyzed travel patterns, hidden infections and
genetic data to show how the epidemic spun out of control.

Invisible outbreaks sprang up everywhere. The United States ignored the
warning signs. We analyzed travel patterns, hidden infections and
genetic data to show how the epidemic spun out of control.

It started small. A man near Seattle had a persistent cough. A woman in
Chicago had a fever and shortness of breath.

By mid-February, there were only \textbf{15 known coronavirus cases} in
the United States, all with direct links to China.

``The 15 within a couple of days is going to be down to close to zero,''
President Trump said.

The patients were isolated. Their contacts were monitored. Travel from
China was restricted.

None of that worked. Only a small part of the picture was visible. Some
\textbf{2,000 hidden infections} were already spreading through major
cities.

\includegraphics{https://static01.nyt.com/newsgraphics/2020/04/14/coronavirus-us-reconstruct/b8ea3804cd1b44e3ac731f706c9e32f6028bc8f6/cases-swatch.png}

Undetected infections \textbf{by Feb. 15}, per estimates from a
Northeastern University modeling team led by Alessandro Vespignani

We traced the hidden spread of the epidemic to explain why the United
States failed to stop it.

At every crucial moment, American officials were weeks or months behind
the reality of the outbreak. Those delays likely cost tens of thousands
of lives.

\hypertarget{how-the-virus-got-in}{%
\section{How The Virus Got In}\label{how-the-virus-got-in}}

The China travel ban was a partial success**:** Only a handful of
infected travelers from China are estimated to have made it into the
country undetected before restrictions were imposed on Feb. 2.

But it wasn't enough.

A vast wave of \textbf{infected travelers} --- roughly 1,000, one model
suggests --- came from other countries in Asia, Europe and the rest of
the world in February, each a dangerous spark that could set off a wider
outbreak.

\includegraphics{https://static01.nyt.com/newsgraphics/2020/04/14/coronavirus-us-reconstruct/b8ea3804cd1b44e3ac731f706c9e32f6028bc8f6/dot-swatch.png}

Infected travelers \textbf{in February}, per estimates by Northeastern

Many of those infections died out. But by mid-February, a few caught
fire and became \textbf{outbreaks}, spreading invisibly.

\includegraphics{https://static01.nyt.com/newsgraphics/2020/04/14/coronavirus-us-reconstruct/b8ea3804cd1b44e3ac731f706c9e32f6028bc8f6/cases-swatch.png}

Estimated infections by \textbf{Feb. 15} in selected cities, per the
Northeastern model. Other cities may have also had early outbreaks.

The country was unaware of its own epidemic. Many tests released by the
C.D.C. didn't work, leaving only enough to test people who had visited
China or had contact with a handful of known cases.

Over the next two weeks, the invisible \textbf{outbreaks doubled in
size, then doubled three more times}.

\includegraphics{https://static01.nyt.com/newsgraphics/2020/04/14/coronavirus-us-reconstruct/b8ea3804cd1b44e3ac731f706c9e32f6028bc8f6/cases-swatch.png}

Estimated infections by \textbf{March 1}

\hypertarget{how-the-first-outbreaks-spread}{%
\section{How the First Outbreaks
Spread}\label{how-the-first-outbreaks-spread}}

Top federal health experts concluded by late February that the virus was
likely to spread widely within the United States and that government
officials would soon need to urge the public to embrace social
distancing measures, such as avoiding crowds and staying home.

But Mr. Trump wanted to avoid disrupting the economy. So some of his
health advisers, at Mr. Trump's urging, told Americans at the end of
February to continue to travel domestically and go on with their normal
lives.

And they did. \textbf{Millions moved across the country}, cellphone data
shows. Some unknowingly carried the virus with them.

\includegraphics{https://static01.nyt.com/newsgraphics/2020/04/14/coronavirus-us-reconstruct/b8ea3804cd1b44e3ac731f706c9e32f6028bc8f6/flow-swatch.png}

Travel volume from \textbf{March 1 to March 14,} based on aggregated
data from Cuebiq, a data intelligence firm. **** Minor routes not shown.

Researchers with the Seattle Flu Study ignored C.D.C. testing
restrictions and uncovered a single case with no travel history in late
February. This was the first sign that the outbreak had spun badly out
of control.

Over the two weeks that followed, people made about \textbf{4.3 million
trips} from the Seattle area.

\includegraphics{https://static01.nyt.com/newsgraphics/2020/04/14/coronavirus-us-reconstruct/b8ea3804cd1b44e3ac731f706c9e32f6028bc8f6/cases-swatch.png}

Estimated cases as of \textbf{March 1}

\includegraphics{https://static01.nyt.com/newsgraphics/2020/04/14/coronavirus-us-reconstruct/b8ea3804cd1b44e3ac731f706c9e32f6028bc8f6/flow-swatch.png}

Travel out of Seattle from \textbf{March 1 to 14}

\textbf{Thousands were contagious.} Genetic samples linked to the
Seattle outbreak appeared in at least 14 states, said Trevor Bedford, a
professor at Fred Hutchinson Cancer Research Center and a leader of the
flu study.

\includegraphics{https://static01.nyt.com/newsgraphics/2020/04/14/coronavirus-us-reconstruct/b8ea3804cd1b44e3ac731f706c9e32f6028bc8f6/dot-swatch.png}

Red dots show proportion of contagious travelers, based on estimates
from Jeffrey Shaman, Columbia University.

Seattle was just the beginning. In New York City, where officials had
found only a single case by March 1, \textbf{roughly 10,000 infections}
had spread undetected.

New Yorkers and visitors continued to travel out of the city. More than
\textbf{5,000 contagious travelers} left in the first two weeks of
March, estimates suggest.

\includegraphics{https://static01.nyt.com/newsgraphics/2020/04/14/coronavirus-us-reconstruct/b8ea3804cd1b44e3ac731f706c9e32f6028bc8f6/dot-swatch.png}

Red dots show proportion of contagious travelers, based on estimates
from Dr. Shaman.

``I'm encouraging New Yorkers to go on with your lives and get out on
the town,'' Mayor Bill de Blasio said on March 2.

People leaving New York City made about 2.8 million trips to the
\textbf{Hudson Valley}. Some carried the virus with them, and outbreaks
there accelerated in mid-March, the likely result of travel from New
York, a Times analysis found.

\includegraphics{https://static01.nyt.com/newsgraphics/2020/04/14/coronavirus-us-reconstruct/b8ea3804cd1b44e3ac731f706c9e32f6028bc8f6/cases-swatch.png}

Known cases in the Hudson Valley as of \textbf{March 22}

People also made more than 25,000 trips to \textbf{New Orleans}, where
genetic data suggests that a large early outbreak stemmed from
infections from New York, according to Karthik Gangavarapu, a
computational scientist at Scripps Research, and Dr. Bedford.

\includegraphics{https://static01.nyt.com/newsgraphics/2020/04/14/coronavirus-us-reconstruct/b8ea3804cd1b44e3ac731f706c9e32f6028bc8f6/cases-swatch.png}

Known cases in New Orleans as of \textbf{March 22}

Tracking signature genetic mutations of the virus allows researchers to
estimate the influence of early outbreaks. Early on, \textbf{variants
prominent in Seattle's outbreak} were found more frequently.

\includegraphics{https://static01.nyt.com/newsgraphics/2020/04/14/coronavirus-us-reconstruct/b8ea3804cd1b44e3ac731f706c9e32f6028bc8f6/seattlevirus.png}

Variant of the virus often found in Seattle

\includegraphics{https://static01.nyt.com/newsgraphics/2020/04/14/coronavirus-us-reconstruct/b8ea3804cd1b44e3ac731f706c9e32f6028bc8f6/othervirus.png}

Other variants

But later samples showed that a variant \textbf{found often in New York
City's outbreak} had become much more widespread. A new analysis of
thousands of mutations also points directly back to New York, Dr.
Bedford said.

\includegraphics{https://static01.nyt.com/newsgraphics/2020/04/14/coronavirus-us-reconstruct/b8ea3804cd1b44e3ac731f706c9e32f6028bc8f6/seattlevirus.png}

Variant of the virus often found in Seattle

\includegraphics{https://static01.nyt.com/newsgraphics/2020/04/14/coronavirus-us-reconstruct/b8ea3804cd1b44e3ac731f706c9e32f6028bc8f6/newyorkvirus.png}

Variant often found in New York

Travel from the city helped to spread that variant across the country.

``New York has acted as a Grand Central Station for this virus,'' said
David Engelthaler of the Translational Genomics Research Institute.

\hypertarget{how-local-hot-spots-emerged}{%
\section{How Local Hot Spots
Emerged}\label{how-local-hot-spots-emerged}}

By the time President Trump blocked travel from Europe on March 13, the
restrictions were essentially pointless.

The outbreak had already been spreading widely in most states for weeks.

A woman celebrated Mardi Gras in \textbf{New Orleans} and then returned
to \textbf{Memphis}, becoming the first known case there. The New
Orleans outbreak helped seed infection across Louisiana and the South.

\includegraphics{https://static01.nyt.com/newsgraphics/2020/04/14/coronavirus-us-reconstruct/b8ea3804cd1b44e3ac731f706c9e32f6028bc8f6/cases-swatch.png}

Known cases as of \textbf{March 22}

\includegraphics{https://static01.nyt.com/newsgraphics/2020/04/14/coronavirus-us-reconstruct/b8ea3804cd1b44e3ac731f706c9e32f6028bc8f6/flow-swatch.png}

Travel volume from \textbf{March 1 to March 14}

A mourner from \textbf{Atlanta} visited \textbf{Albany, Ga.}, to attend
a funeral. Days later, the virus swept through the community in one of
the deadliest outbreaks in the country.

A man returning from a basketball tournament in \textbf{Tucson, Ariz.,}
is thought to have brought the virus to a small town in the Navajo
Nation, the largest Native American reservation within the United
States. An outbreak surged through the reservation after a church rally
there.

\hypertarget{how-the-outbreak-slowed}{%
\section{How the Outbreak Slowed}\label{how-the-outbreak-slowed}}

Faced with an outbreak that had grown beyond their ability to test or
trace, American officials had no option but to ask the public to stay
home.

On March 16, weeks after health officials privately concluded a more
active response would be needed, President Trump asked Americans to
limit travel, avoid groups and stay home from work and school if they
felt sick. One by one, states issued stay-at-home orders and closed
businesses.

As late as March 10, the average person moved around pretty much as
normal, according to cellphone data from Cuebiq.

\includegraphics{https://static01.nyt.com/newsgraphics/2020/04/14/coronavirus-us-reconstruct/b8ea3804cd1b44e3ac731f706c9e32f6028bc8f6/mobility-green.png}

\textbf{Normal movement} on March 10

\includegraphics{https://static01.nyt.com/newsgraphics/2020/04/14/coronavirus-us-reconstruct/b8ea3804cd1b44e3ac731f706c9e32f6028bc8f6/mobility-yellow.png}

\textbf{Some reduction in movement} on March 10

But by March 17, after Mr. Trump's announcement, much of the country
started to shut down.

\includegraphics{https://static01.nyt.com/newsgraphics/2020/04/14/coronavirus-us-reconstruct/b8ea3804cd1b44e3ac731f706c9e32f6028bc8f6/mobility-yellow.png}

\textbf{Some reduction in movement} on March 17

Most people were already staying home in the Bay Area, where some of the
first cases were identified.

\includegraphics{https://static01.nyt.com/newsgraphics/2020/04/14/coronavirus-us-reconstruct/b8ea3804cd1b44e3ac731f706c9e32f6028bc8f6/mobility-red.png}

\textbf{Little movement} on March 17

By March 24, much of the country had followed. Within weeks, those
shutdowns stopped exponential growth of the virus from overwhelming many
parts of the country.

\includegraphics{https://static01.nyt.com/newsgraphics/2020/04/14/coronavirus-us-reconstruct/b8ea3804cd1b44e3ac731f706c9e32f6028bc8f6/mobility-yellow.png}

\textbf{Some reduction in movement} on March 24

\includegraphics{https://static01.nyt.com/newsgraphics/2020/04/14/coronavirus-us-reconstruct/b8ea3804cd1b44e3ac731f706c9e32f6028bc8f6/mobility-red.png}

\textbf{Little movement} on March 24

But every day mattered to halt the virus in New York City, where
political leaders waited crucial days to close schools and impose a
stay-at-home order as the virus spun out of control.

New cases slowed in San Francisco, Seattle and other places where
measures were taken relatively quickly. Hospital beds stayed open. The
worst was kept at bay.

But in New York City, the response was too late, said Lauren Ancel
Meyers, an epidemiologist at the University of Texas at Austin. Other
factors, such as the city's density and its rate of international
travel, also may have played a role, Dr. Meyers said.

\textbf{More than 22,000 deaths} in the New York City area could have
been avoided if the country had started social distancing just one week
earlier, Columbia University researchers estimate.

\includegraphics{https://static01.nyt.com/newsgraphics/2020/04/14/coronavirus-us-reconstruct/b8ea3804cd1b44e3ac731f706c9e32f6028bc8f6/deathswatch.png}

Deaths avoided by \textbf{May 3} with earlier social distancing, based
on estimates by Dr. Shaman and Sen Pei, Columbia University

\textbf{About 36,000 deaths} nationwide could have been avoided **** by
early May had social distancing begun earlier, the estimates say.

Even now, America remains in the dark.

Most infected people are never tested. There is little capacity to trace
and isolate the contacts to those who do test positive. After the
lockdowns expired, new cases spiked once again.

In recent weeks, new outbreaks flared across the South and the West.

These are only the cases we know about. No one can see where the virus
will go next.

\includegraphics{https://static01.nyt.com/newsgraphics/2020/04/14/coronavirus-us-reconstruct/b8ea3804cd1b44e3ac731f706c9e32f6028bc8f6/cases-swatch.png}

New confirmed cases from \textbf{June 9 to June 23}

For more details on the early outbreak, see
\href{https://www.nytimes.com/interactive/2020/03/22/world/coronavirus-spread.html}{how
the virus got out of China.}

Eric Lipton contributed reporting.

\textbf{Notes:}\\
Our understanding of the early outbreak in the United States draws on
case reports, travel patterns, genetic sequencing and disease modeling
that simulates the course of the outbreak based on how it spread and
what is known about the virus. All models are estimates, and it is
impossible to know for certain the origin of each infection or the
number of infections that were not confirmed by testing.

There are no comprehensive, official counts of cases, deaths or tests
throughout the United States. Known cases are from a
\href{https://www.nytimes.com/interactive/2020/us/coronavirus-us-cases.html}{New
York Times database} based on information from federal, state and local
officials. Cases are shown by
\href{https://www.census.gov/topics/housing/housing-patterns/about/core-based-statistical-areas.html}{core-based
statistical areas}, or by county for cases outside of those areas. The
first 15 cases are labeled by statistical area. Those labels may not
match the names of the exact cities in which cases were seen.

The travel patterns we show represent movement between core-based
statistical areas, based on aggregated, anonymous cellphone location
data collected by
\href{https://www.cuebiq.com/visitation-insights-covid19/}{Cuebiq}, a
data intelligence firm that tracks the locations of more than 15 million
cellphones in the United States. The data captures trips, not unique
travelers, and it includes commutes between statistical areas as well as
longer-distance travel. Some minor and short routes are not shown.
Reductions in movement due to social distancing are based measurements
by Cuebiq of the median range that people in each area travel each day.
The effects of social distancing are from Lauren Gardner, et al.,
``\href{https://www.medrxiv.org/content/10.1101/2020.05.07.20092353v1}{Social
Distancing Is Effective at Mitigating Covid-19 Transmission in the
United States}.''

Estimates of the number of
\href{https://www.nytimes.com/2020/04/23/us/coronavirus-early-outbreaks-cities.html}{undetected
infections in 11 American cities} are from modeling by Northeastern
University, as are the estimates of the number of undetected infected
travelers who came into the United States from other countries. See
Matteo Chinazzi and Jessica T. Davis, et al.
"\href{https://science.sciencemag.org/content/368/6489/395}{The effect
of travel restrictions on the spread of the 2019 novel coronavirus
(COVID-19) outbreak}" Science.

Estimates of the number of contagious people who left New York and
Seattle are from modeling by Sen Pei and Jeffrey Shaman at Columbia
University. Estimates of the number of deaths that
\href{https://www.nytimes.com/2020/05/20/us/coronavirus-distancing-deaths.html}{could
have been avoided} with earlier social distancing are from Dr. Pei, et
al.,
``\href{https://www.medrxiv.org/content/10.1101/2020.05.15.20103655v2}{Differential
Effects of Intervention Timing on Covid-19 Spread in the United
States}.''

Genetic samples of the virus are from
\href{https://nextstrain.org/ncov/north-america}{Nextstrain}. We show
the samples grouped by names that were assigned prior to May 1, 2020.
The connection of American outbreaks to
\href{https://www.nytimes.com/2020/05/07/us/new-york-city-coronavirus-outbreak.html}{travel
from New York City} is based on a Times analysis of travel patterns and
analysis of genetic mutations by Trevor Bedford, associate professor,
Fred Hutchinson Cancer Research Center and the University of Washington.

Conclusions from federal health officials in February about the likely
spread of the virus are from
\href{https://www.nytimes.com/2020/04/11/us/politics/coronavirus-trump-response.html}{Times
reporting}. Mr. Trump said that cases would be close to zero within days
at a
\href{https://www.whitehouse.gov/briefings-statements/remarks-president-trump-vice-president-pence-members-coronavirus-task-force-press-conference/}{Feb.
26 press conference}. Health advisers urged Americans to go on with
their normal lives in a
\href{https://www.whitehouse.gov/briefings-statements/remarks-president-trump-vice-president-pence-members-coronavirus-task-force-press-conference-2/}{Feb.
29 press conference}. Mr. de Blasio encouraged New Yorkers to do the
same
\href{https://twitter.com/BilldeBlasio/status/1234648718714036229}{in a
tweet on March 2}. Mr. Trump recommended that Americans avoid travel at
\href{https://www.whitehouse.gov/briefings-statements/remarks-president-trump-vice-president-pence-members-coronavirus-task-force-press-briefing-3/}{a
press conference on March 16}.

\textbf{Additional sources:}\\
•~~Xianding Deng, Charles Chiu et al.,
``\href{https://science.sciencemag.org/content/sci/early/2020/06/05/science.abb9263.full.pdf}{A
Genomic Survey of SARS-CoV-2 Reveals Multiple Introductions into
Northern California without a Predominant Lineage},'' Science

•~~Matthew Maurano, Adriana Heguy et al.,
``\href{https://www.medrxiv.org/content/medrxiv/early/2020/04/23/2020.04.15.20064931.full.pdf}{Sequencing
identifies multiple, early introductions of SARS-CoV2 to New York City
Region},'' medRxiv.org

•~~Ana S. Gonzalez-Reiche, Harm van Bakel et al.,
``\href{https://science.sciencemag.org/content/early/2020/05/28/science.abc1917}{Introductions
and early spread of SARS-CoV-2 in the New York City area},'' Science

Eric Lipton contributed reporting.

Read 819 Comments

\begin{itemize}
\item
\item
\item
\item
\end{itemize}

Advertisement

\protect\hyperlink{after-bottom}{Continue reading the main story}

\hypertarget{site-index}{%
\subsection{Site Index}\label{site-index}}

\hypertarget{site-information-navigation}{%
\subsection{Site Information
Navigation}\label{site-information-navigation}}

\begin{itemize}
\tightlist
\item
  \href{https://help.nytimes.com/hc/en-us/articles/115014792127-Copyright-notice}{©~2020~The
  New York Times Company}
\end{itemize}

\begin{itemize}
\tightlist
\item
  \href{https://www.nytco.com/}{NYTCo}
\item
  \href{https://help.nytimes.com/hc/en-us/articles/115015385887-Contact-Us}{Contact
  Us}
\item
  \href{https://www.nytco.com/careers/}{Work with us}
\item
  \href{https://nytmediakit.com/}{Advertise}
\item
  \href{http://www.tbrandstudio.com/}{T Brand Studio}
\item
  \href{https://www.nytimes.com/privacy/cookie-policy\#how-do-i-manage-trackers}{Your
  Ad Choices}
\item
  \href{https://www.nytimes.com/privacy}{Privacy}
\item
  \href{https://help.nytimes.com/hc/en-us/articles/115014893428-Terms-of-service}{Terms
  of Service}
\item
  \href{https://help.nytimes.com/hc/en-us/articles/115014893968-Terms-of-sale}{Terms
  of Sale}
\item
  \href{https://spiderbites.nytimes.com}{Site Map}
\item
  \href{https://help.nytimes.com/hc/en-us}{Help}
\item
  \href{https://www.nytimes.com/subscription?campaignId=37WXW}{Subscriptions}
\end{itemize}
