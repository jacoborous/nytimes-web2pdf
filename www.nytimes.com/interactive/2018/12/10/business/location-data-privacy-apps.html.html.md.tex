 **NYTimes.com no longer supports Internet Explorer 9 or earlier. Please
upgrade your browser.
\href{http://www.nytimes.com/content/help/site/ie9-support.html}{LEARN
MORE »}

**Sections

**Home

**Search

\hypertarget{the-new-york-times}{%
\subsection{\texorpdfstring{\href{http://www.nytimes.com/}{The New York
Times}}{The New York Times}}\label{the-new-york-times}}

\hypertarget{-business-}{%
\subsubsection{\texorpdfstring{ \href{/section/business}{Business}
}{ Business }}\label{-business-}}

 \href{/section/business}{Business} \textbar{}Your Apps Know Where You
Were Last Night, and They're Not Keeping It Secret

**Close search

\hypertarget{site-search-navigation}{%
\subsection{Site Search Navigation}\label{site-search-navigation}}

Search NYTimes.com

**Clear this text input

Go

\url{https://nyti.ms/2G4rjaG}

\hypertarget{site-navigation}{%
\subsection{Site Navigation}\label{site-navigation}}

\hypertarget{site-mobile-navigation}{%
\subsection{Site Mobile Navigation}\label{site-mobile-navigation}}

\hypertarget{your-apps-know-where-you-were-last-night-and-theyre-not-keeping-it-secret}{%
\section{Your Apps Know Where You Were Last Night, and They're Not
Keeping It
Secret}\label{your-apps-know-where-you-were-last-night-and-theyre-not-keeping-it-secret}}

Dozens of companies use smartphone locations to help advertisers and
even hedge funds. They say it's anonymous, but the data shows how
personal it is.

By Richard Harris \textbar{} Satellite imagery by U.S.D.A. N.A.I.P.

\hypertarget{your-apps-know-where-you-were-last-night-and-theyre-not-keeping-it-secret-1}{%
\section{Your Apps Know Where You Were Last Night, and They're Not
Keeping It
Secret}\label{your-apps-know-where-you-were-last-night-and-theyre-not-keeping-it-secret-1}}

Dozens of companies use smartphone locations to help advertisers and
even hedge funds. They say it's anonymous, but the data shows how
personal it is.

By \href{https://www.nytimes.com/by/jennifer-valentino-devries}{JENNIFER
VALENTINO-DeVRIES},
\href{https://www.nytimes.com/by/natasha-singer}{NATASHA SINGER},
MICHAEL H. KELLER and AARON KROLIK DEC. 10, 2018

The millions of dots on the map trace highways, side streets and bike
trails --- each one following the path of an anonymous cellphone user.

One path tracks someone from a home outside Newark to a nearby Planned
Parenthood, remaining there for more than an hour. Another represents a
person who travels with the mayor of New York during the day and returns
to Long Island at night.

Yet another leaves a house in upstate New York at 7 a.m. and travels to
a middle school 14 miles away, staying until late afternoon each school
day. Only one person makes that trip: Lisa Magrin, a 46-year-old math
teacher. Her smartphone goes with her.

An app on the device gathered her location information, which was then
sold without her knowledge. It recorded her whereabouts as often as
every two seconds, according to a database of more than a million phones
in the New York area that was reviewed by The New York Times. While Ms.
Magrin's identity was not disclosed in those records, The Times was able
to easily connect her to that dot.

The app tracked her as she went to a Weight Watchers meeting and to her
dermatologist's office for a minor procedure. It followed her hiking
with her dog and staying at her ex-boyfriend's home, information she
found disturbing.

``It's the thought of people finding out those intimate details that you
don't want people to know,'' said Ms. Magrin, who allowed The Times to
review her location data.

Like many consumers, Ms. Magrin knew that apps could track people's
movements. But as smartphones have become ubiquitous and technology more
accurate, an industry of snooping on people's daily habits has spread
and grown more intrusive.

\hypertarget{intro}{%
\subsection{intro}\label{intro}}

By Michael H. Keller and Richard Harris \textbar{} Satellite imagery by
Mapbox and DigitalGlobe

At least 75 companies receive anonymous, precise location data from apps
whose users enable location services to get local news and weather or
other information, The Times found. Several of those businesses claim to
track up to 200 million mobile devices in the United States --- about
half those in use last year. The database reviewed by The Times --- a
sample of information gathered in 2017 and held by one company ---
reveals people's travels in startling detail, accurate to within a few
yards and in some cases updated more than 14,000 times a day.

\emph{{[}Learn how to
\href{https://www.nytimes.com/2018/12/10/technology/prevent-location-data-sharing.html?action=click\&module=Intentional\&pgtype=Article}{stop
apps from tracking your location}.{]}}

These companies sell, use or analyze the data to cater to advertisers,
retail outlets and even hedge funds seeking insights into consumer
behavior. It's a hot market, with sales of location-targeted advertising
reaching an
\href{https://shop.biakelsey.com/product/2018-u-s-local-mobile-local-social-ad-forecast}{estimated}
\$21 billion this year. IBM has gotten into the industry, with its
purchase of
\href{https://www.ibm.com/case-studies/mcdonalds-watson-advertising}{the
Weather Channel's apps}. The social network Foursquare
\href{https://venturebeat.com/2018/10/02/foursquare-raises-33-million-for-ad-and-location-analytics/}{remade
itself} as a location marketing company. Prominent investors in location
start-ups include
\href{https://www.businesswire.com/news/home/20180518005397/en/Cuebiq-Raises-27-Million-Growth-Capital}{Goldman
Sachs} and
\href{https://blog.safegraph.com/safegraph-raises-16-million-series-a-e8e88eeb7beb}{Peter
Thiel}, the PayPal co-founder.

Businesses say their interest is in the patterns, not the identities,
that the data reveals about consumers. They note that the information
apps collect is tied not to someone's name or phone number but to a
unique ID. But those with access to the raw data --- including employees
or clients --- could still identify a person without consent. They could
follow someone they knew, by pinpointing a phone that regularly spent
time at that person's home address. Or, working in reverse, they could
attach a name to an anonymous dot, by seeing where the device spent
nights and using public records to figure out who lived there.

Many location companies say that when phone users enable location
services, their data is fair game. But, The Times found, the
explanations people see when prompted to give permission are often
incomplete or misleading. An app may tell users that granting access to
their location will help them get traffic information, but not mention
that the data will be shared and sold. That disclosure is often buried
in a vague privacy policy.

``Location information can reveal some of the most intimate details of a
person's life --- whether you've visited a psychiatrist, whether you
went to an A.A. meeting, who you might date,'' said Senator Ron Wyden,
Democrat of Oregon, who has proposed bills to limit the collection and
sale of such data, which are largely unregulated in the United States.

``It's not right to have consumers kept in the dark about how their data
is sold and shared and then leave them unable to do anything about it,''
he added.

\hypertarget{mobile-surveillance-devices}{%
\paragraph{Mobile Surveillance
Devices}\label{mobile-surveillance-devices}}

After Elise Lee, a nurse in Manhattan, saw that her device had been
tracked to the main operating room at the hospital where she works, she
expressed concern about her privacy and that of her patients.

``It's very scary,'' said Ms. Lee, who allowed The Times to examine her
location history in the data set it reviewed. ``It feels like someone is
following me, personally.''

The mobile location industry began as a way to customize apps and target
ads for nearby businesses, but it has morphed into a data collection and
analysis machine.

Retailers look to tracking companies to tell them about their own
customers and their competitors'. For
\href{https://www.youtube.com/watch?v=BVZc86CZovU}{a web seminar} last
year, Elina Greenstein, an executive at the location company
GroundTruth, mapped out the path of a hypothetical consumer from home to
work to show potential clients how tracking could reveal a person's
preferences. For example, someone may search online for healthy recipes,
but GroundTruth can see that the person often eats at fast-food
restaurants.

``We look to understand who a person is, based on where they've been and
where they're going, in order to influence what they're going to do
next,'' Ms. Greenstein said.

Financial firms can use the information to make investment decisions
before a company reports earnings --- seeing, for example, if more
people are working on a factory floor, or going to a retailer's stores.

\hypertarget{planned-parenthood}{%
\subsection{planned parenthood}\label{planned-parenthood}}

By Michael H. Keller \textbar{} Imagery by Google Earth

Health care facilities are among the more enticing but troubling areas
for tracking, as Ms. Lee's reaction demonstrated. Tell All Digital, a
Long Island advertising firm that is a client of a location company,
says it runs
\href{https://www.npr.org/sections/health-shots/2018/05/25/613127311/digital-ambulance-chasers-law-firms-send-ads-to-patients-phones-inside-ers}{ad
campaigns} for personal injury lawyers targeting people anonymously in
emergency rooms.

``The book `1984,' we're kind of living it in a lot of ways,'' said Bill
Kakis, a managing partner at Tell All.

Jails, schools, a military base and a nuclear power plant --- even crime
scenes --- appeared in the data set The Times reviewed. One person,
perhaps a detective, arrived at the site of a late-night homicide in
Manhattan, then spent time at a nearby hospital, returning repeatedly to
the local police station.

Two location firms, Fysical and
\href{https://blog.safegraph.com/inauguration-attendees-make-significantly-less-money-than-womens-march-attendees-7cb8b056556a}{SafeGraph},
mapped people attending the 2017 presidential inauguration.
\href{https://medium.com/fysicalblog/the-2017-womens-march-in-washington-had-3x-the-attendance-as-trump-s-inauguration-267fc7f6968c}{On
Fysical's map}, a bright red box near the Capitol steps indicated the
general location of President Trump and those around him, cellphones
pinging away. Fysical's chief executive said in an email that the data
it used was anonymous. SafeGraph did not respond to requests for
comment.

\hypertarget{school}{%
\subsection{school}\label{school}}

By Michael H. Keller \textbar{} Imagery by Google Earth

More than 1,000 popular apps contain location-sharing code from such
companies, according to 2018 data from
\href{https://mightysignal.com/}{MightySignal}, a mobile analysis firm.
Google's Android system was found to have about 1,200 apps with such
code, compared with about 200 on Apple's iOS.

The most prolific company was Reveal Mobile, based in North Carolina,
which had location-gathering code in more than 500 apps, including many
that provide local news. A Reveal spokesman said that the popularity of
its code showed that it helped app developers make ad money and
consumers get free services.

To evaluate location-sharing practices, The Times tested 20 apps, most
of which had been flagged by
\href{https://www.appcensus.mobi/}{researchers} and industry insiders as
potentially sharing the data. Together, 17 of the apps sent exact
latitude and longitude to about 70 businesses. Precise location data
from one app, WeatherBug on iOS, was received by 40 companies. When
contacted by The Times, some of the companies that received that data
described it as ``unsolicited'' or ``inappropriate.''

WeatherBug, owned by GroundTruth, asks users' permission to collect
their location and tells them the information will be used to
personalize ads. GroundTruth said that it typically sent the data to ad
companies it worked with, but that if they didn't want the information
they could ask to stop receiving it.

\hypertarget{mayor}{%
\subsection{mayor}\label{mayor}}

By Michael H. Keller \textbar{} Satellite imagery by Mapbox and
DigitalGlobe

The Times also identified more than 25 other companies that have said in
marketing materials or interviews that they sell location data or
services, including targeted advertising.

\emph{{[}Read more about
\href{https://www.nytimes.com/2018/12/10/technology/location-tracking-apps-privacy.html?action=click\&module=Intentional\&pgtype=Article}{how
The Times analyzed location tracking companies}.{]}}

The spread of this information raises questions about how securely it is
handled and whether it is vulnerable to hacking, said Serge Egelman, a
computer security and privacy researcher affiliated with the University
of California, Berkeley.

``There are really no consequences'' for companies that don't protect
the data, he said, ``other than bad press that gets forgotten about.''

\hypertarget{a-question-of-awareness}{%
\paragraph{A Question of Awareness}\label{a-question-of-awareness}}

Companies that use location data say that people agree to share their
information in exchange for customized services, rewards and discounts.
Ms. Magrin, the teacher, noted that she liked that tracking technology
let her record her jogging routes.

Brian Wong, chief executive of Kiip, a mobile ad firm that has also sold
anonymous data from some of the apps it works with, says users give apps
permission to use and share their data. ``You are receiving these
services for free because advertisers are helping monetize and pay for
it,'' he said, adding, ``You would have to be pretty oblivious if you
are not aware that this is going on.''

But Ms. Lee, the nurse, had a different view. ``I guess that's what they
have to tell themselves,'' she said of the companies. ``But come on.''

Ms. Lee had given apps on her iPhone access to her location only for
certain purposes --- helping her find parking spaces, sending her
weather alerts --- and only if they did not indicate that the
information would be used for anything else, she said. Ms. Magrin had
allowed about a dozen apps on her Android phone access to her
whereabouts for services like traffic notifications.

But it is easy to share information without realizing it. Of the 17 apps
that The Times saw sending precise location data, just three on iOS and
one on Android told users in a prompt during the permission process that
the information could be used for advertising. Only one app, GasBuddy,
which identifies nearby gas stations, indicated that data could also be
shared to ``analyze industry trends.''

More typical was theScore, a sports app: When prompting users to grant
access to their location, it said the data would help ``recommend local
teams and players that are relevant to you.'' The app passed precise
coordinates to 16 advertising and location companies.

A spokesman for theScore said that the language in the prompt was
intended only as a ``quick introduction to certain key product
features'' and that the full uses of the data were described in the
app's privacy policy.

The Weather Channel app, owned by an IBM subsidiary, told users that
sharing their locations would let them get personalized local weather
reports. IBM said the subsidiary, the Weather Company, discussed other
uses in its privacy policy and in a separate ``privacy settings''
section of the app. Information on advertising was included there, but a
part of the app called ``location settings'' made no mention of it.

The app did not explicitly disclose that the company had also analyzed
the data for hedge funds --- a pilot program that was promoted on the
company's website. An IBM spokesman said the pilot had ended. (IBM
\href{https://weather.com/en-US/twc/privacy-policy}{updated the app's
privacy policy} on Dec. 5, after queries from The Times, to say that it
might share aggregated location data for commercial purposes such as
analyzing foot traffic.)

Even industry insiders acknowledge that many people either don't read
those policies or may not fully understand their opaque language.
Policies for apps that funnel location information to help investment
firms, for instance, have said the data is used for market analysis, or
simply shared for business purposes.

``Most people don't know what's going on,'' said Emmett Kilduff, the
chief executive of Eagle Alpha, which sells data to financial firms and
hedge funds. Mr. Kilduff said responsibility for complying with
data-gathering regulations fell to the companies that collected it from
people.

Many location companies say they voluntarily take steps to protect
users' privacy, but policies vary widely.

For example, Sense360, which focuses on the restaurant industry, says it
\href{https://sense360.com/portfolio/additional-information-protect-user-privacy/}{scrambles
data} within a 1,000-foot square around the device's approximate home
location. Another company, Factual, says that it collects data from
consumers at home, but that its database doesn't contain their
addresses.

\hypertarget{diptych}{%
\subsection{diptych}\label{diptych}}

By Michael H. Keller \textbar{} Satellite imagery by Mapbox and
DigitalGlobe

Some companies say they delete the location data after using it to serve
ads, some use it for ads and pass it along to data aggregation
companies, and others keep the information for years.

Several people in the location business said that it would be relatively
simple to figure out individual identities in this kind of data, but
that they didn't do it. Others suggested it would require so much effort
that hackers wouldn't bother.

It ``would take an enormous amount of resources,'' said Bill Daddi, a
spokesman for Cuebiq, which analyzes anonymous location data to help
retailers and others, and raised more than \$27 million this year from
investors including Goldman Sachs and Nasdaq Ventures. Nevertheless,
Cuebiq encrypts its information, logs employee queries and sells
aggregated analysis, he said.

There is no federal law limiting the collection or use of such data.
Still, apps that ask for access to users' locations, prompting them for
permission while leaving out important details about how the data will
be used, may run afoul of federal rules on deceptive business practices,
said Maneesha Mithal, a privacy official at the Federal Trade
Commission.

``You can't cure a misleading just-in-time disclosure with information
in a privacy policy,'' Ms. Mithal said.

\hypertarget{following-the-money}{%
\paragraph{Following the Money}\label{following-the-money}}

Apps form the backbone of this new location data economy.

The app developers can make money by directly selling their data, or by
sharing it for location-based ads, which command a premium. Location
data companies pay half a cent to two cents per user per month,
according to offer letters to app makers reviewed by The Times.

Targeted advertising is by far the most common use of the information.

Google and Facebook, which dominate the mobile ad market, also lead in
location-based advertising. Both companies collect the data from their
own apps. They say they don't sell it but keep it for themselves to
personalize their services, sell targeted ads across the internet and
track whether the ads lead to sales at brick-and-mortar stores. Google,
which also receives precise location information from apps that use its
ad services, said it modified that data to make it less exact.

Smaller companies compete for the rest of the market, including by
selling data and analysis to financial institutions. This segment of the
industry is small but growing, expected to reach about \$250 million a
year by 2020, according to the market research firm Opimas.

Apple and Google have a financial interest in keeping developers happy,
but both have taken steps to limit location data collection. In the most
recent version of Android, apps that are not in use can collect
locations ``a few times an hour,'' instead of continuously.

Apple has been stricter, for example requiring apps to justify
collecting location details in pop-up messages. But Apple's
\href{https://developer.apple.com/design/human-interface-guidelines/ios/app-architecture/requesting-permission/}{instructions}
for writing these pop-ups do not mention advertising or data sale, only
features like getting ``estimated travel times.''

A spokesman said the company mandates that developers use the data only
to provide a service directly relevant to the app, or to serve
advertising that met Apple's guidelines.

Apple recently shelved plans that industry insiders say would have
significantly curtailed location collection. Last year, the company said
an upcoming version of iOS would show a blue bar onscreen whenever an
app not in use was gaining access to location data.

The discussion served as a ``warning shot'' to people in the location
industry, David Shim, chief executive of the location company
\href{https://www.placed.com/}{Placed}, said at
\href{https://www.youtube.com/watch?v=ExDwELoLAJ8}{an industry event}
last year.

After examining maps showing the locations extracted by their apps, Ms.
Lee, the nurse, and Ms. Magrin, the teacher, immediately limited what
data those apps could get. Ms. Lee said she told the other
operating-room nurses to do the same.

``I went through all their phones and just told them: `You have to turn
this off. You have to delete this,''' Ms. Lee said. ``Nobody knew.''

Adam Satariano contributed reporting.

\hypertarget{related-coverage}{%
\subsection{Related Coverage}\label{related-coverage}}

\begin{itemize}
\tightlist
\item
  \href{undefined/interactive/2018/09/12/technology/kids-apps-data-privacy-google-twitter.html}{}
\item
  \href{https://www.nytimes.com/2018/05/10/technology/cellphone-tracking-law-enforcement.html}{}
\item
  \href{https://www.nytimes.com/2018/05/19/technology/phone-apps-stalking.html}{}
\end{itemize}

2018

\hypertarget{more-on-nytimescom}{%
\subsection{More on NYTimes.com}\label{more-on-nytimescom}}

Advertisement

\hypertarget{site-information-navigation}{%
\subsection{Site Information
Navigation}\label{site-information-navigation}}

\begin{itemize}
\tightlist
\item
  \href{https://help.nytimes.com/hc/en-us/articles/115014792127-Copyright-notice}{©
  2020 The New York Times Company}
\item
  \href{https://www.nytimes.com}{Home}
\item
  \href{https://www.nytimes.com/search/}{Search}
\item
  Accessibility concerns? Email us at
  \href{mailto:accessibility@nytimes.com}{\nolinkurl{accessibility@nytimes.com}}.
  We would love to hear from you.
\item
  \href{https://help.nytimes.com/hc/en-us/articles/115015385887-Contact-Us}{Contact
  Us}
\item
  \href{https://www.nytco.com/careers/}{Work with us}
\item
  \href{https://nytmediakit.com/}{Advertise}
\item
  \href{https://help.nytimes.com/hc/en-us/articles/115014892108-Privacy-policy\#pp}{Your
  Ad Choices}
\item
  \href{https://help.nytimes.com/hc/en-us/articles/115014892108-Privacy-policy}{Privacy}
\item
  \href{https://help.nytimes.com/hc/en-us/articles/115014893428-Terms-of-service}{Terms
  of Service}
\item
  \href{https://help.nytimes.com/hc/en-us/articles/115014893968-Terms-of-sale}{Terms
  of Sale}
\end{itemize}

\hypertarget{site-information-navigation-1}{%
\subsection{Site Information
Navigation}\label{site-information-navigation-1}}

\begin{itemize}
\tightlist
\item
  \href{https://spiderbites.nytimes.com}{Site Map}
\item
  \href{https://help.nytimes.com/hc/en-us}{Help}
\item
  \href{https://help.nytimes.com/hc/en-us/articles/115015385887-Contact-Us?redir=myacc}{Site
  Feedback}
\item
  \href{https://www.nytimes.com/subscription?campaignId=37WXW}{Subscriptions}
\end{itemize}
