 **NYTimes.com no longer supports Internet Explorer 9 or earlier. Please
upgrade your browser.
\href{http://www.nytimes.com/content/help/site/ie9-support.html}{LEARN
MORE »}

**Sections

**Home

**Search

\hypertarget{the-new-york-times}{%
\subsection{\texorpdfstring{\href{http://www.nytimes.com/}{The New York
Times}}{The New York Times}}\label{the-new-york-times}}

 \href{https://www.nytimes.com/section/us}{U.S.} \textbar{}Tracking the
Events in the Wake of Michael Brown's Shooting

Log In

**0

**Settings

**Close search

\hypertarget{site-search-navigation}{%
\subsection{Site Search Navigation}\label{site-search-navigation}}

Search NYTimes.com

**Clear this text input

Go

\url{https://nyti.ms/1Emw1Um}

\begin{enumerate}
\def\labelenumi{\arabic{enumi}.}
\item
  Loading...
\end{enumerate}

See next articles

See previous articles

\hypertarget{site-navigation}{%
\subsection{Site Navigation}\label{site-navigation}}

\hypertarget{site-mobile-navigation}{%
\subsection{Site Mobile Navigation}\label{site-mobile-navigation}}

Advertisement

\hypertarget{-us-}{%
\subsection{\texorpdfstring{
\href{https://www.nytimes.com/section/us}{U.S.} }{ U.S. }}\label{-us-}}

\hypertarget{tracking-the-events-in-the-wake-of-michael-browns-shooting}{%
\section{Tracking the Events in the Wake of Michael Brown's
Shooting}\label{tracking-the-events-in-the-wake-of-michael-browns-shooting}}

UPDATED Nov. 24, 2014

Updates on the events in Ferguson, Mo., following the shooting of
Michael Brown, an unarmed teenager, by a police officer on Aug. 9.

Michael Brown Is Shot Michael Brown, 18, is shot and killed by a police
officer in Ferguson, Mo. According to reports, Mr. Brown was walking
down the middle of Canfield Drive with a friend, Dorian Johnson, when
the officer stopped his Chevy Tahoe to order them to the sidewalk.
Within seconds, the encounter turned into a physical struggle, as the
officer and Mr. Brown became entangled through the open driver-side
window of the police vehicle. How that encounter began is in dispute,
though most accounts agree that shots were fired while the officer was
in the vehicle. At some point, Mr. Brown broke away. The officer then
got out of the vehicle and fired at Mr. Brown, whose actions at the
point are also in dispute. Some witnesses later said that Mr. Brown
appeared to be surrendering with his hands in the air as he was hit with
the fatal gunshots. Others say that Mr. Brown was moving toward the
officer when he was killed. What is not in dispute is that Mr. Brown was
unarmed. His body would lie in the street for four hours. Angry
Residents Take to the Streets Residents outraged by the shooting take to
the streets, and a portion of West Florissant Avenue becomes a staging
area for protests. ``Hands up, don't shoot'' and ``No justice, no
peace'' become rallying cries. As early protests turn increasingly
violent, the police respond with heavy-handed tactics -- including
military-style weapons and equipment -\/- that seem only to ratchet up
the unrest. Eventually, both sides would make efforts to reduce the
tension of the protests, which continued daily. F.B.I. Opens Civil
Rights Investigation The Federal Bureau of Investigation opens a civil
rights inquiry into the shooting of Mr. Brown. A State Trooper Steps In
Hours after President Obama denounces the actions of both police and
protesters in Ferguson, Gov. Jay Nixon orders the Missouri State Highway
Patrol to take over security operations. Alarm has been rising across
the country at images of a mostly white police force, in a predominantly
African-American community, aiming military-style weapons at protesters
and firing tear gas and rubber bullets. Appointed by Mr. Nixon, Capt.
Ronald S. Johnson of the highway patrol immediately signals a change in
approach. Troopers are ordered to remove tear-gas masks while armored
vehicles and police cars are taken away. The tactics work for a short
time before unrest returns. A curfew is later imposed. At times,
Governor Nixon is jeered or shouted down as he tries to reassure
residents and urge an end to the violence. Officer Involved in Shooting
Is Identified Almost a week after the shooting of Mr. Brown, the officer
who shot him is identified as Darren Wilson, who has five years of
police experience. The release of the name is followed by series of
incomplete accounts by Thomas Jackson, the Ferguson police chief. These
accounts sowed confusion about whether Officer Wilson knew that the
teenager was a suspect in a robbery at a local convenience store that
took place moments before the shooting. Brown's Family Releases Autopsy
Details A preliminary private autopsy
\href{http://stlouis.cbslocal.com/2014/08/17/report-michael-brown-autopsy-shows-teen-was-struck-at-least-6-times/}{shows
that Mr. Brown was shot at least six times,} including twice in the
head. Dr. Michael M. Baden, a former chief New York City medical
examiner who conducted the autopsy for the family, says one bullet
entered the top of Mr. Brown's skull, suggesting that his head was bent
forward when it struck him and caused a fatal injury. Dr. Baden says Mr.
Brown was also shot four times in the right arm, and that all the
bullets were fired into his front. This is the first time that some of
the critical information resulting in Mr. Brown's death has been made
public, but the release of the preliminary autopsy results does little
to explain the circumstances surrounding the shooting. National Guard Is
Ordered to Ferguson After a curfew fails to quiet the streets, Governor
Nixon brings in the National Guard, though in the limited role of
protecting the police command post. Hours later, he lifts the curfew.
The National Guard is ordered to withdraw in four days. Accounts of
Shooting Differ As a county grand jury prepared to hear evidence,
witnesses interviewed by investigators provide sharply conflicting
accounts of the shooting. Some seem to agree on how the fatal
altercation initially unfolded: with a struggle between Officer Wilson
and Mr. Brown. Officer Wilson was inside his patrol car at the time,
while Mr. Brown, who was unarmed, was leaning in through an open window.
Many witnesses also agreed on what happened next: Officer Wilson's
firearm went off inside the car, Mr. Brown ran away, the officer got out
of his car and began firing toward Mr. Brown, and then Mr. Brown stopped
and turned to face the officer. But accounts of the crucial moments that
followed differ sharply. Some witnesses say that Mr. Brown moved toward
Officer Wilson, possibly in a threatening manner, when he was shot.
Others say that Mr. Brown was not moving and that he might have even had
his hands up. Protesters Descend on Ferguson Driven in part by posts on
Twitter and other social media outlets, protesters from across the
country descend on Ferguson, transforming a purely local protest into a
center of national activism. The new protesters include rap and hip-hop
stars as well as veterans of the Occupy Wall Street movement. While they
are welcomed by some in Ferguson, others are deeply suspicious of their
motives and question their behavior. Attorney General Arrives and
Promises Full Inquiry Promising a full and fair investigation, Attorney
General Eric H. Holder Jr. arrives in St. Louis to meet with community
leaders and federal investigators. While in Missouri, he tries to
reassure Ferguson residents about the investigation into Michael Brown's
death and says he understands why many black Americans do not trust the
police. While he promises a full inquiry, Mr. Holder also tries to
temper expectations that charges will be filed. Mourning and Calls for
Action at Brown's Funeral Thousands pay their respects to Mr. Brown.
Infused with Scripture and song, the funeral is a mix of intimate
reflections and national policy plans. Relatives reminisce in choked
voices about Mr. Brown's wide smile as a picture from his high school
graduation flashes on two wide screens, as leaders urge those gathered
to memorialize his life by carrying forward a vocal, strong and unified
effort to seek change across the country. Justice Department Opens
Inquiry The Justice Department announces that it will open a broad civil
rights investigation that will examine whether the Ferguson police have
a history of discrimination or misuse of force beyond the Michael Brown
case. The inquiry is in addition to the F.B.I. civil rights
investigation that is looking specifically into the shooting of Mr.
Brown. Ferguson Leaders Try to Reach Out Responding to complaints that
the Ferguson police are out of touch with the African-American
community, the City Council agrees to establish a citizen review board
to provide guidance. In addition, the council announces sweeping changes
to its court system, which had been criticized as unfairly targeting
low-income blacks, who had become trapped in a cycle of unpaid tickets
and arrest warrants. Young black men in Ferguson and surrounding cities
routinely find themselves passed from jail to jail as they are picked up
on warrants for unpaid fines, one of the many simmering issues in the
city. Ferguson Police Chief Apologizes Thomas Jackson, the Ferguson
police chief, issues a stark apology to the family of Michael Brown,
saying in a \href{http://vimeo.com/107139488}{videotaped statement} that
he was sorry for the death of Mr. Brown and for the four hours that his
body lay in the street after he was fatally shot. ``I want to say this
to the Brown family. No one who has not experienced the loss of a child
can understand what you're feeling,'' he said, facing the camera and
standing in front of an American flag. ``I am truly sorry for the loss
of your son. I'm also sorry that it took so long to remove Michael from
the street. The time that it took involved very important work on the
part of investigators who were trying to collect evidence and gain a
true picture of what happened that day. But it was just too long, and
I'm truly sorry for that.'' A Shift in Police Oversight of Protests The
St. Louis County Police Department takes control of security surrounding
protests in Ferguson, Mo. Chief Thomas Jackson of the Ferguson Police
Department asked the county to step in, citing a ``lack of manpower and
resources'' at the disposal of the relatively small Ferguson police
force, said Brian Schellman, a spokesman for the county police. The
Ferguson police had been criticized for their heavy-handed tactics in
dealing with protesters. Call for Review of Police Tactics and Training
Speaking to mayors and police chiefs gathered in Little Rock, Ark.,
Attorney General Eric H. Holder Jr. says the Justice Department is
working with major police associations to conduct a broad review of
policing tactics, techniques and training. The review is intended to
``help the field swiftly confront emerging threats, better address
persistent challenges, and thoroughly examine the latest tools and
technologies to enhance the safety and the effectiveness of law
enforcement.'' Criticism of the Use of Military-Style Equipment by
Police Images of violent clashes between officers in full body armor,
with military-style equipment, pointing guns at residents in Ferguson
protesting the shooting of Michael Brown, leads to calls for the
demilitarization of local police forces around the country. At a Senate
hearing in Washington, the Department of Homeland Security tried to ease
criticism of the program by reminding lawmakers that the use of the
equipment had been instrumental in the capture of suspects after the
2013 Boston Marathon bombing. A 'Weekend of Resistance' Thousands of
people take part in events around the St. Louis area to protest the
killing of Mr. Brown and to raise awareness of police treatment of
African-Americans. During one event, some younger protesters, part of a
group that had appeared night after night for sometimes rowdy protests,
called out to the older faces on the stage, criticizing older activists
for not being radical enough. During one protest, Cornell William
Brooks, the president of the N.A.A.C.P., and the professor and author
Cornel West were among 43 people arrested outside the Ferguson Police
Department. Policeman Offers His Account Police Officer Darren Wilson,
who fatally shot Mr. Brown, tells investigators that he was pinned in
his vehicle and in fear for his life as he struggled over his gun with
Mr. Brown, said government officials briefed on the federal civil rights
investigation. Officer Wilson, in the first public account of his
testimony, says that Mr. Brown reached for the gun during a scuffle. The
gun was fired twice in the car. Officer Wilson told the authorities that
Mr. Brown had punched and scratched him repeatedly, leaving swelling on
his face and cuts on his neck. The officer's version contradicts some
witness accounts, and does not explain why, after he emerged from his
vehicle, he fired at Mr. Brown multiple times. Governor Activates
National Guard Gov. Jay Nixon declares a state of emergency, allowing
him to activate the Missouri National Guard in preparation for the grand
jury's decision on whether to indict Officer Wilson. He also mobilized
the National Guard in August. The declaration adds to the mounting
tension over the announcement of the decision, which officials maintain
is expected in mid- to late November. At this point, many in Ferguson
say they expect the grand jury to decide against indicting the Ferguson
officer, Darren Wilson, and they anticipate a show of anger and protest
afterward. Some protesters say that calling up the Guard before a grand
jury decision was an antagonistic move that presumed that demonstrations
would be violent. ``My hope and expectation is that peace will
prevail,'' Mr. Nixon said after announcing the decision. ``But I have a
responsibility to plan for any contingency that might arise.'' No
Charges Against Ferguson Officer in Michael Brown Shooting The grand
jury decided to not indict Darren Wilson. Protesters have been
mobilizing for weeks and the St. Louis area in general has been cloaked
in anxiety for months as it has waited for a decision by the grand jury,
which is made up of nine whites and three blacks.

\hypertarget{more-on-nytimescom}{%
\subsection{More on NYTimes.com}\label{more-on-nytimescom}}

Advertisement

\hypertarget{site-information-navigation}{%
\subsection{Site Information
Navigation}\label{site-information-navigation}}

\begin{itemize}
\tightlist
\item
  \href{https://help.nytimes.com/hc/en-us/articles/115014792127-Copyright-notice}{©
  2020 The New York Times Company}
\item
  \href{https://www.nytimes.com}{Home}
\item
  \href{https://www.nytimes.com/search/}{Search}
\item
  Accessibility concerns? Email us at
  \href{mailto:accessibility@nytimes.com}{\nolinkurl{accessibility@nytimes.com}}.
  We would love to hear from you.
\item
  \href{https://help.nytimes.com/hc/en-us/articles/115015385887-Contact-Us}{Contact
  Us}
\item
  \href{https://www.nytco.com/careers/}{Work with us}
\item
  \href{https://nytmediakit.com/}{Advertise}
\item
  \href{https://help.nytimes.com/hc/en-us/articles/115014892108-Privacy-policy\#pp}{Your
  Ad Choices}
\item
  \href{https://help.nytimes.com/hc/en-us/articles/115014892108-Privacy-policy}{Privacy}
\item
  \href{https://help.nytimes.com/hc/en-us/articles/115014893428-Terms-of-service}{Terms
  of Service}
\item
  \href{https://help.nytimes.com/hc/en-us/articles/115014893968-Terms-of-sale}{Terms
  of Sale}
\end{itemize}

\hypertarget{site-information-navigation-1}{%
\subsection{Site Information
Navigation}\label{site-information-navigation-1}}

\begin{itemize}
\tightlist
\item
  \href{https://spiderbites.nytimes.com}{Site Map}
\item
  \href{https://help.nytimes.com/hc/en-us}{Help}
\item
  \href{https://help.nytimes.com/hc/en-us/articles/115015385887-Contact-Us?redir=myacc}{Site
  Feedback}
\item
  \href{https://www.nytimes.com/subscription?campaignId=37WXW}{Subscriptions}
\end{itemize}
