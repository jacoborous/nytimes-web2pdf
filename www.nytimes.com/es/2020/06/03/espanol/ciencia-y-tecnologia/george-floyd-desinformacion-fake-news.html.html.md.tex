Sections

SEARCH

\protect\hyperlink{site-content}{Skip to
content}\protect\hyperlink{site-index}{Skip to site index}

\href{https://www.nytimes.com/es/section/ciencia-y-tecnologia}{Ciencia y
Tecnología}

\href{https://myaccount.nytimes.com/auth/login?response_type=cookie\&client_id=vi}{}

\href{https://www.nytimes.com/section/todayspaper}{Today's Paper}

\href{/es/section/ciencia-y-tecnologia}{Ciencia y
Tecnología}\textbar{}La desinformación sobre las protestas por George
Floyd se propaga en las redes sociales

\url{https://nyti.ms/3csfk1K}

\begin{itemize}
\item
\item
\item
\item
\item
\end{itemize}

\href{https://www.nytimes.com/news-event/george-floyd-protests-minneapolis-new-york-los-angeles?action=click\&pgtype=Article\&state=default\&region=TOP_BANNER\&context=storylines_menu}{Race
and America}

\begin{itemize}
\tightlist
\item
  \href{https://www.nytimes.com/2020/07/26/us/protests-portland-seattle-trump.html?action=click\&pgtype=Article\&state=default\&region=TOP_BANNER\&context=storylines_menu}{Protesters
  Return to Other Cities}
\item
  \href{https://www.nytimes.com/2020/07/24/us/portland-oregon-protests-white-race.html?action=click\&pgtype=Article\&state=default\&region=TOP_BANNER\&context=storylines_menu}{Portland
  at the Center}
\item
  \href{https://www.nytimes.com/2020/07/23/podcasts/the-daily/portland-protests.html?action=click\&pgtype=Article\&state=default\&region=TOP_BANNER\&context=storylines_menu}{Podcast:
  Showdown in Portland}
\item
  \href{https://www.nytimes.com/interactive/2020/07/16/us/black-lives-matter-protests-louisville-breonna-taylor.html?action=click\&pgtype=Article\&state=default\&region=TOP_BANNER\&context=storylines_menu}{45
  Days in Louisville}
\end{itemize}

Advertisement

\protect\hyperlink{after-top}{Continue reading the main story}

Supported by

\protect\hyperlink{after-sponsor}{Continue reading the main story}

Tecnología

\hypertarget{la-desinformaciuxf3n-sobre-las-protestas-por-george-floyd-se-propaga-en-las-redes-sociales}{%
\section{La desinformación sobre las protestas por George Floyd se
propaga en las redes
sociales}\label{la-desinformaciuxf3n-sobre-las-protestas-por-george-floyd-se-propaga-en-las-redes-sociales}}

En el universo de la información falsa en línea, Floyd sigue vivo y
George Soros es el culpable por las protestas.

\includegraphics{https://static01.nyt.com/images/2020/06/01/business/03unrest-disinfo-ES/merlin_173069919_3a77e392-c604-4c90-b4fd-c76e3463c034-articleLarge.jpg?quality=75\&auto=webp\&disable=upscale}

Por \href{https://www.nytimes.com/by/davey-alba}{Davey Alba}

\begin{itemize}
\item
  3 de junio de 2020
\item
  \begin{itemize}
  \item
  \item
  \item
  \item
  \item
  \end{itemize}
\end{itemize}

\href{https://www.nytimes.com/2020/06/01/technology/george-floyd-misinformation-online.html}{Read
in English}

\href{https://www.nytimes.com/newsletters/el-times}{Regístrate para
recibir nuestro boletín} con lo mejor de The New York Times.

\begin{center}\rule{0.5\linewidth}{\linethickness}\end{center}

\emph{{[}}\href{https://www.nytimes.com/2020/06/02/us/protests-today-george-floyd.html}{\emph{Sigue
nuestras actualizaciones en vivo de las protestas por George Floyd en
Estados Unidos}}\emph{.{]}}

En Twitter y en Facebook circulan cientos de publicaciones que dicen que
George Floyd en realidad no está muerto.

Los creadores de conspiraciones están argumentando sin fundamentos que
George Soros, el inversionista multimillonario y donante del partido
demócrata, está financiando las
\href{https://www.nytimes.com/live/2020/george-floyd-protests-today-06-01}{extensas
manifestacione}s en contra de la brutalidad policiaca.

Además, los comentaristas conservadores aseguran con pocas pruebas que
\href{https://www.nytimes.com/es/2020/06/02/espanol/mundo/que-es-antifa.html}{antifa,
el movimiento activista antifascismo de extrema izquierda}, coordinó los
disturbios y saqueos que surgieron de las manifestaciones.

Las mentiras, teorías de conspiración y otra información falsa se desata
en línea conforme se ha acumulado el furor en torno a Floyd, un hombre
negro que
\href{https://www.nytimes.com/2020/05/31/us/george-floyd-investigation.html}{fue
asesinado la semana pasada bajo custodia de la policía} de Mineápolis.
La desinformación ha aparecido conforme las manifestaciones dominan el
diálogo, superando por mucho el volumen de las publicaciones en línea y
las menciones en los medios acerca de las
\href{https://www.nytimes.com/news-event/hong-kong-protests}{manifestaciones
del año pasado en Hong Kong} y el
\href{https://www.nytimes.com/2019/04/15/business/yellow-vests-movement-inequality.html}{movimiento
de los chalecos amarillos} en Francia, de acuerdo con la compañía de
análisis de medios Zignal Labs.

En su punto álgido el viernes, Floyd y las manifestaciones en torno a su
muerte se mencionaron 8,8 millones de veces, dijo Zignal Labs, que
analizó transmisiones globales por televisión y redes sociales. En
contraste, las noticias de las manifestaciones de Hong Kong alcanzaron
1,5 millones de menciones al día y el movimiento de los chalecos
amarillos, 941.000.

``La combinación de sucesos cambiantes, la atención continua y, sobre
todo, las profundas divisiones existentes hacen de este movimiento una
tormenta perfecta para la desinformación'', dijo Graham Brookie,
director del Laboratorio de Investigación Forense Digital del Consejo
Atlántico. ``Toda esa desinformación es tóxica y provoca que nuestros
desafíos y divisiones, tan reales, sean más difíciles de abordar''.

El choque de tensiones raciales y la polarización política durante
l\href{https://www.nytimes.com/news-event/coronavirus?action=click\&pgtype=Article\&state=default\&module=styln-coronavirus\&variant=show\&region=TOP_BANNER\&context=storylines_menu}{a
pandemia del coronavirus} han aumentado el volumen de la desinformación,
dijeron investigadores. Gran parte de la desinformación está siendo
compartida por el grupo conspiracionista QAnon y comentaristas de
extrema derecha, así como por simpatizantes de la izquierda, dijo
Brookie.

El presidente Donald Trump ha atizado la información divisiva. A lo
largo de los últimos días, publicó en Twitter que
\href{https://www.nytimes.com/reuters/2020/05/31/us/31reuters-minneapolis-police-trump-antifa.html}{antifa
era una ``organización terrorista''} y animó a los ciudadanos a
\href{https://www.nytimes.com/2020/05/30/us/politics/trump-threatens-protesters-dogs-weapons.html}{asistir
a una contramanifestación que llamó ``Noche de MAGA} (Hagamos a Estados
Unidos grandioso de nuevo, por su sigla en inglés)'' en la Casa Blanca.

Además, las personas están experimentando altos niveles de temor,
incertidumbre e indignación, dijo Claire Wardle, directora ejecutiva de
First Draft, una organización que combate la desinformación en línea.
Eso crea ``el peor contexto posible para un entorno informativo
saludable'', dijo.

Twitter y Facebook no hicieron comentarios de inmediato.

A continuación tres categorías importantes de mentiras que han surgido
en las plataformas de redes sociales sobre la muerte de Floyd y las
manifestaciones.

\hypertarget{la-muerte-falsa-de-george-floyd}{%
\subsection{La muerte `falsa' de George
Floyd}\label{la-muerte-falsa-de-george-floyd}}

El rumor infundado de que Floyd está vivo es emblemático de la narrativa
de desinformación que asegura que un suceso noticioso fue montado. Esto
se ha vuelto cada vez más común a lo largo de los años, pues los
creadores de conspiraciones dicen, entre otros ejemplos, que el
alunizaje en 1969 y la masacre de 2012 en la Escuela Primaria Sandy Hook
fueron engaños.

El viernes, JonXArmy, un canal de teorías de conspiración en YouTube,
compartió un video de 22 minutos que aseguraba falsamente que la muerte
de Floyd había sido montada. El video fue compartido casi 100 veces en
Facebook, la mayoría en grupos
\href{https://www.nytimes.com/2020/02/09/us/politics/qanon-trump-conspiracy-theory.html}{dirigidos
por QAnon}, por lo que alcanzó a 1,3 millones de personas, de acuerdo
con datos de CrowdTangle, una herramienta que analiza interacciones en
las redes sociales.

Jon Miller, quien dirige el canal JonXArmy, no respondió de inmediato a
las solicitudes para hacer comentarios. YouTube dijo en su sitio que
había eliminado el video, con base en su política contra el discurso de
odio.

En Twitter, las publicaciones que decían ``George Floyd no está muerto''
también fueron tuiteadas cientos de veces a lo largo de la semana
pasada, y la frase tuvo su punto máximo de 15 menciones en un periodo de
diez minutos el lunes por la mañana, de acuerdo con Dataminr, un
servicio de monitoreo de redes sociales.

En miles de otras publicaciones en Facebook y Twitter, hubo gente que
afirmó falsamente que Derek Chauvin, el oficial de policía de Minnesota
que fue
\href{https://www.nytimes.com/2020/05/29/us/minneapolis-police-george-floyd.html}{acusado
de asesinato en tercer grado y homicidio involuntario en segundo grado}
por la muerte de Floyd, era un actor y que todo el incidente había sido
falsificado por el estado profundo.

\hypertarget{la-conspiraciuxf3n-de-george-soros}{%
\subsection{La conspiración de George
Soros}\label{la-conspiraciuxf3n-de-george-soros}}

La idea falsa de que Soros financió las manifestaciones aumentó en las
redes sociales a lo largo de la semana pasada, lo cual muestra la manera
en que los sucesos pueden revivir viejas teorías de conspiración.
Durante años, una red informal de activistas y personajes políticos de
la derecha ha presentado a Soros
\href{https://www.nytimes.com/2018/05/29/us/roseanne-george-soros-twitter.html}{como
un villano que se opone a los conservadores,} y se ha convertido en una
leyenda urbana conveniente para todo tipo de males.

En Twitter, Soros fue mencionado en 34.000 tuits relacionados con la
muerte de Floyd a lo largo de la semana pasada, de acuerdo con Dataminr.
Más de 90 videos en cinco idiomas que mencionan conspiraciones sobre
Soros también fueron publicados en YouTube a lo largo de los últimos
siete días, de acuerdo con un análisis de The New York Times.

En Facebook, 72.000 publicaciones mencionaron a Soros la semana pasada,
un aumento en comparación con las 12.600 de la semana previa, de acuerdo
con el análisis del Times. De las diez publicaciones con más
interacciones acerca de Soros en la red social, nueve tenían
conspiraciones falsas que lo vinculaban con los disturbios. Fueron
compartidas de manera colectiva más de 110.000 veces.

Dos de las primeras publicaciones de Facebook que compartieron las
conspiraciones sobre Soros fueron del comisionado de agricultura de
Texas, \href{https://www.facebook.com/MillerForTexas/}{Sid Miller}, un
abierto defensor de Trump.

``No tengo ninguna duda de que George Soros está financiando estas
llamadas protestas `espontáneas''', escribió Miller en una de sus
publicaciones. ``¡Soros es pura maldad y está empeñado en destruir
nuestro país!''.

Miller no respondió de inmediato a una solicitud de comentarios.

Farshad Shadloo, un portavoz de YouTube, dijo que los videos de
conspiración de Soros no violaban los lineamientos de la compañía, pero
que el sitio no los estaba recomendando.

Una portavoz de Soros dijo: ``Deploramos la idea falsa de que la gente
que sale a las calles para expresar su dolor ha recibido pagos por parte
de George Soros o de cualquier otra persona''.

\hypertarget{desinformaciuxf3n-antifa}{%
\subsection{Desinformación antifa}\label{desinformaciuxf3n-antifa}}

La teoría sin fundamentos de que los activistas antifa son responsables
de los disturbios y los saqueos fue el fragmento de desinformación más
grande acerca de las manifestaciones rastreado por Zignal Labs, que
analizó ciertas categorías de mentiras. De 873.000 piezas de
desinformación relacionadas con las manifestaciones, 575.800 fueron
menciones de antifa, dijo Zignal Labs.

La narrativa antifa ganó popularidad porque ``las redes establecidas
desde hace tiempo por influentes hiperpartidistas en las redes sociales
ahora trabajan en conjunto como una máquina bien lubricada'', dijo Erin
Gallagher, investigadora de redes sociales.

Eso comenzó cuando Trump tuiteó el domingo que ``los anarquistas
dirigidos por antifa'' y ``los anarquistas de la izquierda radical''
eran los responsables de los disturbios, sin proporcionar más detalles.
Después dijo que antifa era ``una organización terrorista''.

Dan Bongino, comentarista político conservador que sin éxito ha sido
candidato a la Cámara de Representantes varias veces, continuó con esta
narrativa. En el programa de televisión ``Fox and Friends'' el lunes,
Bongino dijo que los activistas antifa eran los responsables de un
ataque ``sofisticado'' contra la Casa Blanca y lo describió como una
``insurrección''.

No respondió de inmediato a una solicitud para hacer comentarios.

Esas afirmaciones poco después se propagaron en las redes sociales. Más
de 6000 publicaciones de Facebook que vinculan al movimiento antifa con
las manifestaciones aparecieron en los últimos siete días y han amasado
más de 1,3 millones de me gusta y de publicaciones compartidas, de
acuerdo con el análisis del Times.

Y en Twitter, un falso ``manual'' que especifica ``órdenes de
disturbios'' supuestamente emitido por los demócratas, que ordenaban a
activistas antifa que causaran problemas, circuló ampliamente. Pero el
llamado manual fue la resurrección de un viejo engaño vinculado a los
disturbios de abril de 2015, en Baltimore, por la muerte de Freddie Gray
bajo custodia policial,
i\href{https://www.snopes.com/fact-check/floyd-instruction-manual-protesters/}{nformó
Snope}, un sitio web de verificación de hechos.

Sheera Frenkel colaboró con este reportaje. Ben Decker colaboró con la
investigación.

Davey Alba es una reportera de tecnología que cubre desinformación. En
2019, ganó un premio Livingston a la excelencia por cobertura
internacional y un premio Mirror por la mejor historia sobre periodismo
en peligro. \href{https://twitter.com/daveyalba}{@daveyalba}

\begin{center}\rule{0.5\linewidth}{\linethickness}\end{center}

Advertisement

\protect\hyperlink{after-bottom}{Continue reading the main story}

\hypertarget{site-index}{%
\subsection{Site Index}\label{site-index}}

\hypertarget{site-information-navigation}{%
\subsection{Site Information
Navigation}\label{site-information-navigation}}

\begin{itemize}
\tightlist
\item
  \href{https://help.nytimes.com/hc/en-us/articles/115014792127-Copyright-notice}{©~2020~The
  New York Times Company}
\end{itemize}

\begin{itemize}
\tightlist
\item
  \href{https://www.nytco.com/}{NYTCo}
\item
  \href{https://help.nytimes.com/hc/en-us/articles/115015385887-Contact-Us}{Contact
  Us}
\item
  \href{https://www.nytco.com/careers/}{Work with us}
\item
  \href{https://nytmediakit.com/}{Advertise}
\item
  \href{http://www.tbrandstudio.com/}{T Brand Studio}
\item
  \href{https://www.nytimes.com/privacy/cookie-policy\#how-do-i-manage-trackers}{Your
  Ad Choices}
\item
  \href{https://www.nytimes.com/privacy}{Privacy}
\item
  \href{https://help.nytimes.com/hc/en-us/articles/115014893428-Terms-of-service}{Terms
  of Service}
\item
  \href{https://help.nytimes.com/hc/en-us/articles/115014893968-Terms-of-sale}{Terms
  of Sale}
\item
  \href{https://spiderbites.nytimes.com}{Site Map}
\item
  \href{https://help.nytimes.com/hc/en-us}{Help}
\item
  \href{https://www.nytimes.com/subscription?campaignId=37WXW}{Subscriptions}
\end{itemize}
