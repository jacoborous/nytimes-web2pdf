Sections

SEARCH

\protect\hyperlink{site-content}{Skip to
content}\protect\hyperlink{site-index}{Skip to site index}

\href{https://www.nytimes.com/section/style}{Style}

\href{https://myaccount.nytimes.com/auth/login?response_type=cookie\&client_id=vi}{}

\href{https://www.nytimes.com/section/todayspaper}{Today's Paper}

\href{/section/style}{Style}\textbar{}Tommy Dorfman of `13 Reasons Why'
Volunteers to Make a Kale Salad

\href{https://nyti.ms/2XNBhSw}{https://nyti.ms/2XNBhSw}

\begin{itemize}
\item
\item
\item
\item
\item
\end{itemize}

Advertisement

\protect\hyperlink{after-top}{Continue reading the main story}

Supported by

\protect\hyperlink{after-sponsor}{Continue reading the main story}

Encounters

\hypertarget{tommy-dorfman-of-13-reasons-why-volunteers-to-make-a-kale-salad}{%
\section{Tommy Dorfman of `13 Reasons Why' Volunteers to Make a Kale
Salad}\label{tommy-dorfman-of-13-reasons-why-volunteers-to-make-a-kale-salad}}

The young actor, who identifies as queer and nonbinary, stars in the Off
Broadway play ```Daddy.'''

\includegraphics{https://static01.nyt.com/images/2019/03/08/fashion/08ENCOUNTERS1/08ENCOUNTERS1-articleLarge-v2.jpg?quality=75\&auto=webp\&disable=upscale}

\href{https://www.nytimes.com/by/max-berlinger}{\includegraphics{https://static01.nyt.com/images/2018/11/05/multimedia/author-max-berlinger/author-max-berlinger-thumbLarge.png}}

By \href{https://www.nytimes.com/by/max-berlinger}{Max Berlinger}

\begin{itemize}
\item
  March 8, 2019
\item
  \begin{itemize}
  \item
  \item
  \item
  \item
  \item
  \end{itemize}
\end{itemize}

Tommy Dorfman was holding a kitchen knife that could make Norman Bates
wince, wondering what to do with the kale.

``How should I cut this?'' Mr. Dorfman said.

The 26-year-old actor was preparing lunch at the
\href{https://www.aliforneycenter.org/}{Ali Forney Center}, a nonprofit
in New York City that supports homeless L.G.B.T.Q. youth. The menu
consisted of jambalaya and kale salad, and the salad required a fair
amount of knife work.

A game plan was laid out by Jess Tell, the meal coordinator: Separate
the leaves from the stems and chop them down to bite-size pieces.

Mr. Dorfman, best known for starring in the dark teen drama
\href{https://www.nytimes.com/2017/03/30/arts/television/netflix-13-reasons-why-tv-review.html?action=click\&module=RelatedCoverage\&pgtype=Article\&region=Footer}{``13
Reasons Why,''} was both in and out of his element. In, because he has
volunteered with Ali Forney for more than two years, fund-raising,
posting on social media and lending a hand in person. Out, because he
almost never cooks for himself.

``I can cook, but I just don't,'' he said, noting that he and his
husband, Peter Zurkuhlen, prefer to order in, when they are at home in
Los Angeles. Mr. Dorfman's domestic strengths lie elsewhere. ``I'm a
cleaner, a put-er away-er, an organizer of things,'' he added. ``I'm the
hostess.''

Mr. Dorfman was dressed in a Cher T-shirt and plaid Thom Browne pants,
with his blond curls tucked into a baby blue Nike hat. He was in New
York to perform in
```\href{https://www.vineyardtheatre.org/daddy/}{Daddy},''' a new play
by
\href{https://www.nytimes.com/2016/08/18/style/jeremy-o-harris-actor-playwright-yale-james-franco.html}{Jeremy
O. Harris}, a young and buzzy playwright, and had a day free of
performances and rehearsals. A sous chef was needed at Ali Forney's
drop-in center on West 125th Street in Harlem, and Mr. Dorfman was happy
to step up.

\includegraphics{https://static01.nyt.com/images/2019/03/10/fashion/08ENCOUNTERS3/merlin_151033251_9544f3da-917d-414c-9436-93debd5a42e1-articleLarge.jpg?quality=75\&auto=webp\&disable=upscale}

Image

Credit...Matthew Leifheit for The New York Times

After vigorously washing his hands, he donned a red-and-black apron and,
armed with the knife, started hacking his way into a dense mountain of
kale, saving the stems for later use. He liked being put to work.

``As nice as it is to make an appearance at an event, I don't find that
the most fulfilling way to support an organization,'' he said, dabbing
his face when he started working up a sweat. (He was promptly told to
rewash his hands.) ``The more I'm in service, the less I'm in self. I
try to spend as little time thinking about myself as possible. I find
that's not a constructive way to live. By coming here, by volunteering,
it's a way to get out of my own head.''

During his childhood in Atlanta, with three brothers and one sister,
giving back was part of his family's ethos. His father, who spent
Sundays volunteering for the homeless, ``was always interested in
supporting underprivileged youth,'' Mr. Dorfman said. ``He created a
nonprofit basketball program and housing systems. He had people living
with us from time to time. He's always been that person.''

That may explain Mr. Dorfman's engagement with political and social
causes. His
\href{https://www.instagram.com/tommy.dorfman/?hl=en}{Instagram feed}
not only features
\href{https://www.instagram.com/p/BmUHQ86BD_T/}{shirtless selfies}, but
also photos with \href{https://www.instagram.com/p/BjsJWRpBBQL/}{Emma
González}, **** a gun control activist who survived the Parkland, Fla.,
school shooting, and posts urging his
\href{https://www.instagram.com/p/BpH-245hpNQ/}{followers to vote}.

He was on the February cover of Out magazine and wrote an essay last
year in Teen Vogue about
\href{https://www.teenvogue.com/story/tommy-dorfman-pride-asos-glaad-non-gendered-clothing}{wearing
gendered clothing} as a nonbinary person. As Walt Whitman may have put
it, he contains multitudes.

It is a stark contrast from his role in ```Daddy,''' in which he plays
Max, a vapid young Hollywood actor who can barely conceal his jealousy
when a friend finds a sugar daddy, played by Alan Cumming.

Set entirely on the patio of a modernist Los Angeles manse with a large
pool, which reflects an undulating Hockneyesque light across the stage,
the play deals with art and identity, with Mr. Dorfman moving from
bitter frenemy to a compassionate voice by the play's surreal final act.

``I've never worked this hard in my life professionally,'' he said, as
he squeezed lemon juice over the kale. ``The show's a marathon. I've
never done a play. I honestly do thank God every day when I come to work
because the people are so talented and lovely.''

Next up, Mr. Dorfman has a role in the coming season of
\href{https://www.nytimes.com/watching/recommendations/watching-tv-jane-the-virgin}{``Jane
the Virgin,''} a comedy telenovela on CW. ``I'm the villain,'' he said
in a singsong falsetto (to say Mr. Dorfman's inflection is extremely
expressive would be an understatement).

He's also in ``American Princess,'' a comedy drama from
\href{https://www.nytimes.com/2017/06/08/arts/television/jenji-kohan-interview-orange-is-the-new-black-season-4.html}{Jenji
Kohan}, the co-creator of ``Orange Is the New Black.'' ``I'm basically
playing a drunk, gay baby,'' he said with a shrug.

But first he had to conquer the kale. He was given the arduous task of
massaging the fibrous leaves into a tender salad. After dousing it with
olive oil, he slipped on a pair of plastic gloves and began kneading it
violently. After a good 15 minutes, he was working up a sweat again.
``Does this mean I don't have to go to the gym today?'' he said.

Shortly after 1 p.m., the prep work was done and the proverbial lunch
bell rang. Mr. Dorfman ladled the jambalaya over rice, followed by a
heaping portion of his salad. ``Let's turn on some music,'' he called
out. ``Bodak Yellow'' by Cardi B filled the room. As he worked, his hips
swayed and his lips silently mouthed the lyrics.

A young man in a yellow hoodie approached for his lunch. ``You look
familiar,'' he said.

``Do I?'' Mr. Dorfman said with a sly smile, before giving him his
plate.

Advertisement

\protect\hyperlink{after-bottom}{Continue reading the main story}

\hypertarget{site-index}{%
\subsection{Site Index}\label{site-index}}

\hypertarget{site-information-navigation}{%
\subsection{Site Information
Navigation}\label{site-information-navigation}}

\begin{itemize}
\tightlist
\item
  \href{https://help.nytimes.com/hc/en-us/articles/115014792127-Copyright-notice}{©~2020~The
  New York Times Company}
\end{itemize}

\begin{itemize}
\tightlist
\item
  \href{https://www.nytco.com/}{NYTCo}
\item
  \href{https://help.nytimes.com/hc/en-us/articles/115015385887-Contact-Us}{Contact
  Us}
\item
  \href{https://www.nytco.com/careers/}{Work with us}
\item
  \href{https://nytmediakit.com/}{Advertise}
\item
  \href{http://www.tbrandstudio.com/}{T Brand Studio}
\item
  \href{https://www.nytimes.com/privacy/cookie-policy\#how-do-i-manage-trackers}{Your
  Ad Choices}
\item
  \href{https://www.nytimes.com/privacy}{Privacy}
\item
  \href{https://help.nytimes.com/hc/en-us/articles/115014893428-Terms-of-service}{Terms
  of Service}
\item
  \href{https://help.nytimes.com/hc/en-us/articles/115014893968-Terms-of-sale}{Terms
  of Sale}
\item
  \href{https://spiderbites.nytimes.com}{Site Map}
\item
  \href{https://help.nytimes.com/hc/en-us}{Help}
\item
  \href{https://www.nytimes.com/subscription?campaignId=37WXW}{Subscriptions}
\end{itemize}
