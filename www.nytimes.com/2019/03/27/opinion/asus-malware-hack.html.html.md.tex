Sections

SEARCH

\protect\hyperlink{site-content}{Skip to
content}\protect\hyperlink{site-index}{Skip to site index}

\href{https://myaccount.nytimes.com/auth/login?response_type=cookie\&client_id=vi}{}

\href{https://www.nytimes.com/section/todayspaper}{Today's Paper}

\href{/section/opinion}{Opinion}\textbar{}Stop Ignoring Those `Update
Your Device' Messages

\url{https://nyti.ms/2V1lG00}

\begin{itemize}
\item
\item
\item
\item
\item
\item
\end{itemize}

Advertisement

\protect\hyperlink{after-top}{Continue reading the main story}

\href{/section/opinion}{Opinion}

Supported by

\protect\hyperlink{after-sponsor}{Continue reading the main story}

\hypertarget{stop-ignoring-those-update-your-device-messages}{%
\section{Stop Ignoring Those `Update Your Device'
Messages}\label{stop-ignoring-those-update-your-device-messages}}

Even though the Asus malware attack was spread through software updates,
the best way to protect yourself online is to keep your software
updated.

By Matt Blaze

Mr. Blaze is a professor at Georgetown University.

\begin{itemize}
\item
  March 27, 2019
\item
  \begin{itemize}
  \item
  \item
  \item
  \item
  \item
  \item
  \end{itemize}
\end{itemize}

\includegraphics{https://static01.nyt.com/images/2019/03/27/opinion/27blaze/merlin_152620431_e5a9bf4d-3710-4398-82d1-ea4d8c7484aa-articleLarge.jpg?quality=75\&auto=webp\&disable=upscale}

This week, internet security researchers woke up to
\href{https://motherboard.vice.com/en_us/article/pan9wn/hackers-hijacked-asus-software-updates-to-install-backdoors-on-thousands-of-computers}{disturbing
news}. An attacker had installed malware on as many as half a million
Asus-brand computers running the Windows operating system. Reports of
large-scale malware infections have become almost routine, but what made
this one notable was how it was accomplished: the attacker compromised
the Asus servers used to send periodic operating system and security
updates to customers. In other words, as far as the customer could tell,
the malicious software came directly from the manufacturer, complete
with its digital stamp of approval.

Subverting a software update system (a type of ``supply chain attack''
in security parlance) is usually associated with international espionage
and intelligence operations --- not with run-of-the-mill attempts to
steal credit card numbers or banking passwords --- and for good reason.
It requires considerable resources and skill because the hardware and
software companies' systems that have to be compromised are generally
better managed and more carefully secured than those of individual
consumers. But when they are successful, attacks against the supply
chain are an especially powerful threat: They rely not on repeatedly
fooling each user into, say, opening the wrong email attachment (a
common way malware is spread), but on the trust that users naturally and
reasonably place in the suppliers that sold them their hardware and
software. It's a dangerously effective way for the attacker to reach
many thousands of victims (or a few carefully selected targets).

Fortunately, while the Asus attack exploited the vendor update mechanism
to install unauthorized software on many thousands of computers, it
appears that the malware itself was designed to affect only a few
hundred actual targets and was relatively harmless to everyone else. But
that still leaves us with the uncomfortable fact that a software update
mechanism --- a critical system intended to protect users --- was turned
on its head and used to attack them instead. And it's not just Asus;
almost every other major software and hardware vendor offers an update
mechanism for its products, sometimes enabled by default.

To protect against the insidious threat of malicious updates, it might
be tempting to immediately disable these mechanisms on your computers
and smartphones. But that would be a terrible idea, one that would
expose you to far more harm than it would protect against. In fact, now
would be a fine time to check your devices and make sure the automatic
system update features are turned on and running.

The reason for this counterintuitive advice has to do with the fragile
nature of the software that makes modern computing and the internet
work. In spite of decades of steady technological progress that has made
computers better in almost every way, virtually all software still
suffers from bugs --- the small programming defects that can manifest
themselves as everything from minor unexpected behavior to outright
system crashes.

Some of these bugs have security implications; they can be exploited to
do harm, for example, by exposing sensitive information to attackers. As
systems become larger and more interconnected (as they inevitably do),
the number of bugs, and our exposure to exploitable vulnerabilities,
only increases. In other words, every computer, every smartphone, every
piece of software is delivered to the user with a plethora of hidden
security flaws preinstalled. We just haven't found them yet.

Over time, of course, these vulnerabilities get discovered and used
against users. The only viable protection is to fix them as soon as
they're found. That's where the vendors' software update mechanisms come
into play. The most important updates quietly repair newly discovered
security flaws that have already been, or will soon be, used to attack
end users.

In other words, security in the modern internet can be understood as
something of an ecosystem, where survival depends on continually
adapting to protect against ever-evolving new threats. Vendor software
updates, applied at regular intervals, are, for better or worse, the
only large-scale method we have for adapting our defenses. Those who
fail to update become prominently attractive targets, with their
computers succumbing to automated attacks that might do anything from
steal personal information to installing ``ransomware'' that holds
important files hostage until payment is made. As the ``Internet of
Things'' puts connectivity (and complex software) in everything from
home security systems to light bulbs, the consequences of these attacks,
and the need for regular software updates to prevent them, will only
grow.

Therefore, the most potentially damaging aspect of the Asus attack isn't
whatever malevolent behavior it might have directly exhibited. It is
that people might be frightened away from installing the critical
software updates that keep life on the modern internet relatively safe.
The calculus is simple: Allowing updates subjects us to a small risk of
falling victim to a sophisticated supply chain compromise. But
disallowing updates brings a near certainty over time that we will be
successfully attacked. The danger here lies in overreacting to a small
risk in a way that exposes us to a much more likely --- and even more
undesirable --- one.

So what should we do? The main responsibility lies with the industry.
Asus will no doubt be criticized for allowing its servers to be
compromised and for failing to detect that it had been distributing
malicious software to its customers. Other vendors should take note and
harden their own systems. And especially as the Internet of Things turns
our appliances into computers, lawmakers and regulators should
increasingly understand computer security --- and the requirement for
high-integrity software updates --- as a basic consumer safety issue.

Meanwhile, on the internet, it's update and evolve, or die.

Matt Blaze (@mattblaze) is the McDevitt Professor of Law and Computer
Science at Georgetown University, where he studies security and privacy
technology and its implications for public policy.

\emph{The Times is committed to publishing}
\href{https://www.nytimes.com/2019/01/31/opinion/letters/letters-to-editor-new-york-times-women.html}{\emph{a
diversity of letters}} \emph{to the editor. We'd like to hear what you
think about this or any of our articles. Here are some}
\href{https://help.nytimes.com/hc/en-us/articles/115014925288-How-to-submit-a-letter-to-the-editor}{\emph{tips}}\emph{.
And here's our email:}
\href{mailto:letters@nytimes.com}{\emph{letters@nytimes.com}}\emph{.}

\emph{Follow The New York Times Opinion section on}
\href{https://www.facebook.com/nytopinion}{\emph{Facebook}}\emph{,}
\href{http://twitter.com/NYTOpinion}{\emph{Twitter (@NYTopinion)}}
\emph{and}
\href{https://www.instagram.com/nytopinion/}{\emph{Instagram}}\emph{.}

Advertisement

\protect\hyperlink{after-bottom}{Continue reading the main story}

\hypertarget{site-index}{%
\subsection{Site Index}\label{site-index}}

\hypertarget{site-information-navigation}{%
\subsection{Site Information
Navigation}\label{site-information-navigation}}

\begin{itemize}
\tightlist
\item
  \href{https://help.nytimes.com/hc/en-us/articles/115014792127-Copyright-notice}{©~2020~The
  New York Times Company}
\end{itemize}

\begin{itemize}
\tightlist
\item
  \href{https://www.nytco.com/}{NYTCo}
\item
  \href{https://help.nytimes.com/hc/en-us/articles/115015385887-Contact-Us}{Contact
  Us}
\item
  \href{https://www.nytco.com/careers/}{Work with us}
\item
  \href{https://nytmediakit.com/}{Advertise}
\item
  \href{http://www.tbrandstudio.com/}{T Brand Studio}
\item
  \href{https://www.nytimes.com/privacy/cookie-policy\#how-do-i-manage-trackers}{Your
  Ad Choices}
\item
  \href{https://www.nytimes.com/privacy}{Privacy}
\item
  \href{https://help.nytimes.com/hc/en-us/articles/115014893428-Terms-of-service}{Terms
  of Service}
\item
  \href{https://help.nytimes.com/hc/en-us/articles/115014893968-Terms-of-sale}{Terms
  of Sale}
\item
  \href{https://spiderbites.nytimes.com}{Site Map}
\item
  \href{https://help.nytimes.com/hc/en-us}{Help}
\item
  \href{https://www.nytimes.com/subscription?campaignId=37WXW}{Subscriptions}
\end{itemize}
