Sections

SEARCH

\protect\hyperlink{site-content}{Skip to
content}\protect\hyperlink{site-index}{Skip to site index}

\href{https://www.nytimes.com/section/business}{Business}

\href{https://myaccount.nytimes.com/auth/login?response_type=cookie\&client_id=vi}{}

\href{https://www.nytimes.com/section/todayspaper}{Today's Paper}

\href{/section/business}{Business}\textbar{}U.S. Under Pressure to Cease
Flights of Troubled Boeing Jet

\url{https://nyti.ms/2EUVugJ}

\begin{itemize}
\item
\item
\item
\item
\item
\item
\end{itemize}

Advertisement

\protect\hyperlink{after-top}{Continue reading the main story}

Supported by

\protect\hyperlink{after-sponsor}{Continue reading the main story}

\hypertarget{us-under-pressure-to-cease-flights-of-troubled-boeing-jet}{%
\section{U.S. Under Pressure to Cease Flights of Troubled Boeing
Jet}\label{us-under-pressure-to-cease-flights-of-troubled-boeing-jet}}

\includegraphics{https://static01.nyt.com/images/2019/03/12/business/12boeing-web-sub/merlin_151972782_8ad3f951-08b3-4036-9d60-28267676d16c-articleLarge.jpg?quality=75\&auto=webp\&disable=upscale}

By \href{https://www.nytimes.com/by/david-gelles}{David Gelles},
\href{https://www.nytimes.com/by/thomas-kaplan}{Thomas Kaplan},
\href{https://www.nytimes.com/by/kenneth-p-vogel}{Kenneth P. Vogel} and
\href{https://www.nytimes.com/by/natalie-kitroeff}{Natalie Kitroeff}

\begin{itemize}
\item
  March 12, 2019
\item
  \begin{itemize}
  \item
  \item
  \item
  \item
  \item
  \item
  \end{itemize}
\end{itemize}

\emph{{[}Update: The United States has
now}\href{https://www.nytimes.com/2019/03/13/business/canada-737-max.html}{\emph{grounded
Boeing's 737 Max aircraft}}\emph{, reversing an earlier decision.{]}}

With more countries grounding
\href{https://www.nytimes.com/2019/03/27/business/boeing-hearings.html}{Boeing}
jets and with lawmakers, aviation workers and consumers calling on the
United States to do the same, the head of the aerospace giant on Tuesday
made a personal appeal to President Trump.

\href{https://www.nytimes.com/2019/03/27/business/boeing-hearings.html}{Boeing's}
chief executive, Dennis A. Muilenburg, called from Chicago and expressed
to Mr. Trump his confidence in the safety of the 737 Max 8 jets,
according to two people briefed on the conversation. Two of the planes
flown by overseas carriers have crashed in recent months in similar
accidents.

The brief call had been in the works since Monday, but it came shortly
after Mr. Trump raised concerns that the increasing use of technology in
airplanes was compromising passenger safety. ``Airplanes are becoming
far too complex to fly,''
\href{https://twitter.com/realDonaldTrump/status/1105468569800839169}{he
wrote} on Twitter. ``Pilots are no longer needed, but rather computer
scientists from MIT.''

\href{https://www.nytimes.com/interactive/2019/03/11/world/boeing-737-max-which-airlines.html}{}

\includegraphics{https://static01.nyt.com/images/2019/03/11/world/where-boeing-737-max-flies-1552329118347/where-boeing-737-max-flies-1552329118347-articleLarge-v14.jpg}

\hypertarget{from-8600-flights-to-zero-grounding-the-boeing-737-max-8}{%
\subsection{From 8,600 Flights to Zero: Grounding the Boeing 737 Max
8}\label{from-8600-flights-to-zero-grounding-the-boeing-737-max-8}}

China was the first major country to ground the jets. The U.S. was last.

Soon after the conversation ended, Mr. Muilenburg received more bad
news. The European Union suspended ``all flight operations'' of the
Boeing 737 Max 8 model, a striking move by one of the industry's
important regulators. At the end of the day, the Federal Aviation
Administration said that it was continuing with its review and that the
planes could keep flying.

Yet the decision in Europe means roughly two-thirds of the
\href{https://www.nytimes.com/2019/03/13/business/china-boeing.html}{Boeing
737 Max 8} aircraft in the world have been pulled from use in the two
days since the crash of an Ethiopian Airlines flight that killed
\href{https://www.nytimes.com/2019/03/10/world/africa/ethiopian-airlines-plane-crash-victims.html?module=inline}{157
people}. The swift actions by authorities around the world were driven
in part by concerns about a connection to a similar disaster involving a
Max 8 in Indonesia last October, when a Lion Air flight plunged into the
Java Sea shortly after takeoff, killing all 189 people aboard.

By Tuesday afternoon, the United States was nearly alone among major
countries still allowing the jets to fly.

Elaine Chao, the transportation secretary, said regulators ``will not
hesitate to take immediate and appropriate action'' if a safety issue
arises.

\includegraphics{https://static01.nyt.com/images/2019/03/12/business/12boeing-web2/merlin_151964127_f32f2863-dcf8-4aea-9eb8-dd2b0b2ecae7-articleLarge.jpg?quality=75\&auto=webp\&disable=upscale}

Boeing reiterated in a statement late Tuesday that it had ``full
confidence'' in the 737 Max 8. It noted that the F.A.A. had taken no
action and ``based on the information currently available, we do not
have any basis to issue new guidance to operators.''

Two United States airlines fly the 737 Max 8 aircraft and both said they
planned to keep flying. Southwest Airlines has 34 of the planes and
American Airlines has 24. The airlines have said they have analyzed data
from their thousands of flights with the jets and found no reason to
ground them.

``We don't have any changes planned,'' Southwest said in a statement.
``We have full confidence in the aircraft,'' American said.

The growing pressure left Boeing in an increasingly unfamiliar position.
The company, a major military contractor, has close ties with the
American government, and the F.A.A. in particular.

{[}\href{https://www.nytimes.com/2019/03/12/world/canada/ethiopian-plane-crash-canadian-families.html}{\emph{Three
generations of a Canadian family died}} \emph{in the Ethiopian plane
crash.}{]}

Boeing is a major lobbying force in the nation's capital. Its top
government relations official is a veteran of the Clinton White House,
and last year, the company employed more than a dozen lobbying firms to
advocate for its interests and spent \$15 million in total on lobbying,
\href{https://www.opensecrets.org/lobby/clientsum.php?id=D000000100}{according
to the Center for Responsive Politics}.

The company, through its political action committee, funnels millions of
dollars into the campaign accounts of lawmakers from both political
parties. A
\href{http://www.boeing.com/resources/boeingdotcom/company/key_orgs/pdf/2015_BPAC_Pol_Expenditures.pdf}{list
of a year's worth of political spending} on Boeing's website stretches
on for 14 pages, listing campaign contributions to lawmakers ranging
from a city councilman in South Carolina to Representative Nancy Pelosi
of California, who is now the House speaker.

``Boeing is one of the 800-pound gorillas around here,'' said Senator
Richard Blumenthal, Democrat of Connecticut, who has called for the Max
8 to be grounded. As an example of Boeing's reach in the highest levels
of government, Mr. Blumenthal noted that the acting defense secretary,
Patrick M. Shanahan, is a former Boeing executive.

{[}\emph{While some passengers
balked,}\href{https://www.nytimes.com/2019/03/12/travel/grounded-planes-airline-passengers.html}{\emph{it
was business as usual}} \emph{in the United States and Canada.}{]}

For decades, the F.A.A. has used a network of outside experts, known as
F.A.A. designees, to certify that aircraft meet safety standards. In
2005, the regulator shifted its approach for how it delegated authority
outside the agency, creating a new program through which aircraft
manufacturers like Boeing could choose their own employees to be the
designees and help certify their planes.

Image

President Trump at a tax forum last March with Boeing's chief executive,
Dennis A. Muilenburg, right, and Treasury Secretary Steven Mnuchin in
St. Louis. Boeing's relationship with Mr. Trump has not always been
smooth.Credit...Doug Mills/The New York Times

The program is intended to help the F.A.A. stretch its limited
resources, while also benefiting plane makers who are eager to avoid
delays in the certification process.

The regulator maintains offices inside Boeing's factories, including
those in Renton, Wash., and in Charleston, S.C. ``I've raised this
concern in the past, about people who go to work at the Boeing plant who
work for the F.A.A.,'' said Representative Peter A. DeFazio, Democrat of
Oregon and the chairman of the House transportation committee. ``How
much scrutiny are they applying, and could they be influenced?''

The F.A.A.'s top safety official,
\href{https://www.faa.gov/about/key_officials/bahrami_avs/}{Ali
Bahrami}, has worked closely with Boeing during his career, directing
the agency's certification of the Boeing 787 Dreamliner and the 747-8
passenger and freighter models.

{[}\href{https://www.nytimes.com/2019/03/12/reader-center/737-max-8.html}{\emph{Read
answers to frequently asked questions about Boeing's 737 Max 8.}}{]}

``It's a very cozy relationship,'' said Jim Hall, the former head of the
National Transportation Safety Board. ``The manufacturer essentially
becomes both the manufacturer and the regulator, because of the lack of
the ability of government to do the job.''

At a congressional hearing in 2015, a Boeing executive described the
arrangement as effectively having an ``arm of the F.A.A. within the
Boeing Company,'' and said 1,000 employees were part of the program.

The regulatory policy of allowing manufacturers to essentially sign off
on the safety of their own products has drawn criticism in the past. In
2011, a report from the Transportation Department's Office of Inspector
General found that the ``F.A.A. has significantly reduced its role in
approving individuals who perform work on F.A.A.'s behalf by further
delegating this approval to private companies.''

Boeing's relationship with Mr. Trump has not always been smooth,
however. Shortly after becoming president-elect, Mr. Trump
\href{https://www.nytimes.com/2016/12/06/us/politics/trump-air-force-one-boeing.html?module=inline}{assailed
Boeing} for the estimated cost of its program to build new Air Force One
planes, which provide mobile command centers for the president.

The ``costs are out of control, more than \$4 billion. Cancel order!''
Mr. Trump
\href{https://twitter.com/realDonaldTrump/status/806134244384899072}{wrote
on Twitter} a month after winning the election, but before taking
office. A couple of weeks later, Mr. Muilenburg
\href{https://www.nytimes.com/2016/12/21/business/donald-trump-boeing-lockheed.html?module=inline}{visited
Mr. Trump at his Mar-a-Lago club} in Palm Beach, Fla., to try to smooth
things over.

Image

The Boeing plant in Renton, Wash. Two flight attendant unions called on
the United States to ground the Max 8, but the Federal Aviation
Administration said it had ``no basis to order grounding the
aircraft.''Credit...Ruth Fremson/The New York Times

``It was a terrific conversation,'' Mr. Muilenburg told reporters after
the meeting, explaining that he had given Mr. Trump ``my personal
commitment'' that Boeing would build new Air Force One planes for less
than the \$4 billion estimate. Weeks after the conversation, Boeing
donated \$1 million to Mr. Trump's inaugural committee. The company had
donated the same amount to help finance President Barack Obama's
inauguration in 2013.

As concerns about the airworthiness of the Boeing 737 Max 8 spread
around the globe on Tuesday, pressure was building on the F.A.A. to take
action. Boeing shares fell 6 percent on Tuesday, after falling 5 percent
on Monday. Early on Wednesday in New Zealand, the country's aviation
regulator
\href{https://www.caa.govt.nz/public-and-media-info/caa-releases/boeing-737-max-suspension/}{suspended}
737 Max flights there.

Two unions representing flight attendants called for the jets to be
grounded. ``The F.A.A. must restore public confidence by grounding the
737 Max until the required changes have been implemented and the public
can be fully assured,''
\href{https://www.afacwa.org/afa_statement_on_737_max}{said Sara
Nelson}, president of the Association of Flight Attendants.

Lori Bassani, the president of the Association of Professional Flight
Attendants, which represents flight attendants at American Airlines,
called on the airline to ``strongly consider grounding these planes
until a thorough investigation can be performed.''

In Washington, politicians on both sides of the aisle called for action.
Senator Ted Cruz, the Texas Republican who is the chairman of the Senate
Commerce Committee's Subcommittee on Aviation and Space, said he wanted
``to temporarily ground 737 Max aircraft until the F.A.A. confirms the
safety of these aircraft.'' He also said he planned to hold a hearing to
investigate the crashes.

Senators Elizabeth Warren, Democrat of Massachusetts, and Mitt Romney,
Republican of Utah, also called on the F.A.A. to ground the aircraft
while the cause of the Ethiopian crash is investigated.

``Serious questions have been raised about whether these planes were
pressed into service without additional pilot training in order to save
money,'' Ms. Warren
\href{https://www.masslive.com/politics/2019/03/elizabeth-warren-urges-faa-to-ground-boeing-737-max-8-planes.html}{said}.
Mr. Romney, in a
\href{https://twitter.com/MittRomney/status/1105470096691089413}{Twitter
post}, urged a grounding ``out of an abundance of caution for the flying
public.''

In his phone call with the president, the Boeing chief, Mr. Muilenburg,
outlined the company's position since the crash. He also updated Mr.
Trump on the status of the 737 Max models.

It was unclear whether anything came of the call. The White House did
not respond to questions about the conversation.

In its statement, the F.A.A. said that it had ``no basis to order
grounding the aircraft.''

Advertisement

\protect\hyperlink{after-bottom}{Continue reading the main story}

\hypertarget{site-index}{%
\subsection{Site Index}\label{site-index}}

\hypertarget{site-information-navigation}{%
\subsection{Site Information
Navigation}\label{site-information-navigation}}

\begin{itemize}
\tightlist
\item
  \href{https://help.nytimes.com/hc/en-us/articles/115014792127-Copyright-notice}{©~2020~The
  New York Times Company}
\end{itemize}

\begin{itemize}
\tightlist
\item
  \href{https://www.nytco.com/}{NYTCo}
\item
  \href{https://help.nytimes.com/hc/en-us/articles/115015385887-Contact-Us}{Contact
  Us}
\item
  \href{https://www.nytco.com/careers/}{Work with us}
\item
  \href{https://nytmediakit.com/}{Advertise}
\item
  \href{http://www.tbrandstudio.com/}{T Brand Studio}
\item
  \href{https://www.nytimes.com/privacy/cookie-policy\#how-do-i-manage-trackers}{Your
  Ad Choices}
\item
  \href{https://www.nytimes.com/privacy}{Privacy}
\item
  \href{https://help.nytimes.com/hc/en-us/articles/115014893428-Terms-of-service}{Terms
  of Service}
\item
  \href{https://help.nytimes.com/hc/en-us/articles/115014893968-Terms-of-sale}{Terms
  of Sale}
\item
  \href{https://spiderbites.nytimes.com}{Site Map}
\item
  \href{https://help.nytimes.com/hc/en-us}{Help}
\item
  \href{https://www.nytimes.com/subscription?campaignId=37WXW}{Subscriptions}
\end{itemize}
