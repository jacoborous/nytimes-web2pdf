Sections

SEARCH

\protect\hyperlink{site-content}{Skip to
content}\protect\hyperlink{site-index}{Skip to site index}

\href{https://www.nytimes.com/section/arts/design}{Art \& Design}

\href{https://myaccount.nytimes.com/auth/login?response_type=cookie\&client_id=vi}{}

\href{https://www.nytimes.com/section/todayspaper}{Today's Paper}

\href{/section/arts/design}{Art \& Design}\textbar{}Discovered After 70,
Black Artists Find Success, Too, Has Its Price

\href{https://nyti.ms/2YhhI5r}{https://nyti.ms/2YhhI5r}

\begin{itemize}
\item
\item
\item
\item
\item
\item
\end{itemize}

Advertisement

\protect\hyperlink{after-top}{Continue reading the main story}

Supported by

\protect\hyperlink{after-sponsor}{Continue reading the main story}

\hypertarget{discovered-after-70-black-artists-find-success-too-has-its-price}{%
\section{Discovered After 70, Black Artists Find Success, Too, Has Its
Price}\label{discovered-after-70-black-artists-find-success-too-has-its-price}}

Once on the margins, older African-American artists are suddenly a hot
commodity. They are relishing the attention while dealing with the
market's grueling demands.

\includegraphics{https://static01.nyt.com/images/2019/03/21/arts/00olderartist1/00olderartist1-articleLarge.jpg?quality=75\&auto=webp\&disable=upscale}

\href{https://www.nytimes.com/by/hilarie-m-sheets}{\includegraphics{https://static01.nyt.com/images/2019/04/03/multimedia/author-hilarie-m-sheets/author-hilarie-m-sheets-thumbLarge.png}}

By \href{https://www.nytimes.com/by/hilarie-m-sheets}{Hilarie M. Sheets}

\begin{itemize}
\item
  March 23, 2019
\item
  \begin{itemize}
  \item
  \item
  \item
  \item
  \item
  \item
  \end{itemize}
\end{itemize}

McArthur Binion had been creating art almost completely under the radar
for four decades, handling his own occasional sales and raising two
children in Chicago on a teaching salary.

Now, Mr. Binion has been fully embraced by the mainstream art world ---
at the age of 72. His dealer is a prominent Chelsea gallery. Museums and
international collectors are snapping up his large canvases, minimalist
grids painted in oil stick over collages of personal documents.

With his work selling for up to \$450,000, he can now travel first class
and easily afford his daughter's Brown University tuition. ``I'm totally
ready for it,'' Mr. Binion said of the acclaim.

But he was not totally ready for the coarser realities of the modern-day
art market. Mr. Binion rejected one dealer who he felt patronized him
--- she was hustling his freshly painted works on her cellphone at the
Venice Biennale and coaching him on how to speak to curators and the
press.

``Nobody's going to tell me what to say about my work,'' said Mr.
Binion. ``For me, if it wasn't going to be on your own terms, it's not
worth it.''

Mr. Binion belongs to a generation of African-American artists in their
70s and 80s who are enjoying a market renaissance after decades of
indifference. Museums are mounting popular exhibitions of their work,
and their names, in some cases, are worth millions on the auction block.

And so precisely when many artists are slowing down, they are gearing up
--- and struggling to balance the pressures of success.

\includegraphics{https://static01.nyt.com/images/2019/03/24/arts/24olderartist-print2/merlin_149347170_eeaf27a8-696b-4a6d-b135-0b2a82411140-articleLarge.jpg?quality=75\&auto=webp\&disable=upscale}

With public appreciation have come demands to attend openings, lectures,
interviews and panels at a time when travel can be onerous, or when
artists prefer to be in the studio. Those whose work was never political
or didactic are apprehensive about being framed as black artists, rather
than as just artists.

``The kind of elation I may have had back 30 years, I'm past that
point,'' said
\href{https://mcachicago.org/Exhibitions/2018/Howardena-Pindell}{Howardena
Pindell}, who at 75 uses a walker and can no longer crawl up a ladder to
execute her paintings, often collaged with hundreds of paper dots
covered in layers of acrylic, dye, sequins, glitter and powder.

Like other artists of color discovered later in life, hers is a
different kind of contentment: ``It's a more a sense of feeling
protected and safe in terms of the vicissitudes of the art world.''

In interviews, several of these artists spoke of the urge to create new
works, which can be physically demanding yet invigorating at the same
time. And while all the new money flowing in would have been nice when
they were raising families, they were happy to be able to provide some
security for their children and grandchildren.

Still, they can only do so much. Alexander Gray, a New York dealer who
represents Melvin Edwards, Lorraine O'Grady and Frank Bowling, said he
has been guilty of overestimating the speed of his older artists. ``With
the enthusiasm of the marketplace,'' he said, ``we forget the age of
these human beings and their physical capacity to travel all over the
world, expected of an artist in the art world right now. I think it can
be overwhelming.''

Ms. O'Grady, an 85-year-old conceptual artist, has been inundated with
requests to appear on panels and make presentations. ``Anything that
involves travel at this point, she is declining,'' said Mr. Gray. At 85,
Mr. Bowling, who was born in Guyana and maintains studios in London and
New York, has also put a moratorium on travel and interviews.

``At this point, for artists of advanced age, time is precious and
there's no more precious place to be than in the studio,'' Mr. Gray
said.

Image

Howardena Pindell's thickly painted abstract canvas, ``Autobiography:
Artemis'' (1986) at Garth Greenan Gallery in New York.Credit...Daniel
Dorsa for The New York Times

Image

Howardena Pindell arrived in New York in 1967 to a cool reception; her
abstract art was frowned upon by both white and black members of the art
community. Her breakthrough came in 2015. Now 75, she uses a walker but
is actively painting with the help of assistants.Credit...Daniel Dorsa
for The New York Times

Image

Ms. Pindell's ``Untitled,'' an early work from 1972, spray-paint on
canvas, at Garth Greenan Gallery.Credit...Daniel Dorsa for The New York
Times

\hypertarget{its-about-time}{%
\subsection{`It's About Time'}\label{its-about-time}}

Recent traveling exhibitions such as
\href{https://www.thebroad.org/soul-of-a-nation}{``Soul of a Nation: Art
in the Age of Black Power,''} which opened Saturday
\href{https://www.thebroad.org/soul-of-a-nation}{at the Broad} in Los
Angeles, have illuminated the pantheon of black artists working in the
1960s and 70s. ``There has been a whole parallel universe that existed
that people had not tapped into,'' said Valerie Cassel Oliver, curator
of modern and contemporary art at the
\href{https://www.vmfa.museum/}{Virginia Museum of Fine Arts} in
Richmond.

\emph{{[}}\href{https://www.nytimes.com/2018/09/13/arts/design/soul-of-a-nation-review-brooklyn-museum-black-power.html}{\emph{Read
Holland Cotter's review of ``Soul of a Nation.''}}\emph{{]}}

Shattering the \$2 million threshold, new auction highs were set last
year for Sam Gilliam, 85, as well as
\href{https://www.culturetype.com/2018/05/18/two-1970s-era-portraits-by-barkley-l-hendricks-top-2-million-at-sothebys-shattering-the-artists-previous-record/}{Barkley
Hendricks} and
\href{https://www.culturetype.com/2018/11/15/jack-whittens-ancient-mentor-i-reaches-2-2-million-establishing-new-auction-record-more-than-twice-his-previous-high-mark/}{Jack
Whitten}, both recently deceased. They built on the success of younger
African-American artists such as Kerry James Marshall, 63, who
\href{https://www.nytimes.com/2018/05/18/arts/sean-combs-kerry-james-marshall.html}{recently
broke \$21 million at auction.} (While the artists do not directly
benefit when collectors sell their paintings at auction, the high prices
make their newly created works more valuable.)

Ms. Pindell's breakthrough came in 2014, when she signed with Garth
Greenan, a Chelsea gallerist; since then a survey of her work has
appeared at the Museum of Contemporary Art in Chicago and the Virginia
Museum of Fine Arts and is now on view at
\href{https://www.brandeis.edu/rose/}{Brandeis University's Rose Art
Museum}.

But as a black abstract artist, Ms. Pindell found an inhospitable
reception in New York after graduating from Yale in 1967. She was
bucking the widespread expectation that African-American artists should
create work about social issues.

``Within the African-American community in the 1970s, if you were an
abstract artist you were considered the enemy pandering to the white
world,'' said Ms. Pindell. ``But white dealers would say that
African-Americans who did abstract work were inauthentic.''

Unbowed, she followed her own path. She worked in the curatorial ranks
at the Museum of Modern Art, then
\href{http://art.stonybrook.edu/person/howardena-pindell/}{began
teaching} at what is now Stony Brook University, while showing
sporadically at galleries supporting underrepresented artists.

Even dealers who were starting to show women did not often include black
women. At one gallery exhibition that had no black artists, she said,
``I spoke vocally about that and one white woman came to me said, `Would
you please be cooperative and shut up about race?'''

Since signing with Mr. Greenan she has been asked to produce more work
for fairs and exhibitions than she has been used to.

But on the upside, she can now afford studio assistants and a driver to
take her to her teaching job.

``I'm being well taken care of,'' said Ms. Pindell.

Image

The sculptor Melvin Edwards at his studio in Plainfield, N.J., with
``Nigba Lailai (The Past),'' from 1979.Credit...Melvin Edwards/Artists
Rights Society (ARS), New York; Chester Higgins Jr./The New York Times

Image

Mr. Edwards's welded steel sculpture, ``For Modie 218.'' The show ``Soul
of a Nation: Art in the Age of Black Power,'' at the Broad, shines a
bright light on the work of Mr. Edwards, Ms. O'Grady, Mr. Bowling and
others.Credit...Melvin Edwards/Artists Rights Society (ARS), New York;
via Alexander Gray Associates; Stephen Friedman Gallery

Mr. Edwards, whose abstract, welded-metal sculptures have been acquired
by five institutions in the last 18 months, including the Whitney Museum
and Tate Modern, turns 82 in May. He, too, said that pursuing
abstraction as a black artist was a lonely road.

His geometric barbed-wire and chain structures were shown at the Whitney
in 1970, but it would be another two decades before he had his first
gallery exhibition. Demand for his work has risen dramatically in recent
years as a result of museum shows.

``You invent your own game --- and then you push it forward,'' said Mr.
Edwards, who taught at Rutgers for 30 years. ``It's about time the art
world caught up.''

\hypertarget{not-defined-by-race}{%
\subsection{Not Defined by Race}\label{not-defined-by-race}}

In some cases, the older artists have had younger African-American stars
to thank for the new attention. Mark Bradford, 57, the renowned Los
Angeles artist who represented the United States in 2017 at the Venice
Biennale, lobbied his gallery, Hauser \& Wirth, to take on Mr. Whitten,
one of Mr. Bradford's inspirations. And the conceptual artist Charles
Gaines, 75, Mr. Bradford's former teacher, will have his first show with
the gallery in the fall.

Image

Last year Frank Bowling, 85, created ``Two Blues,'' acrylic and mixed
media on collaged and printed canvas.Credit...Frank Bowling/Artists
Rights Society (ARS), New York; DACS, London; via Alexander Gray
Associates; Hales Gallery

Image

Drawing from his native Guyana and from Color Field painting, Mr.
Bowling's ``Drift II'' (2017), printed with bright stripes, ``is topped
with an eruption of paint as thick as melted ice cream,'' the critic
Roberta Smith wrote.

Credit...Frank Bowling/Artists Rights Society (ARS), New York; DACS,
London; via Alexander Gray Associates; Hales Gallery

Image

``Lorraine O'Grady: Cutting Out CONYT,'' was an exhibition of her work
using cut-out type from The New York Times. Credit...Lorraine
O'Grady/Artists Rights Society (ARS), New York; via Alexander Gray
Associates

Yet there's a rub for older artists whose work has never overtly
addressed identity or politics, as Mr. Bradford's work does.

Michael Rosenfeld, who represents the geometric painter William T.
Williams, 76, and the sculptor Barbara Chase-Riboud, 79, said they both
``purposefully withdrew from the commercial gallery world for decades,''
because they did not want their work to be seen through the lens of
identity.

Still, race is an undeniable factor in the market's new embrace. Rachel
Lehmann, co-owner of Lehmann Maupin, which represents Mr. Binion, said
she has asked buyers what they responded to in the artist's work.

``They were interested in the fact that it is abstract and in the fact
that this is an African-American artist --- part of the spectrum that
has been neglected in our history,'' she said.

Like Mr. Binion, Mr. Edwards has welcomed the new income, which allowed
him to buy a second home, in Senegal, and the freedom to experiment
there with tapestries, which he hopes to show in his fall exhibition at
the Alexander Gray gallery. ``I've got stacks of ideas and things to
do,'' said the sculptor.

On a personal level, he has college-age grandchildren and would ``like
to contribute as much as I can to that,'' he said. ``When the money
comes, we can use it.''

He is philosophical about all the new attention.

``Some is serious, some is fickle and some is not at all positive ---
you just have to find your way through it,'' he said.

``But I'll say this,'' he continued. ``I don't think we'll ever be out
of the western world's art world the way we were before. That just won't
happen anymore.''

Advertisement

\protect\hyperlink{after-bottom}{Continue reading the main story}

\hypertarget{site-index}{%
\subsection{Site Index}\label{site-index}}

\hypertarget{site-information-navigation}{%
\subsection{Site Information
Navigation}\label{site-information-navigation}}

\begin{itemize}
\tightlist
\item
  \href{https://help.nytimes.com/hc/en-us/articles/115014792127-Copyright-notice}{©~2020~The
  New York Times Company}
\end{itemize}

\begin{itemize}
\tightlist
\item
  \href{https://www.nytco.com/}{NYTCo}
\item
  \href{https://help.nytimes.com/hc/en-us/articles/115015385887-Contact-Us}{Contact
  Us}
\item
  \href{https://www.nytco.com/careers/}{Work with us}
\item
  \href{https://nytmediakit.com/}{Advertise}
\item
  \href{http://www.tbrandstudio.com/}{T Brand Studio}
\item
  \href{https://www.nytimes.com/privacy/cookie-policy\#how-do-i-manage-trackers}{Your
  Ad Choices}
\item
  \href{https://www.nytimes.com/privacy}{Privacy}
\item
  \href{https://help.nytimes.com/hc/en-us/articles/115014893428-Terms-of-service}{Terms
  of Service}
\item
  \href{https://help.nytimes.com/hc/en-us/articles/115014893968-Terms-of-sale}{Terms
  of Sale}
\item
  \href{https://spiderbites.nytimes.com}{Site Map}
\item
  \href{https://help.nytimes.com/hc/en-us}{Help}
\item
  \href{https://www.nytimes.com/subscription?campaignId=37WXW}{Subscriptions}
\end{itemize}
