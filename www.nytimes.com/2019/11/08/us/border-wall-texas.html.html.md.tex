Sections

SEARCH

\protect\hyperlink{site-content}{Skip to
content}\protect\hyperlink{site-index}{Skip to site index}

\href{/section/us}{U.S.}\textbar{}Under Construction in Texas: The First
New Section of Border Wall

\url{https://nyti.ms/33xrP8m}

\begin{itemize}
\item
\item
\item
\item
\item
\end{itemize}

\includegraphics{https://static01.nyt.com/images/2019/11/08/us/08borderwall-01/08borderwall-01-videoSixteenByNine3000.jpg}

\hypertarget{under-construction-in-texas-the-first-new-section-of-border-wall}{%
\section{Under Construction in Texas: The First New Section of Border
Wall}\label{under-construction-in-texas-the-first-new-section-of-border-wall}}

Eight miles of original fencing are going up in the Rio Grande Valley,
the first new wall to be built under President Trump. The administration
wants 500 total miles constructed by 2021.

Credit...

Supported by

\protect\hyperlink{after-sponsor}{Continue reading the main story}

By \href{https://www.nytimes.com/by/manny-fernandez}{Manny Fernandez}
and Mitchell Ferman

Photographs and Video by Alyssa Schukar

\begin{itemize}
\item
  Nov. 8, 2019
\item
  \begin{itemize}
  \item
  \item
  \item
  \item
  \item
  \end{itemize}
\end{itemize}

DONNA, Texas --- Two giant construction cranes tower over harvested
sugar cane fields, topped by a pair of checkered flags flapping in the
wind. At a distance along this flat, rural stretch of the Rio Grande
Valley in South Texas, the two structures standing between the cranes
resemble forlorn drive-in movie screens.

The twin panels of 18-foot-tall steel beams do not look like much ---
the only signs indicating it is a construction site warn ``Road Closed''
and ``No Trespassing'' --- but they were milestones of a sort.

Nearly three years after President Trump took office, the two steel
squares, newly installed and still incomplete about a mile north of the
Rio Grande, are the first new sections to be built of the wall the
president has promised to construct along the 1,900-mile southwestern
border.

More than two years ago, the United States Army Corps of Engineers began
drilling and taking soil samples along the border, and work has steadily
proceeded in the years since, in California, Arizona and New Mexico. But
the work until now improved and replaced existing barriers. Construction
on the first new section of border wall, where nothing stood previously,
started just south of Donna, Texas, in late October.

It began months behind schedule. It will cost about \$167 million. And
when it is done, this landmark section of the contentious wall project
--- a symbol of Mr. Trump's presidency and a flash point for his critics
--- will extend the hard border with Mexico by just eight miles.

There are hundreds more miles to go.

The government has been racing to meet a deadline --- the president's
promise to build approximately 500 miles of border fencing by the end of
2020. According to Customs and Border Protection officials, about \$9.8
billion has been set aside in funding from the Department of Defense,
the Department of Homeland Security and a Treasury Department
asset-forfeiture fund.

``We think we can get it close to 500 miles by the end of next year,
depending on certain terrain conditions,'' Mr. Trump told reporters in
September as he showed off the steel bollards recently erected near San
Diego, adding, ``We're building it at breakneck speed.''

\includegraphics{https://static01.nyt.com/images/2019/11/08/us/08borderwall-02alt/merlin_164059584_dd857d26-78db-4fb0-9322-9d0441120300-articleLarge.jpg?quality=75\&auto=webp\&disable=upscale}

Image

The steel squares being erected near Donna, Texas, are the first real
pieces of the Trump border wall.

Image

About \$9.8 billion has been set aside for roughly 500 miles of border
wall construction.

Construction of the eight miles of wall in Donna had been scheduled to
begin in February but did not fully get underway until late October. The
project will take months to complete. About 30 miles west of the Donna
site, near a state park, crews have cleared land in preparation for more
new wall construction that, along with three other sites, constitute the
work to be done in Hidalgo County.

Similar preparations are underway across other stretches of the
southwestern border.

All told, about 76 miles of replacement wall have been completed along
the border, federal officials said, meaning that more than 400 miles
must be installed in about 60 weeks to meet the deadline.

Construction executives and those involved in construction of the
previous sections of border fence built during the administration of
George W. Bush said it was a tough deadline to meet, but was indeed
possible.

``I would've told you 15 years ago: `What are you smoking?''' said
Victor Manjarrez Jr., a former Border Patrol sector chief in El Paso who
helped oversee the erection of the Bush-era border fence in the late
2000s. ``But now, the big difference is it's not just D.O.D. resources.
It's contractors now who are working a hell of a lot faster than we ever
could have imagined.''

Mr. Manjarrez said that during the earlier construction, crews completed
``anywhere from a quarter-mile to a third of a mile a day.'' Agents
overseeing the current South Texas work have been running on short sleep
and fielding phone calls at all hours of the day, a scenario familiar to
Mr. Manjarrez.

Karl Rove, then a senior presidential adviser, wanted daily updates on
the wall's progress, Mr. Manjarrez said. Mr. Rove, he said, tallied the
progress not in miles but in feet, in daily updates dubbed the ``Wall
Dashboard.''

America's southwestern border with Mexico spans four states and nearly
2,000 miles, but it has long been mostly unfenced, with barriers
covering about 650 miles. Some of the controversies that delayed and
complicated efforts by previous presidents to build new fencing continue
to hamper the current project.

Some private landowners along the Rio Grande whose property is needed
for the wall are
\href{https://www.nytimes.com/2017/05/07/us/politics/trump-wall-faces-barrier-in-texas.html}{fighting
the government in court}, with new lawsuits filed ``all the time,'' said
Ricky Garza, a lawyer with the Texas Civil Rights Project. Legal
challenges by landowners who objected when the Bush administration tried
to build a border wall have dragged on for over a decade. As of earlier
this year, there were more than 60 Bush-era cases involving landowners
still pending, Mr. Garza said.

Other delays abound. Fisher Sand \& Gravel, a North Dakota construction
contractor, challenged the government's bidding and selection process
for border-wall contracts in court and in a complaint filed with the
Government Accountability Office. The challenges were later dismissed,
but not before setting back timetables.

Image

An encampment~in Matamoros, Mexico, which is home to hundreds of
migrants seeking asylum in the United States.

For all the public debate over the need for a wall --- and the protests
and congressional standoffs that accompany it --- the actual
construction sites have operated smoothly, with little fanfare.

In South Texas, the work has unfolded in the rural areas of counties
that Hillary Clinton easily won in 2016, often communities dominated by
Mexican-Americans with substantial family and business ties across the
river in Mexico. Some local officials have passed resolutions opposing
the wall, but any public outcry has been muted. Many Hispanic Democrats
in Hidalgo County, where the Donna work is taking place, oppose the
president's tough policies targeting immigrants but support
border-security measures such as fencing; many have friends and
relatives who are Border Patrol agents.

Donna, a town of nearly 17,000 with a median household income of
\$30,000, was where Army soldiers set up a base camp last year during
Mr. Trump's troop deployment to the border. The mayor, Rick Morales,
said he had no problem with the construction work.

``I do believe that there has to be some type of barrier,'' Mr. Morales
said. ``We have a natural barrier, which is the river, but we have a lot
of drug smuggling and human smuggling in this area.''

Other elected officials have been less supportive, questioning whether a
wall is worth the trouble and the money. They discounted promises that
the construction would bring substantial new jobs.

``We have a president that says he got elected on a promise to build a
border wall,'' said the top elected official in Hidalgo County, Richard
Cortez, who serves as the county judge and is a Democrat. ``Well, I
think that we have to have better reasons to build a border wall than a
campaign promise. I'd rather spend the millions on the drainage problems
we have down here.''

Image

A Border Patrol agent apprehended a migrant hiding near Sullivan City,
Texas.

Image

Border Patrol agents said the new wall would ease the pressure in the
Rio Grande Valley Sector, the agency's busiest border zone.

Image

Forty percent of migrant apprehensions along the southwestern border
take place in the sector.

On Thursday, Border Patrol agents led reporters on a 15-minute tour of
the construction site outside Donna.

The site seemed swallowed up by the surrounding farmland. A long stretch
of concrete slab that formed the base of the two steel panels was
stamped ``15,000 lbs.'' With most of the workers gone for the day, there
was little else to see.

A Texas construction company, SLSCO, was awarded the \$167 million
contract for work at the sites in Hidalgo County last year, and is also
contracted with the federal government to do other border-wall projects.

Border Patrol officials said the wall sections under construction in
Hidalgo County would ease the pressure agents face in the Rio Grande
Valley Sector, the agency's busiest border zone. Forty percent of
migrant apprehensions along the southwestern border take place in the
sector. The new wall will likely help stop or slow illegal border
crossings, agents said, as most migrants and smugglers seek out crossing
points that are unfenced. Only 55 of the sector's 277 miles of border
have fencing.

``They're going to the point of least resistance,'' said Christian
Alvarez, a supervisory agent in the sector. ``You just have more
operational control. When there isn't a barrier there, we're trying to
control the flow.''

Advertisement

\protect\hyperlink{after-bottom}{Continue reading the main story}

\hypertarget{site-index}{%
\subsection{Site Index}\label{site-index}}

\hypertarget{site-information-navigation}{%
\subsection{Site Information
Navigation}\label{site-information-navigation}}

\begin{itemize}
\tightlist
\item
  \href{https://help.nytimes.com/hc/en-us/articles/115014792127-Copyright-notice}{©~2020~The
  New York Times Company}
\end{itemize}

\begin{itemize}
\tightlist
\item
  \href{https://www.nytco.com/}{NYTCo}
\item
  \href{https://help.nytimes.com/hc/en-us/articles/115015385887-Contact-Us}{Contact
  Us}
\item
  \href{https://www.nytco.com/careers/}{Work with us}
\item
  \href{https://nytmediakit.com/}{Advertise}
\item
  \href{http://www.tbrandstudio.com/}{T Brand Studio}
\item
  \href{https://www.nytimes.com/privacy/cookie-policy\#how-do-i-manage-trackers}{Your
  Ad Choices}
\item
  \href{https://www.nytimes.com/privacy}{Privacy}
\item
  \href{https://help.nytimes.com/hc/en-us/articles/115014893428-Terms-of-service}{Terms
  of Service}
\item
  \href{https://help.nytimes.com/hc/en-us/articles/115014893968-Terms-of-sale}{Terms
  of Sale}
\item
  \href{https://spiderbites.nytimes.com}{Site Map}
\item
  \href{https://help.nytimes.com/hc/en-us}{Help}
\item
  \href{https://www.nytimes.com/subscription?campaignId=37WXW}{Subscriptions}
\end{itemize}
