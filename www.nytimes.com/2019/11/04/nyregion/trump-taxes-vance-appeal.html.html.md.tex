Sections

SEARCH

\protect\hyperlink{site-content}{Skip to
content}\protect\hyperlink{site-index}{Skip to site index}

\href{https://www.nytimes.com/section/nyregion}{New York}

\href{https://myaccount.nytimes.com/auth/login?response_type=cookie\&client_id=vi}{}

\href{https://www.nytimes.com/section/todayspaper}{Today's Paper}

\href{/section/nyregion}{New York}\textbar{}Trump Taxes: Appeals Court
Rules President Must Turn Over 8 Years of Tax Returns

\url{https://nyti.ms/32c47Np}

\begin{itemize}
\item
\item
\item
\item
\item
\item
\end{itemize}

Advertisement

\protect\hyperlink{after-top}{Continue reading the main story}

Supported by

\protect\hyperlink{after-sponsor}{Continue reading the main story}

\hypertarget{trump-taxes-appeals-court-rules-president-must-turn-over-8-years-of-tax-returns}{%
\section{Trump Taxes: Appeals Court Rules President Must Turn Over 8
Years of Tax
Returns}\label{trump-taxes-appeals-court-rules-president-must-turn-over-8-years-of-tax-returns}}

A three-judge appeals panel said Mr. Trump's accounting firm had to
comply with a subpoena from the Manhattan district attorney, Cyrus R.
Vance, Jr.

\includegraphics{https://static01.nyt.com/images/2019/10/29/nyregion/00nytrump1/merlin_163487526_2bbc55a9-d9cc-4ea6-bb1c-bc174b7a54e6-articleLarge.jpg?quality=75\&auto=webp\&disable=upscale}

\href{https://www.nytimes.com/by/benjamin-weiser}{\includegraphics{https://static01.nyt.com/images/2018/07/16/multimedia/author-benjamin-weiser/author-benjamin-weiser-thumbLarge.png}}\href{https://www.nytimes.com/by/adam-liptak}{\includegraphics{https://static01.nyt.com/images/2018/07/13/multimedia/author-adam-liptak/author-adam-liptak-thumbLarge-v3.png}}

By \href{https://www.nytimes.com/by/benjamin-weiser}{Benjamin Weiser}
and \href{https://www.nytimes.com/by/adam-liptak}{Adam Liptak}

\begin{itemize}
\item
  Published Nov. 4, 2019Updated July 9, 2020
\item
  \begin{itemize}
  \item
  \item
  \item
  \item
  \item
  \item
  \end{itemize}
\end{itemize}

A federal appeals panel said on Monday that President Trump's accounting
firm must turn over
\href{https://www.nytimes.com/2020/08/03/nyregion/donald-trump-taxes-cyrus-vance.html}{eight
years of his personal and corporate tax returns to Manhattan
prosecutors}, a setback for the president's attempt to keep his
financial records private.

Almost immediately after the ruling, one of the president's personal
lawyers, Jay Sekulow, said Mr. Trump would appeal to the Supreme Court.
The president maintains that the Constitution shields him from any
criminal investigation.

``The issue raised in this case goes to the heart of our republic,'' Mr.
Sekulow said. ``The constitutional issues are significant.''

The case will almost certainly be the first one involving Mr. Trump's
personal conduct and business dealings to reach the high court. The
court is not required to hear the case, but the significance of the
issues involved suggests that it will. A decision on the case may come
by June, as the presidential election enters its final stages.

Other cases involving Mr. Trump are also in the pipeline. They involve
matters as diverse as demands from House Democrats for tax and business
records, a request for access to redacted portions of the report
prepared by Robert S. Mueller III, the special counsel, and challenges
to Mr. Trump's business arrangements under the Constitution's emoluments
clauses.

Last month, for instance, a divided three-judge panel of the United
States Court of Appeals for the District of Columbia Circuit
\href{https://www.nytimes.com/2019/10/11/us/politics/mazars-trump-tax-returns.html}{ruled
that Mr. Trump's accounting firm must comply} with the House Oversight
and Reform Committee's demands for eight years of his financial records.
Mr. Trump
\href{http://cdn.cnn.com/cnn/2019/images/10/25/document1.pdf}{has asked
the full appeals court} to rehear that case.

In a different case last month,
\href{https://www.nytimes.com/2019/10/25/us/politics/house-impeachment-subpoenas.html}{a
federal judge in Washington ruled} that the House Judiciary Committee
was entitled to see secret grand jury evidence gathered by Mr. Mueller.

\href{https://www.nytimes.com/2019/08/13/us/politics/trump-house-lawsuits.html?module=inline}{Mr.
Trump has fought vigorously to shield his financial records}, and
prosecutors in Manhattan have agreed not to seek the tax returns until
the case is resolved by the Supreme Court.

In its ruling on Monday, the three-judge appeals panel did not take a
position on the president's biggest argument --- that he was immune from
all criminal investigations. A lower court had called that argument
``repugnant to the nation's governmental structure and constitutional
values.''

Instead, the appeals court said the president's accounting firm, not Mr.
Trump himself, was subpoenaed for the documents, so it did not matter
whether presidents had immunity.

``We emphasize again the narrowness of the issue before us,'' the
decision read. ``This appeal does not require us to consider whether the
president is immune from indictment and prosecution while in office, nor
to consider whether the president may lawfully be ordered to produce
documents for use in a state criminal proceeding.''

Although the panel did not rule on the question of a president's
immunity from investigation, the judges still made it clear they
disagreed with Mr. Trump and thought he was unlikely to prevail on that
argument.

\hypertarget{read-the-decision}{%
\subsection{Read the decision}\label{read-the-decision}}

The Second Circuit Court of Appeals ruled on President Trump's attempt
to shield his tax returns.

\includegraphics{https://int.nyt.com/data/documenthelper/6407-trump-vance-appeal/a0279d1a81a92e6abce8/optimized/thumbnail.png}

Judge Robert A. Katzmann noted in the unanimous ruling that Mr. Trump
had conceded that his immunity would last only as long as he held office
and he could therefore be prosecuted after stepping down.

``There is no obvious reason why a state could not begin to investigate
a president during his term and, with the information secured during
that search, ultimately determine to prosecute him after he leaves
office,'' Judge Katzmann wrote for the panel of the United States Court
of Appeals for the Second Circuit.

By keeping the ruling narrowly focused on the subpoena directed at Mr.
Trump's accounting firm, the effect may be to allow the Supreme Court to
uphold the decision without having to issue a far broader ruling against
the president.

The Second Circuit appeals court typically considers cases with
three-judge panels. In addition to Judge Katzmann, the court's chief
judge, the panel included Judge Denny Chin and Judge Christopher F.
Droney.

Judge Katzmann was placed on the appeals court by former President Bill
Clinton. Judges Chin and Droney were appointed by former President
Barack Obama.

The legal fight began in late August after the office of the Manhattan
district attorney, Cyrus R. Vance Jr., a Democrat, subpoenaed Mr.
Trump's accounting firm, Mazars USA, for his tax returns and those of
his family business dating to 2011.

Prosecutors in the office are
\href{https://www.nytimes.com/2019/08/01/nyregion/trump-cohen-stormy-daniels-vance.html}{examining
the role of the president and his business in hush-money payments} made
to two women just before the 2016 presidential election.

Mr. Vance's office sought the records in connection with an
investigation into whether any New York State laws were broken when Mr.
Trump and his company, the Trump Organization, reimbursed his former
lawyer and fixer,
\href{https://www.nytimes.com/2018/11/29/nyregion/michael-cohen-trump-russia-mueller.html}{Michael
D. Cohen, for payments he made} to the adult film actress Stormy
Daniels, who claimed she had an affair with Mr. Trump.

Mr. Cohen was also involved in money paid to
\href{https://www.nytimes.com/2019/12/05/us/fox-news-mcdougal.html}{Karen
McDougal}, a Playboy model who also said she had a relationship with Mr.
Trump. The president has denied the relationships.

\href{https://www.nytimes.com/2019/09/19/nyregion/trump-tax-returns-lawsuit.html}{Mr.
Trump's lawyers sued to block the subpoena}, writing that the criminal
investigation of the president was unconstitutional. They asserted that
presidents have such unique power and responsibility that they cannot be
subject to the burden of investigations, especially from local
prosecutors who may use the criminal process for political gain.

They pointed to impeachment as the correct way to address any potential
wrongdoing by a president. Mr. Trump's lawyers also have called the
district attorney's action an ``effort to harass the president by
obtaining and exposing his confidential financial information, not a
legitimate attempt to enforce New York law.''

A spokesman for Mr. Vance said the office had no comment on the appeals
court decision.

The immunity argument has never been tested in court. Federal
prosecutors are barred from charging a sitting president with a crime
because the Justice Department has decided that presidents have
temporary immunity from prosecution while they are in office.

But that policy has not precluded investigations of the president. Mr.
Trump and other sitting presidents have been the subjects of federal
criminal investigations, and local prosecutors like Mr. Vance have not
been bound by the policy.

On Oct. 7,
\href{https://www.nytimes.com/2019/10/07/nyregion/trump-tax-returns-new-york.html}{Judge
Victor Marrero of Federal District Court in Manhattan} issued a 75-page
opinion, rejecting Mr. Trump's position.

Mr. Trump appealed to the Second Circuit, and in oral arguments last
month, William S. Consovoy, a lawyer for Mr. Trump, told the panel, ``We
view the entire subpoena as an inappropriate fishing expedition not made
in good faith.''

During the arguments, the president's immunity claim seemed to
crystallize when Judge Chin cited an audacious statement Mr. Trump once
made --- that he could stand on Fifth Avenue and shoot somebody, without
being hurt politically.

\includegraphics{https://static01.nyt.com/images/2019/10/29/nyregion/00nytrump2/merlin_163177542_b3c8c3b3-573b-4c6f-b9ee-b5708a7f66b9-articleLarge.jpg?quality=75\&auto=webp\&disable=upscale}

Judge Chin asked Mr. Consovoy about the potential effect of the
president's immunity claim in such a hypothetical situation. ``Local
authorities couldn't investigate?'' Judge Chin asked, adding: ``Nothing
could be done? That's your position?''

``That is correct. That is correct,'' Mr. Consovoy said.

The
\href{https://www.nytimes.com/2019/10/02/nyregion/trump-taxes-lawsuit.html?module=inline}{Justice
Department, led by William P. Barr, had also weighed in}, writing in
court filings that Mr. Vance's subpoena should be blocked for now but
not adopting Mr. Trump's absolutist view that a sitting president could
never be subject to criminal investigation.

Although the United States is not a party to the lawsuit, it has the
right to give its views.

In an appellate brief, the Justice Department wrote that Mr. Vance's
office should not be able to obtain the president's personal records
unless it could show that they were central to the investigation, not
available elsewhere and were needed immediately, rather than after Mr.
Trump leaves office.

``A subpoena directed at a president's records should be permitted only
as a last resort,'' the department wrote.

Under a deal reached by Mr. Trump's lawyers with Mr. Vance's office, the
subpoena will not be enforced while Mr. Trump seeks review of the
appellate ruling in the Supreme Court, provided that he asks that the
court hear the case in its current term, which ends in June.

Advertisement

\protect\hyperlink{after-bottom}{Continue reading the main story}

\hypertarget{site-index}{%
\subsection{Site Index}\label{site-index}}

\hypertarget{site-information-navigation}{%
\subsection{Site Information
Navigation}\label{site-information-navigation}}

\begin{itemize}
\tightlist
\item
  \href{https://help.nytimes.com/hc/en-us/articles/115014792127-Copyright-notice}{©~2020~The
  New York Times Company}
\end{itemize}

\begin{itemize}
\tightlist
\item
  \href{https://www.nytco.com/}{NYTCo}
\item
  \href{https://help.nytimes.com/hc/en-us/articles/115015385887-Contact-Us}{Contact
  Us}
\item
  \href{https://www.nytco.com/careers/}{Work with us}
\item
  \href{https://nytmediakit.com/}{Advertise}
\item
  \href{http://www.tbrandstudio.com/}{T Brand Studio}
\item
  \href{https://www.nytimes.com/privacy/cookie-policy\#how-do-i-manage-trackers}{Your
  Ad Choices}
\item
  \href{https://www.nytimes.com/privacy}{Privacy}
\item
  \href{https://help.nytimes.com/hc/en-us/articles/115014893428-Terms-of-service}{Terms
  of Service}
\item
  \href{https://help.nytimes.com/hc/en-us/articles/115014893968-Terms-of-sale}{Terms
  of Sale}
\item
  \href{https://spiderbites.nytimes.com}{Site Map}
\item
  \href{https://help.nytimes.com/hc/en-us}{Help}
\item
  \href{https://www.nytimes.com/subscription?campaignId=37WXW}{Subscriptions}
\end{itemize}
