Sections

SEARCH

\protect\hyperlink{site-content}{Skip to
content}\protect\hyperlink{site-index}{Skip to site index}

\href{https://www.nytimes.com/section/business}{Business}

\href{https://myaccount.nytimes.com/auth/login?response_type=cookie\&client_id=vi}{}

\href{https://www.nytimes.com/section/todayspaper}{Today's Paper}

\href{/section/business}{Business}\textbar{}How Bad Is China's Debt? A
City Hospital Is Asking Nurses for Loans

\url{https://nyti.ms/2WXXwFA}

\begin{itemize}
\item
\item
\item
\item
\item
\end{itemize}

Advertisement

\protect\hyperlink{after-top}{Continue reading the main story}

Supported by

\protect\hyperlink{after-sponsor}{Continue reading the main story}

\hypertarget{how-bad-is-chinas-debt-a-city-hospital-is-asking-nurses-for-loans}{%
\section{How Bad Is China's Debt? A City Hospital Is Asking Nurses for
Loans}\label{how-bad-is-chinas-debt-a-city-hospital-is-asking-nurses-for-loans}}

The city of Ruzhou spent big, then used its health care workers to raise
money, as local governments look for ways to keep the economy going.

\includegraphics{https://static01.nyt.com/images/2019/11/10/business/10chinadebt-1/merlin_163827165_5d59e204-3e87-43da-84a4-0f93795477fd-articleLarge.jpg?quality=75\&auto=webp\&disable=upscale}

\href{https://www.nytimes.com/by/alexandra-stevenson}{\includegraphics{https://static01.nyt.com/images/2018/02/20/multimedia/author-alexandra-stevenson/author-alexandra-stevenson-thumbLarge.jpg}}\href{https://www.nytimes.com/by/cao-li}{\includegraphics{https://static01.nyt.com/images/2018/10/15/multimedia/author-cao-li/author-cao-li-thumbLarge.png}}

By \href{https://www.nytimes.com/by/alexandra-stevenson}{Alexandra
Stevenson} and \href{https://www.nytimes.com/by/cao-li}{Cao Li}

\begin{itemize}
\item
  Nov. 10, 2019
\item
  \begin{itemize}
  \item
  \item
  \item
  \item
  \item
  \end{itemize}
\end{itemize}

RUZHOU, China --- When the call came for local doctors and nurses to
step up for their troubled community, the emergency wasn't medical. It
was financial.

Ruzhou, a city of one million people in central China, urgently needed a
new hospital, their bosses said. To pay for it, the administrators were
asking health care workers for loans. If employees didn't have the
money, they were pointed to banks where they could borrow it and then
turn it over to the hospital.

China's doctors and nurses
\href{https://www.nytimes.com/2018/09/30/business/china-health-care-doctors.html}{are
paid a small fraction} of what medical professionals make in the United
States. On message boards online and in the local media, many complained
that they felt pressured to pony up thousands of dollars they could not
afford to give.

``It's like adding insult to injury,'' a message posted to an online
government forum said. Others, speaking to state and local media, asked
why money from lowly employees was needed to build big-ticket government
projects.

Ruzhou is a city with a borrowing problem --- and an emblem of the
\href{https://www.nytimes.com/2017/05/24/business/china-downgrade-explained.html}{trillions
of dollars in debt} threatening the Chinese economy.

Local governments
\href{https://www.nytimes.com/2019/04/10/business/china-economy-debt-tianjin.html}{borrowed
for years} to create jobs and keep factories humming. Now
\href{https://www.nytimes.com/2019/10/17/business/china-economic-growth.html}{China's
economy is slowing} to its weakest pace in nearly three decades, but
Beijing
\href{https://www.nytimes.com/2018/06/14/business/economy/china-economy-debt-interest-rates.html}{has
kept the lending spigots tight} to quell its debt problems.

\includegraphics{https://static01.nyt.com/images/2019/11/10/business/10chinadebt-2/merlin_163827162_d3f5c500-601a-4f43-b86e-7b98765b7864-articleLarge.jpg?quality=75\&auto=webp\&disable=upscale}

In response, a growing number of Chinese cities are raising money using
hospitals, schools and other institutions. Often they use complicated
financial arrangements, like lease agreements or trusts, that stay a
step ahead of regulators in Beijing.

``Whether it is a financial lease or trust, they are just all tools for
local governments to borrow,'' said Chen Zhiwu, director of the Asia
Global Institute at the University of Hong Kong. ``Officials **** stop
one today, and they come up with another tool tomorrow.''

``That's why China has been talking about curbing local government debts
for many years and it's still not solved,'' Mr. Chen said.

Increasingly these deals are going sour, as they did in Ruzhou, and the
loans are going unpaid. Lenders have accused three of Ruzhou's hospitals
and three investment funds tied to the city of not paying back their
debts.

Local officials have long used big spending to keep the economy growing.
Ruzhou is home to a number of white-elephant projects, including a
stadium and sports complex turned e-commerce center, now largely unused.
A shantytown redevelopment project, begun four years ago to give rural
residents new homes, has been slowed for lack of money, locals said.

Ruzhou officials did not respond to repeated requests for comment. Two
employees of The New York Times who traveled to the city were briefly
held by the police and forced to leave.

\href{https://www.nytimes.com/2019/11/10/business/china-reporter-police.html}{{[}}\href{https://www.nytimes.com/2019/11/10/business/china-reporter-police.html}{\emph{These
days in China, the economy is a sensitive
subject}}\href{https://www.nytimes.com/2019/11/10/business/china-reporter-police.html}{.{]}}

The hidden debts of communities like Ruzhou are a major challenge to the
Communist Party. They could disrupt the financial system of the world's
second-largest economy if they cause a chain reaction and spill over
into other parts of the country and the lives of ordinary people. They
also keep Beijing from ramping up lending as a way to juice economic
growth.

Image

A construction site for a four-building cultural complex, across from
the stadium and sports complex.Credit...Gilles Sabrié for The New York
Times

Nobody is quite sure how big the problem might be. Beijing says the
total is
\href{http://yss.mof.gov.cn/zhuantilanmu/dfzgl/sjtj/201904/t20190403_3211696.html}{about
\$2.5 trillion}. Vincent Zhu, an analyst at Rhodium Group, a research
firm, puts the figure at more than \$8 trillion.

``Imagine the economy is a ship like the Titanic,'' Mr. Zhu said.
``Local government debts are like cargo containers piled up on its deck.
There are already lots of cargo containers piling up.''

Ruzhou, a town surrounded by coal mines in Henan Province, borrowed and
spent in line with China's government-driven fads, which helped
guarantee that Beijing would pay for much of it.

When Beijing stressed athletics, the city built the sports complex,
featuring a 15,466-seat stadium, an indoor basketball court and a
convention center complete with an auditorium built in the style of
Beijing's Great Hall of the People.

When technology became a priority for Chinese leaders, Ruzhou relabeled
the sports complex as the Big Data and E-Commerce center and built an
E-Commerce Mansion overlooking the stadium. Today the buildings that
house the basketball court and auditorium stand empty, available to rent
for events. During the visit by the Times employees, a break-dance group
was inspecting the auditorium as a site for a performance.

In China, building these kinds of projects requires some financial
engineering. Local governments have limited power to tax and borrow.
They depend on getting funds from the central government and selling
land to developers. That's not always enough.

To borrow more money, many set up investment-fund-type financial
companies called local government financing vehicles. They help raise
funds for big infrastructure projects without having to record their
debts publicly.

In 2008, when the government unleashed a
\href{https://www.nytimes.com/2008/11/10/world/asia/10china.html}{\$586
billion economic stimulus package} to counter the impact of the global
financial crisis, state-owned banks opened the taps and the money flowed
into these vehicles.

``You could sit at your desk at your company and banks would come to you
and ask if you needed money,'' said Gao Yinliang, a deputy director of
the financing department of Ruzhou Cultural Investment Limited, one of
these vehicles.

Image

When technology became a priority for Chinese leaders, Ruzhou relabeled
the sports complex as the Big Data and E-Commerce center and built an
E-Commerce Mansion overlooking the stadium. Credit...Gilles Sabrié for
The New York Times

Then Beijing had a change of heart. Two years ago, senior officials,
worried about hidden
\href{http://snapshot.sogoucdn.com/websnapshot?ie=utf8\&url=http\%3A\%2F\%2Fwww.cnfinance.cn\%2Fmagzi\%2F2019-03\%2F18-29459.html\&did=e4afc4d0af959068-a7454eb9f6bf81e5-d50b88fa17979eff59feb12c69f52f54\&k=281577afe7d4283eef725f3c2a1851da\&encodedQuery=\%E7\%8E\%8B\%E6\%99\%AF\%E6\%AD\%A6\%20\%E7\%81\%B0\%E7\%8A\%80\%E7\%89\%9B\&query=\%E7\%8E\%8B\%E6\%99\%AF\%E6\%AD\%A6\%20\%E7\%81\%B0\%E7\%8A\%80\%E7\%89\%9B\&\&w=01020400\&m=0\&st=0}{government
debt}, told local governments to clean it up. Beijing officials who
control the state-run banking system tightened lending.

With bills to pay, Ruzhou turned to aggressive private banks that
provided high-interest financing for public projects tied to the city's
hospitals. The city took out loans worth tens of millions of dollars and
soon had trouble paying.

Beginning late last year, these banks sued three of Ruzhou's hospitals,
Ruzhou Cultural Investment and two other government investment funds,
saying they had not paid back the more than \$45 million they owed. In
August, the Cultural Investment fund and the Hospital of Traditional
Chinese Medicine were placed on a national government blacklist, which
will limit their ability to get loans or strike other kinds of business
deals.

Mr. Gao denied being involved in borrowing money. ``We were just
implicated because we were simply the guarantor of the loans,'' he said.

After the hospitals were sued, their administrators began asking doctors
and nurses for money.

In a May memo that was reviewed by The Times, local officials urged
hospital managers to help support a local investment fund that was
selling bonds. ``We encourage managers and staff of local hospitals to
buy the above-mentioned convertible bonds to support the construction of
their own hospitals,'' the memo said.

Some hospitals took this to mean employees were required to give money,
and managers set quotas.

Doctors and nurses at the traditional Chinese medicine hospital
complained to one local state-owned newspaper that they were being
ordered to give between \$14,000 and \$28,000.

At Ruzhou Maternal and Child Health Hospital, nurses and doctors were
told they had to invest between \$8,500 and \$14,000, according to
government online forums and state media.

The government quickly backpedaled.

Zhang Yuhang, the director of Ruzhou Hospital of Traditional Chinese
Medicine, denied that the fund-raising had ever been mandatory and
blamed the hospital itself for misreading government policy.

``It's all voluntary,'' he told the local state-owned newspaper.

It is not clear what will happen to Ruzhou's unfinished projects in the
meantime. Dozens remain half-built, as if parts of the city had been
suddenly abandoned.

Across from the e-commerce center, for example, construction appeared to
have halted on a four-building cultural complex. A bright red banner
hung across one of the buildings.

``Four places holding hands,'' it said, then, invoking the government's
catchphrase for China's rise, ``jointly writing the China Dream.''

Advertisement

\protect\hyperlink{after-bottom}{Continue reading the main story}

\hypertarget{site-index}{%
\subsection{Site Index}\label{site-index}}

\hypertarget{site-information-navigation}{%
\subsection{Site Information
Navigation}\label{site-information-navigation}}

\begin{itemize}
\tightlist
\item
  \href{https://help.nytimes.com/hc/en-us/articles/115014792127-Copyright-notice}{©~2020~The
  New York Times Company}
\end{itemize}

\begin{itemize}
\tightlist
\item
  \href{https://www.nytco.com/}{NYTCo}
\item
  \href{https://help.nytimes.com/hc/en-us/articles/115015385887-Contact-Us}{Contact
  Us}
\item
  \href{https://www.nytco.com/careers/}{Work with us}
\item
  \href{https://nytmediakit.com/}{Advertise}
\item
  \href{http://www.tbrandstudio.com/}{T Brand Studio}
\item
  \href{https://www.nytimes.com/privacy/cookie-policy\#how-do-i-manage-trackers}{Your
  Ad Choices}
\item
  \href{https://www.nytimes.com/privacy}{Privacy}
\item
  \href{https://help.nytimes.com/hc/en-us/articles/115014893428-Terms-of-service}{Terms
  of Service}
\item
  \href{https://help.nytimes.com/hc/en-us/articles/115014893968-Terms-of-sale}{Terms
  of Sale}
\item
  \href{https://spiderbites.nytimes.com}{Site Map}
\item
  \href{https://help.nytimes.com/hc/en-us}{Help}
\item
  \href{https://www.nytimes.com/subscription?campaignId=37WXW}{Subscriptions}
\end{itemize}
