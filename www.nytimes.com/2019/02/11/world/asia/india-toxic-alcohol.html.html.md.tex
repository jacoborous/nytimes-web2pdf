Sections

SEARCH

\protect\hyperlink{site-content}{Skip to
content}\protect\hyperlink{site-index}{Skip to site index}

\href{https://www.nytimes.com/section/world/asia}{Asia Pacific}

\href{https://myaccount.nytimes.com/auth/login?response_type=cookie\&client_id=vi}{}

\href{https://www.nytimes.com/section/todayspaper}{Today's Paper}

\href{/section/world/asia}{Asia Pacific}\textbar{}Toxic Homemade Alcohol
Kills Scores in India

\url{https://nyti.ms/2E5DocJ}

\begin{itemize}
\item
\item
\item
\item
\item
\end{itemize}

Advertisement

\protect\hyperlink{after-top}{Continue reading the main story}

Supported by

\protect\hyperlink{after-sponsor}{Continue reading the main story}

\hypertarget{toxic-homemade-alcohol-kills-scores-in-india}{%
\section{Toxic Homemade Alcohol Kills Scores in
India}\label{toxic-homemade-alcohol-kills-scores-in-india}}

\includegraphics{https://static01.nyt.com/images/2019/02/12/world/12India-print/merlin_150522534_a819d04a-ae40-4a30-bab3-f5fa3c803486-articleLarge.jpg?quality=75\&auto=webp\&disable=upscale}

By \href{https://www.nytimes.com/by/sameer-yasir}{Sameer Yasir},
\href{https://www.nytimes.com/by/jeffrey-gettleman}{Jeffrey Gettleman}
and Ayesha Venkataraman

\begin{itemize}
\item
  Feb. 11, 2019
\item
  \begin{itemize}
  \item
  \item
  \item
  \item
  \item
  \end{itemize}
\end{itemize}

Last weekend, Pintu Kumar, a farmhand in northern India, went to a
friend's memorial service.

On the way back to his village, he bought dozens of small plastic
pouches of homemade alcohol. They weren't labeled --- they usually
aren't --- but the alcohol was incredibly strong and cheap, at about 40
cents per pouch.

It was also unusually milky in color and smelled like diesel fuel. But
that didn't stop Mr. Kumar from tearing holes in a couple of pouches and
sharing them with his friends.

Then tragedy struck.

``All of them died,'' said Bimlesh Kumar, a resident of the same
village. ``The bodies were scattered on the ground as if a massacre had
been committed.''

In the past few days, a poisonous batch of illegal homemade alcohol has
killed as many as 100 people in northern India. The deaths, which have
rattled the country and become front-page news, prompted the authorities
to crack down on underground brewers, arresting more than 3,000 suspects
and seizing tens of thousands of gallons of illicit alcohol.

Indian officials say they have traced the poisonous batch to a criminal
enterprise that brews thousands of pouches of illicit alcohol in an
underground factory hidden in the forest in Uttarakhand State. The
kingpin, they say, is on the run.

Politicians gearing up for national elections in the coming months have
been quick to seize on this disaster and blame each other. Priyanka
Gandhi Vadra, an official for the Indian National Congress party and the
latest member of the Gandhi dynasty to jump into politics, said it was
``unimaginable'' that this could happen ``on such a large scale under
the patronage of the Uttarakhand and Uttar Pradesh governments,'' two
states controlled by the rival Bharatiya Janata Party.

Most of the deaths were in a border area straddling those two states.

Many villagers have vented their fury at police officers, whom they
accuse of taking bribes from bootleggers to look the other way while
dangerous illegal brews are sold openly along the roads and in markets.

Poisonous homemade alcohol is a problem in India, particularly among the
poor. Hundreds die each year from consuming it. In 2015,
\href{https://www.bbc.com/news/world-asia-india-33224514}{at least 100
people} in a Mumbai-area slum were killed, and in 2008, in one of the
largest incidents of this kind in recent decades,
\href{https://timesofindia.indiatimes.com/city/chennai/TN-hooch-tragedy-21-cops-suspended/articleshow/3159848.cms?referral=PM}{more
than 170 people} died after drinking an illicit home brew in slum areas
of Karnataka and Tamil Nadu.

The appeal of illegal liquor is that it is cheap and potent, often far
more potent than what is sold in stores. Two of India's larger states,
Gujarat and Bihar, are dry --- though in both places, a vibrant,
scarcely concealed bootleg industry thrives.

\includegraphics{https://static01.nyt.com/images/2019/02/12/world/12India2/merlin_150523590_05e7ad2d-a7ee-4ef6-a29d-2039559d4937-articleLarge.jpg?quality=75\&auto=webp\&disable=upscale}

On Monday, protesters crammed the streets of Saharanpur, a city in Uttar
Pradesh where the authorities said about 60 people had died after
drinking bad alcohol.

Arvis Lambha, a local activist with the Bhim Army, a volunteer group
working for the welfare of Dalits, a marginalized community among
Hindus, said the hospitals in his area had completely failed.

``Whoever consumed the liquor died within minutes,'' Mr. Lambha said.
``And the majority of them were Dalits.''

Haresh Rawat, the former chief minister of Uttarakhand, blamed organized
crime for the disaster.

``We have failed to put a stop to it for decades,'' he said. ``The
administration has suspended some small officials, but leniency is being
shown to those making millions out of it.''

The alcohol seems to have been distributed over the weekend, with many
people starting to die on Saturday. Several witnesses said it was
cloudier than usual and smelled bad.

Illegal home brew is often referred to as ``desi daru,'' which means
``indigenous alcohol'' and refers to traditional legal brews as well.
The illegal varieties are often laced with methanol, a flammable liquid
used as a fuel that can lead to blindness or death when ingested.

The Uttar Pradesh police services have suspended several officers on
suspicion of allowing the illicit alcohol to cross state lines.

``The public needs to be aware that only authorized liquor shops sell
safe liquor,'' said Anand Kumar, a police official. ``Since this mafia
makes illicit liquor that is sold at cheap rates, people get lured and
fall into a trap, without knowing the ramifications.''

Among those who died was Imran Safi, who purchased a couple of pouches
of the illegal brew from a roadside shack in Saharanpur. Mr. Safi went
to his farm and drank it there. He died at a hospital two hours later
before doctors could intervene.

``He was lying tummy-down on the road, unconscious,'' said his younger
brother, Sulaiman Safi. ``When I took him to the hospital, I have never
seen so many dead people.''

But he said it was not alcohol that had killed his brother.

``My brother is dead,'' he said, ``because the government has been
allowing this poison to be sold in markets so that they can extract
money from its traders.''

Advertisement

\protect\hyperlink{after-bottom}{Continue reading the main story}

\hypertarget{site-index}{%
\subsection{Site Index}\label{site-index}}

\hypertarget{site-information-navigation}{%
\subsection{Site Information
Navigation}\label{site-information-navigation}}

\begin{itemize}
\tightlist
\item
  \href{https://help.nytimes.com/hc/en-us/articles/115014792127-Copyright-notice}{©~2020~The
  New York Times Company}
\end{itemize}

\begin{itemize}
\tightlist
\item
  \href{https://www.nytco.com/}{NYTCo}
\item
  \href{https://help.nytimes.com/hc/en-us/articles/115015385887-Contact-Us}{Contact
  Us}
\item
  \href{https://www.nytco.com/careers/}{Work with us}
\item
  \href{https://nytmediakit.com/}{Advertise}
\item
  \href{http://www.tbrandstudio.com/}{T Brand Studio}
\item
  \href{https://www.nytimes.com/privacy/cookie-policy\#how-do-i-manage-trackers}{Your
  Ad Choices}
\item
  \href{https://www.nytimes.com/privacy}{Privacy}
\item
  \href{https://help.nytimes.com/hc/en-us/articles/115014893428-Terms-of-service}{Terms
  of Service}
\item
  \href{https://help.nytimes.com/hc/en-us/articles/115014893968-Terms-of-sale}{Terms
  of Sale}
\item
  \href{https://spiderbites.nytimes.com}{Site Map}
\item
  \href{https://help.nytimes.com/hc/en-us}{Help}
\item
  \href{https://www.nytimes.com/subscription?campaignId=37WXW}{Subscriptions}
\end{itemize}
