Sections

SEARCH

\protect\hyperlink{site-content}{Skip to
content}\protect\hyperlink{site-index}{Skip to site index}

\href{https://www.nytimes.com/section/world/asia}{Asia Pacific}

\href{https://myaccount.nytimes.com/auth/login?response_type=cookie\&client_id=vi}{}

\href{https://www.nytimes.com/section/todayspaper}{Today's Paper}

\href{/section/world/asia}{Asia Pacific}\textbar{}India Blames Pakistan
for Attack in Kashmir, Promising a Response

\url{https://nyti.ms/2V0ZuTc}

\begin{itemize}
\item
\item
\item
\item
\item
\item
\end{itemize}

Advertisement

\protect\hyperlink{after-top}{Continue reading the main story}

Supported by

\protect\hyperlink{after-sponsor}{Continue reading the main story}

\hypertarget{india-blames-pakistan-for-attack-in-kashmir-promising-a-response}{%
\section{India Blames Pakistan for Attack in Kashmir, Promising a
Response}\label{india-blames-pakistan-for-attack-in-kashmir-promising-a-response}}

\includegraphics{https://static01.nyt.com/images/2019/02/16/world/16india-kashmir1/merlin_150718950_87abe1a3-0d82-467e-804d-c4e2534f84a3-articleLarge.jpg?quality=75\&auto=webp\&disable=upscale}

By \href{https://www.nytimes.com/by/maria-abi-habib}{Maria Abi-Habib},
\href{https://www.nytimes.com/by/sameer-yasir}{Sameer Yasir} and
\href{https://www.nytimes.com/by/hari-kumar}{Hari Kumar}

\begin{itemize}
\item
  Feb. 15, 2019
\item
  \begin{itemize}
  \item
  \item
  \item
  \item
  \item
  \item
  \end{itemize}
\end{itemize}

NEW DELHI --- India accused Pakistan on Friday of orchestrating a
suicide bombing that killed dozens of soldiers in Kashmir, the worst
attack there in decades, promising an appropriate response and calling
on world leaders to isolate its neighbor.

Pakistan has denied involvement in the attack, in which at least 40
Indian soldiers were killed Thursday when a driver slammed an
explosives-packed vehicle into a paramilitary convoy. But by Friday
afternoon, India had recalled its ambassador to Pakistan for
consultations in New Delhi.

{[}\emph{\href{https://www.nytimes.com/2019/08/05/world/asia/india-pakistan-kashmir-jammu.html}{On
August 5, India revoked Kashmir's special status.}}{]}

With national elections in India set to take place by May and Prime
Minister Narendra Modi facing a close contest, analysts say he risks
looking weak if he does not respond. Mr. Modi was elected in 2014 on
promises to crack down on Kashmir's militants and to adopt a tougher
line on Pakistan. The nuclear-armed rivals have gone to war three times
since independence in 1947, with two of the wars fought over Kashmir.

``We will give a befitting reply; our neighbor will not be allowed to
destabilize us,'' Mr. Modi said after an emergency meeting with security
advisers on Friday, according to Reuters. ``Our security forces are
given full freedom'' to respond, he added.

Finance Minister Arun Jaitley said India would use all diplomatic means
to ``ensure the complete isolation from the international community of
Pakistan, of which incontrovertible evidence is available of having a
direct hand in this gruesome terrorist incident.''

The streets of Jammu, in the part of the disputed Himalayan region that
India controls, were generally quiet on Friday after a curfew was
imposed. But anti-Pakistan protests broke out in parts of India, with
demonstrators calling on the government to retaliate.

Scores poured into Delhi's streets, wearing the saffron-colored scarves
of Mr. Modi's Hindu nationalist party, pumping their fists and waving
signs that read: ``Attack Pakistan. Crush it.''

But India's options for putting diplomatic pressure on Pakistan are
limited. Pakistan is largely shielded by its alliance with China, which
has used its veto power at the United Nations Security Council to
protect it, while propping up Pakistan's sputtering, increasingly
isolated economy. Pakistan has grown closer to China as its relations
with the United States have broken down over the past decade.

\includegraphics{https://static01.nyt.com/images/2019/02/16/world/16india-kashmir2/merlin_150716640_85637305-4e38-4490-8974-a7555e717fc6-articleLarge.jpg?quality=75\&auto=webp\&disable=upscale}

India has renewed its call for the United Nations to blacklist Masood
Azhar, the leader of the militant group linked to Thursday's attack,
Jaish-e-Muhammad, or Army of Muhammad. But a Chinese Foreign Ministry
spokesman
\href{https://timesofindia.indiatimes.com/india/china-again-says-no-to-back-indias-bid-to-list-jem-chief-masood-azhar-as-global-terrorist-by-un/articleshow/68005554.cms}{rebuffed
the demand on Friday}.

Putting Mr. Azhar personally on a terrorist blacklist would deliver a
financial blow to Jaish-e-Muhammad. Although the group is banned in
Pakistan, Indian and American officials say it operates and raises funds
in the country under different names.

For years, the United States has tried to get Mr. Azhar designated as an
individual terrorist by the United Nations Security Council, but China
has always blocked the move, a senior American official said on Friday.

On Thursday, the White House demanded that Pakistan end its support to
terrorists, adding that this week's ``attack only strengthens our
resolve to bolster counterterrorism cooperation and coordination between
the United States and India.''

Pakistan has long denied any links to terrorist groups and has bristled
at Washington's warming ties with New Delhi.

India also ended its preferential trade status for Pakistan on Friday
--- a limited move, since their bilateral trade amounts to a
comparatively small \$2 billion annually.

India's options for a military response are also limited, analysts say,
with the disputed border blanketed in thick snow and Pakistani troops on
high alert.

The last time Jaish-e-Muhammad staged a major attack, in 2016, it
infiltrated an Indian Army base in the town of Uri, Kashmir, and killed
19 soldiers in a predawn raid. India's military responded then with what
it described as ``surgical strikes'' in Pakistan.

Image

The site of the attack on Thursday. A bomber drove a vehicle packed with
explosives into an army convoy.Credit...Farooq Khan/EPA, via
Shutterstock

But the nature of Thursday's bombing suggests the insurgency is adapting
and becoming more homegrown, leaving observers to question how deep the
links to Pakistan really run.

The militant who claimed responsibility for the attack, Aadil Ahmad Dar,
was from a village about six miles from where the Indian convoy was
struck, in contrast to the fighters and weapons that once streamed in
from Pakistani-occupied areas to sustain the insurgency. And the
explosives he packed into his car appear to have been locally procured,
security experts said.

An insurgency that was once stoked by Pakistan may have taken on a life
of its own, as Kashmiris become more disenfranchised and angry at the
central government in Delhi and its use of force.

Mr. Dar was a high school dropout working as a day laborer when he
disappeared last March, said his father, Ghulam Hassan Dar, a farmer.
The family searched for their son in vain, until neighbors showed the
Dars a photograph on their phone featuring their son, an automatic rifle
in hand, surrounded by the insignia of Jaish-e-Muhammad.

``I was broken when I saw that picture,'' Ghulam Dar said. ``I knew I
will soon have to shoulder his coffin.'' He added, ``I told my wife our
son was gone forever.''

Friends of the attacker dispute how he turned to militancy. Some say it
was after he was wounded at a protest in 2016, where his leg was struck
by a bullet fired by the Central Reserve Police Force, a paramilitary
unit.

Many Kashmiris loathe the paramilitary unit, viewing it as an occupying
force recruited from across India to suppress them. Mr. Dar's attack on
Thursday was aimed at the force, whose use of pellet guns against
protesters has
\href{https://www.nytimes.com/2016/08/29/world/asia/pellet-guns-used-in-kashmir-protests-cause-dead-eyes-epidemic.html}{blinded
scores of people}.

Others say Mr. Dar was ideologically drawn to Jaish-e-Muhammad,
believing Kashmir should be led by Pakistan, as a Muslim-majority
nation. Kashmir's fate was undecided when the British partitioned India
in 1947. Since then, India has ignored United Nations resolutions to
hold a referendum in the disputed territory, allowing local residents to
decide whether they want to join India or Pakistan.

Image

Students in Amritsar, India, paid tribute to the slain soldiers on
Friday. The attack was the worst in Kashmir in decades.Credit...Narinder
Nanu/Agence France-Presse --- Getty Images

The local news media reported that Mr. Dar used more than 750 pounds of
explosives against the convoy.

``It is not possible to bring such massive amounts of explosives by
infiltrating the border,'' said Lt. Gen. D.S. Hooda, an army commander.

General Hooda added that the material may have been taken from stashes
of explosives being used to blast a mountainside to broaden the highway
to Jammu, the same road where the attack occurred.

{[}In a later interview, the general clarified that he was not ruling
out that the explosives had come from Pakistan but asserting that it
would be very difficult to smuggle in that much material.{]}

The attack has prompted new questions about how tenable Mr. Modi's
hard-line strategy in Kashmir is. India has about 250,000 armed forces
in Kashmir, making it one of the most militarized corners of the world.
The armed presence affects everyday life for most locals, whose
farmland, homes or schools are overshadowed by the military presence.

Happymon Jacob, a professor at Jawaharlal Nehru University in Delhi who
tracks the conflict, said that only a handful of Kashmiri youth joined
the insurgency in 2013 --- the year before Mr. Modi came to power ---
compared with more than 150 last year.

``They aren't joining the militants from Islamic seminaries, but they're
fresh graduates from engineering schools, or they hold jobs. For an
entire generation to be so angry with India says Delhi's policy has been
a failure,'' Professor Jacob said.

He added that the central government had not tried to engage local
people or find meaningful alliances with local politicians. Last year,
Mr. Modi's governing party ended its alliance with a powerful regional
party, leaving the state under the central government's control.

Gowher Nazir, who lives in a village adjacent to Mr. Dar's, said:
``These rebels were once dreamers, looking forward to living their
lives. But they have been pushed to a wall.''

Advertisement

\protect\hyperlink{after-bottom}{Continue reading the main story}

\hypertarget{site-index}{%
\subsection{Site Index}\label{site-index}}

\hypertarget{site-information-navigation}{%
\subsection{Site Information
Navigation}\label{site-information-navigation}}

\begin{itemize}
\tightlist
\item
  \href{https://help.nytimes.com/hc/en-us/articles/115014792127-Copyright-notice}{©~2020~The
  New York Times Company}
\end{itemize}

\begin{itemize}
\tightlist
\item
  \href{https://www.nytco.com/}{NYTCo}
\item
  \href{https://help.nytimes.com/hc/en-us/articles/115015385887-Contact-Us}{Contact
  Us}
\item
  \href{https://www.nytco.com/careers/}{Work with us}
\item
  \href{https://nytmediakit.com/}{Advertise}
\item
  \href{http://www.tbrandstudio.com/}{T Brand Studio}
\item
  \href{https://www.nytimes.com/privacy/cookie-policy\#how-do-i-manage-trackers}{Your
  Ad Choices}
\item
  \href{https://www.nytimes.com/privacy}{Privacy}
\item
  \href{https://help.nytimes.com/hc/en-us/articles/115014893428-Terms-of-service}{Terms
  of Service}
\item
  \href{https://help.nytimes.com/hc/en-us/articles/115014893968-Terms-of-sale}{Terms
  of Sale}
\item
  \href{https://spiderbites.nytimes.com}{Site Map}
\item
  \href{https://help.nytimes.com/hc/en-us}{Help}
\item
  \href{https://www.nytimes.com/subscription?campaignId=37WXW}{Subscriptions}
\end{itemize}
