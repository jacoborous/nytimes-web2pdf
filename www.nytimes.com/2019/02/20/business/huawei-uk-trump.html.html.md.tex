Sections

SEARCH

\protect\hyperlink{site-content}{Skip to
content}\protect\hyperlink{site-index}{Skip to site index}

\href{https://www.nytimes.com/section/business}{Business}

\href{https://myaccount.nytimes.com/auth/login?response_type=cookie\&client_id=vi}{}

\href{https://www.nytimes.com/section/todayspaper}{Today's Paper}

\href{/section/business}{Business}\textbar{}Huawei Risks to Britain Can
Be Blunted, U.K. Official Says, in a Rebuff to U.S.

\url{https://nyti.ms/2V7w5Xx}

\begin{itemize}
\item
\item
\item
\item
\item
\end{itemize}

Advertisement

\protect\hyperlink{after-top}{Continue reading the main story}

Supported by

\protect\hyperlink{after-sponsor}{Continue reading the main story}

\hypertarget{huawei-risks-to-britain-can-be-blunted-uk-official-says-in-a-rebuff-to-us}{%
\section{Huawei Risks to Britain Can Be Blunted, U.K. Official Says, in
a Rebuff to
U.S.}\label{huawei-risks-to-britain-can-be-blunted-uk-official-says-in-a-rebuff-to-us}}

\includegraphics{https://static01.nyt.com/images/2019/02/21/business/21HUAWEI1/merlin_150847260_47be6f04-a1f6-4505-bd3a-19a2b536340a-articleLarge.jpg?quality=75\&auto=webp\&disable=upscale}

By \href{https://www.nytimes.com/by/adam-satariano}{Adam Satariano}

\begin{itemize}
\item
  Feb. 20, 2019
\item
  \begin{itemize}
  \item
  \item
  \item
  \item
  \item
  \end{itemize}
\end{itemize}

\href{https://cn.nytimes.com/business/20190221/huawei-uk-trump/}{阅读简体中文版}\href{https://cn.nytimes.com/business/20190221/huawei-uk-trump/zh-hant/}{閱讀繁體中文版}

LONDON --- The Trump administration has spent a year trying to convince
America's allies in Europe that the Chinese telecommunications giant
Huawei is a grave threat to their national security and should not be
allowed any role in developing new wireless networks.

A top British official indicated Wednesday that the aggressive campaign
may not be working.

The official, Ciaran Martin, who leads Britain's National Cyber Security
Center, expressed confidence at a conference in Brussels that any
security risks Huawei posed could be managed.

Britain, Mr. Martin noted, has successfully managed the company's
presence in the country's telecommunications networks for more than 15
years by subjecting its products to strict security reviews at a
laboratory run by government intelligence officials, and would continue
to do so.

``Our regime is arguably the toughest and most rigorous oversight regime
in the world for Huawei,'' he said. He added that the company's
equipment ``is not in any sensitive networks, including those of the
government.''

``Its kit is part of a balanced supply chain with other suppliers,'' Mr.
Martin said.

As Britain's cellphone carriers begin to build 5G networks, officials
are considering if, and how, Huawei fits into the effort. With a final
decision expected by the end of the year, Mr. Martin's remarks suggest
the British government is unmoved by the Trump administration's
offensive against the company.

Because Britain belongs to the Five Eyes intelligence-sharing alliance
with the United States, whatever it decides on Huawei is likely to
affect how other countries treat the company. Germany is also
considering allowing Huawei in parts of its network,
\href{https://www.wsj.com/articles/in-rebuke-to-u-s-germany-considers-letting-huawei-in-11550577810}{The
Wall Street Journal} reported on Tuesday.

Europe has become a key battleground in the debate over Huawei. While
the company has mostly been blocked in the United States, it is well
established in Europe, working closely with carriers like Deutsche
Telekom, Vodafone and BT Group. The region is Huawei's second-largest
market after China.

\includegraphics{https://static01.nyt.com/images/2019/02/21/business/21huawei3/merlin_149923791_ae8ca94c-e4ec-486a-ad45-8f7a26f052a2-articleLarge.jpg?quality=75\&auto=webp\&disable=upscale}

If Britain does not ban Huawei, it would be a defeat for the White
House. In private intelligence briefings and public speeches, American
officials have warned that Huawei is beholden to the Chinese government,
and that countries allowing its equipment to be installed as part of 5G
networks will open themselves up to espionage.

Vice President Mike Pence, in
\href{https://www.whitehouse.gov/briefings-statements/remarks-vice-president-pence-2019-munich-security-conference-munich-germany/}{a
speech last week} at an international security conference in Munich,
took a direct swipe at Huawei, warning America's allies ``to be vigilant
and to reject any enterprise that would compromise the integrity of our
communications technology or national security systems.''

Administration officials have also said that decisions about where the
United States puts military bases and troops could be affected by
whether countries' networks have such equipment.

On Thursday,
\href{https://twitter.com/realDonaldTrump/status/1098581869233344512?ref_src=twsrc\%5Egoogle\%7Ctwcamp\%5Eserp\%7Ctwgr\%5Etweet}{President
Trump posted on Twitter} that American companies needed to step up the
deployment 5G technology or risk falling behind other countries.

\emph{{[}}\href{https://www.nytimes.com/2018/12/31/technology/personaltech/5g-what-you-need-to-know.html}{\emph{What
is 5G?}} \emph{Here's what you need to know about the next-generation
network.{]}}

Huawei has forcefully denied accusations that it is an instrument of the
Chinese government. The company is the world's largest maker of
telecommunications equipment --- selling antennas, base stations and
other products used by the operators of the world's largest wireless
networks.

The 5G networks are considered critical to the global economy, providing
not just hyper-fast internet speeds, but also new capabilities for
sensors, robots, autonomous vehicles, and other data-hungry devices and
services. In Europe alone, mobile operators are expected to invest at
least \$340 billion to develop the networks, according to the wireless
trade body GSMA.

Huawei's founder,
\href{https://www.nytimes.com/2019/02/18/technology/huawei-ren-zhengfei-bbc.html}{Ren
Zhengfei}, whose daughter, Meng Wanzhou, also a top company executive,
was
\href{https://www.nytimes.com/2018/12/05/business/huawei-cfo-arrest-canada-extradition.html?module=inline}{arrested
in December by the Canadian authorities} at the request of the United
States, denounced America's campaign in an interview with BBC this week.
He called the actions ``politically motivated,'' and vowed that there
was ``no way the U.S. can crush'' the company.

In Britain, one of Huawei's most important markets, the government's
treatment of the company has long been debated. A report published on
Wednesday by the Royal United Services Institute, a defense think tank,
said that ``allowing Huawei's participation is at best naïve, at worst
irresponsible.''

Image

Richard Yu, chief executive of the Huawei Consumer Business Group,
showing the company's 5G chip Balong 5G01 before the Mobile World
Congress in Barcelona, Spain, last year.Credit...Albert Gea/Reuters

The British strategy for dealing with Huawei has traditionally involved
containment. The country operates a research lab outside London to
review Huawei's products and code, and publishes an annual review of the
company's technology. Last year, Britain criticized Huawei for
engineering and supply-chain flaws.

``We will monitor and report on progress, and we will not declare the
problems are on the path to being solved unless and until there is clear
evidence that this is the case,'' Mr. Martin said in his remarks on
Wednesday. ``We will not compromise on the improvements we need to see
from Huawei.''

American officials have argued that 5G networks are much more complex
than existing systems, and that the many lines of constantly updating
code make the systems nearly impossible to protect entirely.

Peter Chase, a senior fellow at the German Marshall Fund specializing in
trans-Atlantic policy, said that a determination by British officials
that Huawei could be handled as a manageable risk would undercut the
argument that the company posed an existential risk.

``They've made a pretty hardheaded evaluation that the United States was
exaggerating the extent of the problem,'' said Mr. Chase, a former
American diplomat to London. ``I don't think they did that to please the
Chinese.''

In his speech in Brussels, Mr. Martin said cybersecurity risks were not
confined to one company.

``The supply chain, and where suppliers are from, is one issue, but it
is not the only issue,'' he said. Last year, he added, his organization
``publicly attributed some attacks on U.K. networks, including telecoms
networks, to Russia.''

``As far as we know, those networks didn't have any Russian kit in them,
anywhere,'' he said.

The dispute over what risk the company presents has complicated
Britain's efforts to avoid becoming entangled in the trade war between
the United States and China. As it prepares to exit the European Union,
Britain is seeking new trade deals with both Washington and Beijing.

China is a small but growing trade partner with Britain. Businesses
exported a record 22.3 billion pounds, about \$28.8 billion, worth of
goods to China in 2017, making it Britain's sixth-largest trading
partner. The United States is the largest, accounting for £112.2 billion
worth of exports in 2017, according to a
\href{https://researchbriefings.parliament.uk/ResearchBriefing/Summary/CBP-7379}{report}
from Parliament this month.

Advertisement

\protect\hyperlink{after-bottom}{Continue reading the main story}

\hypertarget{site-index}{%
\subsection{Site Index}\label{site-index}}

\hypertarget{site-information-navigation}{%
\subsection{Site Information
Navigation}\label{site-information-navigation}}

\begin{itemize}
\tightlist
\item
  \href{https://help.nytimes.com/hc/en-us/articles/115014792127-Copyright-notice}{©~2020~The
  New York Times Company}
\end{itemize}

\begin{itemize}
\tightlist
\item
  \href{https://www.nytco.com/}{NYTCo}
\item
  \href{https://help.nytimes.com/hc/en-us/articles/115015385887-Contact-Us}{Contact
  Us}
\item
  \href{https://www.nytco.com/careers/}{Work with us}
\item
  \href{https://nytmediakit.com/}{Advertise}
\item
  \href{http://www.tbrandstudio.com/}{T Brand Studio}
\item
  \href{https://www.nytimes.com/privacy/cookie-policy\#how-do-i-manage-trackers}{Your
  Ad Choices}
\item
  \href{https://www.nytimes.com/privacy}{Privacy}
\item
  \href{https://help.nytimes.com/hc/en-us/articles/115014893428-Terms-of-service}{Terms
  of Service}
\item
  \href{https://help.nytimes.com/hc/en-us/articles/115014893968-Terms-of-sale}{Terms
  of Sale}
\item
  \href{https://spiderbites.nytimes.com}{Site Map}
\item
  \href{https://help.nytimes.com/hc/en-us}{Help}
\item
  \href{https://www.nytimes.com/subscription?campaignId=37WXW}{Subscriptions}
\end{itemize}
