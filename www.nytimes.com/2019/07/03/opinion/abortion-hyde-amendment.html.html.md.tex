Sections

SEARCH

\protect\hyperlink{site-content}{Skip to
content}\protect\hyperlink{site-index}{Skip to site index}

\href{https://myaccount.nytimes.com/auth/login?response_type=cookie\&client_id=vi}{}

\href{https://www.nytimes.com/section/todayspaper}{Today's Paper}

\href{/section/opinion}{Opinion}\textbar{}I Shouldn't Be Forced to Give
Birth to a Baby Who Won't Live

\href{https://nyti.ms/32gVdjc}{https://nyti.ms/32gVdjc}

\begin{itemize}
\item
\item
\item
\item
\item
\item
\end{itemize}

Advertisement

\protect\hyperlink{after-top}{Continue reading the main story}

\href{/section/opinion}{Opinion}

Supported by

\protect\hyperlink{after-sponsor}{Continue reading the main story}

\hypertarget{i-shouldnt-be-forced-to-give-birth-to-a-baby-who-wont-live}{%
\section{I Shouldn't Be Forced to Give Birth to a Baby Who Won't
Live}\label{i-shouldnt-be-forced-to-give-birth-to-a-baby-who-wont-live}}

Our baby had a fatal birth defect. My federal health insurance plan
refused to cover the abortion.

By Sarah E. Levin

Ms. Levin is a federal employee in Pennsylvania.

\begin{itemize}
\item
  July 3, 2019
\item
  \begin{itemize}
  \item
  \item
  \item
  \item
  \item
  \item
  \end{itemize}
\end{itemize}

\includegraphics{https://static01.nyt.com/images/2019/07/02/opinion/02Levin/02Levin-articleLarge.jpg?quality=75\&auto=webp\&disable=upscale}

When I was 20 weeks pregnant, I and my husband learned during a routine
ultrasound that our baby had not developed a major portion of her brain
and never would. The condition, anencephaly, a type of neural tube
defect that also stunts the growth of the skull, is terminal. If carried
to term, our baby would be very unlikely to survive for more than a few
hours.

One in 1,000 fetuses have this condition. We had no warning signs. No
indications. No idea this was coming. This was a baby we had planned
for. Just three weeks earlier we had told our 5-year-old daughter that
she would soon have a baby sister. We returned home from the hospital
that day and had to tell her that her sister was not coming any more. It
was the first time she saw me sobbing, unable to speak.

We made the decision to terminate the pregnancy immediately. Then came
the roadblocks.

I am a federal worker, and the Hyde Amendment, passed by Congress in
1976, barred my health insurance company from covering my abortion, just
as it does for the millions of other women who are federal employees and
for the millions of women who are federal Medicaid recipients. The
amendment allows abortion coverage only if the pregnancy will endanger a
woman's life or is the result of rape or incest. Some states use their
own funds to cover abortions that don't fall within those bounds.
Pennsylvania, where I live, is not one of them.

I'm lucky to be a federal employee in some respects. I benefited from
regular prenatal care that was entirely covered by my insurer. I
benefited all the way until I needed to have an abortion, when my health
care coverage disappeared --- at the time I needed it most.

Because I was in my second trimester, my abortion this past June cost
\$2,500 up front, not including anesthesia and pathology testing;
anesthesia, alone, usually costs an additional \$1,100. If I were unable
to afford the upfront costs, as would be the case for many Medicaid
recipients, I would have had to carry my pregnancy to term.

Lost in the conversation about forcing women to carry to term is any
acknowledgment of the mental toll it can have, especially on those of us
whose baby is likely to be stillborn, as about 75 percent of those with
anencephaly are, or to die shortly after birth.

What about my husband, who would also miss work, share in our trauma,
and require his own mental health care to work through his pain?

And then there is my 5-year-old daughter, who would have to bear my
grief while watching my pregnant stomach swell for another 20 weeks, and
know that a baby is coming, but not one who would ever be able to be a
sister to her.

I was fortunate enough, despite the financial burden, to still have a
choice. Some states have recently passed laws banning abortions.

Still, the decision to terminate didn't lead to the fast abortion I had
hoped for. First, I had to sit through the state-mandated counseling as
laid out in Pennsylvania's Abortion Control Act, which is designed to
dissuade me from having an abortion. The law also
\href{https://www.plannedparenthood.org/planned-parenthood-southeastern-pennsylvania/patients/pa-abortion-control-act}{requires
a 24-hour waiting period} after ``counseling'' before the two-day
procedure could begin. My hospital provided counseling only on Mondays
at 4 p.m., so I had to wait a week. That's a week in which I felt my
baby kick constantly, a week in which my family began to mourn a loss
that I hadn't even begun to grieve for, because how could I when she was
still growing inside of me?

Waiting one week for the procedure was cruel. Waiting 20 more weeks
would have been intolerable.

This abortion wasn't a choice. It was an urgent medical necessity.

Unfortunately, the courts have sided with the government when the
amendment has been challenged. In 2002, a federal worker who also had a
baby with anencephaly sued the government for its refusal to pay for her
abortion. She won the first case but then
\href{https://www.latimes.com/archives/la-xpm-2005-aug-19-na-milhealth19-story.html}{lost
the appeal}. That means that the federal government continues to choose
to subsidize the increased cost of delivering a baby with anencephaly as
compared to ending the pregnancy. Deliveries typically cost \$30,000 and
\$50,000, and many anencephalic babies are delivered by C-section, which
is even more costly and carries significant risk to the mother. Then
there's the medical care required during the short span of life the baby
may have (very few have lived past their first birthday).

This cannot be what we want for our federal workers, their families, and
for Medicaid recipients who are some of the poorest and most vulnerable
in society. My employer, the largest in the country, made an immoral
decision to refuse me healthcare. It imposed restrictions upon me that
were founded on a religious belief --- not my own --- that hurts my
ability to be a productive public servant.

Sarah Levin is a public defender in Western Pennsylvania.

\emph{The Times is committed to publishing}
\href{https://www.nytimes.com/2019/01/31/opinion/letters/letters-to-editor-new-york-times-women.html}{\emph{a
diversity of letters}} \emph{to the editor. We'd like to hear what you
think about this or any of our articles. Here are some}
\href{https://help.nytimes.com/hc/en-us/articles/115014925288-How-to-submit-a-letter-to-the-editor}{\emph{tips}}\emph{.
And here's our email:}
\href{mailto:letters@nytimes.com}{\emph{letters@nytimes.com}}\emph{.}

\emph{Follow The New York Times Opinion section on}
\href{https://www.facebook.com/nytopinion}{\emph{Facebook}}\emph{,}
\href{http://twitter.com/NYTOpinion}{\emph{Twitter (@NYTopinion)}}
\emph{and}
\href{https://www.instagram.com/nytopinion/}{\emph{Instagram}}\emph{.}

Advertisement

\protect\hyperlink{after-bottom}{Continue reading the main story}

\hypertarget{site-index}{%
\subsection{Site Index}\label{site-index}}

\hypertarget{site-information-navigation}{%
\subsection{Site Information
Navigation}\label{site-information-navigation}}

\begin{itemize}
\tightlist
\item
  \href{https://help.nytimes.com/hc/en-us/articles/115014792127-Copyright-notice}{©~2020~The
  New York Times Company}
\end{itemize}

\begin{itemize}
\tightlist
\item
  \href{https://www.nytco.com/}{NYTCo}
\item
  \href{https://help.nytimes.com/hc/en-us/articles/115015385887-Contact-Us}{Contact
  Us}
\item
  \href{https://www.nytco.com/careers/}{Work with us}
\item
  \href{https://nytmediakit.com/}{Advertise}
\item
  \href{http://www.tbrandstudio.com/}{T Brand Studio}
\item
  \href{https://www.nytimes.com/privacy/cookie-policy\#how-do-i-manage-trackers}{Your
  Ad Choices}
\item
  \href{https://www.nytimes.com/privacy}{Privacy}
\item
  \href{https://help.nytimes.com/hc/en-us/articles/115014893428-Terms-of-service}{Terms
  of Service}
\item
  \href{https://help.nytimes.com/hc/en-us/articles/115014893968-Terms-of-sale}{Terms
  of Sale}
\item
  \href{https://spiderbites.nytimes.com}{Site Map}
\item
  \href{https://help.nytimes.com/hc/en-us}{Help}
\item
  \href{https://www.nytimes.com/subscription?campaignId=37WXW}{Subscriptions}
\end{itemize}
