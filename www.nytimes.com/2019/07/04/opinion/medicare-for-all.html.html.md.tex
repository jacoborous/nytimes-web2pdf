Sections

SEARCH

\protect\hyperlink{site-content}{Skip to
content}\protect\hyperlink{site-index}{Skip to site index}

\href{https://myaccount.nytimes.com/auth/login?response_type=cookie\&client_id=vi}{}

\href{https://www.nytimes.com/section/todayspaper}{Today's Paper}

\href{/section/opinion}{Opinion}\textbar{}How to Straighten Out the
Medicare Maze

\href{https://nyti.ms/2XK49hn}{https://nyti.ms/2XK49hn}

\begin{itemize}
\item
\item
\item
\item
\item
\item
\end{itemize}

Advertisement

\protect\hyperlink{after-top}{Continue reading the main story}

\href{/section/opinion}{Opinion}

Supported by

\protect\hyperlink{after-sponsor}{Continue reading the main story}

\hypertarget{how-to-straighten-out-the-medicare-maze}{%
\section{How to Straighten Out the Medicare
Maze}\label{how-to-straighten-out-the-medicare-maze}}

Expanding insurance coverage isn't the only task ahead.

By Pamela Herd and Donald P. Moynihan

The authors are professors of public policy at Georgetown.

\begin{itemize}
\item
  July 4, 2019
\item
  \begin{itemize}
  \item
  \item
  \item
  \item
  \item
  \item
  \end{itemize}
\end{itemize}

\includegraphics{https://static01.nyt.com/images/2019/07/04/opinion/04herdmoynihan/345a7b24a1204c5186187aa9a45c7a2f-articleLarge.jpg?quality=75\&auto=webp\&disable=upscale}

``Medicare for all'' was a central theme in the initial Democratic
debates and promises to be a defining issue in the primaries. While
nearly all the candidates support expanded access, they should be
pressed on another crucial question: How will they reduce the burdens
involved in dealing with the interwoven public and private insurance
systems that provide our health care coverage?

As parents of a child with a disability caused by a rare genetic
syndrome, we've wasted hundreds of hours sorting out enrollment choices,
completing unending forms and engaging in maddeningly repetitious
conversations, all to ensure that our daughter receives the care she
needs and that we don't get stuck with financially devastating bills.

While many other Americans continue to struggle with these problems,
ours have mostly disappeared because we are spending the year in
Britain. In its National Health Service, we found a system that did not
demand an expertise in navigating bureaucracies. After 10 minutes
filling out a few simple forms, we enrolled our daughter. Within two
days she had an appointment and a filled prescription for medication,
which was free.

We had anticipated the financial relief that can come from a
single-payer system, but not the administrative relief. It had never
occurred to us that it could be so different.

\href{https://www.nytimes.com/2019/06/23/us/politics/2020-democrats-medicare-for-all-public-option.html?module=inline}{All
the likely} Democratic presidential nominees are on board with
\href{https://www.kff.org/medicare/issue-brief/medicare-for-all-and-public-plan-buy-in-proposals-overview-and-key-issues/}{expanding
coverage}, but they disagree on the path forward. Some favor allowing
people to buy into the existing Medicare system. Others support the idea
of Medicare for all, but disagree on whether it would be a comprehensive
single-payer system and what role private insurance would play. The
focus on expanding access has left little room for discussion of the
frustrations embedded in the current system.

Even if Democrats sweep the 2020 elections, the incrementalist history
of health policy reform in the United States suggests that an expansion
of the current Medicare program is the most probable outcome. And yet
the sizable role private insurers already play in Medicare is largely
overlooked, even as they cause substantial administrative burdens for
beneficiaries.

\href{https://www.kff.org/medicare/issue-brief/a-dozen-facts-about-medicare-advantage/}{More
than one-third of Medicare beneficiaries} are covered by private
insurers, in what is known as the Medicare Advantage program. Many of
the remaining beneficiaries have private insurance coverage,
\href{https://www.kff.org/medicare/issue-brief/an-overview-of-medicare/}{through
Medigap and Medicare Part D prescription drug coverage} or their former
employers, to help offset the health care costs not covered by Medicare
Parts A and B, which amount to
\href{https://onlinelibrary.wiley.com/doi/full/10.1111/j.1475-5890.2016.12106}{almost
half} of the overall cost of their care. In fact, 44 percent of Medicare
dollars goes through private insurance plans and a
\href{https://www.kff.org/medicare/issue-brief/an-overview-of-medicare/}{majority}
of Medicare beneficiaries must interact with private insurers.

Private insurers make Medicare extraordinarily confusing, increasing
costs for beneficiaries and their own profits. When enrolling in
Medicare, and then every subsequent year, beneficiaries are required to
make a series of decisions regarding their coverage. Though there is a
base benefit package, there are also many and varied options, ranging
from which prescription drugs are covered to the amount of premiums,
co-payments and deductibles. The plans also change every year.

Making the right choice means finding a match between your fluctuating
health needs and the changing plans. It is as complicated as it sounds.
Getting the best coverage for the
\href{https://www.healthaffairs.org/doi/full/10.1377/hlthaff.2012.0087}{lowest
cost}
\href{https://www.healthaffairs.org/doi/full/10.1377/hlthaff.2012.0087}{often}
\href{https://www.healthaffairs.org/doi/full/10.1377/hlthaff.2012.0087}{requires
switching plans nearly every year} but very few people do this, leaving
them with higher costs and less effective coverage. A
\href{https://www.healthaffairs.org/doi/full/10.1377/hlthaff.2012.0087}{study}
from the University of Pittsburgh, for instance, found that only 5
percent of Medicare beneficiaries in 2009 chose the cheapest plan that
will cover their prescription drug needs.

Medicare beneficiaries are left
\href{https://www.healthaffairs.org/doi/full/10.1377/hlthaff.2011.0132}{feeling
overwhelmed}. As one
\href{https://www.kff.org/report-section/how-are-seniors-choosing-and-changing-health-insurance-plans-what-factors-lead-beneficiaries-to-not-be-enrolled-in-the-lowest-cost-health-plan/}{noted},
``I had papers taped together --- it was six feet wide --- of the
different companies and circles and arrows.'' Even health care experts
struggle
\href{https://www.healthnewsreview.org/2018/11/making-medicare-choices-in-a-marketplace-mess/}{when
they hit age 65 and need to enroll}.

It's not just Medicare. Nearly all coverage expansions over the last 20
years have relied on private insurers, including Obamacare. As the
Medicaid program has grown, private insurers have played a larger role.

There are ways to make the process easier. The Bernie Sanders
\href{https://www.nytimes.com/2019/03/23/health/private-health-insurance-medicare-for-all-bernie-sanders.html}{approach}
calls for effectively eliminating private insurance, including the
private aspects of Medicare. An incremental alternative is to use a
mixture of regulation and government guidance. Regulatory approaches
could more seamlessly standardize plan options so that it's easier to
compare what you're ``buying'' and reduce the number of options to
ensure there isn't a flock of essentially identical plans. Government
can do more to help people enroll and ensure they receive the benefits
to which they are entitled. Yet there has been little meaningful
discussion of these options among the Democratic presidential
contenders.

If Medicare for all merely puts the frustrations that people experience
in the existing system under a public brand, it will be a magnet for
attack. For any policy to be sustainable, voters need to demand that
their leaders not only make health care more accessible, but also that
they make it less burdensome.

Pamela Herd
(\href{https://twitter.com/pamela_herd?lang=en}{@pamela\_herd}) and
Donald Moynihan
(\href{https://twitter.com/donmoyn?ref_src=twsrc\%5Egoogle\%7Ctwcamp\%5Eserp\%7Ctwgr\%5Eauthor}{@donmoyn})
are professors at the McCourt School of Public Policy at Georgetown
University, visiting fellows at the University of Oxford and the authors
of
``\href{https://www.russellsage.org/publications/administrative-burden}{Administrative
Burden: Policymaking by Other Means}.''

\emph{The Times is committed to publishing}
\href{https://www.nytimes.com/2019/01/31/opinion/letters/letters-to-editor-new-york-times-women.html}{\emph{a
diversity of letters}} \emph{to the editor. We'd like to hear what you
think about this or any of our articles. Here are some}
\href{https://help.nytimes.com/hc/en-us/articles/115014925288-How-to-submit-a-letter-to-the-editor}{\emph{tips}}\emph{.
And here's our email:}
\href{mailto:letters@nytimes.com}{\emph{letters@nytimes.com}}\emph{.}

\emph{Follow The New York Times Opinion section on}
\href{https://www.facebook.com/nytopinion}{\emph{Facebook}}\emph{,}
\href{http://twitter.com/NYTOpinion}{\emph{Twitter (@NYTopinion)}}
\emph{and}
\href{https://www.instagram.com/nytopinion/}{\emph{Instagram}}\emph{.}

Advertisement

\protect\hyperlink{after-bottom}{Continue reading the main story}

\hypertarget{site-index}{%
\subsection{Site Index}\label{site-index}}

\hypertarget{site-information-navigation}{%
\subsection{Site Information
Navigation}\label{site-information-navigation}}

\begin{itemize}
\tightlist
\item
  \href{https://help.nytimes.com/hc/en-us/articles/115014792127-Copyright-notice}{©~2020~The
  New York Times Company}
\end{itemize}

\begin{itemize}
\tightlist
\item
  \href{https://www.nytco.com/}{NYTCo}
\item
  \href{https://help.nytimes.com/hc/en-us/articles/115015385887-Contact-Us}{Contact
  Us}
\item
  \href{https://www.nytco.com/careers/}{Work with us}
\item
  \href{https://nytmediakit.com/}{Advertise}
\item
  \href{http://www.tbrandstudio.com/}{T Brand Studio}
\item
  \href{https://www.nytimes.com/privacy/cookie-policy\#how-do-i-manage-trackers}{Your
  Ad Choices}
\item
  \href{https://www.nytimes.com/privacy}{Privacy}
\item
  \href{https://help.nytimes.com/hc/en-us/articles/115014893428-Terms-of-service}{Terms
  of Service}
\item
  \href{https://help.nytimes.com/hc/en-us/articles/115014893968-Terms-of-sale}{Terms
  of Sale}
\item
  \href{https://spiderbites.nytimes.com}{Site Map}
\item
  \href{https://help.nytimes.com/hc/en-us}{Help}
\item
  \href{https://www.nytimes.com/subscription?campaignId=37WXW}{Subscriptions}
\end{itemize}
