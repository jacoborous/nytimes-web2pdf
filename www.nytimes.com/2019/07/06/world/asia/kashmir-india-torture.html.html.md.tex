Sections

SEARCH

\protect\hyperlink{site-content}{Skip to
content}\protect\hyperlink{site-index}{Skip to site index}

\href{https://www.nytimes.com/section/world/asia}{Asia Pacific}

\href{https://myaccount.nytimes.com/auth/login?response_type=cookie\&client_id=vi}{}

\href{https://www.nytimes.com/section/todayspaper}{Today's Paper}

\href{/section/world/asia}{Asia Pacific}\textbar{}Kashmiris Call for
Investigation of Torture Accusations Against India

\url{https://nyti.ms/2S0PHfD}

\begin{itemize}
\item
\item
\item
\item
\item
\end{itemize}

Advertisement

\protect\hyperlink{after-top}{Continue reading the main story}

Supported by

\protect\hyperlink{after-sponsor}{Continue reading the main story}

\hypertarget{kashmiris-call-for-investigation-of-torture-accusations-against-india}{%
\section{Kashmiris Call for Investigation of Torture Accusations Against
India}\label{kashmiris-call-for-investigation-of-torture-accusations-against-india}}

\includegraphics{https://static01.nyt.com/images/2019/07/07/world/07Kashmir/merlin_155996601_611696b7-a6b2-4ae5-a7e6-d5ae7071585b-articleLarge.jpg?quality=75\&auto=webp\&disable=upscale}

By \href{https://www.nytimes.com/by/sameer-yasir}{Sameer Yasir} and
\href{https://www.nytimes.com/by/kai-schultz}{Kai Schultz}

\begin{itemize}
\item
  July 6, 2019
\item
  \begin{itemize}
  \item
  \item
  \item
  \item
  \item
  \end{itemize}
\end{itemize}

RAWALPORA, Kashmir --- Mohammad Ishaq Lone got a call from the Indian
Army one February night, ordering him to meet soldiers at an outpost
near his house in Kashmir. It was only after he was hauled off to a
brightly lit room, bound and beaten that he discovered why.

A soldier began by punching him in the face, drawing blood, Mr. Lone
said. Another smacked him with a metal rod and began demanding that he
disclose the whereabouts of his brother, who had left home months
earlier to join militants waging a campaign to separate Kashmir from
Indian rule.

Mr. Lone, a pharmacist with two young children, begged them to stop,
saying he did not know where his brother had gone. He recalled screaming
for help before losing consciousness.

``The world around me was collapsing,'' he said.

As tensions with the Indian authorities in Kashmir have sharply
increased, Kashmiris are calling for an international investigation into
accounts of abuse and torture by the security forces.

Detail

area

tajik.

New

Delhi

china

afghan.

india

gilgit

baltistan

Controlled

by Pakistan

100 miles

Undefined

border

Srinagar

jammu and

kashmir

Rawalpora

Controlled

by India

pakistan

india

By The New York Times

According
to\href{http://jkccs.net/wp-content/uploads/2019/05/TORTURE-Indian-State\%E2\%80\%99s-Instrument-of-Control-in-Indian-administered-Jammu-and-Kashmir.pdf}{a
lengthy new report from Kashmiri activists}, thousands of civilians have
been summarily arrested and then abused in
\href{https://www.nytimes.com/interactive/2019/world/asia/india-pakistan-crisis.html}{Kashmir},
the center of a long and bitter territorial dispute between India and
Pakistan.

Released in May by rights groups in Srinagar, the capital of the
Indian-administered part of Kashmir, the report profiles 432 victims of
torture in detention since 1990.

It includes accounts alleging that Indian security forces had hung
Kashmiris by their wrists, shocked them, forced them to stare at
high-voltage lamps and dunked them in water mixed with chili powder.
Most were civilians accused of having information about militants, the
report said, and 49 of them died during or after being tortured.

In interviews with The New York Times, more than two dozen Kashmiris,
including 15 whose cases are included in the report, shared similar
accounts. The Times reviewed hospital documents and spoke with victims'
relatives to help verify their stories.

\includegraphics{https://static01.nyt.com/images/2019/07/07/world/07Kashmir2SUB/merlin_134736672_ac511336-e41e-4bfc-861a-fb3aad3ebda1-articleLarge.jpg?quality=75\&auto=webp\&disable=upscale}

Though some forms of torture are explicitly illegal in India, the report
found that security personnel got away with their actions in every case
because of
\href{https://www.indiatoday.in/india/story/afspa-disagreement-jammu-and-kashmir-armed-militancy-cmp-bjp-pdp-281441-2015-07-09}{laws
that give them broad impunity}.

India has emphatically denied accusations of abuses in Kashmir. In an
interview, Dilbag Singh, the director general of the police in the
region, said the report was ``generalizing things based on data that is
fake or fuzzed.''

In a written response, Lt. Col. Mohit Vaishnava, a spokesman for the
Indian Army, said last month that allegations of abuse were ``false and
fabricated propaganda.''

According to data he sent, the Indian Army was aware of 1,052 alleged
human rights abuses between 1994 and May 31, 2019, in Jammu and Kashmir.
Every case was investigated, the data showed, with 997 of them found by
the army to be ``false'' or ``baseless,'' and punishment meted out to 70
personnel in other cases.

Image

Parvez Imroz, the president of the Jammu and Kashmir Coalition of Civil
Society, a group that co-authored the 550-page report on torture, said,
``Fear is used as a weapon.''Credit...Atul Loke for The New York Times

Last year, the United Nations also
\href{https://www.ohchr.org/Documents/Countries/IN/DevelopmentsInKashmirJune2016ToApril2018.pdf}{raised
grave human rights concerns} in Kashmir, logging cases of torture, among
other issues, while detainees were in the custody of Indian security
forces from June 2016 to April 2018.

India's Ministry of External Affairs
\href{http://www.mea.gov.in/media-briefings.htm?dtl/29978/Official_Spokespersons_response_to_a_question_on_the_Report_by_the_Office_of_the_High_Commissioner_for_Human_Rights_on_The_human_rights_situation_in_K}{wrote
in a statement} that the United Nations' findings played down wrongdoing
by Pakistan in spreading terror and were ``fallacious, tendentious and
motivated.''

Accusations of abuses have intensified as the Indian government has
hardened its crackdown against militants and protesters in recent years.

To disperse crowds of protesters, security forces have injured thousands
of people with pellet-firing shotguns. Civilian deaths rose over 200
percent from 2013 through 2018, when at least 160 people were killed,
including from interrogations,
\href{http://jkccs.net/2018-deadliest-year-of-the-decade-jkccs-annual-human-rights-review/}{activists
say}. And this year is
\href{http://jkccs.net/six-monthly-hr-review-271-killings-177-casos-51-internet-blockades/}{on
track} to become one of the deadliest, overall, in the last decade.

Image

Nazir Ahmad Sheikh, 61, a farmer who was featured in the report on
torture, said soldiers accused him of being a militant and detained him
in 1994. He said they crushed his legs with a heavy roller and then
poured scalding hot water on them.Credit...Atul Loke for The New York
Times

Parvez Imroz, the president of the Jammu and Kashmir Coalition of Civil
Society, a group that co-authored the 550-page report on torture, said
the scope of abuse was even larger than the United Nations had reported.

The accusations have come amid a wave of detentions in Kashmir.
\href{https://amnesty.org.in/news-update/amnesty-international-india-calls-for-the-repeal-of-jk-public-safety-act-in-a-new-briefing/}{A
briefing released on June 12} from Amnesty International found that in
the last few years, Indian armed forces had detained many hundreds of
civilians --- including journalists, activists and children --- without
charge or trial. The arrests were made under the Jammu and Kashmir
Public Safety Act, which activists say violates international human
rights law.

``Fear is used as a weapon,'' Mr. Imroz said.

In an interview with The Times, Nazir Ahmad Sheikh, 61, a farmer who was
featured in the report on torture, said soldiers accused him of being a
militant and detained him in 1994. He said they crushed his legs with a
heavy roller and then poured scalding hot water on them. The men then
forced him to open the lid of a coal heater with his bare hands at an
army camp in northern Kashmir, he said.

``The moment I touched it, some of my tortured and numb fingers fell to
the ground,'' said Mr. Sheikh, who also lost both legs below the kneecap
and was forced to beg for money to survive after the ordeal. ``A torture
chamber is like a dark well where you cry out loud and no one hears your
voice.''

Image

The wife of a militant during his funeral procession in Kulgam, South
Kashmir, last month.Credit...Atul Loke for The New York Times

Hundreds of Kashmiris have joined homegrown insurgency groups since
2016. And tensions reached a breaking point in February, when a suicide
vehicle bombing struck a convoy of Indian paramilitary forces, killing
at least 40 of them. It set off
\href{https://www.nytimes.com/2019/02/27/world/asia/kashmir-india-pakistan-aircraft.html?rref=collection\%2Fbyline\%2Fmaria-abi-habib}{a
tense military standoff} between India and Pakistan, where a banned
terrorist group, Jaish-e-Muhammad, claimed responsibility.

Over the last year, activists say, the hunt for separatists has
intensified, pulling ordinary Kashmiris into the fold.

Feroz Ahmad Hajam said he was on the way to meet a friend in September
when the police abducted him, locked him in an interrogation cell in
southern Kashmir and burned his feet and shoulders with cigarettes.

When Mr. Hajam, 25, a laborer, said he had no affiliation with
militants, an officer walked up behind him and cut his throat.

Image

Posters of militants in the village of Kulgam, South
Kashmir.Credit...Atul Loke for The New York Times

The police have denied his claim and accused him of attempting to kill
himself.

``I feel the knife in my dreams, slitting my throat again and again,''
he said in an interview, writing his answers on a piece of paper because
his vocal cords were so damaged that he can no longer speak.

``I want to talk just once,'' he wrote, tears rolling down his cheeks.

Prime Minister Narendra Modi of India, whose Bharatiya Janata Party won
a resounding victory during national elections in May, has vowed to ease
tensions in Kashmir.

But many worry that the government will further alienate locals by
removing special protections that grant the population, which is
majority Muslim, a certain degree of autonomy. Hostility toward the
Indian security
forces\href{https://www.nytimes.com/2018/12/18/world/asia/kashmir-civilians-teenagers.html}{has
only increased}. Almost every day, life is disrupted by gun battles,
bombings or street protests.

Mr. Lone, 39, who said he was tortured for more than two hours while
being questioned by Indian soldiers in February, said peace was
unlikely.

When Mr. Lone regained consciousness at the army camp in the village of
Rawalpora, he said three soldiers standing on him stomped on his thighs,
struck him with bamboo sticks and screamed at him to get up and walk. He
could not manage even one step.

``It was as if someone was taking me to a butcher's shop to have me
chopped into small pieces,'' he said.

Bloodied and exhausted, Mr. Lone passed out again. He woke up in a
hospital bed surrounded by relatives, who had dragged him from the camp
when the soldiers finished with him.

Several months later, Mr. Lone can barely stand for five minutes. He has
trouble kneeling to pray. His brother, the militant, is still missing.
He worries about his children, ages 8 and 12.

He said the cycle of violence showed no signs of letting up.

``Every morning is filled with fear,'' he said. ``How do you expect
justice from one wing of the state when the other is inflicting pain on
you? It is hell.''

Advertisement

\protect\hyperlink{after-bottom}{Continue reading the main story}

\hypertarget{site-index}{%
\subsection{Site Index}\label{site-index}}

\hypertarget{site-information-navigation}{%
\subsection{Site Information
Navigation}\label{site-information-navigation}}

\begin{itemize}
\tightlist
\item
  \href{https://help.nytimes.com/hc/en-us/articles/115014792127-Copyright-notice}{©~2020~The
  New York Times Company}
\end{itemize}

\begin{itemize}
\tightlist
\item
  \href{https://www.nytco.com/}{NYTCo}
\item
  \href{https://help.nytimes.com/hc/en-us/articles/115015385887-Contact-Us}{Contact
  Us}
\item
  \href{https://www.nytco.com/careers/}{Work with us}
\item
  \href{https://nytmediakit.com/}{Advertise}
\item
  \href{http://www.tbrandstudio.com/}{T Brand Studio}
\item
  \href{https://www.nytimes.com/privacy/cookie-policy\#how-do-i-manage-trackers}{Your
  Ad Choices}
\item
  \href{https://www.nytimes.com/privacy}{Privacy}
\item
  \href{https://help.nytimes.com/hc/en-us/articles/115014893428-Terms-of-service}{Terms
  of Service}
\item
  \href{https://help.nytimes.com/hc/en-us/articles/115014893968-Terms-of-sale}{Terms
  of Sale}
\item
  \href{https://spiderbites.nytimes.com}{Site Map}
\item
  \href{https://help.nytimes.com/hc/en-us}{Help}
\item
  \href{https://www.nytimes.com/subscription?campaignId=37WXW}{Subscriptions}
\end{itemize}
