Sections

SEARCH

\protect\hyperlink{site-content}{Skip to
content}\protect\hyperlink{site-index}{Skip to site index}

\href{https://www.nytimes.com/section/opinion/sunday}{Sunday Review}

\href{https://myaccount.nytimes.com/auth/login?response_type=cookie\&client_id=vi}{}

\href{https://www.nytimes.com/section/todayspaper}{Today's Paper}

\href{/section/opinion/sunday}{Sunday Review}\textbar{}You Call It
Starvation. I Call It Biohacking.

\href{https://nyti.ms/32joSIo}{https://nyti.ms/32joSIo}

\begin{itemize}
\item
\item
\item
\item
\item
\item
\end{itemize}

Advertisement

\protect\hyperlink{after-top}{Continue reading the main story}

\href{/section/opinion}{Opinion}

Supported by

\protect\hyperlink{after-sponsor}{Continue reading the main story}

\hypertarget{you-call-it-starvation-i-call-it-biohacking}{%
\section{You Call It Starvation. I Call It
Biohacking.}\label{you-call-it-starvation-i-call-it-biohacking}}

Welcome to the bro-y world of extreme dieting. Or is it disordered
eating?

By Thomas Stackpole

Mr. Stackpole is a writer.

\begin{itemize}
\item
  July 11, 2019
\item
  \begin{itemize}
  \item
  \item
  \item
  \item
  \item
  \item
  \end{itemize}
\end{itemize}

\includegraphics{https://static01.nyt.com/images/2019/07/14/opinion/sunday/14stackpole/14stackpole-articleLarge.jpg?quality=75\&auto=webp\&disable=upscale}

The run happened --- or didn't --- maybe five days into the raw-diet
experiment.

I had formed a sort of fitness pact with a friend to forgo cooked food,
and after days of nothing but salads, almonds, sashimi and black coffee,
my body felt taut and ready for action.

And for about half a mile, it was, my strides floating above the
pavement as a few fistfuls of raw kale percolated in my belly. Then
suddenly I sputtered, feeling an unambiguous alarm go off: Tank is
empty, sorry, this is the end of the line. After a pause, I tried
running again but made it maybe a block before my legs revolted again
and I slowed to a walk. My new healthy diet, it seemed, didn't
accommodate any actual exercise.

When I told all this to my co-workers the next morning, it was fodder
for a good laugh. My obsessions were --- and often still are --- a kind
of running joke. I've been conducting a series of shifting and poorly
planned ``wellness'' experiments on myself for about a decade.

I've eaten keto, low-carb and sometimes not at all. One time, I ate
almost nothing but lean ground turkey and broccoli over greens for maybe
two months as part of a YouTube bodybuilder's plan. More than once, I've
lost 10 pounds in a week. I've also obsessed over bulking up, gaining 25
pounds over about six months of lifting, before pivoting and deciding to
train for a marathon to run it off. Then there were the gut biome
vitamins, the metabolism-boosting mushrooms, the experiments with LSD
microdosing and calorie trackers.

Despite years of cycling through boutique insanities, it didn't occur to
me that I might have a problem until earlier this year, when the Twitter
founder
\href{https://www.nytimes.com/2019/05/02/fashion/jack-dorsey-influencer.html}{turned
Silicon Valley wellness influencer} Jack Dorsey detailed his fasting
regimen. The news that he eats one meal a day during the week and
nothing on the weekend provoked scornful cries that he was advocating
little more than anorexia with a bro-y tech-world veneer. I, on the
other hand, saw a kindred spirit.

My relationship with the extreme margins of the wellness world didn't
start until my mid-20s. And as it does for many people, it started out
about weight.

During my adolescence, I'd had a critical but mostly accepting
relationship with my body. I'd been a high school runner who could clock
a respectable 5:30 mile but just have always had the kind of body that
hangs onto a probably fine amount of fat.

In my early 20s, I had worked service jobs or physical labor, spending
the day on my feet and often exercising before or after. But once I
found myself sitting behind an aging computer in a magazine office in
Washington, I started to gain weight, slowly, but inescapably. The
delicate balance of appreciation and loathing I felt for my body tipped
--- I felt it was betraying me and spiraling out of control.

And so I searched for ways to wrestle it back into line. I ran more and
did hot yoga. I heaved a filing cabinet onto a table and fashioned
myself a sort of Brutalist standing desk.

But the problem, I eventually realized, was my relationship to food ---
always stressed, I chased down my salads with any carbohydrate not
nailed down. Eating raw or straight-up fasting were ways to regain a
modicum of control over my appetites, at least at first --- and to do so
in ways that felt like fun, slightly absurd challenges: There's a
machismo to this sort of explicit bodily abuse that simple healthy
living doesn't offer.

But if this started out about weight, at some point, for me, these
obsessions stopped being about my body; the strain of a new fitness
regimen, a new mania, be it lifting or raw food, became its own draw.

\href{https://www.nytimes.com/newsletters/sunday-best?action=click\&module=Intentional\&pgtype=Article}{\emph{{[}Read
the most thought-provoking, funny, delightful and raw stories from The
New York Times Opinion section. Sign up for our Sunday Best
newsletter.{]}}}

It's clear I'm not the only one --- and not the only guy --- who sees
something appealing here. If fasting started as a
\href{https://amp.theguardian.com/lifeandstyle/2019/feb/21/extreme-fasting-how-silicon-valley-is-rebranding-eating-disorders?CMP=soc_568\&__twitter_impression=true}{life
hack for the billionaire class}, which in turn saw would-be billionaires
follow suit --- as if food was the thing that was holding their
start-ups back --- today, run-of-the-mill bros everywhere are studying
how to eat only during six-hour windows in the pages of
\href{https://www.menshealth.com/nutrition/a27632073/intermittent-fasting-diet-weight-loss/}{Men's
Health} and
\href{https://www.mensjournal.com/health-fitness/3-types-intermittent-fasting-compared/}{Men's
Journal}.

We live in a time of wellness not as health but as transcendence. It's
not a coincidence that all of the supposed cures of wellness-adjacent
diet hacking hinge on extreme behavior --- fasting, or that daily coffee
you put special butter in. The appeal of this brand of wellness has very
little to do with being healthy. After all, most of what maintaining
good health requires feels pretty good: eat well, exercise, get enough
sleep, practice everything in moderation (even moderation), etc. With
``biohacking,'' the effects are ephemeral and the health claims are
dubious. But what these crude approaches \emph{do} offer is a sense of
control in the moment --- a way to tell yourself that you're willing
some change into being.

It would perhaps be going too far to call this kind of behavior ``eating
disorders''; those are conditions that send people to the hospital and
sometimes kill them, not a series of passing, momentary manias. But nor
do I have a healthy relationship with food or exercise, a fact about my
life that up until recently has been more or less obscured by my gender.
After all, if I asked you to picture someone grappling with disordered
eating, would you imagine a skinny teenage girl or me --- a 33-year-old
man who weighs 200 pounds and is flirting with exercise bulimia? I bet
you a cookie you picked the former.

So if there's an upside to the male-driven starvation-as-biohacking era,
it might be that it reveals what disordered eating and exercising,
stripped of their typical gender norms, are actually about.

We typically tend to think of these behaviors as feminine ones. As a
result, there's often an impression that they're primarily about
appearance and, sometimes, vanity. They can be, but this, of course, was
never the whole story.

Today's eating disorder is as likely to come in the guise of a diet that
purports to optimize you to survive and thrive in late capitalism as it
is one that claims to make you beach-body ready. What these iterations
reveal is how much \emph{more} disordered obsessive behavior around food
and exercise can be about, how many kinds of feelings this sort of
behavior can become a vessel for. In an era when so many of us feel the
world spiraling out of control, maybe it's just the promise of being
able to control something --- to will a change, any change, into being
--- that's the draw.

A few days ago, as I was thinking about writing this, I sat down in
front of my computer and filled out a questionnaire from the National
Eating Disorders Association to see whether I was at risk. I clicked
through the questions --- yes, I had gone to extremes to exercise after
eating; no, I don't tend to hide when I eat out of shame. At the end of
it, the website told me I was at risk and should probably talk to
someone.

When I mentioned those results to two close female acquaintances, both
of them laughed before catching themselves, horrified. Both, for the
record, are thoughtful, sensitive women who rebuff gender stereotypes.
They were both familiar with my history of fixation with wellness fads.
Maybe it was just the moment of that absurd history suddenly being
recast with a new, worrisome weight. I laughed, too, for what it's
worth. It had all been a joke for so long. What was it now?

Thomas Stackpole is a senior editor at Boston Magazine.

\emph{The Times is committed to publishing}
\href{https://www.nytimes.com/2019/01/31/opinion/letters/letters-to-editor-new-york-times-women.html}{\emph{a
diversity of letters}} \emph{to the editor. We'd like to hear what you
think about this or any of our articles. Here are some}
\href{https://help.nytimes.com/hc/en-us/articles/115014925288-How-to-submit-a-letter-to-the-editor}{\emph{tips}}\emph{.
And here's our email:}
\href{mailto:letters@nytimes.com}{\emph{letters@nytimes.com}}\emph{.}

\emph{Follow The New York Times Opinion section on}
\href{https://www.facebook.com/nytopinion}{\emph{Facebook}}\emph{,}
\href{http://twitter.com/NYTOpinion}{\emph{Twitter (@NYTopinion)}}
\emph{and}
\href{https://www.instagram.com/nytopinion/}{\emph{Instagram}}\emph{.}

Advertisement

\protect\hyperlink{after-bottom}{Continue reading the main story}

\hypertarget{site-index}{%
\subsection{Site Index}\label{site-index}}

\hypertarget{site-information-navigation}{%
\subsection{Site Information
Navigation}\label{site-information-navigation}}

\begin{itemize}
\tightlist
\item
  \href{https://help.nytimes.com/hc/en-us/articles/115014792127-Copyright-notice}{©~2020~The
  New York Times Company}
\end{itemize}

\begin{itemize}
\tightlist
\item
  \href{https://www.nytco.com/}{NYTCo}
\item
  \href{https://help.nytimes.com/hc/en-us/articles/115015385887-Contact-Us}{Contact
  Us}
\item
  \href{https://www.nytco.com/careers/}{Work with us}
\item
  \href{https://nytmediakit.com/}{Advertise}
\item
  \href{http://www.tbrandstudio.com/}{T Brand Studio}
\item
  \href{https://www.nytimes.com/privacy/cookie-policy\#how-do-i-manage-trackers}{Your
  Ad Choices}
\item
  \href{https://www.nytimes.com/privacy}{Privacy}
\item
  \href{https://help.nytimes.com/hc/en-us/articles/115014893428-Terms-of-service}{Terms
  of Service}
\item
  \href{https://help.nytimes.com/hc/en-us/articles/115014893968-Terms-of-sale}{Terms
  of Sale}
\item
  \href{https://spiderbites.nytimes.com}{Site Map}
\item
  \href{https://help.nytimes.com/hc/en-us}{Help}
\item
  \href{https://www.nytimes.com/subscription?campaignId=37WXW}{Subscriptions}
\end{itemize}
