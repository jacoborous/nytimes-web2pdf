Sections

SEARCH

\protect\hyperlink{site-content}{Skip to
content}\protect\hyperlink{site-index}{Skip to site index}

\href{https://www.nytimes.com/section/politics}{Politics}

\href{https://myaccount.nytimes.com/auth/login?response_type=cookie\&client_id=vi}{}

\href{https://www.nytimes.com/section/todayspaper}{Today's Paper}

\href{/section/politics}{Politics}\textbar{}China Purchases Could
Undercut Trump's Larger Trade Goal

\begin{itemize}
\item
\item
\item
\item
\item
\item
\end{itemize}

Advertisement

\protect\hyperlink{after-top}{Continue reading the main story}

Supported by

\protect\hyperlink{after-sponsor}{Continue reading the main story}

\hypertarget{china-purchases-could-undercut-trumps-larger-trade-goal}{%
\section{China Purchases Could Undercut Trump's Larger Trade
Goal}\label{china-purchases-could-undercut-trumps-larger-trade-goal}}

\includegraphics{https://static01.nyt.com/images/2019/03/29/us/politics/00DC-CHINABUY/merlin_152766375_3d229d5f-8fca-4614-92a0-3544edfae09a-articleLarge.jpg?quality=75\&auto=webp\&disable=upscale}

By \href{https://www.nytimes.com/by/ana-swanson}{Ana Swanson} and
\href{https://www.nytimes.com/by/keith-bradsher}{Keith Bradsher}

\begin{itemize}
\item
  April 1, 2019
\item
  \begin{itemize}
  \item
  \item
  \item
  \item
  \item
  \item
  \end{itemize}
\end{itemize}

\href{https://cn.nytimes.com/usa/20190401/us-china-trade-trump/}{阅读简体中文版}\href{https://cn.nytimes.com/usa/20190401/us-china-trade-trump/zh-hant/}{閱讀繁體中文版}

WASHINGTON --- At the heart of President Trump's negotiations with China
is a troubling contradiction: The United States wants to use the trade
talks to encourage the country to adopt a more market-oriented economy.
But a key element of a prospective deal may end up reinforcing the
economic power of the Chinese state.

Negotiators are still working out deal terms, but any agreement seems
certain to involve China's promise to purchase hundreds of billions of
dollars of American goods. For Mr. Trump, this is an essential element
that will help reduce the United States'
\href{https://www.nytimes.com/2019/03/06/us/politics/us-trade-deficit.html}{record
trade deficit} with China and bolster farmers and other constituencies
hurt by his trade war.

But those purchases will be ordered by the Chinese state, and most will
be carried out by state-controlled Chinese businesses, further cementing
Beijing's role in managing its economy and potentially making United
States industries even more beholden to the Chinese.

``It seems like those types of really simplistic purchasing commitment
type of arrangements would actually reinforce state ownership rather
than discourage it,'' said Rufus Yerxa, the head of the National Foreign
Trade Council, which represents the United States' largest exporters.

After months of talks, the two sides are inching closer to an agreement.
Robert Lighthizer, Mr. Trump's top trade negotiator, and Steven Mnuchin,
the Treasury secretary, discussed the remaining sticking points with
their Chinese counterparts on Thursday evening and Friday in Beijing.
Mr. Mnuchin
\href{https://twitter.com/stevenmnuchin1/status/1111534387328417792}{wrote
on Twitter} on Friday that the talks had been ``constructive.''

Both sides are trying to iron out an agreement by this week, to coincide
with a visit to Washington by Liu He, the Chinese special envoy charged
with negotiating the deal, who will begin meeting with his American
counterparts on Wednesday.

The United States and China had been looking to reach a tentative
agreement by the end of Mr. Liu's visit, with a signing ceremony between
Mr. Trump and President Xi Jinping of China potentially later this
month.

But the two sides are still wrestling with two major sticking points:
how an agreement will be monitored and enforced and how many of Mr.
Trump's tariffs come off and when, said Myron Brilliant, executive vice
president and head of international affairs at the U.S. Chamber of
Commerce.

Mr. Brilliant said there was no question the United States and China
were ``in the endgame with regard to a deal.'' However, he said these
factors were ``complicating the fact that the agreement is 90 percent
done at this point.''

To pave the way toward an agreement, China made several announcements
that could benefit American companies. Last week, Chinese regulators
approved JPMorgan Chase's request to establish a majority owned and
controlled securities brokerage firm in the country, a change China had
discussed since entering the World Trade Organization two decades ago.
Chinese officials have also floated the idea of an expanded trial that
would allow foreign cloud computing companies to operate more freely.

On Sunday evening, China's Finance Ministry issued two statements saying
that Beijing would continue to suspend tariffs it imposed last year on
American cars and car parts in retaliation for Mr. Trump's tariffs on
\$250 billion worth of Chinese imports. Those tariffs, which were
suspended while the two sides tried to reach an agreement, were supposed
to resume at the end of March, but China said it would extend the
suspension indefinitely as a gesture of good will.

\includegraphics{https://static01.nyt.com/images/2019/03/29/us/politics/00DC-CHINABUY-02/merlin_146443122_28b7f71b-91fd-4ec1-a049-ff37b73cb24a-articleLarge.jpg?quality=75\&auto=webp\&disable=upscale}

The Finance Ministry said, ``We hope that the U.S. and China will work
together to step up consultations and make practical efforts toward the
goal of ending trade friction.''

While the two sides are closer to an agreement than at any point in the
past, it remains unclear how successful the Trump administration will be
in achieving its key goals. The president's trade war was initiated in
large part to try to reorient the Chinese economy and force it to become
more open to American companies and investment. Using punishing tariffs
as leverage, the Trump administration has pressed China to roll back its
heavy hand in the economy, including asking Beijing to curtail subsidies
to state-owned firms and to end its practice of forcing foreign
companies doing business in China to transfer their technology to
Chinese competitors.

China has not readily committed to these goals, in part because such
commitments are seen as infringing on China's sovereignty and
undercutting the power of the Chinese state. What the Chinese have
agreed to most readily is purchasing American goods, especially
commodities that can fuel their economy.

While the final list could be different, the United States and China
have discussed the purchase of products including corn, soybeans,
sorghum, natural gas, oil, coal, chemicals, semiconductors and
airplanes, according to people with knowledge of the talks.

The Trump administration views these purchases as necessary to bolster
the president's support across farming and manufacturing communities
hard hit by the trade war and to help narrow the gap between what China
sells to the United States and what it buys. The administration has been
working on various draft lists of what it wants China to purchase,
according to people who have viewed them.

The final purchasing amount is not yet clear. In December, Mr. Mnuchin
said that China had made an offer to buy more than \$1.2 trillion in
American goods as part of the talks. But economists and China analysts
have cautioned that such a large amount could be hard for the United
States to produce and export.

The United States exported just \$120 billion of goods to China last
year. With the American economy hovering near full employment, it lacks
the productive capacity to raise exports by hundreds of billions of
dollars in the short term. The United States could redirect some of the
goods it sells to other countries to China instead, for instance
diverting soybeans headed to Europe to China.

But a deal that would require China to buy even more from the United
States is raising concerns that China could expand its leverage over the
United States.

``If it can be negotiated by government fiat, it can be taken away by
government fiat,'' said Kevin Book, managing director at ClearView
Energy Partners, an energy-focused research firm.

The deal could usher in a wave of new American exports if China agrees
to open its markets more fully. Removing its requirements that American
carmakers and financial services firms team up with a Chinese entity to
do business in the country, for instance, could give those firms more
ability to sell goods and services to China.

But in other industries, including agriculture, energy and aviation,
purchases associated with a trade deal would be made directly by
state-controlled entities. And while that would mean greater revenues
for American companies, skeptics say it could also increase the leverage
that China has over the United States in the future.

``We are handing them the ability to coerce our companies,'' said Derek
Scissors, a resident scholar at the American Enterprise Institute.

Image

While China would like to buy more tech goods, one of the Trump
administration's biggest fears about China is that it is leading a
state-sponsored effort to unseat the United States as the leading
technological power.Credit...Kin Cheung/Associated Press

Such a purchasing arrangement could also provoke challenges at the World
Trade Organization, which bars its members from granting special
advantages or privileges to any one country, as well as directing state
enterprises to purchase goods from a specific country.

China has already demonstrated its ability to influence commodity
purchases depending on the tenor of relations, analysts said. Once trade
tensions flared, China drove its purchases of American oil to nearly
zero last year, Mr. Book said. After a breakthrough in talks this year,
China
\href{https://www.nytimes.com/2019/02/02/business/china-us-soybeans.html}{ramped
up its purchases} of United States soybeans.

``How easy would it be for the government of China to then turn off that
tap when it gets into a dispute with the U.S.?'' Mr. Yerxa asked.

Recent events have also raised a question over some components of the
purchasing package under discussion. China had planned to buy Boeing 737
Max series planes. But the crash of a 737 Max in Ethiopia in early
March, the second deadly crash of the new plane in less than five
months, has thrown those plans into question.

Last week, China suspended the airworthiness certification for the model
entirely, a move Richard Aboulafia, vice president for analysis at the
Teal Group, described as ``pretty aggressive.''

``I think there's a very good chance that this is less about public
safety and more about trade negotiation leverage,'' Mr. Aboulafia said.

It's also unclear how many purchases of semiconductors, the advanced
components that power circuits in laptops, smartphones and other
electronics, will ultimately be in the deal.

While China would like to buy more of the advanced technology, the Trump
administration is concerned about allowing Beijing to gain an upper hand
in the next generation of technology. The administration has pushed
China to roll back its Made in China 2025 campaign, an industrial plan
that pumps state money into building cutting-edge industries, like
advanced manufacturing and aviation.

China had proposed last year to purchase \$200 billion of American
semiconductors, according to people with knowledge of the deal. Beijing
wanted to buy the semiconductors straight from fabrication lines at
American factories and then do the necessary packaging and testing in
China. The United States rejected that idea. The two sides have since
discussed a smaller number, but it is not clear now how many
semiconductor purchases will ultimately be included in a deal.

The semiconductor industry has publicly warned that additional purchases
could result in the Chinese government's gaining more power to control
where Chinese companies buy products. While Chinese companies may buy
more American components in the short term, the industry warns that
Beijing could ultimately use such a system to redirect purchases to
Chinese suppliers instead.

For now, administration officials are still trying to assure lawmakers
that their trade deal will achieve the economic changes the United
States wants and will require much more than purchases.

When pressed by lawmakers in February about whether the deal would
require more than just additional sales to China, Mr. Lighthizer told
Congress: ``I do not think it should just be a purchase agreement.''

Advertisement

\protect\hyperlink{after-bottom}{Continue reading the main story}

\hypertarget{site-index}{%
\subsection{Site Index}\label{site-index}}

\hypertarget{site-information-navigation}{%
\subsection{Site Information
Navigation}\label{site-information-navigation}}

\begin{itemize}
\tightlist
\item
  \href{https://help.nytimes.com/hc/en-us/articles/115014792127-Copyright-notice}{©~2020~The
  New York Times Company}
\end{itemize}

\begin{itemize}
\tightlist
\item
  \href{https://www.nytco.com/}{NYTCo}
\item
  \href{https://help.nytimes.com/hc/en-us/articles/115015385887-Contact-Us}{Contact
  Us}
\item
  \href{https://www.nytco.com/careers/}{Work with us}
\item
  \href{https://nytmediakit.com/}{Advertise}
\item
  \href{http://www.tbrandstudio.com/}{T Brand Studio}
\item
  \href{https://www.nytimes.com/privacy/cookie-policy\#how-do-i-manage-trackers}{Your
  Ad Choices}
\item
  \href{https://www.nytimes.com/privacy}{Privacy}
\item
  \href{https://help.nytimes.com/hc/en-us/articles/115014893428-Terms-of-service}{Terms
  of Service}
\item
  \href{https://help.nytimes.com/hc/en-us/articles/115014893968-Terms-of-sale}{Terms
  of Sale}
\item
  \href{https://spiderbites.nytimes.com}{Site Map}
\item
  \href{https://help.nytimes.com/hc/en-us}{Help}
\item
  \href{https://www.nytimes.com/subscription?campaignId=37WXW}{Subscriptions}
\end{itemize}
