Sections

SEARCH

\protect\hyperlink{site-content}{Skip to
content}\protect\hyperlink{site-index}{Skip to site index}

\href{https://www.nytimes.com/section/politics}{Politics}

\href{https://myaccount.nytimes.com/auth/login?response_type=cookie\&client_id=vi}{}

\href{https://www.nytimes.com/section/todayspaper}{Today's Paper}

\href{/section/politics}{Politics}\textbar{}Kurt Volker, Trump's Envoy
for Ukraine, Resigns

\url{https://nyti.ms/2mex0tH}

\begin{itemize}
\item
\item
\item
\item
\item
\item
\end{itemize}

Advertisement

\protect\hyperlink{after-top}{Continue reading the main story}

Supported by

\protect\hyperlink{after-sponsor}{Continue reading the main story}

\hypertarget{kurt-volker-trumps-envoy-for-ukraine-resigns}{%
\section{Kurt Volker, Trump's Envoy for Ukraine,
Resigns}\label{kurt-volker-trumps-envoy-for-ukraine-resigns}}

\includegraphics{https://static01.nyt.com/images/2019/09/28/world/28-dc-Volker-PRINT/merlin_158531811_aaab7ac0-f029-4144-bda7-a7c374a1378c-articleLarge.jpg?quality=75\&auto=webp\&disable=upscale}

By \href{https://www.nytimes.com/by/peter-baker}{Peter Baker}

\begin{itemize}
\item
  Published Sept. 27, 2019Updated Oct. 3, 2019
\item
  \begin{itemize}
  \item
  \item
  \item
  \item
  \item
  \item
  \end{itemize}
\end{itemize}

WASHINGTON --- Kurt D. Volker, the State Department's special envoy for
\href{https://www.nytimes.com/2019/09/28/us/politics/democrats-impeachment.html}{Ukraine}
who got caught in the middle of the
\href{https://www.nytimes.com/2019/09/25/us/politics/donald-trump-impeachment-probe.html}{pressure
campaign by President Trump and his lawyer, Rudolph W. Giuliani}, to
find damaging information about Democrats, abruptly resigned his post on
Friday.

Mr. Volker, who told Secretary of State Mike Pompeo on Friday that he
was stepping down, offered no public explanation, but a person informed
about his decision said he concluded that it was impossible to be
effective in his assignment given the developments of recent days.

\emph{{[}Update:}
\href{https://www.nytimes.com/2019/10/03/us/politics/kurt-volker-impeachment.html}{\emph{House
impeachment investigators question Kurt Volker}}\emph{.{]}}

His departure was the first resignation since revelations about Mr.
Trump's efforts to
\href{https://www.nytimes.com/interactive/2019/09/25/us/politics/trump-ukraine-transcript.html?module=inline}{pressure
Ukraine's president} to investigate former Vice President Joseph R.
Biden Jr. and other Democrats. The disclosures have triggered a
full-blown House impeachment inquiry, and
\href{https://www.nytimes.com/2019/09/27/us/politics/house-democrats-impeachment-trump.html}{House
leaders announced on Friday that they planned to interview Mr. Volker}
in a deposition on Thursday.

Mr. Volker, a widely respected former ambassador to NATO, served in the
part-time, unpaid position of special envoy to help Ukraine resolve its
armed confrontation with Russia-sponsored separatists. He was among the
government officials who found themselves in an awkward position because
of the search for dirt on Democrats, reluctant to cross the president or
Mr. Giuliani yet wary of getting drawn into politics outside their
purview.

The unidentified intelligence official who filed
\href{https://www.nytimes.com/2019/09/26/us/politics/whistleblower-complaint-released.html?action=click\&module=inline\&pgtype=Homepage}{the
whistle-blower complaint} that brought the president's actions to light
identified Mr. Volker as one of the officials trying to ``contain the
damage'' by advising Ukrainians how to navigate Mr. Giuliani's campaign.

Mr. Volker facilitated an entree for Mr. Giuliani with the newly elected
government in Ukraine, acting not at the instruction of Mr. Trump or Mr.
Pompeo, but at the request of the Ukrainians, who were worried because
Mr. Giuliani was seeking information about Mr. Biden and other Democrats
and had denounced top Ukrainian officials as ``enemies of the
president.''

The Ukrainians were concerned about the impact of their relationship
with the United States, their most important patron against Russia.
Andriy Yermak, a close aide to President
\href{https://www.nytimes.com/2019/09/25/world/europe/ukraine-trump-whistleblower-zelensky.html?module=inline}{Volodymyr
Zelensky}, asked Mr. Volker in July to connect him with Mr. Giuliani, a
former mayor of New York, so the Ukrainian government could hear out his
issues. Mr. Volker agreed to reach out to the former mayor to see if he
would sit down with the Ukrainian official.

Mr. Volker then contacted Mr. Giuliani to ask if he would want to speak
with Mr. Yermak, and the mayor agreed. Mr. Volker and Mr. Giuliani had
breakfast to discuss Ukraine.

``Mr. Mayor --- really enjoyed breakfast this morning,'' Mr. Volker
wrote in a text later that day
\href{https://twitter.com/RudyGiuliani/status/1177346278004539392}{that
Mr. Giuliani posted this week on Twitter}. Mr. Volker set up a
conference call between Mr. Giuliani and Mr. Yermak, who then later met
in person in Madrid, on Aug. 2.

Mr. Giuliani has seized on Mr. Volker's call to him to assert that he
was acting at the behest of the State Department. He said he spoke with
Mr. Volker eight times and displayed the texts on Thursday night on
Laura Ingraham's show on Fox News, calling on Mr. Volker to confirm that
the department initiated contact.

``He should step forward and explain what he did,'' Mr. Giuliani said on
the show. ``The whistle-blower falsely alleges that I was operating on
my own. Well, I wasn't operating on my own!''

Referring to Mr. Volker and Gordon Sondland, the ambassador to the
European Union, Mr. Giuliani added: ``They basically knew everything I
was doing. So, it was being done with the authorization and at the
request --- and then I have a final one in which they --- there is a big
`thank you' about how my honest and straightforward discussion led to
solving a problem in the relationship.''

Reacting to Mr. Volker's resignation on Friday night, Mr. Giuliani said
his release of the texts was not intended to get him ``in trouble,'' and
he defended the special envoy's role in facilitating the talks with Mr.
Yermak.

``I was in the unique position to help them,'' Mr. Giuliani said of the
State Department, adding that he kept both Mr. Volker and Mr. Sondland
apprised of the talks.

The State Department did not respond to a request for comment on Mr.
Volker's resignation on Friday, nor did it leap to his defense after The
New York Times
\href{https://www.nytimes.com/2019/08/21/us/politics/giuliani-ukraine.html}{first
reported on his role} in facilitating Mr. Giuliani's talks with the new
Ukrainian government.

In response to that article, the department said in a statement last
month that Mr. Volker ``has confirmed that, at Presidential Advisor
Andriy Yermak's request, Volker put Yermak in direct contact with Mr.
Giuliani.'' The statement went on to stress that Mr. Giuliani ``is a
private citizen and acts in a personal capacity as a lawyer for
President Trump --- he does not speak on behalf of the U.S.
government.''

Just days after Mr. Giuliani's breakfast with Mr. Volker and the
follow-up phone call with Mr. Yermak, Mr. Trump spoke on the telephone
with Mr. Zelensky. After the Ukrainian president described his need for
more American assistance against Russia, Mr. Trump asked him to
\href{https://www.nytimes.com/2019/09/25/us/politics/donald-trump-impeachment-probe.html?module=inline}{``do
us a favor, though''} and look into Democrats.

Mr. Volker was not on
\href{https://www.nytimes.com/2019/09/26/us/politics/trump-ukraine-timeline.html?module=inline}{that
call}, and he was neither shown a copy of the transcript reconstructed
from the conversation nor told that the president mentioned Mr. Biden,
according to one person informed about the series of events. Mr. Volker
participated in Mr. Trump's meeting with Mr. Zelensky on the sidelines
of the United Nations General Assembly session this week in his last
official duty.

``Kurt was one of the good ones who went in to the administration to
stave off disaster,'' said Thomas Wright, a senior fellow at the
Brookings Institution. ``They all have to speak out now about everything
they know and let the chips fall where they may.''

Mr. Volker, a former career foreign service officer who represented
President George W. Bush at NATO and now serves as the executive
director of the McCain Institute for International Leadership at Arizona
State University based in Washington, spent much of the year trying to
bring Mr. Trump together with Mr. Zelensky to bolster the government
\href{https://www.nytimes.com/2019/04/21/world/europe/Volodymyr-Zelensky-ukraine-elections.html}{elected
in April}.

He argued to Trump administration officials that Mr. Zelensky was a
credible reformer and serious figure who could be his country's last
chance to get its act together in the face of Russian aggression. It was
an uphill task, given Mr. Trump's open disdain for Ukrainians; ``they're
all corrupt and they tried to take me down,'' he said in a private
meeting last spring.

After the Ukrainian inauguration, Mr. Trump agreed to meet with Mr.
Zelensky, but his staff kept delaying putting a date on the calendar.
Like other officials, Mr. Volker was surprised to learn that Mr. Trump
had ordered \$391 million in aid to Ukraine frozen.

But he kept working to bring the two presidents together. Finally, the
White House agreed to schedule a meeting between Mr. Trump and Mr.
Zelensky during the American president's visit to Warsaw, only to
scratch the meeting when Mr. Trump decided to stay home to monitor a
hurricane.

Instead, Vice President Mike Pence, whose trip to Mr. Zelensky's
inauguration had been canceled to increase leverage on the Ukrainian
government,
\href{https://www.nytimes.com/interactive/2019/09/26/us/politics/whistle-blower-complaint.html?action=click\&module=RelatedLinks\&pgtype=Article}{according
to the whistle-blower complaint}, was sent to meet with Mr. Zelensky in
Warsaw in his place.

Mr. Volker's departure, which was
\href{https://www.statepress.com/article/2019/09/sppolitics-mccain-head-steps-down}{first
reported by The State Press}, the student newspaper at Arizona State
University, leaves the Trump administration with few senior officials
versed in Ukraine's struggles with Russia.

{[}\href{https://www.nytimes.com/2019/09/28/us/asu-ukraine-volker.html}{\emph{When
Mr. Volker resigned, a college journalist had the scoop}}.{]}

In recent months, the administration has lost John R. Bolton, the
national security adviser; Fiona Hill, the top Europe official on the
National Security Council staff; and Dan Coats, the director of national
intelligence, all of whom sympathized with Ukraine in its conflict with
Russia.

Moreover, the United States Embassy in Kiev is still without an
ambassador after the administration
\href{https://www.nytimes.com/2019/09/26/us/politics/yovanovitch-trump-ukraine-ambassador.html}{yanked
home Marie L. Yovanovitch}, a career diplomat who was targeted by the
president and Mr. Giuliani for ostensibly being insufficiently loyal, a
charge heatedly disputed by her colleagues.

Senator Christopher S. Murphy, Democrat of Connecticut, expressed regret
at Mr. Volker's resignation. ``He has a well deserved reputation for
fairness, toughness and integrity, which is why I was so disappointed to
see him caught up in this mess,''
\href{https://twitter.com/ChrisMurphyCT/status/1177760491617341440?s=20}{Mr.
Murphy wrote on Twitter}. ``He now must put country first, and tell what
he did and what he knows.''

Advertisement

\protect\hyperlink{after-bottom}{Continue reading the main story}

\hypertarget{site-index}{%
\subsection{Site Index}\label{site-index}}

\hypertarget{site-information-navigation}{%
\subsection{Site Information
Navigation}\label{site-information-navigation}}

\begin{itemize}
\tightlist
\item
  \href{https://help.nytimes.com/hc/en-us/articles/115014792127-Copyright-notice}{©~2020~The
  New York Times Company}
\end{itemize}

\begin{itemize}
\tightlist
\item
  \href{https://www.nytco.com/}{NYTCo}
\item
  \href{https://help.nytimes.com/hc/en-us/articles/115015385887-Contact-Us}{Contact
  Us}
\item
  \href{https://www.nytco.com/careers/}{Work with us}
\item
  \href{https://nytmediakit.com/}{Advertise}
\item
  \href{http://www.tbrandstudio.com/}{T Brand Studio}
\item
  \href{https://www.nytimes.com/privacy/cookie-policy\#how-do-i-manage-trackers}{Your
  Ad Choices}
\item
  \href{https://www.nytimes.com/privacy}{Privacy}
\item
  \href{https://help.nytimes.com/hc/en-us/articles/115014893428-Terms-of-service}{Terms
  of Service}
\item
  \href{https://help.nytimes.com/hc/en-us/articles/115014893968-Terms-of-sale}{Terms
  of Sale}
\item
  \href{https://spiderbites.nytimes.com}{Site Map}
\item
  \href{https://help.nytimes.com/hc/en-us}{Help}
\item
  \href{https://www.nytimes.com/subscription?campaignId=37WXW}{Subscriptions}
\end{itemize}
