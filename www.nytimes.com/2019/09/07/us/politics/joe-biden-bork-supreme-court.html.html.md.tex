\href{/section/politics}{Politics}\textbar{}Biden Wants to Work With
`the Other Side.' This Supreme Court Battle Explains Why.

\url{https://nyti.ms/2A1EjIp}

\begin{itemize}
\item
\item
\item
\item
\item
\item
\end{itemize}

\begin{itemize}
\item
  \href{https://www.nytimes.com/2020/07/31/us/elections/biden-vs-trump.html?action=click\&pgtype=Article\&state=default\&region=TOP_BANNER\&context=storylines_menu}{Election
  Updates}
\item
  \href{https://www.nytimes.com/article/biden-vice-president-2020.html?action=click\&pgtype=Article\&state=default\&region=TOP_BANNER\&context=storylines_menu}{Biden's
  V.P. Search}
\item
  \href{https://www.nytimes.com/interactive/2020/07/24/us/politics/trump-biden-campaign-donors.html?action=click\&pgtype=Article\&state=default\&region=TOP_BANNER\&context=storylines_menu}{Map
  of Donations}
\item
  \href{https://www.nytimes.com/interactive/2020/us/elections/delegate-count-primary-results.html?action=click\&pgtype=Article\&state=default\&region=TOP_BANNER\&context=storylines_menu}{Delegate
  Count}
\item
  \href{https://www.nytimes.com/interactive/2019/us/politics/2020-presidential-candidates.html?action=click\&pgtype=Article\&state=default\&region=TOP_BANNER\&context=storylines_menu}{The
  Candidates}
\item
  \href{https://www.nytimes.com/newsletters/politics?action=click\&pgtype=Article\&state=default\&region=TOP_BANNER\&context=storylines_menu}{Politics
  Newsletter}
\end{itemize}

\includegraphics{https://static01.nyt.com/images/2019/09/05/us/politics/00bidenbork1/merlin_160201764_ee87d1e8-3d12-4797-add3-5fb47d2f6ec8-articleLarge.jpg?quality=75\&auto=webp\&disable=upscale}

Sections

\protect\hyperlink{site-content}{Skip to
content}\protect\hyperlink{site-index}{Skip to site index}

LESSONS IN power

\hypertarget{biden-wants-to-work-with-the-other-side-this-supreme-court-battle-explains-why}{%
\section{Biden Wants to Work With `the Other Side.' This Supreme Court
Battle Explains
Why.}\label{biden-wants-to-work-with-the-other-side-this-supreme-court-battle-explains-why}}

In the clash over Robert H. Bork's nomination, Joe Biden's moderate
instincts defined a winning strategy.

Joseph R. Biden Jr. during the Robert H. Bork hearings.Credit...Jose R.
Lopez/The New York Times

Supported by

\protect\hyperlink{after-sponsor}{Continue reading the main story}

\href{https://www.nytimes.com/by/alexander-burns}{\includegraphics{https://static01.nyt.com/images/2018/09/25/multimedia/author-alexander-burns/author-alexander-burns-thumbLarge-v2.png}}

By \href{https://www.nytimes.com/by/alexander-burns}{Alexander Burns}

\begin{itemize}
\item
  Sept. 7, 2019
\item
  \begin{itemize}
  \item
  \item
  \item
  \item
  \item
  \item
  \end{itemize}
\end{itemize}

Joseph R. Biden Jr. was on the brink of victory, but he was unsatisfied.

Mr. Biden, the 44-year-old chairman of the Senate Judiciary Committee,
was poised to watch his colleagues reject President Ronald Reagan's
formidable nominee to the Supreme Court, Robert H. Bork. The vote was
unlikely to be close. Yet Mr. Biden was hovering in the Senate chamber,
plying Senator John W. Warner of Virginia, a Republican of modestly
conservative politics and regal bearing, with arguments about Bork's
record.

Rejecting a Supreme Court nominee was an extraordinary act of defiance,
and Mr. Biden did not want a narrow vote that could look like an act of
raw partisan politics.

``We already had Bork beat,'' said Mark Gitenstein, who was then chief
counsel to Mr. Biden's committee. ``But Biden really wanted to get
Warner because he had such stature.''

Mr. Biden's entreaties prevailed: Mr. Warner became
\href{https://www.nytimes.com/1987/10/24/politics/borks-nomination-is-rejected-5842-reagan-saddened.html}{one
of 58 senators} to vote against Bork, and one of six Republicans.

The Senate's resounding rejection of Judge Bork in the fall of 1987 was
a turning point, the first time it spurned a nominee to the high court
for primarily ideological reasons. The vote ensured that the court's
swing seat would not go to a man with a long history of criticizing
rulings on the rights of African-Americans and women. It also enraged a
generation of conservatives and transformed the judge's name into an
ominous verb: Fearful of getting ``Borked,'' no nominee would ever again
speak so freely about his views as Bork did.

\includegraphics{https://static01.nyt.com/images/2019/09/05/us/politics/00bidenbork2-sub/merlin_160201737_1e2ce8d2-d706-499a-97fd-d70873992289-articleLarge.jpg?quality=75\&auto=webp\&disable=upscale}

It was also a personal turning point for Mr. Biden. In the Bork debate,
Mr. Biden's political ethos found its most vivid and successful
expression.

A review of Mr. Biden's conduct in the debate --- including interviews
with 16 people directly involved in the nomination fight, and a review
of the hearings and Mr. Biden's speeches --- yielded a portrait of Mr.
Biden as an ambitious young senator determined to achieve a vital
liberal goal by decidedly unradical means.

The strategy Chairman Biden deployed then is the same one he is now
proposing to bring to the White House as President Biden.

In the 1980s, as today,
\href{https://www.nytimes.com/2019/06/21/us/politics/biden-democrats-race.html}{he
saw bipartisan compromise} not as a version of surrender, but as a vital
tool for achieving Democratic goals.

Then, as now, Mr. Biden saw the culture and traditions of the Senate not
as crippling obstacles, but as instruments that could be bent to his
advantage.

And in both defining moments ---~his leadership of the Bork hearings and
his third presidential campaign ---~Mr. Biden made persuading moderates,
rather than exciting liberals, his guiding objective.

Mr. Biden, whose campaign declined to make him available for an
interview, has strained to defend this approach in the 2020 presidential
primary, offering only a halting rationale for a political worldview
that other Democrats see as out of date. His rivals have branded him as
a timid and even reactionary figure --- a creature of the Senate
cloakroom who partnered with former segregationists to pass
\href{https://www.nytimes.com/2019/06/25/us/joe-biden-crime-laws.html}{draconian
anti-crime legislation} and joined with the business lobby to
\href{https://www.nytimes.com/2015/08/31/us/politics/banking-ties-could-hurt-joe-biden-in-race-with-populist-overtone.html}{tighten
bankruptcy laws}.

\hypertarget{latest-updates-2020-election}{%
\section{\texorpdfstring{\href{https://www.nytimes.com/2020/07/31/us/elections/biden-vs-trump.html?action=click\&pgtype=Article\&state=default\&region=MAIN_CONTENT_1\&context=storylines_live_updates}{Latest
Updates: 2020
Election}}{Latest Updates: 2020 Election}}\label{latest-updates-2020-election}}

Updated 2020-08-01T01:26:45.732Z

\begin{itemize}
\tightlist
\item
  \href{https://www.nytimes.com/2020/07/31/us/elections/biden-vs-trump.html?action=click\&pgtype=Article\&state=default\&region=MAIN_CONTENT_1\&context=storylines_live_updates\#link-29fdff45}{Kamala
  Harris, a top vice-presidential contender, confronts double
  standards.}
\item
  \href{https://www.nytimes.com/2020/07/31/us/elections/biden-vs-trump.html?action=click\&pgtype=Article\&state=default\&region=MAIN_CONTENT_1\&context=storylines_live_updates\#link-13ec3d9c}{Karen
  Bass and Susan Rice are rising on Biden's vice-presidential
  shortlist.}
\item
  \href{https://www.nytimes.com/2020/07/31/us/elections/biden-vs-trump.html?action=click\&pgtype=Article\&state=default\&region=MAIN_CONTENT_1\&context=storylines_live_updates\#link-49e9a016}{Trump
  says Russian bounties to kill U.S. troops `never took place.'}
\end{itemize}

\href{https://www.nytimes.com/2020/07/31/us/elections/biden-vs-trump.html?action=click\&pgtype=Article\&state=default\&region=MAIN_CONTENT_1\&context=storylines_live_updates}{See
more updates}

And Mr. Biden's opponents point not to the Bork hearings but a different
confirmation battle as proof that his instincts are flawed. Four years
after Bork was defeated, Mr. Biden would again take an accommodating
approach to his Republican colleagues during Justice Clarence Thomas's
confirmation hearings, allowing harsh and invasive questioning of Anita
Hill, the law professor who accused the nominee of sexual harassment.
Mr. Biden
\href{https://www.nytimes.com/2019/04/26/us/politics/anita-hill-biden-clarence-thomas.html}{would
later expres}s ``regret'' for the treatment she endured.

But he has never regretted the conciliatory style that led him to
triumph against Bork. In that process**,** every important decision Mr.
Biden made was aimed at winning over conservative Democrats and moderate
Republicans --- men like Mr. Warner.

Image

Mr. Biden and Mr. Bork sharing a light moment during the
hearings.Credit...Jose R. Lopez/The New York Times

Now 92, Mr. Warner said in an interview that his memories of the Bork
hearings had grown foggy over the years. But two impressions were
indelible, he said. The first concerned Reagan's nominee: ``I never
encountered a man with a shorter temper,'' Mr. Warner said.

The second concerned the caliber of the Senate's deliberations.

``It was a real, solid, good debate, led by Biden,'' Mr. Warner said.
``He showed extraordinary leadership.''

The outcome was not foreordained, for either Bork or Mr. Biden. The
debate unfolded at a moment of humiliation for Mr. Biden, whose first
\href{https://www.nytimes.com/2019/06/03/us/politics/biden-1988-presidential-campaign.html}{campaign
for president unraveled} as the Bork hearings approached their climax.
And the judge was
\href{https://www.nytimes.com/2012/12/20/us/robert-h-bork-conservative-jurist-dies-at-85.html}{no
timid adversary}, as the journalist Ethan Bronner wrote in a book on the
nomination.

``Robert Bork,'' Mr. Bronner wrote, ``was a man of war.''

\hypertarget{freeze-biden}{%
\subsubsection{\texorpdfstring{\textbf{`Freeze
Biden'}}{`Freeze Biden'}}\label{freeze-biden}}

Mr. Biden was seated behind a desk in a spacious living room adjoining
his study at his Wilmington, Del., home. A few aides sat or stood around
the room, where pizza was in generous supply. Squared off against Mr.
Biden was Robert H. Bork --- or rather, a convincing simulacrum played
by the constitutional scholar Laurence Tribe.

Mr. Tribe and Mr. Biden would spar for hours in a series of sessions
that August, joined occasionally by other legal experts who would help
Mr. Biden hone his queries on subjects from antitrust regulation to
sexual privacy.

``Biden's questions were really smart, and they also needed some
sharpening,'' Mr. Tribe said in an interview, citing Mr. Biden's
tendency to ``ask one thing and mean something slightly different.''

Mr. Biden came to those training sessions by a jagged path, shaped by
pressure from progressive activists and the delicate politics of the
Judiciary Committee. He was arming himself to oppose Bork, but not with
the methods of the left.

Image

Protesters against Mr. Biden's plans to block the nomination of Mr. Bork
for Supreme Court Justice.~Credit...Jim Cole/Associated Press

On the day Bork was nominated, liberals viewed Mr. Biden with suspicion.
Taking over one of the Senate's great committees at a boyish --- for the
Senate --- age of 44, Mr. Biden had already split with progressives on
the issue of
\href{https://www.nytimes.com/2019/06/21/us/politics/joe-biden-james-eastland.html}{busing
as a means of desegregating schools}. Until Bork, the authors Michael
Pertschuk and Wendy Schaetzel would write, Mr. Biden ``had been
reluctant to challenge Reagan's transformation of the federal
judiciary.''

The previous November, the soon-to-be chairman had given liberals new
reason for concern, suggesting to The Philadelphia Inquirer that he
might one day vote to put Bork on the Supreme Court, should he be
Reagan's next nominee.

``I'm not Teddy Kennedy,'' he told the newspaper.

When
\href{https://www.nytimes.com/1998/08/26/us/lewis-powell-crucial-centrist-justice-dies-at-90.html}{Justice
Lewis F. Powell Jr.}, a flexible conservative, resigned from the court
in late June, Mr. Biden found himself in the shadow of Kennedy, the
party's leading liberal, and laboring to reconcile his own moderate
instincts with a mood of alarm on the left. When the White House
announced Bork's nomination on the first day of July, Kennedy delivered
a thunderous warning from the Senate floor: In ``Robert Bork's
America,'' Kennedy said, ``women would be forced into back-alley
abortions, blacks would sit at segregated lunch counters.''

The scathing address was a call to arms for the left, and it helped
animate a coalition of progressives --- led by feminists, civil rights
activists and labor unions --- that applied pressure to undecided
senators throughout the summer.

``His record was so extensive, and it touched almost every issue of
importance to American life,'' said Nan Aron, a leading anti-Bork
activist. ``It wasn't simply a single issue that caused people to be
alarmed.''

Another purpose of Kennedy's speech, his allies have said, was to ensure
Mr. Biden would not cave.

``One of the reasons for `Robert Bork's America' was to freeze Biden,''
Jeffrey Blattner, a Kennedy aide, would say decades later,
\href{https://www.emkinstitute.org/resources/jeffrey-blattner}{in an
oral history} for the Edward M. Kennedy Institute for the United States
Senate. ``He's running for president. We didn't want to leave him any
choice.''

Mr. Biden quickly aligned himself with Kennedy, and, at his liberal
colleague's urging, secured an agreement from Senator Strom Thurmond ---
the 84-year-old former segregationist who was the Judiciary Committee's
top Republican --- to delay Bork's hearings until September.

``Biden was under a lot of pressure, particularly from the liberal
senators,'' said former Senator Dennis DeConcini of Arizona, a centrist
Democrat who said he began the confirmation process favorably disposed
toward Bork. ``At first, I was leaning strongly to vote for him.''

Image

Senator Strom Thurmond, Joseph Biden, and Edward Kennedy.Credit...John
Duricka/Associated Press

Even as he pledged to oppose Bork, Mr. Biden made clear to progressive
leaders in a private meeting that he saw his role as sharply distinct
from theirs. He would play an inside game aimed at swaying Senate
moderates, starting with the four undecided members of his committee:
Mr. DeConcini and two other Democrats, Robert C. Byrd of West Virginia
and Howell Heflin of Alabama, and a Republican, Arlen Specter of
Pennsylvania.

Ralph Neas, a civil rights activist who joined the liberals' initial
meeting with Mr. Biden, said the chairman conveyed ``that he would take
the lead and we would try to put together a bipartisan coalition.''

``Biden's street cred with a lot of the centrists was quite high,'' Mr.
Neas said.

Mr. Biden was blunter with his aides: He would not adopt Kennedy's
rhetoric or make abortion his central cause. According to a book Mr.
Gitenstein published in 1992 about the confirmation fight, Mr. Biden
feared Bork would overturn Roe v. Wade but told aides he did not see the
case as ``great constitutional law.'' More disturbing to him --- and, he
believed, more likely to sway undecided voters --- was a Connecticut
case on contraception that revealed Bork's doubts about a broader right
to privacy.

``It really concerns me more than abortion,'' Mr. Biden is quoted as
saying in the book.

In their sessions, Mr. Tribe said, the future vice president wrestled
not just with Bork's record but also with the idea of disqualifying
nominees based on individual issues.

``I remember pushing back on Biden, saying, `If you think Roe v. Wade
really ought to be the law of the land, shouldn't that count?''' Mr.
Tribe recalled. ``He said, `Yes, it should count a lot, but I still
don't want to have a flat litmus test.'''

Mr. Tribe remembered thinking: ``This guy's a little bit more cautious
than I am. But that's fine, he's playing a different role.''

\hypertarget{defender-of-the-senate}{%
\subsubsection{Defender of the Senate}\label{defender-of-the-senate}}

Mr. Biden's self-assigned role was readily apparent as the Bork hearings
began in mid-September. Beaming down at the judge from a crowded dais,
Mr. Biden praised him as man of towering achievement and ``provocative''
views. Flanked by Kennedy at one elbow and Thurmond at the other, Mr.
Biden said the hearings should not be ``clouded by strident rhetoric
from the far left or the far right.''

``Anytime you feel you want to expand on an answer, you are not bound by
time,'' Mr. Biden encouraged Bork, adding in a tone of levity, ``Go
ahead and bog us down.''

Image

In the Bork hearings, every important decision Mr. Biden made was aimed
at winning over conservative Democrats and moderate
Republicans.Credit...Jose R. Lopez/The New York Times

The judge, bearded and broad shouldered, did not recognize the trap.

Few men could have been more prepared to face a constitutional
interrogation. A former Yale Law School professor who served as the
country's solicitor general and, amid the maelstrom of Watergate, as
acting attorney general, Bork brought to the hearings a reputation for
quick eloquence and utter mastery of the law.

Mr. Biden had no such reputation, and
\href{https://www.washingtonpost.com/archive/opinions/1987/07/02/biden-v-bork/be124295-d2a5-4353-ad3a-a05c20ee0c32/}{the
columnist George F. Will} spoke for much of Washington when he predicted
Bork would be ``more than a match for Biden.''

The chairman gave his colleagues wide latitude to question Bork, whose
testimony consumed five days. It culminated in an unusual Saturday
hearing that was dominated by an hourslong debate between Bork and
Specter, a former district attorney who frequently rode the Amtrak rails
with Mr. Biden, about the meaning of constitutional intent. Mr. Biden
had offered Specter half an hour for his questions; when Specter balked
at the time limit, Mr. Biden relented and opened the way for a crucial
exchange.

``His debate with my father on constitutional law did reveal him to be
not sufficiently respectful of precedent, which pushed my father against
him, and pushed other swing senators against him,'' said Shanin Specter,
the senator's son and a Philadelphia lawyer. ``It would not have
happened if Biden, as chair, hadn't permitted the hearings to go exactly
as long as they needed to go.''

Mr. Biden sought, too, to quash attacks on Bork that he saw as risking
political backlash. He shot down a plan to ambush Bork with a recording
of a speech he gave in 1985, insisting on sharing it with the judge
before airing it in the committee. And Mr. Biden and his aides refused a
request from a number of prominent activists, including Ralph Nader, to
testify in opposition to Bork. The left was applying powerful pressure
from outside the Senate, but Mr. Biden preferred that its leaders stay
there --- on the outside.

Ms. Aron, who would later clash with Mr. Biden over the nomination of
\href{https://www.nytimes.com/2019/04/26/us/politics/anita-hill-biden-clarence-thomas.html}{Justice
Thomas} in 1991, said the combination of popular pressure on the Senate
and Mr. Biden's high-minded hearings doomed the nominee.

``What defeated Robert Bork was public pressure,'' Ms. Aron said. ``But
what allowed the public to engage was a review of Bork's record.''

And Bork did himself few favors: While he assured senators, in his
rumbling voice, that he would not overturn rulings capriciously, he
struggled to explain away past comments decrying ``dozens'' of shoddy
Supreme Court decisions or deriding the Civil Rights Act of 1964, or
ridiculing the concept of a constitutional right to privacy. He startled
even some allies by describing as ``troublesome'' the reasoning behind a
1954 case desegregating public schools in the nation's capital.

In his questions, Mr. Biden posed as a mere mortal grappling with the
ideas of a giant.

``Clearly, I do not want to get into a debate with a professor,'' Mr.
Biden stressed, prodding Mr. Bork about the Griswold v. Connecticut case
that ended a state prohibition on birth control: ``As I hear you, you do
not believe there is a general right of privacy that is in the
Constitution.''

``Not one derived in that fashion,'' Bork said of the popular decision.
``There may be other arguments, and I do not want to pass upon those.''

Watching Bork's testimony, his political backers knew he was losing. He
was articulate, but he was also argumentative. His knowledge of the law
was powerful, his political antennae were not.

``I can't blame Biden,'' reflected Tom Korologos, the Republican
lobbyist tasked with ushering Bork onto the court. ``I blame Bork and
Specter, and the other senators, for going on and on.''

Every swing vote on Mr. Biden's committee swung against Bork, sending
him to the floor with a negative recommendation by a vote of 9 to 5. The
White House offered Bork the chance to withdraw; he chose martyrdom
instead.

Image

Mr. Biden, right, shakes hands with Republican Senator Alan Simpson
after it was voted not to recommend the confirmation of Mr.
Bork.~Credit...John Duricka/Associated Press

His supporters gave him that much, accusing Bork's opponents of bowing
to activists like Mr. Neas and Ms. Aron. ``The man's been trashed in our
house,'' Senator John Danforth, Republican of Missouri, lamented on the
Senate floor. ``Some of us helped generate the trashing. Others of us
yielded to it.''

Mr. Biden called Mr. Danforth's complaint an insult to the Senate.

``I have a higher opinion of the ability of my colleagues to do what's
right than, apparently, the senator from Missouri does,'' he said.

Mr. Biden's approach to the Bork nomination was a legislative and
political success, one he experienced as personal redemption after his
presidential candidacy crumbled. It brought to maturity the strategic
instincts that defined him in subsequent battles --- including his
contested stewardship of the Thomas hearings --- and that shape his
candidacy today.

The fate of Mr. Biden's campaign, and perhaps a future presidency, may
hinge on whether that version of leadership, defined by collegiality and
adherence to procedure, can inspire Democrats and coax cooperation from
Republicans. In the presidential race, there is no Ted Kennedy to sound
a trumpet for the left while Mr. Biden plays a methodical inside game.
And there are no Republicans to be found in the Senate like Specter, who
eventually, at Mr. Biden's urging, quit the G.O.P. to become a Democrat
before his death in 2012.

Still, Mr. Gitenstein said he had encouraged the former vice president
to draw public attention to his role in the 1987 court fight. The defeat
of Robert Bork averted a solidly conservative majority, handing the
court's decisive seat to the more pliant Anthony M. Kennedy, who
\href{https://www.nytimes.com/2018/06/27/us/politics/anthony-kennedy-retire-supreme-court.html}{became
a decisive figure} in a generation's worth of eclectic rulings on
subjects from campaign finance and union rights to abortion and the
legal definition of marriage.

``I don't think he or anyone else makes enough of the fact that, but for
Biden, Roe would be dead 30 years ago, and, but for Biden, we wouldn't
have the gay marriage decision,'' Mr. Gitenstein said. ``I've talked to
him about it. He's got so much on his platter.''

Mr. DeConcini, who at 82 is a supporter of Mr. Biden's campaign, said he
hoped a strategy of moderation could prevail again.

But he admitted to having doubts.

``I'd like to think so, I really would,'' Mr. DeConcini said. ``I'm just
not sure.''

\hypertarget{our-2020-election-guide}{%
\section{Our 2020 Election Guide}\label{our-2020-election-guide}}

Updated July 31, 2020

\begin{itemize}
\item
  \begin{center}\rule{0.5\linewidth}{\linethickness}\end{center}

  \hypertarget{the-latest}{%
  \subsection{The Latest}\label{the-latest}}

  \begin{itemize}
  \tightlist
  \item
    President Trump's assault on the Postal Service is intersecting with
    his attacks on mail-in voting.
    \href{https://www.nytimes.com/2020/07/31/us/politics/trump-usps-mail-delays.html?action=click\&pgtype=Article\&state=default\&region=BELOW_MAIN_CONTENT\&context=storylines_guide}{Voting
    rights groups say it is a recipe for disaster.}
  \end{itemize}
\item
  \begin{center}\rule{0.5\linewidth}{\linethickness}\end{center}

  \hypertarget{bidens-vp-search}{%
  \subsection{Biden's V.P. Search}\label{bidens-vp-search}}

  \begin{itemize}
  \tightlist
  \item
    \href{https://www.nytimes.com/article/biden-vice-president-2020.html?action=click\&pgtype=Article\&state=default\&region=BELOW_MAIN_CONTENT\&context=storylines_guide}{Here
    are 13 women} who have been under consideration to be Joe Biden's
    running mate, and why each might be chosen --- and might not be.
  \end{itemize}
\item
  \begin{center}\rule{0.5\linewidth}{\linethickness}\end{center}

  \hypertarget{keep-up-with-our-coverage}{%
  \subsection{Keep Up With Our
  Coverage}\label{keep-up-with-our-coverage}}

  \begin{itemize}
  \tightlist
  \item
    Get an
    \href{https://www.nytimes.com/newsletters/politics?action=click\&pgtype=Article\&state=default\&region=BELOW_MAIN_CONTENT\&context=storylines_guide}{email}
    recapping the day's news
  \end{itemize}

  \begin{itemize}
  \tightlist
  \item
    Download our mobile app on
    \href{https://apps.apple.com/us/app/nytimes/id284862083?ls=1\&mat_click_id=5c79ae7455014fd1bd66b5610c05b8f2-20191112-16948\&referrer=mat_click_id\%3D5c79ae7455014fd1bd66b5610c05b8f2-20191112-16948\%26link_click_id\%3D722930677036718082}{iOS}
    and
    \href{http://a.localytics.com/android?id=com.nytimes.android\&referrer=utm_source\%3Dother_nyt_mobile_web\%26utm_medium\%3DWeb\%2520page\%26utm_term\%3DGeneral\%2520Mobile\%2520Page\%26utm_campaign\%3DNYT\%2520Mobile\%2520General\%2520Page}{Android}
    and turn on Breaking News and Politics alerts
  \end{itemize}
\end{itemize}

Advertisement

\protect\hyperlink{after-bottom}{Continue reading the main story}

\hypertarget{site-index}{%
\subsection{Site Index}\label{site-index}}

\hypertarget{site-information-navigation}{%
\subsection{Site Information
Navigation}\label{site-information-navigation}}

\begin{itemize}
\tightlist
\item
  \href{https://help.nytimes.com/hc/en-us/articles/115014792127-Copyright-notice}{©~2020~The
  New York Times Company}
\end{itemize}

\begin{itemize}
\tightlist
\item
  \href{https://www.nytco.com/}{NYTCo}
\item
  \href{https://help.nytimes.com/hc/en-us/articles/115015385887-Contact-Us}{Contact
  Us}
\item
  \href{https://www.nytco.com/careers/}{Work with us}
\item
  \href{https://nytmediakit.com/}{Advertise}
\item
  \href{http://www.tbrandstudio.com/}{T Brand Studio}
\item
  \href{https://www.nytimes.com/privacy/cookie-policy\#how-do-i-manage-trackers}{Your
  Ad Choices}
\item
  \href{https://www.nytimes.com/privacy}{Privacy}
\item
  \href{https://help.nytimes.com/hc/en-us/articles/115014893428-Terms-of-service}{Terms
  of Service}
\item
  \href{https://help.nytimes.com/hc/en-us/articles/115014893968-Terms-of-sale}{Terms
  of Sale}
\item
  \href{https://spiderbites.nytimes.com}{Site Map}
\item
  \href{https://help.nytimes.com/hc/en-us}{Help}
\item
  \href{https://www.nytimes.com/subscription?campaignId=37WXW}{Subscriptions}
\end{itemize}
