Sections

SEARCH

\protect\hyperlink{site-content}{Skip to
content}\protect\hyperlink{site-index}{Skip to site index}

\href{https://www.nytimes.com/section/climate}{Climate}

\href{https://myaccount.nytimes.com/auth/login?response_type=cookie\&client_id=vi}{}

\href{https://www.nytimes.com/section/todayspaper}{Today's Paper}

\href{/section/climate}{Climate}\textbar{}Trump's Dorian Tweet Whips Up
a Fight Over a Science Powerhouse

\url{https://nyti.ms/2Aj4fPI}

\begin{itemize}
\item
\item
\item
\item
\item
\end{itemize}

\href{https://www.nytimes.com/section/climate?action=click\&pgtype=Article\&state=default\&region=TOP_BANNER\&context=storylines_menu}{Climate
and Environment}

\begin{itemize}
\tightlist
\item
  \href{https://www.nytimes.com/2020/07/30/climate/sea-level-inland-floods.html?action=click\&pgtype=Article\&state=default\&region=TOP_BANNER\&context=storylines_menu}{Rising
  Seas}
\item
  \href{https://www.nytimes.com/interactive/2020/climate/trump-environment-rollbacks.html?action=click\&pgtype=Article\&state=default\&region=TOP_BANNER\&context=storylines_menu}{Trump's
  Changes}
\item
  \href{https://www.nytimes.com/interactive/2020/04/19/climate/climate-crash-course-1.html?action=click\&pgtype=Article\&state=default\&region=TOP_BANNER\&context=storylines_menu}{Climate
  101}
\item
  \href{https://www.nytimes.com/interactive/2018/08/30/climate/how-much-hotter-is-your-hometown.html?action=click\&pgtype=Article\&state=default\&region=TOP_BANNER\&context=storylines_menu}{Is
  Your Hometown Hotter?}
\item
  \href{https://www.nytimes.com/newsletters/climate-change?action=click\&pgtype=Article\&state=default\&region=TOP_BANNER\&context=storylines_menu}{Newsletter}
\end{itemize}

Advertisement

\protect\hyperlink{after-top}{Continue reading the main story}

Supported by

\protect\hyperlink{after-sponsor}{Continue reading the main story}

\hypertarget{trumps-dorian-tweet-whips-up-a-fight-over-a-science-powerhouse}{%
\section{Trump's Dorian Tweet Whips Up a Fight Over a Science
Powerhouse}\label{trumps-dorian-tweet-whips-up-a-fight-over-a-science-powerhouse}}

\includegraphics{https://static01.nyt.com/images/2019/09/13/climate/13CLI-NOAA1/13CLI-NOAA1-articleLarge.jpg?quality=75\&auto=webp\&disable=upscale}

\href{https://www.nytimes.com/by/christopher-flavelle}{\includegraphics{https://static01.nyt.com/images/2019/06/28/climate/author-chris-flavelle/author-chris-flavelle-thumbLarge-v3.png}}

By \href{https://www.nytimes.com/by/christopher-flavelle}{Christopher
Flavelle}

\begin{itemize}
\item
  Sept. 13, 2019
\item
  \begin{itemize}
  \item
  \item
  \item
  \item
  \item
  \end{itemize}
\end{itemize}

SILVER SPRING, Md. --- On Friday morning in the suburbs of Washington,
D.C., government scientists in khakis and sensible shoes bustled to work
--- beneath a towering bronze sculpture of a hand releasing seabirds ---
heading for a small scientific agency caught up in a political mess
triggered by President Trump's tweet about Hurricane Dorian.

One of the arriving employees was Neil Jacobs, the head of the agency,
the National Oceanic and Atmospheric Administration. ``We're under
investigation,'' a weary looking Dr. Jacobs said, a large messenger bag
slung over his shoulder. ``I can't talk.''

The
\href{https://www.nytimes.com/2019/09/11/climate/noaa-wilbur-ross-dorian.html?rref=collection\%2Fbyline\%2Fchristopher-flavelle\&action=click\&contentCollection=undefined\&region=stream\&module=stream_unit\&version=latest\&contentPlacement=3\&pgtype=collection}{investigations}
are examining an attack on the independence of an agency that, despite
its enormous importance to the United States economy, typically flies
well below the radar. That changed in recent weeks when meteorologists
working for NOAA corrected Mr. Trump on Twitter after he inaccurately
described Hurricane Dorian's path. The president then ordered the agency
to support his version of events, triggering a political clash.

\emph{{[}Want climate news in your inbox?}
\href{https://www.nytimes.com/newsletters/climate-change}{\emph{Sign up
here
for}}\textbf{\href{https://www.nytimes.com/newsletters/climate-change}{\emph{Climate
Fwd:}}}\emph{, our email newsletter.{]}}

The National Oceanic and Atmospheric Administration is hardly a
household name, yet it plays a significant role in modern life.

One of its main jobs is weather forecasting, producing the data that
farmers trust to plant their crops, airlines rely on to design their
routes and millions of Americans check obsessively on their smartphones.
The agency also studies the world's oceans, regulates fisheries and
operates sophisticated satellites that, among other things, detect
threats in space to help protect astronauts.

It ``touches every American life every single day, in a constructive
fashion that's generally appreciated,'' said Kathryn Sullivan, who was
nominated to senior scientific roles by Presidents George Bush and
George W. Bush, and went on to run NOAA under President Barack Obama.

NOAA's scientific research is also central to the United States' ability
to understand climate change --- a role that requires the agency to
conduct independent research, but puts it at cross purposes with a White
House that has repeatedly expressed skepticism of the established
science of global warming.

\includegraphics{https://static01.nyt.com/images/2019/09/13/science/13CLI-NOAA2/merlin_160056072_242b0d74-1871-4a20-9453-fcac09ada922-articleLarge.jpg?quality=75\&auto=webp\&disable=upscale}

Considering that, it is notable that the clash between Mr. Trump and the
agency wasn't about climate science --- which Mr. Trump in the past has
described as a hoax --- but over a statement by meteorologists
reassuring people in Alabama that (contrary to the president's
assertions) they were safe from Hurricane Dorian.

The White House referred a request for comment to the Commerce
Department, which oversees the agency and whose secretary, Wilbur Ross,
\href{https://www.nytimes.com/2019/09/09/climate/hurricane-dorian-trump-tweet.html?rref=collection\%2Fbyline\%2Fchristopher-flavelle\&action=click\&contentCollection=undefined\&region=stream\&module=stream_unit\&version=latest\&contentPlacement=6\&pgtype=collection}{threatened
to fire NOAA employees} amid the clash. In a statement, Kevin Manning, a
spokesman for the department, said that ``Secretary Ross did not
threaten to fire any NOAA staff over forecasting and public statements
about Hurricane Dorian.''

\href{https://www.nytimes.com/section/climate?action=click\&pgtype=Article\&state=default\&region=MAIN_CONTENT_1\&context=storylines_keepup}{}

\hypertarget{climate-and-environment-}{%
\subsubsection{Climate and Environment
›}\label{climate-and-environment-}}

\hypertarget{keep-up-on-the-latest-climate-news}{%
\paragraph{Keep Up on the Latest Climate
News}\label{keep-up-on-the-latest-climate-news}}

Updated July 30, 2020

Here's what you need to know about the latest climate change news this
week:

\begin{itemize}
\item
  \begin{itemize}
  \tightlist
  \item
    \href{https://www.nytimes.com/2020/07/30/climate/bangladesh-floods.html?action=click\&pgtype=Article\&state=default\&region=MAIN_CONTENT_1\&context=storylines_keepup}{Floods
    in}\href{https://www.nytimes.com/2020/07/30/climate/bangladesh-floods.html?action=click\&pgtype=Article\&state=default\&region=MAIN_CONTENT_1\&context=storylines_keepup}{Bangladesh}
    are punishing the people least responsible for climate change.
  \item
    As climate change raises sea levels,
    \href{https://www.nytimes.com/2020/07/30/climate/sea-level-inland-floods.html?action=click\&pgtype=Article\&state=default\&region=MAIN_CONTENT_1\&context=storylines_keepup}{storm
    surges and high tides} are likely to push farther inland.
  \item
    The E.P.A. inspector general plans to investigate whether a rollback
    of fuel efficiency standards
    \href{https://www.nytimes.com/2020/07/27/climate/trump-fuel-efficiency-rule.html?action=click\&pgtype=Article\&state=default\&region=MAIN_CONTENT_1\&context=storylines_keepup}{violated
    government rules}.
  \end{itemize}
\end{itemize}

NOAA's independence is partly structural, according to current and
former staff members. With the exception of the fisheries section, none
of its divisions is primarily focused on regulation. So --- unlike, say,
the Environmental Protection Agency or other government regulatory
bodies --- there are few industries with a financial stake in weakening
the agency or limiting its authority.

The agency has continued to produce a steady stream of climate-related
science. It puts out an annual
\href{https://www.ncdc.noaa.gov/sotc/global/201904}{Global Climate
Report} and told the public just a few weeks ago that
\href{https://www.noaa.gov/news/july-2019-was-hottest-month-on-record-for-planet}{this
July was the hottest on record}.

That stands in contrast to the actions of some other federal agencies.

The Environmental Protection Agency two years ago deleted the climate
page from its website
``\href{https://www.washingtonpost.com/news/energy-environment/wp/2017/05/04/the-epa-is-reviewing-its-climate-change-website-these-scientists-say-it-was-already-accurate/}{to
reflect E.P.A.'s priorities} under the leadership of President Trump.''
At the Centers for Disease Control and Prevention, the head of the
Climate and Health Program recently filed a whistle-blower complaint
alleging
\href{https://insideclimatenews.org/news/16082019/cdc-scientist-whistleblower-complaint-climate-health-research-trump-usda-epa}{retaliation
for speaking out on climate change}.

NOAA's ability to continue pursuing and disseminating climate science
stems partly from its relative anonymity, even within the federal
government. ``NOAA is a very small agency,'' said Paul Sandifer, the
agency's chief science adviser from 2011 to 2014. With fewer staff and a
slimmer budget than other scientific agencies, he said, it has generally
managed to evade scrutiny from the White House.

The agency is actually an amalgam of six separate pieces. The National
Weather Service, whose Birmingham office was the target of Mr. Trump's
ire, is responsible for forecasting. The Marine Fisheries Service
manages the waters off the country's coasts, and a separate office, the
National Ocean Service, produces coastal and oceanic science.

Image

Wilbur Ross Jr. spoke about the 2019 hurricane season outlook in
Arlington, Va., in May.Credit...Win Mcnamee/Getty Images

The Office of Oceanic and Atmospheric Research provides ``science to
better manage the environment,'' according to NOAA, and Marine \&
Aviation Operations runs the ships and planes that gather data. The
agency's satellite service
``\href{https://www.corporateservices.noaa.gov/public/lineoffices.html}{acquires
and manages the Nation's operational environmental satellites}.''

Many industries rely on the climate information produced by NOAA,
according to Eileen Shea, who was chief of the agency's climate services
division from 2007 to 2012. They include insurers, agricultural
producers and anyone deciding where to invest money in building a new
facility.

``There are corporations and lobbying groups with a very big interest in
continuing to see climate data be available,'' Ms. Shea said.

Culture within the agency matters, too. NOAA's scientists, and the
career staff members who oversee them, have a reputation for guarding
the agency's independence. Last year, the Union of Concerned Scientists
asked more than 63,000 scientists at 16 federal agencies to gauge their
perceived independence. Of the scientists at NOAA who responded,
two-thirds agreed with the statement that the agency
``\href{https://www.ucsusa.org/sites/default/files/attach/2018/08/science-under-trump-noaa.pdf}{adheres
to its scientific integrity policy}.'' (For comparison, one-third of
E.P.A. scientists felt the same way.)

Image

Neil Jacobs, NOAA's acting administrator, in May.Credit...Win
Mcnamee/Getty Images

Friday afternoon, Dr. Jacobs sent an all-staff email to try to buck up
the troops. ``Scientific integrity is at the heart of NOAA's mission and
culture, and is essential for maintaining the public's trust,'' he
wrote. ``Our work saves lives.''

The agency's sense of independence partly reflects the fact that, unlike
employees of other federal agencies, the agency's employees tend to live
in the places they serve and see themselves as the defenders of those
places. ``You actually want to tell me to not give my neighbors the best
information I have when a storm is bearing down on them?'' said Dr.
Sullivan, describing the typical view of those scientists.

But it also reflects the incentives facing the agency's staff.
Researchers there often have close relationships with universities,
collaborating with academics on peer-reviewed papers that can advance
their careers, according to Rick Spinrad, who was NOAA's chief scientist
from 2014 to 2017.

The perception that agency scientists are subject to political
interference could cause outside academics to stop working with them, he
said. ``Anything that they view as threatening that relationship is
going to induce a pretty visceral reaction,'' Dr. Spinrad said.

Past administrations have at times tested NOAA's independence, said
Terry Garcia, who was the agency's general counsel under President Bill
Clinton. He recounted other agencies pushing NOAA scientists to
interpret the Endangered Species Act --- which gives NOAA responsibility
for protecting salmon and other animals --- in a way that would help
private landowners. The agency resisted that pressure, he said.

Image

The National Oceanic and Atmospheric Administration offices in Silver
Spring, Md., in January.~Credit...Matt Roth for The New York Times

The Dorian episode isn't NOAA's first test under the Trump
administration.

In 2017, the president nominated Barry L. Myers, then the chief
executive of AccuWeather, to run the agency. Mr. Myers had previously
called to privatize the weather service, a stance that generated
opposition both inside and outside NOAA. His nomination has since
stalled.

Craig McLean, NOAA's acting chief scientist, who has filed a complaint
with the agency alleging it violated its scientific-integrity policy,
said he believes the current problems will dissipate. ``We've had
spurious attacks over time,'' Mr. McLean said. ``We've gotten past
them.''

Meanwhile on Friday afternoon, the National Hurricane Center, an office
of NOAA, was watching a storm of a different kind --- so far known
simply as ``Potential Tropical Cyclone Nine'' --- move northwest off the
coast of the Bahamas in order to start determining when and where it
might strike the United States.

For more news on climate and the environment,
\href{https://twitter.com/nytclimate}{follow @NYTClimate on Twitter}.

Nicholas Bogel-Burroughs contributed reporting from Huntsville, Ala.
Lisa Friedman contributed reporting from Washington.

Advertisement

\protect\hyperlink{after-bottom}{Continue reading the main story}

\hypertarget{site-index}{%
\subsection{Site Index}\label{site-index}}

\hypertarget{site-information-navigation}{%
\subsection{Site Information
Navigation}\label{site-information-navigation}}

\begin{itemize}
\tightlist
\item
  \href{https://help.nytimes.com/hc/en-us/articles/115014792127-Copyright-notice}{©~2020~The
  New York Times Company}
\end{itemize}

\begin{itemize}
\tightlist
\item
  \href{https://www.nytco.com/}{NYTCo}
\item
  \href{https://help.nytimes.com/hc/en-us/articles/115015385887-Contact-Us}{Contact
  Us}
\item
  \href{https://www.nytco.com/careers/}{Work with us}
\item
  \href{https://nytmediakit.com/}{Advertise}
\item
  \href{http://www.tbrandstudio.com/}{T Brand Studio}
\item
  \href{https://www.nytimes.com/privacy/cookie-policy\#how-do-i-manage-trackers}{Your
  Ad Choices}
\item
  \href{https://www.nytimes.com/privacy}{Privacy}
\item
  \href{https://help.nytimes.com/hc/en-us/articles/115014893428-Terms-of-service}{Terms
  of Service}
\item
  \href{https://help.nytimes.com/hc/en-us/articles/115014893968-Terms-of-sale}{Terms
  of Sale}
\item
  \href{https://spiderbites.nytimes.com}{Site Map}
\item
  \href{https://help.nytimes.com/hc/en-us}{Help}
\item
  \href{https://www.nytimes.com/subscription?campaignId=37WXW}{Subscriptions}
\end{itemize}
