Sections

SEARCH

\protect\hyperlink{site-content}{Skip to
content}\protect\hyperlink{site-index}{Skip to site index}

\href{https://www.nytimes.com/section/world/europe}{Europe}

\href{https://myaccount.nytimes.com/auth/login?response_type=cookie\&client_id=vi}{}

\href{https://www.nytimes.com/section/todayspaper}{Today's Paper}

\href{/section/world/europe}{Europe}\textbar{}William Taylor, Top
Diplomat in Ukraine and Key Impeachment Witness, Is Stepping Down

\url{https://nyti.ms/36NodjI}

\begin{itemize}
\item
\item
\item
\item
\item
\item
\end{itemize}

Advertisement

\protect\hyperlink{after-top}{Continue reading the main story}

Supported by

\protect\hyperlink{after-sponsor}{Continue reading the main story}

\hypertarget{william-taylor-top-diplomat-in-ukraine-and-key-impeachment-witness-is-stepping-down}{%
\section{William Taylor, Top Diplomat in Ukraine and Key Impeachment
Witness, Is Stepping
Down}\label{william-taylor-top-diplomat-in-ukraine-and-key-impeachment-witness-is-stepping-down}}

A Vietnam veteran with decades of State Department experience, William
B. Taylor Jr. sharply objected to what he saw as the Trump
administration's shadow foreign policy in Ukraine.

\includegraphics{https://static01.nyt.com/images/2019/12/17/us/politics/17dc-taylor/17dc-taylor-articleLarge.jpg?quality=75\&auto=webp\&disable=upscale}

\href{https://www.nytimes.com/by/lara-jakes}{\includegraphics{https://static01.nyt.com/images/2019/07/25/reader-center/author-lara-jakes/author-lara-jakes-thumbLarge.png}}

By \href{https://www.nytimes.com/by/lara-jakes}{Lara Jakes}

\begin{itemize}
\item
  Dec. 17, 2019
\item
  \begin{itemize}
  \item
  \item
  \item
  \item
  \item
  \item
  \end{itemize}
\end{itemize}

WASHINGTON --- William B. Taylor Jr., the top American diplomat in
Ukraine who described for Congress and the public what he saw as
President Trump's efforts to pressure Kyiv to go after political rivals,
said on Tuesday that he was stepping down from his post.

In a brief email to The New York Times, Mr. Taylor said he would leave
in early January because his temporary appointment to Ukraine last June
is set to expire. Under the Vacancies Act, political appointees in an
acting position can hold office only for about 200 days. Earlier in the
day, people familiar with the planning had suggested he would leave by
the end of December.

``The administration will nominate a permanent ambassador soon,'' Mr.
Taylor said. He did not elaborate. His departure was first reported by
NBC.

Mr. Taylor served as something of a star witness for the House
Intelligence Committee's impeachment inquiry against Mr. Trump. In
\href{https://www.nytimes.com/2019/11/13/us/politics/bill-taylor-impeachment-hearing.html}{public
testimony last month}, he calmly and confidently recounted for lawmakers
what he described as a pressure campaign by the Trump administration to
leverage American security aid to Ukraine in exchange for an
investigation into Mr. Trump's political opponents.

``Security was so important for Ukraine, as well as our own national
interests,'' Mr. Taylor testified at the Nov. 13 hearing. ``To withhold
that assistance for no good reason other than help with a political
campaign made no sense. It was counterproductive to all of what we had
been trying to do. It was illogical. It could not be explained. It was
crazy.''

At the hearing, Mr. Taylor described a growing sense of alarm at
learning that \$391 million in military aid for Ukraine had been held
up. He also said he had discovered that Mr. Trump was conditioning
``everything'' about the United States' relationship with Ukraine ---
including a White House meeting for Ukraine's president --- on the
country's willingness to commit publicly to investigations of his
political rivals.

In a July 25 telephone call, Mr. Trump pressed President Volodymyr
Zelensky of Ukraine to investigate two politically charged allegations:
one was a
\href{https://www.nytimes.com/2019/10/05/us/politics/pompeo-defends-trumps-ukraine-conspiracy-theory.html}{widely
debunked conspiracy theory} about Ukrainian involvement in 2016 election
tampering and the other was related to
\href{https://www.nytimes.com/2019/09/22/us/politics/biden-ukraine-trump.html?module=inline}{corruption
at an energy company} that employed the younger son of former Vice
President Joseph R. Biden Jr., Hunter Biden. There is no evidence that
the Bidens were involved in wrongdoing.

Mr. Taylor, a longtime diplomat, was asked to come out of retirement
after the United States ambassador to Kyiv, Marie L. Yovanovitch, was
ousted for resisting a shadow foreign policy campaign in Ukraine that
was run by Mr. Trump's personal lawyer Rudolph W. Giuliani.

But after Mr. Taylor was pulled into what he called an ``irregular
channel'' of diplomatic relations between Washington and Kyiv, he became
one of the most senior State Department officials to openly challenge
it.

Withholding the security aid as threatened would be ``crazy,'' Mr.
Taylor wrote on Sept. 9 in a text to
\href{https://www.nytimes.com/2019/11/13/us/politics/impeachment-hearings.html}{Gordon
D. Sondland}, the United States ambassador to the European Union. Mr.
Taylor also threatened to quit if Ukrainian officials committed to an
investigation of Mr. Trump's rivals and still did not receive the \$391
million in aid --- what Mr. Taylor called a ``nightmare'' situation.

He was recalled to the State Department from a position helping lead the
United States Institute of Peace after nearly five decades of government
work --- including a tour in Vietnam as an Army infantry soldier, a
stint as a Senate staff member and diplomatic postings including
Brussels, Baghdad and Kabul, Afghanistan.

But he was expected to serve in Kyiv only temporarily.

Names already are being rumored for Mr. Taylor's replacement, including
retired
\href{https://www.marshallcenter.org/MCPUBLICWEB/en/nav-itemid-fix-bios-en/104-cat-bios-command-en/977-art-bio-dayton-keith-en.html}{Lt.
Gen. Keith W. Dayton}, according to a person familiar with the issue.
Mr. Dayton currently is the director of the George C. Marshall European
Center for Security Studies, in Germany, but was appointed in 2018 as
the senior United States defense adviser to Ukraine, according to his
biography.

Kenneth P. Vogel contributed reporting.

Advertisement

\protect\hyperlink{after-bottom}{Continue reading the main story}

\hypertarget{site-index}{%
\subsection{Site Index}\label{site-index}}

\hypertarget{site-information-navigation}{%
\subsection{Site Information
Navigation}\label{site-information-navigation}}

\begin{itemize}
\tightlist
\item
  \href{https://help.nytimes.com/hc/en-us/articles/115014792127-Copyright-notice}{©~2020~The
  New York Times Company}
\end{itemize}

\begin{itemize}
\tightlist
\item
  \href{https://www.nytco.com/}{NYTCo}
\item
  \href{https://help.nytimes.com/hc/en-us/articles/115015385887-Contact-Us}{Contact
  Us}
\item
  \href{https://www.nytco.com/careers/}{Work with us}
\item
  \href{https://nytmediakit.com/}{Advertise}
\item
  \href{http://www.tbrandstudio.com/}{T Brand Studio}
\item
  \href{https://www.nytimes.com/privacy/cookie-policy\#how-do-i-manage-trackers}{Your
  Ad Choices}
\item
  \href{https://www.nytimes.com/privacy}{Privacy}
\item
  \href{https://help.nytimes.com/hc/en-us/articles/115014893428-Terms-of-service}{Terms
  of Service}
\item
  \href{https://help.nytimes.com/hc/en-us/articles/115014893968-Terms-of-sale}{Terms
  of Sale}
\item
  \href{https://spiderbites.nytimes.com}{Site Map}
\item
  \href{https://help.nytimes.com/hc/en-us}{Help}
\item
  \href{https://www.nytimes.com/subscription?campaignId=37WXW}{Subscriptions}
\end{itemize}
