Sections

SEARCH

\protect\hyperlink{site-content}{Skip to
content}\protect\hyperlink{site-index}{Skip to site index}

\href{https://www.nytimes.com/section/arts/television}{Television}

\href{https://myaccount.nytimes.com/auth/login?response_type=cookie\&client_id=vi}{}

\href{https://www.nytimes.com/section/todayspaper}{Today's Paper}

\href{/section/arts/television}{Television}\textbar{}In 2019, Netflix
and Amazon Set Their Sights on India

\href{https://nyti.ms/2tbYAe5}{https://nyti.ms/2tbYAe5}

\begin{itemize}
\item
\item
\item
\item
\item
\end{itemize}

Advertisement

\protect\hyperlink{after-top}{Continue reading the main story}

Supported by

\protect\hyperlink{after-sponsor}{Continue reading the main story}

\hypertarget{in-2019-netflix-and-amazon-set-their-sights-on-india}{%
\section{In 2019, Netflix and Amazon Set Their Sights on
India}\label{in-2019-netflix-and-amazon-set-their-sights-on-india}}

TV's streaming wars are a global contest, and this year India became one
of the most hotly contested fronts.

\includegraphics{https://static01.nyt.com/images/2019/12/30/arts/30indiantv-print-bard-of-blood3/merlin_166389372_5ea04102-d858-486f-91ed-9a6379a7bda5-articleLarge.jpg?quality=75\&auto=webp\&disable=upscale}

By Priya Arora

\begin{itemize}
\item
  Dec. 30, 2019
\item
  \begin{itemize}
  \item
  \item
  \item
  \item
  \item
  \end{itemize}
\end{itemize}

With companies like Netflix and Amazon seeking ever larger international
customer bases, one thing that distinguishes the streaming wars from
previous TV conflicts is the global scale. And in 2019, India became one
of the most hotly contested fronts.

American streamers began dipping their toes in the Indian market in
2017, when Amazon Prime Video released the cricket drama ``Inside
Edge,'' its first original series from India. Netflix followed in 2018
with the crime-thriller ``Sacred Games.''

But both companies dramatically increased their investment in Indian
shows in 2019. They also struck partnerships with some of Bollywood's
most beloved actors, writers and producers.

This year Netflix released five original series and eight original films
produced in India. Among the most high-profile was the series ``Bard of
Blood,'' produced by Red Chillies Entertainment, the production house
owned by the Bollywood star Shah Rukh Khan and his wife, Gauri Khan.

In September, the platform also announced a multiyear collaboration deal
with the Indian producer and director Karan Johar's streaming production
house, Dharmatic Entertainment. Johar's first production with Netflix,
``Drive'' was released in November. New projects slated for 2020 include
debuts for the Bollywood actress Kajol as well as the fashion designer
Masaba Gupta.

Amazon released 10 original Indian shows in 2019, including the spy
series ``Family Man,'' starring the veteran Bollywood actor Manoj
Bajpayee, which was the company's first original series from India to be
dubbed in English. (Previous original offerings, like ``Inside Edge''
and ``Made in Heaven,'' used a mix of Hindi and English and have been
dubbed in other regional languages.) Amazon, too, is pulling in star
power: In addition to Bajpayee, his fellow Bollywood leads Akshay Kumar
and Anushka Sharma have Amazon projects in the works.

It's easy to see why Netflix and Amazon would want to expand their
presence in India, and not just because it is home to 1.3 billion
people. As in much of the viewing world, streaming has exploded in the
country --- at least 30 companies now offer video streaming services
there, up from nine in 2012.

What sets the American streamers apart is a focus on original
programming and an ability to market series for a global audience.

Hotstar, which controls an estimated 75 percent of India's streaming
market, carries critically acclaimed HBO shows like ``Succession'' and a
Hindi version of ``The Office'' alongside plenty of existing programming
from its parent company, Star India. (Like much of the entertainment
world, Star is a subsidiary of
\href{https://www.nytimes.com/2019/03/20/business/media/walt-disney-21st-century-fox-deal.html}{the
Walt Disney Company}.) Indian streamers that do carry originals, like
Zee5 and Eros Now, have had trouble cracking international markets.

``We look for stories that not only resonate with our Indian members but
can also travel globally,'' a Netflix spokeswoman said. For example, two
out of three ``Sacred Games'' viewers were from outside India.

The upshot is that a global audience has more access than ever to an
array of great South Asian series. Here's a look at some of the year's
best original streaming shows from the region.

\hypertarget{on-netflix}{%
\subsection{On Netflix}\label{on-netflix}}

\includegraphics{https://static01.nyt.com/images/2019/12/31/arts/30indiantv-print-sacred-games/merlin_166389249_dca3ca8d-f2bc-4b74-9cdb-9627ec9254a6-articleLarge.jpg?quality=75\&auto=webp\&disable=upscale}

\textbf{`Sacred Games,' Season 2}

The first season of ``Sacred Games'' earned critical acclaim, as well as
an International Emmy nomination. Starring the Bollywood actors Saif Ali
Khan and Nawazuddin Siddiqui,
\href{https://www.youtube.com/watch?v=w-Xe8gLBc5I}{the second season}
was Netflix's most popular release in India and continues the same
cop-and-robber plotline. While the second season has a slower pace than
the first, its deep dive into the past of its lead character, Sartaj
Singh (Khan), is thrilling.

\textbf{`Delhi Crime'}

Created as a true-crime anthology series that will follow a different
crime investigation each season,
\href{https://www.youtube.com/watch?v=jNuKwlKJx2E}{``Delhi Crime''}
focuses in its first season on the aftermath of the
\href{https://www.nytimes.com/2012/12/29/world/asia/condition-worsens-for-victim-of-gang-rape-in-india.html}{horrific
rape and death of a young woman} in New Delhi in 2012. Starring the
Bollywood actress Shefali Shah, the procedural series --- which unlike
in America, are rare on Indian television --- covers the six days
between the attack and the arrests of the perpetrators. Though the
outcome of the case is known to many, the details remain gripping, and
the show also explores the sense of unity that came after the attack in
a country that still struggles with violence against women.

\textbf{`Leila'}

Set in a dystopian future and based on a book of the same name,
\href{https://www.youtube.com/watch?v=5yxjRgwYymg}{``Leila''} follows
Shalini, a mother played by the Bollywood actress Huma Qureshi, who is
searching for her missing daughter, Leila. Arrested and struggling to
survive under a totalitarian regime with segregated communities
(recalling divisions over religion and caste), Shalini endures grueling
conditions as she uncovers the truth about what happened to her
daughter. Striking visuals by the acclaimed director Deepa Mehta and
clear references to contemporary Indian politics make the series even
more chilling.

\textbf{`Selection Day,' Season 2}

In India, you can't go wrong with cricket.
\href{https://www.youtube.com/watch?v=5tpEXBy0R0A}{``Selection Day,''}
based on the 2016 novel, follows two brothers with a father who is
hyperfocused on training them to be the best cricketers in the country.
Conflict arises as the young men cope differently with their father's
intensity, as well as with their own professional aspirations and
desires.

\hypertarget{on-amazon}{%
\subsection{On Amazon}\label{on-amazon}}

Image

Sobhita Dhulipala in ``Made in Heaven.''Credit...Amazon Studios

\textbf{`Made in Heaven'}

Amazon made a big splash earlier this year with
\href{https://www.nytimes.com/2019/04/09/arts/television/amazon-india-made-in-heaven.html}{``Made
in Heaven,''} a series about two friends who run a wedding planning
company. Unlike some of Netflix's most popular Indian series,
\href{https://www.youtube.com/watch?v=Qyg9_a7avTI}{``Made in Heaven''}
didn't just enlist big names from Bollywood, opting instead for some new
faces alongside seasoned actors like Kalki Koechlin. It received
critical acclaim for its nuanced handling of topical cultural issues
like homophobia.

\textbf{`Inside Edge,' Season 2}

The \href{https://www.youtube.com/watch?v=es_cjyjeYbc}{second season} of
Amazon's debut Indian drama was noticeably weaker than its first. But
with star power like Richa Chadha, Angad Bedi and Vivek Oberoi, this
remains one of Amazon's flagship shows.

\textbf{`Comicstaan,' Season 2}

This \href{https://www.youtube.com/watch?v=J4tAE4T72Bk}{reality show},
which highlights young Indian comics, is Amazon's biggest entrant in a
genre that Netflix has typically dominated internationally: stand-up
comedy. (In October, Netflix announced a slate of upcoming specials
featuring eight Indian comedians.)

Image

Manoj Bajpayee in ``The Family Man.''Credit...Amazon Studios

\textbf{`The Family Man'}

\href{https://www.youtube.com/watch?v=XatRGut65VI}{``The Family Man,''}
starring Bajpayee, is the story of Srikant Tiwari, an average
middle-class father who's trying to balance his family duties with,
well, saving the country. His family remains blissfully unaware that
he's a spy, and he tries (and often almost fails) to keep up the ruse
that he is a humble government employee with a boring desk job.

What sets apart ``The Family Man,'' by the writers and directors Raj
Nidimoru and Krishna D.K., is its clever way of fictionalizing current
events. ``The way we saw it was that when you open a newspaper and you
see various new stories in front of you, we wanted to put those stories
on-screen,'' Nidimoru said in a recent interview.

For Bajpayee, though, it's the uniqueness of Srikant, and the relatable
quality of his predicaments, that drew him to the project, he told The
Times. ``I look at him as a very open-minded modern man trying to
understand each and every point of view and still trying to do his
job,'' he said.

\textbf{`Laakhon Mein Ek,' Season 2}

\href{https://www.youtube.com/watch?v=bkVb9fguwuc}{This show} by the
comedian Biswa Kalyan Rath (also one of the judges in ``Comicstaan'')
tackles social issues from the perspective of India's youth. The second
season focused on rural health care and the politically-motivated black
market for generic medicine and stars Shweta Tripathi, a familiar face
to South Asian audiences.

\hypertarget{on-hotstar}{%
\subsection{On Hotstar}\label{on-hotstar}}

\textbf{``The Office''}

The Indian streaming service
\href{https://www.hotstar.com/us/subscribe/get-started}{Hotstar} lacks
the international profile of Netflix and Amazon, but it offers a deep
archive of South Asian series and films (and cricket). One worth
checking out is the Hindi version of
\href{https://www.youtube.com/watch?v=rdqZFIOhes8}{``The Office.''}
While superficially very different from its British and American
predecessors, the sitcom adapts well to new cultural norms while
suggesting that warm (and sometimes awkward) workplace friendships,
petty office politics and obnoxious bosses are universal.

Advertisement

\protect\hyperlink{after-bottom}{Continue reading the main story}

\hypertarget{site-index}{%
\subsection{Site Index}\label{site-index}}

\hypertarget{site-information-navigation}{%
\subsection{Site Information
Navigation}\label{site-information-navigation}}

\begin{itemize}
\tightlist
\item
  \href{https://help.nytimes.com/hc/en-us/articles/115014792127-Copyright-notice}{©~2020~The
  New York Times Company}
\end{itemize}

\begin{itemize}
\tightlist
\item
  \href{https://www.nytco.com/}{NYTCo}
\item
  \href{https://help.nytimes.com/hc/en-us/articles/115015385887-Contact-Us}{Contact
  Us}
\item
  \href{https://www.nytco.com/careers/}{Work with us}
\item
  \href{https://nytmediakit.com/}{Advertise}
\item
  \href{http://www.tbrandstudio.com/}{T Brand Studio}
\item
  \href{https://www.nytimes.com/privacy/cookie-policy\#how-do-i-manage-trackers}{Your
  Ad Choices}
\item
  \href{https://www.nytimes.com/privacy}{Privacy}
\item
  \href{https://help.nytimes.com/hc/en-us/articles/115014893428-Terms-of-service}{Terms
  of Service}
\item
  \href{https://help.nytimes.com/hc/en-us/articles/115014893968-Terms-of-sale}{Terms
  of Sale}
\item
  \href{https://spiderbites.nytimes.com}{Site Map}
\item
  \href{https://help.nytimes.com/hc/en-us}{Help}
\item
  \href{https://www.nytimes.com/subscription?campaignId=37WXW}{Subscriptions}
\end{itemize}
