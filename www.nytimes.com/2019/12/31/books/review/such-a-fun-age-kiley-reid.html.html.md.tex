Sections

SEARCH

\protect\hyperlink{site-content}{Skip to
content}\protect\hyperlink{site-index}{Skip to site index}

\href{https://www.nytimes.com/section/books/review}{Book Review}

\href{https://myaccount.nytimes.com/auth/login?response_type=cookie\&client_id=vi}{}

\href{https://www.nytimes.com/section/todayspaper}{Today's Paper}

\href{/section/books/review}{Book Review}\textbar{}When It Comes to
Race, How Progressive Are the Progressives?

\url{https://nyti.ms/2MK2dPf}

\begin{itemize}
\item
\item
\item
\item
\item
\end{itemize}

Advertisement

\protect\hyperlink{after-top}{Continue reading the main story}

Supported by

\protect\hyperlink{after-sponsor}{Continue reading the main story}

Fiction

\hypertarget{when-it-comes-to-race-how-progressive-are-the-progressives}{%
\section{When It Comes to Race, How Progressive Are the
Progressives?}\label{when-it-comes-to-race-how-progressive-are-the-progressives}}

\includegraphics{https://static01.nyt.com/images/2019/12/31/books/review/31christiansen/merlin_166014954_01c3d4f1-b3f6-4fa0-8ac6-d9cd3df7b5cc-articleLarge.jpg?quality=75\&auto=webp\&disable=upscale}

Buy Book ▾

\begin{itemize}
\tightlist
\item
  \href{https://www.amazon.com/gp/search?index=books\&tag=NYTBSREV-20\&field-keywords=Such+a+Fun+Age+Kiley+Reid}{Amazon}
\item
  \href{https://du-gae-books-dot-nyt-du-prd.appspot.com/buy?title=Such+a+Fun+Age\&author=Kiley+Reid}{Apple
  Books}
\item
  \href{https://www.anrdoezrs.net/click-7990613-11819508?url=https\%3A\%2F\%2Fwww.barnesandnoble.com\%2Fw\%2F\%3Fean\%3D9780525541905}{Barnes
  and Noble}
\item
  \href{https://www.anrdoezrs.net/click-7990613-35140?url=https\%3A\%2F\%2Fwww.booksamillion.com\%2Fp\%2FSuch\%2Ba\%2BFun\%2BAge\%2FKiley\%2BReid\%2F9780525541905}{Books-A-Million}
\item
  \href{https://bookshop.org/a/3546/9780525541905}{Bookshop}
\item
  \href{https://www.indiebound.org/book/9780525541905?aff=NYT}{Indiebound}
\end{itemize}

When you purchase an independently reviewed book through our site, we
earn an affiliate commission.

By \href{https://www.nytimes.com/by/lauren-christensen}{Lauren
Christensen}

\begin{itemize}
\item
  Dec. 31, 2019
\item
  \begin{itemize}
  \item
  \item
  \item
  \item
  \item
  \end{itemize}
\end{itemize}

\textbf{SUCH A FUN AGE}\\
By Kiley Reid

It's 2015 and, in a gentrified variation on ``driving while black,''
20-something Emira is accosted in the freezer aisle of an upscale
Philadelphia supermarket by a security guard accusing her of kidnapping
her white charge. In a midnight crisis, the Chamberlain household has
called Emira in from her night off to watch their toddler, Briar. But
the real crisis unfolds at the store. ``With all due respect,'' the
guard says to an indignant Emira, ``you don't look like you've been
babysitting tonight.'' Somehow this initial confrontation, filmed by a
fellow shopper and defused only by the arrival of Briar's dad, isn't
even the worst offense committed in Kiley Reid's provocative but soapy
debut novel, ``Such a Fun Age.'' It's merely a preamble for the main
narrative about how two white people end up using their proximity to
Emira, a young black woman, as a signifier of their progressiveness.

It's also a setup made for a rom-com: Obviously, Emira ends up dating
Kelley, the handsome white guy who caught the episode on his iPhone.
Less obviously, in one of the many lapses in credibility that beleaguer
Reid's plot, it turns out that Kelley has an unresolved history with
Briar's mom, Alix, a high-profile social media entrepreneur and active
Hillary supporter. Back in high school, he dumped her, she never forgave
him, and the memory of what happened (a lame saga of virginity lost, a
house party ruined) convinces each one of the other's exploitative
attitude toward black people.

Image

Credit...

Emira doesn't discover this connection until midway through Reid's novel
--- and from there the story, told from the oscillating, third-person
perspectives of mother and sitter, takes shape as an interracial love
triangle whose convoluted dynamic lets some of the steam out of its
worthy message. The older, whiter characters' liberal anxieties play out
as a tug-of-war for Emira's affections --- or, as Alix calls it, ``a
losing game called `Which One of Us Is Actually More Racist?''' Alix
grew up garishly rich and now takes pride in things like inviting five
whole black people to her catered Thanksgiving dinner table. She shows
less interest in her daughter than in the ``person she paid to love
her,'' voyeuristically reading Emira's texts and attempting to enfold
the younger, prettier, poorer girl --- the first in her family to go to
college --- in the web of her influence (as flimsy to us as it is to
Emira).

\emph{{[} Read}
\href{https://www.nytimes.com/2019/12/31/books/review/such-a-fun-age-by-kiley-reid-an-excerpt.html}{\emph{an
excerpt from ``Such a Fun Age.''}} \emph{{]}}

Kelley's affection for Emira appears more genuine, but his wokeness has
a whiff of performance. ``Like \ldots{} I get it,'' Emira says to him,
``you have a weirdly large amount of black friends, you saw Kendrick
Lamar in concert and now you have a black girlfriend \ldots{} great.''
He tries to get Emira to post the supermarket video and to quit working
for Alix, who he thinks is trying to use Emira's blackness for personal
gain. But isn't he, too, in a way? In one of the most powerful lines in
the book, Emira articulates to Kelley what a woman of color needs from
her white partner: ``Lemme try to say this. You get real fired up when
we talk about that night at Market Depot. But I don't need you to be mad
that it happened. I need you to be mad that it just like \ldots{}
happens.''

Reid writes scenes and dialogue with a contemporary lilt that feels
deliberately styled for a screen adaptation, inflected throughout with
cringe-inducing ``holup holup''s and ``ohmygod''s, heavy-handed attempts
to mimic millennial parlance. Over all, the characters' melodrama is
unwarranted; the final climactic event that Alix thinks ``felt like the
plot twist of a horror movie'' is actually quite predictable. But the
simple prose and story line belie a more nuanced moral hierarchy: Emira
is clearly the victim of racially motivated manipulation, but the two
white people who profess to care for her shift uncomfortably between the
poles of villain and hero. Both boss and boyfriend engage in distinct
brands of white posturing, defining themselves in part by their
relationships to this young woman --- an adoring, vocationally lost
black woman who must decide whether the benefits of those relationships
are outweighed by the cost to her sense of self. Out of Reid's often
cloying vernacular, then, emerge some surprisingly resonant insights
into the casual racism in everyday life, especially in the America of
the liberal elite.

Advertisement

\protect\hyperlink{after-bottom}{Continue reading the main story}

\hypertarget{site-index}{%
\subsection{Site Index}\label{site-index}}

\hypertarget{site-information-navigation}{%
\subsection{Site Information
Navigation}\label{site-information-navigation}}

\begin{itemize}
\tightlist
\item
  \href{https://help.nytimes.com/hc/en-us/articles/115014792127-Copyright-notice}{©~2020~The
  New York Times Company}
\end{itemize}

\begin{itemize}
\tightlist
\item
  \href{https://www.nytco.com/}{NYTCo}
\item
  \href{https://help.nytimes.com/hc/en-us/articles/115015385887-Contact-Us}{Contact
  Us}
\item
  \href{https://www.nytco.com/careers/}{Work with us}
\item
  \href{https://nytmediakit.com/}{Advertise}
\item
  \href{http://www.tbrandstudio.com/}{T Brand Studio}
\item
  \href{https://www.nytimes.com/privacy/cookie-policy\#how-do-i-manage-trackers}{Your
  Ad Choices}
\item
  \href{https://www.nytimes.com/privacy}{Privacy}
\item
  \href{https://help.nytimes.com/hc/en-us/articles/115014893428-Terms-of-service}{Terms
  of Service}
\item
  \href{https://help.nytimes.com/hc/en-us/articles/115014893968-Terms-of-sale}{Terms
  of Sale}
\item
  \href{https://spiderbites.nytimes.com}{Site Map}
\item
  \href{https://help.nytimes.com/hc/en-us}{Help}
\item
  \href{https://www.nytimes.com/subscription?campaignId=37WXW}{Subscriptions}
\end{itemize}
