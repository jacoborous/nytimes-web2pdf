Sections

SEARCH

\protect\hyperlink{site-content}{Skip to
content}\protect\hyperlink{site-index}{Skip to site index}

\href{https://www.nytimes.com/section/us}{U.S.}

\href{https://myaccount.nytimes.com/auth/login?response_type=cookie\&client_id=vi}{}

\href{https://www.nytimes.com/section/todayspaper}{Today's Paper}

\href{/section/us}{U.S.}\textbar{}Stolen Research: Chinese Scientist Is
Accused of Smuggling Lab Samples

\url{https://nyti.ms/2tng5rK}

\begin{itemize}
\item
\item
\item
\item
\item
\end{itemize}

Advertisement

\protect\hyperlink{after-top}{Continue reading the main story}

Supported by

\protect\hyperlink{after-sponsor}{Continue reading the main story}

\hypertarget{stolen-research-chinese-scientist-is-accused-of-smuggling-lab-samples}{%
\section{Stolen Research: Chinese Scientist Is Accused of Smuggling Lab
Samples}\label{stolen-research-chinese-scientist-is-accused-of-smuggling-lab-samples}}

Zaosong Zheng, a promising cancer researcher, confessed that he had
planned to take the stolen samples to Sun Yat-sen Memorial Hospital, and
publish the results under his own name.

\includegraphics{https://static01.nyt.com/images/2019/12/31/us/31boston-cancer/merlin_88339193_31089367-af81-48c3-b1b2-7b822f84dc5e-articleLarge.jpg?quality=75\&auto=webp\&disable=upscale}

\href{https://www.nytimes.com/by/ellen-barry}{\includegraphics{https://static01.nyt.com/images/2018/10/08/multimedia/author-ellen-barry/author-ellen-barry-thumbLarge.png}}

By \href{https://www.nytimes.com/by/ellen-barry}{Ellen Barry}

\begin{itemize}
\item
  Dec. 31, 2019
\item
  \begin{itemize}
  \item
  \item
  \item
  \item
  \item
  \end{itemize}
\end{itemize}

\href{https://cn.nytimes.com/usa/20200102/chinese-scientist-cancer-research-investigation/}{阅读简体中文版}\href{https://cn.nytimes.com/usa/20200102/chinese-scientist-cancer-research-investigation/zh-hant/}{閱讀繁體中文版}

BOSTON --- Zaosong Zheng was preparing to board Hainan Airlines Flight
482, nonstop from Boston to Beijing, when customs officers pulled him
aside.

Inside his checked luggage, wrapped in a plastic bag and then inserted
into a sock, the officers found what they were looking for: 21 vials of
brown liquid --- cancer cells --- that the authorities say Mr. Zheng,
29, a
\href{https://www.researchgate.net/scientific-contributions/2059782688_Zaosong_Zheng}{cancer
researcher}, took from a laboratory at Beth Israel Deaconess Medical
Center.

Under questioning, court documents say, Mr. Zheng acknowledged that he
had stolen eight of the samples and had replicated 11 more based on a
colleague's research. When he returned to China, he said, he would take
the samples to Sun Yat-sen Memorial Hospital and turbocharge his career
by publishing the results in China, under his own name.

Mr. Zheng's arrest on Dec. 10 signified an escalation in the F.B.I.'s
\href{https://www.nytimes.com/2019/11/04/health/china-nih-scientists.html}{efforts
to root out scientists who, the authorities say, are stealing research
from American laboratories.} Federal prosecutors warn that he may be
charged with transporting stolen goods or with the theft of trade
secrets, a felony that brings a prison term of up to 10 years.

At a hearing on Monday, Magistrate Judge David Hennessy granted
prosecutors' wish to hold Mr. Zheng without bail, noting that the theft
appeared to have the support of the Chinese government. Two other
Chinese scientists who worked in the same lab as Mr. Zheng had
successfully smuggled stolen biological material out of the country,
prosecutors say.

Mr. Zheng's case is the first to unfold in the laboratories clustered
around Harvard University, but it is not likely to be the last. Federal
officials are
\href{https://www.nytimes.com/2019/11/04/health/china-nih-scientists.html}{investigating
hundreds of cases} involving the potential theft of intellectual
property by visiting scientists, nearly all of them Chinese nationals.

Christopher Wray, director of the F.B.I., described the researchers as
``nontraditional collectors'' of intelligence acting at the behest of
the Chinese government, part of a collective effort to ``steal their way
up the economic ladder at our expense.''

Dr. Ross McKinney Jr., chief scientific officer of the Association of
American Medical Colleges, said the actions Mr. Zheng was accused of
were especially bold.

``This is one of the few cases where there's been stealing of physical
material as well as the stealing of ideas,'' he said. ``It's an
escalation over most of what we've been seeing.''

Researchers of Chinese descent make up nearly half of the work force in
American research laboratories, in part because American-born scientists
are drawn to the private sector and less interested in academic careers,
Dr. McKinney said. Among the 6,000 Chinese scientists who have received
grants from the National Institutes of Health, around 180 are under
investigation for possible violation of intellectual property law, he
said.

Harvard University had sponsored Mr. Zheng's visa starting on Sept. 4,
2018, according to Jason A. Newton, a spokesman for the university. The
visa support ended when Mr. Zheng lost his job at Beth Israel Deaconess
Medical Center, he said.

The hospital said in a statement that it was cooperating with the
investigation. ``Any efforts to compromise research undermine the hard
work of our faculty and staff to advance patient care,'' said Jennifer
Kritz, the hospital's director of communication.

A message left for Brendan O. Kelley, Mr. Zheng's lawyer, was not
returned.

Court records sketch out a cat-and-mouse game between Mr. Zheng and Kara
Spice, the F.B.I. special agent assigned to the case. Customs and Border
Protection agents had been warned that he was ``a high risk for possibly
exporting biological undeclared biological material,'' and inspected his
luggage in the airline's bag room.

At first, Mr. Zheng deflected their interest in the 21 vials, telling
the agents that they ``were not important and had nothing to do with his
research.'' Then he offered another explanation, saying that they had
been given to him by a friend and that he had no plans to do anything
with them.

``Zheng could not explain why he was attempting to leave the United
States with the vials concealed in a sock in his checked bag,'' Ms.
Spice's statement says. Shortly thereafter, he confessed to stealing the
material.

Mr. Zheng booked another flight to China the following day, but was
detained by F.B.I. agents before he could board it, court documents say.
Through a Mandarin interpreter, he waived his Miranda rights and told
the agents he intended to use the samples for cancer research. At that
point, he was arrested.

Agents learned more when they visited Mr. Zheng's apartment, according
to court documents. His former roommate, a fellow medical researcher
named Jialin Li, told them that Mr. Zheng had packed all his possessions
in preparation for his Dec. 9 flight, suggesting that he did not intend
to return to the United States.

Mr. Li also told them that two other Chinese researchers, Lei Liu and
Leina Mo, who had worked in the same laboratory at Beth Israel Deaconess
Medical Center, had managed to smuggle biological material into China
without getting caught, according to court documents.

Mr. Zheng's theft ``was not an isolated incident,'' prosecutors stated
in the motion to hold him without bail. ``Rather, it appears to have
been a coordinated crime, with likely involvement by the Chinese
government, as two other Chinese nationals working in the same lab have
also stolen biological materials and smuggled them out of the United
States.''

Advertisement

\protect\hyperlink{after-bottom}{Continue reading the main story}

\hypertarget{site-index}{%
\subsection{Site Index}\label{site-index}}

\hypertarget{site-information-navigation}{%
\subsection{Site Information
Navigation}\label{site-information-navigation}}

\begin{itemize}
\tightlist
\item
  \href{https://help.nytimes.com/hc/en-us/articles/115014792127-Copyright-notice}{©~2020~The
  New York Times Company}
\end{itemize}

\begin{itemize}
\tightlist
\item
  \href{https://www.nytco.com/}{NYTCo}
\item
  \href{https://help.nytimes.com/hc/en-us/articles/115015385887-Contact-Us}{Contact
  Us}
\item
  \href{https://www.nytco.com/careers/}{Work with us}
\item
  \href{https://nytmediakit.com/}{Advertise}
\item
  \href{http://www.tbrandstudio.com/}{T Brand Studio}
\item
  \href{https://www.nytimes.com/privacy/cookie-policy\#how-do-i-manage-trackers}{Your
  Ad Choices}
\item
  \href{https://www.nytimes.com/privacy}{Privacy}
\item
  \href{https://help.nytimes.com/hc/en-us/articles/115014893428-Terms-of-service}{Terms
  of Service}
\item
  \href{https://help.nytimes.com/hc/en-us/articles/115014893968-Terms-of-sale}{Terms
  of Sale}
\item
  \href{https://spiderbites.nytimes.com}{Site Map}
\item
  \href{https://help.nytimes.com/hc/en-us}{Help}
\item
  \href{https://www.nytimes.com/subscription?campaignId=37WXW}{Subscriptions}
\end{itemize}
