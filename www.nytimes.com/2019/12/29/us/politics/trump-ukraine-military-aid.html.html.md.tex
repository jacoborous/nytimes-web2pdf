Sections

SEARCH

\protect\hyperlink{site-content}{Skip to
content}\protect\hyperlink{site-index}{Skip to site index}

\href{https://www.nytimes.com/section/politics}{Politics}

\href{https://myaccount.nytimes.com/auth/login?response_type=cookie\&client_id=vi}{}

\href{https://www.nytimes.com/section/todayspaper}{Today's Paper}

\href{/section/politics}{Politics}\textbar{}Behind the Ukraine Aid
Freeze: 84 Days of Conflict and Confusion

\url{https://nyti.ms/39qrLdQ}

\begin{itemize}
\item
\item
\item
\item
\item
\item
\end{itemize}

\begin{itemize}
\item
  \href{https://www.nytimes.com/2020/07/31/us/elections/biden-vs-trump.html?action=click\&pgtype=Article\&state=default\&region=TOP_BANNER\&context=storylines_menu}{Election
  Updates}
\item
  \href{https://www.nytimes.com/article/biden-vice-president-2020.html?action=click\&pgtype=Article\&state=default\&region=TOP_BANNER\&context=storylines_menu}{Biden's
  V.P. Search}
\item
  \href{https://www.nytimes.com/interactive/2020/07/24/us/politics/trump-biden-campaign-donors.html?action=click\&pgtype=Article\&state=default\&region=TOP_BANNER\&context=storylines_menu}{Map
  of Donations}
\item
  \href{https://www.nytimes.com/interactive/2020/us/elections/delegate-count-primary-results.html?action=click\&pgtype=Article\&state=default\&region=TOP_BANNER\&context=storylines_menu}{Delegate
  Count}
\item
  \href{https://www.nytimes.com/interactive/2019/us/politics/2020-presidential-candidates.html?action=click\&pgtype=Article\&state=default\&region=TOP_BANNER\&context=storylines_menu}{The
  Candidates}
\item
  \href{https://www.nytimes.com/newsletters/politics?action=click\&pgtype=Article\&state=default\&region=TOP_BANNER\&context=storylines_menu}{Politics
  Newsletter}
\end{itemize}

Advertisement

\protect\hyperlink{after-top}{Continue reading the main story}

Supported by

\protect\hyperlink{after-sponsor}{Continue reading the main story}

\hypertarget{behind-the-ukraine-aid-freeze-84-days-of-conflict-and-confusion}{%
\section{Behind the Ukraine Aid Freeze: 84 Days of Conflict and
Confusion}\label{behind-the-ukraine-aid-freeze-84-days-of-conflict-and-confusion}}

The inside story of President Trump's demand to halt military assistance
to an ally shows the price he was willing to pay to carry out his
agenda.

\includegraphics{https://static01.nyt.com/images/2019/12/29/us/politics/28dc-omb1-sub/merlin_161369100_78bab303-39b7-495b-8cd4-48876d4d21e7-articleLarge.jpg?quality=75\&auto=webp\&disable=upscale}

\href{https://www.nytimes.com/by/eric-lipton}{\includegraphics{https://static01.nyt.com/images/2018/12/06/multimedia/author-eric-lipton/author-eric-lipton-thumbLarge.png}}\href{https://www.nytimes.com/by/maggie-haberman}{\includegraphics{https://static01.nyt.com/images/2018/07/12/multimedia/author-maggie-haberman/author-maggie-haberman-thumbLarge.png}}\href{https://www.nytimes.com/by/mark-mazzetti}{\includegraphics{https://static01.nyt.com/images/2018/07/12/multimedia/author-Mark-Mazzetti/author-Mark-Mazzetti-thumbLarge-v4.png}}

By \href{https://www.nytimes.com/by/eric-lipton}{Eric Lipton},
\href{https://www.nytimes.com/by/maggie-haberman}{Maggie Haberman} and
\href{https://www.nytimes.com/by/mark-mazzetti}{Mark Mazzetti}

\begin{itemize}
\item
  Published Dec. 29, 2019Updated Jan. 16, 2020
\item
  \begin{itemize}
  \item
  \item
  \item
  \item
  \item
  \item
  \end{itemize}
\end{itemize}

WASHINGTON --- Deep into a long
\href{https://www.nytimes.com/2019/06/27/world/asia/trump-g20.html}{flight
to Japan} aboard Air Force One with President Trump, Mick Mulvaney, the
acting White House chief of staff, dashed off an email to an aide back
in Washington.

``I'm just trying to tie up some loose ends,'' Mr. Mulvaney wrote. ``Did
we ever find out about the money for Ukraine and whether we can hold it
back?''

It was June 27, more than a week after Mr. Trump had first asked about
putting a hold on security aid to Ukraine, an embattled American ally,
and Mr. Mulvaney needed an answer.

The aide,
\href{https://projects.propublica.org/trump-town/staffers/robert-b-blair-white-house-office}{Robert
B. Blair,} replied that it would be possible, but not pretty. ``Expect
Congress to become unhinged'' if the White House tried to countermand
spending passed by the House and Senate, he wrote in a previously
undisclosed email. And, he wrote, it might further fuel the narrative
that Mr. Trump was pro-Russia.

Mr. Blair was right, even if his prediction of a messy outcome was
wildly understated. Mr. Trump's order to hold \$391 million worth of
sniper rifles, rocket-propelled grenades, night vision goggles, medical
aid and other equipment the Ukrainian military needed to fight a
grinding war against Russian-backed separatists would help pave a path
to the
\href{https://www.nytimes.com/2019/12/18/us/politics/trump-impeached.html?action=click\&pgtype=Article\&state=default\&module=STYLN_trump_playbook\&variant=1_trump_playbook\&region=header\&context=menu}{president's
impeachment}.

The Democratic-led inquiry into Mr. Trump's dealings with Ukraine this
spring and summer established that the president was actively involved
in parallel efforts --- both secretive and highly unusual --- to bring
pressure on a country he viewed with suspicion, if not disdain.

One campaign, spearheaded by Rudolph W. Giuliani, the president's
personal lawyer, aimed to force Ukraine to conduct investigations that
could help Mr. Trump politically, including one focused on a potential
Democratic 2020 rival, former Vice President Joseph R. Biden Jr.

The other, which unfolded nearly simultaneously but has gotten less
attention, was the president's demand to withhold the security
assistance. By late summer, the two efforts merged as American diplomats
used the withheld aid as leverage in the effort to win a public
commitment from the new Ukrainian president, Volodymyr Zelensky, to
carry out the investigations Mr. Trump sought into Mr. Biden and
unfounded or overblown theories about Ukraine interfering in the 2016
election.

\includegraphics{https://static01.nyt.com/images/2019/12/29/us/politics/28dc-omb3-sub/merlin_165104610_5e13285c-2ef9-4061-9a33-a5d14ebd9f3b-articleLarge.jpg?quality=75\&auto=webp\&disable=upscale}

Interviews with dozens of current and former administration officials,
congressional aides and others, previously undisclosed emails and
documents, and a close reading of thousands of pages of impeachment
testimony provide the most complete account yet of the 84 days from when
Mr. Trump first inquired about the money to his decision in September to
relent.

What emerges is the story of how Mr. Trump's demands sent shock waves
through the White House and the Pentagon, created deep rifts within the
senior ranks of his administration, left key aides like Mr. Mulvaney
under intensifying scrutiny --- and ended only after Mr. Trump learned
of a damning whistle-blower report and came under pressure from
influential Republican lawmakers.

In many ways, the havoc Mr. Giuliani and other Trump loyalists set off
in the State Department by pursuing the investigations was matched by
conflicts and confusion in the White House and Pentagon stemming from
Mr. Trump's order to withhold the aid.

Opposition to the order from his top national security advisers was more
intense than previously known. In late August, Defense Secretary Mark T.
Esper joined Secretary of State Mike Pompeo and
\href{https://www.nytimes.com/2020/01/06/us/politics/bolton-testify-impeachment-trial.html}{John
R. Bolton}, the national security adviser at the time, for a previously
undisclosed Oval Office meeting with the president where they tried but
failed to convince him that releasing the aid was in interests of the
United States.

By late summer, top lawyers at the Office of Management and Budget who
had spoken to lawyers at the White House and the Justice Department in
the weeks beforehand, were developing an argument --- not previously
divulged publicly --- that Mr. Trump's role as commander in chief would
simply allow him to override Congress on the issue.

And Mr. Mulvaney is shown to have been deeply involved as a key conduit
for transmitting Mr. Trump's demands for the freeze across the
administration.

The interviews and documents show how Mr. Trump used the bureaucracy to
advance his agenda in the face of questions about its propriety and even
legality from officials in the White House budget office and the
Pentagon, many of whom say they were kept in the dark about the
president's motivations and had grown used to convention-flouting
requests from the West Wing. One veteran budget official who raised
questions about the legal justification was pushed aside.

Those carrying out Mr. Trump's orders on the aid were for the most part
operating in different lanes from those seeking the investigations,
including Mr. Giuliani and a number of senior diplomats, including
\href{https://www.nytimes.com/2019/11/20/us/politics/sondland-statement.html}{Gordon
D. Sondland}, the ambassador to the European Union, and
\href{https://www.nytimes.com/2019/11/19/us/politics/kurt-volker.html}{Kurt
D. Volker}, the State Department's special envoy for Ukraine and Russia.

Image

Those carrying out Mr. Trump's orders on the military aid were for the
most part operating in different lanes from those pressuring Ukraine on
investigations, like Gordon D. Sondland, the ambassador to the European
Union.Credit...Erin Schaff/The New York Times

The New York Times found that some key players are now offering a
defense that they did not know the diplomatic push for the
investigations was playing out at the same time they were implementing
the aid freeze --- or if they were aware of both channels, they did not
connect the two.

Mr. Mulvaney is said by associates to have stepped out of the room
whenever Mr. Trump would talk with Mr. Giuliani to preserve Mr. Trump's
attorney-client privilege, leaving him with limited knowledge about
their efforts regarding Ukraine. Mr. Mulvaney has told associates he
learned of the substance of Mr. Trump's July 25 call weeks after the
fact.

Yet testimony before the House suggests a different picture.
\href{https://www.nytimes.com/2019/11/21/us/politics/who-is-fiona-hill.html}{Fiona
Hill}, a top deputy to Mr. Bolton at the time, told the impeachment
inquiry about a July 10 White House meeting at which Mr. Sondland said
Mr. Mulvaney
\href{https://www.documentcloud.org/documents/6593529-2019-10-14-Fiona-Hill-Deposition.html\#document/p129/a541489}{had
guaranteed} that Mr. Zelensky would be invited to the White House if the
Ukrainians agreed to the investigations --- an arrangement that
\href{https://www.nytimes.com/2019/11/11/us/ukraine-trump.html}{Mr.
Bolton described as a ``drug deal,''} according to Ms. Hill.

Along with Mr. Bolton and others, Mr. Mulvaney and Mr. Blair have
declined to cooperate with impeachment investigators and provide
information to Congress under oath, an intensifying point of friction
between the two parties as the Senate prepares for Mr. Trump's
impeachment trial.

Image

Robert B Blair, left, coordinated with Mr. Mulvaney and the Office of
Management and Budget on freezing the aid.Credit...Jonathan
Ernst/Reuters

At the center of the maelstrom was the Office of Management and Budget,
a seldom-scrutinized arm of the White House that during the Trump
administration has often had to find creative legal reasoning to justify
the president's unorthodox policy proposals, like his demand to divert
Pentagon
\href{https://www.nytimes.com/2019/09/03/us/politics/pentagon-border-wall.html}{funding
to his proposed wall} along the border with Mexico.

In the Ukraine case, however, shock about the president's decision
spread across America's national security apparatus --- from the
National Security Council to the State Department and the Pentagon. By
September, after the freeze had become public and scrutiny was
increasing, the blame game inside the administration was in full swing.

\hypertarget{latest-updates-2020-election}{%
\section{\texorpdfstring{\href{https://www.nytimes.com/2020/07/31/us/elections/biden-vs-trump.html?action=click\&pgtype=Article\&state=default\&region=MAIN_CONTENT_1\&context=storylines_live_updates}{Latest
Updates: 2020
Election}}{Latest Updates: 2020 Election}}\label{latest-updates-2020-election}}

Updated 2020-08-01T01:26:45.732Z

\begin{itemize}
\tightlist
\item
  \href{https://www.nytimes.com/2020/07/31/us/elections/biden-vs-trump.html?action=click\&pgtype=Article\&state=default\&region=MAIN_CONTENT_1\&context=storylines_live_updates\#link-29fdff45}{Kamala
  Harris, a top vice-presidential contender, confronts double
  standards.}
\item
  \href{https://www.nytimes.com/2020/07/31/us/elections/biden-vs-trump.html?action=click\&pgtype=Article\&state=default\&region=MAIN_CONTENT_1\&context=storylines_live_updates\#link-13ec3d9c}{Karen
  Bass and Susan Rice are rising on Biden's vice-presidential
  shortlist.}
\item
  \href{https://www.nytimes.com/2020/07/31/us/elections/biden-vs-trump.html?action=click\&pgtype=Article\&state=default\&region=MAIN_CONTENT_1\&context=storylines_live_updates\#link-49e9a016}{Trump
  says Russian bounties to kill U.S. troops `never took place.'}
\end{itemize}

\href{https://www.nytimes.com/2020/07/31/us/elections/biden-vs-trump.html?action=click\&pgtype=Article\&state=default\&region=MAIN_CONTENT_1\&context=storylines_live_updates}{See
more updates}

On Sept. 10, the day before Mr. Trump changed his mind, a political
appointee at the budget office, Michael P. Duffey, wrote a lengthy email
to the Pentagon's top budget official, with whom he had been at odds
throughout the summer about how long the agency could withhold the aid.

He asserted that the Defense Department had the authority to do more to
ensure that the aid could be released to Ukraine by the congressionally
mandated deadline of the end of that month, suggesting that
responsibility for any failure should not rest with the White House.

Forty-three minutes later, the Pentagon official, Elaine McCusker, hit
send on a brief but stinging reply.

``You can't be serious,'' she wrote. ``I am speechless.''

\hypertarget{we-need-to-hold-it-up}{%
\subsection{`We Need to Hold It Up'}\label{we-need-to-hold-it-up}}

For top officials inside the budget office, the first warning came on
June 19.

Informed that the president had a problem with the aid, Mr. Blair called
Russell T. Vought, the acting head of the Office of Management and
Budget. ``We need to hold it up,'' he said, according to officials
briefed about the conversation.

Typical of the Trump White House, the inquiry was not born of a rigorous
policy process. Aides speculated that someone had shown Mr. Trump a news
article about the Ukraine assistance and he demanded to know more.

Mr. Vought and his team took to Google, and came upon a piece in the
conservative Washington Examiner
\href{https://www.washingtonexaminer.com/policy/defense-national-security/pentagon-to-send-250m-in-weapons-to-ukraine}{saying}
that the Pentagon would pay for weapons and other military equipment for
Ukraine, bringing American security aid to the country to \$1.5 billion
since 2014.

The money, the article noted, was coming at a critical moment: Mr.
Zelensky, a onetime comedian, had called ending the armed conflict with
Russia in eastern Ukraine his top priority --- a move that would likely
only happen if he could negotiate from a position of strength.

Image

Russell T. Vought, the acting head of the Office of Management and
Budget, which had grown used to dealing with unconventional policy
requests from Mr. Trump.Credit...Doug Mills/The New York Times

The budget office officials had little idea of why Mr. Trump was
interested in the topic, but many of the president's more senior aides
were well aware of his feelings about Ukraine. Weeks earlier, in an Oval
Office meeting on May 23, with Mr. Sondland, Mr. Mulvaney and Mr. Blair
in attendance, Mr. Trump batted away assurances that Mr. Zelensky was
committed to confronting corruption.

``They are all corrupt, they are all terrible people,'' Mr. Trump said,
\href{https://www.nytimes.com/2019/11/19/us/politics/volker-statement-testimony.html}{according
to testimony} in the impeachment inquiry.

The United States had been planning to provide \$391 million in military
assistance to Ukraine in two chunks: \$250 million allocated by the
Pentagon for war-fighting equipment
---\href{https://www.defense.gov/Newsroom/Releases/Release/Article/1879340/dod-announces-250m-to-ukraine/}{from
sniper rifles} to rocket-propelled grenade launchers --- and \$141
million
\href{https://www.documentcloud.org/documents/6592555-Congressional-Notification-2019-FMF-Ukraine-115.html}{controlled
by the State Department} to buy night-vision devices, radar systems and
yet more rocket-grenade launchers.

With the money having been appropriated by Congress, it would be hard
for the administration to keep it from being spent by the end of the
fiscal year on Sept. 30.

The task of dealing with the president's demands fell primarily to a
group of political appointees in the West Wing and the budget office,
most with personal and professional ties to Mr. Mulvaney. There was no
public announcement that Mr. Trump wanted the assistance withheld.
Neither Congress nor the Ukrainian government was formally notified.

Mr. Mulvaney had first served in the administration as the budget
director, after three terms in the House, where he earned a reputation
as a firebrand conservative.

The four top political appointees helping Mr. Mulvaney execute the hold
--- Mr. Vought, Mr. Blair, Mr. Duffey and Mark Paoletta, the budget
office's top lawyer --- all had extensive experience in either
congressional budget politics or Republican and conservative causes.

Their efforts would cause tension and at times conflict between
officials at the budget office and the Pentagon, some of whom watched
with growing alarm.

\hypertarget{a-question-of-legality}{%
\subsection{A Question of Legality}\label{a-question-of-legality}}

The single largest chunk of the federal government's annual
discretionary budget, some \$800 billion a year, goes to the Pentagon,
spy agencies and the Department of Veterans Affairs. The career official
in charge of managing the flow of all that money for the budget office
is an Afghanistan war veteran named Mark Sandy.

After learning about the president's June 19 request, Mr. Sandy
contacted the Pentagon to learn more about the aid package. He also
repeatedly pressed Mr. Duffey about why Mr. Trump had imposed the hold
in the first place.

Image

Mark Sandy, who has spent 12 years at the budget office during two
stints, is the top career official overseeing military
spending.Credit...Pete Marovich for The New York Times

``He didn't provide an explicit response on the reason,''
\href{https://www.documentcloud.org/documents/6592845-2019-11-Mark-Sandy-Final-Redacted.html\#document/p142/a541445}{Mr.
Sandy testified} in the impeachment inquiry. ``He simply said we need to
let the hold take place --- and I'm paraphrasing here --- and then
revisit this issue with the president.''

From the start, budget office officials took the position that the money
did not have to go out the door until the end of September, giving them
time to address the president's questions.

It was easy enough for the White House to hold up the State Department
portion of the funding. Since the State Department had not yet notified
Congress of its plans to release the money, all it took was making sure
that the notification did not happen.

Freezing the Pentagon's \$250 million portion was more difficult, since
the Pentagon
\href{https://www.documentcloud.org/documents/6593035-2019-05-23-Rood-DoD-Notification-on-USAI.html}{had
already certified} that Ukraine had met requirements set by Congress to
show that it was addressing its endemic corruption and notified
lawmakers of its intent to spend the money.

So on July 19,
\href{https://www.documentcloud.org/documents/6592845-2019-11-Mark-Sandy-Final-Redacted.html\#document/p33/a541448}{Mr.
Duffey proposed} an unusual solution: Mr. Sandy should attach a footnote
to a routine budget document saying the money was being temporarily
withheld.

Approving such requests is routine; Mr. Sandy processed
\href{https://www.documentcloud.org/documents/6592845-2019-11-Mark-Sandy-Final-Redacted.html\#document/p19/a541447}{hundreds}
each year. But attaching a footnote to block spending that the
administration had already notified Congress was ready to go was not.
Mr. Sandy
\href{https://www.documentcloud.org/documents/6592845-2019-11-Mark-Sandy-Final-Redacted.html\#document/p87/a541442}{said
in testimony} that he had never done it before in his 12 years at the
agency.

And there was a problem with this maneuver:
\href{https://www.documentcloud.org/documents/6592845-2019-11-Mark-Sandy-Final-Redacted.html\#document/p85/a541441}{Mr.
Sandy was concerned} it might
\href{https://www.documentcloud.org/documents/6592845-2019-11-Mark-Sandy-Final-Redacted.html\#document/p131/a541446}{violate}
a law called the
\href{https://history.house.gov/Historical-Highlights/1951-2000/Congressional-Budget-and-Impoundment-Control-Act-of-1974/}{Impoundment
Control Act} that protects Congress's spending power and
\href{https://www.gao.gov/products/B-330330}{prohibits} the
administration from blocking disbursement of the aid unless it notifies
Congress.

``I asked about the duration of the hold and was told there was not
clear guidance on that,''
\href{https://www.documentcloud.org/documents/6592845-2019-11-Mark-Sandy-Final-Redacted.html\#document/p35/a541518}{Mr.
Sandy testified}. ``So that is what prompted my concern.''

Mr. Sandy sought advice from the top lawyers at the budget office.

\hypertarget{a-pivotal-day}{%
\subsection{A Pivotal Day}\label{a-pivotal-day}}

For a full month, the fact that Mr. Trump wanted to halt the aid
remained confined primarily to a small group of officials.

That ended on July 18, when a group of top administration officials
meeting on Ukraine policy --- including some
\href{https://www.documentcloud.org/documents/6593140-2019-10-22-William-Taylor-Testimony.html\#document/p27/a541490}{calling
in from Kyiv} --- learned from a midlevel budget office official that
the president had ordered the aid frozen.

``I and the others on the call sat in astonishment,'' William B. Taylor
Jr., the top United States diplomat in Ukraine,
\href{https://www.documentcloud.org/documents/6593140-2019-10-22-William-Taylor-Testimony.html\#document/p27/a541464}{testified
to} House investigators. ``In an instant, I realized that one of the key
pillars of our strong support for Ukraine was threatened.''

That same day, aides on the House Foreign Affairs Committee received
four calls from administration sources warning them about the hold and
urging them to look into it.

A week later came Mr. Trump's
\href{https://www.nytimes.com/2019/11/19/us/politics/trump-impeachment-hearing-testimony.html}{fateful
July 25 call}with Mr. Zelensky. Mr. Bolton, the national security
adviser, had recommended the call take place in an effort to end the
``incessant lobbying'' from officials like Mr. Sondland that the two
leaders connect.

Image

Mr. Trump pressured Mr. Zelensky to back investigations that would
benefit him politically.Credit...Anna Moneymaker/The New York Times

Some of Mr. Trump's aides had thought the call might lead Mr. Trump to
lift the freeze. But Mr. Trump did not specifically mention the hold,
and instead asked Mr. Zelensky to look into Mr. Biden and his son and
into supposed Ukrainian involvement in the 2016 election. Among those
listening on the call was Mr. Blair.

Mr. Blair has told associates he did not make much of Mr. Trump's
requests during the call for the investigations. He saw the aid freeze
not as a political tool, but as an extension of Mr. Trump's general
aversion to foreign aid and his belief that Ukraine is rife with
corruption.

Just 90 minutes after the call ended, and following days of email
traffic on the topic, Mr. Duffey, Mr. Sandy's boss, sent out a new email
to the Pentagon, where officials were impatient about getting the money
out the door. His message was clear: Do not spend it.

``Given the sensitive nature of the request, I appreciate your keeping
that information closely held to those who need to know to execute the
direction,'' Mr. Duffey
\href{https://www.documentcloud.org/documents/6592561-2019-07-25-Duffey-Email-Re-Hold-CPI-v-DoD-Dec-20.html}{wrote
in his note}, which was released this month to the Center for Public
Integrity.

This caused immediate discomfort at the Pentagon, with a top official
there noting that this hold on military assistance was coming on the
\href{https://www.facebook.com/SecurSerUkraine/photos/a.1539443172952349/2438495979713726/?type=3\&theater}{same
day} Ukraine
\href{https://www.nytimes.com/2019/07/25/world/europe/ukraine-russia-tanker.html}{announced
it had seized a Russian tanker} --- a potential escalation in the
conflict between the two nations.

On that same day, Mr. Sandy, having received the go-ahead from the
budget office's lawyers, took the first official step to legally impose
what they called a ``brief pause,'' inserting a footnote into the budget
document that prohibited the Pentagon from spending any of the aid until
Aug. 5.

By that point, officials in Ukraine
\href{https://www.nytimes.com/2019/12/03/world/europe/ukraine-impeachment-military-aid.html}{were
getting word} that something was up. At the same time, the effort to win
a commitment from the Ukrainians for the investigations sought by Mr.
Trump was intensifying, with Mr. Giuliani and a Zelensky aide,
\href{https://www.nytimes.com/2019/08/21/us/politics/giuliani-ukraine.html}{Andriy
Yermak, meeting in Madrid} on Aug. 2 and the diplomats Mr. Sondland and
Mr. Volker also working the issue.

And inside the intelligence community, a C.I.A. officer was hearing talk
about the two strands of pressure on Ukraine, including the aid freeze.
Seeing how they fit together, he was alarmed enough
\href{https://www.nytimes.com/interactive/2019/09/26/us/politics/whistle-blower-complaint.html}{that
by Aug. 12} he would take the extraordinary step of laying them out in
detail in a confidential whistle-blower complaint.

\hypertarget{a-potus-level-decision}{%
\subsection{A `POTUS-level Decision'}\label{a-potus-level-decision}}

Keeping a hold on the assistance was now a top priority, so officials
moved to tighten control over the money.

In a
\href{https://www.documentcloud.org/documents/6592845-2019-11-Mark-Sandy-Final-Redacted.html\#document/p104/a541466}{very
unusual step}, the White House
\href{https://www.documentcloud.org/documents/6592845-2019-11-Mark-Sandy-Final-Redacted.html\#document/p103/a541465}{removed
Mr. Sandy's authorit}y to oversee the aid freeze. The job was handed in
late July to Mr. Sandy's boss, Mr. Duffey, the political appointee, the
official ultimately responsible for apportionments but one who had
little experience in the nuts and bolts of the budget office process.

As the debate over the aid continued, disagreements flared. Two budget
office staff members left the agency after the summer.
\href{https://www.documentcloud.org/documents/6592845-2019-11-Mark-Sandy-Final-Redacted.html\#document/p149/a541492}{Mr.
Sandy testified} that their departures were related to the aid freeze, a
statement disputed by budget office officials.

Pentagon officials, in the dark about the reason for the holdup, grew
increasingly frustrated. Ms. McCusker, the powerful Pentagon budget
official, notified the budget office that either \$61 million of the
money would have to be spent by Monday, Aug. 12 or it would be lost. The
budget office saw her threat as a ploy to force release of the aid.

At the White House, which had been looped into the dispute by the budget
office, there was a growing consensus that officials could find a legal
rationale for continuing the hold, but with the Monday deadline looming,
it was a ``POTUS-level decision,'' one official said.

Image

A Ukrainian soldier firing at Russian-backed separatists in the Donetsk
region last month. Planned military aid from the United States to
Ukraine included sniper rifles, rocket-propelled grenades, night vision
goggles and medical equipment.Credit...Anatolii Stepanov/Agence
France-Presse --- Getty Images

Complicating matters, another budget battle was escalating. Mr. Vought
was attempting to
\href{https://www.nytimes.com/2019/08/07/us/politics/foreign-aid-freeze-congress.html}{impose
cuts of as much as \$4 billion} on the nation's overall foreign aid
budget. It was an entirely separate initiative from the Ukraine freeze,
and was
\href{https://www.nytimes.com/2019/08/22/us/politics/trump-foreign-aid.html}{quickly
abandoned}, but helped the White House establish that its concern about
aid was not limited to Ukraine.

By the second week of August, Mr. Duffey had taken to issuing footnotes
\href{https://www.documentcloud.org/documents/6592845-2019-11-Mark-Sandy-Final-Redacted.html\#document/p127/a541468}{every
few days} to block the Pentagon spending. Office of Management and
Budget lawyers approved each one.

Mr. Trump \href{https://trumpgolfcount.com/displayoutings}{spent the
weekend} before the Pentagon's Aug. 12 deadline at Bedminster, his New
Jersey golf resort.

In a previously unreported sequence of events, Mr. Mulvaney worked to
schedule a call for that day with Mr. Trump and top aides involved in
the freeze, including Mr. Vought, Mr. Bolton and Pat Cipollone, the
White House counsel. But they waited to set a final time because Mr.
Trump had a golf game planned for
\href{https://twitter.com/pga_johndaly/status/1161105208317026309?lang=en}{Monday
morning with John Daly}, the flamboyant professional golfer, and they
did not know how long it would take.

Late that morning, Ms. McCusker checked in with the budget office.
``Hey, any update for us?''
\href{https://www.documentcloud.org/documents/6592640-2019-08-12-ELAINE-to-DUFFEY-ANY-UPDATE-for-US.html}{she
asked} in an email obtained by Center for Public Integrity.

Mr. Duffey was still waiting for an answer as of late that afternoon.
``Elaine --- I don't have an update,''
\href{https://www.documentcloud.org/documents/6592640-2019-08-12-ELAINE-to-DUFFEY-ANY-UPDATE-for-US.html\#document/p1/a541467}{he
wrote back}. ``I am attempting to get one.''

The planned-for conference call with the president never happened.
Budget office lawyers decided that Ms. McCusker had inaccurately raised
alarms about the Aug. 12 date to try to force their hand.

In Bedminster with Mr. Trump, Mr. Mulvaney finally reached the president
and the answer was clear: Mr. Trump wanted the freeze kept in place. In
Washington, the whistle-blower submitted his report that same day.

\hypertarget{the-national-security-team-intervenes}{%
\subsection{The National Security Team
Intervenes}\label{the-national-security-team-intervenes}}

Inside the administration, pressure was mounting on Mr. Trump to reverse
himself.

Backed by a memo saying the National Security Council, the Pentagon and
the State Department all wanted the aid released, Mr. Bolton made a
personal appeal to Mr. Trump on Aug. 16, but was rebuffed.

On Aug. 28,
\href{https://www.politico.com/story/2019/08/28/trump-ukraine-military-aid-russia-1689531}{Politico
published a story} reporting that the assistance to Ukraine had been
frozen. After more than two months, the issue, the topic of fiery
internal debate, was finally public.

Mr. Bolton's relationship with the president had been deteriorating for
months, and he would leave the White House weeks later, but on this
front he had powerful internal allies.

On a sunny, late-August day, Mr. Bolton, Mr. Esper and Mr. Pompeo
arrayed themselves around the Resolute desk in the Oval Office to
present a united front, the leaders of the president's national security
team seeking to convince him face to face that freeing up the money for
Ukraine was the right thing to do. One by one they made their case.

``This is in America's interest,'' Mr. Bolton argued, according to one
official briefed on the gathering.

Image

``This is in America's interest,'' John R. Bolton, then the national
security adviser, said in an Oval Office meeting intended to persuade
Mr. Trump to release the aid.Credit...Erin Schaff/The New York Times

``This defense relationship, we have gotten some really good benefits
from it,'' Mr. Esper added, noting that most of the money was being
spent on military equipment made in the United States.

Mr. Trump responded that he did not believe Mr. Zelensky's promises of
reform. He emphasized his view that corruption remained endemic and
repeated his position that European nations needed to do more for
European defense.

``Ukraine is a corrupt country,'' the president said. ``We are pissing
away our money.''

The aid remained blocked. On Aug. 31, Senator Ron Johnson, Republican of
Wisconsin, arranged a call with Mr. Trump. Mr. Johnson had been told
days earlier by Mr. Sondland that the aid would be unblocked only if the
Ukrainians gave Mr. Trump the investigations he wanted.

When Mr. Johnson asked Mr. Trump directly if the aid was contingent on
getting a commitment to pursue the investigations,
\href{https://www.ronjohnson.senate.gov/public/_cache/files/e0b73c19-9370-42e6-88b1-b2458eaeeecd/johnson-to-jordan-nunes.pdf}{Mr.
Johnson later said}, Mr. Trump replied, amid a string of expletives,
that there was no such demand and he would never do such a thing.

Around the same time, White House lawyers informed Mr. Trump
\href{https://www.nytimes.com/2019/11/26/us/politics/trump-whistle-blower-complaint-ukraine.html}{about
the whistle-blower's complaint} regarding his pressure campaign. It is
not clear how much detail the lawyers provided the president about the
details of the complaint, which noted the aid freeze.

Mr. Trump was
\href{https://www.militarytimes.com/news/pentagon-congress/2019/07/31/trump-to-visit-poland-for-world-war-ii-anniversary-sept-1/}{scheduled
to travel} to Poland on Sept. 1 to commemorate the 80th anniversary of
the outbreak of World War II, and had planned to get together with Mr.
Zelensky. Some administration officials hoped meeting the new Ukrainian
president in person would change Mr. Trump's mind.

But a hurricane was bearing down on the United States, and Mr. Trump
sent Vice President Mike Pence in his place. When Mr. Zelensky raised
the issue with the vice president, Mr. Pence said he should speak with
Mr. Trump.

Behind the scenes in Warsaw, Mr. Sondland, the American envoy who was
Mr. Trump's point person on getting the Ukrainians to agree to the
investigations, had a blunter message. Until the Ukrainians publicly
announced the investigations, he
\href{https://www.nytimes.com/2019/11/05/us/politics/impeachment-trump.html}{told}
Mr. Yermak, the Zelensky adviser, they should not expect to get the
military aid. (Mr. Yermak has
\href{https://time.com/5746417/ukraine-andriy-yermak-impeachment-interview/}{questioned}
Mr. Sondland's account.)

\hypertarget{an-abrupt-reversal}{%
\subsection{An Abrupt Reversal}\label{an-abrupt-reversal}}

By late summer, top lawyers at the budget office were developing a
proposed legal justification for the hold, based in part on
conversations with White House lawyers as well as the Justice
Department.

Their argument was that lifting the hold would undermine Mr. Trump's
negotiating position in his efforts to fight corruption in Ukraine.

The president, the lawyers believed, could ignore the requirements of
the Impoundment Control Act and continue to hold the aid **** by
asserting constitutional commander in chief powers that give him
authority over diplomacy. He could do so, they believed, if he
determined that, based on existing circumstances, releasing the money
would undermine military or diplomatic efforts.

But divisions within the administration continued to widen; Mr. Bolton
**** was opposed to using an argument proffered by administration
lawyers to block the funding. **** And pressure from Congress was
intensifying. Mr. Johnson and another influential Republican, Senator
Rob Portman of Ohio, were both pushing for the aid to be released.

Image

``We need to release these funds,'' Senator Rob Portman, Republican of
Ohio, told Mr. Trump during a phone call in September.Credit...Anna
Moneymaker/The New York Times

On a call with Mr. Portman on Sept. 11, Mr. Trump repeated his familiar
refrain about other nations not doing enough to support Ukraine.

``Sure, I agree with you,'' Mr. Portman responded, according to an aide
who described the exchange. ``But we should not hold that against
Ukraine. We need to release these funds.''

Democrats in the House were gearing up to limit Mr. Trump's power to
hold up the money to Ukraine, and the chairmen of three House committees
had also announced on Sept. 9 that they were
\href{https://foreignaffairs.house.gov/press-releases?ID=D365D32B-D9D1-4A68-B07E-28B95DA593B0}{opening
an investigation}.

Still, White House officials did not expect anything to change,
especially since Mr. Trump had repeatedly rejected the advice of his
national security team.

But then, just as suddenly as the hold was imposed, it was lifted. Mr.
Trump, apparently unwilling to wage a public battle, told Mr. Portman he
would let the money go.

White House aides rushed to notify their counterparts at the Pentagon
and elsewhere. The freeze had been lifted. The money could be spent. Get
it out the door, they were told.

The debate would now begin as to why the hold was lifted, with Democrats
confident they knew the answer.

``I have no doubt about why the president allowed the assistance to go
forward,'' said Representative Eliot L. Engel, Democrat of New York and
the chairman of the House Foreign Affairs Committee. ``He got caught.''

Adam Goldman, Edward Wong and Peter Baker contributed reporting.

\hypertarget{our-2020-election-guide}{%
\section{Our 2020 Election Guide}\label{our-2020-election-guide}}

Updated July 31, 2020

\begin{itemize}
\item
  \begin{center}\rule{0.5\linewidth}{\linethickness}\end{center}

  \hypertarget{the-latest}{%
  \subsection{The Latest}\label{the-latest}}

  \begin{itemize}
  \tightlist
  \item
    President Trump's assault on the Postal Service is intersecting with
    his attacks on mail-in voting.
    \href{https://www.nytimes.com/2020/07/31/us/politics/trump-usps-mail-delays.html?action=click\&pgtype=Article\&state=default\&region=BELOW_MAIN_CONTENT\&context=storylines_guide}{Voting
    rights groups say it is a recipe for disaster.}
  \end{itemize}
\item
  \begin{center}\rule{0.5\linewidth}{\linethickness}\end{center}

  \hypertarget{bidens-vp-search}{%
  \subsection{Biden's V.P. Search}\label{bidens-vp-search}}

  \begin{itemize}
  \tightlist
  \item
    \href{https://www.nytimes.com/article/biden-vice-president-2020.html?action=click\&pgtype=Article\&state=default\&region=BELOW_MAIN_CONTENT\&context=storylines_guide}{Here
    are 13 women} who have been under consideration to be Joe Biden's
    running mate, and why each might be chosen --- and might not be.
  \end{itemize}
\item
  \begin{center}\rule{0.5\linewidth}{\linethickness}\end{center}

  \hypertarget{keep-up-with-our-coverage}{%
  \subsection{Keep Up With Our
  Coverage}\label{keep-up-with-our-coverage}}

  \begin{itemize}
  \tightlist
  \item
    Get an
    \href{https://www.nytimes.com/newsletters/politics?action=click\&pgtype=Article\&state=default\&region=BELOW_MAIN_CONTENT\&context=storylines_guide}{email}
    recapping the day's news
  \end{itemize}

  \begin{itemize}
  \tightlist
  \item
    Download our mobile app on
    \href{https://apps.apple.com/us/app/nytimes/id284862083?ls=1\&mat_click_id=5c79ae7455014fd1bd66b5610c05b8f2-20191112-16948\&referrer=mat_click_id\%3D5c79ae7455014fd1bd66b5610c05b8f2-20191112-16948\%26link_click_id\%3D722930677036718082}{iOS}
    and
    \href{http://a.localytics.com/android?id=com.nytimes.android\&referrer=utm_source\%3Dother_nyt_mobile_web\%26utm_medium\%3DWeb\%2520page\%26utm_term\%3DGeneral\%2520Mobile\%2520Page\%26utm_campaign\%3DNYT\%2520Mobile\%2520General\%2520Page}{Android}
    and turn on Breaking News and Politics alerts
  \end{itemize}
\end{itemize}

Advertisement

\protect\hyperlink{after-bottom}{Continue reading the main story}

\hypertarget{site-index}{%
\subsection{Site Index}\label{site-index}}

\hypertarget{site-information-navigation}{%
\subsection{Site Information
Navigation}\label{site-information-navigation}}

\begin{itemize}
\tightlist
\item
  \href{https://help.nytimes.com/hc/en-us/articles/115014792127-Copyright-notice}{©~2020~The
  New York Times Company}
\end{itemize}

\begin{itemize}
\tightlist
\item
  \href{https://www.nytco.com/}{NYTCo}
\item
  \href{https://help.nytimes.com/hc/en-us/articles/115015385887-Contact-Us}{Contact
  Us}
\item
  \href{https://www.nytco.com/careers/}{Work with us}
\item
  \href{https://nytmediakit.com/}{Advertise}
\item
  \href{http://www.tbrandstudio.com/}{T Brand Studio}
\item
  \href{https://www.nytimes.com/privacy/cookie-policy\#how-do-i-manage-trackers}{Your
  Ad Choices}
\item
  \href{https://www.nytimes.com/privacy}{Privacy}
\item
  \href{https://help.nytimes.com/hc/en-us/articles/115014893428-Terms-of-service}{Terms
  of Service}
\item
  \href{https://help.nytimes.com/hc/en-us/articles/115014893968-Terms-of-sale}{Terms
  of Sale}
\item
  \href{https://spiderbites.nytimes.com}{Site Map}
\item
  \href{https://help.nytimes.com/hc/en-us}{Help}
\item
  \href{https://www.nytimes.com/subscription?campaignId=37WXW}{Subscriptions}
\end{itemize}
