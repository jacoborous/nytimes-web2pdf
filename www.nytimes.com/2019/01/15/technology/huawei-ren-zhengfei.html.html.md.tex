Sections

SEARCH

\protect\hyperlink{site-content}{Skip to
content}\protect\hyperlink{site-index}{Skip to site index}

\href{https://www.nytimes.com/section/technology}{Technology}

\href{https://myaccount.nytimes.com/auth/login?response_type=cookie\&client_id=vi}{}

\href{https://www.nytimes.com/section/todayspaper}{Today's Paper}

\href{/section/technology}{Technology}\textbar{}Huawei's Reclusive
Founder Rejects Spying and Praises Trump

\url{https://nyti.ms/2Hgjyz4}

\begin{itemize}
\item
\item
\item
\item
\item
\item
\end{itemize}

Advertisement

\protect\hyperlink{after-top}{Continue reading the main story}

Supported by

\protect\hyperlink{after-sponsor}{Continue reading the main story}

\hypertarget{huaweis-reclusive-founder-rejects-spying-and-praises-trump}{%
\section{Huawei's Reclusive Founder Rejects Spying and Praises
Trump}\label{huaweis-reclusive-founder-rejects-spying-and-praises-trump}}

\includegraphics{https://static01.nyt.com/images/2019/01/16/business/16ren/16ren-articleLarge.jpg?quality=75\&auto=webp\&disable=upscale}

By \href{https://www.nytimes.com/by/raymond-zhong}{Raymond Zhong}

\begin{itemize}
\item
  Jan. 15, 2019
\item
  \begin{itemize}
  \item
  \item
  \item
  \item
  \item
  \item
  \end{itemize}
\end{itemize}

\href{https://cn.nytimes.com/technology/20190116/huawei-ren-zhengfei/}{阅读简体中文版}\href{https://cn.nytimes.com/technology/20190116/huawei-ren-zhengfei/zh-hant/}{閱讀繁體中文版}

BEIJING --- To entrepreneurs in China, he is a legend akin to Steve
Jobs.

To United States officials, he is the secretive mastermind behind a
company that is extending the Chinese government's ability to infiltrate
computer systems and data networks around the world.

But for all his fame and power, Ren Zhengfei, the 74-year-old founder
and chief executive of the Chinese technology giant Huawei, may no
longer have the luxury of letting his company's success speak for
itself.

In his first public comments since United States authorities arranged
for the
\href{https://www.nytimes.com/2018/12/05/business/huawei-cfo-arrest-canada-extradition.html}{arrest
of his daughter Meng Wanzhou}, who is also Huawei's chief financial
officer, Mr. Ren told a group of reporters on Tuesday that he missed his
daughter very much, and that he would wait to see if President Trump
intervened in her case. He called Mr. Trump a ``great president,'' and
said that his tax cuts had helped American business.

\href{https://www.nytimes.com/2018/12/05/business/huawei-cfo-arrest-canada-extradition.html}{Ms.
Meng was arrested in Canada last month} on accusations of defrauding
banks to help Huawei's business in Iran. Washington is seeking her
extradition, but
\href{https://www.nytimes.com/2018/12/12/us/politics/trump-meng-wanzhou-huawei-extradition.html}{Mr.
Trump has suggested} that he might intercede if it would help China and
the United States reach a deal to end their trade war. Huawei has said
that it is unaware of any wrongdoing by Ms. Meng.

And last week, the Polish authorities said they had
\href{https://www.nytimes.com/2019/01/11/world/europe/poland-china-huawei-spy.html}{arrested
a Huawei employee} there on charges of spying for Beijing. The company
\href{https://www.nytimes.com/2019/01/12/world/asia/huawei-wang-weijing-poland.html}{fired
the man on Saturday}.

Mr. Ren insisted that his company had not spied for China.

``I love my country. I support the Communist Party. But I will never do
anything to harm any country in the world,'' Mr. Ren said on Tuesday. A
company spokesman confirmed his remarks.

Huawei has 180,000 employees and has become the world's largest maker of
telecommunications equipment. It estimates that it generated more than
\$100 billion in sales last year, and it sells more smartphones around
the world than Apple. Yet Mr. Ren seldom appears in public.

\includegraphics{https://static01.nyt.com/images/2019/01/14/business/00ren-2/00ren-2-articleLarge.jpg?quality=75\&auto=webp\&disable=upscale}

When he has spoken to the news media in the past, he has played down his
achievements, attributing Huawei's success to its employees' hard work.
He has said that his company has never spied for any government --- an
assertion that has not eased the concerns of American
counterintelligence officials.

For most of its existence, Huawei was opaque to people in China, too.

It was founded in 1987, but it did not begin publishing the names and
biographies of its board members until its 2010 annual report. Mr. Ren
spoke to the news media for the first time in 2013. The next year,
\href{https://www.independent.co.uk/news/business/analysis-and-features/huawei-founder-brushes-off-accusations-that-it-acts-as-an-arm-of-the-chinese-state-9319244.html}{he
told The Independent of London} that he had no hobbies, prompting a
colleague to lean in and suggest that he enjoyed reading and drinking
tea.

Mr. Ren was born in 1944, in the mountainous southwestern province of
Guizhou. His parents were teachers; he was one of seven children. His
father, Ren Moxun, was the son of a master ham maker in Zhejiang
Province. When he was growing up, Mr. Ren wrote in a
\href{http://hx.ahxf.gov.cn/show-53-8069-1.html}{2001 article}, the
family was so poor that he did not own a proper shirt until after high
school.

According to
\href{https://www.huawei.com/cn/about-huawei/executives/board-of-directors/ren-zhengfei}{an
official company biography}, he studied engineering in college and
joined the Chinese military's infrastructure engineering corps in 1974
to help build and run a factory manufacturing synthetic fibers for
textiles. At a time when China had no private-sector economy to speak
of, it was not unusual for college graduates to join the military.

The infrastructure engineering corps was disbanded in 1983, according to
the official biography. A few years later, Mr. Ren and business partners
founded Huawei in what he called, in a 2016 interview with the official
news agency Xinhua, a ``run-down shack.'' The company started as a
reseller of telephone equipment imported from Hong Kong, but later
started developing its own technology.

As it expanded around China and then across the world, Huawei inculcated
a die-hard competitive spirit in its employees, pushing them to work
harder and move faster than the company's rivals. Huawei still speaks
proudly of its
\href{https://www.nytimes.com/2018/12/18/technology/huawei-workers-iran-sanctions.html}{``wolf
culture.''}

``We will always have wolf culture,'' Mr. Ren said in
\href{http://www.xinhuanet.com/fortune/2018-04/05/c_1122642170.htm}{an
interview last year} with Xinhua. ``Catching prey might be difficult.
But the wolf is unrelenting.''

Mr. Ren has a reputation for being blunt in conversation. In 2010, Rick
Perry, then the governor of Texas, spoke at the ribbon-cutting for
Huawei's new American headquarters in Plano.

``If you didn't know any better, you'd say he grew up out in West
Texas,'' Mr. Perry, who is now Mr. Trump's energy secretary, said,
according to \href{https://www.youtube.com/watch?v=0eruWGDSYDg}{a video
of the event} posted online by the governor's office.

Image

Huawei has 180,000 employees and sells more smartphones around the world
than Apple.Credit...Gilles Sabrié for The New York Times

That penchant for brutal honesty has not spared the members of Mr. Ren's
family who also have worked for Huawei: Ms. Meng and her husband, plus
two of Mr. Ren's siblings.

For a long time, people in the telecom industry speculated about whether
Mr. Ren would pick one of these relatives to lead the company after his
death. But in a
\href{http://xinsheng.huawei.com/cn/index.php?app=forum\&mod=Detail\&act=index\&id=1353901}{2013
letter to employees} that was shared on a company website, Mr. Ren said
his successor needed to have vision, good character and a deep
understanding of both new technologies and customers' needs.

``My family members do not possess these qualities,'' he wrote.
``Therefore they will never join the line of succession.''

Mr. Ren's plain speaking has not managed to make United States officials
feel comfortable about allowing Huawei's gear into the country's
internet infrastructure.

Huawei executives have said repeatedly that they are independent of the
Chinese government and military. They have
\href{https://www.nytimes.com/2018/12/21/technology/the-week-in-tech-hostages-in-the-us-and-china-tech-cold-war.html}{challenged
Western governments} to produce evidence that the firm's products are
vulnerable to state meddling.

According to
\href{https://search.wikileaks.org/plusd/cables/08GUANGZHOU171_a.html}{a
diplomatic cable} published by WikiLeaks, Mr. Ren told the American
consul general in Guangzhou in 2008 that if Huawei really had ties to
Beijing, the company would be in real estate, not telecom equipment.
That, he said, was where the easy money was.

But for some of Huawei's critics, such assurances do not outweigh larger
concerns about
\href{https://www.nytimes.com/2018/11/29/us/politics/china-trump-cyberespionage.html}{the
Chinese government's behavior in the digital realm}.

``It's not really about Ren's roots in the P.L.A., in my opinion,'' said
Andrew Davenport, the chief operating officer of RWR Advisory Group, a
Washington-based risk consulting firm, referring to the People's
Liberation Army. ``It's just the fact that they're Chinese, and are
tainted by their government's poor record on cyberespionage.''

Mr. Davenport added: ``Any global Chinese tech actor is at risk of being
considered a liability, because they're going to be susceptible to doing
what Chinese government wants them to do.''

Advertisement

\protect\hyperlink{after-bottom}{Continue reading the main story}

\hypertarget{site-index}{%
\subsection{Site Index}\label{site-index}}

\hypertarget{site-information-navigation}{%
\subsection{Site Information
Navigation}\label{site-information-navigation}}

\begin{itemize}
\tightlist
\item
  \href{https://help.nytimes.com/hc/en-us/articles/115014792127-Copyright-notice}{©~2020~The
  New York Times Company}
\end{itemize}

\begin{itemize}
\tightlist
\item
  \href{https://www.nytco.com/}{NYTCo}
\item
  \href{https://help.nytimes.com/hc/en-us/articles/115015385887-Contact-Us}{Contact
  Us}
\item
  \href{https://www.nytco.com/careers/}{Work with us}
\item
  \href{https://nytmediakit.com/}{Advertise}
\item
  \href{http://www.tbrandstudio.com/}{T Brand Studio}
\item
  \href{https://www.nytimes.com/privacy/cookie-policy\#how-do-i-manage-trackers}{Your
  Ad Choices}
\item
  \href{https://www.nytimes.com/privacy}{Privacy}
\item
  \href{https://help.nytimes.com/hc/en-us/articles/115014893428-Terms-of-service}{Terms
  of Service}
\item
  \href{https://help.nytimes.com/hc/en-us/articles/115014893968-Terms-of-sale}{Terms
  of Sale}
\item
  \href{https://spiderbites.nytimes.com}{Site Map}
\item
  \href{https://help.nytimes.com/hc/en-us}{Help}
\item
  \href{https://www.nytimes.com/subscription?campaignId=37WXW}{Subscriptions}
\end{itemize}
