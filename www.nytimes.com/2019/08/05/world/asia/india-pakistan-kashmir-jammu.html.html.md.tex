Sections

SEARCH

\protect\hyperlink{site-content}{Skip to
content}\protect\hyperlink{site-index}{Skip to site index}

\href{https://www.nytimes.com/section/world/asia}{Asia Pacific}

\href{https://myaccount.nytimes.com/auth/login?response_type=cookie\&client_id=vi}{}

\href{https://www.nytimes.com/section/todayspaper}{Today's Paper}

\href{/section/world/asia}{Asia Pacific}\textbar{}India Revokes
Kashmir's Special Status, Raising Fears of Unrest

\url{https://nyti.ms/2T4gN6j}

\begin{itemize}
\item
\item
\item
\item
\item
\item
\end{itemize}

Advertisement

\protect\hyperlink{after-top}{Continue reading the main story}

Supported by

\protect\hyperlink{after-sponsor}{Continue reading the main story}

\hypertarget{india-revokes-kashmirs-special-status-raising-fears-of-unrest}{%
\section{India Revokes Kashmir's Special Status, Raising Fears of
Unrest}\label{india-revokes-kashmirs-special-status-raising-fears-of-unrest}}

\includegraphics{https://static01.nyt.com/images/2019/08/05/world/05kashmir4/merlin_158888133_104917b6-91e8-4406-bec5-2ac566415b24-articleLarge.jpg?quality=75\&auto=webp\&disable=upscale}

By \href{https://www.nytimes.com/by/jeffrey-gettleman}{Jeffrey
Gettleman}, \href{https://www.nytimes.com/by/suhasini-raj}{Suhasini
Raj}, \href{https://www.nytimes.com/by/kai-schultz}{Kai Schultz} and
\href{https://www.nytimes.com/by/hari-kumar}{Hari Kumar}

\begin{itemize}
\item
  Aug. 5, 2019
\item
  \begin{itemize}
  \item
  \item
  \item
  \item
  \item
  \item
  \end{itemize}
\end{itemize}

NEW DELHI --- India's Hindu nationalist government on Monday
unilaterally wiped out the autonomy of the restive Kashmir region,
sending in thousands of army troops to quell any possible unrest the
move would bring in a disputed territory fought over by India and
Pakistan.

Government authorities severed internet connections, mobile phone lines
and even land lines, casting
\href{https://www.nytimes.com/2019/08/07/world/asia/pakistan-kashmir-india.html}{Kashmir}
into an information black hole that made it very difficult to discern
what was unfolding.

For years, India's Hindu nationalists have wanted to curtail the special
freedoms enjoyed by Kashmir, a mountainous, predominantly Muslim
territory that has turned into a tinderbox between
\href{https://www.nytimes.com/2019/08/07/world/asia/pakistan-kashmir-india.html}{India
and Pakistan}, both of which wield nuclear arms.

On Monday, Amit Shah, India's home minister, announced in a quick
speech, which belied years of steady plotting, that the central
government was removing the special, somewhat autonomous status that
served as the foundation for Kashmir joining India more than 70 years
ago.

\emph{{[}See}
\href{https://www.nytimes.com/2019/08/09/world/asia/kashmir-photos-india.html}{\emph{some
of the first images}} \emph{that have emerged from Kashmir since India
instituted a communications blockade.{]}}

While international human rights groups swiftly condemned the action,
Hindu nationalists celebrated, saying this could bring peace and
investment to the war-torn region.

Detail

area

TAJIKISTAN

CHINA

New

Delhi

INDIA

GILGIT--

BALTISTAN

Controlled

by Pakistan

Line of

control

Kashmir

Valley

Ladakh

JAMMU AND KASHMIR

JAMMU

PAKISTAN

INDIA

100 MILES

By The New York Times

But the voice of the Kashmiris was silenced, as government authorities
cut off practically all communication from the area.

Several top Kashmiri politicians were taken into custody. Mehbooba
Mufti, a former chief minister of Kashmir, managed to get out a message
shortly before she was arrested on Monday night.

``The Fifth of August is the blackest day of Indian democracy when its
Parliament, like thieves, snatched away everything from the people of
Jammu and Kashmir,'' she said.

Her daughter, Iltija Javed, who succeeded in transmitting a message to
The New York Times on Monday night, summed up the desperation of many
Kashmiris.

``We feel there is an atmosphere of death looming over us,'' she said.
``We don't know what to expect. We are not allowed to get out of our
houses. Telecommunications are all down. For the first time in 30 years
they snapped landline connections as well. So there is no way even
ordinary Kashmiris here can like communicate with each other, and know
what exactly is going on. Everybody is in a state of absolute shock and
panic.''

The Indian consul general in New York said in a statement that the
action to revoke Kashmir's autonomy, which was granted under Article 370
of India's Constitution, was ``purely administrative'' and was intended
to ``improve good governance and deliver socio-economic justice to the
disadvantaged sections of the people in the State.''

\href{https://www.nytimes.com/interactive/2019/world/asia/india-pakistan-crisis.html}{}

\includegraphics{https://static01.nyt.com/images/2019/02/27/us/india-pakistan-crisis-promo-1551305197760/india-pakistan-crisis-promo-1551305197760-articleLarge-v2.jpg}

\hypertarget{what-is-article-370-and-why-does-it-matter-in-kashmir}{%
\subsection{What Is Article 370, and Why Does It Matter in
Kashmir?}\label{what-is-article-370-and-why-does-it-matter-in-kashmir}}

A simple guide to the roots of the conflict and what could happen next.

The consul general added that restrictions related to Article 370
``seriously discouraged'' investment in the region, limited economic
opportunities and hurt younger generations.

India's government had been carefully preparing for this action, which
instantly raised tensions across the border in Pakistan. For the past
two weeks, tens of thousands of extra troops had been deployed across
Kashmir, and many Kashmiris had been expecting something big.

Still, many people were stunned that the government actually made the
decision. It was widely seen as another bold, muscular move by the
administration of Narendra Modi, India's forceful prime minister, to
consolidate power.

Many Indians believe Kashmir is a legitimate part of India, and several
other political parties, including progressive ones, lined up behind the
government.

With an overwhelming majority, the upper house of Parliament passed a
related bill Monday evening that split the state of Jammu and Kashmir,
which includes the Kashmir Valley and the Ladakh area, into two federal
territories: Jammu and Kashmir, which will have a state legislature, and
Ladakh, a remote, high-altitude area, which will be ruled directly from
New Delhi.

If this clears the lower House, which is expected in the coming days,
Kashmir loses the special status it has enjoyed since 1947 when it chose
to join India.

\includegraphics{https://static01.nyt.com/images/2019/08/05/world/05kashmir5/merlin_158888253_4bc6cd94-1ef7-4d23-8766-9f5d42cdaeaf-articleLarge.jpg?quality=75\&auto=webp\&disable=upscale}

Officials in Pakistan were contacting allies around the world to try and
oppose the action, but many analysts said Pakistan has little
credibility on the issue. Pakistan has a long history of covertly
supporting militant groups in Kashmir, despite pressure from allies to
stop.

Human rights activists said that the moves to change Kashmir's status
were only the first steps in a broader plan to erode Kashmir's core
rights and seed the area with non-Kashmiris, altering the demographics
and eventually destroying its character. Previous laws barred outsiders
from owning property.

Several legal scholars said they believed the government did not have
the legal authority to change Article 370. The issue, they said, was
headed for a showdown in India's Supreme Court.

``The whole bill is not only unconstitutional, it's a fraud,'' said A.
G. Noorani, a constitutional lawyer.

But India's ruling Bharatiya Janata Party, commonly referred to as the
B.J.P., may be difficult to stop. Mr. Modi, the most domineering leader
India has produced in decades, just won
\href{https://www.nytimes.com/2019/05/23/world/asia/narendra-modi-election-win.html}{a
resounding election victory} in May, in part on the promise of revoking
Article 370.

Wiping away Kashmir's special status has been a dream of many B.J.P.
supporters who have spoken of a Greater Hindustan, a Hindu-dominated
land that scoops up Pakistan, Bangladesh and other parts of South Asia.
India is about 80 percent Hindu.

Image

Barricades being set up by the Indian police in Jammu, Kashmir, on
Monday. A sense of panic has spread across Kashmir as millions of
residents woke up Monday to deserted streets.Credit...Channi
Anand/Associated Press

The B.J.P.'s leaders have cast the Kashmir issue as a nationalist cause
and have raised fears of Pakistani infiltrations and terrorist attacks
in the region.

``The application of Article 370 to foster vested interests have created
a climate of separatism,'' the consul general said in the statement,
adding that the ``defense of the security and stability'' of the region
``has cost more than 40,000 lives and been a major drain on resources.''

Kashmir has been
\href{https://www.nytimes.com/2018/08/01/world/asia/kashmir-war-india-pakistan.html}{racked
by bloodshed} for years. Many Kashmiris don't want to be part of India
and a small but stubborn insurgency has been fighting Indian forces.
Countless Hindus across India feel solidarity with Kashmiri Hindus, a
minority in that area, who had been driven out over the years during the
conflict.

Officials in Mr. Modi's party believe it is time to try something
different. They say that if non-Kashmiris are allowed to own land in
Kashmir, more investment and development will follow, increasing the
chances for peace and national unity.

``Imposed divisions between Indians and Kashmiris have been done away
with,'' said Rakesh Sinha, a B.J.P. lawmaker. ``The slogan of `One
Nation, One People' is now a reality for Indians.''

Some analysts say the timing is suspicious. In recent weeks, Mr. Modi's
government has come under increasing criticism over a weakening economy,
with joblessness rising. A sense of malaise is beginning to seep through
just about all sectors of the economy.

Image

A woman and child walking past security personnel in Jammu on Monday.
Analysts say that any steps that reduce Kashmir's autonomy could provoke
an outburst of serious violence.Credit...Rakesh Bakshi/Agence
France-Presse --- Getty Images

Analysts say that Mr. Modi and Mr. Shah, widely considered the Indian
leader's right-hand man, were desperate to shore up their base and shift
the conversation.

``This is exactly what national populists do all over the world,'' said
Christophe Jaffrelot, a research fellow at CERI-Sciences Po/CNRS in
Paris. ``Clearly, India is entering a zone of economic turbulence. So
this is the right time to return to the nationalistic agenda.''

In Kashmir, a sense of doom had been settling in. As federal forces
poured into the valley in recent weeks, many Kashmiris grew to believe
that Mr. Modi's government was preparing to take significant action.
Jammu and Kashmir, with a population of about 13 million, is India's
only Muslim majority state.

Economic and political frustration permeates the Kashmir Valley. The
young have struggled to find work as political turmoil has hampered
development, and many people feel they don't have a voice. Last year,
amid political turmoil, the regional Parliament was dissolved and the
state fell under federal rule.

Kashmir never fit neatly into the bigger India picture. When India and
Pakistan won independence from Britain in 1947, Kashmir originally opted
to remain a small independent state.

Soon after independence, though, militants from Pakistan invaded the
territory, leading it to seek protection from India. Kashmir agreed to
become part of India, but only under the autonomy enshrined in Article
370. That article was like a contract, guaranteeing that Kashmir would
be different from other Indian states and have a say on what kind of
federal laws could be imposed on it.

Image

A celebration in New Delhi on Monday after the Indian government
scrapped the special status for Kashmir.Credit...Danish Siddiqui/Reuters

These protections lead to special property rights for Kashmiris that
blocked non-Kashmiris from owning land.

India and Pakistan then fought several wars over the area. And today
most of Kashmir is administered by India, with a smaller slice
controlled by Pakistan.

Many of the original provisions of the Kashmir-India partnership, like
Kashmir having its own prime minister, have already been done away with.

Long-simmering tensions with Pakistan
\href{https://www.nytimes.com/2019/02/15/world/asia/kashmir-attack-pulwama.html}{reached
a breaking point in February}, when a Kashmiri militant rammed a vehicle
filled with explosives into a convoy of Indian paramilitary forces
traveling on a highway, killing at least 40 soldiers. A banned terrorist
group, Jaish-e-Muhammad, which is based in Pakistan, claimed
responsibility.

It was the deadliest
\href{https://www.nytimes.com/2019/02/15/world/asia/kashmir-attack-pulwama.html?module=inline}{attack
in the region} in three decades, and set off a tense military standoff
between India and Pakistan that culminated in a
\href{https://www.nytimes.com/2019/03/01/world/asia/india-pakistan-plane-abhinandan-varthaman-india.html?module=inline}{dogfight
between Indian and Pakistani warplanes}. Pakistan shot down and captured
an Indian pilot, whose return helped calm the tensions.

President Trump recently met Prime Minister Imran Khan of Pakistan in
Washington and offered to mediate on Kashmir, but India rejected that,
saying Kashmir was a domestic, not an international, issue.

In a written statement on Monday, the State Department said it was
following developments closely and that it was concerned about reports
of detention in Kashmir. It urged all parties to maintain peace along
the dividing line between India- and Pakistan-controlled territory.

On Monday night, Kashmir seemed quiet. According to a few people who
spoke to their relatives in the state government (some officials were
given satellite phones), most streets were deserted. Soldiers were
everywhere. Many people were scared to leave their homes.

Advertisement

\protect\hyperlink{after-bottom}{Continue reading the main story}

\hypertarget{site-index}{%
\subsection{Site Index}\label{site-index}}

\hypertarget{site-information-navigation}{%
\subsection{Site Information
Navigation}\label{site-information-navigation}}

\begin{itemize}
\tightlist
\item
  \href{https://help.nytimes.com/hc/en-us/articles/115014792127-Copyright-notice}{©~2020~The
  New York Times Company}
\end{itemize}

\begin{itemize}
\tightlist
\item
  \href{https://www.nytco.com/}{NYTCo}
\item
  \href{https://help.nytimes.com/hc/en-us/articles/115015385887-Contact-Us}{Contact
  Us}
\item
  \href{https://www.nytco.com/careers/}{Work with us}
\item
  \href{https://nytmediakit.com/}{Advertise}
\item
  \href{http://www.tbrandstudio.com/}{T Brand Studio}
\item
  \href{https://www.nytimes.com/privacy/cookie-policy\#how-do-i-manage-trackers}{Your
  Ad Choices}
\item
  \href{https://www.nytimes.com/privacy}{Privacy}
\item
  \href{https://help.nytimes.com/hc/en-us/articles/115014893428-Terms-of-service}{Terms
  of Service}
\item
  \href{https://help.nytimes.com/hc/en-us/articles/115014893968-Terms-of-sale}{Terms
  of Sale}
\item
  \href{https://spiderbites.nytimes.com}{Site Map}
\item
  \href{https://help.nytimes.com/hc/en-us}{Help}
\item
  \href{https://www.nytimes.com/subscription?campaignId=37WXW}{Subscriptions}
\end{itemize}
