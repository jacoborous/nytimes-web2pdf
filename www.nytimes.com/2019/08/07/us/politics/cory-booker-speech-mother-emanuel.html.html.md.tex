Sections

SEARCH

\protect\hyperlink{site-content}{Skip to
content}\protect\hyperlink{site-index}{Skip to site index}

\href{https://www.nytimes.com/section/politics}{Politics}

\href{https://myaccount.nytimes.com/auth/login?response_type=cookie\&client_id=vi}{}

\href{https://www.nytimes.com/section/todayspaper}{Today's Paper}

\href{/section/politics}{Politics}\textbar{}Biden and Booker Say Trump
Is Fostering Hatred, Not Fighting It

\url{https://nyti.ms/2ThXCWD}

\begin{itemize}
\item
\item
\item
\item
\item
\end{itemize}

\begin{itemize}
\item
  \href{https://www.nytimes.com/2020/07/31/us/elections/biden-vs-trump.html?action=click\&pgtype=Article\&state=default\&region=TOP_BANNER\&context=storylines_menu}{Election
  Updates}
\item
  \href{https://www.nytimes.com/article/biden-vice-president-2020.html?action=click\&pgtype=Article\&state=default\&region=TOP_BANNER\&context=storylines_menu}{Biden's
  V.P. Search}
\item
  \href{https://www.nytimes.com/interactive/2020/07/24/us/politics/trump-biden-campaign-donors.html?action=click\&pgtype=Article\&state=default\&region=TOP_BANNER\&context=storylines_menu}{Map
  of Donations}
\item
  \href{https://www.nytimes.com/interactive/2020/us/elections/delegate-count-primary-results.html?action=click\&pgtype=Article\&state=default\&region=TOP_BANNER\&context=storylines_menu}{Delegate
  Count}
\item
  \href{https://www.nytimes.com/interactive/2019/us/politics/2020-presidential-candidates.html?action=click\&pgtype=Article\&state=default\&region=TOP_BANNER\&context=storylines_menu}{The
  Candidates}
\item
  \href{https://www.nytimes.com/newsletters/politics?action=click\&pgtype=Article\&state=default\&region=TOP_BANNER\&context=storylines_menu}{Politics
  Newsletter}
\end{itemize}

Advertisement

\protect\hyperlink{after-top}{Continue reading the main story}

Supported by

\protect\hyperlink{after-sponsor}{Continue reading the main story}

\hypertarget{biden-and-booker-say-trump-is-fostering-hatred-not-fighting-it}{%
\section{Biden and Booker Say Trump Is Fostering Hatred, Not Fighting
It}\label{biden-and-booker-say-trump-is-fostering-hatred-not-fighting-it}}

\includegraphics{https://static01.nyt.com/images/2019/08/07/us/politics/07booker-1/07booker-1-videoSixteenByNineJumbo1600.jpg}

By \href{https://www.nytimes.com/by/alexander-burns}{Alexander Burns}
and \href{https://www.nytimes.com/by/katie-glueck}{Katie Glueck}

\begin{itemize}
\item
  Aug. 7, 2019
\item
  \begin{itemize}
  \item
  \item
  \item
  \item
  \item
  \end{itemize}
\end{itemize}

CHARLESTON, S.C. --- Democratic presidential candidates lashed President
Trump on Wednesday with their sternest denunciations yet of his
exploitation of racism for political purposes and resistance to gun
control, in a day of biting criticism that also highlighted differences
between Democrats over how best to understand the recent rise of hate
crimes in America.

More than ever, it was clear that last weekend's massacres in El Paso
and Dayton, Ohio, had put Mr. Trump on the defensive and added fierce
new urgency to Democratic efforts to engineer his ouster. Mr. Trump has
not accounted for the echoes of his own rhetoric about immigrants and
minorities in the manifesto composed by the anti-immigrant gunman in
Texas, and he has appeared far more focused on feuding with his critics
than on striking a tone of healing.

Former Vice President Joseph R. Biden Jr., in one of the most fiery
speeches of his campaign so far, argued Wednesday that Mr. Trump had
both explicitly and implicitly ``fanned the flames of white supremacy in
this nation'' with his language.

``Trump readily, eagerly attacks Islamic terrorism but can barely bring
himself to use the words `white supremacy,''' Mr. Biden said in
Burlington, Iowa. ``And even when he says it, he doesn't appear to
believe it. He seems more concerned about losing their votes than
beating back this hateful ideology.''

Speaking in Charleston at Emanuel African Methodist Episcopal Church,
where a white supremacist gunman killed nine black worshipers in 2015,
Senator Cory Booker of New Jersey also blamed Mr. Trump for encouraging
hatred. The weekend's violence, he said, was ``sowed by those who spoke
the same words the El Paso murderer did, warning of an `invasion,''' a
word Mr. Trump has used to describe migrants approaching the Southern
border.

And Senator Elizabeth Warren and former Representative Beto O'Rourke
both said they believe Mr. Trump was a white supremacist.

After the president's visit to El Paso on Wednesday, Mr. O'Rourke also
suggested that Mr. Trump bore responsibility for the mass shooting there
on Saturday by a white supremacist gunman who killed 22 people.
Referring to immigrants, Mexicans and others in the community there, Mr.
O'Rourke said, ``To have been so regularly attacked and vilified and
demonized by this president, for him to have created the conditions that
made an attack like this possible and ultimately likely --- it's very
insulting for us that he was here.''

Mr. Trump has emphatically denied that he is racist, and on Wednesday,
he dismissed reporters' questions about the role of his rhetoric in
dividing the country, saying his language ``brings people together.''

The extraordinary focus this week on white nationalism, gun violence and
domestic terror appeared to reframe a chaotic presidential campaign as a
searing moral debate about the racial history and cultural destiny of
the United States. Mr. Trump, who rose to power railing against the
country's changing ethnic and cultural texture, contends that Democrats
should be punished for opposing his immigration policies and rejecting
the values of the rural white people who make up his base. Democrats,
meanwhile,
\href{https://www.nytimes.com/2019/08/05/us/politics/beto-trump.html}{are
now arguing in the most explicit terms yet that white supremacists are
receiving aid and comfort} from the president.

``His low-energy, vacant-eyed mouthing of the words written for him
condemning white supremacists this week I don't believe fooled anyone,
at home or abroad,'' Mr. Biden said, referring to Mr. Trump's remarks
Monday about the El Paso shooting.

\hypertarget{latest-updates-2020-election}{%
\section{\texorpdfstring{\href{https://www.nytimes.com/2020/07/31/us/elections/biden-vs-trump.html?action=click\&pgtype=Article\&state=default\&region=MAIN_CONTENT_1\&context=storylines_live_updates}{Latest
Updates: 2020
Election}}{Latest Updates: 2020 Election}}\label{latest-updates-2020-election}}

Updated 2020-08-01T01:26:45.732Z

\begin{itemize}
\tightlist
\item
  \href{https://www.nytimes.com/2020/07/31/us/elections/biden-vs-trump.html?action=click\&pgtype=Article\&state=default\&region=MAIN_CONTENT_1\&context=storylines_live_updates\#link-29fdff45}{Kamala
  Harris, a top vice-presidential contender, confronts double
  standards.}
\item
  \href{https://www.nytimes.com/2020/07/31/us/elections/biden-vs-trump.html?action=click\&pgtype=Article\&state=default\&region=MAIN_CONTENT_1\&context=storylines_live_updates\#link-13ec3d9c}{Karen
  Bass and Susan Rice are rising on Biden's vice-presidential
  shortlist.}
\item
  \href{https://www.nytimes.com/2020/07/31/us/elections/biden-vs-trump.html?action=click\&pgtype=Article\&state=default\&region=MAIN_CONTENT_1\&context=storylines_live_updates\#link-49e9a016}{Trump
  says Russian bounties to kill U.S. troops `never took place.'}
\end{itemize}

\href{https://www.nytimes.com/2020/07/31/us/elections/biden-vs-trump.html?action=click\&pgtype=Article\&state=default\&region=MAIN_CONTENT_1\&context=storylines_live_updates}{See
more updates}

There is
\href{https://www.nytimes.com/2019/08/05/us/politics/beto-trump.html}{virtually
no disagreement among the Democratic candidates about Mr. Trump's
character}, or his culpability in what they see as a still-unfurling
disaster of race relations and social cohesion. Where they differ, it is
largely over whether Mr. Trump is the country's chief affliction, or a
symptom of deeper woes.

\includegraphics{https://static01.nyt.com/images/2019/08/07/us/politics/07booker-biden/merlin_158984973_d622bee6-6d1f-4b33-9254-8c6eaf9569da-articleLarge.jpg?quality=75\&auto=webp\&disable=upscale}

But if the starkest contrast this week has been between Mr. Trump and
those vying to unseat him, the speeches on Wednesday by several
Democratic candidates also exposed important gradations in their
worldviews. Mr. Booker spoke at considerable length on racism as an
American heritage, while Mr. Biden acknowledged dark episodes from the
past but leaned more heavily on nostalgia and triumphalism.

In Iowa, Mr. Biden acknowledged that American history was no
``fairy-tale.'' ``I wish I could say that this all began with Donald
Trump and will end with him,'' he said. ``But it didn't and I won't.''

But he also assailed Mr. Trump as representing a wild departure from the
American political tradition, blaming him for stoking hatred and
abandoning the unifying role past presidents have sought to play. He
contrasted Mr. Trump's ambivalent response to racism and tragedy with
the conduct of his predecessors, including Bill Clinton's response to
the Oklahoma City bombing in 1995 and George W. Bush's visit to a mosque
after the terror attacks of Sept. 11, 2001

In a biting one-liner that has become a regular jab on the campaign
trail, Mr. Biden said that Mr. Trump had ``more in common with George
Wallace than he does with George Washington.''

Speaking from the pulpit at the church known as Mother Emanuel, one
floor above the room where the 2015 massacre took place, Mr. Booker
eschewed that kind of nostalgia for the founding fathers in his own
speech against violent racism.

He said instead that white supremacy had been ``ingrained in our
politics since our founding,'' within the text of the Declaration of
Independence and the Constitution. The present moment, he said, demanded
both federal action to regulate guns and investigate white nationalists
and a cleareyed confrontation of the past.

``Racist violence has always been part of the American story, never more
so than in times of transition and times of rapid social change,'' Mr.
Booker said, linking the trauma of the last week to slavery and
segregation, and ``demagogues throughout generations who stoked racist
and anti-immigrant hatred, often for votes, and then enshrined their
bigotry into laws.''

Mr. Booker neither mentioned Mr. Trump by name, nor did he cast the
president as an aberrational figure in American history, as Mr. Biden
did.

Instead, he urged a broad moral reckoning over racism and departed from
his prepared remarks to call, in an echo of Martin Luther King Jr., for
the rise of a ``generation that truly will be free at last.''

``There is no neutrality in this fight,'' he said. ``You are either an
agent of justice or you are contributing to the problem.''

The speech by Mr. Booker, one of two leading black candidates for the
Democratic nomination, had the potential to be one of the most important
moments of his campaign, testing his power as a voice of moral clarity
and racial justice after a week of national pain.

He has lagged in the polls, insisting on a message of healing that has
at times clashed with his party's prevailing mood of hot indignation.
But after months of toiling away in relative obscurity he had a standout
performance in the second round of Democratic debates last week, besting
Mr. Biden in a series of exchanges on race and criminal justice and
displaying for a national audience the kind of sunny pugilism that has
made him a force in New Jersey and in the Senate.

Image

The speech by Mr. Booker had the potential to be one of the most
important moments of his campaign, testing his power as a voice of moral
clarity and racial justice in a crowded Democratic race.Credit...Hilary
Swift for The New York Times

Few venues for Mr. Booker's message could have been as laden with
symbolism as the one he chose. A funeral for the victims of the
massacre, by a gunman who has since been sentenced to die, became the
site of one of the most memorable moments of former President Barack
Obama's time in office. During the ceremony, he broke into a rendition
of ``Amazing Grace'' and called both for the removal of the Confederate
flag from South Carolina's capitol building and a remedy to ``the mayhem
that gun violence inflicts upon this nation.''

Mr. Booker earned applause and murmurs of appreciation throughout his
speech, including with a prominent quotation from ``our beloved Toni
Morrison,''
\href{https://www.nytimes.com/2019/08/06/books/toni-morrison-dead.html}{the
Nobel laureate who died this week}. Borrowing her words, Mr. Booker
said: ``The function of freedom is to free someone else.''

Deirdre McClain, who watched Mr. Booker speak from the pews, said she
found his remarks moving and persuasive, including his recitation of the
names of the nine people murdered in the church.

``I was surprised at how he made the past come to the present, how he
knew the names of those who had passed in this very church and that he
connected it to freedom,'' said Ms. McClain, 53. ``That resonated with
me.''

Ms. McClain said Mr. Booker was among several candidates she was
considering in the presidential race, along with Senator Kamala Harris
of California, Ms. Warren and Mr. Biden.

The Democratic field has been all but unanimous in its criticism of Mr.
Trump this week, and most have also delivered strong criticism of the
gun lobby and congressional Republicans for stymying efforts at even
modest new firearm restrictions. In Iowa on Wednesday evening, Ms.
Warren pledged again that as president she would break the resistance to
gun control led by the National Rifle Association.

``Right now, Congress, the Republicans, are held by the throat by the
N.R.A. Enough is enough,'' she said. ``When I am president, we're going
to
\href{https://www.nytimes.com/2019/07/10/us/politics/climate-change-filibuster.html}{get
rid of the filibuster} and we're going to pass some serious gun
legislation in this country.''

The candidates have also been united in their descriptions of the
president as a racist or as bearing some personal responsibility for the
violence in El Paso.

Mr. Trump has long attacked Latin American migrants as dangerous
criminals and his campaign
\href{https://www.nytimes.com/2019/08/05/us/politics/trump-campaign-facebook-ads-invasion.html}{has
run thousands of digital advertisements describing illegal immigration
as an invasion}. He has spent much of the summer insulting prominent
black and Hispanic Democrats, deriding the predominantly black city of
Baltimore and addressing a rally where his supporters engaged in a chant
of ``send her back'' directed at a Democratic lawmaker, Representative
Ilhan Omar of Minnesota, who is a naturalized citizen who came to the
United States as a child refugee from Somalia.

Though Mr. Trump denounced white supremacy in his speech from the White
House Monday, he has continued to batter his political rivals in
divisive terms, railing on Twitter against Mr. O'Rourke, who has
described Mr. Trump this week as an obvious racist. Mr. Trump mocked Mr.
O'Rourke for taking ``Beto'' as a nickname --- his birth name is Robert
--- tweeting that it was a ``phony name to indicate Hispanic heritage.''

Speaking Wednesday at an El Paso park, Mr. O'Rourke praised his hometown
as a safe, beautiful and welcoming place made stronger by its
bi-nationalism and the immigrants and asylum seekers who live there.

``Though we bore the brunt of this hatred and this racism and this
intolerance and this violence, I believe this community also holds the
answer,'' he said. ``The way that we welcome one another and see our
differences --- not as disqualifying or dangerous, but as a very source
of our strength, as a foundation of our success --- that needs to be the
example to the United States of America today.''

\hypertarget{our-2020-election-guide}{%
\section{Our 2020 Election Guide}\label{our-2020-election-guide}}

Updated July 31, 2020

\begin{itemize}
\item
  \begin{center}\rule{0.5\linewidth}{\linethickness}\end{center}

  \hypertarget{the-latest}{%
  \subsection{The Latest}\label{the-latest}}

  \begin{itemize}
  \tightlist
  \item
    President Trump's assault on the Postal Service is intersecting with
    his attacks on mail-in voting.
    \href{https://www.nytimes.com/2020/07/31/us/politics/trump-usps-mail-delays.html?action=click\&pgtype=Article\&state=default\&region=BELOW_MAIN_CONTENT\&context=storylines_guide}{Voting
    rights groups say it is a recipe for disaster.}
  \end{itemize}
\item
  \begin{center}\rule{0.5\linewidth}{\linethickness}\end{center}

  \hypertarget{bidens-vp-search}{%
  \subsection{Biden's V.P. Search}\label{bidens-vp-search}}

  \begin{itemize}
  \tightlist
  \item
    \href{https://www.nytimes.com/article/biden-vice-president-2020.html?action=click\&pgtype=Article\&state=default\&region=BELOW_MAIN_CONTENT\&context=storylines_guide}{Here
    are 13 women} who have been under consideration to be Joe Biden's
    running mate, and why each might be chosen --- and might not be.
  \end{itemize}
\item
  \begin{center}\rule{0.5\linewidth}{\linethickness}\end{center}

  \hypertarget{keep-up-with-our-coverage}{%
  \subsection{Keep Up With Our
  Coverage}\label{keep-up-with-our-coverage}}

  \begin{itemize}
  \tightlist
  \item
    Get an
    \href{https://www.nytimes.com/newsletters/politics?action=click\&pgtype=Article\&state=default\&region=BELOW_MAIN_CONTENT\&context=storylines_guide}{email}
    recapping the day's news
  \end{itemize}

  \begin{itemize}
  \tightlist
  \item
    Download our mobile app on
    \href{https://apps.apple.com/us/app/nytimes/id284862083?ls=1\&mat_click_id=5c79ae7455014fd1bd66b5610c05b8f2-20191112-16948\&referrer=mat_click_id\%3D5c79ae7455014fd1bd66b5610c05b8f2-20191112-16948\%26link_click_id\%3D722930677036718082}{iOS}
    and
    \href{http://a.localytics.com/android?id=com.nytimes.android\&referrer=utm_source\%3Dother_nyt_mobile_web\%26utm_medium\%3DWeb\%2520page\%26utm_term\%3DGeneral\%2520Mobile\%2520Page\%26utm_campaign\%3DNYT\%2520Mobile\%2520General\%2520Page}{Android}
    and turn on Breaking News and Politics alerts
  \end{itemize}
\end{itemize}

Advertisement

\protect\hyperlink{after-bottom}{Continue reading the main story}

\hypertarget{site-index}{%
\subsection{Site Index}\label{site-index}}

\hypertarget{site-information-navigation}{%
\subsection{Site Information
Navigation}\label{site-information-navigation}}

\begin{itemize}
\tightlist
\item
  \href{https://help.nytimes.com/hc/en-us/articles/115014792127-Copyright-notice}{©~2020~The
  New York Times Company}
\end{itemize}

\begin{itemize}
\tightlist
\item
  \href{https://www.nytco.com/}{NYTCo}
\item
  \href{https://help.nytimes.com/hc/en-us/articles/115015385887-Contact-Us}{Contact
  Us}
\item
  \href{https://www.nytco.com/careers/}{Work with us}
\item
  \href{https://nytmediakit.com/}{Advertise}
\item
  \href{http://www.tbrandstudio.com/}{T Brand Studio}
\item
  \href{https://www.nytimes.com/privacy/cookie-policy\#how-do-i-manage-trackers}{Your
  Ad Choices}
\item
  \href{https://www.nytimes.com/privacy}{Privacy}
\item
  \href{https://help.nytimes.com/hc/en-us/articles/115014893428-Terms-of-service}{Terms
  of Service}
\item
  \href{https://help.nytimes.com/hc/en-us/articles/115014893968-Terms-of-sale}{Terms
  of Sale}
\item
  \href{https://spiderbites.nytimes.com}{Site Map}
\item
  \href{https://help.nytimes.com/hc/en-us}{Help}
\item
  \href{https://www.nytimes.com/subscription?campaignId=37WXW}{Subscriptions}
\end{itemize}
