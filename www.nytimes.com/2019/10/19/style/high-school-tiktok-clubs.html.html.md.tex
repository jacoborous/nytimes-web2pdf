Sections

SEARCH

\protect\hyperlink{site-content}{Skip to
content}\protect\hyperlink{site-index}{Skip to site index}

\href{https://www.nytimes.com/section/style}{Style}

\href{https://myaccount.nytimes.com/auth/login?response_type=cookie\&client_id=vi}{}

\href{https://www.nytimes.com/section/todayspaper}{Today's Paper}

\href{/section/style}{Style}\textbar{}TikTok Clubs in High School? It
Was a Thing.

\url{https://nyti.ms/2BngV8O}

\begin{itemize}
\item
\item
\item
\item
\item
\item
\end{itemize}

Advertisement

\protect\hyperlink{after-top}{Continue reading the main story}

Supported by

\protect\hyperlink{after-sponsor}{Continue reading the main story}

\hypertarget{tiktok-clubs-in-high-school-it-was-a-thing}{%
\section{TikTok Clubs in High School? It Was a
Thing.}\label{tiktok-clubs-in-high-school-it-was-a-thing}}

Teens love the app, and now it's getting the stamp of approval with
teacher-approved clubs. Did school just get ... fun?

\includegraphics{https://static01.nyt.com/images/2019/10/23/fashion/20TIKTOKCLUBS-hallway-dance/merlin_162854286_82985151-de55-4fb2-bf63-37715bdf9928-articleLarge.jpg?quality=75\&auto=webp\&disable=upscale}

\href{https://www.nytimes.com/by/taylor-lorenz}{\includegraphics{https://static01.nyt.com/images/2020/03/18/reader-center/author-taylor-lorenz/author-taylor-lorenz-thumbLarge.png}}

By \href{https://www.nytimes.com/by/taylor-lorenz}{Taylor Lorenz}

\begin{itemize}
\item
  Published Oct. 19, 2019Updated July 11, 2020
\item
  \begin{itemize}
  \item
  \item
  \item
  \item
  \item
  \item
  \end{itemize}
\end{itemize}

WINTER GARDEN, Fla. --- On the wall of a classroom that is home to
\href{http://vm.tiktok.com/5qoKE6/}{the West Orange High School TikTok
club}, large loopy words are scrawled across a whiteboard: ``Wanna be
TikTok famous? Join TikTok club.''

It's working. ``There's a lot of TikTok-famous kids at our school,''
said Amanda DiCastro, who is 14 and a freshman. ``Probably 20 people
have gotten famous off random things.''

The school is on a quiet palm-tree-lined street in a town just outside
Orlando. A hallway by the principal's office is busy with blue plaques
honoring the school's A.P. Scholars. Its choir director, Jeffery
Redding, won the 2019 Grammy Music Educator Award.

Amanda was referring to a different kind of stardom: on TikTok, a social
media app where users post
\href{https://www.nytimes.com/2019/03/10/style/what-is-tik-tok.html}{short
funny videos, usually set to music}, that is enjoying a surge in
popularity among teenagers around the world and has been downloaded 1.4
billion times, according to SensorTower.

\begin{center}\rule{0.5\linewidth}{\linethickness}\end{center}

More news from \href{https://www.nytimes.com/by/taylor-lorenz}{Taylor
Lorenz}:

\begin{itemize}
\item
  \href{https://www.nytimes.com/2020/02/13/style/the-original-renegade.html}{Meet
  the Original Renegade Dance Creator}
\item
  \href{https://www.nytimes.com/2020/07/09/style/tiktok-stars-race-to-land-reality-shows.html}{TikTok
  Stars Race to Land Reality Shows}
\item
  \href{https://www.nytimes.com/2020/06/21/style/tiktok-trump-rally-tulsa.html}{TikTok
  Teens and K-Pop Stans Say They Sunk Trump Rally}
\end{itemize}

\begin{center}\rule{0.5\linewidth}{\linethickness}\end{center}

The embrace of the app at this school is mirrored on scattered campuses
across the United States, where students are forming
\href{http://vm.tiktok.com/5q73YA/}{TikTok clubs} to dance, sing and
perform skits for the app --- essentially drama clubs for the digital
age, but with the potential to reach huge audiences.

And unlike other social media networks, TikTok is winning over some
educators, like Michael Callahan, a teacher at West Orange, who had
never heard of TikTok before the students told him about it.

He is an adviser to the school's club and said he loves how the app
brings students from different friend groups together. ``You see a lot
more teamwork and camaraderie,'' he said, ``and less --- I don't want to
say bullying --- but focus on individuals.''

In many of the videos on the app, which are 15 seconds to a minute long,
school hallways, classrooms and courtyards serve as a recurrent
backdrop. And if kids aren't filming themselves at school, they're
making jokes about school. One popular meme on the app mocks the class
of 2023 (freshmen this year) for being cringey and trying too hard.

``TikTok is such a theatrical platform,'' said Blake Cadwell, the
general manager of Day One agency, a marketing firm in Los Angeles that
works on Chipotle's TikTok account. ``You're trying to build your cast
for whatever you're doing, and high school is a natural environment
where you're with lots of people, so you can do these skits or
challenges.'' (A big part of TikTok culture, challenges are videos users
create that riff on an of-the-moment meme.)

Several students at West Orange have seen their videos shoot to the top
of the popular ``For You'' page of the app. In the spring, the school's
valedictorian went viral for a Minecraft video; another student got more
than three million views for a parody of the film ``Mean Girls.''

\includegraphics{https://static01.nyt.com/images/2019/10/21/fashion/17TIKTOKCLUBS-video/17TIKTOKCLUBS-video-articleLarge.jpg?quality=75\&auto=webp\&disable=upscale}

Ireland McTague, a 15-year-old sophomore at St. Agnes Academy in Texas,
said she spends about 16 hours a week on the app, creating or consuming
videos. Manny Alexander, 16, a high schooler in New York, said he would
diagnose himself as a TikTok addict. ``Not that it's interrupted my
life,'' he said, ``but my life does revolve a bit around it.''

TikTok's addictiveness can be traced, in part, to its use of artificial
intelligence to anticipate what users want and fill their feeds with it.
That technology is so effective that the app's owner,
\href{https://www.nytimes.com/2018/10/29/technology/bytedance-app-funding-china.html}{Bytedance}
(a Chinese tech conglomerate), last year introduced
\href{https://www.scmp.com/tech/apps-social/article/3003796/china-launches-anti-addiction-drive-protect-countrys-short-video}{anti-addiction
measures} in Douyin, the Chinese version, to help both users and the
parents who may be worried about them.

\hypertarget{inside-the-west-orange-high-school-tiktok-club}{%
\subsection{Inside the West Orange High School TikTok
Club}\label{inside-the-west-orange-high-school-tiktok-club}}

The West Orange club meets every other Monday after school. It was
founded in September by Kate Sandoval, a 17-year-old senior. Mr.
Callahan, the adviser, makes sure the students come up with an agenda
for each meeting and don't just sit around goofing off on their phones.

Kate pulled up a series of TikToks on a large screen. The students
sipped Capri Suns and snacked on Cheetos as they watched. The first
TikTok featured a teenager, whose face was obscured by the image of a
giant crying baby's face, dancing to ``Teach Me How to Dougie'' in the
aisle of a sporting goods store.

The second showed a skit between two boys, in which one jokes about
falling for the other after he slides past him to exit a bus seat. (This
is a popular meme for boys; the punch line is a song lyric: ``Oh no, I
think I'm catching feelings.'') The challenge for the week was to riff
on these videos. The winner would receive a Chick-fil-A gift card.

In the hallway, pairs of girls propped their phones against the wall and
attempted to mimic the ``Teach Me How to Dougie'' dance step by step.
Inside the classroom, three boys and two girls prepared skits about
catching feelings for each other.

Amy Sommers and her TikTok partner, Kaylani Heisler, a 16-year-old
senior, danced until they began sweating. ``This is hard work!'' Kaylani
said between steps.

Image

Kate Sandoval, center, plans a video with Lucia
Lopez.~Credit...Charlotte Kesl for The New York Times

Shane Skaling, 17, who is the TikTok club videographer, tracked a pair
of girls' dance moves between lockers. He moved the camera around in
circles, hitting the beats as they danced.

``We finished ours, it's fire,'' said Darcy Friday, 17. Her partner,
Morgan Townsend, 17, agreed.

The school's principal, Melissa Gordon, declared a winner. It was a skit
by three 16-year-old boys --- Benjamin Boucher, Trent Vickersand Zachary
Everidge --- about one boy falling in love with another boy, who picks
his nose. The entire classroom screamed and cheered when the winner was
announced. ``I feel like it might go viral,'' Benjamin said.

Image

From left, Darcy Friday, Trent Vickers, Zachary Everidge, Benjamin
Boucher and Shane Skaling, in a huddle in their club's
classroom.Credit...Charlotte Kesl for The New York Times

\hypertarget{if-its-on-the-internet-its-not-private}{%
\subsection{`If It's on the Internet, It's Not
Private'}\label{if-its-on-the-internet-its-not-private}}

Creating TikToks in class isn't exactly encouraged, but teachers at many
schools say they view TikTok culture as a net positive. Others, like
Emma Peden, a Spanish teacher at Fox Creek High School in South
Carolina, are more hesitant. ``Instagram, TikTok and Snapchat --- all
those interfaces --- can feed bullying,'' she said. ``I think kids can
be recording things that they shouldn't.''

One encouraging sign is that videos about topics that high schoolers are
all studying sometimes generate thousands of views and become memes in
themselves.

Kate Sandoval said she has made TikToks for her role in student
government, and Mr. Callahan, the adviser, is mulling how he can use the
app to teach students about government and social studies. ``We're
thinking this is possibly the new Schoolhouse Rock,'' he said.

``There's a lot more than just funny videos,'' Kaylani Heisler said. ``I
see countless ways to take notes, get organized. I see chemistry study
aids.''

Students occasionally involve their teachers in TikTok stunts, and many
educators have \href{http://vm.tiktok.com/5qKoQd/}{set up their own
accounts}. Sarah Jacobs, a physics teacher at San Jose High School in
California, said some of her students made TikToks explaining Newton's
Laws for extra credit last year.

St. Agnes Academy in Texas has begun releasing musical clips every
Tuesday during its morning broadcast. ``Students make a TikTok to the
sound, then the next Tuesday they post the one that they like the
most,'' said Ireland McTague, the sophomore there.

Whitesboro High School in New York incorporated TikTok memes like
\href{https://www.nytimes.com/2019/08/30/style/vsco-girls.html}{VSCO
girls} --- slang for a subculture involving a lifestyle of scrunchies,
Hydroflasks and environmentalism \emph{---} into homecoming week theme.

Some schools block access to TikTok, along with all other social media
apps, via the school's Wi-Fi systems. At West Orange, Mr. Callahan and
other educators take steps to educate students on their digital
footprint. Students are instructed to think twice before posting
anything online.

Outside the room where the TikTok club meets, paper speech bubbles hang
with messages: ``Google yourself''; ``If it's on the internet, it's not
private''; ``They loved your G.P.A.; then they saw your tweets.''

Image

TikTok club members review the videos they made.Credit...Charlotte Kesl
for The New York Times

``I think you just have to engage students in whatever they're
interested in,'' said Ms. Gordon, the principal. She likes how the app
has unlocked creativity and authenticity in the students.

``On other media you're hiding your flaws,'' Mr. Callahan said. ``Here
you're showing them off.''

Aaron Eddy, 17, a senior at Whitesboro High School in Marcy, N.Y., said
that it's the authenticity part that he thinks makes the app so
compelling. He said he likes how he can be ``crazy'' on it without
judgment.

Morgan Townsend, a 17-year-old senior at West Orange, said that she
makes TikToks of notable life moments for the memories. ``During
homecoming week we'd take a clip of our outfits every day, and it was
fun to watch the end of the week,'' she said.

Ireland McTague said, ``TikTok is a safer space where you can post
videos about you being yourself, rather than worrying about being
perfect.'' She contrasted it with YouTube and Instagram, where more
polished presentations are the norm.

(As with those platforms, there is a potential for inappropriate use of
TikTok by predators. To protect users, a TikTok spokeswoman said, the
app has safeguards like ``privacy settings, controls over who can view
or interact with content, and in-app reporting.'')

Harper Kelly, a 17-year-old senior at Milford High School in Ohio, said,
of her school TikTok club, ``The last TikTok Tuesday, the room was split
in half, one half of the room was watching TikToks, the other half was
people doing dances and making them.''

The TikTok club at
\href{https://fruitanews.org/2634/stories/tiktok-craze-rising/}{Fruita
Monument High School} in Colorado has its own TikTok account,
\href{http://vm.tiktok.com/5qwaEj/}{@TikTokClubbbb}, an early club that
appears to have spawned others. Dennis Allen, a 17-year-old senior and
club member, posted a TikTok, on which one respondent, Sophie Furdek,
wrote: ``I started TikTok club at my school.''

The TikTok club at the Ethical Culture Fieldston School in New York City
was founded just weeks ago but has already attracted big interest: 70
students registered to join at the school's recent club fair.

As the West Orange High School TikTok club wrapped up Monday afternoon,
Kate Sandoval and her friends cleared snacks from the room and discussed
future TikTok ideas. They had already posted a group dance to the
@WestOrangeTikToks handle.

Before they left, they shared one of their ideas: ``Should we make a
TikTok about being in The New York Times?''

Advertisement

\protect\hyperlink{after-bottom}{Continue reading the main story}

\hypertarget{site-index}{%
\subsection{Site Index}\label{site-index}}

\hypertarget{site-information-navigation}{%
\subsection{Site Information
Navigation}\label{site-information-navigation}}

\begin{itemize}
\tightlist
\item
  \href{https://help.nytimes.com/hc/en-us/articles/115014792127-Copyright-notice}{©~2020~The
  New York Times Company}
\end{itemize}

\begin{itemize}
\tightlist
\item
  \href{https://www.nytco.com/}{NYTCo}
\item
  \href{https://help.nytimes.com/hc/en-us/articles/115015385887-Contact-Us}{Contact
  Us}
\item
  \href{https://www.nytco.com/careers/}{Work with us}
\item
  \href{https://nytmediakit.com/}{Advertise}
\item
  \href{http://www.tbrandstudio.com/}{T Brand Studio}
\item
  \href{https://www.nytimes.com/privacy/cookie-policy\#how-do-i-manage-trackers}{Your
  Ad Choices}
\item
  \href{https://www.nytimes.com/privacy}{Privacy}
\item
  \href{https://help.nytimes.com/hc/en-us/articles/115014893428-Terms-of-service}{Terms
  of Service}
\item
  \href{https://help.nytimes.com/hc/en-us/articles/115014893968-Terms-of-sale}{Terms
  of Sale}
\item
  \href{https://spiderbites.nytimes.com}{Site Map}
\item
  \href{https://help.nytimes.com/hc/en-us}{Help}
\item
  \href{https://www.nytimes.com/subscription?campaignId=37WXW}{Subscriptions}
\end{itemize}
