Sections

SEARCH

\protect\hyperlink{site-content}{Skip to
content}\protect\hyperlink{site-index}{Skip to site index}

\href{https://www.nytimes.com/section/politics}{Politics}

\href{https://myaccount.nytimes.com/auth/login?response_type=cookie\&client_id=vi}{}

\href{https://www.nytimes.com/section/todayspaper}{Today's Paper}

\href{/section/politics}{Politics}\textbar{}Mark Esperanto? Trump
Misnames His Defense Secretary in Tweet

\url{https://nyti.ms/2J52evy}

\begin{itemize}
\item
\item
\item
\item
\item
\end{itemize}

Advertisement

\protect\hyperlink{after-top}{Continue reading the main story}

Supported by

\protect\hyperlink{after-sponsor}{Continue reading the main story}

\hypertarget{mark-esperanto-trump-misnames-his-defense-secretary-in-tweet}{%
\section{Mark Esperanto? Trump Misnames His Defense Secretary in
Tweet}\label{mark-esperanto-trump-misnames-his-defense-secretary-in-tweet}}

Even for a president who often lards his online missives with typos,
caps-lock abuses and errant exclamation points, Sunday's Twitter missive
on Syria contained an outsize number of errors.

\includegraphics{https://static01.nyt.com/images/2019/10/20/us/20dc-esperanto/20dc-esperanto-articleLarge.jpg?quality=75\&auto=webp\&disable=upscale}

\href{https://www.nytimes.com/by/katie-rogers}{\includegraphics{https://static01.nyt.com/images/2018/06/12/multimedia/author-katie-rogers/author-katie-rogers-thumbLarge-v2.png}}

By \href{https://www.nytimes.com/by/katie-rogers}{Katie Rogers}

\begin{itemize}
\item
  Oct. 20, 2019
\item
  \begin{itemize}
  \item
  \item
  \item
  \item
  \item
  \end{itemize}
\end{itemize}

WASHINGTON --- President Trump shared an update on Sunday from his
defense secretary that outlined ``minor skirmishes'' between Turkish and
Kurdish fighters in northern Syria as American troops make their way out
of the area. It might have passed by with little notice in the rushing
current of Mr. Trump's Twitter stream, but for one thing.

``Mark Esperanto, Secretary of Defense, `The ceasefire is holding up
very nicely. There are some minor skirmishes that have ended quickly,'''
Mr. Trump wrote on Twitter. ```New areas being resettled with the
Kurds.' USA soldiers are not in combat or ceasefire zones. We have
secured the Oil. Bringing soldiers home!''

Even for Mr. Trump, who often lards his online missives with typos,
caps-lock abuses,
\href{https://www.nytimes.com/2017/05/31/us/politics/covfefe-trump-twitter.html}{occasional
gibberish} and errant exclamation points, Sunday's missive contained an
outsize number of errors. The first and most glaring: The president's
defense secretary is actually named Mark \emph{Esper.}

Questions arose. Was it a typo? How could Mr. Trump's iPhone even make
the jump from ``Esper'' to ``Esperanto'' if it was an auto-correct
situation? It was a mystery that several White House officials could not
solve when asked by a reporter on Sunday.

The larger problem, of course, is that Mr. Trump made a series of false
or unsupported statements about a chaotic situation that has unfolded
since he stood by as President Recep Tayyip Erdogan of Turkey advanced
his forces into the area. In recent days, the
\href{https://www.nytimes.com/2019/10/15/world/middleeast/pence-pompeo-turkey-syria-troops.html}{vice
president traveled to Turkey} to negotiate a brief cease-fire --- a
\href{https://www.nytimes.com/2019/10/17/world/middleeast/trump-pence-syria-turkey-ceasefire.html}{nominal
sacrifice from Mr. Erdogan} that the White House has tried to frame as a
win.

The quote Mr. Trump attributed to Mr. Esper could have come from a
private conversation between them. But it appeared that it might have
been a recounting --- if not an entirely faithful one --- of public
comments made by Mr. Esper, who
\href{https://www.nytimes.com/2019/10/20/world/asia/mark-esper-afghanistan.html}{made
an unannounced visit to meet with American troops in Afghanistan} this
weekend and delivered his own assessment of what was happening in Syria.

``I think overall the cease-fire generally seems to be holding,'' Mr.
Esper said,
\href{https://twitter.com/idreesali114/status/1185911236036157440}{according
to a Reuters correspondent} traveling with him. ``We see a stabilization
of the lines, if you will, on the ground, and we do get reports of
intermittent fires, this and that, that doesn't surprise me
necessarily.''

At the end of the tweet, Mr. Trump added two confusing elements of his
own. The first was that United States had ``secured the oil,'' a claim
he has repeatedly made in recent days without any explanation. The White
House did not clarify what he meant by those remarks, and Mr. Trump has
ignored the question when asked about it by reporters. Last year, there
were about 2.5 billion barrels of oil in the fields in northern Syria,
\href{https://www.bloomberg.com/news/articles/2019-10-18/trump-baffles-with-claim-to-have-taken-control-of-mideast-oil}{according
to industry estimates}.

The president also said that the United States was ``bringing soldiers
home,'' which is also not correct, at least not in the short term: Mr.
Esper has confirmed that the troops leaving Syria are heading to Iraq,
to continue operations against the Islamic State.

Separately, the Trump administration
\href{https://www.nytimes.com/2019/10/11/world/middleeast/trump-saudi-arabia-iran-troops.html}{said
this month} that it would be committing additional troops to Saudi
Arabia, a decision the president has said was made because the Saudis
agreed to pay for the operation.

``A very rich country,'' Mr. Trump said during a news conference with
the Italian president last week. ``They should be paying. And so should
many other countries be paying if they want this kind of protection.''

Hours after the original tweet was posted to the presidential account,
the White House tried again, spelling Mr. Esper's name correctly.

Most of the other questionable assertions remained.

Advertisement

\protect\hyperlink{after-bottom}{Continue reading the main story}

\hypertarget{site-index}{%
\subsection{Site Index}\label{site-index}}

\hypertarget{site-information-navigation}{%
\subsection{Site Information
Navigation}\label{site-information-navigation}}

\begin{itemize}
\tightlist
\item
  \href{https://help.nytimes.com/hc/en-us/articles/115014792127-Copyright-notice}{©~2020~The
  New York Times Company}
\end{itemize}

\begin{itemize}
\tightlist
\item
  \href{https://www.nytco.com/}{NYTCo}
\item
  \href{https://help.nytimes.com/hc/en-us/articles/115015385887-Contact-Us}{Contact
  Us}
\item
  \href{https://www.nytco.com/careers/}{Work with us}
\item
  \href{https://nytmediakit.com/}{Advertise}
\item
  \href{http://www.tbrandstudio.com/}{T Brand Studio}
\item
  \href{https://www.nytimes.com/privacy/cookie-policy\#how-do-i-manage-trackers}{Your
  Ad Choices}
\item
  \href{https://www.nytimes.com/privacy}{Privacy}
\item
  \href{https://help.nytimes.com/hc/en-us/articles/115014893428-Terms-of-service}{Terms
  of Service}
\item
  \href{https://help.nytimes.com/hc/en-us/articles/115014893968-Terms-of-sale}{Terms
  of Sale}
\item
  \href{https://spiderbites.nytimes.com}{Site Map}
\item
  \href{https://help.nytimes.com/hc/en-us}{Help}
\item
  \href{https://www.nytimes.com/subscription?campaignId=37WXW}{Subscriptions}
\end{itemize}
