Sections

SEARCH

\protect\hyperlink{site-content}{Skip to
content}\protect\hyperlink{site-index}{Skip to site index}

\href{https://www.nytimes.com/section/world/europe}{Europe}

\href{https://myaccount.nytimes.com/auth/login?response_type=cookie\&client_id=vi}{}

\href{https://www.nytimes.com/section/todayspaper}{Today's Paper}

\href{/section/world/europe}{Europe}\textbar{}Speaker John Bercow Bows
Out, Loved and Loathed. Much Like Brexit.

\url{https://nyti.ms/2NrYRQB}

\begin{itemize}
\item
\item
\item
\item
\item
\end{itemize}

Advertisement

\protect\hyperlink{after-top}{Continue reading the main story}

Supported by

\protect\hyperlink{after-sponsor}{Continue reading the main story}

\hypertarget{speaker-john-bercow-bows-out-loved-and-loathed-much-like-brexit}{%
\section{Speaker John Bercow Bows Out, Loved and Loathed. Much Like
Brexit.}\label{speaker-john-bercow-bows-out-loved-and-loathed-much-like-brexit}}

Let the `chuntering' begin: The sheriff of Britain's Parliament has laid
down his badge. But maybe not his thesaurus.

\includegraphics{https://static01.nyt.com/images/2019/10/31/world/31bercow/merlin_160490319_d9636c90-12dc-4458-9496-4f1fcbe3f078-articleLarge.jpg?quality=75\&auto=webp\&disable=upscale}

\href{https://www.nytimes.com/by/stephen-castle}{\includegraphics{https://static01.nyt.com/images/2018/10/08/multimedia/author-stephen-castle/author-stephen-castle-thumbLarge.png}}

By \href{https://www.nytimes.com/by/stephen-castle}{Stephen Castle}

\begin{itemize}
\item
  Oct. 31, 2019
\item
  \begin{itemize}
  \item
  \item
  \item
  \item
  \item
  \end{itemize}
\end{itemize}

LONDON --- In Britain's Parliament one day this week, a rising wall of
sound echoed around the chamber as lawmakers jeered Prime Minister Boris
Johnson. Or cheered him.

The prime minister's hand jabbed the air for emphasis as he tried to be
heard above the noise.

Then came the inevitable, familiar rebuke. In booming if strangled
tones, the speaker of the House of Commons,
\href{https://www.nytimes.com/2019/01/19/world/europe/brexit-speaker-john-bercow.html}{John
Bercow,} was demanding ``Order!'' --- for almost the last time.

In a
\href{https://www.nytimes.com/2019/10/20/world/europe/speaker-parliament-john-bercow.html}{decade
as speaker that ended} on Thursday, Mr. Bercow has silenced legislators
this way almost 14,000 times,
\href{https://www.bbc.co.uk/news/uk-politics-50237401}{according to one
analysis}, as well as chiding politicians in famously antiquated
language for ``chuntering from a sedentary position'' (talking while
seated). Sometimes he likened them to mischievous children (``Be a good
boy, young man!'').

With Brexit thrusting Parliament onto center stage, Mr. Bercow's love of
the limelight made him a celebrity, the star of a painful political
drama that at one point drew
\href{https://www.radiotimes.com/news/tv/2019-09-04/bbc-parliament-hits-all-time-rating-high-with-boris-johnsons-brexit-defeat/}{more
than 1.5 million viewers} to the niche Parliament channel on the BBC.

But Mr. Bercow was no mere showman. Some believe he played a unique role
in the modern history of Parliament, and --- for good or ill, depending
on their viewpoint --- in the blighted course of the campaign to pull
Britain out of the European Union.

The speaker does not just preside over debates, but is the ultimate
arbiter of parliamentary rules. That gave Mr. Bercow huge influence at a
time when the government had no majority in the House of Commons.

His decision to allow opposition and rebel Conservative lawmakers to
pass legislation to prevent a disorderly British departure from the
European Union --- that is, one without a formal agreement --- stretched
the rules of the country's unwritten constitution. Philip Hammond, a
former chancellor of the Exchequer who backed the legislation, said Mr.
Bercow's ruling was ``crucial in preventing a no-deal Brexit on Oct.
31.''

In making the decision, Mr. Bercow followed his guiding principles and
championed the rights of backbenchers --- lawmakers who are not part of
the government and who normally have little or no real influence.

\includegraphics{https://static01.nyt.com/images/2019/10/31/world/31bercow2/merlin_161393247_b5426746-41a3-46d1-852b-7a01ee13b1fa-articleLarge.jpg?quality=75\&auto=webp\&disable=upscale}

``You have been singularly brave,'' Pete Wishart, a member of Parliament
from the Scottish National Party, told the speaker on Thursday, praising
him for standing up to the government.

Critics take a less benign view, seeing him as a destructive force, a
pompous and partisan figure in a job that requires strict neutrality,
and so nothing less than a menace to parliamentary democracy.

Perhaps unwisely,
\href{https://www.bbc.com/news/uk-politics-38947257}{Mr. Bercow
admitted} in 2017 that he had voted against withdrawing from the
European Union in the 2016 referendum, though he insisted he was
impartial. (To detractors, an anti-Brexit bumper sticker on his wife's
car suggested otherwise.) Then there was his public opposition to the
idea of President Trump addressing both houses of Parliament. (Unlike
the House speaker in the United States, the British speaker must
renounce his or her party affiliation.)

``I don't think he has been a good speaker,'' said Vernon Bogdanor,
professor of government at King's College, London. ``He has tainted the
system.''

Mr. Bogdanor praised Mr. Bercow's efforts to ``change the equilibrium in
the Commons'' and give backbenchers more power --- but only to a point.

``That is a good thing in itself, but his decisions seem to have favored
remainers,'' he said referring to opponents of Brexit. ``The speaker
should be absolutely neutral.''

That is one of the requirements of one of the most prestigious jobs in
the country, a role that dates back more than 600 years and commands a
salary higher than that taken by Prime Minister Johnson. It comes with
perks like an apartment in the shadow of Big Ben and an entertainment
allowance.

How history will judge Mr. Bercow's reign is unclear. Certainly, he will
go down as one of the more colorful occupants of the post.

Image

Mr. Bercow at the State Opening of Parliament in 2010.Credit...Pool
photo by Johnny Green

The son of a cabdriver, he is an outsider who made his way to the top of
the British establishment. He is so hated by some fellow Conservatives,
who see him as a traitor and as too sympathetic to the opposition, it
can be hard to forget that he was once on the extreme right of his
party. (Reminded of that on Thursday, he buried his head in his hands.)

At times, he was mocked for his stature (he is 5 feet 6½ inches), but if
he ever found this wounding, he appears to have gotten over it. On
Thursday, when one lawmaker, Alan Duncan, confessed in an affectionate
speech to having referred to him as ``Mr. Speaker Hobbit,'' Mr. Bercow
responded, ``I would gently point out that a hobbit is a friendly
creature.''

Mr. Bercow won the job at a time when Parliament was reeling from a
scandal over the expenses claimed by lawmakers. But while he helped turn
the page on that episode, he leaves while faith in Parliament is at a
low ebb because of Brexit. Last year there were
\href{https://www.bbc.co.uk/news/uk-politics-45874284}{calls for his
resignation} following a report about bullying and harassment in
Parliament. Mr. Bercow also faced personal accusations of bullying,
which he denied.

Of the nine lawmakers who hope to succeed him in an election scheduled
for Monday, most say they would do things a little differently, and stay
farther from the spotlight.

One candidate, Shailesh Vara, a Conservative, called Mr. Bercow a
``playground bully'' and said he had ``tarnished the role of speaker.''

Image

Mr. Bercow outside Parliament on his last day as speaker.Credit...Stefan
Rousseau/Press Association, via Associated Press

But as Mr. Bercow left Thursday, the focus was more on his zeal for
reform, his efforts to make Parliament more open and family friendly and
his support of L.G.B.T.Q. rights. One lawmaker also praised the strength
of Mr. Bercow's bladder and his ability to sit through endless debates.

A somewhat more elevated tribute was paid by John Hayes, a veteran
Conservative lawmaker. He described Mr. Bercow as ``indefatigable,
irrepressible, incomparable,'' telling him that he had ``brought theater
to this place and life and art to your role.''

That brought a response rarely heard over the past decade from the
speaker.

``I am almost beyond words,'' he said.

Advertisement

\protect\hyperlink{after-bottom}{Continue reading the main story}

\hypertarget{site-index}{%
\subsection{Site Index}\label{site-index}}

\hypertarget{site-information-navigation}{%
\subsection{Site Information
Navigation}\label{site-information-navigation}}

\begin{itemize}
\tightlist
\item
  \href{https://help.nytimes.com/hc/en-us/articles/115014792127-Copyright-notice}{©~2020~The
  New York Times Company}
\end{itemize}

\begin{itemize}
\tightlist
\item
  \href{https://www.nytco.com/}{NYTCo}
\item
  \href{https://help.nytimes.com/hc/en-us/articles/115015385887-Contact-Us}{Contact
  Us}
\item
  \href{https://www.nytco.com/careers/}{Work with us}
\item
  \href{https://nytmediakit.com/}{Advertise}
\item
  \href{http://www.tbrandstudio.com/}{T Brand Studio}
\item
  \href{https://www.nytimes.com/privacy/cookie-policy\#how-do-i-manage-trackers}{Your
  Ad Choices}
\item
  \href{https://www.nytimes.com/privacy}{Privacy}
\item
  \href{https://help.nytimes.com/hc/en-us/articles/115014893428-Terms-of-service}{Terms
  of Service}
\item
  \href{https://help.nytimes.com/hc/en-us/articles/115014893968-Terms-of-sale}{Terms
  of Sale}
\item
  \href{https://spiderbites.nytimes.com}{Site Map}
\item
  \href{https://help.nytimes.com/hc/en-us}{Help}
\item
  \href{https://www.nytimes.com/subscription?campaignId=37WXW}{Subscriptions}
\end{itemize}
