Sections

SEARCH

\protect\hyperlink{site-content}{Skip to
content}\protect\hyperlink{site-index}{Skip to site index}

\href{https://www.nytimes.com/section/politics}{Politics}

\href{https://myaccount.nytimes.com/auth/login?response_type=cookie\&client_id=vi}{}

\href{https://www.nytimes.com/section/todayspaper}{Today's Paper}

\href{/section/politics}{Politics}\textbar{}The Many Ways That Joe Biden
Trips Over His Own Tongue

\url{https://nyti.ms/2MZmEZh}

\begin{itemize}
\item
\item
\item
\item
\item
\item
\end{itemize}

\begin{itemize}
\item
  \href{https://www.nytimes.com/2020/07/31/us/elections/biden-vs-trump.html?action=click\&pgtype=Article\&state=default\&region=TOP_BANNER\&context=storylines_menu}{Election
  Updates}
\item
  \href{https://www.nytimes.com/article/biden-vice-president-2020.html?action=click\&pgtype=Article\&state=default\&region=TOP_BANNER\&context=storylines_menu}{Biden's
  V.P. Search}
\item
  \href{https://www.nytimes.com/interactive/2020/07/24/us/politics/trump-biden-campaign-donors.html?action=click\&pgtype=Article\&state=default\&region=TOP_BANNER\&context=storylines_menu}{Map
  of Donations}
\item
  \href{https://www.nytimes.com/interactive/2020/us/elections/delegate-count-primary-results.html?action=click\&pgtype=Article\&state=default\&region=TOP_BANNER\&context=storylines_menu}{Delegate
  Count}
\item
  \href{https://www.nytimes.com/interactive/2019/us/politics/2020-presidential-candidates.html?action=click\&pgtype=Article\&state=default\&region=TOP_BANNER\&context=storylines_menu}{The
  Candidates}
\item
  \href{https://www.nytimes.com/newsletters/politics?action=click\&pgtype=Article\&state=default\&region=TOP_BANNER\&context=storylines_menu}{Politics
  Newsletter}
\end{itemize}

Advertisement

\protect\hyperlink{after-top}{Continue reading the main story}

Supported by

\protect\hyperlink{after-sponsor}{Continue reading the main story}

\hypertarget{the-many-ways-that-joe-biden-trips-over-his-own-tongue}{%
\section{The Many Ways That Joe Biden Trips Over His Own
Tongue}\label{the-many-ways-that-joe-biden-trips-over-his-own-tongue}}

Digressions. Mix-ups. Mistakes. Joseph R. Biden Jr. has a choppy
speaking style that could undermine his message at a time when he needs
to attract more voters and donors.

\includegraphics{https://static01.nyt.com/images/2019/10/26/us/politics/26biden-speech/merlin_163208604_a1dbac67-801c-429f-8a8e-dc621e618946-articleLarge.jpg?quality=75\&auto=webp\&disable=upscale}

By \href{https://www.nytimes.com/by/katie-glueck}{Katie Glueck}

\begin{itemize}
\item
  Published Oct. 30, 2019Updated Dec. 20, 2019
\item
  \begin{itemize}
  \item
  \item
  \item
  \item
  \item
  \item
  \end{itemize}
\end{itemize}

WEST POINT, Iowa ---
\href{https://www.nytimes.com/2019/12/20/podcasts/the-daily/joe-biden-2020.html}{Joseph
R. Biden Jr.} was making an impassioned case for protecting undocumented
immigrants one recent Sunday when he abruptly stopped himself.

``There's many more things, but ---'' he said before trailing off.

Minutes later, Mr. Biden interrupted himself again.

``So there's a, there's --- my time up?'' he said, echoing
\href{https://www.nytimes.com/2019/06/27/us/politics/kamala-harris-joe-biden-busing.html?module=inline}{a
line he had used} when he stumbled in the first presidential debate this
year. ``I guess not. I guess it is.''

And as he spoke to reporters here last week about President Trump's
freezing of military
\href{https://www.nytimes.com/2019/10/23/us/politics/ukraine-aid-freeze-impeachment.html}{aid
to Ukraine}, he briefly fumbled his words.

``People are being killed in western, in eastern Afghan --- excuse me,
in eastern, uh, Ukraine,'' he said.

Six months into his presidential campaign, Mr. Biden is still delivering
uneven performances on the debate stage and on the campaign trail in
ways that can undermine his message. He takes circuitous routes to the
ends of sentences, if he finishes them at all. He sometimes says the
opposite of what he means (``I would eliminate the capital gains tax ---
I would raise the capital gains tax'' he said in this month's debate).
He has mixed up countries, cities and dates, embarked on off-message
asides and sometimes he simply cuts himself off.

That choppy speaking style puts Mr. Biden at a disadvantage as his
front-runner status erodes and he confronts growing pressure to expand
his appeal with voters and donors. He faces intensifying competition for
moderate support, a formidable liberal foe in
\href{https://www.nytimes.com/interactive/2020/us/elections/elizabeth-warren.html}{Elizabeth
Warren}, attacks on
\href{https://www.nytimes.com/2019/10/15/us/politics/hunter-biden-interview.html}{his
family} by Mr. Trump and Republicans, and a troubling
\href{https://www.nytimes.com/2019/10/16/us/politics/democratic-fundraising-joe-biden.html}{cash
crunch}.

At a time when he most needs to convey confidence and forcefulness, some
Democrats say, he is instead getting in his own way.

``He does not do well speaking where he isn't giving a speech,'' said
Chris Henning, the Democratic chair in Greene County, Iowa, who caucused
for Mr. Biden when he ran for president in 2008. ``He's not good in
debates and he comes across like he's stumbling around, trying to figure
out what he's going to say.''

\hypertarget{latest-updates-2020-election}{%
\section{\texorpdfstring{\href{https://www.nytimes.com/2020/07/31/us/elections/biden-vs-trump.html?action=click\&pgtype=Article\&state=default\&region=MAIN_CONTENT_1\&context=storylines_live_updates}{Latest
Updates: 2020
Election}}{Latest Updates: 2020 Election}}\label{latest-updates-2020-election}}

Updated 2020-08-01T01:26:45.732Z

\begin{itemize}
\tightlist
\item
  \href{https://www.nytimes.com/2020/07/31/us/elections/biden-vs-trump.html?action=click\&pgtype=Article\&state=default\&region=MAIN_CONTENT_1\&context=storylines_live_updates\#link-29fdff45}{Kamala
  Harris, a top vice-presidential contender, confronts double
  standards.}
\item
  \href{https://www.nytimes.com/2020/07/31/us/elections/biden-vs-trump.html?action=click\&pgtype=Article\&state=default\&region=MAIN_CONTENT_1\&context=storylines_live_updates\#link-13ec3d9c}{Karen
  Bass and Susan Rice are rising on Biden's vice-presidential
  shortlist.}
\item
  \href{https://www.nytimes.com/2020/07/31/us/elections/biden-vs-trump.html?action=click\&pgtype=Article\&state=default\&region=MAIN_CONTENT_1\&context=storylines_live_updates\#link-49e9a016}{Trump
  says Russian bounties to kill U.S. troops `never took place.'}
\end{itemize}

\href{https://www.nytimes.com/2020/07/31/us/elections/biden-vs-trump.html?action=click\&pgtype=Article\&state=default\&region=MAIN_CONTENT_1\&context=storylines_live_updates}{See
more updates}

\includegraphics{https://static01.nyt.com/images/2017/01/29/podcasts/the-daily-album-art/the-daily-album-art-articleInline-v2.jpg?quality=75\&auto=webp\&disable=upscale}

\hypertarget{listen-to-the-daily-the-candidates-joe-biden}{%
\subsubsection{Listen to `The Daily': The Candidates: Joe
Biden}\label{listen-to-the-daily-the-candidates-joe-biden}}

He built a career, and a presidential campaign, on a belief in
bipartisanship. Now critics of the candidate ask: Is political consensus
a dangerous compromise?

transcript

Back to The Daily

bars

0:00/40:36

-40:36

transcript

\hypertarget{listen-to-the-daily-the-candidates-joe-biden-1}{%
\subsection{Listen to `The Daily': The Candidates: Joe
Biden}\label{listen-to-the-daily-the-candidates-joe-biden-1}}

\hypertarget{hosted-by-michael-barbaro-produced-by-rachel-quester-and-eric-krupke-and-edited-by-paige-cowett-and-larissa-anderson}{%
\subsubsection{Hosted by Michael Barbaro, produced by Rachel Quester and
Eric Krupke, and edited by Paige Cowett and Larissa
Anderson}\label{hosted-by-michael-barbaro-produced-by-rachel-quester-and-eric-krupke-and-edited-by-paige-cowett-and-larissa-anderson}}

\hypertarget{he-built-a-career-and-a-presidential-campaign-on-a-belief-in-bipartisanship-now-critics-of-the-candidate-ask-is-political-consensus-a-dangerous-compromise}{%
\paragraph{He built a career, and a presidential campaign, on a belief
in bipartisanship. Now critics of the candidate ask: Is political
consensus a dangerous
compromise?}\label{he-built-a-career-and-a-presidential-campaign-on-a-belief-in-bipartisanship-now-critics-of-the-candidate-ask-is-political-consensus-a-dangerous-compromise}}

\begin{itemize}
\item
  astead herndon\\
  In the summer of 2003, a large crowd gathers in the state capital of
  South Carolina for a funeral of massive proportions. There's a
  horse-drawn carriage that's gliding through the street with a casket
  in the back, draped in an American flag.

  Then the casket is brought in to a large church that's ornately
  decorated with flowers and wreaths.
\item
  archived recording (dick cheney)\\
  We're here to honor the memory of a man whose life was rich in years,
  whose career was filled with accomplishments, and whose calling was to
  serve his state and his country.
\end{itemize}

astead herndon

One by one people, step up to the microphone ---

\begin{itemize}
\tightlist
\item
  archived recording (william wilkins)\\
  A man who understood the art of compromise, but never at the sacrifice
  of principle.
\end{itemize}

astead herndon

--- and praised the man's life and accomplishments.

\begin{itemize}
\tightlist
\item
  archived recording (bettis rainsford)\\
  From early childhood until the day of his death, his life was governed
  by a strong sense of responsibility to help his fellow man.
\end{itemize}

astead herndon

The man that they're there to eulogize is Senator Strom Thurmond, the
longest serving member in the history of the U.S. Senate. He was a noted
segregationist and open racist for much of his early career, including
his opposition to the early Civil Rights Act in the 1960s and his
opposition to the desegregation of schools. And so, because of that
history, which made Senator Thurmond a controversial figure throughout
his career, it's a little surprising who comes to the microphone next to
speak at his funeral.

\begin{itemize}
\tightlist
\item
  archived recording (joe biden)\\
  Strom and I shared a life in the Senate for over 30 years. We shared a
  good life there, and it made a difference.
\end{itemize}

michael barbaro

It's Senator Joe Biden.

\begin{itemize}
\tightlist
\item
  archived recording (joe biden)\\
  I disagreed deeply with Strom on the issue of civil rights and on many
  other issues. But I watched him change. We became good friends.
\end{itemize}

astead herndon

There's a lot of moments throughout Joe Biden's longstanding political
career that point to how he views the world. But this one, when he's
eulogizing Strom Thurmond, I think, is a unique insight to how he views
himself as a bridge-builder, between Republicans and Democrats, between
black communities and white communities, and sees himself as someone who
sees the best in people and can bring that out of them, even as his own
party and maybe sometimes his own supporters doubt it.

michael barbaro

From The New York Times, I'm Michael Barbaro. This is ``The Daily.''
Part 4 in our series on pivotal moments in the lives of the top four
Democratic candidates for president. Today: Joe Biden. It's Friday,
December 20.

Astead Herndon, you pointed us to this moment when Biden is at Strom
Thurmond's funeral, eulogizing him, as particularly revealing of who he
is as a candidate today. And we repeatedly invited Joe Biden to tell us
his story himself, but of the four Democratic presidential candidates
that we decided to profile, he's the only one to have declined to
participate. So where do you think that this story starts for Joe Biden?

astead herndon

Well, for Biden, I think the story starts in the 1960s, in Wilmington,
Delaware.

\begin{itemize}
\tightlist
\item
  archived recording\\
  Good evening. The Reverend Dr. Martin Luther King, 39 years old and a
  Nobel Peace Prize winner, and the leader of the nonviolent civil
  rights movement in the United States, was assassinated in Memphis
  tonight.
\end{itemize}

astead herndon

After the assassination of Martin Luther King ---

\begin{itemize}
\tightlist
\item
  archived recording\\
  The National Guard was called out in several cities to put down riots.
  One of these cities was Wilmington, Delaware.
\end{itemize}

astead herndon

Wilmington was one of the cities that experienced riots that changed the
landscape of the city forever. And those riots really built on the
racial tension that was already existing in Delaware.

\begin{itemize}
\tightlist
\item
  archived recording\\
  But now, in Wilmington, the National Guard is still on duty. And the
  governor, Charles Terry, has no plan to send it back.
\end{itemize}

astead herndon

Now, folks may not know this, but Delaware has always had a pretty
racially fraught history. The southern portions particularly have been
compared to the more Confederate South. It would not be surprising,
according to folks at the time, to see Confederate flags there. And it
was one of the cities and regions that were deeply involved in the
desegregation fights that culminated with Brown v. Board of Education.
And in those northern portions of Delaware and the suburbs of
Wilmington, you have the more liberal areas and the places that fuel the
Democratic electorate. So Wilmington is caught in between those two
worlds.

It's in that tension, it's in the context of that tension, that Joe
Biden gets involved in politics.

michael barbaro

And who is Joe Biden in this moment?

astead herndon

He was a lot of things. He was a son of Delaware and also someone who
had legitimate relationships in black communities in Wilmington. That
included longstanding friendships from his time as a lifeguard at the
black swimming pool in town, but it also included relationships with
civil rights activists, including the leaders who led some of the civil
rights protests and marches for school desegregation.

michael barbaro

Mm-hmm.

astead herndon

So when the city is going through this tumultuous period, Biden sees
those relationships as something that makes him unique in the community
and something that positions him to make change. So he decides to get
involved in politics. He moves from law to run for the city council, and
then later for the Senate in 1971.

\begin{itemize}
\tightlist
\item
  archived recording (joe biden)\\
  I'm Joe Biden, and I'm a candidate for the United States Senate.
\end{itemize}

astead herndon

And in that Senate race, he leans on those relationships to craft a new
brand of politician in the state.

\begin{itemize}
\item
  archived recording (joe biden)\\
  Do you believe politicians when they tell you something in an election
  year?
\item
  archived recording (speaker 1)\\
  No.
\item
  archived recording (speaker 2)\\
  No. Most of the time, no.
\item
  archived recording (speaker 3)\\
  No. No comment.
\item
  archived recording (joe biden)\\
  That's what we've come to.
\end{itemize}

astead herndon

It's a type of politician that is emblematic of generational change and
can tell Wilmington, I'm not like those white politicians of the past.

\begin{itemize}
\tightlist
\item
  archived recording (joe biden)\\
  Politicians have done such a job on the people that the people don't
  believe them anymore. And I'd like a shot at changing that.
\end{itemize}

astead herndon

I come from your community. I know your community. And I'll legislate in
your interest. That's his pitch to voters, that in this time when there
is legitimate tension between Wilmington and the rest of the state,
between black communities and white communities, he's someone who has
good relationships in both. And that pitch to Delaware voters worked.
Joe Biden was elected by a tiny margin, 50 to 49, and he came to the
Senate to embody that new type of politician that he sold himself as.
And as a new senator, he's trying to figure out how to navigate a
Washington that really runs on personal relationships at this time. Joe
Biden, fresh and new, is trying to figure out what he can accomplish and
also how he can serve those dual constituencies, the black and white
communities, in Delaware. And one of the issues he decides to focus on
is crime.

michael barbaro

And why crime? Why that issue?

astead herndon

So since those riots in the `60s, there had been a fear around crime in
Wilmington, some founded, some unfounded. But as you move throughout the
decade, particularly through the `70s, there is a kind of more
increasing nationwide focus on the presence of drugs ---

\begin{itemize}
\tightlist
\item
  archived recording (richard nixon)\\
  America's public enemy number one in the United States is drug abuse.
  In order to fight and defeat this enemy, it is necessary to wage a
  new, all-out offensive.
\end{itemize}

astead herndon

--- and an increasing violent crime rate. That is kind of a whisper in
the `70s that grows to a full-blown chorus by the `80s.

\begin{itemize}
\tightlist
\item
  archived recording\\
  It is a war. Cops against gangs. Gangs against cops. Compared to this
  time last year, the overall crime rate is up by 11 percent. Nearly
  3,000 people killed and 15,000 wounded since 1980. Auto theft up by
  almost 18 percent. Whole neighborhoods of Los Angeles live in fear.
  Many police departments say they're caught in the middle, between
  budget cutbacks, manpower shortages, and what appears to be a national
  crime epidemic.
\end{itemize}

astead herndon

There was a national panic around drugs and drug dealing. There was a
national panic around violent crime. And this crosses racial lines.

\begin{itemize}
\tightlist
\item
  archived recording (joseph riley)\\
  It is by far the most critical problem in the cities of America, large
  and small.
\end{itemize}

astead herndon

Both white and black leaders were seeing their communities upended ---

\begin{itemize}
\tightlist
\item
  archived recording (deborah prothrow-stith)\\
  I'm not talking about heart disease, sickle cell anemia, high blood
  pressure. I'm a physician, but I'm talking about homicide, the leading
  cause of death for young black men.
\end{itemize}

astead herndon

--- seeing their communities really ravaged ---

\begin{itemize}
\tightlist
\item
  archived recording (joseph riley)\\
  They're killing our people. They're destroying our neighborhoods.
  They're eroding our social fabric. They're crippling our cities.
\end{itemize}

astead herndon

--- and were looking for the federal government to intervene and do
something about it.

So with this issue that cuts across race, Biden sees a political
opportunity for himself and for the Democratic Party. For himself, he
sees a chance to really hone in on an issue that can appease both black
and white communities and insulate himself for what was going to be a
tough re-election in the Senate. And for the Democratic Party, he thinks
he can change the reputation that Democrats have as being soft on crime.
He sees focusing on this issue as an opportunity to broaden the
Democrats' national appeal and actually become the leaders on reforming
the criminal justice system.

michael barbaro

So what does he actually do, now that he's landed on this issue as his
focus?

astead herndon

So Biden works his way onto the most important committee that focuses on
this issue --- the Senate Judiciary Committee. And for him to accomplish
anything, he knows that he needs to have working relationships with
Republicans, who, at this point, are in the majority and control the
Senate. And the number one person who could impact Biden's ability to
pass legislation on the Senate Judiciary Committee is its chairman.

\begin{itemize}
\tightlist
\item
  archived recording (strom thurmond)\\
  The committee will come to order.
\end{itemize}

astead herndon

Senator Strom Thurmond of South Carolina.

\begin{itemize}
\tightlist
\item
  archived recording (strom thurmond)\\
  Unfortunately, the state of our criminal justice today favors the
  criminal.
\end{itemize}

astead herndon

Strom Thurmond, like many conservative Republicans at the time, has that
law and order streak ---

\begin{itemize}
\tightlist
\item
  archived recording (strom thurmond)\\
  Our public safety officers are standing as a thin blue line,
  sheltering us from criminal anarchy.
\end{itemize}

astead herndon

--- and has always thought that the way to kind of combat lawlessness
was through the expansion of the prison system.

\begin{itemize}
\tightlist
\item
  archived recording (strom thurmond)\\
  Today, the criminal has four chances in five never to be arrested. A
  person arrested has five chances out of six not to serve time in
  prison. Only about one criminal in 30 ends up behind bars.
\end{itemize}

astead herndon

But let's remember, Thurmond's approach isn't that unique in this era.
Because of that national panic around crime and drugs, it's not just
conservatives who have that law and order mindset like Thurmond. But
Democrats are coming around to that idea, too. And Biden is one of those
people. So while Biden and Thurmond had different rhetoric, came from a
different civil rights background, there is an agreement about the
direction the criminal justice system needs to go. He agrees with
Thurmond that a more punitive approach is necessary. And so Biden and
Thurmond together start working on crime legislation.

michael barbaro

So I get that for any Democrat to get anything done while they're in the
minority, they need to work with Republicans. But I'm wondering how
Biden, someone who thinks of himself and talks about himself as a civil
rights champion, thinks that this partnership with this particular
Republican could end up being good for him, given Strom Thurmond's
well-known reputation on race.

astead herndon

Well, Biden has an incentive to grow his stature on Capitol Hill. That
includes relationships with Republicans and most specifically, it
requires him to have a working relationship with his partner on this
important committee. But Biden is also making a lane for himself. He
sees this as an opportunity for Democrats to make inroads on a very
specific issue. So he's willing to have this relationship with someone
whose reputation might be controversial, because it is helpful for him.
But let's remember that Strom Thurmond gets something out of this, also.
Instead of these issues being seen as completely partisan, or only being
helmed by someone who has a checkered reputation on race, a sordid
reputation on race, he now has a new face for the legislation. There is
a civil rights lawyer from Delaware, someone with a good record in black
communities, who can allow the legislation to move in a way that it
probably wouldn't have if it was just linked with the stench of Strom
Thurmond's racial history. So for both men, it's a marriage of
convenience. And this kind of partnership, it's also just the way it
worked back then. People had relationships because of votes, but also
the collegiality, the old boys' club-ness of it all. That's just the way
the Senate was.

michael barbaro

So how did they approach this legislation once they decide that they are
going to work together?

astead herndon

Well, they go big.

They don't just try some incremental change to criminal justice. They
propose something that is sweeping and bold. Something that would
probably be the most significant overhaul of the criminal justice system
in decades. In 1982, they proposed legislation that would target almost
every area of the criminal justice system. It would limit access to bail
and parole for those who had been arrested. It would create much tougher
sentences for those who are convicted of crimes. And it would just
overall expand the government's ability to pursue the war on drugs. And
the bill passes the Senate by a huge margin, 95 to 1.

michael barbaro

Wow.

astead herndon

And so with this legislation, Biden is able to bring the Democrats along
with him on what has typically been seen as a conservative approach to
the criminal justice issue.

michael barbaro

So the Democrats, following Joe Biden's lead, are now fully embracing
this law and order legislative agenda?

astead herndon

Yeah. And it shows the power of the relationship between Joe Biden and
Strom Thurmond. So from there, the bill heads to President Reagan's
desk, who campaigned for the Oval Office on the tough-on-crime agenda.

\begin{itemize}
\tightlist
\item
  archived recording (ronald reagan)\\
  We live in the midst of a crime epidemic that took the lives of more
  than 22,000 people last year. Many of you have written to me how
  afraid you are to walk the streets alone at night. We must make
  America safe again, especially for women and elderly, who face so many
  moments of fear.
\end{itemize}

astead herndon

So Biden and Thurman feel confident in the president's signature. But
Reagan vetoes it. He thinks that some of the measures are just too much
of a federal government intrusion into the criminal justice space.

michael barbaro

So this is a big defeat.

astead herndon

Well, on one hand, it is. The president killed their big bill, their
sweeping overhaul of criminal justice. But on the other hand, it is a
real testament to their partnership and what they can achieve by
reaching across the aisle. And it signals a real path forward for Biden.
It shows that through building bridges, he can bend the Senate to his
will.

michael barbaro

We'll be right back.

So after this 1982 bill fails, but with this partnership very well
established, how do Biden and Thurmond move their agenda forward?

astead herndon

So even though their big legislation fails, they know they have support
in Congress for the idea.

\begin{itemize}
\tightlist
\item
  archived recording (joe biden)\\
  We said, now let's look at everything we can agree upon and put it on
  this side of the table. Let's take everything we disagree upon and put
  it on this side of the table. And we added up all that we agreed upon.
  And we agreed upon 90 percent of the changes that had to take place.
  Probably 95 percent.
\end{itemize}

astead herndon

So they try to pass each of the major planks of the legislation. They
just do it in separate parts.

michael barbaro

So having failed to do it all in one big package, they try to do these
same reforms piecemeal?

astead herndon

Right. And they're successful at it.

They start with mandatory minimums ---

\begin{itemize}
\tightlist
\item
  archived recording (joe biden)\\
  You get caught, you go to jail.
\end{itemize}

astead herndon

--- which places a baseline amount of time that someone has to spend in
prison for a drug crime.

\begin{itemize}
\tightlist
\item
  archived recording (joe biden)\\
  Where we don't allow judges' discretion to sentence people.
\end{itemize}

astead herndon

They also create a sentencing disparity between crack and powder
cocaine, which are the same drug. Just one is cheaper and more widely
available. It meant that people who were caught using crack, the more
street-level version, were treated more harshly by the criminal justice
system than people who were caught using powder cocaine.

\begin{itemize}
\tightlist
\item
  archived recording (joe biden)\\
  If you have a piece of crack cocaine, no bigger than this quarter that
  I'm holding in my hand, one quarter of one dollar, you go to jail for
  five years. You get no probation. Judge doesn't have a choice.
\end{itemize}

astead herndon

And then they keep going.

\begin{itemize}
\tightlist
\item
  archived recording (joe biden)\\
  A number of other severe penalties.
\end{itemize}

astead herndon

They put something in place which is called civil asset forfeiture.

\begin{itemize}
\tightlist
\item
  archived recording (joe biden)\\
  If you are arrested and you are a drug dealer, the government can take
  everything you own.
\end{itemize}

astead herndon

And what it means is that the government can take your property if they
suspect that you've used it while committing a crime.

\begin{itemize}
\tightlist
\item
  archived recording (joe biden)\\
  Everything from your car to your house, your bank account. They can
  take everything.
\end{itemize}

astead herndon

And most dramatically ---

\begin{itemize}
\tightlist
\item
  archived recording (joe biden)\\
  We've gone from there all the way up to saying --- under the
  leadership of Senator Thurmond, and I'd like to suggest that I take
  some small credit for it myself, as well --- that there is now a death
  penalty.
\end{itemize}

astead herndon

They reinstate the death penalty on the federal level. And the
legislation specifies that it can be applied to drug trafficking.

\begin{itemize}
\tightlist
\item
  archived recording (joe biden)\\
  If you are a major drug dealer involved in the trafficking of drugs
  and murder results from your activities, you go to death.
\end{itemize}

astead herndon

So through pieces of small legislation, they accomplish the overall
overhaul that they initially set out to do. And Joe Biden has
successfully changed the Democratic Party's reputation on the issue of
crime.

\begin{itemize}
\tightlist
\item
  archived recording\\
  The truth is, every major crime bill since 1976 that's come out of
  this Congress, every minor crime bill, has had the name of the
  Democratic senator from the state of Delaware, Joe Biden, on that
  bill, and has had a majority vote of the Democratic members of the
  United States Senate on the bill.
\end{itemize}

astead herndon

By the `90s, you have a Democratic Party that has completely shifted on
criminal justice.

\begin{itemize}
\tightlist
\item
  archived recording (bill clinton)\\
  George Bush talks a good game. But he has no game plan.
\end{itemize}

astead herndon

And the biggest evidence for that shift is Bill Clinton, the Democratic
nominee in 1992.

\begin{itemize}
\tightlist
\item
  archived recording (bill clinton)\\
  He won't streamline the federal government and change the way it
  works. Cut 100,000 bureaucrats and put 100,000 new police officers on
  the streets of American cities. But I will.
\end{itemize}

astead herndon

He is saying, I'm going to legislate kind of tough on crime.

\begin{itemize}
\tightlist
\item
  archived recording (bill clinton)\\
  He's talked a lot about drugs, but he hasn't helped people on the
  front line to wage that war on drugs and crime, but I will.
\end{itemize}

astead herndon

And those aren't just empty words. It is backed by a series of
legislation, helmed by Biden, which give Democrats real credence to say
that we are now the tough-on-crime party.

\begin{itemize}
\tightlist
\item
  archived recording\\
  Members of Congress, I have the high privilege and the distinct honor
  of presenting to you the President of the United States.
\end{itemize}

astead herndon

And once Clinton wins ---

\begin{itemize}
\tightlist
\item
  archived recording (bill clinton)\\
  Members of the 103rd Congress, my fellow Americans ---
\end{itemize}

astead herndon

--- he wants to deliver on that campaign promise of tough-on-crime
legislation.

\begin{itemize}
\tightlist
\item
  archived recording (bill clinton)\\
  Violent crime and the fear it provokes are crippling our society,
  limiting personal freedom, and fraying the ties that bind us.
\end{itemize}

astead herndon

And so, his natural partner in this is Joe Biden, because Biden has had
a decade's worth of practice building consensus on this issue.

michael barbaro

So Clinton is tapping Biden to follow through on Clinton's campaign
promise on criminal justice. And that's going to further the Democratic
Party's agenda to be this party that is tough on crime.

astead herndon

Exactly. And what's important about this time is that Democrats are now
in the majority. So unlike the `80s, when there was Strom Thurmond
leading the Judiciary Committee, it is now Biden at the helm. And that
gives him a unique space of power in which to operate. He is able to
implement those lessons of consensus-building between Democrats and
Republicans and apply them as the head of the Judiciary Committee. And
so with that power and with those skills, he is now able to craft the
most significant legislation of his Senate career, the 1994 crime bill.

\begin{itemize}
\tightlist
\item
  archived recording (george mitchell)\\
  A lot of people deserve credit for the passage of this bill.
\end{itemize}

astead herndon

And it passes.

\begin{itemize}
\tightlist
\item
  archived recording (george mitchell)\\
  But I think no one will disagree when I say that the one person most
  responsible for the passage of this bill is Senator Biden.
\end{itemize}

astead herndon

With big bipartisan support.

\begin{itemize}
\item
  archived recording (george mitchell)\\
  Joe Biden is both the most underrated legislator in the Senate and the
  most effective legislator in the Senate.
\item
  archived recording (joe biden)\\
  That's great, George, thank you. I hope my mom was listening.
\end{itemize}

astead herndon

It provides billions in funding to increase the amount of police
officers on the street, to build new prisons in states, and it
incentivizes states to create harsher sentences on drug crimes. But it
does include some measures that are more progressive, that try to stop
people from committing crimes in the first place. It includes money for
alternative measures that aren't prison.

\begin{itemize}
\tightlist
\item
  archived recording (joe biden)\\
  The thing that has meant more to me than anything I have done in 22
  years in the United States Senate ---
\end{itemize}

astead herndon

It also includes the Violence Against Women Act ---

\begin{itemize}
\tightlist
\item
  archived recording (joe biden)\\
  I can't tell you how much it truly will make a difference in the lives
  of women who are being abused and battered in this country.
\end{itemize}

astead herndon

--- which focuses on preventing domestic violence. And it includes an
assault weapons ban, a rare rebuke to the National Rifle Association.

\begin{itemize}
\tightlist
\item
  archived recording (joe biden)\\
  Because no Republican president, no president that I have served with
  in the 22 years I've been here, was willing to go out on the line and
  say, we're not going to have a bill unless there is the gun ban in the
  bill for assault weapons.
\end{itemize}

astead herndon

But let's be clear. While there are some progressive measures, this is a
continuation of that tough-on-crime approach we saw from Biden in the
`80s.

\begin{itemize}
\tightlist
\item
  archived recording (joe biden)\\
  There's a lot of reasons, I think, for the American people to breathe
  a little sigh of relief today.
\end{itemize}

michael barbaro

So what does this moment represent for Biden?

astead herndon

This 1994 bill is the political culmination of what Biden set out to do
in Washington. It's now that Biden has solidified himself as the
bridge-builder between Republicans and Democrats in the Senate,
particularly on criminal justice. And Biden's no longer reliant on a
Republican like Strom Thurmond to get this legislation done. In fact,
Strom Thurmond votes against the 1994 crime bill, citing some of those
progressive measures. But Biden is able to get it passed anyway, because
he's moved the Democratic Party along with him, and because he has his
own relationships with Republicans to be able to win over some of those
votes. It's a full-circle political moment from where Biden started in
the `70s. He is no longer learning from some of the Senate wheelers and
dealers of the past. This is now Biden's political brand.

michael barbaro

So this is a major accomplishment. And it's clear evidence that Biden
can bridge the parties in Washington. But you also told us that Biden's
focus on criminal justice was also about this desire to serve both the
black and white communities in Delaware. So did that work?

astead herndon

Depends on how you slice it. Politically, it worked well. He keeps
getting re-elected. And he does so with significant support in both
black and white communities in Delaware. Tons of people love him. But
that is not universal. There were definitely people, as early as the
`80s, who were saying that this criminal justice overhaul that was led
by Biden would have particularly devastating effects in black
communities.

\begin{itemize}
\tightlist
\item
  archived recording (jesse jackson)\\
  Reviving the death penalty, spending several billion dollars on
  prisons and longer sentences is not the answer to reducing crime. It's
  settling disproportionately on the poor, on the black. We must break
  the cycle.
\end{itemize}

astead herndon

And now, we have a lot of evidence that those people have been proven
correct.

\begin{itemize}
\tightlist
\item
  archived recording\\
  The U.S. has the world's largest prison population, more than two
  million people behind bars.
\end{itemize}

astead herndon

You can't overstate what the war on drugs did to black communities.

\begin{itemize}
\tightlist
\item
  archived recording\\
  We've got a mass incarceration epidemic in this country. More than 2.2
  million disproportionately black, Latino, non-violent drug offenders.
\end{itemize}

astead herndon

There are 10 times more people in jail for drug offenses by 2017 than
there were in 1980.

\begin{itemize}
\tightlist
\item
  archived recording (dan lungren)\\
  Certainly one of the sad ironies in this entire episode is that a bill
  which was characterized by some as a response to the crack epidemic in
  African-American communities has led to racial sentencing disparities
  which simply cannot be ignored in any reasoned discussion of this
  issue.
\end{itemize}

astead herndon

The longer sentences for crack disproportionately hurt people of color,
specifically black people. And the white drug users, who were often
arrested using cocaine, got away with shorter prison sentences for what
was essentially the same drug.

\begin{itemize}
\tightlist
\item
  archived recording\\
  We can't talk about this without talking about race and poverty. We
  know, for example, that white people in this country are 10 times more
  likely to use drugs than African-Americans. And yet
  disproportionately, African-Americans are in jail about that. The
  whole ``stop-and-frisk'' in New York City ---
\end{itemize}

astead herndon

The amount of police on the streets meant constant surveillance of
communities and report after report of police brutality.

\begin{itemize}
\tightlist
\item
  archived recording\\
  It's the kind of scene that could play out on any given day, in any
  city in America. Men in blue stopping young men of color as tensions
  rise.
\end{itemize}

astead herndon

And what that has is a real human effect on these communities. These
aren't just numbers. These are lives.

\begin{itemize}
\tightlist
\item
  archived recording\\
  Two days after a New York City grand jury cleared a white police
  officer in the chokehold death of an unarmed black man, the protests
  are growing larger and spreading across the country, including Boston
  and Chicago. And now, another New York grand jury, this one in
  Brooklyn, is about to investigate the shooting of another unarmed
  black man.
\end{itemize}

astead herndon

So families are disrupted. Community leaders are gone. And the whole
structure of government's relationship, particularly in black
communities, is forever upended.

\begin{itemize}
\item
  archived recording (crowd)\\
  I can't breathe. I can't breathe. I can't breathe. I can't breathe. I
  can't breathe.
\item
  archived recording (speaker)\\
  You may be charged with additional crimes.
\item
  archived recording (crowd)\\
  Black lives matter. Black lives matter. Black lives matter. Black
  lives matter.
\end{itemize}

astead herndon

So the impact of this legislation has been devastating, particularly for
communities of color. It's been so bad, in fact, that both Democrats and
Republicans have largely moved away from many of these positions and
agree that the measures were overly punitive. Joe Biden himself has
changed positions on a number of these issues and is now arguing the
exact opposite of the legislation that he passed in the `80s and early
`90s. He is against the death penalty. He is against mandatory minimums.
He wants to eliminate the disparity between crack and cocaine in
sentencing. But here's the thing. While he disavows the policies that
were put in place as a result of this legislation, he does not disavow
the politics that helped produced these policies. Biden sees the
bipartisanship across our relationships, the bridge-building that
produced the legislation in the `80s and `90s, as foundational to his
vision of politics. And it's a view of Washington that says what's most
valuable is bringing people together.

\begin{itemize}
\tightlist
\item
  archived recording (joe biden)\\
  The place in which I work is a majestic place. If you're there long
  enough, it has an impact on you.
\end{itemize}

astead herndon

And it's that belief that leads him to eulogize Strom Thurmond in 2003.

\begin{itemize}
\tightlist
\item
  archived recording (joe biden)\\
  This is a man who was opposed to the poll tax. This is a man who I
  watched vote for the extension of the Voting Rights Act. This is a man
  who I watched vote for the Martin Luther King holiday.
\end{itemize}

astead herndon

And when I listen to this eulogy, it strikes me that Biden's political
vision is also a personal one.

\begin{itemize}
\tightlist
\item
  archived recording (joe biden)\\
  It's really easy to say today that that was pure political expediency.
  But I choose to believe otherwise. I choose to believe that Strom
  Thurmond was doing what few do once they pass the age of 50. He was
  continuing to grow, continuing to change.
\end{itemize}

astead herndon

Different from most politicians, it's not just that he thinks
bipartisanship is important because it can make things happen in the
legislative context. He sees reaching across parties and reaching across
communities as a necessary thing to actually personally transform
people.

\begin{itemize}
\tightlist
\item
  archived recording (joe biden)\\
  You cannot, if you respect those with whom you serve, fail to
  understand how deeply they feel about things differently than you. And
  over time, I believe it has an effect on you.
\end{itemize}

astead herndon

He has chosen to believe that if you do the hard work of reaching across
the aisle, you won't just get a policy to happen, but you can make
someone better. You can transform the soul of an individual, but also of
Washington, and in turn, the country.

\begin{itemize}
\tightlist
\item
  archived recording (joe biden)\\
  If we stand together, we will win the battle for the soul of this
  nation.
\end{itemize}

astead herndon

That's why you hear him talking so much about civility in this campaign.

\begin{itemize}
\tightlist
\item
  archived recording (joe biden)\\
  There is not a single thing beyond our capacity if we stand together
  and get up and remember who we are. This is the United States of
  America. Period.
\end{itemize}

astead herndon

His slogan is, ``Restoring the soul of America.''

\begin{itemize}
\tightlist
\item
  archived recording (joe biden)\\
  We are in a battle for the soul of this nation. That's why, primarily,
  I'm running for president.
\end{itemize}

astead herndon

He is evoking an era in which consensus-building and cross-aisle
relationships were the order of the day. And the promise of his
candidacy is to bring that time back. But that is coming into conflict
with a growing wing of the party that is more concerned around ideals
than process. It is their argument that for too long, Democrats have
been concerned with reaching across the aisle to build Republican
support, and should be thinking about how to overcome them to produce
the big solutions that they desire.

michael barbaro

Right. The left wing of the Democratic Party, which is very skeptical of
Joe Biden, says you cannot separate this instinct of his, this kind of
bipartisan politics and dealmaking, however noble it is in intention,
from the policies that those politics have produced, and from their
real-world impact, which, in the case of criminal justice reforms, were
devastating. And Biden seems to be saying, actually, you can separate
them and you should separate them. Don't fixate on one bill or one
legislative partner that I had in the Senate. Focus on the tactics and
the tone. And imagine a world where those are used for whatever it is
you want to get done, because that's what it actually takes to get big
things done in Washington.

astead herndon

Right. But here's why the left disagrees. The left's evidence for their
criticism is not in the 1970s or `80s or `90s, the time in which Biden
was in the Senate. They point to the last Democratic president. They say
that Barack Obama tried to use the same type of strategies to reach out
to Republicans to try to build consensus. And in this era of
polarization, of partisanship, of divisiveness, that it didn't work.
This is the central question that the Biden candidacy is asking of
Democrats --- can the bridge-building still apply in this era? Or, with
the tone that has been set in Washington, is it more important for
Democrats to orient themselves around ideals and around making big
things happen, no matter if Republicans are included in that solution or
not? Biden chooses to believe something different. That even in this
era, even with this tone, that restoring the collegiality and
consensus-building of Washington should still be the priority of any
president. He believes that if you create a Washington that is more
civil, then that could be more transformative than any particular policy
could ever be.

michael barbaro

Astead, thank you very much.

astead herndon

Thank you.

michael barbaro

We'll be right back.

Here's what else you need to know today.

\begin{itemize}
\tightlist
\item
  archived recording (james clyburn)\\
  And until we can get some assurances from the majority leader that he
  is going to allow for a fair and impartial trial to take place, we
  would be crazy to walk in there, knowing he's set up a kangaroo court.
\end{itemize}

michael barbaro

On Thursday, just hours after impeaching President Trump, House
Democratic leaders raised the possibility of withholding the articles of
impeachment from the Senate indefinitely, in order to negotiate better
terms for a trial or avoid a trial altogether.

\begin{itemize}
\item
  archived recording (john berman)\\
  How long are you willing to wait?
\item
  archived recording (james clyburn)\\
  As long as it takes.
\end{itemize}

michael barbaro

But Democratic leaders, including Majority Whip James Clyburn on CNN,
predicted that Senate Majority Leader Mitch McConnell would hold a
rushed and biased trial that would quickly exonerate Trump without
seeking or introducing any new evidence. By not transmitting the
articles of impeachment to the Senate, the Democrats can stall a trial
for weeks or even months until they get the kind of trial that they
want.

\begin{itemize}
\tightlist
\item
  archived recording (lindsey graham)\\
  What they're proposing, to not send the articles for disposition to
  the Senate after being passed in the House, is incredibly dangerous.
\end{itemize}

michael barbaro

Senate Republicans, including Senator Lindsey Graham, a Trump ally,
called the tactic a form of legislative extortion.

\begin{itemize}
\tightlist
\item
  archived recording (lindsey graham)\\
  Just think for a moment. You pass articles of impeachment in the
  House, you refuse to send them into the Senate until the Senate
  constructs a trial of your liking as speaker of the House. We have
  separation of powers for a reason. You can't be speaker of the House
  and majority leader of the Senate at the same time.
\end{itemize}

michael barbaro

``The Daily'' is made by Theo Balcomb, Andy Mills, Lisa Tobin, Rachel
Quester, Lynsea Garrison, Annie Brown, Clare Toeniskoetter, Paige
Cowett, Michael Simon Johnson, Brad Fisher, Larissa Anderson, Wendy
Dorr, Chris Wood, Jessica Cheung, Alexandra Leigh Young, Jonathan Wolfe,
Lisa Chow, Eric Krupke, Marc Georges, Luke Vander Ploeg, Adizah Eghan,
Kelly Prime, Julia Longoria, Sindhu Gnanasambandan, Jazmín Aguilera,
M.J. Davis Lin, Austin Mitchell, Sayre Quevedo, Monika Evstatieva, Neena
Pathak, Dan Powell, and Dave Shaw. Our theme music is by Jim Rutenberg
and Ben Landsverk of Wunderlich. Special thanks to Sam Dolnick, Mikayla
Bouchard, Stella Tan, Lauren Jackson, Julia Simon, Nora Keller, Sydney
Harper and Sheryl Gay Stolberg.

That's it for ``The Daily.'' I'm Michael Barbaro. See you on Monday.

Mr. Biden's verbal miscues have long led to challenges. He was forced to
withdraw from the 1988
\href{https://www.nytimes.com/2019/06/03/us/politics/biden-1988-presidential-campaign.html}{presidential
campaign} after he presented biographical details from the life of the
British Labour Party leader Neil Kinnock as his own. His gaffe-prone
tendencies made national headlines during his 2008 presidential bid when
he referred to Barack Obama as ``the first mainstream African-American
who is articulate and bright and clean.''

But Mr. Biden can now appear less crisp, and more hesitant, than in the
past --- and also in comparison to more polished rivals in the crowded
Democratic primary. As the early leader in 2020 polls he has spent more
time in the national spotlight, and he is competing in a fast-moving
social media environment in which gaffes are magnified and candidates
are rewarded for being quick on their feet.

Mr. Biden, 76,
\href{https://www.realclearpolitics.com/epolls/2020/president/us/2020_democratic_presidential_nomination-6730.html\#polls}{still
leads}\href{https://www.cnn.com/2019/10/23/politics/cnn-poll-biden-lead-increases/index.html}{many
national polls}, and he enjoys
\href{https://www.nytimes.com/2019/05/30/us/politics/joe-biden-beau-biden-death.html}{significant
good will} from many Democratic voters. Some attendees at his events in
Iowa last week said Mr. Biden, who overcame a childhood stutter, is a
relatable raconteur whose decades of experience are comforting amid the
chaos of the Trump era. He can be forceful in denouncing the president
from the podium, and is at his best in individual conversations with
voters.

\href{https://www.nytimes.com/interactive/2020/us/elections/democratic-polls.html}{}

\includegraphics{https://static01.nyt.com/images/2020/01/09/us/democratic-polls-promo-1560481207024/democratic-polls-promo-1560481207024-articleLarge-v30.png}

\hypertarget{which-democrats-are-leading-the-2020-presidential-race}{%
\subsection{Which Democrats Are Leading the 2020 Presidential
Race?}\label{which-democrats-are-leading-the-2020-presidential-race}}

There are two Democrats running for president. Here's the latest data to
track how the candidates are doing.

And of course, if Mr. Biden wins the nomination, he will face a
president who is an undisciplined speaker in his own right --- one who
\href{https://www.nytimes.com/2019/10/20/us/politics/mark-esperanto-trump-tweet.html}{misstated
the name} of his own cabinet secretary recently, tortures grammar and
spelling in his tweets and, most significantly, routinely makes false
claims about matters large and small.

But Mr. Biden's inconsistent performances illustrate why many Democrats
remain skeptical of his candidacy: Whatever his strengths in polls ---
and the data is mixed, especially in the early-voting primary states ---
on the ground his performances are often plainly shaky.

Nowhere are the stakes higher for Mr. Biden than in Iowa, the leadoff
caucus state where Mr. Biden will return for a four-day swing on
Wednesday. Ms. Warren has tied or moved ahead of Mr. Biden in
\href{https://www.realclearpolitics.com/epolls/2020/president/ia/iowa_democratic_presidential_caucus-6731.html}{some}
polls, and party officials on the ground say Mayor
\href{https://www.nytimes.com/interactive/2020/us/elections/pete-buttigieg.html}{Pete
Buttigieg} of South Bend, Ind., appears capable of siphoning some of Mr.
Biden's centrist support. Senator
\href{https://www.nytimes.com/interactive/2020/us/elections/amy-klobuchar.html}{Amy
Klobuchar}, a moderate from neighboring Minnesota, also
\href{https://www.nytimes.com/2019/10/21/us/politics/amy-klobuchar-iowa.html}{gained
attention} after the October debate. Last week, she qualified for the
fifth debate, scheduled for next month.

While Mr. Biden's campaign has publicly sought to downplay expectations
in Iowa, it has also invested heavily in both time and resources. In a
confidential
\href{https://www.nytimes.com/2019/10/26/us/politics/joe-biden-campaign-fundraising.html}{memo}
circulated last week, the campaign manager, Greg Schultz, said that Mr.
Biden was positioned to have a ``narrow advantage'' in the state.

\includegraphics{https://static01.nyt.com/images/2019/10/28/us/politics/00biden-speech-02/merlin_163208652_65e8d271-0eef-411d-bfdb-5ceeb4d5601e-articleLarge.jpg?quality=75\&auto=webp\&disable=upscale}

``I plan on winning Iowa. I'm working like hell to win Iowa,'' Mr. Biden
told reporters here, adding, ``It could end up being a must-win; it
could not make a difference.''

In Iowa and nationally, worries about Mr. Biden's speaking style are
often intertwined with
\href{https://www.nytimes.com/2019/07/29/us/politics/joe-biden-age.html}{concerns
about his age} --- though his allies and former staffers say he has long
been prone to misspeaking.

Mr. Biden has said it is fair to raise his age, and in one of his
stronger moments at the last debate, said that ``with it comes wisdom.''
In
\href{https://www.nytimes.com/2019/07/29/us/politics/joe-biden-age.html}{a
statement} over the summer, his doctor said that he was in ``excellent
physical condition.''

But several public appearances have intensified questions about his
ability to connect in this political moment. At the September debate,
Mr. Biden responded to a question about the legacy of slavery with a
\href{https://www.nytimes.com/2019/09/12/us/politics/biden-record-player.html}{rambling
answer} that included advising the use of a record player to expose
underprivileged children to more words. At a CNN forum focused on
L.G.B.T.Q. issues this month, Mr. Biden --- who was ahead of Mr. Obama
in
\href{https://www.nytimes.com/2012/05/07/us/politics/biden-expresses-support-for-same-sex-marriages.html}{voicing
support} for same-sex marriage --- still raised eyebrows for referencing
``gay bath houses,'' while making a broader point about evolving
attitudes.

``It doesn't necessarily bother me,'' said Steve Drahozal, the
Democratic chairman in Dubuque County, Iowa, of Mr. Biden's ``halting
speaking style,'' allowing that Mr. Biden might simply be thoughtful.
``But there are a lot of voters out there who want to be inspired by a
candidate. Democrats win when we have an inspiring candidate.''

Concerns about Mr. Biden's style aren't confined to members of the
political class: Voters often raise the issue, including at his own
events.

``I like Joe, but I don't think Joe's going to get it because he has not
been doing very well with his speaking, with the debates,'' said Lisa
Kane, 62, of Keokuk, Iowa, as she waited for Mr. Biden to speak. ``Pete
would do a better job. He's younger, a great speaker, he's smart, got
the military background.''

Debbie Hunter, who stood with Ms. Kane, based her assessment on Mr.
Biden's debate performances, which many Democrats continue to see as
lackluster at best.

``He's real hard to follow,'' Ms. Hunter, 57, said. ``I have a lot of
concern about his ability to concentrate, his attention span, the
ability to get the job done.''

But Patty Madden, 69, and Bev Alderson, 60, two former teachers, said
Mr. Biden's informal, story-laden speaking style was part of his charm.

``I love it,'' Ms. Alderson said. ``It appeals to the common people,
working class, Americans, everybody!''

``I know he falls over some of his words, we all do,'' Ms. Madden said.

``Oh, big deal!'' Ms. Alderson interjected. ``He speaks from his
heart.''

At his two events in Iowa last Wednesday, Mr. Biden spoke, relatively
carefully, from teleprompters. There were some apparent tangents --- one
promise to wrap up his address drifted into a digression on illness in
China --- but he also ended with a forceful conclusion about America's
strengths and received a standing ovation.

Mr. Biden has also proven capable of speaking crisply and movingly, and
he was a fierce competitor at the vice-presidential
\href{https://www.nytimes.com/2012/10/12/us/politics/biden-takes-off-gloves-in-vice-presidential-debate.html}{debate}
in 2012.

This year, too, he has had
\href{https://www.nytimes.com/2019/08/07/us/politics/cory-booker-speech-mother-emanuel.html}{some
electric moments}, such as when he eviscerated Mr. Trump for fanning
``the flames of white supremacy.'' And he can be sharper in exchanges
with reporters than he is on a debate stage.

``His
\href{https://www.nytimes.com/2019/10/16/us/politics/joe-biden-elizabeth-warren.html}{day
after debate press conference} have been outstanding,'' Jim Messina, Mr.
Obama's 2012 campaign manager, wrote in an email. ``He shines in that
format, as he did in 2012. It also shows he still had a fastball. The
mass format debates just aren't his strong suit. And that's probably
O.K. The field is beginning to narrow. He still has time.''

Some advisers and allies have said privately that they have little
control over Mr. Biden's speaking style, but they also insist his debate
performances have improved, and note that a string of controversies and
wobbly public appearances have hardly crushed his campaign.

``Not one expert has been right about Joe Biden,'' said former Senator
Barbara Boxer, a California Democrat and friend of Mr. Biden's. ``It
goes on and on. `He can't make it, it's just the name ID.' Then he's
going to fall out because he's
\href{https://www.nytimes.com/2019/08/29/us/politics/joe-biden-women.html}{too
affectionate}. Now he's going to fall down because he's too gaffe prone?
Now it's the money? I just don't buy it.''

In response to multiple questions for this article, a Biden campaign
spokesman, Jamal Brown, pointed to steady poll numbers in head-to-head
matchups with Mr. Trump, and to surveys that show him growing his
support among Democratic voters.

Ms. Hunter, of Keokuk, said seeing Mr. Biden in person allayed some of
her concerns and that she had moved him in her candidate rankings.

``If he wants to win these debates, he needs to speak up, get fired
up,'' she said. Asked if he had done so in the event at West Point, she
replied, ``Well, a little. At the end.''

Shane Goldmacher contributed reporting.

\hypertarget{our-2020-election-guide}{%
\section{Our 2020 Election Guide}\label{our-2020-election-guide}}

Updated July 31, 2020

\begin{itemize}
\item
  \begin{center}\rule{0.5\linewidth}{\linethickness}\end{center}

  \hypertarget{the-latest}{%
  \subsection{The Latest}\label{the-latest}}

  \begin{itemize}
  \tightlist
  \item
    President Trump's assault on the Postal Service is intersecting with
    his attacks on mail-in voting.
    \href{https://www.nytimes.com/2020/07/31/us/politics/trump-usps-mail-delays.html?action=click\&pgtype=Article\&state=default\&region=BELOW_MAIN_CONTENT\&context=storylines_guide}{Voting
    rights groups say it is a recipe for disaster.}
  \end{itemize}
\item
  \begin{center}\rule{0.5\linewidth}{\linethickness}\end{center}

  \hypertarget{bidens-vp-search}{%
  \subsection{Biden's V.P. Search}\label{bidens-vp-search}}

  \begin{itemize}
  \tightlist
  \item
    \href{https://www.nytimes.com/article/biden-vice-president-2020.html?action=click\&pgtype=Article\&state=default\&region=BELOW_MAIN_CONTENT\&context=storylines_guide}{Here
    are 13 women} who have been under consideration to be Joe Biden's
    running mate, and why each might be chosen --- and might not be.
  \end{itemize}
\item
  \begin{center}\rule{0.5\linewidth}{\linethickness}\end{center}

  \hypertarget{keep-up-with-our-coverage}{%
  \subsection{Keep Up With Our
  Coverage}\label{keep-up-with-our-coverage}}

  \begin{itemize}
  \tightlist
  \item
    Get an
    \href{https://www.nytimes.com/newsletters/politics?action=click\&pgtype=Article\&state=default\&region=BELOW_MAIN_CONTENT\&context=storylines_guide}{email}
    recapping the day's news
  \end{itemize}

  \begin{itemize}
  \tightlist
  \item
    Download our mobile app on
    \href{https://apps.apple.com/us/app/nytimes/id284862083?ls=1\&mat_click_id=5c79ae7455014fd1bd66b5610c05b8f2-20191112-16948\&referrer=mat_click_id\%3D5c79ae7455014fd1bd66b5610c05b8f2-20191112-16948\%26link_click_id\%3D722930677036718082}{iOS}
    and
    \href{http://a.localytics.com/android?id=com.nytimes.android\&referrer=utm_source\%3Dother_nyt_mobile_web\%26utm_medium\%3DWeb\%2520page\%26utm_term\%3DGeneral\%2520Mobile\%2520Page\%26utm_campaign\%3DNYT\%2520Mobile\%2520General\%2520Page}{Android}
    and turn on Breaking News and Politics alerts
  \end{itemize}
\end{itemize}

Advertisement

\protect\hyperlink{after-bottom}{Continue reading the main story}

\hypertarget{site-index}{%
\subsection{Site Index}\label{site-index}}

\hypertarget{site-information-navigation}{%
\subsection{Site Information
Navigation}\label{site-information-navigation}}

\begin{itemize}
\tightlist
\item
  \href{https://help.nytimes.com/hc/en-us/articles/115014792127-Copyright-notice}{©~2020~The
  New York Times Company}
\end{itemize}

\begin{itemize}
\tightlist
\item
  \href{https://www.nytco.com/}{NYTCo}
\item
  \href{https://help.nytimes.com/hc/en-us/articles/115015385887-Contact-Us}{Contact
  Us}
\item
  \href{https://www.nytco.com/careers/}{Work with us}
\item
  \href{https://nytmediakit.com/}{Advertise}
\item
  \href{http://www.tbrandstudio.com/}{T Brand Studio}
\item
  \href{https://www.nytimes.com/privacy/cookie-policy\#how-do-i-manage-trackers}{Your
  Ad Choices}
\item
  \href{https://www.nytimes.com/privacy}{Privacy}
\item
  \href{https://help.nytimes.com/hc/en-us/articles/115014893428-Terms-of-service}{Terms
  of Service}
\item
  \href{https://help.nytimes.com/hc/en-us/articles/115014893968-Terms-of-sale}{Terms
  of Sale}
\item
  \href{https://spiderbites.nytimes.com}{Site Map}
\item
  \href{https://help.nytimes.com/hc/en-us}{Help}
\item
  \href{https://www.nytimes.com/subscription?campaignId=37WXW}{Subscriptions}
\end{itemize}
