Sections

SEARCH

\protect\hyperlink{site-content}{Skip to
content}\protect\hyperlink{site-index}{Skip to site index}

\href{https://www.nytimes.com/section/movies}{Movies}

\href{https://myaccount.nytimes.com/auth/login?response_type=cookie\&client_id=vi}{}

\href{https://www.nytimes.com/section/todayspaper}{Today's Paper}

\href{/section/movies}{Movies}\textbar{}`Pain and Glory' Review:
Almodóvar's Dazzling Art of Self-Creation

\href{https://nyti.ms/2AHl0UW}{https://nyti.ms/2AHl0UW}

\begin{itemize}
\item
\item
\item
\item
\item
\end{itemize}

Advertisement

\protect\hyperlink{after-top}{Continue reading the main story}

Supported by

\protect\hyperlink{after-sponsor}{Continue reading the main story}

Critic's Pick

\hypertarget{pain-and-glory-review-almoduxf3vars-dazzling-art-of-self-creation}{%
\section{`Pain and Glory' Review: Almodóvar's Dazzling Art of
Self-Creation}\label{pain-and-glory-review-almoduxf3vars-dazzling-art-of-self-creation}}

Antonio Banderas plays a filmmaker in crisis in Pedro Almodóvar's movie
about loss, love, imagination and memory.

\includegraphics{https://static01.nyt.com/images/2019/10/02/arts/00painandglory-1/merlin_161549757_0e3807a1-a687-42c5-ba71-bcf0de20af8b-articleLarge.jpg?quality=75\&auto=webp\&disable=upscale}

By \href{https://www.nytimes.com/by/manohla-dargis}{Manohla Dargis}

\begin{itemize}
\item
  Oct. 3, 2019
\item
  \begin{itemize}
  \item
  \item
  \item
  \item
  \item
  \end{itemize}
\end{itemize}

\begin{itemize}
\tightlist
\item
  "Pain and Glory"\\
  **NYT Critic's Pick Directed by Pedro Almodóvar Drama R 1h 53m
\end{itemize}

\href{https://www.imdb.com/showtimes/title/tt8291806?ref_=ref_ext_NYT}{Find
Tickets}

When you purchase a ticket for an independently reviewed film through
our site, we earn an affiliate commission.

Every so often in Pedro Almodóvar's sublime ``Pain and Glory,'' Salvador
Mallo (Antonio Banderas) closes his eyes and drifts away. A celebrated
Spanish filmmaker, Salvador has lost his bearings. He's gravely
depressed, and his body seems to have permanently surrendered to his
maladies, to his bad back, migraines, asthma and fits of terrifying,
mysterious choking. When a friend offers him some heroin to smoke,
Salvador readily lights up and disappears. His nagging pains suddenly
give way to images from his childhood, idylls that brighten the screen
like beacons in a fog.

A story of memory and creation, youth and its loss, ``Pain and Glory''
circles around the idea of art as self-creation. The precipitating event
--- the thing that nudges Salvador and the movie forward --- is the
screening of an early triumph, a 1980s film called ``Sabor.'' (Its
poster is suitably Almodóvarian: a strawberry-like tongue licking its
luscious red lips.) Uneasy about the screening, Salvador reaches out to
one of its actors, Alberto (Asier Etxeandia), a debauched looker with
dangerous habits and a thing for skulls. The men haven't spoken for
years, but slip into a thorny intimacy that's almost domestic, pushing
and pulling at each other while picking at old scabs.

The screening turns into a mild farce, but it stirs something in
Salvador, lighting a small fire. The grinning face of death hangs over
``Pain and Glory,'' but it soon emerges that Salvador's most
debilitating issue is that he is a man without desire. He's alone and
hasn't made movie in a while, and a new one doesn't seem on the horizon.
Yet even as he idles, his will to create --- to dream, share stories,
make drama --- remains intact. He may not be shooting a film, but it's
telling how much his life seems like a melodrama or a comedy or even, as
in a gritty scene with slashing knives and blood, a thriller.

One of Almodóvar's talents is his transformational, near-alchemical use
of blunt ideas, how he marshals crude gestures, gaudy flourishes and
melodramatic entanglements. The emotions still sting here, and the
colors glow like traffic lights --- there are eye-popping bursts of
stop-sign red and go-go green --- and the movie is as visually striking
as any Almodóvar has made. But the narrative is elegantly structured
rather than clotted, and its tone is contemplative as opposed to
frantic, as if he had turned down the volume. A great deal happens in
``Pain and Glory,'' just not ritualistically and not at top volume. Its
agonies are tempered, its regrets hushed, its restraint powerful.

All that said, the first time you see Salvador he's at the bottom of a
cerulean-blue swimming pool in a seated position, as still and heavy as
a dropped anchor. He looks like he's meditating, but then again he might
be drowning. Whatever the case, the shot and its uncomfortable duration
(you may find yourself nervously counting off the seconds) create a
sense of mounting unease. Salvador looks so vulnerable with his
near-nakedness and arms akimbo, a vivid scar slashed across his torso.
Keep looking, and he brings to mind iconographic images of Jesus as
\href{https://www.metmuseum.org/toah/works-of-art/1982.480/}{the man of
sorrows}.

This introduction could sink a less gifted director, but Almodóvar is a
virtuoso of quicksilver changes and soon cuts to a young boy at a river
where women wash clothes and break into melodious song. Light and bright
and shimmering with beauty, it is the first in a series of scenes from
Salvador's childhood scattered throughout the movie. Taken together,
they create a wistful, emotionally vibrant counterpoint to the adult
Salvador's lonely, austere odyssey. Yet while they look like flashbacks,
they're closer to idealized reveries than to raw memories. (Asier Flores
plays Salvador when he's around 9; Penélope Cruz lights up the screen as
his mother, Jacinta.)

\includegraphics{https://static01.nyt.com/images/2019/10/02/arts/00painandglory-2/merlin_161549949_3a5c048d-3782-4e2f-9e86-5211990ab26f-articleLarge.jpg?quality=75\&auto=webp\&disable=upscale}

A genre unto himself, Almodóvar has long drawn from his own history for
his movies, most obviously with protagonists who are filmmakers. (He
calls ``Pain and Glory'' the final installment in a triptych that
includes ``Law of Desire'' and ``Bad Education.'') In ``Pain and
Glory,'' Almodóvar's home doubles for Salvador's; Banderas wears some of
the director's clothes and has similarly styled hair and beard. These
teasing biographical gestures blur the line between reality and
representation, but to see this movie as confessional would miss the
point. The point is the blur, that in-between space where art blooms.

Banderas's melancholic presence and subtle, intricate performance add
depth and intensities of feeling both because he draws so flawlessly
from Almodóvar and looks wrung out, with little of the feverish
intensity evident in even their
\href{https://www.nytimes.com/2011/10/14/movies/the-skin-i-live-in-directed-by-pedro-almodovar-review.html}{recent
collaborations}. (This is the eighth movie they've made together in the
last four decades.) With his downcast eyes, sagging posture, silences
and self-imposed isolation, Salvador looks like a man in retreat. He
would be a figure of pure pathos if it weren't clear that Salvador also
suffers from acute vanity. When a friend asks what he will do if he
doesn't make movies, he says, ``live, I suppose,'' quickly lowering and
raising his eyes, like an actor (or coquet) checking the reaction to a
killer line.

Salvador's crisis is real, but its performative quality is a relief; it
lightens the heaviness and gives you permission to laugh. ``Pain and
Glory'' can be achingly sad, but its pleasures, rainbow hues and humor
keep it (and you) aloft. For a depressed man, Salvador still puts on a
lively show, wearing splashes of color. Like his exquisitely appointed
house, his clothing reminds you --- as does Almodóvar's staging of many
conversations --- how we turn ourselves into performers, our homes into
theaters, the world into our audience. The problem with Salvador is that
somewhere along the line, as a visitor suggests, his home became a
museum. It might as well be his mausoleum.

How do you come back from the dead? For Salvador, the answer comes in
fits and starts, in the burnished images of his childhood, in an old
lover's passion, in the power of art. It also comes in his love for
Jacinta, who as an older woman (Julieta Serrano) nearing death, voices
distaste for autobiographical fiction, telling Salvador he wasn't a good
son. It's clear why: He grew up, lived his life, fell in love with a
man, became an artist. His choices were as unforgivable as they were
inescapable. But Salvador listens, and he apologizes. And then he takes
the messiness, the vibrancy and the sensuous pleasures of life as he
remembers it and turns his pain --- and hers --- into glory.

\textbf{Pain and Glory}

Rated R for language, recreational drug use and remembrance of debauches
past. In Spanish, with subtitles. Running time: 1 hour 53 minutes.

Advertisement

\protect\hyperlink{after-bottom}{Continue reading the main story}

\hypertarget{site-index}{%
\subsection{Site Index}\label{site-index}}

\hypertarget{site-information-navigation}{%
\subsection{Site Information
Navigation}\label{site-information-navigation}}

\begin{itemize}
\tightlist
\item
  \href{https://help.nytimes.com/hc/en-us/articles/115014792127-Copyright-notice}{©~2020~The
  New York Times Company}
\end{itemize}

\begin{itemize}
\tightlist
\item
  \href{https://www.nytco.com/}{NYTCo}
\item
  \href{https://help.nytimes.com/hc/en-us/articles/115015385887-Contact-Us}{Contact
  Us}
\item
  \href{https://www.nytco.com/careers/}{Work with us}
\item
  \href{https://nytmediakit.com/}{Advertise}
\item
  \href{http://www.tbrandstudio.com/}{T Brand Studio}
\item
  \href{https://www.nytimes.com/privacy/cookie-policy\#how-do-i-manage-trackers}{Your
  Ad Choices}
\item
  \href{https://www.nytimes.com/privacy}{Privacy}
\item
  \href{https://help.nytimes.com/hc/en-us/articles/115014893428-Terms-of-service}{Terms
  of Service}
\item
  \href{https://help.nytimes.com/hc/en-us/articles/115014893968-Terms-of-sale}{Terms
  of Sale}
\item
  \href{https://spiderbites.nytimes.com}{Site Map}
\item
  \href{https://help.nytimes.com/hc/en-us}{Help}
\item
  \href{https://www.nytimes.com/subscription?campaignId=37WXW}{Subscriptions}
\end{itemize}
