Sections

SEARCH

\protect\hyperlink{site-content}{Skip to
content}\protect\hyperlink{site-index}{Skip to site index}

\href{https://myaccount.nytimes.com/auth/login?response_type=cookie\&client_id=vi}{}

\href{https://www.nytimes.com/section/todayspaper}{Today's Paper}

\href{/section/upshot}{The Upshot}\textbar{}Health Proposals by Walker
and Rubio Are Less Concerned About the Poor

\url{https://nyti.ms/1J2ahlx}

\begin{itemize}
\item
\item
\item
\item
\item
\item
\end{itemize}

Advertisement

\protect\hyperlink{after-top}{Continue reading the main story}

Supported by

\protect\hyperlink{after-sponsor}{Continue reading the main story}

Upshot

Public Health

\hypertarget{health-proposals-by-walker-and-rubio-are-less-concerned-about-the-poor}{%
\section{Health Proposals by Walker and Rubio Are Less Concerned About
the
Poor}\label{health-proposals-by-walker-and-rubio-are-less-concerned-about-the-poor}}

By \href{http://www.nytimes.com/by/margot-sanger-katz}{Margot
Sanger-Katz}

\begin{itemize}
\item
  Aug. 18, 2015
\item
  \begin{itemize}
  \item
  \item
  \item
  \item
  \item
  \item
  \end{itemize}
\end{itemize}

Obamacare gives federal money to poor people to help them get health
insurance. Scott Walker, Wisconsin's governor, has a replacement plan.
It would give federal money to old people instead.

There are many, many other differences between Obamacare and Mr.
Walker's plan, announced Tuesday in a
\href{https://www.scottwalker.com/obamacare-plan/files/Day-One-Patient-Freedom-Plan.pdf}{white
paper} and a
\href{http://www.c-span.org/video/?327681-1/governor-scott-walker-rwi-remarks-health-care}{policy
speech}. But that difference points to the key similarity between the
plans, and the most consequential change.

The similarity: Both embrace the notion that the federal government
should have some responsibility for making insurance affordable for
people who don't get it through their jobs. The Walker plan, and a
similar though
\href{http://www.politico.com/magazine/story/2015/08/marco-rubio-plan-to-fix-health-care-121453.html\#.VdMjkixVikr}{vaguer
plan} announced by Senator Marco Rubio on Tuesday in an editorial in
Politico magazine, thinks that responsibility should, in part, take the
form of federal money. Both Republican presidential candidates would
give individuals tax credits they could use to shop for health plans.

But the Republican plans differ substantially from Obamacare in their
vision of how the money should be doled out, and what it should be used
to buy. The Walker and Rubio proposals call for a much less regulated
insurance market, where the federal government exercises little
oversight over the products in the market.

Their plans are also much less concerned about ensuring health care
access for the poor. In addition to rolling back Obamacare, both would
also reduce future federal spending on state-administered Medicaid
programs.

One of Obamacare's main effects has been to redistribute income. The law
taxes wages, health insurance and medical devices, and raises insurance
prices for wealthy, healthy people. It uses the money to subsidize
insurance for people who are poor or whose health history made them poor
insurance risks in the old system. As my colleague Kevin Quealy and I
wrote last year, the law has had the effect of
\href{http://www.nytimes.com/interactive/2014/10/29/upshot/obamacare-who-was-helped-most.html}{pushing
back against income inequality}. In addition to lowering the cost of
buying insurance, federal dollars also reduce the out-of-pocket costs
that low-income Americans now pay when they use those plans. But
Obamacare has had
\href{http://www.seattletimes.com/nation-world/analysis-federal-subsidies-expand-coverage-but-difficulties-persist/}{more
limited success} in attracting higher-income Americans into the new
markets.

Governor Walker's plan appears to be less generous for many poor
Americans. It would roll back the Medicaid expansion that has provided
free insurance to low-income adults. It would distribute tax credits to
those with private coverage on the basis of age, not income. Such a
system would be far simpler to administer: Every person 50 to 64 would
be given \$3,000 to spend on health insurance, while everyone 18 to 34
would get \$1,200. Older people tend to have higher health care costs,
and are charged higher insurance premiums, the argument for the
age-based subsidy system. But it means that for people without a lot to
spend on insurance, a comprehensive health plan may slip back out of
reach. For others, an affordable plan might be so bare-bones that it
wouldn't kick in before a major health catastrophe.

Wealthier people, on the other hand, could fare better under this plan,
as long as they're healthy. They would get more federal money to buy
insurance plans, and they would have the choice of buying cheaper, less
comprehensive plans than those offered under Obamacare rules.

\includegraphics{https://static01.nyt.com/images/2015/08/19/upshot/19UP-Healthcare/19UP-Healthcare-articleLarge.jpg?quality=75\&auto=webp\&disable=upscale}

The plan could make it harder for people with prior illnesses to buy
insurance. Under Obamacare, insurers have to offer the same products and
charge the same prices to customers of the same age, regardless of their
health histories. The Walker plan would offer similar protections for
people who remain insured for their entire lives. But anyone with a
major gap in coverage could later be either priced out of the insurance
market or disallowed from buying certain health plans.

Governor Walker's white paper says that states could establish special
programs for these people. But such programs in the past have proved
quite expensive. Such a mechanism would provide a financial incentive
for people to stay insured, much as Obamacare's individual mandate uses
tax penalties to keep people in the system. But for anyone who's gone
without insurance, getting back into the system would be much harder.

The Rubio plan shares some of the basics, but offers fewer details. His
editorial says people would get tax credits to buy insurance, but he
doesn't specify how those tax credits would be calculated or what would
happen to people with pre-existing health conditions.

Both plans, however, would strip away Obamacare's myriad health
insurance regulations. Among the rules that would be cut away are
popular ones, like rules preventing insurance companies from capping how
much health care they will pay for during your lifetime, or the
requirement that plans cover adult children up to age 26. They would
also eliminate less popular rules, like the requirement that everyone
obtain insurance or pay a fine and a rule that requires every health
plan to cover a fixed set of benefits, including prenatal and maternity
care.

Without all the rules, and without as many sick people in the system,
insurance would be expected to become less expensive, and perhaps more
inventive. Insurers could offer a wider array of products, including
plans that only cover certain, limited services, or products that charge
high deductibles or cap the amount of care they will pay for. Insurers
may develop innovations that appeal to customers and cost less than
what's currently offered. Plans that include health savings accounts
would be encouraged with additional federal dollars.

States would have the authority to impose insurance rules, so some
markets might have more restrictions. But the plan would also allow
people to buy insurance sold in any state, meaning all Americans would
have access to the least regulated products. The result of erasing the
rules, both campaigns suggest, is that the plan's tax credits would go
further in making insurance affordable for more middle-income people.

All of the Republican presidential candidates say they'd like to repeal
Obamacare. But any of these plans, however meritorious, can only be
accomplished through enormous disruption. Millions of people who have
obtained insurance through the law's expansion of the Medicaid program
would lose it. Millions more would most likely lose the coverage they
bought through new insurance marketplaces.

Of course, repealing Obamacare means
\href{http://www.nytimes.com/2015/03/18/upshot/where-the-republican-budget-plan-meets-reality-on-health-care.html}{more
than just getting rid of insurance regulations}. Obamacare did
\href{http://www.nytimes.com/2015/03/05/upshot/even-with-an-unfavorable-court-ruling-much-of-obamacare-would-live-on.html}{so
many other things} --- including adding a suite of major reforms to the
Medicare program for older Americans. Governor Walker's white paper
includes the word ``Medicare'' just once --- in a footnote. A Medicare
plan may well be forthcoming from him.

But the campaign's silence on this major plank of the health law is
another reminder of just how large and entrenched Obamacare has quickly
become and how difficult it will be to unravel.

Advertisement

\protect\hyperlink{after-bottom}{Continue reading the main story}

\hypertarget{site-index}{%
\subsection{Site Index}\label{site-index}}

\hypertarget{site-information-navigation}{%
\subsection{Site Information
Navigation}\label{site-information-navigation}}

\begin{itemize}
\tightlist
\item
  \href{https://help.nytimes.com/hc/en-us/articles/115014792127-Copyright-notice}{©~2020~The
  New York Times Company}
\end{itemize}

\begin{itemize}
\tightlist
\item
  \href{https://www.nytco.com/}{NYTCo}
\item
  \href{https://help.nytimes.com/hc/en-us/articles/115015385887-Contact-Us}{Contact
  Us}
\item
  \href{https://www.nytco.com/careers/}{Work with us}
\item
  \href{https://nytmediakit.com/}{Advertise}
\item
  \href{http://www.tbrandstudio.com/}{T Brand Studio}
\item
  \href{https://www.nytimes.com/privacy/cookie-policy\#how-do-i-manage-trackers}{Your
  Ad Choices}
\item
  \href{https://www.nytimes.com/privacy}{Privacy}
\item
  \href{https://help.nytimes.com/hc/en-us/articles/115014893428-Terms-of-service}{Terms
  of Service}
\item
  \href{https://help.nytimes.com/hc/en-us/articles/115014893968-Terms-of-sale}{Terms
  of Sale}
\item
  \href{https://spiderbites.nytimes.com}{Site Map}
\item
  \href{https://help.nytimes.com/hc/en-us}{Help}
\item
  \href{https://www.nytimes.com/subscription?campaignId=37WXW}{Subscriptions}
\end{itemize}
