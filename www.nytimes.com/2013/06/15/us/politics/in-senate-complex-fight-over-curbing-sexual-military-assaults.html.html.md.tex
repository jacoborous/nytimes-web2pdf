Sections

SEARCH

\protect\hyperlink{site-content}{Skip to
content}\protect\hyperlink{site-index}{Skip to site index}

\href{https://www.nytimes.com/section/politics}{Politics}

\href{https://myaccount.nytimes.com/auth/login?response_type=cookie\&client_id=vi}{}

\href{https://www.nytimes.com/section/todayspaper}{Today's Paper}

\href{/section/politics}{Politics}\textbar{}Complex Fight in Senate Over
Curbing Military Sex Assaults

\url{https://nyti.ms/12KKgpk}

\begin{itemize}
\item
\item
\item
\item
\item
\item
\end{itemize}

Advertisement

\protect\hyperlink{after-top}{Continue reading the main story}

Supported by

\protect\hyperlink{after-sponsor}{Continue reading the main story}

\hypertarget{complex-fight-in-senate-over-curbing-military-sex-assaults}{%
\section{Complex Fight in Senate Over Curbing Military Sex
Assaults}\label{complex-fight-in-senate-over-curbing-military-sex-assaults}}

\includegraphics{https://static01.nyt.com/images/2013/06/15/us/MILITARY/MILITARY-articleLarge.jpg?quality=75\&auto=webp\&disable=upscale}

By \href{https://www.nytimes.com/by/jennifer-steinhauer}{Jennifer
Steinhauer}

\begin{itemize}
\item
  June 14, 2013
\item
  \begin{itemize}
  \item
  \item
  \item
  \item
  \item
  \item
  \end{itemize}
\end{itemize}

WASHINGTON --- When Senator Carl Levin of Michigan stripped a measure
aimed at curbing sexual assault in the military out of a defense bill
this week, it was widely seen as a trampling by a long-serving male
committee chairman on female lawmakers seeking justice for victims.

But the truth reflects a more complex battle driven by legislative
competition, policy differences and the limits of identity politics in a
chamber where women's numbers and power are increasing.

The vote to replace the measure offered by Senator Kirsten E.
Gillibrand, Democrat of New York, in favor of a more modest provision
pushed by Mr. Levin, the Democrat who is chairman of the Armed Services
Committee, did not break down along gender lines: of the seven women on
the committee, three, including a fellow Democrat, Senator Claire
McCaskill of Missouri, sided with Mr. Levin. ``I think all of us need to
acknowledge that this isn't a gender issue,'' said Senator Deb Fischer,
Republican of Nebraska, at a recent hearing.

Nor was it particularly partisan. Senator Ted Cruz of Texas and Senator
David Vitter of Louisiana, two of the most conservative Republicans on
the committee, sided with Ms. Gillibrand, while seven Democrats and an
independent peeled away.

\href{http://topics.nytimes.com/top/reference/timestopics/people/g/kirsten_gillibrand/index.html?inline1=nyt-per}{Ms.
Gillibrand's} measure, which she is likely to revive on the Senate floor
this fall, would give military prosecutors rather than commanders the
power to decide which sexual assaults to try, with the goal of
increasing the number of people who report crimes without fear of
retaliation.

Mr. Levin's bill requires a senior military officer to review decisions
by commanders who decline to prosecute sexual assault cases. Although
his measure would change the current system, it would keep prosecution
of such cases within the chain of command, as the military wants.

Mr. Cruz did not buy it. ``The data indicates that there is persistent
reluctance to report sexual assault,'' he said. ``Senator Gillibrand
made an effective case.''

Unlike so many Congressional policy battles that end with an empty pot,
the search for sexual assault legislation is likely to result in
significant policy changes to military laws. All told, more than a dozen
sexual assault provisions were approved by the committee this week and
are headed for the Senate floor.

On Friday, the House passed a defense bill that contained some of the
broadest changes to military law intended to curb and more strongly
punish sexual assault. The bill would strip commanders of their
authority to dismiss a finding by a court-martial, establish minimum
sentences for sexual assault convictions, permit victims of sexual
assault to apply for a permanent change of station or unit transfer, and
ensure that convicted offenders leave the military.

Many lawmakers say commanders wield too much power on both the front and
back end of prosecutions. A Navy judge ruled this week in a pretrial
hearing that two defendants in military sexual assault cases could not
be punitively discharged because President Obama, in public remarks,
exercised ``unlawful command influence'' when he said offenders should
be ``prosecuted, stripped of their positions, court-martialed, fired,
dishonorably discharged. Period.'' The decision was first reported by
Stars and Stripes.

``Statements by high officials have always been problematic,'' said
Eugene R. Fidell, who teachers military justice at Yale Law School. ``On
the one hand, people do look to them to take a stand on things; on the
other hand, we are administrating justice here.'' Should Congress pass a
measure that would require sexual assault offenders to be dishonorably
discharged, it would not necessarily render this type of contention
moot, Mr. Fidell said, because ``those comments would have a distorting
effect on the question of conviction.''

The measures in the Senate include a mandatory review of decisions by
commanders not to prosecute sexual assault; making retaliation a crime;
and subjecting sex offenders to automatic dishonorable discharges.
Commanders would also no longer be able to overturn jury convictions
unilaterally.

``What's been lost in all this is that for the first time ever we are
going to have strong legislative changes that are going to make a real
difference in curbing sexual assault,'' said Senator Susan Collins,
Republican of Maine, a co-sponsor of Ms. Gillibrand's bill.

``This was a legitimate policy dispute that resulted in significant,
meaningful reforms.''

A few lawmakers from both parties have pursued changes to military law
to combat the problem of sexual assault in the military, to little
avail, for years.

But a recent series of events --- including startling sexual assault
data released by the Defense Department, a spate of high-profile cases
and a handful of lawmakers who have perceived that their pursuit of the
issue would be politically advantageous --- put a stronger spotlight on
the issue.

Ms. Gillibrand, who is among the most savvy of Senate Democrats in
identifying emotionally-resonating policy issues, attached herself to
the effort last August and then in March oversaw the first Senate
hearing in nearly a decade on sexual assaults in the military. Her
efforts got a boost from several committee members who joined her in
pressing members of the military every time they came to the Hill to
testify this spring, even if the topic was the defense budget.

Mr. Levin, a longtime supporter of the military, was pressured to take
action but, perhaps harboring few illusions about changing the system,
hinted early on that he would not defy Defense Secretary Chuck Hagel and
the majority of military brass who do not want the commanders' powers
removed in sexual assault cases. ``It is the chain of command that can
and must be held accountable if it fails to change a military culture,''
Mr. Levin said in one hearing.

Ms. Gillibrand, undaunted, continued to build a new coalition and fight
for the more far-reaching --- but unlikely --- measure. Other lawmakers,
seeing that her measure was headed for trouble, began to sponsor their
own.

``I think that Kirsten created an environment that allowed us to make a
whole host of reforms that, frankly, a year ago most people would have
said, `That's impossible,'~'' said Ms. McCaskill, who offered several of
her own attention-grabbing measures, and who viewed Ms. Gillibrand's
measure as an impossibility.

At a hearing a few weeks ago, Mr. Levin compiled a witness list of
people who supported keeping the prosecution of sexual assault cases
within the chain of command as Ms. Gillibrand pushed to hear from more
victims' advocates. She and others took the witness list as a sign that
Mr. Levin was stacking the deck. Last week, even as Ms. Gillibrand's
measure cleared a subcommittee, Mr. Levin informed her that he would be
offering his own amendment to replace hers.

For some of Ms. Gillibrand's allies, the committee's actions fell short.
``I wanted to be sure that women understand that they can express
themselves in cases of assault,'' said Senator Kay Hagan, Democrat of
North Carolina. ``I am worried about the consequences here.''

Advertisement

\protect\hyperlink{after-bottom}{Continue reading the main story}

\hypertarget{site-index}{%
\subsection{Site Index}\label{site-index}}

\hypertarget{site-information-navigation}{%
\subsection{Site Information
Navigation}\label{site-information-navigation}}

\begin{itemize}
\tightlist
\item
  \href{https://help.nytimes.com/hc/en-us/articles/115014792127-Copyright-notice}{©~2020~The
  New York Times Company}
\end{itemize}

\begin{itemize}
\tightlist
\item
  \href{https://www.nytco.com/}{NYTCo}
\item
  \href{https://help.nytimes.com/hc/en-us/articles/115015385887-Contact-Us}{Contact
  Us}
\item
  \href{https://www.nytco.com/careers/}{Work with us}
\item
  \href{https://nytmediakit.com/}{Advertise}
\item
  \href{http://www.tbrandstudio.com/}{T Brand Studio}
\item
  \href{https://www.nytimes.com/privacy/cookie-policy\#how-do-i-manage-trackers}{Your
  Ad Choices}
\item
  \href{https://www.nytimes.com/privacy}{Privacy}
\item
  \href{https://help.nytimes.com/hc/en-us/articles/115014893428-Terms-of-service}{Terms
  of Service}
\item
  \href{https://help.nytimes.com/hc/en-us/articles/115014893968-Terms-of-sale}{Terms
  of Sale}
\item
  \href{https://spiderbites.nytimes.com}{Site Map}
\item
  \href{https://help.nytimes.com/hc/en-us}{Help}
\item
  \href{https://www.nytimes.com/subscription?campaignId=37WXW}{Subscriptions}
\end{itemize}
