Sections

SEARCH

\protect\hyperlink{site-content}{Skip to
content}\protect\hyperlink{site-index}{Skip to site index}

\href{https://myaccount.nytimes.com/auth/login?response_type=cookie\&client_id=vi}{}

\href{https://www.nytimes.com/section/todayspaper}{Today's Paper}

\href{/section/opinion}{Opinion}\textbar{}Donald Trump's Denial of
Economic Reality

\url{https://nyti.ms/2en5wI2}

\begin{itemize}
\item
\item
\item
\item
\item
\item
\end{itemize}

Advertisement

\protect\hyperlink{after-top}{Continue reading the main story}

Supported by

\protect\hyperlink{after-sponsor}{Continue reading the main story}

\href{/section/opinion}{Opinion}

Editorial

\hypertarget{donald-trumps-denial-of-economic-reality}{%
\section{Donald Trump's Denial of Economic
Reality}\label{donald-trumps-denial-of-economic-reality}}

By
\href{http://www.nytimes.com/interactive/opinion/editorialboard.html}{The
Editorial Board}

\begin{itemize}
\item
  Nov. 4, 2016
\item
  \begin{itemize}
  \item
  \item
  \item
  \item
  \item
  \item
  \end{itemize}
\end{itemize}

\includegraphics{https://static01.nyt.com/images/2016/11/05/opinion/05sat1web/05sat1web-articleLarge.jpg?quality=75\&auto=webp\&disable=upscale}

To listen to Donald Trump describe the American economy is to hear about
a horrifying alternate reality in which the recession that started at
the \href{http://www.nber.org/cycles.html}{end of 2007} is still with
us. The truth is very different. The recovery that began in the middle
of 2009, though not perfect, has steadily created jobs and lifted wages.

More than
\href{http://beta.bls.gov/dataViewer/view/timeseries/CES0000000001}{15.2
million jobs} have been added since early 2010. The 4.9 percent
unemployment rate is half of what it was in the depths of the recession
in 2009. On Friday, in its monthly employment report, the
\href{http://www.nytimes.com/2016/11/05/business/economy/jobs-report.html?_r=0}{Department
of Labor said} that average hourly wages jumped 2.8 percent, to \$25.92,
in October from a year earlier, the biggest such increase since 2008.

Add these findings to a recent
\href{http://www.census.gov/newsroom/press-releases/2016/cb16-158.html}{report
from the Census Bureau} showing a decline in the number of people in
poverty and the percentage of people without health insurance, and it is
clear that by almost every economic measure the country is doing better
than it was just a few years ago.

Mr. Trump refuses to acknowledge these truths, presumably because he
believes that his best shot at the White House is to insist that the
economy is in terrible shape and that he alone can fix it. When the data
don't agree with his dystopian vision, he calls them
``\href{http://www.nytimes.com/2016/11/04/business/economy/unemployment-labor-department-data-politics.html}{phony
numbers}.'' His campaign called Friday's employment report
``\href{https://www.donaldjtrump.com/press-releases/statement-on-the-disastrous-jobs-report-underscores-the-total-failures-of-t}{disastrous}.''
Little wonder that 370 economists, including eight Nobel Prize winners,
have signed a
\href{http://blogs.wsj.com/economics/2016/11/01/prominent-economists-including-eight-nobel-laureates-do-not-vote-for-donald-trump/}{letter}
denouncing him for peddling ``magical thinking and conspiracy theories
over sober assessments of feasible economic policy options.''

There is no doubt that the economic recovery has not touched every
American equally. About one-quarter of the unemployed, or two million
people, have been out of work for
\href{http://www.bls.gov/news.release/pdf/empsit.pdf}{27 weeks or more}.
Some 81 percent of people between the prime working ages of 25 and 54
are in the labor force (employed or looking for work), an increase from
last year but still lower than in early 2007 when it was 83 percent. The
\href{http://www.census.gov/data/tables/time-series/demo/income-poverty/historical-poverty-people.html}{poverty
rate} fell to 13.5 percent last year, from 14.8 percent, but it is
higher than it was in the late 1990s and early 2000s. For many people,
health insurance remains too expensive.

The next president and Congress need to focus on those whom the recovery
has left behind. Hillary Clinton would increase investments in
infrastructure, help lower the burden of student loans and expand access
to affordable health care. Mr. Trump's main policy idea is to cut taxes
for the wealthy, a reprise of the old trickle-down economics that has
consistently failed. His plans to impose big tariffs on imports would
start a trade war, leading to big job losses, and his vow to repeal the
Affordable Care Act could strip as many as
\href{http://www.hhs.gov/about/news/2016/03/03/20-million-people-have-gained-health-insurance-coverage-because-affordable-care-act-new-estimates}{20
million people} of their health insurance.

Better policies could make the economy stronger. Mrs. Clinton has them.
Mr. Trump does not.

Advertisement

\protect\hyperlink{after-bottom}{Continue reading the main story}

\hypertarget{site-index}{%
\subsection{Site Index}\label{site-index}}

\hypertarget{site-information-navigation}{%
\subsection{Site Information
Navigation}\label{site-information-navigation}}

\begin{itemize}
\tightlist
\item
  \href{https://help.nytimes.com/hc/en-us/articles/115014792127-Copyright-notice}{©~2020~The
  New York Times Company}
\end{itemize}

\begin{itemize}
\tightlist
\item
  \href{https://www.nytco.com/}{NYTCo}
\item
  \href{https://help.nytimes.com/hc/en-us/articles/115015385887-Contact-Us}{Contact
  Us}
\item
  \href{https://www.nytco.com/careers/}{Work with us}
\item
  \href{https://nytmediakit.com/}{Advertise}
\item
  \href{http://www.tbrandstudio.com/}{T Brand Studio}
\item
  \href{https://www.nytimes.com/privacy/cookie-policy\#how-do-i-manage-trackers}{Your
  Ad Choices}
\item
  \href{https://www.nytimes.com/privacy}{Privacy}
\item
  \href{https://help.nytimes.com/hc/en-us/articles/115014893428-Terms-of-service}{Terms
  of Service}
\item
  \href{https://help.nytimes.com/hc/en-us/articles/115014893968-Terms-of-sale}{Terms
  of Sale}
\item
  \href{https://spiderbites.nytimes.com}{Site Map}
\item
  \href{https://help.nytimes.com/hc/en-us}{Help}
\item
  \href{https://www.nytimes.com/subscription?campaignId=37WXW}{Subscriptions}
\end{itemize}
