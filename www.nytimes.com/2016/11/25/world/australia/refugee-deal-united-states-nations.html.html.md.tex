Sections

SEARCH

\protect\hyperlink{site-content}{Skip to
content}\protect\hyperlink{site-index}{Skip to site index}

\href{https://www.nytimes.com/section/world/australia}{Australia}

\href{https://myaccount.nytimes.com/auth/login?response_type=cookie\&client_id=vi}{}

\href{https://www.nytimes.com/section/todayspaper}{Today's Paper}

\href{/section/world/australia}{Australia}\textbar{}U.N. Offers
`One-Off' Help to Australia in Resettling Refugees in U.S.

\url{https://nyti.ms/2fZkE31}

\begin{itemize}
\item
\item
\item
\item
\item
\end{itemize}

Advertisement

\protect\hyperlink{after-top}{Continue reading the main story}

Supported by

\protect\hyperlink{after-sponsor}{Continue reading the main story}

\hypertarget{un-offers-one-off-help-to-australia-in-resettling-refugees-in-us}{%
\section{U.N. Offers `One-Off' Help to Australia in Resettling Refugees
in
U.S.}\label{un-offers-one-off-help-to-australia-in-resettling-refugees-in-us}}

\includegraphics{https://static01.nyt.com/images/2016/11/26/business/26AUSTRALIA-1/26AUSTRALIA-3-articleLarge.jpg?quality=75\&auto=webp\&disable=upscale}

By Michelle Innis and
\href{http://www.nytimes.com/by/somini-sengupta}{Somini Sengupta}

\begin{itemize}
\item
  Nov. 25, 2016
\item
  \begin{itemize}
  \item
  \item
  \item
  \item
  \item
  \end{itemize}
\end{itemize}

SYDNEY, Australia --- For years, the United Nations' refugee agency told
Australia that its policy of banishing asylum seekers to remote Pacific
island detention centers was illegal.

Now, the agency is working with Australia in what both sides call an
unusual, not-to-be-replicated agreement to send some of those refugees
across the world, to be
\href{http://www.nytimes.com/2016/11/13/world/australia/australia-refugees-united-states.html}{resettled
in the United States}.

The deal, announced by Australia last week, is aimed at shutting down
two offshore detention facilities --- one on the island nation of Nauru
and the other on Manus Island in Papua New Guinea --- where hundreds of
people are housed in what rights groups describe as deplorable
conditions. The United States has agreed to take some of them; how many,
and how quickly, remains unclear.

In an interview this week, Volker Turk, an assistant high commissioner
with the United Nations' refugee agency, said his staff would help with
the screening and resettlement of refugees but only as a ``one-off'' to
allay their suffering. ``We think there is an urgent imperative to find
a humanitarian way out of this otherwise very, very, complex
conundrum,'' he said by telephone from Canberra, the Australian capital.

His comments hinted at the dilemmas that the world body can face when
countries flout international law on the rights of people fleeing war
and persecution, as the United Nations and other critics say Australia
has done.

``We do not in any way want to give the impression that we would
continue supporting such types of mechanisms,'' Mr. Turk said, referring
to Australia's offshore detention policy. ``We, all of us, are very
clear that this is a one-off, good offices, exceptional humanitarian
type of involvement because we do not believe that the future of
handling this lies in sending people to Manus Island and Nauru.''

Australia is the only country in the world that sends all seaborne
asylum seekers to other countries, where their claims for refugee status
are assessed, while refusing to let any of them settle within its
borders. The policy is meant to discourage such migrants, many of whose
voyages have ended in disaster after people smugglers pushed them out to
sea from Indonesian ports, crowded onto unseaworthy vessels.

Many of the offshore detainees are from Iran, others from Afghanistan,
Malaysia, Sri Lanka and Vietnam. The government is considering
legislation that would
\href{http://www.nytimes.com/2016/11/01/world/australia/refugee-asylum-seeker-lifetime-ban.html}{bar
them from ever visiting Australia}, regardless of where they settle.
Australia has turned back boats full of migrants and towed them out to
sea; assessed their asylum claims on boats, apparently in violation of
international law, before forcing them back; and has even been accused
of paying a human trafficker to take his passengers back to Indonesia,
where he was arrested.

Australia pays Nauru and Papua New Guinea, both impoverished nations, to
house the detainees. But Papua New Guinea recently said it would
\href{http://www.nytimes.com/2016/08/18/world/australia/manus-detention-center-papua-new-guinea.html}{close
the camp} there, which its Supreme Court
\href{http://www.nytimes.com/2016/04/27/world/australia/papua-new-guinea-asylum-seeker.html}{found
to be in violation} of its Constitution. International rights groups,
the United Nations and domestic critics
\href{http://www.nytimes.com/2016/08/04/world/australia/nauru-refugees-abuse-conditions.html}{have
excoriated} Australian officials for years over
\href{http://www.nytimes.com/2016/08/11/world/australia/nauru-asylum-seeker-refugee-abuse.html}{the
bleak conditions} in which the asylum seekers live.

Image

Volker Turk, an assistant high commissioner with the United Nations'
refugee agency.Credit...Christophe Archambault/Agence France-Presse ---
Getty Images

Neither Australia nor the United States are giving details about
\href{http://www.nytimes.com/2016/11/13/world/australia/australia-refugees-united-states.html}{the
resettlement deal}, including how many people it would involve and how
soon it would happen. Screening and security checks by the United States
authorities, involving multiple intelligence agencies, can usually take
18 to 24 months, making it difficult to imagine that any of the refugees
\href{http://www.nytimes.com/2016/11/18/world/australia/australia-us-refugee-deal.html}{will
arrive} in the United States before Donald J. Trump is sworn in as
president in January.

A spokeswoman for the Australian Department of Immigration and Border
Protection said in a written response to questions that the agency was
``not providing any more details about the arrangements.'' A spokeswoman
at the American State Department declined to say how many refugees would
be resettled in the United States, except that it would be done in
consultation with the United Nations.

The Obama administration has said the United States
\href{http://www.nytimes.com/2016/09/19/world/americas/obama-refugee-united-nations.html}{will
take in 110,000 refugees} from around the world in the current fiscal
year, which began in October, and the State Department has said that the
deal with Australia would not increase that number.

``As two of the world's largest refugee resettlement countries, the
United States and Australia share a commitment to finding long-term
solutions for the world's most vulnerable refugees,'' the State
Department spokeswoman said in an emailed response to questions.

Making the deal even more unusual, Australia has agreed to take in an
unspecified number of Central American refugees who fled gang violence
in their homelands. The United Nations says there are an estimated 2,400
such people from El Salvador, Guatemala and Honduras who have been
screened and recognized as refugees. The United States has long been
reluctant to let them apply for asylum on its territory and
\href{http://www.nytimes.com/2016/07/27/us/politics/obama-refugees-central-america.html}{only
recently agreed} to let the United Nations vet them at a processing
center in Costa Rica.

There appears to be disagreement about how many people are being held on
Manus Island and Nauru. The United Nations said it believed there were
2,200 in total, some of whom had been there, in open-ended detention,
for more than three years. It has urged the United States and Australian
governments to find a humanitarian solution for all of them.

Australia has said that about 1,600 people are housed on both islands,
including some on Nauru who live outside the detention center. Australia
declined to comment when asked why its figures differed from the United
Nations', which are somewhat larger. The government said families on
Nauru would be given priority for resettlement, followed by detainees on
Manus Island, all of whom are men.

The government is proceeding with its proposed lifetime ban on refugees
visiting Australia if they have been held at one of the camps,
regardless of where they eventually gain citizenship. The legislation is
before Australia's Parliament; though the opposition Labor Party has
said it would vote against it, it could pass with the support of a
handful of independent lawmakers in the Senate, where the government
holds a minority of seats.

Mr. Turk of the United Nations sharply criticized the bill, saying it
would divide families, since some of the detainees have relatives in
Australia.

``We do not believe in lifelong bans,'' he said. ``Family unity and
reunification is fundamental to human dignity and also for people to get
on with their lives. It is like a life sentence, never being able to
come.''

Advertisement

\protect\hyperlink{after-bottom}{Continue reading the main story}

\hypertarget{site-index}{%
\subsection{Site Index}\label{site-index}}

\hypertarget{site-information-navigation}{%
\subsection{Site Information
Navigation}\label{site-information-navigation}}

\begin{itemize}
\tightlist
\item
  \href{https://help.nytimes.com/hc/en-us/articles/115014792127-Copyright-notice}{©~2020~The
  New York Times Company}
\end{itemize}

\begin{itemize}
\tightlist
\item
  \href{https://www.nytco.com/}{NYTCo}
\item
  \href{https://help.nytimes.com/hc/en-us/articles/115015385887-Contact-Us}{Contact
  Us}
\item
  \href{https://www.nytco.com/careers/}{Work with us}
\item
  \href{https://nytmediakit.com/}{Advertise}
\item
  \href{http://www.tbrandstudio.com/}{T Brand Studio}
\item
  \href{https://www.nytimes.com/privacy/cookie-policy\#how-do-i-manage-trackers}{Your
  Ad Choices}
\item
  \href{https://www.nytimes.com/privacy}{Privacy}
\item
  \href{https://help.nytimes.com/hc/en-us/articles/115014893428-Terms-of-service}{Terms
  of Service}
\item
  \href{https://help.nytimes.com/hc/en-us/articles/115014893968-Terms-of-sale}{Terms
  of Sale}
\item
  \href{https://spiderbites.nytimes.com}{Site Map}
\item
  \href{https://help.nytimes.com/hc/en-us}{Help}
\item
  \href{https://www.nytimes.com/subscription?campaignId=37WXW}{Subscriptions}
\end{itemize}
