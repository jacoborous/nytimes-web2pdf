Sections

SEARCH

\protect\hyperlink{site-content}{Skip to
content}\protect\hyperlink{site-index}{Skip to site index}

\href{https://www.nytimes.com/section/world/americas}{Americas}

\href{https://myaccount.nytimes.com/auth/login?response_type=cookie\&client_id=vi}{}

\href{https://www.nytimes.com/section/todayspaper}{Today's Paper}

\href{/section/world/americas}{Americas}\textbar{}After Trump's Win, an
Anxious Mexico Asks: What's Next?

\url{https://nyti.ms/2eyLYFK}

\begin{itemize}
\item
\item
\item
\item
\item
\end{itemize}

Advertisement

\protect\hyperlink{after-top}{Continue reading the main story}

Supported by

\protect\hyperlink{after-sponsor}{Continue reading the main story}

\hypertarget{after-trumps-win-an-anxious-mexico-asks-whats-next}{%
\section{After Trump's Win, an Anxious Mexico Asks: What's
Next?}\label{after-trumps-win-an-anxious-mexico-asks-whats-next}}

\includegraphics{https://static01.nyt.com/images/2016/11/15/world/15Mexico3/15Mexico3-articleLarge.jpg?quality=75\&auto=webp\&disable=upscale}

By \href{http://www.nytimes.com/by/azam-ahmed}{Azam Ahmed}

\begin{itemize}
\item
  Nov. 14, 2016
\item
  \begin{itemize}
  \item
  \item
  \item
  \item
  \item
  \end{itemize}
\end{itemize}

\href{http://www.nytimes.com/es/2016/11/15/despues-de-la-victoria-de-trump-mexico-se-pregunta-ahora-que/}{Leer
en español}

MEXICO CITY --- Ever since the election of Donald J. Trump to the
American presidency, Juan Pardinas, a Mexican academic, has been
thinking back to his childhood.

Specifically, the Cold War era, when his days as a young boy were filled
with a medium-grade anxiety that the Russians might incite a nuclear war
that could devastate North America.

``It's the same feeling of uncertainty,'' said Mr. Pardinas, a graduate
of the London School of Economics whose work on anti-corruption
legislation has been roundly praised in Mexico. ``The feeling that
politics has become a source of bitterness, anguish and uncertainty is
really sad.''

Clouds have descended over Mexico, miring it in a state of
\href{http://www.nytimes.com/2016/11/10/world/americas/mexico-donald-trump-peso.html}{anguish
and paralysis} after the election of Mr. Trump to the highest office in
the world. They are clouds of uncertainty and fear, of self-doubt and
insecurity. There were even actual storm clouds hanging over the capital
in recent days, a literal echo of the nation's state of mind.

``This may not affect people on the top of our country, but it can only
mean bad news for us merchants and lower, working-class people in
Mexico,'' said Claudia Rivera, a street vendor who owns a food cart in
Mexico City.

Outside of concerns about the election, violence has been soaring to
levels not seen since the start of the drug war a decade ago. And
corruption and a loss of faith in the political leadership had already
plunged the nation into a state of gloom. Now, the loss for many is
external, too.

``A lot of people see the U.S. as a beacon of freedom, as something to
aspire to,'' said Mr. Pardinas, who works on legislation and economic
competitiveness. ``But what happens when you lose a role model, the role
model of a nation? Now all of us who admired the U.S. are having second
thoughts.''

For most Mexicans, the American election has been a
\href{http://www.nytimes.com/2016/05/23/world/americas/donald-trump-mexico.html}{grim
exercise in self-perception}. Mr. Trump, a candidate who called Mexican
immigrants ``rapists'' and criminals, vowing to deport millions and
build a wall to keep others out, has stoked long-held insecurities in
Mexico over sovereignty and
\href{http://www.nytimes.com/2016/09/01/world/americas/trump-mexico-pena-nieto-reaction.html}{respect
from its northern neighbor}. And his victory was seen by some as
validating the perception that Americans, or at least half of them, see
Mexico through a knot of stereotypes.

Never mind that Mexico's
\href{http://www.nytimes.com/2016/08/24/world/americas/mexico-culture-art-fashion.html?_r=0}{rich
culture} and cuisine, its art and film, are having a
\href{http://www.nytimes.com/interactive/2016/01/07/travel/places-to-visit.html?_r=0}{global
moment}, Mexicans say. Or that a wall between the two countries these
days might actually keep more Mexicans in the United States than out,
given the recent research showing
\href{http://www.nytimes.com/2015/11/20/us/more-mexican-immigrants-leaving-us-than-entering-report-finds.html}{more
Mexicans are returning home} than leaving to seek opportunity in
America.

``We are really in need of some reassurance,'' said Mr. Pardinas,
echoing the sentiment of dozens interviewed in the wake of Mr. Trump's
election. ``But you need political leadership for that, and we are short
on those attributes.''

President Enrique Peña Nieto and his administration have adopted a
diplomatic and hopeful posture toward Mr. Trump's presidency.

In a statement after the election, Mr. Peña Nieto said the results
``open a new chapter in the relationship between Mexico and the United
States, which will imply a change, a challenge, but also, it's necessary
to say, a big opportunity.''

He was sure, he said, that the relationship would be one of ``trust and
mutual respect'' that would ``build prosperity'' for both countries. He
also recounted that he had congratulated Mr. Trump by phone earlier and
that the men had discussed the possibility of meeting again in the
coming months ``to define, with total clarity, the course that the
relationship between the two countries will have to take.''

However, behind the scenes, there was a deep worry regarding the
transition, most immediately the possibility of mass deportations of
Mexicans living in the United States.

\includegraphics{https://static01.nyt.com/images/2016/11/15/world/15Mexico2/15Mexico2-articleLarge.jpg?quality=75\&auto=webp\&disable=upscale}

The Foreign Ministry called back all the Mexican consuls general serving
in the United States for meetings to discuss how to respond to the
incoming administration. Other consular offices issued requests for
Mexicans to report harassment or assaults, as anger stirred by Mr.
Trump's ascendance has turned into
\href{http://www.nytimes.com/2016/11/12/us/reports-of-bias-based-attacks-tick-upward-after-election.html}{racial
threats} and violence in parts of America. Meanwhile, the government has
already expressed a willingness to renegotiate parts of the North
American Free Trade Agreement.

But to some, Mr. Peña Nieto's statement seemed a
\href{http://www.nytimes.com/2016/09/01/world/americas/trump-mexico-pena-nieto-reaction.html}{missed
opportunity} to address the
\href{http://www.nytimes.com/2016/09/08/world/americas/mexico-finance-minister-luis-videgaray-resigns.html}{injury
that many Mexicans still feel} by Mr. Trump's anti-Mexican stance and
the broad concerns about his threats regarding trade between the two
nations.

Armando Ríos Piter, an opposition senator representing the state of
Guerrero, said that after enduring Mr. Trump's hostile discourse for a
year and a half, Mexicans deserved a more robust response from their
president.

``It was a very light response to a very dangerous threat,'' he said.

As Mr. Trump prepared to take office, he continued, Mexico needs to
establish its position regarding the United States wall with ``firmness,
clarity and dignity.''

Instead, ``we are left with a politically light position that doesn't
say anything,'' he said. ``We can't settle for a statement that says, `I
spoke with Trump.'''

In September, in anticipation of a possible Trump victory, Mr. Ríos
submitted bills that would strengthen Mexico's hand. The bills, which
have languished in the Senate, would allow the government to penalize
American investments in Mexico should Mr. Trump follow through on his
promises to tax or block remittances by Mexicans in the United States to
finance his proposed border wall.

The legislation would also make it explicitly illegal for the Mexican
federal government to finance anything that could be interpreted as a
border wall, and it stipulated that if the United States decided to pull
out of Nafta, as Mr. Trump has threatened, the Mexican legislature would
review the dozens of agreements and treaties that govern the bilateral
relationship.

In truth, the Mexican government is in a difficult place. Some Mexicans
say their leaders must be careful not to antagonize the new president of
the United States with their own incendiary comments, given the economic
importance America holds in Mexico.

``It is worrying and frightening to know that the loud guy holding a
stick in his hand, saying he is coming to get you, to beat you up, is
actually in power to do so now,'' said Leticia Vega, a Mexican lawyer.

Business leaders, meanwhile, have begun the process of normalizing Mr.
Trump's presidency. Though most executives have adopted a wait-and-see
approach, they are continuing with business as usual.

``Sometimes the rhetoric is very different from the actual business of
governing,'' said Alejandro Ramirez, the head of the largest business
consortium in Mexico and the chief executive of Cinépolis, which runs
movie theaters across the Americas. ``When you have to face the reality
of governing you have to look much deeper into the facts, to see whether
what you are proposing makes sense.''

Mr. Ramirez buys \$40 million worth of goods from the United States
every year to run his cinemas, from popcorn and nacho cheese to audio
equipment. If free trade were upended, those purchases might be made
from other countries, he said.

Few thought a Trump presidency was possible. Now most are banking on a
stark difference between Candidate Trump and President Trump, meaning
that he will not be as harsh on Mexico as promised. Business consortiums
and trade interest groups have taken a proactive stance on engaging the
president-elect.

``If the Mexican government is smart about this, if they anticipate
correctly the concerns of the incoming administration, they can build an
agenda to which the Trump administration can respond,'' said Duncan
Wood, the director of the Mexico Institute at the Wilson Center, which
promotes relations between the United States and Mexico through
research. ``The immediate reaction I got from board members is that this
is the moment for us to actually engage.''

For some, though, Mexico's own problems loomed larger than a Trump
presidency.

``The problems that we have generated here, in Mexico, ourselves are far
more worrisome and immediate,'' said Juan de la Vega, 42, a lawyer who
has a brother living illegally in San Francisco. ``Those are the ones I
worry about the most because they affect my life directly, like the
stagnant economy, corruption and insecurity.''

``In the grand scale of things, we as Mexicans know how to accept,
assume and transcend this Trump thing,'' he added.

Advertisement

\protect\hyperlink{after-bottom}{Continue reading the main story}

\hypertarget{site-index}{%
\subsection{Site Index}\label{site-index}}

\hypertarget{site-information-navigation}{%
\subsection{Site Information
Navigation}\label{site-information-navigation}}

\begin{itemize}
\tightlist
\item
  \href{https://help.nytimes.com/hc/en-us/articles/115014792127-Copyright-notice}{©~2020~The
  New York Times Company}
\end{itemize}

\begin{itemize}
\tightlist
\item
  \href{https://www.nytco.com/}{NYTCo}
\item
  \href{https://help.nytimes.com/hc/en-us/articles/115015385887-Contact-Us}{Contact
  Us}
\item
  \href{https://www.nytco.com/careers/}{Work with us}
\item
  \href{https://nytmediakit.com/}{Advertise}
\item
  \href{http://www.tbrandstudio.com/}{T Brand Studio}
\item
  \href{https://www.nytimes.com/privacy/cookie-policy\#how-do-i-manage-trackers}{Your
  Ad Choices}
\item
  \href{https://www.nytimes.com/privacy}{Privacy}
\item
  \href{https://help.nytimes.com/hc/en-us/articles/115014893428-Terms-of-service}{Terms
  of Service}
\item
  \href{https://help.nytimes.com/hc/en-us/articles/115014893968-Terms-of-sale}{Terms
  of Sale}
\item
  \href{https://spiderbites.nytimes.com}{Site Map}
\item
  \href{https://help.nytimes.com/hc/en-us}{Help}
\item
  \href{https://www.nytimes.com/subscription?campaignId=37WXW}{Subscriptions}
\end{itemize}
