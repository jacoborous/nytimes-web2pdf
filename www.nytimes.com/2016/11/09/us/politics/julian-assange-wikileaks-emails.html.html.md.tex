Sections

SEARCH

\protect\hyperlink{site-content}{Skip to
content}\protect\hyperlink{site-index}{Skip to site index}

\href{https://www.nytimes.com/section/politics}{Politics}

\href{https://myaccount.nytimes.com/auth/login?response_type=cookie\&client_id=vi}{}

\href{https://www.nytimes.com/section/todayspaper}{Today's Paper}

\href{/section/politics}{Politics}\textbar{}Julian Assange Releases More
Emails and Defends WikiLeaks' Mission

\url{https://nyti.ms/2eBnPtk}

\begin{itemize}
\item
\item
\item
\item
\item
\end{itemize}

Advertisement

\protect\hyperlink{after-top}{Continue reading the main story}

Supported by

\protect\hyperlink{after-sponsor}{Continue reading the main story}

\hypertarget{julian-assange-releases-more-emails-and-defends-wikileaks-mission}{%
\section{Julian Assange Releases More Emails and Defends WikiLeaks'
Mission}\label{julian-assange-releases-more-emails-and-defends-wikileaks-mission}}

\includegraphics{https://static01.nyt.com/images/2016/11/09/us/09fd-assange/09fd-assange-articleInline.jpg?quality=75\&auto=webp\&disable=upscale}

By \href{http://www.nytimes.com/by/steve-eder}{Steve Eder}

\begin{itemize}
\item
  Nov. 8, 2016
\item
  \begin{itemize}
  \item
  \item
  \item
  \item
  \item
  \end{itemize}
\end{itemize}

Julian Assange operates from the Ecuadorean Embassy in London, where his
internet connection was recently cut off. But that has not stopped him
from being one of the most significant and unusual players in the 2016
election campaign in the United States.

His WikiLeaks platform has released tens of thousands of hacked emails
from inside Hillary Clinton's campaign and the Democratic National
Committee, making Mr. Assange a reviled figure among supporters of the
Democratic nominee and a hero to backers of Donald J. Trump.

On Tuesday morning, Mr. Assange
\href{https://wikileaks.org/Assange-Statement-on-the-US-Election.html}{weighed
in one last time as Americans headed to the polls}, releasing a
statement that offered an explanation for making the leaks public, an
assertion of impartiality and a defense of the WikiLeaks mission.

``The real victor is the U.S. public, which is better informed as a
result of our work,'' Mr. Assange wrote, adding that his organization
publishes information as quickly as possible and ``as fast as the public
can absorb it.''

``This is not due to a personal desire to influence the outcome of the
election,'' he said.

Four years ago, Mr. Assange sought and was granted asylum in the
Ecuadorean Embassy in London while he was being pursued in a Swedish
rape investigation, which he said was a cover for a United States effort
to extradite him for publishing a trove of National Security Agency
documents.

Ecuadorean officials have not sought to evict him, but
\href{http://www.nytimes.com/2016/10/19/world/europe/julian-assange-embassy.html}{last
month said that his internet access was disabled} out of fear that the
country was being drawn into Mr. Assange's efforts to interfere with
``electoral processes.''

Mr. Assange's operation did not fold, but continued to publish daily
releases of emails stolen from the accounts of John D. Podesta, the
chairman of Mrs. Clinton's campaign. WikiLeaks published another
installment on Tuesday, bringing the total number of messages made
public to more than 58,000.

The Clinton campaign has declined to authenticate or comment on the
emails, saying that the hack, which American intelligence officials
believe originated in Russia,
\href{http://www.nytimes.com/2016/09/01/world/europe/wikileaks-julian-assange-russia.html}{was
part of the Russian government's efforts} to meddle in the election to
benefit Mr. Trump.

The emails, at times, have been embarrassing to the Clintons and their
closest aides, revealing confidential and candid details about the inner
workings of the
\href{http://www.nytimes.com/2016/10/11/us/politics/hillary-clinton-emails.html}{campaign}
and the
\href{mailto:http://www.nytimes.com/2016/10/27/us/politics/bill-hillary-clinton-foundation-wikileaks.html}{Clinton
Foundation}. It has also shined a light on the Clintons' private work,
including
\href{http://www.nytimes.com/2016/10/16/us/politics/wikileaks-hack-hillary-clinton-emails.html?action=click\&contentCollection=Politics\&module=RelatedCoverage\&region=Marginalia\&pgtype=article}{lucrative
speeches for banks like Goldman Sachs}.

While Mr. Assange has made no secret of his dislike for Mrs. Clinton, he
said in his Tuesday statement that WikiLeaks published the emails
because it had the material and, ``It would be unconscionable for
WikiLeaks to withhold such an archive from the public during an
election.''

The lack of leaks pertaining to Mr. Trump, Mr. Assange said, resulted
from the fact that his organization was not in possession of Mr. Trump's
information. ``We cannot publish what we do not have,'' he wrote.

He went on to criticize The New York Times for withholding ``evidence of
illegal mass surveillance of the U.S. population for a year until after
the 2004 election,'' which he said denied vital information to voters
about George W. Bush.

Mr. Assange, though, has faced similar criticism, with the suggestion
that the leaks of emails from the Clinton campaign and the Democratic
National Committee were timed to inflict maximum damage or blunt
negative stories about Mr. Trump.

A first batch of Democratic National Committee emails ---
\href{http://www.nytimes.com/2016/07/25/us/politics/debbie-wasserman-schultz-dnc-wikileaks-emails.html}{which
led to renewed anger about the party's treatment of Bernie Sanders's
campaign} --- were released on the eve of the party's summer convention.

In the final month of the campaign, WikiLeaks unloaded the emails of Mr.
Podesta, which cover a period of several years ending in the spring of
2016. The first portion of those emails were published the same day that
a recording surfaced where
\href{http://www.nytimes.com/2016/10/08/us/politics/donald-trump-women.html}{Mr.
Trump was caught} on a hot microphone making lewd statements.

Advertisement

\protect\hyperlink{after-bottom}{Continue reading the main story}

\hypertarget{site-index}{%
\subsection{Site Index}\label{site-index}}

\hypertarget{site-information-navigation}{%
\subsection{Site Information
Navigation}\label{site-information-navigation}}

\begin{itemize}
\tightlist
\item
  \href{https://help.nytimes.com/hc/en-us/articles/115014792127-Copyright-notice}{©~2020~The
  New York Times Company}
\end{itemize}

\begin{itemize}
\tightlist
\item
  \href{https://www.nytco.com/}{NYTCo}
\item
  \href{https://help.nytimes.com/hc/en-us/articles/115015385887-Contact-Us}{Contact
  Us}
\item
  \href{https://www.nytco.com/careers/}{Work with us}
\item
  \href{https://nytmediakit.com/}{Advertise}
\item
  \href{http://www.tbrandstudio.com/}{T Brand Studio}
\item
  \href{https://www.nytimes.com/privacy/cookie-policy\#how-do-i-manage-trackers}{Your
  Ad Choices}
\item
  \href{https://www.nytimes.com/privacy}{Privacy}
\item
  \href{https://help.nytimes.com/hc/en-us/articles/115014893428-Terms-of-service}{Terms
  of Service}
\item
  \href{https://help.nytimes.com/hc/en-us/articles/115014893968-Terms-of-sale}{Terms
  of Sale}
\item
  \href{https://spiderbites.nytimes.com}{Site Map}
\item
  \href{https://help.nytimes.com/hc/en-us}{Help}
\item
  \href{https://www.nytimes.com/subscription?campaignId=37WXW}{Subscriptions}
\end{itemize}
