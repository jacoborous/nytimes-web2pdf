Sections

SEARCH

\protect\hyperlink{site-content}{Skip to
content}\protect\hyperlink{site-index}{Skip to site index}

\href{https://www.nytimes.com/section/world/americas}{Americas}

\href{https://myaccount.nytimes.com/auth/login?response_type=cookie\&client_id=vi}{}

\href{https://www.nytimes.com/section/todayspaper}{Today's Paper}

\href{/section/world/americas}{Americas}\textbar{}El Chapo, Escaped
Mexican Drug Lord, Is Recaptured in Gun Battle

\url{https://nyti.ms/22QgWJz}

\begin{itemize}
\item
\item
\item
\item
\item
\item
\end{itemize}

Advertisement

\protect\hyperlink{after-top}{Continue reading the main story}

Supported by

\protect\hyperlink{after-sponsor}{Continue reading the main story}

\hypertarget{el-chapo-escaped-mexican-drug-lord-is-recaptured-in-gun-battle}{%
\section{El Chapo, Escaped Mexican Drug Lord, Is Recaptured in Gun
Battle}\label{el-chapo-escaped-mexican-drug-lord-is-recaptured-in-gun-battle}}

\includegraphics{https://static01.nyt.com/images/2016/01/10/world/10chapo-hp/10chapo-hp-articleLarge.jpg?quality=75\&auto=webp\&disable=upscale}

By \href{http://www.nytimes.com/by/azam-ahmed}{Azam Ahmed}

\begin{itemize}
\item
  Jan. 8, 2016
\item
  \begin{itemize}
  \item
  \item
  \item
  \item
  \item
  \item
  \end{itemize}
\end{itemize}

MEXICO CITY --- He became a byword for government incompetence, a figure
who seemed invincible after he burrowed his way out of the country's
most secure prison.

But on Friday, nearly six months after his escape, Joaquín Guzmán Loera,
the Mexican drug lord known as El Chapo, was captured again after a
fierce gun battle near the coast in his home state, Sinaloa, Mexican
officials said. ``Mission accomplished: We have him,'' President Enrique
Peña Nieto announced.

The arrest ended one of the most extensive manhunts undertaken by the
government, involving every law enforcement agency in the country and
help from the United States.

But it was the marines,
\href{http://topics.nytimes.com/top/news/international/countriesandterritories/mexico/index.html?inline=nyt-geo}{Mexico}'s
most-trusted military force, who managed to capture the fugitive in an
early morning raid that left five people dead, the Mexican authorities
said. An American official also described the raid as ``a Mexican op,
planned and executed by Mexico.''

The government said late Friday that Mr. Guzmán had been planning a
movie about his life and that his people had been in contact with actors
and producers, which had allowed the authorities to track him down. It
said that the authorities had been watching a home in Los Mochis for
more than a month when law enforcement officers finally saw movement on
Thursday. Officials said that during the ensuing raid, Mr. Guzmán
managed to slip away through the sewers, and then he surfaced, stole a
car and was apprehended. The authorities took him to a hotel to wait for
backup.

The capture of the drug lord concludes a deeply embarrassing chapter for
the government of Mr. Peña Nieto, which has been waylaid by a series of
security and corruption scandals that reached their low point with Mr.
Guzmán's daring escape.

Now, a looming question is whether the Mexican authorities will try to
hold Mr. Guzmán for a third time --- he has already escaped from prison
twice --- or whether they will hand him over to the Americans.

Mexican officials are busily debating the issue. Some are arguing for a
``fast-track'' extradition that could put him in the United States
quickly, while others want to continue a previous process that could
take months, according to two people with knowledge of the discussions.
Mr. Guzmán, the head of Mexico's most powerful cartel, is facing
indictments in at least seven American federal courts on charges that
include narcotics trafficking and murder.

With operations that span much of Mexico, his organization has
specialized in smuggling tons of drugs into the United States through
vast networks of tunnels deep beneath the border. His success has made
him among the richest drug dealers in history: Forbes magazine estimated
his net worth at close to \$1 billion.

Mr. Guzmán stunned the world last summer when he stepped into the shower
in his cell, in the most secure wing of the prison, and abruptly
vanished in full view of a video camera. Guards later discovered a small
hole in the shower floor.

It led to a
\href{https://news.vice.com/article/we\%2Dvisited\%2Dthe\%2Dend\%2Dof\%2Dthe\%2Dtunnel\%2Dwhere\%2Del\%2Dchapo\%2Dmade\%2Dhis\%2Dbrazen\%2Djailbreak?utm_source=vicenewstwitter}{mile-long
tunnel} to a construction site. The tunnel was tall enough for Mr.
Guzmán to walk through standing upright --- his nickname translates to
Shorty --- and had been dug more than 30 feet underground. It was
equipped with lighting, ventilation and a motorcycle on rails. Some
engineers estimated that the tunnel took
\href{http://www.nytimes.com/2015/07/17/world/americas/1-million-pricetag-hinted-in-el-chapos-escape.html}{more}
\href{http://www.nytimes.com/2015/07/17/world/americas/1-million-pricetag-hinted-in-el-chapos-escape.html}{than
a year} and at least \$1 million to build.

The prison break humiliated the government of Mr. Peña Nieto, which had
proclaimed the arrests of Mr. Guzmán and the leaders of other drug
cartels as crucial achievements in restoring order and sovereignty to a
country long beleaguered by the horrific violence associated with
organized crime. It was particularly embarrassing because Mr. Guzmán had
already
\href{http://www.nytimes.com/2001/01/29/world/mexican-jail-easy-to-flee-just-pay-up.html}{escaped
from prison}in 2001, when his conspirators managed to smuggle him out.
By some accounts, he escaped that time by hiding in a laundry bin.

There are still major questions ahead, including the potential
extradition of Mr. Guzmán to the United States. Shortly after Mr. Guzmán
was captured in 2014, the attorney general of Mexico at the time refused
to extradite him to the United States, saying that the criminal would
serve his time in Mexico first before he was sent to another country.

Officials and analysts said it was an effort to show sovereignty and put
some distance between the Mexican authorities and their American
counterparts, who often used a heavy hand to influence policy in Mexico.

\includegraphics{https://static01.nyt.com/images/2016/01/09/world/JP-MEXICO2/JP-MEXICO2-articleLarge.jpg?quality=75\&auto=webp\&disable=upscale}

But that stance came to haunt the Peña Nieto administration after the
kingpin escaped. The United States had issued a formal request for his
extradition less than three weeks before
\href{http://www.nytimes.com/2015/07/15/world/americas/mexico-hunts-joaquin-chapo-guzman-united-states-offer-help.html?hp\&action=click\&pgtype=Homepage\&module=first-column-region\&region=top-news\&WT.nav=top-news}{Mr.
Guzmán broke out}.

A few months later, the Mexican government
\href{http://www.nytimes.com/2015/10/01/world/americas/mexico-signaling-shift-extradites-drug-kingpins-to-united-states.html}{extradited
several} top drug lords to the United States, suggesting a new spirit of
cooperation in the wake of Mr. Guzmán's escape. The people extradited
included an American citizen, Edgar Valdez Villarreal, a notorious
figure known as ``La Barbie,'' as well as people charged with
participating in the murders of a United States Consulate worker and an
American immigration and customs agent.

While the likelihood that Mr. Guzmán could escape from an American
maximum security prison is considered low, extradition would still come
at a cost to the image of the Mexican state, some analysts say.
``Extraditing him is a way to say we cannot cope with this with our own
institutions,'' said Pablo A. Piccato, a history professor at Columbia
University. ``While this is something everyone knows, obviously the
government has not been able to publicly recognize this or tackle it in
the past.''

Several senior politicians from Mr. Peña Nieto's party were already
calling for extradition, including Emilio Gamboa, the head of the
governing party in the Senate, who told local news media that he agreed
with the idea.

In a statement on Friday, the American attorney general, Loretta E.
Lynch, commended the Mexican authorities ``who have worked tirelessly in
recent months to bring Guzmán to justice.'' But she did not directly
answer the extradition question.

\includegraphics{https://static01.nyt.com/images/2016/01/08/multimedia/el-chapo-captured/el-chapo-captured-videoSixteenByNine1050.jpg}

The Justice Department ``is proud to maintain a close and effective
relationship with our Mexican counterparts, and we look forward to
continuing our work together to ensure the safety and security of all
our people,'' she said.

In the aftermath of Mr. Guzmán's escape last July, American officials
were frustrated with what they considered Mexico's resistance to
accepting help in the manhunt. After his escape, American officials
offered to give their Mexican counterparts whatever assistance they
could.

When Mexico rebuffed the offer, many officials in both countries worried
that Mr. Guzmán might never be caught.

In October, security forces said they had located Mr. Guzmán in the
remote northwestern mountains where he had been hiding out, an area
known as the Golden Triangle at the border of his home state of Sinaloa,
Durango and Chihuahua. After a gun battle, officials said, he escaped,
with wounds to his face and leg.

The authorities ultimately captured Mr. Guzmán in Los Mochis, a coastal
town of about 250,000 people that has long been known as a center of
boxing in Mexico.

Image

A photo released on Friday by a Mexican website, Plaza de Armas, of Mr.
Guzmán, who was recaptured in Los Mochis.Credit...Agence France-Presse
--- Getty Images

``People knew he would be caught any time; the government was going
after him hard,'' said Adrián Cabrera, a blogger in Culiacán, the
capital of Sinaloa, adding that Mr. Guzmán was still a popular figure in
Sinaloa.

``The new corridos will start coming out pretty soon,'' Mr. Cabrera
said, referring to the songs often used to glorify the exploits of local
drug traffickers. ``He's from here. He has a lot of sympathy here. It's
his turf.''

When Mr. Guzmán was finally recaptured, the president broke the news
himself, via Twitter. Given his sagging popularity --- his ratings are
the lowest of any president in the last 25 years --- Mr. Peña Nieto
seemed eager to declare the success personally.

In a statement on Friday, Mr. Peña Nieto said that the arrest was the
culmination of months of work, an example of Mexico's ability to
overcome adversity. ``Our security institutions have demonstrated that
our citizens can trust them and that they have the stature, strength and
determination to accomplish any mission they are tasked with,'' Mr. Peña
Nieto said.

Mr. Guzmán's lawyers have already filed legal injunctions against his
extradition to the United States. Wherever he ends up, few people here
think that his capture will change much for the government's popularity,
life for Mexicans or the strength of the Sinaloa cartel. ``This won't
have much effect on their internal structure,'' said Eduardo Guerrero, a
Mexico City-based security analyst. ``They are prepared for this kind of
news. They probably even have a protocol for it.''

Advertisement

\protect\hyperlink{after-bottom}{Continue reading the main story}

\hypertarget{site-index}{%
\subsection{Site Index}\label{site-index}}

\hypertarget{site-information-navigation}{%
\subsection{Site Information
Navigation}\label{site-information-navigation}}

\begin{itemize}
\tightlist
\item
  \href{https://help.nytimes.com/hc/en-us/articles/115014792127-Copyright-notice}{©~2020~The
  New York Times Company}
\end{itemize}

\begin{itemize}
\tightlist
\item
  \href{https://www.nytco.com/}{NYTCo}
\item
  \href{https://help.nytimes.com/hc/en-us/articles/115015385887-Contact-Us}{Contact
  Us}
\item
  \href{https://www.nytco.com/careers/}{Work with us}
\item
  \href{https://nytmediakit.com/}{Advertise}
\item
  \href{http://www.tbrandstudio.com/}{T Brand Studio}
\item
  \href{https://www.nytimes.com/privacy/cookie-policy\#how-do-i-manage-trackers}{Your
  Ad Choices}
\item
  \href{https://www.nytimes.com/privacy}{Privacy}
\item
  \href{https://help.nytimes.com/hc/en-us/articles/115014893428-Terms-of-service}{Terms
  of Service}
\item
  \href{https://help.nytimes.com/hc/en-us/articles/115014893968-Terms-of-sale}{Terms
  of Sale}
\item
  \href{https://spiderbites.nytimes.com}{Site Map}
\item
  \href{https://help.nytimes.com/hc/en-us}{Help}
\item
  \href{https://www.nytimes.com/subscription?campaignId=37WXW}{Subscriptions}
\end{itemize}
