Sections

SEARCH

\protect\hyperlink{site-content}{Skip to
content}\protect\hyperlink{site-index}{Skip to site index}

\href{https://www.nytimes.com/section/world/europe}{Europe}

\href{https://myaccount.nytimes.com/auth/login?response_type=cookie\&client_id=vi}{}

\href{https://www.nytimes.com/section/todayspaper}{Today's Paper}

\href{/section/world/europe}{Europe}\textbar{}Russia Is Moving Ahead
With Missile Program That Violates Treaty, U.S. Officials Say

\url{https://nyti.ms/2egOn6W}

\begin{itemize}
\item
\item
\item
\item
\item
\end{itemize}

Advertisement

\protect\hyperlink{after-top}{Continue reading the main story}

Supported by

\protect\hyperlink{after-sponsor}{Continue reading the main story}

\hypertarget{russia-is-moving-ahead-with-missile-program-that-violates-treaty-us-officials-say}{%
\section{Russia Is Moving Ahead With Missile Program That Violates
Treaty, U.S. Officials
Say}\label{russia-is-moving-ahead-with-missile-program-that-violates-treaty-us-officials-say}}

\includegraphics{https://static01.nyt.com/images/2016/10/20/world/20ARMS-web1/20ARMS-web1-articleLarge.jpg?quality=75\&auto=webp\&disable=upscale}

By \href{http://www.nytimes.com/by/michael-r-gordon}{Michael R. Gordon}

\begin{itemize}
\item
  Oct. 19, 2016
\item
  \begin{itemize}
  \item
  \item
  \item
  \item
  \item
  \end{itemize}
\end{itemize}

Russia appears to be moving ahead with a program to produce a
ground-launched cruise missile despite the Obama administration's
protests that the weapon violates a landmark arms control agreement,
according to American officials and lawmakers.

The concern goes beyond those raised by the United States in July 2014,
when the Obama administration said that Russia had violated the 1987
treaty on Intermediate-Range Nuclear Forces by conducting flight tests
of the missile.

The I.N.F. accord, which was signed by President Ronald Reagan and his
Soviet counterpart, Mikhail S. Gorbachev, bans the two nations from
testing, producing and possessing ground-launched ballistic or cruise
missiles that are capable of flying 300 to 3,400 miles.

American officials are now expressing concerns that Russia is producing
more missiles than are needed to sustain a flight-test program, spurring
fears that the Kremlin is moving to build a force that could ultimately
be deployed.

Information about the Russian program was provided by American officials
on the condition of anonymity because they were discussing classified
intelligence assessments.

Two prominent Republican lawmakers have also sent a letter to the White
House asserting a deepening violation by Russia, but without providing
details.

``The I.N.F. Treaty is the only arms control treaty that succeeded in
eliminating a class of nuclear arms,'' wrote Representatives Mac
Thornberry, chairman of the House Committee on Armed Services, and Devin
Nunes, chairman of the House Permanent Select Committee on Intelligence.
``It has become apparent to us that the situation regarding Russia's
violation has worsened and Russia is now in material breach of the
treaty.''

The State Department declined to discuss specifics of the issue. ``We do
not comment on intelligence matters,'' said John Kirby, the State
Department spokesman.

After the charge was leveled two years ago, the Russians insisted that
the United States provide more information about the allegation, and
also responded with their own allegations --- including charges that
American armed drones violate the I.N.F. treaty.

\includegraphics{https://static01.nyt.com/images/2016/10/20/world/20ARMS/20ARMS-articleLarge.jpg?quality=75\&auto=webp\&disable=upscale}

To focus attention on the issue, the United States has called for a rare
meeting of the Special Verification Commission, a body that was
established by the I.N.F. treaty to deal with compliance.

Russia inherited the treaty obligations of the Soviet Union. Other
former Soviet states that also are a party to the treaty --- Ukraine,
Belarus and Kazakhstan --- will also send representatives to the meeting
of the commission, its first since 2003.

The arms control dispute comes against the background of steadily
deteriorating relations, which are already strained over Russian
airstrikes on Aleppo, Syria, as well as its seizure of portions of
Ukraine. A range of American officials also have accused Russia of
meddling in the presidential election by hacking into the email accounts
of Democratic Party figures.

But the arms control issues are important in their own right. The I.N.F.
treaty is regarded as one of the accords that brought an end to the Cold
War. The question of Russian compliance threatens to tarnish the White
House's arms control legacy and President Obama's vision of a world in
which there would be fewer nuclear weapons.

Since the I.N.F. treaty was signed, some Russian officials appear to
have had buyer's remorse, arguing that Moscow needs more ways to respond
to the potential array of threats around its periphery. During the
George W. Bush administration, Russia's defense minister suggested that
the two sides drop the treaty.

The Obama administration says that the treaty is in the overall interest
of the United States even if some of its provisions are being violated.
When the United States charged Russia with violating the accord two
years ago, Mr. Obama sent a letter to President Vladimir V. Putin
stressing his interest in a high-level dialogue to preserve the treaty
and bring the Kremlin back into compliance.

American military officials, for their part, have said that a move by
Russia to actually deploy the new missile system, which is small, mobile
and easily concealed, would be significant. When he served as NATO's top
commander in 2014, Gen. Philip M. Breedlove said that ``a weapons
capability'' that violates the I.N.F. treaty ``can't go unanswered.''

How best to persuade the Russians to rectify the alleged violation is
also a subject of debate.

The Pentagon has produced a list of military steps that could be taken
in response, but the White House has yet to approve them. Two years ago,
the State Department's senior arms control official raised the idea of
imposing ``economic measures,'' but sanctions do not appear to be under
consideration.

It is unlikely that the verification commission will make progress in
resolving the allegation, since the Russians have never acknowledged the
existence of the missile, even though American officials say test
flights may have begun as early as 2008.

This month, Mr. Putin also suspended his country's participation in an
accord that was concluded in 2000 on the disposal of plutonium. That
agreement does not affect the number of nuclear warheads the United
States and Russia have, but the suspension of the accord will deprive
each side of the opportunity to verify what the other is doing to
dispose of plutonium.

Mr. Putin said the step was taken because the deterioration of
American-Russian relations had led to a ``radically changed
environment.''

Advertisement

\protect\hyperlink{after-bottom}{Continue reading the main story}

\hypertarget{site-index}{%
\subsection{Site Index}\label{site-index}}

\hypertarget{site-information-navigation}{%
\subsection{Site Information
Navigation}\label{site-information-navigation}}

\begin{itemize}
\tightlist
\item
  \href{https://help.nytimes.com/hc/en-us/articles/115014792127-Copyright-notice}{©~2020~The
  New York Times Company}
\end{itemize}

\begin{itemize}
\tightlist
\item
  \href{https://www.nytco.com/}{NYTCo}
\item
  \href{https://help.nytimes.com/hc/en-us/articles/115015385887-Contact-Us}{Contact
  Us}
\item
  \href{https://www.nytco.com/careers/}{Work with us}
\item
  \href{https://nytmediakit.com/}{Advertise}
\item
  \href{http://www.tbrandstudio.com/}{T Brand Studio}
\item
  \href{https://www.nytimes.com/privacy/cookie-policy\#how-do-i-manage-trackers}{Your
  Ad Choices}
\item
  \href{https://www.nytimes.com/privacy}{Privacy}
\item
  \href{https://help.nytimes.com/hc/en-us/articles/115014893428-Terms-of-service}{Terms
  of Service}
\item
  \href{https://help.nytimes.com/hc/en-us/articles/115014893968-Terms-of-sale}{Terms
  of Sale}
\item
  \href{https://spiderbites.nytimes.com}{Site Map}
\item
  \href{https://help.nytimes.com/hc/en-us}{Help}
\item
  \href{https://www.nytimes.com/subscription?campaignId=37WXW}{Subscriptions}
\end{itemize}
