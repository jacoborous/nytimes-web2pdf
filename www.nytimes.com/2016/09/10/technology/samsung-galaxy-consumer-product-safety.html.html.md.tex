Sections

SEARCH

\protect\hyperlink{site-content}{Skip to
content}\protect\hyperlink{site-index}{Skip to site index}

\href{https://www.nytimes.com/section/technology}{Technology}

\href{https://myaccount.nytimes.com/auth/login?response_type=cookie\&client_id=vi}{}

\href{https://www.nytimes.com/section/todayspaper}{Today's Paper}

\href{/section/technology}{Technology}\textbar{}Galaxy Note 7 Owners Are
Urged to Stop Using Their Phones

\url{https://nyti.ms/2crTKiP}

\begin{itemize}
\item
\item
\item
\item
\item
\end{itemize}

Advertisement

\protect\hyperlink{after-top}{Continue reading the main story}

Supported by

\protect\hyperlink{after-sponsor}{Continue reading the main story}

\hypertarget{galaxy-note-7-owners-are-urged-to-stop-using-their-phones}{%
\section{Galaxy Note 7 Owners Are Urged to Stop Using Their
Phones}\label{galaxy-note-7-owners-are-urged-to-stop-using-their-phones}}

\includegraphics{https://static01.nyt.com/images/2016/09/10/business/10SAMSUNG/10SAMSUNG-articleLarge.jpg?quality=75\&auto=webp\&disable=upscale}

By \href{http://www.nytimes.com/by/cecilia-kang}{Cecilia Kang}

\begin{itemize}
\item
  Sept. 9, 2016
\item
  \begin{itemize}
  \item
  \item
  \item
  \item
  \item
  \end{itemize}
\end{itemize}

WASHINGTON --- The fallout over the potential for Samsung Galaxy Note 7
smartphones to catch fire is intensifying.

On Friday, the United States Consumer Product Safety Commission urged
owners of the smartphone to power down their Galaxy Note 7 devices and
stop using them altogether.

``C.P.S.C. and Samsung are working cooperatively to formally announce an
official recall of the devices as soon as possible,'' the agency said in
a statement. ``C.P.S.C. is working quickly to determine whether a
replacement Galaxy Note 7 is an acceptable remedy for Samsung or their
phone carriers to provide to consumers.''

The C.P.S.C. is the main consumer product safety agency in the United
States, with broad oversight over toys, tractors, appliances and
electronics. This year, the C.P.S.C. recalled hoverboards that contained
lithium-ion batteries; those batteries exploded or caught fire in dozens
of cases.

The commission's statement is the latest blow to Samsung and the Galaxy
Note 7, which became available only two weeks ago. This month,
\href{http://www.nytimes.com/2016/09/03/business/samsung-galaxy-note-battery.html}{Samsung
said it would recall 2.5 million} of the devices because of an issue
with the lithium-ion batteries in the phones, which can catch fire and
explode. The problem had affected 35 devices globally as of last week.

\href{https://news.samsung.com/global/statement-on-galaxy-note7}{Samsung
said} it was voluntarily recalling the phones and ``conducting a
thorough inspection with our suppliers to identify possible affected
batteries in the market.''

Air safety regulators worldwide have since advised passengers not to
charge or turn on the smartphones inside an aircraft. Three Australian
airlines
\href{http://www.nytimes.com/reuters/2016/09/08/technology/08reuters-airlines-samsung.html}{have
banned them}. On Thursday, the Federal Aviation Administration also said
it ``strongly'' advised passengers onboard planes
\href{http://www.nytimes.com/2016/09/09/business/faa-strongly-advises-against-using-samsung-galaxy-note-7-on-planes.html}{not
to use the Galaxy Note 7}.

In a statement on Friday, Tim Baxter, president of Samsung Electronics
America said, ``We are asking users to power down their Galaxy Note 7s
and exchange them now.''

The recall comes at a tricky time for Samsung. The Galaxy is one of the
South Korean company's most visible consumer product lines, and its
smartphones compete with the Apple iPhone for pre-eminence with
consumers. This week,
\href{http://www.nytimes.com/2016/09/08/technology/iphone-7-apple-headphone-jack.html?ref=technology}{Apple
unveiled its newest smartphones}, the iPhone 7 and 7 Plus, which will
ship later this month.

It was unclear if Samsung would provide refunds for Galaxy Note 7
customers who did not want a replacement provided by the company. All
four major wireless carriers in the United States --- Verizon, AT\&T,
T-Mobile and Sprint --- have halted sales of the Galaxy Note 7 and have
been given instructions to help owners make an exchange, Samsung said.

While the recall of the Galaxy Note 7 is Samsung's largest voluntary
recall, it is not the biggest on record. In 2007, Nokia announced a
recall of 46 million cellphone batteries. In 2006, Dell recalled 4.1
million lithium-ion batteries for notebook computers.

Maria Rerecich, director of electronics testing for Consumer Reports,
said the number of people affected by Samsung's recall was small but
showed ``a serious potential safety hazard.''

Advertisement

\protect\hyperlink{after-bottom}{Continue reading the main story}

\hypertarget{site-index}{%
\subsection{Site Index}\label{site-index}}

\hypertarget{site-information-navigation}{%
\subsection{Site Information
Navigation}\label{site-information-navigation}}

\begin{itemize}
\tightlist
\item
  \href{https://help.nytimes.com/hc/en-us/articles/115014792127-Copyright-notice}{©~2020~The
  New York Times Company}
\end{itemize}

\begin{itemize}
\tightlist
\item
  \href{https://www.nytco.com/}{NYTCo}
\item
  \href{https://help.nytimes.com/hc/en-us/articles/115015385887-Contact-Us}{Contact
  Us}
\item
  \href{https://www.nytco.com/careers/}{Work with us}
\item
  \href{https://nytmediakit.com/}{Advertise}
\item
  \href{http://www.tbrandstudio.com/}{T Brand Studio}
\item
  \href{https://www.nytimes.com/privacy/cookie-policy\#how-do-i-manage-trackers}{Your
  Ad Choices}
\item
  \href{https://www.nytimes.com/privacy}{Privacy}
\item
  \href{https://help.nytimes.com/hc/en-us/articles/115014893428-Terms-of-service}{Terms
  of Service}
\item
  \href{https://help.nytimes.com/hc/en-us/articles/115014893968-Terms-of-sale}{Terms
  of Sale}
\item
  \href{https://spiderbites.nytimes.com}{Site Map}
\item
  \href{https://help.nytimes.com/hc/en-us}{Help}
\item
  \href{https://www.nytimes.com/subscription?campaignId=37WXW}{Subscriptions}
\end{itemize}
