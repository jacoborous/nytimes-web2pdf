Sections

SEARCH

\protect\hyperlink{site-content}{Skip to
content}\protect\hyperlink{site-index}{Skip to site index}

\href{https://www.nytimes.com/section/world/europe}{Europe}

\href{https://myaccount.nytimes.com/auth/login?response_type=cookie\&client_id=vi}{}

\href{https://www.nytimes.com/section/todayspaper}{Today's Paper}

\href{/section/world/europe}{Europe}\textbar{}Russia Denies Any Role in
Deadly Convoy Attack in Syria

\url{https://nyti.ms/2cZw7yr}

\begin{itemize}
\item
\item
\item
\item
\item
\end{itemize}

Advertisement

\protect\hyperlink{after-top}{Continue reading the main story}

Supported by

\protect\hyperlink{after-sponsor}{Continue reading the main story}

\hypertarget{russia-denies-any-role-in-deadly-convoy-attack-in-syria}{%
\section{Russia Denies Any Role in Deadly Convoy Attack in
Syria}\label{russia-denies-any-role-in-deadly-convoy-attack-in-syria}}

\includegraphics{https://static01.nyt.com/images/2016/09/22/world/21Russia-web1/21Russia-web1-articleLarge.jpg?quality=75\&auto=webp\&disable=upscale}

By \href{http://www.nytimes.com/by/neil-macfarquhar}{Neil MacFarquhar}

\begin{itemize}
\item
  Sept. 21, 2016
\item
  \begin{itemize}
  \item
  \item
  \item
  \item
  \item
  \end{itemize}
\end{itemize}

MOSCOW --- Russia sought to distance itself on Wednesday from
\href{http://www.nytimes.com/2016/09/21/world/middleeast/syria-cease-fire.html}{American
accusations} that it was responsible for the
\href{http://www.nytimes.com/2016/09/20/world/middleeast/syria-aid-john-kerry.html}{fiery
destruction} of a humanitarian convoy in Syria, with the foreign
minister, Sergey V. Lavrov, also seeking to absolve Syria of
responsibility for the attack.

The charges and countercharges divided opinion in Russia, with some
accusing the Syrian government and others blaming the United States. If
there was any consensus, it was that the destruction of the much-needed
aid convoy bound for rebel-held parts of Aleppo Province on Monday dealt
a serious blow to already beleaguered attempts by the United States and
Russian to find a way to work together on Syria.

Mr. Lavrov, noting that the Russian Air Force had already announced that
it had not hit the convoy, said the Syrians were not able to fly at
night. ``The Syrian Air Force could not have been at work, because the
convoy was attacked during the hours of darkness,'' Mr. Lavrov said from
the United Nations General Assembly session in New York,
\href{http://www.vesti.ru/doc.html?id=2801245\#/video/https\%3A\%2F\%2Fplayer.vgtrk.com\%2Fiframe\%2Fvideo\%2Fid\%2F1573627\%2Fstart_zoom\%2Ftrue\%2FshowZoomBtn\%2Ffalse\%2Fsid\%2Fvesti\%2FisPlay\%2Ftrue\%2F\%3Facc_video_id\%3D692408}{speaking
to} Russia's state-run Rossiya 1 television.

Later on Wednesday, the Russian Defense Ministry said that a United
States Predator drone had appeared in the area just minutes before the
aid convoy was attacked. There was no immediate response from the
Pentagon.

Russia has said that its yearlong campaign of airstrikes in Syria has
not caused a single civilian casualty, though the monitoring group
\href{http://airwars.org/}{airwars.org} says it conservatively estimates
the number at 3,000.

Mr. Lavrov, in the excerpt from a longer interview broadcast by Rossiya
1, did not elaborate on who might have carried out the attack.

Much of the convoy, carrying aid from government-controlled territory
for 78,000 people in rebel-held territory in the western countryside of
Aleppo Province, went up in flames. The International Committee of the
Red Cross said in a statement that ``around 20 civilians'' had been
killed, including Omar Barakat, the local director of the Syria Red
Crescent Society.

Not everyone accepted the idea of absolving the Syrian government. The
Novaya Gazeta newspaper, one of the few independent voices left in the
Russian news media, blamed Damascus. Alexander Shumilin, a political
scientist, wrote in the newspaper that Syria had ``barbarously bombed''
the convoy, probably dealing a death knell to diplomacy. ``That means
the escalation of the conflict,'' he wrote.

But much of the reaction in Russia blamed the United States. Numerous
analysts found it suspicious that Washington had accused Russia of the
attack so soon after the American military apologized for an airstrike
that killed dozens of Syrian soldiers during last week's cease-fire that
had been negotiated by Mr. Lavrov and Secretary of State John Kerry.

``It seems that such ill-conceived accusations that are not supported by
facts are, among other things, designed to deflect attention away from a
very strange `mistake' made by pilots of an anti-Islamic State coalition
headed by the United States,'' said Roman Kudrin, a reporter commenting
on state-backed Channel One.

On social media, opponents of the Syrian government asserted that the
attack on the convoy was in retaliation for the American strike on the
soldiers, while supporters joked that if the United States could make
``mistakes,'' so, too, could the Syrian government.

Some Russian commentators noted that the American establishment was
divided over reaching an agreement with Russia on Syria. ``This turns
into a rather dangerous situation when agreements reached can eventually
be not supported by the U.S. military,'' Kirill Koktysh, a political
analyst, wrote in the Sobesednik newspaper. ``These are alarming
symptoms that do not bring stability to the rest of the world.''

Some blamed Syrian rebels for the attack, a position already suggested
by the Russian Defense Ministry.

\includegraphics{https://static01.nyt.com/images/2016/09/22/world/21Russia-web2/21Russia-web2-articleLarge.jpg?quality=75\&auto=webp\&disable=upscale}

An official from the Syrian Red Crescent Society, which helped organize
the convoy, said in an interview with the state-run Izvestia daily that
the rebels were responsible for the attack. The official was quoted as
saying that there was no evidence that either the Russian or Syrian Air
Force had struck columns with humanitarian aid.

The daily Kommersant spoke to at least one expert who dismissed the idea
that the destruction of 18 out of 31 trucks had been caused by fires lit
by saboteurs on the ground.

Kommersant noted that the predominate assessment, that the convoy was
hit from the air, seemed to point to Damascus. The paper quoted experts
saying that Russia would have difficulty convincing the world otherwise
and that in defending Syria, Moscow risked bruising its image of being
involved militarily in Syria to help end the conflict. Kommersant also
concluded that the stage was set for the resumption of a full-scale war.

The Obama administration said on Tuesday that it held Russia
responsible, whether for conducting the strike itself or for failing to
restrain Syria, although officials also said that their intelligence
information indicated that Russia itself had bombed the convoy.

The attack was the second calamity in three days that threatened to
destroy the agreement between Russia and the United States. The deal
included a weeklong cease-fire, humanitarian aid deliveries and
cooperation between the two militaries in targeting extremist
organizations in Syria. The longer-term goal was to establish enough
trust to resume peace talks between the Syrian government and the
opposition.

On Saturday, an American bombing run meant to hit Islamic State
militants went awry, instead killing 60 people whom Damascus and its
Russian allies identified as Syrian soldiers. Both Moscow and Damascus
brushed aside apologies from Washington, suggesting that the attack was
deliberate.

Assessing the attack on the convoy, the
\href{http://www.nytimes.com/2016/09/21/world/middleeast/syria-cease-fire.html}{Pentagon
has determined}with ``very high probability'' --- given the United
States' ability to track all aircraft in the region using radar and
other sensors --- that a Russian Su-24 attack aircraft flew directly
above the convoy less than a minute before the airstrike was reported,
officials said. The plane was Russian, not Syrian, they said, while
another said there were no indications of warplanes from other nations
flying nearby at that time.

The strike came shortly after the Syrian military announced that it
regarded the seven-day partial cease-fire as over. Escorted by the
Syrian Arab Red Crescent, the convoy was among the first under the
agreement to deliver critical aid to the rebel-held areas.

The air assault continued on Tuesday night with an airstrike on a
medical clinic in a rebel-held area of Aleppo Province, a Paris-based
coalition of medical organizations reported.

The group, the International Union of Medical Care and Relief
Organizations, said aircraft had attacked a clinic in Khan Touman,
killing two ambulance drivers and two nurses, and demolishing the
three-story building. It was feared that more bodies would be pulled
from the rubble.

Tawfik Chamaa, a Geneva-based spokesman for the group, said the
airstrike appeared to have been carried out by Russian warplanes. The
extent of the damage, he said, indicated that the attackers, ``used
sohisticated missiles which are held mainly by Russian forces.''

While both Russia and Syria denied responsibility for the convoy attack,
the Russian account shifted in the days after the assault.

Some Russian officials first suggested that rebel artillery fire had
ignited the convoy. Later, other officials speculated that the convoy
had been set alight by some manner of sabotage. Then Russia's Defense
Ministry announced on Tuesday that
\href{https://yadi.sk/i/ks9vq2Q8vSg3J}{a drone video} showed terrorists
driving a pickup truck armed with a mortar as part of the convoy. This
suggested that the presence of the pickup truck had prompted the attack.

But the drone video showed the aid convoy stationary, parked along the
road, while what seems to be a truck towing a mortar drives past. There
is no sense from the footage that the pickup is linked to the convoy,
nor anything that would indicate why the aid might have become a
military target.

Advertisement

\protect\hyperlink{after-bottom}{Continue reading the main story}

\hypertarget{site-index}{%
\subsection{Site Index}\label{site-index}}

\hypertarget{site-information-navigation}{%
\subsection{Site Information
Navigation}\label{site-information-navigation}}

\begin{itemize}
\tightlist
\item
  \href{https://help.nytimes.com/hc/en-us/articles/115014792127-Copyright-notice}{©~2020~The
  New York Times Company}
\end{itemize}

\begin{itemize}
\tightlist
\item
  \href{https://www.nytco.com/}{NYTCo}
\item
  \href{https://help.nytimes.com/hc/en-us/articles/115015385887-Contact-Us}{Contact
  Us}
\item
  \href{https://www.nytco.com/careers/}{Work with us}
\item
  \href{https://nytmediakit.com/}{Advertise}
\item
  \href{http://www.tbrandstudio.com/}{T Brand Studio}
\item
  \href{https://www.nytimes.com/privacy/cookie-policy\#how-do-i-manage-trackers}{Your
  Ad Choices}
\item
  \href{https://www.nytimes.com/privacy}{Privacy}
\item
  \href{https://help.nytimes.com/hc/en-us/articles/115014893428-Terms-of-service}{Terms
  of Service}
\item
  \href{https://help.nytimes.com/hc/en-us/articles/115014893968-Terms-of-sale}{Terms
  of Sale}
\item
  \href{https://spiderbites.nytimes.com}{Site Map}
\item
  \href{https://help.nytimes.com/hc/en-us}{Help}
\item
  \href{https://www.nytimes.com/subscription?campaignId=37WXW}{Subscriptions}
\end{itemize}
