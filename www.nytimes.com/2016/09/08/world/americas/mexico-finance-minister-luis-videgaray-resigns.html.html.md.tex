Sections

SEARCH

\protect\hyperlink{site-content}{Skip to
content}\protect\hyperlink{site-index}{Skip to site index}

\href{https://www.nytimes.com/section/world/americas}{Americas}

\href{https://myaccount.nytimes.com/auth/login?response_type=cookie\&client_id=vi}{}

\href{https://www.nytimes.com/section/todayspaper}{Today's Paper}

\href{/section/world/americas}{Americas}\textbar{}Mexico's Finance
Minister Resigns Amid Fallout From Trump Visit

\url{https://nyti.ms/2ckkVMi}

\begin{itemize}
\item
\item
\item
\item
\item
\end{itemize}

Advertisement

\protect\hyperlink{after-top}{Continue reading the main story}

Supported by

\protect\hyperlink{after-sponsor}{Continue reading the main story}

\hypertarget{mexicos-finance-minister-resigns-amid-fallout-from-trump-visit}{%
\section{Mexico's Finance Minister Resigns Amid Fallout From Trump
Visit}\label{mexicos-finance-minister-resigns-amid-fallout-from-trump-visit}}

\includegraphics{https://static01.nyt.com/images/2016/09/07/multimedia/08mexico/08mexico-videoSixteenByNine3000-v2.jpg}

By \href{http://www.nytimes.com/by/kirk-semple}{Kirk Semple} and
\href{https://www.nytimes.com/by/elisabeth-malkin}{Elisabeth Malkin}

\begin{itemize}
\item
  Sept. 7, 2016
\item
  \begin{itemize}
  \item
  \item
  \item
  \item
  \item
  \end{itemize}
\end{itemize}

MEXICO CITY --- One of President Enrique Peña Nieto's top ministers and
closest allies resigned on Wednesday, an apparent casualty of Mr. Peña
Nieto's wildly unpopular meeting last week with
\href{http://www.nytimes.com/topic/person/donald-trump}{Donald J.
Trump.}

The spectacle of the Mexican president standing next to the Republican
candidate who has disparaged Mexicans prompted
\href{http://www.nytimes.com/2016/09/01/world/americas/trump-mexico-pena-nieto-reaction.html}{widespread
dismay and anger} here, and reportedly divided Mr. Peña Nieto's cabinet.
Luis Videgaray, the finance minister who stepped down on Wednesday, had
championed the idea of inviting Mr. Trump to Mexico City over the
objections of other ministers, according to several Mexican news media
reports, though Mr. Peña Nieto insisted it was his own initiative.

Mr. Peña Nieto announced Mr. Videgaray's resignation at a news
conference.

He did not give a reason for Mr. Videgaray's departure. But some
analysts interpreted it as the latest, and most dramatic, effort by the
president to regain the trust of the Mexican public following his
meeting with Mr. Trump, the Republican presidential nominee, who has
made criticism of Mexicans and Mexico an incendiary motif of his
campaign.

``It will help mitigate the anger,'' said José Antonio Crespo Mendoza, a
professor of politics at Centro de Investigación y Docencia Económicas,
a Mexican research and higher education center. ``President Peña Nieto
realized things could not stay as they were, and that they could no
longer insist that it had been a good call'' to invite Mr. Trump.

Mr. Trump cast his visit as a statesmanlike effort to reach out to a
country he had alienated. But hours after his meeting with Mr. Peña
Nieto, he gave a combative speech in Phoenix that struck many of the
anti-immigrant themes that have defined his candidacy. The episode has
left Mr. Peña Nieto and his government embarrassed at a time when his
approval ratings were already at record lows over rising crime, poor
economic growth and conflict of interest scandals.

In the days after the meeting, the president scrambled to contain the
fallout and to repair the damage --- using a television interview, a
newspaper column and a town-hall-style meeting held to mark his annual
``Informe de Gobierno,'' Mexico's equivalent of the State of the Union
address. But these steps seemed to do little to mollify many Mexicans,
who accused the president of humiliating the nation, first by inviting
Mr. Trump and then by failing to use the opportunity to push back
against the candidate's criticism of Mexico and Mexicans.

The discontent has continued to simmer on social media and in the news,
and an anti-Peña Nieto demonstration has been called for Sept. 15. On
Tuesday, an opposition senator --- seeking to provide the defense of
Mexico that he said Mr. Peña Nieto had failed to offer --- submitted a
bill that would empower the government to fight several of Mr. Trump's
foreign policy proposals, including his promise to force Mexico to pay
for the construction of a border wall and his threat to withdraw the
United States from the North American Free Trade Agreement.

Image

Luis VidegarayCredit...Henry Romero/Reuters

``Of all the scandals Peña Nieto has faced, this is the most
devastating,'' said Alfonso Zárate, a political analyst and columnist in
Mexico City.

From Mr. Trump's perspective, the trip was a risky move from the start,
given the potential for embarrassment if the Mexican president were to
criticize him publicly.

It came together after Jared Kushner, the son-in-law and campaign
adviser to Mr. Trump, began to lay the groundwork with Mexican officials
in early August. Mr. Kushner dealt with Mr. Videgaray and other
government officials in negotiating how the meeting would happen,
according to a close ally of Mr. Trump who was involved in the
arrangements.

Mr. Trump's visit was largely incident-free, and provided his campaign
with pleasing visuals of the Republican nominee standing side by side
with a head of state. But any positive effect was undone by Mr. Peña
Nieto's insistence afterward that he had told Mr. Trump that Mexico
would not pay for a border wall, a claim that contradicted Mr. Trump's
account.

Even so, the damage to Mr. Peña Nieto seems to have been far worse.

While Mr. Videgaray's resignation may have been intended to stanch the
political bleeding in Mexico City and salvage the president's standing,
it also shifted the odds in the contest for future leadership of the
governing Institutional Revolutionary Party, or P.R.I. In addition, it
left Mr. Peña Nieto without one of his closest confidantes in his
administration.

Mr. Videgaray, who holds a doctorate in economics from the Massachusetts
Institute of Technology, had been a rising star in the P.R.I., and his
name was often mentioned as a possible presidential contender in 2018.

He had worked for the president since 2005, when Mr. Peña Nieto was the
governor of the State of Mexico, helping him restructure the state's
debt and coordinating Mr. Peña Nieto's 2012 presidential bid.

Mr. Videgaray became the president's closest ally and adviser in the
cabinet --- ``the power behind the throne,'' Mr. Zárate said.

His reputation was tarnished by reports he had bought a house at a golf
club from a government contractor, though not enough to force him out of
the cabinet or out of contention for 2018. The president, who is known
for his loyalty to his inner circle, stuck by him.

\href{https://www.nytimes.com/interactive/2016/09/07/us/politics/trump-products-reaction.html}{}

\includegraphics{https://static01.nyt.com/images/2016/09/07/us/07interactive-trump2/07interactive-trump2-articleLarge.jpg}

\hypertarget{have-you-done-something-to-support-or-oppose-trumps-brands}{%
\subsection{Have You Done Something to Support or Oppose Trump's
Brands?}\label{have-you-done-something-to-support-or-oppose-trumps-brands}}

Have you registered your disapproval or approval for Donald J. Trump's
candidacy by, say, discarding or returning Trump-brand clothes --- or by
booking a night at a Trump hotel? We want to hear your stories.

Indeed, Mr. Peña Nieto was accompanied by Mr. Videgaray when he made the
announcement on Wednesday, and was warm and effusive toward his longtime
adviser, thanking him for being ``committed to Mexico and loyal to the
president of Mexico.''

Mr. Zárate said the departure of Mr. Videgaray was ``very painful for
the president.'' He added, ``we all know the enormous influence that he
has.''

``It leaves the president orphaned,'' Mr. Zárate said.

As finance minister, Mr. Videgaray was at the center of the
administration's efforts to overhaul the country's telecommunications
and energy sectors. He successfully pushed to open the nation's
\href{http://www.nytimes.com/topic/subject/oil-and-gasoline?inline=nyt-classifier}{oil
industry}, which had been a state-run monopoly since the 1930s. He also
championed a tax overhaul that increased tax revenue but earned him some
enemies within the business community.

During his tenure, however, Mexico saw slow economic growth and a weak
peso, and the country's standing with credit-rating agencies has
suffered. Moody's Investors Service and S\&P Global Ratings have each
lowered their outlook from stable to negative in recent months.

Economic performance has been so underwhelming that some analysts
believe Mr. Videgaray's departure was an inevitability discussed well
before Mr. Trump's visit. Furthermore, Mr. Videgaray is rumored to be a
possible P.R.I. candidate for the governorship of the State of Mexico.

``I see this as a clear electoral play,'' said Genaro Lozano, a
professor of United States government and society at the Universidad
Iberoamericana in Mexico City. ``The State of Mexico is what they are
betting on, what they have their eye on. It is one of the largest states
in terms of population, with a solid and powerful P.R.I. structure,
electoral support base and political mobilization. Even if they would
put Trump's cousin as candidate for governor, P.R.I. would win.''

The cabinet shuffle occurred a day before the Mexican government was
scheduled to present its 2017 budget, a job that now shifts to the new
finance minister, José Antonio Meade, whose appointment was announced at
the news conference.

Mr. Meade, 47, who has a doctorate in economics from Yale University,
has been a significant player in the last two administrations. He was
finance minister under Mr. Peña Nieto's predecessor, Felipe Calderón,
and Mr. Peña Nieto made him the foreign minister in 2012. In that post,
he broke from his reputation as an apolitical technocrat when he led an
angry defense of the government against a United Nations report
condemning Mexico for its use of torture on suspects in detention.

Thirteen months ago, Mr. Peña Nieto appointed him to head the Social
Development Ministry, a post that elevated Mr. Meade to the level of a
possible presidential candidate as he traveled around the country
promoting the government's programs for the poor.

Advertisement

\protect\hyperlink{after-bottom}{Continue reading the main story}

\hypertarget{site-index}{%
\subsection{Site Index}\label{site-index}}

\hypertarget{site-information-navigation}{%
\subsection{Site Information
Navigation}\label{site-information-navigation}}

\begin{itemize}
\tightlist
\item
  \href{https://help.nytimes.com/hc/en-us/articles/115014792127-Copyright-notice}{©~2020~The
  New York Times Company}
\end{itemize}

\begin{itemize}
\tightlist
\item
  \href{https://www.nytco.com/}{NYTCo}
\item
  \href{https://help.nytimes.com/hc/en-us/articles/115015385887-Contact-Us}{Contact
  Us}
\item
  \href{https://www.nytco.com/careers/}{Work with us}
\item
  \href{https://nytmediakit.com/}{Advertise}
\item
  \href{http://www.tbrandstudio.com/}{T Brand Studio}
\item
  \href{https://www.nytimes.com/privacy/cookie-policy\#how-do-i-manage-trackers}{Your
  Ad Choices}
\item
  \href{https://www.nytimes.com/privacy}{Privacy}
\item
  \href{https://help.nytimes.com/hc/en-us/articles/115014893428-Terms-of-service}{Terms
  of Service}
\item
  \href{https://help.nytimes.com/hc/en-us/articles/115014893968-Terms-of-sale}{Terms
  of Sale}
\item
  \href{https://spiderbites.nytimes.com}{Site Map}
\item
  \href{https://help.nytimes.com/hc/en-us}{Help}
\item
  \href{https://www.nytimes.com/subscription?campaignId=37WXW}{Subscriptions}
\end{itemize}
