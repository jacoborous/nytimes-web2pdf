Sections

SEARCH

\protect\hyperlink{site-content}{Skip to
content}\protect\hyperlink{site-index}{Skip to site index}

\href{https://www.nytimes.com/section/world/middleeast}{Middle East}

\href{https://myaccount.nytimes.com/auth/login?response_type=cookie\&client_id=vi}{}

\href{https://www.nytimes.com/section/todayspaper}{Today's Paper}

\href{/section/world/middleeast}{Middle East}\textbar{}David Friedman,
Choice for Envoy to Israel, Is Hostile to Two-State Efforts

\url{https://nyti.ms/2hOS1Hx}

\begin{itemize}
\item
\item
\item
\item
\item
\end{itemize}

Advertisement

\protect\hyperlink{after-top}{Continue reading the main story}

Supported by

\protect\hyperlink{after-sponsor}{Continue reading the main story}

\hypertarget{david-friedman-choice-for-envoy-to-israel-is-hostile-to-two-state-efforts}{%
\section{David Friedman, Choice for Envoy to Israel, Is Hostile to
Two-State
Efforts}\label{david-friedman-choice-for-envoy-to-israel-is-hostile-to-two-state-efforts}}

By \href{https://www.nytimes.com/by/isabel-kershner}{Isabel Kershner}
and \href{http://www.nytimes.com/by/sheryl-gay-stolberg}{Sheryl Gay
Stolberg}

\begin{itemize}
\item
  Dec. 16, 2016
\item
  \begin{itemize}
  \item
  \item
  \item
  \item
  \item
  \end{itemize}
\end{itemize}

JERUSALEM --- He is president of the American fund-raising arm for a
yeshiva in a settlement deep in the West Bank headed by a militant rabbi
who has
\href{http://www.israelhayom.com/site/newsletter_article.php?id=230}{called
for}Israeli soldiers to refuse orders to evacuate settlers.

He writes a
\href{http://www.nytimes.com/interactive/2016/12/16/world/middleeast/David-Friedman-Israel-Palestinians-Trump-quotes.html?hp\&action=click\&pgtype=Homepage\&clickSource=story-heading\&module=second-column-region\&region=top-news\&WT.nav=top-news}{column
for a right-wing Israeli news site} in which he has accused President
Obama of ``blatant anti-Semitism,'' dismissed the two-state solution to
the Israeli-Palestinian conflict, likened a liberal American-Jewish
group to ``kapos'' who cooperated with the Nazis, and said American
Jewish leaders ``failed'' Israel on the Iran nuclear deal.

He also supports United Hatzalah, an Israeli emergency medical services
group that prides itself on integrating Arab and Druze volunteers;
helped build a \$42 million village for disabled children --- Bedouin
and Jewish --- in the Negev Desert; and is known as an affable host of
large holiday meals at the penthouse apartment he owns in a well-heeled
Jerusalem neighborhood.

Now, David M. Friedman, an Orthodox Jewish bankruptcy lawyer from Long
Island, is Donald J. Trump's pick for ambassador to Israel, despite his
lack of diplomatic experience and frequent statements that flout decades
of bipartisan American policy.

``Bankruptcy law and involvement with settlements are not normally seen
as an appropriate qualifications for the job,'' one of its former
occupants, Martin S. Indyk, said on Friday. ``But then these are not
normal times.''

Mr. Friedman, 58, has done legal work for Mr. Trump since at least 2001,
when he handled negotiations with bondholders on Mr. Trump's struggling
casinos in Atlantic City. Mr. Friedman represented Mr. Trump's personal
interests in the bankruptcies of the casinos in 2004, 2009 and 2014.

Their relationship was cemented in 2005, friends said, when Mr. Trump
traveled three hours in a snowstorm to pay a condolence call on Mr.
Friedman after the death of his father, a prominent Long Island rabbi.

``He was very taken by Trump spending almost all day just to pay the
shiva,'' said Yossi Kahana, one of the two friends who described the
visit, using the Hebrew term for the week of mourning. ``Barely any
people came, and here is Trump, coming and sitting with him and talking
about things that are important to both of them, their values, their
fathers and their legacies.''

Mr. Friedman did not respond to an interview request made to his office.

A person close to the Trump transition who spoke on the condition of
anonymity said the ambassadorship had been negotiated directly between
the two men over many months. Mr. Friedman, who donated a total of
\$50,000 to the Trump campaign and the Republican National Committee in
2016, according to federal election records, had been openly saying even
before the election that the job --- one of the most sensitive and high
profile in the diplomatic corps --- would be his, according to friends.

Image

David M. FriedmanCredit...Kasowitz, Benson, Torres \& Friedman L.L.P.,
via Associated Press

Israel's conservative settlement supporters and their American backers
rejoiced at the selection, while believers in a Palestinian state and
the American-brokered peace process were perplexed and close to despair.
Mr. Friedman is a staunch opponent of basic tenets of Washington's
longstanding approach to much of the ambassadorial portfolio.

He refers to the West Bank by its biblical name, Judea and Samaria,
something hard to imagine his predecessors doing publicly. Upon being
nominated Thursday night, he said he looked forward to working ``from
the U.S. Embassy in Israel's eternal capital, Jerusalem,'' rather than
Tel Aviv, where the American Embassy has been for decades, under the
State Department's insistence that the holy city's status be determined
as part of a broader deal between Israel and the Palestinians.

The State Department has not allowed its ambassadors to set foot in West
Bank settlements. Tax forms list Mr. Friedman as president of the
American Friends of Bet El Yeshiva, which has raised about \$2 million a
year in recent years. He is also described as president of
\href{http://betelinstitutions.com/2016-dinner-reservations/}{Bet El
Institutions}, which supports, among other things, the news site for
which Mr. Friedman wrote columns, IsraelNationalNews.com, known as Arutz
7.

Beit El, as the settlement is more usually spelled, was founded in 1977
and is now home to about 7,000 religious residents. It was a hotbed of
controversy in 2012 when the Israeli authorities followed a court order
to
\href{http://www.nytimes.com/2012/06/27/world/middleeast/jewish-settlers-begin-evacuation-of-ulpana.html}{evacuate
30 families} from five buildings built illegally on private Palestinian
land.

According to an \href{http://www.the7eye.org.il/181976}{investigation}
by The Seventh Eye, an Israeli magazine, the contested neighborhood was
built by a company linked to the one registered in the Marshall Islands
that controls Arutz 7.

Baruch Gordon, the director of development for Bet El Institutions, told
Arutz 7 on Friday that it was ``proud to be closely associated with Mr.
Friedman,'' calling him ``a pioneer philanthropist and builder of Jewish
institutions and housing projects in Judea and Samaria (a.k.a. the `West
Bank') and throughout the country.''

Mr. Friedman, whose middle name is Melech --- Hebrew for king --- grew
up in North Woodmere, N.Y., one of four children of Rabbi Morris S.
Friedman, who held the pulpit at Temple Hillel there for 46 years. In
October 1984, President Ronald Reagan visited the synagogue and went to
the Friedman family home for lunch, perhaps an early political influence
on the ambassador-desginate.

He graduated from New York University School of Law in 1981, and has
worked since 1994 at Kasowitz, Benson, Torres \& Friedman L.L.P., where
he is a partner.

The firm represented Mr. Trump in his unsuccessful libel lawsuit against
a former New York Times reporter, Timothy L. O'Brien, and its founding
partner, Marc E. Kasowitz, twice this year threatened to sue The Times
in relation to articles it was preparing regarding Mr. Trump's
\href{https://www.nytimes.com/2016/05/15/us/politics/donald-trump-women.html}{treatment
of women} and
\href{https://www.nytimes.com/2016/10/02/us/politics/donald-trump-taxes.html?_r=0}{income
tax returns}.

Mr. Friedman's connections to Israel date back to his bar mitzvah at the
Western Wall. Friends describe him as a strong Zionist who spends many
Jewish holidays and most of his summers in his Jerusalem apartment. He
and his wife are renowned for gathering people for dinners in their
sukkah, a hut observant Jews build on their balconies during a fall
harvest festival.

\href{https://www.nytimes.com/interactive/2016/12/16/world/middleeast/David-Friedman-Israel-Palestinians-Trump-quotes.html}{}

\includegraphics{https://static01.nyt.com/images/2016/12/17/world/17FRIEDMAN-LISTY/17FRIEDMAN-LISTY-thumbLarge.jpg}

\hypertarget{david-friedman-trumps-ambassador-to-israel-on-the-issues}{%
\subsection{David Friedman, Trump's Ambassador to Israel, on the
Issues}\label{david-friedman-trumps-ambassador-to-israel-on-the-issues}}

David M. Friedman, the nominee to become the ambassador to Israel, has
contributed to a right-leaning Israeli news site.

``His whole life, he's been focused and extremely thoughtful about
Israel and about the political situation there,'' said Philip Rosen,
whose friendship with Mr. Friedman began in law school.

Anon Geva, the founder of an Israeli winery in which Mr. Friedman's
son's company invested, said Mr. Freidman had invited everyone connected
with the winery --- about 30 people --- for dinner one year in the
sukkah. Mr. Geva recalled Mr. Friedman saying that he decided to buy a
home in Jerusalem on the day in 2002 that a Palestinian suicide bomber
blew himself up at Café Moment, a popular bar in the city, killing 11
Israelis.

Mr. Kahana, who directs a task force on disabilities for the Jewish
National Fund, said Mr. Friedman and some friends raised and donated
several hundred thousand dollars to help build
\href{https://aleh.org/aleh-branches/aleh-negev-nahalat-eran/?v=7516fd43adaa}{Aleh
Negev}, the village for disabled people, a joint project of the fund and
the Israeli government.

``He visited, and I must say, he was very concerned that this village is
not only for Jewish kids, that it is also for Bedouin kids,'' Mr. Kahana
said. He called Mr. Friedman ``very generous, very caring for needy
people and especially people with disabilities.''

Mr. Rosen, who was co-chairman of Mitt Romney's presidential campaign in
2012 and Senator Marco Rubio's 2016 bid, said Mr. Friedman had developed
a strong rapport with Mr. Trump that would allow him to be effective as
his envoy. ``They've worked together closely for a very long time, and
he knows what Donald is thinking, and what Donald wants to accomplish,''
Mr. Rosen said.

Many of Mr. Friedman's views are far to the right of the stated
positions of Prime Minister Benjamin Netanyahu, who has endorsed the
principle of a Palestinian state alongside Israel. Mr. Netanyahu did not
respond to Mr. Friedman's selection, nor did Israel's Foreign Ministry.

But the deputy foreign minister, Tzipi Hotovely, who hails from the
right flank of Mr. Netanyahu's Likud Party, rushed to praise it, saying,
``His positions reflect the desire to strengthen the standing of
Israel's capital Jerusalem at this time and to underscore that the
settlements have never been the true problem in the area.''

A senior Palestinian cleric, Sheikh Ikrama Sabri, said during Friday
Prayers that if Mr. Friedman managed to move the embassy to Jerusalem,
``the U.S. is declaring a new war on the Palestinians and all Muslim
Arabs.''

Saeb Erekat, the secretary general of the Palestine Liberation
Organization, told reporters in the West Bank on Friday that Mr. Trump's
appointments were ``his business,'' but that it was ``not up to Trump or
anybody else'' to take steps like moving the embassy to Jerusalem.

Daniel C. Kurtzer, who served President George W. Bush as ambassador to
Israel from 2001 to 2005, was alarmed by the appointment.

``He has made clear that he will appeal to a small minority of Israeli
--- and American --- extremists, ignoring the majority of Israelis who
continue to seek peace,'' Mr. Kurtzer, now a professor at Princeton,
said in an interview. ``Friedman's appointment as ambassador runs
directly contrary to Mr. Trump's professed desire to make the `ultimate
deal' between Israelis and Palestinians.''

Advertisement

\protect\hyperlink{after-bottom}{Continue reading the main story}

\hypertarget{site-index}{%
\subsection{Site Index}\label{site-index}}

\hypertarget{site-information-navigation}{%
\subsection{Site Information
Navigation}\label{site-information-navigation}}

\begin{itemize}
\tightlist
\item
  \href{https://help.nytimes.com/hc/en-us/articles/115014792127-Copyright-notice}{©~2020~The
  New York Times Company}
\end{itemize}

\begin{itemize}
\tightlist
\item
  \href{https://www.nytco.com/}{NYTCo}
\item
  \href{https://help.nytimes.com/hc/en-us/articles/115015385887-Contact-Us}{Contact
  Us}
\item
  \href{https://www.nytco.com/careers/}{Work with us}
\item
  \href{https://nytmediakit.com/}{Advertise}
\item
  \href{http://www.tbrandstudio.com/}{T Brand Studio}
\item
  \href{https://www.nytimes.com/privacy/cookie-policy\#how-do-i-manage-trackers}{Your
  Ad Choices}
\item
  \href{https://www.nytimes.com/privacy}{Privacy}
\item
  \href{https://help.nytimes.com/hc/en-us/articles/115014893428-Terms-of-service}{Terms
  of Service}
\item
  \href{https://help.nytimes.com/hc/en-us/articles/115014893968-Terms-of-sale}{Terms
  of Sale}
\item
  \href{https://spiderbites.nytimes.com}{Site Map}
\item
  \href{https://help.nytimes.com/hc/en-us}{Help}
\item
  \href{https://www.nytimes.com/subscription?campaignId=37WXW}{Subscriptions}
\end{itemize}
