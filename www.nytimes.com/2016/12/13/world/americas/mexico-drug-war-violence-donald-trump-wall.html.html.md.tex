Sections

SEARCH

\protect\hyperlink{site-content}{Skip to
content}\protect\hyperlink{site-index}{Skip to site index}

\href{https://www.nytimes.com/section/world/americas}{Americas}

\href{https://myaccount.nytimes.com/auth/login?response_type=cookie\&client_id=vi}{}

\href{https://www.nytimes.com/section/todayspaper}{Today's Paper}

\href{/section/world/americas}{Americas}\textbar{}Mexico Grapples With a
Surge in Violence

\url{https://nyti.ms/2hzagR5}

\begin{itemize}
\item
\item
\item
\item
\item
\end{itemize}

Advertisement

\protect\hyperlink{after-top}{Continue reading the main story}

Supported by

\protect\hyperlink{after-sponsor}{Continue reading the main story}

\hypertarget{mexico-grapples-with-a-surge-in-violence}{%
\section{Mexico Grapples With a Surge in
Violence}\label{mexico-grapples-with-a-surge-in-violence}}

\includegraphics{https://static01.nyt.com/images/2016/12/14/us/14mexico4/14mexico4-articleLarge.jpg?quality=75\&auto=webp\&disable=upscale}

By \href{http://www.nytimes.com/by/kirk-semple}{Kirk Semple}

\begin{itemize}
\item
  Dec. 13, 2016
\item
  \begin{itemize}
  \item
  \item
  \item
  \item
  \item
  \end{itemize}
\end{itemize}

CIUDAD JUÁREZ, Mexico --- Five men shot dead in a barbershop, their
bodies slumped near the doorway. A decapitated body dumped next to a
housing development. Three others killed behind a pool hall and several
more in a bar called Tres Mentiras, or Three Lies.

By the end of October, at least 96 people had been killed in the border
city of Ciudad Juárez. It was the highest monthly tally since 2012,
sowing fears of a return to the gangland mayhem that once earned this
city the title of the most violent place in the world.

Back then, the bloodshed in this city was in a class of its own. But now
it has company, with other Mexican cities that are as bad or worse. In
the last year, the number of homicides around Mexico has soared to
levels not seen in several years.

In the first 10 months of this year, there were 17,063 homicide cases in
Mexico, already more than last year's total and the highest 10-month
tally since 2012. The relapse in security has unnerved Mexico and led
many to wonder whether the country is on the brink of a bloody, all-out
war between criminal groups.

``It's a trauma, it's a kind of fear, among all of us who saw a killing,
who heard gunshots,'' said Carlos Nájera, an activist in Juárez.
``Everyone's worried about a slide to the past.''

The surge in violence around Mexico reflects an increasingly volatile
criminal landscape and the limitations of North America's
counternarcotics strategy, and it has contributed to the plummeting
approval ratings of President Enrique Peña Nieto.

A longstanding cornerstone of the Mexican government's fight against
organized crime --- backed by hundreds of millions of dollars in
\href{http://www.nytimes.com/2011/08/07/world/07drugs.html}{American
aid} --- has been to aim at the kingpins, on the theory that
\href{https://www.nytimes.com/2016/01/09/world/americas/El-Chapo-captured-mexico.html}{cutting
off the head} will wither the body. But the tactic has helped to
fragment monolithic, hierarchical criminal enterprises into an array of
groups that are more violent and uncontrollable, analysts said.

The rising insecurity poses
\href{https://www.nytimes.com/2016/11/10/world/americas/mexico-donald-trump-peso.html}{a
problem for President-elect Donald J. Trump}, who has offered few
insights into how he intends to approach the battle against
narco-trafficking and crime in the hemisphere.

His campaign language suggested
\href{https://www.nytimes.com/2016/11/15/world/americas/mexico-donald-trump-enrique-pena-nieto.html}{a
strategy of containment}, its centerpiece being the construction of
\href{https://www.nytimes.com/2016/09/02/us/us-mexico-border-wall-tunnels.html?rref=collection\%2Ftimestopic\%2FMexican\%20Drug\%20Trafficking\&action=click\&contentCollection=world\&region=stream\&module=stream_unit\&version=latest\&contentPlacement=5\&pgtype=collection}{a
wall} along the American border to thwart drugs and illegal immigration.
Some analysts worry that, as part of this approach, Mr. Trump may
withdraw the limited American support for initiatives in Mexico that
seek to strengthen the rule of law, fortify state institutions and
repair communities damaged by crime.

\includegraphics{https://static01.nyt.com/images/2016/12/13/world/14mexico/14mexico-articleLarge.jpg?quality=75\&auto=webp\&disable=upscale}

But a hands-off American approach may only give more space to violent
criminal groups in Mexico and elsewhere, destabilizing the region,
analysts said.

``A fortress America response is probably going to prove insufficient
very quickly,'' said Alejandro Hope, a leading security analyst in
Mexico.

He noted that all the heroin consumed annually in the United States,
most of which comes from Mexico, ``would fit into 1,800 to 2,000 pieces
of luggage.''

``You don't stop that with a wall,'' he said.

The Mexican government has been battling drug traffickers for decades,
but the fight acquired new intensity in 2006 when the president at the
time, Felipe Calderón, declared ``war'' on organized crime.

The Mexican military was partly successful in that approach, capturing
or killing many of the most-wanted drug traffickers in the country.
Monthly tallies of homicide cases, after climbing to a peak of 2,131 in
May 2011, eventually began to fall.

Juárez saw some of the worst of the violence, becoming a symbol of
Mexican dysfunction and tragedy: At the peak of the bloodshed, in
October 2010, the city suffered 359 homicides, according to the Security
and Justice Working Group in Juárez, an independent task force that
includes representatives of civil society and government. But an
intensive response --- including the saturation of the city by
government security forces and a robust engagement by civil society ---
helped
\href{http://www.nytimes.com/2013/12/15/world/americas/a-border-city-known-for-killing-gets-back-to-living.html}{turn
things around}.

The national kingpin strategy, however, fell short in one important
respect: Drug trafficking continued to flourish. And as leaders fell,
the large drug organizations splintered into smaller criminal gangs,
which waged battles of succession that led to greater violence.

``These groups, if you just kind of leave them alone, they're very
powerful,'' said Steven Dudley, co-director of InSight Crime, a
foundation that studies organized crime in the Americas. ``And if you
mess with them and they fragment, they're multiple, unwieldy beasts.''

Since late 2014, the homicide numbers have trended upward, an increase
that Eduardo Guerrero, a security consultant in Mexico City, has named
``the second wave of violence.''

Image

A crime scene in a neighborhood in Ciudad Juárez in 2011, when Mexico's
homicide cases reached a peak of 2,131 in May.Credit...Katie Orlinsky
for The New York Times

September --- with 1,976 homicide cases around the country --- was the
deadliest month in Mexico since May 2012, and one of the deadliest on
record, according to Mexico's Interior Ministry.

And while the violence that was a part of Mr. Calderón's presidency was
mostly concentrated in a few places, like Juárez, the recent rise in
homicides has been dispersed. Violence has erupted in places that had
experienced relatively little of it until recently, including Colima, a
once-tranquil Pacific Coast state, and the state of Guanajuato, a
growing hub of the automotive industry and the location of San Miguel de
Allende, a popular tourist destination for foreigners.

In September 2015, for instance, only two states had more than 100
homicide victims over the course of the month. In September 2016, 11
states suffered more than 100.

Though the clashes between remnant drug groups are widely thought to be
a significant cause in the rising violence, analysts and government
officials also point to other factors, including changes in political
control of state and municipal governments after recent elections.

As old political power structures make way for new ones, cooperation
between the corrupt authorities and criminal groups fall apart, analysts
said.

``Groups try to mobilize themselves to have a better position to
negotiate with the incoming government,'' Mr. Guerrero said. ``The
uncertainty of the criminals is very high, so their best weapon in the
negotiations is to `heat up the plaza.'''

In addition, criminal organizations have diversified their business
models, branching out into extortion, theft, kidnapping, prostitution,
illegal gambling, intellectual property piracy and fuel theft, analysts
said.

``What you have is a transition in the criminal underworld that is from
large-scale, relatively identifiable, hierarchically structured criminal
organizations whose business was mainly about smuggling drugs to the
United States, to diversified, smaller gangs, more local in scope, more
predatory in nature,'' Mr. Hope said.

But while the nature of Mexico's criminal operations has shifted, the
government response has not, he said. ``They're great at capturing El
Chapo but not so good at addressing the extortion of mom and pop stores
in Guerrero,'' he said, referring to the captured drug kingpin Joaquín
Guzmán Loera.

Image

Neighbors at a crime scene where two men had been shot and killed
moments before in 2012 in Ciudad Juárez.Credit...Adriana Zehbrauskas for
The New York Times

In August, the administration of Mr. Peña Nieto announced a plan to
reinforce security in 50 municipalities that account for 40 percent of
the country's homicides. The government has yet to name the
municipalities and for months offered few details about the strategy.
But in response to written questions this week, the Interior Ministry
said the plan involved the coordination of local, state and federal
authorities and included the deployment of quick-reaction forces in each
of the 50 municipalities, among other measures.

Even while acknowledging the increase in homicides, officials have
apparently sought to play it down. At a news conference last month,
Renato Sales Heredia, the national security commissioner, dismissed the
increase as ``not substantial.'' His office later clarified in an
interview that he had not been referring to this year's surging
violence, but to the smaller increase from 2014 to 2015.

Officials have also denied that the problem is widespread. In its
responses to questions this week, the Interior Ministry said that 42
percent of homicides in Mexico were concentrated in 2 percent of the
nation's municipalities, though it did not provide a time frame for that
statistic.

The responses have left many analysts to conclude that the
administration lacks a coherent strategy to address the problem.

``The only thing they do is to confront the consequences but not the
causes, and they do so in a very marginal way,'' said Francisco Rivas,
director of the Observatorio Nacional Ciudadano, a group that studies
security and justice issues in Mexico.

Still, administration officials privately express deep concern about the
rising numbers and even the possibility of a return to an all-out drug
war.

In Juárez, that possibility is palpable. This year's increase in
homicides has aggravated a kind of communal post-traumatic stress
disorder, even if the numbers are still well off the peak of the
violence that engulfed this city several years ago --- dropping to 33 in
November from 96 in October, according to El Diario de Ciudad Juárez.

``They say Juárez is reborn, it's new. Horrible lies!'' said Sergio Meza
de Anda, director of Plan Estratégico de Juárez, a community-based
organization. ``The underlying causes persist.''

He rattled off problems as much national as local, including corruption,
impunity, weak public institutions, poverty, income inequality and
insufficient development.

``The state is an accomplice to the disorder,'' he said.

The Rev. Mario Manríquez, a prominent priest in Juárez, has seen the
cost of neglect on the streets and in the homes of his parish in a
southern neighborhood of the city --- the broken families, the lives cut
short.

``The violence never went away,'' he said.

On the edge of the park in front of his church, he has built a monument
to the victims of the city's drug war. It is covered with plaques
bearing the names of some of those who have been killed. The memorial is
only three years old, but he is already running out of space for new
names.

Advertisement

\protect\hyperlink{after-bottom}{Continue reading the main story}

\hypertarget{site-index}{%
\subsection{Site Index}\label{site-index}}

\hypertarget{site-information-navigation}{%
\subsection{Site Information
Navigation}\label{site-information-navigation}}

\begin{itemize}
\tightlist
\item
  \href{https://help.nytimes.com/hc/en-us/articles/115014792127-Copyright-notice}{©~2020~The
  New York Times Company}
\end{itemize}

\begin{itemize}
\tightlist
\item
  \href{https://www.nytco.com/}{NYTCo}
\item
  \href{https://help.nytimes.com/hc/en-us/articles/115015385887-Contact-Us}{Contact
  Us}
\item
  \href{https://www.nytco.com/careers/}{Work with us}
\item
  \href{https://nytmediakit.com/}{Advertise}
\item
  \href{http://www.tbrandstudio.com/}{T Brand Studio}
\item
  \href{https://www.nytimes.com/privacy/cookie-policy\#how-do-i-manage-trackers}{Your
  Ad Choices}
\item
  \href{https://www.nytimes.com/privacy}{Privacy}
\item
  \href{https://help.nytimes.com/hc/en-us/articles/115014893428-Terms-of-service}{Terms
  of Service}
\item
  \href{https://help.nytimes.com/hc/en-us/articles/115014893968-Terms-of-sale}{Terms
  of Sale}
\item
  \href{https://spiderbites.nytimes.com}{Site Map}
\item
  \href{https://help.nytimes.com/hc/en-us}{Help}
\item
  \href{https://www.nytimes.com/subscription?campaignId=37WXW}{Subscriptions}
\end{itemize}
