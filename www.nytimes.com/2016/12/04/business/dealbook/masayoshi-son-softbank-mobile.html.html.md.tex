Sections

SEARCH

\protect\hyperlink{site-content}{Skip to
content}\protect\hyperlink{site-index}{Skip to site index}

\href{https://myaccount.nytimes.com/auth/login?response_type=cookie\&client_id=vi}{}

\href{https://www.nytimes.com/section/todayspaper}{Today's Paper}

\href{/section/business/dealbook}{DealBook}\textbar{}SoftBank's
Masayoshi Son Chases First Place With Tech Deals

\url{https://nyti.ms/2gEiOoo}

\begin{itemize}
\item
\item
\item
\item
\item
\end{itemize}

Advertisement

\protect\hyperlink{after-top}{Continue reading the main story}

Supported by

\protect\hyperlink{after-sponsor}{Continue reading the main story}

DealBook Business and Policy

\hypertarget{softbanks-masayoshi-son-chases-first-place-with-tech-deals}{%
\section{SoftBank's Masayoshi Son Chases First Place With Tech
Deals}\label{softbanks-masayoshi-son-chases-first-place-with-tech-deals}}

\includegraphics{https://static01.nyt.com/images/2016/11/03/business/DB-SOFTBANK/DB-SOFTBANK-articleLarge.jpg?quality=75\&auto=webp\&disable=upscale}

By Jonathan Soble

\begin{itemize}
\item
  Dec. 4, 2016
\item
  \begin{itemize}
  \item
  \item
  \item
  \item
  \item
  \end{itemize}
\end{itemize}

TOKYO --- Masayoshi Son is a Japanese billionaire who wants to control
the way your car talks to street lamps.

The 59-year-old technology investor --- a grandson of South Korean
immigrants who has amassed one of Japan's largest personal fortunes by
pursuing grand, Silicon Valley-inspired visions --- is best known
outside his homeland for buying the American mobile phone carrier Sprint
for \$21.6 billion in 2013.

Now, after a short flirtation with retirement, Mr. Son has embarked on
two of the most ambitious projects of his career. He is making
\href{http://www.nytimes.com/2016/07/19/business/dealbook/softbank-buys-chip-designer-arm.html}{his
largest acquisition yet}: the British microchip designer ARM Holdings,
whose products are at the core of most of the world's smartphones. And
he is collaborating with Saudi Arabia's ruling family
\href{http://www.nytimes.com/2016/10/14/business/dealbook/softbank-and-saudi-arabia-partner-to-form-giant-investment-fund.html?_r=0}{to
create} what could become the world's largest technology investment
fund.

His goal is nothing less than to change how the material world works ---
and to turn his company, the SoftBank Group, into the future's most
important technology business. Specifically, he is planning for a day
when millions of everyday objects run on chips and tiny computers that
talk to one another --- allowing street lamps, for example, to save
power by switching themselves off when cars are not around.

The bets are risky. SoftBank is buying ARM for a rich \$32 billion, and
it plans to plow another \$25 billion in the Saudi joint venture over
the next five years. Neither guarantees that SoftBank will stay globally
relevant in a fast-changing digital world.

Mr. Son's track record includes big victories, like an early investment
in the Chinese e-commerce giant Alibaba Group. But his deal for Sprint
has so far resulted mostly in red ink. And his ambitions led to the
departure of his handpicked successor, leaving questions over who will
oversee SoftBank's future.

``Son acts on instinct,'' said Satoshi Shima, a former senior lieutenant
to Mr. Son who is now a professor at Tama University. ``It's a genius
instinct, but it's not logic.''

The ARM purchase is crucial to Mr. Son's ambitions. The British company
aims to wire up self-driving cars, internet-enabled home appliances,
medical devices and even clothing.

``SoftBank has had a lot of businesses, but they've always ranked, say,
third in Japan, or fourth in the U.S.,'' Mr. Son said in July after
announcing the ARM deal. ``This is the first time we're directly
controlling a business that is No. 1 in the world.''

With a \$14.9 billion personal fortune, Mr. Son is ranked by Forbes as
Japan's second-richest man, behind Tadashi Yanai, the owner of Fast
Retailing and its Uniqlo brand. But he spent his early life in poverty:
His father started out as a bootlegger and small-time pig farmer before
finding success in restaurants and pachinko, the Japanese offshoot of
pinball. By the time he was a teenager, his family was wealthy enough
that he was able to travel to the United States for college, eventually
ending up at the University of California, Berkeley.

Mr. Son founded SoftBank in 1981 as a distributor of computer software.
For start-up money, he has told biographers, he sold a prototype
electronic translation machine to Sharp, the Japanese electronics
company, for about \$1 million. Mr. Son, who trained in economics, not
engineering, came up with the general idea for the machine while still
in college, then recruited others to create it --- an approach he has
followed throughout his career.

Takenobu Miki worked closely with Mr. Son for eight years before leaving
SoftBank in 2006 to found an online language-training start-up.
Initially, he was the only full-time employee at SoftBank, which,
despite sounding like a tech start-up, operated ``like an investment
fund, with a portfolio of holdings that Son would adjust based on
changing growth rates,'' Mr. Miki said.

One item in the portfolio was Yahoo, an early investment that earned Mr.
Son a fortune. The investment displayed Mr. Son's willingness to act on
a general conviction: in this case, that web portals would be crucial to
the early commercial development of the internet. He did relatively
little research on Yahoo itself, Mr. Miki recalled. ``He asked some
contacts in Silicon Valley which was the best web portal, and they said
Yahoo. So he bought.''

Mr. Miki also remembers working with Mr. Son to create a growth
projection for SoftBank that extended 300 years into the future.

``The goal was to become a 100 trillion-yen company,'' Mr. Miki said, an
amount equal to about \$1 trillion. ``Even at that time, Son wanted to
become No. 1 in the world.''

SoftBank declined to make Mr. Son available for an interview for this
article.

Mr. Son revels in confrontation, a trait that sets him apart in
harmony-obsessed Japan. Twice, he has threatened to set fire to himself
or the offices of Japanese telecommunications regulators --- the first
time in a dispute over access to fiber-optic cable, the second in a
fight over internet censorship. He apologized in the second instance, in
2010, calling the threat an inappropriate joke.

In 2013, he apologized again at a news conference after he became
involved in a shouting match with government officials over plans to
allocate cellular spectrum to KDDI, a SoftBank rival. ``I thought I had
grown up,'' he said.

Mr. Son has made bold, expensive acquisitions before. In 2006, he
borrowed heavily to buy the Japanese arm of Vodafone, the British
cellphone carrier, which was badly lagging two Japanese rivals. He tore
up the company's pricing strategy, using discounts to undercut the
industry's cozy near duopoly, then negotiated exclusive rights to carry
Apple's iPhone. Subscriber numbers and profits soared.

Hoping to repeat his success in the United States, SoftBank spent \$21.6
billion to acquire Sprint in 2013. But there were no exclusive iPhone
deals to be had, and American officials blocked a plan by Mr. Son to buy
another carrier, T-Mobile, which would have given the business more
scale. Sprint has hemorrhaged money --- it lost \$302 million last
quarter --- though lately its subscriber base has been expanding.

SoftBank's deal with Saudi Arabia will give Mr. Son even more money to
play with, though it could also make future investments more complicated
because the involvement of the Saudi government could lead to extra
scrutiny when striking deals in other countries.

After the Sprint deal, it seemed Mr. Son was looking ahead to
retirement. That would have fit with a life plan he drew up while still
in his 20s, in which he resolved to build a business empire and hand it
over to a successor in his 60s. In 2014, he hired a star Google
executive, Nikesh Arora, and all but anointed him to that role.

But with the ARM acquisition, the chance for another big, transformative
deal appears to have proved too tempting. Mr. Arora announced his
resignation in June, a month before the acquisition. Mr. Son said he
wanted to stay on for another decade or so to ``work on a few more crazy
ideas.''

Advertisement

\protect\hyperlink{after-bottom}{Continue reading the main story}

\hypertarget{site-index}{%
\subsection{Site Index}\label{site-index}}

\hypertarget{site-information-navigation}{%
\subsection{Site Information
Navigation}\label{site-information-navigation}}

\begin{itemize}
\tightlist
\item
  \href{https://help.nytimes.com/hc/en-us/articles/115014792127-Copyright-notice}{©~2020~The
  New York Times Company}
\end{itemize}

\begin{itemize}
\tightlist
\item
  \href{https://www.nytco.com/}{NYTCo}
\item
  \href{https://help.nytimes.com/hc/en-us/articles/115015385887-Contact-Us}{Contact
  Us}
\item
  \href{https://www.nytco.com/careers/}{Work with us}
\item
  \href{https://nytmediakit.com/}{Advertise}
\item
  \href{http://www.tbrandstudio.com/}{T Brand Studio}
\item
  \href{https://www.nytimes.com/privacy/cookie-policy\#how-do-i-manage-trackers}{Your
  Ad Choices}
\item
  \href{https://www.nytimes.com/privacy}{Privacy}
\item
  \href{https://help.nytimes.com/hc/en-us/articles/115014893428-Terms-of-service}{Terms
  of Service}
\item
  \href{https://help.nytimes.com/hc/en-us/articles/115014893968-Terms-of-sale}{Terms
  of Sale}
\item
  \href{https://spiderbites.nytimes.com}{Site Map}
\item
  \href{https://help.nytimes.com/hc/en-us}{Help}
\item
  \href{https://www.nytimes.com/subscription?campaignId=37WXW}{Subscriptions}
\end{itemize}
