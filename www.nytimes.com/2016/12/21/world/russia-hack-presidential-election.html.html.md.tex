Sections

SEARCH

\protect\hyperlink{site-content}{Skip to
content}\protect\hyperlink{site-index}{Skip to site index}

\href{https://www.nytimes.com/section/world}{World}

\href{https://myaccount.nytimes.com/auth/login?response_type=cookie\&client_id=vi}{}

\href{https://www.nytimes.com/section/todayspaper}{Today's Paper}

\href{/section/world}{World}\textbar{}Совершенное орудие: как российская
кибермощь проникла в США.

\url{https://nyti.ms/2icgw1z}

\begin{itemize}
\item
\item
\item
\item
\item
\end{itemize}

Advertisement

\protect\hyperlink{after-top}{Continue reading the main story}

Supported by

\protect\hyperlink{after-sponsor}{Continue reading the main story}

\hypertarget{ux441ux43eux432ux435ux440ux448ux435ux43dux43dux43eux435-ux43eux440ux443ux434ux438ux435-ux43aux430ux43a-ux440ux43eux441ux441ux438ux439ux441ux43aux430ux44f-ux43aux438ux431ux435ux440ux43cux43eux449ux44c-ux43fux440ux43eux43dux438ux43aux43bux430-ux432-ux441ux448ux430}{%
\section{Совершенное орудие: как российская кибермощь проникла в
США.}\label{ux441ux43eux432ux435ux440ux448ux435ux43dux43dux43eux435-ux43eux440ux443ux434ux438ux435-ux43aux430ux43a-ux440ux43eux441ux441ux438ux439ux441ux43aux430ux44f-ux43aux438ux431ux435ux440ux43cux43eux449ux44c-ux43fux440ux43eux43dux438ux43aux43bux430-ux432-ux441ux448ux430}}

\includegraphics{https://static01.nyt.com/images/2016/12/14/us/14hack-russian-translation-top/14hack-top1-sub-articleInline.jpg?quality=75\&auto=webp\&disable=upscale}

By Эрик Липтон, Дэвид Э. Сангер И Скотт Шейн

\begin{itemize}
\item
  Dec. 21, 2016
\item
  \begin{itemize}
  \item
  \item
  \item
  \item
  \item
  \end{itemize}
\end{itemize}

\href{https://www.nytimes.com/2016/12/13/us/politics/russia-hack-election-dnc.html}{Read
in English}

Вашинготон --- Когда в сентябре 2015 года, спецагент ФБР Эдриан Хокинс
позвонил в Национальный комитет Демократической партии (DNC), чтобы
передать тревожное известие об их компьютерной сети, его, естественно,
перевели в службу техподдержки. Его сообщение было коротким, но
тревожным. По крайней мере одна из компьютерных систем, принадлежащих
DNC была скомпрометирована хакерами, группой кибершпионов, связанных с
российским правительством, которых федеральные агенты называли ``The
Dukes''.

Они были хорошо известны ФБР: последние несколько лет Бюро занималось
тем, чтобы избавить от проникновения The Dukes в незасекреченные системы
электронной почты Белого дома, Госдепартамента и даже Объединенного
комитета начальников штабов с его самой защищенной сетью.

Йэерд Тэмин, сотрудник службы технической поддержки DNC, принявший
телефонный звонок, не был экспертом по кибератакам. Для начала он
``залез'' в Гугл, чтобы получить информацию о группе The Dukes и провел
поверхностный поиск в регистрационных журналах компьютерных сетей с
целью обнаружить следы киберпроникновения. По его собственному
признанию, он не стал проверять слишком внимательно, даже после того как
спецагент Хокинс перезвонил ему несколько раз в следующие несколько
недель - отчасти потому, что не был уверен, что звонивший действительно
был агентом, а не самозванцем.

``Я никак не мог определить не был ли этот звонок телефонным
розыгрышем,''

\begin{itemize}
\tightlist
\item
  написал г-н Тэмин в служебной записке, характеризующей его контакт в
  ФБР и имеющейся в распоряжении ``Нью-Йорк Таймс''.
\end{itemize}

Это был первый косвенный сигнал, свидетельствующий о кампании
кибершпионажа и информационной войны, направленной на срыв президентских
выборов 2016 года, первая в истории США попытка такого рода,
осуществляемая со стороны иностранного государства.

По мнению представителей разведки, то, что началось как операция по
сбору информации, в результате вылилось в попытку нанести вред одному
кандидату, Хиллари Клинтон и изменить ситуацию в пользу ее оппонента
Дональда Трампа.

Подобно другому известному скандалу, также связанному с президентскими
выборами в США, все началось с незаконного проникновения в DNC. В первый
раз, 44 года назад, в старом здании Комитета в комплексе Уотергейт,
взломщики заложили подслушивающие устройства и вскрыли канцелярский
шкаф. На этот раз взлом осуществлялся дистанционно, издалека, под
руководством Кремля, посредством целевого ``фишинг-мошенничества'' через
электронную почту.

Расследование этой операции русских, проведенное ``Нью-Йорк Таймс'',
основано на интервью с десятками людей - жертвами атаки, с
представителями разведки, занимавшимися расследованием этого дела и с
чиновниками Администрации президента Обамы, подыскивавшими надлежащую
форму ответа на кибератаку. Это расследование обнаружило целый ряд
пропущенных сигналов, примеров замедленной реакции и затянувшегося
периода недооценки серьезности этой кибератаки.

Некомпетентный контакт DNC с ФБР привел к тому, что самая удачная
возможность остановить проникновение русских была упущена. Неспособность
оценить масштаб атак практически перечеркнула усилия по минимизированию
их последствий. А нежелание Белого дома дать решительный ответ означает,
что русские так и не поплатились за свои действия, что может привести к
критическим результатам при отражении будущих кибератак.

Неспешный подход ФБР привел к тому, что российские хакеры получили
возможность в течение почти семи месяцев свободно прогуливаться по сетям
Комитета, пока высшие чиновники DNC не были оповещены о хакерской атаке
и привлекли, наконец, специалистов для защиты своих систем связи. Тем
временем хакеры перешли на цели за пределами DNC, в частности, на Джона
Д. Подеста, главу избирательного штаба г-жи Клинтон. Несколько месяцев
спустя его личный электронный почтовый ящик был взломан хакерами. Даже
г-н Подеста, человек искушенный в этих вопросах, сам имеющий доступ к
конфиденциальной информации, автор доклада о кибербезопасности,
написанном в 2014 году для президента Обамы не смог в полной мере
оценить серьезность происходящего.

Прошлым летом Демократы в беспомощной ярости наблюдали как их личная
почта и конфиденциальные документы, добытые агентами российской
разведки, день за днем появлялись на Wikileaks и на других сайтах, а
затем и в репортажах американских СМИ, не исключая и ``Таймс''. Г-н
Трамп в ходе своей кампании с удовольствием цитировал многие из
похищенных писем.

Как результат происходящего, последовала отставка г-жи Дебби Вассерман
Шульц, члена Палаты Представителей и председателя DNC, а вслед за ней и
ее ближайших помощников по партии. Ведущие деятели Демократической
партии были выведены из игры на самом пике кампании, вынужденные
замолчать под напором откровений, обнаруженных в компрометирующих
электронных письмах, или необходимостью принятия срочных мер для борьбы
с хакерами. Хоть и привлекшие меньшее внимание публики, конфиденциальные
документы, добытые российскими хакерами у одной из родственных DNC
организаций, в Комитете Демократической партии США по выборам в
Конгресс, выплыли в дюжине штатов в ходе выборных компаний в Конгресс,
омрачив некоторые из них обвинениями в совершении недостойных поступков.

Совсем недавно скептически настроенный вновь избранный президент,
разведывательные органы страны и две основные политические партии
оказались втянутыми в необычайную публичную дискуссию по вопросу: какие
существуют свидетельства того, что президент Владимир Путин вышел за
рамки просто шпионажа и намеренно попытался подорвать американскую
демократию и лично подобрать победителя президентских выборов.

Многие из ближайших помощников г-жи Клинтон сходятся во мнении, что
российское вмешательство оказало очень серьезное влияние на выборы,
признавая при этом, что и другие факторы - слабость г-жи Клинтон как
кандидата, история с сервером ее личной почты, публичные заявления
Директора ФБР Джеймса Коми относительно ее обращения с конфиденциальной
информацией - также сыграли важную роль.

Не имея возможности точно оценить конечный результат хакерской атаки,
можно сделать следующий вывод: низкозатратное, высокоэффективное оружие,
которое Россия испытала на выборах от Украины до Европы было нацелено на
Соединенные Штаты, произведя сокрушительный эффект. Для России, с ее
ослабленной экономикой и ядерным арсеналом, который она не может
использовать кроме как в полномасштабной войне, кибероружие оказалось
самым совершенным: недорогое, предусмотреть приближение которого или
обнаружить трудно.

``Ни у кого не должно быть никаких сомнений,'' - сказал Адмирал Майкл С.
Роджерс, директор Национального агентства безопасности и командующий
Кибернетическим Командованием США на конференции, состоявшейся после
выборов. ``Это не было что-то, сделанное непреднамеренно, это не было
чем-то, сделанным случайно, это не была цель, избранная произвольно,''

\begin{itemize}
\tightlist
\item
  продолжал он. ``Это было сознательным усилием одного государства с
  целью добиться конкретного эффекта.''
\end{itemize}

В душах людей, чьи электронные письма были украдены, эта новая форма
политической диверсии оставила след от шока и нанесла большой вред их
профессиональной деятельности. Нира Тэнден, президент Центра прогресса
США и один из ключевых сторонников г-жи Клинтон, вспоминает, как она
вошла в заполненный людьми предвыборный офис Хиллари Клинтон и испытала
чувство глубокого унижения, увидев свое лицо на экранах телевизоров,
когда политические обозреватели обсуждали ``утечку'' из ее электронной
переписки, в которой она назвала инстинкты г-жи Клинтон далекими от
оптимальных.'' ``Это было как ежедневный удар ниже пояса,'' сказала г-жа
Тэнден. ``Это было самым ужасным испытанием в моей жизни.''

Соединенные Штаты тоже наносили киберудары. И в прошедшие десятилетия
ЦРУ делало попытки подрывать избирательные процессы в других странах.
Однако российская кибератака все больше воспринимается всем политическим
спектром как поворотное событие, несущее серьезную угрозу. За одним
примечательным исключением: Г-н Трамп отверг выводы разведывательных
агентств, которыми он скоро будет руководить как ``смешными'',
настаивая, что хакер мог быть американцем, или китайцем, но об этом
``они понятия не имеют''. Г-н Трамп упоминал известные разногласия между
агентствами относительно того, намеревался ли Г-н Путин помочь ему в
избрании на пост президента. Во вторник официальный представитель
российского правительства эхом отозвалась на насмешки Трампа.

«Эта история со ``взломами'' похожа на банальную разборку между
американскими силовиками за сферы влияния», написала в Facebook Мария
Захарова, официальный представитель МИД России.

В прошедший уикенд четыре сенатора, два Республиканца и два Демократа
объединились, чтобы начать расследование, намеренно игнорируя
скептические замечания г-на Трампа.

``Демократы и Республиканцы должны работать вместе и под юрисдикцией
Конгресса с тем, чтобы самым тщательным образом расследовать эти
недавние инциденты и выработать кардинальные решения для предотвращения
и защиты от дальнейших кибератак'', сказали сенаторы Джон Маккейн,
Линдси Грэхем, Чак Шумер и Джек Рид.

``К этому вопросу не должно быть одностороннего подхода'', - сказали
они. ``Ставки слишком высоки для нашей страны''.

\hypertarget{ux43cux438ux448ux435ux43dux44c-ux434ux43bux44f-ux432ux437ux43bux43eux43cux43eux432}{%
\subsection{\texorpdfstring{\textbf{МИШЕНЬ ДЛЯ
ВЗЛОМОВ}}{МИШЕНЬ ДЛЯ ВЗЛОМОВ}}\label{ux43cux438ux448ux435ux43dux44c-ux434ux43bux44f-ux432ux437ux43bux43eux43cux43eux432}}

В подвале штаб-квартиры DNC под большущим портретом Барака Обамы стоит
канцелярский шкаф 1960-х годов, на нижнем выдвижном ящике которого не
хватает ручки. Только газетная статья в рамочке, висящая на стене дает
представление о значении этого устаревшего предмета офисной мебели.

``Ответственный сотрудник службы безопасности Республиканской партии -
один из пяти арестованных по делу о незаконной прослушке'', - гласит
заголовок статьи Боба Вудварда и Карла Бернстайна на первой полосе
``Washington Post'' oт 19 июня 1972 года.

Эндрю Браун, директор отдела технологий DNC, ему 37, и он родился много
позже этого знаменитого взлома. Но когда он начал готовиться к
электоральному циклу этого года, он хорошо сознавал, что DNC снова может
стать объектом для взломщиков.

Было стремление обеспечить надежную защиту DNC от киберпроникновения, а
затем наступила реальность, г-н Браун и его начальство признают: DNC -
некоммерческая организация, зависящая от поступления взносов, бюджет ее
службы безопасности составляет только крупицу того, что необходимо для
корпорации такого размера.

``У нас никогда не было достаточно средств, чтобы делать то, что было
необходимо'', - говорит г-н Браун.

У DNC была стандартная система фильтрования почты от спама,
предназначенная также для блокировки атак ``фишинг-мошенников'' и
вредоносных программ, замаскированных под легитимные электронные
сообщения. Однако, как подтверждают внутренние меморандумы DNC, когда
российские хакеры взялись за DNC, у комитета не было современных
продвинутых систем, способных отслеживать подозрительный поток
сообщений.

Г-н Тэмин, который принял звонок от агента ФБР и работает под началом
г-на Брауна, на самом деле не является штатным сотрудником DNC. Он
работает в подрядной фирме, базирующейся в Чикаго. На его собственное
усмотрение было оставлено решение, как отреагировать на предупреждение,
и более того, решить был ли человек, позвонивший на коммутатор DNC
агентом ФБР, или нет.

``ФБР считает, что в сети DNC есть по крайней мере один
``скомпрометированный'', т.е. подвергшийся проникновению извне,
компьютер, и ФБР хочет знать, известно ли это DNC, и если известно, то
что DNC предпринимает в этой связи'', написал г-н Тэмин в служебной
записке о своих контактах с ФБР. Он также добавил: ``спецагент
рекомендовал мне проследить за определенным типом вредоносной программы,
известной в американских службах разведки и кибербезопасности под
названием ``Dukes''.

Отчасти проблема заключается в том, что спецагент Хокинс не явился лично
в офис DNC . В тоже время он не мог известить их об опасности по
электронной почте, т.к. это могло стать предупреждением хакерам о том,
что ФБР знает об их проникновении в систему.

Первоначальное сканирование системы, проведенное г-ном Тэмином, не дало
никаких результатов, поскольку он пользовался далеким от оптимального
инструментарием и данными, недостаточными для постановки задачи по
поиску цели. Поэтому, когда спецагент Хокинс звонил снова в октябре и
оставлял для Тэмина сообщения на автоответчике с просьбой перезвонить
ему, ``Я не стал ему перезванивать, потому что мне нечего было ему
доложить,'' - писал Тэмин в служебной записке.

В ноябре спецагент Хокинс позвонил, чтобы сообщить еще более угрожающую
новость. Один из компьютеров в системе DNC ``позвонил домой'', под домом
подразумевалась Россия'', говорится в служебной записке Тэмина, где
имеется в виду программное обеспечение, которое пересылало информацию в
Москву. ``Спецагент Хокинс добавил, что, по мнению ФБР, эти ``звонки
домой'' могут быть признаком атаки, осуществляемой каким-то
государством''.

Г-н Браун знал, что г-н Тэмин, который отказался от комментариев по
этому вопросу, принимал телефонные звонки из ФБР. Но он сам был занят
решением другой проблемы: свидетельствами, указывающими на то, что
сенатор от штата Вермонт Берни Сандерс, главный конкурент г-жи Клинтон в
лагере Демократов, незаконно получил доступ к данным по проведению ее
кампании.

Г-жа Вассерман Шульц и Эйми Дэси, в то время занимавшие посты,
соответственно, председателя и президента DNC, сообщили в интервью, что
ни одна из них не была оповещена о ранних сигналах, указывавших на
возможный захват их компьютерной сети извне.

Шон Генри, в свое время возглавлявший службу кибербезопасности ФБР, ныне
занимает пост президента CrowdStrike Services, фирму, специализирующуюся
на кибербезопасности и нанятую DNC в апреле, сказал, что он был
шокирован, узнав, что ФБР не удосужилось позвонить кому-то из
руководства DNC или послать своего агента в штаб-квартиру партии, чтобы
заставить принять более решительные меры.

``Ведь речь не идет об офисе, затерянном где-то в лесах Монтаны'', -
сказал г-н Генри. ``Мы говорим об офисе в полумиле от здания ФБР''.

``Это ведь не какая-то семейная забегаловка или библиотека на вашей
улице, это наиважнейшая часть инфраструктуры США, т.к. она имеет
непосредственное отношение к электоральному процессу, нашим выборным
чиновникам, нашему законодательному процессу и к нашим органам
исполнительной власти'', - добавил он. ``По моему мнению, это
серьезнейший вопрос, имеющий отношение к самым высоким сферам, и если
через пару месяцев мы не увидим результатов, кому-то придется вынести
этот вопрос на более высокий уровень.''

ФБР отказалось комментировать действия агентства в связи с хакерской
атакой. ``ФБР очень серьезно относится к любому незаконному
проникновению в системы общественного или частного сектора'', - сообщило
бюро в официальном заявлении, подчеркивая, что агенты ``будут продолжать
делиться информацией'', чтобы помочь жертвам атак ``обеспечивать
безопасность их систем от действий настойчивых киберпреступников''.

К началу марта г-н Тэмин и его команда как минимум два раза встречались
с сотрудниками ФБР и убедились, что агент Хокинс на самом деле является
сотрудником федеральной службы. Но к тому времени ситуация приняла
совершенно жуткий оборот.

Вторая команда хакеров, связанных с Россией, начала нацеливаться на DNC
и других игроков на политическом поле, проявляя особое внимание к
Демократам. Билли Райнхарт, бывший региональный директор DNC, который в
то время работал в избирательном штабе г-жи Клинтон, получил по
электронной почте странное предупреждение от Google.

``Кто-то только что воспользовался вашим паролем для входа в ваш личный
кабинет в Google'', гласило сообщение от 22 марта. В нем также
сообщалось, что попытка входа осуществлялась из Украины. ``Google пресек
эту попытку, вам следует немедленно изменит свой пароль''. Г-н Райнхарт
был в это время на Гавайях, он вспоминает, что проверял свою почту в 4
часа утра, чтобы прочесть сообщения от своих коллег на Восточном
побережье. Не долго думая над полученным предупреждением, он кликнул
кнопку ``заменить пароль'' и, насколько он помнит, в полусонном
состоянии, набрал новый пароль. Только несколько месяцев спустя он
узнал, что в тот момент он дал российским хакерам доступ к учетной
записи своей электронной почты.

Сотни подобных ``фишинговых'' электронных сообщений рассылались
политическим деятелям в США, включая похожее сообщение, отправленное 19
марта г-ну Подеста, главе предвыборного штаба Хиллари Клинтон. Принимая
во внимание огромное количество электронных сообщений, полученных г-ном
Подеста на его личный адрес, доступ к нему имели и несколько его
помощников. Один из них обратил внимание на предупреждающее сообщение.
Он переправил его сотруднику техподдержки, чтобы убедиться, что это
сообщение было легитимным, прежде чем кто-нибудь нажмет кнопку ``сменить
пароль''.

``Это сообщение легитимное'', - Чарльз Делаван, сотрудник избирательного
штаба Клинтон ответил другому помощнику г-на Подеста, заметившему
предупреждение, - ``Джону нужно немедленно сменить свой пароль''.

Один клик, и накопленные за десяток лет электронные письма, которые г-н
Подеста сохранил в своем Gmail-аккаунте, всего около 60 тысяч сообщений,
стали доступны для российских хакеров. В интервью г-н Делаван признался,
что вредный совет он дал коллегам в результате опечатки. Он знал, что
это была фишинг-атака и хотел напечатать слово ``нелигитимное''
сообщение. Эта ошибка не дает ему покоя с тех пор.

В ходе этой второй волны хакерам удалось получить доступ к системе
Комитета Демократической партии по выборам в Конгресс, а затем, через
виртуальную корпоративную сеть, и в главный компьютер сети DNC.

ФБР тоже обратило внимание на этот всплеск активности и вновь обратилось
к г-ну Тэмину, чтобы предупредить его. А г-н Тэмин все еще не видел
причины для беспокойства: он обнаружил, что копии фишинг-сообщений осели
в спам-фильтрах систем DNC. Но он сказал, что не видел оснований
считать, что в компьютерные системы было совершено проникновение извне.

К середине апреля наконец был достигнут определенный прогресс: через
семь месяцев после первого предупреждения DNC, в конце концов,
установила ``надежный комплекс средств мониторинга'', - говорится в
служебной записке Тэмина.

\hypertarget{ux43eux442ux442ux430ux447ux438ux432ux430ux43dux438ux435-ux442ux430ux43aux442ux438ux43aux438-ux441ux43aux440ux44bux442ux43dux43eux441ux442ux438}{%
\subsection{\texorpdfstring{\textbf{ОТТАЧИВАНИЕ ТАКТИКИ
СКРЫТНОСТИ}}{ОТТАЧИВАНИЕ ТАКТИКИ СКРЫТНОСТИ}}\label{ux43eux442ux442ux430ux447ux438ux432ux430ux43dux438ux435-ux442ux430ux43aux442ux438ux43aux438-ux441ux43aux440ux44bux442ux43dux43eux441ux442ux438}}

В течение двух десятилетий США предупреждали, что российские
разведслужбы пытались взломать самые важные и секретные компьютерные
сети США. Однако россиянам всегда удавалось быть на шаг впереди.

Их первая крупная атака была обнаружена 7 октября 1996 года, когда
оператор ЭВМ Высшего горного училища Колорадо обнаружил следы ночной
работы компьютера, которые он не мог объяснить. У института был крупный
контракт с военно-морским ведомством и оператор предупредил об этом
случае свои контакты в ВМФ. Но, как и спустя два десятилетия в DNC,
``никому не удалось сопоставить все факты'', - сказал Томас Рид, ученый
в Королевском колледже в Лондоне, изучавший атаку.

Учёные дали атаке название -- «Лунный свет» - и провели два года, часто
работая днем и ночью, прослеживая, как хакеры перескакивали из ВМФ в
Департамент энергетики, в ВВС и НАСА. В конце концов, они пришли к
выводу, что было украдено столько файлов, что если их распечатать и
сложить вместе, то кипа будет выше, чем Монумент Вашингтону.

Оружейные проекты и разработки улетучивались целиком, и это был только
первый звоночек, предупреждавший о том, что надвигалось: эскалация
спланированных кибератак по всему миру.

Однако в течение многих лет россияне оставались в тени благодаря
китайцам - бравшим на себя большие риски и потому часто попадавшимся.
Они украли технологические разработки истребителя F-35, фирменные
секреты прокатки стали, даже чертежи газопроводов, питающих большую
часть Соединенных Штатов. А во время президентских выборов 2008 года
китайская разведка взломала компьютеры в предвыборном штабе г-на Обамы и
г-на Маккейна с внутренними документами, определяющими позицию
кандидатов по различным вопросам, с рекламными и агитационными
материалами.

Правда, ничего из этого не было опубликовано.

Конечно, русские никуда не делись. ``Они были просто гораздо более
скрытными'', - сказал Кевин Мэндая, бывший офицер разведки
военно-воздушных сил, большую часть жизни занимавшийся отражением
российских кибератак. Он основал специализирующуюся на кибербезопасности
фирму Mandiant, которая в настоящее время является подразделением
FireEye. Она обеспечивала защиту президентской кампании Хиллари Клинтон.

Русские совершали свои атаки скорее в политических целях. Кибератака
против Эстонии, бывшей советской республики, вступившей в НАТО,
осуществленная в 2007 году, продемонстрировала, что Россия могла бы
парализовать эту страну без вторжения в нее. В следующем году кибератаки
были использованы во время войны России с Грузией.

Однако американские официальные лица не могли представить, что русские
посмеют опробовать эти методы в Соединенных Штатах. Американские власти
были в значительной степени сосредоточены на предотвращении
``кибернетического Перл-Харбора'' -- полное отключение электросетей и
сотовой связи, - того, о чём предупреждал бывший министр обороны Леон
Панетта.

Однако в 2014 и 2015 годах русская хакерская группа начала
систематические атаки на Государственный департамент, Белый дом и
Объединенный комитет начальников штабов. ``Каждый раз их вылазки
увенчивались успехом в той или иной форме'', написали недавно Майкл
Салмайер, бывший киберэксперт министра обороны, и Бен Бьюкенен в
докладе, который скоро будет опубликован фондом Карнеги. В настоящее
время оба участвуют в проекте Harvard Cyber Security.

Русские возвели свою скрытность в абсолют, обманом вынуждая
правительственные компьютеры передавать данные, одновременно маскируя
электронные сигналы ``управления и контроля'' таким образом, что они
предупреждали о попытках пресекать их вредоносные действия. В результате
Государственный департамент был настолько парализован, что ему
приходилось неоднократно закрывать свои системы, чтобы избавиться от
вторгшихся злоумышленников. В какой-то момент чиновникам, отправившимся
в Вену с госсекретарем Джоном Керри на переговоры с Ираном по ядерному
вопросу, пришлось создать платные аккаунты Gmail, чтобы просто общаться
друг с другом и с сопровождающими их журналистами.

Г-ну Обаме регулярно докладывали об этих случаях, но он принял решение,
о котором многие в Белом доме жалеют до сих пор: он не стал публично
указывать на русских или вводить санкции. Для этого всегда находилось
объяснение: страх перед эскалацией кибервойны и мысль о том, что США
нужно сотрудничество России в переговорах по Сирии.

``Мы заседали на всех этих многочисленных встречах'', - рассказывал один
высокопоставленный чиновник Госдепа, «на которых все соглашались, что
пора нанести ответный удар, и очень сильный. Но этого так и не
произошло».

А русские снова пошли на эскалацию, взламывая системы не ради шпионажа,
а чтобы широко обнародовать то, что им удалось найти. Они занимались
тем, что среди кибер-специалистов называется «доксинг».

Это была дерзкая смена тактики: переход от шпионажа к операциям влияния.
В феврале 2014 года они предали широкой гласности перехваченный
телефонный разговор Виктории Нуланд, помощницы госсекретаря по
взаимоотношениям с Россией, у которой были довольно напряженные
отношения с Путиным, и Джеффри Пьятта, посла США в Украине. Г-жа Нуланд
рассказывала о малоизвестной попытке США выступить посредником в одной
сделке в Украине, которая тогда переживала сильные потрясения.

Они были не единственными на ком русские практиковали эту стратегию типа
«перехватил-огласил». Фонд «Открытое общество» Джорджа Сороса тоже стал
одной из главных мишеней, и когда документы фонда были преданы
гласности, некоторые из них были искажены так, чтобы создать впечатление
будто Фонд финансирует членов оппозиции в России.

В прошлом году атаки стали более агрессивными. Россия взломала
компьютеры крупной французской телевизионной станции, было выведено из
строя важное оборудование. В Рождественские дни они атаковали
электросети Украины, оставив часть страны в темноте, выведя из строя
резервные генераторы. Сейчас становится понятным, что это был
предупредительный выстрел.

Эти атаки «еще не были военными операциями в полном смысле этого
слова'',- сказал г-н Салмейер, но они продемонстрировали растущую
дерзость.

\hypertarget{ux43cux438ux43bux44bux439-ux43cux438ux448ux43aux430-ux438-ux43aux440ux443ux442ux43eux439-ux43cux438ux448ux43aux430}{%
\subsection{\texorpdfstring{\textbf{МИЛЫЙ МИШКА И КРУТОЙ
МИШКА}}{МИЛЫЙ МИШКА И КРУТОЙ МИШКА}}\label{ux43cux438ux43bux44bux439-ux43cux438ux448ux43aux430-ux438-ux43aux440ux443ux442ux43eux439-ux43cux438ux448ux43aux430}}

В апреле, накануне ужина Ассоциации корреспондентов Белого дома, г-жа
Дэйси, президент DNC готовилась к праздничному вечеру, когда раздался
телефонный звонок.

После установки новой системы мониторинга, г-н Тэмин тщательно проверил
административный архив компьютерной системы DNC и обнаружил нечто
подозрительное: кто-то посторонний, обладающий допуском
административного уровня, проник в компьютеры DNC.

Точно никто не знал, насколько велика брешь, пробитая взломщиком, но
было совершенно ясно, что объем похищенной информации был гораздо
больше, чем содержимое одного канцелярского шкафа. Сохраняя секретность,
немедленно был созван специальный комитет, в который вошли г-жа Дэйси,
г-жа Вассерман Шульц, г-н Браун и Майкл Сассмэнн, бывший прокурор по
киберпреступности, в настоящее время работающий в Perkins Coie,
юридической фирме, ведущей политические дела DNC.

«Три самых главных вопроса», - написал г-н Сассмэнн своим клиентам в тот
вечер, когда был подтвержден взлом, - «1) К каким данным был получен
доступ?; 2) Как это было сделано?; 3) Как это остановить?»

Г-н Сассмэнн запретил своим клиентам пользоваться электронной почтой
DNC, так как у них был только один шанс выставить хакеров за дверь -- и
он будет упущен, если хакеры узнают, что они обнаружены.

«У вас только один шанс поднять мост», - сказал г-н Сассмэнн, - «Если
противники узнают, что вам известно об их присутствии, они примут меры,
чтобы спрятаться или стереть следы своего присутствия».

DNC немедленно обратилась в CrowdStrike, фирму, занимающуюся
кибербезопасностью, с просьбой просканировать компьютеры,
идентифицировать незваных гостей и построить новую компьютерную и
телефонную систему с нуля. В течение дня CrowdStrike подтвердила, что
вторжение было из России, сообщил г-н Сассмэнн.

Работа, которую проводят подобные компании, является компьютерной
версией старомодного расследования на месте преступления, с поиском
отпечатков пальцев, стреляных гильз и взятием образцов ДНК, только в
этом случае ведется поиск электронных следов, которые также могут быть
инкриминирующими. Как полицейские учатся определять характерные методы
взломщика со стажем, также следователи из CrowdStrike определили
«почерк» Cozy Bear и Fancy Bear.

Это клички, которые CrowdStrike дала двум российским группам хакеров,
следы работы которых фирма обнаружила в сети DNC. Cozy Bear (Милый
Мишка) -- группа, также известная как Dukes, или АРТ-29, сокращение от
Advanced Persistent Threat (постоянно совершенствуемая угроза). Группа
может ассоциироваться с ФСБ (но это не наверняка), однако широко
распространено мнение, что так или иначе эта операция проводится
государственными органами России. Впервые Cozy Bear появилась в 2014
году, сказал Дмитрий Альперович, соучредитель и главный технический
директор CrowdStrike.

В CrowdStrike пришли к заключению, что Cozy Bear впервые проникла в DNC
летом 2015 года путём рассылки фишинговых электронных писем большому
количеству вашингтонских государственных агентств, некоммерческих
организаций и правительственных подрядчиков. Всякий раз, когда кто-то
открывал фишинговое сообщение, русские входили в сеть, находили
интересующие их документы и накапливали их в разведывательных целях.

«Как только они попали в DNC, то нашли ценную информацию и решили
продолжить операцию», сообщил г-н Альперович, который родился в России и
переехал в США в отрочестве.

Fancy Bear (Крутой Мишка) появился только в марте 2016 -- сперва
проникнув в компьютеры Комитета Демократической партии по выборам в
Конгресс США, а потом переместившись в DNC. Так считает следствие. Fancy
Bear, известный также как АРТ29 и, похоже, управляемый ГРУ, военной
разведкой России, был создан раньше и отслеживается Западом в течение
почти десяти лет. Именно Fancy Bear получили доступ к электронной почте
г-на Подесты.

Атрибуция как умение идентифицировать кибер-хакера больше похожа на
искусство, чем на науку. Часто невозможно с абсолютной уверенностью
назвать имя атакующего. Но с течением времени, при накоплении справочных
материалов, описывающих методы и цели хакеров, становится возможным
определить рецидивистов. Например, Fancy Bear интересовался военными и
политическими целями на Украине и в Грузии, а также базами НАТО.

Это во многом исключает рядовых киберпреступников и большинство стран,
сообщил господин Альперович. «Кроме России, нет других вероятных
киберспециалистов, интересующихся таким количеством жертв», сказал он.
Ещё один ключ к разгадке: российские группировки хакеров, как правило,
действовали в рабочие часы по московскому времени.

К своему изумлению, сказал господин Альперович, эксперты CrowdStrike
нашли признаки того, что российские группировки хакеров не согласовывали
свои атаки между собой. Fancy Bear, по всей видимости, не знала, что
Cozy Bear в течение полугода рылась в файлах DNC, и захватила множество
тех же самых документов.

В течение шести недель после начала работы CrowdStrike, в обстановке
абсолютной секретности, заменила компьютерную систему DNC. В выходные
дни электронные адреса и телефоны отключались; сотрудникам сообщили, что
ведется модернизация системы. Содержимое всех ноутбуков «прошерстили»,
жёсткие диски были вычищены, а незаражённая информация была перенесена
на новые диски.

Хотя чиновники DNC знали, что компьютеры Комитета Демократической партии
по выборам в Конгресс США также были заражены, они не оповестили своих
коллег, работающих в том же здании, так как боялись утечки информации.

Всё это происходило на фоне продолжающегося соперничества за выдвижение
Демократами г-жи Клинтон и г-на Сандерса, и сильно отвлекало г-жу
Вассерман Шульц и исполнительного директора DNC от избирательной
кампании.

«Это вам не помеха на дороге, что случается довольно часто», - сказала
она в интервью. -- «Два разных российских разведывательных агентства
взломали нашу сеть и украли нашу интеллектуальную собственность. И мы не
знаем, что им удалось захватить. Однако нам известно, что им удалось
получить широкий доступ к нашей сети. Мы в полной неизвестности. Это
просто выбивает из колеи».

Руководители DNC со своим юристом впервые провели официальную встречу с
высокопоставленными чиновниками из ФБР в середине июня, спустя девять
месяцев после того, как спецслужба впервые обратилась в службу
техподдержки. По словам участников, в первую очередь были сделаны
следующие запросы: чтобы Федеральное правительство провело срочную
атрибуцию злоумышленников, формально обвинив их в связях с
правительством России, чтобы не оставалось сомнений в том, что это была
не обычная хакерская атака, а шпионаж, осуществляемый иностранным
государством.

«У вас полным ходом идут президентские выборы, и вы знаете, что русские
взломали DNC», - возмутился г-н Сассмэнн, ссылаясь на сообщение ФБР. --
«Нам необходимо сообщить об этом американскому народу. И немедленно».

\hypertarget{ux440ux43eux43bux44c-ux441ux43cux438}{%
\subsection{\texorpdfstring{\textbf{Роль
СМИ}}{Роль СМИ}}\label{ux440ux43eux43bux44c-ux441ux43cux438}}

В середине июня по совету г-на Сассмэнна руководство DNC решилось на
отчаянный шаг. Боясь утечки информации о факте взлома, они решили
сделать это достоянием общественности посредством публикации в
``Washington Post'' о том, что Комитет был атакован. Таким образом, они
могли опередить журналистов, завоевать немного сочувствия избирателей
как жертвы российских хакеров и сконцентрироваться на избирательной
кампании.

Но буквально на следующий день их ждал ещё один неприятный сюрприз.
Некто, именующий себя Guccifer 2.0, появился в сети, утверждая, что он
является хакером, взломавшим DNC. Он опубликовал секретный документ DNC
со сведениями из досье на г-на Трампа и еще с десяток других документов,
подтверждающих, что это не подделки, а подлинные документы.

«И это лишь малая часть всех документов, которые я скачал из сетей
Демократов», написал он. А затем что-то более угрожающее: «Большую часть
документов, тысячи файлов и электронных писем, я отдал WikiLeaks. Они
скоро их опубликуют».

Плохо уже то, что российские хакеры месяцами шпионили в сети комитета.
Теперь же публикация документов превратила обычный шпионаж в что-то
более угрожающее: политический саботаж, непредсказуемую, не поддающуюся
контролю угрозу Демократам.

Guccifer 2.0 позаимствовал кличку у другого хакера, румына, называвшего
себя Guccifer, который был отправлен в тюрьму за взлом личных
компьютеров бывшего президента Джорджа Буша-младшего, бывшего
Госсекретаря Колина Пауэлла и других выдающихся людей. Этот новый хакер,
похоже, хотел показать, что киберэксперты DNC в CrowdStrike были
неправы, обвинив Россию. Guccifer 2.0 назвал себя хакером-одиночкой и
высмеял CrowdStrike за то, что они называли атакующих «искушёнными».

Но следователи, работающие в онлайн-режиме, быстро вычислили его. Шутки
ради Лоренцо Франчески-Биккерай, журналист, пишущий для Интернет-издания
Motherboard, попытался связаться с Guccifer 2.0, написав ему сообщение в
Twitter.

«Удивительно, но он почти сразу ответил», поделился Франчески-Биккерай.
Однако казалось, что на другом конце над ним глумятся. «Я спросил его,
зачем он это делает, и он сказал, что хотел разоблачить иллюминатов,
мировую «закулису». Он назвал себя любителем Гуччи. Он утверждал, что он
румын».

Это натолкнуло Франчески-Биккерая на мысль. При помощи Google Translate
он отправил предполагаемому хакеру несколько вопросов на румынском языке
и получил ответы. Когда тот был вне сети, г-н Франчески-Биккерай
проверил ответы у пары носителей этого языка, которые сообщили ему, что
Guccifer 2.0 также использовал Google Translate -- и стало понятно, что
он не румын, как утверждал.

Кибер-исследователи нашли другие улики, указывающие на Россию. Документы
Microsoft Word, опубликованные Guccifer 2.0, были отредактированы
кем-то, кто называл себя по-русски Феликсом Эдмундовичем -- псевдонимом,
очевидно выбранным в память об основателе советской тайной полиции
Феликсе Эдмундовиче Дзержинском. Неработающие ссылки в тексте были
отмечены предупреждениями на русском языке, сделанными с помощью
русскоязычной версией Word.

Когда Франчески-Биккераю удалось втянуть Guccifer 2.0 в активную
переписку, то он обнаружил, что тон и манеры его собеседника изменились.
«Сперва он был беззаботным и болтливым. Спустя несколько недель он стал
отрывисто-грубым и более расчётливым», сказал он. «Это было похоже на
группу людей и очень неловкую попытку скрыть это».

Эксперты по кибербезопасности пришли к такому же заключению по поводу
DCLeaks.com, сайта, который возник в июне, позиционирующийся как работа
«хактивистов», но публикующий в основном украденные документы. И этот
сайт выглядел как грубое прикрытие для тех же самых русских, которые
украли документы. Примечательно, что сайт был зарегистрирован в апреле,
наводя на мысль, что российская группировка хакеров подготовилась
заранее, чтобы выложить перед честным народом украденное ими.

Кроме того, то, что Guccifer 2.0 выкладывал на своем сайте, он напрямую
посылал материалы по просьбе разных блогеров и изданий. Непрекращающийся
поток документов от Guccifer 2.0 постоянно подрывал обмен сообщениями
Демократов. 6 июля, за 12 дней до Национального съезда Республиканской
партии Guccifer выложил «план боя», подготовленный DNC, и бюджет на
организацию его противодействия. Для Республиканцев это было стало
золотыми копями инсайдерской информации.

Затем Wikileaks, куда более известное издание, стал публиковать
материал, полученный в ходе взлома -- как Guccifer 2.0 и обещал. 22
июля, за три дня до начала Национального съезда Республиканской партии
США в Филадельфии, Wikileaks вбросил 44053 электронных письма,
украденных в DNC с 17761 приложением. Некоторые из этих сообщений давали
понять, что руководство DNC питало большую симпатию к г-же Клинтон по
сравнению с её соперником г-ном Сандерсом.

Это не шокировало людей; в конце концов, г-н Сандерс был независимым
социалистом, а не демократом в течение долгого периода его работы в
Конгрессе, в то время как г-жа Клинтон была одной из звёзд партии на
протяжении десятков лет. Но электронные письма, содержавшие иногда
непристойности и оскорбления, разозлили делегатов Сандерса, прибывших в
Филадельфию. Г-жа Вассерман-Шульц была вынуждена уйти в отставку
накануне открытия съезда, где она планировала быть председателем.

Г-н Трамп, к этому времени кандидат от республиканцев, был в восторге от
продолжительного шока его оппонентов и начал использовать Twitter и
выступать с речами, заостряя внимания на публикациях в Wikileaks. 25
июля он опубликовал легкомысленное сообщение в Twitter: «Новая хохма в
городе», писал он, «Россия слила губительные электронные письма DNC,
которые не следовало писать, потому что Путин мне симпатизирует».

Но объявленная Wikileaks война ещё не завершилась. 7 октября, за месяц
до выборов, сайт начал серию публикаций тысяч личных электронных писем,
адресованных к и от имени г-на Подеста, руководителя избирательного
штаба г-жи Клинтон.

В тот же день США официально обвинили правительство России в том, что
оно стоит за хакерскими атаками в совместном заявлении директора
Национальной разведки и Министерства внутренней безопасности, а г-н
Трамп получил сильнейший удар в результате публикации записи, на которой
он бахвалится сексуальными домогательствами к женщинам.

Электронные письма Подеста и рядом не стояли с сенсационным видео
Трампа. Но, публикуемые сайтом Wikileaks день за днём в течение
последнего месяца избирательной кампании, они предоставляли материалы
для бесчисленных новостных сообщений. Они обнародовали речи г-жи
Клинтон, обращенные к крупным банкам, которые она отказалась
публиковать. Они разоблачали трения, существующие внутри избирательной
кампании, включая несогласие штатных сотрудников по поводу денежных
пожертвований в Фонд Клинтон, которые, по их мнению, вредят имиджу
кандидата, не говоря уже о жалобе г-жи Тэнден, что инстинкты г-жи
Клинтон «далеки от оптимальных».

«Я сгорала от стыда», поделилась г-жа Тэнден в интервью. Она вспоминает,
что её электронные письма были обнародованы накануне президентских
дебатов. «Я схватилась за голову и сказала, что не могу поверить, что
это происходит со мной». Хотя она регулярно появлялась с г-жой Клинтон в
телеэфирах, потом она прекратила это делать, так как все вопросы к ней
сводились к содержанию этих писем.

Г-жа Тэнден, как и другие Демократы, чьи сообщения стали достоянием
общественности, сказала, что ей очевидно, что Wikileaks изо всех сил
пытался навредить штабу Клинтон. «Если вы заботитесь о прозрачности, то
сразу отправляете все электронные письма», сказала она. «Но они хотели
причинить ей боль. То есть они публиковали от 1800 до 3000 писем в
день».

Штаб Трампа заранее знал о планах Wikileaks. За несколько дней до
публикации писем г-на Подеста, Роджер Стоун, работник аппарата в штабе
Трампа, опубликовал восторженное сообщение в Twitter о том, что грядет.

``Среда@ С Хилари Клинтон покончено. \# Wikileaks''

Однако в интервью г-н Стоун поведал, что он не сыграл в утечках никакой
роли; он всего лишь слышал от американца, имевшего связи с Wikileaks,
что эти убийственные электронные письма скоро будут достоянием
гласности.

Основатель и редактор Wikileaks Джулиан Ассандж сопротивлялся выводам,
что его сайт стал транзитом для российских хакеров, работающих на
правительство г-на Путина или что он намеренно пытался компрометировать
кандидатуру г-жи Клинтон. Но свидетельства по обеим этим позициям
оказались весьма убедительными

Во время обмена электронными письмами г-н Ассандж отказался что-либо
говорить по поводу источника взломанных материалов для Wikileaks. Он
отрицал резко негативное отношение к г-же Клинтон в своих публичных
выступлениях («Неправда. Но что это? Детский сад») или что сайт
приурочивал публикации для максимально негативного воздействия на её
избирательную кампанию. «Wikileaks принимает решения на основании
событийности, включая недавние потрясающие сенсации», написал он.

Г-н Ассанж оспорил вывод о заявлении разведывательных агентств от 7
октября, которое гласит, что утечки «направлены на вмешательство в
избирательный процесс в США».

«Это неверно», писал он. «Как разглашающая сторона, мы знаем, что это не
было сделано намеренно. Издатели, публикующие достойную освещения
информацию во время выборов, и составляют часть свободных выборов».

Однако когда его спросили о том, верит ли он, что утечки были одной из
причин победы г-на Трампа, г-н Ассандж был рад приписать это себе в
заслугу. «Американцы проявляли огромный интерес к нашими публикациями»,
писал он. «По статистике Facebook, Wikileaks был самой упоминаемой
политической темой в октябре».

Хотя г-н Ассандж так не говорил, лучшей защитой для Wikileaks может
стать поведение основных американских СМИ. Каждое крупное издание,
включая The Times, публиковало многочисленные истории, ссылаясь на
электронные письма DNC и Подесты, выложенные на Wikileaks, фактически
становясь инструментом российских спецслужб.

Г-н Путин, изучавший восточные единоборства, использовал два основных
столпа американской демократии -- политические кампании и независимые
СМИ -- в своих интересах. Интерес СМИ к полученным в результате взлома
материалам и их внимание, обращенное на их скандальный характер, а не на
факт их происхождения из российских источников, обеспокоило многих, чья
личная переписка была опубликована в сети.

«Что меня действительно удивило?» - сказала г-жа Тэнден, - «Я не могла
поверить в то, что всё это освещали журналисты».

\hypertarget{ux43fux43eux434ux433ux43eux442ux43eux432ux43aux430-ux440ux435ux430ux43aux446ux438ux438-ux430ux434ux43cux438ux43dux438ux441ux442ux440ux430ux446ux438ux438}{%
\subsection{\texorpdfstring{\textbf{ПОДГОТОВКА РЕАКЦИИ
АДМИНИСТРАЦИИ}}{ПОДГОТОВКА РЕАКЦИИ АДМИНИСТРАЦИИ}}\label{ux43fux43eux434ux433ux43eux442ux43eux432ux43aux430-ux440ux435ux430ux43aux446ux438ux438-ux430ux434ux43cux438ux43dux438ux441ux442ux440ux430ux446ux438ux438}}

В самом Белом доме советники г-на Обамы, обсуждая свой ответ на эти
события, незаметно перешли на тему Северной Кореи.

В конце 2014 года хакеры, работавшие на Ким Чен Ына, молодого и
непредсказуемого лидера Северной Кореи, осуществили хорошо
спланированную атаку на компанию Sony Pictures Entertainment, чтобы
предотвратить приуроченный к Рождеству выход комедии о заговоре ЦРУ с
целью убить г-на Кима.

В этом случае тоже были опубликованы компрометирующие электронные
сообщения. Но самый большой ущерб был нанесен системам самой компании
Sony: более 70 процентов их компьютеров были практически уничтожены
после их заражения чрезвычайно мощной формой вирусной программы. В
течение нескольких недель разведслужбы проследили источник заражения.
Это были Северная Корея и ее руководство. Г-н Обама публично обвинил в
этом Северную Корею и наложил санкции, оказавшиеся не слишком
эффективными. Тогда даже Китай оказал содействие, ненадолго вырубив
Интернет-связь с Кореей.

Когда в июле начались первые совещания Ситуационного центра, «стало
ясно, что этот случай с российским вмешательством окажется гораздо более
трудным», сказал один из участников этих встреч. Совершенно очевидно,
что русские обладали гораздо более глубоким пониманием американской
политики и мастерски владели искусством использования компрометирующих
материалов, «компромата», как они говорят.

Однако официальный доклад с результатами «аттрибуции» все еще не был
направлен президенту.

«Это тянулось бесконечно», - сказал один из высоких чиновников
Администрации, жалуясь на систему за слишком медленное продвижение
оценок, сделанных разведорганами.

В августе группа под названием «Shadow Brokers» выпустила комплект
программного обеспечения, похожий на то, чем пользуется Агентство
Национальной безопасности (АНБ) для взлома иностранных компьютерных
систем для установки «закладок», вредоносных программ, используемых для
сбора разведданных или кибератак. Код для получения этой информации был
получен от Tailored Access Operations, подразделения АНБ, засекреченной
группы, отрабатывающей приемы сбора информации и ведения кибервойны.

Есть мнение, пока еще не подтвержденное, что код был выложен в открытый
доступ как предупреждение: попробуйте поквитаться с нами за DNC, и мы
выложим гораздо больше секретов, полученных от взлома систем Госдепа,
Белого дома и Пентагона. Один высокопоставленный чиновник сравнил эту
ситуацию со сценой из «Крестного отца», когда отрезанная голова любимой
лошади кинопродюсера оказывается в его постели в качестве
предупреждения.

АНБ промолчало. Но в конце августа адмирал Роджерс, его директор, начал
настаивать на более жестком ответе русским. В качестве главы
Киберкомандования Пентагона он предложил нанести серию мощных
кибернетических контрударов.

Сами контрмеры не обсуждались в деталях, но, по имеющимся сведениям, эти
контрудары должны были побить Путина его же оружием, разоблачая его
финансовые связи с российскими олигархами, а также пробивая бреши в
российском сегменте Интернета, чтобы дать диссидентам возможность
изложить свои взгляды и позицию. Чиновники Пентагона посчитали эти меры
слишком грубыми и взялись за разработку своих предложений. В результате
ни одно из них не было официально представлено президенту для
рассмотрения.

В ходе целой серии совещаний «на уровне заместителей» под
председательством Эврила Хэйанса, заместителя Советника президента по
национальной безопасности и бывшего заместителя Директора ЦРУ, многие
участники предупреждали, что чрезмерная, слишком резкая реакция
Администрации может сыграть на руку Путину.

«Если мы объявим положение «Defcon-4» (полная боевая готовность для
начала войны)», - сказал один из участников этих совещаний, прибегая к
терминологии времен Холодной войны, - «мы тем самым распишемся в
отсутствии доверия к целостности и непоколебимости нашей избирательной
системы».

Даже такие простые и очевидные меры, как использование исполнительных
полномочий президента, подкрепленных после инцидента с Sony, для
применения экономических санкций и ограничения перемещения для
участников кибератак казалось слишком рискованным.

«Никто не рвался требовать расплаты до Дня выборов», - сказал другой
участник этих секретных совещаний. «Любые ответные меры рассматривались
сквозь призму того, что может произойти в День выборов».

Вместо этого, когда советники президента Обамы по национальной
безопасности собрались после летних каникул, все внимание было
сконцентрировано на срочных мерах по обеспечению безопасности машин для
голосования и списков учета избирателей. Наиболее часто обсуждался
следующий сценарий, который в итоге не сбылся, - Клинтон побеждает с
незначительным перевесом, затем следует заявление Трампа о подтасовке
выборов и новые утечки, направленные на подрыв легитимности г-жи
Клинтон.

Донна Бразил, временно исполняющая обязанности председателя DNC,
испытывала все большее разочарование по мере приближения выборов, однако
по-прежнему ни Белый дом, ни руководство Республиканской партии не
выступили с резким осуждением кибератаки как акта шпионажа со стороны
иностранного государства.

Г-жа Бразил даже сделала попытку и лично обратилась к Рейнсу Прайбасу,
председателю Национального комитета Республиканской партии. Дважды в
частных беседах и в письме к нему она призывала его присоединиться к ней
в осуждении кибератак. Он отказался.

«Мы только и слышали, правительство даст достойный ответ, правительство
даст ответ», сказала она. «Когда-нибудь, если иностранное государство
вмешается в наши выборы, мы дадим ответ как единая страна, а не как
политическая партия».

Но г-н Обама все же принял решение лично сделать предупреждение г-ну
Путину на саммите Двадцатки в китайском Ханчжоу. То был последний
случай, когда они оказались вместе до истечения президентских полномочий
г-на Обамы. Когда они, отойдя в сторонку, встретились в напряженном
разговоре, г-н Обама недвусмысленно предупредил г-на Путина о жесткой
реакции США в случае продолжения попыток оказывать влияние на выборный
процесс или манипулировать голосованием, - сообщают чиновники Белого
дома, которые, правда, не участвовали в этой беседе один на один.

Позже в этот же день г-н Обама сделал редкое для него замечание
относительно американской наступательной кибермощи, темы, которой он
практически никогда не касался. «Откровенно говоря, мы обладаем бОльшими
возможностями, как оборонительными, так и наступательными», - сказал он,
обращаясь к журналистам.

Однако когда пришло время сделать официальное заявление о роли России в
событиях начала октября, это было сделано в письменном заявлении от
имени директора национальной разведки и министра внутренней
безопасности. Это, конечно, было гораздо менее эффектно, чем появление
президента в зале для прессы Белого дома за два года до этого, когда он
прямо обвинил Северную Корею в атаке на Sony.

Упоминающиеся в заявлении взломы «политических организаций», - говорят
чиновники, - подразумевают также и похищение данных у Республиканцев.
Два крупных чиновника утверждают, что данные криминалистической
экспертизы были дополнены «человеческими и техническими» источниками на
территории России, что скорее всего означает, что американские
«закладки» или подключения к российским компьютерным и телефонным сетям
помогли подтвердить участие России в кибератаках.

Но все это может продолжать оставаться тайной в течение десятилетий,
пока не будут сняты грифы секретности.

Неделей позже Вице-президенту Джону Байдену было поручено передать
публично предупреждение г-ну Путину: Соединенные Штаты примут ответные
меры «в нужное время по своему выбору и в обстоятельствах, гарантирующих
наибольший эффект».

Позже, после того как г-н Байден сказал, что он не думает, что Россия
может «оказать фундаментальное влияние на выборы», его спросили, узнает
ли американская общественность, что послание г-ну Путину отправлено.

«Надеюсь, что нет», - ответил г-н Байден.

Некоторые из его бывших коллег считают, что это был неправильный ответ.
Американский контрудар, сказал Майкл Морелл, бывший замдиректора ЦРУ в
администрации президента Обамы, «должен быть открытым и очевидным. Он
должен быть виден».

Скрытый ответ значительно снижает эффект устрашения, добавил он. «Если
вы не можете этого увидеть, это не сработает как сдерживающий фактор для
китайцев, северокорейцев или иранцев, или кого-то еще».

Администрация президента Обамы говорит, что у них есть еще тридцать
дней, чтобы сделать именно это.

\hypertarget{ux441ux43bux435ux434ux443ux44eux449ux430ux44f-ux43cux438ux448ux435ux43dux44c}{%
\subsection{\texorpdfstring{\textbf{СЛЕДУЮЩАЯ
МИШЕНЬ}}{СЛЕДУЮЩАЯ МИШЕНЬ}}\label{ux441ux43bux435ux434ux443ux44eux449ux430ux44f-ux43cux438ux448ux435ux43dux44c}}

По мере приближения нового года кажется, появляются возможности для
дальнейшего расследования российской хакерской атаки: Разведывательная
сводка, которую г-н Обама распорядился закончить к 20 января, ко дню,
когда он покинет свой пост; и еще одно или несколько расследований в
Конгрессе. Среди прочих вопросов, они будут биться над вопросом о
мотивах, которыми руководствовался г-н Путин.

Стремился ли он испортить имидж американской демократии или предупредить
антироссийские выступления? Может быть, он стремился ослабить следующего
американского президента, поскольку у него не было причин не верить в
предсказание легкой победы г-жи Клинтон? А может быть, как предположило
ЦРУ, это было намеренной попыткой избрать Трампа?

На самом деле российский план под девизом «взламывай и публикуй
найденное» достиг всех трех вышеназванных целей.

Принимая во внимание успех российских хакеров, представляется ясным, что
они не остановятся на этом. Две недели назад Бруно Каал, шеф германской
разведки, предупредил о возможности вмешательства России в будущем году
в выборы в Германии. «Злоумышленники заинтересованы в лишении
легитимности самого демократического процесса как такового», сказал г-н
Каал. ``Теперь, - добавил он, - в фокусе этих попыток нарушения общего
порядка оказывается Европа, и в особенности Германия''.

Но Россия ни в коей мере не забывает объекты в Америке. На следующий
день после президентских выборов компания Volexity, занимающаяся
кибербезопасностью, обнаружила пять новых волн фишинговых сообщений,
явно отправленных группой Cozy Bear в различные аналитические центры и
неправительственные организации в Соединенных Штатах.

К одному из сообщений, якобы отправленному из Гарвардского университета,
была приложена поддельная статья под заголовком: «В чем причина
дискредитации выборов в Америке».

Advertisement

\protect\hyperlink{after-bottom}{Continue reading the main story}

\hypertarget{site-index}{%
\subsection{Site Index}\label{site-index}}

\hypertarget{site-information-navigation}{%
\subsection{Site Information
Navigation}\label{site-information-navigation}}

\begin{itemize}
\tightlist
\item
  \href{https://help.nytimes.com/hc/en-us/articles/115014792127-Copyright-notice}{©~2020~The
  New York Times Company}
\end{itemize}

\begin{itemize}
\tightlist
\item
  \href{https://www.nytco.com/}{NYTCo}
\item
  \href{https://help.nytimes.com/hc/en-us/articles/115015385887-Contact-Us}{Contact
  Us}
\item
  \href{https://www.nytco.com/careers/}{Work with us}
\item
  \href{https://nytmediakit.com/}{Advertise}
\item
  \href{http://www.tbrandstudio.com/}{T Brand Studio}
\item
  \href{https://www.nytimes.com/privacy/cookie-policy\#how-do-i-manage-trackers}{Your
  Ad Choices}
\item
  \href{https://www.nytimes.com/privacy}{Privacy}
\item
  \href{https://help.nytimes.com/hc/en-us/articles/115014893428-Terms-of-service}{Terms
  of Service}
\item
  \href{https://help.nytimes.com/hc/en-us/articles/115014893968-Terms-of-sale}{Terms
  of Sale}
\item
  \href{https://spiderbites.nytimes.com}{Site Map}
\item
  \href{https://help.nytimes.com/hc/en-us}{Help}
\item
  \href{https://www.nytimes.com/subscription?campaignId=37WXW}{Subscriptions}
\end{itemize}
