Sections

SEARCH

\protect\hyperlink{site-content}{Skip to
content}\protect\hyperlink{site-index}{Skip to site index}

\href{https://www.nytimes.com/section/politics}{Politics}

\href{https://myaccount.nytimes.com/auth/login?response_type=cookie\&client_id=vi}{}

\href{https://www.nytimes.com/section/todayspaper}{Today's Paper}

\href{/section/politics}{Politics}\textbar{}C.I.A. Judgment on Russia
Built on Swell of Evidence

\url{https://nyti.ms/2hCsrBj}

\begin{itemize}
\item
\item
\item
\item
\item
\item
\end{itemize}

Advertisement

\protect\hyperlink{after-top}{Continue reading the main story}

Supported by

\protect\hyperlink{after-sponsor}{Continue reading the main story}

\hypertarget{cia-judgment-on-russia-built-on-swell-of-evidence}{%
\section{C.I.A. Judgment on Russia Built on Swell of
Evidence}\label{cia-judgment-on-russia-built-on-swell-of-evidence}}

\includegraphics{https://static01.nyt.com/images/2016/12/12/us/12INTEL/12INTEL-articleInline.jpg?quality=75\&auto=webp\&disable=upscale}

By \href{http://www.nytimes.com/by/mark-mazzetti}{Mark Mazzetti} and
\href{http://www.nytimes.com/by/eric-lichtblau}{Eric Lichtblau}

\begin{itemize}
\item
  Dec. 11, 2016
\item
  \begin{itemize}
  \item
  \item
  \item
  \item
  \item
  \item
  \end{itemize}
\end{itemize}

WASHINGTON --- American spy and law enforcement agencies were united in
the belief, in the weeks before the presidential election, that the
Russian government had deployed computer hackers to sow chaos during the
campaign. But they had conflicting views about the specific goals of the
subterfuge.

Last week, Central Intelligence Agency officials presented lawmakers
with
\href{https://www.nytimes.com/2016/12/09/us/obama-russia-election-hack.html}{a
stunning new judgment} that upended the debate: Russia, they said, had
intervened with the primary aim of helping make Donald J. Trump
president.

The C.I.A.'s conclusion does not appear to be the product of specific
new intelligence obtained since the election, several American
officials, including some who had read the agency's briefing, said on
Sunday. Rather, it was an analysis of what many believe is overwhelming
circumstantial evidence --- evidence that others feel does not support
firm judgments --- that the Russians put a thumb on the scale for Mr.
Trump, and got their desired outcome.

It is unclear why the C.I.A. did not produce this formal assessment
before the election, although several officials said that parts of it
had been made available to President Obama in the presidential daily
briefing in the weeks before the vote. But the conclusion that Moscow
ran an operation to help install the next president is one of the most
consequential analyses by American spy agencies in years.

Mr. Trump's response
\href{https://www.nytimes.com/2016/12/11/us/politics/trump-russia-democrats.html}{has
been to dismiss} the reports by citing another famous intelligence
assessment --- the botched 2002 conclusion that the Iraqi leader, Saddam
Hussein, had weapons of mass destruction --- and portraying American
spies as bumbling and biased.

``I think it's ridiculous. I think it's just another excuse. I don't
believe it,'' Mr. Trump said on Sunday in an interview on Fox News. Some
top Republican congressmen have said the same, although with less
bombastic language, arguing that there is no clear proof that the
Russians tried to rig the election for Mr. Trump.

Yet there is a loud chorus of bipartisan voices, including Senator John
McCain, Republican of Arizona, going public to accuse the Russians of
election interference.

Representative Adam B. Schiff of California, the top Democrat on the
House Intelligence Committee, said the public evidence alone made it
clear that Moscow had intervened to help the ``most ostentatiously
pro-Russian candidate in history.''

``If the Russians were going to interfere, why on earth would they do it
to the detriment of the candidate that was pro-Russian?'' Mr. Schiff
asked.

The dispute cuts to core realities of intelligence analysis. Judgments
are often made in a fog of uncertainty, are sometimes based on putting
together shards of a mosaic that do not reveal a full picture, and can
always be affected by human biases.

\includegraphics{https://static01.nyt.com/images/2017/07/08/us/27TRUMP-PUTIN-COMBO/27TRUMP-PUTIN-COMBO-videoSixteenByNine3000.jpg}

``This is why I hate the term `we speak truth to power,''' said Mark M.
Lowenthal, a former senior C.I.A. analyst. ``We don't have truth. We
have really good ideas.''

Mr. Lowenthal said that determining the motives of foreign leaders ---
in this case, what drove President Vladimir V. Putin of Russia to order
the hacking --- was one of the most important missions for C.I.A.
analysts. In 2002, one of the critical failures of American spy agencies
was their inability to understand Saddam Hussein's goals and motives.

At the same time, Mr. Lowenthal said, intelligence agencies have always
been loath to be seen as taking sides in disputes about American
politics.

``This is the one place you don't want to be as an intelligence officer:
the meat in someone's partisan sandwich,'' he said.

Both intelligence and law enforcement officials agree that there is a
mountain of circumstantial evidence suggesting that the Russian hacking
was primarily aimed at helping Mr. Trump and damaging his opponent,
Hillary Clinton.

In July, the infiltration of the Democratic National Committee's
computer servers produced
\href{http://www.nytimes.com/2016/07/23/us/politics/dnc-emails-sanders-clinton.html}{embarrassing
emails} and other internal party documents, the publication of which
caused a backlash that
\href{http://www.nytimes.com/2016/07/25/us/politics/debbie-wasserman-schultz-dnc-wikileaks-emails.html}{led
to the resignation} of the committee's chairwoman, Debbie Wasserman
Schultz, and her top staff. Just weeks before the election, hacked
emails from the account of John D. Podesta, Mrs. Clinton's top campaign
manager,
\href{https://www.nytimes.com/2016/10/21/us/private-security-group-says-russia-was-behind-john-podestas-email-hack.html}{were
made public} and produced numerous stories about the internal dynamics
of the campaign. That hack also produced
\href{http://www.nytimes.com/2016/10/08/us/politics/hillary-clinton-speeches-wikileaks.html}{the
text of speeches} Mrs. Clinton had given to Wall Street banks.

American intelligence officials believe that Russia also penetrated
databases housing Republican National Committee data, but chose to
release documents only on the Democrats. The committee has denied that
it was hacked.

Beyond the specific targets of the hacks, American officials cite broad
evidence that Mr. Putin and the Russian government favored Mr. Trump
over Mrs. Clinton.

After demonstrators marched through Moscow in 2011 chanting ``Putin is a
thief'' and ``Russia without Putin,'' Mr. Putin
\href{http://www.nytimes.com/2011/12/09/world/europe/putin-accuses-clinton-of-instigating-russian-protests.html}{publicly
accused} Mrs. Clinton, then the secretary of state, of instigating the
protests. ``She set the tone for some actors in our country and gave
them a signal,'' he said.

More generally, the Russian government has blamed Mrs. Clinton, along
with the C.I.A. and other American officials, for encouraging
anti-Russian revolts during the 2003 Rose Revolution in Georgia and the
2004 Orange Revolution in Ukraine. What Americans saw as legitimate
democracy promotion, Mr. Putin saw as an unwarranted intrusion into
Russia's geographic sphere of interest, as the United States once saw
Soviet meddling in Cuba.

By contrast, Mr. Trump and Mr. Putin have had a very public mutual
admiration society. In December 2015, the Russian president
\href{http://abcnews.go.com/International/russian-president-vladimir-putin-praises-donald-trump-talented/story?id=35816611}{called
Mr. Trump} ``very colorful'' --- using a Russian word that Mr. Trump and
some news outlets mistranslated as ``brilliant'' --- as well as
``talented'' and ``absolutely the leader in the presidential race.'' Mr.
Trump
\href{http://www.nbcnews.com/meet-the-press/meet-press-december-20-2015-n483421}{called
Mr. Putin} ``a strong leader'' and further pleased him by questioning
\href{http://www.nytimes.com/2016/07/21/us/politics/donald-trump-issues.html}{whether
the United States should defend} NATO members that did not spend enough
on their militaries.

Russian television, which is tightly controlled by the government, has
generally
\href{http://www.cnbc.com/2016/11/05/russian-media-backs-trump-questions-us-democracy.html}{portrayed}
Mr. Trump as a strong, friendly potential partner while often airing
scathing assessments of Mrs. Clinton.

And yet, there is skepticism within the American government,
particularly at the F.B.I., that this evidence adds up to proof that the
Russians had the specific objective of getting Mr. Trump elected.

A senior American law enforcement official said the F.B.I. believed that
the Russians probably had a combination of goals, including damaging
Mrs. Clinton and undermining American democratic institutions. Whether
one of those goals was to install Mr. Trump remains unclear to the
F.B.I., he said.

The official played down any disagreement between the F.B.I. and the
C.I.A., and suggested that the C.I.A.'s conclusions were probably more
nuanced than they were being framed in the news media.

The agencies' differences in judgment may also reflect different methods
of investigating the Russian interference. The F.B.I., which has both a
law enforcement and an intelligence role, is held to higher standards of
proof in examining people involved in the hacking because it has an eye
toward eventual criminal prosecutions. The C.I.A. has a broader mandate
to develop intelligence assessments.

Law enforcement officials said that if F.B.I. agents had the evidence to
charge Russians with specific crimes, they would do so. The F.B.I. and
federal prosecutors have already gone aggressively after Russian
hackers, including two men detained in Thailand and the Czech Republic
whom the United States is trying to extradite.

Russia has tried to block those efforts and has accused the United
States of harassing its citizens.

The F.B.I. began investigating Russia's apparent attempts to meddle in
the election over the summer. Agents examined numerous possible
connections between Russians and members of Mr. Trump's inner circle,
including former Trump aides like Paul Manafort and Carter Page, as well
as a mysterious and unexplained trail of computer activity between the
Trump Organization and an email account at a large Russian bank, Alfa
Bank.

At the height of its investigation before the election, the F.B.I. saw
some indications that the Russians might be explicitly seeking to get
Mr. Trump elected, officials said, and investigators collected online
evidence and conducted interviews overseas and inside the United States
to test that theory.

The F.B.I. was concerned enough about Russia's influence and possible
connections to the Trump campaign that it briefed congressional leaders
--- including Senator Harry Reid, the Nevada Democrat and Senate
minority leader --- on some of the evidence this summer and fall. Mr.
Reid, in particular, pressed for the F.B.I. to find out more and charged
that the agency was sitting on important information that could
implicate Russia.

But the agency's suspicions about a direct effort by Russia to help Mr.
Trump, or about possible connections between the two camps, appear to
have waned as the investigation continued into September and October.
The reasons are not entirely clear, and F.B.I. officials declined to
comment.

Now that a partisan squall has erupted over exactly what role Russia
played in influencing the election, there is growing momentum among both
Republicans and Democrats on Capitol Hill to have a congressional
investigation.

``I'm not trying to relitigate the election,'' said Senator Angus King,
independent of Maine, who is one of the lawmakers calling for such an
investigation. ``I'm just trying to prevent this from happening again.''

Advertisement

\protect\hyperlink{after-bottom}{Continue reading the main story}

\hypertarget{site-index}{%
\subsection{Site Index}\label{site-index}}

\hypertarget{site-information-navigation}{%
\subsection{Site Information
Navigation}\label{site-information-navigation}}

\begin{itemize}
\tightlist
\item
  \href{https://help.nytimes.com/hc/en-us/articles/115014792127-Copyright-notice}{©~2020~The
  New York Times Company}
\end{itemize}

\begin{itemize}
\tightlist
\item
  \href{https://www.nytco.com/}{NYTCo}
\item
  \href{https://help.nytimes.com/hc/en-us/articles/115015385887-Contact-Us}{Contact
  Us}
\item
  \href{https://www.nytco.com/careers/}{Work with us}
\item
  \href{https://nytmediakit.com/}{Advertise}
\item
  \href{http://www.tbrandstudio.com/}{T Brand Studio}
\item
  \href{https://www.nytimes.com/privacy/cookie-policy\#how-do-i-manage-trackers}{Your
  Ad Choices}
\item
  \href{https://www.nytimes.com/privacy}{Privacy}
\item
  \href{https://help.nytimes.com/hc/en-us/articles/115014893428-Terms-of-service}{Terms
  of Service}
\item
  \href{https://help.nytimes.com/hc/en-us/articles/115014893968-Terms-of-sale}{Terms
  of Sale}
\item
  \href{https://spiderbites.nytimes.com}{Site Map}
\item
  \href{https://help.nytimes.com/hc/en-us}{Help}
\item
  \href{https://www.nytimes.com/subscription?campaignId=37WXW}{Subscriptions}
\end{itemize}
