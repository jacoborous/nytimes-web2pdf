Sections

SEARCH

\protect\hyperlink{site-content}{Skip to
content}\protect\hyperlink{site-index}{Skip to site index}

\href{https://www.nytimes.com/section/world/europe}{Europe}

\href{https://myaccount.nytimes.com/auth/login?response_type=cookie\&client_id=vi}{}

\href{https://www.nytimes.com/section/todayspaper}{Today's Paper}

\href{/section/world/europe}{Europe}\textbar{}Глубоководная подлодка.
Смертельный пожар. Тайные планы России в подводном мире.

\url{https://nyti.ms/2Kl8Fus}

\begin{itemize}
\item
\item
\item
\item
\item
\end{itemize}

Advertisement

\protect\hyperlink{after-top}{Continue reading the main story}

Supported by

\protect\hyperlink{after-sponsor}{Continue reading the main story}

\hypertarget{ux433ux43bux443ux431ux43eux43aux43eux432ux43eux434ux43dux430ux44f-ux43fux43eux434ux43bux43eux434ux43aux430-ux441ux43cux435ux440ux442ux435ux43bux44cux43dux44bux439-ux43fux43eux436ux430ux440-ux442ux430ux439ux43dux44bux435-ux43fux43bux430ux43dux44b-ux440ux43eux441ux441ux438ux438-ux432-ux43fux43eux434ux432ux43eux434ux43dux43eux43c-ux43cux438ux440ux435}{%
\section{Глубоководная подлодка. Смертельный пожар. Тайные планы России
в подводном
мире.}\label{ux433ux43bux443ux431ux43eux43aux43eux432ux43eux434ux43dux430ux44f-ux43fux43eux434ux43bux43eux434ux43aux430-ux441ux43cux435ux440ux442ux435ux43bux44cux43dux44bux439-ux43fux43eux436ux430ux440-ux442ux430ux439ux43dux44bux435-ux43fux43bux430ux43dux44b-ux440ux43eux441ux441ux438ux438-ux432-ux43fux43eux434ux432ux43eux434ux43dux43eux43c-ux43cux438ux440ux435}}

Мало кто готов говорить о 14 моряках, погибших на борту российского чуда
техники. Еще меньше желающих рассказать о том, что они делали в водах
близ Норвегии.

\includegraphics{https://static01.nyt.com/images/2020/04/05/world/20russian-02russian-sub/05russia-sub-articleLarge.jpg?quality=75\&auto=webp\&disable=upscale}

By \href{https://www.nytimes.com/by/james-glanz}{James Glanz} and Thomas
Nilsen

\begin{itemize}
\item
  April 20, 2020
\item
  \begin{itemize}
  \item
  \item
  \item
  \item
  \item
  \end{itemize}
\end{itemize}

\href{https://www.nytimes.com/2020/04/20/world/europe/russian-submarine-fire-losharik.html}{Read
in English}

РАСПОЛОЖЕНИЕ БЕРЕГОВОЙ ОХРАНЫ НОРВЕГИИ НА БАРЕНЦЕВОМ МОРЕ - Вряд ли
найдется более жуткое место для борьбы с пожаром, чем в чреве
«Лошарика», загадочной российской глубоководной подлодки.

1 Июля, в аккумуляторном отсеке подводной лодки шедшей в российских
водах в четырехстах километрах к северу от Северного полярного круга,
что-то пошло не так, как надо.

На любой подводной лодке пожар -- наверное, самый жуткий кошмар для
подводника, но пожар на борту «Лошарика» представлял угрозу совершенно
другого порядка. Этот корабль способен погружаться глубже, чем
практически любая другая субмарина. Однако именно достижения инженерной
мысли, давшие ей эту способность, вероятно и решили судьбу 14 моряков,
погибших в этой катастрофе.

Единственное, что остается еще большей загадкой, чем то, что произошло
на борту, -- это вопрос о том, что же делала это подлодка на глубине 300
метров в 60 морских милях от Норвегии.

Это чудовищное происшествие может стать ключом к пониманию российских
военных амбиций на глубине морей и океанов. Того, как они вписываются в
план максимального использования военно-морских сил в Арктике для
достижения Россией стратегических целей по всему земному шару, включая
возможность блокирования жизненно важных каналов международной связи.

Официальные власти в Москве не распространяются о катастрофе с
«Лошариком» и настаивают на том, что это был всего лишь
исследовательский аппарат. Норвежские военные, чьи наблюдательные посты,
корабли и разведывательные самолеты следят за российским Северным флотом
в рамках сотрудничества по НАТО, отказываются сообщить об увиденном.
Единственными гражданскими свидетелями последовавшей за пожаром
спасательной операции, могли быть стаи российских рыбацких судов,
нелегально промышляющих в этом районе.

Однако совершенно очевидно лишь одно: речь идет о совершенно секретной
операции, а в списке погибших ряд наиболее опытных, награжденных самыми
высокими знаками отличия, офицеров-подводников.

\includegraphics{https://static01.nyt.com/images/2020/04/21/world/20RUSSIAN-01russia-sub/05russia-sub-03-articleLarge.jpg?quality=75\&auto=webp\&disable=upscale}

Чтобы понять, почему эти подводники оказались на борту субмарины,
которая, возможно, способна погружаться на глубину до 6,000 метров
(глубина в десять раз большая, чем та, на которой, как полагают,
способны оперировать американские обитаемые подводные аппараты),
достаточно вспомнить то, что пересекает дно Северной Атлантики вдоль и
поперек.

Это бесконечные километры оптоволоконных кабелей, несущих большую долю
мирового интернет-трафика, включая финансовые транзакции на триллионы
долларов. Здесь же находятся кабели, соединяющие сонары
гидроакустической разведки.

Президент России Владимир Путин и военное руководство страны все чаще
подчеркивают, насколько важно контролировать потоки информации. Они
говорят, что это необходимо для сохранения преимущества в случае
конфликта, -- говорит Катаржина Зюск, руководитель Центра политики
безопасности норвежского Института оборонных исследований. Независимо от
того, где в мире назревает конфликт, выведение из строя этих кабелей, -
говорит профессор Зюск, -- может заставить противника дважды подумать,
прежде чем рискнуть пойти на эскалацию конфликта.

«Степень возможного ущерба, делающего такой риск неприемлемым, гораздо
ниже в Европе и на Западе, чем во времена холодной войны», - говорит
профессор. -- «Так что вам и делать-то особенно ничего не потребуется».

Не всякая подводная лодка способна на такое, во всяком случае, не в
масштабах практически всего пространства морского дна.

Но «Лошарик» -- не простая подводная лодка. Ее внутренний корпус, как
полагают, состоит из ряда титановых сфер, в которых помещаются
центральный пост, жилые помещения, ядерный реактор и другое
оборудование.

Сферы эти очень тесные и соединяются еще более узкими переходами.

Если в тот июльский день люки были задраены, чтобы замедлить горение и
распространение огня, члены экипажа, по-видимому, оказались, как в
западне, в тесных, полутемных и задымленных камерах. Если они были в той
камере, где находился аккумуляторный отсек, в котором, по-видимому и
начались проблемы, им пришлось бороться с пламенем, бушевавшем в
пространстве шириной около одного метра, - говорит Питер Лобнер, бывший
судовой механик подводной лодки ВМС США.

«Это самое жуткое место, в котором только можно оказаться на подводной
лодке», - говорит г-н Лобнер.

\hypertarget{ux44dux442ux43e-ux43eux447ux435ux43dux44c-ux43fux43e-ux440ux443ux441ux441ux43aux438}{%
\subsection{Это очень
по-русски}\label{ux44dux442ux43e-ux43eux447ux435ux43dux44c-ux43fux43e-ux440ux443ux441ux441ux43aux438}}

Русские рыбаки на небольшом суденышке шли в восточном направлении.
Возможно, они находились в закрытых морских водах, в то время, когда
прямо по курсу перед ними из глубин вырвалась и всплыла подводная лодка,
-- рассказал один из них в интервью мурманской газете «СеверПост».

«Мы шли на Кильдин, это островок неподалеку», -- рассказал рыбак
репортеру «Северпост» по телефону, -- «и тут, около полдесятого вечера
всплывает подводная лодка. Неожиданно и полностью выходит на
поверхность. Я в жизни такого никогда не видел. А по палубе в страшной
суматохе бегают люди».

Этой подлодкой был не «Лошарик», это было судно гораздо большее по
размеру: его корабль-матка или судно-носитель. Конструкция «Лошарика»
позволяет ему пристыковываться к днищу носителя. Таким образом, он может
быть транспортирован для обслуживания или при переброске на большие
расстояния, или, как это видимо произошло 1 июля у берегов Норвегии, --
для спасательных работ при аварии.

Почему Россия не перекрыла этот район неизвестно, но рассказ рыбака
очень подробный и точный; он и его товарищи, похоже, оказались
единственными посторонними свидетелями тайной спасательной операции. Они
промышляли в закрытом районе, и, тем не менее, они решились рассказать о
том, что они увидели.

«Это очень по-русски», -- сказал Джеффри Манкофф, заместитель директора
и старший научный сотрудник Российской и Евразийской программы Центра
Стратегических и международных исследований в Вашингтоне.

Подводная лодка поспешно удалилась, однако немедленного оповещения
Россией норвежских органов радиационной и ядерной безопасности о
происшествии с возможной утечкой радиоактивных веществ в Баренцевом море
не последовало, -- сообщила Астрид Лайленд, руководитель отдела ядерной
безопасности.

Официальное российское информационное агентство ТАСС сообщило о
происшествии на следующий день, не упомянув, что на подводной лодке была
ядерная энергетическая установка. Статья в газете «СеверПост» появилась
на следующее утро.

Image

Фьорд неподалеку от Киркенеса, небольшого норвежского городка рядом с
границей России.Credit...Фото Матиаса Свольда для «Нью-Йорк Таймс»

Как сообщила г-жа Лайланд, между Россией и Норвегией существует
соглашение о взаимном оповещении в случаях происшествий с ядерными
установками. «К сожалению», - сказала она, - «Россия считает, что это
соглашение не распространяется на военные установки подводные лодки в
частности».

Сложная и запутанная история «Лошарика», как и растущая мощь российского
Северного флота, проистекает из одного очень простого объяснения,
считает профессор Зюск, норвежский аналитик.

«Морской флот занимает в сердце Путина особое место,» - полагает она. -
«Это один из символов великой державы».

Северный флот занимает верхнюю строчку военного бюджета г-на Путина и
включает такие статьи первостепенной важности, как обеспечение наиболее
совершенными надводными кораблями и крылатыми ракетами. В 2014 году
Северный флот получил под свое командование арктические бригады и
солдат, экипированных самым современным снаряжением для ведения военных
действий в условиях холодных климатических условий. Кроме того,
развертываются баллистические ракеты и ударные подводные лодки
последнего поколения.

Обладая всей этой военно-морской мощью, самый короткий путь, чтобы
застать Соединенные Штаты врасплох -- пройти на всех парах из Арктики в
Северную Атлантику, -- считает Хизер А. Конли, первый вице-президент по
вопросам Европы, Евразии и Арктики Центра стратегических и международных
исследований.

«Несомненно, эта сфера развивается все более динамично», - сказал г-жа
Конли. «Это на самом деле выглядит как обновленная версия сюжета
кинофильма «Охота за ``Красным Октябрем''».

Не стоит забывать и об экономических выгодах, -- считает г-жа Конли, --
Россия не скрывает своего стремления контролировать северные морские
коммуникации через Арктику и увеличивать объемы добычи нефти и газа, по
мере того как ледяной покров отступает под влиянием климатических
изменений.

За последние пять лет были построены или модернизированы 14 аэродромов
вдоль Северного морского пути; на арктических архипелагах были открыты
три полностью автономных базы. Миллиарды долларов вложены в развитие
газовых месторождений на полуострове Ямал, где общие запасы оцениваются
почти в 17 триллионов кубометров. В перспективе природный газ Ямала
будет питать газопровод, строящийся ныне через Балтийское море для
снабжения газом Западной Европы.

Однако, процесс добычи нефти и газа севернее Ямала крайне трудоемок, а
перспективы развития туризма и международного судоходства не ясны.
Поэтому экономика может и не прибавить роста в ближайшие пятьдесят лет,
а может быть, и в дальнейшей перспективе, - считает Андреас Остаген,
старший научный сотрудник Института Фритьофа Нансена в Осло и соавтора
работы «Европейский союз и геополитика Арктики».

Помимо того, что России самой необходимо следить за сохранением ядерного
сдерживания, разгадка пристального интереса России к Арктике в том, чего
Москва хочет меньше всего: а именно оказаться втянутой в продолжительный
конфликт с НАТО, говорит профессор Зюск, . Россия знает, что у нее нет
необходимых ресурсов, чтобы одержать верх в конфликте такого рода,
считает профессор Зюск.

По этой причине неважно, где конфликт начнется, говорит она, но «Россия
сделает все, чтобы сохранить стратегическую инициативу». Она считает,
что «на первый план здесь выходит информационное превосходство».

Например, российские генералы открыто говорят о том, что необходимо
сеять раздор в государственной финансовой системе противника, --
продолжает профессор Зюск, -- и вывод из строя подводных кабелей
«безусловно, способствовал бы достижению этой цели».

В докладе, сделанном в 2017 году исследовательским и образовательным
центром Policy Exchange в Великобритании, говорится, что подводные
кабели передают 97\% мировой информации, включая финансовые сделки на
сумму порядка 10 триллионов долларов ежедневно. В большинстве своем
кабели не защищены, и их легко обнаружить. Всего лишь несколько лет
назад официальные представители американских вооруженных сил и разведки
сообщали, что российские подводные лодки часто действовали около этих
магистральных кабелей.

Поскольку Интернет в случае повреждения кабелей может перенаправить
информацию, западные специалисты часто игнорировали угрозу диверсии. «Но
с учетом огромного значения информации для любых структур на Западе», --
говорит профессор Зюск, -- «достаточно оказать воздействие, снизив
эффективность работы сети».

«Что случится, когда люди лишатся Фейсбука и Твиттера? О Боже!», -
сказала она не совсем в шутку.

Матье Булег, научный сотрудник Программы по России и Евразии центра
Chatham House в Великобритании, сказал, что такой специализированный
аппарат как «Лошарик» мог бы помочь оценить способность Запада на
ответные меры при повреждении кабелей.

«Это часть новых российских возможностей насолить нам», -- сказал г-н
Булег.

Image

Норвежская береговая линия вблизи границы с Российской Федерацией, как
она видна через бинокль с борта корабля береговой охраны Норвегии
«Фарм».Credit...Фото Матиаса Свольда для «Нью-Йорк Таймс»

\hypertarget{ux43dux435ux431ux44cux44eux449ux435ux435ux441ux44f-ux44fux439ux446ux43e}{%
\subsection{НЕБЬЮЩЕЕСЯ
ЯЙЦО}\label{ux43dux435ux431ux44cux44eux449ux435ux435ux441ux44f-ux44fux439ux446ux43e}}

Что же до самой аварии, немногие удивились из-за того, что на жемчужине
российского подводного флота произошел пожар. Это случилось на не очень
большом расстоянии от базы -- предположительно, на глубине не более 300
метров. В результате погибла большая часть команды. Некоторые эксперты
считают, что у русских более высокая толерантность к риску, чем на
Западе.

«Лошарик» был спроектирован в 80-е годы, но из-за распада Советского
Союза, согласно последнему изданию «Подводных лодок холодной войны»
историков Нормана Полмара и К. Дж. Мура, на воду судно было спущено лишь
в 2003 году.

В 2012 году «Лошарик» участвовал в научной экспедиции, связанной с
бурением трехкилометровой скважины в арктической коре для получения
образцов породы. Лучшее из опубликованных изображение подлодки появилось
несколькими годами позже -- в 2015 году, когда она неожиданно всплыла во
время рекламной съемки «Мерседеса» для российского издания журнала Top
Gear.

«Подобно яичной скорлупе, титановые сферы противостоят колоссальному

давлению более успешно, чем традиционный, вытянутый корпус, -- говорит
г-н Полмар. «Она может медленно опуститься на дно, не треснув при этом»
-- говорит он.

Г-н Полмар, считает, что «в американском флоте не было ничего, что могло
бы сравниться» с «Лошариком» с точки зрения глубины, на которую он
способен доставить свою команду. Разные источники, -- упоминает он, --

«приводят разные цифры, от 2,5 км до 6 км в качестве максимальной
глубины, доступной этому загадочному аппарату».

Г-н Лобнер, бывший офицер американского подводного флота, сказал: «У нас
нет ничего, кроме беспилотных аппаратов», которые действуют на таких
глубинах. А с помощью подводной лодки русские «могут действовать там,
используя различные инструменты», добавил он.

Тем не менее, в то время как одни видят в этом инженерное чудо, другие
считают это свидетельством того, что Россия, возможно, не способна
создать подобие сложных, автономных, подводных дронов, на которые,
по-видимому, делает ставку США.

«Они скорее приспособят существующие системы, модернизируют их и

попытаются как-нибудь обойтись этим», -- сказал г-н Булег. «Так что,
неудивительно, что эти устройства продолжают взрываться», -- продолжил
он. Г-н Булег считает, что подобные происшествия были гораздо более
распространенным явлением, чем это представляется общественности.

Джон Пайк, директор аналитического центра GlobalSecurity.org, сказал,
что

пожар на «Лошарике» позволяет предположить, что российским военным
по-прежнему приходится бороться с некоторыми извечными проблемами:
коррумпированными подрядчиками, проблемами контроля качества в процессе
производства, цепочкой поставщиков запчастей и обслуживанием.

«Я предполагаю, что каждая вторая подлодка российского флота
сталкивается с похожими проблемами», -- сказал г-н Пайк. -- «Я думаю ко
всему этому относятся наплевательски».

Российская деловая газета «Коммерсант», цитируя источники близкие к
расследованию аварии на «Лошарике», пишет, что когда на подлодке
заметили первое появление дыма, это не было воспринято как что-то
катастрофическое. «Коммерсант» пишет, что в это время «Лошарик» еще мог
быть пристыкован к кораблю-носителю.

После частичной эвакуации, 10 членов экипажа остались бороться с огнем
вместе с четырьмя пожарными, прибывшими к ним на помощь с
корабля-носителя. Ситуация становилась все более тяжелой по мере того,
как в двух аварийных дыхательных системах на борту лодки заканчивался
кислород, пишет «Коммерсант». «Члены экипажа начали задыхаться от дыма,
а в аккумуляторном отсеке, по-видимому, произошел взрыв», -- сообщает
газета.

Г-н Лобнер говорит, что даже на обычной атомной подводной лодке проходы
в аккумуляторном отсеке настолько узки, что даже при дежурных
инспекционных работах приходится пробираться через них ползком или на
спине. Жилые помещения экипажа также малы и могли быстро заполниться
дымом, -- добавил он.

«Это совсем не то же самое, что войти в горящий дом», - сказал г-н
Лобнер.

Image

\hypertarget{ux433ux43bux430ux437ux430-ux43eux442ux43aux440ux44bux442ux44b-ux440ux43eux442-ux437ux430ux43aux440ux44bux442}{%
\subsection{ГЛАЗА ОТКРЫТЫ. РОТ
ЗАКРЫТ.}\label{ux433ux43bux430ux437ux430-ux43eux442ux43aux440ux44bux442ux44b-ux440ux43eux442-ux437ux430ux43aux440ux44bux442}}

Русские - не единственные кто не хочет говорить о «Лошарике».

Адмирал Джеймс Дж. Фогго III, командующий Шестым Флотом США, сфера
действий которого включает Европу, отказался дать интервью для этой
статьи. Так же поступил Хаакон Брюн-Ханссен, командующий оборонным
сектором вооруженных сил Норвегии.

Даже рядовой Сандер Бадар, 20-летний новобранец норвежской армии,

тщательно подбирал слова, одновременно осматривая через огромный бинокль
воды вблизи северного побережья России с наблюдательного поста,
расположенного на скале на почти трехсотметровой высоте над Баренцевым
морем. Именно в этом направлении по другую сторону побережья полуострова
Рыбачий находился горевший «Лошарик».

«То, что мы ведем наблюдение за их границей и видим, что там происходит,
не является тайной», - сказал рядовой Бадар октябрьским днем, когда
арктическое солнце уже угасало.

Со своими наблюдательными постами, подобными посту рядового Бадара, а
также с разведывательными самолетами и военными кораблями, Вооруженные
силы Норвегии играют роль глаз и ушей НАТО на рубежах России. Но когда
его спросили о российских подводных лодках, рядовой Бадар отказался
поделиться тем, что он, возможно, видел.

В первом сообщении российского информационного агентства ТАСС о пожаре
на «Лошарике» было написано, что 14 моряков погибли на борту
«глубоководного аппарата», не упомянув при этом имевшийся на нем ядерный
реактор. На следующий день, пресс-секретарь г-на Путина сказал, что
информация о происшествии «относится к категории совершенно секретной
информации».

В последующие дни, г-н Путин посмертно присвоил высшую награду страны
«Герой Российской Федерации» четырем членам команды и другие менее
почетные награды остальным десяти. Во время похорон в Санкт-Петербурге
один офицер ВМФ сказал, что команда «предотвратила катастрофу
планетарного масштаба».

Российские источники говорят, что планируется полностью восстановить
подводную лодку и вернуть ее в рабочее состояние. Похоже, что это
вызывает беспокойство далеко не у всех.

Image

Джон Б. Паджет III, контр-адмирал в отставке, бывший командующий
Тихоокеанским подводным флотом, сказал в телефонном интервью, что он не
считает, что «Лошарик» -- это пример того, что США отстает в этой сфере.

«Мы опускаемся настолько глубоко, насколько нам нужно и продвигаемся
настолько быстро, насколько нам нужно», -- подчеркнул адмирал Паджет.

Но полковник Ейстейн Кварвинг, начальник службы общественных связей
Норвежского Объединенного Штаба дал ясно понять, что ставки высоки.

У норвежских военных, сказал полковник Кварвинг, есть прямая линия связи
по Скайпу с командующим Северным Флотом России, которая проверяется
каждую неделю. В течение месяцев, прошедших после пожара, сказал он,
Россия провела крупнейшие морские учения после холодной войны.

Как сюда вписывается «Лошарик»?

«Вы погружаетесь глубоко и движетесь беззвучно», - говорит Кварвинг.
«Ключевым является слово «необнаруживаемый». Если они идут
необнаруживаемыми, куда им заблагорассудится, -- это серьёзная
проблема».

Томас Нильсен является редактором «Индепендент Баренц Обсервер»

Image

Полковник Ейстейн Кварвинг, начальник службы общественных
связей.Credit...Фото Матиаса Свольда для «Нью-Йорк Таймс»

Advertisement

\protect\hyperlink{after-bottom}{Continue reading the main story}

\hypertarget{site-index}{%
\subsection{Site Index}\label{site-index}}

\hypertarget{site-information-navigation}{%
\subsection{Site Information
Navigation}\label{site-information-navigation}}

\begin{itemize}
\tightlist
\item
  \href{https://help.nytimes.com/hc/en-us/articles/115014792127-Copyright-notice}{©~2020~The
  New York Times Company}
\end{itemize}

\begin{itemize}
\tightlist
\item
  \href{https://www.nytco.com/}{NYTCo}
\item
  \href{https://help.nytimes.com/hc/en-us/articles/115015385887-Contact-Us}{Contact
  Us}
\item
  \href{https://www.nytco.com/careers/}{Work with us}
\item
  \href{https://nytmediakit.com/}{Advertise}
\item
  \href{http://www.tbrandstudio.com/}{T Brand Studio}
\item
  \href{https://www.nytimes.com/privacy/cookie-policy\#how-do-i-manage-trackers}{Your
  Ad Choices}
\item
  \href{https://www.nytimes.com/privacy}{Privacy}
\item
  \href{https://help.nytimes.com/hc/en-us/articles/115014893428-Terms-of-service}{Terms
  of Service}
\item
  \href{https://help.nytimes.com/hc/en-us/articles/115014893968-Terms-of-sale}{Terms
  of Sale}
\item
  \href{https://spiderbites.nytimes.com}{Site Map}
\item
  \href{https://help.nytimes.com/hc/en-us}{Help}
\item
  \href{https://www.nytimes.com/subscription?campaignId=37WXW}{Subscriptions}
\end{itemize}
