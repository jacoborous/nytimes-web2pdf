Sections

SEARCH

\protect\hyperlink{site-content}{Skip to
content}\protect\hyperlink{site-index}{Skip to site index}

\href{https://www.nytimes.com/section/world}{World}

\href{https://myaccount.nytimes.com/auth/login?response_type=cookie\&client_id=vi}{}

\href{https://www.nytimes.com/section/todayspaper}{Today's Paper}

\href{/section/world}{World}\textbar{}SANDS DIES IN NORTHERN IRELAND
JAIL ON THE 66TH DAY OF HUNGER STRIKE

\url{https://nyti.ms/29z5rVz}

\begin{itemize}
\item
\item
\item
\item
\item
\end{itemize}

Advertisement

\protect\hyperlink{after-top}{Continue reading the main story}

Supported by

\protect\hyperlink{after-sponsor}{Continue reading the main story}

\hypertarget{sands-dies-in-northern-ireland-jail-on-the-66th-day-of-hunger-strike}{%
\section{SANDS DIES IN NORTHERN IRELAND JAIL ON THE 66TH DAY OF HUNGER
STRIKE}\label{sands-dies-in-northern-ireland-jail-on-the-66th-day-of-hunger-strike}}

By William Borders, Special To the New York Times

\begin{itemize}
\item
  May 5, 1981
\item
  \begin{itemize}
  \item
  \item
  \item
  \item
  \item
  \end{itemize}
\end{itemize}

\includegraphics{https://s1.nyt.com/timesmachine/pages/1/1981/05/05/112457_360W.png?quality=75\&auto=webp\&disable=upscale}

See the article in its original context from\\
May 5, 1981, Section A, Page
1\href{https://store.nytimes.com/collections/new-york-times-page-reprints?utm_source=nytimes\&utm_medium=article-page\&utm_campaign=reprints}{Buy
Reprints}

\href{http://timesmachine.nytimes.com/timesmachine/1981/05/05/112457.html}{View
on timesmachine}

TimesMachine is an exclusive benefit for home delivery and digital
subscribers.

About the Archive

This is a digitized version of an article from The Times's print
archive, before the start of online publication in 1996. To preserve
these articles as they originally appeared, The Times does not alter,
edit or update them.

Occasionally the digitization process introduces transcription errors or
other problems; we are continuing to work to improve these archived
versions.

Robert Sands, the Irish Republican Army hunger striker and Member of
Parliament, died early today on his 66th day without food. He had begun
his fast in an effort to force the British Government to recognize
I.R.A. inmates as political prisoners rather than common criminals.

Death came to Mr. Sands at 1:17 A.M. (8:17 P.M. Monday, New York time)
in the hospital wing of the Maze Prison, where members of his family had
been conducting an anxious vigil for several days.

''He just finally starved to death,'' said a Government official. The
I.R.A. said this morning that Mr. Sands would be buried later this week
''with all the ceremony due a republican volunteer.'' Soon after 2 A.M.,
I.R.A. sympathizers in cars with loudspeakers and sirens began spreading
the news of Mr. Sand's death through Roman Catholic neighborhoods of
Belfast.

''Bobby Sands is dead,'' they broadcast through the quiet streets.
''Come out! Come out!'' In response, hundreds of residents poured into
the streets. Some held brief prayer ceremonies for Mr. Sands, and
others, largely young men wearing ski masks to preserve their anonymity,
began engaging the police.

In several parts of town, youths hurled rocks and firebombs at the
police, who responded by firing rubber bullets. At intersections leading
into scattered Catholic neighborhoods, gangs of youths erected
barricades, some of them made out of hijacked vehicles that had been set
on fire.

In Londonderry, the other major city in the province, the response to
Mr. Sands death was generally more peaceful, as several hundred people
poured out of their houses to take part in a 15-minute silent prayer
service.

Even in Belfast, as dawn broke over the tense city, there was still no
widespread pattern of violence, and most of the streets were quiet.
There were no reports of serious injuries either among the demonstators
or the security forces.

In the expectation of disruptions after Mr. Sands's death, there have
been heavy police and army patrols on the streets for several days.

Humphrey Atkins, the British Cabinet minister responsible for Northern
Ireland, said this morning: ''I regret this needless and senseless
death. Too many have died in Northern Ireland. In this case it was
self-inflicted.'' Mr. Atkins also said, ''We should not forget the many
others who have died.'' Since the late 1960's, when the present phase of
violence in Northern Ireland's ancient sectarian dispute began, more
than 2,000 people have died in what the Irish call ''the troubles.''

Mr. Atkins has been keeping long hours as Mr. Sands's death neared,
conferring with security advisers and, presumably, with Prime Minister
Margaret Thatcher, who has dictated the unyielding posture that the
Government has taken with Mr. Sands and three other I.R.A. hunger
strikers.

Before he lost consciousness Sunday, the 27-year-old Mr. Sands had been
drinking water. The fact that he was no longer drinking hastened his
death.

Prison authorities could presumably have force-fed Mr. Sands since he
could no longer resist, but in 1974 the British Government instituted a
policy of not force-feeding prisoners who go on hunger strikes.

Anticipating outbreaks of violence, the police barred vehicular traffic
from some Catholic neighborhoods of Belfast. Four-man patrols of
soldiers wearing battle fatigues and carrying automatic weapons picked
their way gingerly through littered streets, the last man usually
covering the rear.

There were taunts of ''Brits out!'' from the windows of the brick row
houses, and occasional bottles or firebombs were thrown at the armored
vehicles.

But in general the level of violence was normal, or less than normal, by
the standards of Northern Ireland. After two boys, aged 12 and 13, were
arrested for throwing rocks at police cars in a Catholic neighborhood,
the police appealed to parents to keep teen-agers at home, and
apparently many of them did.

Mr. Sands was elected to Parliament a month ago in a by-election called
to fill a vacancy caused by death. But he was not permitted to leave
prison to take his seat.

In a statement last night, the political wing of the I.R.A. said that
''the British now prepare for the murder of the elected representative
of the people of Fermanagh and South Tyrone.'' Right to Wear Civilian
Clothes

The statement reiterated the demand that Mr. Sands had made that he and
the other Irish nationalists jailed on terrorism charges be granted
political status, since they view themselves as soldiers in a war of
independence, not criminals. They want the rights to wear civilian
clothing and to be excused from routine prison work.

''To accept the status of criminal would be to degrade myself and to
admit that the cause I believe in and cherish is wrong,'' Mr. Sands once
explained, in an article that the I.R.A. has been recirculating during
his hunger strike.

But Prime Minister Thatcher's Government insists that ''there can be no
compromise with murder and terrorism,'' as she put it recently. Mr.
Sands, who was serving a 14-year sentence for the possession of
firearms, began his fast on March 1 and was joined in it in subsequent
weeks by three other Irish nationalist prisoners.

Mr. Sands joined the Republican movement as a teen-ager in the early
1970's after an outbreak of secctarian violence forced his family to
move from a Protestant neighborhood into a Roman Catholic one.

''I had seen too many homes wrecked, fathers and sons arrested,
neighbors hurt, friends murdered, too much gas, shooting and blood,''
Mr. Sands said, explaining his decision to join the I.R.A. -\/-\/-\/-
State Dept. Expresses Regret

In a statement issued in Washington last night, the State Department
said that it deeply regretted the death of Robert Sands. In New York,
Governor Carey and Mayor Koch criticized the British Government but said
they opposed any violence in reaction to the death.

''We deeply regret Mr. Sands's death,'' a State Department spokesman
said. ''We hope that the hunger strike by three other inmates at the
Maze Prison will not end in the same tragic fashion.''

Governor Carey accused the British Government of ''intransigence'' on
the issue of the political status of imprisoned members of the Irish
Republican Army, adding: ''I deeply regret that the British Government
has let Bobby Sands bring his hunger strike to its bitter conclusion.''

But he and Mayor Koch said they were opposed to violence. ''It is my
fervent hope that the death of Bobby Sands will not lead to further
bloodshed,'' the Mayor said in his statement. That plea was joined by
Senator Edward M. Kennedy, Democrat of Massachusetts. ''At this time of
heightened tension, I urge all sides in Northern Ireland to resist calls
for further violence,'' he said.

Immediately after Mr. Sands's death was announced, a small group of
placard-carrying demonstrators gathered in front of the British
Consulate on Third Avenue between 51st and 52nd Streets, vowing to stay
through the night.

''He's a free man now,'' Veronica Pugh, a Bronx resident, said of Mr.
Sands. ''He beat the Brits.''

Advertisement

\protect\hyperlink{after-bottom}{Continue reading the main story}

\hypertarget{site-index}{%
\subsection{Site Index}\label{site-index}}

\hypertarget{site-information-navigation}{%
\subsection{Site Information
Navigation}\label{site-information-navigation}}

\begin{itemize}
\tightlist
\item
  \href{https://help.nytimes.com/hc/en-us/articles/115014792127-Copyright-notice}{©~2020~The
  New York Times Company}
\end{itemize}

\begin{itemize}
\tightlist
\item
  \href{https://www.nytco.com/}{NYTCo}
\item
  \href{https://help.nytimes.com/hc/en-us/articles/115015385887-Contact-Us}{Contact
  Us}
\item
  \href{https://www.nytco.com/careers/}{Work with us}
\item
  \href{https://nytmediakit.com/}{Advertise}
\item
  \href{http://www.tbrandstudio.com/}{T Brand Studio}
\item
  \href{https://www.nytimes.com/privacy/cookie-policy\#how-do-i-manage-trackers}{Your
  Ad Choices}
\item
  \href{https://www.nytimes.com/privacy}{Privacy}
\item
  \href{https://help.nytimes.com/hc/en-us/articles/115014893428-Terms-of-service}{Terms
  of Service}
\item
  \href{https://help.nytimes.com/hc/en-us/articles/115014893968-Terms-of-sale}{Terms
  of Sale}
\item
  \href{https://spiderbites.nytimes.com}{Site Map}
\item
  \href{https://help.nytimes.com/hc/en-us}{Help}
\item
  \href{https://www.nytimes.com/subscription?campaignId=37WXW}{Subscriptions}
\end{itemize}
