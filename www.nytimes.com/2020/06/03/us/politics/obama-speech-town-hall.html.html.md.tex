Sections

SEARCH

\protect\hyperlink{site-content}{Skip to
content}\protect\hyperlink{site-index}{Skip to site index}

\href{https://www.nytimes.com/section/politics}{Politics}

\href{https://myaccount.nytimes.com/auth/login?response_type=cookie\&client_id=vi}{}

\href{https://www.nytimes.com/section/todayspaper}{Today's Paper}

\href{/section/politics}{Politics}\textbar{}Obama Voices Support for
George Floyd Protesters and Calls for Police Reform

\url{https://nyti.ms/2MwSmf3}

\begin{itemize}
\item
\item
\item
\item
\item
\item
\end{itemize}

\href{https://www.nytimes.com/news-event/george-floyd-protests-minneapolis-new-york-los-angeles?action=click\&pgtype=Article\&state=default\&region=TOP_BANNER\&context=storylines_menu}{Race
and America}

\begin{itemize}
\tightlist
\item
  \href{https://www.nytimes.com/2020/07/26/us/protests-portland-seattle-trump.html?action=click\&pgtype=Article\&state=default\&region=TOP_BANNER\&context=storylines_menu}{Protesters
  Return to Other Cities}
\item
  \href{https://www.nytimes.com/2020/07/24/us/portland-oregon-protests-white-race.html?action=click\&pgtype=Article\&state=default\&region=TOP_BANNER\&context=storylines_menu}{Portland
  at the Center}
\item
  \href{https://www.nytimes.com/2020/07/23/podcasts/the-daily/portland-protests.html?action=click\&pgtype=Article\&state=default\&region=TOP_BANNER\&context=storylines_menu}{Podcast:
  Showdown in Portland}
\item
  \href{https://www.nytimes.com/interactive/2020/07/16/us/black-lives-matter-protests-louisville-breonna-taylor.html?action=click\&pgtype=Article\&state=default\&region=TOP_BANNER\&context=storylines_menu}{45
  Days in Louisville}
\end{itemize}

Advertisement

\protect\hyperlink{after-top}{Continue reading the main story}

Supported by

\protect\hyperlink{after-sponsor}{Continue reading the main story}

\hypertarget{obama-voices-support-for-george-floyd-protesters-and-calls-for-police-reform}{%
\section{Obama Voices Support for George Floyd Protesters and Calls for
Police
Reform}\label{obama-voices-support-for-george-floyd-protesters-and-calls-for-police-reform}}

Mr. Obama, offering a starkly more upbeat assessment of peaceful
protesters and their motives than President Trump has, urged people to
``remember that this country was founded on protest --- it is called the
American Revolution.''

\includegraphics{https://static01.nyt.com/images/2020/06/03/us/politics/03vid-Obama-Live/03vid-Obama-Live--videoSixteenByNineJumbo1600.jpg}

By \href{https://www.nytimes.com/by/glenn-thrush}{Glenn Thrush}

\begin{itemize}
\item
  June 3, 2020
\item
  \begin{itemize}
  \item
  \item
  \item
  \item
  \item
  \item
  \end{itemize}
\end{itemize}

WASHINGTON --- Former President Barack Obama threw his support behind
the efforts of peaceful protesters demanding police reforms during his
first on-camera remarks since
\href{https://www.nytimes.com/2020/06/03/us/live-george-floyd-protests-today.html}{a
wave of protests over the killing of George Floyd} convulsed the country
and upended the 2020 election.

Mr. Obama, offering a strikingly more upbeat assessment of the
protesters than
\href{https://www.nytimes.com/interactive/2020/us/elections/donald-trump.html}{President
Trump} and White House officials, said he believed only a ``tiny''
percentage had acted violently.

``For those who have been talking about protest, just remember that this
country was founded on protest --- it is called the American
Revolution,'' Mr. Obama said from his home in Washington. He made the
comments during an online round-table event with his former attorney
general Eric H. Holder Jr. and activists from Minneapolis sponsored by
My Brother's Keeper Alliance, a nonprofit group Mr. Obama founded.

``Every step of progress in this country, every expansion of freedom,
every expression of our deepest ideals have been won through efforts
that made the status quo uncomfortable,'' said Mr. Obama, who adopted a
conciliatory tone that contrasted sharply with Mr. Trump's tweets and
public remarks. ``And we should all be thankful for folks who are
willing, in a peaceful, disciplined way, to be out there making a
difference.''

Mr. Obama called on every mayor in the United States to review
use-of-force policies and to aggressively pursue an eight-point slate of
police reforms that include mandatory de-escalation of conflicts, a ban
on shooting at moving vehicles, timely reporting of violent incidents,
and prohibitions on some forms of restraint used by the police.

``Chokeholds and strangleholds, that's not what we do,'' Mr. Obama said
as he sat, tieless in blue shirt sleeves, in front of a bookcase.

He said officials in New York City and Chicago had already agreed to
adopt the measures. Other localities, including Atlanta, quickly
followed suit.

Mr. Obama also said that the ``vast majority'' of police officers, in
his view, were not violent, and predicted many would ultimately support
reforms despite the opposition of some unions.

Reflecting on the larger meaning of the protests, Mr. Obama said the
unrest after Mr. Floyd's death was ``unlike anything I have seen in my
lifetime'' and expressed hope that Americans would be ``reawakened'' to
unite around racial justice.

``In a lot of ways, what has happened in the last several weeks is that
challenges and structural problems here in the United States have been
thrown into high relief,'' he said. ``They are the outcome of not just
an immediate moment in time, but as the result of a long host of things
--- slavery, Jim Crow, redlining and institutional racism.''

With the exception of his support for protesters, Mr. Obama confined his
remarks to the issues of policing and racial disparities in health care
during the coronavirus pandemic that have led to higher rates of
infection and death in nonwhite communities.

Mr. Obama, as he often does, tried to avoid a one-on-one battle with his
successor, a fight he thinks will energize the president's conservative
base and overshadow his friend
\href{https://www.nytimes.com/interactive/2020/us/elections/joe-biden.html}{Joseph
R. Biden Jr.}, the Democrats' presumptive nominee.

Mr. Obama did not directly address Mr. Trump's bellicose comments or the
president's demand that the authorities ``dominate'' protesters,
although people close to the former president said he was outraged by
\href{https://www.nytimes.com/2020/06/02/us/politics/trump-walk-lafayette-square.html}{the
use of chemical spray on protesters} before Mr. Trump walked to a
fire-damaged church near the White House and brandished a Bible.

Instead, Mr. Obama expressed optimism that the reform effort could
transcend political divisions. He said that he was heartened by polls
showing broad support for their grievances, and that this made the
current situation more heartening than the protests in the late 1960s.

Mr. Obama's remarks tracked closely with two essays he posted online
over the last week in which he implored young protesters to channel
their rage into political action by turning out for Mr. Biden in
November and to embrace local reforms to hold police officers
accountable for abuses of power.

``We should be fighting to make sure that we have a president, a
Congress, a U.S. Justice Department, and a federal judiciary that
actually recognize the ongoing, corrosive role that racism plays in our
society and want to do something about it,''
\href{https://medium.com/@BarackObama/how-to-make-this-moment-the-turning-point-for-real-change-9fa209806067}{he
wrote in a post on Medium on Monday}.

In recent appearances, Mr. Obama has become more forceful in his
criticism of the White House,
\href{https://www.nytimes.com/2020/05/09/us/politics/obama-flynn-coronavirus-trump.html}{hammering
Mr. Trump's actions without invoking his successor's name}. Mr. Obama
rebuked the current administration's response to the coronavirus
pandemic as ``chaotic'' and questioned Mr. Trump's commitment to the
``rule of law'' in a call with former members of his White House team
last month.

For all his outward calm, Mr. Obama's passions are running high, and the
former president is finding it harder to stay on script, friends said.
Over the last few days, he has been working the phones with close
associates, including Mr. Holder, and strategizing about the best way to
address the issues without inflaming the crisis.

On Tuesday,
\href{https://www.kvrr.com/2020/06/02/twin-cities-law-enforcement-briefed-on-possible-obama-pence-visits-this-week/}{a
Minneapolis radio station reported} that Secret Service officials were
making preliminary preparations for a high-level visitor, perhaps Mr.
Obama. But people close to the former president said he had no intention
of traveling there this week --- although they did not rule out Mr.
Obama's participation in related events in the future.

Shortly before Mr. Obama spoke on Wednesday, former President Jimmy
Carter issued a statement calling for peaceful protest and systemic
change. ``As a white male of the South, I know all too well the impact
of segregation and injustice to African-Americans,'' the 95-year-old
former president
\href{https://www.cartercenter.org/news/pr/2020/statement-060320.html}{wrote}.
``We need a government as good as its people, and we are better than
this.''

Those comments came a day after another former president also presented
an alternative vision of the protests to Mr. Trump. In a
\href{https://www.bushcenter.org/about-the-center/newsroom/press-releases/2020/06/statement-by-president-george-w-bush.html}{lengthy
statement}, former President George W. Bush expressed solidarity with
the demonstrators in the streets and, without naming the incumbent
president, warned against trying to suppress the protests.

``It is a strength when protesters, protected by responsible law
enforcement, march for a better future,'' Mr. Bush said on Tuesday.
``This tragedy --- in a long series of similar tragedies --- raises a
long overdue question: How do we end systemic racism in our society? The
only way to see ourselves in a true light is to listen to the voices of
so many who are hurting and grieving.''

Mr. Bush, the only living Republican former president --- and one who
refused to vote for Mr. Trump in 2016 --- made no direct reference to
the current president. But Mr. Bush spoke after Mr. Trump's photo op
havoc, and the former president's comments read like a rebuke.

``Those who set out to silence those voices,'' Mr. Bush said, ``do not
understand the meaning of America --- or how it becomes a better
place.''

Mr. Obama struck a similar tone Wednesday, saying the overall message of
the protests was simple, admirable and unifying:

``See me, I'm human,'' he said.

Peter Baker contributed reporting.

Advertisement

\protect\hyperlink{after-bottom}{Continue reading the main story}

\hypertarget{site-index}{%
\subsection{Site Index}\label{site-index}}

\hypertarget{site-information-navigation}{%
\subsection{Site Information
Navigation}\label{site-information-navigation}}

\begin{itemize}
\tightlist
\item
  \href{https://help.nytimes.com/hc/en-us/articles/115014792127-Copyright-notice}{©~2020~The
  New York Times Company}
\end{itemize}

\begin{itemize}
\tightlist
\item
  \href{https://www.nytco.com/}{NYTCo}
\item
  \href{https://help.nytimes.com/hc/en-us/articles/115015385887-Contact-Us}{Contact
  Us}
\item
  \href{https://www.nytco.com/careers/}{Work with us}
\item
  \href{https://nytmediakit.com/}{Advertise}
\item
  \href{http://www.tbrandstudio.com/}{T Brand Studio}
\item
  \href{https://www.nytimes.com/privacy/cookie-policy\#how-do-i-manage-trackers}{Your
  Ad Choices}
\item
  \href{https://www.nytimes.com/privacy}{Privacy}
\item
  \href{https://help.nytimes.com/hc/en-us/articles/115014893428-Terms-of-service}{Terms
  of Service}
\item
  \href{https://help.nytimes.com/hc/en-us/articles/115014893968-Terms-of-sale}{Terms
  of Sale}
\item
  \href{https://spiderbites.nytimes.com}{Site Map}
\item
  \href{https://help.nytimes.com/hc/en-us}{Help}
\item
  \href{https://www.nytimes.com/subscription?campaignId=37WXW}{Subscriptions}
\end{itemize}
