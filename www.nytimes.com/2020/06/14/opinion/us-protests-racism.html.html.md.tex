\href{/section/opinion}{Opinion}\textbar{}An Insatiable Rage

\url{https://nyti.ms/3hsyWXi}

\begin{itemize}
\item
\item
\item
\item
\item
\item
\end{itemize}

\includegraphics{https://static01.nyt.com/images/2020/06/14/opinion/14Blow/merlin_173230803_fc4a3693-0fb3-4695-b912-a579464afb05-articleLarge.jpg?quality=75\&auto=webp\&disable=upscale}

Sections

\protect\hyperlink{site-content}{Skip to
content}\protect\hyperlink{site-index}{Skip to site index}

\href{/section/opinion}{Opinion}

\hypertarget{an-insatiable-rage}{%
\section{An Insatiable Rage}\label{an-insatiable-rage}}

It is an everyday struggle to neither fall into despair nor explode in
anger.

Protesters gathered in Manhattan in support of Black Lives Matter last
month.Credit...Ashley Gilbertson for The New York Times

Supported by

\protect\hyperlink{after-sponsor}{Continue reading the main story}

\href{https://www.nytimes.com/by/charles-m-blow}{\includegraphics{https://static01.nyt.com/images/2018/04/02/opinion/charles-m-blow/charles-m-blow-thumbLarge.png}}

By \href{https://www.nytimes.com/by/charles-m-blow}{Charles M. Blow}

Opinion Columnist

\begin{itemize}
\item
  June 14, 2020
\item
  \begin{itemize}
  \item
  \item
  \item
  \item
  \item
  \item
  \end{itemize}
\end{itemize}

In the wake of the killing of George Floyd and the massive wave of
protests that have swept the country and the world, New York State on
Friday passed a package of policing reforms, banning chokeholds and
opening police disciplinary records, among other things.

After signing the bills into law, Gov. Andrew M. Cuomo said at one of
his coronavirus news conferences: ``You don't need to protest, you won.
You accomplished your goal. Society says you're right, the police need
systemic reform.''

Cuomo's statement betrays a fundamental misunderstanding of this moment.

Yes, the package of bills he signed, and the steps being taken in other
cities and states, represent movement in the right direction on the
issue of policing, but people aren't only in the streets because of a
single killing or a single issue.

People are marching as a way of screaming, a way of exhaling pain, as an
enormous group catharsis.

This isn't only about the pain of police brutality, it's about
\emph{all} the pain. This is about all the injustice and disrespect and
oppression. This is about ancestry and progeny.

In fact, with every word of solidarity, with every overture by
governments and companies, with every new law passed or reform
instituted, the cry draws strength, because these actions are all
acknowledgment that those in pain have been right all along, that all of
their heretofore unheard and unheeded protestations had been wrongfully
ignored.

People are in the streets because their backs have too long borne the
weight of racism, or because for too long they have averted their eyes
from it.

Black people are saying: ``See me! See what you have done to me and
continue to do to me. Stand naked in your sin, and stare, unflinching,
at your reflection. You did this.''

They are saying, ``Stop killing us!''

And in that, they mean killing in every conceivable way.

Stop underfunding schools and overfunding police. Stop anti-black bias
in all fields, from medicine to employment to entertainment. Stop using
911 calls as a deadly weapon. America, just stop.

And, contrary to what Cuomo might have thought, his package of bills
represents a win in only one battle in a larger war.

It took centuries for America to hone its instruments of oppression.
Every time part of it fell, it simply re-emerged in a more elegant form.

After slavery was abolished, the black codes were instituted, keeping
many restrictions that slavery had enforced and guaranteeing that black
people would continue to exist as a cheap source of labor.

After Reconstruction was allowed to fail, Southern states rushed to
rewrite their constitutions to institute and codify white supremacy,
ushering in Jim Crow.

For instance, at the
\href{https://www.demos.org/blog/overcoming-white-supremacy-louisiana}{Louisiana
constitutional convention} in 1898, Thomas J. Semmes stated that the
``mission'' of the delegates had been ``to establish the supremacy of
the white race in this state.'' In his closing remarks, E.B.
Kruttschnitt, the president of that convention, bemoaned that the
delegates had been constrained by the 15th Amendment such that they
could not provide ``universal white manhood suffrage and the exclusion
from the suffrage of every man with a trace of African blood in his
veins.''

He went on to proclaim:

``I say to you, that we can appeal to the conscience of the nation, both
judicial and legislative, and I don't believe that they will take the
responsibility of striking down the system which we have reared in order
to protect the purity of the ballot box and to perpetuate the supremacy
of the Anglo--Saxon race in Louisiana.''

This wretched language repeated itself, in some form, at other
conventions.

Terror became a tool to keep black people underfoot.
\href{https://www.vox.com/identities/2017/8/15/16153220/trump-confederate-statues}{Confederate
monuments}sprang up everywhere, the Ku Klux Klan flourished and
lynchings surged.

After the Civil Rights Act of 1964,
\href{https://www.vox.com/2015/7/13/8913297/mass-incarceration-maps-charts}{mass
incarceration began its climb}, accomplishing many of the same things
Jim Crow did before --- voter disenfranchisement, employment and housing
restrictions, and just overall punishment and disrespect.

Racial oppression is infinitely transmutable.

So people are in the streets because they are tired of chopping heads
off the hydra. They are tired of fighting this oppression only to see it
spring right back, or multiply.

It is exhausting and infuriating and maddening to be forced to fight,
always, for what for others is free. It enrages, when you realize that
you're still fighting the same fight that your parents fought, and that
their parents fought.

It is an everyday struggle to neither fall into despair nor explode in
anger.

So, these people are in the streets, having their moment and having
their say. And America would do well to listen and not try to silence
them or soothe them.

In fact, America listening and responding to these protests, respecting
them, is one of the healthiest things the country can do, because as
protester Kimberly Latrice Jones said at the end of
\href{https://www.instagram.com/tv/CA5gksAgvxJ/?utm_source=ig_embed}{her
viral video}, ``They are lucky that what black people are looking for is
equality and not revenge.''

\emph{The Times is committed to publishing}
\href{https://www.nytimes.com/2019/01/31/opinion/letters/letters-to-editor-new-york-times-women.html}{\emph{a
diversity of letters}} \emph{to the editor. We'd like to hear what you
think about this or any of our articles. Here are some}
\href{https://help.nytimes.com/hc/en-us/articles/115014925288-How-to-submit-a-letter-to-the-editor}{\emph{tips}}\emph{.
And here's our email:}
\href{mailto:letters@nytimes.com}{\emph{letters@nytimes.com}}\emph{.}

\emph{Follow The New York Times Opinion section on}
\href{https://www.facebook.com/nytopinion}{\emph{Facebook}} \emph{and}
\href{http://twitter.com/NYTOpinion}{\emph{Twitter
(@NYTopinion)}}\emph{, and}
\href{https://www.instagram.com/nytopinion/}{\emph{Instagram}}\emph{.}

Advertisement

\protect\hyperlink{after-bottom}{Continue reading the main story}

\hypertarget{site-index}{%
\subsection{Site Index}\label{site-index}}

\hypertarget{site-information-navigation}{%
\subsection{Site Information
Navigation}\label{site-information-navigation}}

\begin{itemize}
\tightlist
\item
  \href{https://help.nytimes.com/hc/en-us/articles/115014792127-Copyright-notice}{©~2020~The
  New York Times Company}
\end{itemize}

\begin{itemize}
\tightlist
\item
  \href{https://www.nytco.com/}{NYTCo}
\item
  \href{https://help.nytimes.com/hc/en-us/articles/115015385887-Contact-Us}{Contact
  Us}
\item
  \href{https://www.nytco.com/careers/}{Work with us}
\item
  \href{https://nytmediakit.com/}{Advertise}
\item
  \href{http://www.tbrandstudio.com/}{T Brand Studio}
\item
  \href{https://www.nytimes.com/privacy/cookie-policy\#how-do-i-manage-trackers}{Your
  Ad Choices}
\item
  \href{https://www.nytimes.com/privacy}{Privacy}
\item
  \href{https://help.nytimes.com/hc/en-us/articles/115014893428-Terms-of-service}{Terms
  of Service}
\item
  \href{https://help.nytimes.com/hc/en-us/articles/115014893968-Terms-of-sale}{Terms
  of Sale}
\item
  \href{https://spiderbites.nytimes.com}{Site Map}
\item
  \href{https://help.nytimes.com/hc/en-us}{Help}
\item
  \href{https://www.nytimes.com/subscription?campaignId=37WXW}{Subscriptions}
\end{itemize}
