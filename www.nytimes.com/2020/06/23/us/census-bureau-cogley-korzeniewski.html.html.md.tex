Sections

SEARCH

\protect\hyperlink{site-content}{Skip to
content}\protect\hyperlink{site-index}{Skip to site index}

\href{https://www.nytimes.com/section/us}{U.S.}

\href{https://myaccount.nytimes.com/auth/login?response_type=cookie\&client_id=vi}{}

\href{https://www.nytimes.com/section/todayspaper}{Today's Paper}

\href{/section/us}{U.S.}\textbar{}Census Bureau Adds Top-Level Political
Posts, Raising Fears for 2020 Count

\url{https://nyti.ms/2Va02Jd}

\begin{itemize}
\item
\item
\item
\item
\item
\end{itemize}

Advertisement

\protect\hyperlink{after-top}{Continue reading the main story}

Supported by

\protect\hyperlink{after-sponsor}{Continue reading the main story}

\hypertarget{census-bureau-adds-top-level-political-posts-raising-fears-for-2020-count}{%
\section{Census Bureau Adds Top-Level Political Posts, Raising Fears for
2020
Count}\label{census-bureau-adds-top-level-political-posts-raising-fears-for-2020-count}}

The census, which is constitutionally mandated to count every person in
the country, has traditionally been carried out in a rigidly nonpartisan
fashion.

\includegraphics{https://static01.nyt.com/images/2020/07/22/us/22census/merlin_170753940_88eb31f9-ce4f-48d2-bdfb-92e5ef59c515-articleLarge.jpg?quality=75\&auto=webp\&disable=upscale}

By \href{https://www.nytimes.com/by/michael-wines}{Michael Wines}

\begin{itemize}
\item
  Published June 23, 2020Updated July 28, 2020
\item
  \begin{itemize}
  \item
  \item
  \item
  \item
  \item
  \end{itemize}
\end{itemize}

WASHINGTON --- The
\href{https://www.nytimes.com/2020/07/28/us/trump-census.html}{Census
Bureau} said on Tuesday that it had created two new top-level positions
and filled them with political appointees from outside the agency, an
unprecedented move that revived concerns the national population count
has turned increasingly partisan.

The census, which is constitutionally mandated to count every person in
the country every decade, has traditionally been carried out in a
rigidly nonpartisan fashion.

But critics fear that the appointments are the latest sign that the
census, which is used to apportion federal dollars and political
representation, has become increasingly politicized --- and a way for
Republicans to bend census results to advance their electoral interests.

Tuesday's announcement comes almost one year after the Supreme Court
ruled that the administration could not ask census respondents whether
they were American citizens. This ended a bitter legal battle over
charges that Republicans were trying to deter immigrants, ethnic
minorities and others who tend to vote Democratic from responding to the
survey.

Until now, only the director of the Census Bureau, its congressional
liaison and its spokesperson have been political appointees. And for
decades, the agency's directors and top managers have been career
statisticians, economists and survey methodologists --- sometimes
eminent ones.

But neither appointee announced on Tuesday appears to have the extensive
experience in census issues or administration that is traditional for
such senior roles at the bureau.

\href{https://www.census.gov/newsroom/press-releases/2020/statement-new-staff.html}{In
a news release}, the bureau's director, Steven Dillingham, said
Nathaniel T. Cogley, a professor who heads the government department at
a Texas university, would take a new position as deputy director for
policy.

Mr. Cogley, who received a Ph.D. in political science from Yale
University in 2013, is an assistant professor at Tarleton State
University in Stephenville, Texas.
\href{http://nathanielcogley.com/Nathaniel-Terence-Cogley-CV.pdf}{His
résumé} lists dozens of appearances on television and radio programs as
a commentator on political issues, as well as opinion pieces in which he
has criticized technical aspects of the Democratic House's impeachment
case against President Trump.

Mr. Dillingham also said Mr. Cogley's senior adviser would be Adam
Korzeniewski, described on a LinkedIn page bearing his name as a former
political consultant for Republican candidates who most recently worked
for five months in a Census Bureau field job in New York.

Mr. Korzeniewski earned a bachelor's degree in statistics and computer
science in 2017 from Columbia University, according to the LinkedIn
page, which appears to have been taken down. Before then, according to
the profile, Mr. Korzeniewski spent five years in the Marines, including
a stretch in Afghanistan.

The two men have been working since April as advisers to a deputy of
Commerce Secretary Wilbur L. Ross Jr., who oversees the Census Bureau.
The news release said they would ``help the bureau achieve a complete
and accurate 2020 census and study future improvements'' in technology
and data collection.

``The importance of more and better data for decision making will
continue as the heart of the Census Bureau mission,'' the release
stated.

The announcement quickly drew sharp criticism from outside experts and
groups that have been pressing for a complete and nonpartisan census.
And it comes on the heels of recent White House moves to fire the heads
of two other traditionally nonpartisan federal agencies ---
\href{https://www.nytimes.com/2020/05/07/us/politics/postmaster-general-louis-dejoy.html}{the
Post Office} and
\href{https://www.nytimes.com/2020/06/15/us/politics/voice-of-american-resignations.html}{Voice
of America} --- and replace them with Trump loyalists.

A veteran private consultant on census issues for business and nonprofit
groups, Terri Ann Lowenthal, called the Census Bureau appointments
``deeply disturbing.''

``Their proximity to the director and lack of relevant expertise suggest
a thinly veiled effort to interfere in the implementation and outcome of
the 2020 census for the administration's benefit,'' said Ms. Lowenthal,
who oversaw a review of federal statistical operations for President
Barack Obama's 2008 transition team. ``It's hard to draw any other
conclusion.''

Representative Carolyn B. Maloney of New York, the Democratic chair of a
House committee overseeing the bureau, called the appointees ``political
operatives'' chosen by the Trump administration and accused officials of
``using the census for political gain.''

Kenneth Prewitt, a Columbia University professor who headed the Census
Bureau during the 2000 count, called the appointments ``a frightening
development.''

``Two decades ago, I said it was impossible for the White House to
manipulate data in such a way as to affect the distribution of seats in
Congress,'' he said. ``I was wrong. If this plays out as we fear, this
would be a partisan use of the census that is unprecedented.''

The bureau did not immediately respond to requests for comment on the
reaction or for interviews with the two appointees.

Last year the administration
\href{https://www.nytimes.com/2019/07/02/us/trump-census-citizenship-question.html}{lost
a pitched legal battle} over the census that centered on charges that
the White House was trying to rig the population count to benefit
Republicans.

In that fight, Mr. Ross sought to add a question on citizenship to the
census questionnaire, saying it was needed to better enforce civil
rights laws. The Supreme Court rejected that explanation as not
credible, and much of the evidence suggested that his
\href{https://www.nytimes.com/2019/05/30/us/census-citizenship-question-hofeller.html}{real
motive was to discourage racial and ethnic minorities} from filling out
census forms.

The resulting population totals would have depicted an older, whiter
America than actually exists, boosting the power of the Republican
Party's core constituency when census totals are used to draw new
political boundaries next year.

Some veteran Census Bureau officials are increasingly worried that the
new appointees will seek to skew the 2020 census totals in a similarly
inaccurate way, accomplishing what the battle over the citizenship
question failed to achieve.

Since arriving at the Commerce Department in April, Mr. Cogley and Mr.
Korzeniewski have met with several Census Bureau officials to discuss
the agency's operations. According to one senior census official who
spoke on condition of anonymity for fear of retribution, they have
repeatedly questioned the need for census operations that focus on
accurately counting the nation's hardest-to-reach residents.

Those so-called hard-to-count populations --- overwhelmingly minorities
and lower-income residents --- are the last crucial segment that the
2020 census has yet to reach, the people who have not responded to
repeated requests to turn in their census forms.

In a normal census, the bureau dispatches an army of field workers
during the summer to knock on doors and count those populations in
person. But the coronavirus pandemic has halted those efforts for
months, and the bureau only now is starting to gear up for that part of
the count.

Sheelagh McNeill contributed research

Advertisement

\protect\hyperlink{after-bottom}{Continue reading the main story}

\hypertarget{site-index}{%
\subsection{Site Index}\label{site-index}}

\hypertarget{site-information-navigation}{%
\subsection{Site Information
Navigation}\label{site-information-navigation}}

\begin{itemize}
\tightlist
\item
  \href{https://help.nytimes.com/hc/en-us/articles/115014792127-Copyright-notice}{©~2020~The
  New York Times Company}
\end{itemize}

\begin{itemize}
\tightlist
\item
  \href{https://www.nytco.com/}{NYTCo}
\item
  \href{https://help.nytimes.com/hc/en-us/articles/115015385887-Contact-Us}{Contact
  Us}
\item
  \href{https://www.nytco.com/careers/}{Work with us}
\item
  \href{https://nytmediakit.com/}{Advertise}
\item
  \href{http://www.tbrandstudio.com/}{T Brand Studio}
\item
  \href{https://www.nytimes.com/privacy/cookie-policy\#how-do-i-manage-trackers}{Your
  Ad Choices}
\item
  \href{https://www.nytimes.com/privacy}{Privacy}
\item
  \href{https://help.nytimes.com/hc/en-us/articles/115014893428-Terms-of-service}{Terms
  of Service}
\item
  \href{https://help.nytimes.com/hc/en-us/articles/115014893968-Terms-of-sale}{Terms
  of Sale}
\item
  \href{https://spiderbites.nytimes.com}{Site Map}
\item
  \href{https://help.nytimes.com/hc/en-us}{Help}
\item
  \href{https://www.nytimes.com/subscription?campaignId=37WXW}{Subscriptions}
\end{itemize}
