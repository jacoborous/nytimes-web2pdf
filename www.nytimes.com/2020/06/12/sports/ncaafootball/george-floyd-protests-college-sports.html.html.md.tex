Sections

SEARCH

\protect\hyperlink{site-content}{Skip to
content}\protect\hyperlink{site-index}{Skip to site index}

\href{https://www.nytimes.com/section/sports/ncaafootball}{College
Football}

\href{https://myaccount.nytimes.com/auth/login?response_type=cookie\&client_id=vi}{}

\href{https://www.nytimes.com/section/todayspaper}{Today's Paper}

\href{/section/sports/ncaafootball}{College Football}\textbar{}College
Athletes, Phones in Hand, Force Shift in Protest Movement

\url{https://nyti.ms/37ld8YU}

\begin{itemize}
\item
\item
\item
\item
\item
\end{itemize}

\href{https://www.nytimes.com/news-event/george-floyd-protests-minneapolis-new-york-los-angeles?action=click\&pgtype=Article\&state=default\&region=TOP_BANNER\&context=storylines_menu}{Race
and America}

\begin{itemize}
\tightlist
\item
  \href{https://www.nytimes.com/2020/07/26/us/protests-portland-seattle-trump.html?action=click\&pgtype=Article\&state=default\&region=TOP_BANNER\&context=storylines_menu}{Protesters
  Return to Other Cities}
\item
  \href{https://www.nytimes.com/2020/07/24/us/portland-oregon-protests-white-race.html?action=click\&pgtype=Article\&state=default\&region=TOP_BANNER\&context=storylines_menu}{Portland
  at the Center}
\item
  \href{https://www.nytimes.com/2020/07/23/podcasts/the-daily/portland-protests.html?action=click\&pgtype=Article\&state=default\&region=TOP_BANNER\&context=storylines_menu}{Podcast:
  Showdown in Portland}
\item
  \href{https://www.nytimes.com/interactive/2020/07/16/us/black-lives-matter-protests-louisville-breonna-taylor.html?action=click\&pgtype=Article\&state=default\&region=TOP_BANNER\&context=storylines_menu}{45
  Days in Louisville}
\end{itemize}

Advertisement

\protect\hyperlink{after-top}{Continue reading the main story}

Supported by

\protect\hyperlink{after-sponsor}{Continue reading the main story}

\hypertarget{college-athletes-phones-in-hand-force-shift-in-protest-movement}{%
\section{College Athletes, Phones in Hand, Force Shift in Protest
Movement}\label{college-athletes-phones-in-hand-force-shift-in-protest-movement}}

``Things that are important to African-Americans in a sports setting get
labeled a distraction. That's a painful word when you're talking about
matters of life and death.''

\includegraphics{https://static01.nyt.com/images/2020/06/13/sports/00unrest-collegefootball1-print/merlin_161231535_319499fc-b80d-4ff6-89a4-966df4586dfc-articleLarge.jpg?quality=75\&auto=webp\&disable=upscale}

\href{https://www.nytimes.com/by/alan-blinder}{\includegraphics{https://static01.nyt.com/images/2018/08/24/multimedia/author-alan-blinder/author-alan-blinder-thumbLarge.png}}\href{https://www.nytimes.com/by/billy-witz}{\includegraphics{https://static01.nyt.com/images/2018/02/16/multimedia/author-billy-witz/author-billy-witz-thumbLarge.jpg}}

By \href{https://www.nytimes.com/by/alan-blinder}{Alan Blinder} and
\href{https://www.nytimes.com/by/billy-witz}{Billy Witz}

\begin{itemize}
\item
  June 12, 2020
\item
  \begin{itemize}
  \item
  \item
  \item
  \item
  \item
  \end{itemize}
\end{itemize}

They knelt on campuses and outside courthouses and a capitol. They
filmed videos and challenged coaches and gripped megaphones to call out
racism they knew from their classrooms and stadiums. They led protest
chants, registered voters and started to strategize for Nov. 3, Election
Day.

In some instances, the nation's college athletes even pledged not to
play.

Until recently, before
\href{https://www.nytimes.com/news-event/george-floyd-protests-minneapolis-new-york-los-angeles}{the
death of a black man, George Floyd}, in police custody in Minneapolis,
many university administrators and coaches would have instinctively
sought to silence college athletes' public expressions of racial furor,
pain or politics. But over a matter of weeks, players and coaches have
seized their influence for a display of political action that historians
and executives say recalls the 1960s, another era when people took to
the streets to protest racial inequality.

``There are a lot of things we're not going to stand for anymore,''
Marvin Wilson, a defensive tackle at Florida State, said in an interview
this week. ``People are starting to realize we have a say-so in how this
country should be run.''

The gravity of the national moment emboldened the players, especially
because it was a challenge to a justice system that many believed stood
poised to oppress them or their black teammates when they were away from
the field. Often cheered by their vast followings on social media, they
also drew motivation from a long-simmering debate that has recently
driven student-athletes
\href{https://www.murphy.senate.gov/download/madness-inc}{to question
their place in a \$14 billion industry} and consider
\href{https://www.nytimes.com/2020/04/29/sports/ncaabasketball/ncaa-athlete-endorsements.html}{whether
they deserve to profit off their fame and talents}.

In a shift that could alter the relationship between college activism
and athletics, universities suddenly became willing to lend the power of
their sports brands to social causes. And when players felt their
schools had fallen short, university leaders found themselves challenged
behind closed doors and, tellingly, sometimes in public.

``It's where we are as a society that there's no room to quiet the
voices or stifle them,'' said Mike Locksley, the football coach at
Maryland, where in 2014 university officials squashed plans by the team
to wear shirts publicly opposing police brutality after Eric Garner's
\href{https://www.nytimes.com/2019/05/12/nyregion/eric-garner-death-daniel-pantaleo-chokehold.html}{gasped
words} --- ``I can't breathe'' --- became a protest cry.

On Friday, student-athletes at Texas, which was supportive of its
students speaking out after Floyd's death, released
\href{https://twitter.com/_BrennanEagles_/status/1271518098248667139}{a
list of requests} to the university, including that it rename certain
buildings and replace the school song, ``The Eyes of Texas,'' with a new
tune lacking ``racist undertones.'' The athletes said that without an
``official commitment'' from the university, they would not assist in
recruitment or donor events.

\includegraphics{https://static01.nyt.com/images/2020/06/13/sports/11unrest-collegefootball2-print/merlin_172866267_7f4890c5-ebb8-4c34-ac4a-9c33cb1a5087-articleLarge.jpg?quality=75\&auto=webp\&disable=upscale}

But the current effort gained one of its earliest and firmest footholds
at Florida State.

When Mike Norvell, the new football coach, told a reporter
\href{https://twitter.com/tashanreed/status/1267967609007267841}{that he
had spoken individually with each of his players} in the wake of Floyd's
death, it sounded empathetic.

But Florida State's players knew Norvell had merely sent out a mass text
message and followed up with only some players. Quickly, teammates
pinged messages across a players-only chat to hatch a response.

There would be no call to meet with Norvell or complain to a senior
administrator. Instead, the players went to social media and lined up
behind Wilson, their senior captain.
\href{https://twitter.com/marvinwilson21/status/1268395525495193601}{In
a blistering tweet, he called Norvell's statement a lie} and said his
teammates would stop voluntary workouts.

The coach apologized at a team meeting hours later. The players pledged
to register to vote and devote 10 hours each to community service, and
they made plans to raise money for an African-American college
scholarship fund and to help lower income students near the university
in Tallahassee, Fla.

``We used that to get everybody at the university and Coach Norvell's
attention,'' said Wilson, who is projected as a top N.F.L. prospect.
``And then we handled things behind closed doors.''

Social media has provided a megaphone for athletes, who understand its
potential power.

A video Wilson posted after the meeting has been viewed more than
400,000 times. Former gymnasts at Alabama and Auburn, and former
football players at Clemson, Iowa and Utah have used Twitter to describe
racist behavior in their programs, leading to public apologies and
several coaches being placed on leave.

A post by Ashlynn Dunbar, a two-sport athlete at Oklahoma, telling fans
not to support her on the court if they don't support her right to speak
out drew more than 40,000 likes.

``It's so powerful to be able to speak our minds and have people
actually hear us,'' said Dunbar, an all-Big 12 volleyball player who
will play basketball next season while she finishes her master's degree
in college athletic administration. ``I don't have the biggest
following, but the ability for people to share it with their friends who
share it with their friends has allowed my voice to be heard farther
than just the people I know.''

Sometimes, powerful conversations are taking place in private.

When Dabo Swinney, Clemson's football coach, invited his 16 seniors to
his home recently, he was asked to explain why he did not discipline an
assistant coach when he learned several years ago that he had used a
racial slur in conversation with a player.

Darien Rencher, a senior running back, said that while Swinney told them
his assistant was wrong, the timing made for ``a sticky situation'' ---
one that involved a familiar dynamic in college football, a sport in
which more than 80 percent of Football Bowl Subdivision coaches in the
2018 season were white though nearly half of the players were black.

Rencher, who described the conversation as a family talk, said topics
``that are kind of touchy were able to be talked about behind the scenes
that gave us all perspective.''

The door had been cracked, at least a bit, by players of the past.

Members of Howard's 1936 football team staged a strike over a lack of
food. There have been demonstrations over living conditions, the Vietnam
War and civil rights. In one episode in 1969, Wyoming's coach dismissed
14 black athletes who wanted to demonstrate during a game against
Brigham Young.

In February 2019, just a few years after
\href{https://www.nytimes.com/2015/11/09/us/missouri-football-players-boycott-in-protest-of-university-president.html}{Missouri
football players helped force the resignation of their university
president}, many Mississippi basketball players knelt during the
national anthem to protest a campus demonstration that included white
nationalists.

Image

In 2019, six Mississippi basketball players --- eventually eight ---
knelt during the national anthem to protest a campus demonstration that
included white nationalists.Credit...Nathanael Gabler/The Oxford Eagle,
via Associated Press

``It's sporadic, but when it happens, it can be widespread and have a
major impact,'' said Lane Demas, a history professor at Central Michigan
and the author of ``Integrating the Gridiron: Black Civil Rights and
American College Football.'' ``Today, more than ever, players understand
the power they have.''

Few appreciate that shift more than Ramogi Huma, the executive director
of the National College Players Association, the closest thing that
college players have to a formal advocate.

Huma was a senior linebacker at U.C.L.A. in 1998 when the
African-American players on the unbeaten football team met with the
school's Black Student Union, which brought a powerful speaker: John
Carlos, whose black-gloved, raised-fist salute with Tommie Smith at the
1968 Olympics remains an iconic moment of the civil rights era.

The organization asked the players to use their platform of a nationally
televised game to protest Proposition 209, a ballot measure that had
amended California's Constitution to prohibit universities from
considering race or gender in admissions, which caused black enrollment
to plummet at U.C.L.A. The players settled on wearing black wristbands.

When Bob Toledo, U.C.L.A.'s coach, got wind of the protest planned for
the final game of the regular season at Miami, he argued that the
players should stand down, according to Huma, and found a reason that
struck a nerve: that Southern sportswriters might not vote the Bruins
into the national title game.

The issue was so contentious that the team canceled its usual meetings
the night before the game to instead discuss the protest. The Bruins
nixed their plans and then lost, ruining their shot at a national
championship.

Coaches, Huma said, blamed the protest idea and, he added, one white
player shouted across the locker room, ``I don't want to hear a damn
thing about those wristbands anymore.''

``It was similar to the reaction to Colin Kaepernick when it was very
clear he was trying to raise the issue of police brutality,'' Huma said
of the rift. ``Things that are important to African-Americans in a
sports setting get labeled a distraction. That's a painful word when
you're talking about matters of life and death.''

Now, at least in many cases, universities are intertwining themselves
institutionally with the wave of unrest. Ohio State, with one of
America's most influential and lucrative athletic departments, aligned
itself with Black Lives Matter and offered resources that universities
\href{https://www.nytimes.com/2019/10/24/sports/ncaafootball/heisman-trophy-wisconsin-jonathan-taylor.html}{tend
to reserve for Heisman Trophy campaigns}, not political ones: Twitter
accounts with hundreds of thousands of followers, video production
wizardry, publicists and long-cultivated prestige.

``We can't walk away because we know that people will disagree with
us,'' Gene Smith, Ohio State's athletic director, once one of two black
men running a Football Bowl Subdivision program, said. ``If it's right,
then we stand tall, and we accept whatever negatives come with it.''

With
\href{https://www.nytimes.com/interactive/2020/06/10/upshot/black-lives-matter-attitudes.html}{polls
showing rising support for Black Lives Matter}, the risks of a public
stand are not what they were even a month ago. Locksley, Maryland's
football coach, who was an assistant when players were kept from
protesting in 2014, said a dispiriting message emerged when his team
convened to talk about Floyd: ``It keeps happening, Coach.''

This time, there was no suppression reminiscent of 2014, not after it
had taken Locksley, one of the few black men leading a Power Five
football program,
\href{https://twitter.com/CoachLocks/status/1266761658334097408}{nearly
a week to write his own response}. Instead, the team broke into groups
and
\href{https://twitter.com/TerpsFootball/status/1267905585674158081}{crafted
a statement} that forcefully declared its torment and, just as
important, a plan of action.

``Many of our teammates,'' the statement said, ``are inconsolable as yet
another Black life has been taken at the hands of law enforcement and
injustice.'' They announced a push to promote voter registration and to
take people to the polls in November.

In wanting action --- and not just words ``where it's one time, we put
it out and forget it,'' as Locksley said --- the Maryland players are
not unique. Some, like Florida State's Wilson and his teammates, have
grand ambitions in their community. Others, like athletes at Wisconsin,
asked administrators and coaches to review hiring practices and provided
reading and movie lists on topics of racial injustice.

The most elementary step gaining traction has been simple: to vote.

When Georgia Tech's men's basketball team gathered over Zoom, each
participant was asked to express his emotions in two words. Players
described being ``frustrated,'' ``angry'' and ``tired.'' The two words
for Eric Reveno, a 54-year-old white assistant coach, were
``embarrassed'' and ``disgusted.''

That night, Reveno stewed on something that Malachi Rice, one of the
team's leaders, had said: that too many people protest injustice but do
not bother to vote.

Reveno woke up the next morning with an idea: that the N.C.A.A. should
ban athletic activities on Election Day to encourage its more than
460,000 athletes to vote.

No practice, no meetings, no games.

In response, the N.C.A.A.'s powerful Board of Governors said Friday
afternoon that it was encouraging its member schools to make Nov. 3 a
day off for athletic activities.

``We commend N.C.A.A. student-athletes who recognized the need for
change and took action though safe and peaceful protest,'' the board
\href{https://www.ncaa.org/about/resources/media-center/ncaa-board-governors-statement-social-activism}{said
in a statement}.

Before the association's statement, U.C.L.A. had announced voter
education sessions for all 25 of its teams, and Georgia Tech said nine
of its in-season teams, including football, would not hold mandatory
activities on Election Day.

Reveno said support with words was ``not enough.''

``We teach them financial literacy --- the power of interest over time,
the dangers of credit card debt, about how much that daily latte is
costing them,'' Reveno said. ``What about investing in your community so
that your kid's life is shaped more the way you wanted? Being an engaged
and active citizen is the most powerful thing we do as an American.''

Increasingly, the nation's college athletes are acting on this.

Image

Sanaá Dotson, a volleyball player at Oklahoma, with her message about
athletes speaking on political and social issues.

Image

Some, like Sanaá Dotson, a volleyball player at Oklahoma, sense an
awakening. She came of age
\href{https://www.nytimes.com/2020/06/05/sports/football/george-floyd-kaepernick-kneeling-nfl-protests.html}{when
Kaepernick could not find an N.F.L. job} and
\href{https://www.nytimes.com/2018/02/16/sports/basketball/lebron-laura-ingraham.html}{LeBron
James was told to shut up and dribble}.

``It sent a message that people don't really care about what we have to
say,'' she said. ``They just care about what we do on the court.
Obviously, that hurts --- especially as a young black person trying to
find their identity.''

But there is a shift, she said, something that she sensed profoundly
when she attended a march in her hometown, Houston, where Floyd's family
members spoke. There is an opportunity that did not previously exist,
she said. It comes not from a louder voice, but from an audience that is
ready to listen.

``It's different speaking about things,'' Dotson said. ``And then
wanting to be heard.''

Advertisement

\protect\hyperlink{after-bottom}{Continue reading the main story}

\hypertarget{site-index}{%
\subsection{Site Index}\label{site-index}}

\hypertarget{site-information-navigation}{%
\subsection{Site Information
Navigation}\label{site-information-navigation}}

\begin{itemize}
\tightlist
\item
  \href{https://help.nytimes.com/hc/en-us/articles/115014792127-Copyright-notice}{©~2020~The
  New York Times Company}
\end{itemize}

\begin{itemize}
\tightlist
\item
  \href{https://www.nytco.com/}{NYTCo}
\item
  \href{https://help.nytimes.com/hc/en-us/articles/115015385887-Contact-Us}{Contact
  Us}
\item
  \href{https://www.nytco.com/careers/}{Work with us}
\item
  \href{https://nytmediakit.com/}{Advertise}
\item
  \href{http://www.tbrandstudio.com/}{T Brand Studio}
\item
  \href{https://www.nytimes.com/privacy/cookie-policy\#how-do-i-manage-trackers}{Your
  Ad Choices}
\item
  \href{https://www.nytimes.com/privacy}{Privacy}
\item
  \href{https://help.nytimes.com/hc/en-us/articles/115014893428-Terms-of-service}{Terms
  of Service}
\item
  \href{https://help.nytimes.com/hc/en-us/articles/115014893968-Terms-of-sale}{Terms
  of Sale}
\item
  \href{https://spiderbites.nytimes.com}{Site Map}
\item
  \href{https://help.nytimes.com/hc/en-us}{Help}
\item
  \href{https://www.nytimes.com/subscription?campaignId=37WXW}{Subscriptions}
\end{itemize}
