Sections

SEARCH

\protect\hyperlink{site-content}{Skip to
content}\protect\hyperlink{site-index}{Skip to site index}

\href{/section/opinion}{Opinion}\textbar{}A Statue Was Toppled. Can We
Finally Talk About the British Empire?

\url{https://nyti.ms/30BiUV0}

\begin{itemize}
\item
\item
\item
\item
\item
\end{itemize}

\includegraphics{https://static01.nyt.com/images/2020/06/12/opinion/12Bhambra1/12Bhambra1-articleLarge.jpg?quality=75\&auto=webp\&disable=upscale}

\href{/section/opinion}{Opinion}

\hypertarget{a-statue-was-toppled-can-we-finally-talk-about-the-british-empire}{%
\section{A Statue Was Toppled. Can We Finally Talk About the British
Empire?}\label{a-statue-was-toppled-can-we-finally-talk-about-the-british-empire}}

The removal of a slave trader's statue in England begins a conversation
about how we are shaped by our past and how it configures the present.

The statue of the slave trader Edward Colston falling into the water on
Sunday after protesters in Bristol, England, pulled it
down.Credit...Keir Gravil, via Reuters

Supported by

\protect\hyperlink{after-sponsor}{Continue reading the main story}

By Gurminder K Bhambra

Dr. Bhambra is a professor of postcolonial studies at the University of
Sussex.

\begin{itemize}
\item
  June 12, 2020
\item
  \begin{itemize}
  \item
  \item
  \item
  \item
  \item
  \end{itemize}
\end{itemize}

BRIGHTON, England --- Tens of thousands of people protested in British
cities in solidarity with those rising up against police brutality
against black Americans in the past week. They highlighted similar
\href{https://www.theguardian.com/commentisfree/2020/jun/09/protests-british-history}{injustices
in Britain}. Protesters in the city of Bristol drew connections between
a white police officer's killing of George Floyd, a black man in
Minneapolis, and the histories of colonialism and the slave trade. On
Sunday, they
\href{https://www.theguardian.com/uk-news/2020/jun/07/blm-protesters-topple-statue-of-bristol-slave-trader-edward-colston}{toppled
the statue of Edward Colston}, a 17th-century slave trader, trampled
over it and rolled it into Bristol Harbor.

Between 1672 and 1689, Colston's Royal African Company shipped about
100,000 enslaved people from West Africa to the Americas and the
Caribbean, branding them on their chests with his corporation's acronym,
RAC. Disease and dehydration killed more than 20,000 people taken onto
those ships by Colston's company, and their bodies were thrown into the
ocean. Yet Colston's bronze statue, which was erected in 1895 in
Bristol, was engraved with the inscription `` \ldots{} one of the most
virtuous and wise sons'' of the city.

Toppling statues is one way of unsettling accounts of the past that fail
to acknowledge the
\href{https://www.theguardian.com/news/2017/nov/10/how-colonial-violence-came-home-the-ugly-truth-of-the-first-world-war}{broader
truths of the British Empire}. Attempts had been made through petitions
and letters and engagements with local authorities to change the
inscription and to reconsider the names of civic and public institutions
that continued to honor him. But to no avail. This week, people of all
backgrounds joined together to highlight the multiple injustices
embodied in the statue and took matters into their own hands.

The glorification of the British Empire despite its histories of
\href{https://lareviewofbooks.org/article/blighted-by-empire-what-the-british-did-to-india/}{colonization,
plunder} and
\href{http://www.nationalarchives.gov.uk/slavery/pdf/britain-and-the-trade.pdf}{enslavement}is
evident in the plethora of statues to its architects. The toppling of
Colston's statue begins a conversation about how we are shaped by our
past and that we are accountable for how it configures the present.

Dispossession, appropriation,
\href{https://www.youtube.com/watch?v=3NXC4Q_4JVg}{elimination and
enslavement} were central to the British Empire and to the making of
modern Britain. Its initial expansion westward into the territories of
the Americas was followed by
\href{https://www.nytimes.com/2019/09/04/opinion/east-india-company.html}{commercial
and colonial initiatives} in the East. This was compounded by Britain's
involvement in the Europe-wide trade in human beings from Africa and
\href{https://www.striking-women.org/module/map-major-south-asian-migration-flows/indentured-labour-south-asia-1834-1917}{circuits
of indentured labor} from Asia.

These histories rarely make it into the standard narratives of how
Britain came to be. Instead, there is either
\href{https://www.independent.co.uk/news/uk/politics/british-people-are-proud-of-colonialism-and-the-british-empire-poll-finds-a6821206.html}{a
glorification of the empire} or amnesiac histories that either ignore it
or consider it benign. The end of the empire is similarly elided.

\includegraphics{https://static01.nyt.com/images/2020/06/12/opinion/12Bhambra2/12Bhambra2-articleLarge.jpg?quality=75\&auto=webp\&disable=upscale}

The limited public understanding of the empire, through education,
popular histories and
\href{https://www.radiotimes.com/travel/2018-04-03/michael-portillo-rides-the-rails-in-india-for-once-i-felt-quite-understated-in-my-attire/}{television
shows}, is largely an exercise in forgetting or celebration. There is no
requirement to teach British students about it. Few Britons have an
adequate understanding of the
\href{https://www.nytimes.com/2017/08/02/opinion/dunkirk-indians-world-war.html}{histories
that produced Britain} or why the unquestioning glorification of some
aspects of that history is
\href{https://www.independent.co.uk/voices/poll-shows-brits-are-proud-of-colonialism-clearly-they-havent-heard-of-these-colonial-crimes-a6823151.html}{wholly
inappropriate}.

Across the 20th century, decolonization movements --- from Ireland to
India and across the African continent --- began systematically to
dismantle the imperial state. Britain's decline from an imperial global
power to a ``small island'' coincided with its entry into
\href{https://ukandeu.ac.uk/fact-figures/when-did-britain-decide-to-join-the-european-union/}{the
European Economic Community}. This masked the loss of global power and
status that came with the loss of empire and enabled Britain to continue
to exert disproportionate influence upon the world stage.

It is significant that it was when Britain sought to leave the European
Union that questions both of the ``breakup'' of Britain (previously
united by the imperial project) and unresolved issues of its imperial
past emerged center stage. The discourses around the Brexit referendum
sought
\href{https://discoversociety.org/2016/07/05/viewpoint-brexit-class-and-british-national-identity/}{to
reclaim national sovereignty} with little recognition that Britain had
never been a nation, but an empire.

This inadequate historical understanding disfigured contemporary
arguments about who belongs and has rights, as was evident in the
\href{https://www.nytimes.com/2018/04/24/world/europe/britain-windrush-immigrants.html}{illegitimate
deportations of Commonwealth citizens} known as the Windrush scandal.

The parochiality of Brexit has been disrupted by two more immediate
contexts. The resurgence of the global Black Lives Matter movement in
light of the death of George Floyd and the
\href{https://www.theguardian.com/world/2020/jun/02/covid-19-death-rate-in-england-higher-among-bame-people}{disproportionate
deaths} of black, Asian and other minority ethnic citizens in Britain
--- mostly people with origins in the former colonies --- from Covid-19.

The inherited consequences of colonialism are evident across all British
ethnic minority populations. And their roles as front-line workers,
keeping the country going during this crisis, has shifted the public
sense of who constitutes the
\href{https://discoversociety.org/2020/04/22/rethinking-brexit-in-the-light-of-covid-19/}{social
and political community}. This conjunction provokes all of us to
reconsider the nature of the inequalities that structure our communities
and the complicity of particular forms of public representation in this.

The inequalities and injustices created by colonialism, enslavement and
empire are manifest in the public display of statues of men such as
Edward Colston, Cecil Rhodes, Henry Dundas and Robert Clive. They are
manifest in statues of King Leopold II in Belgium or any number of
Confederate statues in the United States. They represent and glorify
those histories and call us to agree to be defined by them, to be
represented by them.

It is only if you are unaffected by Colston's trade in human beings that
it is possible to value his philanthropy separated from it. If it is
understood that his philanthropy is intimately connected to the slave
trade and the imperial project and that we continue to live the
hierarchies and inequalities established through such historical
processes, then a reckoning is necessary. This is particularly so when
we acknowledge the subjects of empire, and those who were subjected by
it, as also being who we, collectively, are today.

A mature political community addresses historical wrongs by recognizing
and acting upon the just claims of others. In the process, it tackles
the contemporary inequalities that flow from those histories and comes
to a more expansive self-understanding.

The toppling of Colston's statue has made a public conversation about
our colonial past possible. Equality and freedom from domination is not
given; it has to be struggled for. Those who
\href{https://www.theguardian.com/politics/live/2020/jun/08/uk-coronavirus-johnson-says-anti-racist-protests-were-subverted-by-thuggery-live-news-covid19-updates}{condemn
disruptive actions} that precipitate change should recognize the
violence intrinsic to previous acquiescence.

A statue topples and the veil of colonial ignorance is torn. This is not
a moment to reinscribe that ignorance. Rather, it is a moment to
acknowledge what has now become collectively visible and to represent
ourselves anew.

Gurminder K. Bhambra is a professor of postcolonial and decolonial
studies in the School of Global Studies, University of Sussex.

\emph{The Times is committed to publishing}
\href{https://www.nytimes.com/2019/01/31/opinion/letters/letters-to-editor-new-york-times-women.html}{\emph{a
diversity of letters}} \emph{to the editor. We'd like to hear what you
think about this or any of our articles. Here are some}
\href{https://help.nytimes.com/hc/en-us/articles/115014925288-How-to-submit-a-letter-to-the-editor}{\emph{tips}}\emph{.
And here's our email:}
\href{mailto:letters@nytimes.com}{\emph{letters@nytimes.com}}\emph{.}

\emph{Follow The New York Times Opinion section on}
\href{https://www.facebook.com/nytopinion}{\emph{Facebook}}\emph{,}
\href{http://twitter.com/NYTOpinion}{\emph{Twitter (@NYTopinion)}}
\emph{and}
\href{https://www.instagram.com/nytopinion/}{\emph{Instagram}}\emph{.}

Advertisement

\protect\hyperlink{after-bottom}{Continue reading the main story}

\hypertarget{site-index}{%
\subsection{Site Index}\label{site-index}}

\hypertarget{site-information-navigation}{%
\subsection{Site Information
Navigation}\label{site-information-navigation}}

\begin{itemize}
\tightlist
\item
  \href{https://help.nytimes.com/hc/en-us/articles/115014792127-Copyright-notice}{©~2020~The
  New York Times Company}
\end{itemize}

\begin{itemize}
\tightlist
\item
  \href{https://www.nytco.com/}{NYTCo}
\item
  \href{https://help.nytimes.com/hc/en-us/articles/115015385887-Contact-Us}{Contact
  Us}
\item
  \href{https://www.nytco.com/careers/}{Work with us}
\item
  \href{https://nytmediakit.com/}{Advertise}
\item
  \href{http://www.tbrandstudio.com/}{T Brand Studio}
\item
  \href{https://www.nytimes.com/privacy/cookie-policy\#how-do-i-manage-trackers}{Your
  Ad Choices}
\item
  \href{https://www.nytimes.com/privacy}{Privacy}
\item
  \href{https://help.nytimes.com/hc/en-us/articles/115014893428-Terms-of-service}{Terms
  of Service}
\item
  \href{https://help.nytimes.com/hc/en-us/articles/115014893968-Terms-of-sale}{Terms
  of Sale}
\item
  \href{https://spiderbites.nytimes.com}{Site Map}
\item
  \href{https://help.nytimes.com/hc/en-us}{Help}
\item
  \href{https://www.nytimes.com/subscription?campaignId=37WXW}{Subscriptions}
\end{itemize}
