Sections

SEARCH

\protect\hyperlink{site-content}{Skip to
content}\protect\hyperlink{site-index}{Skip to site index}

\href{https://www.nytimes.com/section/world/middleeast}{Middle East}

\href{https://myaccount.nytimes.com/auth/login?response_type=cookie\&client_id=vi}{}

\href{https://www.nytimes.com/section/todayspaper}{Today's Paper}

\href{/section/world/middleeast}{Middle East}\textbar{}Syria's Economy
Collapses Even as Civil War Winds to a Close

\url{https://nyti.ms/37vSVQ9}

\begin{itemize}
\item
\item
\item
\item
\item
\item
\end{itemize}

Advertisement

\protect\hyperlink{after-top}{Continue reading the main story}

Supported by

\protect\hyperlink{after-sponsor}{Continue reading the main story}

\hypertarget{syrias-economy-collapses-even-as-civil-war-winds-to-a-close}{%
\section{Syria's Economy Collapses Even as Civil War Winds to a
Close}\label{syrias-economy-collapses-even-as-civil-war-winds-to-a-close}}

President Bashar al-Assad faces threats he cannot bomb into submission,
as protests erupt over living standards and a rift opens among the
ruling elite.

\includegraphics{https://static01.nyt.com/images/2020/06/11/world/xxsyria/xxsyria-articleLarge-v2.jpg?quality=75\&auto=webp\&disable=upscale}

\href{https://www.nytimes.com/by/ben-hubbard}{\includegraphics{https://static01.nyt.com/images/2018/10/10/multimedia/author-ben-hubbard/author-ben-hubbard-thumbLarge.png}}

By \href{https://www.nytimes.com/by/ben-hubbard}{Ben Hubbard}

\begin{itemize}
\item
  June 15, 2020
\item
  \begin{itemize}
  \item
  \item
  \item
  \item
  \item
  \item
  \end{itemize}
\end{itemize}

BEIRUT, Lebanon --- President Bashar al-Assad, who has mostly won
Syria's civil war, now faces an acute economic crisis that has
impoverished his people, brought about the collapse of the currency and
fueled a rare public rift in the ruling elite.

Government salaries have become nearly worthless.
\href{https://www.facebook.com/watch/?v=962481407516349}{Protests
against falling living standards} have broken out in the southeast.

The Syrian pound is worth so little that people have
posted\href{https://twitter.com/Ferhad49838610/status/1269950887218360320}{images
on social media of bank notes used to roll cigarettes}.

The government is so strapped for cash that it is squeezing wealthy
businessmen to help fund the state, a move that prompted a powerful
Syrian tycoon to openly criticize the government.

For nine years, Mr. al-Assad has relied on brute force to beat back the
rebels who sought to end his family's decades-old grip on power. But
now, with the war's biggest battles behind him, he faces new threats
that he cannot bomb his way out of or count on his few allies to help
him surmount.

\includegraphics{https://static01.nyt.com/images/2020/06/11/world/xxsyria11/xxsyria11-articleLarge.jpg?quality=75\&auto=webp\&disable=upscale}

That the tycoon, a member of Mr. al-Assad's inner circle, had the
temerity to go public with his dispute suggests a weakening of his
power. And strict
\href{https://www.nytimes.com/2019/12/16/us/politics/us-syria-sanctions-war-crimes.html}{American
economic sanctions} that take effect Wednesday are likely to make
matters worse.

``The problem for al-Assad is that he does not have a solution,'' said
Danny Makki, a Syria analyst at the Middle East Institute in Washington.
``It is going to create a high intensity crisis, and he either has to
talk to the Americans and make concessions or endure what could be a
major economic collapse.''

The war has throttled Syria's economy, reducing it to a third the size
it was before the war and taking a toll thought to be in the hundreds of
billions of dollars.

An estimated 80 percent of Syrians live in poverty. About 40 percent
were unemployed at the end of 2019, the latest figures available, and
joblessness has only increased because of government restrictions to
control the coronavirus.

The collapse of Syria's currency has compounded the crisis.

Worth about 50 to the United States dollar before the war, the Syrian
pound traded in the hundreds per dollar in recent years, but began
plummeting last fall in connection with a
\href{https://www.nytimes.com/2020/05/10/world/middleeast/lebanon-economic-crisis.html}{financial
crisis in neighboring Lebanon}, where many Syrians kept their money.
Unofficial capital controls aimed at stopping a run on Lebanese banks
have also blocked Syrians who bank there from withdrawing dollars.

Last week the Syrian pound fell to 3,500 to the dollar on the black
market, destroying the purchasing power of government employees. Prices
for imported staples such as sugar, coffee, flour and rice have doubled
or tripled.

Image

A market street in Ariha, in Idlib Province in northwestern Syria, the
last pocket of the country still under rebel control.Credit...Omar Haj
Kadour/Agence France-Presse --- Getty Images

The government has been hitting up Syrian business leaders for money to
help finance government salaries and services, according to Mr. Makki
and Jamil al-Sayyed, a former Lebanese security chief who meets with
Syrian officials.

Most of those approached have quietly acquiesced, and how much they paid
has not been made public.

But Syria's best known tycoon --- Rami Makhlouf, a billionaire financier
with holdings in electricity, oil and telecommunications --- pushed
back, creating a rare open divide at the top levels of Syrian society.

Mr. Makhlouf is a first cousin and childhood companion of Mr. al-Assad's
who used his connections to the ruling family to become one of Syria's
wealthiest men.

The United States Treasury
Department\href{https://www.treasury.gov/press-center/press-releases/Pages/hp834.aspx}{sanctioned
Mr. Makhlouf in 2008 for corruption}, freezing any assets he held in
American banks. Calling him ``one of the primary centers of corruption
in Syria,'' the department said he had ``amassed his commercial empire
by exploiting his relationships with Syrian regime members'' and
``manipulated the Syrian judicial system and used Syrian intelligence
officials to intimidate his business rivals.''

When Mr. Makhlouf refused to pay, Mr. al-Assad's government turned the
screws, barring him from state contracts, freezing his assets and
leveling about \$180 million in fees on SyriaTel, the country's main
mobile phone provider and once a cash cow for Mr. Makhlouf.

That's when Mr. Makhlouf went public, posting a series of videos on
Facebook complaining about the arrest of his employees, casting himself
as a patron of the security services and calling on Mr. al-Assad to
rectify matters.

Image

Last month, Mr. Makhlouf posted videos on Facebook calling Mr.
al-Assad's policies ``dangerous.''Credit...Agence France-Presse ---
Getty Images

``The situation is dangerous,'' Mr. Makhlouf said in one video. If the
pressure on him and his employees continued, he said, there would be
``divine justice because we have started a dangerous turn.''

Efforts to reach Mr. Makhlouf through his social media accounts were not
successful.

Analysts and former associates of Mr. al-Assad said that Mr. Makhlouf's
public campaign revealed a new fragility in Mr. al-Assad's inner circle.

``The regime is suffering from many economic and other problems, or Rami
would have never dared to do these videos,'' said Firas Tlass, a former
associate of the al-Assad family who defected early in the war.

In another sign of turmoil in the government, Mr. al-Assad dismissed the
prime minister, Imad Khamis, on Thursday in a move analysts said sought
to deflect blame for the country's economic distress.

Fearing that public grumbling in pacified areas of the country could
fuel unrest, the security forces have detained a number of citizens who
have written about corruption and the economic decline on social media.

In April, an economics professor at Damascus University, Ziad Zanboua,
wrote on Facebook that he had been fired after speaking publicly about
the erosion of Syria's middle class.

``Why is a university professor punished in a state of institutions and
law?'' Mr. Zanboua wrote. ``Because I committed the greatest of all
indecencies: I talked?!''

Anger about sinking livelihoods has flared even among members of Mr.
al-Assad's Alawite minority, whose young men fought in large numbers
with his forces only to find that they will share in the country's
poverty instead of reaping the benefits of victory.

One Alawite man with relatives in the military said the currency
collapse had made their salaries virtually worthless, with army generals
earning the equivalent of less than \$50 per month and soldiers earning
less than a third of that.

Image

Syrian government soldiers in the city of Qamishli, Syria, where they
control only a small pocket in the part of northeastern Syria run by the
Kurds.Credit...Delil Souleiman/AFP, via Getty Image

``We sacrificed tens of thousands of our sons and men to defend the
united, strong Syria and live in dignity,'' the man said, speaking on
condition of anonymity to avoid arrest. ``The president should find a
solution or leave.''

Mr. al-Assad, who occasionally appears in public wearing dark suits and
conservative ties, has not responded publicly to Mr. Makhlouf and has
blamed conspiracies by foreign adversaries --- the United States, Israel
and Saudi Arabia, among others --- for his country's problems.

He has rarely addressed the economic pain facing his citizens, but last
month he told a committee that restrictions on business and movement
aimed at preventing the coronavirus had trapped Syrians ``between hunger
and poverty and deprivation on one side and death on the other.''

Mr. al-Assad has managed to reclaim most of the country, aside from
\href{https://www.nytimes.com/2020/02/26/world/middleeast/syria-idlib-refugees.html}{pockets
in the north} and
\href{https://www.nytimes.com/interactive/2019/10/30/world/middleeast/syria-turkey-maps.html}{most
of the northeast}, with generous military assistance from Russia and
Iran.

But those allies, both struggling under Western sanctions, are unlikely
to bail him out financially. Officials in both countries have raised
questions about how Mr. al-Assad will pay them back for their support.

``The Russians, the Iranians, the allies --- they are not going to plow
money into Syria,'' said Mr. Makki, the Syria analyst. ``They want a
return on their investment.''

Image

Mr. al-Assad with Russia's defense minister, Sergei Shoigu, center, and
President Vladimir Putin in Damascus in January. Russian forces have
been crucial to helping Mr. al-Assad win the civil war.Credit...Pool
photo by Alexei Nikolsky

More pain looms.

The United States will impose
\href{https://www.nytimes.com/2019/12/16/us/politics/us-syria-sanctions-war-crimes.html}{sweeping
new sanctions} this week that could target the businesspeople Mr.
al-Assad needs to rebuild
\href{https://www.nytimes.com/2019/08/20/world/middleeast/syria-recovery-aleppo-douma.html}{his
shattered cities}.

The Caesar Act, named after a Syrian police photographer who defected
with photos of thousands of prisoners
\href{https://www.nytimes.com/2019/05/11/world/middleeast/syria-torture-prisons.html}{tortured
and killed in Syrian custody}, requires the United States president to
sanction anyone who does business with or provides significant support
to the Syrian government or its officials.

It specifically targets anyone who provides aircraft parts, offers
services to the Syrian oil industry or engages in engineering or
construction projects for the state or people linked to it.

Analysts said the legislation is so broad that it is unclear how it will
be applied, but that it has already sent a chill through companies in
the region that were eyeing opportunities to profit from Syria's
reconstruction efforts.

``If I am a businessman and I have a few million dollars to invest, I
won't go to Syria today,'' said Kheder Khaddour, a Syria analyst at the
Carnegie Middle East Center in Beirut. ``It's too risky.''

Hwaida Saad contributed reporting from Beirut, and an employee of The
New York Times from Damascus, Syria.

Advertisement

\protect\hyperlink{after-bottom}{Continue reading the main story}

\hypertarget{site-index}{%
\subsection{Site Index}\label{site-index}}

\hypertarget{site-information-navigation}{%
\subsection{Site Information
Navigation}\label{site-information-navigation}}

\begin{itemize}
\tightlist
\item
  \href{https://help.nytimes.com/hc/en-us/articles/115014792127-Copyright-notice}{©~2020~The
  New York Times Company}
\end{itemize}

\begin{itemize}
\tightlist
\item
  \href{https://www.nytco.com/}{NYTCo}
\item
  \href{https://help.nytimes.com/hc/en-us/articles/115015385887-Contact-Us}{Contact
  Us}
\item
  \href{https://www.nytco.com/careers/}{Work with us}
\item
  \href{https://nytmediakit.com/}{Advertise}
\item
  \href{http://www.tbrandstudio.com/}{T Brand Studio}
\item
  \href{https://www.nytimes.com/privacy/cookie-policy\#how-do-i-manage-trackers}{Your
  Ad Choices}
\item
  \href{https://www.nytimes.com/privacy}{Privacy}
\item
  \href{https://help.nytimes.com/hc/en-us/articles/115014893428-Terms-of-service}{Terms
  of Service}
\item
  \href{https://help.nytimes.com/hc/en-us/articles/115014893968-Terms-of-sale}{Terms
  of Sale}
\item
  \href{https://spiderbites.nytimes.com}{Site Map}
\item
  \href{https://help.nytimes.com/hc/en-us}{Help}
\item
  \href{https://www.nytimes.com/subscription?campaignId=37WXW}{Subscriptions}
\end{itemize}
