Sections

SEARCH

\protect\hyperlink{site-content}{Skip to
content}\protect\hyperlink{site-index}{Skip to site index}

\href{https://myaccount.nytimes.com/auth/login?response_type=cookie\&client_id=vi}{}

\href{https://www.nytimes.com/section/todayspaper}{Today's Paper}

\href{/section/opinion}{Opinion}\textbar{}Donald Trump Is Our National
Catastrophe

\href{https://nyti.ms/2XF1FAh}{https://nyti.ms/2XF1FAh}

\begin{itemize}
\item
\item
\item
\item
\item
\item
\end{itemize}

Advertisement

\protect\hyperlink{after-top}{Continue reading the main story}

\href{/section/opinion}{Opinion}

Supported by

\protect\hyperlink{after-sponsor}{Continue reading the main story}

\hypertarget{donald-trump-is-our-national-catastrophe}{%
\section{Donald Trump Is Our National
Catastrophe}\label{donald-trump-is-our-national-catastrophe}}

With malice toward all; with charity for none.

\href{https://www.nytimes.com/by/bret-stephens}{\includegraphics{https://static01.nyt.com/images/2017/08/27/insider/bretstephens/bretstephens-thumbLarge-v6.png}}

By \href{https://www.nytimes.com/by/bret-stephens}{Bret Stephens}

Opinion Columnist

\begin{itemize}
\item
  June 5, 2020
\item
  \begin{itemize}
  \item
  \item
  \item
  \item
  \item
  \item
  \end{itemize}
\end{itemize}

\includegraphics{https://static01.nyt.com/images/2020/06/06/opinion/06stephens_print/merlin_172688259_4f84b201-2da4-42ed-ac60-754a266d75c4-articleLarge.jpg?quality=75\&auto=webp\&disable=upscale}

This spring I taught a seminar (via Zoom, of course) at the University
of Chicago on the art of political persuasion. We read Lincoln,
Pericles, King, Orwell, Havel and Churchill, among other great
practitioners of the art. We ended with a study of Donald Trump's
tweets, as part of a class on demagogy.

If the closing subject was depressing, at least the timing was
appropriate.

We are in the midst of an unprecedented national catastrophe. The
catastrophe is not the pandemic, or an economic depression, or killer
cops, or looted cities, or racial inequities. These are all too
precedented. What's unprecedented is that never before have we been led
by a man who so completely inverts the spirit of Lincoln's Second
Inaugural Address.

With malice toward all; with charity for none: eight words that
encapsulate everything this president is, does and stands for.

What does one learn when reading great political speeches and writings?
That well-chosen words are the way by which past deeds acquire meaning
and future deeds acquire purpose. ``The world will little note, nor long
remember what we say here,'' are the only false notes in the Gettysburg
Address. The Battle of Gettysburg is etched in national memory less for
its military significance than because Lincoln reinvented the goals of
the Civil War in that speech --- and, in doing so, reimagined the
possibilities of America.

Political writing doesn't just provide meaning and purpose. It also
offers determination, hope and instruction.

In
``\href{https://hac.bard.edu/amor-mundi/the-power-of-the-powerless-vaclav-havel-2011-12-23}{The
Power of the Powerless},'' written at one of the grimmer moments of
Communist tyranny, Václav Havel laid out why the system was so much
weaker, and the individual so much stronger, than either side knew. In
his
``\href{https://winstonchurchill.org/resources/speeches/1940-the-finest-hour/we-shall-fight-on-the-beaches/}{Fight
on the beaches}'' speech after Dunkirk, Winston Churchill told Britons
of ``a victory inside this deliverance'' --- a reason, however remote,
for resolve and optimism. In
``\href{https://www.africa.upenn.edu/Articles_Gen/Letter_Birmingham.html}{Letter
From Birmingham Jail},'' Martin Luther King Jr., explained why patience
was no answer to injustice: ``When you have seen hate-filled policemen
curse, kick, brutalize, and even kill your black brothers and sisters
with impunity \ldots{} then you will understand why we find it difficult
to wait.''

In a word, great political writing aims to elevate. What, by contrast,
does one learn by studying Trump's utterances?

The purpose of Trump's presidency is to debase, first by debasing the
currency of speech. It's why he refuses to hire reasonably competent
speechwriters to craft reasonably competent speeches. It's why his
communication team has been filled by people like Dan Scavino and
Stephanie Grisham and Sarah Sanders.

And it's why Twitter is his preferred medium of communication. It is
speech designed for provocations and put-downs; for making supporters
feel smug; for making opponents seethe; for reducing national discourse
to the level of grunts and counter-grunts.

That's a level that suits Trump because it's the level at which he
excels. Anyone who studies Trump's tweets carefully must come away
impressed by the way he has mastered the demagogic arts. He doesn't lead
his base, as most politicians do. He \emph{personifies} it. He speaks to
his followers as if he were them. He cultivates their resentments,
demonizes their opponents, validates their hatreds. He glorifies himself
so they may bask in the reflection.

Whatever this has achieved for him, or them, it's a calamity for us. At
a moment when disease has left more than 100,000 American families
bereft, we have a president incapable of expressing the nation's
heartbreak. At a moment of the most bitter racial grief since the 1960s,
we have a president who has bankrupted the moral capital of the office
he holds.

And at a moment when many Americans, particularly conservatives, are
aghast at the outbursts of looting and rioting that have come in the
wake of peaceful protests, we have a president who wants to replace rule
of law with rule by the gun. If Trump now faces a revolt by the
Pentagon's
\href{https://www.nytimes.com/2020/06/03/us/politics/esper-milley-trump-protest.html}{civilian}
and
\href{https://int.nyt.com/data/documenthelper/6990-milley-memo/fc4fb1c4459fbdbc87a7/optimized/full.pdf\#page=1}{military
leadership} (both current and
\href{https://www.cnn.com/2020/06/05/politics/john-kelly-agrees-with-jim-mattis-on-trump/index.html}{former})
against his desire to deploy active-duty troops in American cities, it's
because his words continue to drain whatever is left of his credibility
as commander in chief.

I write this as someone who doesn't lay every national problem at
Trump's feet and tries to give him credit when I think it's due.

Trump is no more responsible for the policing in Minneapolis than Barack
Obama was responsible for policing in Ferguson. I doubt the pandemic
would have been handled much better by a Hillary Clinton administration,
especially considering the catastrophic errors of judgment by people
like Bill de Blasio and Andrew Cuomo. And our economic woes are largely
the result of a lockdown strategy most avidly embraced by the
president's critics.

But the point here isn't that Trump is responsible for the nation's
wounds. It's that he is the reason some of those wounds have festered
and why none of them can heal, at least for as long as he remains in
office. Until we have a president who can say, as Lincoln did in his
first inaugural, ``We are not enemies, but friends'' --- and be believed
in the bargain --- our national agony will only grow worse.

\emph{The Times is committed to publishing}
\href{https://www.nytimes.com/2019/01/31/opinion/letters/letters-to-editor-new-york-times-women.html}{\emph{a
diversity of letters}} \emph{to the editor. We'd like to hear what you
think about this or any of our articles. Here are some}
\href{https://help.nytimes.com/hc/en-us/articles/115014925288-How-to-submit-a-letter-to-the-editor}{\emph{tips}}\emph{.
And here's our email:}
\href{mailto:letters@nytimes.com}{\emph{letters@nytimes.com}}\emph{.}

\emph{Follow The New York Times Opinion section on}
\href{https://www.facebook.com/nytopinion}{\emph{Facebook}}\emph{,}
\href{http://twitter.com/NYTOpinion}{\emph{Twitter (@NYTopinion)}}
\emph{and}
\href{https://www.instagram.com/nytopinion/}{\emph{Instagram}}\emph{.}

Advertisement

\protect\hyperlink{after-bottom}{Continue reading the main story}

\hypertarget{site-index}{%
\subsection{Site Index}\label{site-index}}

\hypertarget{site-information-navigation}{%
\subsection{Site Information
Navigation}\label{site-information-navigation}}

\begin{itemize}
\tightlist
\item
  \href{https://help.nytimes.com/hc/en-us/articles/115014792127-Copyright-notice}{©~2020~The
  New York Times Company}
\end{itemize}

\begin{itemize}
\tightlist
\item
  \href{https://www.nytco.com/}{NYTCo}
\item
  \href{https://help.nytimes.com/hc/en-us/articles/115015385887-Contact-Us}{Contact
  Us}
\item
  \href{https://www.nytco.com/careers/}{Work with us}
\item
  \href{https://nytmediakit.com/}{Advertise}
\item
  \href{http://www.tbrandstudio.com/}{T Brand Studio}
\item
  \href{https://www.nytimes.com/privacy/cookie-policy\#how-do-i-manage-trackers}{Your
  Ad Choices}
\item
  \href{https://www.nytimes.com/privacy}{Privacy}
\item
  \href{https://help.nytimes.com/hc/en-us/articles/115014893428-Terms-of-service}{Terms
  of Service}
\item
  \href{https://help.nytimes.com/hc/en-us/articles/115014893968-Terms-of-sale}{Terms
  of Sale}
\item
  \href{https://spiderbites.nytimes.com}{Site Map}
\item
  \href{https://help.nytimes.com/hc/en-us}{Help}
\item
  \href{https://www.nytimes.com/subscription?campaignId=37WXW}{Subscriptions}
\end{itemize}
