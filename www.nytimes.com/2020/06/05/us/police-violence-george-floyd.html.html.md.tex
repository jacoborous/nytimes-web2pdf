Sections

SEARCH

\protect\hyperlink{site-content}{Skip to
content}\protect\hyperlink{site-index}{Skip to site index}

\href{https://www.nytimes.com/section/us}{U.S.}

\href{https://myaccount.nytimes.com/auth/login?response_type=cookie\&client_id=vi}{}

\href{https://www.nytimes.com/section/todayspaper}{Today's Paper}

\href{/section/us}{U.S.}\textbar{}A Crisis That Began With an Image of
Police Violence Keeps Providing More

\url{https://nyti.ms/2MsEaDY}

\begin{itemize}
\item
\item
\item
\item
\item
\end{itemize}

\href{https://www.nytimes.com/news-event/george-floyd-protests-minneapolis-new-york-los-angeles?action=click\&pgtype=Article\&state=default\&region=TOP_BANNER\&context=storylines_menu}{Race
and America}

\begin{itemize}
\tightlist
\item
  \href{https://www.nytimes.com/2020/07/26/us/protests-portland-seattle-trump.html?action=click\&pgtype=Article\&state=default\&region=TOP_BANNER\&context=storylines_menu}{Protesters
  Return to Other Cities}
\item
  \href{https://www.nytimes.com/2020/07/24/us/portland-oregon-protests-white-race.html?action=click\&pgtype=Article\&state=default\&region=TOP_BANNER\&context=storylines_menu}{Portland
  at the Center}
\item
  \href{https://www.nytimes.com/2020/07/23/podcasts/the-daily/portland-protests.html?action=click\&pgtype=Article\&state=default\&region=TOP_BANNER\&context=storylines_menu}{Podcast:
  Showdown in Portland}
\item
  \href{https://www.nytimes.com/interactive/2020/07/16/us/black-lives-matter-protests-louisville-breonna-taylor.html?action=click\&pgtype=Article\&state=default\&region=TOP_BANNER\&context=storylines_menu}{45
  Days in Louisville}
\end{itemize}

Advertisement

\protect\hyperlink{after-top}{Continue reading the main story}

Supported by

\protect\hyperlink{after-sponsor}{Continue reading the main story}

\hypertarget{a-crisis-that-began-with-an-image-of-police-violence-keeps-providing-more}{%
\section{A Crisis That Began With an Image of Police Violence Keeps
Providing
More}\label{a-crisis-that-began-with-an-image-of-police-violence-keeps-providing-more}}

Chaos in the streets and the ubiquity of cellphones have created a
disturbing array of videos reflecting violent police behavior.

\includegraphics{https://static01.nyt.com/images/2020/06/04/us/04UNREST-ABUSE-atlanta/merlin_173115912_96f2353d-49a0-40ef-83d2-d1c254b7968f-articleLarge.jpg?quality=75\&auto=webp\&disable=upscale}

\href{https://www.nytimes.com/by/shawn-hubler}{\includegraphics{https://static01.nyt.com/images/2020/06/05/reader-center/author-shawn-hubler/author-shawn-hubler-thumbLarge.png}}\href{https://www.nytimes.com/by/julie-bosman}{\includegraphics{https://static01.nyt.com/images/2018/11/09/multimedia/author-julie-bosman/author-julie-bosman-thumbLarge.png}}

By \href{https://www.nytimes.com/by/shawn-hubler}{Shawn Hubler} and
\href{https://www.nytimes.com/by/julie-bosman}{Julie Bosman}

\begin{itemize}
\item
  Published June 5, 2020Updated July 8, 2020
\item
  \begin{itemize}
  \item
  \item
  \item
  \item
  \item
  \end{itemize}
\end{itemize}

SACRAMENTO --- A protest movement that was ignited by a horrific video
of police violence --- a white police officer pressing his knee against
the neck of
\href{https://www.nytimes.com/2020/07/08/us/george-floyd-body-camera-transcripts.html}{George
Floyd}, a black man, for nearly nine minutes --- has now prompted
hundreds of other incidents and videos documenting violent tactics by
police.

In Atlanta, a half-dozen officers have been criminally charged after
bystanders tweeted footage of an abrupt **** attack **** on two college
students sitting in a car during protests. In Austin**, **a 20-year-old
protester shot in the head by police officers aiming at someone else
with what's described as nonlethal beanbag ammunition was left with a
fractured skull and brain damage. Video shows volunteers being shot,
too, as they carry him off.

In California, an officer sitting in a police car in Vallejo shot and
killed a 22-year-old man who was on his knees with his hands up. Only
later did the police discover that the man, who was suspected of trying
to loot a Walgreens, had a hammer in his sweatshirt pocket, not a gun.
The family's lawyer says he has requested the videotape from
\href{https://www.nytimes.com/2020/07/08/us/george-floyd-body-camera-transcripts.html}{body
cam} and store cameras, but police haven't yet released them.

Cellphone videos show New York City police officers beating unarmed
protesters and
\href{https://nyc.streetsblog.org/2020/05/30/nypd-out-of-control-videos-depict-cops-on-rampages-across-city/}{sideswiping}
demonstrators with opened squad car doors. Others around the country
show the police indiscriminately
\href{https://www.cleveland.com/crime/2020/06/video-photo-show-cleveland-police-pepper-spraying-shooting-projectiles-at-legal-observers-during-george-floyd-protests.html}{using}
pepper spray on protesters or pedestrians. On live television, police
officers in Louisville, Ky.,
\href{https://www.wave3.com/2020/05/29/lmpd-officer-fires-pepper-balls-wave-news-reporter-photographer-during-louisville-protest/}{fired}
pepper-spray balls at journalists.

\includegraphics{https://static01.nyt.com/images/2020/05/27/autossell/flyod-site-1-white-box/flyod-site-1-white-box-videoSixteenByNineJumbo1600.jpg}

In Fort Lauderdale, Fla., Miami Herald reporters filmed officers who
were
\href{https://www.miamiherald.com/news/local/community/broward/article243193481.html}{shooting
a nonviolent protester in the head} with foam
\href{https://www.nytimes.com/2020/06/12/health/protests-rubber-bullets-beanbag.html}{rubber
bullets}, fracturing her eye socket and leaving her screaming and
bloody. In Kansas City, Mo., the police walked onto a sidewalk to use
pepper spray on protesters yelling at them.

\includegraphics{https://static01.nyt.com/images/2020/06/06/autossell/Police-Push1/Police-Push-videoSixteenByNineJumbo1600.jpg}

In Buffalo,
\href{https://twitter.com/davidbegnaud/status/1268716877355810818?s=21}{a
video from WBFO}, the local National Public Radio station, on Thursday
showed
\href{https://www.nytimes.com/2020/06/05/us/buffalo-police-shove-protester-unrest.html}{police
officers in riot gear shoving a 75-year-old man to the ground} and
walking away as he lay unconscious on the sidewalk, blood coming out of
his ear.

A compilation of videotaped incidents
\href{https://twitter.com/greg_doucette}{posted on Twitter}by a North
Carolina lawyer stood at more than 300 clips by Friday morning.

The episodes have occurred in cities large and small, in the heat of
mass protests and in their quiet aftermath. Some have occurred when the
police confronted people who were suspected of looting. Experts on
policing said that the videos showed, in many cases, examples of abrupt
escalation on the part of law enforcement that was difficult to justify.

``It feels like the police are being challenged in ways that they
haven't been challenged in some time,'' said Chuck Wexler, the executive
director of the Police Executive Research Forum. ``They are responding.
And sometimes, that response is totally inappropriate.''

Ed Obayashi, a California-based expert on the use of force by law
enforcement, said the incidents, while often disturbing, were an overall
improvement on past police conduct during episodes of unrest.

A lawyer who advises the California Association of Police Training
Officers on use of force, Mr. Obayashi said that, considering the level
of chaos across the nation, ``from my standpoint, there's been
considerable restraint.''

The
\href{https://www.nytimes.com/2020/06/03/us/rodney-king-george-floyd-los-angeles.html}{Los
Angeles riots in 1992}, he said, left more than 50 people dead and more
than 1,000 injured just in that city. These protests have been national,
involving thousands of officers and demonstrators in scores of cities,
with far fewer deaths or injuries to civilians, and several have
resulted in arrests, firings or internal investigations of officers.

``I'm seeing, consistently, a command-and-control system in place in
many of these police departments and much better training,'' he said.
``In years past, you'd have officers going this way and that way with no
organization. It was just everyone for himself and `Who can I catch?'''

Those improvements, however, have not necessarily changed the culture of
rank-and-file law enforcement, said John Burris, a longtime civil rights
lawyer in Oakland, Calif., who in 1994 helped represent Rodney King
against the Los Angeles Police Department.

``From what I can see, for most of them, this is still just a moment to
get through, like Ferguson,'' said Mr. Burris, who is now representing
the family of the
\href{https://www.nbcnews.com/news/us-news/police-vallejo-calif-fatally-shoot-man-hammer-kneeling-outside-walgreens-n1224621}{unarmed
man shot Tuesday} just after midnight in a Vallejo parking lot. That
shooting, he noted, was only the most recent for a department that,
despite hiring a new chief last year, has had a long
\href{https://www.kqed.org/news/11768008/the-life-and-death-of-willie-mccoy}{history
of police abuse accusations}.

``We're seeing a lot of bad conduct by police officers throughout the
nation,'' Mr. Burris said. ``And none of it suggests to me that they
appreciate the rage people are feeling as a consequence of longstanding
police brutality that has taken place in the African-American and
Hispanic communities.''

\includegraphics{https://static01.nyt.com/images/2020/05/30/autossell/NYPD-CAR/NYPD-CAR-videoSixteenByNineJumbo1600.jpg}

One difference is that, in a number of cases, police departments are
disciplining officers for their actions.

For example, after footage of the Buffalo incident went viral, the
department suspended two officers and launched an internal affairs
investigation.

The United States Park Police said two officers were
\href{https://twitter.com/JvittalTV/status/1268306972685844480?s=20}{assigned
to administrative duties} after assaulting an Australian news crew in
Lafayette Square in Washington on Monday. In Sacramento, police opened a
use-of-force investigation after bystanders videotaped officers
\href{https://twitter.com/RationalDis/status/1268278251899797505}{using
an apparent chokehold} to subdue a person suspected of looting after
demonstrations late Sunday night.

In Atlanta, where a group of police officers on Saturday yanked two
students out of a car and set upon them with a Taser, two officers were
immediately fired. Criminal charges were filed this week against all six
--- the product, the police chief said, of
``\href{https://www.ajc.com/news/local/atlanta-police-chief-says-charges-against-officers-are-political/afQRm3nM431DlyGyqEjY3K/}{a
tsunami of political jockeying}.''

``We understand that our officers are working very long hours under an
enormous amount of stress,'' said Mayor Keisha Lance Bottoms. ``But we
also understand that the use of excessive force is never acceptable.''

\includegraphics{https://static01.nyt.com/images/2020/06/04/us/04UNREST-ABUSE-atl-camera/merlin_173067234_d2420aab-ad93-4b3c-844c-f62579132e0d-articleLarge.jpg?quality=75\&auto=webp\&disable=upscale}

After a police officer in Houston on a horse ran over a woman who had
her back to the officer, knocking her to the ground, the city's mayor,
Sylvester Turner, apologized to the woman. Art Acevedo, the police
chief, suggested that the officer could have been distracted by
protesters in the crowd throwing rocks and bottles.

Austin's police chief, Brian Manley,
\href{https://www.statesman.com/news/20200601/lsquothis-is-not-what-we-set-out-to-dorsquo-chief-says-after-police-injure-some-during-protests}{also
apologized} after injuries to protesters in that city, including to
Justin Howell, 20, who was badly hurt by a beanbag round after a person
next to him hurled a water bottle at the police.

Injuring the public ``is not what we set out to do as a police
department,'' Chief Manley said on Monday.

Mr. Howell's mother, Myra Howell, said on Thursday that her son, who was
about to start his junior year in college, was ``doing better'' from his
injuries, but that ``it's day by day.''

The Austin City Council on Thursday opened a two-day review of police
handling of protests over the weekend, with several members expressing
outrage over the police use of ``less-lethal'' munitions that injured
four demonstrators, including Mr. Howell.

``People shouldn't be in a hospital for attending a demonstration,''
said Greg Casar, a council member, who called for a ``transformational''
change in policing ``from top to bottom.''

Image

Justin Howell, who was protesting in Texas, was hit with a beanbag round
in the head, leaving him seriously injured.Credit...Family photo

Some videos have displayed a more empathetic response, like one that
showed officers
\href{https://publish.twitter.com/?query=https\%3A\%2F\%2Ftwitter.com\%2FRexChapman\%2Fstatus\%2F1268379338019418113\&widget=Tweet}{dancing
with protesters in Lincoln, Neb.}

But in many cellphone videos that have proliferated on social media,
officers often appear to be unapologetic for rough treatment of
demonstrators.
``\href{https://twitter.com/Bishop_Krystal/status/1268009974170451968}{Don't
kill 'em, but hit 'em hard,}'' a state trooper in riot gear exhorts in
video posted this week during protests in Seattle.

Even medical personnel trying to provide care for protesters have
reported being targeted by law enforcement. When Geoff Markowitz was
working as a medic last weekend at a protest in Denver, he was surprised
when
\href{https://www.nytimes.com/2020/06/12/health/protests-rubber-bullets-beanbag.html}{rubber
bullets} were fired in his direction. Mr. Markowitz, a third-year
medical student was tending to a protester who had been hit in the
temple by a foam bullet fired by the Denver Police Department.

Mr. Markowitz estimated that he was tear-gassed more than a dozen times,
even though he was only approaching if someone was seriously injured and
needed to be moved.

``I believe that it is wrong to target medical professionals attempting
to render aid,'' he said.

\includegraphics{https://static01.nyt.com/images/2020/06/01/world/01unrest-briefing-wh/01unrest-briefing-wh-videoSixteenByNine3000.jpg}

The American Civil Liberties Union has responded to police tactics with
lawsuits: After journalists were attacked by the police during protests
in Minneapolis, the organization sued the City of Minneapolis, accusing
it of abridging the constitutionally mandated freedom of the press. On
Thursday, the A.C.L.U. filed a lawsuit against President Trump and
Attorney General William P. Barr on behalf of Black Lives Matter D.C.
and protesters who were in Lafayette Park in Washington on Monday,
saying that the police firing tear gas at peaceful protesters violated
their constitutional rights.

The emergence of videos from demonstrations has brought other episodes
to light that might have otherwise been overlooked. In one video taken
on Sunday in Chicago, police officers surrounded a car in a parking lot
and forcibly pulled out people inside. The women who were in the car
told
\href{https://blockclubchicago.org/2020/06/04/in-vicious-police-attack-of-black-woman-outside-of-northwest-side-mall-family-demands-criminal-investigation-it-has-to-stop/}{Block
Club Chicago} they were simply trying to purchase items at a store and
realized that it was closed when the police pounced.

Mayor Lori Lightfoot, who ran for office promising to reform the Police
Department, said there would be an investigation.

``If any wrongdoing is discovered, officers will be held accountable,''
Ms. Lightfoot said in a statement. Kim Foxx, the Cook County state's
attorney, said she would conduct an independent investigation.

Shawn Hubler reported from Sacramento, and Julie Bosman from Chicago.
David Montgomery contributed reporting from Austin, Texas, and Wudan Yan
from Seattle.

Advertisement

\protect\hyperlink{after-bottom}{Continue reading the main story}

\hypertarget{site-index}{%
\subsection{Site Index}\label{site-index}}

\hypertarget{site-information-navigation}{%
\subsection{Site Information
Navigation}\label{site-information-navigation}}

\begin{itemize}
\tightlist
\item
  \href{https://help.nytimes.com/hc/en-us/articles/115014792127-Copyright-notice}{©~2020~The
  New York Times Company}
\end{itemize}

\begin{itemize}
\tightlist
\item
  \href{https://www.nytco.com/}{NYTCo}
\item
  \href{https://help.nytimes.com/hc/en-us/articles/115015385887-Contact-Us}{Contact
  Us}
\item
  \href{https://www.nytco.com/careers/}{Work with us}
\item
  \href{https://nytmediakit.com/}{Advertise}
\item
  \href{http://www.tbrandstudio.com/}{T Brand Studio}
\item
  \href{https://www.nytimes.com/privacy/cookie-policy\#how-do-i-manage-trackers}{Your
  Ad Choices}
\item
  \href{https://www.nytimes.com/privacy}{Privacy}
\item
  \href{https://help.nytimes.com/hc/en-us/articles/115014893428-Terms-of-service}{Terms
  of Service}
\item
  \href{https://help.nytimes.com/hc/en-us/articles/115014893968-Terms-of-sale}{Terms
  of Sale}
\item
  \href{https://spiderbites.nytimes.com}{Site Map}
\item
  \href{https://help.nytimes.com/hc/en-us}{Help}
\item
  \href{https://www.nytimes.com/subscription?campaignId=37WXW}{Subscriptions}
\end{itemize}
