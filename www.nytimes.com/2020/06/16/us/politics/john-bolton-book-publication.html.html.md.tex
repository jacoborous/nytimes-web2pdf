Sections

SEARCH

\protect\hyperlink{site-content}{Skip to
content}\protect\hyperlink{site-index}{Skip to site index}

\href{https://www.nytimes.com/section/politics}{Politics}

\href{https://myaccount.nytimes.com/auth/login?response_type=cookie\&client_id=vi}{}

\href{https://www.nytimes.com/section/todayspaper}{Today's Paper}

\href{/section/politics}{Politics}\textbar{}Trump Administration Sues to
Try to Delay Publication of Bolton's Book

\url{https://nyti.ms/3e88Gz7}

\begin{itemize}
\item
\item
\item
\item
\item
\item
\end{itemize}

Advertisement

\protect\hyperlink{after-top}{Continue reading the main story}

Supported by

\protect\hyperlink{after-sponsor}{Continue reading the main story}

\hypertarget{trump-administration-sues-to-try-to-delay-publication-of-boltons-book}{%
\section{Trump Administration Sues to Try to Delay Publication of
Bolton's
Book}\label{trump-administration-sues-to-try-to-delay-publication-of-boltons-book}}

The request comes a week before the highly anticipated memoir was set to
be published.

\includegraphics{https://static01.nyt.com/images/2020/06/17/us/politics/16dc-bolton-print/merlin_173470857_9bd374a6-d7d3-41a4-9ce9-bae598d5af35-articleLarge.jpg?quality=75\&auto=webp\&disable=upscale}

\href{https://www.nytimes.com/by/maggie-haberman}{\includegraphics{https://static01.nyt.com/images/2018/07/12/multimedia/author-maggie-haberman/author-maggie-haberman-thumbLarge.png}}\href{https://www.nytimes.com/by/katie-benner}{\includegraphics{https://static01.nyt.com/images/2018/02/16/multimedia/author-katie-benner/author-katie-benner-thumbLarge-v2.png}}

By \href{https://www.nytimes.com/by/maggie-haberman}{Maggie Haberman}
and \href{https://www.nytimes.com/by/katie-benner}{Katie Benner}

\begin{itemize}
\item
  Published June 16, 2020Updated July 3, 2020
\item
  \begin{itemize}
  \item
  \item
  \item
  \item
  \item
  \item
  \end{itemize}
\end{itemize}

The Trump administration sued the former national security adviser John
R. Bolton on Tuesday to try to delay publication of his highly
anticipated memoir about his time in the White House, saying the
\href{https://www.nytimes.com/2020/06/17/us/politics/bolton-book-trump-impeached.html}{book
contained classified information} that would compromise national
security if it became public.

\href{https://www.nytimes.com/2020/06/17/books/review-room-where-it-happened-john-bolton-memoir.html}{The
book, ``The Room Where It Happened,}'' is set for release on June 23.
Administration officials have repeatedly warned Mr. Bolton against
publishing it.

Mr. Bolton made clear in a statement this week that his book contained
explosive details about his time at the White House. He and Mr. Trump
clashed on significant policy issues like Iran, North Korea and
Afghanistan, and in his book, Mr. Bolton also confirmed accusations at
the heart of the Democratic impeachment case over the president's
dealings with Ukraine, according to
\href{https://www.nytimes.com/2020/01/31/us/politics/trump-bolton-ukraine.html}{details
from his manuscript} previously reported by The New York Times.

The Justice Department accused him of short-circuiting a government
review that he had agreed to participate in for any eventual manuscript
before even accepting the post in 2018.

Mr. Bolton is breaking that agreement, ``unilaterally deciding that the
prepublication review process is complete and deciding for himself
whether classified information should be made public,'' department
lawyers wrote in a breach of contract
\href{https://int.nyt.com/data/documenthelper/7030-john-bolton-lawsuit/ce3b8c4bf5f6687fa454/optimized/full.pdf\#page=1}{lawsuit
against Mr. Bolton} filed in federal court in Washington.

The book's publisher, Simon \& Schuster, has already printed and
distributed copies, and the lawsuit did not name it as a party, in an
apparent nod to the constitutional and practical impediments to trying
to stop it. Instead, the Justice Department asked a judge to seize Mr.
Bolton's proceeds from the book deal and to order him to try to persuade
Simon \& Schuster to pull back the book and dispose of copies until the
review is completed.

Mr. Bolton's lawyer, Charles J. Cooper, did not immediately respond to a
request for comment. He has said that his client acted in good faith and
that the Trump administration is abusing a standard review process to
prevent Mr. Bolton from revealing information that is merely
embarrassing to President Trump, but not a threat to national security.

A spokesman for Simon \& Schuster called the lawsuit ``nothing more than
the latest in a long-running series of efforts by the administration to
quash publication of a book it deems unflattering to the president.''

While insider books vex many administrations, it is rare for one to sue
to delay them before publication. Several former White House lawyers
from Democratic and Republican administrations said they could not
recall a similar legal effort to stop a book by a former White House
official.

Mr. Bolton grew convinced that the prepublication review was no longer
on the level, if it ever was, after he agreed to make the changes he was
asked to make but the White House still gave him no written confirmation
that the review was complete. The White House has now taken longer to
review the book than Mr. Bolton did to write it after
\href{https://www.nytimes.com/2019/09/10/us/politics/john-bolton-national-security-adviser-trump.html}{he
resigned} in September.

On Monday, Mr. Trump accused Mr. Bolton of violating policies on
classified information by moving ahead with the book. The president also
threatened Mr. Bolton with criminal charges for moving ahead, though
there is no indication that federal prosecutors plan to pursue any.

The Justice Department did accuse Mr. Bolton in the lawsuit of leaking
the manuscript, which contained classified information, without
approval. Disclosing classified information is a federal crime.

But in a further sign that the Justice Department is not mounting a
serious bid to try to block the book's imminent release, the complaint
does not seek a temporary restraining order --- a legal step to freeze
an action so the court can evaluate disputes --- to block any further
distribution of copies, Richard Hasen, a law professor at the University
of California, Irvine,
\href{https://twitter.com/rickhasen/status/1273007151938592768}{said on
Twitter}.

Mr. Trump also faces the prospect of unfavorable revelations from a
forthcoming book on another front. A niece of his,
\href{https://www.nytimes.com/2020/07/03/us/politics/mary-trump-book-publication.html}{Mary
Trump}, plans to divulge damaging information about him in a book to be
published next month, also by Simon \& Schuster, the publisher said this
week.

Mr. Bolton submitted the manuscript to the administration for review in
January. At the time, the impeachment trial was underway into whether
Mr. Trump's dealings with Ukraine constituted an abuse of power.

Democrats asked Mr. Bolton to testify voluntarily in the House
impeachment inquiry, but he declined, and they never sought a subpoena,
fearing a protracted court fight. Mr. Bolton did offer to testify in the
Senate trial if subpoenaed, only to have Republicans block such a
request by Democrats, and ultimately the president was acquitted.

Mr. Trump has been enraged about Mr. Bolton's pending book for months,
and has told his advisers he wanted to try to stop it. On Monday,
Attorney General William P. Barr criticized Mr. Bolton for publishing a
book while the president he served under was still in office,
erroneously calling it unprecedented. Other officials, including Robert
M. Gates, a former defense secretary and C.I.A. director under
presidents of both parties, have published books while the
administration they worked in was still in power.

The government's system for reviewing books and other material by former
officials was created to ensure that classified and other sensitive
information remained secret. Officials must agree to submit any works to
the review process in order to obtain a security clearance.

Mr. Bolton submitted his materials to the National Security Council,
which found ``significant quantities of classified information,'' the
Justice Department said in its complaint.

According to the complaint, a security council official reviewing it
told Mr. Bolton in April that it did not contain classified information.
But she also told Mr. Bolton that the review was still underway. A
second review began in May.

In a letter this month to Mr. Cooper, John A. Eisenberg, a deputy White
House counsel, informed him that the manuscript still contained
classified information, that Mr. Bolton ``himself classified and
designated for declassification only after the lapse of 25 years'' and
that the book could not be published and distributed, the complaint
said.

Mr. Cooper replied in a letter that ``the book has now been printed,
bound and shipped to distributors across the country,'' the lawsuit
said.

Rather than wait for the government to complete the review, the Justice
Department said, Mr. Bolton ``decided to take matters into his own
hands'' and decide with his publisher to disclose last week that the
book would be published this month, without giving notice to the
National Security Council.

The government's 27-page civil complaint was bolstered by 19 exhibits,
including the security agreement that he signed, a White House memo that
detailed Mr. Bolton's postemployment obligations and letters between Mr.
Bolton, his lawyer and the government sent in January, February and
March that discuss the fact that the manuscript contains classified
materials.

The government can try to take several types of action when a former
official is about to publish what it says is classified information.

The first type, so-called prior restraint, is to seek to block
publication of the material. Courts have repeatedly taken a dim view of
such attempts in light of the First Amendment's protections for a free
press, including in \href{https://www.oyez.org/cases/1970/1873}{a famous
case in 1971} striking down the Nixon administration's attempt to stop
The Times from publishing the Pentagon Papers.

Separately, the government can try to seize the proceeds from books on
the grounds that the writers breached the agreements they signed as a
condition of receiving access to classified information.

The lawsuit filed on Tuesday gestured at blocking publication, but it
seemed more squarely focused on seizing Mr. Bolton's profits.

Filed against Mr. Bolton --- not Simon \& Schuster --- it asked for the
court to take control of the money he made from the book, and to order
that he ``instruct or request his publisher, insofar as he has the
authority to do so,'' to retrieve and dispose of current copies of the
book and further delay its release ``until completion of the
prepublication review process.''

A group of former national security officials said last year in a
\href{https://www.nytimes.com/2019/04/02/us/politics/prepublication-censorship-system.html}{lawsuit}
that the review process for books and articles unjustifiably restricted
their rights to free speech and due process.

They claimed that the review system, which is governed by several
ambiguous policies, gives reviewing officials too much discretionary
power over what is published and allows them to quickly clear reviews
for former officials who write positively about the government.

The head of the Justice Department's Civil Division, Joseph R. Hunt, who
signed the lawsuit, told staff members on Tuesday that
\href{https://www.nytimes.com/2020/06/16/us/politics/justice-department-jody-hunt.html}{he
planned to resign after nearly two years} in the post, according to an
email obtained by The Times, making him the third top department
official to step down in the past week.

Charlie Savage and Peter Baker contributed reporting.

Advertisement

\protect\hyperlink{after-bottom}{Continue reading the main story}

\hypertarget{site-index}{%
\subsection{Site Index}\label{site-index}}

\hypertarget{site-information-navigation}{%
\subsection{Site Information
Navigation}\label{site-information-navigation}}

\begin{itemize}
\tightlist
\item
  \href{https://help.nytimes.com/hc/en-us/articles/115014792127-Copyright-notice}{©~2020~The
  New York Times Company}
\end{itemize}

\begin{itemize}
\tightlist
\item
  \href{https://www.nytco.com/}{NYTCo}
\item
  \href{https://help.nytimes.com/hc/en-us/articles/115015385887-Contact-Us}{Contact
  Us}
\item
  \href{https://www.nytco.com/careers/}{Work with us}
\item
  \href{https://nytmediakit.com/}{Advertise}
\item
  \href{http://www.tbrandstudio.com/}{T Brand Studio}
\item
  \href{https://www.nytimes.com/privacy/cookie-policy\#how-do-i-manage-trackers}{Your
  Ad Choices}
\item
  \href{https://www.nytimes.com/privacy}{Privacy}
\item
  \href{https://help.nytimes.com/hc/en-us/articles/115014893428-Terms-of-service}{Terms
  of Service}
\item
  \href{https://help.nytimes.com/hc/en-us/articles/115014893968-Terms-of-sale}{Terms
  of Sale}
\item
  \href{https://spiderbites.nytimes.com}{Site Map}
\item
  \href{https://help.nytimes.com/hc/en-us}{Help}
\item
  \href{https://www.nytimes.com/subscription?campaignId=37WXW}{Subscriptions}
\end{itemize}
