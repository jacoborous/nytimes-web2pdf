Sections

SEARCH

\protect\hyperlink{site-content}{Skip to
content}\protect\hyperlink{site-index}{Skip to site index}

\href{https://www.nytimes.com/section/politics}{Politics}

\href{https://myaccount.nytimes.com/auth/login?response_type=cookie\&client_id=vi}{}

\href{https://www.nytimes.com/section/todayspaper}{Today's Paper}

\href{/section/politics}{Politics}\textbar{}How Trump and the Black
Lives Matter Movement Changed White Voters' Minds

\url{https://nyti.ms/3g66DvY}

\begin{itemize}
\item
\item
\item
\item
\item
\end{itemize}

\begin{itemize}
\item
  \href{https://www.nytimes.com/2020/07/31/us/elections/biden-vs-trump.html?action=click\&pgtype=Article\&state=default\&region=TOP_BANNER\&context=storylines_menu}{Election
  Updates}
\item
  \href{https://www.nytimes.com/article/biden-vice-president-2020.html?action=click\&pgtype=Article\&state=default\&region=TOP_BANNER\&context=storylines_menu}{Biden's
  V.P. Search}
\item
  \href{https://www.nytimes.com/interactive/2020/07/24/us/politics/trump-biden-campaign-donors.html?action=click\&pgtype=Article\&state=default\&region=TOP_BANNER\&context=storylines_menu}{Map
  of Donations}
\item
  \href{https://www.nytimes.com/interactive/2020/us/elections/delegate-count-primary-results.html?action=click\&pgtype=Article\&state=default\&region=TOP_BANNER\&context=storylines_menu}{Delegate
  Count}
\item
  \href{https://www.nytimes.com/interactive/2019/us/politics/2020-presidential-candidates.html?action=click\&pgtype=Article\&state=default\&region=TOP_BANNER\&context=storylines_menu}{The
  Candidates}
\item
  \href{https://www.nytimes.com/newsletters/politics?action=click\&pgtype=Article\&state=default\&region=TOP_BANNER\&context=storylines_menu}{Politics
  Newsletter}
\end{itemize}

Advertisement

\protect\hyperlink{after-top}{Continue reading the main story}

Supported by

\protect\hyperlink{after-sponsor}{Continue reading the main story}

\hypertarget{how-trump-and-the-black-lives-matter-movement-changed-white-voters-minds}{%
\section{How Trump and the Black Lives Matter Movement Changed White
Voters'
Minds}\label{how-trump-and-the-black-lives-matter-movement-changed-white-voters-minds}}

A majority of American voters support demonstrations on police
brutality, and many see the president as ill-equipped on racial justice.

\includegraphics{https://static01.nyt.com/images/2020/06/27/us/politics/27UNREST-POLL1/merlin_173730312_44880e61-1bee-4baa-98d9-2a0db6e9b383-articleLarge.jpg?quality=75\&auto=webp\&disable=upscale}

\href{https://www.nytimes.com/by/astead-w-herndon}{\includegraphics{https://static01.nyt.com/images/2018/09/14/us/author-head-astead/author-head-astead-thumbLarge-v2.png}}\href{https://www.nytimes.com/by/dionne-searcey}{\includegraphics{https://static01.nyt.com/images/2018/10/15/multimedia/author-dionne-searcey/author-dionne-searcey-thumbLarge-v2.png}}

By \href{https://www.nytimes.com/by/astead-w-herndon}{Astead W. Herndon}
and \href{https://www.nytimes.com/by/dionne-searcey}{Dionne Searcey}

\begin{itemize}
\item
  Published June 27, 2020Updated July 3, 2020
\item
  \begin{itemize}
  \item
  \item
  \item
  \item
  \item
  \end{itemize}
\end{itemize}

A majority of American voters support the demonstrations against police
brutality and racial injustice that have roiled the country over the
past month, embracing ideas about bias within the criminal justice
system and the persistence of systemic racism that are central tenets of
the Black Lives Matter movement, according to a new national poll of
registered voters by The New York Times and Siena College.

Fifty-nine percent of voters, including 52 percent of white voters,
believe the death of George Floyd at the hands of the police in
Minneapolis was ``part of a broader pattern of excessive police violence
toward African Americans,'' the poll found. The Black Lives Matter
movement and the police had similar favorability ratings, with 44
percent of registered voters having a ``very favorable'' view of the
movement, almost identical to the 43 percent rating for the police.

The numbers add to the mounting evidence that recent protests have
significantly shifted public opinion on race, creating potential
political allies for a movement that was, within the past decade,
dismissed as fringe and divisive. It also highlights how President Trump
is increasingly out of touch with a country he is seeking to lead for a
second term: While he has shown little sympathy for the protesters and
their fight for racial justice, and has continued to use racist language
that many have denounced, voters feel favorably toward the protests and
their cause.

A survey of battleground states critical to November's election largely
mirrored the national results. Fifty-four percent of voters in those
states said the way the criminal justice system treats black Americans
was a bigger problem than the incidents of rioting seen during some
demonstrations. Just 37 percent said rioting was a bigger problem,
though Mr. Trump and his allies have tried to discredit the protests by
focusing on some isolated incidents of violence.

It has not worked.

``I probably didn't understand what bringing people together meant until
Trump started talking the way he does,'' said Rita Hopkins, 55, from
rural Clark County, Mo., in the northeastern part of the state. ``Now I
see what a president says can divide people.''

Ms. Hopkins, a white registered Democrat who describes herself as a
centrist, said she was particularly galled by Mr. Trump's comments at
one point during protests over Mr. Floyd's death that the Secret Service
had been prepared to sic the ``most vicious dogs'' on protesters outside
the White House gates.

The words immediately brought to mind photos of the
\href{https://www.washingtonpost.com/history/2020/06/01/trump-vicious-dogs-protesters-civil-rights-slavery/}{Alabama
police aiming snarling dogs at peaceful black protesters}.

``I hate to say it, but I had forgotten about those pictures I had
seen,'' said Ms. Hopkins, who lives in an overwhelmingly Republican
county. ``I kind of thought we had gotten past that.''

The attitudes cut across race, geography and educational status, and
speak to a country that has been awakened through protests to complaints
that black Americans have long made about police brutality and systemic
racism. What began in the Democratic primary, in which white liberals
showed a new openness to candidates speaking frankly about systemic
injustice, has continued into the general election, with a spotlight on
Mr. Trump's response.

\includegraphics{https://static01.nyt.com/images/2020/06/27/us/politics/27UNREST-POLL2/27UNREST-POLL2-articleLarge.jpg?quality=75\&auto=webp\&disable=upscale}

The coalition of people sympathetic to the protesters' cause, including
Latino voters, exposes the limits of Mr. Trump's tendency to exclusively
speak directly to his overwhelmingly white and conservative base. As
with other issues, including the coronavirus pandemic, the
administration's narrow focus has been derided by experts and voters,
who say the governing strategy does not reflect the country's broader
interests, or the current political realities.

``Over the past six years, so much of the work has been focused on
convincing the country --- and convincing policymakers and white
communities that there's an actual problem,'' said Samuel Sinyangwe, an
activist and co-founder of Mapping Police Violence. ``Now there's been
universal condemnation of the George Floyd incident and a recognition
that things needs to change.''

\hypertarget{latest-updates-2020-election}{%
\section{\texorpdfstring{\href{https://www.nytimes.com/2020/07/31/us/elections/biden-vs-trump.html?action=click\&pgtype=Article\&state=default\&region=MAIN_CONTENT_1\&context=storylines_live_updates}{Latest
Updates: 2020
Election}}{Latest Updates: 2020 Election}}\label{latest-updates-2020-election}}

Updated 2020-08-01T01:26:45.732Z

\begin{itemize}
\tightlist
\item
  \href{https://www.nytimes.com/2020/07/31/us/elections/biden-vs-trump.html?action=click\&pgtype=Article\&state=default\&region=MAIN_CONTENT_1\&context=storylines_live_updates\#link-29fdff45}{Kamala
  Harris, a top vice-presidential contender, confronts double
  standards.}
\item
  \href{https://www.nytimes.com/2020/07/31/us/elections/biden-vs-trump.html?action=click\&pgtype=Article\&state=default\&region=MAIN_CONTENT_1\&context=storylines_live_updates\#link-13ec3d9c}{Karen
  Bass and Susan Rice are rising on Biden's vice-presidential
  shortlist.}
\item
  \href{https://www.nytimes.com/2020/07/31/us/elections/biden-vs-trump.html?action=click\&pgtype=Article\&state=default\&region=MAIN_CONTENT_1\&context=storylines_live_updates\#link-49e9a016}{Trump
  says Russian bounties to kill U.S. troops `never took place.'}
\end{itemize}

\href{https://www.nytimes.com/2020/07/31/us/elections/biden-vs-trump.html?action=click\&pgtype=Article\&state=default\&region=MAIN_CONTENT_1\&context=storylines_live_updates}{See
more updates}

Aaron Perry, an alderman on the City Council of Waukesha, Wis., said he
doesn't support the looting that took place after Mr. Floyd's death in
various cities but said it occurred on a small scale relative to the
peaceful protests that broke out.

Mr. Perry, a 40-year-old white man, describes himself as a centrist but
was compelled to switch parties last year from Republican to Democrat
because of his support for marriage equality and legalization of
cannabis. He called himself a ``never-Trumper'' and said that in 2016,
he wrote in John Kasich, then the governor of Ohio, on the presidential
ballot.

The death of Mr. Floyd was an urgent message to the nation, he said, to
make changes to end the kinds of police and societal behavior that led
to the incident.

``This is the last time we have a chance to get this right. I'm on board
with that,'' Mr. Perry said, emphasizing, with an expletive, that he
really didn't care if ``most of the people I represent don't look like''
Mr. Floyd, and that issues of racial justice matter for a majority-white
area, too.

Darrell Keaton Sr., a 49-year-old black Democrat from Wausau, Wis.,
several hours north of Mr. Perry, said the protests after Mr. Floyd's
death were monumental for changing views on structural racism in
America. Finally, he said, it feels like white people are listening and
joining in the protests.

``We have just been racking our brains and screaming at the top of our
lungs for so many years that we're going to need other people to stand
up alongside the black community to change anything,'' he said.

Though the poll over all shows former Vice President Joseph R. Biden Jr.
in a very strong position, especially on racial justice, and voters'
belief in his ability to unite a divided country, it also indicates how
difficult a task that could be: More than 40 percent of white
respondents agreed in some measure that discrimination against whites
has become as big a problem as other forms of discrimination,
reinforcing a theme of white grievance politics that the president and
his supporters have long expressed.

There are also broad generational gaps between how voters are responding
to the national moment of unrest. Every age bracket said the use of
force by the police against black Americans was a bigger problem than
looting at demonstrations. But support for Black Lives Matter got more
tepid among older voters, the polls found. Sixty-seven percent of voters
ages 18 to 29 had a ``very favorable'' view of the Black Lives Matter
movement as did 54 percent of voters ages 30 to 44.

Among people 45 to 64, the support dropped to 37 percent, while 22
percent had a ``somewhat favorable'' view. Voters 65 and over were the
least persuaded: Only 31 percent had a ``very favorable'' view of the
Black Lives Matter movement, and 25 percent had a ``somewhat favorable''
opinion.

Michael Berlinger, 67, who lives in Lancaster, Pa., and considers
himself an independent voter, said he thought the Black Lives Matter
movement is too myopic. The protests over Mr. Floyd's death have been
too destructive, he said.

``The whole message has been undermined in a lot of ways,'' said Mr.
Berlinger, a white retired teacher. ``I'm not a big fan of people who
break the law to say they're working for a cause. I don't think that's
the correct way of doing it.''

The looting and the property destruction were ``a dilution of the
message and the results they wanted to achieve.'' Mr. Berlinger is
likely to vote for Mr. Trump, he said, but he described the choices on
the Republican and Democratic ballots, respectively, as one between ``a
lunatic and a senile senior citizen.''

``I think all lives matter,'' Mr. Berlinger said. ``The black and blue
lives, and red, white and blue lives.''

Charles Defever, a 28-year-old Minneapolis Democrat, said he felt this
was a moment to get more involved. His activism was limited to the
occasional comment in support of Black Lives Matter on social media ---
until he saw the video of George Floyd's arrest, he said.

``I was not very active, and my interest would fade,'' said Mr. Defever,
who is white and works as a produce buyer for a food wholesaler. ``I
would write on the Black Lives Matter page and see truth and pain and
hurt in so many people I know but would not go out and protest.''

``I've spent a lot of time at the State Capitol listening to young black
and brown youth speak about the world they want, and that's the world I
have,'' he said.

Mr. Defever was a supporter of Senator Bernie Sanders of Vermont but
said he planned to vote for Mr. Biden because he was ``better than
having destructive Republican policies and leadership.''

Many activists, progressive political groups, and civil rights
organizations draw a direct line to these changing attitudes and the
events of the recent months. Renewed attention ignited by the death of
Mr. Floyd --- as well as others who died at the hands of police,
including Breonna Taylor of Kentucky, Rayshard Brooks of Georgia, and
Elijah McClain of Colorado --- has built on other moments of awakening,
like the surprise of the 2016 election of Mr. Trump, said Nell Irvin
Painter, a historian and the author of ``The History of White People.''

``The great stall point after the civil rights movement was white people
not being able to talk to other white people about whiteness,'' Ms.
Painter said. ``That has to happen before anything can change. Now, many
white people are stepping up and saying, `Oh we've got to talk about
this.'''

D'Atra Jackson, national director for Black Youth Project 100, the
progressive political organization that has been on the front lines of
the national protests, agreed that this is a unique political moment.
However, Ms. Jackson said it is important to maintain pressure on people
seeking elected office so that public sympathy can be transformed into
action --- getting people elected and getting legislation passed.

``It's one thing to be hopeful and believe that new things are
possible,'' Ms. Jackson said. ``It's another thing to build power.''

\hypertarget{our-2020-election-guide}{%
\section{Our 2020 Election Guide}\label{our-2020-election-guide}}

Updated July 31, 2020

\begin{itemize}
\item
  \begin{center}\rule{0.5\linewidth}{\linethickness}\end{center}

  \hypertarget{the-latest}{%
  \subsection{The Latest}\label{the-latest}}

  \begin{itemize}
  \tightlist
  \item
    President Trump's assault on the Postal Service is intersecting with
    his attacks on mail-in voting.
    \href{https://www.nytimes.com/2020/07/31/us/politics/trump-usps-mail-delays.html?action=click\&pgtype=Article\&state=default\&region=BELOW_MAIN_CONTENT\&context=storylines_guide}{Voting
    rights groups say it is a recipe for disaster.}
  \end{itemize}
\item
  \begin{center}\rule{0.5\linewidth}{\linethickness}\end{center}

  \hypertarget{bidens-vp-search}{%
  \subsection{Biden's V.P. Search}\label{bidens-vp-search}}

  \begin{itemize}
  \tightlist
  \item
    \href{https://www.nytimes.com/article/biden-vice-president-2020.html?action=click\&pgtype=Article\&state=default\&region=BELOW_MAIN_CONTENT\&context=storylines_guide}{Here
    are 13 women} who have been under consideration to be Joe Biden's
    running mate, and why each might be chosen --- and might not be.
  \end{itemize}
\item
  \begin{center}\rule{0.5\linewidth}{\linethickness}\end{center}

  \hypertarget{keep-up-with-our-coverage}{%
  \subsection{Keep Up With Our
  Coverage}\label{keep-up-with-our-coverage}}

  \begin{itemize}
  \tightlist
  \item
    Get an
    \href{https://www.nytimes.com/newsletters/politics?action=click\&pgtype=Article\&state=default\&region=BELOW_MAIN_CONTENT\&context=storylines_guide}{email}
    recapping the day's news
  \end{itemize}

  \begin{itemize}
  \tightlist
  \item
    Download our mobile app on
    \href{https://apps.apple.com/us/app/nytimes/id284862083?ls=1\&mat_click_id=5c79ae7455014fd1bd66b5610c05b8f2-20191112-16948\&referrer=mat_click_id\%3D5c79ae7455014fd1bd66b5610c05b8f2-20191112-16948\%26link_click_id\%3D722930677036718082}{iOS}
    and
    \href{http://a.localytics.com/android?id=com.nytimes.android\&referrer=utm_source\%3Dother_nyt_mobile_web\%26utm_medium\%3DWeb\%2520page\%26utm_term\%3DGeneral\%2520Mobile\%2520Page\%26utm_campaign\%3DNYT\%2520Mobile\%2520General\%2520Page}{Android}
    and turn on Breaking News and Politics alerts
  \end{itemize}
\end{itemize}

Advertisement

\protect\hyperlink{after-bottom}{Continue reading the main story}

\hypertarget{site-index}{%
\subsection{Site Index}\label{site-index}}

\hypertarget{site-information-navigation}{%
\subsection{Site Information
Navigation}\label{site-information-navigation}}

\begin{itemize}
\tightlist
\item
  \href{https://help.nytimes.com/hc/en-us/articles/115014792127-Copyright-notice}{©~2020~The
  New York Times Company}
\end{itemize}

\begin{itemize}
\tightlist
\item
  \href{https://www.nytco.com/}{NYTCo}
\item
  \href{https://help.nytimes.com/hc/en-us/articles/115015385887-Contact-Us}{Contact
  Us}
\item
  \href{https://www.nytco.com/careers/}{Work with us}
\item
  \href{https://nytmediakit.com/}{Advertise}
\item
  \href{http://www.tbrandstudio.com/}{T Brand Studio}
\item
  \href{https://www.nytimes.com/privacy/cookie-policy\#how-do-i-manage-trackers}{Your
  Ad Choices}
\item
  \href{https://www.nytimes.com/privacy}{Privacy}
\item
  \href{https://help.nytimes.com/hc/en-us/articles/115014893428-Terms-of-service}{Terms
  of Service}
\item
  \href{https://help.nytimes.com/hc/en-us/articles/115014893968-Terms-of-sale}{Terms
  of Sale}
\item
  \href{https://spiderbites.nytimes.com}{Site Map}
\item
  \href{https://help.nytimes.com/hc/en-us}{Help}
\item
  \href{https://www.nytimes.com/subscription?campaignId=37WXW}{Subscriptions}
\end{itemize}
