Sections

SEARCH

\protect\hyperlink{site-content}{Skip to
content}\protect\hyperlink{site-index}{Skip to site index}

\href{https://www.nytimes.com/section/politics}{Politics}

\href{https://myaccount.nytimes.com/auth/login?response_type=cookie\&client_id=vi}{}

\href{https://www.nytimes.com/section/todayspaper}{Today's Paper}

\href{/section/politics}{Politics}\textbar{}Young Protesters Say Voting
Isn't Enough. Will They Do It Anyway?

\url{https://nyti.ms/2AGMOMy}

\begin{itemize}
\item
\item
\item
\item
\item
\end{itemize}

\begin{itemize}
\item
  \href{https://www.nytimes.com/2020/07/31/us/elections/biden-vs-trump.html?action=click\&pgtype=Article\&state=default\&region=TOP_BANNER\&context=storylines_menu}{Election
  Updates}
\item
  \href{https://www.nytimes.com/article/biden-vice-president-2020.html?action=click\&pgtype=Article\&state=default\&region=TOP_BANNER\&context=storylines_menu}{Biden's
  V.P. Search}
\item
  \href{https://www.nytimes.com/interactive/2020/07/24/us/politics/trump-biden-campaign-donors.html?action=click\&pgtype=Article\&state=default\&region=TOP_BANNER\&context=storylines_menu}{Map
  of Donations}
\item
  \href{https://www.nytimes.com/interactive/2020/us/elections/delegate-count-primary-results.html?action=click\&pgtype=Article\&state=default\&region=TOP_BANNER\&context=storylines_menu}{Delegate
  Count}
\item
  \href{https://www.nytimes.com/interactive/2019/us/politics/2020-presidential-candidates.html?action=click\&pgtype=Article\&state=default\&region=TOP_BANNER\&context=storylines_menu}{The
  Candidates}
\item
  \href{https://www.nytimes.com/newsletters/politics?action=click\&pgtype=Article\&state=default\&region=TOP_BANNER\&context=storylines_menu}{Politics
  Newsletter}
\end{itemize}

Advertisement

\protect\hyperlink{after-top}{Continue reading the main story}

Supported by

\protect\hyperlink{after-sponsor}{Continue reading the main story}

\hypertarget{young-protesters-say-voting-isnt-enough-will-they-do-it-anyway}{%
\section{Young Protesters Say Voting Isn't Enough. Will They Do It
Anyway?}\label{young-protesters-say-voting-isnt-enough-will-they-do-it-anyway}}

``Don't boo --- vote,'' has been Barack Obama's mantra. Now, Democrats
want to adapt it: Protest, then vote.

\includegraphics{https://static01.nyt.com/images/2020/06/06/us/politics/06voting1/merlin_146267856_d2e59313-e560-4248-9e3d-e0c8d7255576-articleLarge.jpg?quality=75\&auto=webp\&disable=upscale}

\href{https://www.nytimes.com/by/matt-flegenheimer}{\includegraphics{https://static01.nyt.com/images/2018/10/02/multimedia/author-matt-flegenheimer/author-matt-flegenheimer-thumbLarge.png}}

By \href{https://www.nytimes.com/by/matt-flegenheimer}{Matt
Flegenheimer}

\begin{itemize}
\item
  June 6, 2020
\item
  \begin{itemize}
  \item
  \item
  \item
  \item
  \item
  \end{itemize}
\end{itemize}

Barack Obama has a favorite saying on the campaign trail: ``Don't boo
--- vote.''

And young protesters, galvanized by police brutality and a rash of
political disappointments, seem to be sketching out a present-day
response:

Sure, maybe. But first, some well-directed fury.

``I'm tired. I'm literally tired. I'm tired of having to do this,'' said
Aalayah Eastmond, 19, who survived the 2018 massacre at her high school
in Parkland, Fla., became a gun control advocate, saw many legislative
efforts stall --- and is now organizing protests in Washington over
police violence against fellow black Americans.

Ms. Eastmond could be forgiven, she suggested, for doubting that the
electoral system would meet the moment on its own: ``We do our job,''
she said, ``and then we don't see the people we vote in doing their
job.''

As nationwide demonstrations continue to simmer, interviews with
millennial and Generation Z protesters and activists across racial lines
reflect a steady suspicion about the value and effectiveness of voting
alone. Their disillusionment threatens to perpetuate a consistent
generational gap in election turnout, hinting at a key challenge facing
Joseph R. Biden Jr. The former vice president, who announced Friday
evening that he had earned a majority of delegates in the Democratic
primary contest, has struggled to generate youth enthusiasm despite the
demographic's broad disapproval of President Trump.

To some degree, this dynamic has figured in political fights across the
decades: Voters are disproportionately old; marchers are
disproportionately young. (Even in the
\href{https://www.census.gov/library/stories/2019/04/behind-2018-united-states-midterm-election-turnout.html}{2018
midterms}, when youth engagement spiked compared with four years prior,
turnout registered at about 36 percent for voting-age citizens under 30
and nearly twice that for those 65 and up, according to Census Bureau
data.)

But the frustrations of today's younger Americans also speak to the
particular conditions of the era, with a preferred candidate in the last
two Democratic presidential primaries, Senator Bernie Sanders of
Vermont, falling short twice and a sense that those in office have done
little to stem a flood of crises.

The deaths of black people at the hands of law enforcement. The
relentless creep of climate change. Recurring economic uncertainty ---
this time amid a pandemic exacerbated by missteps across the federal
government.

\hypertarget{latest-updates-2020-election}{%
\section{\texorpdfstring{\href{https://www.nytimes.com/2020/07/31/us/elections/biden-vs-trump.html?action=click\&pgtype=Article\&state=default\&region=MAIN_CONTENT_1\&context=storylines_live_updates}{Latest
Updates: 2020
Election}}{Latest Updates: 2020 Election}}\label{latest-updates-2020-election}}

Updated 2020-08-01T01:26:45.732Z

\begin{itemize}
\tightlist
\item
  \href{https://www.nytimes.com/2020/07/31/us/elections/biden-vs-trump.html?action=click\&pgtype=Article\&state=default\&region=MAIN_CONTENT_1\&context=storylines_live_updates\#link-29fdff45}{Kamala
  Harris, a top vice-presidential contender, confronts double
  standards.}
\item
  \href{https://www.nytimes.com/2020/07/31/us/elections/biden-vs-trump.html?action=click\&pgtype=Article\&state=default\&region=MAIN_CONTENT_1\&context=storylines_live_updates\#link-13ec3d9c}{Karen
  Bass and Susan Rice are rising on Biden's vice-presidential
  shortlist.}
\item
  \href{https://www.nytimes.com/2020/07/31/us/elections/biden-vs-trump.html?action=click\&pgtype=Article\&state=default\&region=MAIN_CONTENT_1\&context=storylines_live_updates\#link-49e9a016}{Trump
  says Russian bounties to kill U.S. troops `never took place.'}
\end{itemize}

\href{https://www.nytimes.com/2020/07/31/us/elections/biden-vs-trump.html?action=click\&pgtype=Article\&state=default\&region=MAIN_CONTENT_1\&context=storylines_live_updates}{See
more updates}

``In an ideal world, all of these issues would be solved by going out
and voting,'' said Zoe Demkovitz, 27, who had supported Mr. Sanders's
presidential campaign, as she marched against police violence in
Philadelphia. ``I tried that. I voted for the right people.''

``And this,'' she concluded, adding an expletive, ``still happens.''

Democratic leaders are plainly aware of this perception and mindful that
a stronger showing from Hillary Clinton among young voters four years
ago
\href{https://www.npr.org/2016/11/14/501727488/millennials-just-didnt-love-hillary-clinton-the-way-they-loved-barack-obama}{probably
would have} turned her fortunes.

Some have moved in recent days to explicitly urge protesters not to
overlook November.

In a
\href{https://medium.com/@BarackObama/how-to-make-this-moment-the-turning-point-for-real-change-9fa209806067}{post
on Medium}, Mr. Obama disputed the notion that racial bias in criminal
justice ``proves that only protests and direct action can bring about
change, and that voting and participation in electoral politics is a
waste of time.''

``Eventually, \emph{aspirations have to be translated into specific laws
and institutional practices},'' the former president wrote, italicizing
liberally, ``and in a democracy, that only happens when we elect
government officials who are responsive to our demands.''

Representative James E. Clyburn of South Carolina, the highest-ranking
African-American in Congress, suggested that protests were so valuable
in part because they helped introduce new leaders to old systems. At 79,
Mr. Clyburn
\href{https://www.washingtonpost.com/history/2020/01/10/clyburn-recounts-1960-meet-cute-with-his-future-wife-jail/}{still
delights} in reminding audiences that he met his wife in jail after a
civil rights march in 1960.

``I stayed involved,'' Mr. Clyburn said, ``and I'm now in the United
States Congress.''

Some younger protesters do not dismiss this prospective path --- or the
wisdom of voting, however grudgingly.

But they say several of the most stinging policy letdowns in recent
years have come after nominal election successes.

In New York, Mayor Bill de Blasio won office in 2013 with a pledge to
dramatically reform the city's police culture, memorably showcasing his
biracial family throughout his campaign. Through a recent stretch of
demonstrations that included the arrest of his own daughter, Mr. de
Blasio has largely defended the department's approach despite news
accounts and videos of officers responding to peaceful protests with
often striking aggression.

``The mayor's transformation has been so pronounced that I have trouble
wrapping my head around it,'' said Ritchie Torres, 32, a Bronx city
councilman now running for Congress.

\includegraphics{https://static01.nyt.com/images/2020/06/06/us/politics/06voting2/merlin_157886589_5c730897-94ce-48f6-a87e-117bb16af780-articleLarge.jpg?quality=75\&auto=webp\&disable=upscale}

For younger New Yorkers, he said, it was a reminder that electing
ostensibly like-minded leadership was not enough. ``Young people rightly
and clearly see the limitations of voting,'' he said, calling it ``a
necessary but insufficient condition for political engagement.''

Even Mr. Obama's White House tenure, made possible in large part by his
strength with younger voters, has come in for mixed appraisals.

Evan Weber, 28, the political director for the Sunrise Movement, a group
of young liberal environmental activists, cited the dissatisfaction
among progressives his age over Mr. Obama's record on financial reform
and some climate issues. ``People are turning to protest out of
necessity,'' Mr. Weber said. ``We have grown up --- millennials and
especially Generation Z --- with a system that has either delivered too
little or not at all.''

People of color have signaled a particular weariness with the
implication that voting is a cure-all, especially given the scale of
voter suppression efforts and other barriers to the ballot.

Jess Morales Rocketto, 33, a progressive strategist and former campaign
aide to Mr. Obama and Mrs. Clinton, said the standard get-out-and-vote
message tended to sound most palatable to people who were planning to
vote anyway.

``What we're really wrestling with is not whether or not people vote but
whether people believe institutions matter,'' she said. ``That
disillusionment is actually about the fight for a generation of civic
participation.''

Image

Jess Morales Rocketto, left, worked for campaigns of President Barack
Obama and Hillary Clinton.Credit...Audra Melton for The New York Times

On that score, some academics say, the protests might help.

Daniel Q. Gillion, a professor of political science at the University of
Pennsylvania, said that his research --- detailed in a recent book,
``The Loud Minority,'' about the importance of demonstrations since the
1960s --- showed that areas with meaningful protest activity often saw
increased turnout in subsequent elections.

Whether younger Americans find a candidate to believe in is another
matter. Jason Culler, 38, who also attended the march in Philadelphia,
predicted that the current election cycle would not produce leaders who
adequately reflected the crowds filling the streets.

``Not this election, not the Democratic Party, not the Republican
Party,'' he said. ``These people don't represent us, that's why we're
out here still fighting the same thing.''

If nothing else, such persistence has proved a point, especially for
certain participants.

Ms. Eastmond, the Parkland survivor, recalled the skepticism two years
ago that she and other teens
\href{https://www.nytimes.com/2018/04/01/us/politics/gun-control-marches-protests.html}{stirred
to action} by the shooting would remain as engaged in political activism
as the months passed.

She does not hear those doubts so much anymore.

``People were questioning: `A lot of the people in that movement, where
are they now?''' she said. ``I'm here. I'm just one person, but I'm
here.''

Jon Hurdle contributed reporting from Philadelphia, and Isabella Grullón
Paz from New York.

\hypertarget{our-2020-election-guide}{%
\section{Our 2020 Election Guide}\label{our-2020-election-guide}}

Updated July 31, 2020

\begin{itemize}
\item
  \begin{center}\rule{0.5\linewidth}{\linethickness}\end{center}

  \hypertarget{the-latest}{%
  \subsection{The Latest}\label{the-latest}}

  \begin{itemize}
  \tightlist
  \item
    President Trump's assault on the Postal Service is intersecting with
    his attacks on mail-in voting.
    \href{https://www.nytimes.com/2020/07/31/us/politics/trump-usps-mail-delays.html?action=click\&pgtype=Article\&state=default\&region=BELOW_MAIN_CONTENT\&context=storylines_guide}{Voting
    rights groups say it is a recipe for disaster.}
  \end{itemize}
\item
  \begin{center}\rule{0.5\linewidth}{\linethickness}\end{center}

  \hypertarget{bidens-vp-search}{%
  \subsection{Biden's V.P. Search}\label{bidens-vp-search}}

  \begin{itemize}
  \tightlist
  \item
    \href{https://www.nytimes.com/article/biden-vice-president-2020.html?action=click\&pgtype=Article\&state=default\&region=BELOW_MAIN_CONTENT\&context=storylines_guide}{Here
    are 13 women} who have been under consideration to be Joe Biden's
    running mate, and why each might be chosen --- and might not be.
  \end{itemize}
\item
  \begin{center}\rule{0.5\linewidth}{\linethickness}\end{center}

  \hypertarget{keep-up-with-our-coverage}{%
  \subsection{Keep Up With Our
  Coverage}\label{keep-up-with-our-coverage}}

  \begin{itemize}
  \tightlist
  \item
    Get an
    \href{https://www.nytimes.com/newsletters/politics?action=click\&pgtype=Article\&state=default\&region=BELOW_MAIN_CONTENT\&context=storylines_guide}{email}
    recapping the day's news
  \end{itemize}

  \begin{itemize}
  \tightlist
  \item
    Download our mobile app on
    \href{https://apps.apple.com/us/app/nytimes/id284862083?ls=1\&mat_click_id=5c79ae7455014fd1bd66b5610c05b8f2-20191112-16948\&referrer=mat_click_id\%3D5c79ae7455014fd1bd66b5610c05b8f2-20191112-16948\%26link_click_id\%3D722930677036718082}{iOS}
    and
    \href{http://a.localytics.com/android?id=com.nytimes.android\&referrer=utm_source\%3Dother_nyt_mobile_web\%26utm_medium\%3DWeb\%2520page\%26utm_term\%3DGeneral\%2520Mobile\%2520Page\%26utm_campaign\%3DNYT\%2520Mobile\%2520General\%2520Page}{Android}
    and turn on Breaking News and Politics alerts
  \end{itemize}
\end{itemize}

Advertisement

\protect\hyperlink{after-bottom}{Continue reading the main story}

\hypertarget{site-index}{%
\subsection{Site Index}\label{site-index}}

\hypertarget{site-information-navigation}{%
\subsection{Site Information
Navigation}\label{site-information-navigation}}

\begin{itemize}
\tightlist
\item
  \href{https://help.nytimes.com/hc/en-us/articles/115014792127-Copyright-notice}{©~2020~The
  New York Times Company}
\end{itemize}

\begin{itemize}
\tightlist
\item
  \href{https://www.nytco.com/}{NYTCo}
\item
  \href{https://help.nytimes.com/hc/en-us/articles/115015385887-Contact-Us}{Contact
  Us}
\item
  \href{https://www.nytco.com/careers/}{Work with us}
\item
  \href{https://nytmediakit.com/}{Advertise}
\item
  \href{http://www.tbrandstudio.com/}{T Brand Studio}
\item
  \href{https://www.nytimes.com/privacy/cookie-policy\#how-do-i-manage-trackers}{Your
  Ad Choices}
\item
  \href{https://www.nytimes.com/privacy}{Privacy}
\item
  \href{https://help.nytimes.com/hc/en-us/articles/115014893428-Terms-of-service}{Terms
  of Service}
\item
  \href{https://help.nytimes.com/hc/en-us/articles/115014893968-Terms-of-sale}{Terms
  of Sale}
\item
  \href{https://spiderbites.nytimes.com}{Site Map}
\item
  \href{https://help.nytimes.com/hc/en-us}{Help}
\item
  \href{https://www.nytimes.com/subscription?campaignId=37WXW}{Subscriptions}
\end{itemize}
