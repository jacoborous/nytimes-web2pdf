Sections

SEARCH

\protect\hyperlink{site-content}{Skip to
content}\protect\hyperlink{site-index}{Skip to site index}

\href{https://www.nytimes.com/section/politics}{Politics}

\href{https://myaccount.nytimes.com/auth/login?response_type=cookie\&client_id=vi}{}

\href{https://www.nytimes.com/section/todayspaper}{Today's Paper}

\href{/section/politics}{Politics}\textbar{}Pentagon Ordered National
Guard Helicopters' Aggressive Response in D.C.

\url{https://nyti.ms/2MwBF3w}

\begin{itemize}
\item
\item
\item
\item
\item
\end{itemize}

\href{https://www.nytimes.com/news-event/george-floyd-protests-minneapolis-new-york-los-angeles?action=click\&pgtype=Article\&state=default\&region=TOP_BANNER\&context=storylines_menu}{Race
and America}

\begin{itemize}
\tightlist
\item
  \href{https://www.nytimes.com/2020/07/26/us/protests-portland-seattle-trump.html?action=click\&pgtype=Article\&state=default\&region=TOP_BANNER\&context=storylines_menu}{Protesters
  Return to Other Cities}
\item
  \href{https://www.nytimes.com/2020/07/24/us/portland-oregon-protests-white-race.html?action=click\&pgtype=Article\&state=default\&region=TOP_BANNER\&context=storylines_menu}{Portland
  at the Center}
\item
  \href{https://www.nytimes.com/2020/07/23/podcasts/the-daily/portland-protests.html?action=click\&pgtype=Article\&state=default\&region=TOP_BANNER\&context=storylines_menu}{Podcast:
  Showdown in Portland}
\item
  \href{https://www.nytimes.com/interactive/2020/07/16/us/black-lives-matter-protests-louisville-breonna-taylor.html?action=click\&pgtype=Article\&state=default\&region=TOP_BANNER\&context=storylines_menu}{45
  Days in Louisville}
\end{itemize}

Advertisement

\protect\hyperlink{after-top}{Continue reading the main story}

Supported by

\protect\hyperlink{after-sponsor}{Continue reading the main story}

\hypertarget{pentagon-ordered-national-guard-helicopters-aggressive-response-in-dc}{%
\section{Pentagon Ordered National Guard Helicopters' Aggressive
Response in
D.C.}\label{pentagon-ordered-national-guard-helicopters-aggressive-response-in-dc}}

The high-profile episode, after days of protests in Washington, was a
turning point in the military's response to unrest in the city.

\includegraphics{https://static01.nyt.com/images/2020/06/06/us/politics/06dc-helicopter/06dc-helicopter-articleLarge.jpg?quality=75\&auto=webp\&disable=upscale}

\href{https://www.nytimes.com/by/thomas-gibbons-neff}{\includegraphics{https://static01.nyt.com/images/2018/07/12/multimedia/author-thomas-gibbons-neff/author-thomas-gibbons-neff-thumbLarge.png}}\href{https://www.nytimes.com/by/eric-schmitt}{\includegraphics{https://static01.nyt.com/images/2018/06/12/multimedia/author-eric-schmitt/author-eric-schmitt-thumbLarge-v2.png}}

By \href{https://www.nytimes.com/by/thomas-gibbons-neff}{Thomas
Gibbons-Neff} and \href{https://www.nytimes.com/by/eric-schmitt}{Eric
Schmitt}

\begin{itemize}
\item
  June 6, 2020
\item
  \begin{itemize}
  \item
  \item
  \item
  \item
  \item
  \end{itemize}
\end{itemize}

WASHINGTON --- Top Pentagon officials ordered National Guard helicopters
to use what they called ``persistent presence'' to disperse protests in
the capital this week, according to military officials. The loosely
worded order prompted a series of low-altitude maneuvers that human
rights organizations quickly criticized as a show of force usually
reserved for combat zones.

Ryan D. McCarthy, the Army secretary and one of the officials who
authorized part of the planning for the helicopters' mission Monday
night, said on Friday that the Army had opened an investigation into the
episode.

Two Army National Guard helicopters flew low over the protesters, with
the downward blast from their rotor blades sending protesters scurrying
for cover and ripping signs from the sides of buildings. The pilots of
one of the helicopters have been grounded pending the outcome of the
inquiry.

The high-profile episode, after days of protests in Washington --- some
of which turned violent --- was a turning point in the military's
response to unrest in the city. After days of operating on the periphery
of the crowds, National Guard forces suddenly became a focus of the
controversy over the military's role in urban law enforcement.

Military officials said that the National Guard's aggressive approach to
crowd control was prompted by a pointed threat from the Pentagon: If the
Guard was unable to handle the situation, then active-duty military
units, such as a rapid-reaction unit of the 82nd Airborne Division,
would be sent into the city.

Senior Pentagon officials, including Gen. Mark A. Milley, the chairman
of the Joint Chiefs of Staff,
\href{https://www.nytimes.com/2020/06/05/us/politics/protests-milley-trump.html}{were
trying to persuade President Trump} that active-duty troops should not
be sent into the streets to impose order, and that law enforcement and
National Guard personnel could contain the level of unrest.

On Monday night, both Mr. McCarthy and the Army's chief of staff, Gen.
James C. McConville, pressed Maj. Gen. William J. Walker, the commanding
general of the District of Columbia National Guard, to increase his
forces' presence in the city, according to a senior Defense Department
official.

An Army official declined to comment, saying that the investigation was
continuing.

The episode has stirred outrage among lawmakers. ``What we saw on Monday
night was our military using its equipment to threaten and put Americans
at risk on American soil,'' said Senator Tammy Duckworth, an Illinois
Democrat and former Army Black Hawk pilot.

Documents obtained by The New York Times show that planning for the
National Guard mission included oversight by Mr. McCarthy and General
McConville. The operation had been reviewed by a judge advocate team ---
military lawyers --- before aviation units were instructed to apply
``persistent presence.'' These types of maneuvers are well known to Mr.
McCarthy, who served in the Army's elite Ranger Regiment during the
opening operations of the war in Afghanistan.

The episode, which occurred about three hours after a 7 p.m. curfew in
the capital went into effect on Monday, began when a Black Hawk
helicopter, assigned to the District of Columbia National Guard, began a
low and slow pass over a group of roughly 200 peaceful protesters in the
Chinatown neighborhood.

The downward force of the helicopter's rotor blades snapped a small
tree, with debris almost hitting several people. The second helicopter
tried a similar maneuver. Roaring overhead, the Lakota, adorned with a
red-and-white cross denoting its medical affiliation, hovered over the
crowd, staying at rooftop level, blowing debris and sending protesters
scattering.

The red cross with white background is a ``universally recognized symbol
of medical aid and is protected under the Geneva Conventions,'' Human
Rights Watch said in
\href{https://www.hrw.org/news/2020/06/05/reckless-use-us-helicopters-intimidate-protesters}{a
report Friday}. ``Its misuse is prohibited under the conventions and it
has no place in a `show of force' or to forcibly disperse protesters.''

``The wind speeds created by a low-hovering helicopter can lift objects
and cause serious damage, potentially leading to injury or death,'' the
report said. ``These risks are amplified in congested urban
environments, where the consequences would be exceptionally dangerous if
something were to go wrong.''

On Wednesday, Defense Secretary Mark T. Esper said that he was told that
the helicopters had been asked by law enforcement to look at a
checkpoint to ``see if there were protesters around.''

``We need to let the Army conduct its inquiry and get back and see what
the facts actually are,'' Mr. Esper told reporters. The District of
Columbia National Guard is the only unit of the Guard that reports
directly to the president because of the capital's unusual political
status --- it has no governor, who usually commands the units.

A District of Columbia National Guard spokeswoman declined to comment,
citing the open investigation.

\href{https://www.nytimes.com/2020/06/02/us/politics/military-national-guard-trump-protests.html}{During
the operation Monday night}, the helicopters followed the crowd through
several well-lit intersections and repeatedly hovered over protesters
for close to an hour.

People at the scene expressed their disbelief and fear. One protester,
asked by a friend if he wanted to stay out later, responded curtly that
he was just ``trying not to die.''

There is no formal training for the type of maneuvers conducted Monday
night, said one military official with direct knowledge of the episode,
so any guidance about ``persistent presence'' is left to the
interpretation of the pilots.

The use of a medical evacuation helicopter, the official added, appeared
to result from the fact that command levels of the District of Columbia
National Guard did not realize that the majority of the Lakota
helicopters available for law enforcement missions are deployed to the
Texas border for Customs and Border Protection missions there to halt
illegal immigration.

While many Army aviation units have the Red Cross symbol in a detachable
form, by way of magnets, the District of Columbia National Guard has the
cross painted on the airframes of its helicopters since they are so
often involved in a patient transfer program that moves people among
routine, urgent and critical care facilities in the Washington area, the
official added.

The unit responsible for Monday's episode performed a lifesaving
transfer mission the next day, transporting a deteriorating patient from
Ft. Belvoir's community hospital in Virginia to Walter Reed National
Military Medical Center in Bethesda, Md., the official said. On Friday,
The Washington Post
\href{https://www.washingtonpost.com/national-security/pentagon-disarms-guardsmen-in-washington-dc-in-signal-of-de-escalation/2020/06/05/324da91a-a733-11ea-8681-7d471bf20207_story.html}{reported}
that all District of Columbia National Guard helicopter operations had
been suspended pending the results of the investigation, although it was
unclear if that affects medical patient transfers.

Advertisement

\protect\hyperlink{after-bottom}{Continue reading the main story}

\hypertarget{site-index}{%
\subsection{Site Index}\label{site-index}}

\hypertarget{site-information-navigation}{%
\subsection{Site Information
Navigation}\label{site-information-navigation}}

\begin{itemize}
\tightlist
\item
  \href{https://help.nytimes.com/hc/en-us/articles/115014792127-Copyright-notice}{©~2020~The
  New York Times Company}
\end{itemize}

\begin{itemize}
\tightlist
\item
  \href{https://www.nytco.com/}{NYTCo}
\item
  \href{https://help.nytimes.com/hc/en-us/articles/115015385887-Contact-Us}{Contact
  Us}
\item
  \href{https://www.nytco.com/careers/}{Work with us}
\item
  \href{https://nytmediakit.com/}{Advertise}
\item
  \href{http://www.tbrandstudio.com/}{T Brand Studio}
\item
  \href{https://www.nytimes.com/privacy/cookie-policy\#how-do-i-manage-trackers}{Your
  Ad Choices}
\item
  \href{https://www.nytimes.com/privacy}{Privacy}
\item
  \href{https://help.nytimes.com/hc/en-us/articles/115014893428-Terms-of-service}{Terms
  of Service}
\item
  \href{https://help.nytimes.com/hc/en-us/articles/115014893968-Terms-of-sale}{Terms
  of Sale}
\item
  \href{https://spiderbites.nytimes.com}{Site Map}
\item
  \href{https://help.nytimes.com/hc/en-us}{Help}
\item
  \href{https://www.nytimes.com/subscription?campaignId=37WXW}{Subscriptions}
\end{itemize}
