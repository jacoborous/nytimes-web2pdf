Sections

SEARCH

\protect\hyperlink{site-content}{Skip to
content}\protect\hyperlink{site-index}{Skip to site index}

\href{https://www.nytimes.com/section/politics}{Politics}

\href{https://myaccount.nytimes.com/auth/login?response_type=cookie\&client_id=vi}{}

\href{https://www.nytimes.com/section/todayspaper}{Today's Paper}

\href{/section/politics}{Politics}\textbar{}Trump Issues Executive Order
Targeting Vandalism Against Monuments

\url{https://nyti.ms/3dC01Ei}

\begin{itemize}
\item
\item
\item
\item
\item
\end{itemize}

\href{https://www.nytimes.com/news-event/george-floyd-protests-minneapolis-new-york-los-angeles?action=click\&pgtype=Article\&state=default\&region=TOP_BANNER\&context=storylines_menu}{Race
and America}

\begin{itemize}
\tightlist
\item
  \href{https://www.nytimes.com/2020/07/26/us/protests-portland-seattle-trump.html?action=click\&pgtype=Article\&state=default\&region=TOP_BANNER\&context=storylines_menu}{Protesters
  Return to Other Cities}
\item
  \href{https://www.nytimes.com/2020/07/24/us/portland-oregon-protests-white-race.html?action=click\&pgtype=Article\&state=default\&region=TOP_BANNER\&context=storylines_menu}{Portland
  at the Center}
\item
  \href{https://www.nytimes.com/2020/07/23/podcasts/the-daily/portland-protests.html?action=click\&pgtype=Article\&state=default\&region=TOP_BANNER\&context=storylines_menu}{Podcast:
  Showdown in Portland}
\item
  \href{https://www.nytimes.com/interactive/2020/07/16/us/black-lives-matter-protests-louisville-breonna-taylor.html?action=click\&pgtype=Article\&state=default\&region=TOP_BANNER\&context=storylines_menu}{45
  Days in Louisville}
\end{itemize}

Advertisement

\protect\hyperlink{after-top}{Continue reading the main story}

Supported by

\protect\hyperlink{after-sponsor}{Continue reading the main story}

\hypertarget{trump-issues-executive-order-targeting-vandalism-against-monuments}{%
\section{Trump Issues Executive Order Targeting Vandalism Against
Monuments}\label{trump-issues-executive-order-targeting-vandalism-against-monuments}}

The order also threatens to withhold funding from local governments that
fail to protect their own statues from vandals.

\includegraphics{https://static01.nyt.com/images/2020/06/26/us/politics/26dc-trump/merlin_173824434_c1acd490-2a10-4b1a-94cf-2b5fe7d372b7-articleLarge.jpg?quality=75\&auto=webp\&disable=upscale}

\href{https://www.nytimes.com/by/michael-d-shear}{\includegraphics{https://static01.nyt.com/images/2018/06/13/multimedia/author-michael-d-shear/author-michael-d-shear-thumbLarge-v2.png}}

By \href{https://www.nytimes.com/by/michael-d-shear}{Michael D. Shear}

\begin{itemize}
\item
  June 26, 2020
\item
  \begin{itemize}
  \item
  \item
  \item
  \item
  \item
  \end{itemize}
\end{itemize}

WASHINGTON --- President Trump issued an executive order on Friday that
instructed federal law enforcement authorities to prosecute people who
damage federal monuments or statues and that threatened to withhold
funding from local governments that fail to protect their own statues
from vandals.

The order, which Mr. Trump announced
\href{https://twitter.com/realDonaldTrump/status/1276633518433538049}{on
Twitter}, comes as he seeks to seize on a cultural divide in the United
States during his re-election campaign, suggesting that Democrats are
waging an assault on the nation's history.

``Anarchists and left-wing extremists have sought to advance a fringe
ideology that paints the United States of America as fundamentally
unjust,''
\href{https://www.whitehouse.gov/presidential-actions/executive-order-protecting-american-monuments-memorials-statues-combating-recent-criminal-violence/}{Mr.
Trump writes in the order}, which is titled, ``Protecting American
Monuments, Memorials and Statues and Combating Recent Criminal
Violence.''

The order adds: ``Key targets in the violent extremists' campaign
against our country are public monuments, memorials and statues.''

It is a response to the toppling of statues and monuments in recent
weeks after the
\href{https://www.nytimes.com/2020/05/31/us/george-floyd-investigation.html}{killing
of George Floyd in Minneapolis} prompted protests for police reform and
social justice.

But the order offers little in the way of new authority. It directs
federal law enforcement officials to prosecute ``to the fullest extent
permitted'' people who violate existing federal laws that already make
it a crime to damage or destroy a monument or statue.

The order also urges prosecution of anyone who is caught ``attacking,
removing or defacing depictions of Jesus or other religious figures or
religious artwork.''

Protesters across the country have knocked down monuments, mostly of
Confederate generals. In Raleigh, N.C., the statues of two Confederate
soldiers were torn down. And in San Francisco, a crowd toppled a bust of
Ulysses S. Grant, despite the fact that he was a Union general who beat
the Confederate Army. (Protesters noted that he was also a slave owner.)

In Washington, protesters knocked over a statue of Albert Pike, the only
Confederate general honored in the city, and they tried ---
unsuccessfully --- to take down a statue near the White House of Andrew
Jackson, the nation's seventh president.

In the order, Mr. Trump accuses local governments, like Washington's, of
having ``surrendered to mob rule, imperiling community safety, allowing
for the wholesale violation of our laws, and privileging the violent
impulses of the mob over the rights of law-abiding citizens.''

In an attempt to punish those governments that the president claims have
looked the other way during monument destruction, the order directs
officials to consider holding back funding and grants.

But it is unclear whether the Trump administration could actually follow
through on that threat. The president made a similar threat to withhold
funds from so-called sanctuary cities, which limit their cooperation
with federal immigration agents. This year, an appeals court blocked it.

Advertisement

\protect\hyperlink{after-bottom}{Continue reading the main story}

\hypertarget{site-index}{%
\subsection{Site Index}\label{site-index}}

\hypertarget{site-information-navigation}{%
\subsection{Site Information
Navigation}\label{site-information-navigation}}

\begin{itemize}
\tightlist
\item
  \href{https://help.nytimes.com/hc/en-us/articles/115014792127-Copyright-notice}{©~2020~The
  New York Times Company}
\end{itemize}

\begin{itemize}
\tightlist
\item
  \href{https://www.nytco.com/}{NYTCo}
\item
  \href{https://help.nytimes.com/hc/en-us/articles/115015385887-Contact-Us}{Contact
  Us}
\item
  \href{https://www.nytco.com/careers/}{Work with us}
\item
  \href{https://nytmediakit.com/}{Advertise}
\item
  \href{http://www.tbrandstudio.com/}{T Brand Studio}
\item
  \href{https://www.nytimes.com/privacy/cookie-policy\#how-do-i-manage-trackers}{Your
  Ad Choices}
\item
  \href{https://www.nytimes.com/privacy}{Privacy}
\item
  \href{https://help.nytimes.com/hc/en-us/articles/115014893428-Terms-of-service}{Terms
  of Service}
\item
  \href{https://help.nytimes.com/hc/en-us/articles/115014893968-Terms-of-sale}{Terms
  of Sale}
\item
  \href{https://spiderbites.nytimes.com}{Site Map}
\item
  \href{https://help.nytimes.com/hc/en-us}{Help}
\item
  \href{https://www.nytimes.com/subscription?campaignId=37WXW}{Subscriptions}
\end{itemize}
