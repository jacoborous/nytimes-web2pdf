Sections

SEARCH

\protect\hyperlink{site-content}{Skip to
content}\protect\hyperlink{site-index}{Skip to site index}

\href{https://www.nytimes.com/section/us}{U.S.}

\href{https://myaccount.nytimes.com/auth/login?response_type=cookie\&client_id=vi}{}

\href{https://www.nytimes.com/section/todayspaper}{Today's Paper}

\href{/section/us}{U.S.}\textbar{}From Cosmetics to NASCAR, Calls for
Racial Justice Are Spreading

\url{https://nyti.ms/2YvtWse}

\begin{itemize}
\item
\item
\item
\item
\item
\item
\end{itemize}

\href{https://www.nytimes.com/news-event/george-floyd-protests-minneapolis-new-york-los-angeles?action=click\&pgtype=Article\&state=default\&region=TOP_BANNER\&context=storylines_menu}{Race
and America}

\begin{itemize}
\tightlist
\item
  \href{https://www.nytimes.com/2020/07/26/us/protests-portland-seattle-trump.html?action=click\&pgtype=Article\&state=default\&region=TOP_BANNER\&context=storylines_menu}{Protesters
  Return to Other Cities}
\item
  \href{https://www.nytimes.com/2020/07/24/us/portland-oregon-protests-white-race.html?action=click\&pgtype=Article\&state=default\&region=TOP_BANNER\&context=storylines_menu}{Portland
  at the Center}
\item
  \href{https://www.nytimes.com/2020/07/23/podcasts/the-daily/portland-protests.html?action=click\&pgtype=Article\&state=default\&region=TOP_BANNER\&context=storylines_menu}{Podcast:
  Showdown in Portland}
\item
  \href{https://www.nytimes.com/interactive/2020/07/16/us/black-lives-matter-protests-louisville-breonna-taylor.html?action=click\&pgtype=Article\&state=default\&region=TOP_BANNER\&context=storylines_menu}{45
  Days in Louisville}
\end{itemize}

Advertisement

\protect\hyperlink{after-top}{Continue reading the main story}

Supported by

\protect\hyperlink{after-sponsor}{Continue reading the main story}

\hypertarget{from-cosmetics-to-nascar-calls-for-racial-justice-are-spreading}{%
\section{From Cosmetics to NASCAR, Calls for Racial Justice Are
Spreading}\label{from-cosmetics-to-nascar-calls-for-racial-justice-are-spreading}}

What started as a renewed push for police reform has now touched
seemingly every aspect of American life.

\includegraphics{https://static01.nyt.com/images/2020/06/12/us/00UNREST-RECKONING-top/merlin_173350512_63361a86-b944-4671-9521-b35cd294351f-articleLarge.jpg?quality=75\&auto=webp\&disable=upscale}

By \href{https://www.nytimes.com/by/amy-harmon}{Amy Harmon},
\href{https://www.nytimes.com/by/apoorva-mandavilli}{Apoorva
Mandavilli}, \href{https://www.nytimes.com/by/sapna-maheshwari}{Sapna
Maheshwari} and \href{https://www.nytimes.com/by/jodi-kantor}{Jodi
Kantor}

\begin{itemize}
\item
  Published June 13, 2020Updated June 16, 2020
\item
  \begin{itemize}
  \item
  \item
  \item
  \item
  \item
  \item
  \end{itemize}
\end{itemize}

\href{https://www.nytimes.com/es/2020/06/15/espanol/mundo/racismo-george-floyd-protestas.html}{Leer
en español}

The reckonings have been swift and dizzying.

On Monday, it was the dictionary, with Merriam-Webster
\href{https://www.nytimes.com/2020/06/10/us/merriam-webster-racism-definition.html}{saying}
it was revising its entry on racism to illustrate the ways in which it
``can be systemic.''

On Tuesday, the University of Washington
\href{https://www.seattletimes.com/sports/uw-huskies/uw-removes-dance-coach-asks-black-members-to-rejoin-team-and-pledges-to-include-diversity-in-tryout-process/?utm_medium=notification\&utm_source=pushly\&utm_campaign=564808}{removed
the coach of its dance team} after the only two black members of the
group were cut. The two women were invited to return.

On Wednesday, after a black racecar driver called on NASCAR to ban the
\href{https://www.nytimes.com/news-event/confederate-flags-monuments-statues}{Confederate
battle flag} from its events, the organization did just that.

On Thursday, Nike joined a wave of American companies that have made
Juneteenth, which celebrates the end of slavery in America,
\href{https://www.nytimes.com/aponline/2020/06/12/business/ap-juneteenth-company-holiday.html}{an
official paid holiday}, ``to better commemorate and celebrate Black
history and culture.''

And on Friday, ABC Entertainment named the franchise's
\href{https://www.nytimes.com/2020/06/12/arts/television/matt-james-black-bachelor.html}{first
black man to star in ``The Bachelor}'' in the show's 18-year history,
acceding to longstanding demands from fans.

In just under three weeks since the killing of George Floyd set off
widespread protests, what started as a renewed demand for police reform
has now roiled seemingly every sphere of American life, prompting
institutions and individuals around the country to confront enduring
forms of racial discrimination.

Many black Americans have been
\href{https://medium.com/the-faculty/white-academia-do-better-fa96cede1fc5}{inundated}
with testaments and
\href{https://twitter.com/sarahcpr/status/1268231406351613952}{queries
from white friends} about fighting racism. And anti-racist activists
have watched with some amazement as powerful white leaders and
corporations acknowledge concepts like ``structural racism'' and pledge
to make sweeping changes in personal and institutional behavior.

But those who have been in the trenches for decades fighting racism in
America wonder how lasting the soul searching will be.

The flood of corporate statements denouncing racism ``feels like a
series of mea culpas written by the press folks and run by the top black
folks'' inside each organization, said Dream Hampton, a writer and
\href{https://vimeo.com/307655948}{filmmaker}. ``Show us a picture of
your C-suite, who is on your board. Then we can have a conversation
about diversity, equity and inclusion.''

``Stop sending positive vibes,'' begged Chad Sanders, a writer, in a
recent
\href{https://www.nytimes.com/2020/06/05/opinion/whites-anti-blackness-protests.html}{New
York Times Op-Ed}, directing his white friends to instead help protect
black protesters, donate to black politicians and funds fighting racial
injustice, and urge others to do the same.

The protests have so far yielded some tangible changes in policing
itself. On Friday, New York banned the use of chokeholds by law
enforcement and repealed a law that kept police disciplinary records
secret.

But their power is also cultural. A run on books about racism has
reordered
\href{https://www.nytimes.com/2020/06/05/books/antiracism-books-race-racism.html}{best-seller
lists}, driving titles like ``How to Be an Antiracist'' and ``White
Fragility'' to the top. And language about American racial dynamics that
was once the purview of academia and activism appears to have gone
mainstream.

In a video
\href{https://twitter.com/NFL/status/1269034074552721408}{released June
5} apologizing for the N.F.L.'s previous failure to support players who
protested police violence, Roger Goodell, the commissioner of the
league, condemned the ``systematic oppression'' of black people, a term
used to convey that racism is embedded in the policies of public and
private institutions. The Denver Board of Education, in voting to
\href{https://www.nytimes.com/2020/06/12/us/schools-police-resource-officers.html}{end
its contract with the city police department} for school resource
officers, cited a desire to avoid the ``perpetuation of the
school-to-prison pipeline,'' a reference to how school policies can
\href{https://www.nytimes.com/2020/04/04/us/politics/black-girls-school-racism.html}{lay
the groundwork} for the incarceration of young black Americans.

``One of the exhilarating things about this moment is that black people
are articulating to the world that this isn't just an issue of the state
literally killing us, it's also about psychic death,'' said Jeremy O.
Harris, a playwright whose
``\href{https://www.nytimes.com/2020/01/27/theater/slave-play-broadway-interviews.html}{Slave
Play}'' addresses the failure of white liberals to admit their
complicity in America's ongoing racial inequities.

\includegraphics{https://static01.nyt.com/images/2020/06/12/us/00UNREST-RECKONING-harris/merlin_167459187_90ab01e0-e5f7-4c63-be51-9a2dc3786efb-articleLarge.jpg?quality=75\&auto=webp\&disable=upscale}

He added, ``It's exhilarating because for the first time, in a macro
sense, people are saying names and showing up and showing receipts.''

Sensing a rare, and perhaps fleeting, opportunity to be heard, many
black Americans are sharing painful stories on social media about racism
and mistreatment in the workplace, accounts that some said they were too
scared to disclose before. They are using hashtags like
\href{https://twitter.com/search?q=\%23BlackintheIvory\&src=typeahead_click}{\#BlackInTheIvory}
or
\href{https://twitter.com/search?q=\%23WeSeeYouwat\&src=typeahead_click}{\#WeSeeYouWAT},
referring to bias in academia and ``White American Theater.''

The feeling of a dam breaking has drawn analogies to the fall and winter
of 2017, when sexual abuse allegations against Harvey Weinstein
triggered a deluge of disturbing accounts from women and provoked frank
conversations in which friends, colleagues and neighbors confessed to
one another: I've suffered in that manner as well. Or: I now realize I
have wronged someone, and I'd like to do better.

Though racism is hardly a secret, ``a huge awakening is just the
awareness of people who don't face the headwinds,'' said Drew Dixon, a
music producer, activist and subject of the documentary
``\href{https://www.nytimes.com/2020/05/27/movies/on-the-record-russell-simmons-review.html}{On
the Record},'' about her decision to come forward with
\href{https://www.nytimes.com/2017/12/13/arts/music/russell-simmons-rape.html}{rape
allegations} against the music producer Russell Simmons, which he has
denied. ``Many people had no idea what women deal with every single day,
and I think many non-black people had no idea what black people deal
with every day.''

\hypertarget{a-shift-in-the-making}{%
\subsection{A shift in the making}\label{a-shift-in-the-making}}

While the outpouring may seem sudden, there have been signs that
perceptions on race were already in flux.

Opinion polls over the last decade have shown a self-reported turn by
Democrats toward a more sympathetic view of black Americans, with more
attributing disparities in areas like income and education to
discrimination rather than personal failure. By 2018, white liberals
said they felt more positively about blacks, Latinos and Asians than
they did about whites.

The reason for the shift is unclear --- and those attitudes have so far
not translated into desegregated schools or neighborhoods --- but may
help explain the cascade of responses to Mr. Floyd's killing.

The outpouring is also related to the horrific nature of Mr. Floyd's
death --- a white police officer kneeling on his neck for nearly nine
minutes --- captured in a stark video at a moment of rising national
frustration with the government's handling of the coronavirus pandemic
and the lockdown.

The protests
\href{https://www.nytimes.com/interactive/2020/06/13/us/george-floyd-protests-cities-photos.html?action=click\&module=Top\%20Stories\&pgtype=Homepage}{still
surging through the streets of America's cities}, said the civil rights
movement scholar Aldon Morris, are ``unprecedented in terms of the high
levels of white participation in a movement targeting black oppression
and grievances.''

Younger Americans are also much more racially diverse than earlier
generations. They tend to have different views on race. And their
imprint on society is only growing.

Brands trying to appeal to younger consumers have in recent years
increasingly proclaimed their belief in equality and justice. Two years
ago, Nike featured in a major ad campaign the former San Francisco 49ers
quarterback Colin Kaepernick, who knelt during the national anthem to
protest racism. The tagline for MAC, the cosmetics company, is ``All
Ages, All Races, All Genders.''

In the wake of the Floyd protests, everyone from
\href{https://www.cnbc.com/2020/06/01/wall-street-ceos-speak-out-about-george-floyd-and-protests-rocking-us-cities.html}{Wall
Street C.E.O.s} and the
\href{https://www.nytimes.com/2020/06/10/business/adidas-black-employees-discrimination.html?searchResultPosition=1}{sportswear
giant Adidas} to the
\href{https://www.themarysue.com/gushers-finally-weighs-in-on-black-lives-matter/}{fruit
snack Gushers} and a company that
\href{https://www.axon.com/news/axon-statement-on-the-death-of-george-floyd}{sells
stun guns} put out statements of support of diversity, flooding
Instagram with vague messages.

Image

Colin Kaepernick (center) and teammates kneeled during the national
anthem before an N.F.L. game in Santa Clara, Calif., in
2016.Credit...Marcio Jose Sanchez/Associated Press

These prompted
\href{https://www.nytimes.com/2020/06/06/business/corporate-america-has-failed-black-america.html?searchResultPosition=1}{cries
of hypocrisy} from those who said the companies don't practice the
values they're espousing.

At several companies, what employees saw as an inadequate response to
Mr. Floyd's death seemed to serve as a catalyst for a long-simmering
contention over questions of racial equity. At Adidas, dozens of
employees stopped working to attend daily protests outside the company's
North American headquarters in Portland, Ore.

The tumult has been especially fraught at Estée Lauder, the beauty
giant, stemming from the political donations of Ronald S. Lauder, a
76-year-old board member and a son of the company's founders. He has
also been a prominent supporter of President Trump.

On May 29, employees at Estée Lauder, like those in much of the rest of
corporate America, began receiving emails from the company's leadership
addressing racial discrimination.

There was ``considerable pain'' in black communities, one missive noted.
According to copies of the internal communications obtained by The New
York Times, the company, whose vast portfolio includes Clinique, MAC,
Bobbi Brown, La Mer and Aveda, encouraged employees to pause working on
June 2 in honor of ``Blackout Tuesday.''

At a video meeting on June 4 among an internal group called NOBLE, or
Network of Black Leaders and Executives, company leaders said Estée
Lauder was donating \$1 million to support racial and social justice
organizations. But employees pinpointed Mr. Lauder's political donations
to Mr. Trump as being in conflict with the company's stance on race. The
president has tweeted
\href{https://www.nytimes.com/2020/06/09/nyregion/who-is-martin-gugino-buffalo-police.html}{conspiracy
theories} about injured protesters, described demonstrators as
``\href{https://twitter.com/realDonaldTrump/status/1266231100780744704?s=20}{THUGS},''
and
\href{https://www.nytimes.com/2020/06/08/us/politics/defund-police-trump.html}{praised
most law enforcement officers} as ``great people.''

Image

Aerin Lauder, left, and Ronald S. Lauder attended the Lincoln Center
Corporate Fashion Gala in November in New York.Credit...Krista Schlueter
for The New York Times

Employees left dissatisfied. Later that night, a petition appeared on
Change.org.

The company's donation did ``not match, or exceed Ronald Lauder's
personal donations in support of state-sanctioned violence,'' organizers
of the petition, which has amassed more than 6,000 signatures, wrote.
``Ronald Lauder's involvement with the Estée Lauder Companies is
damaging to our corporate values, our relationship with the Black
community, our relationship with this company's Black employees, and
this company's legacy.''

In his first public comment on the situation, Mr. Lauder told The Times
in a statement Friday that he had spent decades ``fighting
anti-Semitism, hate and bigotry in all its forms in New York and around
the world as president of the World Jewish Congress.''

``As a country, we must recommit ourselves to the fight against
anti-Semitism and racism,'' he said. ``In this urgent moment of change,
I am expanding the scope of my anti-Semitism campaign to include causes
for racial justice, especially in the Black community, as well as other
forms of dangerous ethnic and religious intolerance around the world.''

On Monday, Estée Lauder said it would donate \$5 million in coming weeks
to ``support racial and social justice and to continue to support
greater access to education,'' and donate an additional \$5 million over
the following two years.

Other companies have also pledged money. On Thursday alone, PayPal,
Apple and YouTube collectively pledged \$730 million to racial justice
and equity efforts.

\hypertarget{jobs-on-the-line}{%
\subsection{Jobs on the line}\label{jobs-on-the-line}}

As companies face restive employees, pressure has also grown to remove
those who have made offensive statements. Others have had to apologize
publicly.
\href{https://www.nytimes.com/2020/06/08/dining/bon-appetit-adam-rapoport.html}{Adam
Rapoport} resigned as editor in chief of the magazine Bon Appétit on
Monday after a 2004 photo showing him in an offensive costume resurfaced
on social media.

And Greg Glassman, the founder and chief executive of CrossFit,
\href{https://www.nytimes.com/2020/06/09/style/crossfit-gyms-founder-protests.html}{stepped
down} on Tuesday following comments about race and racism on a Zoom call
to gym owners.

``We're not mourning for George Floyd, I don't think me or any of my
staff are,'' said Mr. Glassman on the Zoom call, according to a
recording of the call provided to The Times.

``Can you tell me why I should mourn for him?'' he said. ``Other than
it's the `white' thing to do. I get that pressure, but give me another
reason.''

NBCUniversal, a division of Comcast that includes the NBC broadcast
network and cable channels like Bravo, has encountered fires on multiple
fronts as the reckoning has swept the country.

For NBC, the problems started the morning after Mr. Floyd's death, when
Jimmy Fallon found himself under attack on Twitter for performing in
blackface on ``Saturday Night Live'' in 2000. A video of the sketch had
\href{https://www.youtube.com/watch?v=XPau1pLm3jQ}{resurfaced online}.
Mr. Fallon, who has been an NBC star for 22 years, first at ``SNL'' and
more recently leading the ``Tonight'' show,
\href{https://www.nytimes.com/2020/05/26/us/jimmy-fallon-chris-rock-blackface.html}{issued
a written apology} that afternoon. He
\href{https://deadline.com/2020/06/jimmy-fallon-emotional-apology-snl-blackface-skit-tonight-show-return-naacp-derrick-johnson-cnn-don-lemon-saturday-night-live-video-1202949008/}{apologized
at length} on camera the following week.

On June 2, a
\href{https://variety.com/2020/tv/news/dick-wolf-craig-gore-fired-law-and-order-spinoff-controversial-facebook-posts-1234623190/}{writer
was fired} from an upcoming NBC series, ``Law \& Order: Organized
Crime,'' after posting photos of himself on Facebook holding a weapon
and threatening to ``light up'' looters.

Then came an explosion from NBCUniversal's cable division. The hit
reality series ``Vanderpump Rules,'' an anchor tenant on Bravo since
2013,
\href{https://www.today.com/popculture/andy-cohen-absolutely-supports-decision-fire-vanderpump-rules-stars-t183908}{fired
four cast members} for past racist behavior. Some of the incidents were
already known. Others were disclosed on Instagram after Mr. Floyd's
death.

Image

NBC, a division of Comcast, has encountered fires on multiple fronts as
the reckoning on race and police practices has swept the
country.Credit...Hannah Yoon for The New York Times

On June 8, Brian Roberts, Comcast's chief executive, said in a memo to
employees that the company would give \$75 million to social justice
organizations, along with \$25 million worth of advertising inventory,
including on Sky, its pay-television unit in Britain.

``We know that Comcast alone can't remedy this complex issue,'' Mr.
Roberts wrote. ``But you have my commitment that our company will try to
play an integral role in driving lasting reform.''

\hypertarget{a-boiling-point}{%
\subsection{A `boiling point'}\label{a-boiling-point}}

Late last Saturday night, two women who study black health and
communication were talking to each other, for what seemed like the
thousandth time, about the racism they have encountered in their
careers.

The killings of
\href{https://www.nytimes.com/article/breonna-taylor-police.html}{Breonna
Taylor}, George Floyd and too many others had brought them to a
``boiling point,'' recalled one of the women, Joy Melody Woods, a
graduate student at Moody College of Communication. But the national
conversation was still focused primarily on police brutality.

``That's not the only system that perpetuates white supremacy,'' Ms.
Woods said. ``There are other systems, and academia is one of those.''

Ms. Woods called on black scholars to begin sharing their experiences
using the hashtag \#BlackInTheIvory, which her friend Shardé M. Davis,
an assistant professor at the University of Connecticut, had just
coined.

The women went to sleep that night, not knowing they had opened the
floodgates. The hashtag was trending by Sunday night, and as of Thursday
evening had collected nearly 90,000 tweets.

The stories of exclusion, humiliation and hostility were all too
familiar. But the difference was that they had mostly been shared behind
closed doors. In the past, nonblack colleagues could be sympathetic but
were more often dismissive or worse, sometimes labeling a black
colleague as ``difficult.''

``What feels different this time is that white folks are listening,''
Dr. Davis said.

Particularly important, she and others said, is that white scholars seem
to be having conversations about racism in their institutions without a
black colleague around to prompt or guide them.

``You need to be willing to get in the mix and have the conversation and
not expect us to hold your hand through the whole thing --- and so maybe
that's something that is beginning to gain momentum,'' said Chanda
Prescod-Weinstein, a theoretical physicist and feminist scholar at the
University of New Hampshire.

There's a tendency among nonblack scholars to view their black
colleagues as exempt from police brutality and violent hate crimes. But,
Dr. Prescod-Weinstein said, ``That sense of safety isn't real --- our
Ph.D.s are not bulletproof.''

The danger is particularly acute for black naturalists, as shown in the
recent incident with Christian Cooper, the
\href{https://www.nytimes.com/2020/05/27/nyregion/amy-cooper-christian-central-park-video.html}{birder
in Central Park} who asked a white woman to leash her dog, only to have
her call 911.

``Our job means going into the field and being visible and moving in
spaces that are not always welcoming to us,'' said Earyn McGee, a
herpetologist and birder at the University of Arizona. ``We understood
what the danger was.''

The viral video prompted Ms. McGee and others to organize
\#BlackBirdersWeek. Jeffrey Ward, a co-organizer and
\href{https://www.nytimes.com/2019/06/28/movies/birding-people-of-color-rolling-stone.html}{well-known
birder}, said he always keeps his binoculars visible to reassure people
who act fearful when they see him. After two police officers followed
and questioned him two years ago at Crotona Park in the Bronx, he
recalled, he told some white friends. They were sympathetic then, but
seem to better grasp the breadth and gravity of systemic racism now, he
said.

``They reached out to me and said, `We didn't understand it was this
serious. We apologize for not listening to you before.'''

Dr. Prescod-Weinstein was one of several researchers who called for a
strike on Wednesday to
\href{https://www.nytimes.com/2020/06/10/science/science-diversity-racism-protests.html}{protest
racism in science}. Nearly 6,000 scientists, professional societies and
institutions pledged to join.

But she also noted that academic institutions are unrelentingly
hierarchical and resistant to change.

As a postdoctoral fellow at M.I.T., Dr. Prescod-Weinstein was the only
black physicist with a Ph.D. in a department of about 100. Students of
color sought her out for advice and mentoring, she said --- unpaid labor
that she was never recognized or compensated for --- and they felt the
pressure of having to represent their entire race.

``That kind of pressure is extraordinary,'' she said.

Inequity in universities manifests at multiple levels. Black academics
are disproportionately hired to positions with weaker long-term
prospects. They receive fewer grants, and their papers are cited less
often.

Changing these systems will take ``an incredible amount of energy at the
right pressure points in the system,'' said Dr. Kafui Dzirasa, a
psychiatrist at Duke University.

For any system --- say, applying for grants from the National Institutes
of Health --- making things more equitable would come at a cost, either
to the system or to nonblack applicants. ``And that's the cost that it's
unclear if the system is ready to take on,'' Dr. Dzirasa said.

Dr. Davis was more blunt.

``We've received nothing but empty platitudes and empty promises, and
the wound just scabs right back up,'' she said. ``We're walking around
in institutions with a whole bunch of Band-Aids and scabbed-over wounds.
Enough, enough.''

Brooks Barnes contributed reporting. Susan Beachy contributed research.

Advertisement

\protect\hyperlink{after-bottom}{Continue reading the main story}

\hypertarget{site-index}{%
\subsection{Site Index}\label{site-index}}

\hypertarget{site-information-navigation}{%
\subsection{Site Information
Navigation}\label{site-information-navigation}}

\begin{itemize}
\tightlist
\item
  \href{https://help.nytimes.com/hc/en-us/articles/115014792127-Copyright-notice}{©~2020~The
  New York Times Company}
\end{itemize}

\begin{itemize}
\tightlist
\item
  \href{https://www.nytco.com/}{NYTCo}
\item
  \href{https://help.nytimes.com/hc/en-us/articles/115015385887-Contact-Us}{Contact
  Us}
\item
  \href{https://www.nytco.com/careers/}{Work with us}
\item
  \href{https://nytmediakit.com/}{Advertise}
\item
  \href{http://www.tbrandstudio.com/}{T Brand Studio}
\item
  \href{https://www.nytimes.com/privacy/cookie-policy\#how-do-i-manage-trackers}{Your
  Ad Choices}
\item
  \href{https://www.nytimes.com/privacy}{Privacy}
\item
  \href{https://help.nytimes.com/hc/en-us/articles/115014893428-Terms-of-service}{Terms
  of Service}
\item
  \href{https://help.nytimes.com/hc/en-us/articles/115014893968-Terms-of-sale}{Terms
  of Sale}
\item
  \href{https://spiderbites.nytimes.com}{Site Map}
\item
  \href{https://help.nytimes.com/hc/en-us}{Help}
\item
  \href{https://www.nytimes.com/subscription?campaignId=37WXW}{Subscriptions}
\end{itemize}
