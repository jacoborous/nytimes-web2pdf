Sections

SEARCH

\protect\hyperlink{site-content}{Skip to
content}\protect\hyperlink{site-index}{Skip to site index}

\href{https://www.nytimes.com/section/world/europe}{Europe}

\href{https://myaccount.nytimes.com/auth/login?response_type=cookie\&client_id=vi}{}

\href{https://www.nytimes.com/section/todayspaper}{Today's Paper}

\href{/section/world/europe}{Europe}\textbar{}Erdogan and Trump Form New
Bond as Interests Align

\url{https://nyti.ms/37gRfdj}

\begin{itemize}
\item
\item
\item
\item
\item
\end{itemize}

Advertisement

\protect\hyperlink{after-top}{Continue reading the main story}

Supported by

\protect\hyperlink{after-sponsor}{Continue reading the main story}

\hypertarget{erdogan-and-trump-form-new-bond-as-interests-align}{%
\section{Erdogan and Trump Form New Bond as Interests
Align}\label{erdogan-and-trump-form-new-bond-as-interests-align}}

Relations between President Trump and President Recep Tayyip Erdogan of
Turkey have long blown hot and cold. For the moment, they are finding
common cause.

\includegraphics{https://static01.nyt.com/images/2020/06/10/world/10turkey-erdogan-trump001/merlin_164343726_eee1c0a3-b545-4621-a51a-077e2052f404-articleLarge.jpg?quality=75\&auto=webp\&disable=upscale}

\href{https://www.nytimes.com/by/carlotta-gall}{\includegraphics{https://static01.nyt.com/images/2018/10/10/multimedia/author-carlotta-gall/author-carlotta-gall-thumbLarge.png}}

By \href{https://www.nytimes.com/by/carlotta-gall}{Carlotta Gall}

\begin{itemize}
\item
  Published June 10, 2020Updated June 12, 2020
\item
  \begin{itemize}
  \item
  \item
  \item
  \item
  \item
  \end{itemize}
\end{itemize}

ISTANBUL --- Relations between President Trump and his Turkish
counterpart, Recep Tayyip Erdogan, were in the worst state anyone could
remember 10 months ago, veering toward armed clashes between their
armies across the Syrian-Turkish border, while Mr. Trump threatened to
annihilate Turkey's economy.

But these days, as the coronavirus threatens recession and rallies their
opponents, both men are under pressure at home,
\href{https://www.nytimes.com/2020/06/02/world/europe/trump-merkel-allies.html?searchResultPosition=1}{with
not many friends abroad}, and may feel the need of some friendly
comfort. This week, according to the Turkish account, they shared a few
jokes during a phone call.

``To be honest, after our conversation tonight, a new era can begin
between the United States and Turkey,'' Mr. Erdogan said during a
television interview afterward on Monday.

Relations between the two leaders have long blown hot and cold.

A Turkish court sentenced a U.S. consulate employee to eight years in
prison on a terror-related charge on Thursday, a setback for U.S.
officials who have been struggling for three years to exonerate three
U.S. employees who they say are being used as political hostages.

But the presidents' stars have aligned for the moment, with the
interests of Turkey and the United States converging on several of the
biggest issues that had driven them apart in recent years.

It helps that even when interests diverge, the two men like and
understand each other, share a love of strongman politics and have
thrust their family members together
\href{https://www.nytimes.com/2019/11/12/us/politics/trump-erdogan-family-turkey.html}{to
nurture potentially mutually beneficial business deals}.

In recent months, Mr. Trump has not stood in the way of and even
assisted Turkey's interventions in both Syria and Libya. He thanked
Turkey for freeing an American evangelical pastor, even though diplomats
accused Turkey of political hostage taking. And the F.B.I. has opened a
budding investigation into Mr. Erdogan's bête noire, the Islamic
preacher Fethullah Gulen, whom he accuses of masterminding a failed coup
in 2016 from his self-imposed exile in Pennsylvania.

\includegraphics{https://static01.nyt.com/images/2020/06/10/world/10turkey-erdogan-trump02/merlin_172870938_e83b908d-444e-439e-990a-9d5fd7a691f0-articleLarge.jpg?quality=75\&auto=webp\&disable=upscale}

Equally important, Mr. Trump has held off imposing sanctions against
Turkey for its purchase of a Russian S-400 missile system, something
that has prevented Turkey drifting further away from the West, said Asli
Aydintasbas, a senior fellow at the European Council on Foreign
Relations.

``He saved this relationship,'' she said of Mr. Trump. ``If not for this
strange Trump factor, we really would have been in a Turkey-Russian
axis.''

Libya is the latest place where the two men have seemingly reached
agreement, with Mr. Trump effectively greenlighting Mr. Erdogan's
military intervention, which has reshaped the conflict.

``We came to some agreements during our call,'' Mr. Erdogan said this
week about their conversation on Libya, without specifying exactly what
these were.

President Trump has shown little interest in Libya and signaled an
ambivalence over the outcome of the war.

His administration formally supports the United Nations-backed
government of Prime Minister Fayez al-Sarraj. But Mr. Trump also held a
phone call with the Libyan strongman Khalifa Hifter,
\href{https://www.nytimes.com/2020/02/18/us/politics/hifter-torture-lawsuit-libya.html?searchResultPosition=8}{a
former C.I.A. asset} who opened an offensive against Tripoli last year
with the backing of Russia, Egypt and the United Arab Emirates.

This spring Turkish forces came to the aid of the al-Sarraj government,
rescuing it and
\href{https://www.nytimes.com/2020/05/21/world/middleeast/libya-turkey-russia-hifter.html?searchResultPosition=5}{turning
the tide in the war}, and there are signs that Washington is not opposed
to the Turkish intervention.

Image

Fighters loyal to the U.N.-backed government in Libya celebrated last
week after taking over an area of Tripoli from forces loyal to the
strongman Khalifa Hifter.Credit...Mahmud Turkia/Agence France-Presse ---
Getty Images

Washington has not protested Turkey's use of American weapons in its
operations, for example, said Ozgur Unluhisarcikli, Ankara director of
the German Marshall Fund of the United States. The U.S. Africa Command,
based in Europe, is probably also
\href{https://www.nytimes.com/2020/04/14/world/middleeast/libya-russia-john-bolton.html?searchResultPosition=6}{not
unhappy to see Russia restrained in Libya}, he added.

``What Turkey has done in containing Russia, I believe also suits the
U.S. perfectly well,'' he said.

For Mr. Erdogan it is a dramatic turnaround in his dealings with the
United States. Last fall he was on the verge of going to war in northern
Syria against American troops --- NATO allies --- and was castigating
Washington daily for its armed support for the Kurdish forces there.

Turkey had long complained that the Kurdish People's Protection Units,
who collaborated with U.S. forces in fighting the Islamic State in
Syria, were the same organization that has been mounting an insurgency
inside Turkey for three decades.

The Pentagon's arming and training of the Kurdish forces along Turkey's
southern border represented not only a security threat to Turkey but
became an enormous diplomatic dispute with Washington.

Image

U.S. military vehicles, part of a joint convoy with the Kurdish People's
Protection Units, in the northeastern Hasaka Province of Syria in
November.Credit...Delil Souleiman/Agence France-Presse --- Getty Images

That problem has more or less gone away after Mr. Trump pulled American
troops away from Syria's northern border and reduced their footprint to
a smaller area in the south of the country.

Mr. Trump's move set off anguished protests in Congress and even among
his own military over what many saw as its betrayal of longstanding
Kurdish allies. But the sudden withdrawal cleared the way for Turkey to
seize control of a narrow band of territory along the border inside
Syria, with Russia moving into the remaining border areas.

Mr. Erdogan barely mentions American support for the Kurdish forces
these days, even though it continues.

He has also dropped mention of another thorny issue, the extradition of
the Islamic preacher Mr. Gulen, which the United States has refused
saying there is lack of evidence. Turkey seems to have accepted an
alternative that the F.B.I. is conducting an investigation into Mr.
Gulen's affairs instead.

Image

The cleric Fethullah Gulen at his home in Saylorsburg, Pennsylvania, in
2016.Credit...Charles Mostoller/Reuters

A new offensive by Russian and Syrian government forces in December and
January in Idlib, the last rebel-held province of Syria, then brought a
new convergence of interests between Turkey and the United States.

The rapid and ruthless offensive obliterated a swath of towns and
villages, sending nearly a million people fleeing toward Turkey's border
in desperate conditions of cold and misery. Turkey, aided by U.S.
intelligence and surveillance, sent in troops to stem the advance.

Mr. Erdogan had until then been relying on his own relationship with
President Vladimir V. Putin of Russia to negotiate a series of
cease-fires, but the winter offensive was of such a devastating scale
that it tipped Turkey firmly into open opposition. A Russian
\href{https://www.nytimes.com/2020/02/27/world/middleeast/russia-turkey-syria-war-strikes.html}{strike
on a Turkish military convoy} that killed 34 soldiers in February was a
decider.

``There was always an illusion that Turkey was in this big power game
with Russia,'' said Ms. Aydintasbas. ``That is all shattered.''

The Russian aggression in Idlib was one of the main drivers that has
pushed Turkey into a closer cooperation with the United States, said
Sinan Ulgen, chairman of the Center for Economics and Foreign Policy
Studies. ``That was a turning point,'' he said.

Image

A camp for displaced families in Idlib Province, on the Syria-Turkey
border, in March.Credit...Ivor Prickett for The New York Times

The U.S. special representative for Syria, James Jeffrey, has been open
in his praise for Turkey's actions in stemming the Russian-Syrian
government advance and retaining a parcel of territory for the Syrian
opposition. ``We strongly support the cease-fire; we strongly support
the Turkish military action,'' he told a video conference with the
Atlantic Council in April.

This shift does not mean that Turkey will turn its back on Russia,
analysts said. Turkey is conducting a ``balancing act,'' Ms. Aydintasbas
said.

The biggest thorn of all in the relationship,
\href{https://www.nytimes.com/2019/07/12/world/russia-turkey-missile-explain.html}{Turkey's
purchase of a Russian S-400 missile system}, remains unresolved.

Hit by a serious economic downturn as a result of the coronavirus
pandemic, Mr. Erdogan has also softened his tone in an effort to buy
some time for recovery. He did not activate the missile system in April,
as had been scheduled. Many analysts suggest he has held off in order to
avert U.S. sanctions and even to negotiate a swap deal with the Federal
Reserve.

Even if a swap is not successful, easing relations with Washington could
at least help with improving the general investment climate, Ms.
Aydintasbas said.

``Is there a new phase of cooperation? I think there is a window of
opportunity,'' Mr. Unluhisarcikli said.

Mr. Erdogan may still want to bring out the S-400 to rally supporters at
home nearer elections, or at least show them that he did not waste the
\$2.5 billion only to keep the system unused, Mr. Unluhisarcikli added.

``The window is that Turkey is not operationalizing the S-400s,'' he
said. ``But if they do, it's back to square one.''

Advertisement

\protect\hyperlink{after-bottom}{Continue reading the main story}

\hypertarget{site-index}{%
\subsection{Site Index}\label{site-index}}

\hypertarget{site-information-navigation}{%
\subsection{Site Information
Navigation}\label{site-information-navigation}}

\begin{itemize}
\tightlist
\item
  \href{https://help.nytimes.com/hc/en-us/articles/115014792127-Copyright-notice}{©~2020~The
  New York Times Company}
\end{itemize}

\begin{itemize}
\tightlist
\item
  \href{https://www.nytco.com/}{NYTCo}
\item
  \href{https://help.nytimes.com/hc/en-us/articles/115015385887-Contact-Us}{Contact
  Us}
\item
  \href{https://www.nytco.com/careers/}{Work with us}
\item
  \href{https://nytmediakit.com/}{Advertise}
\item
  \href{http://www.tbrandstudio.com/}{T Brand Studio}
\item
  \href{https://www.nytimes.com/privacy/cookie-policy\#how-do-i-manage-trackers}{Your
  Ad Choices}
\item
  \href{https://www.nytimes.com/privacy}{Privacy}
\item
  \href{https://help.nytimes.com/hc/en-us/articles/115014893428-Terms-of-service}{Terms
  of Service}
\item
  \href{https://help.nytimes.com/hc/en-us/articles/115014893968-Terms-of-sale}{Terms
  of Sale}
\item
  \href{https://spiderbites.nytimes.com}{Site Map}
\item
  \href{https://help.nytimes.com/hc/en-us}{Help}
\item
  \href{https://www.nytimes.com/subscription?campaignId=37WXW}{Subscriptions}
\end{itemize}
