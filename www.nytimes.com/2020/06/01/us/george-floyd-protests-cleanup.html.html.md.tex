Sections

SEARCH

\protect\hyperlink{site-content}{Skip to
content}\protect\hyperlink{site-index}{Skip to site index}

\href{https://www.nytimes.com/section/us}{U.S.}

\href{https://myaccount.nytimes.com/auth/login?response_type=cookie\&client_id=vi}{}

\href{https://www.nytimes.com/section/todayspaper}{Today's Paper}

\href{/section/us}{U.S.}\textbar{}In Cities Battered by Protest, the
Cleaning Crews Come Out

\url{https://nyti.ms/3gJPj15}

\begin{itemize}
\item
\item
\item
\item
\item
\end{itemize}

\href{https://www.nytimes.com/news-event/george-floyd-protests-minneapolis-new-york-los-angeles?action=click\&pgtype=Article\&state=default\&region=TOP_BANNER\&context=storylines_menu}{Race
and America}

\begin{itemize}
\tightlist
\item
  \href{https://www.nytimes.com/2020/07/26/us/protests-portland-seattle-trump.html?action=click\&pgtype=Article\&state=default\&region=TOP_BANNER\&context=storylines_menu}{Protesters
  Return to Other Cities}
\item
  \href{https://www.nytimes.com/2020/07/24/us/portland-oregon-protests-white-race.html?action=click\&pgtype=Article\&state=default\&region=TOP_BANNER\&context=storylines_menu}{Portland
  at the Center}
\item
  \href{https://www.nytimes.com/2020/07/23/podcasts/the-daily/portland-protests.html?action=click\&pgtype=Article\&state=default\&region=TOP_BANNER\&context=storylines_menu}{Podcast:
  Showdown in Portland}
\item
  \href{https://www.nytimes.com/interactive/2020/07/16/us/black-lives-matter-protests-louisville-breonna-taylor.html?action=click\&pgtype=Article\&state=default\&region=TOP_BANNER\&context=storylines_menu}{45
  Days in Louisville}
\end{itemize}

Advertisement

\protect\hyperlink{after-top}{Continue reading the main story}

Supported by

\protect\hyperlink{after-sponsor}{Continue reading the main story}

\hypertarget{in-cities-battered-by-protest-the-cleaning-crews-come-out}{%
\section{In Cities Battered by Protest, the Cleaning Crews Come
Out}\label{in-cities-battered-by-protest-the-cleaning-crews-come-out}}

Many protesters have volunteered to clean up after looting and
vandalism, in the hope that the movement will not be defined by
destruction.

\includegraphics{https://static01.nyt.com/images/2020/06/01/us/01UNREST-CLEANUP/merlin_173032311_a2ce4936-071f-4589-a1e4-7044eb96292d-articleLarge.jpg?quality=75\&auto=webp\&disable=upscale}

\href{https://www.nytimes.com/by/ellen-barry}{\includegraphics{https://static01.nyt.com/images/2018/10/08/multimedia/author-ellen-barry/author-ellen-barry-thumbLarge.png}}

By \href{https://www.nytimes.com/by/ellen-barry}{Ellen Barry}

\begin{itemize}
\item
  June 1, 2020
\item
  \begin{itemize}
  \item
  \item
  \item
  \item
  \item
  \end{itemize}
\end{itemize}

BOSTON --- Sierra Rothberg awoke on Monday to a battered city.

Venturing into downtown Boston at 6:30 a.m., she saw red and black
graffiti scrawled on the walls of the State House. Restaurants and shops
gaped open to the street, their windows smashed. Mailboxes were
overturned. The tall grass of Boston Common was littered with broken
glass, looted shoes and discarded signs.

Worse than all of these was her feeling of sadness that a protest that
had inspired her had ended with destruction.

So Ms. Rothberg, 45, did what she often does when faced with a problem.
She started gathering together cleaning supplies.

By 9 a.m., she was out on the sidewalk, sweeping debris into a small
dustbin.

``I thought, `I'm a doer, I need to get my hands in there,' '' said Ms.
Rothberg, an event planner. By early afternoon, more than 100 people had
joined her, some in response to her Facebook callout and some
spontaneously.

``They would say, `Can I help? I have 15 minutes,' and they would stay
for three hours,'' Ms. Rothberg said. ``People could not give up on it.
It took on a life of its own. A guy jumped out of his jeep and brought
us 10 pizzas.''

A hallmark of recent days in America is that, in cities troubled by
contagion, grief and now violence, people are coming out of the woodwork
to clean.

Residents of Los Angeles were out with glass cleaner and long-handled
brooms. Volunteers scrubbed the walls of the State Capitol in Denver,
swept up the shattered windows of stores in Fargo, N.D., and repaired
damaged buildings in downtown San Antonio.

\href{https://www.nytimes.com/article/california-george-floyd-protests.html}{In
San Jose, Calif.}, on Monday, Mayor Sam Liccardo was handing out
cleaning kits --- a bucket, a scraper, some rags and Goo Gone, a
chemical solvent --- and offering a short course in graffiti removal.

``You end up having to scrub --- a lot,'' he said. He called the cleanup
``an ongoing task,'' and said that not all of the protesters were
``doing so in true community spirit.'' Several dozen small businesses in
San Jose had their windows smashed in protests over the weekend.

``It tears your heart out,'' he said, ``because they are struggling so
much already.''

In interviews, organizers of cleanup events said they were motivated, in
part, by fear that the protests would come to be defined by looting and
vandalism.

``So many people were worried that the message was getting lost in the
violence,'' said Justine Sandoval, 34, the president of the Denver Young
Democrats, who organized a cleanup event on Sunday in Denver.

``They want to show up and say, `These protests are important, but we're
going to be there to pick up the pieces afterward,' '' she said. ``It
felt good, because we want to keep this conversation going.''

She said local politicians were quick to latch onto the cleanups,
perhaps seeing them as a sign of conciliation --- but that, she said,
was a misunderstanding.

``It still doesn't erase the fact that we're fighting, because black
people are being killed by the police in this country,'' she said.

\includegraphics{https://static01.nyt.com/images/2020/06/01/us/01UNREST-CLEANUP-mlps/merlin_173078655_5441aba7-ad3a-419b-8208-53d1058abf6c-articleLarge.jpg?quality=75\&auto=webp\&disable=upscale}

Some who cleaned up said it helped calm them after days of intense
emotion.

``I thought, `OK, I'm back at the peaceful part of it,' '' said Rachel
Madden, 55, who, along with her sisters, spent part of the weekend
picking up garbage on Lake Street in Minneapolis.

Ms. Madden, who is white, said she could understand the rage that led to
looting and vandalism, and that she might do the same if she had grown
up black in those neighborhoods.

``I feel like buildings need to burn, sadly, for people to listen,''
said Ms. Madden, an artist. ``And then I'm happy to come and clean up
and do what I can.''

In Boston, the cleanup began at around 3 a.m., when the city's public
works department deployed 12 street sweepers. At 4 a.m., they were
joined by sidewalk sweepers --- not machines, but humans with brooms. At
6 a.m. the city sent the team of workers who specialize in graffiti
removal.

Much of the work was finished by 9 a.m., a city official said.

It was nearly 10 by the time Audrey Markoff, an artist, and her husband
biked to the Common, surveying smashed storefronts along the way.

It made them sad; they had watched the protests with sympathy, and did
not expect violence on Sunday. ``I thought Boston was going to be the
peaceful one,'' said her husband, Greg Dunn.

They wanted to help clean, in part to show that the violence was not
representative of all of the protesters.

``There was a huge turnout this morning from the protest people, because
they're separate from the riot people,'' said Ms. Markoff, 33. ``They're
illustrating that these are separate groups of people, separate
events.''

By they time they reached the Common, though, much of the cleaning had
already been done, and there were plenty of volunteers already. Ms.
Markoff left relieved at how quickly the city had responded. ``I hope it
doesn't happen again tonight or tomorrow,'' she said.

If it does, Ms. Rothberg said, she would organize another cleanup.
Already, she is coordinating with the Parks Department to clean up after
a rally planned for Tuesday.

``There was so much unrest and uncertainty and scary feelings,'' she
said. ``What I really found is that the most beautiful moments were the
aftermath --- that people were coming together and saying, `Oh, come on,
this doesn't represent us.'''

John Eligon contributed reporting from Minneapolis, Jennifer Medina from
Los Angeles and Thomas Fuller from San Francisco.

Advertisement

\protect\hyperlink{after-bottom}{Continue reading the main story}

\hypertarget{site-index}{%
\subsection{Site Index}\label{site-index}}

\hypertarget{site-information-navigation}{%
\subsection{Site Information
Navigation}\label{site-information-navigation}}

\begin{itemize}
\tightlist
\item
  \href{https://help.nytimes.com/hc/en-us/articles/115014792127-Copyright-notice}{©~2020~The
  New York Times Company}
\end{itemize}

\begin{itemize}
\tightlist
\item
  \href{https://www.nytco.com/}{NYTCo}
\item
  \href{https://help.nytimes.com/hc/en-us/articles/115015385887-Contact-Us}{Contact
  Us}
\item
  \href{https://www.nytco.com/careers/}{Work with us}
\item
  \href{https://nytmediakit.com/}{Advertise}
\item
  \href{http://www.tbrandstudio.com/}{T Brand Studio}
\item
  \href{https://www.nytimes.com/privacy/cookie-policy\#how-do-i-manage-trackers}{Your
  Ad Choices}
\item
  \href{https://www.nytimes.com/privacy}{Privacy}
\item
  \href{https://help.nytimes.com/hc/en-us/articles/115014893428-Terms-of-service}{Terms
  of Service}
\item
  \href{https://help.nytimes.com/hc/en-us/articles/115014893968-Terms-of-sale}{Terms
  of Sale}
\item
  \href{https://spiderbites.nytimes.com}{Site Map}
\item
  \href{https://help.nytimes.com/hc/en-us}{Help}
\item
  \href{https://www.nytimes.com/subscription?campaignId=37WXW}{Subscriptions}
\end{itemize}
