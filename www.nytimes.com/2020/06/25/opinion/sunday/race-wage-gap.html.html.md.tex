Sections

SEARCH

\protect\hyperlink{site-content}{Skip to
content}\protect\hyperlink{site-index}{Skip to site index}

\href{https://www.nytimes.com/section/opinion/sunday}{Sunday Review}

\href{https://myaccount.nytimes.com/auth/login?response_type=cookie\&client_id=vi}{}

\href{https://www.nytimes.com/section/todayspaper}{Today's Paper}

\href{/section/opinion/sunday}{Sunday Review}\textbar{}The Black-White
Wage Gap Is as Big as It Was in 1950

\href{https://nyti.ms/3eYeNGU}{https://nyti.ms/3eYeNGU}

\begin{itemize}
\item
\item
\item
\item
\item
\item
\end{itemize}

\href{https://www.nytimes.com/interactive/2020/opinion/america-inequality-coronavirus.html?action=click\&pgtype=Article\&state=default\&region=TOP_BANNER\&context=storylines_menu}{\includegraphics{https://static01.nyt.com/newsgraphics/2020/04/10/storylines-op-inequality/128d73ea016db3e2791158d151b6485f57f635a8/NYTOpEd-Inequality-Icon.jpg}
The America We Need is a Times Opinion series on emerging from this
crisis with a fair, resilient society. }

debugid:204

Advertisement

\protect\hyperlink{after-top}{Continue reading the main story}

\href{/section/opinion}{Opinion}

Supported by

\protect\hyperlink{after-sponsor}{Continue reading the main story}

\hypertarget{the-black-white-wage-gap-is-as-big-as-it-was-in-1950}{%
\section{The Black-White Wage Gap Is as Big as It Was in
1950}\label{the-black-white-wage-gap-is-as-big-as-it-was-in-1950}}

Recent research indicates little progress since the Truman
administration.

\href{https://www.nytimes.com/by/david-leonhardt}{\includegraphics{https://static01.nyt.com/images/2020/05/01/multimedia/David-Leonhardt-Headshot-The-Morning/David-Leonhardt-Headshot-The-Morning-thumbLarge-v3.png}}

By \href{https://www.nytimes.com/by/david-leonhardt}{David Leonhardt}

Mr. Leonhardt writes
\href{https://www.nytimes.com/series/us-morning-briefing}{The Morning}
newsletter.

\begin{itemize}
\item
  June 25, 2020
\item
  \begin{itemize}
  \item
  \item
  \item
  \item
  \item
  \item
  \end{itemize}
\end{itemize}

\includegraphics{https://static01.nyt.com/images/2020/07/05/opinion/sunday/InequalityIconsCh3-06/InequalityIconsCh3-06-articleLarge.jpg?quality=75\&auto=webp\&disable=upscale}

\hypertarget{black-share-of-wages}{%
\subsection{Black Share of Wages}\label{black-share-of-wages}}

\begin{itemize}
\item
\item
\end{itemize}

Government statistics suggest that the earnings gap between black and
white men is substantially smaller than it was 75 years ago. It shrunk
in the 1950s, `60s and `70s and has remained largely stable since then.

But these statistics are misleading. A more comprehensive look at the
data, based on recent academic research, shows that the black-white wage
gap is roughly as large today as it was in 1950.

That's remarkable. Despite decades of political change --- the end of
enforced segregation across the South, the legalization of interracial
marriage, the passage of multiple civil rights laws and more --- the
wages of black men trail those of white men by as much as when Harry
Truman was president. That gap indicates that there have also been
powerful forces pushing against racial equality.

Before getting into the causes, though, I want to explain the difference
between the best-known wage statistics and the more accurate version.
The
\href{https://web.archive.org/web/20110607035101/http://www.gpoaccess.gov/eop/ca/pdfs/ch4.pdf}{traditional
numbers} are incomplete in a way that many people do not realize: They
cover only workers. People who don't work are ignored. This group
includes students, full-time parents, people who have given up on
finding work and people who are incarcerated.

Excluding them wouldn't present a problem if the percentage of
nonworkers had remained fairly stable over time. But it has not.
``There's been a tremendous run-up in non-work among prime-age men,''
says Kerwin Kofi Charles, an economist and the dean of the Yale School
of Management.

One reason is that many middle-aged men --- of all races, although
disproportionately black --- have dropped out of the labor force, and
are neither working nor looking for work. The shrinking number of
decent-paying blue-collar jobs has left many people who didn't attend
college without good job opportunities, and they have responded
\href{https://obamawhitehouse.archives.gov/sites/default/files/page/files/20160620_cea_primeage_male_lfp.pdf}{by
no longer actively looking for work}.

A second reason that more men aren't working is that
\href{https://sentencingproject.org/wp-content/uploads/2016/01/Trends-in-US-Corrections.pdf}{vastly
more of them are incarcerated}. Incarceration rates are especially high
for black men --- about twice as high as those of Hispanic men, six
times higher than those of white men and at least 25 times higher than
those of black women, Hispanic women or white women.

Becky Pettit, a sociologist at the University of Texas, refers to these
incarcerated men as invisible. She has written a book titled,
\href{https://www.russellsage.org/publications/invisible-men}{``Invisible
Men: Mass incarceration and the myth of black progress.''}

\hypertarget{whos-not-working}{%
\subsection{Who's Not Working?}\label{whos-not-working}}

People considered ``unemployed'' represent a small --- and declining ---
share of those out of work.

Unemployed

Out of labor force

Institutionalized (mostly people in prison)

40\%

40\%

White men

not working

Black men

not working

7\%

13\%

30

30

11\%

8\%

9\%

8\%

20

20

19

16

16

16

4\%

7\%

8\%

6\%

6\%

13

4\%

3\%

13

10

10

4\%

10

4\%

7

3\%

7

2\%

10

3\%

9

9

9

9

8

8

8

6

5

5

4

4

6

4

4

3

3

0

0

1950

`90

`14

1950

`90

`14

Institutionalized (mostly people in prison)

Unemployed

Out of labor force

40\%

40\%

White men not working

Black men not working

7\%

13\%

30

30

11\%

8\%

9\%

8\%

20

20

19

16

16

16

4\%

7\%

8\%

6\%

6\%

13

4\%

3\%

13

10

10

4\%

10

4\%

7

3\%

7

2\%

10

3\%

9

9

9

9

8

8

8

6

5

5

4

4

6

4

4

3

3

0

0

1950

1970

1990

2007

2014

1950

1970

1990

2007

2014

Source: Patrick Bayer and Kerwin Kofi Charles, "Divergent Paths." Note:
Men aged 25-54. \textbar{} The New York Times

The traditional statistics on the black-white wage gap ignore these
trends, because they examine only people with earnings. As social
scientists put it, the traditional numbers ignore the ``zero values.''

This means that the statistics on the wage gap are looking at a
shrinking share of the population over time. They overlook the roughly
30 percent of black men and 15 percent of white men between the ages of
25 and 54 who had not been working in a given week during recent years.
(Those shares are even higher now, given the economic downturn.)

``It's a weird hole,'' Mr. Charles says.

He and another economist --- Patrick Bayer of Duke ---
\href{https://academic.oup.com/qje/article/133/3/1459/4830121}{undertook
a research project to fill that hole}. They collected census data dating
back to 1940 and constructed wage statistics that included men who were
not working. They are also conducting a follow-up project about women,
Mr. Bayer said. The gap between black and white women may have narrowed,
but only modestly.

More from ``The America We Need''
\href{https://www.nytimes.com/2020/07/04/opinion/sunday/women-work-coronavirus.html?action=click\&pgtype=Article\&state=default\&region=MAIN_CONTENT_2\&context=storylines_related_links}{}

\includegraphics{https://static01.nyt.com/images/2020/07/05/opinion/05nashtop/05nashtop-threeByTwoSmallAt2X.jpg}

Women Ask Themselves, `How Can I Do This for One More Day?' By Leah Nash

\href{https://www.nytimes.com/2020/07/02/opinion/sunday/income-inequality-solutions.html?action=click\&pgtype=Article\&state=default\&region=MAIN_CONTENT_2\&context=storylines_related_links}{}

\includegraphics{https://static01.nyt.com/images/2020/07/02/opinion/02solutionWeb/02solutionWeb-threeByTwoSmallAt2X-v2.jpg}

America Needs Some Repairs. Here's Where to Start. By The Editorial
Board

\href{https://www.nytimes.com/2020/07/02/opinion/private-equity-inequality.html?action=click\&pgtype=Article\&state=default\&region=MAIN_CONTENT_2\&context=storylines_related_links}{}

\includegraphics{https://static01.nyt.com/images/2020/07/08/opinion/02-inequalityB/02-inequalityB-threeByTwoSmallAt2X-v2.jpg}

The Neoliberal Looting of America By Mehrsa Baradaran

The research by Mr. Charles and Mr. Bayer shows that once all men ---
working and not working --- are included, the picture changes:

\hypertarget{a-persistent-wage-gap}{%
\subsection{A Persistent Wage Gap}\label{a-persistent-wage-gap}}

The wage gap between black and white men is virtually unchanged when
including all black men.

Black male earnings for every \$1 earned by white men

All black workers

All black men

\$0.60

0.40

0.20

1950

2014

1950

2014

Black male earnings for every \$1 earned by white men

All black men

All black workers

\$0.60

\$0.60

0.40

0.40

0.20

0.20

1950

2014

1950

2014

Note: Data shows the median for each group. Source: Patrick Bayer and
Kerwin Kofi Charles, "Divergent Paths." \textbar{} By The New York Times

The black-white wage gap shrunk substantially from 1950 to 1980, and
especially during the 1960s. Civil-rights laws and a decline in legally
sanctioned racism most likely played some role. But the main reasons,
Mr. Charles said, appear to have been trends that benefited all
blue-collar workers, like strong unions and a rising minimum wage.
Because black workers were disproportionately in blue-collar jobs, the
general rise of incomes for the poor and middle class shrank the racial
wage gap.

One law was especially important:
\href{https://www.law.cornell.edu/cfr/text/29/779.338}{the 1966
amendment} to the Fair Labor Standards Act. When Congress passed the
original law, during the New Deal, it deliberately exempted service and
other industries with many black workers from the minimum wage. ``Just
expanding the minimum wage to those industries,'' Ellora Derenoncourt, a
University of California, Berkeley, economist, said, ``boosted the
relative wages of black workers substantially.''

Since 1980, however, the wage gap has increased again, and is now back
roughly to where it was in 1950. The same economic forces are at work,
only in the opposite direction: The minimum wage has stagnated in some
states, unions have shrunk, tax rates on the wealthy have fallen more
than they have for anyone else and incomes for the bottom 90 percent ---
and especially the bottom half ---
\href{https://www.nytimes.com/2019/02/24/opinion/income-inequality-upper-middle-class.html}{have
trailed economic growth}. Black workers, again, are disproportionately
in these lower-income groups.

One nuance is that the racial wage gap has shrunk somewhat among
higher-income men. That's a sign that more African-Americans have broken
into the upper middle class than was the case in prior decades:

\hypertarget{upper-middle-class}{%
\subsection{Upper Middle Class}\label{upper-middle-class}}

The wage gap between white and black men has shrunk somewhat for workers
with higher incomes.

Black male earnings for every \$1 earned by white men

Black men with

positive earnings

All black men

90th quantile

Median

\$0.60

90th

quantile

Median

0.40

0.20

1940

2010

1940

2010

Black male earnings for every \$1 earned by white men

Black men with positive earnings

All black men

Median

90th quantile

90th quantile

\$0.60

\$0.60

Median

0.40

0.40

0.20

0.20

1950

1970

1990

2010

1950

1970

1990

2010

Source: Patrick Bayer and Kerwin Kofi Charles, "Divergent Paths."
\textbar{} The New York Times

This history also points to some of the likely solutions for closing the
racial wage gap. An end to mass incarceration would help. So would
policies that attempt to
\href{https://www.nytimes.com/interactive/2020/06/24/magazine/reparations-slavery.html}{reverse
decades of government-encouraged racism} --- especially in housing. But
it's possible that nothing would have a bigger impact than policies that
lifted the pay of all working-class families, across races.

``Black people are concentrated in low-paying jobs if they have jobs,''
Ms. Derenoncourt said. ``This has been one of the most egregious forms
of inequality over the last 40 years: There has been almost no wage
growth for the bottom half of the wage distribution.''

\emph{The Times is committed to publishing}
\href{https://www.nytimes.com/2019/01/31/opinion/letters/letters-to-editor-new-york-times-women.html}{\emph{a
diversity of letters}} \emph{to the editor. We'd like to hear what you
think about this or any of our articles. Here are some}
\href{https://help.nytimes.com/hc/en-us/articles/115014925288-How-to-submit-a-letter-to-the-editor}{\emph{tips}}\emph{.
And here's our email:}
\href{mailto:letters@nytimes.com}{\emph{letters@nytimes.com}}\emph{.}

\emph{Follow The New York Times Opinion section on}
\href{https://www.facebook.com/nytopinion}{\emph{Facebook}}\emph{,}
\href{http://twitter.com/NYTOpinion}{\emph{Twitter (@NYTopinion)}}
\emph{and}
\href{https://www.instagram.com/nytopinion/}{\emph{Instagram}}\emph{.}

Advertisement

\protect\hyperlink{after-bottom}{Continue reading the main story}

\hypertarget{site-index}{%
\subsection{Site Index}\label{site-index}}

\hypertarget{site-information-navigation}{%
\subsection{Site Information
Navigation}\label{site-information-navigation}}

\begin{itemize}
\tightlist
\item
  \href{https://help.nytimes.com/hc/en-us/articles/115014792127-Copyright-notice}{©~2020~The
  New York Times Company}
\end{itemize}

\begin{itemize}
\tightlist
\item
  \href{https://www.nytco.com/}{NYTCo}
\item
  \href{https://help.nytimes.com/hc/en-us/articles/115015385887-Contact-Us}{Contact
  Us}
\item
  \href{https://www.nytco.com/careers/}{Work with us}
\item
  \href{https://nytmediakit.com/}{Advertise}
\item
  \href{http://www.tbrandstudio.com/}{T Brand Studio}
\item
  \href{https://www.nytimes.com/privacy/cookie-policy\#how-do-i-manage-trackers}{Your
  Ad Choices}
\item
  \href{https://www.nytimes.com/privacy}{Privacy}
\item
  \href{https://help.nytimes.com/hc/en-us/articles/115014893428-Terms-of-service}{Terms
  of Service}
\item
  \href{https://help.nytimes.com/hc/en-us/articles/115014893968-Terms-of-sale}{Terms
  of Sale}
\item
  \href{https://spiderbites.nytimes.com}{Site Map}
\item
  \href{https://help.nytimes.com/hc/en-us}{Help}
\item
  \href{https://www.nytimes.com/subscription?campaignId=37WXW}{Subscriptions}
\end{itemize}
