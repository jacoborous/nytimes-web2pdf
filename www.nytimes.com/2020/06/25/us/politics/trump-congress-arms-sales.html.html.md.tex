Sections

SEARCH

\protect\hyperlink{site-content}{Skip to
content}\protect\hyperlink{site-index}{Skip to site index}

\href{https://www.nytimes.com/section/politics}{Politics}

\href{https://myaccount.nytimes.com/auth/login?response_type=cookie\&client_id=vi}{}

\href{https://www.nytimes.com/section/todayspaper}{Today's Paper}

\href{/section/politics}{Politics}\textbar{}Trump Administration May End
Congressional Review of Foreign Arms Sales

\url{https://nyti.ms/2CGRsey}

\begin{itemize}
\item
\item
\item
\item
\item
\end{itemize}

Advertisement

\protect\hyperlink{after-top}{Continue reading the main story}

Supported by

\protect\hyperlink{after-sponsor}{Continue reading the main story}

\hypertarget{trump-administration-may-end-congressional-review-of-foreign-arms-sales}{%
\section{Trump Administration May End Congressional Review of Foreign
Arms
Sales}\label{trump-administration-may-end-congressional-review-of-foreign-arms-sales}}

Mike Pompeo and other aides to President Trump are discussing ending a
bipartisan review process because lawmakers have held up sales to Saudi
Arabia over civilian casualties in Yemen.

\includegraphics{https://static01.nyt.com/images/2020/06/25/us/politics/25dc-weapons/25dc-weapons-articleLarge.jpg?quality=75\&auto=webp\&disable=upscale}

By \href{https://www.nytimes.com/by/michael-laforgia}{Michael LaForgia},
\href{https://www.nytimes.com/by/edward-wong}{Edward Wong} and
\href{https://www.nytimes.com/by/eric-schmitt}{Eric Schmitt}

\begin{itemize}
\item
  June 25, 2020
\item
  \begin{itemize}
  \item
  \item
  \item
  \item
  \item
  \end{itemize}
\end{itemize}

Senior Trump administration officials are quietly discussing whether to
end a decades-old process for congressional review that has allowed
lawmakers from both parties to block weapons sales to foreign
governments over humanitarian concerns, according to current and former
administration officials and congressional aides. The move could quickly
advance sales of bombs to Saudi Arabia, among other deals.

If adopted, the change would effectively end congressional oversight of
the sale of American weapons and offers of training to countries engaged
in wars with high civilian casualties or human rights abuses. It would
also certainly widen rifts between the administration and Congress.

Senior administration officials have been especially frustrated in the
past three years by bipartisan efforts in Congress to hold up arms sales
to Saudi Arabia, which, along with the United Arab Emirates, has used
American weapons to wage a devastating war in Yemen that has killed
thousands of civilians.

President Trump has championed the sales ---
\href{https://www.washingtonpost.com/business/2018/11/21/trump-again-uses-arms-sales-justify-saudi-ties-dragging-us-defense-contractors-into-an-unwelcome-debate/}{even
holding up charts} in the White House during a news conference to stress
their importance --- as have Jared Kushner, his son-in-law and senior
adviser, and
\href{https://www.nytimes.com/2020/05/16/us/arms-deals-raytheon-yemen.html}{Peter
Navarro}, a trade adviser.

In May 2019, Secretary of State Mike Pompeo declared an emergency to
bypass Congress and fast-track more than \$8 billion in bombs and other
weapons, mainly to Saudi Arabia and the United Arab Emirates --- citing
a need to
``\href{https://apnews.com/4a1fef7a381045a783b27479f191809d}{deter
further Iranian adventurism}'' in the region.

A State Department inspector general began an inquiry into the
declaration in June 2019 and was
\href{https://www.nytimes.com/2020/05/18/us/politics/pompeo-trump-linick-inspector-general-firing.html}{close
to completing that investigation} into whether Mr. Pompeo and other
aides had acted illegally when the inspector general was fired by Mr.
Trump last month at Mr. Pompeo's urging.

More recently, the administration has chafed at decisions by Senator Bob
Menendez, Democrat of New Jersey, and others this winter to block
\href{https://www.nytimes.com/2020/05/28/us/politics/congress-saudi-arabia-arms-sales.html}{the
sale of \$478 million worth of precision-guided bombs} to Saudi Arabia
and a license for an American company, Raytheon Technologies, to expand
its manufacturing footprint in the kingdom.

In the late winter, lawmakers also placed a hold on a new package to
Saudi Arabia --- technology that would connect military databases with
those of domestic security forces. That package, which has not been
reported publicly, has raised concerns among lawmakers because of
domestic human rights abuses by Saudi authorities.

A decision by the Trump administration to block Congress from the arms
sales review process would not just free up deals with Saudi Arabia. It
would also effectively push through sales of Predator drones to the
United Arab Emirates; a refurbishment package for Egyptian attack
helicopters; sophisticated radars for Pakistan; and missiles, bombs and
machine guns for Turkey, among other items, officials said.

All of those deals, which have not previously been reported, are being
held up by lawmakers over questions about how the items will be used.
Under the current system, the State Department gives informal
notification to relevant foreign policy committees in Congress of
proposed arms sales. The lawmakers then give input to administration
officials, which helps agencies in making adjustments to ensure the
sales get approved by Congress as a whole.

Under this informal process, lawmakers can hold up sales, which is what
both Republican and Democratic senators have done with arms sales to
Gulf Arab nations.

Once any differences are resolved, the administration gives Congress
formal notification of the arms sales, which then starts a 30-day period
when lawmakers can object.

If the administration scraps the informal notification process, it would
tell Congress of proposed arms sales only through the formal process.
That framework allows members of Congress to introduce and vote on
resolutions to disapprove of certain sales. But to actually block a
deal, a measure would require support from two-thirds of both chambers
to overcome an inevitable presidential veto.

Any decision to end the informal process would further alienate
lawmakers from both parties, current and former officials said.

``This is not just a thumb in the eye to Democrats, this is a thumb in
the eye to Republicans and to all of Congress,'' said Max Bergmann, a
senior adviser at the State Department during the Obama administration
who had helped oversee military sales. ``The way the arms sale process
has worked is one of the rare bipartisan mechanisms that has existed no
matter who controls the White House and Congress.''

Mr. Menendez, the Democratic senator, said ending the informal
notification system would make the process harder for all sides. ``The
American public has a right to insist that the sales of U.S. weapons to
foreign governments are consistent with U.S. values and national
security objectives,'' he said.

The State Department and the Pentagon both declined to comment.

Defense Secretary Mark T. Esper supports ending the informal
notification process, but is letting Mr. Pompeo take the lead in the
efforts, officials with knowledge of the matter said. Mr. Esper thinks a
faster process for arms sales would help him on a range of security
issues in the Middle East, these officials said.

Career officials in both departments have warned political appointees
against ending the process.

The discussions come at a particularly sensitive time for Mr. Pompeo.

Three congressional committees are investigating whether Mr. Pompeo
urged Mr. Trump to fire the State Department's inspector general, Steve
A. Linick, over inquiries Mr. Linick was conducting into the secretary.
One of those focused on whether Mr. Pompeo and other administration
officials
\href{https://www.nytimes.com/2020/05/18/us/politics/pompeo-trump-linick-inspector-general-firing.html}{acted
illegally} when he announced the ``emergency'' declaration to push
through the \$8.1 billion sales of weapons in 22 batches mainly to Saudi
Arabia and the United Arab Emirates. Those sales had been held up since
2017 by lawmakers from both parties in the informal notification
process.

Mr. Pompeo was aware of Mr. Linick's investigation and had submitted a
written statement in response to questions from the inspector generals'
office. In early March, investigators briefed several senior State
Department officials on their findings, but the report has not been
completed.

Mr. Pompeo has said Mr. Linick was
``\href{https://www.nytimes.com/2020/05/18/us/politics/pompeo-trump-linick-inspector-general-firing.html}{undermining}''
the department.

\href{https://pomed.org/team/andrew-miller/}{Andrew Miller,} a former
department official, said he had heard that discussions had been
underway for months among administration officials over ending the
informal notification process.

He said some congressional offices became aware of the discussions at
the time that State Department officials
\href{https://www.nytimes.com/2020/05/28/us/politics/congress-saudi-arabia-arms-sales.html}{gave
informal notification} to those offices about the new \$478 million
package of precision-guided bombs to Saudi Arabia, which includes the
license for Raytheon. Congressional aides say that license is just as
troubling as the bombs.

``In terms of the policy, it has two contradictory effects,'' said Mr.
Miller, a deputy director for policy at the Project on Middle East
Democracy. ``On one hand, it could circumvent congressional oversight
and lead to more reckless sales. On the other hand, it deprives the
administration of an early opportunity to adjust sales to reflect
congressional concerns, which could actually lead to delays.''

Congressional aides said the fact that Mr. Kushner
\href{https://www.nytimes.com/2018/12/08/world/middleeast/saudi-mbs-jared-kushner.html}{has
a direct channel} with Crown Prince Mohammed bin Salman of Saudi Arabia
--- the two communicate using WhatsApp and other methods --- raises more
questions about the arms deals and increases the importance of
oversight. Among other things, U.S. intelligence agencies believe the
prince ordered
\href{https://www.nytimes.com/2019/06/19/world/middleeast/jamal-khashoggi-Mohammed-bin-Salman.html}{the
2018 killing of Jamal Khashoggi}, a Virginia resident and Washington
Post columnist; that killing has contributed to lawmakers putting holds
on arms sales.

Under the review process, Congress scrutinizes hundreds of proposed arms
sales packages each year. The vast majority go through the process
smoothly, but there have been prominent instances in which lawmakers and
the Trump administration have clashed.

The informal notification process has existed in one form or another
since at least the 1980s. Executive branch agencies agreed with
lawmakers on the latest iteration in early 2013. Under a ``gentleman's
agreement,'' a legislative aide said, congressional committees would
approve packages during the informal process in 20, 30 or 40 days,
depending on the sensitivity of the package. However, on those few
packages where lawmakers had additional questions, the committees could
continue the freeze them until they got satisfactory answers.

Trump administration officials are frustrated by that part of the
process. In a classified briefing with congressional committee members
this month, R. Clarke Cooper, the assistant secretary of state in the
bureau of political-military affairs,
\href{https://www.nytimes.com/2019/06/12/us/politics/arms-sales-saudi-arabia.html}{urged
lawmakers to lift holds} on contentious sales within reasonable amounts
of time.

Some current and former national security officials said the informal
process did need revamping.

``Both sides have a legitimate argument,'' said Bilal Saab, the director
of the Middle East Institute's defense and security program and a former
Pentagon official who worked on security cooperation. ``You have
congressional staffers who can put an indefinite hold on an arms sales
case because their boss has a personal vendetta against political rivals
in the administration. That's excessive, and it's bad for foreign
policy. But Congress does not want to be cornered.''

Mr. Saab recommended a compromise that would allow Congress to put a
30-day informal hold on a sale to resolve any grievances, after which
the hold would be lifted.

Advertisement

\protect\hyperlink{after-bottom}{Continue reading the main story}

\hypertarget{site-index}{%
\subsection{Site Index}\label{site-index}}

\hypertarget{site-information-navigation}{%
\subsection{Site Information
Navigation}\label{site-information-navigation}}

\begin{itemize}
\tightlist
\item
  \href{https://help.nytimes.com/hc/en-us/articles/115014792127-Copyright-notice}{©~2020~The
  New York Times Company}
\end{itemize}

\begin{itemize}
\tightlist
\item
  \href{https://www.nytco.com/}{NYTCo}
\item
  \href{https://help.nytimes.com/hc/en-us/articles/115015385887-Contact-Us}{Contact
  Us}
\item
  \href{https://www.nytco.com/careers/}{Work with us}
\item
  \href{https://nytmediakit.com/}{Advertise}
\item
  \href{http://www.tbrandstudio.com/}{T Brand Studio}
\item
  \href{https://www.nytimes.com/privacy/cookie-policy\#how-do-i-manage-trackers}{Your
  Ad Choices}
\item
  \href{https://www.nytimes.com/privacy}{Privacy}
\item
  \href{https://help.nytimes.com/hc/en-us/articles/115014893428-Terms-of-service}{Terms
  of Service}
\item
  \href{https://help.nytimes.com/hc/en-us/articles/115014893968-Terms-of-sale}{Terms
  of Sale}
\item
  \href{https://spiderbites.nytimes.com}{Site Map}
\item
  \href{https://help.nytimes.com/hc/en-us}{Help}
\item
  \href{https://www.nytimes.com/subscription?campaignId=37WXW}{Subscriptions}
\end{itemize}
