Sections

SEARCH

\protect\hyperlink{site-content}{Skip to
content}\protect\hyperlink{site-index}{Skip to site index}

\href{https://www.nytimes.com/section/well/family}{Family}

\href{https://myaccount.nytimes.com/auth/login?response_type=cookie\&client_id=vi}{}

\href{https://www.nytimes.com/section/todayspaper}{Today's Paper}

\href{/section/well/family}{Family}\textbar{}Caring for Children With
Multisystem Inflammatory Syndrome

\href{https://nyti.ms/2NJOpnT}{https://nyti.ms/2NJOpnT}

\begin{itemize}
\item
\item
\item
\item
\item
\item
\end{itemize}

\href{https://www.nytimes.com/news-event/coronavirus?action=click\&pgtype=Article\&state=default\&region=TOP_BANNER\&context=storylines_menu}{The
Coronavirus Outbreak}

\begin{itemize}
\tightlist
\item
  live\href{https://www.nytimes.com/2020/08/08/world/coronavirus-updates.html?action=click\&pgtype=Article\&state=default\&region=TOP_BANNER\&context=storylines_menu}{Latest
  Updates}
\item
  \href{https://www.nytimes.com/interactive/2020/us/coronavirus-us-cases.html?action=click\&pgtype=Article\&state=default\&region=TOP_BANNER\&context=storylines_menu}{Maps
  and Cases}
\item
  \href{https://www.nytimes.com/interactive/2020/science/coronavirus-vaccine-tracker.html?action=click\&pgtype=Article\&state=default\&region=TOP_BANNER\&context=storylines_menu}{Vaccine
  Tracker}
\item
  \href{https://www.nytimes.com/interactive/2020/world/coronavirus-tips-advice.html?action=click\&pgtype=Article\&state=default\&region=TOP_BANNER\&context=storylines_menu}{F.A.Q.}
\item
  \href{https://www.nytimes.com/live/2020/08/07/business/stock-market-today-coronavirus?action=click\&pgtype=Article\&state=default\&region=TOP_BANNER\&context=storylines_menu}{Markets
  \& Economy}
\end{itemize}

Advertisement

\protect\hyperlink{after-top}{Continue reading the main story}

Supported by

\protect\hyperlink{after-sponsor}{Continue reading the main story}

The Checkup

\hypertarget{caring-for-children-with-multisystem-inflammatory-syndrome}{%
\section{Caring for Children With Multisystem Inflammatory
Syndrome}\label{caring-for-children-with-multisystem-inflammatory-syndrome}}

Now, nearly two months after the first cases were reported, doctors can
reassure parents that the syndrome remains rare, while continuing to
urge vigilance.

\includegraphics{https://static01.nyt.com/images/2020/06/29/well/29klass-misc/29klass-misc-articleLarge.jpg?quality=75\&auto=webp\&disable=upscale}

By \href{https://www.nytimes.com/by/perri-klass-md}{Perri Klass, M.D.}

\begin{itemize}
\item
  June 29, 2020
\item
  \begin{itemize}
  \item
  \item
  \item
  \item
  \item
  \item
  \end{itemize}
\end{itemize}

When doctors began to recognize a
\href{https://www.nytimes.com/2020/05/05/nyregion/children-Kawasaki-syndrome-coronavirus.html}{serious
inflammatory syndrome} affecting children, apparently related in some
way to Covid-19 infections, it was scary news for parents and
pediatricians alike. Everyone had taken some comfort in the idea that
this terrible and terrifying pandemic was in large part sparing the
young.

Nobody could tell how common this new syndrome would turn out to be, and
people worried that the surges of coronavirus infections and deaths
among adults might be trailed by increasing numbers of children who had
perhaps had mild or asymptomatic infections, but would later develop
this mysterious illness.

Now, nearly two months after the first cases were reported, doctors can
reassure parents that the syndrome remains rare, while continuing to
urge vigilance. We do need to identify these rare children, because
while they may start out with very common symptoms, they can become very
sick --- and also because we have therapies which seem to be working ---
which is the other cautiously positive news.

In the United States, we call the new pediatric inflammatory syndrome
related to Covid-19 infection MIS-C, for multisystem inflammatory
syndrome in children, while in Britain, it's called PIMS-TS, for
pediatric inflammatory multisystem syndrome, temporally associated with
SARS-CoV-2. The hallmarks of the syndrome include prolonged fever,
without any clear cause, and a variety of symptoms of inflammation,
including rash. Laboratory tests show marked inflammation.

The doctors taking care of these children around the world, pediatric
hospitalists and intensivists, rheumatologists, infectious disease
specialists, cardiologists, immunologists, are absorbed in the unusual
experience of managing and figuring out a new syndrome, a new entity, a
new disease. And as with other aspects of the current pandemic, clinical
cooperation and medical research are happening on an unprecedented
level.

Dr. Gail Shust, an associate professor in the division of pediatric
infectious diseases at New York University School of Medicine, said that
different specialties within the same hospital, and different hospitals
across the city and around the world, are connecting on daily calls to
discuss patients and therapies.

In New York City, Dr. Shust said, doctors are seeing a decline in these
cases, though everyone is worried about what is going to happen in other
states, as the pandemic flares.

\hypertarget{latest-updates-the-coronavirus-outbreak}{%
\section{\texorpdfstring{\href{https://www.nytimes.com/2020/08/07/world/covid-19-news.html?action=click\&pgtype=Article\&state=default\&region=MAIN_CONTENT_1\&context=storylines_live_updates}{Latest
Updates: The Coronavirus
Outbreak}}{Latest Updates: The Coronavirus Outbreak}}\label{latest-updates-the-coronavirus-outbreak}}

Updated 2020-08-08T12:04:28.992Z

\begin{itemize}
\tightlist
\item
  \href{https://www.nytimes.com/2020/08/07/world/covid-19-news.html?action=click\&pgtype=Article\&state=default\&region=MAIN_CONTENT_1\&context=storylines_live_updates\#link-1f86d03a}{As
  the U.S. relief talks falter again, Trump says he is prepared to act
  on his own.}
\item
  \href{https://www.nytimes.com/2020/08/07/world/covid-19-news.html?action=click\&pgtype=Article\&state=default\&region=MAIN_CONTENT_1\&context=storylines_live_updates\#link-3f64a70a}{Cuomo
  says N.Y. schools can reopen in-person but leaves it up to districts
  to determine if, when and how.}
\item
  \href{https://www.nytimes.com/2020/08/07/world/covid-19-news.html?action=click\&pgtype=Article\&state=default\&region=MAIN_CONTENT_1\&context=storylines_live_updates\#link-14e70066}{Thousands
  of cases went unreported in California when a computer server failed.}
\end{itemize}

\href{https://www.nytimes.com/2020/08/07/world/covid-19-news.html?action=click\&pgtype=Article\&state=default\&region=MAIN_CONTENT_1\&context=storylines_live_updates}{See
more updates}

More live coverage:
\href{https://www.nytimes.com/live/2020/08/07/business/stock-market-today-coronavirus?action=click\&pgtype=Article\&state=default\&region=MAIN_CONTENT_1\&context=storylines_live_updates}{Markets}

The symptoms are common in any pediatric practice: fever, rash, upset
stomachs. ``When I get calls about kids with fever and rash, my thought
process has changed, and I worry about things I didn't worry about
before,'' said Dr. Leora Mogilner, an associate professor of pediatrics
at Mount Sinai who sees patients at their pediatric associates practice.

Back in March or April, if a child had vomiting and diarrhea, doctors
sought to counsel parents on how to manage it at home. ``Now we're
looking at it through a different lens,'' Dr. Mogilner said. ``We're
asking, did the child have Covid, or more usually a family member?''

If there's any concern that a child is not looking well, she said, that
child needs to come in, get vital signs checked, and get lab work done.
``It's very anxiety-provoking for pediatricians in primary care, and
also for very seasoned parents,'' she said.

Dr. Philip Kahn, a pediatric rheumatologist at New York University
School of Medicine, said some need intensive care, and others have
relatively mild disease and may not get treated at all. But even most of
the sick ones do well, he said: ``The vast majority need treatment,
they're treated, they go home,'' he said.

More detailed reports are appearing, detailing the clinical presentation
and the clinical course of these children. An
\href{https://jamanetwork.com/journals/jama/fullarticle/2767209}{article}
published this month in JAMA looked at 58 children hospitalized in
England. The median age was 9 years old. All of the children had
persistent fevers, ranging from 3 to 19 days, and 45 of them had
evidence of current or past Covid-19 infection.

Dr. Michael Levin, who is professor of international child health at
Imperial College London, who was the senior author on the study, said
that the children fall into three groups. Most concerning are the group
with critical illness, with shock and multi-organ failure, particularly
affecting the heart muscle.

Then there is a group of children who don't require intensive care, but
meet criteria for Kawasaki disease, a different illness with a
constellation of similar clinical features, including fever, rash,
conjunctivitis, red swollen hands and feet, and swollen lymph nodes.

And finally, he said, there is a much larger group of children with
persistent fever who may have one or two of those features, but whose
laboratory results indicate a high degree of inflammation.

Gastrointestinal symptoms were common, with half of the children in the
study having abdominal pain, and many with vomiting and diarrhea. Thirty
of the 58 had rashes, 26 had conjunctivitis. Some had sore throats,
headaches, red swollen hands and feet, swollen lymph nodes. And their
blood tests showed elevated markers of inflammation.

Dr. Shust in New York was also struck, she said, by how many of the
children there were coming in with gastrointestinal symptoms. ``It's
really pretty severe and prolonged abdominal pain,'' she said, ``not
just, `I have a tummy ache after I ate dinner,''' she said.

Some children with heart problems may complain of chest pain, Dr. Shust
said, or they may have cardiac dysfunction discovered as they are
checked for other symptoms. In Kawasaki disease, cardiac complications
tend to come late, after the acute illness, but in MIS-C, she said,
there may be problems early on.

\href{https://www.nytimes.com/news-event/coronavirus?action=click\&pgtype=Article\&state=default\&region=MAIN_CONTENT_3\&context=storylines_faq}{}

\hypertarget{the-coronavirus-outbreak-}{%
\subsubsection{The Coronavirus Outbreak
›}\label{the-coronavirus-outbreak-}}

\hypertarget{frequently-asked-questions}{%
\paragraph{Frequently Asked
Questions}\label{frequently-asked-questions}}

Updated August 6, 2020

\begin{itemize}
\item ~
  \hypertarget{why-are-bars-linked-to-outbreaks}{%
  \paragraph{Why are bars linked to
  outbreaks?}\label{why-are-bars-linked-to-outbreaks}}

  \begin{itemize}
  \tightlist
  \item
    Think about a bar. Alcohol is flowing. It can be loud, but it's
    definitely intimate, and you often need to lean in close to hear
    your friend. And strangers have way, way fewer reservations about
    coming up to people in a bar. That's sort of the point of a bar.
    Feeling good and close to strangers. It's no surprise, then, that
    \href{https://www.nytimes.com/2020/07/02/us/coronavirus-bars.html?action=click\&pgtype=Article\&state=default\&region=MAIN_CONTENT_3\&context=storylines_faq}{bars
    have been linked to outbreaks in several states.} Louisiana health
    officials have tied
    \href{https://www.nytimes.com/2020/06/22/us/new-coronavirus-phase.html?action=click\&pgtype=Article\&state=default\&region=MAIN_CONTENT_3\&context=storylines_faq}{at
    least 100 coronavirus cases} to bars in the Tigerland nightlife
    district in Baton Rouge. Minnesota has traced 328 recent cases to
    bars across the state.
    \href{https://www.boisestatepublicradio.org/post/bars-large-venues-close-ada-county-after-surge-coronavirus-prompts-rollback\#stream/0}{In
    Idaho}, health officials shut down bars in Ada County after
    reporting clusters of infections among young adults who had visited
    several bars in downtown Boise. Governors in
    \href{https://www.nytimes.com/2020/07/01/us/california-coronavirus-reopening.html?action=click\&pgtype=Article\&state=default\&region=MAIN_CONTENT_3\&context=storylines_faq}{California},
    \href{https://www.nytimes.com/2020/06/14/us/coronavirus-united-states.html?action=click\&pgtype=Article\&state=default\&region=MAIN_CONTENT_3\&context=storylines_faq}{Texas
    and Arizona}, where coronavirus cases are soaring, have ordered
    hundreds of newly reopened bars to shut down. Less than two weeks
    after Colorado's bars reopened at limited capacity, Gov. Jared Polis
    \href{https://www.denverpost.com/2020/06/30/colorado-bars-closed-coronavirus/}{ordered
    them to close}.
  \end{itemize}
\item ~
  \hypertarget{i-have-antibodies-am-i-now-immune}{%
  \paragraph{I have antibodies. Am I now
  immune?}\label{i-have-antibodies-am-i-now-immune}}

  \begin{itemize}
  \tightlist
  \item
    As of right now,
    \href{https://www.nytimes.com/2020/07/22/health/covid-antibodies-herd-immunity.html?action=click\&pgtype=Article\&state=default\&region=MAIN_CONTENT_3\&context=storylines_faq}{that
    seems likely, for at least several months.} There have been
    frightening accounts of people suffering what seems to be a second
    bout of Covid-19. But experts say these patients may have a
    drawn-out course of infection, with the virus taking a slow toll
    weeks to months after initial exposure. People infected with the
    coronavirus typically
    \href{https://www.nature.com/articles/s41586-020-2456-9}{produce}
    immune molecules called antibodies, which are
    \href{https://www.nytimes.com/2020/05/07/health/coronavirus-antibody-prevalence.html?action=click\&pgtype=Article\&state=default\&region=MAIN_CONTENT_3\&context=storylines_faq}{protective
    proteins made in response to an
    infection}\href{https://www.nytimes.com/2020/05/07/health/coronavirus-antibody-prevalence.html?action=click\&pgtype=Article\&state=default\&region=MAIN_CONTENT_3\&context=storylines_faq}{.
    These antibodies may} last in the body
    \href{https://www.nature.com/articles/s41591-020-0965-6}{only two to
    three months}, which may seem worrisome, but that's perfectly normal
    after an acute infection subsides, said Dr. Michael Mina, an
    immunologist at Harvard University. It may be possible to get the
    coronavirus again, but it's highly unlikely that it would be
    possible in a short window of time from initial infection or make
    people sicker the second time.
  \end{itemize}
\item ~
  \hypertarget{im-a-small-business-owner-can-i-get-relief}{%
  \paragraph{I'm a small-business owner. Can I get
  relief?}\label{im-a-small-business-owner-can-i-get-relief}}

  \begin{itemize}
  \tightlist
  \item
    The
    \href{https://www.nytimes.com/article/small-business-loans-stimulus-grants-freelancers-coronavirus.html?action=click\&pgtype=Article\&state=default\&region=MAIN_CONTENT_3\&context=storylines_faq}{stimulus
    bills enacted in March} offer help for the millions of American
    small businesses. Those eligible for aid are businesses and
    nonprofit organizations with fewer than 500 workers, including sole
    proprietorships, independent contractors and freelancers. Some
    larger companies in some industries are also eligible. The help
    being offered, which is being managed by the Small Business
    Administration, includes the Paycheck Protection Program and the
    Economic Injury Disaster Loan program. But lots of folks have
    \href{https://www.nytimes.com/interactive/2020/05/07/business/small-business-loans-coronavirus.html?action=click\&pgtype=Article\&state=default\&region=MAIN_CONTENT_3\&context=storylines_faq}{not
    yet seen payouts.} Even those who have received help are confused:
    The rules are draconian, and some are stuck sitting on
    \href{https://www.nytimes.com/2020/05/02/business/economy/loans-coronavirus-small-business.html?action=click\&pgtype=Article\&state=default\&region=MAIN_CONTENT_3\&context=storylines_faq}{money
    they don't know how to use.} Many small-business owners are getting
    less than they expected or
    \href{https://www.nytimes.com/2020/06/10/business/Small-business-loans-ppp.html?action=click\&pgtype=Article\&state=default\&region=MAIN_CONTENT_3\&context=storylines_faq}{not
    hearing anything at all.}
  \end{itemize}
\item ~
  \hypertarget{what-are-my-rights-if-i-am-worried-about-going-back-to-work}{%
  \paragraph{What are my rights if I am worried about going back to
  work?}\label{what-are-my-rights-if-i-am-worried-about-going-back-to-work}}

  \begin{itemize}
  \tightlist
  \item
    Employers have to provide
    \href{https://www.osha.gov/SLTC/covid-19/standards.html}{a safe
    workplace} with policies that protect everyone equally.
    \href{https://www.nytimes.com/article/coronavirus-money-unemployment.html?action=click\&pgtype=Article\&state=default\&region=MAIN_CONTENT_3\&context=storylines_faq}{And
    if one of your co-workers tests positive for the coronavirus, the
    C.D.C.} has said that
    \href{https://www.cdc.gov/coronavirus/2019-ncov/community/guidance-business-response.html}{employers
    should tell their employees} -\/- without giving you the sick
    employee's name -\/- that they may have been exposed to the virus.
  \end{itemize}
\item ~
  \hypertarget{what-is-school-going-to-look-like-in-september}{%
  \paragraph{What is school going to look like in
  September?}\label{what-is-school-going-to-look-like-in-september}}

  \begin{itemize}
  \tightlist
  \item
    It is unlikely that many schools will return to a normal schedule
    this fall, requiring the grind of
    \href{https://www.nytimes.com/2020/06/05/us/coronavirus-education-lost-learning.html?action=click\&pgtype=Article\&state=default\&region=MAIN_CONTENT_3\&context=storylines_faq}{online
    learning},
    \href{https://www.nytimes.com/2020/05/29/us/coronavirus-child-care-centers.html?action=click\&pgtype=Article\&state=default\&region=MAIN_CONTENT_3\&context=storylines_faq}{makeshift
    child care} and
    \href{https://www.nytimes.com/2020/06/03/business/economy/coronavirus-working-women.html?action=click\&pgtype=Article\&state=default\&region=MAIN_CONTENT_3\&context=storylines_faq}{stunted
    workdays} to continue. California's two largest public school
    districts --- Los Angeles and San Diego --- said on July 13, that
    \href{https://www.nytimes.com/2020/07/13/us/lausd-san-diego-school-reopening.html?action=click\&pgtype=Article\&state=default\&region=MAIN_CONTENT_3\&context=storylines_faq}{instruction
    will be remote-only in the fall}, citing concerns that surging
    coronavirus infections in their areas pose too dire a risk for
    students and teachers. Together, the two districts enroll some
    825,000 students. They are the largest in the country so far to
    abandon plans for even a partial physical return to classrooms when
    they reopen in August. For other districts, the solution won't be an
    all-or-nothing approach.
    \href{https://bioethics.jhu.edu/research-and-outreach/projects/eschool-initiative/school-policy-tracker/}{Many
    systems}, including the nation's largest, New York City, are
    devising
    \href{https://www.nytimes.com/2020/06/26/us/coronavirus-schools-reopen-fall.html?action=click\&pgtype=Article\&state=default\&region=MAIN_CONTENT_3\&context=storylines_faq}{hybrid
    plans} that involve spending some days in classrooms and other days
    online. There's no national policy on this yet, so check with your
    municipal school system regularly to see what is happening in your
    community.
  \end{itemize}
\end{itemize}

Depending on their clinical appearance and the severity of their
illness, these children are being treated in a variety of ways. The
sickest ones require intensive care support and ventilators. Many
receive IV immunoglobulin, which is the treatment for Kawasaki disease,
and others are getting immunomodulating drugs which affect the so-called
``cytokine storm,'' the severe immune reaction that causes many of the
symptoms.

``The good news is that even the children who have had pretty severe
dysfunction seem to be turning around pretty quickly,'' Dr. Shust said.
``We're following those kids really closely with EKGs and serial
echocardiograms; we don't know what the natural history of this syndrome
will be.''

Of the 58 children in the British study, 29 developed shock and needed
I.C.U. care, and 23 of those ended up needing time on a ventilator.
Eight developed abnormalities of their coronary arteries. Many of them
got immunoglobulin and corticosteroids, and a few got the immune
modulator drugs.

Another
\href{https://onlinelibrary.wiley.com/doi/abs/10.1002/jmv.26224}{article},
published last week by researchers at Mount Sinai, looked at 15 children
in New York City, and noted a disproportionate burden of disease in
Hispanic and African-American children.

Dr. Usha Ramachandran, an associate professor of pediatrics at Rutgers
Robert Wood Johnson Medical School, who provides primary care to
patients at Eric B. Chandler Health Center in New Brunswick, said that
parents ask her, ```You don't think it's that bad thing we're hearing
about on TV that kids are getting?'''

If children aren't sick enough to get blood tests done, she has been
following up closely with phone calls and telehealth visits. ``Luckily,
I've been able to say, I don't think so, he doesn't have the fever, but
we're going to watch him closely, and if things take a turn for the
worse, call me right back, take him to the emergency room.''

``Hospitals and E.R.s are set up to look for this and pick up on this
diagnosis and triage the kids appropriately,'' Dr. Kahn said,
``Everyone's on heightened alert.''

Dr. Levin said, ``While it is a serious disease, it is very rare.'' He
said that the many children in New York City who have contracted the
coronavirus, most will have had very mild or asymptomatic infections and
only a small proportion will end up with multisystem inflammatory
syndrome.

``Don't panic, but if your child has a prolonged fever and these
symptoms, it's worth calling your pediatrician,'' Dr. Shust said.
``Parents know better than anyone else if a kid is not acting right.''

Advertisement

\protect\hyperlink{after-bottom}{Continue reading the main story}

\hypertarget{site-index}{%
\subsection{Site Index}\label{site-index}}

\hypertarget{site-information-navigation}{%
\subsection{Site Information
Navigation}\label{site-information-navigation}}

\begin{itemize}
\tightlist
\item
  \href{https://help.nytimes.com/hc/en-us/articles/115014792127-Copyright-notice}{©~2020~The
  New York Times Company}
\end{itemize}

\begin{itemize}
\tightlist
\item
  \href{https://www.nytco.com/}{NYTCo}
\item
  \href{https://help.nytimes.com/hc/en-us/articles/115015385887-Contact-Us}{Contact
  Us}
\item
  \href{https://www.nytco.com/careers/}{Work with us}
\item
  \href{https://nytmediakit.com/}{Advertise}
\item
  \href{http://www.tbrandstudio.com/}{T Brand Studio}
\item
  \href{https://www.nytimes.com/privacy/cookie-policy\#how-do-i-manage-trackers}{Your
  Ad Choices}
\item
  \href{https://www.nytimes.com/privacy}{Privacy}
\item
  \href{https://help.nytimes.com/hc/en-us/articles/115014893428-Terms-of-service}{Terms
  of Service}
\item
  \href{https://help.nytimes.com/hc/en-us/articles/115014893968-Terms-of-sale}{Terms
  of Sale}
\item
  \href{https://spiderbites.nytimes.com}{Site Map}
\item
  \href{https://help.nytimes.com/hc/en-us}{Help}
\item
  \href{https://www.nytimes.com/subscription?campaignId=37WXW}{Subscriptions}
\end{itemize}
