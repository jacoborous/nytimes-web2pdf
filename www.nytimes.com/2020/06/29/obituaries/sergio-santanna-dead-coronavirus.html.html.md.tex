Sections

SEARCH

\protect\hyperlink{site-content}{Skip to
content}\protect\hyperlink{site-index}{Skip to site index}

\href{https://www.nytimes.com/section/obituaries}{Obituaries}

\href{https://myaccount.nytimes.com/auth/login?response_type=cookie\&client_id=vi}{}

\href{https://www.nytimes.com/section/todayspaper}{Today's Paper}

\href{/section/obituaries}{Obituaries}\textbar{}Sérgio Sant'Anna,
Brazilian Master of the Short Story, Dies at 78

\url{https://nyti.ms/3idJOsl}

\begin{itemize}
\item
\item
\item
\item
\item
\end{itemize}

\href{https://www.nytimes.com/news-event/coronavirus?action=click\&pgtype=Article\&state=default\&region=TOP_BANNER\&context=storylines_menu}{The
Coronavirus Outbreak}

\begin{itemize}
\tightlist
\item
  live\href{https://www.nytimes.com/2020/08/03/world/coronavirus-covid-19.html?action=click\&pgtype=Article\&state=default\&region=TOP_BANNER\&context=storylines_menu}{Latest
  Updates}
\item
  \href{https://www.nytimes.com/interactive/2020/us/coronavirus-us-cases.html?action=click\&pgtype=Article\&state=default\&region=TOP_BANNER\&context=storylines_menu}{Maps
  and Cases}
\item
  \href{https://www.nytimes.com/interactive/2020/science/coronavirus-vaccine-tracker.html?action=click\&pgtype=Article\&state=default\&region=TOP_BANNER\&context=storylines_menu}{Vaccine
  Tracker}
\item
  \href{https://www.nytimes.com/2020/08/02/us/covid-college-reopening.html?action=click\&pgtype=Article\&state=default\&region=TOP_BANNER\&context=storylines_menu}{College
  Reopening}
\item
  \href{https://www.nytimes.com/live/2020/08/03/business/stock-market-today-coronavirus?action=click\&pgtype=Article\&state=default\&region=TOP_BANNER\&context=storylines_menu}{Economy}
\end{itemize}

Advertisement

\protect\hyperlink{after-top}{Continue reading the main story}

Supported by

\protect\hyperlink{after-sponsor}{Continue reading the main story}

Those We've Lost

\hypertarget{suxe9rgio-santanna-brazilian-master-of-the-short-story-dies-at-78}{%
\section{Sérgio Sant'Anna, Brazilian Master of the Short Story, Dies at
78}\label{suxe9rgio-santanna-brazilian-master-of-the-short-story-dies-at-78}}

Mr. Sant'Anna wrote novels and poetry, but was most famous for stories
that used a sardonic humor to skewer the fractures within Brazilian
society. He died of the coronavirus.

\includegraphics{https://static01.nyt.com/images/2020/07/02/obituaries/26Santanna/merlin_173950731_d4a020c2-b196-46c4-9676-20f1de036170-articleLarge.jpg?quality=75\&auto=webp\&disable=upscale}

By Michael Astor

\begin{itemize}
\item
  Published June 29, 2020Updated July 1, 2020
\item
  \begin{itemize}
  \item
  \item
  \item
  \item
  \item
  \end{itemize}
\end{itemize}

\emph{This obituary is part of a series about people who have died in
the coronavirus pandemic. Read about others}
\href{https://www.nytimes.com/interactive/2020/obituaries/people-died-coronavirus-obituaries.html}{\emph{here}}\emph{.}

In one of Sérgio Sant'Anna's last stories, a goal post narrates an
imaginary soccer scrimmage. The main character in his best-known novel
is tortured by the government to reveal the answers to questions on an
elementary school test.

Such experimentation, combined with a knack for finding philosophical
dilemmas in everyday situations, made Mr. Sant'Anna one of Brazil's most
popular and influential authors.

He died on May 10 in a Rio de Janeiro hospital from Covid-19, his sister
Sonia said. He was 78.

Mr. Sant'Anna wrote novels and poetry but was most famous for his short
stories, which skewered the fractures within Brazilian society with
sardonic humor. His deceptively simple writing and allusions mostly went
over the heads of the government censors during the country's 1964-86
military dictatorship, even if personally he was not always so lucky.

He was put before a military inquiry and fired from his administrative
job with the government oil company, Petrobras, for union activity when
the military took power.

He also worked as a typist and editor with a labor tribunal, and was
later a professor at the Federal University of Rio de Janeiro until
retiring in 1990.

Sérgio Andrade Sant'Anna e Silva was born on Oct. 30, 1941, in Rio de
Janeiro to Sebastiao de Sant'Anna e Silva, an economics professor, and
Maria Jose Andrade de Sant'Anna e Silva, a homemaker. His family moved
to Belo Horizonte, the capital of Minas Gerais state, in 1959, and he
graduated from the federal university there in 1966.

He did postgraduate work at the Paris Institute of Political Studies,
commonly known as Sciences Po, from 1967 to 1968 and attended the
International Writing Program at the University of Iowa in 1970 on a
fellowship.

He self-published his first book of short stories, ``Survivor,'' in
1969. Among his best-known books are ``Confessions of Ralfo: An
Imaginary Autobiography'' (1975), which contains the absurdist torture
scenes, and ``The Monster'' (1994), a story collection. His last book,
``Nocturnal Angel,'' was published in 2017.

Little of Mr. Sant'Anna's work has been translated into English, but his
short story ``Miss Simpson'' became the basis for
\href{https://www.nytimes.com/2000/04/28/movies/film-review-sleepwalking-to-dreamy-rhythms-in-sumptuous-rio.html}{the
film ``Bossa Nova,''} a romantic farce starring Amy Irving and directed
by Bruno Barreto.

The experimental nature of Mr. Sant'Anna's work inspired the next
generation of Brazilian writers to explore different forms and narrative
techniques.

He was also a two-time winner of the Jabuti Prize, Brazil's main
literary award.

He is survived by his siblings, Ivan and Sonia; his wife, Mariza
Werneck-Muniz; his children, Andre and Paula; and a grandchild.

Nelson Vieira, a professor emeritus of Portuguese and Brazilian studies
at Brown University and a friend of Mr. Sant'Anna's, said his writing
was characterized by a constant search for new ways to tell a story.

``He didn't try style for style's sake, but he would always take a very
unconventional approach,'' Professor Vieira said in a telephone
interview. ``He was not a cookie cutter.''

\href{https://www.nytimes.com/interactive/2020/obituaries/people-died-coronavirus-obituaries.html?action=click\&pgtype=Article\&state=default\&region=BELOW_MAIN_CONTENT\&context=covid_obits_promo}{}

\hypertarget{those-weve-lost}{%
\section{Those We've Lost}\label{those-weve-lost}}

The coronavirus pandemic has taken an incalculable death toll. This
series is designed to put names and faces to the numbers.

Read more

\includegraphics{https://static01.nyt.com/images/2020/07/30/obituaries/30Pedro/30Pedro-square640.jpg}

\hypertarget{bernaldina-josuxe9-pedro}{%
\section{Bernaldina José Pedro}\label{bernaldina-josuxe9-pedro}}

d. Boa Vista, Brazil

Leader among the Indigenous Macuxi

\includegraphics{https://static01.nyt.com/images/2020/07/31/obituaries/31Swing/merlin_175167783_8913bc90-0d64-43f3-a655-1bb1bf1601c9-square640.jpg}

\hypertarget{john-eric-swing}{%
\section{John Eric Swing}\label{john-eric-swing}}

d. Fountain Valley, Calif.

Champion of Filipino-Americans

\includegraphics{https://static01.nyt.com/images/2020/07/27/obituaries/27Victor/merlin_175001436_38b11f8e-227a-4e2c-9821-7618af9b2524-square640.jpg}

\hypertarget{victor-victor}{%
\section{Victor Victor}\label{victor-victor}}

d. Santo Domingo, Dominican Republic

Beloved musician of the Dominican Republic

\includegraphics{https://static01.nyt.com/images/2020/07/31/obituaries/31Negron/merlin_175160169_516322ae-fd23-4969-b6b2-193ced371105-square640.jpg}

\hypertarget{dr-eddie-negruxf3n}{%
\section{Dr. Eddie Negrón}\label{dr-eddie-negruxf3n}}

d. Fort Walton Beach, Fla.

Internist on Florida's Emerald Coast

\includegraphics{https://static01.nyt.com/images/2020/07/30/obituaries/30Dobson/merlin_175115928_f6b9271c-8f05-4fe1-a38a-5ca4a58f8935-square640.jpg}

\hypertarget{dobby-dobson}{%
\section{Dobby Dobson}\label{dobby-dobson}}

d. Coral Springs, Fla.

Jamaican singer and songwriter

\includegraphics{https://static01.nyt.com/images/2020/08/01/obituaries/28Gonzalez/merlin_175002771_beb57888-3951-409a-ae13-03a94b2e962e-square640.jpg}

\hypertarget{waldemar-gonzalez}{%
\section{Waldemar Gonzalez}\label{waldemar-gonzalez}}

d. White Plains, N.Y.

Teacher and social worker

Advertisement

\protect\hyperlink{after-bottom}{Continue reading the main story}

\hypertarget{site-index}{%
\subsection{Site Index}\label{site-index}}

\hypertarget{site-information-navigation}{%
\subsection{Site Information
Navigation}\label{site-information-navigation}}

\begin{itemize}
\tightlist
\item
  \href{https://help.nytimes.com/hc/en-us/articles/115014792127-Copyright-notice}{©~2020~The
  New York Times Company}
\end{itemize}

\begin{itemize}
\tightlist
\item
  \href{https://www.nytco.com/}{NYTCo}
\item
  \href{https://help.nytimes.com/hc/en-us/articles/115015385887-Contact-Us}{Contact
  Us}
\item
  \href{https://www.nytco.com/careers/}{Work with us}
\item
  \href{https://nytmediakit.com/}{Advertise}
\item
  \href{http://www.tbrandstudio.com/}{T Brand Studio}
\item
  \href{https://www.nytimes.com/privacy/cookie-policy\#how-do-i-manage-trackers}{Your
  Ad Choices}
\item
  \href{https://www.nytimes.com/privacy}{Privacy}
\item
  \href{https://help.nytimes.com/hc/en-us/articles/115014893428-Terms-of-service}{Terms
  of Service}
\item
  \href{https://help.nytimes.com/hc/en-us/articles/115014893968-Terms-of-sale}{Terms
  of Sale}
\item
  \href{https://spiderbites.nytimes.com}{Site Map}
\item
  \href{https://help.nytimes.com/hc/en-us}{Help}
\item
  \href{https://www.nytimes.com/subscription?campaignId=37WXW}{Subscriptions}
\end{itemize}
