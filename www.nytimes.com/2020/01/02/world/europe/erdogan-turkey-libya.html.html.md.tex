Sections

SEARCH

\protect\hyperlink{site-content}{Skip to
content}\protect\hyperlink{site-index}{Skip to site index}

\href{https://www.nytimes.com/section/world/europe}{Europe}

\href{https://myaccount.nytimes.com/auth/login?response_type=cookie\&client_id=vi}{}

\href{https://www.nytimes.com/section/todayspaper}{Today's Paper}

\href{/section/world/europe}{Europe}\textbar{}Turkey, Flexing Its
Muscles, Will Send Troops to Libya

\url{https://nyti.ms/2MNI0YN}

\begin{itemize}
\item
\item
\item
\item
\item
\item
\end{itemize}

Advertisement

\protect\hyperlink{after-top}{Continue reading the main story}

Supported by

\protect\hyperlink{after-sponsor}{Continue reading the main story}

\hypertarget{turkey-flexing-its-muscles-will-send-troops-to-libya}{%
\section{Turkey, Flexing Its Muscles, Will Send Troops to
Libya}\label{turkey-flexing-its-muscles-will-send-troops-to-libya}}

A new wrinkle in the battle for an oil-rich country, and a signal from
President Erdogan that Turkey aims to be a power broker in a volatile
region.

\includegraphics{https://static01.nyt.com/images/2019/12/31/world/00turkey-foreign-policy1/00turkey-foreign-policy1-articleLarge-v2.jpg?quality=75\&auto=webp\&disable=upscale}

\href{https://www.nytimes.com/by/carlotta-gall}{\includegraphics{https://static01.nyt.com/images/2018/10/10/multimedia/author-carlotta-gall/author-carlotta-gall-thumbLarge.png}}

By \href{https://www.nytimes.com/by/carlotta-gall}{Carlotta Gall}

\begin{itemize}
\item
  Jan. 2, 2020
\item
  \begin{itemize}
  \item
  \item
  \item
  \item
  \item
  \item
  \end{itemize}
\end{itemize}

ISTANBUL --- Turkey's Parliament approved plans on Thursday to send
troops to Libya, escalating what has become a chaotic proxy war between
multiple powers for control of the oil-rich country.

The Turkish deployment, a dramatic intervention championed by President
Recep Tayyip Erdogan, aims to bolster the fragile United Nations-backed
government in Tripoli after nine months of siege from rebel forces based
in eastern Libya.

The size and nature of the military deployment was unclear. But coming
just months after Turkey's third military incursion into Syria, it
expands Turkey's military footprint in a volatile region and, analysts
say, offers new evidence of its growing self-confidence as a regional
power.

For Libya's embattled government, Mr. Erdogan has become an essential
patron. Already this year, he has sent military advisers, arms and a
fleet of 20 drones to defend Tripoli from the forces of Gen. Khalifa
Hifter, which control much of eastern Libya and are backed by Russia,
Saudi Arabia, the United Arab Emirates and Egypt.

According the Syrian Observatory for Human Rights, an independent
monitoring organization, Turkey has already sent Syrian proxy fighters
to Libya, and more have assembled in training camps in Turkey ahead of
deployment.

But in recent weeks, General Hifter's forces have gained the upper hand
in the battle for Tripoli. Buoyed by the
\href{https://www.nytimes.com/2019/11/05/world/middleeast/russia-libya-mercenaries.html}{arrival
of Kremlin-backed Russian mercenaries}, and armed with
\href{https://www.bloomberg.com/news/articles/2019-12-21/libya-security-chief-says-russians-spearheading-tripoli-fighting}{sophisticated
drone-jamming technology}, they have pushed farther into Tripoli,
tightening their grip on the capital.

``In recent days, things have been quite bad for the government on the
front line,'' said Emad Badi, a Libya scholar at the European University
Institute in Florence, Italy. ``They might have held on for just another
week. Now I don't expect Turkey to allow that to happen.''

Much depends, Mr. Badi said, on what Turkey will bring to the fight and
how quickly those reinforcements will arrive. Turkey could deploy a
warship off the coast of Misurata, a govern-controlled city east of
Tripoli, he said. Or it could deploy fighter jets to combat General
Hifter's attack drones in the air.

In any event, such maneuvers might pause the fight for Libya, but not
decide it. ``It will be an incremental approach,'' Mr. Badi said, adding
that Turkey was likely to deploy the minimum resources needed to repel
the offensive on Tripoli, and no more.

President Trump spoke with Mr. Erdogan on Thursday, the White House
said, and ``pointed out that foreign interference is complicating the
situation in Libya.'' The White House did not say whether Mr. Trump had
asked Turkey to refrain from sending troops.

Mr. Erdogan has long held an ambition for a kind of restoration of the
Ottoman Empire, re-establishing Turkey's position of leadership in the
Muslim world with an expansive foreign policy.

His stance, in alliance with the wealthy Gulf state Qatar, has pitted
Turkey against Saudi Arabia, Egypt and the United Arab Emirates. The
opposing groups represent a new fault line in the Middle East, having
backed opposite sides of the Arab Spring uprisings and rival forces in
Libya and Syria.

An intervention in Libya may deepen Turkey's feud with the Emirates,
which is General Hifter's main backer. But it may also open a new
conflict with Russia, which has been a partner of Turkey's in Syria and
recently sold Turkey an advanced antiaircraft missile system.

At home, though. Mr. Erdogan's geopolitical aspirations are popular. The
mission in Libya, part of the former Ottoman domain, fits neatly into
his vision of restoration. His assertive foreign policy has also given
Mr. Erdogan a handy slate of challengers that he can point to abroad,
helping him nurture nationalism and maintain his support domestically.

Six months after his party's
\href{https://www.nytimes.com/2019/06/23/world/europe/istanbul-mayor-election-erdogan.html}{loss
of Istanbul's mayoralty in local elections} --- his most significant
electoral setback in a 25-year political career --- Mr. Erdogan, 65, is
pondering holding general elections in 2020, according to some political
analysts. Although his term runs until 2023,
\href{https://www.nytimes.com/2019/12/23/world/middleeast/istanbul-mayor-erdogan.html}{his
slide in the polls} and the
\href{https://www.nytimes.com/2019/12/13/world/europe/ahmet-davutoglu-erdogan-turkey.html}{splintering
of his party} are making him consider calling a snap election in the
fall, Mehmet Ali Kulat, a political consultant and pollster in Ankara,
said.

A
\href{https://www.nytimes.com/2019/07/08/business/turkey-economy-crisis.html}{faltering
Turkish economy} may only add urgency to the president's considerations.

\includegraphics{https://static01.nyt.com/images/2019/12/31/world/00turkey-foreign-policy2/merlin_162481755_4522f8a2-9e5e-409f-a2da-950cb9450d49-articleLarge.jpg?quality=75\&auto=webp\&disable=upscale}

Mr. Erdogan's assertive posturing has helped stir up nationalist
feelings and rally his core supporters, Ali Bayramoglu, who was close to
Mr. Erdogan's party in its early years, said.

``Our right-wing parties did not use to act like they did not care about
the United States,'' Mr. Bayramoglu said. ``This independence, this
challenging is a new thing. Turkish right-wing voters love it.''

With some justification Mr. Erdogan has argued that he has security
interests in Iraq and Syria, since Turkey shares a long border with both
and has suffered from instability spilling over from their conflicts.
With Libya he has made similar arguments, as well as historical ones.

Mr. Erdogan has noted that Libya was the last of the Ottoman territories
to be lost and that Turkey's founder, Mustafa Kemal Ataturk, fought and
was wounded there as a young officer.

``It's not difficult to convince Turkish public about the need for an
intervention in Libya, in part because of the Ottoman legacy,'' Asli
Aydintasbas, a senior fellow of the European Council for Human Rights,
said in written comments.

There are important Turkish commercial interests at stake, too. Beneath
Mr. Erdogan's agreement with Libya is a desire to position Turkey for
oil and gas exploration in the eastern Mediterranean, off the coast of
Cyprus, in competition with Greece, Cyprus, Egypt and Israel, analysts
say.

``Turkey does not want to be frozen out of the great game which revolves
around the hydrocarbon deposits in the Eastern Mediterranean,'' Ms.
Aydintasbas said.

As in Syria, Turkey wants to have troops on the ground in Libya to gain
a place at the table, she said.

Image

Mr. Erdogan and his Russian counterpart, Vladimir V. Putin, in August.
Though on opposing sides in Syria and Libya, the two have found ways to
avoid direct confrontation. Credit...Pool photo by Maxim Shipenkov

There is also an ethnic link. The main Libyan faction backing Tripoli is
from Misurata, whose population is mostly ethnically Turkish and traces
its roots to Turkey.

Turkey has already signed an agreement for an exclusive economic area
with the Tripoli government. If the Libyan government falls, the
agreement would fall with it. So Mr. Erdogan is trying to protect that
agreement, said Ozgur Unluhisarcikli, director in Ankara of the German
Marshall Fund of the United States.

Image

A Turkish drilling vessel being escorted by a ship from the country's
navy in the eastern Mediterranean, off Cyprus, in August.Credit...Murad
Sezer/Reuters

Mr. Erdogan's ever more aggressive angling has unnerved his neighbors,
especially in Greece, who now openly worry about confrontation.
Diplomats in Athens and Brussels say the situation is tensest it has
been in more than two decades, since 1996, when the two neighbors
exchanged fire in the Aegean.

Under pressure, Prime Minister Kyriakos Mitsotakis of Greece is poised
to change decades of foreign policy, announcing on Sunday that he was
prepared to take Turkey to international arbitration at the
International Court of Justice in The Hague.

But diplomats in Brussels readily admit that there is only so far they
can press Turkey, considering how desperately they depend on Mr. Erdogan
to control the flow of migrants to Europe, which was destabilized by the
arrival of more than a million asylum seekers in 2015.

Now, Mr. Erdogan faces the possibility of a new refugee crisis coming
from Syria, where Russian and Syrian government forces have redoubled
their offensive in Idlib, the last rebel-held province.

Seeking leverage from the potential for a crisis as Chancellor Angela
Merkel of Germany prepares to visit Turkey this month, Mr. Erdogan has
warned that he will be
\href{https://www.nytimes.com/2019/12/23/world/middleeast/syria-idlib-russia-aid-refugees.html}{forced
to open the gates for the refugees} to enter Europe again.

Image

Syrian children at a camp for displaced people near the border with
Turkey.Credit...Aaref Watad/Agence France-Presse --- Getty Images

Elsewhere, Mr. Erdogan's flirtation with Russia has stirred American and
European ire. So has his foray into Syria, despite President Trump's
apparent green light for it.

Typically, Mr. Erdogan is having none of it.

``Of course, everyone is giving advice to us: `What are you doing in
Syria?' they say. `When will you leave Syria?''' Mr. Erdogan said in
London.

``We have only one answer to them: `What are you doing in Syria? Do you
have a border there? No. And, what are you doing there? You go there
from a distance of 10,000 kilometers, 3,000, 5,000. But we have a 911
kilometer-long border.'''

Mr. Trump has so far not acted on attempts by Congress to punish Turkey
for its purchase of the Russian S400 missile system and for violating
United States sanctions against Iran.

That has not stopped Mr. Erdogan from threatening to close off United
States access to Turkish bases, including Incirlik Air Base, which
houses roughly
\href{https://www.nytimes.com/2019/10/14/world/middleeast/trump-turkey-syria.html}{50
American tactical nuclear weapons}.

Mr. Erdogan has also remained loyal to the Arab Spring uprisings since
they began in 2011, placing him at odds with longstanding dictators in
Egypt, Tunisia and Libya, as well as the United States.

Even as other Western nations ceased support when extremists took over
the uprisings or, as in the case of Egypt, a counterrevolution ousted
the elected Islamist government, Turkey supported the Islamist-leaning
groups that emerged from the uprisings.

That includes Libya, underlining Mr. Erdogan's desire for an independent
foreign policy.

Image

Turkey-backed rebel groups based out of Tripoli and Misurata from the
start of the uprising in Libya. Credit...Bryan Denton for The New York
Times

``Today there is a Turkey with an independent foreign policy, making
operations for its own national security without looking for permission
from anyone,'' he said in London.

Declan Walsh contributed reporting from Ballina, Ireland, Michael
Crowley from Washington, Hwaida Saad from Beirut, and Matina
Stevis-Gridneff from Brussels.

Advertisement

\protect\hyperlink{after-bottom}{Continue reading the main story}

\hypertarget{site-index}{%
\subsection{Site Index}\label{site-index}}

\hypertarget{site-information-navigation}{%
\subsection{Site Information
Navigation}\label{site-information-navigation}}

\begin{itemize}
\tightlist
\item
  \href{https://help.nytimes.com/hc/en-us/articles/115014792127-Copyright-notice}{©~2020~The
  New York Times Company}
\end{itemize}

\begin{itemize}
\tightlist
\item
  \href{https://www.nytco.com/}{NYTCo}
\item
  \href{https://help.nytimes.com/hc/en-us/articles/115015385887-Contact-Us}{Contact
  Us}
\item
  \href{https://www.nytco.com/careers/}{Work with us}
\item
  \href{https://nytmediakit.com/}{Advertise}
\item
  \href{http://www.tbrandstudio.com/}{T Brand Studio}
\item
  \href{https://www.nytimes.com/privacy/cookie-policy\#how-do-i-manage-trackers}{Your
  Ad Choices}
\item
  \href{https://www.nytimes.com/privacy}{Privacy}
\item
  \href{https://help.nytimes.com/hc/en-us/articles/115014893428-Terms-of-service}{Terms
  of Service}
\item
  \href{https://help.nytimes.com/hc/en-us/articles/115014893968-Terms-of-sale}{Terms
  of Sale}
\item
  \href{https://spiderbites.nytimes.com}{Site Map}
\item
  \href{https://help.nytimes.com/hc/en-us}{Help}
\item
  \href{https://www.nytimes.com/subscription?campaignId=37WXW}{Subscriptions}
\end{itemize}
