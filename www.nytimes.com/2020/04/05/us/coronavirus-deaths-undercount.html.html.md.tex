Sections

SEARCH

\protect\hyperlink{site-content}{Skip to
content}\protect\hyperlink{site-index}{Skip to site index}

\href{/section/us}{U.S.}\textbar{}Official Counts Understate the U.S.
Coronavirus Death Toll

\url{https://nyti.ms/2JIkmuS}

\begin{itemize}
\item
\item
\item
\item
\item
\end{itemize}

\href{https://www.nytimes.com/news-event/coronavirus?action=click\&pgtype=Article\&state=default\&region=TOP_BANNER\&context=storylines_menu}{The
Coronavirus Outbreak}

\begin{itemize}
\tightlist
\item
  live\href{https://www.nytimes.com/2020/08/04/world/coronavirus-cases.html?action=click\&pgtype=Article\&state=default\&region=TOP_BANNER\&context=storylines_menu}{Latest
  Updates}
\item
  \href{https://www.nytimes.com/interactive/2020/us/coronavirus-us-cases.html?action=click\&pgtype=Article\&state=default\&region=TOP_BANNER\&context=storylines_menu}{Maps
  and Cases}
\item
  \href{https://www.nytimes.com/interactive/2020/science/coronavirus-vaccine-tracker.html?action=click\&pgtype=Article\&state=default\&region=TOP_BANNER\&context=storylines_menu}{Vaccine
  Tracker}
\item
  \href{https://www.nytimes.com/2020/08/02/us/covid-college-reopening.html?action=click\&pgtype=Article\&state=default\&region=TOP_BANNER\&context=storylines_menu}{College
  Reopening}
\item
  \href{https://www.nytimes.com/live/2020/08/04/business/stock-market-today-coronavirus?action=click\&pgtype=Article\&state=default\&region=TOP_BANNER\&context=storylines_menu}{Economy}
\end{itemize}

\includegraphics{https://static01.nyt.com/images/2020/04/03/us/00virus-dead01/merlin_171254238_54123ca9-32e9-4a98-815a-6eb1aa2e523a-articleLarge.jpg?quality=75\&auto=webp\&disable=upscale}

\hypertarget{official-counts-understate-the-us-coronavirus-death-toll}{%
\section{Official Counts Understate the U.S. Coronavirus Death
Toll}\label{official-counts-understate-the-us-coronavirus-death-toll}}

Inconsistent protocols, limited resources and a patchwork of decision
making have led to an undercounting of people with the coronavirus who
have died, health experts say.

Lina Evans, the coroner in Shelby County, Ala., said she is now
suspicious about a surge in deaths in her county earlier this year, many
of which involved severe pneumonias.Credit...Bob Miller for The New York
Times

Supported by

\protect\hyperlink{after-sponsor}{Continue reading the main story}

By \href{https://www.nytimes.com/by/sarah-kliff}{Sarah Kliff} and
\href{https://www.nytimes.com/by/julie-bosman}{Julie Bosman}

\begin{itemize}
\item
  Published April 5, 2020Updated April 7, 2020
\item
  \begin{itemize}
  \item
  \item
  \item
  \item
  \item
  \end{itemize}
\end{itemize}

WASHINGTON --- A coroner in Indiana wanted to know if the coronavirus
had killed a man in early March, but said that her health department
denied a test. Paramedics in New York City say that many patients who
died at home were never tested for the coronavirus, even if they showed
telltale signs of infection.

In Virginia, a funeral director prepared the remains of three people
after health workers cautioned her that they each had tested positive
for the coronavirus. But only one of the three had the virus noted on
the death certificate.

Across the United States, even as
\href{https://www.nytimes.com/interactive/2020/04/21/world/coronavirus-missing-deaths.html}{coronavirus
deaths} are being recorded in terrifying numbers --- many hundreds each
day --- the true death toll is likely much higher.

More than 9,400 people with the coronavirus have been reported to have
died in this country as of this weekend, but hospital officials,
doctors, public health experts and medical examiners say that official
counts have failed to capture the true number of Americans dying in this
pandemic. The undercount is a result of inconsistent protocols, limited
resources and a patchwork of decision making from one state or county to
the next.

In many rural areas, coroners say they don't have the tests they need to
detect the disease. Doctors now believe that some deaths in February and
early March, before the coronavirus reached epidemic levels in the
United States, were likely misidentified as influenza or only described
as pneumonia.

With no uniform system for reporting coronavirus-related deaths in the
United States, and a continued shortage of tests, some states and
counties have improvised, obfuscated and, at times, backtracked in
counting the dead.

``We definitely think there are deaths that we have not accounted for,''
said Jennifer Nuzzo, a senior scholar at the Johns Hopkins University
Center for Health Security, which studies global health threats and is
\href{http://www.centerforhealthsecurity.org/}{closely tracking} the
coronavirus pandemic.

Late last week, the Centers for Disease Control and Prevention issued
new guidance for how to certify coronavirus deaths, underscoring the
need for uniformity and reinforcing the sense by health care workers and
others that deaths have not been consistently tracked. In its
\href{https://www.cdc.gov/nchs/data/nvss/vsrg/vsrg03-508.pdf}{guidance},
the C.D.C. instructed officials to report deaths where the patient has
tested positive or, in an absence of testing, ``if the circumstances are
compelling within a reasonable degree of certainty.''

In infectious outbreaks, public health experts say that under typical
circumstances it takes months or years to compile data that is as
accurate as possible on deaths. The reporting system during an epidemic
of this scale is particularly strained. And while experts say they
believe that virus-related deaths have been missed, the extent of the
problem is not clear.

But as mayors and governors hold daily news conferences reporting the
latest figures of infections and deaths related to Covid-19, Americans
have paid close attention to the locations and numbers of the sick and
dead --- one of the few metrics available for understanding the new and
mysterious disease threatening their communities.

\hypertarget{latest-updates-global-coronavirus-outbreak}{%
\section{\texorpdfstring{\href{https://www.nytimes.com/2020/08/04/world/coronavirus-cases.html?action=click\&pgtype=Article\&state=default\&region=MAIN_CONTENT_1\&context=storylines_live_updates}{Latest
Updates: Global Coronavirus
Outbreak}}{Latest Updates: Global Coronavirus Outbreak}}\label{latest-updates-global-coronavirus-outbreak}}

Updated 2020-08-04T20:57:54.346Z

\begin{itemize}
\tightlist
\item
  \href{https://www.nytimes.com/2020/08/04/world/coronavirus-cases.html?action=click\&pgtype=Article\&state=default\&region=MAIN_CONTENT_1\&context=storylines_live_updates\#link-1228a480}{Novavax
  sees encouraging results from two studies of its experimental
  vaccine.}
\item
  \href{https://www.nytimes.com/2020/08/04/world/coronavirus-cases.html?action=click\&pgtype=Article\&state=default\&region=MAIN_CONTENT_1\&context=storylines_live_updates\#link-4825b93}{Public
  and private schools in Maryland and elsewhere are divided over
  in-person instruction.}
\item
  \href{https://www.nytimes.com/2020/08/04/world/coronavirus-cases.html?action=click\&pgtype=Article\&state=default\&region=MAIN_CONTENT_1\&context=storylines_live_updates\#link-50f7386d}{The
  United Nations calls on policymakers to `plan thoroughly for school
  reopenings.'}
\end{itemize}

\href{https://www.nytimes.com/2020/08/04/world/coronavirus-cases.html?action=click\&pgtype=Article\&state=default\&region=MAIN_CONTENT_1\&context=storylines_live_updates}{See
more updates}

More live coverage:
\href{https://www.nytimes.com/live/2020/08/04/business/stock-market-today-coronavirus?action=click\&pgtype=Article\&state=default\&region=MAIN_CONTENT_1\&context=storylines_live_updates}{Markets}

Public health experts say that an accurate count of deaths is an
essential tool to understand a disease outbreak as it unfolds: The more
deadly a disease, the more aggressively the authorities are willing to
disrupt normal life. Precise death counts can also inform the federal
government on how to target resources, like ventilators from the
national stockpile, to the areas of the country with the most desperate
need.

For families who have lost a loved one in the midst of this epidemic,
there is an urge simply to know: Was it the coronavirus?

\hypertarget{lingering-questions}{%
\subsection{Lingering questions}\label{lingering-questions}}

As the coronavirus outbreak began sweeping across the country last
month, Julio Ramirez, a 43-year-old salesman in San Gabriel, Calif.,
came home from a business trip and began feeling unwell, suffering from
a fever, cough and body aches. By the next day, he had lost his sense of
taste and smell.

His wife, Julie Murillo, took him to an urgent care clinic several days
later, where he was so weak he had to be pushed in a wheelchair. Doctors
prescribed antibiotics, a cough syrup and gave him a chest X-ray, but
they did not test for the coronavirus, she said. Just over a week after
he returned from his trip, Ms. Murillo found him dead in his bed.

``I kept trying to get him tested from the beginning,'' Ms. Murillo
said. ``They told me no.''

Frustrated, Ms. Murillo enlisted friends to call the C.D.C. on her
behalf, urging a post-mortem test. Then she hired a private company to
conduct an autopsy; the owner pleaded for a coronavirus test from local
and federal authorities.

On Saturday afternoon, 19 days after the death, Ms. Murillo received a
call from the Los Angeles County Department of Public Health, she said.
The health department had gone to the funeral home where her husband's
body was resting and taken a sample for a coronavirus test. He tested
positive.

In a statement, the health department said that post-mortem testing has
been conducted on ``a number of cases,'' but did not provide specifics
or comment on Mr. Ramirez's case.

The work of counting deaths related to the virus falls to an assortment
of health care providers, medical examiners, coroners, funeral homes and
local health departments that fill out America's death certificates. The
documents typically include information on the immediate cause of death,
such as a heart attack or pneumonia, as well as on any underlying
disease. In coronavirus cases, that would be Covid-19.

\includegraphics{https://static01.nyt.com/images/2020/04/03/us/00virus-dead02/merlin_171074487_6d60859a-8812-4e68-9f5d-27b55993331e-articleLarge.jpg?quality=75\&auto=webp\&disable=upscale}

The federal government does not expect to produce a final tally of
coronavirus deaths until 2021, when it publishes an annual compilation
of the country's leading causes of death.

A
\href{https://www.nytimes.com/interactive/2020/us/coronavirus-us-cases.html}{New
York Times tally} of known Covid-related deaths, based on reports from
state and local officials, showed 9,470 deaths as of Sunday. On Friday,
the National Center for Health Statistics, part of the C.D.C., began
publishing preliminary estimates of coronavirus deaths, although a
spokesman said that information would have a ``lag of 1-2 weeks.''
Its\href{https://www.cdc.gov/nchs/nvss/vsrr/COVID19/}{first estimate}
noted 1,150 **** deaths, based on the number of death certificates that
included Covid-19 as an underlying disease.

``It is not a `real time' count of Covid deaths, like what the states
are currently reporting,'' Jeff Lancashire, a spokesman for the National
Center for Health Statistics, said.

But those who work with death certificates say they worry that relying
only on those documents may leave out a significant number of cases in
which coronavirus was confirmed by testing, but not written down in the
section where doctors and coroners are asked to note relevant underlying
diseases. Generally,
\href{https://www.cdc.gov/nchs/data/dvs/blue_form.pdf}{certificates
require} an immediate cause, and encourage --- but do not require ---
officials to take note of an underlying disease.

Then there are the many suspected cases.

Susan Perry, the funeral director from Virginia, said that she was
informed by health workers and families that three recently deceased
people had tested positive for the virus so that she and her staff could
take necessary precautions with the bodies. Only one death certificate
mentioned the virus.

``This probably happens all the time with different diseases, but this
is the first time I'm paying attention to it,'' Ms. Perry said. ``If we
don't know the numbers, how are we going to be able to prepare ourselves
and protect ourselves?''

\hypertarget{now-were-having-the-aha-moment}{%
\subsection{`Now we're having the ``aha!''
moment'}\label{now-were-having-the-aha-moment}}

Early in the U.S. outbreak, virus-linked deaths may have been
overlooked, hospital officials said. A late start to coronavirus testing
hampered hospitals' ability to detect the infection among patients with
flulike symptoms in February and early March. Doctors at several
hospitals reported treating pneumonia patients who eventually died
before testing was available.

``When I was working before we had testing, we had a ton of patients
with pneumonia,'' said Geraldine M\textbf{é}nard, chief of general
internal medicine at Tulane Medical Center in New Orleans. ``I remember
thinking it was weird. I'm sure some of those patients did have it. But
no one knew back then.''

Image

Geraldine Ménard, chief of general internal medicine at Tulane Medical
Center in New Orleans.Credit...William Widmer for The New York Times

An emergency department physician in San Francisco recalled two deaths
that were probably coronavirus but not identified as such. One patient
died at home; a relative in the same home later tested positive for the
disease. Another patient was an older man who came to the hospital with
typical coronavirus symptoms, and who had been in contact with someone
recently traveling to China, but arrived at the hospital before testing
was available.

\href{https://www.nytimes.com/news-event/coronavirus?action=click\&pgtype=Article\&state=default\&region=MAIN_CONTENT_3\&context=storylines_faq}{}

\hypertarget{the-coronavirus-outbreak-}{%
\subsubsection{The Coronavirus Outbreak
›}\label{the-coronavirus-outbreak-}}

\hypertarget{frequently-asked-questions}{%
\paragraph{Frequently Asked
Questions}\label{frequently-asked-questions}}

Updated August 4, 2020

\begin{itemize}
\item ~
  \hypertarget{i-have-antibodies-am-i-now-immune}{%
  \paragraph{I have antibodies. Am I now
  immune?}\label{i-have-antibodies-am-i-now-immune}}

  \begin{itemize}
  \tightlist
  \item
    As of right
    now,\href{https://www.nytimes.com/2020/07/22/health/covid-antibodies-herd-immunity.html?action=click\&pgtype=Article\&state=default\&region=MAIN_CONTENT_3\&context=storylines_faq}{that
    seems likely, for at least several months.} There have been
    frightening accounts of people suffering what seems to be a second
    bout of Covid-19. But experts say these patients may have a
    drawn-out course of infection, with the virus taking a slow toll
    weeks to months after initial exposure. People infected with the
    coronavirus typically
    \href{https://www.nature.com/articles/s41586-020-2456-9}{produce}
    immune molecules called antibodies, which are
    \href{https://www.nytimes.com/2020/05/07/health/coronavirus-antibody-prevalence.html?action=click\&pgtype=Article\&state=default\&region=MAIN_CONTENT_3\&context=storylines_faq}{protective
    proteins made in response to an
    infection}\href{https://www.nytimes.com/2020/05/07/health/coronavirus-antibody-prevalence.html?action=click\&pgtype=Article\&state=default\&region=MAIN_CONTENT_3\&context=storylines_faq}{.
    These antibodies may} last in the body
    \href{https://www.nature.com/articles/s41591-020-0965-6}{only two to
    three months}, which may seem worrisome, but that's perfectly normal
    after an acute infection subsides, said Dr. Michael Mina, an
    immunologist at Harvard University. It may be possible to get the
    coronavirus again, but it's highly unlikely that it would be
    possible in a short window of time from initial infection or make
    people sicker the second time.
  \end{itemize}
\item ~
  \hypertarget{im-a-small-business-owner-can-i-get-relief}{%
  \paragraph{I'm a small-business owner. Can I get
  relief?}\label{im-a-small-business-owner-can-i-get-relief}}

  \begin{itemize}
  \tightlist
  \item
    The
    \href{https://www.nytimes.com/article/small-business-loans-stimulus-grants-freelancers-coronavirus.html?action=click\&pgtype=Article\&state=default\&region=MAIN_CONTENT_3\&context=storylines_faq}{stimulus
    bills enacted in March} offer help for the millions of American
    small businesses. Those eligible for aid are businesses and
    nonprofit organizations with fewer than 500 workers, including sole
    proprietorships, independent contractors and freelancers. Some
    larger companies in some industries are also eligible. The help
    being offered, which is being managed by the Small Business
    Administration, includes the Paycheck Protection Program and the
    Economic Injury Disaster Loan program. But lots of folks have
    \href{https://www.nytimes.com/interactive/2020/05/07/business/small-business-loans-coronavirus.html?action=click\&pgtype=Article\&state=default\&region=MAIN_CONTENT_3\&context=storylines_faq}{not
    yet seen payouts.} Even those who have received help are confused:
    The rules are draconian, and some are stuck sitting on
    \href{https://www.nytimes.com/2020/05/02/business/economy/loans-coronavirus-small-business.html?action=click\&pgtype=Article\&state=default\&region=MAIN_CONTENT_3\&context=storylines_faq}{money
    they don't know how to use.} Many small-business owners are getting
    less than they expected or
    \href{https://www.nytimes.com/2020/06/10/business/Small-business-loans-ppp.html?action=click\&pgtype=Article\&state=default\&region=MAIN_CONTENT_3\&context=storylines_faq}{not
    hearing anything at all.}
  \end{itemize}
\item ~
  \hypertarget{what-are-my-rights-if-i-am-worried-about-going-back-to-work}{%
  \paragraph{What are my rights if I am worried about going back to
  work?}\label{what-are-my-rights-if-i-am-worried-about-going-back-to-work}}

  \begin{itemize}
  \tightlist
  \item
    Employers have to provide
    \href{https://www.osha.gov/SLTC/covid-19/standards.html}{a safe
    workplace} with policies that protect everyone equally.
    \href{https://www.nytimes.com/article/coronavirus-money-unemployment.html?action=click\&pgtype=Article\&state=default\&region=MAIN_CONTENT_3\&context=storylines_faq}{And
    if one of your co-workers tests positive for the coronavirus, the
    C.D.C.} has said that
    \href{https://www.cdc.gov/coronavirus/2019-ncov/community/guidance-business-response.html}{employers
    should tell their employees} -\/- without giving you the sick
    employee's name -\/- that they may have been exposed to the virus.
  \end{itemize}
\item ~
  \hypertarget{should-i-refinance-my-mortgage}{%
  \paragraph{Should I refinance my
  mortgage?}\label{should-i-refinance-my-mortgage}}

  \begin{itemize}
  \tightlist
  \item
    \href{https://www.nytimes.com/article/coronavirus-money-unemployment.html?action=click\&pgtype=Article\&state=default\&region=MAIN_CONTENT_3\&context=storylines_faq}{It
    could be a good idea,} because mortgage rates have
    \href{https://www.nytimes.com/2020/07/16/business/mortgage-rates-below-3-percent.html?action=click\&pgtype=Article\&state=default\&region=MAIN_CONTENT_3\&context=storylines_faq}{never
    been lower.} Refinancing requests have pushed mortgage applications
    to some of the highest levels since 2008, so be prepared to get in
    line. But defaults are also up, so if you're thinking about buying a
    home, be aware that some lenders have tightened their standards.
  \end{itemize}
\item ~
  \hypertarget{what-is-school-going-to-look-like-in-september}{%
  \paragraph{What is school going to look like in
  September?}\label{what-is-school-going-to-look-like-in-september}}

  \begin{itemize}
  \tightlist
  \item
    It is unlikely that many schools will return to a normal schedule
    this fall, requiring the grind of
    \href{https://www.nytimes.com/2020/06/05/us/coronavirus-education-lost-learning.html?action=click\&pgtype=Article\&state=default\&region=MAIN_CONTENT_3\&context=storylines_faq}{online
    learning},
    \href{https://www.nytimes.com/2020/05/29/us/coronavirus-child-care-centers.html?action=click\&pgtype=Article\&state=default\&region=MAIN_CONTENT_3\&context=storylines_faq}{makeshift
    child care} and
    \href{https://www.nytimes.com/2020/06/03/business/economy/coronavirus-working-women.html?action=click\&pgtype=Article\&state=default\&region=MAIN_CONTENT_3\&context=storylines_faq}{stunted
    workdays} to continue. California's two largest public school
    districts --- Los Angeles and San Diego --- said on July 13, that
    \href{https://www.nytimes.com/2020/07/13/us/lausd-san-diego-school-reopening.html?action=click\&pgtype=Article\&state=default\&region=MAIN_CONTENT_3\&context=storylines_faq}{instruction
    will be remote-only in the fall}, citing concerns that surging
    coronavirus infections in their areas pose too dire a risk for
    students and teachers. Together, the two districts enroll some
    825,000 students. They are the largest in the country so far to
    abandon plans for even a partial physical return to classrooms when
    they reopen in August. For other districts, the solution won't be an
    all-or-nothing approach.
    \href{https://bioethics.jhu.edu/research-and-outreach/projects/eschool-initiative/school-policy-tracker/}{Many
    systems}, including the nation's largest, New York City, are
    devising
    \href{https://www.nytimes.com/2020/06/26/us/coronavirus-schools-reopen-fall.html?action=click\&pgtype=Article\&state=default\&region=MAIN_CONTENT_3\&context=storylines_faq}{hybrid
    plans} that involve spending some days in classrooms and other days
    online. There's no national policy on this yet, so check with your
    municipal school system regularly to see what is happening in your
    community.
  \end{itemize}
\end{itemize}

In New York City, emergency medical workers say that infection and death
rates are probably far higher than reported. Given a record number of
calls, many ambulance crews have encouraged anyone not critically ill to
stay home. The result, medics say, is that many presumed coronavirus
patients may never know for sure if they had the virus, so any who later
die at home may never be categorized as having had it.

Across the country, coroners are going through a process of
re-evaluation, reconsidering deaths that occurred before testing was
widely available. Coroners and medical examiners generally investigate
deaths that are considered unusual, or result from accidents or
suicides, or occur at home.

Joani Shields, the coroner in Monroe County, Ind., said she wondered
about a man diagnosed with pneumonia who died in early March.

A coronavirus test was requested at the time, but the local health
department denied it, Ms. Shields said, on the ground that the supply of
tests was too limited.

``I wish we could have tested him,'' she said.

In Shelby County, Ala., Lina Evans, the coroner, said she was now
suspicious of a surge in deaths in her county earlier this year, many of
which involved severe pneumonia: ``We had a lot of hospice deaths this
year, and now it makes me go back and think, wow, did they have Covid?
Did that accelerate their death?''

Ms. Evans, who is also a nurse, is frustrated that she will never know.

``When we go back to those deaths that occurred earlier this year,
people who were negative for flu, now we're having the `aha!' moment,''
she said. ``They should have been tested for the coronavirus. As far as
underreporting, I would say, definitely.''

\hypertarget{disparate-reporting-more-waiting}{%
\subsection{Disparate reporting, more
waiting}\label{disparate-reporting-more-waiting}}

Even now, as testing is more widely available, there is a patchwork of
standards about information being reported by state and local health
officials on deaths in the United States.

Around the world, keeping an accurate death toll has been a challenge
for governments. Availability of testing and other resources have
affected the official counts in some places, and
\href{https://www.nytimes.com/2020/04/02/us/politics/cia-coronavirus-china.html}{significant
questions} have emerged about official government tallies in places such
as China and Iran.

In the U.S., uncertainties and inconsistencies have emerged, and health
departments have had to backtrack on cases of previously reported
deaths. Florida officials rescinded an announcement of a Covid death in
Pasco County. In Hawaii, the state's first announced coronavirus death
was later re-categorized as unrelated after officials admitted
misreading test results. Los Angeles county officials announced that a
child had died from the virus, then said they were unsure whether the
virus caused the death, then declined to explain the confusion.

Adding to the complications, different jurisdictions are using distinct
standards for attributing a death to the coronavirus and, in some cases,
are relying on techniques that would lower the overall count of
fatalities.

In Blaine County, Idaho, the local health authority requires a positive
test to certify a death the result of coronavirus. But in Alabama, the
state health department requires a physician to review a person's
medical records to determine whether the virus was actually the root
cause of death.

``This is in the interest of having the most accurate, and most
transparent data that we can provide,'' said Karen Landers, a district
medical officer with the Alabama Department of Public Health. ``We
recognize that different sites might do it differently.''

So far, the state has received reports of 45 people with the coronavirus
dying, but has only certified 31 of those deaths as a result of the
virus.

Image

Paramedics in New York City said that many patients who died at home
were never tested for coronavirus.Credit...Johnny Milano for The New
York Times

Experts who study mortality statistics caution that it may take months
for scientists to calculate a fatality rate for coronavirus in the
United States that is as accurate as possible.

Some researchers say there may never be a truly accurate, complete count
of deaths. It has happened before. Experts
\href{https://www.ncbi.nlm.nih.gov/pmc/articles/PMC3827586/}{believe}
that widespread news coverage in 1976 of a potential swine flu epidemic
--- one that never materialized --- led to a rash of deaths recorded as
influenza that, in years prior, would have been categorized as
pneumonia.

``We're still debating the death toll of the Spanish flu'' of 1918-19,
said St\textbf{é}phane Helleringer, associate professor at the Johns
Hopkins University Bloomberg School of Public Health. ``It might take a
long time. It's not just that the data is messy, but because the effects
of a pandemic disease are very complex.''

Sarah Kliff reported from Washington, and Julie Bosman from Chicago.
Reporting was contributed by Mitch Smith in Overland Park, Kan., and Ali
Watkins in New York. Susan C. Beachy contributed research from New York.

Advertisement

\protect\hyperlink{after-bottom}{Continue reading the main story}

\hypertarget{site-index}{%
\subsection{Site Index}\label{site-index}}

\hypertarget{site-information-navigation}{%
\subsection{Site Information
Navigation}\label{site-information-navigation}}

\begin{itemize}
\tightlist
\item
  \href{https://help.nytimes.com/hc/en-us/articles/115014792127-Copyright-notice}{©~2020~The
  New York Times Company}
\end{itemize}

\begin{itemize}
\tightlist
\item
  \href{https://www.nytco.com/}{NYTCo}
\item
  \href{https://help.nytimes.com/hc/en-us/articles/115015385887-Contact-Us}{Contact
  Us}
\item
  \href{https://www.nytco.com/careers/}{Work with us}
\item
  \href{https://nytmediakit.com/}{Advertise}
\item
  \href{http://www.tbrandstudio.com/}{T Brand Studio}
\item
  \href{https://www.nytimes.com/privacy/cookie-policy\#how-do-i-manage-trackers}{Your
  Ad Choices}
\item
  \href{https://www.nytimes.com/privacy}{Privacy}
\item
  \href{https://help.nytimes.com/hc/en-us/articles/115014893428-Terms-of-service}{Terms
  of Service}
\item
  \href{https://help.nytimes.com/hc/en-us/articles/115014893968-Terms-of-sale}{Terms
  of Sale}
\item
  \href{https://spiderbites.nytimes.com}{Site Map}
\item
  \href{https://help.nytimes.com/hc/en-us}{Help}
\item
  \href{https://www.nytimes.com/subscription?campaignId=37WXW}{Subscriptions}
\end{itemize}
