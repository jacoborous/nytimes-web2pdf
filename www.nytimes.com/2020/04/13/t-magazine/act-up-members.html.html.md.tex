Sections

SEARCH

\protect\hyperlink{site-content}{Skip to
content}\protect\hyperlink{site-index}{Skip to site index}

\href{https://myaccount.nytimes.com/auth/login?response_type=cookie\&client_id=vi}{}

\href{https://www.nytimes.com/section/todayspaper}{Today's Paper}

12 People on Joining ACT UP: `I Went to That First Meeting and Never
Left'

\href{https://nyti.ms/2JX5YiH}{https://nyti.ms/2JX5YiH}

\begin{itemize}
\item
\item
\item
\item
\item
\end{itemize}

Advertisement

\protect\hyperlink{after-top}{Continue reading the main story}

Supported by

\protect\hyperlink{after-sponsor}{Continue reading the main story}

\hypertarget{12-people-on-joining-act-up-i-went-to-that-first-meeting-and-never-left}{%
\section{12 People on Joining ACT UP: `I Went to That First Meeting and
Never
Left'}\label{12-people-on-joining-act-up-i-went-to-that-first-meeting-and-never-left}}

A dozen participants of the pioneering AIDS activist group, including
Larry Kramer and Peter Staley, on the moment they decided to take
action.

\includegraphics{https://static01.nyt.com/images/2020/04/13/t-magazine/art/13tmag-actup-slide-GSGE/13tmag-actup-slide-GSGE-articleLarge.jpg?quality=75\&auto=webp\&disable=upscale}

By Kyle Turner

\begin{itemize}
\item
  Published April 13, 2020Updated April 17, 2020
\item
  \begin{itemize}
  \item
  \item
  \item
  \item
  \item
  \end{itemize}
\end{itemize}

It's been 33 years since \href{https://actupny.com/}{ACT UP} --- the
AIDS Coalition to Unleash Power, a direct action group made up of men
and women united in their fury at the lack of response from the
government and pharmaceutical industry to the AIDS crisis beginning in
the 1980s --- took to Wall Street on March 24, 1987, to protest public
apathy about and medical profiteering from the epidemic. Since that day,
much has changed: how we think about grass-roots organizing, the ways in
which community among marginalized groups is cultivated and fostered,
the life expectancy and treatment of people living with H.I.V. and AIDS.
**** While many of the issues the group labored to address **** also
**** persist, ACT UP has created an indelible blueprint for ****
on-the-ground activism that has paved the way both for other movements,
including Black Lives Matter and Occupy Wall Street, and for subsequent
generations of L.G.B.T.Q. people to continue to fight, unrelentingly,
for justice.

For
\href{https://www.nytimes.com/interactive/2020/04/13/t-magazine/culture-issue-2020.html}{T's
Culture issue}, which highlights a range of influential creative
communities,
\href{https://www.nytimes.com/interactive/2020/04/13/t-magazine/act-up-aids.html}{we
gathered 98 members} of ACT UP, past and present, for a reunion of sorts
at New York's Manny Cantor Center in February. As the writer and
director David France explains of the group's participants in his
accompanying essay, **** ``They had little in common beyond what
political scientists call a linked fate: Everyone in those meetings knew
someone who was dying or had died, or else they were marked for death
themselves. This brought a ferocious urgency into the room.'' **** Here,
in their own words, 12 members explain why they joined.

\includegraphics{https://static01.nyt.com/images/2020/04/13/t-magazine/art/13tmag-actup-slide-NNCT/13tmag-actup-slide-NNCT-articleLarge.jpg?quality=75\&auto=webp\&disable=upscale}

\hypertarget{larry-kramer-84}{%
\subsubsection{\texorpdfstring{\href{https://www.nytimes.com/2020/05/27/us/larry-kramer-dead.html}{LARRY
KRAMER}, 84}{LARRY KRAMER, 84}}\label{larry-kramer-84}}

\textbf{Joined in 1987 (founding member)}

From the day I read the July 1981 New York Times
\href{https://www.nytimes.com/1981/07/03/us/rare-cancer-seen-in-41-homosexuals.html}{article}
titled ``Rare Cancer Seen in 41 Homosexuals,'' I just knew that no one
was going to do anything to respond. Several friends of mine had already
died. So I started writing tirades in the gay press; it scared the crap
out of everyone, and my pieces were published in other gay newspapers
around the country. I worked with the Gay Men's Health Crisis, which is
a patient care organization, from January 1982, but they wouldn't
protest visibly. In contrast, ACT UP was a direct-action group and did
nothing \emph{but} protest. It was an exceptionally moving organization.
There was so much love between the men and women fighting side by side.

Image

Gran Fury's ``Women Don't Get AIDS. They Just Die From It'' poster
(1992).Credit...Gran Fury Collection, Manuscripts and Archives Division,
The New York Public Library

\hypertarget{ivy-kwan-arce-55}{%
\subsubsection{IVY KWAN ARCE, 55}\label{ivy-kwan-arce-55}}

\textbf{Joined in 1990}

In 1990, when I first moved to New York, I saw a couple of the graphics
that ACT UP had posted all over the city. One of them said, ``Women
don't get AIDS. They just die.'' Underneath that, it listed the risk
factors for women. At the time, I was dating someone who was an IV drug
user and they would tell me, ``There's no reason to get a test, because
women don't get it.'' When I finally got checked, I tested positive. The
doctor who gave me the results said, ``The only place that will give you
advice is ACT UP.'' I immediately remembered the poster that I had seen,
and I went to a meeting that same day. I met people who were key figures
within the group, and they helped me a lot; I learned not only how to
take part in actions but also how to be part of a supportive community.
They taught me --- someone who wasn't a great speaker and who had no
experience working for AIDS organizations --- to take up space in this
movement.

Image

An undated still from a New York news broadcast reporting on ACT UP's
Wall Street protest.Credit...ACT UP New York Records, Manuscripts and
Archives Division, The New York Public Library

\hypertarget{peter-staley-59}{%
\subsubsection{PETER STALEY, 59}\label{peter-staley-59}}

\textbf{Joined in 1987}

I found out I was H.I.V. positive in late 1985 while I was a bond trader
at JPMorgan. I was 24 years old and deep in the closet then. About a
year and a half later, in March of '87, I was handed a flyer on my way
to work by a new organization called ACT UP that was doing its very
first demonstration on Wall Street. I didn't see the protest myself ---
I was on the trading floor when it happened --- but I saw it that night
on TV **** and was really impressed that they made the national news and
that the F.D.A. commissioner held a news conference. I could tell that
it was real, the kind of power that I wanted to tap into right away. I
got myself to the very next meeting and never looked back.

I had a crazy year where I was a closeted bond trader by day and a
radical AIDS activist by night. It ended when **** my CD4 count ****
{[}the number of T cells in the body that fight infection and are the
primary target of the virus{]} crashed in early '88. I walked into my
boss's office the next morning, told him everything and dedicated myself
to AIDS activism from that point on. I felt like I had entered the
movement for very selfish reasons --- I was desperate to buy myself some
time ---~but within months of being part of this extraordinary response
from a community that was filled with love and passion and determination
and anger, I realized I was a part of something far larger than myself,
something that could change the lives of millions of people. I got
totally swept up in that. ACT UP became my church, my social life, what
I did every moment of the day~--- it's where I found all my boyfriends!
And it's where I became an activist, which has been my title ever since.

\hypertarget{eric-sawyer-66}{%
\subsubsection{ERIC SAWYER, 66}\label{eric-sawyer-66}}

\textbf{Joined in 1987 (founding member)}

I got involved with ACT UP mainly because I knew that there were a lot
of homeless people with AIDS and I wanted to try to develop housing for
them. In 1987, I called Larry Kramer, who had been a friend of mine
since 1980 --- he helped me find doctors when I got sick, and helped my
partner, who died in '86, with access to medical care --- and he told me
about this speech he was going to give at the Gay and Lesbian Community
Center, in which he was going to call for a group to be civilly
disobedient. He asked me if I would be a plant in the audience and bring
some attractive friends to encourage other people to come. So my initial
attendance was driven by a request from Larry to come and help form a
group. But the experience of organizing that first demonstration at the
center was empowering: It allowed me to turn my feelings of anger at the
loss of my partner into some type of action. I became hooked after that.

\hypertarget{robert-vazquez-pacheco-64}{%
\subsubsection{ROBERT VAZQUEZ-PACHECO,
64}\label{robert-vazquez-pacheco-64}}

\textbf{Joined in 1988}

In September 1980, my boyfriend at the time, Jeff, whom I was living
with, was diagnosed with Kaposi's sarcoma. So, my relationship with the
epidemic started early on. After his death in February 1986, I started
to figure out what I would do with my life. I became a facilitator in my
gay male consciousness-raising group, which met on Mondays in one of the
rooms at the Gay and Lesbian Community Center. When my friend, the
activist David Kirschenbaum, and I left each week, we would walk
downstairs and through the ACT UP meeting. We entered the meeting one
day, and the room was filled with mainly white gay men. I'm a person of
color, and I have to get my bearings when I walk into an overwhelmingly
white space. David asked me, ``Well, where do we stand?'' I looked
around and found **** that, in that old ACT UP room, the power brokers
were in the right-hand corner at the back. So I said to David, ``That's
where we stand.''

I think the group had done its first Wall Street protest by then, and we
were trying to figure out whether we wanted to get involved. Of course,
I did. After having dealt with what Jeff went through, I was certainly
pissed off enough. I thought, ``This is a good way for me to deal with
what I'm feeling now.'' Recently, I've been trying to write an account,
in my own words, of the racial politics of ACT UP --- and about the
whitewashing that's going on. A lot of academics of color are actually
now starting to write about this. It's very important, because there
were so many people of color who were part of ACT UP, even at the
beginning --- who were there and then died.

Image

A poster for the 10th anniversary protest for ACT UP on Wall
Street.Credit...ACT UP New York Records, Manuscripts and Archives
Division, The New York Public Library

\hypertarget{jamie-bauer-61}{%
\subsubsection{JAMIE BAUER, 61}\label{jamie-bauer-61}}

\textbf{Joined in 1987}

In 1981, I became active in the
\href{http://www.wloe.org/WLOE-en/background/wpastatem.html}{Women's
Pentagon Action}, which is a feminist, anti-militarist group with
connections to the War Resisters **** League (WRL), one of the oldest
pacifist organizations in the United States. I was trained in nonviolent
civil disobedience: We would discuss a demonstration and really walk
through the safety of it to make sure that we weren't doing anything to
intentionally hurt people. When ACT UP started in 1987, some of the
organizers of the first meeting reached out to the WRL. David
McReynolds, a longtime member, and I dispatched WRL members who
understood nonviolent civil disobedience --- and that included the two
of us, both queer --- to go to the ACT UP meeting and do a brief
training. After that, I just stuck with it.

As a pacifist, for me, it was always about acknowledging your anger and
channeling it into something productive --- and, of course, with people
dying, there was so much anger. Although ACT UP did not take a vow of
nonviolence, we had a series of principles that were built upon that; we
had very clearly drawn lines. For me, the biggest struggle was working
with people to make sure that we didn't overstep those boundaries, that
we didn't turn into the Weathermen {[}the '60s and '70s-era radical
faction of the left-wing Students for a Democratic Society{]}, that we
didn't bomb buildings --- which, in a time of desperation, when you're
watching all your friends die, would have been an easy direction for us
to go in.

\hypertarget{kendall-thomas-63}{%
\subsubsection{KENDALL THOMAS, 63}\label{kendall-thomas-63}}

\textbf{Joined in 1987}

The first meeting I went to was in late '87 at the Gay and Lesbian
Community Center. I walked into the room with a friend and it was
jam-packed. The first thing I noticed was the energy, which was
palpable, of all of those people gathered together. Until that point, I
hadn't been in a room of people who had come together to talk about what
our community response ought to be to the absolutely crushing
indifference and outright hostility to the AIDS epidemic and people who
had it. It was a transformative moment for me. I felt that something was
possible if we fought **** together --- and I had never felt that way
before about H.I.V. and AIDS.

In 1988, **** I became a founding member of the Majority Actions
Committee, which was an affinity group --- one of the smaller caucuses
organized around issues such as housing, treatment and data --- within
ACT UP. It was made up of people of color. A number of them died before
we had access to effective treatment. That was a moment in which I was
able to connect --- in a way, frankly, that I had not been able to do
before --- my commitments to racial justice and sexual justice, to
racial freedom and to sexual freedom. The self-identified Black, Latinx
and Asian members who came together in the Majority Actions Committee
were able to bring both a set of experiences and a political
understanding to ACT UP that it would otherwise have lacked.

\hypertarget{jay-blotcher-59}{%
\subsubsection{JAY BLOTCHER, 59}\label{jay-blotcher-59}}

\textbf{Joined in 1987}

In 1987, I was working as a volunteer for the Gay Men's Health Crisis,
the AIDS service organization started by a group including Larry Kramer,
helping with the annual AIDS Walk in New York, and I had decided to take
a more active role in manning the phones. I remember one evening, when
we were making calls for donations, a friend of mine walked into our
office in Chelsea and said, ``Hey, did you hear there's going to be a
demonstration on Wall Street tomorrow? They're going to protest the fact
that AZT is the only drug available to fight AIDS and how expensive it
is.'' That resonated with me, so I got up the next morning and went down
to march. That group became ACT UP. I didn't join officially until the
fall of 1987, after the March on Washington. When I saw ACT UP then, in
all its fiery fury, all of its grandeur, all of its sexy anger, I just
thought ``Wow, I want to be a part of this.''

\hypertarget{avram-finkelstein-68}{%
\subsubsection{AVRAM FINKELSTEIN, 68}\label{avram-finkelstein-68}}

\textbf{Joined in 1987 (founding member)}

I was meeting, once a week, with the graphic designers, art directors
and artists that developed the iconography that ACT UP would later use
--- we created the ``Silence=Death'' poster in 1986-87 --- and I was
aware that Larry Kramer was one of the few people in New York who was
writing about AIDS as a political crisis. When I saw the announcement at
the Gay and Lesbian Community Center that he was going to speak there in
March 1987, replacing Nora Ephron the night that she wasn't able to make
it, I suggested to the collective **** that instead of meeting that
week, we go hear Larry talk. So I didn't simply **** hear about ACT UP
and go to a meeting: I was there at the genesis of it. Coming from a
family that was engaged in politics, I probably had a very different
experience of being part of ACT UP than people who came in cold. My
parents were members of the American Communist Party, and my grandmother
used to come to peace marches with me, so a lot of it felt very
familiar. It wasn't so much formative for me as it was a way of
connecting my own political past to my queer identity.

Image

Demonstrators from ACT UP, angry with the federal government's response
to the AIDS crisis, protest in front of the headquarters of the Food and
Drug Administration in Rockville, Md., Oct. 11, 1988, and effectively
shut it down. By mid-morning some 50 of the protesters had been
arrested.Credit...J. Scott Applewhite/AP Images

\hypertarget{garance-franke-ruta-48}{%
\subsubsection{GARANCE FRANKE-RUTA, 48}\label{garance-franke-ruta-48}}

\textbf{Joined in 1988}

I started the first gay and lesbian youth group in Santa Fe, N.M., in
1987 or '88. But that was really just with friends in high school. The
first action I participated in with ACT UP was the October 1988 protest
against the F.D.A.'s slow drug approval process and the lack of support
for affected communities. That was fairly early in ACT UP's history, and
the group invited everybody in the broader community to come down to
Rockville, Md., **** for that action. A lot of people I knew were going,
and it seemed like what people who were engaged with the community did.
After that first year, I was involved with the protests primarily.

\hypertarget{asantewaa-gail-harris-68}{%
\subsubsection{ASANTEWAA GAIL HARRIS,
68}\label{asantewaa-gail-harris-68}}

\textbf{Joined in 1987}

I was a founding member of the Brooklyn AIDS Task Force {[}now
\href{https://bac-ny.org/}{Bridging Access to Care NYC}, Brooklyn's
oldest H.I.V. and AIDS service organization{]} and had attended a
national conference at Emory University in 1987 that was sponsored by
the American Medical Association. That experience led me to other
conferences, and **** I can't remember exactly where the idea of ACT UP
membership came up for me, but it did, and I attended a meeting. So my
involvement with ACT UP started very early on, and it was at around that
time that I also joined the Women's Caucus and the Majority Actions
Committee, two of the several splinter groups within ACT UP. Hidden
since the beginning of this crisis was its impact on women and children.
I'm a third-generation Brooklynite, so New York City is my community;
**** if there was social justice or human rights work to be done, then I
would find it. Together with the women I was working with in the Caucus,
we **** were writing and including our stories in this movement, and I
was pleased with that.

\hypertarget{jason-rosenberg-28}{%
\subsubsection{JASON ROSENBERG, 28}\label{jason-rosenberg-28}}

\textbf{Joined in 2016}

I've been a member **** of activist groups since high school, and I
first got involved in H.I.V. work when I moved back to New York after
college in 2015. I was active in this group called Impulse Group NYC,
which is part of the AIDS Healthcare Foundation, but because I thought
they weren't doing enough politically to advocate **** for
\href{https://www.cdc.gov/hiv/basics/prep.html}{PrEP} and destigmatize
H.I.V., I decided to channel all of my energy into ACT UP in 2016. The
people who are part of this long history --- it's now been 33 years ---
really changed the way I view a lot of activist work. It's helpful for
young organizers to seek mentorship from our elders and long-term
survivors. It's sort of the best practices of protesting. We need to
reference our history to reflect on what's next in confronting H.I.V.
and the stigma that continues to exist.

\emph{These quotes have been edited and condensed.}

Advertisement

\protect\hyperlink{after-bottom}{Continue reading the main story}

\hypertarget{site-index}{%
\subsection{Site Index}\label{site-index}}

\hypertarget{site-information-navigation}{%
\subsection{Site Information
Navigation}\label{site-information-navigation}}

\begin{itemize}
\tightlist
\item
  \href{https://help.nytimes.com/hc/en-us/articles/115014792127-Copyright-notice}{©~2020~The
  New York Times Company}
\end{itemize}

\begin{itemize}
\tightlist
\item
  \href{https://www.nytco.com/}{NYTCo}
\item
  \href{https://help.nytimes.com/hc/en-us/articles/115015385887-Contact-Us}{Contact
  Us}
\item
  \href{https://www.nytco.com/careers/}{Work with us}
\item
  \href{https://nytmediakit.com/}{Advertise}
\item
  \href{http://www.tbrandstudio.com/}{T Brand Studio}
\item
  \href{https://www.nytimes.com/privacy/cookie-policy\#how-do-i-manage-trackers}{Your
  Ad Choices}
\item
  \href{https://www.nytimes.com/privacy}{Privacy}
\item
  \href{https://help.nytimes.com/hc/en-us/articles/115014893428-Terms-of-service}{Terms
  of Service}
\item
  \href{https://help.nytimes.com/hc/en-us/articles/115014893968-Terms-of-sale}{Terms
  of Sale}
\item
  \href{https://spiderbites.nytimes.com}{Site Map}
\item
  \href{https://help.nytimes.com/hc/en-us}{Help}
\item
  \href{https://www.nytimes.com/subscription?campaignId=37WXW}{Subscriptions}
\end{itemize}
