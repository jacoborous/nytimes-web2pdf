Sections

SEARCH

\protect\hyperlink{site-content}{Skip to
content}\protect\hyperlink{site-index}{Skip to site index}

\href{https://myaccount.nytimes.com/auth/login?response_type=cookie\&client_id=vi}{}

\href{https://www.nytimes.com/section/todayspaper}{Today's Paper}

\href{/section/opinion}{Opinion}\textbar{}The Virus vs. Journalism

\href{https://nyti.ms/35jt9xj}{https://nyti.ms/35jt9xj}

\begin{itemize}
\item
\item
\item
\item
\item
\end{itemize}

Advertisement

\protect\hyperlink{after-top}{Continue reading the main story}

\href{/section/opinion}{Opinion}

Supported by

\protect\hyperlink{after-sponsor}{Continue reading the main story}

\hypertarget{the-virus-vs-journalism}{%
\section{The Virus vs. Journalism}\label{the-virus-vs-journalism}}

The disappearance of local information.

\href{https://www.nytimes.com/by/david-leonhardt}{\includegraphics{https://static01.nyt.com/images/2018/04/02/opinion/david-leonhardt/david-leonhardt-thumbLarge.png}}

By \href{https://www.nytimes.com/by/david-leonhardt}{David Leonhardt}

Opinion Columnist

\begin{itemize}
\item
  April 30, 2020
\item
  \begin{itemize}
  \item
  \item
  \item
  \item
  \item
  \end{itemize}
\end{itemize}

\includegraphics{https://static01.nyt.com/images/2020/04/30/opinion/30leonhardt-newsletter-onsite/merlin_171051516_4128bec9-ec44-4d45-be6c-d2faf33c25c0-articleLarge.jpg?quality=75\&auto=webp\&disable=upscale}

\emph{This article is part of David Leonhardt's newsletter. You can}
\href{https://www.nytimes.com/newsletters/opiniontoday?action=click\&module=Intentional\&pgtype=Article}{\emph{sign
up here}} \emph{to receive it each weekday.}

Local journalism was in deep trouble before the coronavirus.

The internet has taken away the main source of revenue for newspapers
--- print advertisements --- leading to a rapid shrinking of the
industry. Nationwide, the number of people employed in newsrooms
\href{https://www.pewresearch.org/fact-tank/2020/04/20/u-s-newsroom-employment-has-dropped-by-a-quarter-since-2008/}{fell
about 25 percent} between 2008 and 2019, and it's probably down more
than 50 percent from its peak.

If local papers were being replaced by digital publications covering
local news, this trend wouldn't be a problem. But that's not happening.
Instead, many Americans lack basic information about their communities
--- like what their mayor, school board, local employers and more are
doing.

The disappearance of this information has big effects.
\href{https://medium.com/office-of-citizen/how-we-know-journalism-is-good-for-democracy-9125e5c995fb}{Academic
research has found} that voter turnout and civic engagement tend to
decline when newspapers shrink or close. Fewer people run for office.
Political corruption and polarization rise.

``Local newspapers are basically little machines that spit out healthier
democracies,''
\href{https://www.niemanlab.org/2019/04/when-local-newspapers-shrink-fewer-people-bother-to-run-for-mayor/}{Joshua
Benton}, director of the Nieman Journalism Lab, has written.

Now the virus is taking this crisis to a new level.

The rapid shrinking of the economy --- at the fastest pace
\href{https://www.nytimes.com/2020/04/14/us/politics/coronavirus-economy-recession-depression.html}{since
the Great Depression} --- has led to a further decline in advertising.
Some newspapers that were on the brink may not survive. And many more
journalists
\href{https://www.nytimes.com/2020/04/10/business/media/news-media-coronavirus-jobs.html}{have
been laid off}. As The Washington Post's
\href{https://www.washingtonpost.com/lifestyle/media/local-journalism-needs-a-coronavirus-stimulus-plan-too/2020/03/25/08358062-6ec6-11ea-b148-e4ce3fbd85b5_story.html}{Margaret
Sullivan} has noted, ``it's happening around the world,'' with
newspapers in Australia and Britain announcing that ``they were going
out of business or suspending print publication.''

What's the solution? In the short term, Sullivan and some media
observers have called for government stimulus money to be directed at
local news outlets, as is happening for many other industries.

Writing in The Atlantic,
\href{https://www.theatlantic.com/ideas/archive/2020/03/coronavirus-killing-local-news/608695/}{Steven
Waldman and Charles Sennott} of Report for America offer an intriguing
idea:

\begin{quote}
The federal government can do something quite concrete right now: As
part of its stimulus plans, it should funnel \$500 million in spending
for public-health ads through local media. The government already spends
about \$1 billion on public-service ads that promote initiatives such as
military recruitment and census participation. The stimulus should add
another \$1 billion to support the communication of accurate
health-related information. Some of those ads should go to social-media
platforms and national news networks, but half should go to local news
organizations. This is not a bailout; the government will be buying an
effective way of getting health messages to the public, and could even
customize the notices to specific audiences.
\end{quote}

Long term, however, stimulus isn't the answer. Local journalism needs a
new business model. (National journalism, by the way,
\href{https://www.niemanlab.org/2020/02/the-wall-street-journal-joins-the-new-york-times-in-the-2-million-digital-subscriber-club/}{is
doing OK}, thanks in part to the growth of subscription-based
journalism, at The New York Times and elsewhere.)

My hope is that somebody will eventually find a way to make money
providing useful local information. Until then, the answer will almost
certainly need to involve philanthropy, much as philanthropy has long
supported public radio.

You've heard me
\href{https://www.nytimes.com/2018/11/21/opinion/local-journalism-news-media.html}{say
this before}, and it's never been more true: If you have a local source
of news that you trust, I hope you can find a way to support it
financially.

That source may still be a traditional local newspaper, which sells
subscriptions. But I know many people now live in communities where
companies like Alden Global Capital have taken over newspapers and are
bleeding them for some final profits. (See
\href{https://www.vanityfair.com/news/2020/02/hedge-fund-vampire-alden-global-capital-that-bleeds-newspapers-dry-has-chicago-tribune-by-the-throat}{Vanity
Fair's Joe Pompeo} for more on this.)

In that case, see if your community now has a nonprofit start-up as
well,
\href{https://www.nytimes.com/2019/12/10/opinion/local-news.html}{in the
mold of the Texas Tribune}.

And if you have no good local options, you may even want to think about
starting a movement to change that.

\textbf{For more \ldots{}}

\begin{itemize}
\item
  \href{https://www.poynter.org/business-work/2020/here-are-the-newsroom-layoffs-furloughs-and-closures-caused-by-the-coronavirus/}{Poynter}
  has a running list of the newsroom layoffs, furloughs and closures
  caused by the coronavirus.
\item
  \href{https://www.nbcnews.com/think/opinion/coronavirus-revealing-why-local-news-so-important-it-s-also-ncna1186261}{Matt
  Laslo}, NBC News Think:
\end{itemize}

\begin{quote}
The ability for people to get timely, unbiased information on local
conditions in their communities is more important than ever. Doing so,
however, is increasingly more difficult than ever before --- and could
get even worse. Many newsrooms were already facing hard times before the
coronavirus pandemic shuttered much of America's economy. \ldots{} And
in the absence of local news organizations, we could all face an
unprecedented attack from a second invisible enemy: Fake news parading
as fact, with nothing and nobody to counter its spread.
\end{quote}

\begin{itemize}
\tightlist
\item
  Politico's
  \href{https://www.politico.com/news/magazine/2020/04/20/dont-waste-stimulus-money-on-newspapers-197015}{Jack
  Shafer} argues against stimulus for newspapers:
\end{itemize}

\begin{quote}
It might make sense for the government to assist otherwise healthy
companies --- such as the airlines --- that need a couple of months of
breathing space from the viral shock to recover and are in a theoretical
position to repay government loans sometime soon. But it's quite another
thing to fling a life buoy to a drowning swimmer who doesn't have the
strength to hold on. Newspapers are such a drowning industry. Readers
have abandoned them in the tens of millions. Advertisers have largely
abandoned them. For the most part, the virus isn't causing them to sink.
They're already sunk.

In the triage of rescuing flailing firms, some sectors must be left dead
unless we want to make permanent welfare cases out of them --- and
that's a much different argument than a bailout. It would also be a
grievous error to bail out papers controlled by the Alden Global Capital
hedge fund --- and other firms like them --- that have made a practice
of squeezing high profits while simultaneously cutting staff and
escalating subscription prices.
\end{quote}

\emph{If you are not a subscriber to this newsletter, you can}
\href{https://www.nytimes.com/newsletters/david-leonhardt}{\emph{subscribe
here}}\emph{. You can also join me on}
\href{https://twitter.com/DLeonhardt}{\emph{Twitter (@DLeonhardt)}}
\emph{and}
\href{https://www.facebook.com/DavidRLeonhardt/}{\emph{Facebook}}\emph{.}

\emph{Follow The New York Times Opinion section on}
\href{https://www.facebook.com/nytopinion}{\emph{Facebook}}\emph{,}
\href{http://twitter.com/NYTOpinion}{\emph{Twitter (@NYTopinion)}}
\emph{and}
\href{https://www.instagram.com/nytopinion/}{\emph{Instagram}}\emph{.}

Advertisement

\protect\hyperlink{after-bottom}{Continue reading the main story}

\hypertarget{site-index}{%
\subsection{Site Index}\label{site-index}}

\hypertarget{site-information-navigation}{%
\subsection{Site Information
Navigation}\label{site-information-navigation}}

\begin{itemize}
\tightlist
\item
  \href{https://help.nytimes.com/hc/en-us/articles/115014792127-Copyright-notice}{©~2020~The
  New York Times Company}
\end{itemize}

\begin{itemize}
\tightlist
\item
  \href{https://www.nytco.com/}{NYTCo}
\item
  \href{https://help.nytimes.com/hc/en-us/articles/115015385887-Contact-Us}{Contact
  Us}
\item
  \href{https://www.nytco.com/careers/}{Work with us}
\item
  \href{https://nytmediakit.com/}{Advertise}
\item
  \href{http://www.tbrandstudio.com/}{T Brand Studio}
\item
  \href{https://www.nytimes.com/privacy/cookie-policy\#how-do-i-manage-trackers}{Your
  Ad Choices}
\item
  \href{https://www.nytimes.com/privacy}{Privacy}
\item
  \href{https://help.nytimes.com/hc/en-us/articles/115014893428-Terms-of-service}{Terms
  of Service}
\item
  \href{https://help.nytimes.com/hc/en-us/articles/115014893968-Terms-of-sale}{Terms
  of Sale}
\item
  \href{https://spiderbites.nytimes.com}{Site Map}
\item
  \href{https://help.nytimes.com/hc/en-us}{Help}
\item
  \href{https://www.nytimes.com/subscription?campaignId=37WXW}{Subscriptions}
\end{itemize}
