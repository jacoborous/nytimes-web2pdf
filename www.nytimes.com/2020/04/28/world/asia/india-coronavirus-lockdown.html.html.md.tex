Sections

SEARCH

\protect\hyperlink{site-content}{Skip to
content}\protect\hyperlink{site-index}{Skip to site index}

\href{https://www.nytimes.com/section/world/asia}{Asia Pacific}

\href{https://myaccount.nytimes.com/auth/login?response_type=cookie\&client_id=vi}{}

\href{https://www.nytimes.com/section/todayspaper}{Today's Paper}

\href{/section/world/asia}{Asia Pacific}\textbar{}Unified in Coronavirus
Lockdown, India Splinters Over Reopening

\url{https://nyti.ms/35b3neJ}

\begin{itemize}
\item
\item
\item
\item
\item
\item
\end{itemize}

\href{https://www.nytimes.com/news-event/coronavirus?action=click\&pgtype=Article\&state=default\&region=TOP_BANNER\&context=storylines_menu}{The
Coronavirus Outbreak}

\begin{itemize}
\tightlist
\item
  live\href{https://www.nytimes.com/2020/08/02/world/coronavirus-updates.html?action=click\&pgtype=Article\&state=default\&region=TOP_BANNER\&context=storylines_menu}{Latest
  Updates}
\item
  \href{https://www.nytimes.com/interactive/2020/us/coronavirus-us-cases.html?action=click\&pgtype=Article\&state=default\&region=TOP_BANNER\&context=storylines_menu}{Maps
  and Cases}
\item
  \href{https://www.nytimes.com/interactive/2020/science/coronavirus-vaccine-tracker.html?action=click\&pgtype=Article\&state=default\&region=TOP_BANNER\&context=storylines_menu}{Vaccine
  Tracker}
\item
  \href{https://www.nytimes.com/interactive/2020/07/29/us/schools-reopening-coronavirus.html?action=click\&pgtype=Article\&state=default\&region=TOP_BANNER\&context=storylines_menu}{What
  School May Look Like}
\item
  \href{https://www.nytimes.com/live/2020/07/31/business/stock-market-today-coronavirus?action=click\&pgtype=Article\&state=default\&region=TOP_BANNER\&context=storylines_menu}{Economy}
\end{itemize}

Advertisement

\protect\hyperlink{after-top}{Continue reading the main story}

Supported by

\protect\hyperlink{after-sponsor}{Continue reading the main story}

\hypertarget{unified-in-coronavirus-lockdown-india-splinters-over-reopening}{%
\section{Unified in Coronavirus Lockdown, India Splinters Over
Reopening}\label{unified-in-coronavirus-lockdown-india-splinters-over-reopening}}

As India starts loosening restrictions, a once-broad consensus among
leaders on how to proceed has weakened --- and many people are staying
home anyway.

\includegraphics{https://static01.nyt.com/images/2020/04/28/world/28virus-india-top/merlin_171916635_6d61172a-6217-4039-9a9f-b6d7e0731b51-articleLarge.jpg?quality=75\&auto=webp\&disable=upscale}

\href{https://www.nytimes.com/by/kai-schultz}{\includegraphics{https://static01.nyt.com/images/2019/11/22/reader-center/author-kai-schultz/author-kai-schultz-thumbLarge.png}}\href{https://www.nytimes.com/by/sameer-yasir}{\includegraphics{https://static01.nyt.com/images/2019/11/22/reader-center/author-sameer-yasir/author-sameer-yasir-thumbLarge.png}}

By \href{https://www.nytimes.com/by/kai-schultz}{Kai Schultz} and
\href{https://www.nytimes.com/by/sameer-yasir}{Sameer Yasir}

\begin{itemize}
\item
  April 28, 2020
\item
  \begin{itemize}
  \item
  \item
  \item
  \item
  \item
  \item
  \end{itemize}
\end{itemize}

NEW DELHI --- When the Indian government
\href{https://www.business-standard.com/article/current-affairs/govt-eases-lockdown-rules-allows-all-local-and-standalone-shops-to-open-120042500051_1.html}{eased
coronavirus restrictions last week}, allowing many shops to reopen in
rural parts of the country, Uday Shankar Sharma, a retail store owner in
a small farming village, said he had no intention of complying.

Over the past few weeks, Mr. Sharma said fear had deepened in Sabna,
where he lives in northern India. Community meetings held under a clock
tower have stopped. Neighbors barely talk to each other. Streets are so
silent that people can hear grasshoppers in the daytime.

Mr. Sharma said resuming business was simply too dangerous right now,
even though his district of more than three million people has only
reported one case of the coronavirus.

``It is better to stay hungry than to get the coronavirus,'' he said in
a telephone interview. ``Why should I risk the lives of my family
members for a few hundred rupees?''

For five weeks, Indians of all stripes have united to
\href{https://www.nytimes.com/2020/04/19/world/asia/india-coronavirus-lockdown.html}{zealously}
conduct a nationwide lockdown, the largest and one of the most severe
anywhere. But as the central government has started lifting restrictions
in areas with few or no known cases of the coronavirus, officials are
now facing a new challenge:
\href{https://indianexpress.com/article/india/kerala-gujarat-set-to-open-shops-maharashtra-punjab-stay-shut-coronavirus-lockdown-6379495/}{persuading
fearful residents, and their leaders, to consider a partial reopening}.

\includegraphics{https://static01.nyt.com/images/2020/04/28/world/28virus-india-1/merlin_171735528_efc8c801-f2e5-4f6d-8064-abc015e492ee-articleLarge.jpg?quality=75\&auto=webp\&disable=upscale}

By many measures, the
\href{https://www.nytimes.com/2020/03/24/world/asia/india-coronavirus-lockdown.html}{nationwide
lockdown} imposed last month by Prime Minister Narendra Modi has helped
blunt the spread of the coronavirus. India's doubling rate for cases has
slowed to around nine days, and though testing is still restricted,
infections have remained relatively low for a nation of 1.3 billion,
with nearly 30,000 confirmed cases and 900 deaths.

Last Monday, India took a step toward reviving the economy to
``\href{https://twitter.com/PIBHomeAffairs/status/1250284039048556545}{mitigate
hardship to the public},'' allowing construction, plantation work and
some manufacturing to resume. By Friday, the central government had
further eased restrictions, permitting many shops to reopen in rural
parts of the country and outside hot spots, which have largely been
traced to bigger cities like Mumbai and New Delhi.

\emph{{[}Update:}
\href{https://www.nytimes.com/2020/05/06/world/asia/india-coronavirus-lockdown-infections.html}{\emph{As
India loosens its strict lockdown, coronavirus deaths jump
sharply}}\emph{.{]}}

But unlike the initial lockdown, which Indians widely endorsed despite
the clear cost of shutting a country where around half the population
\href{https://databank.worldbank.org/data/download/poverty/33EF03BB-9722-4AE2-ABC7-AA2972D68AFE/Global_POVEQ_IND.pdf}{lives
on less than \$3 a day}, the lifting of restrictions has divided state
leaders. They have some autonomy to set their own coronavirus guidelines
as long as they are no less strict than those imposed by the central
government.

While critics of a prolonged shutdown in the United States, for
instance, have often grounded arguments for reopening in notions of
individual liberty, Indian officials have almost uniformly rallied
around Mr. Modi's framing of the pandemic as a collectively felt crisis
that required cooperation at every rung of society.

\hypertarget{latest-updates-global-coronavirus-outbreak}{%
\section{\texorpdfstring{\href{https://www.nytimes.com/2020/08/01/world/coronavirus-covid-19.html?action=click\&pgtype=Article\&state=default\&region=MAIN_CONTENT_1\&context=storylines_live_updates}{Latest
Updates: Global Coronavirus
Outbreak}}{Latest Updates: Global Coronavirus Outbreak}}\label{latest-updates-global-coronavirus-outbreak}}

Updated 2020-08-02T17:52:35.962Z

\begin{itemize}
\tightlist
\item
  \href{https://www.nytimes.com/2020/08/01/world/coronavirus-covid-19.html?action=click\&pgtype=Article\&state=default\&region=MAIN_CONTENT_1\&context=storylines_live_updates\#link-34047410}{The
  U.S. reels as July cases more than double the total of any other
  month.}
\item
  \href{https://www.nytimes.com/2020/08/01/world/coronavirus-covid-19.html?action=click\&pgtype=Article\&state=default\&region=MAIN_CONTENT_1\&context=storylines_live_updates\#link-780ec966}{Top
  U.S. officials work to break an impasse over the federal jobless
  benefit.}
\item
  \href{https://www.nytimes.com/2020/08/01/world/coronavirus-covid-19.html?action=click\&pgtype=Article\&state=default\&region=MAIN_CONTENT_1\&context=storylines_live_updates\#link-2bc8948}{Its
  outbreak untamed, Melbourne goes into even greater lockdown.}
\end{itemize}

\href{https://www.nytimes.com/2020/08/01/world/coronavirus-covid-19.html?action=click\&pgtype=Article\&state=default\&region=MAIN_CONTENT_1\&context=storylines_live_updates}{See
more updates}

More live coverage:
\href{https://www.nytimes.com/live/2020/07/31/business/stock-market-today-coronavirus?action=click\&pgtype=Article\&state=default\&region=MAIN_CONTENT_1\&context=storylines_live_updates}{Markets}

Many embraced Mr. Modi's order for a ``total ban of coming out of your
homes,'' heeding his directives to police one another and fight the
virus like a ``dedicated soldier.''

Image

A photo released by the Indian government showing Prime Minister
Narendra Modi interacting with the chief ministers of states via
videoconference in New Delhi on Monday.Credit...Government of India/EPA,
via Shutterstock

But as India's economy suffers, the consensus has started to fray.

After lockdown measures were eased last week, the states of
\href{https://www.newindianexpress.com/states/kerala/2020/apr/25/with-centre-relaxing-lockdown-restrictions-kerala-allows-shops-to-open-in-rural-areas-2135151.html}{Kerala}
and
\href{https://www.outlookindia.com/newsscroll/small-shops-trades-to-be-allowed-from-sunday-in-gujarat-ld/1814354}{Gujarat}
were among those that planned to move forward with reopening shops.
Tamil Nadu, Jharkhand and Maharashtra indicated that they would keep
businesses shut until at least May 3, when Mr. Modi will decide whether
to extend the lockdown or let it expire. Other states barely said
anything.

Crafting enforceable orders is challenging in a country as diverse and
fragmented as India, with nearly two dozen official languages and vast
cultural chasms across states and even neighboring villages. The cryptic
nature of the government's news releases has not helped.

After announcing that many shops selling nonessential items could reopen
late Friday, the Ministry of Home Affairs issued multiple corrections
over the next 24 hours. On Twitter, Vasudha Gupta, a ministry
spokeswoman,
\href{https://twitter.com/PIBHomeAffairs/status/1253969490142003200}{revised}
an earlier announcement that ``ALL shops'' outside municipalities could
reopen by exempting liquor stores, then restaurants, then salons.

The Confederation of All India Traders, a group that represents small
retailers, urged the government to clarify even further. Over the
weekend, the group said it expected millions of businesses to open
nationwide, but only a few had been successful.

Image

A woman passing through a sanitization tunnel in Mumbai on
Saturday.Credit...Atul Loke for The New York Times

``There is lack of consensus among the administration and law enforcing
agencies with the result that traders are not allowed to open the
shops,''
\href{https://www.outlookindia.com/newsscroll/cait-awaits-clarity-on-relaxed-lockdown-norms-on-shops/1815243}{the
group said in a statement}.

In the southern state of Karnataka, Subhash Chandra, the managing
director of Sangeetha Mobiles,
\href{https://economictimes.indiatimes.com/industry/services/retail/retail-outlets-stay-shut-despite-mha-relaxation/articleshow/75396154.cms?from=mdr}{told
the Economic Times} that nearly half of the chain's 260 outlets had
reopened on Sunday only to be promptly shut by the local police.

Even business owners who faced fewer roadblocks in resuming operations
said supply chain wrinkles had made it nearly impossible to complete
most of their work.

After Mr. Modi announced the lockdown on March 24, migrant workers
typically hired for construction jobs left cities for their home
villages, some of them hundreds of miles away. With train and bus
service suspended, they have no easy way to return.

Mukesh Goel, a government official who oversees construction projects in
the state of Punjab, said his office reopened last week with a
``skeleton crew'' and no business.

``We are trying to find a way to fully resume work, but it doesn't seem
likely anytime soon,'' he said. ``We need machinery, labor that is
almost impossible to get at the moment.''

Arunoday Singh Parawar, a social worker in the state of Madhya Pradesh,
said skepticism to reopen went beyond fears of the coronavirus.

Image

Burying a Covid-19 victim in New Delhi this month.Credit...Manish
Swarup/Associated Press

In Chhatarpur, the town where he lives, local leaders have imposed
harsher restrictions than most by allowing food shops to open only on
alternate days, even though the area has been mostly unaffected by the
coronavirus.

\href{https://www.nytimes.com/news-event/coronavirus?action=click\&pgtype=Article\&state=default\&region=MAIN_CONTENT_3\&context=storylines_faq}{}

\hypertarget{the-coronavirus-outbreak-}{%
\subsubsection{The Coronavirus Outbreak
›}\label{the-coronavirus-outbreak-}}

\hypertarget{frequently-asked-questions}{%
\paragraph{Frequently Asked
Questions}\label{frequently-asked-questions}}

Updated July 27, 2020

\begin{itemize}
\item ~
  \hypertarget{should-i-refinance-my-mortgage}{%
  \paragraph{Should I refinance my
  mortgage?}\label{should-i-refinance-my-mortgage}}

  \begin{itemize}
  \tightlist
  \item
    \href{https://www.nytimes.com/article/coronavirus-money-unemployment.html?action=click\&pgtype=Article\&state=default\&region=MAIN_CONTENT_3\&context=storylines_faq}{It
    could be a good idea,} because mortgage rates have
    \href{https://www.nytimes.com/2020/07/16/business/mortgage-rates-below-3-percent.html?action=click\&pgtype=Article\&state=default\&region=MAIN_CONTENT_3\&context=storylines_faq}{never
    been lower.} Refinancing requests have pushed mortgage applications
    to some of the highest levels since 2008, so be prepared to get in
    line. But defaults are also up, so if you're thinking about buying a
    home, be aware that some lenders have tightened their standards.
  \end{itemize}
\item ~
  \hypertarget{what-is-school-going-to-look-like-in-september}{%
  \paragraph{What is school going to look like in
  September?}\label{what-is-school-going-to-look-like-in-september}}

  \begin{itemize}
  \tightlist
  \item
    It is unlikely that many schools will return to a normal schedule
    this fall, requiring the grind of
    \href{https://www.nytimes.com/2020/06/05/us/coronavirus-education-lost-learning.html?action=click\&pgtype=Article\&state=default\&region=MAIN_CONTENT_3\&context=storylines_faq}{online
    learning},
    \href{https://www.nytimes.com/2020/05/29/us/coronavirus-child-care-centers.html?action=click\&pgtype=Article\&state=default\&region=MAIN_CONTENT_3\&context=storylines_faq}{makeshift
    child care} and
    \href{https://www.nytimes.com/2020/06/03/business/economy/coronavirus-working-women.html?action=click\&pgtype=Article\&state=default\&region=MAIN_CONTENT_3\&context=storylines_faq}{stunted
    workdays} to continue. California's two largest public school
    districts --- Los Angeles and San Diego --- said on July 13, that
    \href{https://www.nytimes.com/2020/07/13/us/lausd-san-diego-school-reopening.html?action=click\&pgtype=Article\&state=default\&region=MAIN_CONTENT_3\&context=storylines_faq}{instruction
    will be remote-only in the fall}, citing concerns that surging
    coronavirus infections in their areas pose too dire a risk for
    students and teachers. Together, the two districts enroll some
    825,000 students. They are the largest in the country so far to
    abandon plans for even a partial physical return to classrooms when
    they reopen in August. For other districts, the solution won't be an
    all-or-nothing approach.
    \href{https://bioethics.jhu.edu/research-and-outreach/projects/eschool-initiative/school-policy-tracker/}{Many
    systems}, including the nation's largest, New York City, are
    devising
    \href{https://www.nytimes.com/2020/06/26/us/coronavirus-schools-reopen-fall.html?action=click\&pgtype=Article\&state=default\&region=MAIN_CONTENT_3\&context=storylines_faq}{hybrid
    plans} that involve spending some days in classrooms and other days
    online. There's no national policy on this yet, so check with your
    municipal school system regularly to see what is happening in your
    community.
  \end{itemize}
\item ~
  \hypertarget{is-the-coronavirus-airborne}{%
  \paragraph{Is the coronavirus
  airborne?}\label{is-the-coronavirus-airborne}}

  \begin{itemize}
  \tightlist
  \item
    The coronavirus
    \href{https://www.nytimes.com/2020/07/04/health/239-experts-with-one-big-claim-the-coronavirus-is-airborne.html?action=click\&pgtype=Article\&state=default\&region=MAIN_CONTENT_3\&context=storylines_faq}{can
    stay aloft for hours in tiny droplets in stagnant air}, infecting
    people as they inhale, mounting scientific evidence suggests. This
    risk is highest in crowded indoor spaces with poor ventilation, and
    may help explain super-spreading events reported in meatpacking
    plants, churches and restaurants.
    \href{https://www.nytimes.com/2020/07/06/health/coronavirus-airborne-aerosols.html?action=click\&pgtype=Article\&state=default\&region=MAIN_CONTENT_3\&context=storylines_faq}{It's
    unclear how often the virus is spread} via these tiny droplets, or
    aerosols, compared with larger droplets that are expelled when a
    sick person coughs or sneezes, or transmitted through contact with
    contaminated surfaces, said Linsey Marr, an aerosol expert at
    Virginia Tech. Aerosols are released even when a person without
    symptoms exhales, talks or sings, according to Dr. Marr and more
    than 200 other experts, who
    \href{https://academic.oup.com/cid/article/doi/10.1093/cid/ciaa939/5867798}{have
    outlined the evidence in an open letter to the World Health
    Organization}.
  \end{itemize}
\item ~
  \hypertarget{what-are-the-symptoms-of-coronavirus}{%
  \paragraph{What are the symptoms of
  coronavirus?}\label{what-are-the-symptoms-of-coronavirus}}

  \begin{itemize}
  \tightlist
  \item
    Common symptoms
    \href{https://www.nytimes.com/article/symptoms-coronavirus.html?action=click\&pgtype=Article\&state=default\&region=MAIN_CONTENT_3\&context=storylines_faq}{include
    fever, a dry cough, fatigue and difficulty breathing or shortness of
    breath.} Some of these symptoms overlap with those of the flu,
    making detection difficult, but runny noses and stuffy sinuses are
    less common.
    \href{https://www.nytimes.com/2020/04/27/health/coronavirus-symptoms-cdc.html?action=click\&pgtype=Article\&state=default\&region=MAIN_CONTENT_3\&context=storylines_faq}{The
    C.D.C. has also} added chills, muscle pain, sore throat, headache
    and a new loss of the sense of taste or smell as symptoms to look
    out for. Most people fall ill five to seven days after exposure, but
    symptoms may appear in as few as two days or as many as 14 days.
  \end{itemize}
\item ~
  \hypertarget{does-asymptomatic-transmission-of-covid-19-happen}{%
  \paragraph{Does asymptomatic transmission of Covid-19
  happen?}\label{does-asymptomatic-transmission-of-covid-19-happen}}

  \begin{itemize}
  \tightlist
  \item
    So far, the evidence seems to show it does. A widely cited
    \href{https://www.nature.com/articles/s41591-020-0869-5}{paper}
    published in April suggests that people are most infectious about
    two days before the onset of coronavirus symptoms and estimated that
    44 percent of new infections were a result of transmission from
    people who were not yet showing symptoms. Recently, a top expert at
    the World Health Organization stated that transmission of the
    coronavirus by people who did not have symptoms was ``very rare,''
    \href{https://www.nytimes.com/2020/06/09/world/coronavirus-updates.html?action=click\&pgtype=Article\&state=default\&region=MAIN_CONTENT_3\&context=storylines_faq\#link-1f302e21}{but
    she later walked back that statement.}
  \end{itemize}
\end{itemize}

Mr. Parawar said the reason was simple: Officials feared that if they
eased restrictions too soon, or by too much, they risked the ability to
reimpose coronavirus rules and persuade millions of people, many of them
without a formal education, to return to a life indoors.

``They do not want to lose control of the public,'' he said.

Still, economists say an indefinite lockdown is hardly sustainable. With
so many Indians out of work, the country's public distribution system,
which provides food and other handouts to hundreds of millions of
people, has been severely stressed.

And in remote areas of the country, the authorities have sometimes used
force to keep people inside, making it difficult to reach markets and
ration shops.

Nazia Errum, a widow who supports three children as a seamstress, said
the police have beaten people for trying to leave their homes in the
village of Hajipara, where she lives in the northeastern state of Assam.
With no work for a month, Ms. Errum feared that her family would starve
if the lockdown continued.

Image

Migrant workers stranded in Mumbai.Credit...Atul Loke for The New York
Times

``When you can't step out of your house for a minute, how will you
earn?'' she said. ``We have been eating rice only once a day instead of
three because we don't know what will happen tomorrow. We are
terrified.''

In India's largest state, Uttar Pradesh, state leaders have already
indicated that
\href{https://timesofindia.indiatimes.com/india/after-delhi-5-more-states-want-lockdown-extended-beyond-may-3/articleshow/75384555.cms}{they
will not implement the central government's loosening of restrictions.}
In Sabna, a community of farmers, Rajesh Kumar Jaiswal, the village's
leader, said changing people's psychology was one of the biggest
stumbling blocks to reopening.

``People have developed a habit of following restrictions,'' he said.
``Even if the government eases them, would people come out? No one is
gathering.''

Mr. Sharma, the small business owner, said he had tentatively opened his
shop on Sunday so that people could buy dry goods and home hardware
products.

But within a few hours, he had shut it again, fearing that he might
endanger the village and become a pariah if somebody got sick.

For now, he said, stores would stay closed.

``Those who outlive this will remember a time when people had an
opportunity to earn money, but they feared the very people who would
give them the currency note,'' he said.

Image

The normally bustling courtyard of the Jama Masjid in New Delhi on
Friday.Credit...Rebecca Conway for The New York Times

Advertisement

\protect\hyperlink{after-bottom}{Continue reading the main story}

\hypertarget{site-index}{%
\subsection{Site Index}\label{site-index}}

\hypertarget{site-information-navigation}{%
\subsection{Site Information
Navigation}\label{site-information-navigation}}

\begin{itemize}
\tightlist
\item
  \href{https://help.nytimes.com/hc/en-us/articles/115014792127-Copyright-notice}{©~2020~The
  New York Times Company}
\end{itemize}

\begin{itemize}
\tightlist
\item
  \href{https://www.nytco.com/}{NYTCo}
\item
  \href{https://help.nytimes.com/hc/en-us/articles/115015385887-Contact-Us}{Contact
  Us}
\item
  \href{https://www.nytco.com/careers/}{Work with us}
\item
  \href{https://nytmediakit.com/}{Advertise}
\item
  \href{http://www.tbrandstudio.com/}{T Brand Studio}
\item
  \href{https://www.nytimes.com/privacy/cookie-policy\#how-do-i-manage-trackers}{Your
  Ad Choices}
\item
  \href{https://www.nytimes.com/privacy}{Privacy}
\item
  \href{https://help.nytimes.com/hc/en-us/articles/115014893428-Terms-of-service}{Terms
  of Service}
\item
  \href{https://help.nytimes.com/hc/en-us/articles/115014893968-Terms-of-sale}{Terms
  of Sale}
\item
  \href{https://spiderbites.nytimes.com}{Site Map}
\item
  \href{https://help.nytimes.com/hc/en-us}{Help}
\item
  \href{https://www.nytimes.com/subscription?campaignId=37WXW}{Subscriptions}
\end{itemize}
