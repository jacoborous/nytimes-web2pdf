Sections

SEARCH

\protect\hyperlink{site-content}{Skip to
content}\protect\hyperlink{site-index}{Skip to site index}

\href{https://www.nytimes.com/section/nyregion}{New York}

\href{https://myaccount.nytimes.com/auth/login?response_type=cookie\&client_id=vi}{}

\href{https://www.nytimes.com/section/todayspaper}{Today's Paper}

\href{/section/nyregion}{New York}\textbar{}He Needs Jobless Benefits.
He Was Told to Find a Fax Machine.

\url{https://nyti.ms/34bHprj}

\begin{itemize}
\item
\item
\item
\item
\item
\end{itemize}

\href{https://www.nytimes.com/news-event/coronavirus?action=click\&pgtype=Article\&state=default\&region=TOP_BANNER\&context=storylines_menu}{The
Coronavirus Outbreak}

\begin{itemize}
\tightlist
\item
  live\href{https://www.nytimes.com/2020/08/04/world/coronavirus-cases.html?action=click\&pgtype=Article\&state=default\&region=TOP_BANNER\&context=storylines_menu}{Latest
  Updates}
\item
  \href{https://www.nytimes.com/interactive/2020/us/coronavirus-us-cases.html?action=click\&pgtype=Article\&state=default\&region=TOP_BANNER\&context=storylines_menu}{Maps
  and Cases}
\item
  \href{https://www.nytimes.com/interactive/2020/science/coronavirus-vaccine-tracker.html?action=click\&pgtype=Article\&state=default\&region=TOP_BANNER\&context=storylines_menu}{Vaccine
  Tracker}
\item
  \href{https://www.nytimes.com/2020/08/02/us/covid-college-reopening.html?action=click\&pgtype=Article\&state=default\&region=TOP_BANNER\&context=storylines_menu}{College
  Reopening}
\item
  \href{https://www.nytimes.com/live/2020/08/04/business/stock-market-today-coronavirus?action=click\&pgtype=Article\&state=default\&region=TOP_BANNER\&context=storylines_menu}{Economy}
\end{itemize}

Advertisement

\protect\hyperlink{after-top}{Continue reading the main story}

Supported by

\protect\hyperlink{after-sponsor}{Continue reading the main story}

\hypertarget{he-needs-jobless-benefits-he-was-told-to-find-a-fax-machine}{%
\section{He Needs Jobless Benefits. He Was Told to Find a Fax
Machine.}\label{he-needs-jobless-benefits-he-was-told-to-find-a-fax-machine}}

Thousands of newly unemployed New Yorkers desperate to stay afloat are
being frustrated by the state's 1970s-era technology.

\includegraphics{https://static01.nyt.com/images/2020/04/03/nyregion/00nyvirus-unemployed1/merlin_171263199_c45860ee-a2d3-43af-b58a-15b0e3edde63-articleLarge.jpg?quality=75\&auto=webp\&disable=upscale}

By \href{https://www.nytimes.com/by/patrick-mcgeehan}{Patrick McGeehan}

\begin{itemize}
\item
  Published April 4, 2020Updated April 16, 2020
\item
  \begin{itemize}
  \item
  \item
  \item
  \item
  \item
  \end{itemize}
\end{itemize}

Mohammed Saiful Islam got a taste of how antiquated the technology that
runs New York State's unemployment-insurance system is when he had to go
to a Staples store in the middle of a pandemic to fax his pay stubs to
Albany.

Mr. Islam, a Lyft driver who lives in Queens and has been idled by the
outbreak, is among more than 450,000 New Yorkers who have tried, often
in vain, to apply for
\href{https://www.nytimes.com/2020/04/16/us/california-unemployment-edd-coronavirus.html}{unemployment}
benefits in the past three weeks.

As he and many others discovered, the state's archaic systems were
woefully unprepared for the deluge of claims. In Mr. Islam's case, he
said it took him four days to reach someone who could explain what he
had to do to complete the application process.

State officials admitted as recently as last summer that there were
problems with the technology used for such applications, describing New
York's unemployment-insurance systems as relics from the heyday of
mainframe computers.

The software programs that run the systems were ``written in the 1970s
and 1980s and remain constrained by the technology of that era,''
officials wrote while seeking bids as part of a planned modernization
project.

In March, when hundreds of thousands of workers whose jobs had suddenly
evaporated started trying to log onto the Labor Department website or
call its phone lines, the systems failed.

Would-be applicants' frustration grew as their computer screens froze
repeatedly and their calls went unanswered for days. Some attempts to
apply for benefits yielded a pop-up message that suggested using
Netscape, a browser that effectively no longer exists.

Mr. Islam, who had never applied for jobless benefits in the 35 years
since he immigrated from Bangladesh, said he was taken aback to hear
that he had to find a fax machine to complete his claim.

But he put on a face mask and gloves and warily trudged off to a Staples
store. Late this week, he was still waiting to hear how much he would
receive, and when.

``Scary things are going on in our life right now,'' Mr. Islam, 49, said
in an interview from the home he shares with his wife and four children.

\hypertarget{latest-updates-global-coronavirus-outbreak}{%
\section{\texorpdfstring{\href{https://www.nytimes.com/2020/08/04/world/coronavirus-cases.html?action=click\&pgtype=Article\&state=default\&region=MAIN_CONTENT_1\&context=storylines_live_updates}{Latest
Updates: Global Coronavirus
Outbreak}}{Latest Updates: Global Coronavirus Outbreak}}\label{latest-updates-global-coronavirus-outbreak}}

Updated 2020-08-04T20:50:09.557Z

\begin{itemize}
\tightlist
\item
  \href{https://www.nytimes.com/2020/08/04/world/coronavirus-cases.html?action=click\&pgtype=Article\&state=default\&region=MAIN_CONTENT_1\&context=storylines_live_updates\#link-1228a480}{Novavax
  sees encouraging results from two studies of its experimental
  vaccine.}
\item
  \href{https://www.nytimes.com/2020/08/04/world/coronavirus-cases.html?action=click\&pgtype=Article\&state=default\&region=MAIN_CONTENT_1\&context=storylines_live_updates\#link-4825b93}{Public
  and private schools in Maryland and elsewhere are divided over
  in-person instruction.}
\item
  \href{https://www.nytimes.com/2020/08/04/world/coronavirus-cases.html?action=click\&pgtype=Article\&state=default\&region=MAIN_CONTENT_1\&context=storylines_live_updates\#link-50f7386d}{The
  United Nations calls on policymakers to `plan thoroughly for school
  reopenings.'}
\end{itemize}

\href{https://www.nytimes.com/2020/08/04/world/coronavirus-cases.html?action=click\&pgtype=Article\&state=default\&region=MAIN_CONTENT_1\&context=storylines_live_updates}{See
more updates}

More live coverage:
\href{https://www.nytimes.com/live/2020/08/04/business/stock-market-today-coronavirus?action=click\&pgtype=Article\&state=default\&region=MAIN_CONTENT_1\&context=storylines_live_updates}{Markets}

New York's governor, Andrew M. Cuomo, acknowledged the problems with the
unemployment-claims process on Tuesday.

``I apologize for the pain,'' Mr. Cuomo said at a news conference. ``It
must be infuriating to deal with.''

The Labor Department, he said, had received 1.2 million calls the day
before, after getting more than seven million calls last week. But the
state reported just 80,000 claims for the week that ended March 20 and
just 370,000 last week, far fewer than either California and
Pennsylvania reported.

``The staff at the Department of Labor are killing themselves to try to
deal with this situation,'' said Richard Blum, a lawyer with the Legal
Aid Society who advocates for worker protections and benefits. ``But the
problems that they and applicants are facing are the results of
long-term disinvestment in the system.''

Department officials did not respond to repeated inquiries about the
computer systems.

New York is not the only state having trouble handling the tidal wave of
unemployment claims.

On Thursday, the executive director of Florida's Department of Economic
Opportunity publicly
\href{https://www.tampabay.com/news/health/2020/04/02/i-apologize-for-floridas-unemployment-website-fiasco-director-says/}{apologized}
after the state's unemployment website failed. Auditors had warned the
governor, Ron DeSantis,
about\href{https://www.tampabay.com/news/health/2020/03/31/ron-desantis-was-warned-about-floridas-broken-unemployment-website-last-year-audit-shows/}{problems}
with the website last year.

Connecticut has a backlog that could take five weeks to process because
its computer system is also at least 40 years old, said Nancy Steffens,
a spokeswoman for the state's Labor Department.

Ms. Steffens said that Connecticut has had to resort to recruiting
retirees who knew how to program in COBOL, a nearly extinct computer
language. Connecticut and four other states are involved in a joint
project meant to overhaul their systems but it will not be finished
before next year, she said.

New York is not part of that effort. Instead, in 2017, the state sought
bids for a ``solution'' to its unemployment-insurance system. Last year,
it
\href{https://www.osc.state.ny.us/press/releases/july19/070119.htm}{awarded}
a \$56 million, five-year contract for that solution to Tata Consultancy
Services, which is based in Mumbai.

In a subsequent solicitation last July, the state used similar wording
to describe its ``outdated and expensive mainframe-based'' systems,
suggesting that the modernization effort would take more time.

The repeated crashing of New York's online application system was a
relatively minor setback for Elizabeth Lucia, considering what she has
been through lately. Ms. Lucia, who is 30 and several months pregnant,
lost two jobs on the same night last month.

\includegraphics{https://static01.nyt.com/images/2020/04/04/nyregion/04nyvirus-unemployed2/merlin_171267138_9d52e816-bbc9-4fdc-98e6-f64bdf717c09-articleLarge.jpg?quality=75\&auto=webp\&disable=upscale}

After Mr. Cuomo ordered nonessential work to stop, she could no longer
do either her main job, at the furniture chain Raymour \& Flanigan, or
her side gig in real-estate sales. Raymour \& Flanigan later furloughed
her, but the company said it would keep her on its health-insurance
plan.

Ms. Lucia, who lives in Vestal, N.Y., heard recently that property
showings were still allowed, but she decided to stay home anyway.

``I need to be healthy to give birth in a month,'' she explained.

Now, she said, she was counting on unemployment checks to cover her rent
and mounting expenses.

\href{https://www.nytimes.com/news-event/coronavirus?action=click\&pgtype=Article\&state=default\&region=MAIN_CONTENT_3\&context=storylines_faq}{}

\hypertarget{the-coronavirus-outbreak-}{%
\subsubsection{The Coronavirus Outbreak
›}\label{the-coronavirus-outbreak-}}

\hypertarget{frequently-asked-questions}{%
\paragraph{Frequently Asked
Questions}\label{frequently-asked-questions}}

Updated August 4, 2020

\begin{itemize}
\item ~
  \hypertarget{i-have-antibodies-am-i-now-immune}{%
  \paragraph{I have antibodies. Am I now
  immune?}\label{i-have-antibodies-am-i-now-immune}}

  \begin{itemize}
  \tightlist
  \item
    As of right
    now,\href{https://www.nytimes.com/2020/07/22/health/covid-antibodies-herd-immunity.html?action=click\&pgtype=Article\&state=default\&region=MAIN_CONTENT_3\&context=storylines_faq}{that
    seems likely, for at least several months.} There have been
    frightening accounts of people suffering what seems to be a second
    bout of Covid-19. But experts say these patients may have a
    drawn-out course of infection, with the virus taking a slow toll
    weeks to months after initial exposure. People infected with the
    coronavirus typically
    \href{https://www.nature.com/articles/s41586-020-2456-9}{produce}
    immune molecules called antibodies, which are
    \href{https://www.nytimes.com/2020/05/07/health/coronavirus-antibody-prevalence.html?action=click\&pgtype=Article\&state=default\&region=MAIN_CONTENT_3\&context=storylines_faq}{protective
    proteins made in response to an
    infection}\href{https://www.nytimes.com/2020/05/07/health/coronavirus-antibody-prevalence.html?action=click\&pgtype=Article\&state=default\&region=MAIN_CONTENT_3\&context=storylines_faq}{.
    These antibodies may} last in the body
    \href{https://www.nature.com/articles/s41591-020-0965-6}{only two to
    three months}, which may seem worrisome, but that's perfectly normal
    after an acute infection subsides, said Dr. Michael Mina, an
    immunologist at Harvard University. It may be possible to get the
    coronavirus again, but it's highly unlikely that it would be
    possible in a short window of time from initial infection or make
    people sicker the second time.
  \end{itemize}
\item ~
  \hypertarget{im-a-small-business-owner-can-i-get-relief}{%
  \paragraph{I'm a small-business owner. Can I get
  relief?}\label{im-a-small-business-owner-can-i-get-relief}}

  \begin{itemize}
  \tightlist
  \item
    The
    \href{https://www.nytimes.com/article/small-business-loans-stimulus-grants-freelancers-coronavirus.html?action=click\&pgtype=Article\&state=default\&region=MAIN_CONTENT_3\&context=storylines_faq}{stimulus
    bills enacted in March} offer help for the millions of American
    small businesses. Those eligible for aid are businesses and
    nonprofit organizations with fewer than 500 workers, including sole
    proprietorships, independent contractors and freelancers. Some
    larger companies in some industries are also eligible. The help
    being offered, which is being managed by the Small Business
    Administration, includes the Paycheck Protection Program and the
    Economic Injury Disaster Loan program. But lots of folks have
    \href{https://www.nytimes.com/interactive/2020/05/07/business/small-business-loans-coronavirus.html?action=click\&pgtype=Article\&state=default\&region=MAIN_CONTENT_3\&context=storylines_faq}{not
    yet seen payouts.} Even those who have received help are confused:
    The rules are draconian, and some are stuck sitting on
    \href{https://www.nytimes.com/2020/05/02/business/economy/loans-coronavirus-small-business.html?action=click\&pgtype=Article\&state=default\&region=MAIN_CONTENT_3\&context=storylines_faq}{money
    they don't know how to use.} Many small-business owners are getting
    less than they expected or
    \href{https://www.nytimes.com/2020/06/10/business/Small-business-loans-ppp.html?action=click\&pgtype=Article\&state=default\&region=MAIN_CONTENT_3\&context=storylines_faq}{not
    hearing anything at all.}
  \end{itemize}
\item ~
  \hypertarget{what-are-my-rights-if-i-am-worried-about-going-back-to-work}{%
  \paragraph{What are my rights if I am worried about going back to
  work?}\label{what-are-my-rights-if-i-am-worried-about-going-back-to-work}}

  \begin{itemize}
  \tightlist
  \item
    Employers have to provide
    \href{https://www.osha.gov/SLTC/covid-19/standards.html}{a safe
    workplace} with policies that protect everyone equally.
    \href{https://www.nytimes.com/article/coronavirus-money-unemployment.html?action=click\&pgtype=Article\&state=default\&region=MAIN_CONTENT_3\&context=storylines_faq}{And
    if one of your co-workers tests positive for the coronavirus, the
    C.D.C.} has said that
    \href{https://www.cdc.gov/coronavirus/2019-ncov/community/guidance-business-response.html}{employers
    should tell their employees} -\/- without giving you the sick
    employee's name -\/- that they may have been exposed to the virus.
  \end{itemize}
\item ~
  \hypertarget{should-i-refinance-my-mortgage}{%
  \paragraph{Should I refinance my
  mortgage?}\label{should-i-refinance-my-mortgage}}

  \begin{itemize}
  \tightlist
  \item
    \href{https://www.nytimes.com/article/coronavirus-money-unemployment.html?action=click\&pgtype=Article\&state=default\&region=MAIN_CONTENT_3\&context=storylines_faq}{It
    could be a good idea,} because mortgage rates have
    \href{https://www.nytimes.com/2020/07/16/business/mortgage-rates-below-3-percent.html?action=click\&pgtype=Article\&state=default\&region=MAIN_CONTENT_3\&context=storylines_faq}{never
    been lower.} Refinancing requests have pushed mortgage applications
    to some of the highest levels since 2008, so be prepared to get in
    line. But defaults are also up, so if you're thinking about buying a
    home, be aware that some lenders have tightened their standards.
  \end{itemize}
\item ~
  \hypertarget{what-is-school-going-to-look-like-in-september}{%
  \paragraph{What is school going to look like in
  September?}\label{what-is-school-going-to-look-like-in-september}}

  \begin{itemize}
  \tightlist
  \item
    It is unlikely that many schools will return to a normal schedule
    this fall, requiring the grind of
    \href{https://www.nytimes.com/2020/06/05/us/coronavirus-education-lost-learning.html?action=click\&pgtype=Article\&state=default\&region=MAIN_CONTENT_3\&context=storylines_faq}{online
    learning},
    \href{https://www.nytimes.com/2020/05/29/us/coronavirus-child-care-centers.html?action=click\&pgtype=Article\&state=default\&region=MAIN_CONTENT_3\&context=storylines_faq}{makeshift
    child care} and
    \href{https://www.nytimes.com/2020/06/03/business/economy/coronavirus-working-women.html?action=click\&pgtype=Article\&state=default\&region=MAIN_CONTENT_3\&context=storylines_faq}{stunted
    workdays} to continue. California's two largest public school
    districts --- Los Angeles and San Diego --- said on July 13, that
    \href{https://www.nytimes.com/2020/07/13/us/lausd-san-diego-school-reopening.html?action=click\&pgtype=Article\&state=default\&region=MAIN_CONTENT_3\&context=storylines_faq}{instruction
    will be remote-only in the fall}, citing concerns that surging
    coronavirus infections in their areas pose too dire a risk for
    students and teachers. Together, the two districts enroll some
    825,000 students. They are the largest in the country so far to
    abandon plans for even a partial physical return to classrooms when
    they reopen in August. For other districts, the solution won't be an
    all-or-nothing approach.
    \href{https://bioethics.jhu.edu/research-and-outreach/projects/eschool-initiative/school-policy-tracker/}{Many
    systems}, including the nation's largest, New York City, are
    devising
    \href{https://www.nytimes.com/2020/06/26/us/coronavirus-schools-reopen-fall.html?action=click\&pgtype=Article\&state=default\&region=MAIN_CONTENT_3\&context=storylines_faq}{hybrid
    plans} that involve spending some days in classrooms and other days
    online. There's no national policy on this yet, so check with your
    municipal school system regularly to see what is happening in your
    community.
  \end{itemize}
\end{itemize}

``I already have a medical bill in the mail for \$1,100 that I'm trying
to figure out what to do with,'' she said, adding that ``getting
unemployment would be a matter of sinking or swimming.''

Navigating the Labor Department's overloaded system has been a challenge
even for tech-savvy applicants like Eric Saari, who said he had once
designed mobile apps for IBM.

Mr. Saari, 50, said he worried that he might become homeless if he was
unable to get unemployment benefits. He had been driving a taxi in
Saratoga Springs, N.Y., until early March when a visibly ill passenger
wearing a mask said he might have the virus.

Mr. Saari quit driving after that and said he was down to his last
several hundred dollars, which he needed to buy food for the next month.
He has tried seven different websites and the phone numbers of at least
three government agencies to try to get help filing a claim, he said.

``Right now,'' he said, ``it's unfortunately impossible, as far as I can
tell.''

Congress has promised those who are now unemployed especially generous
benefits for the next few months: \$600 a week through July, on top of
what they would typically get from their home states.

But New York residents stand to collect less than recipients in nearby
states. The maximum weekly benefit in New York is \$504 a week, compared
with
\href{https://www.cbia.com/news/issues-policies/ct-raises-unemployment-benefit/}{\$631}
in Connecticut,
\href{https://www.nj.gov/labor/lwdhome/press/2019/20200113_benefitrates.shtml}{\$713}
in New Jersey and
\href{https://www.mass.gov/info-details/how-your-unemployment-benefits-are-determined\#calculating-your-maximum-benefit-credit-}{\$823}
in Massachusetts.

``New York's benefits system is unusually stingy,'' said Paul K. Sonn,
the state policy program director at the National Employment Law
Project, which advocates for low-wage workers. Despite the state's
progressive image, Mr. Sonn said, ``New York is not a leading state in
providing economic security for jobless workers.''

But the \$2 trillion stimulus package that Congress approved has made
far more people eligible for unemployment benefits, including artists
and other freelancers, said Nicole Salk, senior staff attorney at Legal
Services N.Y.C.

She said she had been counseling some people about how to apply for
benefits and avoid pitfalls in the process.

New York, she added, also had a distinctly punitive approach when
calculating benefits for part-time workers that withholds 25 percent of
what they receive in a week for each day they work, no matter how many
hours they put in.

That means that if an unemployed actress spent just an hour or two a day
three days a week delivering groceries for a service like Instacart, she
would forfeit 75 percent of her weekly check.

The rule, which is inconsistent with how other states handle part-time
work, gives unemployed New Yorkers a strong incentive to remain idle
while they collect benefits, Ms. Salk said.

Legislation to change the rule appeared to have the support of Mr. Cuomo
and legislative leaders last year, she said, but no bill was ever signed
into law.

Mr. Blum said he was told why: The Labor Department's primitive
computers could not be reprogrammed quickly enough to make the
adjustment.

Alex Traub contributed reporting

Advertisement

\protect\hyperlink{after-bottom}{Continue reading the main story}

\hypertarget{site-index}{%
\subsection{Site Index}\label{site-index}}

\hypertarget{site-information-navigation}{%
\subsection{Site Information
Navigation}\label{site-information-navigation}}

\begin{itemize}
\tightlist
\item
  \href{https://help.nytimes.com/hc/en-us/articles/115014792127-Copyright-notice}{©~2020~The
  New York Times Company}
\end{itemize}

\begin{itemize}
\tightlist
\item
  \href{https://www.nytco.com/}{NYTCo}
\item
  \href{https://help.nytimes.com/hc/en-us/articles/115015385887-Contact-Us}{Contact
  Us}
\item
  \href{https://www.nytco.com/careers/}{Work with us}
\item
  \href{https://nytmediakit.com/}{Advertise}
\item
  \href{http://www.tbrandstudio.com/}{T Brand Studio}
\item
  \href{https://www.nytimes.com/privacy/cookie-policy\#how-do-i-manage-trackers}{Your
  Ad Choices}
\item
  \href{https://www.nytimes.com/privacy}{Privacy}
\item
  \href{https://help.nytimes.com/hc/en-us/articles/115014893428-Terms-of-service}{Terms
  of Service}
\item
  \href{https://help.nytimes.com/hc/en-us/articles/115014893968-Terms-of-sale}{Terms
  of Sale}
\item
  \href{https://spiderbites.nytimes.com}{Site Map}
\item
  \href{https://help.nytimes.com/hc/en-us}{Help}
\item
  \href{https://www.nytimes.com/subscription?campaignId=37WXW}{Subscriptions}
\end{itemize}
