Sections

SEARCH

\protect\hyperlink{site-content}{Skip to
content}\protect\hyperlink{site-index}{Skip to site index}

\href{https://www.nytimes.com/section/business/economy}{Economy}

\href{https://myaccount.nytimes.com/auth/login?response_type=cookie\&client_id=vi}{}

\href{https://www.nytimes.com/section/todayspaper}{Today's Paper}

\href{/section/business/economy}{Economy}\textbar{}Private Equity Firm
Pushes for Broader Access to Fed Lending Program

\url{https://nyti.ms/3bTQIPg}

\begin{itemize}
\item
\item
\item
\item
\item
\end{itemize}

\href{https://www.nytimes.com/news-event/coronavirus?action=click\&pgtype=Article\&state=default\&region=TOP_BANNER\&context=storylines_menu}{The
Coronavirus Outbreak}

\begin{itemize}
\tightlist
\item
  live\href{https://www.nytimes.com/2020/08/03/world/coronavirus-covid-19.html?action=click\&pgtype=Article\&state=default\&region=TOP_BANNER\&context=storylines_menu}{Latest
  Updates}
\item
  \href{https://www.nytimes.com/interactive/2020/us/coronavirus-us-cases.html?action=click\&pgtype=Article\&state=default\&region=TOP_BANNER\&context=storylines_menu}{Maps
  and Cases}
\item
  \href{https://www.nytimes.com/interactive/2020/science/coronavirus-vaccine-tracker.html?action=click\&pgtype=Article\&state=default\&region=TOP_BANNER\&context=storylines_menu}{Vaccine
  Tracker}
\item
  \href{https://www.nytimes.com/2020/08/02/us/covid-college-reopening.html?action=click\&pgtype=Article\&state=default\&region=TOP_BANNER\&context=storylines_menu}{College
  Reopening}
\item
  \href{https://www.nytimes.com/live/2020/08/03/business/stock-market-today-coronavirus?action=click\&pgtype=Article\&state=default\&region=TOP_BANNER\&context=storylines_menu}{Economy}
\end{itemize}

Advertisement

\protect\hyperlink{after-top}{Continue reading the main story}

Supported by

\protect\hyperlink{after-sponsor}{Continue reading the main story}

\hypertarget{private-equity-firm-pushes-for-broader-access-to-fed-lending-program}{%
\section{Private Equity Firm Pushes for Broader Access to Fed Lending
Program}\label{private-equity-firm-pushes-for-broader-access-to-fed-lending-program}}

Apollo has led a push to expand a program intended to keep loans flowing
to small businesses and households by allowing for more kinds of assets
to be offered as collateral.

\includegraphics{https://static01.nyt.com/images/2020/04/05/business/03DC-FedLobby-print/merlin_168049713_6a730bf0-507f-4d01-879e-d12a69a0bea4-articleLarge.jpg?quality=75\&auto=webp\&disable=upscale}

By \href{https://www.nytimes.com/by/kate-kelly}{Kate Kelly},
\href{https://www.nytimes.com/by/jeanna-smialek}{Jeanna Smialek} and
\href{https://www.nytimes.com/by/alan-rappeport}{Alan Rappeport}

\begin{itemize}
\item
  April 4, 2020
\item
  \begin{itemize}
  \item
  \item
  \item
  \item
  \item
  \end{itemize}
\end{itemize}

As government officials fight to prevent an economic depression by
setting up emergency lending programs to keep credit flowing to
taxpayers and small businesses, a prominent private equity firm is
pushing to ensure that a broader spectrum of investments are included.

Apollo Global Management, the large private-equity and financing firm,
has been pressing government officials in recent weeks to expand the
types of assets eligible to be offered as collateral in a Federal
Reserve lending program, according to six people who have been briefed
on the firm's initiative and a draft of an Apollo presentation that was
reviewed by The New York Times.

The presentation, which was drafted by Marc J. Rowan, a co-founder of
Apollo, on March 29, and shared widely within the investor community,
argues that a Fed lending program called the Term Asset-Backed
Securities Loan Facility, or TALF, should become ``a broad program''
that would allow a wider array of assets to participate. Doing so, he
argued, could be essential to keeping the economy afloat.

TALF, which was deployed during the 2008 financial crisis to help
stabilize markets and keep loans flowing to businesses and households,
was revived by the Fed on March 23 as part of a package of programs.
While it is not yet up and running, the program will offer cheap loans
in exchange for bundles of debt, called asset-backed securities.

Those securities must be
\href{https://www.federalreserve.gov/newsevents/pressreleases/files/monetary20200323b3.pdf}{built
on} certain types of borrowing, like credit cards, auto loans, or small
business loans. To qualify for Fed assistance, those asset-backed
securities must hold the
\href{https://www.federalreserve.gov/monetarypolicy/talf.htm}{highest
possible credit rating} --- a particularly high bracket of
investment-grade, or extremely safe, credit --- meaning the loans those
securities back are highly unlikely to default.

In the presentation, Mr. Rowan argues that the program should be
expanded to include ``all investment-grade'' or relatively safe ``market
participants'' including mortgages and commercial real estate,
certificates of deposit, and many other types of assets.

In a statement late on Friday, Apollo confirmed that it had been arguing
for a ``broad application'' of TALF. ``The investment-grade market
provides funding to U.S. consumers and businesses of all sizes,'' the
company said. ``It is integral to the proper functioning of the U.S.
economy and must be restarted.''

\hypertarget{latest-updates-economy}{%
\section{\texorpdfstring{\href{https://www.nytimes.com/live/2020/08/03/business/stock-market-today-coronavirus?action=click\&pgtype=Article\&state=default\&region=MAIN_CONTENT_1\&context=storylines_live_updates}{Latest
Updates:
Economy}}{Latest Updates: Economy}}\label{latest-updates-economy}}

\href{https://www.nytimes.com/live/2020/08/03/business/stock-market-today-coronavirus?action=click\&pgtype=Article\&state=default\&region=MAIN_CONTENT_1\&context=storylines_live_updates\#the-chicago-fed-president-says-its-up-to-congress-to-save-the-economy}{3h
ago}

\href{https://www.nytimes.com/live/2020/08/03/business/stock-market-today-coronavirus?action=click\&pgtype=Article\&state=default\&region=MAIN_CONTENT_1\&context=storylines_live_updates\#the-chicago-fed-president-says-its-up-to-congress-to-save-the-economy}{The
Chicago Fed president says it's up to Congress to save the economy.}

\href{https://www.nytimes.com/live/2020/08/03/business/stock-market-today-coronavirus?action=click\&pgtype=Article\&state=default\&region=MAIN_CONTENT_1\&context=storylines_live_updates\#faa-says-boeing-has-effectively-mitigated-defects-in-the-737-max}{4h
ago}

\href{https://www.nytimes.com/live/2020/08/03/business/stock-market-today-coronavirus?action=click\&pgtype=Article\&state=default\&region=MAIN_CONTENT_1\&context=storylines_live_updates\#faa-says-boeing-has-effectively-mitigated-defects-in-the-737-max}{F.A.A.
says Boeing has `effectively mitigated' defects in the 737 Max.}

\href{https://www.nytimes.com/live/2020/08/03/business/stock-market-today-coronavirus?action=click\&pgtype=Article\&state=default\&region=MAIN_CONTENT_1\&context=storylines_live_updates\#small-businesses-got-emergency-loans-but-not-what-they-expected}{6h
ago}

\href{https://www.nytimes.com/live/2020/08/03/business/stock-market-today-coronavirus?action=click\&pgtype=Article\&state=default\&region=MAIN_CONTENT_1\&context=storylines_live_updates\#small-businesses-got-emergency-loans-but-not-what-they-expected}{Small
businesses got emergency loans, but not what they expected.}

\href{https://www.nytimes.com/live/2020/08/03/business/stock-market-today-coronavirus?action=click\&pgtype=Article\&state=default\&region=MAIN_CONTENT_1\&context=storylines_live_updates}{See
more updates}

More live coverage:
\href{https://www.nytimes.com/2020/08/03/world/coronavirus-covid-19.html?action=click\&pgtype=Article\&state=default\&region=MAIN_CONTENT_1\&context=storylines_live_updates}{Global}

Other private-equity firms, including the large investor TPG, have
backed Apollo's push, according to two of the people who were briefed on
the initiative, and Apollo noted in its statement that ``insurance
companies, retirement plans, and industry organizations'' shared its
concerns.

A spokesman for TPG declined to comment, and the Fed and the Treasury
Department declined to comment.

Since the coronavirus began spreading across the United States, the Fed
has scrambled to keep credit flowing through an increasingly turbulent
economy, dusting off 2008-era programs like TALF to backstop troubled
markets. Those efforts received a booster shot last week, when Congress
appropriated an additional \$454 billion in rescue lending, which could
be leveraged to back more than \$4 trillion in cheap loans and asset
purchases.

Apollo has argued that additional aid for the underpinning of the
economy --- notably including real estate --- is needed.

``Too little attention'' wrote Mr. Rowan in the March 29 presentation
draft, ``has been paid to the financial plumbing of the economy,''
including the markets for both complex assets known as structured
products and for real estate. ``These financial markets have seized
up,'' he added, ``and already are starting to exhibit patterns of a
full-blown panic.''

The presentation argues that TALF should be broadened to include all
investment-grade structured products --- basically financial products
built on underlying securities --- and a type of short-term debt, known
as commercial paper, that real-estate and other firms use, among other
securities.

Apollo also argued against limits on executive compensation and other
restrictions tied to those who use the facility saying that ``would
render the program unpalatable.''

The degree to which Apollo could benefit from Mr. Rowan's
recommendations are not clear. The firm oversees \$331 billion in
assets, including a range of companies in its private-equity portfolio,
but none of those carry investment-grade ratings, according to its
spokeswoman.

The firm manages a significant portfolio of products known as
collateralized loan obligations, which could be protected in Mr. Rowan's
proposed scenario if they were investment grade. Apollo also invests in
and finances real estate around the country, though that is a relatively
small part of its overall business mix. And it has a stake in the
insurance company Athene,
\href{https://www.bloomberg.com/news/articles/2019-10-28/apollo-global-management-to-buy-an-18-stake-in-athene-holdings}{which
allows Apollo to manage the money of its annuity holders}, providing it
with a valuable source of cash to invest in its fund.

``Better functioning capital markets and potentially more support in
asset prices would be beneficial for them as an organization,'' said
Devin Ryan, a research analyst at JMP Securities who covers banks and
alternative-investment firms. The potential benefits of Mr. Rowan's
recommendations were hard to gauge, Mr. Ryan said, without having much
more detail on both his proposals and Apollo's many holdings.

Apollo, whose presentation was initially reported by Bloomberg News, and
TPG aren't alone in pushing for more out of the emergency lending
program. The Structured Finance Association, an industry trade group,
has\href{https://structuredfinance.org/wp-content/uploads/2020/03/106711__113068379v7_SFA-SPARCC-comment-letter_SFALetterhead.pdf}{previously
urged} the Fed to include older securities, not just newly-issued
assets.

Members of the House Financial Services Committee, which helps to
oversee the central bank, wrote to the Fed Chair, Jerome H. Powell, on
Wednesday to urge him to expand TALF to include additional types of
consumer credit as collateral **** to help keep credit flowing as
non-bank lenders and financial technology firms struggle.

The Fed and the Treasury Department face a tough trade-off when it comes
to broadening TALF, which remains a possibility, according to a person
familiar with the matter. It could make the program riskier for the
central bank, requiring more backup from Treasury and siphoning the
backstop away from other programs --- including one that could help
midsize businesses and others that can help state and local government
debt markets.

Helping out slightly riskier companies could also reward firms that have
not carefully minded their credit ratings, putting a floor under them
even if they made less-responsible choices when times were good.

Yet the Fed's programs are meant to improve market functioning and if
corporate debt markets come under extended strain, it could prove bad
for the economy as a whole.

``The policy goal should be to address the parts of the market that are
the most critical and require the most help,'' according to the draft
presentation.

The document states that given the severity of the coronavirus shock,
and compared to economic packages rolled out in Europe and China,
Congress' \$2 trillion rescue package ``is severely underestimating the
size of the required response.''

Advertisement

\protect\hyperlink{after-bottom}{Continue reading the main story}

\hypertarget{site-index}{%
\subsection{Site Index}\label{site-index}}

\hypertarget{site-information-navigation}{%
\subsection{Site Information
Navigation}\label{site-information-navigation}}

\begin{itemize}
\tightlist
\item
  \href{https://help.nytimes.com/hc/en-us/articles/115014792127-Copyright-notice}{©~2020~The
  New York Times Company}
\end{itemize}

\begin{itemize}
\tightlist
\item
  \href{https://www.nytco.com/}{NYTCo}
\item
  \href{https://help.nytimes.com/hc/en-us/articles/115015385887-Contact-Us}{Contact
  Us}
\item
  \href{https://www.nytco.com/careers/}{Work with us}
\item
  \href{https://nytmediakit.com/}{Advertise}
\item
  \href{http://www.tbrandstudio.com/}{T Brand Studio}
\item
  \href{https://www.nytimes.com/privacy/cookie-policy\#how-do-i-manage-trackers}{Your
  Ad Choices}
\item
  \href{https://www.nytimes.com/privacy}{Privacy}
\item
  \href{https://help.nytimes.com/hc/en-us/articles/115014893428-Terms-of-service}{Terms
  of Service}
\item
  \href{https://help.nytimes.com/hc/en-us/articles/115014893968-Terms-of-sale}{Terms
  of Sale}
\item
  \href{https://spiderbites.nytimes.com}{Site Map}
\item
  \href{https://help.nytimes.com/hc/en-us}{Help}
\item
  \href{https://www.nytimes.com/subscription?campaignId=37WXW}{Subscriptions}
\end{itemize}
