Sections

SEARCH

\protect\hyperlink{site-content}{Skip to
content}\protect\hyperlink{site-index}{Skip to site index}

\href{https://www.nytimes.com/section/us}{U.S.}

\href{https://myaccount.nytimes.com/auth/login?response_type=cookie\&client_id=vi}{}

\href{https://www.nytimes.com/section/todayspaper}{Today's Paper}

\href{/section/us}{U.S.}\textbar{}How George Floyd Was Killed in Police
Custody

\url{https://nyti.ms/2XMtUMa}

\begin{itemize}
\item
\item
\item
\item
\item
\item
\end{itemize}

\href{https://www.nytimes.com/news-event/george-floyd-protests-minneapolis-new-york-los-angeles?action=click\&pgtype=Article\&state=default\&region=TOP_BANNER\&context=storylines_menu}{Race
and America}

\begin{itemize}
\tightlist
\item
  \href{https://www.nytimes.com/2020/07/26/us/protests-portland-seattle-trump.html?action=click\&pgtype=Article\&state=default\&region=TOP_BANNER\&context=storylines_menu}{Protesters
  Return to Other Cities}
\item
  \href{https://www.nytimes.com/2020/07/24/us/portland-oregon-protests-white-race.html?action=click\&pgtype=Article\&state=default\&region=TOP_BANNER\&context=storylines_menu}{Portland
  at the Center}
\item
  \href{https://www.nytimes.com/2020/07/23/podcasts/the-daily/portland-protests.html?action=click\&pgtype=Article\&state=default\&region=TOP_BANNER\&context=storylines_menu}{Podcast:
  Showdown in Portland}
\item
  \href{https://www.nytimes.com/interactive/2020/07/16/us/black-lives-matter-protests-louisville-breonna-taylor.html?action=click\&pgtype=Article\&state=default\&region=TOP_BANNER\&context=storylines_menu}{45
  Days in Louisville}
\end{itemize}

Advertisement

\protect\hyperlink{after-top}{Continue reading the main story}

Supported by

\protect\hyperlink{after-sponsor}{Continue reading the main story}

\hypertarget{how-george-floyd-was-killed-in-police-custody}{%
\section{How George Floyd Was Killed in Police
Custody}\label{how-george-floyd-was-killed-in-police-custody}}

The Times has reconstructed the death of George Floyd on May 25.
Security footage, witness videos and official documents show how a
series of actions by officers turned fatal.

\includegraphics{https://static01.nyt.com/images/2020/05/27/autossell/flyod-site-1-white-box/flyod-site-1-white-box-videoSixteenByNineJumbo1600.jpg}

By \href{https://www.nytimes.com/by/evan-hill}{Evan Hill},
\href{https://www.nytimes.com/by/ainara-tiefenthaler}{Ainara
Tiefenthäler},
\href{http://nytimes.com/by/christiaan-triebert}{Christiaan Triebert},
\href{https://www.nytimes.com/by/drew-jordan}{Drew Jordan},
\href{https://www.nytimes.com/by/haley-willis}{Haley Willis} and Robin
Stein

\begin{itemize}
\item
  Published May 31, 2020Updated July 28, 2020
\item
  \begin{itemize}
  \item
  \item
  \item
  \item
  \item
  \item
  \end{itemize}
\end{itemize}

\href{https://cn.nytimes.com/usa/20200602/george-floyd-investigation/}{阅读简体中文版}\href{https://cn.nytimes.com/usa/20200602/george-floyd-investigation/zh-hant}{閱讀繁體中文版}

On May 25,
\href{https://www.nytimes.com/2020/07/28/us/umbrella-man-identified-minneapolis.html}{Minneapolis
police} officers arrested
\href{https://www.nytimes.com/2020/07/28/us/umbrella-man-identified-minneapolis.html}{George
Floyd}, a 46-year-old black man, after a convenience store employee
called 911 and told the police that Mr. Floyd had bought cigarettes with
a counterfeit \$20 bill. Seventeen minutes after the first squad car
arrived at the scene, Mr. Floyd was unconscious and pinned beneath three
police officers, showing no signs of life.

By combining videos from bystanders and security cameras, reviewing
official documents and consulting experts, The New York Times
reconstructed in detail the minutes leading to Mr. Floyd's death. Our
video shows officers taking a series of actions that violated the
policies of the Minneapolis Police Department and turned fatal, leaving
Mr. Floyd unable to breathe, even as he and onlookers called out for
help.

The day after Mr. Floyd's death, the Police Department fired all four of
the officers involved in the episode. On May 29, the Hennepin County
attorney, Mike Freeman, announced third-degree murder and second-degree
manslaughter charges against
\href{https://www.nytimes.com/2020/07/22/us/derek-chauvin-tax-fraud.html}{Derek
Chauvin}, the officer seen most clearly in
\href{https://www.nytimes.com/2020/06/04/us/politics/george-floyd-witness-maurice-lester-hall.html}{witness
videos} pinning Mr. Floyd to the ground. Mr. Chauvin, who is white, kept
his knee on Mr. Floyd's neck for at least eight minutes and 15 seconds,
according to a Times analysis of timestamped video. Our video
investigation shows that Mr. Chauvin did not remove his knee even after
Mr. Floyd lost consciousness and for a full minute and 20 seconds after
paramedics arrived at the scene.

On June 3, Hennepin County prosecutors added a more serious
second-degree murder charge against Mr. Chauvin and also charged each of
the three other former officers --- Thomas Lane,
\href{https://www.nytimes.com/2020/06/27/us/minneapolis-police-officer-kueng.html}{J.
Alexander Kueng} and Tou Thao --- with aiding and abetting second-degree
murder.

On June 18,
\href{https://m.startribune.com/7-minutes-46-seconds-county-says-error-in-timeline-of-floyd-s-killing-won-t-affect-charges/571322872/}{the
Hennepin County attorney's office said} that its criminal complaint
misstated the amount of time Mr. Chauvin kept his knee on Mr. Floyd's
neck. The complaint originally said that Mr. Chauvin had done so for
eight minutes and 46 seconds, a length of time that
\href{https://www.nytimes.com/2020/06/18/us/george-floyd-timing.html}{became
a symbol and rallying cry for protesters}. Responding to inquiries from
journalists who noted a discrepancy with the durations listed in the
complaint, the office said the actual time was seven minutes and 46
seconds. But The Times's own analysis of the video shows that
\href{https://www.nytimes.com/2020/06/18/us/george-floyd-timing.html}{this
revised time is also incorrect}.

``It makes no difference,'' said Jamar Nelson, who works with the
families of crime victims in Minneapolis. ``The bottom line is, it was
long enough to kill him, long enough to execute him.''

Nicholas Bogel-Burroughs contributed reporting.

Advertisement

\protect\hyperlink{after-bottom}{Continue reading the main story}

\hypertarget{site-index}{%
\subsection{Site Index}\label{site-index}}

\hypertarget{site-information-navigation}{%
\subsection{Site Information
Navigation}\label{site-information-navigation}}

\begin{itemize}
\tightlist
\item
  \href{https://help.nytimes.com/hc/en-us/articles/115014792127-Copyright-notice}{©~2020~The
  New York Times Company}
\end{itemize}

\begin{itemize}
\tightlist
\item
  \href{https://www.nytco.com/}{NYTCo}
\item
  \href{https://help.nytimes.com/hc/en-us/articles/115015385887-Contact-Us}{Contact
  Us}
\item
  \href{https://www.nytco.com/careers/}{Work with us}
\item
  \href{https://nytmediakit.com/}{Advertise}
\item
  \href{http://www.tbrandstudio.com/}{T Brand Studio}
\item
  \href{https://www.nytimes.com/privacy/cookie-policy\#how-do-i-manage-trackers}{Your
  Ad Choices}
\item
  \href{https://www.nytimes.com/privacy}{Privacy}
\item
  \href{https://help.nytimes.com/hc/en-us/articles/115014893428-Terms-of-service}{Terms
  of Service}
\item
  \href{https://help.nytimes.com/hc/en-us/articles/115014893968-Terms-of-sale}{Terms
  of Sale}
\item
  \href{https://spiderbites.nytimes.com}{Site Map}
\item
  \href{https://help.nytimes.com/hc/en-us}{Help}
\item
  \href{https://www.nytimes.com/subscription?campaignId=37WXW}{Subscriptions}
\end{itemize}
