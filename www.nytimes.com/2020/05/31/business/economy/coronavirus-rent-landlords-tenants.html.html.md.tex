Sections

SEARCH

\protect\hyperlink{site-content}{Skip to
content}\protect\hyperlink{site-index}{Skip to site index}

\href{https://www.nytimes.com/section/business/economy}{Economy}

\href{https://myaccount.nytimes.com/auth/login?response_type=cookie\&client_id=vi}{}

\href{https://www.nytimes.com/section/todayspaper}{Today's Paper}

\href{/section/business/economy}{Economy}\textbar{}Tenants Largely Stay
Current on Rent, for Now

\href{https://nyti.ms/2Mh8t0d}{https://nyti.ms/2Mh8t0d}

\begin{itemize}
\item
\item
\item
\item
\item
\end{itemize}

\href{https://www.nytimes.com/news-event/coronavirus?action=click\&pgtype=Article\&state=default\&region=TOP_BANNER\&context=storylines_menu}{The
Coronavirus Outbreak}

\begin{itemize}
\tightlist
\item
  live\href{https://www.nytimes.com/2020/08/08/world/coronavirus-updates.html?action=click\&pgtype=Article\&state=default\&region=TOP_BANNER\&context=storylines_menu}{Latest
  Updates}
\item
  \href{https://www.nytimes.com/interactive/2020/us/coronavirus-us-cases.html?action=click\&pgtype=Article\&state=default\&region=TOP_BANNER\&context=storylines_menu}{Maps
  and Cases}
\item
  \href{https://www.nytimes.com/interactive/2020/science/coronavirus-vaccine-tracker.html?action=click\&pgtype=Article\&state=default\&region=TOP_BANNER\&context=storylines_menu}{Vaccine
  Tracker}
\item
  \href{https://www.nytimes.com/interactive/2020/world/coronavirus-tips-advice.html?action=click\&pgtype=Article\&state=default\&region=TOP_BANNER\&context=storylines_menu}{F.A.Q.}
\item
  \href{https://www.nytimes.com/live/2020/08/07/business/stock-market-today-coronavirus?action=click\&pgtype=Article\&state=default\&region=TOP_BANNER\&context=storylines_menu}{Markets
  \& Economy}
\end{itemize}

Advertisement

\protect\hyperlink{after-top}{Continue reading the main story}

Supported by

\protect\hyperlink{after-sponsor}{Continue reading the main story}

\hypertarget{tenants-largely-stay-current-on-rent-for-now}{%
\section{Tenants Largely Stay Current on Rent, for
Now}\label{tenants-largely-stay-current-on-rent-for-now}}

Collections have been surprisingly strong through the pandemic, but
there are troubling signs --- for landlords and tenants alike.

\includegraphics{https://static01.nyt.com/images/2020/06/01/business/31virus-rent2-print/merlin_172225968_8711da2e-1083-413e-8bda-ca79c346b008-articleLarge.jpg?quality=75\&auto=webp\&disable=upscale}

\href{https://www.nytimes.com/by/conor-dougherty}{\includegraphics{https://static01.nyt.com/images/2018/07/27/multimedia/author-conor-dougherty/author-conor-dougherty-thumbLarge.png}}

By \href{https://www.nytimes.com/by/conor-dougherty}{Conor Dougherty}

\begin{itemize}
\item
  May 31, 2020
\item
  \begin{itemize}
  \item
  \item
  \item
  \item
  \item
  \end{itemize}
\end{itemize}

Since April, landlords have looked to the first of the month fearing
that tenants will stop paying their rent. For the most part, that has
not happened. Despite a
\href{https://www.nytimes.com/interactive/2020/05/08/business/economy/april-jobs-report.html}{14.7
percent} unemployment rate and millions of new jobless claims each week,
collections at many buildings are only slightly below where they were
last year, when the economy was booming.

How can this be? The answer is a little negotiation and a lot of
government money. The \$2 trillion CARES Act, which
\href{https://www.nytimes.com/2020/05/28/business/economy/coronavirus-stimulus-unemployment.html}{backstopped
household finances with stimulus checks and extended unemployment
benefits}, has kept a surprising number of tenants current on their
monthly balances. At the same time, many landlords have reduced rents or
are forgiving overdue payments in full or in part.

The trend cannot continue without a swift and robust recovery, which is
becoming
\href{https://www.nytimes.com/2020/05/11/upshot/virus-lasting-economic-effects.html}{increasingly
unlikely}, or without another big injection of government money, which
Senate Republicans say is not happening anytime soon. American
households were
\href{https://www.mprnews.org/story/2020/01/31/report-more-middleincome-renters-burdened-by-housing-costs}{struggling}
with rent long before the economy went into free fall, and there are
signs --- from an increase in partial payments to surveys that show many
tenants are
\href{https://www.gozego.com/articles/may-rent-payment-data-reveals-april-trends-have-continued-as-a-result-of-covid-19/}{putting
rent} on their
\href{https://www.wsj.com/articles/out-of-work-apartment-tenants-putting-monthly-rent-on-plastic-11586966251}{credit
cards} and struggling to
\href{https://www.urban.org/urban-wire/when-people-cant-pay-their-rent-what-comes-next}{pay
for essentials} like food --- that this pressure is building.

\hypertarget{there-are-cracks-under-the-surface}{%
\subsection{There are cracks under the
surface.}\label{there-are-cracks-under-the-surface}}

\includegraphics{https://static01.nyt.com/images/2020/06/01/business/31virus-rent1-print/merlin_172686150_85b9e4c4-b6fb-41bc-94a4-fd308c530b68-articleLarge.jpg?quality=75\&auto=webp\&disable=upscale}

When the coronavirus outbreak started shutting down the economy in
March, there was widespread fear that millions of tenants would fall
behind on their monthly bills. Renters were already struggling with
housing costs, with a
\href{https://www.jchs.harvard.edu/sites/default/files/Harvard_JCHS_Americas_Rental_Housing_2020.pdf}{quarter
of tenant households} paying more than half their before-tax income on
rent and utilities, and the loss of jobs and hours seemed almost certain
to worsen those troubles.

Much of the available data has told a different story. In April, the
National Multifamily Housing Council started releasing weekly
\href{https://www.nmhc.org/research-insight/nmhc-rent-payment-tracker/}{payment
tallies} covering about a quarter of the nation's rental units, and
aside from a
\href{https://www.nytimes.com/2020/04/08/business/economy/coronavirus-rent.html}{substantial
dip} in the first week, the collection rates have been only slightly
below where they were last year. Through May 20, landlords in the
council's survey had received 90.8 percent of rents, compared with 93
percent a year earlier. A similar story has played out in state surveys
and
\href{https://www.wsj.com/articles/high-end-apartment-owners-dodge-economic-slump-11590494401?emailToken=7af94d1507aba35de1477aea302a17e246hHQSXuT8AowEI0zTGOdHODX64FAs5KBUKvaHttT+ROlMHWE/QuAqz0LNAL454jv8gJc2RCvYF+n6ctvCL88yTvyf22xn2yycQF/R7OmDg\%3D\&reflink=article_gmail_share}{earnings
reports} from real estate investment trusts like
\href{http://ir.maac.com/Cache/IRCache/5f994f94-850d-3395-cd7d-3dc6624075c9.PDF?O=PDF\&T=\&Y=\&D=\&FID=5f994f94-850d-3395-cd7d-3dc6624075c9\&iid=103123}{Mid-America
Apartment Company} and
\href{http://investors.equityapartments.com/file/Index?KeyFile=403555723}{Equity
Residential}.

But many of these numbers skew toward higher-end buildings. Other
surveys show that buildings with poorer tenants have lower
\href{https://olis.oregonlegislature.gov/liz/2019I1/Downloads/CommitteeMeetingDocument/221959}{collection
rates}. Meantime, deferrals and partial payments appear to be
increasing:
\href{https://www.apartmentlist.com/rentonomics/may-housing-payments/}{Apartment
List}, a rental listing service, said 31 percent of respondents failed
to make the full May payment on time, up from a quarter the month
before. Hoping for a swift recovery, many landlords are telling tenants
they can pay later, knowing this often won't happen.

\hypertarget{latest-updates-the-coronavirus-outbreak-and-the-economy}{%
\section{\texorpdfstring{\href{https://www.nytimes.com/live/2020/08/07/business/stock-market-today-coronavirus?action=click\&pgtype=Article\&state=default\&region=MAIN_CONTENT_1\&context=storylines_live_updates}{Latest
Updates: The Coronavirus Outbreak and the
Economy}}{Latest Updates: The Coronavirus Outbreak and the Economy}}\label{latest-updates-the-coronavirus-outbreak-and-the-economy}}

\href{https://www.nytimes.com/live/2020/08/07/business/stock-market-today-coronavirus?action=click\&pgtype=Article\&state=default\&region=MAIN_CONTENT_1\&context=storylines_live_updates\#wealthy-families-are-throwing-a-lifeline-to-distressed-businesses}{15h
ago}

\href{https://www.nytimes.com/live/2020/08/07/business/stock-market-today-coronavirus?action=click\&pgtype=Article\&state=default\&region=MAIN_CONTENT_1\&context=storylines_live_updates\#wealthy-families-are-throwing-a-lifeline-to-distressed-businesses}{Wealthy
families are throwing a lifeline to distressed businesses.}

\href{https://www.nytimes.com/live/2020/08/07/business/stock-market-today-coronavirus?action=click\&pgtype=Article\&state=default\&region=MAIN_CONTENT_1\&context=storylines_live_updates\#the-publisher-of-the-onion-jezebel-and-other-websites-lays-off-15-employees}{16h
ago}

\href{https://www.nytimes.com/live/2020/08/07/business/stock-market-today-coronavirus?action=click\&pgtype=Article\&state=default\&region=MAIN_CONTENT_1\&context=storylines_live_updates\#the-publisher-of-the-onion-jezebel-and-other-websites-lays-off-15-employees}{The
publisher of The Onion, Jezebel and other websites lays off 15
employees.}

\href{https://www.nytimes.com/live/2020/08/07/business/stock-market-today-coronavirus?action=click\&pgtype=Article\&state=default\&region=MAIN_CONTENT_1\&context=storylines_live_updates\#canada-outlines-its-response-to-the-new-us-aluminum-tariff}{21h
ago}

\href{https://www.nytimes.com/live/2020/08/07/business/stock-market-today-coronavirus?action=click\&pgtype=Article\&state=default\&region=MAIN_CONTENT_1\&context=storylines_live_updates\#canada-outlines-its-response-to-the-new-us-aluminum-tariff}{Canada
outlines its response to the new U.S. aluminum tariff.}

\href{https://www.nytimes.com/live/2020/08/07/business/stock-market-today-coronavirus?action=click\&pgtype=Article\&state=default\&region=MAIN_CONTENT_1\&context=storylines_live_updates}{See
more updates}

More live coverage:
\href{https://www.nytimes.com/2020/08/07/world/covid-19-news.html?action=click\&pgtype=Article\&state=default\&region=MAIN_CONTENT_1\&context=storylines_live_updates}{Global}

``Landlords and renters will share in the pain,'' said John Pawlowski,
an analyst with Green Street Advisors, a real estate research firm in
Newport Beach, Calif. ``We just don't know what the sharing balance will
look like.''

New Story, a San Francisco-based nonprofit organization that builds and
finances affordable housing, recently raised \$2 million to help renters
struggling to make their bills because of coronavirus-related job
losses. Alexandria Lafci, a founder of the organization and its chief
operating officer, has spent the last few weeks calling landlords to
haggle on behalf of tenants.

``I called 21 properties and got eight yeses with an average of 20
percent off,'' she said. Only three landlords rejected any
accommodation, with the rest agreeing to arrangements like lower
payments for the next three months or shaving down past-due balances to
give tenants a break without lowering their advertised rents.

\hypertarget{smaller-landlords-and-affordable-housing-are-in-trouble}{%
\subsection{Smaller landlords and affordable housing are in
trouble.}\label{smaller-landlords-and-affordable-housing-are-in-trouble}}

Image

``Landlords and renters will share in the pain,'' an analyst said, but
it's not clear in what measure.Credit...Tom Brenner/Reuters

Rental housing is a fragmented business, with purveyors ranging from
publicly traded corporations that own tens of thousands of units to
operators of only one or two. And falling rent collections are more
likely to affect smaller landlords, who tend to have a limited financial
cushion and less capacity to borrow.

These landlords play an important role in the housing system ---
especially for lower-income tenants. Individual investors own about half
the supply of low-cost units, and many are what housing advocates call
``naturally occurring affordable housing,'' or homes and apartments that
carry below-market rents even without a subsidy, according to the Joint
Center for Housing Studies at Harvard. These units, which overwhelmingly
consist of small apartment buildings and single-family homes, are also
more likely to have tenants affected by the coronavirus, with
\href{https://www.jchs.harvard.edu/blog/covid-19-rent-shortfalls-in-small-buildings/}{more
than half} of renters in the hardest-hit occupations living in
single-family homes and duplexes, according to the center.

Naturally occurring affordable housing is often overlooked, but these
units are crucial. Government housing programs like Section 8 rental
vouchers and the Low-Income Housing Tax Credit do not come close to
satisfying the demand for lower-cost housing. This is why cities have
\href{https://www.nytimes.com/2017/05/09/magazine/how-homeownership-became-the-engine-of-american-inequality.html}{yearslong
lists for vouchers} and
\href{https://www.nytimes.com/2018/05/12/upshot/these-95-apartments-promised-affordable-rent-in-san-francisco-then-6580-people-applied.html}{lotteries}
for the tiny number of newly built subsidized units.

Getting such housing is laborious and invasive, and it leaves out
workers like undocumented immigrants and families whose incomes put them
just beyond the threshold to qualify. Naturally occurring affordable
housing is in a sense more valuable, because it represents units that
anyone --- someone switching jobs or fleeing an abusive spouse, for
instance --- can find on Craigslist.

This housing can also be easily lost, not because it disappears, but
because it is purchased by a homeowner or investor who renovates in
hopes of increasing rents. This is what has happened over the last two
decades: Since 2014, according to Harvard's Joint Center, the nation has
lost about
\href{https://www.jchs.harvard.edu/blog/the-continuing-decline-of-low-cost-rentals/}{2.7
million affordable units}, defined as those carrying rents less than
\$600.

Carline Chery, 50, owns three Boston duplexes. Two-bedroom units go for
\$1,800, more than what the lowest-income renters can pay but roughly
\$900 less than the typical rent in the metropolitan area, according to
Zillow. Compared with a public company, Ms. Chery runs a shoestring
operation, with no reserves and little capacity to absorb a missed
month.

So when tenants in one of her buildings recently stopped paying, she
borrowed from family members to make the mortgage payment, then put the
building up for sale. The strongest interest has come not from another
landlord, but a first-time home buyer.

``I cannot afford it anymore,'' Ms. Chery said.

\hypertarget{landlords-and-tenants-both-want-more-money}{%
\subsection{Landlords and tenants both want more
money.}\label{landlords-and-tenants-both-want-more-money}}

Fearing a surge in homelessness, state and local governments spent March
and April instituting triage measures, like
\href{https://nlihc.org/eviction-and-foreclosure-moratoriums}{bans on
evictions} and utility shut-offs, along with limited subsidies for
struggling renters. The CARES Act also offered aid to public-housing
providers and grants to state governments that could be used for rental
assistance.

Since then, tenant activists have unified around a cry of
\href{https://www.nytimes.com/2020/05/01/nyregion/rent-strike-coronavirus.html}{\#CancelRent},
staging car rallies and
\href{https://www.nytimes.com/2020/04/23/business/economy/coronavirus-tenants-rent-protests.html}{roadside
protests} to demand that the government halt rent and mortgage payments
--- without the accrual of back payments --- as long as the economy is
battered by the coronavirus. Representative Ilhan Omar, Democrat of
Minnesota, introduced a bill that roughly mirrors that desire.

Although the bill has little chance of passing, housing advocates and
landlords' groups have pressed for more direct help to renters. The
CARES Act allotted \$12 billion in housing grants to cities, homeless
shelters, affordable-housing providers and states, but the money was
largely directed to renters and landlords in public or subsidized
housing. That leaves out most moderate- and low-income tenants who live
in market-rate developments, and small landlords like Ms. Chery, whose
loans are held by private lenders and not backed by the federal
government.

The House of Representatives recently passed the \$3 trillion
\href{https://www.congress.gov/bill/116th-congress/house-bill/6800/text}{HEROES
Act}, which in addition to more financial stimulus to households
included \$100 billion in rental subsidies for tenants affected by
coronavirus-related job loss. That bill has no prospect of Senate
approval, but landlord and tenant groups continue to push for expanded
aid for tenants.

``Small landlords and renters depend on each other, and both need
emergency assistance to stay afloat during this time,'' said Diane
Yentel, chief executive of the National Low Income Housing Coalition.
``Without it, we will end this crisis having saddled low-income renters
with more debt, and having lost more of our country's critical housing
stock.''

Advertisement

\protect\hyperlink{after-bottom}{Continue reading the main story}

\hypertarget{site-index}{%
\subsection{Site Index}\label{site-index}}

\hypertarget{site-information-navigation}{%
\subsection{Site Information
Navigation}\label{site-information-navigation}}

\begin{itemize}
\tightlist
\item
  \href{https://help.nytimes.com/hc/en-us/articles/115014792127-Copyright-notice}{©~2020~The
  New York Times Company}
\end{itemize}

\begin{itemize}
\tightlist
\item
  \href{https://www.nytco.com/}{NYTCo}
\item
  \href{https://help.nytimes.com/hc/en-us/articles/115015385887-Contact-Us}{Contact
  Us}
\item
  \href{https://www.nytco.com/careers/}{Work with us}
\item
  \href{https://nytmediakit.com/}{Advertise}
\item
  \href{http://www.tbrandstudio.com/}{T Brand Studio}
\item
  \href{https://www.nytimes.com/privacy/cookie-policy\#how-do-i-manage-trackers}{Your
  Ad Choices}
\item
  \href{https://www.nytimes.com/privacy}{Privacy}
\item
  \href{https://help.nytimes.com/hc/en-us/articles/115014893428-Terms-of-service}{Terms
  of Service}
\item
  \href{https://help.nytimes.com/hc/en-us/articles/115014893968-Terms-of-sale}{Terms
  of Sale}
\item
  \href{https://spiderbites.nytimes.com}{Site Map}
\item
  \href{https://help.nytimes.com/hc/en-us}{Help}
\item
  \href{https://www.nytimes.com/subscription?campaignId=37WXW}{Subscriptions}
\end{itemize}
