Sections

SEARCH

\protect\hyperlink{site-content}{Skip to
content}\protect\hyperlink{site-index}{Skip to site index}

\href{https://www.nytimes.com/section/science}{Science}

\href{https://myaccount.nytimes.com/auth/login?response_type=cookie\&client_id=vi}{}

\href{https://www.nytimes.com/section/todayspaper}{Today's Paper}

\href{/section/science}{Science}\textbar{}SpaceX Lifts NASA Astronauts
to Orbit, Launching New Era of Spaceflight

\url{https://nyti.ms/2BiKyeL}

\begin{itemize}
\item
\item
\item
\item
\item
\item
\end{itemize}

\href{https://www.nytimes.com/2020/08/02/science/spacex-astronauts-splashdown.html?action=click\&pgtype=Article\&state=default\&region=TOP_BANNER\&context=storylines_menu}{SpaceX's
Astronaut Trip}

\begin{itemize}
\tightlist
\item
  \href{https://www.nytimes.com/2020/08/02/science/spacex-astronauts-splashdown.html?action=click\&pgtype=Article\&state=default\&region=TOP_BANNER\&context=storylines_menu}{`Thanks
  for Flying SpaceX'}
\item
  \href{https://www.nytimes.com/2020/05/26/science/spacex-launch-nasa.html?action=click\&pgtype=Article\&state=default\&region=TOP_BANNER\&context=storylines_menu}{Why
  NASA Picked SpaceX}
\item
  \href{https://www.nytimes.com/interactive/2020/05/26/science/spacex-nasa.html?action=click\&pgtype=Article\&state=default\&region=TOP_BANNER\&context=storylines_menu}{Inside
  the Capsule}
\item
  \href{https://www.nytimes.com/2020/05/27/science/bob-behnken-doug-hurley.html?action=click\&pgtype=Article\&state=default\&region=TOP_BANNER\&context=storylines_menu}{Meet
  the Astronauts}
\end{itemize}

Advertisement

\protect\hyperlink{after-top}{Continue reading the main story}

Supported by

\protect\hyperlink{after-sponsor}{Continue reading the main story}

\hypertarget{spacex-lifts-nasa-astronauts-to-orbit-launching-new-era-of-spaceflight}{%
\section{SpaceX Lifts NASA Astronauts to Orbit, Launching New Era of
Spaceflight}\label{spacex-lifts-nasa-astronauts-to-orbit-launching-new-era-of-spaceflight}}

The trip to the space station was the first from American soil since
2011 when the space shuttles were retired.

\includegraphics{https://static01.nyt.com/images/2020/05/30/science/30nasa-launch1/30nasa-launch1-videoSixteenByNine3000.jpg}

\href{https://www.nytimes.com/by/kenneth-chang}{\includegraphics{https://static01.nyt.com/images/2018/02/16/multimedia/author-kenneth-chang/author-kenneth-chang-thumbLarge.jpg}}

By \href{https://www.nytimes.com/by/kenneth-chang}{Kenneth Chang}

\begin{itemize}
\item
  Published May 30, 2020Updated Aug. 1, 2020
\item
  \begin{itemize}
  \item
  \item
  \item
  \item
  \item
  \item
  \end{itemize}
\end{itemize}

KENNEDY SPACE CENTER, Fla. --- The United States opened a new era of
human space travel on Saturday as a private company for the first time
launched astronauts into orbit, nearly a decade after the government
retired the storied space shuttle program in the aftermath of national
tragedy.

Two American astronauts lifted off at 3:22 p.m. from a familiar setting,
the same Florida launchpad that once served Apollo missions and the
space shuttles. But the rocket and capsule that lofted them out of the
atmosphere were a new sight for many --- built and operated not by NASA
but SpaceX, the company founded by the billionaire Elon Musk to pursue
his dream of sending colonists to Mars.

\emph{\emph{\emph{{[}}\href{https://www.nytimes.com/2020/08/02/science/spacex-nasa-return.html}{\emph{Follow
two NASA astronauts' return trip aboard SpaceX's Crew Dragon
capsule}}}.{]}}**

Crowds of spectators including President Trump and Vice President Mike
Pence watched and cheered as the countdown ticked to zero, and the
engines of a Falcon 9 rocket roared to life.

Rising slowly at first, the rocket then shot like a sleek, silvery
javelin into humid skies, three days after an earlier launch was
canceled because of concerns about lightning and other unsafe weather.

It was a moment of triumph and perhaps nostalgia for the country, a
welcome reminder of America's global pre-eminence in science,
technological innovation and private enterprise at a time its prospects
and ambitions have been clouded by the coronavirus pandemic, economic
uncertainty and political strife. Millions around the world watched the
launch online and on television, many from self-imposed quarantine in
their homes.

Mr. Trump, who watched from a rooftop at the space center, called it
``an inspiration for our country'' after the ship lifted off.

``Today's launch makes clear the commercial space industry is the
future,''
\href{https://www.nytimes.com/2020/05/30/us/politics/trump-spacex-launch.html}{Mr.
Trump said in later remarks in the giant Vehicle Assembly Building}
where the Saturn 5 moon rocket was once stacked.

``We have created the envy of the world and will soon be landing on Mars
and will soon have the greatest weapons ever imagined in history,'' he
added.

The Falcon 9 carried a Crew Dragon capsule, guided by teams of SpaceX
personnel in control centers in Florida and Hawthorne, Calif. It was
scheduled to rendezvous with the International Space Station on Sunday
morning.

Aboard are two veterans of the astronauts corps, Robert L. Behnken and
Douglas O. Hurley. NASA selected the two men along with a group of their
colleagues to be the first customers of space capsules built by private
companies.

It was the first launch of NASA astronauts from the United States since
the retirement of the space shuttles in 2011. In the years since, NASA
has paid Russia's space program to transport its astronauts to the space
station. And with this success, NASA, to its own delight, has begun
ceding this task to SpaceX and other companies, and it opens new
possibilities for entrepreneurs looking to make money off the planet.

The transformation at NASA, especially of its human spaceflight program,
has been slow in coming, and Mr. Musk, a 48-year-old native of South
Africa who started SpaceX with little knowledge of rockets, is an
unlikely person to be at the vanguard.

``I'm really quite overcome with emotion on this day,'' Mr. Musk said
during a news conference after the launch. ``It's hard to talk,
frankly.''

At the turn of the millennium, Mr. Musk, still in his 20s, was chief
executive of PayPal, making his first fortune when eBay bought that
company.

Mr. Musk selected a quixotic project as the focus of his newfound
wealth: sending a small greenhouse to Mars.

Because rockets were so expensive in the United States, he went to
Russia seeking to buy a launch on a converted Russian ballistic missile.
Unable to close a deal with the Russians, Mr. Musk turned to Jim
Cantrell, an aerospace consultant accompanying him on the trip, and
Michael D. Griffin, who would later become NASA administrator, and said,
``Guys, I think we can build these rockets ourselves.''

He founded SpaceX in 2002 and soon vaulted into the popular imagination,
a strong-willed figure who clashed with regulators and can sound both
humble and full of ambition verging on hubris.

\includegraphics{https://static01.nyt.com/images/2020/05/30/science/30nasa-prep/merlin_172993566_8d9e6edc-b361-492f-afff-3b84955eb304-articleLarge.jpg?quality=75\&auto=webp\&disable=upscale}

The first three launches of his Falcon 1 rocket failed. With one more
failure, SpaceX would have joined the graveyard of failed rocket
start-ups. The fourth Falcon 1 made it to orbit.

``The fourth launch worked, or that would have been it for SpaceX,'' Mr.
Musk said in a presentation at the International Astronautical Congress
in 2017. ``But fate liked us that day.''

During this time, NASA was going through upheaval after the loss of the
space shuttle Columbia in 2003, which disintegrated during re-entry into
the atmosphere, killing the seven astronauts onboard.

President George W. Bush decided the remaining three shuttles would be
retired and then the money devoted toward the shuttles could be used for
returning to the moon.

The rockets for that moon program, Constellation, would be built and
operated by NASA, just as in the past.

But Dr. Griffin, who had accompanied Mr. Musk on that trip to Moscow,
opened a competition for companies to bid on contracts to carry cargo to
the space station.

SpaceX was one of the two companies that NASA selected, and investments
from the space agency helped Mr. Musk's company develop the Falcon 9
rocket and the Dragon capsule, which evolved into the Crew Dragon
spacecraft that is now en route to the space station.

During a flight in 2015, SpaceX pulled off a feat that seemed like
something out of science fiction. Usually, the first stage of a rocket,
also known as a booster, falls into the ocean, becoming trash after just
one use. On this flight, the Falcon 9 booster, after it dropped away,
turned around and returned, landing vertically at Cape Canaveral. SpaceX
now routinely lands and reuses its boosters --- including the one used
on Saturday --- passing the cost savings onto its customers.

The Obama administration canceled the Constellation program. But it
needed a replacement for taking astronauts to the space station.

Taking inspiration from the successful cargo program, NASA sought
proposals from companies to start transporting people as well. In 2014,
the space agency selected Boeing and SpaceX.

A goal of the program was to free up money in NASA's budget to devote to
more ambitious projects like sending astronauts back to the moon and
eventually to Mars.

It will certainly save taxpayers the expense of paying Russia to take
NASA astronauts to space. This month, the space agency agreed to buy one
more seat on a Russian Soyuz rocket for \$90 million.

The developmental costs to NASA of commercial crew programs totaled
about \$6 billion, Philip McAlister, NASA's director of commercial
spaceflight development reported this month. That's a considerably
cheaper way to get to the space station than NASA's previous plan of
using an Orion capsule designed for deep-space missions and the Ares I
rocket that was being developed as part of Constellation. Mr. McAllister
said cost estimates for getting that system to the launchpad were \$24.5
billion to \$34.5 billion.

But while NASA may spend less on trips to the space station, it is still
developing a larger rocket for human spaceflight at Congress's direction
called the Space Launch System and has continued work on Orion. Reports
by NASA's inspector general find that the program is now billions of
dollars over budget and years behind schedule.

Still, by supporting SpaceX's and Boeing's spacecraft, NASA may help the
companies incubate other kinds of business in space.

Two companies, Axiom Space and Space Adventures, have already announced
plans to buy launches in SpaceX's capsule to transport non-NASA
passengers --- space tourists --- to the space station or on shorter
orbital trips. NASA also confirmed this month that
\href{https://www.nytimes.com/2020/05/05/science/nasa-tom-cruise-space-station.html}{Tom
Cruise expressed interest} in using the space station for a film.

NASA's next aim is to try this commercial approach on the moon, where
its future seems less certain.

The agency signed three companies, including Space X, last month to
produce three landers that could be used to take astronauts to the
surface of the moon.

Early prototypes of SpaceX's proposed lander,
\href{https://www.nytimes.com/2019/09/29/science/elon-musk-spacex-starship.html}{a
giant vessel named Starship}, have so far failed spectacularly; its
latest model was
\href{https://www.valleycentral.com/news/explosion-at-spacex-south-texas-facility-caught-on-camera/}{destroyed
in an explosion on Friday} after the end of an engine test firing.

\href{https://www.nytimes.com/2020/04/30/science/nasa-moon-lander.html}{Another
recipient of this lunar lander program's funding is Blue Origin}, the
rocket company started by Jeffrey P. Bezos, the founder of Amazon. Mr.
Bezos's wealth ensures that he can make long-term investments,
regardless of the whims of space policy in the United States, in his
pursuit of what he describes as a future with millions of people living
and working in space.

NASA officials noted that the lunar lander proposals were optimistic
about how quickly the spacecraft could be developed; the Trump
administration has committed to a 2024 mission. Yet, the designs rely on
rockets that have yet to launch.

NASA, spurred by the National Space Council, of which Mr. Pence is
chairman, is also taking steps it hopes will help commercial space
stations replace the International Space Station near the end of the
decade. Axiom, a company led by a former NASA manager of the space
station, is building a commercial module to be attached to I.S.S. and
when I.S.S. is retired, that module will be detached to form a piece of
an independent station owned and operated by Axiom.

Commercial ventures have already used the International Space Station to
conduct pharmaceutical research, manufacture optical fibers and launch
satellites. New space stations could make it easier for other ventures
to take advantage of near gravity-free environments.

``Let's have a robotic factory so the gravity disturbances are nil and
you can produce high-quality films and process products and
pharmaceutical drugs,'' said Jeffrey Manber, chief executive of
NanoRacks, a company that flies experiments to the space station. ``I'm
not dependent on the government to send my tourists, my honeymooners or
my scientists.''

However uncertain and far-off that future feels, SpaceX and NASA
appeared energized by the prospects of the successful launch on
Saturday.

Just after noon, the astronauts were seen off by their families ahead of
their drive to the launchpad. Mr. Behnken asked his son, Theodore, ``Are
you going to listen to Mommy and make her life easy?'' referring to
Megan McArthur, his wife and a fellow astronaut. Theodore, 6, replied,
``Let's light this candle!'' (Mr. Hurley, too, is married to a fellow
astronaut, Karen Nyberg.)

Within the hour, the men had boarded the Crew Dragon capsule and began
the hours of prelaunch procedures. Once they were sealed in the capsule,
Mr. Behnken and Mr. Hurley, and all watching them, waited patiently for
about two hours to see if the weather would cooperate with the day's
trip.

And their patience paid off. The weather cleared as the rocket was
loaded with propellant. And after the countdown and liftoff, a voice
from SpaceX's launch team bid the astronauts on a safe mission: ``Go
NASA. Go SpaceX. Godspeed, Bob and Doug!''

\emph{Peter Baker and Mariel Padilla contributed reporting.}

Advertisement

\protect\hyperlink{after-bottom}{Continue reading the main story}

\hypertarget{site-index}{%
\subsection{Site Index}\label{site-index}}

\hypertarget{site-information-navigation}{%
\subsection{Site Information
Navigation}\label{site-information-navigation}}

\begin{itemize}
\tightlist
\item
  \href{https://help.nytimes.com/hc/en-us/articles/115014792127-Copyright-notice}{©~2020~The
  New York Times Company}
\end{itemize}

\begin{itemize}
\tightlist
\item
  \href{https://www.nytco.com/}{NYTCo}
\item
  \href{https://help.nytimes.com/hc/en-us/articles/115015385887-Contact-Us}{Contact
  Us}
\item
  \href{https://www.nytco.com/careers/}{Work with us}
\item
  \href{https://nytmediakit.com/}{Advertise}
\item
  \href{http://www.tbrandstudio.com/}{T Brand Studio}
\item
  \href{https://www.nytimes.com/privacy/cookie-policy\#how-do-i-manage-trackers}{Your
  Ad Choices}
\item
  \href{https://www.nytimes.com/privacy}{Privacy}
\item
  \href{https://help.nytimes.com/hc/en-us/articles/115014893428-Terms-of-service}{Terms
  of Service}
\item
  \href{https://help.nytimes.com/hc/en-us/articles/115014893968-Terms-of-sale}{Terms
  of Sale}
\item
  \href{https://spiderbites.nytimes.com}{Site Map}
\item
  \href{https://help.nytimes.com/hc/en-us}{Help}
\item
  \href{https://www.nytimes.com/subscription?campaignId=37WXW}{Subscriptions}
\end{itemize}
