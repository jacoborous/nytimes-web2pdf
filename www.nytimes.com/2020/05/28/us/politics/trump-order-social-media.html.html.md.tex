Sections

SEARCH

\protect\hyperlink{site-content}{Skip to
content}\protect\hyperlink{site-index}{Skip to site index}

\href{https://www.nytimes.com/section/politics}{Politics}

\href{https://myaccount.nytimes.com/auth/login?response_type=cookie\&client_id=vi}{}

\href{https://www.nytimes.com/section/todayspaper}{Today's Paper}

\href{/section/politics}{Politics}\textbar{}Trump Signs Executive Order
on Social Media, Claiming to Protect `Free Speech'

\url{https://nyti.ms/2XFsov9}

\begin{itemize}
\item
\item
\item
\item
\item
\end{itemize}

Advertisement

\protect\hyperlink{after-top}{Continue reading the main story}

Supported by

\protect\hyperlink{after-sponsor}{Continue reading the main story}

\hypertarget{trump-signs-executive-order-on-social-media-claiming-to-protect-free-speech}{%
\section{Trump Signs Executive Order on Social Media, Claiming to
Protect `Free
Speech'}\label{trump-signs-executive-order-on-social-media-claiming-to-protect-free-speech}}

The president and his allies have often accused Twitter and Facebook of
bias against conservatives, and had resisted taking action until this
week, when Twitter fact-checked his own false statements.

\includegraphics{https://static01.nyt.com/images/2020/05/28/us/politics/28dc-trumporder-1/merlin_172931088_a848151e-7b2d-400b-bbe4-17b168bb992c-articleLarge.jpg?quality=75\&auto=webp\&disable=upscale}

By \href{https://www.nytimes.com/by/maggie-haberman}{Maggie Haberman}
and \href{https://www.nytimes.com/by/kate-conger}{Kate Conger}

\begin{itemize}
\item
  Published May 28, 2020Updated June 2, 2020
\item
  \begin{itemize}
  \item
  \item
  \item
  \item
  \item
  \end{itemize}
\end{itemize}

Denouncing what he said was the power of social media ``to shape the
interpretation of public events,'' President Trump signed an executive
order on Thursday directing federal regulators to crack down on
companies like Twitter and to consider taking away the legal protections
that shield them from liability for what gets posted on their platforms.

Mr. Trump and his allies have often accused Twitter and Facebook of bias
against conservative voices, and the president has been urged for years
to take a harder line against them. He had resisted until this week,
when
\href{https://www.nytimes.com/2020/05/26/technology/twitter-trump-mail-in-ballots.html}{Twitter
fact-checked his own false statements} in two posts.

That move by Twitter prompted an outcry from conservatives, who said
that the platform should not be able to selectively choose whose
statements it was fact-checking. But while the order sought to impose
new regulatory pressure on social media companies, legal experts said it
would be difficult to enforce.

``We're here today to defend free speech from one of the greatest
dangers it has faced in American history,'' Mr. Trump told reporters in
signing the order in the Oval Office, with the attorney general, William
P. Barr, standing nearby.

``They've had unchecked power to censure, restrict, edit, shape, hide,
alter virtually any form of communication between private citizens or
large public audiences,'' Mr. Trump said, adding that there was ``no
precedent'' for it. ``We cannot allow that to happen, especially when
they go about doing what they're doing.''

Twitter, the president said, was making ``editorial decisions.''

``In these moments, Twitter ceases to be a neutral public platform ---
they become an editor with a viewpoint,'' he said, saying that Facebook
and Google are included in his critiques.

Brandon Borrman, a spokesman for
\href{https://twitter.com/Policy/status/1266170586197262337?s=20}{Twitter,
responded} on Thursday night to the president's executive order: ``This
EO is a reactionary and politicized approach to a landmark law.
\#Section230 protects American innovation and freedom of expression, and
it's underpinned by democratic values. Attempts to unilaterally erode it
threaten the future of online speech and Internet freedoms.''

With its order, the administration sought to curtail the protections
currently given to technology companies under
\href{https://www.nytimes.com/2020/05/28/business/section-230-internet-speech.html}{Section
230 of the Communications Decency Act}, which limits the liability that
companies face for content posted by their users.

The law has enabled technology companies to flourish, allowing them to
mostly set their own rules for their platforms and to collect a vast
amount of free content from users against which to sell ads. The
executive order is aimed at removing that shield, Mr. Trump said.

The companies, along with many free speech advocates, have maintained
that amending Section 230 would cripple online discussion and bury
platforms under endless legal bills.

``We have clear content policies and we enforce them without regard to
political viewpoint,'' said Riva Sciuto, a Google spokeswoman. ``Our
platforms have empowered a wide range of people and organizations from
across the political spectrum, giving them a voice and new ways to reach
their audiences. Undermining Section 230 in this way would hurt
America's economy and its global leadership on internet freedom.''

Liz Bourgeois, a Facebook spokeswoman, said that ``by exposing companies
to potential liability for everything that billions of people around the
world say, this would penalize companies that choose to allow
controversial speech and encourage platforms to censor anything that
might offend anyone.''

Conservative pundits have said the companies remove their posts more
frequently than their liberal counterparts or ban them from social media
services altogether.

But if protection from liability was ended, the order could
\href{https://www.nytimes.com/2020/05/28/us/politics/trump-social-media-executive-order.html}{end
up backfiring} on Mr. Trump, who has used Twitter to lob insults at
rivals and to interact freely with his supporters. Without the liability
shield that Section 230 provides, social media platforms could be forced
to remove posts that are considered false or defamatory --- and the
president often pushes the boundaries with his commentary.

Moments after saying free speech was under attack from tech companies,
Mr. Trump suggested he would shut down Twitter if it were legally
possible, although he acknowledged there would be substantial obstacles.
But he suggested he was planning legislation dealing with social media
platforms.

Administration officials initially said the executive order would be
released on Wednesday after the president said he would make an
aggressive move related to social media companies. But with officials
scrambling to fill in the details, the order was not released until
after Mr. Trump answered questions from reporters on Thursday afternoon.

Legal experts said that the enforcement actions suggested by the
president were largely toothless and unlikely to withstand legal
challenges.

``Regardless of the circumstances that led up to this, this is not how
public policy is made in the United States. An executive order cannot be
properly used to change federal law,'' the U.S. Chamber of Commerce said
in an unusually pointed statement, one that echoed the concerns
conservatives once voiced about President Barack Obama's use of
executive orders.

Mr. Trump's order proposes three ways to crack down on the companies:
requiring the federal government to review its spending on social media
advertisements, giving the Federal Communications Commission the
authority to make new rules applying to social media platforms, and
asking the Federal Trade Commission to investigate whether social media
companies have misled users about the kinds of content they can post
online.

It also called on states to pursue their own enforcement actions and
directed the attorney general to draft a proposal for legislation.

Of the three tactics laid out in the order, the review of federal
spending is the most feasible, legal experts said.

``The government does have broad authority to promote legitimate public
policy goals through its spending power, and it does this a lot,'' said
Harold Feld, a senior vice president at Public Knowledge, a public
policy nonprofit.

But withholding advertising dollars may not sway the social media
companies' behavior. Last year, Twitter swore off political advertising
altogether, and many social media companies rely heavily on major brands
for their advertising dollars.

Daisuke Wakabayashi and Sheera Frenkel contributed reporting from
Oakland, Calif.

Advertisement

\protect\hyperlink{after-bottom}{Continue reading the main story}

\hypertarget{site-index}{%
\subsection{Site Index}\label{site-index}}

\hypertarget{site-information-navigation}{%
\subsection{Site Information
Navigation}\label{site-information-navigation}}

\begin{itemize}
\tightlist
\item
  \href{https://help.nytimes.com/hc/en-us/articles/115014792127-Copyright-notice}{©~2020~The
  New York Times Company}
\end{itemize}

\begin{itemize}
\tightlist
\item
  \href{https://www.nytco.com/}{NYTCo}
\item
  \href{https://help.nytimes.com/hc/en-us/articles/115015385887-Contact-Us}{Contact
  Us}
\item
  \href{https://www.nytco.com/careers/}{Work with us}
\item
  \href{https://nytmediakit.com/}{Advertise}
\item
  \href{http://www.tbrandstudio.com/}{T Brand Studio}
\item
  \href{https://www.nytimes.com/privacy/cookie-policy\#how-do-i-manage-trackers}{Your
  Ad Choices}
\item
  \href{https://www.nytimes.com/privacy}{Privacy}
\item
  \href{https://help.nytimes.com/hc/en-us/articles/115014893428-Terms-of-service}{Terms
  of Service}
\item
  \href{https://help.nytimes.com/hc/en-us/articles/115014893968-Terms-of-sale}{Terms
  of Sale}
\item
  \href{https://spiderbites.nytimes.com}{Site Map}
\item
  \href{https://help.nytimes.com/hc/en-us}{Help}
\item
  \href{https://www.nytimes.com/subscription?campaignId=37WXW}{Subscriptions}
\end{itemize}
