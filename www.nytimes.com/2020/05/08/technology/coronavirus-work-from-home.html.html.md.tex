Sections

SEARCH

\protect\hyperlink{site-content}{Skip to
content}\protect\hyperlink{site-index}{Skip to site index}

\href{https://www.nytimes.com/section/technology}{Technology}

\href{https://myaccount.nytimes.com/auth/login?response_type=cookie\&client_id=vi}{}

\href{https://www.nytimes.com/section/todayspaper}{Today's Paper}

\href{/section/technology}{Technology}\textbar{}White-Collar Companies
Race to Be Last to Return to the Office

\url{https://nyti.ms/2SJFIwU}

\begin{itemize}
\item
\item
\item
\item
\item
\end{itemize}

\href{https://www.nytimes.com/spotlight/at-home?action=click\&pgtype=Article\&state=default\&region=TOP_BANNER\&context=at_home_menu}{At
Home}

\begin{itemize}
\tightlist
\item
  \href{https://www.nytimes.com/2020/07/28/books/time-for-a-literary-road-trip.html?action=click\&pgtype=Article\&state=default\&region=TOP_BANNER\&context=at_home_menu}{Take:
  A Literary Road Trip}
\item
  \href{https://www.nytimes.com/2020/07/29/magazine/bored-with-your-home-cooking-some-smoky-eggplant-will-fix-that.html?action=click\&pgtype=Article\&state=default\&region=TOP_BANNER\&context=at_home_menu}{Cook:
  Smoky Eggplant}
\item
  \href{https://www.nytimes.com/2020/07/27/travel/moose-michigan-isle-royale.html?action=click\&pgtype=Article\&state=default\&region=TOP_BANNER\&context=at_home_menu}{Look
  Out: For Moose}
\item
  \href{https://www.nytimes.com/interactive/2020/at-home/even-more-reporters-editors-diaries-lists-recommendations.html?action=click\&pgtype=Article\&state=default\&region=TOP_BANNER\&context=at_home_menu}{Explore:
  Reporters' Obsessions}
\end{itemize}

Advertisement

\protect\hyperlink{after-top}{Continue reading the main story}

Supported by

\protect\hyperlink{after-sponsor}{Continue reading the main story}

\hypertarget{white-collar-companies-race-to-be-last-to-return-to-the-office}{%
\section{White-Collar Companies Race to Be Last to Return to the
Office}\label{white-collar-companies-race-to-be-last-to-return-to-the-office}}

Google, Facebook, Amazon, Capital One and others are extending
work-from-home policies to September and sometimes far beyond.

\includegraphics{https://static01.nyt.com/images/2020/05/11/business/08wfh/07wfh-articleLarge.jpg?quality=75\&auto=webp\&disable=upscale}

\href{https://www.nytimes.com/by/david-streitfeld}{\includegraphics{https://static01.nyt.com/images/2019/03/01/multimedia/author-david-streitfeld/author-david-streitfeld-thumbLarge.png}}

By \href{https://www.nytimes.com/by/david-streitfeld}{David Streitfeld}

\begin{itemize}
\item
  May 8, 2020
\item
  \begin{itemize}
  \item
  \item
  \item
  \item
  \item
  \end{itemize}
\end{itemize}

Even as President Trump
\href{https://www.nytimes.com/2020/05/06/us/politics/trump-coronavirus-recovery.html}{has
said} ``we have to get our country open again,'' much of corporate
America is in no rush to return employees to their campuses and
skyscrapers. The companies are racing not to be the first back, but the
last.

An increasing number of them, which mostly have white-collar employees,
have recently extended work-from-home policies far beyond the
shelter-in-place timelines mandated by state and local authorities.

Google and Facebook employees were told Thursday that they could stay
home until next year. Capital One informed 40,000 workers that they will
be out through Labor Day and possibly longer. Amazon is saying October.
Nationwide Insurance is moving more aggressively than other firms,
shuttering five offices around the country and having its 4,000
employees telecommute permanently.

The moves reflect the reality that no one is sure how
\href{https://www.nytimes.com/news-event/coronavirus?action=click\&pgtype=Article\&state=default\&module=STYLN_coronahub\&variant=show\&region=header\&context=menu}{the
coronavirus pandemic} will evolve. While
\href{https://www.nytimes.com/2020/05/05/us/coronavirus-deaths-cases-united-states.html}{deaths
from the virus in hot zones} like New York City have come down, new
outbreaks have emerged elsewhere. Almost every day, there are at least
20,000 new cases in the U.S., bringing the country's total to more than
1.2 million.

But even after the coronavirus no longer requires it, working from home
is likely to retain a significant presence in corporate life. It will
affect the shape of cities and the commercial real-estate industry, and
change the culture at companies that for years have been building
elaborate temples for their workers.

For many companies, which started having employees work from home in
March, prolonging the policy is not just a safety measure. It is a
pragmatic approach that helps workers with young children plan for a
difficult summer, and gives management time to reconfigure open-office
plans into
\href{https://www.nytimes.com/2020/05/04/health/coronavirus-office-makeover.html}{something
safer}.

Some companies said there is another reason: Working from home is
working out well.

``Working from home is a great thing for the company and for the
employees, who don't want to get back in cars and commute for two hours.
That's lost productivity,'' said Joan Burke, the chief people officer of
DocuSign, a San Francisco tech company that enables electronic
agreements. ``I see it happening way more often in the future.''

DocuSign recently announced a September return but said it could easily
be later. California is in lockdown
\href{https://www.mercurynews.com/2020/04/27/bay-areas-shelter-in-place-to-last-through-may/}{until
May 31}, its governor, Gavin Newsom, has said.

It is no coincidence that tech companies are in the front ranks of the
stay-at-home movement. Their software promotes working at a distance.
Tattoo parlors, bars and hairdressing salons, all of which need
face-to-face interaction with customers, have no such luxury.

Before the coronavirus struck, 8 percent of all wage and salaried
employees worked from home at least one day a week,
\href{https://www.bls.gov/news.release/flex2.t03.htm}{according to the
Bureau of Labor Statistics}; about 2 percent worked from home full time.
In a matter of days, the pandemic pushed telecommuting from marginal to
mandatory in many parts of the country.

Now, even as states like Georgia and Illinois roll out phased
re-openings, companies see a future for remote work. Gartner, the
research firm and consultant, said its clients --- mostly large firms
that have little direct interaction with the public --- expected as many
as half their employees to work at home at least part time.

A broad shift could have major implications for traffic congestion,
office culture and corporate profits. Smaller firms could draw on a much
larger pool of potential workers who live beyond the radius of
headquarters. And for some, it would erase the boundary between work and
home.

There are risks to companies, too. Employee loyalty could become more
tenuous, making retention more difficult. Managing could also become
harder. But the bottom line exerts a powerful pull.

``There are real cost benefits to doing this, and companies are in a
period where cost matters a lot,'' said Brian Kropp, a Gartner vice
president. ``Even if employees who are working remotely are 5 percent
less productive, companies can save 20 percent on real estate and end up
with a higher return.''

\includegraphics{https://static01.nyt.com/images/2020/05/07/business/00virus-workfromhome2/merlin_80913889_45b85031-8d10-41f1-90d6-350b291b7ca3-articleLarge.jpg?quality=75\&auto=webp\&disable=upscale}

Few are embracing the remote future as avidly as Zillow, the online real
estate firm based in Seattle. It said on April 24 that its 5,000
employees could work at home until 2021.

Three months ago, Zillow had traditional views about the workplace.
About 2 percent of its employees worked remotely; another 4 percent
worked from home part of the time. Everyone else went in every day.

``I don't see those numbers ever going back to where they were,'' Dan
Spaulding, Zillow's chief people officer, said in an interview. ``Our
bias against working from home has been completely exploded.'' He said
employees have stayed engaged while at home and the company was ``not
seeing any discernible drop in productivity.''

When Rich Barton, Zillow's chief executive,
\href{https://twitter.com/Rich_Barton/status/1254187509459742720}{tweeted
his emphatic support for working from home} late last month, a critic
responded by quoting a post from the employment rating site Glassdoor
that ``the constant check-ins, daily reports and hours of meetings a day
make it impossible to get your job completed.''

Mr. Spaulding acknowledged that ``there are pieces that are negative
here. The Zoom calls are great on some days, not on other days, and
downright atrocious for some kinds of collaboration.''

The open-office plan favored by Zillow and many other companies, however
maligned, at least in theory encouraged a collaborative environment. Now
they all need to think about reconfiguring to lower the risk of
contagion.

``If we're going back to the 1980s office for health reasons'' --- where
everyone had an office with a door --- ``I don't know how many employees
are interested in that,'' Mr. Spaulding said.

The notion of
\href{https://www.citylab.com/life/2015/12/the-invention-of-telecommuting/418047/}{telecommuting
was invented by Jack Nilles,} a former NASA engineer, in 1973. It
originally was not about working from home, which was largely impossible
before the commercial internet was developed in the late 1990s. Instead,
people would go to convenient satellite offices to reduce commuting
time.

Progress was fitful. New York, Washington, Seattle and San Francisco
flourished while other cities lagged. The disparity kept growing.

``Companies tried regional hubs, but it turned out you don't want to be
in Phoenix when all the decisions are made in San Francisco,'' said
Nicholas Bloom, a Stanford economics professor and co-director of the
productivity, innovation and entrepreneurship program at the National
Bureau of Economic Research.

In a 2015 study of work-from-home productivity, Mr. Bloom concluded
that\href{https://www.gsb.stanford.edu/faculty-research/publications/does-working-home-work-evidence-chinese-experiment}{it
went up}, but he has mixed feelings about the current situation. While
Covid-19 may help banish the stigma, he said, he doubted that working
from home five days a week would grow much.

``It's hard to remain motivated or innovative sitting in your living
room,'' he said. ``That sounds more like being a gig worker.''

That may be the fate of Nationwide Insurance employees in Gainesville,
Fla.; Harleysville, Penn.; Raleigh, N.C.; Wausau, Wis., and Richmond,
Va., whose offices will be closed permanently by Nov. 1.

Nationwide has 28,000 employees, about 20 percent of whom were already
working remotely. The company said it was ``permanently transitioning to
a hybrid operating model.'' Executives declined to be interviewed.

Other financial firms, which face more telecommuting security issues
than other industries, are also beginning to push back return dates.
Capital One said Tuesday that any return to offices this fall would be
``slow'' and ``staggered.''

Image

Amazon, which has its headquarters in Seattle, said on April 30 that
employees are ``welcome'' to work from home until October.Credit...Grant
Hindsley for The New York Times

Amazon, which spent billions on its new Seattle urban campus, said on
April 30 that employees are ``welcome'' to work from home until October.
Facebook and Google made internal announcements Thursday that most
employees could telecommute until the end of the year, but also said
they would reopen offices this summer for employees who need to be
there. The companies declined to comment.

Slack, which makes messaging technology that allows teams to communicate
and work together, is seeing its business boom during the quarantine.
But
\href{https://www.nytimes.com/2019/06/02/technology/slack-stewart-butterfield.html}{the
San Francisco company} plans to take as much time as necessary to
determine any changes for its 2,000 employees.

``It's easier to manage a company that is 100 percent remote than one
where employees are 50 percent remote and 50 percent in the office,''
said Robby Kwok, Slack's senior vice president for people.

That's because completely virtual companies need to write everything
down for employees. Companies that combine the two approaches risk that
some employees are more informed than others.

And in a world where crowds are now dangerous, Slack can help workers
stay safe by keeping them at home. The earliest employees will return to
the office is September, Mr. Kwok said.

``We have this community obligation to be the last to go back,'' he
said.

Advertisement

\protect\hyperlink{after-bottom}{Continue reading the main story}

\hypertarget{site-index}{%
\subsection{Site Index}\label{site-index}}

\hypertarget{site-information-navigation}{%
\subsection{Site Information
Navigation}\label{site-information-navigation}}

\begin{itemize}
\tightlist
\item
  \href{https://help.nytimes.com/hc/en-us/articles/115014792127-Copyright-notice}{©~2020~The
  New York Times Company}
\end{itemize}

\begin{itemize}
\tightlist
\item
  \href{https://www.nytco.com/}{NYTCo}
\item
  \href{https://help.nytimes.com/hc/en-us/articles/115015385887-Contact-Us}{Contact
  Us}
\item
  \href{https://www.nytco.com/careers/}{Work with us}
\item
  \href{https://nytmediakit.com/}{Advertise}
\item
  \href{http://www.tbrandstudio.com/}{T Brand Studio}
\item
  \href{https://www.nytimes.com/privacy/cookie-policy\#how-do-i-manage-trackers}{Your
  Ad Choices}
\item
  \href{https://www.nytimes.com/privacy}{Privacy}
\item
  \href{https://help.nytimes.com/hc/en-us/articles/115014893428-Terms-of-service}{Terms
  of Service}
\item
  \href{https://help.nytimes.com/hc/en-us/articles/115014893968-Terms-of-sale}{Terms
  of Sale}
\item
  \href{https://spiderbites.nytimes.com}{Site Map}
\item
  \href{https://help.nytimes.com/hc/en-us}{Help}
\item
  \href{https://www.nytimes.com/subscription?campaignId=37WXW}{Subscriptions}
\end{itemize}
