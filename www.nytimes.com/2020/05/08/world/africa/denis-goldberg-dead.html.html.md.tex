Sections

SEARCH

\protect\hyperlink{site-content}{Skip to
content}\protect\hyperlink{site-index}{Skip to site index}

\href{https://www.nytimes.com/section/world/africa}{Africa}

\href{https://myaccount.nytimes.com/auth/login?response_type=cookie\&client_id=vi}{}

\href{https://www.nytimes.com/section/todayspaper}{Today's Paper}

\href{/section/world/africa}{Africa}\textbar{}Denis Goldberg, South
African Freedom Fighter, Is Dead at 87

\url{https://nyti.ms/2WhbG5G}

\begin{itemize}
\item
\item
\item
\item
\item
\end{itemize}

Advertisement

\protect\hyperlink{after-top}{Continue reading the main story}

Supported by

\protect\hyperlink{after-sponsor}{Continue reading the main story}

\hypertarget{denis-goldberg-south-african-freedom-fighter-is-dead-at-87}{%
\section{Denis Goldberg, South African Freedom Fighter, Is Dead at
87}\label{denis-goldberg-south-african-freedom-fighter-is-dead-at-87}}

He was the only white defendant to be convicted alongside Nelson Mandela
and others in 1964 for resisting apartheid. He spent 22 years in prison.

\includegraphics{https://static01.nyt.com/images/2020/05/09/obituaries/09Goldberg-obit2/07Goldberg2-articleLarge.jpg?quality=75\&auto=webp\&disable=upscale}

By \href{https://www.nytimes.com/by/alan-cowell}{Alan Cowell}

\begin{itemize}
\item
  May 8, 2020
\item
  \begin{itemize}
  \item
  \item
  \item
  \item
  \item
  \end{itemize}
\end{itemize}

Denis Goldberg, one of two surviving political activists convicted in
the so-called Rivonia Trial, which put Nelson Mandela and seven others
in prison for many years and proved a turning-point in South Africa's
long struggle against apartheid, died on April 29 in Cape Town. He was
87.

His family, in confirming the death, said he had been treated for lung
cancer.

Mr. Goldberg's career, first in the armed resistance movement and later
in the post-apartheid era, encapsulated much of his country's modern
history, from the racial nuances of the struggle against white minority
rule to the reluctant acknowledgment of --- and disillusion with --- the
corruption that became a byword in early 21st-century South Africa.

At the trial, which lasted from 1963 to 1964, many of those accused of
sabotage were expecting the death sentence. Indeed, in a celebrated
address from the dock, Mr. Mandela said his ideal of a democratic and
free South Africa was, ``if needs be, an ideal for which I am prepared
to die.''

When Judge Quartus de Wet pronounced life sentences on eight defendants,
Mr. Goldberg's mother, Annie Goldberg, who was in the public gallery,
did not hear what he said.

``Denis, what is it?'' she called out. ``What did the judge say?''

Mr. Goldberg replied: ``Life! Life is wonderful!''

In all, 11 people faced charges as the trial approached. Of those, the
state withdrew its accusations against one potential defendant,
\href{https://www.theguardian.com/law/2015/aug/26/sir-bob-hepple}{Robert
Hepple}, and he was released. Two others ---
\href{https://www.theguardian.com/news/2002/jun/26/guardianobituaries1}{Lionel
Bernstein}, who was known as Rusty, and
\href{https://www.sahistory.org.za/people/james-kantor}{James Kantor}
--- were acquitted. All three fled to London.

\includegraphics{https://static01.nyt.com/images/2020/05/09/obituaries/09Goldberg-obit6/merlin_172059192_8bc8761b-5bc5-4c1a-ad1e-c5d7023024e1-articleLarge.jpg?quality=75\&auto=webp\&disable=upscale}

Those convicted along with Mr. Mandela and Mr. Goldberg were
\href{https://www.nytimes.com/2003/05/06/world/walter-sisulu-mandela-mentor-and-comrade-dies-at-90.html}{Walter
Sisulu},
\href{https://www.nytimes.com/2001/08/31/world/govan-mbeki-91-an-enemy-of-apartheid-system-dies.html}{Govan
Mbeki},
\href{https://www.theguardian.com/news/2005/feb/25/guardianobituaries.southafrica}{Raymond
Mhlaba},
\href{https://www.sahistory.org.za/people/elias-mathope-motsoaledi}{Elias
Motsoaledi},
\href{https://www.sahistory.org.za/people/andrew-mokete-mlangeni}{Andrew
Mlangeni} and
\href{https://www.nytimes.com/2017/03/28/world/africa/ahmed-kathrada-dies-nelson-mandela.html}{Ahmed
Kathrada}. With Mr. Goldberg's death, the sole survivor of those
convicted is Mr. Mlangeni, now 94.

At 31 Mr. Goldberg was the youngest of those convicted and the only
white person among them.

The hearings came at a crucial juncture in South African history. The
authorities there had increasingly resorted to force in suppressing
opposition to apartheid, the white rulers' draconian system of racial
separation, and their adversaries had turned to armed struggle in
response. The trial was intended to crush and silence Mr. Mandela and
his followers.

But the prisoners turned the occasion into a global indictment of
apartheid, particularly with Mr. Mandela's speech from the dock.

``It was the most important trial in South Africa's history,'' Nick
Stadlen, a former High Court judge in Britain who made a documentary
film about the trial in 2017 that featured Mr. Goldberg and others,
\href{https://www.theguardian.com/commentisfree/2020/may/03/denis-goldberg-the-man-who-offered-to-sign-his-own-death-warrant-to-save-nelson-mandela}{wrote
in The Guardian} after Mr. Goldberg's death.

The origins of the trial date to July 1963, when the South African
security forces raided \href{http://www.liliesleaf.co.za/}{Liliesleaf
Farm}, a hideout in the Rivonia district in the northern suburbs of
Johannesburg. Members of
\href{https://www.sahistory.org.za/article/umkhonto-wesizwe-mk}{mKhonto
we Sizwe} (Spear of the Nation), the clandestine military wing of the
African National Congress --- both of them outlawed organizations ---
were meeting there when the police stormed in. At the time, Mr.
Goldberg, a member of the banned South African Communist Party, had been
a technical officer in the military unit, cloaking his sabotage
activities behind a day job in the construction of a power station in
Cape Town.

Image

Mr. Goldberg, center, and others who had been involved in the
anti-apartheid struggle spoke to the news media in 2001 at Liliesleaf
Farm, which had been a hideout for Mr. Mandela and other freedom
fighters in the Rivonia district north of Johannesburg. The others, from
left, were Lionel Bernstein, Andrew Mlangeni, Raymond Mhlaba and Arthur
Goldreich.Credit...Juda Ngwenya/Reuters

Many of the documents produced at the trial had been written by him.
Indeed, he offered to assume responsibility for all the charges so that
his co-defendants could be acquitted. But they rejected his offer.

Before the trial, Mr. Goldberg was interrogated and threatened in a
police effort to secure confessions or persuade their captives to
testify against their fellow detainees. Under harsh laws permitting
detention without trial for 90 days, Mr. Goldberg's wife, Esmé Goldberg,
was also held for many days.

Even Mr. Goldberg's sentencing did not escape the strictures of
apartheid. While Mr. Mandela and six defendants were sent to serve their
sentences on Robben Island, off Cape Town, Mr. Goldberg was ordered to
the Central Prison in Pretoria, the administrative capital.

In more recent times the facility has been known as the prison where the
Olympic and Paralympic sprinter Oscar Pistorius served part of a
sentence for killing his girlfriend, Reeva Steenkamp. It is also the
setting for a 2020 movie
\href{https://variety.com/2020/film/reviews/escape-from-pretoria-film-review-daniel-radcliffe-1203525520/}{``Escape
from Pretoria,''}starring Daniel Ratcliffe, which chronicles a real-life
breakout by three prisoners in 1979. Mr. Goldberg, who helped facilitate
the escape but did not participate in it, was played in the film by Ian
Hart, an English actor.

Image

The entrance to the former prison on Robben Island, off Cape Town, where
Mr. Mandela and other South African political prisoners were held for
many years. Mr. Goldberg was sent to the Central Prison in
Pretoria.Credit...Rodger Bosch/Agence France-Presse --- Getty Images

Mr. Goldberg remained in prison until 1985, serving 22 years, during
which he was allowed visits by his wife amounting to only a few hours in
the entire period of his incarceration. The few letters he was allowed
to send were intercepted or censored. He studied law while behind bars.

Mr. Mandela was freed in 1990 as part of the maneuvering that led to
South Africa's first democratic, all-race election in 1994.

Early in his sentence Mr. Goldberg tended to the terminally ill fellow
prisoner
\href{https://www.sahistory.org.za/people/abram-bram-fischer}{Abram
Fischer}, who was known as Bram, a lawyer of Afrikaner descent who had
led the Rivonia defendants' defense and who had himself been tried and
jailed in 1966 on charges of conspiring to overthrow the government and
furthering the aims of communism. He was found to have cancer in 1974.

In the final stages of Mr. Fischer's incarceration, the prison
authorities allowed Mr. Goldberg to spend nights in his cell to nurse
him, as Mr. Goldberg recounted in his 2016 memoir, ``A Life for
Freedom.'' Mr. Fischer was allowed to be placed under house arrest at
his brother's home only in 1975, a few weeks before his death.

Denis Theodore Goldberg was born in Cape Town on April 11, 1933, to
English-born Jewish parents, Sam and Annie (Fineberg) Goldberg. His
mother was a seamstress, his father a truck driver. The couple's
forebears were active Communists who had fled Lithuania to escape
Russian pogroms, and their son inherited their ideology. After Hitler
invaded Poland in 1939, he recalled, his teachers and fellow schoolmates
in South Africa taunted him for being Jewish.

Image

Mr. Goldberg in 2013 at Liliesleaf Farm. A raid there by the police in
1963 led to his arrest, conviction and 22-year
imprisonment.Credit...Christopher Furlong/Getty Images

Mr. Goldberg grew up in a blue-collar, mixed-race neighborhood of Cape
Town, but in his teens South Africa's National Party won elections in
1948 and began erecting the system known as apartheid.

In his early 20s he joined several left-wing and anti-apartheid
movements, including the South African Communist Party. During this
period he met Esmé Bodenstein, a fellow member of a multiracial group
called the Modern Youth Society. They married in 1954 and had two
children, Hilary and David. Ms. Goldberg, a physiotherapist,
\href{https://www.theguardian.com/news/2000/mar/08/guardianobituaries}{died
in 2000}, and their daughter died at 47 in 2002, the same year Mr.
Goldberg married Edelgard Nkobi. Ms. Nkobi died in 2006. He is survived
by his son.

At the University of Cape Town, Mr. Goldberg studied civil engineering
and graduated in 1955. He was first detained, along with his mother,
during the state of emergency following the
\href{https://www.sahistory.org.za/article/sharpeville-massacre-21-march-1960}{Sharpeville
massacre} of March 1960, when police opened fire on protesters and
killed 69 of them.

He joined the mKhonto we Sizwe military unit in 1961 with the job of
building weapons and explosives to sabotage electric power pylons and
other targets. At the time of his arrest in 1963 he had been expecting
to be spirited out of the country for training by the East Bloc
countries that supported Mr. Mandela's armed struggle.

Mr. Goldberg was granted release from prison in 1985 under an agreement
with the white authorities that he renounce violence. ``I reckoned I had
been in prison long enough,'' he wrote in his memoir. His daughter,
Hilary, who was living at the time in Israel, had campaigned for his
release.

As soon as he was out of prison he left South Africa for London, where
he worked for the African National Congress and raised funds for
charities. He returned to live in South Africa only in 2002 and worked
there as a ministerial aide. He became a fierce critic of corruption
among the political elite that had grown around President Jacob Zuma,
who resigned in 2018.

Mr. Goldberg devoted much of his final years to promoting an arts and
educational center called the Denis Goldberg House of Hope in the town
of Hout Bay near Cape Town. The center reflected his conviction,
\href{https://www.theguardian.com/global-development/2017/nov/12/denis-goldberg-south-africa-anc-nelson-mandela-jacob-zuma}{he
told The Guardian in 2017}, that ``people matter.''

``I feel the whole point of being in politics is about people,'' he
said. ``For me it's not about power.''

Advertisement

\protect\hyperlink{after-bottom}{Continue reading the main story}

\hypertarget{site-index}{%
\subsection{Site Index}\label{site-index}}

\hypertarget{site-information-navigation}{%
\subsection{Site Information
Navigation}\label{site-information-navigation}}

\begin{itemize}
\tightlist
\item
  \href{https://help.nytimes.com/hc/en-us/articles/115014792127-Copyright-notice}{©~2020~The
  New York Times Company}
\end{itemize}

\begin{itemize}
\tightlist
\item
  \href{https://www.nytco.com/}{NYTCo}
\item
  \href{https://help.nytimes.com/hc/en-us/articles/115015385887-Contact-Us}{Contact
  Us}
\item
  \href{https://www.nytco.com/careers/}{Work with us}
\item
  \href{https://nytmediakit.com/}{Advertise}
\item
  \href{http://www.tbrandstudio.com/}{T Brand Studio}
\item
  \href{https://www.nytimes.com/privacy/cookie-policy\#how-do-i-manage-trackers}{Your
  Ad Choices}
\item
  \href{https://www.nytimes.com/privacy}{Privacy}
\item
  \href{https://help.nytimes.com/hc/en-us/articles/115014893428-Terms-of-service}{Terms
  of Service}
\item
  \href{https://help.nytimes.com/hc/en-us/articles/115014893968-Terms-of-sale}{Terms
  of Sale}
\item
  \href{https://spiderbites.nytimes.com}{Site Map}
\item
  \href{https://help.nytimes.com/hc/en-us}{Help}
\item
  \href{https://www.nytimes.com/subscription?campaignId=37WXW}{Subscriptions}
\end{itemize}
