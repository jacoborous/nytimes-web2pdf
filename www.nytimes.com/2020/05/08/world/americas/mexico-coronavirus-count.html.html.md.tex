Sections

SEARCH

\protect\hyperlink{site-content}{Skip to
content}\protect\hyperlink{site-index}{Skip to site index}

\href{/section/world/americas}{Americas}\textbar{}Hidden Toll: Mexico
Ignores Wave of Coronavirus Deaths in Capital

\url{https://nyti.ms/3dv4okW}

\begin{itemize}
\item
\item
\item
\item
\item
\item
\end{itemize}

\href{https://www.nytimes.com/news-event/coronavirus?action=click\&pgtype=Article\&state=default\&region=TOP_BANNER\&context=storylines_menu}{The
Coronavirus Outbreak}

\begin{itemize}
\tightlist
\item
  live\href{https://www.nytimes.com/2020/08/01/world/coronavirus-covid-19.html?action=click\&pgtype=Article\&state=default\&region=TOP_BANNER\&context=storylines_menu}{Latest
  Updates}
\item
  \href{https://www.nytimes.com/interactive/2020/us/coronavirus-us-cases.html?action=click\&pgtype=Article\&state=default\&region=TOP_BANNER\&context=storylines_menu}{Maps
  and Cases}
\item
  \href{https://www.nytimes.com/interactive/2020/science/coronavirus-vaccine-tracker.html?action=click\&pgtype=Article\&state=default\&region=TOP_BANNER\&context=storylines_menu}{Vaccine
  Tracker}
\item
  \href{https://www.nytimes.com/interactive/2020/07/29/us/schools-reopening-coronavirus.html?action=click\&pgtype=Article\&state=default\&region=TOP_BANNER\&context=storylines_menu}{What
  School May Look Like}
\item
  \href{https://www.nytimes.com/live/2020/07/31/business/stock-market-today-coronavirus?action=click\&pgtype=Article\&state=default\&region=TOP_BANNER\&context=storylines_menu}{Economy}
\end{itemize}

\includegraphics{https://static01.nyt.com/images/2020/05/08/world/08mexico-virus-top/merlin_172318887_a2f538e6-e2ad-4286-b75f-b305a2df38bf-articleLarge.jpg?quality=75\&auto=webp\&disable=upscale}

\hypertarget{hidden-toll-mexico-ignores-wave-of-coronavirus-deaths-in-capital}{%
\section{Hidden Toll: Mexico Ignores Wave of Coronavirus Deaths in
Capital}\label{hidden-toll-mexico-ignores-wave-of-coronavirus-deaths-in-capital}}

More than three times as many people may have died from Covid-19 in
Mexico City than federal statistics show, according to a Times analysis.

Workers at a crematorium at Xilotepec Cemetery bringing the body of a
Covid-19 victim to be cremated on Friday in Mexico City.Credit...

Supported by

\protect\hyperlink{after-sponsor}{Continue reading the main story}

By \href{https://www.nytimes.com/by/azam-ahmed}{Azam Ahmed}

Photographs by Daniel Berehulak

\begin{itemize}
\item
  Published May 8, 2020Updated May 28, 2020
\item
  \begin{itemize}
  \item
  \item
  \item
  \item
  \item
  \item
  \end{itemize}
\end{itemize}

\href{https://www.nytimes.com/es/2020/05/08/espanol/america-latina/mexico-coronavirus.html}{Leer
en español}

MEXICO CITY --- The Mexican government is not reporting hundreds,
possibly thousands, of
\href{https://www.nytimes.com/2020/04/30/world/americas/coronavirus-mexico-factories.html}{deaths
from the coronavirus in Mexico City}, dismissing anxious officials who
have tallied more than three times as many fatalities in the capital
than the government publicly acknowledges, according to officials and
confidential data.

The tensions have come to a head in recent weeks, with Mexico City
alerting the government to the deaths repeatedly, hoping it will come
clean to the public about the
\href{https://www.nytimes.com/interactive/2020/world/coronavirus-maps.html}{true
toll of the virus} on the nation's biggest city and, by extension, the
country at large.

But that has not happened. Doctors in overwhelmed hospitals in Mexico
City say the reality of the epidemic is being hidden from the country.
In some hospitals, patients lie on the floor, splayed on mattresses.
Elderly people are propped up on metal chairs because there are not
enough beds, while patients are turned away to search for space in
\href{https://www.nytimes.com/2020/05/28/world/americas/virus-mexico-doctors.html}{less-prepared
hospitals}. Many die while searching, several doctors said.

``It's like we doctors are living in two different worlds,'' said Dr.
Giovanna Avila, who works at Hospital de Especialidades Belisario
Domínguez. ``One is inside of the hospital with patients dying all the
time. And the other is when we walk out onto the streets and see people
walking around, clueless of what is going on and how bad the situation
really is.''

Mexico City officials have tabulated more than 2,500 deaths from the
virus and from serious respiratory illnesses that doctors suspect were
related to Covid-19, according to the data, which was reviewed by The
New York Times. Yet the federal government is reporting about 700 in the
area, which includes Mexico City and the municipalities on its
outskirts.

Nationwide, the federal government has reported about 3,000 confirmed
deaths from the virus, plus nearly 250 suspected of being related, in a
country of more than 120 million people. But experts say Mexico has only
a minimal sense of the real scale of the epidemic because it is testing
so few people.

\includegraphics{https://static01.nyt.com/images/2020/05/08/world/08virus-mexico5-sub/merlin_172296483_5bae9db4-e6fb-4a8b-852a-17b660373e00-articleLarge.jpg?quality=75\&auto=webp\&disable=upscale}

Far fewer than one in 1,000 people in Mexico are tested for the virus
---
\href{https://www.weforum.org/agenda/2020/04/these-are-the-oecd-countries-testing-most-for-covid-19/}{by
far the lowest}of the dozens of nations in the Organization for Economic
Cooperation and Development, which average about 23 tests for every
1,000 people.

The government says Mexico has been faring better than many of the
world's largest countries, and on Monday its Covid-19 czar estimated
that the final death toll would be around 6,000 people.

``We have flattened the curve,'' Hugo López-Gatell, the health ministry
official who has become the face of the country's response, said this
week.

\hypertarget{latest-updates-global-coronavirus-outbreak}{%
\section{\texorpdfstring{\href{https://www.nytimes.com/2020/08/01/world/coronavirus-covid-19.html?action=click\&pgtype=Article\&state=default\&region=MAIN_CONTENT_1\&context=storylines_live_updates}{Latest
Updates: Global Coronavirus
Outbreak}}{Latest Updates: Global Coronavirus Outbreak}}\label{latest-updates-global-coronavirus-outbreak}}

Updated 2020-08-02T01:29:11.393Z

\begin{itemize}
\tightlist
\item
  \href{https://www.nytimes.com/2020/08/01/world/coronavirus-covid-19.html?action=click\&pgtype=Article\&state=default\&region=MAIN_CONTENT_1\&context=storylines_live_updates\#link-34047410}{The
  U.S. reels as July cases more than double the total of any other
  month.}
\item
  \href{https://www.nytimes.com/2020/08/01/world/coronavirus-covid-19.html?action=click\&pgtype=Article\&state=default\&region=MAIN_CONTENT_1\&context=storylines_live_updates\#link-3ac56579}{Top
  officials work to break impasse over jobless benefit.}
\item
  \href{https://www.nytimes.com/2020/08/01/world/coronavirus-covid-19.html?action=click\&pgtype=Article\&state=default\&region=MAIN_CONTENT_1\&context=storylines_live_updates\#link-25930521}{Thousands
  in Berlin protest Germany's coronavirus measures.}
\end{itemize}

\href{https://www.nytimes.com/2020/08/01/world/coronavirus-covid-19.html?action=click\&pgtype=Article\&state=default\&region=MAIN_CONTENT_1\&context=storylines_live_updates}{See
more updates}

More live coverage:
\href{https://www.nytimes.com/live/2020/07/31/business/stock-market-today-coronavirus?action=click\&pgtype=Article\&state=default\&region=MAIN_CONTENT_1\&context=storylines_live_updates}{Markets}

But the government did not respond to questions about the deaths in
Mexico City. It also denied repeated requests by The Times over the
course of three weeks to identify all deaths related to respiratory
illnesses since January, saying the data was incomplete.

One former health secretary, José Narro Robles, has accused Mr.
López-Gatell of lying to the people of Mexico. And some state
governments are beginning to draw similar conclusions: that, much like
Mexico City found, the data presented by the government does not reflect
reality.

Image

Workers sterilizing a traditional farm in the Xochimilco neighborhood of
Mexico City.

Official counts in many countries have understated the number of deaths
during the pandemic, especially where limited testing has prevented the
virus from being diagnosed,
\href{https://www.nytimes.com/interactive/2020/04/21/world/coronavirus-missing-deaths.html}{a
Times review of mortality data} has found. In Ecuador, six times more
people have died than official figures reflect, the data show. In Italy,
the overall increase in deaths in March was nearly twice official
counts.

In Mexico City, the doubts started a month ago, when the city's mayor,
Claudia Sheinbaum, began to suspect that federal data and modeling on
the epidemic were flawed, according to three people with knowledge of
the matter.

She had already instructed her staff to call every public hospital in
the Mexico City area to ask about all confirmed and suspected Covid-19
deaths, the people said. In the last week, that effort found that the
deaths were more than three times what the federal government reported.

The disagreements have taken place largely behind the scenes, as Ms.
Sheinbaum, who declined to comment for this article, has been loath to
publicly embarrass President Andrés Manuel López Obrador, her close
political ally. The city and the federal government continue to work
together on a number of fronts, including getting ventilators.

But the data from Mexico City calls into question the federal
government's grasp of the crisis in the country.

With such limited testing and doubts about the government's models,
experts say federal estimates for when the nation will reach its peak,
how long the epidemic will last and how bad the damage will be may not
be reliable.

That disconnect has left cities and states across the country scrambling
to meet the demand for protective equipment and ventilators. It also
underplays the severity of the epidemic for millions of Mexicans, making
it hard for them to determine how bad the situation is --- and how
seriously to take it.

Image

Coffins of Covid-19 victims were stacked behind the crematorium at a
cemetery in Xochimilco on Thursday.

``That is shocking,'' said Fernando Alarid-Escudero, who has a Ph.D. in
health decision sciences and who developed an independent model in
collaboration with scientists at Stanford University to chart the curve
of the epidemic in Mexico. ``If that is case, and we are not really
capturing all those people who eventually die, we are not getting a
sense of the picture.''

``We are way underestimating the magnitude of the epidemic,'' he added.

In Tijuana, hospitals are already overwhelmed. Doctors and nurses across
the country have held public protests against the lack of protective
gear, and several hospitals along the border have suffered outbreaks of
the virus among medical personnel. Federal officials have been
scrambling to buy respirators, long after seeing the outbreaks grip
China, Europe and the United States.

One big reason for the competing death tolls in Mexico has to do with
the way the federal government is testing, vetting and reporting the
data. The official results include a two-week lag, people familiar with
the process say, which means timely information is not available
publicly.

More worrisome, they say, are the many deaths absent from the data
altogether, as suggested by the figures from Mexico City, where the
virus has struck hardest of all. Some people die from acute respiratory
illness and are cremated without ever getting tested, officials say.
Others are dying at home without being admitted to a hospital --- and
are not even counted under Mexico City's statistics.

Image

An ambulance team transporting a suspected coronavirus patient at the
General Hospital of Mexico.

Beyond that, Mexico appears to be vastly underreporting suspected deaths
from coronavirus. Data published by the federal government on May 7 show
only 245 suspicious deaths nationwide.

\href{https://www.nytimes.com/news-event/coronavirus?action=click\&pgtype=Article\&state=default\&region=MAIN_CONTENT_3\&context=storylines_faq}{}

\hypertarget{the-coronavirus-outbreak-}{%
\subsubsection{The Coronavirus Outbreak
›}\label{the-coronavirus-outbreak-}}

\hypertarget{frequently-asked-questions}{%
\paragraph{Frequently Asked
Questions}\label{frequently-asked-questions}}

Updated July 27, 2020

\begin{itemize}
\item ~
  \hypertarget{should-i-refinance-my-mortgage}{%
  \paragraph{Should I refinance my
  mortgage?}\label{should-i-refinance-my-mortgage}}

  \begin{itemize}
  \tightlist
  \item
    \href{https://www.nytimes.com/article/coronavirus-money-unemployment.html?action=click\&pgtype=Article\&state=default\&region=MAIN_CONTENT_3\&context=storylines_faq}{It
    could be a good idea,} because mortgage rates have
    \href{https://www.nytimes.com/2020/07/16/business/mortgage-rates-below-3-percent.html?action=click\&pgtype=Article\&state=default\&region=MAIN_CONTENT_3\&context=storylines_faq}{never
    been lower.} Refinancing requests have pushed mortgage applications
    to some of the highest levels since 2008, so be prepared to get in
    line. But defaults are also up, so if you're thinking about buying a
    home, be aware that some lenders have tightened their standards.
  \end{itemize}
\item ~
  \hypertarget{what-is-school-going-to-look-like-in-september}{%
  \paragraph{What is school going to look like in
  September?}\label{what-is-school-going-to-look-like-in-september}}

  \begin{itemize}
  \tightlist
  \item
    It is unlikely that many schools will return to a normal schedule
    this fall, requiring the grind of
    \href{https://www.nytimes.com/2020/06/05/us/coronavirus-education-lost-learning.html?action=click\&pgtype=Article\&state=default\&region=MAIN_CONTENT_3\&context=storylines_faq}{online
    learning},
    \href{https://www.nytimes.com/2020/05/29/us/coronavirus-child-care-centers.html?action=click\&pgtype=Article\&state=default\&region=MAIN_CONTENT_3\&context=storylines_faq}{makeshift
    child care} and
    \href{https://www.nytimes.com/2020/06/03/business/economy/coronavirus-working-women.html?action=click\&pgtype=Article\&state=default\&region=MAIN_CONTENT_3\&context=storylines_faq}{stunted
    workdays} to continue. California's two largest public school
    districts --- Los Angeles and San Diego --- said on July 13, that
    \href{https://www.nytimes.com/2020/07/13/us/lausd-san-diego-school-reopening.html?action=click\&pgtype=Article\&state=default\&region=MAIN_CONTENT_3\&context=storylines_faq}{instruction
    will be remote-only in the fall}, citing concerns that surging
    coronavirus infections in their areas pose too dire a risk for
    students and teachers. Together, the two districts enroll some
    825,000 students. They are the largest in the country so far to
    abandon plans for even a partial physical return to classrooms when
    they reopen in August. For other districts, the solution won't be an
    all-or-nothing approach.
    \href{https://bioethics.jhu.edu/research-and-outreach/projects/eschool-initiative/school-policy-tracker/}{Many
    systems}, including the nation's largest, New York City, are
    devising
    \href{https://www.nytimes.com/2020/06/26/us/coronavirus-schools-reopen-fall.html?action=click\&pgtype=Article\&state=default\&region=MAIN_CONTENT_3\&context=storylines_faq}{hybrid
    plans} that involve spending some days in classrooms and other days
    online. There's no national policy on this yet, so check with your
    municipal school system regularly to see what is happening in your
    community.
  \end{itemize}
\item ~
  \hypertarget{is-the-coronavirus-airborne}{%
  \paragraph{Is the coronavirus
  airborne?}\label{is-the-coronavirus-airborne}}

  \begin{itemize}
  \tightlist
  \item
    The coronavirus
    \href{https://www.nytimes.com/2020/07/04/health/239-experts-with-one-big-claim-the-coronavirus-is-airborne.html?action=click\&pgtype=Article\&state=default\&region=MAIN_CONTENT_3\&context=storylines_faq}{can
    stay aloft for hours in tiny droplets in stagnant air}, infecting
    people as they inhale, mounting scientific evidence suggests. This
    risk is highest in crowded indoor spaces with poor ventilation, and
    may help explain super-spreading events reported in meatpacking
    plants, churches and restaurants.
    \href{https://www.nytimes.com/2020/07/06/health/coronavirus-airborne-aerosols.html?action=click\&pgtype=Article\&state=default\&region=MAIN_CONTENT_3\&context=storylines_faq}{It's
    unclear how often the virus is spread} via these tiny droplets, or
    aerosols, compared with larger droplets that are expelled when a
    sick person coughs or sneezes, or transmitted through contact with
    contaminated surfaces, said Linsey Marr, an aerosol expert at
    Virginia Tech. Aerosols are released even when a person without
    symptoms exhales, talks or sings, according to Dr. Marr and more
    than 200 other experts, who
    \href{https://academic.oup.com/cid/article/doi/10.1093/cid/ciaa939/5867798}{have
    outlined the evidence in an open letter to the World Health
    Organization}.
  \end{itemize}
\item ~
  \hypertarget{what-are-the-symptoms-of-coronavirus}{%
  \paragraph{What are the symptoms of
  coronavirus?}\label{what-are-the-symptoms-of-coronavirus}}

  \begin{itemize}
  \tightlist
  \item
    Common symptoms
    \href{https://www.nytimes.com/article/symptoms-coronavirus.html?action=click\&pgtype=Article\&state=default\&region=MAIN_CONTENT_3\&context=storylines_faq}{include
    fever, a dry cough, fatigue and difficulty breathing or shortness of
    breath.} Some of these symptoms overlap with those of the flu,
    making detection difficult, but runny noses and stuffy sinuses are
    less common.
    \href{https://www.nytimes.com/2020/04/27/health/coronavirus-symptoms-cdc.html?action=click\&pgtype=Article\&state=default\&region=MAIN_CONTENT_3\&context=storylines_faq}{The
    C.D.C. has also} added chills, muscle pain, sore throat, headache
    and a new loss of the sense of taste or smell as symptoms to look
    out for. Most people fall ill five to seven days after exposure, but
    symptoms may appear in as few as two days or as many as 14 days.
  \end{itemize}
\item ~
  \hypertarget{does-asymptomatic-transmission-of-covid-19-happen}{%
  \paragraph{Does asymptomatic transmission of Covid-19
  happen?}\label{does-asymptomatic-transmission-of-covid-19-happen}}

  \begin{itemize}
  \tightlist
  \item
    So far, the evidence seems to show it does. A widely cited
    \href{https://www.nature.com/articles/s41591-020-0869-5}{paper}
    published in April suggests that people are most infectious about
    two days before the onset of coronavirus symptoms and estimated that
    44 percent of new infections were a result of transmission from
    people who were not yet showing symptoms. Recently, a top expert at
    the World Health Organization stated that transmission of the
    coronavirus by people who did not have symptoms was ``very rare,''
    \href{https://www.nytimes.com/2020/06/09/world/coronavirus-updates.html?action=click\&pgtype=Article\&state=default\&region=MAIN_CONTENT_3\&context=storylines_faq\#link-1f302e21}{but
    she later walked back that statement.}
  \end{itemize}
\end{itemize}

The gap in information has left many Mexicans with a sense that their
country has avoided the harrowing outbreaks afflicting nations like the
United States, where nearly 1.2 million people have been infected and
more than 70,000 people have died, according to the Centers for Disease
Control.

Publicly, Mr. López-Gatell, the health ministry official, has become
something of a celebrity, steering nightly news conferences in which he
assures the public that things are moving according to plan.

But there have been problems with the government's assumptions from the
very beginning, according to three people familiar with its
preparations. As early as February, they said, the government was using
Wuhan, China --- the city where the pandemic originated --- to model the
potential needs and response in Mexico.

But those calculations quickly went awry, the people said, as officials
realized the dynamic in China was entirely different from the one in
Mexico. As the outbreak spread in Wuhan, Chinese officials locked down
the city and the surrounding province, prohibiting tens of millions of
people from traveling.

In Mexico, by contrast, the lockdown measures have been optional, with
officials simply urging people to go to hospitals or stay at home,
depending on symptoms. There are no travel restrictions in or out of
Mexico City.

Image

Commuters at a metro station in Mexico City last month.

In the last month, the government has added experts to review the data
and analysis, after urging from the country's foreign minister, Marcelo
Ebrard, and other officials. But even those newer models make
assumptions that experts feel are inadequate.

The main model the country is believed to now be using assumes only 5
percent of the infected population show symptoms, and that only 5
percent of those patients will go to the hospital, according to modeling
documents obtained by The Times.

``Their model is wrong,'' said Laurie Ann Ximénez-Fyvie, a
Harvard-trained Ph.D. at the National Autonomous University of Mexico,
adding that symptomatic and severe cases could be significantly higher.
``There is very good consensus on that.''

Several experts also questioned Mexico's assumptions of how quickly the
epidemic will pass. Its model shows a sharp rise in infections, followed
by a sharp decline. But in almost no other country in the world has
there been a rapid decline after a peak.

``There is a long tail for the curve, and the number of deaths does not
drop to zero anytime in the near future,'' said Nilanjan Chatterjee, a
professor in the department of biostatistics at the Bloomberg School of
Public Health at Johns Hopkins University. ``The graph they are using is
inconsistent with the shapes of the curve in other countries.''

Image

Sterilizing the chapel at the Xilotepec Cemetery in the Xochimilco
neighborhood of Mexico City on Thursday.

Paulina Villegas contributed reporting.

Advertisement

\protect\hyperlink{after-bottom}{Continue reading the main story}

\hypertarget{site-index}{%
\subsection{Site Index}\label{site-index}}

\hypertarget{site-information-navigation}{%
\subsection{Site Information
Navigation}\label{site-information-navigation}}

\begin{itemize}
\tightlist
\item
  \href{https://help.nytimes.com/hc/en-us/articles/115014792127-Copyright-notice}{©~2020~The
  New York Times Company}
\end{itemize}

\begin{itemize}
\tightlist
\item
  \href{https://www.nytco.com/}{NYTCo}
\item
  \href{https://help.nytimes.com/hc/en-us/articles/115015385887-Contact-Us}{Contact
  Us}
\item
  \href{https://www.nytco.com/careers/}{Work with us}
\item
  \href{https://nytmediakit.com/}{Advertise}
\item
  \href{http://www.tbrandstudio.com/}{T Brand Studio}
\item
  \href{https://www.nytimes.com/privacy/cookie-policy\#how-do-i-manage-trackers}{Your
  Ad Choices}
\item
  \href{https://www.nytimes.com/privacy}{Privacy}
\item
  \href{https://help.nytimes.com/hc/en-us/articles/115014893428-Terms-of-service}{Terms
  of Service}
\item
  \href{https://help.nytimes.com/hc/en-us/articles/115014893968-Terms-of-sale}{Terms
  of Sale}
\item
  \href{https://spiderbites.nytimes.com}{Site Map}
\item
  \href{https://help.nytimes.com/hc/en-us}{Help}
\item
  \href{https://www.nytimes.com/subscription?campaignId=37WXW}{Subscriptions}
\end{itemize}
