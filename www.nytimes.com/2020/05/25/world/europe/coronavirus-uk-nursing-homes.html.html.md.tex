Sections

SEARCH

\protect\hyperlink{site-content}{Skip to
content}\protect\hyperlink{site-index}{Skip to site index}

\href{https://www.nytimes.com/section/world/europe}{Europe}

\href{https://myaccount.nytimes.com/auth/login?response_type=cookie\&client_id=vi}{}

\href{https://www.nytimes.com/section/todayspaper}{Today's Paper}

\href{/section/world/europe}{Europe}\textbar{}On a Scottish Isle,
Nursing Home Deaths Expose a Covid-19 Scandal

\url{https://nyti.ms/3gnKdHw}

\begin{itemize}
\item
\item
\item
\item
\item
\end{itemize}

\href{https://www.nytimes.com/news-event/coronavirus?action=click\&pgtype=Article\&state=default\&region=TOP_BANNER\&context=storylines_menu}{The
Coronavirus Outbreak}

\begin{itemize}
\tightlist
\item
  live\href{https://www.nytimes.com/2020/08/01/world/coronavirus-covid-19.html?action=click\&pgtype=Article\&state=default\&region=TOP_BANNER\&context=storylines_menu}{Latest
  Updates}
\item
  \href{https://www.nytimes.com/interactive/2020/us/coronavirus-us-cases.html?action=click\&pgtype=Article\&state=default\&region=TOP_BANNER\&context=storylines_menu}{Maps
  and Cases}
\item
  \href{https://www.nytimes.com/interactive/2020/science/coronavirus-vaccine-tracker.html?action=click\&pgtype=Article\&state=default\&region=TOP_BANNER\&context=storylines_menu}{Vaccine
  Tracker}
\item
  \href{https://www.nytimes.com/interactive/2020/07/29/us/schools-reopening-coronavirus.html?action=click\&pgtype=Article\&state=default\&region=TOP_BANNER\&context=storylines_menu}{What
  School May Look Like}
\item
  \href{https://www.nytimes.com/live/2020/07/31/business/stock-market-today-coronavirus?action=click\&pgtype=Article\&state=default\&region=TOP_BANNER\&context=storylines_menu}{Economy}
\end{itemize}

Advertisement

\protect\hyperlink{after-top}{Continue reading the main story}

Supported by

\protect\hyperlink{after-sponsor}{Continue reading the main story}

\hypertarget{on-a-scottish-isle-nursing-home-deaths-expose-a-covid-19-scandal}{%
\section{On a Scottish Isle, Nursing Home Deaths Expose a Covid-19
Scandal}\label{on-a-scottish-isle-nursing-home-deaths-expose-a-covid-19-scandal}}

At the Home Farm nursing home on the Isle of Skye, more than a quarter
of its residents died and nearly all were infected with coronavirus.
Families are furious.

\includegraphics{https://static01.nyt.com/images/2020/05/22/world/virus-uk-carehome1/merlin_172744356_74577628-c0fd-4beb-999f-a3bec9a17042-articleLarge.jpg?quality=75\&auto=webp\&disable=upscale}

\href{https://www.nytimes.com/by/benjamin-mueller}{\includegraphics{https://static01.nyt.com/images/2018/02/20/multimedia/author-benjamin-mueller/author-benjamin-mueller-thumbLarge.jpg}}

By \href{https://www.nytimes.com/by/benjamin-mueller}{Benjamin Mueller}

\begin{itemize}
\item
  May 25, 2020
\item
  \begin{itemize}
  \item
  \item
  \item
  \item
  \item
  \end{itemize}
\end{itemize}

On the Isle of Skye off the western coast of Scotland, residents thought
they had sealed themselves off from the coronavirus. They shuttered
hotels. Officials warned of police checks. Traffic emptied on the only
bridge from the mainland.

But the frailest spot on the island remained catastrophically exposed:
Home Farm, a 40-bed nursing home for people with dementia. Owned by a
private equity firm, Home Farm has become a grim monument of the push to
maximize profits at Britain's largest nursing home chains, and of the
government's failure to protect its most vulnerable citizens.

Today, all but seven of the residents have been stricken. More than a
quarter are dead.

The virus has ravaged nursing homes across Europe and the United States.
But the death toll in British homes --- 14,000, official figures say,
with thousands more dying as an indirect result of the virus --- is
becoming a defining scandal of the pandemic for Prime Minister Boris
Johnson.

By focusing at first on protecting the health system, Mr. Johnson's
strategy meant that some infected patients were unwittingly moved from
hospitals and into nursing homes. Residents and staff members were
denied tests, while nursing home workers begged in vain for protective
gear.

``We were witnessing horrendous images in Spain and Italy, so a lot of
attention was paid to maintaining and securing the National Health
Service,'' said Dr. Donald Macaskill, the chief executive of Scottish
Care, which represents nursing homes. ``The N.H.S. was prioritized at
the expense of social care.''

\includegraphics{https://static01.nyt.com/images/2020/05/22/world/virus-uk-carehome2/merlin_172527693_24a050a4-9400-4716-86d8-e8cf0225e429-articleLarge.jpg?quality=75\&auto=webp\&disable=upscale}

At Home Farm, set above a silvery loch on a northeastern finger of the
island, employees do not know how the virus got inside. But early in the
pandemic, they expressed fears to their bosses about the company
bringing in workers from outside the island. And they fretted over the
half-dozen new residents who were deposited in empty beds, some of them
from hospitals and others from their own homes.

Responding to the outbreak, the Scottish health secretary said a
\href{https://www.bbc.co.uk/news/uk-scotland-52789991}{review should be
conducted of its entire nursing home system}, which falls under Scottish
government control. In England, an independent commission is
\href{https://www.telegraph.co.uk/news/2020/05/14/equalities-watchdog-considers-whether-sending-patients-hospitals/}{looking
into ``serious potential breaches'' of human rights} in nursing homes.

Problems funding nursing homes, a bugaboo in British politics since
Margaret Thatcher privatized them in 1990, hobbled Mr. Johnson's
predecessor, Theresa May, whose
\href{https://www.ft.com/content/82ff3a76-3c98-11e7-ac89-b01cc67cfeec}{proposal
to raise resident fees} was nicknamed the ``dementia tax.''

Now Mr. Johnson is feeling the heat. In the House of Commons, he faces a
weekly barrage over nursing homes, including accusations that he lied
about government guidance playing down the chance of outbreaks.

\hypertarget{latest-updates-global-coronavirus-outbreak}{%
\section{\texorpdfstring{\href{https://www.nytimes.com/2020/08/01/world/coronavirus-covid-19.html?action=click\&pgtype=Article\&state=default\&region=MAIN_CONTENT_1\&context=storylines_live_updates}{Latest
Updates: Global Coronavirus
Outbreak}}{Latest Updates: Global Coronavirus Outbreak}}\label{latest-updates-global-coronavirus-outbreak}}

Updated 2020-08-02T07:33:01.580Z

\begin{itemize}
\tightlist
\item
  \href{https://www.nytimes.com/2020/08/01/world/coronavirus-covid-19.html?action=click\&pgtype=Article\&state=default\&region=MAIN_CONTENT_1\&context=storylines_live_updates\#link-34047410}{The
  U.S. reels as July cases more than double the total of any other
  month.}
\item
  \href{https://www.nytimes.com/2020/08/01/world/coronavirus-covid-19.html?action=click\&pgtype=Article\&state=default\&region=MAIN_CONTENT_1\&context=storylines_live_updates\#link-780ec966}{Top
  U.S. officials work to break an impasse over the federal jobless
  benefit.}
\item
  \href{https://www.nytimes.com/2020/08/01/world/coronavirus-covid-19.html?action=click\&pgtype=Article\&state=default\&region=MAIN_CONTENT_1\&context=storylines_live_updates\#link-2bc8948}{Its
  outbreak untamed, Melbourne goes into even greater lockdown.}
\end{itemize}

\href{https://www.nytimes.com/2020/08/01/world/coronavirus-covid-19.html?action=click\&pgtype=Article\&state=default\&region=MAIN_CONTENT_1\&context=storylines_live_updates}{See
more updates}

More live coverage:
\href{https://www.nytimes.com/live/2020/07/31/business/stock-market-today-coronavirus?action=click\&pgtype=Article\&state=default\&region=MAIN_CONTENT_1\&context=storylines_live_updates}{Markets}

``People in nursing homes, they don't have the same voice as I have,''
said John Gordon, a member of the local council from Skye, whose
83-year-old father is one of 10 Home Farm residents to have died. ``The
government has failed our old people.''

Image

John Gordon, right, with his father, John, who died in Home
Farm.Credit...via Gordon family

Britain's hospitals are revered for providing free, universal health
coverage. But the nursing home system is a decidedly American export,
with corporate giants based in offshore tax havens often paying workers
the minimum wage and trying to wring profits out of an aging population.

For-profit nursing homes now control even more of the British market ---
86 percent --- than the American market. And the biggest chain, HC-One,
which owns Home Farm, has been hit hard. Cases have broken out in
two-thirds of its 328 homes. Four employees and 934 residents have died.

Among the dead was Colin Harris, 66, a witty Home Farm resident with
dementia and Parkinson's disease.

In the months before Mr. Harris died on May 6, staffing was so thin that
his incontinence pads were often left wet, eroding the skin on his
thighs, Mandie Harris, his wife, wrote in a complaint. His dentures came
unglued when he ate.

After a video call on April 8, Ms. Harris complained to HC-One that she
saw aides without protective gear and a resident's husband walking down
the corridor in street clothes. ``Where is the infection control?'' she
asked. In response, the company told her in an email that the man was
being hired as a cleaner and that Mr. Harris's ``teeth appeared clean
and secure.''

Image

Colin Harris, 66, who died in May.Credit...via Harris family

Inside the home, staff members were becoming panicked, said three
workers, who spoke on the condition of anonymity because they had been
instructed not to talk publicly. In early April staff meetings, they
pleaded for better protective gear, and in some cases ordered their own.

But management told workers to wear masks only around suspected
coronavirus patients --- an approach that Ms. Harris, in her complaint,
compared to ``closing the gate after the horse has bolted.'' The company
told her that aides who wanted masks were provided with them starting
April 9. Not until April 18, a week before the outbreak, were masks
required.

Even so, managers sometimes refused to wear masks themselves, including
on medicine rounds to residents' rooms, complaining that they itched,
the three workers said.

Soon after a nationwide lockdown went into effect in March, a new deputy
manager arrived from Kent, in southeastern England. HC-One has said she
isolated before starting work. But that was before she made the 650-mile
journey to the island, the employees and HC-One said. She eventually
became sick and stopped working, the company said.

Feeling unprotected by management, employees cleaned the home
obsessively and enforced their own distancing rules. When residents were
startled, as they often were, aides held their hands and stroked them.
Sometimes employees broke down crying.

``People were petrified,'' one of the employees said.

For HC-One, the nursing home business has been lucrative, as the company
paid more than 50 million pounds, or nearly \$61 million, in dividends
from 2017 to 2019.

But staff members at Home Farm suffered. During 12-hour shifts, they
sometimes made do with no more than one nurse and two aides on an 18-bed
floor, employees said. The residents' help buttons buzzed incessantly.

Staffing shortages were so dire, regulators said in January, that Home
Farm stopped accepting new residents. An inspection found that the home
was unclean, staffing was uneven and ``the level and quality of care and
support people received was not always adequate.''

HC-One said it ``faced chronic recruitment challenges,'' forcing the
home to rely on temporary workers.

Image

The hospital in London where Mr. Johnson was treated for
coronavirus.~Britain's hospitals are revered for providing free,
universal health coverage.Credit...Andrew Testa for The New York Times

But as the pandemic raged in Britain, the moratorium on new admissions
at Home Farm was lifted. When employees complained about the risk of
transmission and even volunteered to temporarily move into the home to
avoid carrying the virus inside, management said beds needed to be
filled with paying customers, two workers said.

\href{https://www.nytimes.com/news-event/coronavirus?action=click\&pgtype=Article\&state=default\&region=MAIN_CONTENT_3\&context=storylines_faq}{}

\hypertarget{the-coronavirus-outbreak-}{%
\subsubsection{The Coronavirus Outbreak
›}\label{the-coronavirus-outbreak-}}

\hypertarget{frequently-asked-questions}{%
\paragraph{Frequently Asked
Questions}\label{frequently-asked-questions}}

Updated July 27, 2020

\begin{itemize}
\item ~
  \hypertarget{should-i-refinance-my-mortgage}{%
  \paragraph{Should I refinance my
  mortgage?}\label{should-i-refinance-my-mortgage}}

  \begin{itemize}
  \tightlist
  \item
    \href{https://www.nytimes.com/article/coronavirus-money-unemployment.html?action=click\&pgtype=Article\&state=default\&region=MAIN_CONTENT_3\&context=storylines_faq}{It
    could be a good idea,} because mortgage rates have
    \href{https://www.nytimes.com/2020/07/16/business/mortgage-rates-below-3-percent.html?action=click\&pgtype=Article\&state=default\&region=MAIN_CONTENT_3\&context=storylines_faq}{never
    been lower.} Refinancing requests have pushed mortgage applications
    to some of the highest levels since 2008, so be prepared to get in
    line. But defaults are also up, so if you're thinking about buying a
    home, be aware that some lenders have tightened their standards.
  \end{itemize}
\item ~
  \hypertarget{what-is-school-going-to-look-like-in-september}{%
  \paragraph{What is school going to look like in
  September?}\label{what-is-school-going-to-look-like-in-september}}

  \begin{itemize}
  \tightlist
  \item
    It is unlikely that many schools will return to a normal schedule
    this fall, requiring the grind of
    \href{https://www.nytimes.com/2020/06/05/us/coronavirus-education-lost-learning.html?action=click\&pgtype=Article\&state=default\&region=MAIN_CONTENT_3\&context=storylines_faq}{online
    learning},
    \href{https://www.nytimes.com/2020/05/29/us/coronavirus-child-care-centers.html?action=click\&pgtype=Article\&state=default\&region=MAIN_CONTENT_3\&context=storylines_faq}{makeshift
    child care} and
    \href{https://www.nytimes.com/2020/06/03/business/economy/coronavirus-working-women.html?action=click\&pgtype=Article\&state=default\&region=MAIN_CONTENT_3\&context=storylines_faq}{stunted
    workdays} to continue. California's two largest public school
    districts --- Los Angeles and San Diego --- said on July 13, that
    \href{https://www.nytimes.com/2020/07/13/us/lausd-san-diego-school-reopening.html?action=click\&pgtype=Article\&state=default\&region=MAIN_CONTENT_3\&context=storylines_faq}{instruction
    will be remote-only in the fall}, citing concerns that surging
    coronavirus infections in their areas pose too dire a risk for
    students and teachers. Together, the two districts enroll some
    825,000 students. They are the largest in the country so far to
    abandon plans for even a partial physical return to classrooms when
    they reopen in August. For other districts, the solution won't be an
    all-or-nothing approach.
    \href{https://bioethics.jhu.edu/research-and-outreach/projects/eschool-initiative/school-policy-tracker/}{Many
    systems}, including the nation's largest, New York City, are
    devising
    \href{https://www.nytimes.com/2020/06/26/us/coronavirus-schools-reopen-fall.html?action=click\&pgtype=Article\&state=default\&region=MAIN_CONTENT_3\&context=storylines_faq}{hybrid
    plans} that involve spending some days in classrooms and other days
    online. There's no national policy on this yet, so check with your
    municipal school system regularly to see what is happening in your
    community.
  \end{itemize}
\item ~
  \hypertarget{is-the-coronavirus-airborne}{%
  \paragraph{Is the coronavirus
  airborne?}\label{is-the-coronavirus-airborne}}

  \begin{itemize}
  \tightlist
  \item
    The coronavirus
    \href{https://www.nytimes.com/2020/07/04/health/239-experts-with-one-big-claim-the-coronavirus-is-airborne.html?action=click\&pgtype=Article\&state=default\&region=MAIN_CONTENT_3\&context=storylines_faq}{can
    stay aloft for hours in tiny droplets in stagnant air}, infecting
    people as they inhale, mounting scientific evidence suggests. This
    risk is highest in crowded indoor spaces with poor ventilation, and
    may help explain super-spreading events reported in meatpacking
    plants, churches and restaurants.
    \href{https://www.nytimes.com/2020/07/06/health/coronavirus-airborne-aerosols.html?action=click\&pgtype=Article\&state=default\&region=MAIN_CONTENT_3\&context=storylines_faq}{It's
    unclear how often the virus is spread} via these tiny droplets, or
    aerosols, compared with larger droplets that are expelled when a
    sick person coughs or sneezes, or transmitted through contact with
    contaminated surfaces, said Linsey Marr, an aerosol expert at
    Virginia Tech. Aerosols are released even when a person without
    symptoms exhales, talks or sings, according to Dr. Marr and more
    than 200 other experts, who
    \href{https://academic.oup.com/cid/article/doi/10.1093/cid/ciaa939/5867798}{have
    outlined the evidence in an open letter to the World Health
    Organization}.
  \end{itemize}
\item ~
  \hypertarget{what-are-the-symptoms-of-coronavirus}{%
  \paragraph{What are the symptoms of
  coronavirus?}\label{what-are-the-symptoms-of-coronavirus}}

  \begin{itemize}
  \tightlist
  \item
    Common symptoms
    \href{https://www.nytimes.com/article/symptoms-coronavirus.html?action=click\&pgtype=Article\&state=default\&region=MAIN_CONTENT_3\&context=storylines_faq}{include
    fever, a dry cough, fatigue and difficulty breathing or shortness of
    breath.} Some of these symptoms overlap with those of the flu,
    making detection difficult, but runny noses and stuffy sinuses are
    less common.
    \href{https://www.nytimes.com/2020/04/27/health/coronavirus-symptoms-cdc.html?action=click\&pgtype=Article\&state=default\&region=MAIN_CONTENT_3\&context=storylines_faq}{The
    C.D.C. has also} added chills, muscle pain, sore throat, headache
    and a new loss of the sense of taste or smell as symptoms to look
    out for. Most people fall ill five to seven days after exposure, but
    symptoms may appear in as few as two days or as many as 14 days.
  \end{itemize}
\item ~
  \hypertarget{does-asymptomatic-transmission-of-covid-19-happen}{%
  \paragraph{Does asymptomatic transmission of Covid-19
  happen?}\label{does-asymptomatic-transmission-of-covid-19-happen}}

  \begin{itemize}
  \tightlist
  \item
    So far, the evidence seems to show it does. A widely cited
    \href{https://www.nature.com/articles/s41591-020-0869-5}{paper}
    published in April suggests that people are most infectious about
    two days before the onset of coronavirus symptoms and estimated that
    44 percent of new infections were a result of transmission from
    people who were not yet showing symptoms. Recently, a top expert at
    the World Health Organization stated that transmission of the
    coronavirus by people who did not have symptoms was ``very rare,''
    \href{https://www.nytimes.com/2020/06/09/world/coronavirus-updates.html?action=click\&pgtype=Article\&state=default\&region=MAIN_CONTENT_3\&context=storylines_faq\#link-1f302e21}{but
    she later walked back that statement.}
  \end{itemize}
\end{itemize}

HC-One said that, like other homes, it was asked to help hospitals by
admitting some patients.

In late April, employees' fears were realized: An aide tested positive.
Employees said they learned the news not from management, but from
Facebook, where the aide's mother posted about it.

Residents, too, were showing symptoms, like slackening appetites and
high temperatures. By April 27, a Monday, staff members were adamant
that residents needed help. Management urged them not to worry, arguing
that it was just the flu, the three workers said.

At the time, testing was scarce: Not until days later did Scotland say
it would
\href{https://www.bbc.co.uk/news/uk-scotland-scotland-politics-52495827}{offer
tests to any nursing homes with cases}. So only four of the home's 36
residents were initially given tests. Blanket tests later that week
revealed the calamitous extent of the outbreak: 28 residents were
infected, along with 26 of 52 staff members.

Residents' families said administrators were slow to acknowledge the
likely spread. Up until her husband's positive result, Ms. Harris said,
management told her he was being treated as though he had a chest or
urinary tract infection. Later, Ms. Harris said management insisted he
was tired, but nothing worse, only for her to see him on a video call
``looking deathly.''

Image

Home Farm employees pleaded for better protective masks.Credit...Peter
Jolly/Shutterstock

HC-One attributed the number of infections in part to Home Farm being
``one of the first care homes where everyone was tested.'' The company
said it was ``confident the manager acted appropriately with regards to
Mr. Harris's health.''

The police are now investigating the deaths of three residents. The
local health service has stepped in to help run Home Farm. Regulators
tried this month to take HC-One's license in court but have since backed
off.

In the absence of details about the outbreak, islanders said, rumors
multiplied and staff members were unfairly blamed.

``It's a well-connected community,'' said Keith MacKenzie, the lone
reporter for the local West Highland Free Press during the outbreak.
``But of course in well-connected communities, it's not always the right
information that gets circulated.''

Britain's nursing home chains, already carrying substantial debt, have
been imperiled by rising costs and plunging occupancy rates during the
pandemic. HC-One
\href{https://www.ft.com/content/edf259f8-7cb0-4d6f-b58f-7202a5d28e28}{warned
that its ``ability to continue as a going concern''} was in jeopardy.

But nursing home finances are difficult to trace. The HC-One group
includes 62 companies, 19 of them registered offshore, and its parent
company is based in the Cayman Islands.

``It's money before care all the time,'' Ms. Harris said. ``The staff
they did have worked so hard, but they've been let down.''

On the afternoon of May 6, nurses called Ms. Harris to tell her that her
husband's breathing was failing. She hurried over. Zoe Docherty, their
daughter, was already in protective gear, holding her father's hand.

Ms. Docherty asked that her mother be allowed inside; management had
said they could take turns visiting. But she and a nurse disagreed about
how to choreograph the swap, and at Ms. Docherty's frantic urging, the
nurse left to consult colleagues.

Meanwhile, Mr. Harris died, with his wife looking through the glass from
outside.

Advertisement

\protect\hyperlink{after-bottom}{Continue reading the main story}

\hypertarget{site-index}{%
\subsection{Site Index}\label{site-index}}

\hypertarget{site-information-navigation}{%
\subsection{Site Information
Navigation}\label{site-information-navigation}}

\begin{itemize}
\tightlist
\item
  \href{https://help.nytimes.com/hc/en-us/articles/115014792127-Copyright-notice}{©~2020~The
  New York Times Company}
\end{itemize}

\begin{itemize}
\tightlist
\item
  \href{https://www.nytco.com/}{NYTCo}
\item
  \href{https://help.nytimes.com/hc/en-us/articles/115015385887-Contact-Us}{Contact
  Us}
\item
  \href{https://www.nytco.com/careers/}{Work with us}
\item
  \href{https://nytmediakit.com/}{Advertise}
\item
  \href{http://www.tbrandstudio.com/}{T Brand Studio}
\item
  \href{https://www.nytimes.com/privacy/cookie-policy\#how-do-i-manage-trackers}{Your
  Ad Choices}
\item
  \href{https://www.nytimes.com/privacy}{Privacy}
\item
  \href{https://help.nytimes.com/hc/en-us/articles/115014893428-Terms-of-service}{Terms
  of Service}
\item
  \href{https://help.nytimes.com/hc/en-us/articles/115014893968-Terms-of-sale}{Terms
  of Sale}
\item
  \href{https://spiderbites.nytimes.com}{Site Map}
\item
  \href{https://help.nytimes.com/hc/en-us}{Help}
\item
  \href{https://www.nytimes.com/subscription?campaignId=37WXW}{Subscriptions}
\end{itemize}
