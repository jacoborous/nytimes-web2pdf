Sections

SEARCH

\protect\hyperlink{site-content}{Skip to
content}\protect\hyperlink{site-index}{Skip to site index}

\href{https://www.nytimes.com/section/world/europe}{Europe}

\href{https://myaccount.nytimes.com/auth/login?response_type=cookie\&client_id=vi}{}

\href{https://www.nytimes.com/section/todayspaper}{Today's Paper}

\href{/section/world/europe}{Europe}\textbar{}Bafflement Greets Boris
Johnson's Plan for Reopening Britain

\url{https://nyti.ms/2WpK2Ul}

\begin{itemize}
\item
\item
\item
\item
\item
\end{itemize}

\href{https://www.nytimes.com/news-event/coronavirus?action=click\&pgtype=Article\&state=default\&region=TOP_BANNER\&context=storylines_menu}{The
Coronavirus Outbreak}

\begin{itemize}
\tightlist
\item
  live\href{https://www.nytimes.com/2020/08/01/world/coronavirus-covid-19.html?action=click\&pgtype=Article\&state=default\&region=TOP_BANNER\&context=storylines_menu}{Latest
  Updates}
\item
  \href{https://www.nytimes.com/interactive/2020/us/coronavirus-us-cases.html?action=click\&pgtype=Article\&state=default\&region=TOP_BANNER\&context=storylines_menu}{Maps
  and Cases}
\item
  \href{https://www.nytimes.com/interactive/2020/science/coronavirus-vaccine-tracker.html?action=click\&pgtype=Article\&state=default\&region=TOP_BANNER\&context=storylines_menu}{Vaccine
  Tracker}
\item
  \href{https://www.nytimes.com/interactive/2020/07/29/us/schools-reopening-coronavirus.html?action=click\&pgtype=Article\&state=default\&region=TOP_BANNER\&context=storylines_menu}{What
  School May Look Like}
\item
  \href{https://www.nytimes.com/live/2020/07/31/business/stock-market-today-coronavirus?action=click\&pgtype=Article\&state=default\&region=TOP_BANNER\&context=storylines_menu}{Economy}
\end{itemize}

Advertisement

\protect\hyperlink{after-top}{Continue reading the main story}

Supported by

\protect\hyperlink{after-sponsor}{Continue reading the main story}

\hypertarget{bafflement-greets-boris-johnsons-plan-for-reopening-britain}{%
\section{Bafflement Greets Boris Johnson's Plan for Reopening
Britain}\label{bafflement-greets-boris-johnsons-plan-for-reopening-britain}}

Critics say the government has failed to answer some basic questions.
Among them: 1) when to return to work and 2) how to get there.

\includegraphics{https://static01.nyt.com/images/2020/05/11/world/11-virus-britain-1/merlin_172365441_1d3eab80-041a-4a96-8517-bc8c0a870f83-articleLarge.jpg?quality=75\&auto=webp\&disable=upscale}

\href{https://www.nytimes.com/by/stephen-castle}{\includegraphics{https://static01.nyt.com/images/2018/10/08/multimedia/author-stephen-castle/author-stephen-castle-thumbLarge.png}}\href{https://www.nytimes.com/by/mark-landler}{\includegraphics{https://static01.nyt.com/images/2019/10/22/reader-center/author-mark-landler/author-mark-landler-thumbLarge-v3.png}}

By \href{https://www.nytimes.com/by/stephen-castle}{Stephen Castle} and
\href{https://www.nytimes.com/by/mark-landler}{Mark Landler}

\begin{itemize}
\item
  May 11, 2020
\item
  \begin{itemize}
  \item
  \item
  \item
  \item
  \item
  \end{itemize}
\end{itemize}

LONDON --- When the coronavirus first struck,
\href{https://www.nytimes.com/2020/05/12/podcasts/the-daily/boris-johnson-uk-coronavirus.html}{Prime
Minister Boris Johnson} of Britain agonized long and hard before closing
stores, pubs and restaurants as part of the country's fight against the
disease. But with the spread of the virus now curbed, easing the
lockdown is proving harder still.

On Monday Mr. Johnson's long-anticipated
\href{https://www.nytimes.com/2020/05/10/world/europe/coronavirus-britain-boris-johnson.html}{blueprint
for reopening of the economy} ran into a barrage of opposition, as
critics pointed to gaps and contradictions in a plan that left many
pondering basic questions such as when to return to work and how to get
there.

``What the country needs is clarity and reassurance, and at the moment
both are in short supply,'' Keir Starmer, leader of the opposition
Labour Party, told Parliament.

Mr. Starmer accused Mr. Johnson of spreading ``considerable confusion''
in a country that is among the worst hit in Europe by the pandemic.

Mr. Johnson, making his first statement to Parliament on the virus, said
Monday that the nation's ``shared effort has averted a still worse
catastrophe.'' He rejected criticism that his proposals were too vague,
saying he trusted the public to apply ``good, solid British common
sense.''

But political leaders in Scotland and Wales have been quick to reject
parts of the new strategy. And with contradictory official advice over
when a return to work should start, even those in some quarters that
generally support the government were unforgiving.

``Boris Johnson's big lockdown speech descends into farce'' was
\href{https://www.dailymail.co.uk/news/article-8306539/Commuters-crowd-Tubes-Boris-Johnson-urges-Britons-work.html}{the
headline in MailOnline}, the digital edition of The Daily Mail.

\includegraphics{https://static01.nyt.com/images/2017/01/29/podcasts/the-daily-album-art/the-daily-album-art-articleInline-v2.jpg?quality=75\&auto=webp\&disable=upscale}

\hypertarget{listen-to-the-daily-boris-johnsons-change-of-heart}{%
\subsubsection{Listen to `The Daily': Boris Johnson's Change of
Heart}\label{listen-to-the-daily-boris-johnsons-change-of-heart}}

The Prime Minister was relaxed about the pandemic. Until he was
hospitalized with the virus.

transcript

Back to The Daily

bars

0:00/27:36

-27:36

transcript

\hypertarget{listen-to-the-daily-boris-johnsons-change-of-heart-1}{%
\subsection{Listen to `The Daily': Boris Johnson's Change of
Heart}\label{listen-to-the-daily-boris-johnsons-change-of-heart-1}}

\hypertarget{hosted-by-michael-barbaro-produced-by-rachel-quester-and-luke-vander-ploeg-with-help-from-asthaa-chaturvedi-and-mj-davis-lin-and-edited-by-lisa-tobin}{%
\subsubsection{Hosted by Michael Barbaro; produced by Rachel Quester and
Luke Vander Ploeg; with help from Asthaa Chaturvedi and M.J. Davis Lin;
and edited by Lisa
Tobin}\label{hosted-by-michael-barbaro-produced-by-rachel-quester-and-luke-vander-ploeg-with-help-from-asthaa-chaturvedi-and-mj-davis-lin-and-edited-by-lisa-tobin}}

\hypertarget{the-prime-minister-was-relaxed-about-the-pandemic-until-he-was-hospitalized-with-the-virus}{%
\paragraph{The Prime Minister was relaxed about the pandemic. Until he
was hospitalized with the
virus.}\label{the-prime-minister-was-relaxed-about-the-pandemic-until-he-was-hospitalized-with-the-virus}}

\begin{itemize}
\item
  michael barbaro\\
  From The New York Times, I'm Michael Barbaro. This is ``The Daily.''
\item
  {[}music{]}\\
  Today: As Europe begins to reopen, and countries like Germany send
  children back to school. Why the country that was the most skeptical
  of the virus may be the slowest to reopen. Mark Landler on the
  situation in Britain.

  It's Tuesday, May 12.

  Mark, how would you describe Boris Johnson's original approach to the
  coronavirus?
\item
  mark landler\\
  Well I'd say initially, he was really kind of out to lunch on it. You
  know, remember, the earliest days of the virus in late January, early
  February coincided with Britain formally leaving the European Union.
\item
  michael barbaro\\
  Right.
\item
  archived recording (boris johnson)\\
  This is the moment when the dawn breaks and the curtain goes up on a
  new act in our great national drama.
\end{itemize}

mark landler

In those early days, Johnson was really very preoccupied and distracted
by the milestone of Brexit.

\begin{itemize}
\tightlist
\item
  archived recording (boris johnson)\\
  I know that we can turn this opportunity into a stunning success. And
  whatever the bumps in the road ahead, I know that we will succeed.
\end{itemize}

mark landler

So there were several very key early meetings of his cabinet that he
didn't attend, where the subject of the virus came up. And then even
when you began to see the first cases in Britain, and it became clear
that Britain also faced a major risk, he affected an air of nonchalance.

\begin{itemize}
\tightlist
\item
  archived recording (boris johnson)\\
  Wash your hands with soap and hot water for the length of time it
  takes to sing Happy Birthday twice.
\end{itemize}

mark landler

He even sort of made light of some of the caution of the scientists.
There was a very early press conference on March 3 where he made a point
of saying he'd visited coronavirus patients at a hospital, and he went
around and shook all their hands.

\begin{itemize}
\tightlist
\item
  archived recording (boris johnson)\\
  I was at a hospital the other night where --- I think there were
  actually a few coronavirus patients, and I shook hands with everybody,
  you'll be pleased to know. And I continue to shake hands and I think
  it's very important that we --- people obviously can make up their own
  minds, I think ---
\end{itemize}

mark landler

And then, as if seeking validation, he turned to the government's chief
scientific adviser, who was standing next to him ---

\begin{itemize}
\tightlist
\item
  archived recording (boris johnson)\\
  I think the scientific evidence is --- well, I'll hand it over to the
  experts.
\end{itemize}

mark landler

And the scientific advisor was shaking his head with pursed lips.

\begin{itemize}
\tightlist
\item
  archived recording (boris johnson)\\
  But our judgement is wash. Washing your hands is the crucial thing.
\end{itemize}

mark landler

This is a guy who has his own way of living, and very much a
live-and-let-live philosophy, and I think he brought an element of that
to the early days of this crisis.

michael barbaro

Mark, there are obvious ways in which the story you're describing echoes
what we saw here in the U.S., and with President Trump's initial
response to the coronavirus. Having covered both our White House and now
Downing Street, are there notable differences in the way the two
countries and the two governments responded to their leaders dismissing
the severity of the situation?

mark landler

Oh, yeah. I think there are stark differences. And I would group them in
a couple of categories. First, in the United States, you obviously have
independent minded governors who chose to respond to the crisis for
their own states, regardless of what President Trump's decisions and
signals were on the federal level. You don't have that here in Britain.
Here, Downing Street, the rime minister, really sets the policy for the
whole country. So that's one obvious difference. I think somewhat more
subtly is the role of the scientific community. In the United States,
you had the C.D.C., Dr. Anthony Fauci and others very quickly and
urgently calling for a very rigorous response for lockdowns. Here, on
the other hand, you had a scientific community that took a slightly
maverick initial response to this virus. There were very prominent
scientists who advised the government who were arguing that maybe the
best way to deal with the virus is to let it rip through the population,
to let the population develop what they call herd immunity. And that
that would build up a natural immunity that would make Britain more
resilient to subsequent potential waves of infection. So this was an
idea that was circulating in the scientific community in those early
weeks --- late February, early March --- and it, to some extent, also
informed where Boris Johnson was coming down on these issues. Unlike in
the United States, where Donald Trump has blown hot and cold on his
scientific advisors, Boris Johnson has said from the very start, I will
be guided by the science. The wrinkle is that in those early days, the
science was also calling for a more casual reaction. And that's what you
saw in the first two weeks of March.

michael barbaro

That's fascinating. So given that, what ends up being the official
approach to this in Britain?

mark landler

Well, in the second week of March ---

\begin{itemize}
\tightlist
\item
  archived recording (boris johnson)\\
  Good afternoon, everybody, and thank you very much for coming.
\end{itemize}

mark landler

Boris Johnson has a daily news conference, very similar to what you saw
in the White House with President Trump.

\begin{itemize}
\tightlist
\item
  archived recording (boris johnson)\\
  We've done what can be done to contain this disease.
\end{itemize}

mark landler

And at that news conference, he begins to urge a very mild form of
social distancing.

\begin{itemize}
\tightlist
\item
  archived recording (boris johnson)\\
  We are considering the question of banning major public events, such
  as sporting fixtures. And the scientific advice, as we've said over
  the last couple of weeks, is that this banning such events will have
  little effect on the spread.
\end{itemize}

mark landler

But he's stopping well short of asking people to quarantine themselves
in their houses, closing restaurants, closing bars. All of things, by
the way, that are being done by France, by parts of Germany, by Spain,
by Italy. So at that point, Britain is a clear outrider, and you're
really seeing the British go off in a very different direction that is
really very striking, and beginning to draw notice all over Europe.

michael barbaro

Mm hmm.

mark landler

And then on March 16, Imperial College, which is one of the leading
institutions in Britain that studies infectious diseases, published a
report that changed everything. The headline of the report was basically
that if the British government allowed the virus to spread through the
population unchecked, it risked anywhere from 250,000 to 500,000
fatalities.

michael barbaro

Wow.

mark landler

It's worth noting that this report was written by an epidemiologist
called Neil Ferguson. He's probably the leading figure on epidemics and
infectious disease in this country, and his reports and his
pronouncements have almost a sort of a holy writ quality to them. He's
really viewed as something of an oracle in this area. And the report
essentially terrified everyone inside the British government. And you
could almost see the change in the thinking take shape over the
subsequent days as Boris Johnson continued to appear on television, each
day looking a little more rattled, a little more anxious. And then
finally on Monday, March 23 ---

\begin{itemize}
\tightlist
\item
  archived recording (boris johnson)\\
  Good evening. The coronavirus is the biggest threat this country has
  faced for decades.
\end{itemize}

mark landler

Johnson addressed the nation at around 7 o'clock in the evening.

\begin{itemize}
\tightlist
\item
  archived recording (boris johnson)\\
  From this evening, I must give the British people a very simple
  instruction: You must stay at home. You should not be meeting friends.
  If your friends ask you to meet, you should say no. If you don't
  follow the rules, the police will have the powers to enforce them.
\end{itemize}

mark landler

So you now finally had Britain in alignment with Italy, with Spain, with
France, in a total lockdown.

\begin{itemize}
\tightlist
\item
  archived recording (boris johnson)\\
  Stay at home, protect our NHS and save lives. Thank you.
\end{itemize}

mark landler

But it was at least a week to 10 days later than those other countries
had acted, so in essence, Britain lost some valuable days.

And sure enough, events began to vindicate that very dark picture.

\begin{itemize}
\tightlist
\item
  archived recording\\
  Some breaking news coming into us from NHS England, and it is an
  update of some of the numbers of those who have sadly died.
\end{itemize}

mark landler

The death toll begins mounting. The hospitals begin filling up. And in
fact, and most dramatically, perhaps, the virus actually makes its way
into the political establishment of the government.

\begin{itemize}
\tightlist
\item
  archived recording\\
  Now proof that the virus can affect anyone came earlier on Wednesday,
  when it was confirmed ---
\end{itemize}

mark landler

The first dramatic moment comes when Prince Charles, the heir to the
throne, the eldest son of Queen Elizabeth, announces that he's tested
positive.

\begin{itemize}
\tightlist
\item
  archived recording\\
  He has been displaying mild symptoms, but otherwise remains in good
  health, and has been working from home throughout the last few days as
  usual.
\end{itemize}

mark landler

A few days later, you begin hearing that senior officials in the
government are testing positive. The health secretary, the man most
directly responsible for leading the response to the crisis, has to
quarantine himself. He has symptoms. The chief medical officer is
quarantining himself. He has symptoms. And then on March 27 ---

\begin{itemize}
\tightlist
\item
  archived recording (boris johnson)\\
  Hi, folks. I want to bring you up to speed with something that's
  happening today, which is that I've developed mild symptoms of the
  coronavirus. That's to say, a temperature and a persistent cough.
\end{itemize}

mark landler

Boris Johnson announces to the British public that he has tested
positive for coronavirus and will be going into isolation.

\begin{itemize}
\tightlist
\item
  archived recording (boris johnson)\\
  But be in no doubt that I can continue, thanks to the wizardry of
  modern technology, to communicate with all my top team to lead the
  national fight back against coronavirus. And I want to thank ---
\end{itemize}

mark landler

So what you see in that first week is Boris Johnson recording videos in
which he says, I'm still at the helm, I'm still directing policy. He
runs several crisis meetings from his apartment adjacent to 10 Downing
Street. So he's isolated, but he's in command. That's the very
reassuring message that he's trying to send. But then you arrive at an
important weekend, the weekend of April 5, when Queen Elizabeth is going
to address the nation on this national challenge of the pandemic. And
she's delivered a handful of such addresses in her 68-year reign, so
this is a very big, historic moment for the country.

\begin{itemize}
\tightlist
\item
  archived recording (queen elizabeth)\\
  I'm speaking to you at what I know is an increasingly challenging
  time.
\end{itemize}

mark landler

She addresses the nation early in the evening.

\begin{itemize}
\tightlist
\item
  archived recording (queen elizabeth)\\
  We should take comfort that while we may have more still to endure,
  better days will return.
\end{itemize}

mark landler

She finishes her address ---

\begin{itemize}
\tightlist
\item
  archived recording (queen elizabeth)\\
  But for now, I send my thanks and warmest good wishes to you all.
\end{itemize}

mark landler

And about an hour later, Downing Street puts out a press release saying
Boris Johnson has entered the hospital.

{[}music{]}

michael barbaro

We'll be right back.

So Mark, what happens once Boris Johnson enters the hospital?

mark landler

Well, the first thing that happens is there's a great deal of
uncertainty over who would actually run the government were he to be
incapacitated for a long time, or God forbid, not come back at all.
Britain, unlike the United States, doesn't have a clear line of
succession for the head of the government. But then you kind of enter
this very unsettling period where the government is continuing to put
out reassuring press releases. They invariably say the prime minister is
in, quote, ``good spirits.'' But his top aides acknowledge that no one's
talked to him for several days. His doctors are not releasing any status
reports on his condition. And there starts to be this undercurrent of
worry in the country that things are actually maybe worse than we think.
And then sure enough, a few days later comes the really alarming news
---

\begin{itemize}
\tightlist
\item
  archived recording (dominic raab)\\
  During the course of this afternoon, the prime minister's condition
  worsened, and on the advice of the medical team, he was moved into a
  critical care unit.
\end{itemize}

mark landler

He's actually been admitted to the intensive care unit. Statistically,
the odds of patients decline rapidly when they go into the I.C.U. And so
there is a sense, a real sense, that Boris Johnson may not make it.

\begin{itemize}
\item
  archived recording 1\\
  This is obviously an extremely serious situation. I mean, how worried
  should people be about his health and about who's in charge of the
  government?
\item
  archived recording 2\\
  Well, the government's business will continue. And the focus of the
  government will continue to be on making sure, the prime minister's
  direction, all the plans for making sure that we can defeat
  coronavirus and pull the country through this challenge, will be taken
  forward.
\end{itemize}

mark landler

But then there's sort of good news.

\begin{itemize}
\tightlist
\item
  archived recording\\
  We got this statement from 10 Downing Street. The prime minister has
  been moved this evening from intensive care back to the ward, where he
  will receive close monitoring during the early phase of his recovery.
\end{itemize}

mark landler

You start to hear better things. He's beginning to walk around a bit,
he's beginning to read some briefing papers. And then on Easter Sunday,
of all days ---

\begin{itemize}
\tightlist
\item
  archived recording\\
  The prime minister, Boris Johnson, has now been discharged from
  hospital as he continues his recovery from Covid-19.
\end{itemize}

mark landler

We get the news that Boris Johnson has been discharged from the
hospital. And of course, the timing occasions some classic British
humor. Senior government officials are heard saying to each other, he
has risen, and referring to Boris as the Messiah. And for the country,
the moment of peak anxiety, the moment where they genuinely thought they
might have lost their leader, has passed.

michael barbaro

Mark, I'm curious how Brits responded to all of this. I could see this
being an obvious moment for the British people to turn on Boris Johnson.
He was slow to act, he took risks, he was pretty dismissive of the
coronavirus. It would be easy to see all of this as the actions of an
irresponsible leader that almost cost him his own life, and deprived the
country of its prime minister.

mark landler

That's a very plausible assumption to make. But oddly, it isn't the way
it played out. And I think the reason it is isn't has to do with the
rather clever way that Boris Johnson handled his own illness.

\begin{itemize}
\tightlist
\item
  archived recording (boris johnson)\\
  Good afternoon. I have today left hospital after a week in which the
  NHS has saved my life. No question.
\end{itemize}

mark landler

On the day he was released from the hospital, he recorded this very
heartfelt video, in which he thanked the doctors and nurses at the
hospital for saving his life.

\begin{itemize}
\tightlist
\item
  archived recording (boris johnson)\\
  I'm going to forget some names, so please forgive me. But I want to
  thank Po Ling and Shannon and Emily and Angel and Connie and Becky and
  Rachael and Nicky ---
\end{itemize}

mark landler

He singled out two of the nurses ---

\begin{itemize}
\tightlist
\item
  archived recording (boris johnson)\\
  Jenny from New Zealand --- Invercargill, on the South Island, to be
  exact --- and Luis from Portugal near Porto.
\end{itemize}

mark landler

--- who kept a vigil at his bedside all night long, giving him oxygen.

\begin{itemize}
\tightlist
\item
  archived recording (boris johnson)\\
  For every second of the night, they were watching, and they were
  thinking and they were caring and making the interventions I needed.
\end{itemize}

mark landler

The NHS in this country is a revered institution, and so Boris Johnson
very much tied himself to that institution and made his own personal
story part of a broader narrative. And in so doing, he kept the support
of the public far from turning on him, far from telling him you got what
you deserved. I think a lot of people in this country were sympathetic
to him, and it's really all the more remarkable when you consider that
in the days and weeks since he left the hospital, the death toll from
coronavirus in this country has continued to spiral upward. And as of a
week ago or so, Britain now has the largest number of deaths of any
country in Europe, and the second largest number of deaths of any
country in the world, after the United States. And yet, even with that
tragic human cost of this virus, Boris Johnson and his government retain
the support of something like 51 percent , 52 percent of the people who
approve of the way that they've handled this virus.

michael barbaro

So Mark, where are we right now in Britain? I mean, we're watching the
world --- Europe in particular --- start to figure out what reopening is
supposed to look like. Given this up-and-down history that you just
recounted, what are you seeing that look like in Britain?

mark landler

Well, so Britain is facing this increasing pressure to restart its
economy, to ease the lockdown, to try to bring society back to some
level of normalcy. And you're beginning to see the Italians, the
Spanish, the Germans lift their lockdown. Britain reaches that same
moment, and Boris Johnson announces he's going to give a speech to the
nation in which he's going to lay out a roadmap. And there's a
tremendous burst of hope that we are at the end of this difficult
period. Some of the tabloids talk about the end of the lockdown, and the
happy days that lie ahead. And so on Sunday night ---

\begin{itemize}
\tightlist
\item
  archived recording (boris johnson)\\
  It is now almost two months since the people of this country began to
  put up with restrictions on their freedom --- your freedom.
\end{itemize}

mark landler

Boris Johnson does appear and he delivers this fairly detailed speech.

\begin{itemize}
\tightlist
\item
  archived recording (boris johnson)\\
  So I want to provide, tonight, for you, the shape of a plan, both to
  beat the virus and provide the first sketch of a roadmap for reopening
  society.
\end{itemize}

mark landler

He lays out all sorts of benchmarks and criterias.

\begin{itemize}
\tightlist
\item
  archived recording (boris johnson)\\
  And the first step is a change of emphasis that we hope that people
  will act on this week. We said that you should work from home if you
  can, and only go to work if you must. We now need to stress that
  anyone who can't work from home, for instance, those in construction
  or manufacturing, should be actively encouraged to go to work. And we
  want it to be safe for you to get to work. In step two, at the
  earliest, by June the first, after half term, we believe we may be in
  a position to begin the phased reopening of shops and to get primary
  pupils back into schools.
\end{itemize}

mark landler

But when you actually go back and read his words ---

\begin{itemize}
\tightlist
\item
  archived recording (boris johnson)\\
  And step three, at the earliest by July, and subject to all these
  conditions and further the scientific advice --- if and only if the
  numbers support it --- we will hope to reopen at least some of the
  hospitality industry and other public places, provided they're safe
  and enforce social distancing.
\end{itemize}

mark landler

Really, nothing much has changed. He's given a very dramatic speech in
which he has barely budged the policy.

\begin{itemize}
\tightlist
\item
  archived recording (boris johnson)\\
  And so, no. This is not the time simply to end the lockdown this week.
\end{itemize}

michael barbaro

Huh. So his speech about reopening is really about how Britain is not
about to truly reopen.

mark landler

Yes. This turns out to be a speech that, when you really analyze it, is
about why we are going to continue to live under a lockdown --- one that
might be tweaked at the edges here and there, there might be slight
shifts --- but fundamentally, nothing is changing in this country.

michael barbaro

It's kind of remarkable how everything is flipped. Meantime,
neighboring, more progressive countries like Germany, which took the
virus so seriously from the beginning, are now announcing relatively
aggressive reopening plans, and here we have Britain taking this far
more cautious approach. It all seems quite scrambled, and I wonder what
you make of that.

mark landler

Well, it's interesting. Every country had to make a choice about how to
deal with something that is fundamentally so mysterious and so
unpredictable. And the choices these countries made played out in ways
that were either fortunate and successful, or tragic and largely
unsuccessful. And that in turn has very much driven the way that these
countries have dealt with this moment --- the moment where they lift the
lockdown. And so you have in Germany a country that, as you say, took
the threat extremely seriously at the beginning, was very conservative,
was very cautious, went to a lockdown, took the science of it extremely
seriously. In Germany, the caution paid off. The death tolls stayed low.
And now, as we come to this moment of lifting the lockdown, the Germans
have the confidence to be brave, to take some risks, to lift elements of
the lockdown, even at the risk of kicking up the infection rate. And
conversely, here in England, where they came into this crisis with such
an air of self-confidence, they've now been rattled to their core.
They've really had their confidence shaken. And far from being brave or
bold at this moment, they're now the country that's reacting with
extreme caution. This libertarian country is now ready to keep the heavy
hand of government in place, as long as it takes.

And so you really see a role reversal that reflects how this virus has
struck different countries, how the experiences that they've undergone
have been so starkly different. And as a result, how they're going to
emerge from this period is sometimes extremely unexpected. And in
Britain, it's been perhaps the most unexpected of all. The country that
took this pandemic the least seriously is arguably going to now take it
more seriously than anybody else.

michael barbaro

Mark, thank you very much.

mark landler

Thank you, Michael.

{[}music{]}

michael barbaro

Here's what else you need to know today. On Monday, countries across the
world took some of their biggest steps yet toward easing restrictions on
their citizens' movements.

Spain began permitting small gatherings of up to 10 people, and for
small shops to reopen. France allowed residents to leave their homes
without filling out release forms for the first time in eight weeks.

\begin{itemize}
\tightlist
\item
  archived recording (vladimir putin)\\
  {[}SPEAKING RUSSIAN{]}
\end{itemize}

michael barbaro

And Russia announced the end of nationwide stay at home restrictions,
despite a recent rise in infections. And ---

\begin{itemize}
\tightlist
\item
  archived recording (donald trump)\\
  If you look at all of those people over there, every one of them, from
  what I see, these are White House staffers, they're White House
  representatives, they're White House executives, and everybody has a
  mask on. We've had ---
\end{itemize}

michael barbaro

The White House has instituted a new rule, requiring that all employees
wear masks inside the West Wing after two aides tested positive for the
coronavirus.

\begin{itemize}
\item
  archived recording\\
  Were you the one who required that, sir?
\item
  archived recording (donald trump)\\
  Yeah, I did. I did. I required it, yes.
\end{itemize}

michael barbaro

But the policy is not expected to apply to either the president or the
vice president, who, for weeks have avoided wearing masks. An attitude
that The Times reports have trickled down to staff members, creating a
dangerous situation.

{[}music{]}

That's it for ``The Daily.'' I'm Michael Barbaro. See you tomorrow.

\includegraphics{https://static01.nyt.com/images/2020/05/11/world/11-virus-britain-2/merlin_172382874_3ca2fc07-79c2-4f62-a5ed-738cd320bfe2-articleLarge.jpg?quality=75\&auto=webp\&disable=upscale}

Under Mr. Johnson's
\href{https://assets.publishing.service.gov.uk/government/uploads/system/uploads/attachment_data/file/884171/FINAL_6.6637_CO_HMG_C19_Recovery_FINAL_110520_v2_WEB__1_.pdf}{new
proposals}, announced on Sunday and Monday, those unable to work from
home will be encouraged to return to workplaces --- but also to avoid
public transport.

People will be advised to wear face coverings on buses and trains and in
some stores --- but not obliged to.

\hypertarget{latest-updates-global-coronavirus-outbreak}{%
\section{\texorpdfstring{\href{https://www.nytimes.com/2020/08/01/world/coronavirus-covid-19.html?action=click\&pgtype=Article\&state=default\&region=MAIN_CONTENT_1\&context=storylines_live_updates}{Latest
Updates: Global Coronavirus
Outbreak}}{Latest Updates: Global Coronavirus Outbreak}}\label{latest-updates-global-coronavirus-outbreak}}

Updated 2020-08-02T07:42:09.613Z

\begin{itemize}
\tightlist
\item
  \href{https://www.nytimes.com/2020/08/01/world/coronavirus-covid-19.html?action=click\&pgtype=Article\&state=default\&region=MAIN_CONTENT_1\&context=storylines_live_updates\#link-34047410}{The
  U.S. reels as July cases more than double the total of any other
  month.}
\item
  \href{https://www.nytimes.com/2020/08/01/world/coronavirus-covid-19.html?action=click\&pgtype=Article\&state=default\&region=MAIN_CONTENT_1\&context=storylines_live_updates\#link-780ec966}{Top
  U.S. officials work to break an impasse over the federal jobless
  benefit.}
\item
  \href{https://www.nytimes.com/2020/08/01/world/coronavirus-covid-19.html?action=click\&pgtype=Article\&state=default\&region=MAIN_CONTENT_1\&context=storylines_live_updates\#link-2bc8948}{Its
  outbreak untamed, Melbourne goes into even greater lockdown.}
\end{itemize}

\href{https://www.nytimes.com/2020/08/01/world/coronavirus-covid-19.html?action=click\&pgtype=Article\&state=default\&region=MAIN_CONTENT_1\&context=storylines_live_updates}{See
more updates}

More live coverage:
\href{https://www.nytimes.com/live/2020/07/31/business/stock-market-today-coronavirus?action=click\&pgtype=Article\&state=default\&region=MAIN_CONTENT_1\&context=storylines_live_updates}{Markets}

They will be allowed to exercise more and meet with one other person in
open spaces like parks --- so long as they remain two meters, or roughly
six feet, apart.

There is also a vague timetable for the reopening next month of some
schools, and the possibility of resuming some sporting events behind
closed doors.

Carolyn Fairbairn, director general of the Confederation of British
Industry, a business lobby group, called the prime minister's plans
``the first glimmer of light for our faltering economy'' and said ``a
phased and careful return to work is the only way to protect jobs and
pay for future public services.''

But while the government laid down objectives for easing the lockdown,
trade unions said it left many questions unanswered, including some
relating to the safety of workplaces and transport networks.

Many questions also remained about a plan, likely to be introduced in a
few weeks, to quarantine those flying into Britain.

``The government will require all international arrivals not on a short
list of exemptions to self-isolate in their accommodation for 14 days on
arrival into the U.K.'' a government document said.

It said that ``where international travelers are unable to demonstrate
where they would self-isolate, they will be required to do so in
accommodation arranged by the government.''

The government said that travelers from Ireland and France would be
exempt from the quarantine rules but has not explained how it would make
sure that people just transiting through those two countries to Britain
would be isolated.

Part of Mr. Johnson's latest troubles relate to poor presentation and
media management, perhaps a surprising failing for someone regarded as
an effective political communicator.

Last week he raised expectations of a swift relaxation of some lockdown
measures, prompting excited media speculation that had to be dampened.

Then Mr. Johnson annoyed lawmakers by saying he would announce his new
plan on TV on Sunday, not in Parliament, where he has appeared much less
frequently than his predecessor, Theresa May.

As a concession, the prime minister delayed the publication of the
official document on the new strategy until he appeared in the House of
Commons on Monday. But there followed confusion over differences between
what the government said on Sunday and the document made public on
Monday.

Image

A park in North London on Friday. Last week Mr. Johnson raised
expectations of a swift relaxation of some lockdown
measures.Credit...Andrew Testa for The New York Times

Mr. Johnson's new strategy is notably laissez-faire in its approach to
balancing health and economic risks, leaving more to personal judgment.
That has struck some critics as reminiscent of the early stages of the
pandemic, when halfheartedness characterized Mr. Johnson's approach at
almost every step of the crisis.

\href{https://www.nytimes.com/news-event/coronavirus?action=click\&pgtype=Article\&state=default\&region=MAIN_CONTENT_3\&context=storylines_faq}{}

\hypertarget{the-coronavirus-outbreak-}{%
\subsubsection{The Coronavirus Outbreak
›}\label{the-coronavirus-outbreak-}}

\hypertarget{frequently-asked-questions}{%
\paragraph{Frequently Asked
Questions}\label{frequently-asked-questions}}

Updated July 27, 2020

\begin{itemize}
\item ~
  \hypertarget{should-i-refinance-my-mortgage}{%
  \paragraph{Should I refinance my
  mortgage?}\label{should-i-refinance-my-mortgage}}

  \begin{itemize}
  \tightlist
  \item
    \href{https://www.nytimes.com/article/coronavirus-money-unemployment.html?action=click\&pgtype=Article\&state=default\&region=MAIN_CONTENT_3\&context=storylines_faq}{It
    could be a good idea,} because mortgage rates have
    \href{https://www.nytimes.com/2020/07/16/business/mortgage-rates-below-3-percent.html?action=click\&pgtype=Article\&state=default\&region=MAIN_CONTENT_3\&context=storylines_faq}{never
    been lower.} Refinancing requests have pushed mortgage applications
    to some of the highest levels since 2008, so be prepared to get in
    line. But defaults are also up, so if you're thinking about buying a
    home, be aware that some lenders have tightened their standards.
  \end{itemize}
\item ~
  \hypertarget{what-is-school-going-to-look-like-in-september}{%
  \paragraph{What is school going to look like in
  September?}\label{what-is-school-going-to-look-like-in-september}}

  \begin{itemize}
  \tightlist
  \item
    It is unlikely that many schools will return to a normal schedule
    this fall, requiring the grind of
    \href{https://www.nytimes.com/2020/06/05/us/coronavirus-education-lost-learning.html?action=click\&pgtype=Article\&state=default\&region=MAIN_CONTENT_3\&context=storylines_faq}{online
    learning},
    \href{https://www.nytimes.com/2020/05/29/us/coronavirus-child-care-centers.html?action=click\&pgtype=Article\&state=default\&region=MAIN_CONTENT_3\&context=storylines_faq}{makeshift
    child care} and
    \href{https://www.nytimes.com/2020/06/03/business/economy/coronavirus-working-women.html?action=click\&pgtype=Article\&state=default\&region=MAIN_CONTENT_3\&context=storylines_faq}{stunted
    workdays} to continue. California's two largest public school
    districts --- Los Angeles and San Diego --- said on July 13, that
    \href{https://www.nytimes.com/2020/07/13/us/lausd-san-diego-school-reopening.html?action=click\&pgtype=Article\&state=default\&region=MAIN_CONTENT_3\&context=storylines_faq}{instruction
    will be remote-only in the fall}, citing concerns that surging
    coronavirus infections in their areas pose too dire a risk for
    students and teachers. Together, the two districts enroll some
    825,000 students. They are the largest in the country so far to
    abandon plans for even a partial physical return to classrooms when
    they reopen in August. For other districts, the solution won't be an
    all-or-nothing approach.
    \href{https://bioethics.jhu.edu/research-and-outreach/projects/eschool-initiative/school-policy-tracker/}{Many
    systems}, including the nation's largest, New York City, are
    devising
    \href{https://www.nytimes.com/2020/06/26/us/coronavirus-schools-reopen-fall.html?action=click\&pgtype=Article\&state=default\&region=MAIN_CONTENT_3\&context=storylines_faq}{hybrid
    plans} that involve spending some days in classrooms and other days
    online. There's no national policy on this yet, so check with your
    municipal school system regularly to see what is happening in your
    community.
  \end{itemize}
\item ~
  \hypertarget{is-the-coronavirus-airborne}{%
  \paragraph{Is the coronavirus
  airborne?}\label{is-the-coronavirus-airborne}}

  \begin{itemize}
  \tightlist
  \item
    The coronavirus
    \href{https://www.nytimes.com/2020/07/04/health/239-experts-with-one-big-claim-the-coronavirus-is-airborne.html?action=click\&pgtype=Article\&state=default\&region=MAIN_CONTENT_3\&context=storylines_faq}{can
    stay aloft for hours in tiny droplets in stagnant air}, infecting
    people as they inhale, mounting scientific evidence suggests. This
    risk is highest in crowded indoor spaces with poor ventilation, and
    may help explain super-spreading events reported in meatpacking
    plants, churches and restaurants.
    \href{https://www.nytimes.com/2020/07/06/health/coronavirus-airborne-aerosols.html?action=click\&pgtype=Article\&state=default\&region=MAIN_CONTENT_3\&context=storylines_faq}{It's
    unclear how often the virus is spread} via these tiny droplets, or
    aerosols, compared with larger droplets that are expelled when a
    sick person coughs or sneezes, or transmitted through contact with
    contaminated surfaces, said Linsey Marr, an aerosol expert at
    Virginia Tech. Aerosols are released even when a person without
    symptoms exhales, talks or sings, according to Dr. Marr and more
    than 200 other experts, who
    \href{https://academic.oup.com/cid/article/doi/10.1093/cid/ciaa939/5867798}{have
    outlined the evidence in an open letter to the World Health
    Organization}.
  \end{itemize}
\item ~
  \hypertarget{what-are-the-symptoms-of-coronavirus}{%
  \paragraph{What are the symptoms of
  coronavirus?}\label{what-are-the-symptoms-of-coronavirus}}

  \begin{itemize}
  \tightlist
  \item
    Common symptoms
    \href{https://www.nytimes.com/article/symptoms-coronavirus.html?action=click\&pgtype=Article\&state=default\&region=MAIN_CONTENT_3\&context=storylines_faq}{include
    fever, a dry cough, fatigue and difficulty breathing or shortness of
    breath.} Some of these symptoms overlap with those of the flu,
    making detection difficult, but runny noses and stuffy sinuses are
    less common.
    \href{https://www.nytimes.com/2020/04/27/health/coronavirus-symptoms-cdc.html?action=click\&pgtype=Article\&state=default\&region=MAIN_CONTENT_3\&context=storylines_faq}{The
    C.D.C. has also} added chills, muscle pain, sore throat, headache
    and a new loss of the sense of taste or smell as symptoms to look
    out for. Most people fall ill five to seven days after exposure, but
    symptoms may appear in as few as two days or as many as 14 days.
  \end{itemize}
\item ~
  \hypertarget{does-asymptomatic-transmission-of-covid-19-happen}{%
  \paragraph{Does asymptomatic transmission of Covid-19
  happen?}\label{does-asymptomatic-transmission-of-covid-19-happen}}

  \begin{itemize}
  \tightlist
  \item
    So far, the evidence seems to show it does. A widely cited
    \href{https://www.nature.com/articles/s41591-020-0869-5}{paper}
    published in April suggests that people are most infectious about
    two days before the onset of coronavirus symptoms and estimated that
    44 percent of new infections were a result of transmission from
    people who were not yet showing symptoms. Recently, a top expert at
    the World Health Organization stated that transmission of the
    coronavirus by people who did not have symptoms was ``very rare,''
    \href{https://www.nytimes.com/2020/06/09/world/coronavirus-updates.html?action=click\&pgtype=Article\&state=default\&region=MAIN_CONTENT_3\&context=storylines_faq\#link-1f302e21}{but
    she later walked back that statement.}
  \end{itemize}
\end{itemize}

In early March, he urged people to keep their distance from one another
but declined to ban large public gatherings. He implored people to stay
away from pubs and restaurants but declined to order them to close.

When he switched directions and ordered the lockdown, the government
proved more effective in securing compliance than many expected, in part
by adopting a clear slogan --- ``Stay Home'' --- and urging people to
protect the National Health Service and save lives.

On Monday, Mr. Johnson said Britons had heeded the call to stay home
``more thoroughly than many other populations.''

That slogan has now been scrapped in favor of the vaguer ``Stay Alert.''

Mr. Johnson is under pressure from some of his own lawmakers, who want a
quick reopening to limit what some experts predict could be the severest
damage to the economy in three centuries.

Yet Mr. Johnson knows from personal experience how grueling the virus
can be, having himself been hospitalized with Covid-19. And he wants to
avoid a second spike of infections and a renewed lockdown that could
destroy business confidence.

The government's attempt to balance those considerations has frustrated
not only politicians but also some public health experts. They lamented
the lack of detail in the reopening road map about an integrated program
of testing, contact tracing and isolation for people who are infected.

``I think it is a decent starting point and sets out the challenge
well,'' said Devi Sridhar, director of the global health governance
program at Edinburgh University. ``But it is lacking in clear
communication on what each phase entails and has almost nothing on
isolation.''

Even after the government sets up an extensive testing and tracing
operation, Dr. Sridhar said, it will have to provide support to
households whose members are quarantined. Public-health authorities will
have to monitor people to see whether they develop symptoms.

Others critics pointed to the lack of clarity on whether people should
wear face coverings in public places. Scientific advisers initially
played down their effectiveness in curbing the spread of the virus and
noted that the World Health Organization did not recommend them. But as
France and other countries have begun to require their use, Britain has
reconsidered.

Some of Britain's reluctance has been the result of an acute shortage of
masks and other protective gear for health workers. Officials fear that
if they require the public to wear masks, it could siphon off the supply
for hospitals and especially nursing homes, where the lack of such gear
appears to have aggravated the death toll.

So the advice is voluntary, not compulsory, leaving the final decision
to those who Mr. Johnson hopes will soon start going back to work.

Advertisement

\protect\hyperlink{after-bottom}{Continue reading the main story}

\hypertarget{site-index}{%
\subsection{Site Index}\label{site-index}}

\hypertarget{site-information-navigation}{%
\subsection{Site Information
Navigation}\label{site-information-navigation}}

\begin{itemize}
\tightlist
\item
  \href{https://help.nytimes.com/hc/en-us/articles/115014792127-Copyright-notice}{©~2020~The
  New York Times Company}
\end{itemize}

\begin{itemize}
\tightlist
\item
  \href{https://www.nytco.com/}{NYTCo}
\item
  \href{https://help.nytimes.com/hc/en-us/articles/115015385887-Contact-Us}{Contact
  Us}
\item
  \href{https://www.nytco.com/careers/}{Work with us}
\item
  \href{https://nytmediakit.com/}{Advertise}
\item
  \href{http://www.tbrandstudio.com/}{T Brand Studio}
\item
  \href{https://www.nytimes.com/privacy/cookie-policy\#how-do-i-manage-trackers}{Your
  Ad Choices}
\item
  \href{https://www.nytimes.com/privacy}{Privacy}
\item
  \href{https://help.nytimes.com/hc/en-us/articles/115014893428-Terms-of-service}{Terms
  of Service}
\item
  \href{https://help.nytimes.com/hc/en-us/articles/115014893968-Terms-of-sale}{Terms
  of Sale}
\item
  \href{https://spiderbites.nytimes.com}{Site Map}
\item
  \href{https://help.nytimes.com/hc/en-us}{Help}
\item
  \href{https://www.nytimes.com/subscription?campaignId=37WXW}{Subscriptions}
\end{itemize}
