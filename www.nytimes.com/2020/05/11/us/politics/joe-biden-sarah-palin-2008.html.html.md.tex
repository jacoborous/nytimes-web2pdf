Sections

SEARCH

\protect\hyperlink{site-content}{Skip to
content}\protect\hyperlink{site-index}{Skip to site index}

\href{https://www.nytimes.com/section/politics}{Politics}

\href{https://myaccount.nytimes.com/auth/login?response_type=cookie\&client_id=vi}{}

\href{https://www.nytimes.com/section/todayspaper}{Today's Paper}

\href{/section/politics}{Politics}\textbar{}Joe Biden's Time in Sarah
Palin's Shadow

\url{https://nyti.ms/3fEw0py}

\begin{itemize}
\item
\item
\item
\item
\item
\item
\end{itemize}

\begin{itemize}
\item
  \href{https://www.nytimes.com/2020/07/31/us/elections/biden-vs-trump.html?action=click\&pgtype=Article\&state=default\&region=TOP_BANNER\&context=storylines_menu}{Election
  Updates}
\item
  \href{https://www.nytimes.com/article/biden-vice-president-2020.html?action=click\&pgtype=Article\&state=default\&region=TOP_BANNER\&context=storylines_menu}{Biden's
  V.P. Search}
\item
  \href{https://www.nytimes.com/interactive/2020/07/24/us/politics/trump-biden-campaign-donors.html?action=click\&pgtype=Article\&state=default\&region=TOP_BANNER\&context=storylines_menu}{Map
  of Donations}
\item
  \href{https://www.nytimes.com/interactive/2020/us/elections/delegate-count-primary-results.html?action=click\&pgtype=Article\&state=default\&region=TOP_BANNER\&context=storylines_menu}{Delegate
  Count}
\item
  \href{https://www.nytimes.com/interactive/2019/us/politics/2020-presidential-candidates.html?action=click\&pgtype=Article\&state=default\&region=TOP_BANNER\&context=storylines_menu}{The
  Candidates}
\item
  \href{https://www.nytimes.com/newsletters/politics?action=click\&pgtype=Article\&state=default\&region=TOP_BANNER\&context=storylines_menu}{Politics
  Newsletter}
\end{itemize}

Advertisement

\protect\hyperlink{after-top}{Continue reading the main story}

Supported by

\protect\hyperlink{after-sponsor}{Continue reading the main story}

The Long Run

\hypertarget{joe-bidens-time-in-sarah-palins-shadow}{%
\section{Joe Biden's Time in Sarah Palin's
Shadow}\label{joe-bidens-time-in-sarah-palins-shadow}}

What two strange months in 2008 taught the former vice president about
the politics of grievance, and how that might help him pick a running
mate of his own to take on Donald Trump.

\includegraphics{https://static01.nyt.com/images/2020/04/10/us/politics/00biden-2008-5/00biden-2008-5-articleLarge-v2.jpg?quality=75\&auto=webp\&disable=upscale}

\href{https://www.nytimes.com/by/matt-flegenheimer}{\includegraphics{https://static01.nyt.com/images/2018/10/02/multimedia/author-matt-flegenheimer/author-matt-flegenheimer-thumbLarge.png}}

By \href{https://www.nytimes.com/by/matt-flegenheimer}{Matt
Flegenheimer}

\begin{itemize}
\item
  May 11, 2020
\item
  \begin{itemize}
  \item
  \item
  \item
  \item
  \item
  \item
  \end{itemize}
\end{itemize}

Joe Biden was getting the hang of being overshadowed. It was not a bad
life.

Less than a week had passed since Barack Obama, the Democratic supernova
of 2008, had announced Mr. Biden, a recent presidential also-ran, as his
running mate. And after a well-turned nominating convention in Denver in
late August --- ``This is \emph{his} time,'' Mr. Biden told the crowd,
pumping his fist on the key word, ``this is \emph{our} time'' --- the
two were jetting off on a joint campaign swing when the patter of
breaking news consumed their plane.

John McCain, their Republican opponent, had made his selection for vice
president. Mr. Obama's chief strategist, David Axelrod, briefed the
front of the cabin. Mr. Biden scrunched his face a bit, searching his
mental database:

``Sarah Palin, Sarah Palin,'' he repeated, thinking aloud.

He had nothing to add. ``He couldn't even place the name,'' Mr. Axelrod
recalled.

Neither of these things would happen again.

Twelve years later, with Mr. Biden the presumptive 2020 Democratic
nominee, the frenetic final months of the 2008 race stand as perhaps the
most consequential stretch of his campaign career. It is a chapter at
once critical to understanding Mr. Biden's present thinking, according
to former aides and allies --- a moment, like this one, shadowed by
grave national uncertainty and economic crisis --- and freshly relevant
after his
\href{https://www.nytimes.com/2020/03/15/us/politics/joe-biden-female-vice-president.html}{pledge
in March} to name a woman to the ticket.

Mr. Biden has long described himself as a champion of women, and his
competition with Ms. Palin, the last female vice-presidential nominee of
a major party, is consistent with a public arc in which he has seemed to
figure prominently, Forrest Gump-like, in signal episodes of complicated
gender politics in modern American history.

He led the confirmation hearings of Justice Clarence Thomas, attracting
criticism for his Senate committee's treatment of Anita Hill. He pushed
for the passage of the Violence Against Women Act, which Mr. Biden
\href{https://obamawhitehouse.archives.gov/blog/2014/09/13/vice-president-biden-20-years-ago-today}{has
called} his ``proudest legislative achievement.'' And he and his
campaign have emphasized that work when faced with allegations of
\href{https://www.nytimes.com/2019/04/02/us/politics/joe-biden-women-me-too.html}{unwanted
touching} and, more recently, a
\href{https://www.nytimes.com/2020/04/12/us/politics/joe-biden-tara-reade-sexual-assault-complaint.html}{sexual
assault}, which he has forcefully denied.

Now, as Mr. Biden considers his options for prospective vice presidents,
his position mirrors Mr. McCain's in 2008, to an extent: a
septuagenarian statesman-candidate, primed to face a political celebrity
in the general election, hoping that his choice can inject urgent energy
into his campaign while sending a powerful signal to female voters who
might have hoped to see a woman atop the ballot in November.

It is not lost on Mr. Biden that whomever he chooses might well be
elected the nation's first female president after his turn, or at least
become a new front-runner for the distinction. He has called himself a
``bridge'' to the next generation of Democratic leaders, a transitional
figure whose chief goal is the removal of President Trump. That Mr.
Biden is a 77-year-old man likely to accept the nomination during a
pandemic has attached even weightier stakes to his decision.

In private encounters before this campaign, Mr. Biden has likened
running-mate evaluation to deciding among calendar models, with three
broad categories (and outdated honorifics): Contenders can be a ``Mr.
August'' (a shot of momentum in the summer), a ``Mr. October'' (a
reliable and effective campaigner for the fall) or a ``Mr. January'' (a
governing partner, politics notwithstanding).

Some close to Mr. Biden say that his process will be informed by one
intuitive, if often overlooked, fact: He thinks he was a very good pick
--- a combination of Mr. October and Mr. January, at minimum --- and
views his own blend of résumé and campaign chops with high regard.

``It was a governing pick with political benefits,'' Anita Dunn, a top
adviser to Mr. Biden in 2020 and to Mr. Obama in 2008, said with a
laugh. ``The best kind of governing pick.''

Yet as much as any figure in modern politics, Mr. Biden appreciates the
power and peril of an ``August''-style spectacle: He was once the
Democrat responsible for neutralizing one, while subsisting in her
reflected glow.

The
\href{https://www.nytimes.com/2008/08/30/us/politics/30veep.html}{initial
shock} of Mr. McCain's gambit would be difficult to overstate. Ms.
Palin, then 44, had been on none of the presumed shortlists and was
little known outside Alaska, where she had served as governor for less
than two years after a stint as the mayor of Wasilla, a town with a
population of about 7,000 at the time.

\includegraphics{https://static01.nyt.com/images/2020/04/10/us/politics/00biden-2008-2/merlin_25049191_7de54dee-0087-4e1e-88ab-d21f0b80e398-articleLarge.jpg?quality=75\&auto=webp\&disable=upscale}

Her raw talent at a microphone made her an instant phenomenon, catching
the Obama operation flat-footed. To date, Ms. Palin remains Mr. Biden's
most salient preparation for an adversary like Mr. Trump, gifted in the
politics of grievance and belittling: ``I guess a small-town mayor is
sort of like a community organizer,'' she said at the Republican
National Convention, cutting down Mr. Obama's credentials, ``except that
you have actual responsibilities.''

Still, in Ms. Palin's down-home appeal as an accessible ``hockey mom,''
Mr. Biden also seemed to recognize political kinship of a sort as he
watched her from his campaign bus.

``A lot of people are going to see themselves in her,'' he told aides,
recalling the upset that propelled him to the Senate as a largely
anonymous 29-year-old with an instinct for human connection. ``People
forget how I won in '72.''

\hypertarget{latest-updates-2020-election}{%
\section{\texorpdfstring{\href{https://www.nytimes.com/2020/07/31/us/elections/biden-vs-trump.html?action=click\&pgtype=Article\&state=default\&region=MAIN_CONTENT_1\&context=storylines_live_updates}{Latest
Updates: 2020
Election}}{Latest Updates: 2020 Election}}\label{latest-updates-2020-election}}

Updated 2020-08-01T01:26:45.732Z

\begin{itemize}
\tightlist
\item
  \href{https://www.nytimes.com/2020/07/31/us/elections/biden-vs-trump.html?action=click\&pgtype=Article\&state=default\&region=MAIN_CONTENT_1\&context=storylines_live_updates\#link-29fdff45}{Kamala
  Harris, a top vice-presidential contender, confronts double
  standards.}
\item
  \href{https://www.nytimes.com/2020/07/31/us/elections/biden-vs-trump.html?action=click\&pgtype=Article\&state=default\&region=MAIN_CONTENT_1\&context=storylines_live_updates\#link-13ec3d9c}{Karen
  Bass and Susan Rice are rising on Biden's vice-presidential
  shortlist.}
\item
  \href{https://www.nytimes.com/2020/07/31/us/elections/biden-vs-trump.html?action=click\&pgtype=Article\&state=default\&region=MAIN_CONTENT_1\&context=storylines_live_updates\#link-49e9a016}{Trump
  says Russian bounties to kill U.S. troops `never took place.'}
\end{itemize}

\href{https://www.nytimes.com/2020/07/31/us/elections/biden-vs-trump.html?action=click\&pgtype=Article\&state=default\&region=MAIN_CONTENT_1\&context=storylines_live_updates}{See
more updates}

Mr. Biden's political prognoses did not always find a receptive audience
among Obama advisers, who chafed at his well-earned reputation for loose
talk and strained to fold a self-described ``gut politician'' into their
whirring campaign machine.

Tensions were often more pronounced than Mr. Biden cares to dwell on
today, with trust so mutually fragile at times that he wondered if some
traveling staff members were sending reports about him back to the
Chicago headquarters surreptitiously.

``Are you one of my guys?'' Mr. Biden would ask, according to a person
present. The implication was clear: There were Biden guys and
\emph{their} guys.

In the end, of course, Mr. Biden did what he had signed on to do. He
reassured voters, shaken by financial catastrophe, that Mr. Obama was
ready to lead. He girded himself for the most anticipated (maybe the
only anticipated) vice-presidential debate on record. He wended
dutifully through the kinds of old-guard union towns where he had long
made his electoral living.

Now, Mr. Biden holds up the work that the 2008 campaign made possible as
the most significant of his professional life.

Then, it was not always easy to feel essential.

``Remember,'' he told supporters in Ohio that September, ``no one
decides who they're going to vote for based on the vice president.''

\hypertarget{the-forgotten-candidate}{%
\subsection{The `forgotten candidate'}\label{the-forgotten-candidate}}

Image

Tensions between Mr. Biden and Barack Obama's advisers were often more
pronounced than Mr. Biden cares to dwell on today.Credit...Ozier
Muhammad/The New York Times

The nominee made his case diplomatically.

``I want you to view this as the capstone of your career,'' Mr. Obama
said when he asked Mr. Biden to run with him, according to Mr. Biden's
\href{https://www.nytimes.com/2009/03/29/us/politics/29biden.html}{retelling
at the time}.

``And not the tombstone,'' Mr. Biden clarified.

It was a strange fit, in theory: a Washington veteran who cherished his
committee seniority and a hyper-disciplined prodigy a few years removed
from the Illinois State Senate.

In fact, Obama advisers viewed the presumed limitations of Mr. Biden's
long-term political future as an asset. He would be 74 by the end of Mr.
Obama's second term, hardly the profile of a natural heir, ostensibly
making Mr. Biden less likely to prize executive ambition over day-to-day
loyalty.

But the merger was not frictionless. Speaking in New Hampshire weeks
after being chosen, Mr. Biden
\href{https://www.nytimes.com/2008/09/11/world/americas/11iht-biden.4.16081515.html}{suggested}
that Hillary Clinton, Mr. Obama's leading rival from the primary,
``might have been a better pick than me.''

Another comparison was dodgier: Mr. Biden, believing he was speaking in
confidence, told reporters on his plane that he was more qualified than
Mr. Obama to be president, reasoning that he would not have run in the
primary himself if he thought otherwise.

When such flourishes distressed Obama advisers, Mr. Biden --- who had
been his own boss for decades --- smarted at having his political
intuition questioned or overruled, former aides say.

Ms. Dunn, the top Biden 2020 strategist who worked for Mr. Obama 12
years ago, played down any squabbles without dismissing them entirely.
``There may have been a couple of times, but there was never anything
major,'' she said of Mr. Biden's frustrations. ``If he had conflicts, he
never took them public.''

The factions lurched toward compromise. Early on, some on the Obama team
watched incredulously as Mr. Biden began his remarks with exhaustive
thank-you lists, sparing no detail in saluting the town, local leaders,
the fire marshal on hand. This, to the campaign's eye, gave cable
networks every reason to cut away from Mr. Biden, who valued the
reaction in the room over telegenic considerations.

Mr. Biden agreed to add an accessory to his speech materials for certain
addresses: a stopwatch, ticking away in front of him as he spoke. He
became helpfully competitive with himself, proudly marking any triumphs
over long-windedness.

``Twelve minutes!'' Mr. Biden boasted once, coming offstage and waving
the timer. (Such brevity more often eluded him.)

Mr. Obama's staff also came to admire Mr. Biden's skills as a retail
campaigner, sending him across the Industrial Midwest as a kind of
ambassador to the white working class.

``I tell you, man, this is nice,'' Mr. Biden said in Michigan, revving a
Mustang engine at an auto plant.

``I'm dripping here, man,'' he reported in Ohio, carrying a vanilla ice
cream cone out of a diner.

``Get her on the phone, man!'' he urged well-wishers on his rope lines,
whenever they informed him that their mothers loved him.

Mr. Biden's crowds were respectable and often animated enough. But there
was no comparison to Ms. Palin.

``She was like a fireworks display in full technicolor,'' Mr. Axelrod
said. ``And he was kind of your standard vice-presidential candidate.''

One Pew analysis, comparing Mr. Biden's media coverage to his
counterpart's,
\href{https://www.journalism.org/2008/09/15/pej-campaign-coverage-index-september-8-14-2008/}{labeled}
him ``the virtually forgotten candidate.'' His campaign plane was
sometimes so neglected that reporters could commandeer full rows for
naps.

Privately, some in the party tossed around Mr. Biden's self-deprecating
assessment themselves: Might Mrs. Clinton have been a better pick?

Ms. Palin was making an unsubtle --- and, Democrats insisted, reductive
--- play for Clinton voters, invoking her by name despite their
disparate views and telling supporters that ``the women of America
aren't finished yet.''

During Ms. Palin's grand introduction at the Republican National
Convention, a group of former Clinton aides in Chicago listened with
alarm.

``I remember sitting on my couch with other former Hillary staffers who
had joined the Obama campaign,'' said Patti Solis Doyle, Mr. Biden's
chief of staff for the general election. ``We all sort of looked at each
other and said, `Uh oh.'''

\hypertarget{the-rare-female}{%
\subsection{`The rare female'}\label{the-rare-female}}

Image

Ms. Palin~``was like a fireworks display in full technicolor,'' said
David Axelrod, Mr. Obama's chief strategist at the time, adding that Mr.
Biden ``was kind of your standard vice-presidential
candidate.''Credit...Todd Heisler/The New York Times

There was an upside to Ms. Palin's rise.

Mr. Biden --- eager for a meaningful portfolio, a vital task to perform
--- had one in front of him now: He would have to debate a dynamic
younger woman, in a suddenly much-hyped mega-event, while both avoiding
condescension and holding her to account.

And the political press seemed unconvinced that he was capable.

``I remember him kind of laughing at the way that the question kept
coming at him,'' said David Wade, then a traveling communications aide
to Mr. Biden. ``It sort of sounded as if it was like a National
Geographic expedition to confront the rare female.''

Mr. Biden found the skepticism bizarre. To his mind, he had spent much
of his Senate life in the company of accomplished women, even if the
Capitol remained male-dominated on balance.

If he was visited by any self-doubt dating to the Justice Thomas
hearings of 1991, for which Mr. Biden
\href{https://www.nytimes.com/2019/04/26/us/politics/anita-hill-biden-clarence-thomas.html}{has
faced considerable blowback} over his committee's posture toward Ms.
Hill, aides did not recall it.

Still, Mr. Biden was often uncharacteristically careful when discussing
Ms. Palin on the trail, rarely mentioning her explicitly and conceding
that their meeting might be fraught.

``Are there pitfalls? Yeah, there are pitfalls if two people of
different genders or races, different ethnicities, debate one another,''
he said in Wisconsin. ``Either person may say something that comes off
the wrong way.''

Mr. Biden seemed liable to be that person. At one event in Ohio, he had
summarized their differences
\href{http://blogs.reuters.com/talesfromthetrail/2008/08/31/difference-between-biden-and-palin-shes-good-looking/}{like
this}: ``She's good-looking.''

Ms. Palin's gaffes were of a different type. In the weeks after her
rousing debut, she appeared out of her depth on many basic questions of
policy and readiness, most memorably staggering through an
\href{https://www.youtube.com/watch?v=-ZVh_u5RyiU}{interview with Katie
Couric} in which Ms. Palin declined to name a news source she read.
(Attempts to reach Ms. Palin were unsuccessful.)

Image

After a rousing debut, Ms. Palin appeared out of her depth on questions
of policy and readiness.~Credit...Jim Wilson/The New York Times

In recent years, Mr. Biden has been loath to criticize Mr. McCain, a
genuine friend from the Senate whom
\href{https://www.nytimes.com/2018/08/30/us/politics/john-mccain-memorial.html}{he
eulogized in 2018}. But former aides say Mr. Biden seemed to lament that
his Republican peer had, in Mr. Biden's estimation, ignored his better
judgment in potentially placing Ms. Palin a 72-year-old's heartbeat away
from the presidency.

Ms. Palin's stumbles also made Mr. Biden's debate challenge more
delicate, lowering expectations for her and ensuring that even a
technically proficient showing from him would be appraised harshly if he
seemed patronizing.

As their forum in early October approached, Mr. Biden and his team
gathered at a hotel prep space in Delaware, calling in Jennifer
Granholm, then the Michigan governor, to play Ms. Palin.

Sprinkling her answers with known Palin flourishes like ``you betcha''
or a well-placed wink, Ms. Granholm also slipped in mistakes or outright
nonsense to test Mr. Biden's restraint.

``We had to take a step back and ask, `What does wiping the floor with
her look like to the American people?''' Ms. Solis Doyle said. ``There
were several women in the debate prep, and he very much wanted to hear
from us: `How does that sound? How does that look?'''

In one session, Ms. Granholm made inexplicable reference to ``Guess
Who's Coming to Dinner,'' the 1967 film about the discomfort of a
well-to-do liberal white couple whose daughter plans to marry a black
man.

Almost any response from Mr. Biden might have landed awkwardly. So he
trained himself, at least this once, to eliminate the risk.

``Well,'' Mr. Biden replied eventually, ``I guess I'll just leave it at
that.''

\hypertarget{sarah-and-joe}{%
\subsection{Sarah and Joe}\label{sarah-and-joe}}

Image

``Can I call you Joe?'' Ms. Palin asked Mr. Biden as the two met onstage
in St. Louis for the 2008 vice presidential debate.Credit...Richard
Perry/The New York Times

The first round went to Ms. Palin.

``Can I call you Joe?'' she asked as the two met onstage in St. Louis, a
disarming touch for a national newcomer. Mr. Biden gave his blessing.

``Candidly, I kind of liked it,'' recalled Terrell McSweeny, a Biden
aide at the time. ``Like, `Wow, good move.'''

It was not Ms. Palin's only triumph of style. Though she frequently
retreated to talking points, Ms. Palin
\href{https://www.nytimes.com/elections/2008/president/debates/transcripts/vice-presidential-debate.html}{spoke}
inclusively of ``Main Streeters like me,'' ``Joe Sixpack'' and ``hockey
moms across the nation.''

Ms. Palin's mistakes came mostly when she strayed from such rhetorical
comforts. Discussing counterinsurgency strategy abroad, she repeatedly
referred to Gen. David D. McKiernan, the top American commander in
Afghanistan, as ``McClellan,'' the surname of a 19th-century general.

Mr. Biden declined to correct her, or even use the name himself, for
fear of appearing litigious. ``Well,'' he began, after Ms. Palin
finished, ``our commanding general did say that.''

By the end of a mostly cordial 90 minutes, each side seemed pleased. Ms.
Palin had avoided disaster, defying predictions of full-scale
embarrassment, and Mr. Biden was a faithful messenger for a campaign
focused on presenting Mr. McCain as out of step with the times.

``He understood what his job was in that debate,'' Ms. Dunn said. ``It
was not to make it about her and not to let her make it about her.''

That weekend on ``Saturday Night Live'' --- where Tina Fey's skewering
of Ms. Palin came to define her in the public consciousness more than
any opponent could --- a
\href{https://www.nbc.com/saturday-night-live/video/vp-debate-open-palin--biden/n12319}{fictionalized
Mr. Biden} enjoyed a victory lap.

``My goal tonight was a simple one: to come up here and at no point seem
like a condescending, egomaniacal bully,'' said the character, played by
Jason Sudeikis. ``And I'm gonna be honest: I think I nailed it.''

The real Mr. Biden was more gracious in public. But stepping offstage
that night in St. Louis, something was still eating at him.

He had cleared his most significant obstacle of the fall. Advisers were
congratulating him on his performance. At last, Mr. Biden could unburden
himself:

``McClellan,'' he told an aide, discreetly enough, ``was a Civil War
general.''

Kitty Bennett contributed research.

\hypertarget{our-2020-election-guide}{%
\section{Our 2020 Election Guide}\label{our-2020-election-guide}}

Updated July 31, 2020

\begin{itemize}
\item
  \begin{center}\rule{0.5\linewidth}{\linethickness}\end{center}

  \hypertarget{the-latest}{%
  \subsection{The Latest}\label{the-latest}}

  \begin{itemize}
  \tightlist
  \item
    President Trump's assault on the Postal Service is intersecting with
    his attacks on mail-in voting.
    \href{https://www.nytimes.com/2020/07/31/us/politics/trump-usps-mail-delays.html?action=click\&pgtype=Article\&state=default\&region=BELOW_MAIN_CONTENT\&context=storylines_guide}{Voting
    rights groups say it is a recipe for disaster.}
  \end{itemize}
\item
  \begin{center}\rule{0.5\linewidth}{\linethickness}\end{center}

  \hypertarget{bidens-vp-search}{%
  \subsection{Biden's V.P. Search}\label{bidens-vp-search}}

  \begin{itemize}
  \tightlist
  \item
    \href{https://www.nytimes.com/article/biden-vice-president-2020.html?action=click\&pgtype=Article\&state=default\&region=BELOW_MAIN_CONTENT\&context=storylines_guide}{Here
    are 13 women} who have been under consideration to be Joe Biden's
    running mate, and why each might be chosen --- and might not be.
  \end{itemize}
\item
  \begin{center}\rule{0.5\linewidth}{\linethickness}\end{center}

  \hypertarget{keep-up-with-our-coverage}{%
  \subsection{Keep Up With Our
  Coverage}\label{keep-up-with-our-coverage}}

  \begin{itemize}
  \tightlist
  \item
    Get an
    \href{https://www.nytimes.com/newsletters/politics?action=click\&pgtype=Article\&state=default\&region=BELOW_MAIN_CONTENT\&context=storylines_guide}{email}
    recapping the day's news
  \end{itemize}

  \begin{itemize}
  \tightlist
  \item
    Download our mobile app on
    \href{https://apps.apple.com/us/app/nytimes/id284862083?ls=1\&mat_click_id=5c79ae7455014fd1bd66b5610c05b8f2-20191112-16948\&referrer=mat_click_id\%3D5c79ae7455014fd1bd66b5610c05b8f2-20191112-16948\%26link_click_id\%3D722930677036718082}{iOS}
    and
    \href{http://a.localytics.com/android?id=com.nytimes.android\&referrer=utm_source\%3Dother_nyt_mobile_web\%26utm_medium\%3DWeb\%2520page\%26utm_term\%3DGeneral\%2520Mobile\%2520Page\%26utm_campaign\%3DNYT\%2520Mobile\%2520General\%2520Page}{Android}
    and turn on Breaking News and Politics alerts
  \end{itemize}
\end{itemize}

Advertisement

\protect\hyperlink{after-bottom}{Continue reading the main story}

\hypertarget{site-index}{%
\subsection{Site Index}\label{site-index}}

\hypertarget{site-information-navigation}{%
\subsection{Site Information
Navigation}\label{site-information-navigation}}

\begin{itemize}
\tightlist
\item
  \href{https://help.nytimes.com/hc/en-us/articles/115014792127-Copyright-notice}{©~2020~The
  New York Times Company}
\end{itemize}

\begin{itemize}
\tightlist
\item
  \href{https://www.nytco.com/}{NYTCo}
\item
  \href{https://help.nytimes.com/hc/en-us/articles/115015385887-Contact-Us}{Contact
  Us}
\item
  \href{https://www.nytco.com/careers/}{Work with us}
\item
  \href{https://nytmediakit.com/}{Advertise}
\item
  \href{http://www.tbrandstudio.com/}{T Brand Studio}
\item
  \href{https://www.nytimes.com/privacy/cookie-policy\#how-do-i-manage-trackers}{Your
  Ad Choices}
\item
  \href{https://www.nytimes.com/privacy}{Privacy}
\item
  \href{https://help.nytimes.com/hc/en-us/articles/115014893428-Terms-of-service}{Terms
  of Service}
\item
  \href{https://help.nytimes.com/hc/en-us/articles/115014893968-Terms-of-sale}{Terms
  of Sale}
\item
  \href{https://spiderbites.nytimes.com}{Site Map}
\item
  \href{https://help.nytimes.com/hc/en-us}{Help}
\item
  \href{https://www.nytimes.com/subscription?campaignId=37WXW}{Subscriptions}
\end{itemize}
