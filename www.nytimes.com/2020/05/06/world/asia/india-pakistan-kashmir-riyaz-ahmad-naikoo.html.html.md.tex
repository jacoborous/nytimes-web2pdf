Sections

SEARCH

\protect\hyperlink{site-content}{Skip to
content}\protect\hyperlink{site-index}{Skip to site index}

\href{https://www.nytimes.com/section/world/asia}{Asia Pacific}

\href{https://myaccount.nytimes.com/auth/login?response_type=cookie\&client_id=vi}{}

\href{https://www.nytimes.com/section/todayspaper}{Today's Paper}

\href{/section/world/asia}{Asia Pacific}\textbar{}Top Kashmiri Militant
Is Killed, Sparking Protests and Rage

\url{https://nyti.ms/2Wwa9b8}

\begin{itemize}
\item
\item
\item
\item
\item
\end{itemize}

Advertisement

\protect\hyperlink{after-top}{Continue reading the main story}

Supported by

\protect\hyperlink{after-sponsor}{Continue reading the main story}

\hypertarget{top-kashmiri-militant-is-killed-sparking-protests-and-rage}{%
\section{Top Kashmiri Militant Is Killed, Sparking Protests and
Rage}\label{top-kashmiri-militant-is-killed-sparking-protests-and-rage}}

Over years of fighting, Riyaz Ahmad Naikoo recruited scores of young
Kashmiris in an armed quest for independence from India. His death has
set off a fresh wave of unrest.

\includegraphics{https://static01.nyt.com/images/2020/05/06/world/00kashmir-1/00kashmir-1-articleLarge-v2.jpg?quality=75\&auto=webp\&disable=upscale}

By \href{https://www.nytimes.com/by/sameer-yasir}{Sameer Yasir},
\href{https://www.nytimes.com/by/kai-schultz}{Kai Schultz} and Iqbal
Kirmani

\begin{itemize}
\item
  May 6, 2020
\item
  \begin{itemize}
  \item
  \item
  \item
  \item
  \item
  \end{itemize}
\end{itemize}

NEW DELHI --- Before disappearing into the woods almost a decade ago,
Riyaz Ahmad Naikoo was a schoolteacher living a relatively quiet life in
Kashmir.

But then he resurfaced in videos on social media sitting next to
automatic weapons and grenades, rebranding himself as a fighter, and
demanding that India pull out of the disputed Himalayan territory.

\href{https://indianexpress.com/article/india/top-militant-commander-trapped-encounter-ongoing-in-jk/}{Mr.
Naikoo's quest came to a violent end on Wednesday.} Indian security
forces surrounded the small village where he was hiding in the district
of Pulwama, dug trenches to prevent civilians from coming to his rescue,
and shot him dead.

The gunning down of Mr. Naikoo, a senior leader in Hizbul Mujahideen, a
banned militant group, was one of the most significant recent victories
for Indian troops fighting the secession movement in Kashmir, a rugged,
contested territory between India and Pakistan.

His death has roiled the region, casting the insurgency into a
leaderless future and immediately heightening tensions between Indian
security forces and a local population that has often supported a brutal
fight for independence.

By late Wednesday,
\href{https://thekashmirwalla.com/2020/05/pulwama-gunfight-14-civilians-injured-in-clashes/}{violent
protests had broken out in Pulwama over his death}, and at least 14
people were injured.

``The insurgency in Kashmir could nose dive now,'' said Aijaz Ahmad
Wani, a professor of political science at Kashmir University. ``There
has been a period of highs and lows, and this is one of the lowest
points in the last 30 years.''

In recent months, Kashmir has weathered a grim resurgence of violence,
with India and Pakistan trading gunfire along the Line of Control, which
divides the territory. Last weekend, a battle between the Indian Army
and militants
\href{https://www.nytimes.com/2020/05/03/world/asia/kashmir-gun-battle-army-militants.html}{left
seven people dead}.

\includegraphics{https://static01.nyt.com/images/2020/05/06/world/00kashmir-2/00kashmir-2-articleLarge.jpg?quality=75\&auto=webp\&disable=upscale}

At least 50 militants and 20 soldiers have been killed this year,
according to data collected by the South Asia Terrorism Portal, a
research project that tracks casualties in the conflict.

For decades, Kashmir, a stunningly beautiful valley in the Himalayas,
has been caught in the throes of the conflict between Hindu-majority
India, which controls much of the territory, and Muslim-majority
Pakistan.

Three wars between the countries have claimed tens of thousands of
lives. A secession movement that started in the late 1980s saw many
guerrilla fighters crossing the border from Pakistan into India.

Mr. Naikoo, who was in his 30s, was perhaps the most powerful militant
leader in Kashmir today.

After becoming a fighter in 2012, he steadily built a loyal base of
followers, dressing up in army fatigues and spreading his message of
resistance on social media. When he attended the funerals of other
militants, civilians young and old called him ``master,'' and jostled
through the crowds just to touch him.

He also boosted support for an independence movement among the masses,
taking control of Hizbul Mujahideen in Kashmir after the death of a
former leader, Burhan Wani,
\href{https://www.nytimes.com/2016/07/17/world/asia/how-killing-of-prominent-separatist-set-off-turmoil-in-kashmir.html}{who
was gunned down by Indian troops in 2016.}

After Mr. Wani was killed, thousands of Kashmiris poured into the
streets to protest, pelting stones at the Indian authorities,
\href{https://www.nytimes.com/2016/08/29/world/asia/pellet-guns-used-in-kashmir-protests-cause-dead-eyes-epidemic.html}{who
used pump-action shotguns to fire lead pellets into the crowds, blinding
many}.

Mr. Naikoo filled the vacuum in power, encouraging young Kashmiris to
quit their jobs, leave their homes and take up arms.

Image

The village of Beighpora in the Pulwama district, on
Wednesday.Credit...Reuters

They hid in Kashmir's forests, occasionally emerging to attack Indian
security forces. Most lost their lives within months of joining the
insurgency, though Mr. Naikoo has said that violence was not the goal.

\href{https://www.aljazeera.com/news/2018/11/qa-hizbul-mujahideen-leader-surrender-181109111719903.html}{In
a 2018 interview with Al Jazeera}, Mr. Naikoo said the death of a family
member at the hands of Indian security forces had prompted his embrace
of militancy, along with India's ``repression'' of Kashmir's right to
protest.

``Yes, we have chosen the path of armed struggle, but primarily, we are
for peace, not war,'' he said. ``It is the nature of the occupying
Indian state that has compelled us to resort to violent methods of
resistance.''

Last August, Kashmir spiraled into uncertainty once again, when the
Indian government
\href{https://www.nytimes.com/2019/08/05/world/asia/india-pakistan-kashmir-jammu.html}{revoked
the region's partial autonomy}. Since then, many have been on edge in
the Kashmir Valley, where businesses were shuttered, streets cleared and
Indian security forces moved to find militants.

On Tuesday, an eight-year hunt for Mr. Naikoo, who was at the top of the
government's most wanted list, reached its apex.

Working off a tip that he was hiding in the village of Beighpora, in
southern Kashmir, Indian army and police officers surrounded the area to
prevent him from slipping away. According to witnesses and videos shared
with The New York Times, they used bulldozers to dig deep trenches at
night, searching for possible underground hide outs.

By Wednesday morning, they had traced him to a house in the village.
During a shootout, officials said, Mr. Naikoo was killed, along with
another militant. The authorities
\href{https://www.thehindu.com/news/national/other-states/mobile-internet-suspended-in-kashmir-as-top-hizb-commander-riaz-naikoo-trapped-in-encounter/article31515428.ece}{cut
internet} in the Kashmir Valley, but protests quickly broke out.

In a statement, the Jammu and Kashmir police heralded his death as a
major victory, and implicated Mr. Naikoo in the deaths and abductions of
at least a dozen civilians and security officers. ``He resorted to
brutal killings,'' it read.

But many Kashmiris said they saw Mr. Naikoo differently.

Shortly after his death, an angry crowd of men surrounded a police
vehicle near Beighpora, climbing atop it and battering its windows with
wooden poles and sheets of tin.

Irfan Mir, a student living in Beighpora, said Mr. Naikoo's death was
the beginning of another wave of resistance. When one militant dies, he
said, ``another will be there to take his place.''

``He was a hero, not just here, but in all of Kashmir,'' Mr. Mir said.
``He will live in our hearts forever.''

Sameer Yasir and Kai Schultz reported from New Delhi, and Iqbal Kirmani
from Pulwama, Kashmir.

Advertisement

\protect\hyperlink{after-bottom}{Continue reading the main story}

\hypertarget{site-index}{%
\subsection{Site Index}\label{site-index}}

\hypertarget{site-information-navigation}{%
\subsection{Site Information
Navigation}\label{site-information-navigation}}

\begin{itemize}
\tightlist
\item
  \href{https://help.nytimes.com/hc/en-us/articles/115014792127-Copyright-notice}{©~2020~The
  New York Times Company}
\end{itemize}

\begin{itemize}
\tightlist
\item
  \href{https://www.nytco.com/}{NYTCo}
\item
  \href{https://help.nytimes.com/hc/en-us/articles/115015385887-Contact-Us}{Contact
  Us}
\item
  \href{https://www.nytco.com/careers/}{Work with us}
\item
  \href{https://nytmediakit.com/}{Advertise}
\item
  \href{http://www.tbrandstudio.com/}{T Brand Studio}
\item
  \href{https://www.nytimes.com/privacy/cookie-policy\#how-do-i-manage-trackers}{Your
  Ad Choices}
\item
  \href{https://www.nytimes.com/privacy}{Privacy}
\item
  \href{https://help.nytimes.com/hc/en-us/articles/115014893428-Terms-of-service}{Terms
  of Service}
\item
  \href{https://help.nytimes.com/hc/en-us/articles/115014893968-Terms-of-sale}{Terms
  of Sale}
\item
  \href{https://spiderbites.nytimes.com}{Site Map}
\item
  \href{https://help.nytimes.com/hc/en-us}{Help}
\item
  \href{https://www.nytimes.com/subscription?campaignId=37WXW}{Subscriptions}
\end{itemize}
