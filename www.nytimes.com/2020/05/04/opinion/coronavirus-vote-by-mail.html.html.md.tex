Sections

SEARCH

\protect\hyperlink{site-content}{Skip to
content}\protect\hyperlink{site-index}{Skip to site index}

\href{https://myaccount.nytimes.com/auth/login?response_type=cookie\&client_id=vi}{}

\href{https://www.nytimes.com/section/todayspaper}{Today's Paper}

\href{/section/opinion}{Opinion}\textbar{}We Should Never Have to Vote
in Person Again

\url{https://nyti.ms/2SxZrzM}

\begin{itemize}
\item
\item
\item
\item
\item
\end{itemize}

Advertisement

\protect\hyperlink{after-top}{Continue reading the main story}

\href{/section/opinion}{Opinion}

Supported by

\protect\hyperlink{after-sponsor}{Continue reading the main story}

\hypertarget{we-should-never-have-to-vote-in-person-again}{%
\section{We Should Never Have to Vote in Person
Again}\label{we-should-never-have-to-vote-in-person-again}}

Our new paper shows vote-by-mail dramatically boosted Colorado's
turnout. It could change this fall's election and every election after.

By Charlotte Hill, Jacob Grumbach, Adam Bonica and Hakeem Jefferson

The authors are political science researchers.

\begin{itemize}
\item
  May 4, 2020
\item
  \begin{itemize}
  \item
  \item
  \item
  \item
  \item
  \end{itemize}
\end{itemize}

\includegraphics{https://static01.nyt.com/images/2020/05/04/opinion/04hill1/merlin_169966806_380ec337-19e0-445b-bcbf-5d7a60e33301-articleLarge.jpg?quality=75\&auto=webp\&disable=upscale}

\hypertarget{mailing-every-coloradan-a-ballot-increased-voting-across-all-age-groups}{%
\subsection{Mailing every Coloradan a ballot increased voting across all
age
groups}\label{mailing-every-coloradan-a-ballot-increased-voting-across-all-age-groups}}

Mailing every Coloradan a ballot increased voting across all age groups

100\%

Actual 2018 turnout

80

Projected turnout

without all-mail voting

60

Turnout increased the most for young voters

40

30

40

50

60

70

80

90

Mailing every Coloradan a ballot increased voting across all age groups

100\%

Actual 2018 turnout

Projected turnout

without all-mail voting

80

60

Turnout increased the most for young voters

40

30

35

40

45

50

55

60

65

70

75

80

85

90

Gen X

Silent Generation

Millennials

Boomers

The New York Times·Source: Research by Adam Bonica, Jacob M. Grumbach,
Charlotte HIll and Hakeem Jefferson

As the coronavirus continues to spread throughout the country, there are
growing concerns about whether in-person voting can be conducted safely
in the months ahead, including for the November presidential election. A
huge expansion of
\href{https://www.nytimes.com/2020/06/19/us/politics/nyc-vote-by-mail.html}{mail
voting} is one way to ensure that participating in democracy won't
undermine public health.

The idea of ``all-mail voting'' is straightforward: Every registered
voter gets sent a ballot via mail to their home address, then after
making their choices,
\href{https://www.nytimes.com/2020/05/25/us/vote-by-mail-coronavirus.html}{voters
mail} it back; and those who want to still travel to vote in person can
do so. In the midst of this pandemic, it's an adjustment that every
state legislature should try to make.

But should we expand mail voting beyond the Covid-19 crisis?

Nathaniel Persily, a professor at Stanford Law School, and Charles
Stewart III, a professor of political science at M.I.T., argue in a
recent
\href{https://www.lawfareblog.com/ten-recommendations-ensure-healthy-and-trustworthy-2020-election}{article},
``States should approach this situation as an emergency, not as an
opportunity to make long-term changes to election policy.'' We disagree.

\href{https://www.dropbox.com/s/8n4zjvgmytim1rv/Bonica_Grumbach_Hill_Jefferson_Mail_Voting.pdf?raw=1}{Our
new research, published yesterday, shows} that elections with all-mail
voting increase turnout among everyone, especially groups that tend to
vote less frequently. Those results merit permanent, wide-scale shifts.
Currently, registered voters automatically get a ballot by mail in five
states: Oregon, Washington, Utah, Colorado and Hawaii. A few other
states have
\href{https://www.ncsl.org/research/elections-and-campaigns/all-mail-elections.aspx}{all-mail
voting in small jurisdictions}, and California has been gradually
rolling it out.

Before this year, the results of research into all-mail voting's turnout
effect had been mixed. Past studies of all-mail voting, mostly of its
early years in Oregon and California, argued that it does boost turnout,
but mainly for those who already vote. If that remained true, then mail
voting could actually exacerbate present inequalities in political
participation.

So when we began
\href{https://www.dropbox.com/s/8n4zjvgmytim1rv/Bonica_Grumbach_Hill_Jefferson_Mail_Voting.pdf?raw=1}{our
research}, we wouldn't have been surprised by unequal outcomes. Young
people,
\href{https://www.nytimes.com/2020/04/12/opinion/biden-sanders-young-voters.html}{notorious
for their low turnout rates}, use traditional mail less than other
groups. And people of color --- who have been subjected to centuries of
voter discrimination --- might be skeptical of adopting big changes to
an electoral system that has disadvantaged them.

\href{https://www.dropbox.com/s/8n4zjvgmytim1rv/Bonica_Grumbach_Hill_Jefferson_Mail_Voting.pdf?raw=1}{Our
findings} show, however, that low-turnout groups are the very groups
that stand to benefit most from all-mail voting. Focusing on Colorado's
recent switch to vote-by-mail in 2013 and using
\href{https://www.pewresearch.org/fact-tank/2018/02/15/voter-files-study-qa/}{the
voter file} --- a comprehensive record of who turns out in American
elections --- we find that turnout goes up among everyone, especially
the historically disenfranchised: young people, voters of color,
less-educated people and blue-collar workers.

In Colorado, a traditional swing state, ballots are mailed to all
registered voters, who can then choose to mail back their completed
ballot or drop it in one of many secure collection boxes. (Denver alone
has about 30 throughout the city.) Or voters take it to a county vote
center, staffed with personnel, to cast their ballot in person. Vote
centers are open during an early voting period as well as on Election
Day.

To study the effect of all-mail voting in Colorado,
\href{https://www.dropbox.com/s/8n4zjvgmytim1rv/Bonica_Grumbach_Hill_Jefferson_Mail_Voting.pdf?raw=1}{we
first looked at how turnout changed} after the state instituted all-mail
voting. We then compared that with how turnout changed during the same
period in similar nearby states. We found that all-mail voting has a
tremendously large effect, boosting overall voting rates in Colorado by
more than 9 percentage points.

\hypertarget{less-wealthy-coloradans-benefited-most-from-all-mail-voting}{%
\subsection{Less wealthy Coloradans benefited most from all-mail
voting}\label{less-wealthy-coloradans-benefited-most-from-all-mail-voting}}

Which Coloradans benefited most from

all-mail voting

Difference between actual and projected turnout in Colorado's 2018
election, by race, educational attainment and household wealth

0

+5

+10 pct. points

African-American

Asian

Latino

Larger

increase

in turnout

White

\$0-5k household

wealth

\$5-10k

\$10-25k

\$25-50k

\$100-250k

\$50-100k

\$250-500k

More than \$500k

Less than high school diploma

High school

diploma

Some college

Bachelor's degree

Graduate degree

0

+5

+10

Which Coloradans benefited most from all-mail voting

\$0-5k household

wealth

Difference between actual and projected turnout in Colorado's 2018
election,

by race, educational attainment and household wealth

\$5-10k

0

+5

+10

\$10-25k

Less than high school diploma

0

+5

+10 pct. points

\$25-50k

High school diploma

\$100-250k

African-American

Some college

\$50-100k

Asian

Bachelor's degree

\$250-500k

Latino

Larger

increase

in turnout

Graduate degree

More than \$500k

White

Which Coloradans benefited most from all-mail voting

Difference between actual and projected turnout in Colorado's 2018
election, by race, educational attainment and household wealth

0

+2

+4

+6

+8

+10

+12 pct. points

African-American

Asian

Latino

Higher turnout

due to all-mail voting

White

\$0-5k household

wealth

\$5-10k

\$10-25k

\$25-50k

\$100-250k

\$50-100k

\$250-500k

More than \$500k

Less than high school diploma

High school diploma

Some college

Bachelor's degree

Graduate degree

The New York Times·Source: Research by Adam Bonica, Jacob M. Grumbach,
Charlotte HIll and Hakeem Jefferson

But the good news in the state doesn't stop there. Under all-mail
voting,
\href{https://www.dropbox.com/s/8n4zjvgmytim1rv/Bonica_Grumbach_Hill_Jefferson_Mail_Voting.pdf?raw=1}{youth
turnout increases by 16 percentage points}. Blue-collar workers see a 10
percentage-point jump in turnout. People without a high school diploma
are 9.6 percentage points more likely to vote. And voters of color
benefit immensely:
\href{https://www.dropbox.com/s/8n4zjvgmytim1rv/Bonica_Grumbach_Hill_Jefferson_Mail_Voting.pdf?raw=1}{Our
research} finds a 13 percentage-point turnout boost for
African-Americans, a 10 percentage-point boost for Latino voters and an
11 percentage-point increase for Asian-Americans.

All-mail voting also helps
\href{https://www.dropbox.com/s/8n4zjvgmytim1rv/Bonica_Grumbach_Hill_Jefferson_Mail_Voting.pdf?raw=1}{reduce
the wealth-turnout gap}. Households with less than \$10,000 in wealth
see a 10 percentage-point turnout boost from all-mail voting, while the
effect for those with \$250,000 or more in wealth is about half that
size.

The explanation is simple: Mail voting makes participating in elections
a lot easier. This is particularly helpful for young people, who
disproportionately cite time constraints as their reason for not voting.

We also examine the inevitable question on politicians' minds: What will
this do for my re-election prospects? Looking at voters by political
party, we find that Democrats and Republicans benefit about the same
amount: around 8 percentage points.

This is somewhat surprising, given that groups historically associated
with voting for Democrats benefit most from mail voting. One explanation
may be that in Colorado, young people are
\href{https://www.npr.org/2016/02/24/467914349/explaining-indepedent-voters-in-colorado}{choosing
to register as independents} rather than as Democrats. In fact, we found
that Colorado's shift to vote-by-mail increased the turnout of
independents by 12 percentage points, more than among members of either
major party.

It should be noted that while our findings suggest national voting by
mail could do wonders alone, voting experts rate Colorado's system so
highly because it also allows for same-day registration. This ensures
that people who miss the state's registration deadline for mail voting
can still register and vote in person. (Colorado also proactively
updates voter addresses using the United States Postal Service's
National Change of Address database and, as of 2017, provides automatic
voter registration throughout the state.)

Colorado also had years to prepare for its expansion of mail voting.
Last-minute changes could undermine trust and depress turnout. This is
why a leading consortium of civil rights groups called for primaries to
\href{https://civilrights.org/2020/03/13/voting-rights-groups-elections-must-proceed-while-states-protect-public-health/}{continue
as scheduled} in the face of Covid-19. Election administrators looking
to institute mail voting in time for November should be careful to
communicate all changes --- and the reasoning behind them --- to voters
of all backgrounds.

For most voters, mail voting is not a partisan issue. The reform draws
strong support among both Democratic and Republican voters, according to
a
\href{https://www.reuters.com/article/us-usa-election-poll/most-americans-unlike-trump-want-mail-in-ballots-for-november-if-coronavirus-threatens-reuters-ipsos-poll-idUSKBN21P3G0}{recent
Reuters/Ipsos poll}. Support is even stronger among Democrats and
Republicans living in states that already have all-mail voting. All-mail
voting appears to be that rarest of democracy reforms: a shift that
helps everyone get more involved, that reduces inequities and that
attracts support across parties --- if only at the grass-roots level.

A number of Republican leaders --- most notably President Trump --- have
come out in opposition to mail voting. To justify their position, many
have bandied the Republican Party's evergreen excuse for opposing
democracy reforms: the specter of voter fraud. This is a bad-faith take.

As election security experts have
\href{https://www.nytimes.com/article/mail-in-voting-explained.html}{pointed
out}, fraud is exceptionally rare, hard to commit without getting caught
and nearly impossible to do on the scale necessary to affect election
results. And because mail voting
\href{https://www.washingtonpost.com/news/powerpost/paloma/the-cybersecurity-202/2018/05/10/the-cybersecurity-202-how-colorado-became-the-safest-state-to-cast-a-vote/5af317c930fb042db5797427/}{leaves
behind a paper trail} --- which election officials can audit to verify
that votes were counted as cast --- it may actually be even more secure
than in-person voting.

There's solid evidence that Republican politicians may not believe their
own rhetoric on this issue. Some Republican legislatures recently
introduced
\href{https://nymag.com/intelligencer/2020/04/seven-states-limit-voting-by-mail-to-trump-approved-groups.html}{proposed
changes} that allow for mail voting in November 2020, but only for those
age 65 and older, or those in the military, both of whom more reliably
support Republican candidates.

At this point, the burden of argument regarding the merit of mail voting
should be on its opponents, not its proponents.
\href{https://www.dropbox.com/s/8n4zjvgmytim1rv/Bonica_Grumbach_Hill_Jefferson_Mail_Voting.pdf?raw=1}{Our
research suggests} policymakers looking to to maximize democratic
participation can expand mail voting ahead of this challenging November
election --- and also put it on the books for years to come.

Charlotte Hill is a doctoral candidate at the University of California,
Berkeley. Jacob Grumbach is a professor of political science at the
University of Washington. Adam Bonica and Hakeem Jefferson are
professors of political science at Stanford.

\emph{The Times is committed to publishing}
\href{https://www.nytimes.com/2019/01/31/opinion/letters/letters-to-editor-new-york-times-women.html}{\emph{a
diversity of letters}} \emph{to the editor. We'd like to hear what you
think about this or any of our articles. Here are some}
\href{https://help.nytimes.com/hc/en-us/articles/115014925288-How-to-submit-a-letter-to-the-editor}{\emph{tips}}\emph{.
And here's our email:}
\href{mailto:letters@nytimes.com}{\emph{letters@nytimes.com}}\emph{.}

\emph{Follow The New York Times Opinion section on}
\href{https://www.facebook.com/nytopinion}{\emph{Facebook}}\emph{,}
\href{http://twitter.com/NYTOpinion}{\emph{Twitter (@NYTopinion)}}
\emph{and}
\href{https://www.instagram.com/nytopinion/}{\emph{Instagram}}\emph{.}

Advertisement

\protect\hyperlink{after-bottom}{Continue reading the main story}

\hypertarget{site-index}{%
\subsection{Site Index}\label{site-index}}

\hypertarget{site-information-navigation}{%
\subsection{Site Information
Navigation}\label{site-information-navigation}}

\begin{itemize}
\tightlist
\item
  \href{https://help.nytimes.com/hc/en-us/articles/115014792127-Copyright-notice}{©~2020~The
  New York Times Company}
\end{itemize}

\begin{itemize}
\tightlist
\item
  \href{https://www.nytco.com/}{NYTCo}
\item
  \href{https://help.nytimes.com/hc/en-us/articles/115015385887-Contact-Us}{Contact
  Us}
\item
  \href{https://www.nytco.com/careers/}{Work with us}
\item
  \href{https://nytmediakit.com/}{Advertise}
\item
  \href{http://www.tbrandstudio.com/}{T Brand Studio}
\item
  \href{https://www.nytimes.com/privacy/cookie-policy\#how-do-i-manage-trackers}{Your
  Ad Choices}
\item
  \href{https://www.nytimes.com/privacy}{Privacy}
\item
  \href{https://help.nytimes.com/hc/en-us/articles/115014893428-Terms-of-service}{Terms
  of Service}
\item
  \href{https://help.nytimes.com/hc/en-us/articles/115014893968-Terms-of-sale}{Terms
  of Sale}
\item
  \href{https://spiderbites.nytimes.com}{Site Map}
\item
  \href{https://help.nytimes.com/hc/en-us}{Help}
\item
  \href{https://www.nytimes.com/subscription?campaignId=37WXW}{Subscriptions}
\end{itemize}
