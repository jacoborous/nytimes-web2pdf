Sections

SEARCH

\protect\hyperlink{site-content}{Skip to
content}\protect\hyperlink{site-index}{Skip to site index}

\href{https://www.nytimes.com/section/health}{Health}

\href{https://myaccount.nytimes.com/auth/login?response_type=cookie\&client_id=vi}{}

\href{https://www.nytimes.com/section/todayspaper}{Today's Paper}

\href{/section/health}{Health}\textbar{}`Straight-Up Fire' in His Veins:
Teen Battles New Covid Syndrome

\href{https://nyti.ms/2LCl3qG}{https://nyti.ms/2LCl3qG}

\begin{itemize}
\item
\item
\item
\item
\item
\item
\end{itemize}

\href{https://www.nytimes.com/news-event/coronavirus?action=click\&pgtype=Article\&state=default\&region=TOP_BANNER\&context=storylines_menu}{The
Coronavirus Outbreak}

\begin{itemize}
\tightlist
\item
  live\href{https://www.nytimes.com/2020/08/08/world/coronavirus-updates.html?action=click\&pgtype=Article\&state=default\&region=TOP_BANNER\&context=storylines_menu}{Latest
  Updates}
\item
  \href{https://www.nytimes.com/interactive/2020/us/coronavirus-us-cases.html?action=click\&pgtype=Article\&state=default\&region=TOP_BANNER\&context=storylines_menu}{Maps
  and Cases}
\item
  \href{https://www.nytimes.com/interactive/2020/science/coronavirus-vaccine-tracker.html?action=click\&pgtype=Article\&state=default\&region=TOP_BANNER\&context=storylines_menu}{Vaccine
  Tracker}
\item
  \href{https://www.nytimes.com/interactive/2020/world/coronavirus-tips-advice.html?action=click\&pgtype=Article\&state=default\&region=TOP_BANNER\&context=storylines_menu}{F.A.Q.}
\item
  \href{https://www.nytimes.com/live/2020/08/07/business/stock-market-today-coronavirus?action=click\&pgtype=Article\&state=default\&region=TOP_BANNER\&context=storylines_menu}{Markets
  \& Economy}
\end{itemize}

Advertisement

\protect\hyperlink{after-top}{Continue reading the main story}

Supported by

\protect\hyperlink{after-sponsor}{Continue reading the main story}

\hypertarget{straight-up-fire-in-his-veins-teen-battles-new-covid-syndrome}{%
\section{`Straight-Up Fire' in His Veins: Teen Battles New Covid
Syndrome}\label{straight-up-fire-in-his-veins-teen-battles-new-covid-syndrome}}

Jack McMorrow, 14, awoke in agony, with heart failure. His case may help
doctors understand a frightening new affliction in children linked to
the coronavirus.

\includegraphics{https://static01.nyt.com/images/2020/05/19/science/00virus-teen-1/merlin_172404669_1536a1ab-f1d9-4611-b071-3050f4680440-articleLarge.jpg?quality=75\&auto=webp\&disable=upscale}

\href{https://www.nytimes.com/by/pam-belluck}{\includegraphics{https://static01.nyt.com/images/2018/02/16/multimedia/author-pam-belluck/author-pam-belluck-thumbLarge-v2.png}}

By \href{https://www.nytimes.com/by/pam-belluck}{Pam Belluck}

\begin{itemize}
\item
  Published May 17, 2020Updated May 21, 2020
\item
  \begin{itemize}
  \item
  \item
  \item
  \item
  \item
  \item
  \end{itemize}
\end{itemize}

\href{https://www.nytimes.com/es/2020/05/18/espanol/sindrome-coronavirus-ninos.html}{Leer
en español}

\includegraphics{https://static01.nyt.com/images/2017/01/29/podcasts/the-daily-album-art/the-daily-album-art-articleInline-v2.jpg?quality=75\&auto=webp\&disable=upscale}

\hypertarget{listen-to-the-daily-a-teenagers-medical-mystery}{%
\subsubsection{Listen to `The Daily': A Teenager's Medical
Mystery}\label{listen-to-the-daily-a-teenagers-medical-mystery}}

How the case of one 14-year-old could help doctors understand a
frightening new illness linked to the coronavirus.

transcript

Back to The Daily

bars

0:00/33:01

-33:01

transcript

\hypertarget{listen-to-the-daily-a-teenagers-medical-mystery-1}{%
\subsection{Listen to `The Daily': A Teenager's Medical
Mystery}\label{listen-to-the-daily-a-teenagers-medical-mystery-1}}

\hypertarget{hosted-by-michael-barbaro-produced-by-clare-toeniskoetter-and-jessica-cheung-with-help-from-rachel-quester-edited-by-liz-o-baylen-and-lisa-tobin}{%
\subsubsection{Hosted by Michael Barbaro; produced by Clare
Toeniskoetter and Jessica Cheung; with help from Rachel Quester; edited
by Liz O. Baylen and Lisa
Tobin}\label{hosted-by-michael-barbaro-produced-by-clare-toeniskoetter-and-jessica-cheung-with-help-from-rachel-quester-edited-by-liz-o-baylen-and-lisa-tobin}}

\hypertarget{how-the-case-of-one-14-year-old-could-help-doctors-understand-a-frightening-new-illness-linked-to-the-coronavirus}{%
\paragraph{How the case of one 14-year-old could help doctors understand
a frightening new illness linked to the
coronavirus.}\label{how-the-case-of-one-14-year-old-could-help-doctors-understand-a-frightening-new-illness-linked-to-the-coronavirus}}

\begin{itemize}
\item
  michael barbaro\\
  From The New York Times, I'm Michael Barbaro. This is ``The Daily.''
\item
  {[}music{]}\\
  Today: From the earliest days of the coronavirus, health officials
  believed that it largely spared children and teenagers, but recently
  that belief has been challenged. My colleague Pam Belluck on the story
  of a 14-year-old boy whose case is being studied to better understand
  the impact of the virus on children.

  It's Thursday, May 21.

  Pam, when does this understanding that we all seem to have about the
  coronavirus and how it spares children, when does that start to
  change?
\item
  pam belluck\\
  In late April, there was this bulletin that was sent out by a
  pediatric health service in the United Kingdom. It just said we're
  noticing some kids, not very many. They seem to have these symptoms of
  inflammation. We don't really know what this is about. Some have
  tested positive for coronavirus. Some haven't. And it was just kind of
  saying we think we're seeing something.

  So I talked to my editors about it, and we were trying to figure out
  whether we should explore it more at that point. And we decided, well,
  we don't really know a whole lot. It seems like a small number of
  cases. We can't even say for certain that it's connected to
  coronavirus, and so we just kind of put it aside for a bit and watch
  it.

  And then I think a couple days later I got an email from a hospital in
  New York City. The person said we've got two cases of this syndrome
  that they've been talking about in the UK. If you want to talk to
  somebody, let us know. And that's how I got to know Jack McMorrow and
  his family in their apartment in Queens.
\item
  michael barbaro\\
  And tell me about this visit.
\item
  pam belluck\\
  It's a very warm, kind of cozy apartment. There are all these
  ``welcome home'' banners and balloons for Jack. And the family, Jack
  and his father John and his mother Doris, they just immediately
  welcomed me and our photographer Gabriela in, and they all just start
  talking. I just know immediately I have to put on my tape recorder
  because there's no way I'm going to capture all this writing things
  down.
\item
  john mcmorrow\\
  He sent him a letter.
\item
  jack mcmorrow\\
  Yes. Can I explain something ---
\item
  john mcmorrow\\
  Oh, come on. I'm giving backdrop here. You can explain all you want.
\item
  jack mcmorrow\\
  No. First of all, I wanted to talk about how the virus was behaving
  like a bacteria.
\item
  john mcmorrow\\
  OK, you're gonna, but I just want to say ---
\item
  jack mcmorrow\\
  That was way back when.
\item
  john mcmorrow\\
  --- Randall ---
\item
  jack mcmorrow\\
  This thing's probably not picking up anything.
\item
  doris stroman\\
  I hope you know she's taping all of you because ---
\item
  john mcmorrow\\
  Yeah, we barter like this all day long.
\item
  doris stroman\\
  I got a bell from school ---
\item
  jack mcmorrow\\
  Stop with the bell.
\item
  doris stroman\\
  --- because they bicker. And I have to do this ---
\item
  john mcmorrow\\
  And it's a timeout bell.
\item
  doris stroman\\
  --- and tell them to both go to each corner.
\item
  jack mcmorrow\\
  I'm sorry about all this chaotic ---
\item
  pam belluck\\
  It's wonderful. It's wonderful.
\item
  {[}bell ringing{]}
\item
  john mcmorrow\\
  And these cleaned arteries, they have to bring you right back.
\item
  jack mcmorrow\\
  I was coherent at this time.
\item
  john mcmorrow\\
  No, I know. But I know you've been jumping things, and I know when
  you're excited.
\item
  jack mcmorrow\\
  Me jumping things?
\item
  pam belluck\\
  You're doing great. You're doing great.
\item
  jack mcmorrow\\
  Dad, you went from day one to me in the new I.C.U. That's like a ---
\item
  doris stroman\\
  Do I need to get my bell?
\item
  jack mcmorrow\\
  --- huge jump. Dude, if ---
\item
  speaker\\
  Oh my God.
\end{itemize}

michael barbaro

And tell me about this family. Who are they?

pam belluck

So Jack's father is John McMorrow.

\begin{itemize}
\tightlist
\item
  john mcmorrow\\
  I know it's your story, son.
\end{itemize}

pam belluck

He is a truck driver. He works as a Teamster for the film industry. He
was recently laid off because of the pandemic. And his mother, Jack's
mother, is Doris Stroman. She works at a lab school with five and
six-year-old kids. She was wearing a mask that had The Rolling Stones
tongue logo on it.

\begin{itemize}
\tightlist
\item
  doris stroman\\
  --- to figure out what was going on, starting with his pediatrician.
\end{itemize}

pam belluck

And Jack is 14. He's a ninth grader. He goes to Catholic school in
Queens.

\begin{itemize}
\item
  jack mcmorrow\\
  Yeah, and I have a whole bunch of other prop replica stuff.
\item
  pam belluck\\
  So you're a Star Wars fan?
\item
  jack mcmorrow\\
  I like Marvel a lot more than I do ---
\item
  pam belluck\\
  Oh, you're more of a Marvel person. OK.
\item
  jack mcmorrow\\
  This is the Infinity Gauntlet from ``Avengers: Infinity War.'' I
  really have to ---
\item
  doris stroman\\
  They don't have time for that, Jack.
\item
  jack mcmorrow\\
  This is why ---
\end{itemize}

michael barbaro

And, Pam, what is the story that Jack and his parents tell you about
this mysterious condition that he has?

pam belluck

So Jack was living the world of a New York City teenager in a pandemic.

\begin{itemize}
\item
  doris stroman\\
  He never left the house.
\item
  jack mcmorrow\\
  I haven't left ---
\item
  john mcmorrow\\
  Since March 13, he's been in the house.
\item
  doris stroman\\
  You never left the house. His school ---
\item
  john mcmorrow\\
  In his room, not even in here.
\item
  doris stroman\\
  His Catholic school was one of the first that were closed.
\item
  pam belluck\\
  Oh wow.
\item
  doris stroman\\
  Didn't leave the house.
\end{itemize}

pam belluck

March 12 was his last day of school, and he was doing the online
learning thing.

\begin{itemize}
\item
  doris stroman\\
  The one time he left the house other than --- was to help me with the
  laundry and didn't want to touch anything.
\item
  jack mcmorrow\\
  I took a shower after I came up from the laundry room.
\item
  doris stroman\\
  Yeah, the kid just ---
\item
  jack mcmorrow\\
  I'm a germophobe.
\end{itemize}

pam belluck

They just kind of stayed in. He was playing video games. He was chatting
with his friends and that kind of thing. That was Jack's world.

{[}music{]}

Then in mid-April, Jack's parents start to notice some unusual things.

\begin{itemize}
\item
  john mcmorrow\\
  Three weeks ago, he came out to me with a rash on the backside of his
  hands.
\item
  jack mcmorrow\\
  Yeah.
\item
  john mcmorrow\\
  I thought it was ---
\item
  jack mcmorrow\\
  It was bad.
\item
  john mcmorrow\\
  --- from the antibacterial soap. You know, Purell. Maybe he's doing it
  too much. He's sensitive.
\item
  jack mcmorrow\\
  Yeah, for like --- we thought it was nothing more than eczema.
\item
  john mcmorrow\\
  And then I think a day or two later he ---
\item
  jack mcmorrow\\
  No, it was like ---
\item
  john mcmorrow\\
  --- your mother told you something about your eyes. She thought you
  were playing video games too much.
\item
  jack mcmorrow\\
  Yeah, the eyes definitely, but I don't know if that was ---
\end{itemize}

pam belluck

They went on, and then the next week --- and this was April 21 ---

\begin{itemize}
\tightlist
\item
  jack mcmorrow\\
  I had got a normal fever, like 101, 102.
\end{itemize}

pam belluck

--- Jack gets a fever.

\begin{itemize}
\item
  jack mcmorrow\\
  I woke up one day with ---
\item
  doris stroman\\
  Sore throat.
\item
  jack mcmorrow\\
  --- sore throat. That was the first inflammation symptom that we had,
  which was ---
\item
  doris stroman\\
  On his hands, on his feet ---
\item
  jack mcmorrow\\
  On my hands, on my feet ---
\item
  doris stroman\\
  --- on his neck.
\item
  jack mcmorrow\\
  --- and on my neck. That was the first ---
\end{itemize}

pam belluck

And then around Friday, April 24, things start to get more severe.

\begin{itemize}
\item
  jack mcmorrow\\
  That ended up being a swollen lymph node that grew to about the size
  of a tennis ball that you could visibly see coming on the side of my
  neck.
\item
  doris stroman\\
  That was alarming.
\end{itemize}

pam belluck

By the next day, Saturday morning, he wakes up and he's got a 104.7
fever.

michael barbaro

That is a real fever.

pam belluck

That is a serious fever. They call their pediatrician at 7:30 in the
morning, and she says, you guys, you got to get to an urgent care
clinic, and they do. And there he gets a coronavirus test, but it's
going to be a couple days before he gets the results.

michael barbaro

So at this point they think it might perhaps be Covid-19.

pam belluck

It doesn't look like Covid-19, but we're living in a world of Covid-19,
and so I think that they are just sort of saying, well, let's just test
him. We don't really know what this is. They send him home. Things just
keep getting worse and worse. And by Monday morning, Jack wakes up. He
cannot move. He can't move.

\begin{itemize}
\tightlist
\item
  jack mcmorrow\\
  Because I wake up, and to even sit up, I screamed for them. And I had
  105 almost.
\end{itemize}

pam belluck

And he's lying on the couch.

\begin{itemize}
\item
  jack mcmorrow\\
  I was sleeping with my socks on, and he kind of saw red. And he takes
  off my socks to reveal my entire feet, right here, had just rashes on
  the insides and bottom.
\item
  doris stroman\\
  At that time ---
\item
  jack mcmorrow\\
  And my hands.
\item
  doris stroman\\
  --- we thought that was the apex.
\item
  jack mcmorrow\\
  And my hands.
\item
  doris stroman\\
  But it wasn't until days later.
\item
  jack mcmorrow\\
  Yeah, they thought that was bad. My hands here on my palms, a little
  bit at the back, all rashes. So my skin --- to even touch my skin and
  feel ---
\end{itemize}

pam belluck

It's terribly, terribly frightening. And he says to me, I was very
emotional.

\begin{itemize}
\tightlist
\item
  jack mcmorrow\\
  I'm using the word emotional to try and cover up the fact I was crying
  like a baby. It was so bad.
\end{itemize}

pam belluck

They happen to have a home blood pressure monitor, so they take his
blood pressure. And this is where, as if all of these symptoms weren't
alarming enough and frightening enough, the blood pressure is very low.
And so they know they had to take him to the hospital. They had to
figure out how to get him out of the house. He can't move. So John and
Jack kind of demonstrate this for me.

\begin{itemize}
\item
  jack mcmorrow\\
  I put my hands on his arms like this and, not kidding, shuffled my
  way.
\item
  john mcmorrow\\
  And I had to then hold him up ---
\item
  jack mcmorrow\\
  With his arms.
\end{itemize}

pam belluck

John picks him up, puts Jack's feet on top of John's feet, and then
walks backward out the apartment door, sort of shuffling Jack along.

\begin{itemize}
\item
  doris stroman\\
  And when we got to the hospital ---
\item
  john mcmorrow\\
  They took a wheelchair.
\item
  doris stroman\\
  --- they took a wheelchair.
\item
  jack mcmorrow\\
  Yeah, I took a wheelchair.
\item
  doris stroman\\
  He couldn't walk.
\item
  john mcmorrow\\
  He couldn't walk no more. He couldn't bend his legs.
\end{itemize}

pam belluck

So he gets to the hospital. They are trying to figure out, again, what's
going on. They don't know.

\begin{itemize}
\tightlist
\item
  john mcmorrow\\
  --- everything back and forth. You had a cardiologist department. You
  had the pulmonary specialist, infectious disease experts, and then you
  had the immunology all throwing numbers and prescriptions and how they
  count through each other to deal with him. And this is stuff that I
  --- it's French to me. You might as well just tell me ---
\end{itemize}

pam belluck

And while he's there, they get the coronavirus test results back from
the clinic that he went to on Saturday two days earlier.

michael barbaro

And what does it say?

pam belluck

They're negative. So they're crossing that off the list. They say, we
really should probably send you home because we don't really know what
this is, and we think maybe you can just kind of watch it at home.

\begin{itemize}
\tightlist
\item
  doris stroman\\
  Because they were riding the wave that he tested negative.
\end{itemize}

pam belluck

Well, Doris is not happy about that. She says ---

\begin{itemize}
\tightlist
\item
  doris stroman\\
  And I said, well, he needs to be tested again. And she said, we only
  test those who are admitted. And I said, well, then he needs to be
  admitted. We have nowhere to be ---
\end{itemize}

pam belluck

So there's a communication around that. And they agree there's no harm
in doing another coronavirus test. Why not? We don't really know what's
happening. Why not? So they do another coronavirus test. And then while
they're there waiting, another symptom emerges.

\begin{itemize}
\item
  doris stroman\\
  When he woke up, his eyes were like this. And I was just like, what
  just ---
\item
  jack mcmorrow\\
  Yeah, they were rolling in the back of my head.
\item
  doris stroman\\
  And they were red.
\end{itemize}

pam belluck

His eyes turned bright red. As his mom is telling me about this, she is
pointing to a red pillow on their couch, and she says, it's like this.
And his eyes are rolling back into his head, and they're bright red.

\begin{itemize}
\item
  doris stroman\\
  He was like, I'm fine, I'm fine, like this. I'm fine. I'm fine.
\item
  john mcmorrow\\
  When was this?
\end{itemize}

pam belluck

Then the doctor comes in and tells them that, guess what? The new
coronavirus test, the second one, it was positive.

michael barbaro

Pam, how could that be that he has a negative test and just a few days
later, suddenly a positive test?

pam belluck

Well, unfortunately this is kind of the reality of coronavirus testing
right now that they are not 100 percent reliable. It's a little bit of a
Wild West situation. So there are cases of false negatives, and that's
obviously what was the case with Jack. So once they realize that he is
Covid positive, they decide at that hospital that he's got to go to a
children's hospital. And Jack is not on board with this. He does not
want to go.

And the doctor says to him, ``If I send you home today, you will be dead
by tomorrow.''

\begin{itemize}
\tightlist
\item
  jack mcmorrow\\
  That, I would say, had scared me to death. But it more scared me to
  life. It scared me to fight.
\end{itemize}

pam belluck

So Jack gets to the children's hospital in the ambulance. And the
doctors take one look, and they realize, this is not what we thought
coronavirus infection looks like. This is not the way it usually affects
patients. And they know that by looking at Jack and figuring out what's
going on with him, they are going to learn a lot more about what this
virus can do to kids.

{[}music{]}

michael barbaro

We'll be right back.

\begin{itemize}
\tightlist
\item
  jack mcmorrow\\
  I'm getting to the pain now. It was a throbbing, stinging rush of,
  like, you could feel it going through your veins.
\end{itemize}

pam belluck

So when Jack gets to the hospital, he is just exhausted and in so much
pain.

\begin{itemize}
\tightlist
\item
  jack mcmorrow\\
  You could feel --- it was almost like someone injected you with
  straight up fire. Just fire.
\end{itemize}

pam belluck

The major symptom that's going on with Jack is that he has very low
blood pressure.

\begin{itemize}
\tightlist
\item
  jack mcmorrow\\
  You've got to remember, my heart rate was at 165 while I was sleeping.
  That's like a marathon runner.
\end{itemize}

pam belluck

And he has a very, very fast heart rate, because his heart is trying
very hard to compensate for that low blood pressure that is preventing
him from pumping oxygen and nutrients throughout his body to his
critical organs. So that's what they've got to treat. That is a
condition that is called cardiogenic shock. It is heart failure. It is
fatal if not treated. And he was telling me that he started to focus his
energy. He started to feel like, I have got to understand what is going
on with my body. I've got to know, because if I don't know what I'm
fighting, then I can't fight it. So he starts to talk to the doctors.

\begin{itemize}
\tightlist
\item
  jack mcmorrow\\
  They don't get a lot of kids that can actually talk to them since it's
  pediatrics.
\end{itemize}

pam belluck

And he's a ninth-grade kid, and he's been taking biology, and he has
some understanding about the heart and the lungs and how they all work.
And so he's asking them lots of questions.

\begin{itemize}
\tightlist
\item
  jack mcmorrow\\
  We were going back and forth with the whole --- especially the way my
  heart related to my cardiovascular and circulatory system, never mind
  my ---
\end{itemize}

pam belluck

And that made him feel much more in control, or at least it was a little
bit less terrifying for him once he kind of realized what he could
understand. But in that first day or two ---

\begin{itemize}
\tightlist
\item
  jack mcmorrow\\
  It was scary.
\end{itemize}

pam belluck

--- he did feel like he wasn't going to come out of it.

\begin{itemize}
\tightlist
\item
  speaker\\
  It didn't look like I was coming out of it the same, if at all.
\end{itemize}

michael barbaro

And how do the doctors try to treat Jack during this time?

pam belluck

So the first thing that they're trying to do is give him blood pressure
medication to try to get his blood pressure up, but it's just not
working. It's been 48 hours. And they are so worried about his heart,
which is not pumping enough oxygen to his body, that they think they're
going to need to put him on a ventilator.

michael barbaro

Wow.

\begin{itemize}
\item
  john mcmorrow\\
  They were going to intubate him. And I said, you know, that was
  breaking my heart.
\item
  doris stroman\\
  If they were to ---
\item
  john mcmorrow\\
  And so did they. They didn't want it, because they know that they had
  to brace me on the realistic approach that only 20 percent come off.
\end{itemize}

pam belluck

So they say, well, you know, why don't we try some steroids? Now,
steroids are this widely used medication that works in a lot of
different ways and works for some things, it doesn't work for other
things, and it's really hard to know whether it's going to help him or
not. But within a few hours, he starts to stabilize. They decide they
don't need the ventilator, and ---

\begin{itemize}
\item
  jack mcmorrow\\
  They were bringing me Icees and ginger ale ---
\item
  john mcmorrow\\
  They were bringing him everything, lollipops ---
\item
  jack mcmorrow\\
  --- and I hadn't had water.
\item
  john mcmorrow\\
  He hadn't no water, nothing in his mouth for over 48 hours because
  they were ---
\item
  jack mcmorrow\\
  For 48 hours.
\item
  john mcmorrow\\
  --- preparing him to do the tube.
\item
  jack mcmorrow\\
  My mouth was --- I felt like I was dying. And then they were throwing
  Icees my way. They were like, here you go, kid. They gave me
  lollipops. They gave me ginger ales. I was, like, living the life.
\end{itemize}

pam belluck

So it seems like the steroids worked, but doctors actually don't know
that 100 percent. And John, Jack's father, called the pediatrician,
their longtime pediatrician, and said, what happened? I don't know what
happened. And ---

\begin{itemize}
\tightlist
\item
  john mcmorrow\\
  He laughed. And I said, why? Why? Why? How did this happen? What did
  he do? And he goes, I don't know. I said, you know my family's going
  to believe this was the power of prayer. And he goes, I'll go with
  that, because we don't know why. We don't know.
\end{itemize}

pam belluck

My family is going to think that it's a miracle. And the pediatrician
says, well, that works for me because I don't really know either.

michael barbaro

And Pam, beyond the steroids and whether or not those worked, what did
the doctors understand about what was going on here?

pam belluck

Well, they're kind of mystified. I mean, they've got this kid, and they
know that he has a positive coronavirus test, but he doesn't have
symptoms that kind of look like what they've come to expect from
coronavirus. And at the same time, just that very morning they've had
two or three other kids show up with the same symptoms, very similar
symptoms. And those kids have tested negative for coronavirus. So they
don't have a live coronavirus infection, but the doctors are wondering.

And so they have another test in their toolkit. They have what's called
an antibody test, which can tell you not whether you have the live
infection right now, but it can tell you whether somebody has ever had
coronavirus infection. And they think, let's just give these kids ---
these other kids that test and see.

And lo and behold, those kids end up being positive for coronavirus
antibodies. And that means that all of these kids who are showing up
with these mysterious symptoms that cannot be explained by anything else
that doctors know have this one common denominator. They have all had
coronavirus.

michael barbaro

Pam, at this point, what do the doctors think that this is exactly?
Because all of these kids have had coronavirus, but most of them don't
still have it.

pam belluck

What they think is this may be a kind of second-stage effect of
coronavirus that we didn't know was possible, that we didn't know was
part of the way this virus worked. These kids didn't get the lung
problems, the breathing problems, that kind of assault on the lungs that
is the primary way that coronavirus works.

And so what the doctors think is that at the time of their infection,
their immune system did a really good job of just swatting the infection
away, of battling it away --- that's why they didn't have any symptoms
at the time. But that somehow in the course of that fight, their immune
system got so revved up and so hyperactive that it generated this
inflammatory response weeks later, and their bodies had this incredible
overzealous reaction that went throughout their bodies and caused all
sorts of havoc.

michael barbaro

So this is not coronavirus for kids. It's some kind of
later-down-the-line, affiliated set of horrible conditions that follows
it.

pam belluck

Exactly.

michael barbaro

I mean, what seems particularly scary about this is that theoretically
any kid who has had the coronavirus --- and I have to imagine there are
tens of thousands, maybe hundreds of thousands of these across the
United States, people like Jack who probably showed no symptoms
whatsoever from the original infection --- it now seems possible that
they could develop these really awful new secondary symptoms.

pam belluck

That's exactly the risk here. That's exactly the worry. We know that
kids are just as likely to get infected as adults. They don't have any
protection from infection. A whole lot of them end up showing no
symptoms. And we wanted to think that that meant that they really
weren't getting that sick. But now we have this thing that shows up
weeks later, and we don't have any idea who will end up with this
inflammatory syndrome and when.

michael barbaro

I mean, what are the implications of that as we think about reopening
schools, for example? I mean, one of the kind of saving graces, silver
linings of this pandemic was that kids were supposed to be spared, and
that understanding seems to have been the basis for plans to reopen
schools. What does it mean that this second-stage set of symptoms is now
starting to show up among children?

pam belluck

It definitely puts a serious complication in those plans. It's something
that governors, federal officials, they are already thinking about ---
they are going to have to think about. It's not like you can test kids
and say, OK, you're negative, or you have antibodies, you're going to be
fine. Because you could have antibodies, and then you could end up with
this. So it makes that issue much more tenuous and much more
complicated, and I don't think anybody has a good answer for that right
now.

michael barbaro

And Pam, how is Jack doing at this point?

pam belluck

He's doing OK. He's home.

\begin{itemize}
\item
  jack mcmorrow\\
  And I came home to take the best shower I've ever had in my entire
  life. Not even gassing it. It was like 30 minutes.
\item
  doris stroman\\
  You can't get him in, and then you can't get him out.
\item
  jack mcmorrow\\
  No, no, no. It was like ---
\item
  doris stroman\\
  You know. You have kids.
\item
  jack mcmorrow\\
  It was like 30 minutes, this one. And it felt fine, and then I was
  like, I got to stop running around because I'm going to fall. I'm
  going to get lightheaded and pass out. But completely ignoring my own
  self advice, just ran into my room, put on my headphones, talked to my
  friends, and I said, I'm home! And they were all like, yeah!
\item
  doris stroman\\
  Any time he runs around ---
\item
  jack mcmorrow\\
  And it was the best.
\item
  doris stroman\\
  --- and says I'm alive, I'm alive ---
\item
  jack mcmorrow\\
  No.
\item
  doris stroman\\
  --- we go, ``I'm a real boy!''
\item
  jack mcmorrow\\
  I'm a real boy.
\item
  doris stroman\\
  I'm a real boy!
\item
  jack mcmorrow\\
  No, no, no, because I said that.
\item
  doris stroman\\
  From ``Pinocchio.''
\item
  jack mcmorrow\\
  No, because I was in the hospital, and I was like there are no strings
  on me because ---
\item
  john mcmorrow\\
  Because he did IVs ---
\end{itemize}

pam belluck

He has some residual heart issues, but they think that his issues,
because he's so young and otherwise healthy, that he'll probably emerge
from this with no real issues. They are going to be following him.
They're going to be following these other kids, too, because this is
still a mystery, and they don't really know whether it's going to have
any long-term effects.

And since his case, since his successful treatment, doctors have been
using the same playbook on other kids with his issue. So they think that
the steroids were what helped him, and they are giving other kids
steroids a lot earlier when they come into the hospital. So far,
apparently the results have been pretty encouraging. They are writing up
Jack's case, along with some of the other kids, in an article that's
going to be published in a medical journal. Jack was very excited to
learn about that. And he said to me ---

\begin{itemize}
\tightlist
\item
  jack mcmorrow\\
  It's been really good being back home, and I just want to do more with
  my life now, now that I have it back.
\end{itemize}

pam belluck

I really want to do something with my life, now that I have it back.

\begin{itemize}
\item
  jack mcmorrow\\
  In any way that I can.
\item
  pam belluck\\
  That is awesome.
\item
  jack mcmorrow\\
  Yeah.
\end{itemize}

pam belluck

He said this while holding his Captain America shield. So I thought ---
{[}LAUGHTER{]}

michael barbaro

He is, after all, a 14-year-old.

pam belluck

He is, after all, a 14-year-old boy.

\begin{itemize}
\item
  jack mcmorrow\\
  I literally sent my biology teacher an email, saying thank you for
  educating me.
\item
  pam belluck\\
  Really?
\item
  john mcmorrow\\
  Oh, that was the first thing he did.
\item
  jack mcmorrow\\
  I can show you it if you want.
\item
  john mcmorrow\\
  Yes. You should actually ---
\item
  pam belluck\\
  I would love to see it.
\item
  john mcmorrow\\
  --- show it.
\item
  doris stroman\\
  No, no, not now. Not now. Let her have it so ---
\item
  pam belluck\\
  Yeah, why don't you email it to me.
\item
  doris stroman\\
  It's long, so just let her read it when she gets a minute.
\end{itemize}

pam belluck

{[}READING JACK'S EMAIL{]} ``OK, I'll try to make this email quick,
because I'm still in the hospital recovering. The complications of this
virus have left me with pneumonia. And more serious than that, heart
issues. A mild heart blockage, as explained by the doctors. This heart
blockage is the main reason I'm not at home recovering right now, but
rather in a cardio-monitoring room.

``As hard as it is to keep up with all of this and understand many
aspects of these complications, because of how little they know of
Covid, I have to say, once it came around to them talking to me about my
heart and my systems, I'm confident that I was able to keep up with the
conversation and understand what was wrong with me and what to do to
keep fighting --- or rather, to keep my vitals in check.

``To summarize what I'm trying to say --- and this is the honest truth
--- I would like to thank you for educating me as you did and for
providing me the educational support to understand my body when I need
to most. Because based off of my knowledge on my heart and circulatory
system, I'm now able to work off of that knowledge and help myself
understand the doctors and communicate to them.

``I don't want to drag this out, and I know I said that I'd try to make
this short, but I really do have to thank you for educating me enough to
know what I needed to know. I'm sorry for making this email so long, and
I really feel bad for disturbing you on a Saturday night. But seriously,
I'm genuinely thanking you for educating me as you did, and I look
forward to seeing you on Zoom or in class if we return this school year.

``I hope your family and yourself stay safe. Thank you.''

michael barbaro

That's lovely.

pam belluck

Isn't that amazing?

michael barbaro

Is he back in school remotely?

pam belluck

He is back. Jack is back in school remotely. He's taking that biology
class and he's seeing his friends. And he is --- he is being Jack.

michael barbaro

Thank you, Pam. We really appreciate it.

pam belluck

Thank you.

{[}music{]}

michael barbaro

Last week, health officials gave Jack's condition a name: Multisystem
inflammatory syndrome in children. So far, it has been found in about
200 children in the U.S. and Europe, and has killed several of them.
Because the condition was just identified, it's unclear how many cases
have remained unreported. We'll be right back.

{[}music{]}

Here's what else you need to know today.

\begin{itemize}
\tightlist
\item
  archived recording (andy beshear)\\
  Retail opened today. Big day, big step. And what we saw out there from
  everything that we could see is people trying really hard.
\end{itemize}

michael barbaro

On Wednesday, two months after the pandemic began, all 50 states began
reopening to varying degrees.

\begin{itemize}
\tightlist
\item
  archived recording (andy beshear)\\
  And that's important, because we have one shot at reopening the
  economy the right way.
\end{itemize}

michael barbaro

Kentucky permitted retailers to let in customers. Connecticut allowed
restaurants and malls to reopen with significant limits. And New York
allowed religious gatherings of up to 10 people.

\begin{itemize}
\tightlist
\item
  archived recording (andrew cuomo)\\
  I understand their desire to get back to religious ceremonies as soon
  as possible. As a former altar boy, I get it. I think those religious
  ceremonies can be very comforting.
\end{itemize}

michael barbaro

But there were signs on Wednesday that the reopenings would be slow and
risky. Ford, which restarted its U.S. assembly lines earlier this week,
said it would halt operations at plants in Illinois and Michigan after
workers there tested positive for the virus.

{[}music{]}

That's it for ``The Daily.'' I'm Michael Barbaro. See you tomorrow.

\emph{To hear more audio stories from publishers like The New York
Times, download}
\href{https://www.audm.com/?utm_source=nyt\&utm_medium=embed\&utm_campaign=fire_in_veins}{\emph{Audm
for iPhone or Android}}\emph{.}

When a sprinkling of a reddish rash appeared on Jack McMorrow's hands in
mid-April, his father figured the 14-year-old was overusing hand
sanitizer --- not a bad thing during a global pandemic.

When Jack's parents noticed that his eyes looked glossy, they attributed
it to late nights of video games and TV.

When he developed a stomachache and didn't want dinner, ``they thought
it was because I ate too many cookies or whatever,'' said Jack, a ninth
grader in Woodside, Queens, who loves Marvel Comics and has ambitions to
teach himself ``Stairway to Heaven'' on the guitar.

But over the next 10 days, Jack felt increasingly unwell. His parents
consulted his pediatricians in video appointments and took him to a
weekend urgent care clinic. Then, one morning, he awoke unable to move.

He had a tennis-ball-size lymph node, raging fever, racing heartbeat and
dangerously low blood pressure. Pain deluged his body in ``a throbbing,
stinging rush,'' he said.

``You could feel it going through your veins and it was almost like
someone injected you with straight-up fire,'' he said.

Jack, who was previously healthy, was hospitalized with heart failure
that day, in a stark example of the newly discovered severe inflammatory
syndrome linked to
\href{https://www.nytimes.com/2020/05/21/podcasts/the-daily/coronavirus-children-sick.html?action=click\&module=Briefings\&pgtype=Homepage}{the
coronavirus} that has already been identified in about 200 children in
the United States and Europe and killed several.

The condition, which the Centers for Disease Control and Prevention are
calling
\href{https://emergency.cdc.gov/han/2020/han00432.asp}{Multisystem
Inflammatory Syndrome in Children}, has shaken widespread confidence
that children were largely spared from the pandemic. Instead of
targeting lungs as the primary coronavirus infection does, it causes
inflammation throughout the body and can cripple the heart. It has been
compared to a rare childhood inflammatory illness called
\href{https://www.cdc.gov/kawasaki/index.html}{Kawasaki disease}, but
doctors have learned that the new syndrome affects the heart differently
and erupts mostly in school-age children, rather than infants and
toddlers. The syndrome often appears weeks after infection in children
who did not experience first-phase coronavirus symptoms.

At a Senate hearing last week, Dr. Anthony S. Fauci, a leader of the
government's coronavirus response, warned that because of the syndrome,
``we've got to be careful that we are not cavalier and thinking that
children are completely immune to the deleterious effects.''

\includegraphics{https://static01.nyt.com/images/2020/05/15/science/00virus-teen-2/merlin_172404741_6abeb3f0-c536-461c-92ef-8de4a7628a63-articleLarge.jpg?quality=75\&auto=webp\&disable=upscale}

Jack's recovery and the experience of other survivors are Rosetta stones
for doctors, health officials and parents anxious to understand the
mysterious condition.

``He could have definitely died,'' said Dr. Gheorghe Ganea, who, along
with his wife, Dr. Camelia Ganea, has been Jack's primary doctor for
years. ``When there's cardiovascular failure, other things can follow.
Other organs can fail one after another, and survival becomes very
difficult.''

New York State has reported three deaths and, as of Sunday,
\href{https://www.nytimes.com/2020/05/17/nyregion/coronavirus-new-york-update.html}{137
cases were being investigated in the city alone}. Last week,
\href{https://emergency.cdc.gov/han/2020/han00432.asp}{a C.D.C. alert}
urged doctors nationwide to report suspected cases.

\hypertarget{latest-updates-the-coronavirus-outbreak}{%
\section{\texorpdfstring{\href{https://www.nytimes.com/2020/08/07/world/covid-19-news.html?action=click\&pgtype=Article\&state=default\&region=MAIN_CONTENT_1\&context=storylines_live_updates}{Latest
Updates: The Coronavirus
Outbreak}}{Latest Updates: The Coronavirus Outbreak}}\label{latest-updates-the-coronavirus-outbreak}}

Updated 2020-08-08T12:04:28.992Z

\begin{itemize}
\tightlist
\item
  \href{https://www.nytimes.com/2020/08/07/world/covid-19-news.html?action=click\&pgtype=Article\&state=default\&region=MAIN_CONTENT_1\&context=storylines_live_updates\#link-1f86d03a}{As
  the U.S. relief talks falter again, Trump says he is prepared to act
  on his own.}
\item
  \href{https://www.nytimes.com/2020/08/07/world/covid-19-news.html?action=click\&pgtype=Article\&state=default\&region=MAIN_CONTENT_1\&context=storylines_live_updates\#link-3f64a70a}{Cuomo
  says N.Y. schools can reopen in-person but leaves it up to districts
  to determine if, when and how.}
\item
  \href{https://www.nytimes.com/2020/08/07/world/covid-19-news.html?action=click\&pgtype=Article\&state=default\&region=MAIN_CONTENT_1\&context=storylines_live_updates\#link-14e70066}{Thousands
  of cases went unreported in California when a computer server failed.}
\end{itemize}

\href{https://www.nytimes.com/2020/08/07/world/covid-19-news.html?action=click\&pgtype=Article\&state=default\&region=MAIN_CONTENT_1\&context=storylines_live_updates}{See
more updates}

More live coverage:
\href{https://www.nytimes.com/live/2020/08/07/business/stock-market-today-coronavirus?action=click\&pgtype=Article\&state=default\&region=MAIN_CONTENT_1\&context=storylines_live_updates}{Markets}

``Everyone is doing everything they can to help look into this from all
different angles just to get the answers that parents want, that we
want,'' said Dr. Thomas Connors, a pediatric critical care physician who
treated Jack at NewYork-Presbyterian Morgan Stanley Children's Hospital.

Neither Jack nor his parents, John McMorrow and Doris Stroman, know how
he became infected with the coronavirus. After cleaning out his locker
at Monsignor McClancy High School on March 18 to continue school online
at home, he only left the apartment once, they said, to help his mother
wash clothes in their high-rise building's laundry room. His parents and
22-year-old sister also avoided going out and the tests they have had
turned up negative.

Image

After Jack's father, John McMorrow, choked up recounting his son's
illness, Jack embraced him.Credit...Gabriela Bhaskar for The New York
Times

Last week, in their apartment festooned with welcome-home balloons, the
family --- Jack wearing a blue bandanna as a mask, his mother in a mask
with the Rolling Stones tongue logo on it --- recounted their story. His
father, a recently laid-off truck driver for the film industry, briefly
choked up and Jack bounded over to hug him.

The week after his hand rash and stomachache, about a month after he had
last set foot in school, Jack developed a 102-degree fever and a sore
throat. Worried, his mother arranged a video visit with their
pediatricians, who started him on an antibiotic for possible bacterial
infection. For several days, he felt about the same, but then other
symptoms rapidly emerged: swollen neck, nausea, dry cough, a metallic
taste.

On Saturday, April 25, his fever spiked to 104.7, his chest felt tight,
and when he took deep breaths, ``it hurt down in the bottom,'' he said.

Image

Jack arriving at the children's hospital in an ambulance.

Credit...via McMorrow family

Image

Jack in his hospital bed.

Credit...via McMorrow family

That morning, Dr. Camelia Ganea video-conferenced with the family while
still in her pajamas, discovering Jack could barely open his mouth. She
prescribed steroids and suggested they visit an urgent care clinic.
There, Jack was tested for the coronavirus, but it would be two days
before results arrived.

By Monday, pain was ``flowing through me like lightning,'' Jack said,
and a rosy rash covered his feet.

``I was very very emotional,'' Jack said. He paused. ``I'm using the
word emotional to cover up the fact I was crying like a baby.''

Lying on the sofa, he could not move on his own and grasped for words to
describe what was happening.

``Rooftop,'' he implored his parents, seeking a shorthand way to ask
them to bend his leg like a peaked roof.

``I didn't know what I was trying to say, but I knew what I meant,'' he
explained later.

With a home monitor, they discovered his blood pressure was very low.
Mr. McMorrow lifted him, placing Jack's feet on top of his own, and
shuffled him to the car. At NewYork-Presbyterian/Weill Cornell hospital,
doctors gave Jack intravenous fluids and tried to diagnose his
condition. He did not have the obvious respiratory distress of Covid-19.
And then they got the results of his Saturday coronavirus test:
negative.

Suspecting he might have a condition like mononucleosis, they prepared
to discharge him, thinking he could be safely watched at home with
instructions to return if his blood pressure dropped again, his parents
said.

His mother was urging them to keep Jack longer when his eyes turned red
with a ``raging case of pinkeye'' and rolled back in his head, she said.
After a conversation with Jack's pediatrician, the hospital conducted
its own coronavirus test. It was positive.

The doctor decided Jack should be transferred to NewYork-Presbyterian's
pediatric affiliate, Morgan Stanley Children's Hospital, which is
treating many coronavirus cases. Jack begged to go home.

The doctor responded bluntly, saying she knew that teenagers often think
they are invincible.

``She told me if I go home now, by tomorrow, I'll be dead,'' Jack said.
``I would say that scared me to death, but it more scared me to life. It
scared me to fight as hard as I could.''

Image

Pain was ``flowing through me like lightning,'' Jack said, recalling the
morning he woke up and couldn't move.Credit...Gabriela Bhaskar for The
New York Times

Jack arrived at the children's hospital so feverish that his father was
``washing me down with ice-cold water and it only felt like a tingle,''
he said.

His resting heart rate was 165 beats per minute, about twice as high as
normal, as his heart struggled to compensate for his alarmingly low
blood pressure, which was hampering its ability to circulate blood and
supply his vital organs with oxygen and nutrients.

\href{https://www.nytimes.com/news-event/coronavirus?action=click\&pgtype=Article\&state=default\&region=MAIN_CONTENT_3\&context=storylines_faq}{}

\hypertarget{the-coronavirus-outbreak-}{%
\subsubsection{The Coronavirus Outbreak
›}\label{the-coronavirus-outbreak-}}

\hypertarget{frequently-asked-questions}{%
\paragraph{Frequently Asked
Questions}\label{frequently-asked-questions}}

Updated August 6, 2020

\begin{itemize}
\item ~
  \hypertarget{why-are-bars-linked-to-outbreaks}{%
  \paragraph{Why are bars linked to
  outbreaks?}\label{why-are-bars-linked-to-outbreaks}}

  \begin{itemize}
  \tightlist
  \item
    Think about a bar. Alcohol is flowing. It can be loud, but it's
    definitely intimate, and you often need to lean in close to hear
    your friend. And strangers have way, way fewer reservations about
    coming up to people in a bar. That's sort of the point of a bar.
    Feeling good and close to strangers. It's no surprise, then, that
    \href{https://www.nytimes.com/2020/07/02/us/coronavirus-bars.html?action=click\&pgtype=Article\&state=default\&region=MAIN_CONTENT_3\&context=storylines_faq}{bars
    have been linked to outbreaks in several states.} Louisiana health
    officials have tied
    \href{https://www.nytimes.com/2020/06/22/us/new-coronavirus-phase.html?action=click\&pgtype=Article\&state=default\&region=MAIN_CONTENT_3\&context=storylines_faq}{at
    least 100 coronavirus cases} to bars in the Tigerland nightlife
    district in Baton Rouge. Minnesota has traced 328 recent cases to
    bars across the state.
    \href{https://www.boisestatepublicradio.org/post/bars-large-venues-close-ada-county-after-surge-coronavirus-prompts-rollback\#stream/0}{In
    Idaho}, health officials shut down bars in Ada County after
    reporting clusters of infections among young adults who had visited
    several bars in downtown Boise. Governors in
    \href{https://www.nytimes.com/2020/07/01/us/california-coronavirus-reopening.html?action=click\&pgtype=Article\&state=default\&region=MAIN_CONTENT_3\&context=storylines_faq}{California},
    \href{https://www.nytimes.com/2020/06/14/us/coronavirus-united-states.html?action=click\&pgtype=Article\&state=default\&region=MAIN_CONTENT_3\&context=storylines_faq}{Texas
    and Arizona}, where coronavirus cases are soaring, have ordered
    hundreds of newly reopened bars to shut down. Less than two weeks
    after Colorado's bars reopened at limited capacity, Gov. Jared Polis
    \href{https://www.denverpost.com/2020/06/30/colorado-bars-closed-coronavirus/}{ordered
    them to close}.
  \end{itemize}
\item ~
  \hypertarget{i-have-antibodies-am-i-now-immune}{%
  \paragraph{I have antibodies. Am I now
  immune?}\label{i-have-antibodies-am-i-now-immune}}

  \begin{itemize}
  \tightlist
  \item
    As of right now,
    \href{https://www.nytimes.com/2020/07/22/health/covid-antibodies-herd-immunity.html?action=click\&pgtype=Article\&state=default\&region=MAIN_CONTENT_3\&context=storylines_faq}{that
    seems likely, for at least several months.} There have been
    frightening accounts of people suffering what seems to be a second
    bout of Covid-19. But experts say these patients may have a
    drawn-out course of infection, with the virus taking a slow toll
    weeks to months after initial exposure. People infected with the
    coronavirus typically
    \href{https://www.nature.com/articles/s41586-020-2456-9}{produce}
    immune molecules called antibodies, which are
    \href{https://www.nytimes.com/2020/05/07/health/coronavirus-antibody-prevalence.html?action=click\&pgtype=Article\&state=default\&region=MAIN_CONTENT_3\&context=storylines_faq}{protective
    proteins made in response to an
    infection}\href{https://www.nytimes.com/2020/05/07/health/coronavirus-antibody-prevalence.html?action=click\&pgtype=Article\&state=default\&region=MAIN_CONTENT_3\&context=storylines_faq}{.
    These antibodies may} last in the body
    \href{https://www.nature.com/articles/s41591-020-0965-6}{only two to
    three months}, which may seem worrisome, but that's perfectly normal
    after an acute infection subsides, said Dr. Michael Mina, an
    immunologist at Harvard University. It may be possible to get the
    coronavirus again, but it's highly unlikely that it would be
    possible in a short window of time from initial infection or make
    people sicker the second time.
  \end{itemize}
\item ~
  \hypertarget{im-a-small-business-owner-can-i-get-relief}{%
  \paragraph{I'm a small-business owner. Can I get
  relief?}\label{im-a-small-business-owner-can-i-get-relief}}

  \begin{itemize}
  \tightlist
  \item
    The
    \href{https://www.nytimes.com/article/small-business-loans-stimulus-grants-freelancers-coronavirus.html?action=click\&pgtype=Article\&state=default\&region=MAIN_CONTENT_3\&context=storylines_faq}{stimulus
    bills enacted in March} offer help for the millions of American
    small businesses. Those eligible for aid are businesses and
    nonprofit organizations with fewer than 500 workers, including sole
    proprietorships, independent contractors and freelancers. Some
    larger companies in some industries are also eligible. The help
    being offered, which is being managed by the Small Business
    Administration, includes the Paycheck Protection Program and the
    Economic Injury Disaster Loan program. But lots of folks have
    \href{https://www.nytimes.com/interactive/2020/05/07/business/small-business-loans-coronavirus.html?action=click\&pgtype=Article\&state=default\&region=MAIN_CONTENT_3\&context=storylines_faq}{not
    yet seen payouts.} Even those who have received help are confused:
    The rules are draconian, and some are stuck sitting on
    \href{https://www.nytimes.com/2020/05/02/business/economy/loans-coronavirus-small-business.html?action=click\&pgtype=Article\&state=default\&region=MAIN_CONTENT_3\&context=storylines_faq}{money
    they don't know how to use.} Many small-business owners are getting
    less than they expected or
    \href{https://www.nytimes.com/2020/06/10/business/Small-business-loans-ppp.html?action=click\&pgtype=Article\&state=default\&region=MAIN_CONTENT_3\&context=storylines_faq}{not
    hearing anything at all.}
  \end{itemize}
\item ~
  \hypertarget{what-are-my-rights-if-i-am-worried-about-going-back-to-work}{%
  \paragraph{What are my rights if I am worried about going back to
  work?}\label{what-are-my-rights-if-i-am-worried-about-going-back-to-work}}

  \begin{itemize}
  \tightlist
  \item
    Employers have to provide
    \href{https://www.osha.gov/SLTC/covid-19/standards.html}{a safe
    workplace} with policies that protect everyone equally.
    \href{https://www.nytimes.com/article/coronavirus-money-unemployment.html?action=click\&pgtype=Article\&state=default\&region=MAIN_CONTENT_3\&context=storylines_faq}{And
    if one of your co-workers tests positive for the coronavirus, the
    C.D.C.} has said that
    \href{https://www.cdc.gov/coronavirus/2019-ncov/community/guidance-business-response.html}{employers
    should tell their employees} -\/- without giving you the sick
    employee's name -\/- that they may have been exposed to the virus.
  \end{itemize}
\item ~
  \hypertarget{what-is-school-going-to-look-like-in-september}{%
  \paragraph{What is school going to look like in
  September?}\label{what-is-school-going-to-look-like-in-september}}

  \begin{itemize}
  \tightlist
  \item
    It is unlikely that many schools will return to a normal schedule
    this fall, requiring the grind of
    \href{https://www.nytimes.com/2020/06/05/us/coronavirus-education-lost-learning.html?action=click\&pgtype=Article\&state=default\&region=MAIN_CONTENT_3\&context=storylines_faq}{online
    learning},
    \href{https://www.nytimes.com/2020/05/29/us/coronavirus-child-care-centers.html?action=click\&pgtype=Article\&state=default\&region=MAIN_CONTENT_3\&context=storylines_faq}{makeshift
    child care} and
    \href{https://www.nytimes.com/2020/06/03/business/economy/coronavirus-working-women.html?action=click\&pgtype=Article\&state=default\&region=MAIN_CONTENT_3\&context=storylines_faq}{stunted
    workdays} to continue. California's two largest public school
    districts --- Los Angeles and San Diego --- said on July 13, that
    \href{https://www.nytimes.com/2020/07/13/us/lausd-san-diego-school-reopening.html?action=click\&pgtype=Article\&state=default\&region=MAIN_CONTENT_3\&context=storylines_faq}{instruction
    will be remote-only in the fall}, citing concerns that surging
    coronavirus infections in their areas pose too dire a risk for
    students and teachers. Together, the two districts enroll some
    825,000 students. They are the largest in the country so far to
    abandon plans for even a partial physical return to classrooms when
    they reopen in August. For other districts, the solution won't be an
    all-or-nothing approach.
    \href{https://bioethics.jhu.edu/research-and-outreach/projects/eschool-initiative/school-policy-tracker/}{Many
    systems}, including the nation's largest, New York City, are
    devising
    \href{https://www.nytimes.com/2020/06/26/us/coronavirus-schools-reopen-fall.html?action=click\&pgtype=Article\&state=default\&region=MAIN_CONTENT_3\&context=storylines_faq}{hybrid
    plans} that involve spending some days in classrooms and other days
    online. There's no national policy on this yet, so check with your
    municipal school system regularly to see what is happening in your
    community.
  \end{itemize}
\end{itemize}

This condition is a form of heart failure called cardiogenic shock, and
Jack's was ``pretty severe,'' said Dr. Steven Kernie, chief of pediatric
critical care medicine at the hospital and Columbia University. ``Over
all, his heart wasn't working very well,'' he said. ``It wasn't pumping
as strongly as normal.''

Doctors could not explain why Jack's heart function had suddenly become
impaired. Its structure and rhythm were normal. But blood vessels
throughout his body were inflamed, a condition called vasculitis, so the
vessels' muscles were ``not controlling blood flow as well as they
should,'' Dr. Kernie said.

Doctors also suspected that the heart was inflamed, known as
myocarditis, which in untreated serious cases can cause lasting damage.

Jack's condition was not only distressing, it reflected a frightening
new pattern. ``I remember that morning having admitted multiple children
with a similar syndrome,'' Dr. Connors said, ``and it was kind of like,
`What's going on here?'''

The inflammation seemed driven by a hyperactive immune response, and
Jack received medication for bacterial infection until tests ruled that
out. ``Whenever kids come in in shock you have to treat for
everything,'' Dr. Kernie said.

Jack's positive coronavirus test was a clue, but others with similar
symptoms had negative diagnostic test results, Dr. Connors said. The
doctors then decided to check the other children for evidence of the
coronavirus with a different test, one for antibodies, which signal they
had an earlier, no-longer-active infection. Most children ended up
having either a positive diagnostic or antibody test result.

By April 29, Jack's third day in the I.C.U., the blood pressure
medication was not helping enough and doctors began planning to insert a
central line through his groin to deliver additional medications. They
also prepared to put Jack, who was receiving nasal oxygen, on a
ventilator, something doctors deem necessary when ``your heart's not
doing its job,'' Dr. Connors said. ``We didn't know which way this was
going.''

The situation, especially the prospect of a ventilator, was terrifying
to Mr. McMorrow, 51, who stayed in Jack's hospital room round-the-clock,
and Ms. Stroman, 52, who was at home communicating by text and FaceTime
because only one parent was allowed in the hospital.

``You had a cardiologist, a pulmonary specialist, infectious disease
experts all throwing numbers and prescriptions to each other, and this
is stuff that's French to me,'' Mr. McMorrow said.

Jack mustered the energy to ask the doctors questions. ``I needed to
know because how am I supposed to fight something I don't know I'm
fighting,'' he said.

He concluded that his condition essentially boiled down to: ``Your
coronary and pulmonary responses come back and bite you in the butt.''

But then doctors began giving Jack steroids, which can have
anti-inflammatory and immunosuppressant effects. At last, something
seemed to work. Within hours, Jack needed less blood pressure
medication. As the family's pediatrician, Dr. Ganea, who has training in
infectious diseases and spoke to the hospital team, put it: ``Jack
turned into a normal Jack.''

Doctors are not sure the steroids made the difference, but since then,
they have administered them much earlier to children with the syndrome,
with encouraging results, Dr. Kernie said.

But Jack was not out of the woods even after moving to a regular
hospital room. His heart rate was in the 30s, about half what it should
be. The low heart rate might have been because of the steroids, doctors
said, but they could not be sure, so they moved Jack to a unit with
continual cardiac monitoring.

Over the next week, Jack recovered. He emailed his biology teacher from
his hospital bed: ``I would like to thank you for educating me as you
did, and for providing me the educational support to understand my body
when I need to most.''

His mother knew Jack was his old self when, on the phone, he asked to
speak with his sister, quoting the family's favorite movie, ``Midnight
Run'': ``Is this moron No. 1? Put moron No. 2 on the phone.''

Image

Jack's mother, Doris Stroman, clasped his wrist as she recalled his
frightening ordeal.Credit...Gabriela Bhaskar for The New York Times

Image

Jack and his parents on the balcony of their Woodside, Queens,
apartment, a few days after he came home from the
hospital.Credit...Gabriela Bhaskar for The New York Times

On May 7, 10 days after being hospitalized, Jack went home and traipsed
around the apartment channeling Pinocchio: ``I'm a boy! There are no
strings on me!''

He will require follow-up cardiology appointments and will take steroids
and blood thinners for a while. He may have some heart-valve tears and
residual cardiac inflammation, but doctors expect those to heal on their
own. Jack and his family have taken genetic tests as part of research
into the syndrome, and he and other survivors will be followed as
doctors strive to learn how to recognize and treat it.

Pausing near a model of Darth Vader's castle on his desk, Jack said he
once considered becoming an actor. He was even an extra on the TV show
``Gotham,'' playing a kidnapped orphan. But before getting sick, he was
thinking about studying medicine. ``I was really into the heart,'' he
said. Now, he is even more interested.

``I just want to do more with my life now that I have it back,'' he
said, gesturing with his Captain America shield.

Advertisement

\protect\hyperlink{after-bottom}{Continue reading the main story}

\hypertarget{site-index}{%
\subsection{Site Index}\label{site-index}}

\hypertarget{site-information-navigation}{%
\subsection{Site Information
Navigation}\label{site-information-navigation}}

\begin{itemize}
\tightlist
\item
  \href{https://help.nytimes.com/hc/en-us/articles/115014792127-Copyright-notice}{©~2020~The
  New York Times Company}
\end{itemize}

\begin{itemize}
\tightlist
\item
  \href{https://www.nytco.com/}{NYTCo}
\item
  \href{https://help.nytimes.com/hc/en-us/articles/115015385887-Contact-Us}{Contact
  Us}
\item
  \href{https://www.nytco.com/careers/}{Work with us}
\item
  \href{https://nytmediakit.com/}{Advertise}
\item
  \href{http://www.tbrandstudio.com/}{T Brand Studio}
\item
  \href{https://www.nytimes.com/privacy/cookie-policy\#how-do-i-manage-trackers}{Your
  Ad Choices}
\item
  \href{https://www.nytimes.com/privacy}{Privacy}
\item
  \href{https://help.nytimes.com/hc/en-us/articles/115014893428-Terms-of-service}{Terms
  of Service}
\item
  \href{https://help.nytimes.com/hc/en-us/articles/115014893968-Terms-of-sale}{Terms
  of Sale}
\item
  \href{https://spiderbites.nytimes.com}{Site Map}
\item
  \href{https://help.nytimes.com/hc/en-us}{Help}
\item
  \href{https://www.nytimes.com/subscription?campaignId=37WXW}{Subscriptions}
\end{itemize}
