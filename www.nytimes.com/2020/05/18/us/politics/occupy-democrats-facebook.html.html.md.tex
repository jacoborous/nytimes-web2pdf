Sections

SEARCH

\protect\hyperlink{site-content}{Skip to
content}\protect\hyperlink{site-index}{Skip to site index}

\href{https://www.nytimes.com/section/politics}{Politics}

\href{https://myaccount.nytimes.com/auth/login?response_type=cookie\&client_id=vi}{}

\href{https://www.nytimes.com/section/todayspaper}{Today's Paper}

\href{/section/politics}{Politics}\textbar{}How Immigrant Twin Brothers
Are Beating Trump's Team on Facebook

\url{https://nyti.ms/3dSyu1C}

\begin{itemize}
\item
\item
\item
\item
\item
\item
\end{itemize}

\begin{itemize}
\item
  \href{https://www.nytimes.com/2020/07/31/us/elections/biden-vs-trump.html?action=click\&pgtype=Article\&state=default\&region=TOP_BANNER\&context=storylines_menu}{Election
  Updates}
\item
  \href{https://www.nytimes.com/article/biden-vice-president-2020.html?action=click\&pgtype=Article\&state=default\&region=TOP_BANNER\&context=storylines_menu}{Biden's
  V.P. Search}
\item
  \href{https://www.nytimes.com/interactive/2020/07/24/us/politics/trump-biden-campaign-donors.html?action=click\&pgtype=Article\&state=default\&region=TOP_BANNER\&context=storylines_menu}{Map
  of Donations}
\item
  \href{https://www.nytimes.com/interactive/2020/us/elections/delegate-count-primary-results.html?action=click\&pgtype=Article\&state=default\&region=TOP_BANNER\&context=storylines_menu}{Delegate
  Count}
\item
  \href{https://www.nytimes.com/interactive/2019/us/politics/2020-presidential-candidates.html?action=click\&pgtype=Article\&state=default\&region=TOP_BANNER\&context=storylines_menu}{The
  Candidates}
\item
  \href{https://www.nytimes.com/newsletters/politics?action=click\&pgtype=Article\&state=default\&region=TOP_BANNER\&context=storylines_menu}{Politics
  Newsletter}
\end{itemize}

Advertisement

\protect\hyperlink{after-top}{Continue reading the main story}

Supported by

\protect\hyperlink{after-sponsor}{Continue reading the main story}

\hypertarget{how-immigrant-twin-brothers-are-beating-trumps-team-on-facebook}{%
\section{How Immigrant Twin Brothers Are Beating Trump's Team on
Facebook}\label{how-immigrant-twin-brothers-are-beating-trumps-team-on-facebook}}

Occupy Democrats, a Facebook page that Rafael and Omar Rivero started
eight years ago, has emerged as a counterweight to right-wing meme
machines.

\includegraphics{https://static01.nyt.com/images/2020/05/18/us/politics/00occupydems-1/00occupydems-1-articleLarge-v2.jpg?quality=75\&auto=webp\&disable=upscale}

\href{https://www.nytimes.com/by/nick-corasaniti}{\includegraphics{https://static01.nyt.com/images/2018/06/13/multimedia/author-nick-corasaniti/author-nick-corasaniti-thumbLarge-v2.png}}

By \href{https://www.nytimes.com/by/nick-corasaniti}{Nick Corasaniti}

\begin{itemize}
\item
  May 18, 2020
\item
  \begin{itemize}
  \item
  \item
  \item
  \item
  \item
  \item
  \end{itemize}
\end{itemize}

\href{https://www.nytimes.com/es/2020/05/21/espanol/occupy-democrats-facebook-trump.html}{Leer
en español}

It was a campaign video that reached seemingly every Democratic corner
of the internet: former President Barack Obama's 12-minute endorsement
of his former vice president and indictment of the current president. On
Mr. Obama's Facebook page, one of the most popular destinations in
politics with 55.3 million followers, his endorsement of
\href{https://www.nytimes.com/interactive/2020/us/elections/joe-biden.html}{Joseph
R. Biden Jr.} was viewed more than four million times.

But another Facebook page, run by twin brothers who immigrated from
Mexico, reached substantially more eyes. Their reposting of Mr. Obama's
endorsement, with a simple ``BREAKING'' text over the video, clocked
over 23 million views.

Meet Rafael and Omar Rivero, the co-founders of Occupy Democrats, the
social media mavens of the left who are quickly emerging as a
counterweight to the dominance of right-wing online sites.

In a presidential race playing out on iPhones and screens more than any
in history, in part because of the coronavirus pandemic, these digital
entrepreneurs can drive the political conversation online and influence
how candidates are seen as much as a campaign's well-funded digital
advisers can.

The twins, 33, started the Occupy Democrats Facebook page eight years
ago and, combined with an accompanying website, they have reached a
digital dominance rarely seen among liberals --- one that keeps pace
with viral news sites and regularly outperforms
\href{https://www.nytimes.com/interactive/2020/us/elections/donald-trump.html}{President
Trump}'s own page, as well as the Daily Caller, Fox News and other
right-wing websites or personalities. What was once a hobby between gigs
has grown into a full-fledged, full-time operation with five additional
staffers.

Over the past month, nearly half of the 40 top-performing videos on
Facebook that mention ``Trump'' were from Occupy Democrats. They have
had a top-10 performing post on Facebook regularly for months. A video
they recently posted
\href{https://www.facebook.com/watch/?v=236673061013278}{called ``The
Liar Tweets Tonight,''} sung by a choir of individually recorded voters
to the tune of ``The Lion Sleeps Tonight,'' was viewed 41 million times,
among the most-watched videos on Facebook over the last month.

``Democratic voters are tired of the Democratic Party kind of taking
barrages from Republicans on the right on social media and Trump
himself, taking that lying down and not fighting back,'' Omar Rivero
said. ``So we fight back with the truth. But we make sure that we punch
them in the mouth with the truth.''

Though they claim not to have taken tactics from the right, there are
some clear commonalities between the Occupy Democrats posts and some of
the right-wing sites that have mastered the art of writing shareable
copy that acts like gasoline on a social media outrage fire ---
amplified by anger-inducing adjectives contextualizing the news, or an
all-caps ``BREAKING'' to introduce a post.

They also are relentless in their posting on Facebook. On Sunday, a
relatively slow news day by the Trump-era pandemic standards, the
Facebook page published 80 items, a mix of original, text-heavy memes;
cross-posts from Mr. Biden's social media pages; commentary with links
to mainstream news stories and videos; and links to original posts on
the Occupy Democrats website.

It has helped them earn 25 million more interactions than Mr. Trump's
page, and 63 million more interactions than Mr. Biden's over the past 30
days on Facebook, according to CrowdTangle, a Facebook-owned data tool.

\hypertarget{latest-updates-2020-election}{%
\section{\texorpdfstring{\href{https://www.nytimes.com/2020/07/31/us/elections/biden-vs-trump.html?action=click\&pgtype=Article\&state=default\&region=MAIN_CONTENT_1\&context=storylines_live_updates}{Latest
Updates: 2020
Election}}{Latest Updates: 2020 Election}}\label{latest-updates-2020-election}}

Updated 2020-08-01T01:26:45.732Z

\begin{itemize}
\tightlist
\item
  \href{https://www.nytimes.com/2020/07/31/us/elections/biden-vs-trump.html?action=click\&pgtype=Article\&state=default\&region=MAIN_CONTENT_1\&context=storylines_live_updates\#link-29fdff45}{Kamala
  Harris, a top vice-presidential contender, confronts double
  standards.}
\item
  \href{https://www.nytimes.com/2020/07/31/us/elections/biden-vs-trump.html?action=click\&pgtype=Article\&state=default\&region=MAIN_CONTENT_1\&context=storylines_live_updates\#link-13ec3d9c}{Karen
  Bass and Susan Rice are rising on Biden's vice-presidential
  shortlist.}
\item
  \href{https://www.nytimes.com/2020/07/31/us/elections/biden-vs-trump.html?action=click\&pgtype=Article\&state=default\&region=MAIN_CONTENT_1\&context=storylines_live_updates\#link-49e9a016}{Trump
  says Russian bounties to kill U.S. troops `never took place.'}
\end{itemize}

\href{https://www.nytimes.com/2020/07/31/us/elections/biden-vs-trump.html?action=click\&pgtype=Article\&state=default\&region=MAIN_CONTENT_1\&context=storylines_live_updates}{See
more updates}

Occupy Democrats is a rare bright spot for a party and political wing
that once was proudly ``the party of tech'' but has since ceded nearly
every digital stronghold to the right. As Mr. Biden, the presumptive
Democratic presidential nominee, is moving headlong into a general
election with a digital operation that is dwarfed by the Trump campaign,
the Occupy team has started to step in.

``I think one of the big mistakes of 2016 was not immediately embracing
Hillary as a change agent and as someone to get excited about,'' Rafael
Rivero said.

Rafael also wanted to prove that, yes, Mr. Biden could indeed go viral.

On the same day Senator Bernie Sanders suspended his campaign, Rafael
started the ``Ridin' with Biden 2020'' page, employing meme tactics,
social media copy and video promotion similar to those that power the
central Occupy Democrats page.

An
\href{https://www.facebook.com/ridinwithbiden2020/posts/119595033036839}{Avengers-esque
meme of Mr. Biden}, Jill Biden, Mr. Obama and Michelle Obama striding
across the White House lawn overlaid with the text ``When America Was
Great'' reached 2.2 million viewers.

Soon, the ``Ridin' With Biden'' page was outperforming the campaign's
own account, with their own content.

A
\href{https://www.facebook.com/7860876103/videos/3085674444800331}{digital
video ad released} by the Biden campaign received more than one million
views on Facebook, a successful showing for most campaigns. But on
``Ridin' With Biden,'' it got 8.6 million views, with little added
window dressing than text on the video that read: ``Holy cow \ldots{}
this Biden ad is GOOD 🔥🔥🔥.''

When the Occupy page shared the live video of Hillary Clinton endorsing
Mr. Biden, the live viewership jumped from 15,000 to 25,000 in a matter
of minutes.

Democratic campaign operatives note that these types of booming online
communities benefit from being a bit rougher around the edges.

``They're able to say things that are not quite as polished as what the
parties are going to produce or what the Biden campaign is going to
produce, or any campaign,'' said Kenneth Pennington, a Democratic
digital strategist who was Mr. Sanders' digital director in 2016. ``But
it's kind of the unvarnished, unpolished stuff that actually does really
well online because people are seeking that kind of authentic sass.''

Mr. Pennington added that these types of pages can help boost a campaign
as well, crediting a different Facebook page --- The People for Bernie
Sanders --- as one of the reasons Mr. Sanders catapulted ``from a
no-shot candidate into an online sensation that raised \$230 million in
2016.''

While the social media dominance of Occupy Democrats may surprise some,
social media experts note that there has always been a liveliness among
liberal groups online, but they just get less attention.

Whitney Phillips, a media studies and communications professor at
Syracuse University, said the reported distress on the left about
``losing the edge on social media'' wasn't the full picture. ``The
framing is maybe not fully representing all of the activity and all the
vibrancy that's happening on the left because all the stories get
written about what Ben Shapiro is up to,'' she said, referring to the
popular conservative writer.

\includegraphics{https://static01.nyt.com/images/2020/05/17/us/politics/00occupydems-02/00occupydems-02-articleLarge.png?quality=75\&auto=webp\&disable=upscale}

Occupy Democrats does try to give readers their vegetables, too. A post
about Senator Mitch McConnell's comments on the newest Democratic
proposal for a coronavirus relief package, for example, included
highlights from the Democratic proposal.

``People clicked to be mad about McConnell, but while they read about
that, they learn about what the Democrats are doing,'' said Colin
Taylor, the editor in chief of Occupy Democrats. ``In this kind of
outrage-heavy online sphere, it's kind of hard to get people's attention
with the more wonky stuff.''

The group's origins date to the movement that informs its name --- and a
dissatisfaction with it. The Rivero twins found themselves in Zuccotti
Park back in 2012, when the Occupy movement had camped out in Lower
Manhattan and quickly garnered a national news profile. Both brothers
were drawn to the ideals of the movement --- economic equality, social
justice and addressing climate change --- but they saw the Occupy
movement's leaderless ethos as a critical failure, and one that would
never allow it to grow.

``I looked around and thought, wow, there's not a single Occupy
congressman,'' Omar Rivero said. ``In the end, we're not pulling the
levers of power. So I thought, well, you know, maybe we should try to
make Occupy a force that not only helps Democrats but also keeps them
honest. Similar to what the Tea Party is doing to Republicans.''

Omar started the Facebook page after leaving a job at an investment bank
--- ``working for the man at a bank'' --- heavy in debt to both the
federal government and his mother after earning an undergraduate degree
from Cornell University and a master's degree from ESCP Europe. In
between side gigs cleaning short-term rental apartments, and sometimes
while cleaning, he continued to build an audience. But he needed help
with the visuals.

Omar turned to his twin brother, who was running his own real-estate
rental business in Miami at the time, to put his graphic design
background to use and join the effort.

``My mom staged an intervention, literally, with my aunt and uncle,''
Rafael said. ``They thought, you guys got the scholarships to Swarthmore
and Cornell, and you guys are throwing all that away to focus on
something called a Facebook page.''

But with Rafael making memes, the page began to grow. Quickly. The two
moved to California and lived in a friend's pool house. When the demand
for content grew beyond the twins' capabilities, they posted hiring ads
on Craigslist. Mr. Taylor, a former line cook who blogged on the side,
was their first hire, and the pool house soon doubled as the Occupy
newsroom, now with multiple writers churning out dozens of posts a day,
building both the Facebook page and the website into traffic machines.

They survived the reorientation of the Facebook algorithm after the 2016
election --- which pushed down independent, less verified sites in favor
of more mainstream news content --- by repeatedly boosting and sharing
mainstream news articles, introduced with their own spin. Though they
had to lay off a few writers in the wake of those Facebook changes, they
have kept churning out content.

And, of course, countless memes.

The memes and videos are what generate the most engagement, and Occupy
Democrats' white-and-yellow text on a black background both grabs the
eye with its harsh color contrast and conveys a sense of urgency.
Distilling the news into a single shareable photo that remains on
Facebook has quickly caught on, particularly among older users.

But with this newfound power, the Rivero brothers want to expand and
build a broader network with other Democrats.

``We're not only the largest political network on Facebook, but we're
the largest partisan political network on Facebook,'' said Omar. ``And I
think that the Democrats should take advantage of that.''

\hypertarget{our-2020-election-guide}{%
\section{Our 2020 Election Guide}\label{our-2020-election-guide}}

Updated July 31, 2020

\begin{itemize}
\item
  \begin{center}\rule{0.5\linewidth}{\linethickness}\end{center}

  \hypertarget{the-latest}{%
  \subsection{The Latest}\label{the-latest}}

  \begin{itemize}
  \tightlist
  \item
    President Trump's assault on the Postal Service is intersecting with
    his attacks on mail-in voting.
    \href{https://www.nytimes.com/2020/07/31/us/politics/trump-usps-mail-delays.html?action=click\&pgtype=Article\&state=default\&region=BELOW_MAIN_CONTENT\&context=storylines_guide}{Voting
    rights groups say it is a recipe for disaster.}
  \end{itemize}
\item
  \begin{center}\rule{0.5\linewidth}{\linethickness}\end{center}

  \hypertarget{bidens-vp-search}{%
  \subsection{Biden's V.P. Search}\label{bidens-vp-search}}

  \begin{itemize}
  \tightlist
  \item
    \href{https://www.nytimes.com/article/biden-vice-president-2020.html?action=click\&pgtype=Article\&state=default\&region=BELOW_MAIN_CONTENT\&context=storylines_guide}{Here
    are 13 women} who have been under consideration to be Joe Biden's
    running mate, and why each might be chosen --- and might not be.
  \end{itemize}
\item
  \begin{center}\rule{0.5\linewidth}{\linethickness}\end{center}

  \hypertarget{keep-up-with-our-coverage}{%
  \subsection{Keep Up With Our
  Coverage}\label{keep-up-with-our-coverage}}

  \begin{itemize}
  \tightlist
  \item
    Get an
    \href{https://www.nytimes.com/newsletters/politics?action=click\&pgtype=Article\&state=default\&region=BELOW_MAIN_CONTENT\&context=storylines_guide}{email}
    recapping the day's news
  \end{itemize}

  \begin{itemize}
  \tightlist
  \item
    Download our mobile app on
    \href{https://apps.apple.com/us/app/nytimes/id284862083?ls=1\&mat_click_id=5c79ae7455014fd1bd66b5610c05b8f2-20191112-16948\&referrer=mat_click_id\%3D5c79ae7455014fd1bd66b5610c05b8f2-20191112-16948\%26link_click_id\%3D722930677036718082}{iOS}
    and
    \href{http://a.localytics.com/android?id=com.nytimes.android\&referrer=utm_source\%3Dother_nyt_mobile_web\%26utm_medium\%3DWeb\%2520page\%26utm_term\%3DGeneral\%2520Mobile\%2520Page\%26utm_campaign\%3DNYT\%2520Mobile\%2520General\%2520Page}{Android}
    and turn on Breaking News and Politics alerts
  \end{itemize}
\end{itemize}

Advertisement

\protect\hyperlink{after-bottom}{Continue reading the main story}

\hypertarget{site-index}{%
\subsection{Site Index}\label{site-index}}

\hypertarget{site-information-navigation}{%
\subsection{Site Information
Navigation}\label{site-information-navigation}}

\begin{itemize}
\tightlist
\item
  \href{https://help.nytimes.com/hc/en-us/articles/115014792127-Copyright-notice}{©~2020~The
  New York Times Company}
\end{itemize}

\begin{itemize}
\tightlist
\item
  \href{https://www.nytco.com/}{NYTCo}
\item
  \href{https://help.nytimes.com/hc/en-us/articles/115015385887-Contact-Us}{Contact
  Us}
\item
  \href{https://www.nytco.com/careers/}{Work with us}
\item
  \href{https://nytmediakit.com/}{Advertise}
\item
  \href{http://www.tbrandstudio.com/}{T Brand Studio}
\item
  \href{https://www.nytimes.com/privacy/cookie-policy\#how-do-i-manage-trackers}{Your
  Ad Choices}
\item
  \href{https://www.nytimes.com/privacy}{Privacy}
\item
  \href{https://help.nytimes.com/hc/en-us/articles/115014893428-Terms-of-service}{Terms
  of Service}
\item
  \href{https://help.nytimes.com/hc/en-us/articles/115014893968-Terms-of-sale}{Terms
  of Sale}
\item
  \href{https://spiderbites.nytimes.com}{Site Map}
\item
  \href{https://help.nytimes.com/hc/en-us}{Help}
\item
  \href{https://www.nytimes.com/subscription?campaignId=37WXW}{Subscriptions}
\end{itemize}
