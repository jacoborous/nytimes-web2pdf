Sections

SEARCH

\protect\hyperlink{site-content}{Skip to
content}\protect\hyperlink{site-index}{Skip to site index}

\href{https://www.nytimes.com/section/books}{Books}

\href{https://myaccount.nytimes.com/auth/login?response_type=cookie\&client_id=vi}{}

\href{https://www.nytimes.com/section/todayspaper}{Today's Paper}

\href{/section/books}{Books}\textbar{}A `Hunger Games' Prequel Focuses
on an Unlikely Character

\url{https://nyti.ms/2zaLqRX}

\begin{itemize}
\item
\item
\item
\item
\item
\item
\end{itemize}

\href{https://www.nytimes.com/spotlight/at-home?action=click\&pgtype=Article\&state=default\&region=TOP_BANNER\&context=at_home_menu}{At
Home}

\begin{itemize}
\tightlist
\item
  \href{https://www.nytimes.com/2020/07/28/books/time-for-a-literary-road-trip.html?action=click\&pgtype=Article\&state=default\&region=TOP_BANNER\&context=at_home_menu}{Take:
  A Literary Road Trip}
\item
  \href{https://www.nytimes.com/2020/07/29/magazine/bored-with-your-home-cooking-some-smoky-eggplant-will-fix-that.html?action=click\&pgtype=Article\&state=default\&region=TOP_BANNER\&context=at_home_menu}{Cook:
  Smoky Eggplant}
\item
  \href{https://www.nytimes.com/2020/07/27/travel/moose-michigan-isle-royale.html?action=click\&pgtype=Article\&state=default\&region=TOP_BANNER\&context=at_home_menu}{Look
  Out: For Moose}
\item
  \href{https://www.nytimes.com/interactive/2020/at-home/even-more-reporters-editors-diaries-lists-recommendations.html?action=click\&pgtype=Article\&state=default\&region=TOP_BANNER\&context=at_home_menu}{Explore:
  Reporters' Obsessions}
\end{itemize}

Advertisement

\protect\hyperlink{after-top}{Continue reading the main story}

Supported by

\protect\hyperlink{after-sponsor}{Continue reading the main story}

\href{/column/books-of-the-times}{Books of The Times}

\hypertarget{a-hunger-games-prequel-focuses-on-an-unlikely-character}{%
\section{A `Hunger Games' Prequel Focuses on an Unlikely
Character}\label{a-hunger-games-prequel-focuses-on-an-unlikely-character}}

\includegraphics{https://static01.nyt.com/images/2020/05/20/books/17BOOKCOLLINS1/17BOOKCOLLINS1-articleLarge-v3.png?quality=75\&auto=webp\&disable=upscale}

Buy Book ▾

\begin{itemize}
\tightlist
\item
  \href{https://www.amazon.com/gp/search?index=books\&tag=NYTBSREV-20\&field-keywords=The+Ballad+of+Songbirds+and+Snakes+Suzanne+Collins}{Amazon}
\item
  \href{https://du-gae-books-dot-nyt-du-prd.appspot.com/buy?title=The+Ballad+of+Songbirds+and+Snakes\&author=Suzanne+Collins}{Apple
  Books}
\item
  \href{https://www.anrdoezrs.net/click-7990613-11819508?url=https\%3A\%2F\%2Fwww.barnesandnoble.com\%2Fw\%2F\%3Fean\%3D9781338635171}{Barnes
  and Noble}
\item
  \href{https://www.anrdoezrs.net/click-7990613-35140?url=https\%3A\%2F\%2Fwww.booksamillion.com\%2Fp\%2FThe\%2BBallad\%2Bof\%2BSongbirds\%2Band\%2BSnakes\%2FSuzanne\%2BCollins\%2F9781338635171}{Books-A-Million}
\item
  \href{https://bookshop.org/a/3546/9781338635171}{Bookshop}
\item
  \href{https://www.indiebound.org/book/9781338635171?aff=NYT}{Indiebound}
\end{itemize}

When you purchase an independently reviewed book through our site, we
earn an affiliate commission.

By \href{https://www.nytimes.com/by/sarah-lyall}{Sarah Lyall}

\begin{itemize}
\item
  Published May 19, 2020Updated May 20, 2020
\item
  \begin{itemize}
  \item
  \item
  \item
  \item
  \item
  \item
  \end{itemize}
\end{itemize}

Reading Suzanne Collins's ``Hunger Games'' trilogy, which concluded a
decade ago, was a feverish, disturbing, exhilarating, all-consuming
experience. Its premise was horrible: that in the dystopian world of
Panem, built on the ruins of what was once North America, young people
who were selected each year by lottery would fight to the death,
gladiator-style, while an enthralled nation followed along on TV.

More than 100 million copies of the books are in print, and their
immense popularity is due largely to their spectacularly
\href{https://www.nytimes.com/2018/10/18/books/katniss-everdeen-hunger-games.html}{charismatic
heroine, Katniss Everdeen}, with her rebel's bravery, her hunter's
cunning and her burning desire for justice. If she was the best thing
about Panem, then its president, the creepy Coriolanus Snow, was the
worst. A horrible mix of Machiavelli, Nero and Richard III, he famously
wore a rose to mask the stench of blood in his ulcerated mouth (the
result of ingesting poison).

And now, in the tradition of movies like
``\href{https://www.nytimes.com/2019/10/03/movies/joker-review.html}{Joker},''
which reveals that one of Batman's nemeses was a standup comedian who
lost access to his medication, and the
\href{https://www.nytimes.com/2019/05/17/movies/star-wars-phantom-menace-anniversary.html}{Star
Wars prequels}, in which it turns out that Darth Vader was once a heroic
Jedi knight, comes ``The Ballad of Songbirds and Snakes.'' This prequel,
set 64 years before the original books, stars Coriolanus as a confused,
impoverished 18-year-old high school student yearning for good grades
and world domination.

\emph{{[}}
\href{https://www.nytimes.com/2020/04/23/arts/new-may-books.html}{\emph{This
book was one of our most anticipated titles of May. See the full
list}}\emph{. {]}}

It is a steep challenge to write a book whose hero is, everyone knows,
destined to become deeply evil. Do we want to hear --- now, after we
know the endgame --- that the young Voldemort was unfairly saddled with
a demerit in class or that the adolescent Sauron fretted because he had
to wear hand-me-down clothes?

Yes, please. (Apologies to those who like their closed fictional worlds
to remain intact.) ``The Ballad of Songbirds and Snakes'' takes us to a
Panem still in the dark days of reconstruction after the districts'
failed rebellion against the dictatorial Capitol. Much like the outlying
lands subjugated by a rapacious central government in ancient Rome ---
one of the inspirations for the story, Collins has said, and the reason
so many characters' names are plucked from Roman history --- the
districts have paid a dear price for their treachery, living under
martial law, laboring to provide products for the much richer Capitol
and giving up their children as tributes in the Hunger Games.

\emph{{[}}
\href{https://www.nytimes.com/2008/11/09/books/review/Green-t.html}{\emph{Read
John Green's review of ``The Hunger Games.''}} \emph{{]}}

After a Stalingrad-like siege in which people starved outside their
homes, their corpses cannibalized by neighbors, the Capitol isn't yet
the rich center of decadent excess it will become. Food is scarce.
Rubble lines the streets. Everything is used and reused. Everyone has
PTSD. ``What a luxury trash would be,'' young Coriolanus thinks.

Life is not easy for him. As the scion of an upper-crust family fallen
into shameful penury, he has to keep up appearances. (``Snow lands on
top,'' he and a cousin tell one another.) He has to excel at the elite
Academy, earn a scholarship to college and fulfill what he sees as his
manifest destiny. And, as the book begins, he has to navigate his way
through one of the hardest assignments his class has ever faced: serving
as mentors for the tributes forced to participate in the 10th Hunger
Games.

As much as this is Coriolanus's origin story, it is an origin story for
the Games themselves, an answer to the questions about their history
posed by Katniss in ``Mockingjay,'' the final volume of the trilogy:
``Did a group of people sit around and cast their votes on initiating
the Hunger Games? Was there dissent? Did someone make a case for
mercy?''

Image

Suzanne CollinsCredit...Todd Pitt

People who love finding out the back stories in fictional universes ---
why Sherlock Holmes wears a deerstalker hat; where Indiana Jones got his
scar --- will relish the chance to learn these details. Here, the Games
are still a miserable spectacle, a poor version of the lavish, grotesque
extravaganza they will become.

The children are shackled, carted to the Capitol on cattle trains and
then dumped in the monkey cage at the zoo. The Games take place in a
crumbling, dilapidated stadium still stained with the blood of past
losers; the participants are as likely to die of starvation or illness
as they are to be shot or cleaved to death by their opponents.

Which makes them no fun at all.

So Coriolanus and his classmates are asked to come up with ideas to make
the games more engaging, to get the public involved, to raise the TV
ratings. Think Pierre de Coubertin and the origins of the modern
Olympics. On second thought, don't.

One student suggests executing anyone who refuses to watch. Coriolanus's
proposal --- enabling viewers to place bets on the tributes, and to send
them food or water via drone --- is more like it. Not everyone is an
enthusiast. ``Who wants to watch a group of children kill each other?
Only a vicious, twisted person,'' grouses the most rebellious of the
students.

As in the trilogy, the descriptions of the Games themselves --- scenes
in which blameless teenagers poison, beat, stab, trident and ax each
other to death while adults debate tactics from afar --- are hard to
read but hard to turn away from. This is violence porn. It is disturbing
that we find it so compelling. It also means that the book inevitably
loses some of its propulsive bite when the Games end and the action
moves out of the Capitol. Parts of the last fourth of the novel feel
flat and desultory after the excitement we have just been through.

The standout heroine is Coriolanus's mentee, Lucy Gray Baird from
District 12. (That was Katniss's district. Alert readers will recognize
some neat connective threads.) She is scrappy, charming, fearless,
grown-up before her time and a natural on camera. Like Harry Potter, she
appears to be a parselmouth, able to talk to snakes. Her jauntiness
occasionally grates. Coriolanus falls for her, but will they end up
together? I'm afraid we can imagine the answer to that.

At times, Coriolanus is a sympathetic character. He recoils from
injustice. He is repulsed by the woman in charge of the Games, Volumnia
Gaul, a Mengele-esque scientist who has her own affinity for snakes and
whose idea of a good time is to melt the flesh off lab rats ``with some
sort of laser.''

But he is a snob and an opportunist, skilled in the art of looking out
for No. 1, even as he and his classmates debate human nature and the
morality of the Hunger Games. His slide into evil seems the result of
inertia and greed, not a specific come-to-Satan moment.

It isn't until the final pages that you learn the real answer to
Katniss's question, how the very first Hunger Games began, and the
anguish they have brought to the architect behind the original proposal.
As that character says, before meeting an untimely end: ``Who but the
vilest monster would stage it?''

Advertisement

\protect\hyperlink{after-bottom}{Continue reading the main story}

\hypertarget{site-index}{%
\subsection{Site Index}\label{site-index}}

\hypertarget{site-information-navigation}{%
\subsection{Site Information
Navigation}\label{site-information-navigation}}

\begin{itemize}
\tightlist
\item
  \href{https://help.nytimes.com/hc/en-us/articles/115014792127-Copyright-notice}{©~2020~The
  New York Times Company}
\end{itemize}

\begin{itemize}
\tightlist
\item
  \href{https://www.nytco.com/}{NYTCo}
\item
  \href{https://help.nytimes.com/hc/en-us/articles/115015385887-Contact-Us}{Contact
  Us}
\item
  \href{https://www.nytco.com/careers/}{Work with us}
\item
  \href{https://nytmediakit.com/}{Advertise}
\item
  \href{http://www.tbrandstudio.com/}{T Brand Studio}
\item
  \href{https://www.nytimes.com/privacy/cookie-policy\#how-do-i-manage-trackers}{Your
  Ad Choices}
\item
  \href{https://www.nytimes.com/privacy}{Privacy}
\item
  \href{https://help.nytimes.com/hc/en-us/articles/115014893428-Terms-of-service}{Terms
  of Service}
\item
  \href{https://help.nytimes.com/hc/en-us/articles/115014893968-Terms-of-sale}{Terms
  of Sale}
\item
  \href{https://spiderbites.nytimes.com}{Site Map}
\item
  \href{https://help.nytimes.com/hc/en-us}{Help}
\item
  \href{https://www.nytimes.com/subscription?campaignId=37WXW}{Subscriptions}
\end{itemize}
