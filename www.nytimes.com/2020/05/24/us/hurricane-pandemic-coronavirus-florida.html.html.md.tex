Sections

SEARCH

\protect\hyperlink{site-content}{Skip to
content}\protect\hyperlink{site-index}{Skip to site index}

\href{https://www.nytimes.com/section/us}{U.S.}

\href{https://myaccount.nytimes.com/auth/login?response_type=cookie\&client_id=vi}{}

\href{https://www.nytimes.com/section/todayspaper}{Today's Paper}

\href{/section/us}{U.S.}\textbar{}What Happens if a Hurricane Hits
During the Pandemic?

\url{https://nyti.ms/3cZVcVI}

\begin{itemize}
\item
\item
\item
\item
\item
\end{itemize}

\href{https://www.nytimes.com/news-event/coronavirus?action=click\&pgtype=Article\&state=default\&region=TOP_BANNER\&context=storylines_menu}{The
Coronavirus Outbreak}

\begin{itemize}
\tightlist
\item
  live\href{https://www.nytimes.com/2020/08/02/world/coronavirus-updates.html?action=click\&pgtype=Article\&state=default\&region=TOP_BANNER\&context=storylines_menu}{Latest
  Updates}
\item
  \href{https://www.nytimes.com/interactive/2020/us/coronavirus-us-cases.html?action=click\&pgtype=Article\&state=default\&region=TOP_BANNER\&context=storylines_menu}{Maps
  and Cases}
\item
  \href{https://www.nytimes.com/interactive/2020/science/coronavirus-vaccine-tracker.html?action=click\&pgtype=Article\&state=default\&region=TOP_BANNER\&context=storylines_menu}{Vaccine
  Tracker}
\item
  \href{https://www.nytimes.com/interactive/2020/07/29/us/schools-reopening-coronavirus.html?action=click\&pgtype=Article\&state=default\&region=TOP_BANNER\&context=storylines_menu}{What
  School May Look Like}
\item
  \href{https://www.nytimes.com/live/2020/07/31/business/stock-market-today-coronavirus?action=click\&pgtype=Article\&state=default\&region=TOP_BANNER\&context=storylines_menu}{Economy}
\end{itemize}

Advertisement

\protect\hyperlink{after-top}{Continue reading the main story}

Supported by

\protect\hyperlink{after-sponsor}{Continue reading the main story}

\hypertarget{what-happens-if-a-hurricane-hits-during-the-pandemic}{%
\section{What Happens if a Hurricane Hits During the
Pandemic?}\label{what-happens-if-a-hurricane-hits-during-the-pandemic}}

Florida is trying to figure out the daunting prospect of asking
residents to evacuate for their safety during a storm after asking them
to stay at home for the coronavirus.

\includegraphics{https://static01.nyt.com/images/2020/05/25/us/25virus-hurricane-01/merlin_145122714_7105ee51-9c19-44b5-ad3a-f6c7d8f587a2-articleLarge.jpg?quality=75\&auto=webp\&disable=upscale}

\href{https://www.nytimes.com/by/patricia-mazzei}{\includegraphics{https://static01.nyt.com/images/2018/11/28/multimedia/author-patricia-mazzei/author-patricia-mazzei-thumbLarge.png}}

By \href{https://www.nytimes.com/by/patricia-mazzei}{Patricia Mazzei}

\begin{itemize}
\item
  Published May 24, 2020Updated June 18, 2020
\item
  \begin{itemize}
  \item
  \item
  \item
  \item
  \item
  \end{itemize}
\end{itemize}

MIAMI --- It's deep into the summer and a massive
\href{https://www.nytimes.com/2020/07/26/us/hurricane-douglas-hawaii.html}{hurricane}
looms off the Florida coast, threatening enormous destruction and
widespread power blackouts. In normal times, in such a scenario, the
orders would come down for millions of coastal residents: Evacuate.

But in the middle of a pandemic, the most consequential of disaster
decisions become complicated by fears of contagion.

Temporarily moving in with a relative might expose older family members
to the coronavirus. Friends might be wary of letting in evacuees from
outside their quarantine bubble. People who might otherwise book a
flight out of town worry about getting infected on a plane. And the more
than 1.5 million
\href{https://www.nytimes.com/2020/04/23/us/florida-coronavirus-unemployment.html}{Floridians
who are out of work} might be unable to afford gas or a motel room.

What is left are emergency shelters, where hundreds of people crowd into
high school gymnasiums, share public bathrooms and line up for
buffet-style meals.

Gulp.

This is the planning dilemma now facing emergency managers across the
Southeast ahead of June 1, the official start of a
\href{https://www.nytimes.com/interactive/2020/07/25/us/hurricane-hanna-tracker-map.html}{hurricane}
season that meteorologists expect to be quite active. The National
Oceanic and Atmospheric Administration
\href{https://www.nytimes.com/2020/05/21/climate/hurricane-season-2020-noaa.html}{has
forecast} as many as six storms rated Category 3 or higher. A named
system,
\href{https://www.nytimes.com/2020/05/17/us/tropical-storm-arthur-2020-path.html}{Tropical
Storm Arthur}, already formed in May.

If a big storm comes this summer, people in harm's way may hear advice
from the authorities that is somewhat contradictory and perhaps
confusing: Stay at home and remain socially distant from others to avoid
contracting the coronavirus. But leave home --- even if that means
coming into closer contact with other people --- to be safe during a
dangerous hurricane.

\hypertarget{latest-updates-global-coronavirus-outbreak}{%
\section{\texorpdfstring{\href{https://www.nytimes.com/2020/08/01/world/coronavirus-covid-19.html?action=click\&pgtype=Article\&state=default\&region=MAIN_CONTENT_1\&context=storylines_live_updates}{Latest
Updates: Global Coronavirus
Outbreak}}{Latest Updates: Global Coronavirus Outbreak}}\label{latest-updates-global-coronavirus-outbreak}}

Updated 2020-08-02T17:52:35.962Z

\begin{itemize}
\tightlist
\item
  \href{https://www.nytimes.com/2020/08/01/world/coronavirus-covid-19.html?action=click\&pgtype=Article\&state=default\&region=MAIN_CONTENT_1\&context=storylines_live_updates\#link-34047410}{The
  U.S. reels as July cases more than double the total of any other
  month.}
\item
  \href{https://www.nytimes.com/2020/08/01/world/coronavirus-covid-19.html?action=click\&pgtype=Article\&state=default\&region=MAIN_CONTENT_1\&context=storylines_live_updates\#link-780ec966}{Top
  U.S. officials work to break an impasse over the federal jobless
  benefit.}
\item
  \href{https://www.nytimes.com/2020/08/01/world/coronavirus-covid-19.html?action=click\&pgtype=Article\&state=default\&region=MAIN_CONTENT_1\&context=storylines_live_updates\#link-2bc8948}{Its
  outbreak untamed, Melbourne goes into even greater lockdown.}
\end{itemize}

\href{https://www.nytimes.com/2020/08/01/world/coronavirus-covid-19.html?action=click\&pgtype=Article\&state=default\&region=MAIN_CONTENT_1\&context=storylines_live_updates}{See
more updates}

More live coverage:
\href{https://www.nytimes.com/live/2020/07/31/business/stock-market-today-coronavirus?action=click\&pgtype=Article\&state=default\&region=MAIN_CONTENT_1\&context=storylines_live_updates}{Markets}

``We're going to need to get people out, because that is the emergent
threat,'' said Jared Moskowitz, director of Florida's division of
emergency management. ``We will undoubtedly have to balance the risks.''

Some people in India and Bangladesh resisted evacuations when
\href{https://www.nytimes.com/2020/05/21/world/asia/cyclone-amphan-india-bangladesh.html}{a
powerful cyclone struck} last week. Communities in Michigan,
\href{https://www.nytimes.com/2020/05/20/us/michigan-flooding-dams-midland.html}{after
a river flooded and two dams were breached}, and in Arkansas,
\href{https://www.nytimes.com/2020/03/29/us/tornado-coronavirus-arkansas.html}{after
a tornado}, recently struggled with how to safely shelter large numbers
of people.

There is plenty of hurricane fatigue in Florida, which has endured hits
or brushes with at least five hurricanes over the past four years,
including
\href{https://www.nytimes.com/2017/09/09/us/irmas-fearsome-winds-reach-florida-shores-with-full-strike-yet-to-come.html}{Hurricane
Irma} in 2017,
\href{https://www.nytimes.com/2018/10/10/us/hurricane-michael-florida.html}{Hurricane
Michael} in 2018 and
\href{https://www.nytimes.com/2019/09/03/us/hurricane-dorian-updates.html}{Hurricane
Dorian} in 2019. The prospect of
\href{https://www.nytimes.com/2020/06/05/us/tropical-storm-cristobal-louisiana.html}{another
busy storm season} felt exhausting even before the arrival of Covid-19,
which has led to 50,000 cases and more than 2,000 deaths since the
beginning of March.

A mild storm might not require many evacuations. People with newer homes
built to withstand strong winds could be safer sheltering in place than
leaving their homes, Mr. Moskowitz said, as long as they do not live in
a low-lying area prone to storm surge.

\includegraphics{https://static01.nyt.com/images/2020/05/25/us/25virus-hurricane-02/merlin_127230902_d45c6dd3-493d-428d-9aad-7b5b0548b08e-articleLarge.jpg?quality=75\&auto=webp\&disable=upscale}

But experts always prepare for the worst case: a behemoth storm riding
up the entirety of the peninsula, or hitting a big city like Miami or
Tampa directly. During Irma, which made landfall in the Florida Keys and
moved north, some 350,000 people sought refuge in shelters.

In new storm guidelines, the
\href{https://www.cdc.gov/coronavirus/2019-ncov/downloads/Guidance-for-Gen-Pop-Disaster-Shelters-a-Pandemic_cleared_JIC_ADS_final.pdf}{Centers
for Disease Control and Prevention} recommended small shelters of fewer
than 50 people. But the
\href{https://www.fema.gov/media-library-data/1589997234798-adb5ce5cb98a7a89e3e1800becf0eb65/2020_Hurricane_Pandemic_Plan.pdf}{Federal
Emergency Management Agency} acknowledged that big shelters ``will still
be necessary.''

To find alternatives where evacuees might be more spread out, Mr.
Moskowitz's team created a map of hotels --- along with their wind
rating and whether they have a power generator --- that might be
commandeered as shelters. The division of emergency management also
developed an app that counties could use to assign evacuees to those
hotels.

Traditional school shelters will be unavoidable, at least in densely
populated areas, said Frank K. Rollason, the emergency management
director for Miami-Dade County. Only 20 hotels in Miami-Dade are outside
of a storm evacuation zone, he said, and many might be booked with
guests evacuated from coastal hotels or with crews deployed in advance
to restore electricity or phone service after the storm.

``We're looking at those, but this is the 11th hour,'' Mr. Rollason
said. ``It's a long shot.''

He has focused instead on how the county's 81 shelters, the largest of
which can usually accommodate up to 1,500 people, might adapt to prevent
virus spread: Set aside 36 square feet per person, up from the usual 20
square feet. Stagger meal times. Empty classrooms of furniture so they
could be used for large families, groups of symptomatic people or those
who have tested positive for the virus. It may be possible to designate
a specific shelter for those evacuees.

Mr. Rollason is not counting on rapid testing to become widely available
to reliably determine which evacuees are sick. Those entering shelters
will have their temperatures taken and be asked questions about symptoms
and exposure when they arrive, he said --- preferably indoors in some
sort of anteroom so that they do not have to stand in line outside being
pelted by rain.

``But they will come in,'' Mr. Rollason said. ``We're not going to turn
anyone away.''

Volunteers to work in the shelters alongside county employees
\href{https://www.nytimes.com/2020/05/22/climate/fema-volunteer-disaster-response.html}{would
be difficult to come by}. The state may assign its own workers or
temporarily hire unemployed people, Mr. Moskowitz said. Florida has set
aside 10 million masks for use during hurricanes, he added.

\href{https://www.nytimes.com/news-event/coronavirus?action=click\&pgtype=Article\&state=default\&region=MAIN_CONTENT_3\&context=storylines_faq}{}

\hypertarget{the-coronavirus-outbreak-}{%
\subsubsection{The Coronavirus Outbreak
›}\label{the-coronavirus-outbreak-}}

\hypertarget{frequently-asked-questions}{%
\paragraph{Frequently Asked
Questions}\label{frequently-asked-questions}}

Updated July 27, 2020

\begin{itemize}
\item ~
  \hypertarget{should-i-refinance-my-mortgage}{%
  \paragraph{Should I refinance my
  mortgage?}\label{should-i-refinance-my-mortgage}}

  \begin{itemize}
  \tightlist
  \item
    \href{https://www.nytimes.com/article/coronavirus-money-unemployment.html?action=click\&pgtype=Article\&state=default\&region=MAIN_CONTENT_3\&context=storylines_faq}{It
    could be a good idea,} because mortgage rates have
    \href{https://www.nytimes.com/2020/07/16/business/mortgage-rates-below-3-percent.html?action=click\&pgtype=Article\&state=default\&region=MAIN_CONTENT_3\&context=storylines_faq}{never
    been lower.} Refinancing requests have pushed mortgage applications
    to some of the highest levels since 2008, so be prepared to get in
    line. But defaults are also up, so if you're thinking about buying a
    home, be aware that some lenders have tightened their standards.
  \end{itemize}
\item ~
  \hypertarget{what-is-school-going-to-look-like-in-september}{%
  \paragraph{What is school going to look like in
  September?}\label{what-is-school-going-to-look-like-in-september}}

  \begin{itemize}
  \tightlist
  \item
    It is unlikely that many schools will return to a normal schedule
    this fall, requiring the grind of
    \href{https://www.nytimes.com/2020/06/05/us/coronavirus-education-lost-learning.html?action=click\&pgtype=Article\&state=default\&region=MAIN_CONTENT_3\&context=storylines_faq}{online
    learning},
    \href{https://www.nytimes.com/2020/05/29/us/coronavirus-child-care-centers.html?action=click\&pgtype=Article\&state=default\&region=MAIN_CONTENT_3\&context=storylines_faq}{makeshift
    child care} and
    \href{https://www.nytimes.com/2020/06/03/business/economy/coronavirus-working-women.html?action=click\&pgtype=Article\&state=default\&region=MAIN_CONTENT_3\&context=storylines_faq}{stunted
    workdays} to continue. California's two largest public school
    districts --- Los Angeles and San Diego --- said on July 13, that
    \href{https://www.nytimes.com/2020/07/13/us/lausd-san-diego-school-reopening.html?action=click\&pgtype=Article\&state=default\&region=MAIN_CONTENT_3\&context=storylines_faq}{instruction
    will be remote-only in the fall}, citing concerns that surging
    coronavirus infections in their areas pose too dire a risk for
    students and teachers. Together, the two districts enroll some
    825,000 students. They are the largest in the country so far to
    abandon plans for even a partial physical return to classrooms when
    they reopen in August. For other districts, the solution won't be an
    all-or-nothing approach.
    \href{https://bioethics.jhu.edu/research-and-outreach/projects/eschool-initiative/school-policy-tracker/}{Many
    systems}, including the nation's largest, New York City, are
    devising
    \href{https://www.nytimes.com/2020/06/26/us/coronavirus-schools-reopen-fall.html?action=click\&pgtype=Article\&state=default\&region=MAIN_CONTENT_3\&context=storylines_faq}{hybrid
    plans} that involve spending some days in classrooms and other days
    online. There's no national policy on this yet, so check with your
    municipal school system regularly to see what is happening in your
    community.
  \end{itemize}
\item ~
  \hypertarget{is-the-coronavirus-airborne}{%
  \paragraph{Is the coronavirus
  airborne?}\label{is-the-coronavirus-airborne}}

  \begin{itemize}
  \tightlist
  \item
    The coronavirus
    \href{https://www.nytimes.com/2020/07/04/health/239-experts-with-one-big-claim-the-coronavirus-is-airborne.html?action=click\&pgtype=Article\&state=default\&region=MAIN_CONTENT_3\&context=storylines_faq}{can
    stay aloft for hours in tiny droplets in stagnant air}, infecting
    people as they inhale, mounting scientific evidence suggests. This
    risk is highest in crowded indoor spaces with poor ventilation, and
    may help explain super-spreading events reported in meatpacking
    plants, churches and restaurants.
    \href{https://www.nytimes.com/2020/07/06/health/coronavirus-airborne-aerosols.html?action=click\&pgtype=Article\&state=default\&region=MAIN_CONTENT_3\&context=storylines_faq}{It's
    unclear how often the virus is spread} via these tiny droplets, or
    aerosols, compared with larger droplets that are expelled when a
    sick person coughs or sneezes, or transmitted through contact with
    contaminated surfaces, said Linsey Marr, an aerosol expert at
    Virginia Tech. Aerosols are released even when a person without
    symptoms exhales, talks or sings, according to Dr. Marr and more
    than 200 other experts, who
    \href{https://academic.oup.com/cid/article/doi/10.1093/cid/ciaa939/5867798}{have
    outlined the evidence in an open letter to the World Health
    Organization}.
  \end{itemize}
\item ~
  \hypertarget{what-are-the-symptoms-of-coronavirus}{%
  \paragraph{What are the symptoms of
  coronavirus?}\label{what-are-the-symptoms-of-coronavirus}}

  \begin{itemize}
  \tightlist
  \item
    Common symptoms
    \href{https://www.nytimes.com/article/symptoms-coronavirus.html?action=click\&pgtype=Article\&state=default\&region=MAIN_CONTENT_3\&context=storylines_faq}{include
    fever, a dry cough, fatigue and difficulty breathing or shortness of
    breath.} Some of these symptoms overlap with those of the flu,
    making detection difficult, but runny noses and stuffy sinuses are
    less common.
    \href{https://www.nytimes.com/2020/04/27/health/coronavirus-symptoms-cdc.html?action=click\&pgtype=Article\&state=default\&region=MAIN_CONTENT_3\&context=storylines_faq}{The
    C.D.C. has also} added chills, muscle pain, sore throat, headache
    and a new loss of the sense of taste or smell as symptoms to look
    out for. Most people fall ill five to seven days after exposure, but
    symptoms may appear in as few as two days or as many as 14 days.
  \end{itemize}
\item ~
  \hypertarget{does-asymptomatic-transmission-of-covid-19-happen}{%
  \paragraph{Does asymptomatic transmission of Covid-19
  happen?}\label{does-asymptomatic-transmission-of-covid-19-happen}}

  \begin{itemize}
  \tightlist
  \item
    So far, the evidence seems to show it does. A widely cited
    \href{https://www.nature.com/articles/s41591-020-0869-5}{paper}
    published in April suggests that people are most infectious about
    two days before the onset of coronavirus symptoms and estimated that
    44 percent of new infections were a result of transmission from
    people who were not yet showing symptoms. Recently, a top expert at
    the World Health Organization stated that transmission of the
    coronavirus by people who did not have symptoms was ``very rare,''
    \href{https://www.nytimes.com/2020/06/09/world/coronavirus-updates.html?action=click\&pgtype=Article\&state=default\&region=MAIN_CONTENT_3\&context=storylines_faq\#link-1f302e21}{but
    she later walked back that statement.}
  \end{itemize}
\end{itemize}

To send evacuees to other counties --- out of the vulnerable Keys, for
example --- emergency managers might have to rent more buses so
passengers can sit at a safe distance from each other. Mr. Moskowitz
said the state is in talks with Uber to possibly provide individual
rides if needed.

Florida ordered all nursing homes and assisted living facilities to
install generators for cooling systems after as many as 12 people
\href{https://www.nytimes.com/2019/08/24/us/4-charged-holywood-hills-deaths-hurricane-irma-florida.html}{died
from the sweltering heat} in a Broward County nursing home during
Hurricane Irma. Some homes with temporary generators were granted
variances as they work toward installing permanent ones, but almost all
are in at least basic compliance, said Mary C. Mayhew, who runs the
Agency for Health Care Administration, which oversees long-term care
facilities.

But nursing homes in evacuation zones
\href{https://www.nytimes.com/2019/09/03/us/hurricane-dorian-florida-evacuation.html}{might
have to send residents to facilities} out of the storm's path that have
extra beds, she said, or to one of the various sites that have been set
up recently to relieve overcrowded hospitals in the event of a Covid-19
surge.

After a storm, a host of other concerns would emerge. Amid the economic
crisis, more people could need meals, perhaps for a week or more. And
electricity would likely take longer to restore because utility crews
would be working under unusual conditions.

Image

Shoppers linned up to get supplies after Hurricane Michael in
Florida.Credit...Chang W. Lee/The New York Times

Duke Energy Florida sent crews to help restore power in South Carolina
after a severe storm last month and found that the special fireproof
face masks needed for virus protection in areas with fire hazards made
workers hot and required them to take more water breaks, said Jason
Cutliffe, the company's storm director.

Then there is the question of accommodations for storm workers. Gone
would be the usual large tent cities for up to 2,000 workers with
centralized cafeterias, showers and laundry. Instead, crews would have
to stay in smaller staging areas that allow for social distancing but
also result in less efficient replenishing of equipment.

That slows workers down, said Eric Silagy, chief executive of Florida
Power \& Light, the state's largest utility.

``The things that we're going to have to do to keep folks safe from a
virus will lead to inefficiencies in our ability to respond normally,''
he said.

As it happens, the company dusted off its old pandemic plan last year
after the human resources department suggested an update.

``But even our pandemic plan didn't have a global pandemic component to
it: interrupted supply chains globally and everybody sheltering in place
and grounding the airlines,'' he said. ``I'm glad that we have it in
place, but you have to adapt.''

Advertisement

\protect\hyperlink{after-bottom}{Continue reading the main story}

\hypertarget{site-index}{%
\subsection{Site Index}\label{site-index}}

\hypertarget{site-information-navigation}{%
\subsection{Site Information
Navigation}\label{site-information-navigation}}

\begin{itemize}
\tightlist
\item
  \href{https://help.nytimes.com/hc/en-us/articles/115014792127-Copyright-notice}{©~2020~The
  New York Times Company}
\end{itemize}

\begin{itemize}
\tightlist
\item
  \href{https://www.nytco.com/}{NYTCo}
\item
  \href{https://help.nytimes.com/hc/en-us/articles/115015385887-Contact-Us}{Contact
  Us}
\item
  \href{https://www.nytco.com/careers/}{Work with us}
\item
  \href{https://nytmediakit.com/}{Advertise}
\item
  \href{http://www.tbrandstudio.com/}{T Brand Studio}
\item
  \href{https://www.nytimes.com/privacy/cookie-policy\#how-do-i-manage-trackers}{Your
  Ad Choices}
\item
  \href{https://www.nytimes.com/privacy}{Privacy}
\item
  \href{https://help.nytimes.com/hc/en-us/articles/115014893428-Terms-of-service}{Terms
  of Service}
\item
  \href{https://help.nytimes.com/hc/en-us/articles/115014893968-Terms-of-sale}{Terms
  of Sale}
\item
  \href{https://spiderbites.nytimes.com}{Site Map}
\item
  \href{https://help.nytimes.com/hc/en-us}{Help}
\item
  \href{https://www.nytimes.com/subscription?campaignId=37WXW}{Subscriptions}
\end{itemize}
