Sections

SEARCH

\protect\hyperlink{site-content}{Skip to
content}\protect\hyperlink{site-index}{Skip to site index}

\href{https://www.nytimes.com/section/us}{U.S.}

\href{https://myaccount.nytimes.com/auth/login?response_type=cookie\&client_id=vi}{}

\href{https://www.nytimes.com/section/todayspaper}{Today's Paper}

\href{/section/us}{U.S.}\textbar{}Ralph W. McGehee, Agent Who Exposed
the C.I.A., Dies at 92

\url{https://nyti.ms/2T3LNV4}

\begin{itemize}
\item
\item
\item
\item
\item
\end{itemize}

\href{https://www.nytimes.com/news-event/coronavirus?action=click\&pgtype=Article\&state=default\&region=TOP_BANNER\&context=storylines_menu}{The
Coronavirus Outbreak}

\begin{itemize}
\tightlist
\item
  live\href{https://www.nytimes.com/2020/08/03/world/coronavirus-covid-19.html?action=click\&pgtype=Article\&state=default\&region=TOP_BANNER\&context=storylines_menu}{Latest
  Updates}
\item
  \href{https://www.nytimes.com/interactive/2020/us/coronavirus-us-cases.html?action=click\&pgtype=Article\&state=default\&region=TOP_BANNER\&context=storylines_menu}{Maps
  and Cases}
\item
  \href{https://www.nytimes.com/interactive/2020/science/coronavirus-vaccine-tracker.html?action=click\&pgtype=Article\&state=default\&region=TOP_BANNER\&context=storylines_menu}{Vaccine
  Tracker}
\item
  \href{https://www.nytimes.com/2020/08/02/us/covid-college-reopening.html?action=click\&pgtype=Article\&state=default\&region=TOP_BANNER\&context=storylines_menu}{College
  Reopening}
\item
  \href{https://www.nytimes.com/live/2020/08/03/business/stock-market-today-coronavirus?action=click\&pgtype=Article\&state=default\&region=TOP_BANNER\&context=storylines_menu}{Economy}
\end{itemize}

Advertisement

\protect\hyperlink{after-top}{Continue reading the main story}

Supported by

\protect\hyperlink{after-sponsor}{Continue reading the main story}

Those We've Lost

\hypertarget{ralph-w-mcgehee-agent-who-exposed-the-cia-dies-at-92}{%
\section{Ralph W. McGehee, Agent Who Exposed the C.I.A., Dies at
92}\label{ralph-w-mcgehee-agent-who-exposed-the-cia-dies-at-92}}

A crisis of conscience in Vietnam led him to conclude that the agency
was ``a malevolent force'' and to lay it bare in a memoir, ``Deadly
Deceits.''

\includegraphics{https://static01.nyt.com/images/2020/05/15/obituaries/15McGehee1/14McGehee1-articleLarge.jpg?quality=75\&auto=webp\&disable=upscale}

By \href{https://www.nytimes.com/by/tim-weiner}{Tim Weiner}

\begin{itemize}
\item
  May 14, 2020
\item
  \begin{itemize}
  \item
  \item
  \item
  \item
  \item
  \end{itemize}
\end{itemize}

\emph{This obituary is part of a series about people who have died in
the coronavirus pandemic. Read about others}
\href{https://www.nytimes.com/series/people-who-have-died-of-the-coronavirus}{\emph{here}}\emph{.}

Ralph W. McGehee, a veteran of the Central Intelligence Agency's
clandestine crusades in Vietnam who went to war against the C.I.A.
itself, died May 2 at an assisted-living facility in Falmouth, Maine. He
was 92.

The cause was Covid-19, his son, Dan McGehee, said.

Mr. McGehee's 1983 memoir, ``Deadly Deceits,'' was a scathing critique,
a chronicle of the C.I.A.'s Cold War covert operations in Southeast Asia
and his dawning realization that the American cause in Vietnam was
doomed. He recalled his epiphany: At the end of 1968, he sat drinking
alone in a sparsely furnished villa outside Saigon, listening to a
tragic pop song, ``The End of the World,'' as helicopter gunships
circled overhead and B-52s dropped bombs in the distance.

``My idealism, my patriotism, my ambition, my plans to be a good
intelligence officer to help my country fight the Communist scourge ---
what the hell had happened?'' he wrote. ``Why did we have to bomb the
people we were trying to save? Why were we napalming young children? Why
did the C.I.A., my employer for 16 years, report lies instead of the
truth?'' He struggled to answer those questions for the rest of his
life.

After growing up on the South Side of Chicago, starring on Notre Dame's
undefeated college football teams from 1946 to 1949, failing a tryout
with the Green Bay Packers, and working as a management trainee at
Montgomery Ward, he received a telegram from out of the blue in January
1952. It asked: Would you like to serve your country in an unusual way?
Football players, given their brawn and affinity for teamwork, were
prime candidates for paramilitary missions, in the eyes of the C.I.A.

The Korean War was at its height and the C.I.A., founded in 1947, was
expanding exponentially, from 200 officers in the beginning to roughly
15,000 in 1952, with some 50 overseas stations and a budget exceeding
\$5 billion in today's money. The agency searched frantically for
Americans capable of conducting covert operations overseas.

Mr. McGehee made the grade. After training and indoctrination, the
agency sent him out into the world. Serving over the years in Japan, the
Philippines, Taiwan, Thailand and South Vietnam, he confronted
confounding problems: for example, a richly compensated foreign agent
from Taiwan whose highly touted secret reports on Communist China were
based on nothing but newspaper clippings. In northern Thailand, he
worked on counterinsurgency operations with opium-smoking hill
tribesmen, to little avail. He tried, with some success, to train the
Thai national police to gather intelligence.

Mr. McGehee rose to the very middle of the C.I.A.'s ranks, and in 1968
he landed in Saigon to work in liaison with the chief of the secret
police. He then faced a spiritual crisis. The war was going badly for
the United States, and as bad turned to worse, it shattered him. He
questioned America's role in the world, the C.I.A.'s role in Vietnam,
his role in the C.I.A., and his very existence. He wrote that he had
contemplated unfurling a banner reading ``THE C.I.A. LIES'' and then
killing himself to protest the war.

\includegraphics{https://static01.nyt.com/images/2020/05/15/obituaries/15McGehee2/14McGehee2-articleLarge.jpg?quality=75\&auto=webp\&disable=upscale}

By 1973, after he returned to headquarters, labeled a malcontent and
relegated to a backwater desk, the agency confronted its own existential
crisis. The wars of Watergate would breach the ramparts of its secrecy.
Cold War skeletons tumbled from the closet: assassination plots, covert
support for autocrats, spying on Americans. Presidents had approved such
exploits in secret, but the C.I.A. was blamed and shamed. By the time he
retired in 1977, Mr. McGehee was convinced that the agency was a
malevolent force.

``Deadly Deceits: My 25 Years in the C.I.A.'' appeared six years later,
after the agency had sought and won significant deletions. Though C.I.A.
veterans had published memoirs since the 1960s, few had accused the
agency of distorting intelligence to deceive American presidents and the
American public to protect its power.

``The American people are the primary target audience of its lies,'' Mr.
McGehee wrote.

Now-declassified Cold War records tell a more complicated story. The
C.I.A.'s primary audience was presidents, not the public. Presidents
Lyndon B. Johnson and Richard M. Nixon had rejected the C.I.A.'s
pessimistic reporting on Vietnam, telling the American people that
victory, or peace with honor, was at hand when it wasn't. The
presidents, their national security advisers and the Pentagon had
pressured the C.I.A. to confirm their political preconceptions.
Sometimes the agency bent to their will, but not often.

Those records do bear out Mr. McGehee's critique that the C.I.A. had
neglected the gathering and analysis of intelligence, its founding
mission, in favor of bold covert operations that changed the world,
often for the worse, especially in the years leading up to the
disastrous Bay of Pigs invasion in Cuba, approved by President John F.
Kennedy, in 1961.

Ralph Walter McGehee Jr. was born in Moline, Ill., on April 9, 1928. His
parents managed an apartment complex, his mother as a bookkeeper and his
father as a maintenance man.

His wife of 63 years, Norma (Galbreath) McGehee,
\href{https://www.legacy.com/obituaries/tampabaytimes/obituary.aspx?n=norma-mcgehee\&pid=156086530}{died
in 2012}. In addition to his son Dan (who is also known as Keenan
Dakota), his survivors include another son, Scott; two daughters, Jean
Marteski and Peggy McGehee Horton; 10 grandchildren; and four
great-grandchildren.

In later years Mr. McGehee developed and maintained
\href{https://archive.org/details/CIABASE-Ralph_McGehee}{CIABASE}, an
online collection of open-source information, and gave lectures,
occasionally laced with conspiracy theories. He once told a reporter for
The New York Times that he realized that his book would not change the
C.I.A. But, he said, ``I guess I justify myself by thinking that I
fought for what I thought was right.''

\href{https://www.nytimes.com/interactive/2020/obituaries/people-died-coronavirus-obituaries.html?action=click\&pgtype=Article\&state=default\&region=BELOW_MAIN_CONTENT\&context=covid_obits_promo}{}

\hypertarget{those-weve-lost}{%
\section{Those We've Lost}\label{those-weve-lost}}

The coronavirus pandemic has taken an incalculable death toll. This
series is designed to put names and faces to the numbers.

Read more

\includegraphics{https://static01.nyt.com/images/2020/07/30/obituaries/30Pedro/30Pedro-square640.jpg}

\hypertarget{bernaldina-josuxe9-pedro}{%
\section{Bernaldina José Pedro}\label{bernaldina-josuxe9-pedro}}

d. Boa Vista, Brazil

Leader among the Indigenous Macuxi

\includegraphics{https://static01.nyt.com/images/2020/07/31/obituaries/31Swing/merlin_175167783_8913bc90-0d64-43f3-a655-1bb1bf1601c9-square640.jpg}

\hypertarget{john-eric-swing}{%
\section{John Eric Swing}\label{john-eric-swing}}

d. Fountain Valley, Calif.

Champion of Filipino-Americans

\includegraphics{https://static01.nyt.com/images/2020/07/27/obituaries/27Victor/merlin_175001436_38b11f8e-227a-4e2c-9821-7618af9b2524-square640.jpg}

\hypertarget{victor-victor}{%
\section{Victor Victor}\label{victor-victor}}

d. Santo Domingo, Dominican Republic

Beloved musician of the Dominican Republic

\includegraphics{https://static01.nyt.com/images/2020/07/31/obituaries/31Negron/merlin_175160169_516322ae-fd23-4969-b6b2-193ced371105-square640.jpg}

\hypertarget{dr-eddie-negruxf3n}{%
\section{Dr. Eddie Negrón}\label{dr-eddie-negruxf3n}}

d. Fort Walton Beach, Fla.

Internist on Florida's Emerald Coast

\includegraphics{https://static01.nyt.com/images/2020/07/30/obituaries/30Dobson/merlin_175115928_f6b9271c-8f05-4fe1-a38a-5ca4a58f8935-square640.jpg}

\hypertarget{dobby-dobson}{%
\section{Dobby Dobson}\label{dobby-dobson}}

d. Coral Springs, Fla.

Jamaican singer and songwriter

\includegraphics{https://static01.nyt.com/images/2020/08/01/obituaries/28Gonzalez/merlin_175002771_beb57888-3951-409a-ae13-03a94b2e962e-square640.jpg}

\hypertarget{waldemar-gonzalez}{%
\section{Waldemar Gonzalez}\label{waldemar-gonzalez}}

d. White Plains, N.Y.

Teacher and social worker

Advertisement

\protect\hyperlink{after-bottom}{Continue reading the main story}

\hypertarget{site-index}{%
\subsection{Site Index}\label{site-index}}

\hypertarget{site-information-navigation}{%
\subsection{Site Information
Navigation}\label{site-information-navigation}}

\begin{itemize}
\tightlist
\item
  \href{https://help.nytimes.com/hc/en-us/articles/115014792127-Copyright-notice}{©~2020~The
  New York Times Company}
\end{itemize}

\begin{itemize}
\tightlist
\item
  \href{https://www.nytco.com/}{NYTCo}
\item
  \href{https://help.nytimes.com/hc/en-us/articles/115015385887-Contact-Us}{Contact
  Us}
\item
  \href{https://www.nytco.com/careers/}{Work with us}
\item
  \href{https://nytmediakit.com/}{Advertise}
\item
  \href{http://www.tbrandstudio.com/}{T Brand Studio}
\item
  \href{https://www.nytimes.com/privacy/cookie-policy\#how-do-i-manage-trackers}{Your
  Ad Choices}
\item
  \href{https://www.nytimes.com/privacy}{Privacy}
\item
  \href{https://help.nytimes.com/hc/en-us/articles/115014893428-Terms-of-service}{Terms
  of Service}
\item
  \href{https://help.nytimes.com/hc/en-us/articles/115014893968-Terms-of-sale}{Terms
  of Sale}
\item
  \href{https://spiderbites.nytimes.com}{Site Map}
\item
  \href{https://help.nytimes.com/hc/en-us}{Help}
\item
  \href{https://www.nytimes.com/subscription?campaignId=37WXW}{Subscriptions}
\end{itemize}
