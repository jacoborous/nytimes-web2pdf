Sections

SEARCH

\protect\hyperlink{site-content}{Skip to
content}\protect\hyperlink{site-index}{Skip to site index}

\href{https://www.nytimes.com/section/technology}{Technology}

\href{https://myaccount.nytimes.com/auth/login?response_type=cookie\&client_id=vi}{}

\href{https://www.nytimes.com/section/todayspaper}{Today's Paper}

\href{/section/technology}{Technology}\textbar{}Twitter Adds Warnings to
Trump and White House Tweets, Fueling Tensions

\url{https://nyti.ms/36HJflc}

\begin{itemize}
\item
\item
\item
\item
\item
\item
\end{itemize}

\href{https://www.nytimes.com/news-event/george-floyd-protests-minneapolis-new-york-los-angeles?action=click\&pgtype=Article\&state=default\&region=TOP_BANNER\&context=storylines_menu}{Race
and America}

\begin{itemize}
\tightlist
\item
  \href{https://www.nytimes.com/2020/07/26/us/protests-portland-seattle-trump.html?action=click\&pgtype=Article\&state=default\&region=TOP_BANNER\&context=storylines_menu}{Protesters
  Return to Other Cities}
\item
  \href{https://www.nytimes.com/2020/07/24/us/portland-oregon-protests-white-race.html?action=click\&pgtype=Article\&state=default\&region=TOP_BANNER\&context=storylines_menu}{Portland
  at the Center}
\item
  \href{https://www.nytimes.com/2020/07/23/podcasts/the-daily/portland-protests.html?action=click\&pgtype=Article\&state=default\&region=TOP_BANNER\&context=storylines_menu}{Podcast:
  Showdown in Portland}
\item
  \href{https://www.nytimes.com/interactive/2020/07/16/us/black-lives-matter-protests-louisville-breonna-taylor.html?action=click\&pgtype=Article\&state=default\&region=TOP_BANNER\&context=storylines_menu}{45
  Days in Louisville}
\end{itemize}

Advertisement

\protect\hyperlink{after-top}{Continue reading the main story}

Supported by

\protect\hyperlink{after-sponsor}{Continue reading the main story}

\hypertarget{twitter-adds-warnings-to-trump-and-white-house-tweets-fueling-tensions}{%
\section{Twitter Adds Warnings to Trump and White House Tweets, Fueling
Tensions}\label{twitter-adds-warnings-to-trump-and-white-house-tweets-fueling-tensions}}

Twitter said the tweets, which implied that protesters in Minneapolis
could be shot, glorified violence --- the first time it had applied such
warnings to any public figure's posts.

By \href{https://www.nytimes.com/by/davey-alba}{Davey Alba},
\href{https://www.nytimes.com/by/kate-conger}{Kate Conger} and
\href{https://www.nytimes.com/by/raymond-zhong}{Raymond Zhong}

\begin{itemize}
\item
  Published May 29, 2020Updated June 3, 2020
\item
  \begin{itemize}
  \item
  \item
  \item
  \item
  \item
  \item
  \end{itemize}
\end{itemize}

\includegraphics{https://static01.nyt.com/images/2020/05/29/world/29trump-twitter-1/merlin_172931115_a46552ea-338d-4dcd-b9d1-33132a437045-articleLarge.jpg?quality=75\&auto=webp\&disable=upscale}

Twitter escalated its confrontation with President Trump on Friday,
adding warning labels to two
\href{https://www.nytimes.com/2020/06/03/us/politics/trump-twitter-fact-check.html}{tweets
by Mr. Trump} and the official White House Twitter account that implied
that
\href{https://www.nytimes.com/2020/05/30/us/george-floyd-minneapolis.html}{protesters
in Minneapolis} could be shot.

Amid the unrest in Minnesota, Mr. Trump
\href{https://twitter.com/realDonaldTrump/status/1266231100780744704}{posted}a
message on Twitter early Friday saying that ``when the looting starts,
the shooting starts.'' Twitter quickly prevented users from viewing the
tweet without reading a brief notice that the post glorified violence,
the first time it had applied such a warning on any public figure's
tweets. The
\href{https://twitter.com/WhiteHouse/status/1266342941649506304}{official
White House account then reposted} Mr. Trump's message; Twitter
responded by adding the same notice.

Twitter's actions came a day after Mr. Trump
\href{https://www.nytimes.com/2020/05/28/us/politics/trump-jack-dorsey.html}{signed
an executive order} to limit its legal protections under a statute that
shields social media companies from liability for the content posted on
their platforms. Twitter
\href{https://twitter.com/Policy/status/1266170586197262337}{publicly
opposed the executive order}, calling it ``a reactionary and politicized
approach to a landmark law,'' ramping up a conflict with Mr. Trump that
has exploded this week.

The decision to add the new warning labels was approved by Jack Dorsey,
Twitter's chief executive, after a late-night debate among company
officials, said a person with knowledge of the deliberations. Twitter
further tightened restrictions on the messages from Mr. Trump and the
White House by blocking users from liking or replying to them, though
people could still retweet the messages if they added a comment of their
own.

But Twitter did not go as far as taking the posts down, saying it was in
the public's interest that the messages remain accessible.

Image

The back-and-forth between
\href{https://www.nytimes.com/2020/06/03/us/politics/trump-twitter-fact-check.html}{Mr.
Trump and Twitter} on Friday punctuated a week of conflict between the
two.

The tussle began after Mr. Trump tweeted a hurtful and unsubstantiated
conspiracy theory
\href{https://twitter.com/realDonaldTrump/status/1257258214615367680?ref_src=twsrc\%5Etfw\%7Ctwcamp\%5Etweetembed\%7Ctwterm\%5E1257258214615367680\%7Ctwgr\%5E\&ref_url=https\%3A\%2F\%2Fwww.nytimes.com\%2F2020\%2F05\%2F26\%2Fbusiness\%2Fletter-to-twitter-ceo.html}{this}
\href{https://twitter.com/realDonaldTrump/status/1260161295019630592?ref_src=twsrc\%5Etfw\%7Ctwcamp\%5Etweetembed\%7Ctwterm\%5E1260161295019630592\%7Ctwgr\%5E\&ref_url=https\%3A\%2F\%2Fwww.nytimes.com\%2F2020\%2F05\%2F26\%2Fbusiness\%2Fletter-to-twitter-ceo.html}{month}
to attack the MSNBC host Joe Scarborough, which
\href{https://www.nytimes.com/2020/05/26/opinion/trump-scarborough-twitter.html}{caused
critics to call on Twitter to remove the messages}. While Twitter did
not take those posts down,
\href{https://www.nytimes.com/2020/05/26/technology/twitter-trump-mail-in-ballots.html}{it
added fact-checking labels for the first time} to two of the president's
election-related posts on Tuesday. The labels stood out because Twitter
for years did little to moderate Mr. Trump's often inaccurate and
threatening posts.

That immediately ignited Mr. Trump's ire. He accused Twitter of stifling
free speech and said he would not allow the social media companies to
operate unfettered. And in an apparent act of retaliation, he
\href{https://www.nytimes.com/2020/05/28/us/politics/trump-jack-dorsey.html}{signed
the executive order} on Thursday taking aim at Section 230 of the
Communications Decency Act, which provides the liability shield to the
tech companies.

Twitter and Mr. Trump are now in a standoff. The company has said it
will continue putting warning labels and restrictions on tweets that
incite violence or spread false information about elections and the
coronavirus. And Mr. Trump, who once tweeted up to 108 times a day this
month, shows no signs of stopping his usage of the service, lashing out
on Friday on Twitter about Twitter itself.

``Twitter is doing nothing about all of the lies \& propaganda being put
out by China or the Radical Left Democrat Party,''
\href{https://twitter.com/realDonaldTrump/status/1266326065833824257}{he
wrote}. ``They have targeted Republicans, Conservatives \& the President
of the United States. Section 230 should be revoked by Congress. Until
then, it will be regulated!''

He posted several other tweets citing similar views by his favorite Fox
News hosts. And as if daring Twitter, he posted another message about
looting leading to shooting on Friday afternoon.

In its separate Twitter account, the White House
\href{https://twitter.com/WhiteHouse/status/1266373803870806023}{jabbed
directly at Mr. Dorsey}: ``The President did not glorify violence. He
clearly condemned it. @Jack and Twitter's biased, bad-faith
`fact-checkers' have made it clear: Twitter is a publisher, not a
platform.''

And Dan Scavino, the president's deputy chief of staff, said Twitter
should be targeting the protesters in Minneapolis. ``Twitter is
targeting the President of the United States 24/7, while turning their
heads to protest organizers who are planning, plotting, and
communicating their next moves daily on this very platform,''
\href{https://twitter.com/Scavino45/status/1266343153466060803}{he
wrote}. He added that Twitter was full of it and ``more and more people
are beginning to get it.''

\href{https://twitter.com/TwitterComms/status/1266267446979129345}{Twitter
said} it had decided to restrict Mr. Trump's tweet about the protests in
Minnesota ``based on the historical context of the last line, its
connection to violence, and the risk it could inspire similar actions
today.'' It had applied a warning label
\href{https://twitter.com/OsmarTerra/status/1246474430676643842}{on a
tweet} from the Brazilian minister of citizenship, Osmar Terra, last
month, but the label was not for glorifying violence.

Mr. Dorsey also
\href{https://twitter.com/jack/status/1266390510278569984}{tweeted} on
Friday morning that social networks' fact-checking process should use
open-source technology --- software that is created and shared for
general use --- and be verifiable by everyone. He did not respond to a
request for comment.

The conflict has thrown Twitter into chaos, with employees racing to
take action on Mr. Trump's tweets while also scrambling to protect
themselves from harassment.
\href{https://www.nytimes.com/2020/05/27/technology/trump-twitter.html}{After
Mr. Trump and his allies lashed out} at one Twitter employee who had
publicly criticized Mr. Trump and other Republican leaders, other
employees removed their company affiliation from their social media
profiles or locked their accounts from public view.

First Amendment scholars said Friday that Mr. Trump and his allies had
it backward and that he was the one trying to stifle speech that clashed
with his own views.

``Fundamentally this dispute is about whether Twitter has the right to
disagree with, criticize, and respond to the president,'' said Jameel
Jaffer, executive director at the Knight First Amendment Institute at
Columbia University. ``Obviously, it does. It is remarkable and truly
chilling that the president and his advisers seem to believe
otherwise.''

Revoking Section 230 protections would expose Twitter and other online
platforms to expansive potential legal vulnerability that could
undermine the fundamentals of their businesses. But it would also remove
the very legal standard that has allowed Mr. Trump to use Twitter so
effectively to communicate with his more than 80 million followers ---
no matter how incendiary, false and even defamatory his messages may be.

Without a liability shield, Twitter and online companies would be forced
to
\href{https://www.nytimes.com/2020/05/28/us/politics/trump-jack-dorsey.html}{police
accounts like Mr. Trump's even more closely} to guard themselves against
legal action.

Twitter has for years
\href{https://www.nytimes.com/2019/10/15/technology/trump-twitter-account.html}{faced
criticism} over Mr. Trump's posts on the platform. The company has said
repeatedly that the president did not violate its terms of service,
however much he appeared to skirt the line. It has also said that
\href{https://blog.twitter.com/en_us/topics/company/2018/world-leaders-and-twitter.html}{blocking
world leaders} from the service or removing their tweets would hinder
public debate.

Twitter did
\href{https://www.nytimes.com/2019/06/27/technology/twitter-politicans-labels-abuse.html}{announce
last year}, however, that it would in certain cases place warning labels
on posts from political figures that broke its rules, the feature it
used with Mr. Trump's tweet about Minneapolis.

Mr. Trump's message implying that the Minneapolis protesters could be
shot was also
\href{https://www.facebook.com/DonaldTrump/posts/10164767134275725}{posted
on his official Facebook page}, where it appears without any warning
labels. Mark Zuckerberg, the company's chief executive,
\href{https://www.foxnews.com/media/facebook-mark-zuckerberg-twitter-fact-checking-trump}{told
Fox News} this week that he was uncomfortable with Facebook's being
``the arbiter of truth of everything that people say online.''

Protests have raged in Minneapolis this week over the death on Monday of
George Floyd, a black man who had been pinned down by a white police
officer who pressed his knee on Mr. Floyd's neck.

Frederike Kaltheuner, a tech policy fellow at the Mozilla Foundation,
said Twitter's confrontation with Mr. Trump raised questions about how
the platform would treat other world leaders. In March, the company
deleted posts by the presidents of Brazil and Venezuela that contained
unproven information about Covid-19 treatments.

``I doubt that Twitter has the resources to consistently apply rules to
all heads of states that use their platform in all sorts of languages,''
Ms. Kaltheuner said. ``From all we know about the many inconsistent ways
in which other policies are being enforced, my guess is that places that
rarely make U.S. news will likely be overlooked.''

In Mr. Trump's tweets about Minneapolis on Friday, he also criticized
the response by Mayor Jacob Frey, a Democrat. Mr. Trump said Mr. Frey
must ``get his act together and bring the City under control, or I will
send in the National Guard \& get the job done right.''

Mr. Frey did not know about Mr. Trump's tweets until a reporter read
them aloud during a news conference early on Friday. The mayor shook his
head and slammed a podium for emphasis.

``Weakness is refusing to take responsibility for your own actions,'' he
said. ``Weakness is pointing your finger at somebody else during a time
of crisis.''

He added, ``Donald Trump knows nothing about the strength of
Minneapolis.''

Peter Baker, Russell Goldman and Adam Satariano contributed reporting.

Advertisement

\protect\hyperlink{after-bottom}{Continue reading the main story}

\hypertarget{site-index}{%
\subsection{Site Index}\label{site-index}}

\hypertarget{site-information-navigation}{%
\subsection{Site Information
Navigation}\label{site-information-navigation}}

\begin{itemize}
\tightlist
\item
  \href{https://help.nytimes.com/hc/en-us/articles/115014792127-Copyright-notice}{©~2020~The
  New York Times Company}
\end{itemize}

\begin{itemize}
\tightlist
\item
  \href{https://www.nytco.com/}{NYTCo}
\item
  \href{https://help.nytimes.com/hc/en-us/articles/115015385887-Contact-Us}{Contact
  Us}
\item
  \href{https://www.nytco.com/careers/}{Work with us}
\item
  \href{https://nytmediakit.com/}{Advertise}
\item
  \href{http://www.tbrandstudio.com/}{T Brand Studio}
\item
  \href{https://www.nytimes.com/privacy/cookie-policy\#how-do-i-manage-trackers}{Your
  Ad Choices}
\item
  \href{https://www.nytimes.com/privacy}{Privacy}
\item
  \href{https://help.nytimes.com/hc/en-us/articles/115014893428-Terms-of-service}{Terms
  of Service}
\item
  \href{https://help.nytimes.com/hc/en-us/articles/115014893968-Terms-of-sale}{Terms
  of Sale}
\item
  \href{https://spiderbites.nytimes.com}{Site Map}
\item
  \href{https://help.nytimes.com/hc/en-us}{Help}
\item
  \href{https://www.nytimes.com/subscription?campaignId=37WXW}{Subscriptions}
\end{itemize}
