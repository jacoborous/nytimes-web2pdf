Sections

SEARCH

\protect\hyperlink{site-content}{Skip to
content}\protect\hyperlink{site-index}{Skip to site index}

\href{https://www.nytimes.com/section/us}{U.S.}

\href{https://myaccount.nytimes.com/auth/login?response_type=cookie\&client_id=vi}{}

\href{https://www.nytimes.com/section/todayspaper}{Today's Paper}

\href{/section/us}{U.S.}\textbar{}Lockdown Delays Cost at Least 36,000
Lives, Data Show

\url{https://nyti.ms/2LK7JR7}

\begin{itemize}
\item
\item
\item
\item
\item
\end{itemize}

\href{https://www.nytimes.com/news-event/coronavirus?action=click\&pgtype=Article\&state=default\&region=TOP_BANNER\&context=storylines_menu}{The
Coronavirus Outbreak}

\begin{itemize}
\tightlist
\item
  live\href{https://www.nytimes.com/2020/08/02/world/coronavirus-updates.html?action=click\&pgtype=Article\&state=default\&region=TOP_BANNER\&context=storylines_menu}{Latest
  Updates}
\item
  \href{https://www.nytimes.com/interactive/2020/us/coronavirus-us-cases.html?action=click\&pgtype=Article\&state=default\&region=TOP_BANNER\&context=storylines_menu}{Maps
  and Cases}
\item
  \href{https://www.nytimes.com/interactive/2020/science/coronavirus-vaccine-tracker.html?action=click\&pgtype=Article\&state=default\&region=TOP_BANNER\&context=storylines_menu}{Vaccine
  Tracker}
\item
  \href{https://www.nytimes.com/interactive/2020/07/29/us/schools-reopening-coronavirus.html?action=click\&pgtype=Article\&state=default\&region=TOP_BANNER\&context=storylines_menu}{What
  School May Look Like}
\item
  \href{https://www.nytimes.com/live/2020/07/31/business/stock-market-today-coronavirus?action=click\&pgtype=Article\&state=default\&region=TOP_BANNER\&context=storylines_menu}{Economy}
\end{itemize}

Advertisement

\protect\hyperlink{after-top}{Continue reading the main story}

Supported by

\protect\hyperlink{after-sponsor}{Continue reading the main story}

\hypertarget{lockdown-delays-cost-at-least-36000-lives-data-show}{%
\section{Lockdown Delays Cost at Least 36,000 Lives, Data
Show}\label{lockdown-delays-cost-at-least-36000-lives-data-show}}

Even small differences in timing would have prevented the worst
exponential growth, which by April had subsumed New York City, New
Orleans and other major cities, researchers found.

By \href{https://www.nytimes.com/by/james-glanz}{James Glanz} and
\href{https://www.nytimes.com/by/campbell-robertson}{Campbell Robertson}

\begin{itemize}
\item
  Published May 20, 2020Updated May 22, 2020
\item
  \begin{itemize}
  \item
  \item
  \item
  \item
  \item
  \end{itemize}
\end{itemize}

Total reported deaths in the

United States on May 3

Estimated deaths on May 3 if social distancing

started one week earlier than it did

New York City

17,581

65,307

29,410

New York City

2,838

Los Angeles

Los Angeles

1,223

451

Total reported deaths in the

United States on May 3

Estimated deaths on May 3 if social distancing

started one week earlier than it did

65,307

29,410

New York City

New York City

17,581

2,838

Los Angeles

Los Angeles

1,223

451

Total reported deaths in the

United States on May 3

65,307

New York City

17,581

Los Angeles

1,223

Estimated deaths on May 3 if social distancing

started one week earlier than it did

29,410

New York City

2,838

Los Angeles

451

Total reported deaths in the

United States on May 3

65,307

New York City

17,581

Los Angeles

1,223

Estimated deaths on May 3 if social distancing

started one week earlier than it did

29,410

New York City

2,838

Los Angeles

451

Estimated deaths on May 3 if social distancing

started one week earlier than it did

Total reported deaths in the

United States on May 3

New York City

17,581

65,307

29,410

New York City

2,838

Los Angeles

Los Angeles

1,223

451

By Lazaro Gamio·Source: ``Differential Effects of Intervention Timing on
COVID-19 Spread in the United States,'' by Sen Pei, Sasikiran Kandula
and Jeffrey Shaman, Columbia University

If the United States had begun imposing social distancing measures one
week earlier than it did in March, about 36,000 fewer people would have
died in the coronavirus outbreak, according to
\href{https://www.medrxiv.org/content/10.1101/2020.05.15.20103655v1}{new
estimates} from Columbia University disease modelers.

And if the country had begun locking down cities and limiting social
contact on March 1, two weeks earlier than most people started staying
home, the vast majority of the nation's deaths --- about 83 percent ---
would have been avoided, the researchers estimated.

Under that scenario, about 54,000 fewer people would have died by early
May.

The enormous cost of waiting to take action reflects the unforgiving
dynamics of the outbreak that swept through American cities in early
March. Even small differences in timing would have prevented the worst
exponential growth, which by April had subsumed New York City, New
Orleans and other major cities, the researchers found.

``It's a big, big difference. That small moment in time, catching it in
that growth phase, is incredibly critical in reducing the number of
deaths,'' said Jeffrey Shaman, an epidemiologist at Columbia and the
leader of the research team.

\hypertarget{how-earlier-control-measures-could-have-saved-lives}{%
\subsubsection{How Earlier Control Measures Could Have Saved
Lives}\label{how-earlier-control-measures-could-have-saved-lives}}

Number of reported

deaths by May 3

65,307

60,000 deaths

Estimated deaths if

social distancing

started \ldots{}

40,000

\ldots{} one week earlier

than it did in March

Range of

estimates

29,410

20,000

\ldots{} two weeks earlier

11,253

0

March 1

April 1

May 3

Number of reported

deaths by May 3

65,307

60,000 deaths

Estimated deaths

if social distancing

started \ldots{}

40,000

\ldots{} one week earlier

than it did in March

Range of

estimates

29,410

20,000

\ldots{} two weeks earlier

11,253

0

March 1

April 1

May 3

Number of reported

deaths by May 3

65,307

60,000 deaths

Estimated deaths

if social distancing

started \ldots{}

40,000

\ldots{} one week earlier

than it did in March

29,410

20,000

\ldots{} two weeks earlier

11,253

0

March 1

April 1

May 3

Number of reported

deaths by May 3

65,307

60,000 deaths

Estimated deaths if

social distancing

started \ldots{}

40,000

\ldots{} one week earlier

than it did in March

29,410

20,000

\ldots{} two weeks earlier

11,253

0

March 1

April 1

May 3

By Weiyi Cai·Source: ``Differential Effects of Intervention Timing on
COVID-19 Spread in the United States,'' by Sen Pei, Sasikiran Kandula
and Jeffrey Shaman, Columbia University

The findings are based on infectious disease modeling that gauges how
reduced contact between people starting in mid-March slowed transmission
of the virus. Dr. Shaman's team modeled what would have happened if
those same changes had taken place one or two weeks earlier and
estimated the spread of infections and deaths until May 3.

The results show that as states reopen, outbreaks can easily get out of
control unless officials closely monitor infections and immediately
clamp down on new flare-ups. And they show that each day that officials
waited to impose restrictions in early March came at a great cost.

After Italy and South Korea had started aggressively responding to the
virus, President Trump resisted canceling campaign rallies or telling
people to stay home or avoid crowds. The risk of the virus to most
Americans was very low, he said.

``Nothing is shut down, life \& the economy go on,'' Mr. Trump
\href{https://twitter.com/realDonaldTrump/status/1237027356314869761}{tweeted}
on March 9, suggesting that the flu was worse than the coronavirus. ``At
this moment there are 546 confirmed cases of CoronaVirus, with 22
deaths. Think about that!''

\hypertarget{latest-updates-global-coronavirus-outbreak}{%
\section{\texorpdfstring{\href{https://www.nytimes.com/2020/08/01/world/coronavirus-covid-19.html?action=click\&pgtype=Article\&state=default\&region=MAIN_CONTENT_1\&context=storylines_live_updates}{Latest
Updates: Global Coronavirus
Outbreak}}{Latest Updates: Global Coronavirus Outbreak}}\label{latest-updates-global-coronavirus-outbreak}}

Updated 2020-08-02T17:52:35.962Z

\begin{itemize}
\tightlist
\item
  \href{https://www.nytimes.com/2020/08/01/world/coronavirus-covid-19.html?action=click\&pgtype=Article\&state=default\&region=MAIN_CONTENT_1\&context=storylines_live_updates\#link-34047410}{The
  U.S. reels as July cases more than double the total of any other
  month.}
\item
  \href{https://www.nytimes.com/2020/08/01/world/coronavirus-covid-19.html?action=click\&pgtype=Article\&state=default\&region=MAIN_CONTENT_1\&context=storylines_live_updates\#link-780ec966}{Top
  U.S. officials work to break an impasse over the federal jobless
  benefit.}
\item
  \href{https://www.nytimes.com/2020/08/01/world/coronavirus-covid-19.html?action=click\&pgtype=Article\&state=default\&region=MAIN_CONTENT_1\&context=storylines_live_updates\#link-2bc8948}{Its
  outbreak untamed, Melbourne goes into even greater lockdown.}
\end{itemize}

\href{https://www.nytimes.com/2020/08/01/world/coronavirus-covid-19.html?action=click\&pgtype=Article\&state=default\&region=MAIN_CONTENT_1\&context=storylines_live_updates}{See
more updates}

More live coverage:
\href{https://www.nytimes.com/live/2020/07/31/business/stock-market-today-coronavirus?action=click\&pgtype=Article\&state=default\&region=MAIN_CONTENT_1\&context=storylines_live_updates}{Markets}

In fact,
\href{https://www.nytimes.com/2020/04/23/us/coronavirus-early-outbreaks-cities.html}{tens
of thousands of people} had already been infected by that point,
researchers later estimated. But a lack of widespread testing allowed
those infections to go undetected, hiding the urgency of an outbreak
that most Americans still identified as a foreign threat.

In a statement released late Wednesday night in response to the new
estimates, the White House reiterated Mr. Trump's assertion that
restrictions on travel from China in January and Europe in mid-March
slowed the spread of the virus.

On March 16, Mr. Trump urged Americans to limit travel, avoid groups and
stay home from school. Bill de Blasio, mayor of New York City, closed
the city's schools on March 15, and Gov. Andrew M. Cuomo issued a
stay-at-home order that took effect on March 22. Changes to personal
behavior across the country in mid-March slowed the epidemic, a number
of disease researchers have found.

But in cities where the virus arrived early and spread quickly, those
actions were too late to avoid a calamity.

In the New York metro area alone, 21,800 people had died by May 3. Fewer
than 4,300 would have died by then if control measures had been put in
place and adopted nationwide just a week earlier, on March 8, the
researchers estimated.

All models are only estimates, and it is impossible to know for certain
the exact number of people who would have died. But Lauren Ancel Meyers,
a University of Texas at Austin epidemiologist who was not involved in
the research, said that it ``makes a compelling case that even slightly
earlier action in New York could have been game changing.''

``This implies that if interventions had occurred two weeks earlier,
many Covid-19 deaths and cases would have been prevented by early May,
not just in New York City but throughout the U.S.,'' Dr. Meyers said.

The fates of specific people cannot be captured by a computer model. But
there is a name, a story and a town for every person who was infected
and later showed symptoms and died in March and early April. Around the
country, people separate from this study have wondered what might have
been.

Rushia Stephens, a music teacher who had become a county court records
technician in an Atlanta suburb, collapsed on her bedroom floor, unable
to breathe, and died on March 19. Adolph Mendez, a businessman in New
Braunfels, Texas, was confined to his own bedroom as his terrified
family tended to him until he died on March 26. Richard Walts, a retired
firefighter in Oklahoma, was ferried to a hospital in an ambulance and
died two weeks later, on April 3.

Mr. Mendez's widow, Angela Mendez, said she still couldn't say for sure
whether action should have been taken earlier. It didn't matter now
anyway, not for her husband.

``They probably could have had earlier a better way to not let this
pandemic go that far,'' she said. ``But they didn't.''

Official social distancing measures don't work unless people follow
them. While the measures have enjoyed generally widespread support among
Americans, the findings rely on the assumption that millions of people
would have been willing to change their behavior sooner.

People are apt to take restrictions much more seriously when the
devastation of a disease is visible, said Natalie Dean, an assistant
professor of biostatistics at the University of Florida who specializes
in emerging infectious diseases. But in early March, there had been few
deaths, and infections were still spreading silently through the
population.

\href{https://www.nytimes.com/news-event/coronavirus?action=click\&pgtype=Article\&state=default\&region=MAIN_CONTENT_3\&context=storylines_faq}{}

\hypertarget{the-coronavirus-outbreak-}{%
\subsubsection{The Coronavirus Outbreak
›}\label{the-coronavirus-outbreak-}}

\hypertarget{frequently-asked-questions}{%
\paragraph{Frequently Asked
Questions}\label{frequently-asked-questions}}

Updated July 27, 2020

\begin{itemize}
\item ~
  \hypertarget{should-i-refinance-my-mortgage}{%
  \paragraph{Should I refinance my
  mortgage?}\label{should-i-refinance-my-mortgage}}

  \begin{itemize}
  \tightlist
  \item
    \href{https://www.nytimes.com/article/coronavirus-money-unemployment.html?action=click\&pgtype=Article\&state=default\&region=MAIN_CONTENT_3\&context=storylines_faq}{It
    could be a good idea,} because mortgage rates have
    \href{https://www.nytimes.com/2020/07/16/business/mortgage-rates-below-3-percent.html?action=click\&pgtype=Article\&state=default\&region=MAIN_CONTENT_3\&context=storylines_faq}{never
    been lower.} Refinancing requests have pushed mortgage applications
    to some of the highest levels since 2008, so be prepared to get in
    line. But defaults are also up, so if you're thinking about buying a
    home, be aware that some lenders have tightened their standards.
  \end{itemize}
\item ~
  \hypertarget{what-is-school-going-to-look-like-in-september}{%
  \paragraph{What is school going to look like in
  September?}\label{what-is-school-going-to-look-like-in-september}}

  \begin{itemize}
  \tightlist
  \item
    It is unlikely that many schools will return to a normal schedule
    this fall, requiring the grind of
    \href{https://www.nytimes.com/2020/06/05/us/coronavirus-education-lost-learning.html?action=click\&pgtype=Article\&state=default\&region=MAIN_CONTENT_3\&context=storylines_faq}{online
    learning},
    \href{https://www.nytimes.com/2020/05/29/us/coronavirus-child-care-centers.html?action=click\&pgtype=Article\&state=default\&region=MAIN_CONTENT_3\&context=storylines_faq}{makeshift
    child care} and
    \href{https://www.nytimes.com/2020/06/03/business/economy/coronavirus-working-women.html?action=click\&pgtype=Article\&state=default\&region=MAIN_CONTENT_3\&context=storylines_faq}{stunted
    workdays} to continue. California's two largest public school
    districts --- Los Angeles and San Diego --- said on July 13, that
    \href{https://www.nytimes.com/2020/07/13/us/lausd-san-diego-school-reopening.html?action=click\&pgtype=Article\&state=default\&region=MAIN_CONTENT_3\&context=storylines_faq}{instruction
    will be remote-only in the fall}, citing concerns that surging
    coronavirus infections in their areas pose too dire a risk for
    students and teachers. Together, the two districts enroll some
    825,000 students. They are the largest in the country so far to
    abandon plans for even a partial physical return to classrooms when
    they reopen in August. For other districts, the solution won't be an
    all-or-nothing approach.
    \href{https://bioethics.jhu.edu/research-and-outreach/projects/eschool-initiative/school-policy-tracker/}{Many
    systems}, including the nation's largest, New York City, are
    devising
    \href{https://www.nytimes.com/2020/06/26/us/coronavirus-schools-reopen-fall.html?action=click\&pgtype=Article\&state=default\&region=MAIN_CONTENT_3\&context=storylines_faq}{hybrid
    plans} that involve spending some days in classrooms and other days
    online. There's no national policy on this yet, so check with your
    municipal school system regularly to see what is happening in your
    community.
  \end{itemize}
\item ~
  \hypertarget{is-the-coronavirus-airborne}{%
  \paragraph{Is the coronavirus
  airborne?}\label{is-the-coronavirus-airborne}}

  \begin{itemize}
  \tightlist
  \item
    The coronavirus
    \href{https://www.nytimes.com/2020/07/04/health/239-experts-with-one-big-claim-the-coronavirus-is-airborne.html?action=click\&pgtype=Article\&state=default\&region=MAIN_CONTENT_3\&context=storylines_faq}{can
    stay aloft for hours in tiny droplets in stagnant air}, infecting
    people as they inhale, mounting scientific evidence suggests. This
    risk is highest in crowded indoor spaces with poor ventilation, and
    may help explain super-spreading events reported in meatpacking
    plants, churches and restaurants.
    \href{https://www.nytimes.com/2020/07/06/health/coronavirus-airborne-aerosols.html?action=click\&pgtype=Article\&state=default\&region=MAIN_CONTENT_3\&context=storylines_faq}{It's
    unclear how often the virus is spread} via these tiny droplets, or
    aerosols, compared with larger droplets that are expelled when a
    sick person coughs or sneezes, or transmitted through contact with
    contaminated surfaces, said Linsey Marr, an aerosol expert at
    Virginia Tech. Aerosols are released even when a person without
    symptoms exhales, talks or sings, according to Dr. Marr and more
    than 200 other experts, who
    \href{https://academic.oup.com/cid/article/doi/10.1093/cid/ciaa939/5867798}{have
    outlined the evidence in an open letter to the World Health
    Organization}.
  \end{itemize}
\item ~
  \hypertarget{what-are-the-symptoms-of-coronavirus}{%
  \paragraph{What are the symptoms of
  coronavirus?}\label{what-are-the-symptoms-of-coronavirus}}

  \begin{itemize}
  \tightlist
  \item
    Common symptoms
    \href{https://www.nytimes.com/article/symptoms-coronavirus.html?action=click\&pgtype=Article\&state=default\&region=MAIN_CONTENT_3\&context=storylines_faq}{include
    fever, a dry cough, fatigue and difficulty breathing or shortness of
    breath.} Some of these symptoms overlap with those of the flu,
    making detection difficult, but runny noses and stuffy sinuses are
    less common.
    \href{https://www.nytimes.com/2020/04/27/health/coronavirus-symptoms-cdc.html?action=click\&pgtype=Article\&state=default\&region=MAIN_CONTENT_3\&context=storylines_faq}{The
    C.D.C. has also} added chills, muscle pain, sore throat, headache
    and a new loss of the sense of taste or smell as symptoms to look
    out for. Most people fall ill five to seven days after exposure, but
    symptoms may appear in as few as two days or as many as 14 days.
  \end{itemize}
\item ~
  \hypertarget{does-asymptomatic-transmission-of-covid-19-happen}{%
  \paragraph{Does asymptomatic transmission of Covid-19
  happen?}\label{does-asymptomatic-transmission-of-covid-19-happen}}

  \begin{itemize}
  \tightlist
  \item
    So far, the evidence seems to show it does. A widely cited
    \href{https://www.nature.com/articles/s41591-020-0869-5}{paper}
    published in April suggests that people are most infectious about
    two days before the onset of coronavirus symptoms and estimated that
    44 percent of new infections were a result of transmission from
    people who were not yet showing symptoms. Recently, a top expert at
    the World Health Organization stated that transmission of the
    coronavirus by people who did not have symptoms was ``very rare,''
    \href{https://www.nytimes.com/2020/06/09/world/coronavirus-updates.html?action=click\&pgtype=Article\&state=default\&region=MAIN_CONTENT_3\&context=storylines_faq\#link-1f302e21}{but
    she later walked back that statement.}
  \end{itemize}
\end{itemize}

``If things are really taking off, people are likely to clamp down
more,'' Dr. Dean said. ``Do people need to hear the sirens for them to
stay home?''

Dr. Shaman's team estimated the effect of relaxing all control measures
across the country. The model finds that because of the lag between the
time infections occur and symptoms begin emerging, without extensive
testing and rapid action, many more infections will occur, leading to
more deaths --- as many as tens of thousands across the country.

The timing and circumstances of those who were infected in March raise
haunting questions.

It was a Friday night in mid-March when Devin Taquino began feeling
sick. Neither he nor his wife was thinking at all about the coronavirus.
There were already more than 200 cases in the state by that time, but
most of those cases were in the eastern part of the state, not in the
small city of Donora, south of Pittsburgh.

Plus, Mr. Taquino did not fit the profile: he was only 47 years old with
no underlying conditions and his main symptom --- diarrhea --- was not
something broadly associated with the disease. He was planning to work a
Saturday morning overtime shift at a call center half an hour away, but
he called in sick. Offices all over the area were asking people not to
come in, but Mr. Taquino's had not taken that step.

He worked on Monday, but on Tuesday he returned home sick from work,
passed out in bed and didn't wake up for 16 hours. The next morning, his
wife, Rebecca Taquino, 42, woke him up and told him they needed to get
tested. She didn't think he had the virus, but she thought it was the
smart thing to do.

Without primary care doctors, they went to a nearby urgent care clinic,
where they learned that his blood oxygen level was very low. The people
at the clinic offered to call an ambulance, but fearing the cost, and
still skeptical that this was that serious, the Taquinos chose to drive
to an emergency room.

At the hospital, he was given an X-ray and diagnosed with pneumonia. He
stayed, kept in an isolation unit just in case, and she returned home.
The next evening, March 26, he called her with two developments. One:
his work had emailed with the news that someone at the call center,
where the work stations sat about a foot apart, had tested positive for
the virus. The other bit of news was that he had tested positive.

There has been a lot for Ms. Taquino to think about in the weeks since
that phone call, including the long days during which she never left the
house and her husband's situation got more horrifyingly worse.

Should the call center have sent the employees home earlier? When she
called the center on Friday to report his condition, it was already
empty: the workers had been sent home. Did they act too late?

``I kind of tossed that one back and forth myself,'' she said. ``I
really want to blame it on them, I really do.''

Could she know definitively where he got it? It was hard to say for
sure. Still, given that email the day of his diagnosis, it seemed by far
the most likely possibility that he got it at work.

After three weeks of agony, Mr. Taquino died on April 10. Whether he was
one of the thousands of people who might be alive if social distancing
measures had been put in place a week earlier can never be known.

Ms. Taquino said officials should have known.

``If it's spreading that fast you have to know it would have come
here,'' Ms. Taquino said. ``They should have been implementing programs.
I think it was a giant lapse in our country. There was no way to think
that we were going to be spared from this.''

Campbell Robertson reported from Pittsburgh.

Advertisement

\protect\hyperlink{after-bottom}{Continue reading the main story}

\hypertarget{site-index}{%
\subsection{Site Index}\label{site-index}}

\hypertarget{site-information-navigation}{%
\subsection{Site Information
Navigation}\label{site-information-navigation}}

\begin{itemize}
\tightlist
\item
  \href{https://help.nytimes.com/hc/en-us/articles/115014792127-Copyright-notice}{©~2020~The
  New York Times Company}
\end{itemize}

\begin{itemize}
\tightlist
\item
  \href{https://www.nytco.com/}{NYTCo}
\item
  \href{https://help.nytimes.com/hc/en-us/articles/115015385887-Contact-Us}{Contact
  Us}
\item
  \href{https://www.nytco.com/careers/}{Work with us}
\item
  \href{https://nytmediakit.com/}{Advertise}
\item
  \href{http://www.tbrandstudio.com/}{T Brand Studio}
\item
  \href{https://www.nytimes.com/privacy/cookie-policy\#how-do-i-manage-trackers}{Your
  Ad Choices}
\item
  \href{https://www.nytimes.com/privacy}{Privacy}
\item
  \href{https://help.nytimes.com/hc/en-us/articles/115014893428-Terms-of-service}{Terms
  of Service}
\item
  \href{https://help.nytimes.com/hc/en-us/articles/115014893968-Terms-of-sale}{Terms
  of Sale}
\item
  \href{https://spiderbites.nytimes.com}{Site Map}
\item
  \href{https://help.nytimes.com/hc/en-us}{Help}
\item
  \href{https://www.nytimes.com/subscription?campaignId=37WXW}{Subscriptions}
\end{itemize}
