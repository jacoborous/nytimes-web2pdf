Sections

SEARCH

\protect\hyperlink{site-content}{Skip to
content}\protect\hyperlink{site-index}{Skip to site index}

\href{https://www.nytimes.com/section/health}{Health}

\href{https://myaccount.nytimes.com/auth/login?response_type=cookie\&client_id=vi}{}

\href{https://www.nytimes.com/section/todayspaper}{Today's Paper}

\href{/section/health}{Health}\textbar{}A New Entry in the Race for a
Coronavirus Vaccine: Hope

\url{https://nyti.ms/2X82BLK}

\begin{itemize}
\item
\item
\item
\item
\item
\item
\end{itemize}

\href{https://www.nytimes.com/news-event/coronavirus?action=click\&pgtype=Article\&state=default\&region=TOP_BANNER\&context=storylines_menu}{The
Coronavirus Outbreak}

\begin{itemize}
\tightlist
\item
  live\href{https://www.nytimes.com/2020/08/02/world/coronavirus-updates.html?action=click\&pgtype=Article\&state=default\&region=TOP_BANNER\&context=storylines_menu}{Latest
  Updates}
\item
  \href{https://www.nytimes.com/interactive/2020/us/coronavirus-us-cases.html?action=click\&pgtype=Article\&state=default\&region=TOP_BANNER\&context=storylines_menu}{Maps
  and Cases}
\item
  \href{https://www.nytimes.com/interactive/2020/science/coronavirus-vaccine-tracker.html?action=click\&pgtype=Article\&state=default\&region=TOP_BANNER\&context=storylines_menu}{Vaccine
  Tracker}
\item
  \href{https://www.nytimes.com/interactive/2020/07/29/us/schools-reopening-coronavirus.html?action=click\&pgtype=Article\&state=default\&region=TOP_BANNER\&context=storylines_menu}{What
  School May Look Like}
\item
  \href{https://www.nytimes.com/live/2020/07/31/business/stock-market-today-coronavirus?action=click\&pgtype=Article\&state=default\&region=TOP_BANNER\&context=storylines_menu}{Economy}
\end{itemize}

Advertisement

\protect\hyperlink{after-top}{Continue reading the main story}

Supported by

\protect\hyperlink{after-sponsor}{Continue reading the main story}

\hypertarget{a-new-entry-in-the-race-for-a-coronavirus-vaccine-hope}{%
\section{A New Entry in the Race for a Coronavirus Vaccine:
Hope}\label{a-new-entry-in-the-race-for-a-coronavirus-vaccine-hope}}

Scientists are increasingly optimistic that a vaccine can be produced in
record time. But getting it manufactured and distributed will pose huge
challenges.

\includegraphics{https://static01.nyt.com/images/2020/05/19/science/19VIRUS-VACCINES1/19VIRUS-VACCINES1-articleLarge.jpg?quality=75\&auto=webp\&disable=upscale}

\href{https://www.nytimes.com/by/carl-zimmer}{\includegraphics{https://static01.nyt.com/images/2018/06/12/multimedia/author-carl-zimmer/author-carl-zimmer-thumbLarge.png}}\href{https://www.nytimes.com/by/knvul-sheikh}{\includegraphics{https://static01.nyt.com/images/2020/01/03/reader-center/author-knvul-sheikh/author-knvul-sheikh-thumbLarge.png}}\href{https://www.nytimes.com/by/noah-weiland}{\includegraphics{https://static01.nyt.com/images/2019/07/23/reader-center/author-noah-weiland/author-noah-weiland-thumbLarge.png}}

By \href{https://www.nytimes.com/by/carl-zimmer}{Carl Zimmer},
\href{https://www.nytimes.com/by/knvul-sheikh}{Knvul Sheikh} and
\href{https://www.nytimes.com/by/noah-weiland}{Noah Weiland}

\begin{itemize}
\item
  Published May 20, 2020Updated July 1, 2020
\item
  \begin{itemize}
  \item
  \item
  \item
  \item
  \item
  \item
  \end{itemize}
\end{itemize}

\href{https://www.nytimes.com/es/2020/05/22/espanol/ciencia-y-tecnologia/vacuna-coronavirus.html}{Leer
en español}

In a medical research project nearly unrivaled in its ambition and
scope, volunteers worldwide are rolling up their sleeves to receive
experimental vaccines against the
\href{https://www.nytimes.com/2020/07/01/health/coronavirus-vaccine-trials.html}{coronavirus}
--- only months after the virus was identified.

Companies like Inovio and Pfizer have begun early tests of candidates in
people to determine whether their
\href{https://www.nytimes.com/2020/07/01/health/coronavirus-vaccine-trials.html}{vaccines}
are safe. Researchers at the University of Oxford in England are testing
vaccines in human subjects, too, and say they could have one ready for
emergency use as soon as September.

\href{https://www.nytimes.com/2020/05/18/health/coronavirus-vaccine-moderna.html}{Moderna
on Monday announced encouraging results} of a safety trial of its
vaccine in eight volunteers. There were no published data, but the news
alone kindled hopes and sent the company's stock soaring.

Animal studies have raised expectations, too. Researchers at Beth Israel
Deaconess Medical Center on Wednesday published research showing that a
prototype vaccine
\href{https://www.nytimes.com/2020/05/20/health/coronavirus-vaccine-harvard.html}{effectively
protected monkeys from infection with the virus}.

The findings will pave the way to development of a human vaccine, said
the investigators. They have already partnered with Janssen, a division
of Johnson \& Johnson.

In labs around the world, there is now cautious optimism that a
\href{https://www.nytimes.com/2020/06/03/us/politics/coronavirus-vaccine-trump-moderna.html}{coronavirus
vaccine}, and perhaps more than one, will be ready sometime next year.

\emph{{[}}\href{https://www.nytimes.com/interactive/2020/science/coronavirus-vaccine-tracker.html}{\emph{Follow
our Live Coronavirus Vaccine Tracker}}\emph{.{]}}

Scientists are exploring not just one approach to creating the vaccine,
but at least four. So great is the urgency that they are combining trial
phases and shortening a process that usually takes years, sometimes more
than a decade.

The coronavirus itself has turned out to be clumsy prey, a stable
pathogen
\href{https://www.nytimes.com/interactive/2020/04/30/science/coronavirus-mutations.html}{unlikely
to mutate significantly and dodge a vaccine}.

``It's an easier target, which is terrific news,'' said Michael Farzan,
a virologist at Scripps Research in Jupiter, Fla.

An effective vaccine will be crucial to ending the pandemic, which has
sickened
\href{https://www.nytimes.com/interactive/2020/world/coronavirus-maps.html}{at
least 4.7 million worldwide and killed at least 324,000}. Widespread
immunity would reopen the door to lives without social distancing and
face masks.

\hypertarget{latest-updates-global-coronavirus-outbreak}{%
\section{\texorpdfstring{\href{https://www.nytimes.com/2020/08/01/world/coronavirus-covid-19.html?action=click\&pgtype=Article\&state=default\&region=MAIN_CONTENT_1\&context=storylines_live_updates}{Latest
Updates: Global Coronavirus
Outbreak}}{Latest Updates: Global Coronavirus Outbreak}}\label{latest-updates-global-coronavirus-outbreak}}

Updated 2020-08-02T17:52:35.962Z

\begin{itemize}
\tightlist
\item
  \href{https://www.nytimes.com/2020/08/01/world/coronavirus-covid-19.html?action=click\&pgtype=Article\&state=default\&region=MAIN_CONTENT_1\&context=storylines_live_updates\#link-34047410}{The
  U.S. reels as July cases more than double the total of any other
  month.}
\item
  \href{https://www.nytimes.com/2020/08/01/world/coronavirus-covid-19.html?action=click\&pgtype=Article\&state=default\&region=MAIN_CONTENT_1\&context=storylines_live_updates\#link-780ec966}{Top
  U.S. officials work to break an impasse over the federal jobless
  benefit.}
\item
  \href{https://www.nytimes.com/2020/08/01/world/coronavirus-covid-19.html?action=click\&pgtype=Article\&state=default\&region=MAIN_CONTENT_1\&context=storylines_live_updates\#link-2bc8948}{Its
  outbreak untamed, Melbourne goes into even greater lockdown.}
\end{itemize}

\href{https://www.nytimes.com/2020/08/01/world/coronavirus-covid-19.html?action=click\&pgtype=Article\&state=default\&region=MAIN_CONTENT_1\&context=storylines_live_updates}{See
more updates}

More live coverage:
\href{https://www.nytimes.com/live/2020/07/31/business/stock-market-today-coronavirus?action=click\&pgtype=Article\&state=default\&region=MAIN_CONTENT_1\&context=storylines_live_updates}{Markets}

``What people don't realize is that normally vaccine development takes
many years, sometimes decades,'' said Dr. Dan Barouch, a virologist at
Beth Israel Deaconess Medical Center in Boston who led the monkey
trials. ``And so trying to compress the whole vaccine process into 12 to
18 months is really unheard-of.''

``If that happens, it will be the fastest vaccine development program
ever in history.''

More than 100 research teams around the world are taking aim at the
virus from multiple angles.

Moderna's vaccine is based on a relatively new mRNA technology that
delivers bits of the virus's genes into human cells. The goal is for
cells to begin making a viral protein that the immune system recognizes
as foreign. The body builds defenses against that protein, priming
itself to attack if the actual coronavirus invades.

\includegraphics{https://static01.nyt.com/images/2020/05/19/science/19VIRUS-VACCINES3/merlin_172650756_1b50e804-a091-4421-951a-4e8f58651288-articleLarge.jpg?quality=75\&auto=webp\&disable=upscale}

Some vaccine makers, including Inovio, are developing vaccines based on
DNA variations of this approach.

But the technology used by both companies has never produced a vaccine
approved for clinical use, let alone one that can be made in industrial
quantities. Moderna was criticized for making rosy predictions, based on
a handful of patients, without providing any scientific data.

Other research teams have turned to more traditional strategies.

Some scientists are using harmless viruses to deliver coronavirus genes
into cells, forcing them to produce proteins that may teach the immune
system to watch out for the coronavirus. CanSino Biologics, a company in
China, has begun human testing of a coronavirus vaccine that relies on
this approach, as has the University of Oxford team.

Other traditional approaches rely on fragments of a coronavirus protein
to make a vaccine, while some use killed, or inactivated, versions of
the whole coronavirus. In China, such vaccines have already entered
human trials.

Florian Krammer, a virologist at Icahn School of Medicine at Mount Sinai
in New York, predicted that at least 20 additional vaccine candidates
will make their way into clinical trials in the weeks to come.

``I'm not worried at all about it,'' he said of the prospects for a new
vaccine.

Many of these vaccines will stumble as the trials progress. As more
people are inoculated, some candidates will fail to protect against the
virus, and side effects will become more apparent.

But from what scientists are learning about the coronavirus, it ought to
be a relatively easy target.

The coronavirus sports tempting targets on its surface, unique ``spike''
proteins the pathogen needs to enter human cells. The immune system
readily learns to recognize these proteins, it appears, and to attack
them, killing the virus.

Viruses can challenge vaccine makers by mutating rapidly, changing shape
so that antibodies that work on one viral strain fail on another.
Thankfully,
\href{https://www.nytimes.com/2020/05/06/health/coronavirus-mutation-transmission.html}{the
new coronavirus seems to be a slow mutator}, and a vaccine that proves
effective in trials should work anywhere in the world.

\href{https://www.nytimes.com/interactive/2020/04/30/science/coronavirus-mutations.html}{}

\includegraphics{https://static01.nyt.com/images/2020/04/29/us/coronavirus-mutations-promo-1588180690446/coronavirus-mutations-promo-1588180690446-articleLarge.jpg}

\hypertarget{how-coronavirus-mutates-and-spreads}{%
\subsection{How Coronavirus Mutates and
Spreads}\label{how-coronavirus-mutates-and-spreads}}

The virus has mutated. But that doesn't mean it's getting deadlier.

When work on a coronavirus vaccine started, some researchers worried
that antibodies actually might worsen Covid-19, the illness caused by
the coronavirus. But in early studies, no serious risks have emerged.

``That doesn't mean that there won't be, but so far there hasn't been
any indication, so I'm cautiously optimistic on that point,'' said Dr.
Alyson Kelvin, a researcher at the Canadian Center for Vaccinology and
Dalhousie University.

\hypertarget{scaling-up}{%
\subsection{Scaling Up}\label{scaling-up}}

Ensuring that vaccines are safe and effective demands large trials that
require careful planning and execution. If successful vaccines emerge
from those trials, someone's going to have to make an awful lot of them.

Almost everyone on the planet is vulnerable to the new coronavirus. Each
person may need two doses of a new vaccine to receive protective
immunity. That's 16 billion doses.

``When companies promise of delivering a vaccine in a year or less, I am
not sure what stage they are talking about,'' said Akiko Iwasaki, an
immunobiologist at Yale University. ``I doubt they are talking about
global distributions in billions of doses.''

Manufacturing vaccines is profoundly more complex than manufacturing,
say, shoes or bicycles. Vaccines typically require large vats in which
their ingredients are grown, and these have to be maintained in sterile
conditions. Also, no factories have ever churned out millions of doses
of approved vaccines made with the cutting-edge technology being tested
by companies like Inovio and Moderna.

\href{https://www.nytimes.com/news-event/coronavirus?action=click\&pgtype=Article\&state=default\&region=MAIN_CONTENT_3\&context=storylines_faq}{}

\hypertarget{the-coronavirus-outbreak-}{%
\subsubsection{The Coronavirus Outbreak
›}\label{the-coronavirus-outbreak-}}

\hypertarget{frequently-asked-questions}{%
\paragraph{Frequently Asked
Questions}\label{frequently-asked-questions}}

Updated July 27, 2020

\begin{itemize}
\item ~
  \hypertarget{should-i-refinance-my-mortgage}{%
  \paragraph{Should I refinance my
  mortgage?}\label{should-i-refinance-my-mortgage}}

  \begin{itemize}
  \tightlist
  \item
    \href{https://www.nytimes.com/article/coronavirus-money-unemployment.html?action=click\&pgtype=Article\&state=default\&region=MAIN_CONTENT_3\&context=storylines_faq}{It
    could be a good idea,} because mortgage rates have
    \href{https://www.nytimes.com/2020/07/16/business/mortgage-rates-below-3-percent.html?action=click\&pgtype=Article\&state=default\&region=MAIN_CONTENT_3\&context=storylines_faq}{never
    been lower.} Refinancing requests have pushed mortgage applications
    to some of the highest levels since 2008, so be prepared to get in
    line. But defaults are also up, so if you're thinking about buying a
    home, be aware that some lenders have tightened their standards.
  \end{itemize}
\item ~
  \hypertarget{what-is-school-going-to-look-like-in-september}{%
  \paragraph{What is school going to look like in
  September?}\label{what-is-school-going-to-look-like-in-september}}

  \begin{itemize}
  \tightlist
  \item
    It is unlikely that many schools will return to a normal schedule
    this fall, requiring the grind of
    \href{https://www.nytimes.com/2020/06/05/us/coronavirus-education-lost-learning.html?action=click\&pgtype=Article\&state=default\&region=MAIN_CONTENT_3\&context=storylines_faq}{online
    learning},
    \href{https://www.nytimes.com/2020/05/29/us/coronavirus-child-care-centers.html?action=click\&pgtype=Article\&state=default\&region=MAIN_CONTENT_3\&context=storylines_faq}{makeshift
    child care} and
    \href{https://www.nytimes.com/2020/06/03/business/economy/coronavirus-working-women.html?action=click\&pgtype=Article\&state=default\&region=MAIN_CONTENT_3\&context=storylines_faq}{stunted
    workdays} to continue. California's two largest public school
    districts --- Los Angeles and San Diego --- said on July 13, that
    \href{https://www.nytimes.com/2020/07/13/us/lausd-san-diego-school-reopening.html?action=click\&pgtype=Article\&state=default\&region=MAIN_CONTENT_3\&context=storylines_faq}{instruction
    will be remote-only in the fall}, citing concerns that surging
    coronavirus infections in their areas pose too dire a risk for
    students and teachers. Together, the two districts enroll some
    825,000 students. They are the largest in the country so far to
    abandon plans for even a partial physical return to classrooms when
    they reopen in August. For other districts, the solution won't be an
    all-or-nothing approach.
    \href{https://bioethics.jhu.edu/research-and-outreach/projects/eschool-initiative/school-policy-tracker/}{Many
    systems}, including the nation's largest, New York City, are
    devising
    \href{https://www.nytimes.com/2020/06/26/us/coronavirus-schools-reopen-fall.html?action=click\&pgtype=Article\&state=default\&region=MAIN_CONTENT_3\&context=storylines_faq}{hybrid
    plans} that involve spending some days in classrooms and other days
    online. There's no national policy on this yet, so check with your
    municipal school system regularly to see what is happening in your
    community.
  \end{itemize}
\item ~
  \hypertarget{is-the-coronavirus-airborne}{%
  \paragraph{Is the coronavirus
  airborne?}\label{is-the-coronavirus-airborne}}

  \begin{itemize}
  \tightlist
  \item
    The coronavirus
    \href{https://www.nytimes.com/2020/07/04/health/239-experts-with-one-big-claim-the-coronavirus-is-airborne.html?action=click\&pgtype=Article\&state=default\&region=MAIN_CONTENT_3\&context=storylines_faq}{can
    stay aloft for hours in tiny droplets in stagnant air}, infecting
    people as they inhale, mounting scientific evidence suggests. This
    risk is highest in crowded indoor spaces with poor ventilation, and
    may help explain super-spreading events reported in meatpacking
    plants, churches and restaurants.
    \href{https://www.nytimes.com/2020/07/06/health/coronavirus-airborne-aerosols.html?action=click\&pgtype=Article\&state=default\&region=MAIN_CONTENT_3\&context=storylines_faq}{It's
    unclear how often the virus is spread} via these tiny droplets, or
    aerosols, compared with larger droplets that are expelled when a
    sick person coughs or sneezes, or transmitted through contact with
    contaminated surfaces, said Linsey Marr, an aerosol expert at
    Virginia Tech. Aerosols are released even when a person without
    symptoms exhales, talks or sings, according to Dr. Marr and more
    than 200 other experts, who
    \href{https://academic.oup.com/cid/article/doi/10.1093/cid/ciaa939/5867798}{have
    outlined the evidence in an open letter to the World Health
    Organization}.
  \end{itemize}
\item ~
  \hypertarget{what-are-the-symptoms-of-coronavirus}{%
  \paragraph{What are the symptoms of
  coronavirus?}\label{what-are-the-symptoms-of-coronavirus}}

  \begin{itemize}
  \tightlist
  \item
    Common symptoms
    \href{https://www.nytimes.com/article/symptoms-coronavirus.html?action=click\&pgtype=Article\&state=default\&region=MAIN_CONTENT_3\&context=storylines_faq}{include
    fever, a dry cough, fatigue and difficulty breathing or shortness of
    breath.} Some of these symptoms overlap with those of the flu,
    making detection difficult, but runny noses and stuffy sinuses are
    less common.
    \href{https://www.nytimes.com/2020/04/27/health/coronavirus-symptoms-cdc.html?action=click\&pgtype=Article\&state=default\&region=MAIN_CONTENT_3\&context=storylines_faq}{The
    C.D.C. has also} added chills, muscle pain, sore throat, headache
    and a new loss of the sense of taste or smell as symptoms to look
    out for. Most people fall ill five to seven days after exposure, but
    symptoms may appear in as few as two days or as many as 14 days.
  \end{itemize}
\item ~
  \hypertarget{does-asymptomatic-transmission-of-covid-19-happen}{%
  \paragraph{Does asymptomatic transmission of Covid-19
  happen?}\label{does-asymptomatic-transmission-of-covid-19-happen}}

  \begin{itemize}
  \tightlist
  \item
    So far, the evidence seems to show it does. A widely cited
    \href{https://www.nature.com/articles/s41591-020-0869-5}{paper}
    published in April suggests that people are most infectious about
    two days before the onset of coronavirus symptoms and estimated that
    44 percent of new infections were a result of transmission from
    people who were not yet showing symptoms. Recently, a top expert at
    the World Health Organization stated that transmission of the
    coronavirus by people who did not have symptoms was ``very rare,''
    \href{https://www.nytimes.com/2020/06/09/world/coronavirus-updates.html?action=click\&pgtype=Article\&state=default\&region=MAIN_CONTENT_3\&context=storylines_faq\#link-1f302e21}{but
    she later walked back that statement.}
  \end{itemize}
\end{itemize}

Facilities have sprung up in recent years to make viral-vector vaccines,
including a Johnson \& Johnson plant in the Netherlands. But meeting
pandemic demand would be an enormous challenge. Manufacturers have the
most experience mass-producing inactivated vaccines, made with killed
viruses, so this type may be the easiest to produce in large quantities.

Image

A participant in the first phase of Inovio's vaccine trial receiving an
injection in Kansas City, Mo., last month.Credit...Center for
Pharmaceutical Research, via Associated Press

But there cannot be just one vaccine. If that were to happen, the
company that made it would have no chance of meeting the world's demand.

``The hope is that they will all, at some level, be effective, and
certainly that's important because we need more than just one,'' said
Emilio Emini, a director of the vaccine program at the Bill and Melinda
Gates Foundation, which is providing financial support to many competing
vaccine efforts.

As part of a public-private partnership the White House calls Operation
Warp Speed, the Trump administration has promised to design a kind of
parallel manufacturing track to run alongside the clinical trials,
building up capacity well before trials are concluded, in hopes that one
or more vaccines could be distributed immediately upon approval.

President Trump said on Friday that the goal of the project was to
distribute a vaccine ``prior to the end of the year.'' To do that, Mr.
Trump is relying on the Defense Department to manage the manufacturing
logistics related to vaccine development.

But in an interview on Thursday, Gen. Gustave F. Perna, who will manage
the manufacturing logistics, said discussions about the equipment and
facilities needed for production were just beginning.

He described his work as a ``math problem'': how to get 300 million
doses of a vaccine that doesn't yet exist to Americans --- by January.

\textbf{\emph{{[}}\href{http://on.fb.me/1paTQ1h}{\emph{Like the Science
Times page on Facebook.}}} ****** \emph{\textbar{} Sign up for the}
\textbf{\href{http://nyti.ms/1MbHaRU}{\emph{Science Times
newsletter.}}\emph{{]}}}

Finding the supplies and planning their distribution would occur at the
same time, he said. ``I need to have syringes,'' General Perna said. ``I
need to have wipes, right? I need to have Band-Aids. I need to have the
vaccine.''

He added: ``Now, how am I going to distribute it? What is it going to be
distributed in? What do I need to order now to make sure I have the
distribution capability? The small bottles, the trucks.''

Dr. Amesh Adalja, an infectious disease physician and senior scholar at
the Johns Hopkins University Center for Health Security, said that
seemingly minor aspects of production and distribution could complicate
progress later on.

``This is on a scale we've never seen since the polio vaccine,'' he
said. ``It's the little things like the syringes, the needles, the glass
vials. All of that has to be thought about. You don't want something
that seems so simple to be the bottleneck in your vaccination program.''

A coronavirus vaccine doesn't yet exist, but already there are questions
about who will be able to afford it.

At the World Health Assembly meeting this week, a proposal from the
European Union
\href{https://www.reuters.com/article/us-health-coronavirus-who-resolution/eu-resolution-on-pandemic-adopted-at-who-assembly-official-idUSKBN22V1RS}{was
adopted recommending a voluntary patent pool}, which would put pressure
on companies to give up their monopolies on vaccines they've developed.

Oxfam, an international charity, has published an open letter from 140
world leaders and experts calling for a ``people's vaccine,'' which
would be ``made available for all people, in all countries, free of
charge.''

``These vaccines have to be a public good,'' said Helen Clark, a former
prime minister of New Zealand, who signed the letter. ``We're not safe
till everyone is safe.''

Sui-Lee Wee contributed reporting from Singapore.

Advertisement

\protect\hyperlink{after-bottom}{Continue reading the main story}

\hypertarget{site-index}{%
\subsection{Site Index}\label{site-index}}

\hypertarget{site-information-navigation}{%
\subsection{Site Information
Navigation}\label{site-information-navigation}}

\begin{itemize}
\tightlist
\item
  \href{https://help.nytimes.com/hc/en-us/articles/115014792127-Copyright-notice}{©~2020~The
  New York Times Company}
\end{itemize}

\begin{itemize}
\tightlist
\item
  \href{https://www.nytco.com/}{NYTCo}
\item
  \href{https://help.nytimes.com/hc/en-us/articles/115015385887-Contact-Us}{Contact
  Us}
\item
  \href{https://www.nytco.com/careers/}{Work with us}
\item
  \href{https://nytmediakit.com/}{Advertise}
\item
  \href{http://www.tbrandstudio.com/}{T Brand Studio}
\item
  \href{https://www.nytimes.com/privacy/cookie-policy\#how-do-i-manage-trackers}{Your
  Ad Choices}
\item
  \href{https://www.nytimes.com/privacy}{Privacy}
\item
  \href{https://help.nytimes.com/hc/en-us/articles/115014893428-Terms-of-service}{Terms
  of Service}
\item
  \href{https://help.nytimes.com/hc/en-us/articles/115014893968-Terms-of-sale}{Terms
  of Sale}
\item
  \href{https://spiderbites.nytimes.com}{Site Map}
\item
  \href{https://help.nytimes.com/hc/en-us}{Help}
\item
  \href{https://www.nytimes.com/subscription?campaignId=37WXW}{Subscriptions}
\end{itemize}
