Sections

SEARCH

\protect\hyperlink{site-content}{Skip to
content}\protect\hyperlink{site-index}{Skip to site index}

\href{https://www.nytimes.com/section/health}{Health}

\href{https://myaccount.nytimes.com/auth/login?response_type=cookie\&client_id=vi}{}

\href{https://www.nytimes.com/section/todayspaper}{Today's Paper}

\href{/section/health}{Health}\textbar{}Prototype Vaccine Protects
Monkeys From Coronavirus

\url{https://nyti.ms/3cPxYS6}

\begin{itemize}
\item
\item
\item
\item
\item
\item
\end{itemize}

\href{https://www.nytimes.com/news-event/coronavirus?action=click\&pgtype=Article\&state=default\&region=TOP_BANNER\&context=storylines_menu}{The
Coronavirus Outbreak}

\begin{itemize}
\tightlist
\item
  live\href{https://www.nytimes.com/2020/08/02/world/coronavirus-updates.html?action=click\&pgtype=Article\&state=default\&region=TOP_BANNER\&context=storylines_menu}{Latest
  Updates}
\item
  \href{https://www.nytimes.com/interactive/2020/us/coronavirus-us-cases.html?action=click\&pgtype=Article\&state=default\&region=TOP_BANNER\&context=storylines_menu}{Maps
  and Cases}
\item
  \href{https://www.nytimes.com/interactive/2020/science/coronavirus-vaccine-tracker.html?action=click\&pgtype=Article\&state=default\&region=TOP_BANNER\&context=storylines_menu}{Vaccine
  Tracker}
\item
  \href{https://www.nytimes.com/interactive/2020/07/29/us/schools-reopening-coronavirus.html?action=click\&pgtype=Article\&state=default\&region=TOP_BANNER\&context=storylines_menu}{What
  School May Look Like}
\item
  \href{https://www.nytimes.com/live/2020/07/31/business/stock-market-today-coronavirus?action=click\&pgtype=Article\&state=default\&region=TOP_BANNER\&context=storylines_menu}{Economy}
\end{itemize}

Advertisement

\protect\hyperlink{after-top}{Continue reading the main story}

Supported by

\protect\hyperlink{after-sponsor}{Continue reading the main story}

matter

\hypertarget{prototype-vaccine-protects-monkeys-from-coronavirus}{%
\section{Prototype Vaccine Protects Monkeys From
Coronavirus}\label{prototype-vaccine-protects-monkeys-from-coronavirus}}

A series of animal experiments may point the way to an effective human
vaccine, scientists said.

\includegraphics{https://static01.nyt.com/images/2020/05/26/science/20VIRUS-MONKEYS1/merlin_172650474_1f9da522-ec11-4a9a-a6b9-0661946e1b06-articleLarge.jpg?quality=75\&auto=webp\&disable=upscale}

\href{https://www.nytimes.com/by/carl-zimmer}{\includegraphics{https://static01.nyt.com/images/2018/06/12/multimedia/author-carl-zimmer/author-carl-zimmer-thumbLarge.png}}

By \href{https://www.nytimes.com/by/carl-zimmer}{Carl Zimmer}

\begin{itemize}
\item
  Published May 20, 2020Updated May 25, 2020
\item
  \begin{itemize}
  \item
  \item
  \item
  \item
  \item
  \item
  \end{itemize}
\end{itemize}

A prototype
\href{https://www.nytimes.com/2020/05/20/health/coronavirus-vaccines.html}{vaccine}
has protected monkeys from the
\href{https://www.nytimes.com/2020/05/20/health/coronavirus-vaccines.html}{coronavirus},
researchers reported on Wednesday, a finding that offers new hope for
effective human vaccines.

Scientists are already testing
\href{https://www.nytimes.com/2020/06/03/us/politics/coronavirus-vaccine-trump-moderna.html}{coronavirus
vaccines} in people, but the initial trials are designed to determine
safety, not how well a vaccine works. The research published Wednesday
offers insight into what a vaccine must do to be effective and how to
measure that.

``To me, this is convincing that a vaccine is possible,'' said Dr.
Nelson Michael, the director of the Center for Infectious Diseases
Research at Walter Reed Army Institute of Research.

Scientists are engaged in a worldwide scramble to create a vaccine
against the new coronavirus. Over a hundred research projects have been
launched; early safety trials in humans have been started or completed
in nine of them.

Next to come are larger trials to determine whether these candidate
vaccines are not just safe, but effective. But those results won't
arrive for months.

In the meantime, Dr. Dan Barouch, a virologist at Beth Israel Deaconess
Medical Center in Boston, and his colleagues have started a series of
experiments on monkeys to get a broader look at how coronaviruses affect
monkeys --- and whether vaccines might fight the pathogens. Their report
was published in Science.

Dr. Barouch is working in a partnership with Johnson \& Johnson, which
is developing a coronavirus vaccine that uses a specially modified
virus, called Ad26, that he developed.

The new research in monkeys ``lays the scientific foundations'' for
those efforts, Dr. Barouch said. In March, the federal government
awarded \$450 million to Janssen Pharmaceuticals, a division of Johnson
\& Johnson, to develop a coronavirus vaccine.

The scientists started by studying whether the monkeys become immune to
the virus after getting sick. The team infected nine unvaccinated rhesus
macaques with the new coronavirus.

The monkeys developed symptoms that resembled a moderate case of
Covid-19, including inflammation in their lungs that led to pneumonia.
The monkeys recovered after a few days, and Dr. Barouch and his
colleagues found that the animals had begun making antibodies to the
coronavirus.

\hypertarget{latest-updates-global-coronavirus-outbreak}{%
\section{\texorpdfstring{\href{https://www.nytimes.com/2020/08/01/world/coronavirus-covid-19.html?action=click\&pgtype=Article\&state=default\&region=MAIN_CONTENT_1\&context=storylines_live_updates}{Latest
Updates: Global Coronavirus
Outbreak}}{Latest Updates: Global Coronavirus Outbreak}}\label{latest-updates-global-coronavirus-outbreak}}

Updated 2020-08-02T17:52:35.962Z

\begin{itemize}
\tightlist
\item
  \href{https://www.nytimes.com/2020/08/01/world/coronavirus-covid-19.html?action=click\&pgtype=Article\&state=default\&region=MAIN_CONTENT_1\&context=storylines_live_updates\#link-34047410}{The
  U.S. reels as July cases more than double the total of any other
  month.}
\item
  \href{https://www.nytimes.com/2020/08/01/world/coronavirus-covid-19.html?action=click\&pgtype=Article\&state=default\&region=MAIN_CONTENT_1\&context=storylines_live_updates\#link-780ec966}{Top
  U.S. officials work to break an impasse over the federal jobless
  benefit.}
\item
  \href{https://www.nytimes.com/2020/08/01/world/coronavirus-covid-19.html?action=click\&pgtype=Article\&state=default\&region=MAIN_CONTENT_1\&context=storylines_live_updates\#link-2bc8948}{Its
  outbreak untamed, Melbourne goes into even greater lockdown.}
\end{itemize}

\href{https://www.nytimes.com/2020/08/01/world/coronavirus-covid-19.html?action=click\&pgtype=Article\&state=default\&region=MAIN_CONTENT_1\&context=storylines_live_updates}{See
more updates}

More live coverage:
\href{https://www.nytimes.com/live/2020/07/31/business/stock-market-today-coronavirus?action=click\&pgtype=Article\&state=default\&region=MAIN_CONTENT_1\&context=storylines_live_updates}{Markets}

Some of them turned out to be so-called neutralizing antibodies, meaning
that they stopped the virus from entering cells and reproducing.

Thirty-five days after inoculating the monkeys, the researchers carried
out a ``re-challenge,'' spraying a second dose of the coronavirus into
the noses of the animals.

The monkeys
\href{https://science.sciencemag.org/content/early/2020/05/19/science.abc4776}{produced
a surge of protective neutralizing antibodies}. The coronavirus briefly
managed to establish a small infection in the monkey's noses but was
soon wiped out.

\includegraphics{https://static01.nyt.com/images/2020/05/20/science/20VIRUS-MONKEYS2/20VIRUS-MONKEYS2-articleLarge.jpg?quality=75\&auto=webp\&disable=upscale}

These results don't necessarily mean that humans also develop strong
immunity to the coronavirus. Still, Dr. Barouch and others found the
research encouraging.

``If we did the re-challenge study and it didn't work, the implication
would be that the entire vaccine effort would fail,'' he said. ``That
would have been really, really bad news for seven billion people.''

In a separate experiment, Dr. Barouch and his colleagues tested
prototype vaccines on rhesus macaques. Each monkey received pieces of
DNA, which their cells turned into viral proteins designed to train the
immune system to recognize the virus.

Both macaques and humans make neutralizing antibodies against
coronaviruses that target one part in particular: a protein that covers
the virus's surface, called the spike protein.

Most coronavirus vaccines are intended to coax the immune system to make
antibodies that latch onto the spike protein and destroy the virus. Dr.
Barouch and his colleagues tried out six variations.

The researchers gave each vaccine to four or five monkeys. They let the
monkeys develop an immune response for three weeks, and then sprayed
viruses in their noses.

Some of the vaccines provided only partial protection. The virus wasn't
entirely eliminated from the animals' lungs or noses, although levels
were lower than in unvaccinated monkeys.

But other vaccines worked better. The one that worked best trained the
immune system to recognize and attack the entire spike protein of the
coronavirus. In eight monkeys, the researchers couldn't detect the virus
at all.

``I think that overall this will be seen as very good news for the
vaccine effort,'' said Dr. Barouch. ``This increases our optimism that a
vaccine for Covid-19 will be possible.''

Florian Krammer, a virologist at the Icahn School of Medicine at Mount
Sinai in New York who was not involved in the study, said that the
levels of antibodies seen in the monkeys were promising.

``This is something that would protect you from disease,'' he said.
``It's not perfect, but you certainly see protection.''

\href{https://www.nytimes.com/news-event/coronavirus?action=click\&pgtype=Article\&state=default\&region=MAIN_CONTENT_3\&context=storylines_faq}{}

\hypertarget{the-coronavirus-outbreak-}{%
\subsubsection{The Coronavirus Outbreak
›}\label{the-coronavirus-outbreak-}}

\hypertarget{frequently-asked-questions}{%
\paragraph{Frequently Asked
Questions}\label{frequently-asked-questions}}

Updated July 27, 2020

\begin{itemize}
\item ~
  \hypertarget{should-i-refinance-my-mortgage}{%
  \paragraph{Should I refinance my
  mortgage?}\label{should-i-refinance-my-mortgage}}

  \begin{itemize}
  \tightlist
  \item
    \href{https://www.nytimes.com/article/coronavirus-money-unemployment.html?action=click\&pgtype=Article\&state=default\&region=MAIN_CONTENT_3\&context=storylines_faq}{It
    could be a good idea,} because mortgage rates have
    \href{https://www.nytimes.com/2020/07/16/business/mortgage-rates-below-3-percent.html?action=click\&pgtype=Article\&state=default\&region=MAIN_CONTENT_3\&context=storylines_faq}{never
    been lower.} Refinancing requests have pushed mortgage applications
    to some of the highest levels since 2008, so be prepared to get in
    line. But defaults are also up, so if you're thinking about buying a
    home, be aware that some lenders have tightened their standards.
  \end{itemize}
\item ~
  \hypertarget{what-is-school-going-to-look-like-in-september}{%
  \paragraph{What is school going to look like in
  September?}\label{what-is-school-going-to-look-like-in-september}}

  \begin{itemize}
  \tightlist
  \item
    It is unlikely that many schools will return to a normal schedule
    this fall, requiring the grind of
    \href{https://www.nytimes.com/2020/06/05/us/coronavirus-education-lost-learning.html?action=click\&pgtype=Article\&state=default\&region=MAIN_CONTENT_3\&context=storylines_faq}{online
    learning},
    \href{https://www.nytimes.com/2020/05/29/us/coronavirus-child-care-centers.html?action=click\&pgtype=Article\&state=default\&region=MAIN_CONTENT_3\&context=storylines_faq}{makeshift
    child care} and
    \href{https://www.nytimes.com/2020/06/03/business/economy/coronavirus-working-women.html?action=click\&pgtype=Article\&state=default\&region=MAIN_CONTENT_3\&context=storylines_faq}{stunted
    workdays} to continue. California's two largest public school
    districts --- Los Angeles and San Diego --- said on July 13, that
    \href{https://www.nytimes.com/2020/07/13/us/lausd-san-diego-school-reopening.html?action=click\&pgtype=Article\&state=default\&region=MAIN_CONTENT_3\&context=storylines_faq}{instruction
    will be remote-only in the fall}, citing concerns that surging
    coronavirus infections in their areas pose too dire a risk for
    students and teachers. Together, the two districts enroll some
    825,000 students. They are the largest in the country so far to
    abandon plans for even a partial physical return to classrooms when
    they reopen in August. For other districts, the solution won't be an
    all-or-nothing approach.
    \href{https://bioethics.jhu.edu/research-and-outreach/projects/eschool-initiative/school-policy-tracker/}{Many
    systems}, including the nation's largest, New York City, are
    devising
    \href{https://www.nytimes.com/2020/06/26/us/coronavirus-schools-reopen-fall.html?action=click\&pgtype=Article\&state=default\&region=MAIN_CONTENT_3\&context=storylines_faq}{hybrid
    plans} that involve spending some days in classrooms and other days
    online. There's no national policy on this yet, so check with your
    municipal school system regularly to see what is happening in your
    community.
  \end{itemize}
\item ~
  \hypertarget{is-the-coronavirus-airborne}{%
  \paragraph{Is the coronavirus
  airborne?}\label{is-the-coronavirus-airborne}}

  \begin{itemize}
  \tightlist
  \item
    The coronavirus
    \href{https://www.nytimes.com/2020/07/04/health/239-experts-with-one-big-claim-the-coronavirus-is-airborne.html?action=click\&pgtype=Article\&state=default\&region=MAIN_CONTENT_3\&context=storylines_faq}{can
    stay aloft for hours in tiny droplets in stagnant air}, infecting
    people as they inhale, mounting scientific evidence suggests. This
    risk is highest in crowded indoor spaces with poor ventilation, and
    may help explain super-spreading events reported in meatpacking
    plants, churches and restaurants.
    \href{https://www.nytimes.com/2020/07/06/health/coronavirus-airborne-aerosols.html?action=click\&pgtype=Article\&state=default\&region=MAIN_CONTENT_3\&context=storylines_faq}{It's
    unclear how often the virus is spread} via these tiny droplets, or
    aerosols, compared with larger droplets that are expelled when a
    sick person coughs or sneezes, or transmitted through contact with
    contaminated surfaces, said Linsey Marr, an aerosol expert at
    Virginia Tech. Aerosols are released even when a person without
    symptoms exhales, talks or sings, according to Dr. Marr and more
    than 200 other experts, who
    \href{https://academic.oup.com/cid/article/doi/10.1093/cid/ciaa939/5867798}{have
    outlined the evidence in an open letter to the World Health
    Organization}.
  \end{itemize}
\item ~
  \hypertarget{what-are-the-symptoms-of-coronavirus}{%
  \paragraph{What are the symptoms of
  coronavirus?}\label{what-are-the-symptoms-of-coronavirus}}

  \begin{itemize}
  \tightlist
  \item
    Common symptoms
    \href{https://www.nytimes.com/article/symptoms-coronavirus.html?action=click\&pgtype=Article\&state=default\&region=MAIN_CONTENT_3\&context=storylines_faq}{include
    fever, a dry cough, fatigue and difficulty breathing or shortness of
    breath.} Some of these symptoms overlap with those of the flu,
    making detection difficult, but runny noses and stuffy sinuses are
    less common.
    \href{https://www.nytimes.com/2020/04/27/health/coronavirus-symptoms-cdc.html?action=click\&pgtype=Article\&state=default\&region=MAIN_CONTENT_3\&context=storylines_faq}{The
    C.D.C. has also} added chills, muscle pain, sore throat, headache
    and a new loss of the sense of taste or smell as symptoms to look
    out for. Most people fall ill five to seven days after exposure, but
    symptoms may appear in as few as two days or as many as 14 days.
  \end{itemize}
\item ~
  \hypertarget{does-asymptomatic-transmission-of-covid-19-happen}{%
  \paragraph{Does asymptomatic transmission of Covid-19
  happen?}\label{does-asymptomatic-transmission-of-covid-19-happen}}

  \begin{itemize}
  \tightlist
  \item
    So far, the evidence seems to show it does. A widely cited
    \href{https://www.nature.com/articles/s41591-020-0869-5}{paper}
    published in April suggests that people are most infectious about
    two days before the onset of coronavirus symptoms and estimated that
    44 percent of new infections were a result of transmission from
    people who were not yet showing symptoms. Recently, a top expert at
    the World Health Organization stated that transmission of the
    coronavirus by people who did not have symptoms was ``very rare,''
    \href{https://www.nytimes.com/2020/06/09/world/coronavirus-updates.html?action=click\&pgtype=Article\&state=default\&region=MAIN_CONTENT_3\&context=storylines_faq\#link-1f302e21}{but
    she later walked back that statement.}
  \end{itemize}
\end{itemize}

Two vaccine teams --- one at the University of Oxford and one at the
China-based company Sinovac --- have tested vaccines on rhesus macaques.
This month they reported that their vaccines also offered the animals
protection.

The new study provides a deeper look at how vaccines protect monkeys,
and perhaps one day humans.

Along with neutralizing antibodies, the immune system has a huge arsenal
of weapons it can deploy against pathogens. Some immune cells can
recognize infected cells and destroy them, for example.

Dr. Barouch and his colleagues found a strong connection between
neutralizing antibodies and how well a vaccine worked: The vaccines that
gave monkeys stronger protection produced more neutralizing antibodies.

Pamela Bjorkman, a structural biologist at Caltech who was not involved
in the study, said that this correlation gave her more confidence in Dr.
Barouch's findings. ``I think that's really reassuring,'' she said.

Dr. Michael said that link could help scientists running safety trials
in humans. They may be able to get some early clues about whether the
vaccines are effective.

When a new vaccine goes into testing, the first round of trials are
designed to see if it's safe. Only then do researchers move forward with
bigger trials to determine if the vaccine actually protects against a
disease.

Vaccine designers often try different doses in a safety trial, looking
for the lowest dose that provides the greatest protection. Dr. Barouch's
study suggests that measuring neutralizing antibodies can give an
indication if a dose will be potent enough to give protection.

Malcolm Martin, a virologist at the National Institutes of Health who
was not involved in the study, cautioned that monkeys are different from
humans in important ways.

The unvaccinated monkeys in this study didn't develop any of the severe
symptoms that some people get following a coronavirus infection. ``It
looks like they got a cold,'' Dr. Martin said.

Lisa Tostanoski, a postdoctoral fellow working with Dr. Barouch and
co-author of the new study, noted that the study only offers a glimpse
at how the vaccine works three weeks after injection.

It's possible that the vaccines may defend the monkeys for many years to
come, she noted --- or the protection may fade much sooner.

How long immunity to the coronavirus lasts may determine whether people
will need just one shot of a vaccine or more. People may need boosters
from time to time to rev up their defenses again and keep the pandemic
at bay.

``Every three years is thinkable,'' Dr. Krammer said. ``That doesn't
mean a vaccine doesn't work.''

Advertisement

\protect\hyperlink{after-bottom}{Continue reading the main story}

\hypertarget{site-index}{%
\subsection{Site Index}\label{site-index}}

\hypertarget{site-information-navigation}{%
\subsection{Site Information
Navigation}\label{site-information-navigation}}

\begin{itemize}
\tightlist
\item
  \href{https://help.nytimes.com/hc/en-us/articles/115014792127-Copyright-notice}{©~2020~The
  New York Times Company}
\end{itemize}

\begin{itemize}
\tightlist
\item
  \href{https://www.nytco.com/}{NYTCo}
\item
  \href{https://help.nytimes.com/hc/en-us/articles/115015385887-Contact-Us}{Contact
  Us}
\item
  \href{https://www.nytco.com/careers/}{Work with us}
\item
  \href{https://nytmediakit.com/}{Advertise}
\item
  \href{http://www.tbrandstudio.com/}{T Brand Studio}
\item
  \href{https://www.nytimes.com/privacy/cookie-policy\#how-do-i-manage-trackers}{Your
  Ad Choices}
\item
  \href{https://www.nytimes.com/privacy}{Privacy}
\item
  \href{https://help.nytimes.com/hc/en-us/articles/115014893428-Terms-of-service}{Terms
  of Service}
\item
  \href{https://help.nytimes.com/hc/en-us/articles/115014893968-Terms-of-sale}{Terms
  of Sale}
\item
  \href{https://spiderbites.nytimes.com}{Site Map}
\item
  \href{https://help.nytimes.com/hc/en-us}{Help}
\item
  \href{https://www.nytimes.com/subscription?campaignId=37WXW}{Subscriptions}
\end{itemize}
