Sections

SEARCH

\protect\hyperlink{site-content}{Skip to
content}\protect\hyperlink{site-index}{Skip to site index}

\href{https://www.nytimes.com/section/politics}{Politics}

\href{https://myaccount.nytimes.com/auth/login?response_type=cookie\&client_id=vi}{}

\href{https://www.nytimes.com/section/todayspaper}{Today's Paper}

\href{/section/politics}{Politics}\textbar{}On Trump's To-Do List: Take
Back The Suburbs. Court Black Voters. Expand the Electoral Map. Win.

\url{https://nyti.ms/38clIsD}

\begin{itemize}
\item
\item
\item
\item
\item
\item
\end{itemize}

\begin{itemize}
\item
  \href{https://www.nytimes.com/2020/07/31/us/elections/biden-vs-trump.html?action=click\&pgtype=Article\&state=default\&region=TOP_BANNER\&context=storylines_menu}{Election
  Updates}
\item
  \href{https://www.nytimes.com/article/biden-vice-president-2020.html?action=click\&pgtype=Article\&state=default\&region=TOP_BANNER\&context=storylines_menu}{Biden's
  V.P. Search}
\item
  \href{https://www.nytimes.com/interactive/2020/07/24/us/politics/trump-biden-campaign-donors.html?action=click\&pgtype=Article\&state=default\&region=TOP_BANNER\&context=storylines_menu}{Map
  of Donations}
\item
  \href{https://www.nytimes.com/interactive/2020/us/elections/delegate-count-primary-results.html?action=click\&pgtype=Article\&state=default\&region=TOP_BANNER\&context=storylines_menu}{Delegate
  Count}
\item
  \href{https://www.nytimes.com/interactive/2019/us/politics/2020-presidential-candidates.html?action=click\&pgtype=Article\&state=default\&region=TOP_BANNER\&context=storylines_menu}{The
  Candidates}
\item
  \href{https://www.nytimes.com/newsletters/politics?action=click\&pgtype=Article\&state=default\&region=TOP_BANNER\&context=storylines_menu}{Politics
  Newsletter}
\end{itemize}

Advertisement

\protect\hyperlink{after-top}{Continue reading the main story}

Supported by

\protect\hyperlink{after-sponsor}{Continue reading the main story}

\hypertarget{on-trumps-to-do-list-take-back-the-suburbs-court-black-voters-expand-the-electoral-map-win}{%
\section{On Trump's To-Do List: Take Back The Suburbs. Court Black
Voters. Expand the Electoral Map.
Win.}\label{on-trumps-to-do-list-take-back-the-suburbs-court-black-voters-expand-the-electoral-map-win}}

With impeachment behind the president, his re-election campaign wants to
address political weaknesses exacerbated by his policies and behavior.

\includegraphics{https://static01.nyt.com/images/2020/02/07/us/politics/07trump-campaign1/merlin_168538518_b9b031e0-4d02-4eaf-a0a2-73f3fdce92c7-articleLarge.jpg?quality=75\&auto=webp\&disable=upscale}

\href{https://www.nytimes.com/by/maggie-haberman}{\includegraphics{https://static01.nyt.com/images/2018/07/12/multimedia/author-maggie-haberman/author-maggie-haberman-thumbLarge.png}}\href{https://www.nytimes.com/by/annie-karni}{\includegraphics{https://static01.nyt.com/images/2019/02/05/multimedia/author-annie-karni/author-annie-karni-thumbLarge.png}}\href{https://www.nytimes.com/by/jonathan-martin}{\includegraphics{https://static01.nyt.com/images/2018/11/06/multimedia/author-jonathan-martin/author-jonathan-martin-thumbLarge.png}}

By \href{https://www.nytimes.com/by/maggie-haberman}{Maggie Haberman},
\href{https://www.nytimes.com/by/annie-karni}{Annie Karni} and
\href{https://www.nytimes.com/by/jonathan-martin}{Jonathan Martin}

\begin{itemize}
\item
  Published Feb. 8, 2020Updated Feb. 11, 2020
\item
  \begin{itemize}
  \item
  \item
  \item
  \item
  \item
  \item
  \end{itemize}
\end{itemize}

WASHINGTON --- Buoyed by his impeachment acquittal and the muddled
Democratic primary race, President Trump and his campaign are turning to
address his re-election bid's greatest weaknesses with an aggressive,
well-funded but uncertain effort to win back suburban voters turned off
by his policies and behavior.

His campaign is aiming to regain these voters in battleground states
like Pennsylvania and Michigan, after
\href{https://www.nytimes.com/2018/11/01/us/politics/republicans-trump-house.html}{losing
many of them} to Democrats in the 2018 midterms. Advisers hope to expand
the electoral map for November by winning moderate-leaning states like
Minnesota and New Hampshire. And the White House is gearing up to help
with policy issues directed at swing states, such as the new trade deal
with Mexico and Canada and paid family leave for federal workers.

Trump campaign officials are also stockpiling cash to help with these
efforts, with \$200 million in the bank now and fund-raising continuing
at a brisk pace. They have put up television ads relatively early in the
race, allocating \$6 million for the final three months of 2019 to
highlight a booming economy and the low unemployment numbers.

Among the goals is trying to appeal to
\href{https://www.nytimes.com/2020/02/04/us/politics/trump-super-bowl-ad.html}{black
voters} and suburban and upper-income white voters with ads such as
\href{https://www.nytimes.com/2018/06/06/us/politics/trump-alice-johnson-sentence-commuted-kim-kardashian-west.html}{a
spot} focusing on criminal justice reform that first aired during the
Super Bowl and is continuing on cable channels with large female
audiences, like Bravo and Lifetime.

Yet Mr. Trump's messaging, like so much else about his approach to
politics, is contradictory. For all the focus on appealing to moderates,
the campaign is also engaging the president's hard-core supporters with
\href{https://www.nytimes.com/2019/10/20/us/elections/trump-campaign-ads-democrats.html}{Facebook
ads} warning of the danger of undocumented ``aliens'' and their
``invasion'' of the U.S., and decrying ``the impeachment hoax,'' while
also promoting polarizing policies like curtailing immigration.

Those inflammatory, targeted ads are ones that suburban voters may never
see, a reflection of the campaign's broad strategy: Keep his
conservative base energized and chip away at his problems in the suburbs
and communities of color.

The challenge facing Mr. Trump's advisers remains the same as it has
been since 2017: The president is among the most deeply divisive leaders
in the nation's history, whose conduct has helped accelerate a
\href{https://www.nytimes.com/2019/11/06/us/politics/kentucky-governor-virginia-election.html}{realignment
of moderate suburban voters} toward Democrats. These voters have been
the cornerstone of Democrats' electoral revival since 2016, helping them
flip governorships and propelling their capture of the House.

Mr. Trump cannot win a second term without winning back suburban voters
and independents in a handful of states he carried in 2016. But he is
highly averse to staying on script and delivering a consistent message
aimed at moderate voters rather than his hard-core admirers, or ** his
own need to get things off his chest.

Mr. Trump's advisers argue that the suburban voters who eschewed
Republicans in the 2018 midterms will vote differently when the
president's name is on the ballot. And they are lacing the strong
economy into much of their messaging and policy plans: Mr. Trump himself
sees the economy as his calling card and is monitoring fluctuations in
stock market closely, and his team thinks the economy is one of their
best selling points in the suburbs.

``Suburban women is where he has a challenge,'' said Senator Kevin
Cramer, Republican from North Dakota.

\hypertarget{latest-updates-2020-election}{%
\section{\texorpdfstring{\href{https://www.nytimes.com/2020/07/31/us/elections/biden-vs-trump.html?action=click\&pgtype=Article\&state=default\&region=MAIN_CONTENT_1\&context=storylines_live_updates}{Latest
Updates: 2020
Election}}{Latest Updates: 2020 Election}}\label{latest-updates-2020-election}}

Updated 2020-08-01T01:26:45.732Z

\begin{itemize}
\tightlist
\item
  \href{https://www.nytimes.com/2020/07/31/us/elections/biden-vs-trump.html?action=click\&pgtype=Article\&state=default\&region=MAIN_CONTENT_1\&context=storylines_live_updates\#link-29fdff45}{Kamala
  Harris, a top vice-presidential contender, confronts double
  standards.}
\item
  \href{https://www.nytimes.com/2020/07/31/us/elections/biden-vs-trump.html?action=click\&pgtype=Article\&state=default\&region=MAIN_CONTENT_1\&context=storylines_live_updates\#link-13ec3d9c}{Karen
  Bass and Susan Rice are rising on Biden's vice-presidential
  shortlist.}
\item
  \href{https://www.nytimes.com/2020/07/31/us/elections/biden-vs-trump.html?action=click\&pgtype=Article\&state=default\&region=MAIN_CONTENT_1\&context=storylines_live_updates\#link-49e9a016}{Trump
  says Russian bounties to kill U.S. troops `never took place.'}
\end{itemize}

\href{https://www.nytimes.com/2020/07/31/us/elections/biden-vs-trump.html?action=click\&pgtype=Article\&state=default\&region=MAIN_CONTENT_1\&context=storylines_live_updates}{See
more updates}

``I think the biggest problem that he has with suburban women is the
part that so many in his base like about him,'' Mr. Cramer said. ``His
rhetoric, his punching down at his opponents. It's so different than
anything they've seen.''

\includegraphics{https://static01.nyt.com/images/2020/02/07/us/politics/07trump-campaign3/07trump-campaign3-articleLarge.jpg?quality=75\&auto=webp\&disable=upscale}

Scott Reed, the top political adviser to the U.S. Chamber of Commerce,
nodded to the fleeting nature of Trump-era politics as he assessed the
electoral landscape for the president.

``Politics in Trumpville are great right now, but these days, a week
feels like three months and we have a long way to go,'' Mr. Reed said.

Republican National Committee officials are tracking the suburban
problem. In 2016, about 100,000 Michigan residents who voted in state
legislative races left the box for president empty. Many of those voters
were men in the suburbs, R.N.C. officials said, and were people who
didn't believe Mr. Trump was truly a conservative, but who have come
back after seeing him deliver on conservative judicial appointments and
a tax-cut bill.

But suburban women remain difficult to sway, Trump advisers acknowledge.
Some messages have moved the dial, if only temporarily: When Mr. Trump
talks about Democrats wanting to provide government health care benefits
to undocumented immigrants, for instance, Republican officials have seen
an uptick of support in their own surveys of the suburbs of
Pennsylvania. When Mr. Trump paints the entire Democratic field,
falsely, as supporting ending private health insurance, his advisers see
room for him to grow. But they admit that it's a difficult line to walk.

The G.O.P. strategy ultimately depends on who his Democratic opponent
turns out to be. And Mr. Trump faces an unknown in Michael R. Bloomberg,
a billionaire former New York City mayor running a general election
strategy, who is spending so much money that Mr. Trump's advisers
acknowledged that he cannot be ignored even if Mr. Bloomberg loses the
Democratic nomination.

\href{https://www.nytimes.com/interactive/2019/us/politics/2020-presidential-candidates.html}{}

\includegraphics{https://static01.nyt.com/images/2019/01/20/us/2020-presidential-candidates-promo-1548014688187/2020-presidential-candidates-promo-1548014688187-articleLarge-v68.png}

\hypertarget{whos-running-for-president-in-2020}{%
\subsection{Who's Running for President in
2020?}\label{whos-running-for-president-in-2020}}

The field of Democratic presidential candidates has been historically
large, but all have dropped out except Joe Biden, the presumptive
Democratic nominee to challenge President Trump.

With
\href{https://www.nytimes.com/2020/02/06/us/politics/democratic-iowa-caucuses.html}{the
Democrats enmeshed} in the start of their primary season, Mr. Trump is
beginning his own new phase: He has reasons to feel reassured about his
prospects as he turns more fully to his re-election effort, and the
apparatus of the White House and the Republican Party are more able to
focus on winning him a second term.

Mr. Trump's approval ratings have inched up and he's now around where
the last three incumbent presidents were at the start of their own,
successful, re-elections. And
\href{https://www.nytimes.com/2020/02/07/upshot/election-year-economy-trump.html}{the
economy} shows no signs of slowing.

``The White House and the campaign should focus 100 percent on the
economic growth and opportunity society Trump is creating for America,''
Mr. Reed said, somewhat hopefully.

But greater confidence and a freer hand can lead Mr. Trump to take
risks: His phone call with the Ukrainian president on July 25, 2019,
which ultimately helped lead to his impeachment in the House, came after
he had seen the end of the two-year investigation by the special
counsel, Robert S. Mueller III. Just this past week, Mr. Trump fired
from the White House two witnesses and an ambassador who testified in
the House impeachment inquiry, including Lt. Col. Alexander Vindman, a
decorated war veteran, prompting outrage from Democrats and private
concern among some Republican lawmakers. On Saturday, he tweeted that
Colonel Vindman had earned his dismissal.

As Mr. Trump has repeatedly shown, he can show a measure of discipline
in one moment --- like his teleprompter-ready speech at the State of the
Union that was sprinkled with appeals to different demographic groups
--- and then do or say something that alienates swing voters.

His 62-minute stemwinder of retribution in the East Room of the White
House the day after the acquittal was the type of ventilating
performance Mr. Trump had been craving, but which some advisers
acknowledge undermines the carefully-crafted efforts at broadening his
appeal.

``Many people are evaluating the president based on his conduct and
behavior in office rather than the state of the economy,'' said Whit
Ayres, a longtime Republican pollster. ``It's his conduct and behavior
in office that have kept a foot on his job approval rating. Any other
president would be in the upper 50s or even low 60s with this economy.''

Image

Mr. Trump's approval ratings have inched up and he's now around where
the last three incumbent presidents were at the start of their own,
successful, re-elections. And the roaring economy shows no signs of
slowing.Credit...Erin Schaff/The New York Times

Most of the president's aides concede that his base of supporters is not
enough to re-elect him, and that he must attract the voters who were
repelled by his behavior and voted against Republicans in the 2018
midterms --- particularly upscale whites, suburban women and
self-described independent voters who polls repeatedly show think the
president is racist, or has a troubling temperament, or both.

To that end, the president's campaign aired a Super Bowl ad featuring
\href{https://www.nytimes.com/2020/02/06/us/politics/alice-johnson-trump-super-bowl-ad.html}{Alice
Johnson}, a black woman convicted on charges related to drug trafficking
whose sentence the president commuted. The president also awarded an
``Opportunity Zone'' scholarship to a young African-American girl during
his State of the Union address, and tailored other moments during the
speech to appeal to members of the military.

Trump advisers are focused not just on the three states that elected Mr.
Trump in 2016 --- Wisconsin, Michigan and Pennsylvania --- but also the
forever battleground of Florida, and battleground states with
competitive Senate races that could help the Democratic nominee in
Georgia, Arizona and North Carolina.

The campaign also sees opportunities for pickups in
\href{https://www.nytimes.com/live/2020/new-hampshire-primary-02-11}{New
Hampshire} and especially in Minnesota, states that have voted for
Democrats in recent presidential races but where the margins were close
in 2016. But while the campaign manager Brad Parscale has insisted New
Mexico is within reach, other Trump advisers say there's been little
movement, in part because of the president's disinterest in taking the
day trips he favors to the western part of the country.

In an interview, Ronna McDaniel, the chairwoman of the R.N.C., said they
have the resources to appeal to multiple groups of voters. ``That gives
us an advantage to focus on the rural vote that we need to turn out, but
then also go after places where we've lost voters to bring them back
in,'' she said. And Tim Murtaugh, a campaign spokesman, said they had
always planned to woo various demographics, ``regardless of what
Democrats in Congress were trying to do to him.''

The administration is pulling out the policy stops. Vice President Mike
Pence has recently made stops and bus tours in Wisconsin and
Pennsylvania, highlighting Trump administration efforts like the
``school choice'' initiative to help low-income students enter private
schools.

On Thursday, Mr. Trump tweeted that he was looking to move away from a
proposal pushed by his former energy secretary, Rick Perry: storing
nuclear waste in Nevada's Yucca Mountain, an effort his two top
political advisers, Bill Stepien and Justin Clark, opposed for years.
And officials are expected to hold events in the Midwest highlighting
provisions aimed at helping domestic automakers that were included in
the U.S.M.C.A. trade deal.

``We've been chopping wood for a while, and it feels like everyone else
is seeing what we've been seeing for a long time,'' said Jared Kushner,
the president's son-in-law who is overseeing his campaign. ``Everyone
else has been distracted, but it's not like we invented these policies
for the State of the Union.''

What's unclear, and what could prove decisive, is whether the country is
exhausted by Mr. Trump and is ready for a so-called return to normalcy,
or if voters have grown inured to his eruptions and have effectively
priced in his behavior.

A key factor will be the candidate the Democrats eventually nominate.
Interviews with more than a dozen Republican strategists, lawmakers and
state chairs reveal a consensus that Senator Bernie Sanders would be the
easiest Democrat for them to beat because they believe his avowed
socialism would help them reclaim suburbanites and better frame the
election as a choice.

``It's easy to call him a socialist because he admits it,'' said Tim
Pawlenty, the former Minnesota governor. ``At least Warren tries to deny
it.''

Mr. Sanders's aides, of course, see it very differently and believe that
they would tear up Mr. Trump's 2016 electoral map by reclaiming
working-class white voters in states like Michigan and Wisconsin,
something some Trump advisers agree with. And Trump advisers have been
caught by surprise by the success of Pete Buttigieg, the former mayor of
South Bend, Ind.

``We don't have a Democratic opponent yet,'' said Mr. Cramer. ``It's
always harder to run against the unnamed opponent. Once you have the
opponent, you get to draw the distinctions.''

\hypertarget{our-2020-election-guide}{%
\section{Our 2020 Election Guide}\label{our-2020-election-guide}}

Updated July 31, 2020

\begin{itemize}
\item
  \begin{center}\rule{0.5\linewidth}{\linethickness}\end{center}

  \hypertarget{the-latest}{%
  \subsection{The Latest}\label{the-latest}}

  \begin{itemize}
  \tightlist
  \item
    President Trump's assault on the Postal Service is intersecting with
    his attacks on mail-in voting.
    \href{https://www.nytimes.com/2020/07/31/us/politics/trump-usps-mail-delays.html?action=click\&pgtype=Article\&state=default\&region=BELOW_MAIN_CONTENT\&context=storylines_guide}{Voting
    rights groups say it is a recipe for disaster.}
  \end{itemize}
\item
  \begin{center}\rule{0.5\linewidth}{\linethickness}\end{center}

  \hypertarget{bidens-vp-search}{%
  \subsection{Biden's V.P. Search}\label{bidens-vp-search}}

  \begin{itemize}
  \tightlist
  \item
    \href{https://www.nytimes.com/article/biden-vice-president-2020.html?action=click\&pgtype=Article\&state=default\&region=BELOW_MAIN_CONTENT\&context=storylines_guide}{Here
    are 13 women} who have been under consideration to be Joe Biden's
    running mate, and why each might be chosen --- and might not be.
  \end{itemize}
\item
  \begin{center}\rule{0.5\linewidth}{\linethickness}\end{center}

  \hypertarget{keep-up-with-our-coverage}{%
  \subsection{Keep Up With Our
  Coverage}\label{keep-up-with-our-coverage}}

  \begin{itemize}
  \tightlist
  \item
    Get an
    \href{https://www.nytimes.com/newsletters/politics?action=click\&pgtype=Article\&state=default\&region=BELOW_MAIN_CONTENT\&context=storylines_guide}{email}
    recapping the day's news
  \end{itemize}

  \begin{itemize}
  \tightlist
  \item
    Download our mobile app on
    \href{https://apps.apple.com/us/app/nytimes/id284862083?ls=1\&mat_click_id=5c79ae7455014fd1bd66b5610c05b8f2-20191112-16948\&referrer=mat_click_id\%3D5c79ae7455014fd1bd66b5610c05b8f2-20191112-16948\%26link_click_id\%3D722930677036718082}{iOS}
    and
    \href{http://a.localytics.com/android?id=com.nytimes.android\&referrer=utm_source\%3Dother_nyt_mobile_web\%26utm_medium\%3DWeb\%2520page\%26utm_term\%3DGeneral\%2520Mobile\%2520Page\%26utm_campaign\%3DNYT\%2520Mobile\%2520General\%2520Page}{Android}
    and turn on Breaking News and Politics alerts
  \end{itemize}
\end{itemize}

Advertisement

\protect\hyperlink{after-bottom}{Continue reading the main story}

\hypertarget{site-index}{%
\subsection{Site Index}\label{site-index}}

\hypertarget{site-information-navigation}{%
\subsection{Site Information
Navigation}\label{site-information-navigation}}

\begin{itemize}
\tightlist
\item
  \href{https://help.nytimes.com/hc/en-us/articles/115014792127-Copyright-notice}{©~2020~The
  New York Times Company}
\end{itemize}

\begin{itemize}
\tightlist
\item
  \href{https://www.nytco.com/}{NYTCo}
\item
  \href{https://help.nytimes.com/hc/en-us/articles/115015385887-Contact-Us}{Contact
  Us}
\item
  \href{https://www.nytco.com/careers/}{Work with us}
\item
  \href{https://nytmediakit.com/}{Advertise}
\item
  \href{http://www.tbrandstudio.com/}{T Brand Studio}
\item
  \href{https://www.nytimes.com/privacy/cookie-policy\#how-do-i-manage-trackers}{Your
  Ad Choices}
\item
  \href{https://www.nytimes.com/privacy}{Privacy}
\item
  \href{https://help.nytimes.com/hc/en-us/articles/115014893428-Terms-of-service}{Terms
  of Service}
\item
  \href{https://help.nytimes.com/hc/en-us/articles/115014893968-Terms-of-sale}{Terms
  of Sale}
\item
  \href{https://spiderbites.nytimes.com}{Site Map}
\item
  \href{https://help.nytimes.com/hc/en-us}{Help}
\item
  \href{https://www.nytimes.com/subscription?campaignId=37WXW}{Subscriptions}
\end{itemize}
