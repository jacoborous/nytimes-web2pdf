Sections

SEARCH

\protect\hyperlink{site-content}{Skip to
content}\protect\hyperlink{site-index}{Skip to site index}

\href{https://www.nytimes.com/section/books/review}{Book Review}

\href{https://myaccount.nytimes.com/auth/login?response_type=cookie\&client_id=vi}{}

\href{https://www.nytimes.com/section/todayspaper}{Today's Paper}

\href{/section/books/review}{Book Review}\textbar{}In 1980s Glasgow, a
World of Pain Made Bearable by Love

\url{https://nyti.ms/3bvZ4gH}

\begin{itemize}
\item
\item
\item
\item
\item
\end{itemize}

Advertisement

\protect\hyperlink{after-top}{Continue reading the main story}

Supported by

\protect\hyperlink{after-sponsor}{Continue reading the main story}

Fiction

\hypertarget{in-1980s-glasgow-a-world-of-pain-made-bearable-by-love}{%
\section{In 1980s Glasgow, a World of Pain Made Bearable by
Love}\label{in-1980s-glasgow-a-world-of-pain-made-bearable-by-love}}

\includegraphics{https://static01.nyt.com/images/2020/01/29/books/Cohen1/Cohen1-articleLarge.jpg?quality=75\&auto=webp\&disable=upscale}

Buy Book ▾

\begin{itemize}
\tightlist
\item
  \href{https://www.amazon.com/gp/search?index=books\&tag=NYTBSREV-20\&field-keywords=Shuggie+Bain+Douglas+Stuart}{Amazon}
\item
  \href{https://du-gae-books-dot-nyt-du-prd.appspot.com/buy?title=Shuggie+Bain\&author=Douglas+Stuart}{Apple
  Books}
\item
  \href{https://www.anrdoezrs.net/click-7990613-11819508?url=https\%3A\%2F\%2Fwww.barnesandnoble.com\%2Fw\%2F\%3Fean\%3D9780802148049}{Barnes
  and Noble}
\item
  \href{https://www.anrdoezrs.net/click-7990613-35140?url=https\%3A\%2F\%2Fwww.booksamillion.com\%2Fp\%2FShuggie\%2BBain\%2FDouglas\%2BStuart\%2F9780802148049}{Books-A-Million}
\item
  \href{https://bookshop.org/a/3546/9780802148049}{Bookshop}
\item
  \href{https://www.indiebound.org/book/9780802148049?aff=NYT}{Indiebound}
\end{itemize}

When you purchase an independently reviewed book through our site, we
earn an affiliate commission.

By Leah Hager Cohen

\begin{itemize}
\item
  Feb. 11, 2020
\item
  \begin{itemize}
  \item
  \item
  \item
  \item
  \item
  \end{itemize}
\end{itemize}

\textbf{SHUGGIE BAIN}

By Douglas Stuart

The body --- especially the body in pain --- blazes on the pages of
``Shuggie Bain.'' Hair is ripped from heads, people are dragged up the
stairs and down the street, faces and groins are bloodied and bruised,
and all with a nearly quotidian inevitability. The most common form of
suffering in this novel is that which characters inflict on themselves:
poisoning themselves with drink, putting their heads in the oven,
setting the bedroom on fire, egging on aggressors, rebuffing love,
refusing help.

Even the nonviolent bits fairly pulsate with descriptions of body parts
and fluids: mustaches coated with cream or ``congealing pink sauce,''
thighs gone ``tartan blue'' with cold, children spitting through the
letter slot on a front door, ``big long gobbets of sugary phlegm that
stuck on the metal flap and slid slowly down the inside of the wood.''

This is the world of Shuggie Bain, a little boy growing up in Glasgow in
the 1980s. And this is the world of Agnes Bain, his glamorous,
calamitous mother, drinking herself ever so slowly to death. The wonder
is how crazily, improbably alive it all is: this world of slag heaps and
council houses, of unemployed miners and women stuck at home with their
``weans,'' forced to supplement weekly benefit payments by prying open
the electric meter and reclaiming the coins therein. Douglas Stuart
writes in a sense-drenched Glaswegian prose so studded with slang
(``papped,'' ``boak,'' ``laldy,'' ``smirr'') and phonetically rendered
dialogue (``Wit are the pair of ye stauning there all glaikit fur?'')
that the language itself adds up to another layer of physicality, a
rhythmic reel coursing through the reader's blood.

At the center of all this, little Shuggie is just beginning to perceive,
and trying to puzzle out why, nearly everyone he meets considers him
``no right.'' He is different from other boys. There's a gorgeous scene
early on where 5-year-old Shuggie is playing with dolls. Only they're
not dolls, they're cans of Tennent's beer that have ``half-naked
beauties photographed on the side.'' He strokes ``their tinny hair'' and
makes them ``talk to each other in imagined conversations.'' When his
father catches him at it he's proud, misinterpreting the little boy's
play for precocious lust, but his mother looks on sadly, ``knowing what
was really going on.''

Soon enough, Shuggie's father abandons his wife and children, Agnes
sinks deeper into the drink, Shuggie's older siblings find ways to
escape, and Shuggie is left alone to absorb his own pain and assuage his
mother's. We follow them, the ``wee poofter'' and his ``hoor'' of a
mammy, through roughly a decade of heartbreak and squalor, a more or
less Jobian arc of things going from bad to worse to excruciating, and
the book would be just about unbearable were it not for the author's
astonishing capacity for love.

He's lovely, Douglas Stuart, fierce and loving and lovely. He shows us
lots of monstrous behavior, but not a single monster --- only damage. If
he has a sharp eye for brokenness, he is even keener on the
inextinguishable flicker of love that remains. The book is long, more
than 400 pages, but its length seems crucial to its overall effect. Like
Agnes, we're all doomed to our patterns. How often we repeat the same
disastrous mistakes, make the same wrong turn again and again. And yet,
like Shuggie, how often we rise, against all odds, to stumble forward
once more. The book leaves us gutted and marveling: Life may be short,
but it takes forever.

Advertisement

\protect\hyperlink{after-bottom}{Continue reading the main story}

\hypertarget{site-index}{%
\subsection{Site Index}\label{site-index}}

\hypertarget{site-information-navigation}{%
\subsection{Site Information
Navigation}\label{site-information-navigation}}

\begin{itemize}
\tightlist
\item
  \href{https://help.nytimes.com/hc/en-us/articles/115014792127-Copyright-notice}{©~2020~The
  New York Times Company}
\end{itemize}

\begin{itemize}
\tightlist
\item
  \href{https://www.nytco.com/}{NYTCo}
\item
  \href{https://help.nytimes.com/hc/en-us/articles/115015385887-Contact-Us}{Contact
  Us}
\item
  \href{https://www.nytco.com/careers/}{Work with us}
\item
  \href{https://nytmediakit.com/}{Advertise}
\item
  \href{http://www.tbrandstudio.com/}{T Brand Studio}
\item
  \href{https://www.nytimes.com/privacy/cookie-policy\#how-do-i-manage-trackers}{Your
  Ad Choices}
\item
  \href{https://www.nytimes.com/privacy}{Privacy}
\item
  \href{https://help.nytimes.com/hc/en-us/articles/115014893428-Terms-of-service}{Terms
  of Service}
\item
  \href{https://help.nytimes.com/hc/en-us/articles/115014893968-Terms-of-sale}{Terms
  of Sale}
\item
  \href{https://spiderbites.nytimes.com}{Site Map}
\item
  \href{https://help.nytimes.com/hc/en-us}{Help}
\item
  \href{https://www.nytimes.com/subscription?campaignId=37WXW}{Subscriptions}
\end{itemize}
