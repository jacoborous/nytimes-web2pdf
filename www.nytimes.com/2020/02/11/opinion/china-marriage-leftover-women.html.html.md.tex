Sections

SEARCH

\protect\hyperlink{site-content}{Skip to
content}\protect\hyperlink{site-index}{Skip to site index}

\href{https://myaccount.nytimes.com/auth/login?response_type=cookie\&client_id=vi}{}

\href{https://www.nytimes.com/section/todayspaper}{Today's Paper}

\href{/section/opinion}{Opinion}\textbar{}Where Being a Single Woman Is
Not OK

\href{https://nyti.ms/2uER10D}{https://nyti.ms/2uER10D}

\begin{itemize}
\item
\item
\item
\item
\item
\item
\end{itemize}

Advertisement

\protect\hyperlink{after-top}{Continue reading the main story}

Supported by

\protect\hyperlink{after-sponsor}{Continue reading the main story}

\href{/section/opinion}{Opinion}

Op-Docs

\hypertarget{where-being-a-single-woman-is-not-ok}{%
\section{Where Being a Single Woman Is Not
OK}\label{where-being-a-single-woman-is-not-ok}}

Dating is hard. A government campaign to get you married is worse.

\includegraphics{https://static01.nyt.com/images/2020/02/06/opinion/single-in-china-img/single-in-china-img-videoSixteenByNineJumbo1600.jpg}

By Shosh Shlam and Hilla Medalia

Feb. 11, 2020

``Sheng nu'' ** (``leftover women'') is a term used to describe single
women who are 27 or older in China. Most of these women live in cities
and lead rewarding professional lives. The term was coined in 2007 by a
government organization responsible for the protection and promotion of
women's rights and policies. That same year, the Ministry of Education
added ``sheng nu'' ** to the official lexicon.

In this Op-Doc, based on the Independent Lens feature documentary
``\href{http://www.pbs.org/independentlens/leftover-women/}{Leftover
Women},'' we follow one of those women --- Qiu Huamei, contending with
the stigma and social pressure forcing her to go on a grueling quest in
search of a husband. She grew up in a small village five hours south of
Beijing and is the second youngest of five sisters. Ms. Qiu is a
successful lawyer, fluent in English and opinionated --- but those
qualities do not outweigh one key flaw: She is not married.

In recent years, the Chinese government has been waging an aggressive
campaign to pressure women into marriage. Single women are caricatured
in news reports, editorials and social media. The orchestrated campaign
is a byproduct of China's one-child policy, which created a great gender
imbalance in the population.

Ms. Qiu does all she can to comply with expectations and find a partner.
But the search sometimes feels incompatible with the life she envisions
for herself. When she goes on dates, she hears again and again how a
woman's place is at home. Her intellectual and professional achievements
are irrelevant. She is measured only by traditional values. And so with
every year that passes, her value in the marriage market diminishes.

\emph{The Times is committed to publishing}
\href{https://www.nytimes.com/2019/01/31/opinion/letters/letters-to-editor-new-york-times-women.html}{\emph{a
diversity of letters}} \emph{to the editor. We'd like to hear what you
think about this or any of our articles. Here are some}
\href{https://help.nytimes.com/hc/en-us/articles/115014925288-How-to-submit-a-letter-to-the-editor}{\emph{tips}}\emph{.
And here's our email:}
\href{mailto:letters@nytimes.com}{\emph{letters@nytimes.com}}\emph{.}

Shosh Shlam is a filmmaker whose documentaries have been shown in
theaters and on ARTE, ZDF, PBS and the BBC. Hilla Medalia is a Peabody
Award-winning filmmaker and founder of Medalia Productions, whose films
have been commissioned by HBO, Arte and the BBC and have been shown at
Cannes, Berlinale and Sundance. Their previous Op-Doc is
\href{https://www.nytimes.com/2014/01/20/opinion/chinas-web-junkies.html}{``China's
Web Junkies.''}

\begin{center}\rule{0.5\linewidth}{\linethickness}\end{center}

\href{https://www.nytimes.com/column/op-docs}{Op-Docs} is a forum for
short, opinionated documentaries by independent filmmakers.
\href{http://www.nytimes.com/ref/opinion/about-op-docs.html}{Learn more}
about Op-Docs and
\href{http://www.nytimes.com/content/help/site/editorial/op-video/video.html?ref=opinion}{how
to submit} to the series. Follow The New York Times Opinion section on
\href{https://www.facebook.com/nytopinion}{Facebook},
\href{http://twitter.com/NYTOpinion}{Twitter (@NYTopinion)} and
\href{https://www.instagram.com/nytopinion/}{Instagram}.

Advertisement

\protect\hyperlink{after-bottom}{Continue reading the main story}

\hypertarget{site-index}{%
\subsection{Site Index}\label{site-index}}

\hypertarget{site-information-navigation}{%
\subsection{Site Information
Navigation}\label{site-information-navigation}}

\begin{itemize}
\tightlist
\item
  \href{https://help.nytimes.com/hc/en-us/articles/115014792127-Copyright-notice}{©~2020~The
  New York Times Company}
\end{itemize}

\begin{itemize}
\tightlist
\item
  \href{https://www.nytco.com/}{NYTCo}
\item
  \href{https://help.nytimes.com/hc/en-us/articles/115015385887-Contact-Us}{Contact
  Us}
\item
  \href{https://www.nytco.com/careers/}{Work with us}
\item
  \href{https://nytmediakit.com/}{Advertise}
\item
  \href{http://www.tbrandstudio.com/}{T Brand Studio}
\item
  \href{https://www.nytimes.com/privacy/cookie-policy\#how-do-i-manage-trackers}{Your
  Ad Choices}
\item
  \href{https://www.nytimes.com/privacy}{Privacy}
\item
  \href{https://help.nytimes.com/hc/en-us/articles/115014893428-Terms-of-service}{Terms
  of Service}
\item
  \href{https://help.nytimes.com/hc/en-us/articles/115014893968-Terms-of-sale}{Terms
  of Sale}
\item
  \href{https://spiderbites.nytimes.com}{Site Map}
\item
  \href{https://help.nytimes.com/hc/en-us}{Help}
\item
  \href{https://www.nytimes.com/subscription?campaignId=37WXW}{Subscriptions}
\end{itemize}
