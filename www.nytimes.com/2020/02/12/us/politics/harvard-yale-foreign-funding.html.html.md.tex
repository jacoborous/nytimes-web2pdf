Sections

SEARCH

\protect\hyperlink{site-content}{Skip to
content}\protect\hyperlink{site-index}{Skip to site index}

\href{https://www.nytimes.com/section/politics}{Politics}

\href{https://myaccount.nytimes.com/auth/login?response_type=cookie\&client_id=vi}{}

\href{https://www.nytimes.com/section/todayspaper}{Today's Paper}

\href{/section/politics}{Politics}\textbar{}Harvard and Yale Ensnared in
Education Dept. Crackdown on Foreign Funding

\url{https://nyti.ms/39yrfKn}

\begin{itemize}
\item
\item
\item
\item
\item
\end{itemize}

Advertisement

\protect\hyperlink{after-top}{Continue reading the main story}

Supported by

\protect\hyperlink{after-sponsor}{Continue reading the main story}

\hypertarget{harvard-and-yale-ensnared-in-education-dept-crackdown-on-foreign-funding}{%
\section{Harvard and Yale Ensnared in Education Dept. Crackdown on
Foreign
Funding}\label{harvard-and-yale-ensnared-in-education-dept-crackdown-on-foreign-funding}}

The department told the Ivy League universities to hand over records on
millions of dollars in gifts, grants and contracts from foreign
countries, including China, Iran and Russia.

\includegraphics{https://static01.nyt.com/images/2020/02/12/us/12ivys/merlin_168073176_d365e8eb-8c63-4711-8639-c2043bde0879-articleLarge.jpg?quality=75\&auto=webp\&disable=upscale}

\href{https://nytimes.com/by/erica-l-green}{\includegraphics{https://static01.nyt.com/images/2018/06/14/multimedia/author-erica-l-green/author-erica-l-green-thumbLarge-v2.png}}\href{https://www.nytimes.com/by/ellen-barry}{\includegraphics{https://static01.nyt.com/images/2018/10/08/multimedia/author-ellen-barry/author-ellen-barry-thumbLarge.png}}

By \href{https://nytimes.com/by/erica-l-green}{Erica L. Green} and
\href{https://www.nytimes.com/by/ellen-barry}{Ellen Barry}

\begin{itemize}
\item
  Feb. 12, 2020
\item
  \begin{itemize}
  \item
  \item
  \item
  \item
  \item
  \end{itemize}
\end{itemize}

WASHINGTON --- A federal crackdown on universities that fail to disclose
donations and contracts from foreign governments has ensnared Harvard
and Yale,
\href{https://content.govdelivery.com/accounts/USED/bulletins/27b7801}{the
Education Department said} on Wednesday.

In letters to the schools on Tuesday, the department wrote that it was
investigating whether the two Ivy League universities had failed to
report at least \$375 million **** from countries including China, Iran,
Russia, Qatar and Saudi Arabia. The department is seeking extensive
records related to grants, gifts, contracts and overseas programming.

In a
\href{https://www2.ed.gov/policy/highered/leg/harvard-20200211.pdf}{letter
to Harvard}, the department said it was ``aware of information
suggesting Harvard University lacks appropriate institutional
controls,'' and as a result, the university's reports to Washington may
not include or fully reflect ``all reportable gifts'' and contracts
``from or with foreign sources.''

In the case of Yale,
\href{https://www2.ed.gov/policy/highered/leg/yale-20200211.pdf}{officials
wrote} that although the university had ``a considerable presence
abroad, represented by sites in dozens of cities and countries,'' it
appeared to have ``failed to report a single foreign source gift or
contract in 2014, 2015, 2016 and 2017.''

The Education Department zeroed in on records related to two Chinese
telecommunications companies,
\href{https://www.nytimes.com/2020/02/11/us/politics/white-house-huawei-back-door.html}{Huawei}
and
\href{https://www.nytimes.com/2018/06/07/business/what-is-zte.html}{ZTE},
that the Trump administration has labeled security risks or sanctions
violators. The
\href{https://www.nytimes.com/2017/09/13/us/politics/kaspersky-lab-antivirus-federal-government.html}{Russian
computer security firm Kaspersky Lab} has also fallen under suspicion.
The letters also named the Skolkovo Foundation of Russia, the
Iran-linked Alavi Foundation and the Qatar National Research Fund, among
other organizations.

The inquiry was first
\href{https://www.wsj.com/articles/education-department-investigating-harvard-yale-over-foreign-funding-11581539042}{reported
by The Wall Street Journal}.

``This is about transparency,'' Education Secretary Betsy DeVos said in
a statement. ``If colleges and universities are accepting foreign money
and gifts, their students, donors and taxpayers deserve to know how much
and from whom. Moreover, it's what the law requires.''

Jonathan Swain, a Harvard spokesman, confirmed that the university was
informed on Tuesday of the records request.

``We're reviewing it and beginning to start to compile our response to
the Department of Education, which is going to take some time,'' he
said.

Karen Peart, the director of Yale's media relations, also confirmed that
the university had received the request on Tuesday.

``We are reviewing the request and preparing to respond to it,'' she
said in a written statement.

The letter to Harvard appears to have been prompted in part by an
investigation into Charles M. Lieber, the chairman of the university's
chemistry department, who was
\href{https://www.nytimes.com/2020/01/28/us/charles-lieber-harvard.html}{charged
with lying to federal officials} about grants he had received from
China. **** The Education Department request asks for all records
regarding Dr. Lieber's Chinese benefactors, the Thousand Talents
recruitment program and the Wuhan University of Technology.

Dr. Lieber was arrested on Jan. 28 and released early this month on \$1
million bond.

But the inquiry is also part of a broad crackdown that began last summer
and was designed to force more scrutiny on funding for U.S. higher
education institutions from countries that are often at odds with
American policies but eager to tap the country's brightest minds.

The Education Department said Wednesday that since July, its enforcement
efforts have prompted the reporting of about \$6.5 billion in
undisclosed foreign gifts, grants and contracts. Ten schools --- Boston
University, Carnegie Mellon, the University of Chicago, the University
of Colorado at Boulder, Cornell, M.I.T., the University of Pennsylvania,
Texas A\&M, the University of Texas M.D. Anderson Cancer Center and Yale
--- declared approximately \$3.6 billion in previously unreported
foreign gifts.

The department
\href{https://s3.amazonaws.com/public-inspection.federalregister.gov/2019-13904.pdf?utm_source=federalregister.gov\&utm_medium=email\&utm_campaign=pi+subscription+mailing+list}{announced
in June that it was investigating} whether Cornell, Georgetown, Rutgers
and Texas A\&M were fully complying with a federal law that required
colleges to report all gifts and contracts from foreign sources that
exceeded \$250,000. In letters sent to the universities in July,
department officials wrote that they were seeking records dating as far
back as nine years, outlining agreements, communication and financial
transactions with entities and governments in countries such as China,
Qatar, Russia and Saudi Arabia.

The federal government demanded thousands of records that could reveal
millions of dollars in foreign aid for campus operations overseas,
academic research and other cultural and academic partnerships.

The investigations have caused friction between the Education
Department\href{https://www.acenet.edu/Documents/Letter-ED-Associations-Response-on-Section-117.pdf}{and
several higher education groups}, which have urged the department to
clarify the rules around an obscure provision, called Section 117, in
the Higher Education Act. The provision requires colleges to report all
gifts and contracts from foreign sources that exceed \$250,000.

Education Department officials revealed last February in
\href{https://www.hsgac.senate.gov/imo/media/doc/2019-02-28\%20Zais\%20Testimony\%20-\%20PSI.pdf}{congressional
testimony} that fewer than 3 percent of 3,700 higher education
institutions that receive foreign funding reported receiving foreign
gifts or contracts exceeding \$250,000.

``Unfortunately, the more we dig, the more we find that too many are
underreporting or not reporting at all,'' Ms. DeVos said. ``We will
continue to hold colleges and universities accountable and work with
them to ensure their reporting is full, accurate and transparent, as
required by the law.''

Advertisement

\protect\hyperlink{after-bottom}{Continue reading the main story}

\hypertarget{site-index}{%
\subsection{Site Index}\label{site-index}}

\hypertarget{site-information-navigation}{%
\subsection{Site Information
Navigation}\label{site-information-navigation}}

\begin{itemize}
\tightlist
\item
  \href{https://help.nytimes.com/hc/en-us/articles/115014792127-Copyright-notice}{©~2020~The
  New York Times Company}
\end{itemize}

\begin{itemize}
\tightlist
\item
  \href{https://www.nytco.com/}{NYTCo}
\item
  \href{https://help.nytimes.com/hc/en-us/articles/115015385887-Contact-Us}{Contact
  Us}
\item
  \href{https://www.nytco.com/careers/}{Work with us}
\item
  \href{https://nytmediakit.com/}{Advertise}
\item
  \href{http://www.tbrandstudio.com/}{T Brand Studio}
\item
  \href{https://www.nytimes.com/privacy/cookie-policy\#how-do-i-manage-trackers}{Your
  Ad Choices}
\item
  \href{https://www.nytimes.com/privacy}{Privacy}
\item
  \href{https://help.nytimes.com/hc/en-us/articles/115014893428-Terms-of-service}{Terms
  of Service}
\item
  \href{https://help.nytimes.com/hc/en-us/articles/115014893968-Terms-of-sale}{Terms
  of Sale}
\item
  \href{https://spiderbites.nytimes.com}{Site Map}
\item
  \href{https://help.nytimes.com/hc/en-us}{Help}
\item
  \href{https://www.nytimes.com/subscription?campaignId=37WXW}{Subscriptions}
\end{itemize}
