Sections

SEARCH

\protect\hyperlink{site-content}{Skip to
content}\protect\hyperlink{site-index}{Skip to site index}

\href{https://myaccount.nytimes.com/auth/login?response_type=cookie\&client_id=vi}{}

\href{https://www.nytimes.com/section/todayspaper}{Today's Paper}

\href{/section/opinion}{Opinion}\textbar{}The Government Uses `Near
Perfect Surveillance' Data on Americans

\url{https://nyti.ms/2S7IPyG}

\begin{itemize}
\item
\item
\item
\item
\item
\end{itemize}

Advertisement

\protect\hyperlink{after-top}{Continue reading the main story}

\href{/section/opinion}{Opinion}

Supported by

\protect\hyperlink{after-sponsor}{Continue reading the main story}

\hypertarget{the-government-uses-near-perfect-surveillance-data-on-americans}{%
\section{The Government Uses `Near Perfect Surveillance' Data on
Americans}\label{the-government-uses-near-perfect-surveillance-data-on-americans}}

Congressional hearings are urgently needed to address location tracking.

By
\href{https://www.nytimes.com/interactive/opinion/editorialboard.html}{The
Editorial Board}

The editorial board is a group of opinion journalists whose views are
informed by expertise, research, debate and certain longstanding
\href{https://www.nytimes.com/interactive/2018/opinion/editorialboard.html}{values}.
It is separate from the newsroom.

\begin{itemize}
\item
  Feb. 7, 2020
\item
  \begin{itemize}
  \item
  \item
  \item
  \item
  \item
  \end{itemize}
\end{itemize}

\includegraphics{https://static01.nyt.com/images/2020/02/08/opinion/08tracking-web/08tracking-web-articleLarge.png?quality=75\&auto=webp\&disable=upscale}

``When the government tracks the location of a cellphone it achieves
near perfect surveillance, as if it had attached an ankle monitor to the
phone's user,'' wrote John Roberts, the chief justice of the Supreme
Court, in a
\href{https://www.supremecourt.gov/opinions/17pdf/16-402_h315.pdf}{2018
ruling} that prevented the government from obtaining location data from
\href{https://www.nytimes.com/2018/06/22/us/politics/supreme-court-warrants-cell-phone-privacy.html}{cellphone
towers without a warrant}.

``We decline to grant the state unrestricted access to a wireless
carrier's database of physical location information,'' Chief Justice
Roberts
\href{https://www.nytimes.com/2018/06/22/us/politics/supreme-court-warrants-cell-phone-privacy.html}{wrote}
in the decision, Carpenter v. United States.

With that judicial intent in mind, it is alarming to read
\href{https://www.wsj.com/articles/federal-agencies-use-cellphone-location-data-for-immigration-enforcement-11581078600?mod=hp_lead_pos5}{a
new report} in The Wall Street Journal that found the Trump
administration ``has bought access to a commercial database that maps
the movements of millions of cellphones in America and is using it for
immigration and border enforcement.''

The data used by the government comes not from the phone companies but
from a
\href{https://www.nytimes.com/interactive/2019/12/19/opinion/location-tracking-cell-phone.html}{location
data company}, one of many that are quietly and relentlessly collecting
the precise movements of all smartphone-owning Americans through their
phone apps.

Many apps --- weather apps or coupon apps, for instance --- gather and
record location data without users' understanding what the code is up
to. That data can then be sold to third party buyers including,
apparently, the government.

Since that data is available for sale, it seems the government believes
that no court oversight is necessary. ``The federal government has
essentially found a workaround by purchasing location data used by
marketing firms rather than going to court on a case-by-case basis,''
The Journal reported. ``Because location data is available through
numerous commercial ad exchanges, government lawyers have approved the
programs and concluded that the Carpenter ruling doesn't apply.''

A spokesman from Customs and Border Protection defended the practice in
a statement to The Times: ``While C.B.P. is being provided access to
location information, it is important to note that such information does
not include cellular phone tower data, is not ingested in bulk and does
not include the individual user's identity.''

Use of this type of location-tracking data by the government has not
been tested in court. And in the private sector, location data --- and
the multibillion dollar advertising ecosystem that has eagerly embraced
it --- are both opaque and largely unregulated.

Last year, a
\href{https://www.nytimes.com/interactive/2019/12/19/opinion/location-tracking-cell-phone.html}{Times
Opinion investigation} found that claims about the anonymity of location
data are untrue since comprehensive records of time and place easily
identify real people. Consider a commute: Even without a name, how many
phones travel between a specific home and specific office every day?

This week's revelations dredge up many questions about C.B.P.'s
workflow: What precisely does the agency mean when it claims that the
data is not ingested in bulk? Who in the agency gets to look at the data
and for what purposes? Where is it stored? How long is it stored for? If
the government plans to outsource the surveillance state to commercial
entities to bypass Supreme Court rulings, both parties ought to be
questioned under oath about the specifics of their practices.

The use of location data to aid in deportations also demonstrates how
out of date the notion of informed consent has become. When users accept
the terms and conditions for various digital products, not only are they
uninformed about how their data is gathered, they are also consenting to
future uses that they could never predict.

Without oversight, it is inconceivable that tactics turned against
undocumented immigrants won't eventually be turned to the enforcement of
other laws. As the world has seen in the streets of Hong Kong, where
protesters wear masks to avoid a network of government
facial-recognition cameras, once a surveillance technology is widely
deployed in a society it is almost impossible to uproot.

Chief Justice Roberts
\href{https://www.supremecourt.gov/opinions/17pdf/16-402_h315.pdf}{outlined}
those stakes in his Carpenter ruling. ``The retrospective quality of the
data here gives police access to a category of information otherwise
unknowable. In the past, attempts to reconstruct a person's movements
were limited by a dearth of records and the frailties of recollection.
With access to {[}cellphone location data{]}, the Government can now
travel back in time to retrace a person's whereabouts, subject only to
the retention polices of the wireless carriers, which currently maintain
records for up to five years. Critically, because location information
is continually logged for all of the 400 million devices in the United
States --- not just those belonging to persons who might happen to come
under investigation --- this newfound tracking capacity runs against
everyone.''

The courts are a ponderous and imperfect venue for protecting Fourth
Amendment rights in an age of rapid technological advancement. Exhibit A
is the notion that the Carpenter ruling applies only to location data
captured by cellphone towers and not to location data streamed from
smartphone apps, which can produce nearly identical troves of
information.

For far, far too long, lawmakers have neglected their critical role in
overseeing how these technologies are used. After all, concern about
location tracking is bipartisan, as Republican and Democratic lawmakers
\href{https://www.nytimes.com/interactive/2019/12/19/opinion/location-tracking-cell-phone.html}{told
Times Opinion} last year.

``I am deeply concerned by reports that the Trump administration has
been secretly collecting cellphone data --- without warrants --- to
track the location of millions of people across the United States to
target individuals for deportation,'' Representative Carolyn Maloney,
who leads the Oversight and Reform Committee, told The Times. ``Such
Orwellian government surveillance threatens the privacy of every
American. The federal government should not have the unfettered ability
to track us in our homes, at work, at the doctor or at church. The
Oversight Committee plans to fully investigate this issue to ensure that
Americans' privacy is protected.''

Surely, Congress has time to hold hearings about a matter of urgent
concern to everyone who owns a smartphone or cares about the government
using the most invasive corporate surveillance system ever devised
against its own people.

\emph{Like other media companies, The Times collects data on its
visitors when they read stories like this one. For more detail please
see}
\href{https://help.nytimes.com/hc/en-us/articles/115014892108-Privacy-policy?module=inline}{\emph{our
privacy policy}} \emph{and}
\href{https://www.nytimes.com/2019/04/10/opinion/sulzberger-new-york-times-privacy.html?rref=collection\%2Fspotlightcollection\%2Fprivacy-project-does-privacy-matter\&action=click\&contentCollection=opinion\&region=stream\&module=stream_unit\&version=latest\&contentPlacement=8\&pgtype=collection}{\emph{our
publisher's description}} \emph{of The Times's practices and continued
steps to increase transparency and protections.}

\emph{Follow}
\href{https://twitter.com/privacyproject}{\emph{@privacyproject}}
\emph{on Twitter and The New York Times Opinion Section on}
\href{https://www.facebook.com/nytopinion}{\emph{Facebook}}
\emph{and}\href{https://www.instagram.com/nytopinion/}{\emph{Instagram}}\emph{.}

\hypertarget{glossary-replacer}{%
\subsection{glossary replacer}\label{glossary-replacer}}

Advertisement

\protect\hyperlink{after-bottom}{Continue reading the main story}

\hypertarget{site-index}{%
\subsection{Site Index}\label{site-index}}

\hypertarget{site-information-navigation}{%
\subsection{Site Information
Navigation}\label{site-information-navigation}}

\begin{itemize}
\tightlist
\item
  \href{https://help.nytimes.com/hc/en-us/articles/115014792127-Copyright-notice}{©~2020~The
  New York Times Company}
\end{itemize}

\begin{itemize}
\tightlist
\item
  \href{https://www.nytco.com/}{NYTCo}
\item
  \href{https://help.nytimes.com/hc/en-us/articles/115015385887-Contact-Us}{Contact
  Us}
\item
  \href{https://www.nytco.com/careers/}{Work with us}
\item
  \href{https://nytmediakit.com/}{Advertise}
\item
  \href{http://www.tbrandstudio.com/}{T Brand Studio}
\item
  \href{https://www.nytimes.com/privacy/cookie-policy\#how-do-i-manage-trackers}{Your
  Ad Choices}
\item
  \href{https://www.nytimes.com/privacy}{Privacy}
\item
  \href{https://help.nytimes.com/hc/en-us/articles/115014893428-Terms-of-service}{Terms
  of Service}
\item
  \href{https://help.nytimes.com/hc/en-us/articles/115014893968-Terms-of-sale}{Terms
  of Sale}
\item
  \href{https://spiderbites.nytimes.com}{Site Map}
\item
  \href{https://help.nytimes.com/hc/en-us}{Help}
\item
  \href{https://www.nytimes.com/subscription?campaignId=37WXW}{Subscriptions}
\end{itemize}
