Sections

SEARCH

\protect\hyperlink{site-content}{Skip to
content}\protect\hyperlink{site-index}{Skip to site index}

\href{https://myaccount.nytimes.com/auth/login?response_type=cookie\&client_id=vi}{}

\href{https://www.nytimes.com/section/todayspaper}{Today's Paper}

\href{/section/opinion}{Opinion}\textbar{}Chinese Hacking Is Alarming.
So Are Data Brokers.

\url{https://nyti.ms/2SMpkLx}

\begin{itemize}
\item
\item
\item
\item
\item
\item
\end{itemize}

Advertisement

\protect\hyperlink{after-top}{Continue reading the main story}

\href{/section/opinion}{Opinion}

Supported by

\protect\hyperlink{after-sponsor}{Continue reading the main story}

\hypertarget{chinese-hacking-is-alarming-so-are-data-brokers}{%
\section{Chinese Hacking Is Alarming. So Are Data
Brokers.}\label{chinese-hacking-is-alarming-so-are-data-brokers}}

Companies like Equifax threaten our personal privacy and our national
security.

\href{https://www.nytimes.com/by/charlie-warzel}{\includegraphics{https://static01.nyt.com/images/2019/03/15/opinion/charlie-warzel/charlie-warzel-thumbLarge-v3.png}}

By \href{https://www.nytimes.com/by/charlie-warzel}{Charlie Warzel}

Mr. Warzel is an opinion writer at large.

\begin{itemize}
\item
  Feb. 10, 2020
\item
  \begin{itemize}
  \item
  \item
  \item
  \item
  \item
  \item
  \end{itemize}
\end{itemize}

\includegraphics{https://static01.nyt.com/images/2020/02/10/opinion/10warzel_web/10warzel_web-articleLarge-v6.jpg?quality=75\&auto=webp\&disable=upscale}

On Monday, the Justice Department
\href{https://www.nytimes.com/2020/02/10/us/politics/equifax-hack-china.html}{announced
that it was charging} four members of China's People's Liberation Army
with the 2017 Equifax breach that resulted in the theft of personal data
of about 145 million Americans.

The attack, according to the charges, was part of a coordinated effort
by Chinese intelligence to steal trade secrets and personal information
to target Americans.

Using the personal data of millions of Americans against their will is
certainly alarming. But what's the difference between the Chinese
government stealing all that information and a data broker amassing it
legally without user consent and selling it on the open market?

Both are predatory practices to invade privacy for insights and
strategic leverage. Yes, one is corporate and legal and the other
geopolitical and decidedly not legal. But the hack wasn't a malfunction
of the system; it was a direct result of how the system was designed.

Equifax is eager to play the hapless victim in all this. Don't believe
it. In a
\href{https://investor.equifax.com/news-and-events/news/2020/02-10-2020-160714269}{statement}
praising the Justice Department, Equifax's chief executive, Mark Begor,
deflected responsibility, highlighting the hack as the work of ``a
well-funded and sophisticated military'' operation. ``The attack on
Equifax was an attack on U.S. consumers as well as the United States,''
he said.

While the state-sponsored attack was indeed well funded and
sophisticated, Equifax, by way of apparent negligence, was also
responsible for the theft of our private information by a foreign
government.

According to the indictment, the Chinese military exploited a
vulnerability in Apache Struts software, which Equifax used. As soon as
Apache disclosed the vulnerability, it offered a patch to prevent
breaches. Equifax's security team, according to the indictment, didn't
employ the patch, leaving the drawbridge down for People's Liberation
Army attackers. From there, the hackers gained access to Equifax's web
servers and ultimately got a hold of employee credentials.

Though the attack was quite sophisticated --- the hackers sneaked out
information in small, hard to detect chunks and routed internet traffic
through 34 servers in over a dozen countries to cover their tracks ---
Equifax's apparent carelessness made it a perfect target.

According to a 2019
\href{http://securities.stanford.edu/filings-documents/1063/EI00_15/2019128_r01x_17CV03463.pdf}{class-action
lawsuit}, the company's cybersecurity practices were a nightmare. The
suit alleged that ``sensitive personal information relating to hundreds
of millions of Americans was not encrypted, but instead was stored in
plain text'' and ``was accessible through a public-facing, widely used
website.'' Another example of the company's weak safeguards, according
to the suit, shows the company struggling to use a competent password
system. ``Equifax employed the username `admin' and the password `admin'
to protect a portal used to manage credit disputes,'' it read.

The takeaway: While almost anything digital is at some risk of being
hacked, the Equifax attack was largely preventable.

Since its establishment in 1899 (it was originally named Retail Credit),
Equifax\href{https://www.wired.com/1995/09/equifax/}{has prompted
concerns} over the sheer volume of data it amasses. Those fears
increased as the company entered the digital age. In
\href{https://www.nytimes.com/1970/03/12/archives/23-to-study-computer-threat.html}{a
March 1970 Times article} about the company, Alan Westin, a professor at
Columbia University, offered this warning: ``Almost inevitably,
transferring information from a manual file to a computer triggers a
threat to civil liberties, to privacy, to a man's very humanity \ldots{}
because access is so simple.''

Five decades on, that statement rings especially true. Moreover, it's a
useful frame to understand why, in a world where everything can be
hacked, bloated data brokers like Equifax present an untenable risk to
our personal and national security.

It's helpful to think about a hack like what happened to Equifax as part
of a chain of events where, the further down the chain you go, the more
intrusive and potentially damaging the results. The Equifax data we know
was stolen is a perfect example of what's known as Personally
Identifiable Information. Obtaining the names, birth dates and Social
Security numbers of almost half of all Americans is troubling on its
own, but that basic information can then be used to procure even more
personal information, including medical or financial records.

That more sensitive information can then be used to target vulnerable
Americans for blackmail or simply to glean detailed information about
the country by analyzing the metadata of its citizens. And so the
revelations in the indictment in the Equifax case are alarming. The
theft is one in a string of successful hacks, including of the federal
\href{https://www.nytimes.com/2015/08/01/world/asia/us-decides-to-retaliate-against-chinas-hacking.html}{Office
of Personnel Management},
\href{https://www.nytimes.com/2019/01/04/us/politics/marriott-hack-passports.html}{Marriott
International} and
\href{https://www.nytimes.com/2019/05/09/technology/anthem-hack-indicted-breach.html}{the
insurance company Anthem}. Given the volume and granularity of the data
and the ability of attackers to use the information to gain \emph{even
more} data, it's not unreasonable to ask, Does China now know as much
about American citizens as our own government does?

In his statement on Monday, Mr. Begor, Equifax's chief executive, noted
that ``cybercrime is one of the greatest threats facing our nation
today.'' But what he ignored was his own company's role in creating a
glaring vulnerability in the system. If we're to think of cybercrime
like an analog counterpart, then Equifax is a bank on Main Street that
forgot to lock its vault.

Why rob a bank? Because that's where the money is. Why hack a data
broker? Because that's where the information is.

The analogy isn't quite apt, though, because Equifax, like other data
brokers, doesn't fill its vaults with deposits from willing customers.
Equifax amasses personal data on millions of Americans whether we want
it to or not, creating valuable profiles that can be used to approve or
deny loans or insurance claims. That data, which can help dictate the
outcome of major events in our lives (where we live, our finances, even
potentially our health), then becomes a target.

From this vantage, it's unclear why data brokers should continue to
collect such sensitive information at scale. Setting aside Equifax's
\href{https://www.wired.com/1995/09/equifax/}{long, sordid history} of
privacy concerns and its refusal to let Americans opt out of collection,
the very existence of such information, stored by private companies with
little oversight, is a systemic risk.

In an endless cyberwar, information is power. Equifax's services as a
data broker offer something similar to its customers, promising data and
insights it can leverage for corporate power. China is behaving a lot
like any other data broker. The big difference is that it isn't paying.

\emph{Like other media companies, The Times collects data on its
visitors when they read stories like this one. For more detail please
see}
\href{https://help.nytimes.com/hc/en-us/articles/115014892108-Privacy-policy?module=inline}{\emph{our
privacy policy}} \emph{and}
\href{https://www.nytimes.com/2019/04/10/opinion/sulzberger-new-york-times-privacy.html?rref=collection\%2Fspotlightcollection\%2Fprivacy-project-does-privacy-matter\&action=click\&contentCollection=opinion\&region=stream\&module=stream_unit\&version=latest\&contentPlacement=8\&pgtype=collection}{\emph{our
publisher's description}} \emph{of The Times's practices and continued
steps to increase transparency and protections.}

\emph{Follow}
\href{https://twitter.com/privacyproject}{\emph{@privacyproject}}
\emph{on Twitter and The New York Times Opinion Section on}
\href{https://www.facebook.com/nytopinion}{\emph{Facebook}}
\emph{and}\href{https://www.instagram.com/nytopinion/}{\emph{Instagram}}\emph{.}

\hypertarget{glossary-replacer}{%
\subsection{glossary replacer}\label{glossary-replacer}}

Advertisement

\protect\hyperlink{after-bottom}{Continue reading the main story}

\hypertarget{site-index}{%
\subsection{Site Index}\label{site-index}}

\hypertarget{site-information-navigation}{%
\subsection{Site Information
Navigation}\label{site-information-navigation}}

\begin{itemize}
\tightlist
\item
  \href{https://help.nytimes.com/hc/en-us/articles/115014792127-Copyright-notice}{©~2020~The
  New York Times Company}
\end{itemize}

\begin{itemize}
\tightlist
\item
  \href{https://www.nytco.com/}{NYTCo}
\item
  \href{https://help.nytimes.com/hc/en-us/articles/115015385887-Contact-Us}{Contact
  Us}
\item
  \href{https://www.nytco.com/careers/}{Work with us}
\item
  \href{https://nytmediakit.com/}{Advertise}
\item
  \href{http://www.tbrandstudio.com/}{T Brand Studio}
\item
  \href{https://www.nytimes.com/privacy/cookie-policy\#how-do-i-manage-trackers}{Your
  Ad Choices}
\item
  \href{https://www.nytimes.com/privacy}{Privacy}
\item
  \href{https://help.nytimes.com/hc/en-us/articles/115014893428-Terms-of-service}{Terms
  of Service}
\item
  \href{https://help.nytimes.com/hc/en-us/articles/115014893968-Terms-of-sale}{Terms
  of Sale}
\item
  \href{https://spiderbites.nytimes.com}{Site Map}
\item
  \href{https://help.nytimes.com/hc/en-us}{Help}
\item
  \href{https://www.nytimes.com/subscription?campaignId=37WXW}{Subscriptions}
\end{itemize}
