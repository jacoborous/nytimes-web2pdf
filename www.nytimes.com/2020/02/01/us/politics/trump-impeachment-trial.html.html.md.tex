Sections

SEARCH

\protect\hyperlink{site-content}{Skip to
content}\protect\hyperlink{site-index}{Skip to site index}

\href{https://www.nytimes.com/section/politics}{Politics}

\href{https://myaccount.nytimes.com/auth/login?response_type=cookie\&client_id=vi}{}

\href{https://www.nytimes.com/section/todayspaper}{Today's Paper}

\href{/section/politics}{Politics}\textbar{}While Stained in History,
Trump Will Emerge From Trial Triumphant and Unshackled

\url{https://nyti.ms/2S9BzkP}

\begin{itemize}
\item
\item
\item
\item
\item
\item
\end{itemize}

\begin{itemize}
\item
  \href{https://www.nytimes.com/2020/07/31/us/elections/biden-vs-trump.html?action=click\&pgtype=Article\&state=default\&region=TOP_BANNER\&context=storylines_menu}{Election
  Updates}
\item
  \href{https://www.nytimes.com/article/biden-vice-president-2020.html?action=click\&pgtype=Article\&state=default\&region=TOP_BANNER\&context=storylines_menu}{Biden's
  V.P. Search}
\item
  \href{https://www.nytimes.com/interactive/2020/07/24/us/politics/trump-biden-campaign-donors.html?action=click\&pgtype=Article\&state=default\&region=TOP_BANNER\&context=storylines_menu}{Map
  of Donations}
\item
  \href{https://www.nytimes.com/interactive/2020/us/elections/delegate-count-primary-results.html?action=click\&pgtype=Article\&state=default\&region=TOP_BANNER\&context=storylines_menu}{Delegate
  Count}
\item
  \href{https://www.nytimes.com/interactive/2019/us/politics/2020-presidential-candidates.html?action=click\&pgtype=Article\&state=default\&region=TOP_BANNER\&context=storylines_menu}{The
  Candidates}
\item
  \href{https://www.nytimes.com/newsletters/politics?action=click\&pgtype=Article\&state=default\&region=TOP_BANNER\&context=storylines_menu}{Politics
  Newsletter}
\end{itemize}

Advertisement

\protect\hyperlink{after-top}{Continue reading the main story}

Supported by

\protect\hyperlink{after-sponsor}{Continue reading the main story}

news analysis

\hypertarget{while-stained-in-history-trump-will-emerge-from-trial-triumphant-and-unshackled}{%
\section{While Stained in History, Trump Will Emerge From Trial
Triumphant and
Unshackled}\label{while-stained-in-history-trump-will-emerge-from-trial-triumphant-and-unshackled}}

His acquittal in the Senate assured, the emboldened president will take
his victory and grievance to the campaign trail, no longer worried about
congressional constraint.

\includegraphics{https://static01.nyt.com/images/2020/02/01/us/politics/01dc-assess-1/01dc-assess-1-articleLarge-v2.jpg?quality=75\&auto=webp\&disable=upscale}

\href{https://www.nytimes.com/by/peter-baker}{\includegraphics{https://static01.nyt.com/images/2018/06/13/multimedia/peter-baker/peter-baker-thumbLarge-v2.png}}

By \href{https://www.nytimes.com/by/peter-baker}{Peter Baker}

\begin{itemize}
\item
  Feb. 1, 2020
\item
  \begin{itemize}
  \item
  \item
  \item
  \item
  \item
  \item
  \end{itemize}
\end{itemize}

WASHINGTON --- Ralph Waldo Emerson seemed to foresee the lesson of the
Senate impeachment trial of President Trump. ``When you strike at a
king,'' Emerson famously said, ``you must kill him.''

Mr. Trump's foes struck at him but did not take him down.

With
\href{https://www.nytimes.com/2020/01/31/us/politics/trump-impeachment-trial.html?action=click\&module=Spotlight\&pgtype=Homepage}{the
end of the impeachment trial now in sight} and acquittal assured, a
triumphant Mr. Trump emerges from the biggest test of his presidency
emboldened, ready to claim exoneration and take his case of grievance,
persecution and resentment to the campaign trail.

The president's Democratic adversaries rolled out the biggest
constitutional weapon they had and failed to defeat him, or even to
force a full trial with witnesses testifying to the allegations against
him. Now Mr. Trump, who has said that the Constitution
\href{https://abcnews.go.com/Politics/exclusive-trump-cites-lessons-nixon-fire-mueller/story?id=63722267}{``allows
me to do whatever I want''} and pushed so many boundaries that curtailed
past
presidents\href{https://abcnews.go.com/Politics/exclusive-trump-cites-lessons-nixon-fire-mueller/story?id=63722267}{,}
has little reason to fear the legislative branch nor any inclination to
reach out in conciliation.

``I don't think in any way Trump is willing to move on,'' said Mickey
Edwards, a former Republican congressman who teaches at Princeton
University. ``I think he will just have been given a green light and he
will claim not just acquittal but vindication and he can do those things
and they can't impeach him again. I think this is going to empower him
to be much bolder. I would expect to see him even more let loose.''

\href{https://www.nytimes.com/2019/12/18/us/politics/trump-impeached.html}{Impeachment
will always be a stain} on Mr. Trump's historical record, a reality that
has stung him in private, according to some close to him. But he will be
the
\href{https://www.nytimes.com/2019/11/23/us/politics/trump-impeachment-voters.html}{first
president in American history to face voters} after an impeachment trial
and that will give him the chance to argue for the next nine months that
his enemies have spent his entire presidency plotting against him to
undo the 2016 election.

``This was clearly a political coup d'état carried out by a group of
people who were amazingly, openly dishonest and I think it's going to be
repudiated,'' said former Speaker Newt Gingrich, a strong ally of the
president's. ``He's been beaten up for three solid years and he's still
standing. That's an amazing achievement if you think about it.''

Even before a final vote on the impeachment charges on Wednesday, Mr.
Trump has several high-profile opportunities in the next few days to
begin framing the new post-trial environment to his advantage.

On Sunday, he will be interviewed by Sean Hannity of Fox News during the
pregame of the Super Bowl, one of the most watched television events of
the year. Then on Tuesday, he will deliver his State of the Union
address from the dais in the House chamber where he
\href{https://www.nytimes.com/2019/12/18/us/politics/trump-impeached.html}{was
impeached in December}.

A senior administration official briefing reporters on Friday said the
president will use his State of the Union address to celebrate ``the
great American comeback'' and present ``a vision of relentless
optimism'' encouraging Congress to work with him. Mr. Trump plans to
pursue an agenda of cutting taxes again, bringing down prescription drug
prices, completing his trade negotiations with China and further
restricting immigration.

From there, Mr. Trump will head back to the campaign trail, starting
with a rally in New Hampshire on Feb. 10, the night before the state's
first-in-the-nation primary race, an effort to upstage the Democrats as
they try to pick a nominee to face him in the fall.

\hypertarget{latest-updates-2020-election}{%
\section{\texorpdfstring{\href{https://www.nytimes.com/2020/07/31/us/elections/biden-vs-trump.html?action=click\&pgtype=Article\&state=default\&region=MAIN_CONTENT_1\&context=storylines_live_updates}{Latest
Updates: 2020
Election}}{Latest Updates: 2020 Election}}\label{latest-updates-2020-election}}

Updated 2020-08-01T01:26:45.732Z

\begin{itemize}
\tightlist
\item
  \href{https://www.nytimes.com/2020/07/31/us/elections/biden-vs-trump.html?action=click\&pgtype=Article\&state=default\&region=MAIN_CONTENT_1\&context=storylines_live_updates\#link-29fdff45}{Kamala
  Harris, a top vice-presidential contender, confronts double
  standards.}
\item
  \href{https://www.nytimes.com/2020/07/31/us/elections/biden-vs-trump.html?action=click\&pgtype=Article\&state=default\&region=MAIN_CONTENT_1\&context=storylines_live_updates\#link-13ec3d9c}{Karen
  Bass and Susan Rice are rising on Biden's vice-presidential
  shortlist.}
\item
  \href{https://www.nytimes.com/2020/07/31/us/elections/biden-vs-trump.html?action=click\&pgtype=Article\&state=default\&region=MAIN_CONTENT_1\&context=storylines_live_updates\#link-49e9a016}{Trump
  says Russian bounties to kill U.S. troops `never took place.'}
\end{itemize}

\href{https://www.nytimes.com/2020/07/31/us/elections/biden-vs-trump.html?action=click\&pgtype=Article\&state=default\&region=MAIN_CONTENT_1\&context=storylines_live_updates}{See
more updates}

Democrats insist that Mr. Trump has been damaged by the evidence
presented to the public that he sought to use the power of his office to
illicitly benefit his own re-election chances. Even as they line up to
acquit him,
\href{https://www.nytimes.com/2020/01/31/us/politics/republican-response-impeachment.html}{some
Senate Republicans have acknowledged}that the House managers prosecuting
the case proved that Mr. Trump withheld \$391 million in security aid to
Ukraine as part of an effort to pressure it to announce political
investigations into his domestic rivals.

But the public comes out of the impeachment trial pretty close to where
it was when it started, divided starkly down the middle with somewhat
more Americans against Mr. Trump than for him.

When the House impeached him in December, 47.4 percent supported the
move and 46.5 percent opposed it, according to
\href{https://projects.fivethirtyeight.com/impeachment-polls/}{an
analysis of multiple surveys} by the polling analysis site
FiveThirtyEight. Now as the trial wraps up, 49.5 percent favor
impeachment versus 46.4 percent who do not.

Those numbers are strikingly close to
\href{https://www.nytimes.com/elections/2016/results/president}{the
popular vote results from 2016}, when Mr. Trump trailed Hillary Clinton
46 percent to 48 percent even as he prevailed in the Electoral College.
That means that the public today is roughly where it was three years
ago; few seem to have changed their minds. And the president has done
nothing to expand his base and by traditional measures is a weak
candidate for a second term, forcing him to try to pull the same
Electoral College inside straight he did last time.

Mr. Trump is the only president in the history of Gallup polling who has
\href{https://news.gallup.com/poll/203207/trump-job-approval-weekly.aspx}{never
had the support of a majority of Americans} for even a single day, a
troubling indicator for re-election. Nine months is an eternity in
American politics these days and, given his history, Mr. Trump could
easily create another furor that will change the campaign dynamics, the
economy could become an issue, and with all the accumulated allegations
some analysts anticipate a certain scandal fatigue could weigh him down.

But Mr. Trump is gambling that he can rally his most fervent supporters
by making the case that he was the victim and not the villain of
impeachment while keeping disenchanted supporters on board with steady
economic growth, rising military spending and conservative judicial
appointments. He has made clear he will paint former Vice President
Joseph R. Biden Jr. as corrupt if he faces him in the fall and will
assail other possible Democratic challengers as socialists.

If Mr. Trump does win a second term, it would be the first time an
impeached president had the opportunity to serve five years after his
trial and Mr. Trump's critics worry that he would feel unbound. He has
already used his power in ways that presidents since Richard M. Nixon
considered out of line, like
\href{https://www.nytimes.com/2017/05/09/us/politics/james-comey-fired-fbi.html}{firing
an F.B.I. director} who was investigating him and
\href{https://www.nytimes.com/2017/11/03/us/politics/trump-says-justice-dept-and-fbi-must-do-what-is-right-and-investigate-democrats.html}{browbeating
the Justice Department} to investigate his political foes.

While in theory nothing in the Constitution would prevent the House from
impeaching him again, as a political matter that seems implausible given
that he has demonstrated his complete command over congressional
Republicans led by Senator Mitch McConnell of Kentucky, leaving the
president less to fear from a Democratic House. Some House managers
warned that acquittal would lower the bar for presidential misconduct,
meaning that Mr. Trump would feel even freer to use his power for his
own benefit because he got away with it.

``He is going to ratchet it up to another level now,'' said Anthony
Scaramucci, the onetime White House communications director who has
broken with Mr. Trump. ``He's going to be Trump to the third power now.
He's not going to be exponential Trump because that's not enough Trump.
It's going to be Trump to the third power.''

But in that, Mr. Scaramucci said, are the seeds of Mr. Trump's own
downfall because he could go so far that he finally alienates enough of
the public to lose. ``The one person who absolutely can beat Trump is
Trump,'' he said.

No other impeached president had the opportunity or challenge that Mr.
Trump does. President Andrew Johnson, who was acquitted in 1868, was a
man without a party, a Democrat who had joined the Republican Abraham
Lincoln's ticket, and was so disliked that both parties nominated other
candidates shortly after his Senate trial, leaving him to finish his
last 10 months in office a lame duck.

Indeed, while Johnson was not removed from office, impeachment reduced
him to a shadow president, said Brenda Wineapple, author of
\href{https://www.nytimes.com/2019/05/15/books/review-impeachers-andrew-johnson-brenda-wineapple.html}{``The
Impeachers,''} an account of his trial.

``The Republicans still had a majority in Congress so they could reject
some of his appointments, which they did, and override his vetoes of
their legislation --- and they could allow the states that conformed to
the Reconstruction Acts to re-enter the Union,'' she said. ``So in that
sense, Johnson was hamstrung, if not powerless.''

\includegraphics{https://static01.nyt.com/images/2020/02/01/us/politics/01dc-assess-2/merlin_9597494_35b077c0-c6c7-49f0-9fd4-59eeb90eefbe-articleLarge.jpg?quality=75\&auto=webp\&disable=upscale}

President Bill Clinton was in his second term
\href{https://www.washingtonpost.com/politics/clinton-impeachment/clinton-impeached-house-approves-articles-alleging-perjury-obstruction/}{when
he was impeached} and acquitted, never to be on a ballot again. With
nearly two years left in office, he tried to move on from his
impeachment, all but pretending it had not happened.
\href{https://www.washingtonpost.com/politics/clinton-impeachment/senate-acquits-president-clinton/}{On
the day of his acquittal} in 1999, he appeared in the Rose Garden alone
and expressed regret rather than vindication.

``I want to say again to the American people how profoundly sorry I am
for what I said and did to trigger these events and the great burden
they have imposed on the Congress and on the American people,'' Mr.
Clinton said, calling for ``a time of reconciliation and renewal.''

As he turned to leave,
\href{http://movies2.nytimes.com/library/politics/021399impeach-clinton-text.html}{a
reporter called after him}. ``In your heart, sir, can you forgive and
forget?''

Mr. Clinton paused as if deciding whether to take the bait, then turned
and answered, ``I believe any person who asks for forgiveness has to be
prepared to give it.''

Mr. Clinton, who unlike Mr. Trump admitted wrongdoing without agreeing
that he committed felonies, never truly forgave his opponents, or
reconciled with them, but for the most part he avoided expressing those
feelings publicly.

``Clinton saw the acquittal as a humbling end to that chapter and I
think Trump sees it as a way to start his re-elect,'' said Jennifer
Palmieri, who was a top aide to Mr. Clinton. ``He just wanted to shut
the door on that and move on and have a fresh start. And Trump sees it
as a jump start --- `this is what I'm going to run on.'''

Mr. Clinton had some help in that Republicans themselves emerged from
his trial feeling bruised by their failure to remove him and the clear
public repudiation of the impeachment in polls and the midterm
elections. Unlike Mr. Trump, whose approval ratings remain
\href{https://projects.fivethirtyeight.com/trump-approval-ratings/?ex_cid=rrpromo}{mired
in the mid-40s}, Mr. Clinton's popularity reached its highest level
during impeachment,
\href{https://news.gallup.com/poll/116584/presidential-approval-ratings-bill-clinton.aspx}{with
73 percent of the public backing him} just days after the House charged
him with high crimes.

``I don't think Clinton was emboldened. I think he was embarrassed about
the mess he caused and he wanted to somehow move on and fix his own
reputation,'' said John Feehery, a Republican strategist who was a top
adviser to Speaker J. Dennis Hastert at the time.

And so did the Republicans. ``I think both the president and the speaker
had a vested interest in moving past impeachment to getting things
done,'' he said. ``We were very conscious about how polarizing
impeachment was and we were dedicated to healing the country and
repairing the G.O.P. brand.''

That does not seem like the likeliest path forward for Mr. Trump, more
of a pugilist than a peacemaker. ``He's obviously legitimately pretty
angry,'' said Mr. Gingrich, who was
\href{https://www.nytimes.com/1998/11/07/us/speaker-steps-down-overview-facing-revolt-gingrich-won-t-run-for-speaker-will.html}{forced
out as speaker} after Republicans lost the midterm elections during the
drive to impeach Mr. Clinton. ``Given that he's a natural
counterpuncher, he may decide to go after them.''

``That's not his best strategy,'' Mr. Gingrich said. ``His best strategy
is to assume that the Democrats are totally out of control, that they
will not be able to keep from fighting. If he appears conciliatory,
they're going to very badly damage themselves with average voters who
are going to say these guys are pathological.''

``He has that option,'' he added. ``I'm not saying he's going to take
it.''

\hypertarget{our-2020-election-guide}{%
\section{Our 2020 Election Guide}\label{our-2020-election-guide}}

Updated July 31, 2020

\begin{itemize}
\item
  \begin{center}\rule{0.5\linewidth}{\linethickness}\end{center}

  \hypertarget{the-latest}{%
  \subsection{The Latest}\label{the-latest}}

  \begin{itemize}
  \tightlist
  \item
    President Trump's assault on the Postal Service is intersecting with
    his attacks on mail-in voting.
    \href{https://www.nytimes.com/2020/07/31/us/politics/trump-usps-mail-delays.html?action=click\&pgtype=Article\&state=default\&region=BELOW_MAIN_CONTENT\&context=storylines_guide}{Voting
    rights groups say it is a recipe for disaster.}
  \end{itemize}
\item
  \begin{center}\rule{0.5\linewidth}{\linethickness}\end{center}

  \hypertarget{bidens-vp-search}{%
  \subsection{Biden's V.P. Search}\label{bidens-vp-search}}

  \begin{itemize}
  \tightlist
  \item
    \href{https://www.nytimes.com/article/biden-vice-president-2020.html?action=click\&pgtype=Article\&state=default\&region=BELOW_MAIN_CONTENT\&context=storylines_guide}{Here
    are 13 women} who have been under consideration to be Joe Biden's
    running mate, and why each might be chosen --- and might not be.
  \end{itemize}
\item
  \begin{center}\rule{0.5\linewidth}{\linethickness}\end{center}

  \hypertarget{keep-up-with-our-coverage}{%
  \subsection{Keep Up With Our
  Coverage}\label{keep-up-with-our-coverage}}

  \begin{itemize}
  \tightlist
  \item
    Get an
    \href{https://www.nytimes.com/newsletters/politics?action=click\&pgtype=Article\&state=default\&region=BELOW_MAIN_CONTENT\&context=storylines_guide}{email}
    recapping the day's news
  \end{itemize}

  \begin{itemize}
  \tightlist
  \item
    Download our mobile app on
    \href{https://apps.apple.com/us/app/nytimes/id284862083?ls=1\&mat_click_id=5c79ae7455014fd1bd66b5610c05b8f2-20191112-16948\&referrer=mat_click_id\%3D5c79ae7455014fd1bd66b5610c05b8f2-20191112-16948\%26link_click_id\%3D722930677036718082}{iOS}
    and
    \href{http://a.localytics.com/android?id=com.nytimes.android\&referrer=utm_source\%3Dother_nyt_mobile_web\%26utm_medium\%3DWeb\%2520page\%26utm_term\%3DGeneral\%2520Mobile\%2520Page\%26utm_campaign\%3DNYT\%2520Mobile\%2520General\%2520Page}{Android}
    and turn on Breaking News and Politics alerts
  \end{itemize}
\end{itemize}

Advertisement

\protect\hyperlink{after-bottom}{Continue reading the main story}

\hypertarget{site-index}{%
\subsection{Site Index}\label{site-index}}

\hypertarget{site-information-navigation}{%
\subsection{Site Information
Navigation}\label{site-information-navigation}}

\begin{itemize}
\tightlist
\item
  \href{https://help.nytimes.com/hc/en-us/articles/115014792127-Copyright-notice}{©~2020~The
  New York Times Company}
\end{itemize}

\begin{itemize}
\tightlist
\item
  \href{https://www.nytco.com/}{NYTCo}
\item
  \href{https://help.nytimes.com/hc/en-us/articles/115015385887-Contact-Us}{Contact
  Us}
\item
  \href{https://www.nytco.com/careers/}{Work with us}
\item
  \href{https://nytmediakit.com/}{Advertise}
\item
  \href{http://www.tbrandstudio.com/}{T Brand Studio}
\item
  \href{https://www.nytimes.com/privacy/cookie-policy\#how-do-i-manage-trackers}{Your
  Ad Choices}
\item
  \href{https://www.nytimes.com/privacy}{Privacy}
\item
  \href{https://help.nytimes.com/hc/en-us/articles/115014893428-Terms-of-service}{Terms
  of Service}
\item
  \href{https://help.nytimes.com/hc/en-us/articles/115014893968-Terms-of-sale}{Terms
  of Sale}
\item
  \href{https://spiderbites.nytimes.com}{Site Map}
\item
  \href{https://help.nytimes.com/hc/en-us}{Help}
\item
  \href{https://www.nytimes.com/subscription?campaignId=37WXW}{Subscriptions}
\end{itemize}
