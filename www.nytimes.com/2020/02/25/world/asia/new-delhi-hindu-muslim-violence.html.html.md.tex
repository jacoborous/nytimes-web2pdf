Sections

SEARCH

\protect\hyperlink{site-content}{Skip to
content}\protect\hyperlink{site-index}{Skip to site index}

\href{https://www.nytimes.com/section/world/asia}{Asia Pacific}

\href{https://myaccount.nytimes.com/auth/login?response_type=cookie\&client_id=vi}{}

\href{https://www.nytimes.com/section/todayspaper}{Today's Paper}

\href{/section/world/asia}{Asia Pacific}\textbar{}New Delhi Streets Turn
Into Battleground, Hindus vs. Muslims

\url{https://nyti.ms/2vZMJkK}

\begin{itemize}
\item
\item
\item
\item
\item
\item
\end{itemize}

Advertisement

\protect\hyperlink{after-top}{Continue reading the main story}

Supported by

\protect\hyperlink{after-sponsor}{Continue reading the main story}

\hypertarget{new-delhi-streets-turn-into-battleground-hindus-vs-muslims}{%
\section{New Delhi Streets Turn Into Battleground, Hindus vs.
Muslims}\label{new-delhi-streets-turn-into-battleground-hindus-vs-muslims}}

As President Trump toured India's capital, at least 11 people were
killed in mob violence that upended a working-class neighborhood.

\includegraphics{https://static01.nyt.com/images/2020/02/25/world/25india-sub/merlin_169520820_c8478583-1c28-4370-9b00-b604d00bf822-articleLarge.jpg?quality=75\&auto=webp\&disable=upscale}

\href{https://www.nytimes.com/by/jeffrey-gettleman}{\includegraphics{https://static01.nyt.com/images/2018/10/10/multimedia/author-jeffrey-gettleman/author-jeffrey-gettleman-thumbLarge.png}}\href{https://www.nytimes.com/by/suhasini-raj}{\includegraphics{https://static01.nyt.com/images/2019/11/22/reader-center/author-Suhasini-Raj/author-Suhasini-Raj-thumbLarge.png}}\href{https://www.nytimes.com/by/sameer-yasir}{\includegraphics{https://static01.nyt.com/images/2019/11/22/reader-center/author-sameer-yasir/author-sameer-yasir-thumbLarge.png}}

By \href{https://www.nytimes.com/by/jeffrey-gettleman}{Jeffrey
Gettleman}, \href{https://www.nytimes.com/by/suhasini-raj}{Suhasini Raj}
and \href{https://www.nytimes.com/by/sameer-yasir}{Sameer Yasir}

\begin{itemize}
\item
  Published Feb. 25, 2020Updated April 2, 2020
\item
  \begin{itemize}
  \item
  \item
  \item
  \item
  \item
  \item
  \end{itemize}
\end{itemize}

NEW DELHI --- A mob of Hindu men, their foreheads marked by a saffron
stripe, angrily patrolled the streets carrying iron bars, clubs and a
bright blue aluminum baseball bat. They were itching for a fight.

The streets in the New Delhi neighborhood were littered with scraps of
bricks. All shops were closed and almost no women or children were out
--- except for two Hindu women brandishing sticks and threatening
journalists.

Gangs of Hindus and Muslims have been clashing in the neighborhood,
Maujpur, and surrounding areas since Sunday, killing at least 11 people,
including a police officer bashed in the head with a rock.

\emph{{[}Analysis:}\href{http://www.nytimes.com/2020/04/02/world/asia/modi-india-press-media.html}{\emph{Under
Modi, India's press is not so free anymore}}\emph{.{]}}

While President Trump and his host, Prime Minister Narendra Modi of
India, discussed geopolitics and lunched together in another part of the
capital, thousands of furious residents faced off again, hurling petrol
bombs, attacking vehicles, hospitalizing several journalists and drawing
more and more police officers and paramilitary troops.

The violence is connected to the continuing protests against
\href{https://www.nytimes.com/2019/12/16/world/asia/india-citizenship-protests.html}{India's
divisive citizenship law}, but this was the first time that the protests
have set off major bloodshed between Hindus and Muslims. It is an old
and dangerous fault line, and any sign of communal violence raises alarm
instantly.

``The situation is volatile and tense,'' said Alok Kumar, a senior
police officer. ``It's a mixed neighborhood, and in seconds you can have
crowds of tens of thousands. Even a small thing can lead to violence.''

\includegraphics{https://static01.nyt.com/images/2020/02/25/world/25india-1/merlin_169453650_ea1e3120-43c1-4860-97f4-a958bda1b328-articleLarge.jpg?quality=75\&auto=webp\&disable=upscale}

In the Muslim quarters, many people felt victimized and accused Mr.
Modi's government of abandoning them. This is a longstanding grievance:
that Mr. Modi's governing political party, which is rooted in a
Hindu-nationalist worldview, has taken sides and abetted violent
religious extremists.

Mr. Modi had choreographed Mr. Trump's visit as a demonstration of
India's rising stature on the world stage, seeking to turn the page on
months of street protests.

But
d\href{https://www.nytimes.com/2020/01/17/world/asia/india-protests-aishe-ghosh.html}{emonstrations
keep breaking out} against the citizenship law, which makes it easier
for migrants of every significant South Asian religion except Islam to
become Indian citizens. Hundreds of thousands of Indian Muslims have
protested, joined by students, academics, human rights activists and
those worried about the country's direction. Many of them say the new
law is a grave threat to India's traditions as a secular and inclusive
nation.

Since last year's election handed Mr. Modi and his Bharatiya Janata
Party another term in power, many Indians feared a resurgence of
communal violence, sparked by Hindu triumphalism and Muslim desperation.
Until now, however, most of the demonstrations remained peaceful.

Image

Smoke rises from a clash in New Delhi on Monday, where Hindus supporting
a new citizenship law faced off against Muslims opposed to
it.Credit...EPA, via Shutterstock

Maujpur is a working-class neighborhood about a half-hour's drive from
the center of Delhi. Gray two- and three-story buildings stand along its
roads, housing small factories and many migrant workers.

For the past several weeks, Muslim residents, many of them women, have
been protesting the citizenship law. On Saturday night,
\href{https://www.firstpost.com/india/jaffrabad-anti-caa-protests-over-500-women-block-road-connecting-seelampur-with-maujpur-and-yamuna-vihar-delhi-metro-shuts-station-8076371.html}{they
began to block a major road}.

The next day, Kapil Mishra, a local leader from Mr. Modi's political
party, showed up. He threatened to mobilize a mob to clear out the
protesters. He said he did not want to create trouble while Mr. Trump
was visiting, but he warned the police that as soon as Mr. Trump left
India on Tuesday night, his followers would clear the streets if the
police did not.

Tensions shot up. As Sunday evening approached, gangs of Hindu men and
Muslim men began throwing rocks at each other. This quickly degenerated
into wider violence, with Hindu residents accusing Muslims of attacking
Hindu statues and Muslim residents expressing fear that a Hindu mob was
forming to get them.

Shoaib Ahmad, a Muslim businessman who makes a living repairing tires,
said his shop was burned down Monday night by a Hindu mob as he stood on
the roof of his house.

``All my dreams were destroyed in those flames,'' Mr. Ahmad said.

Image

A burned shop in the Bhajanpura area of New Delhi.Credit...Sajjad
Hussain/Agence France-Presse --- Getty Images

What made it even worse, he said, was that police officers encouraged
the mobs to burn down Muslims' property.

Images circulating on social media showed a group of Hindu men beating a
Muslim man with sticks, leaving him on the ground, curled up in a ball
and covered in blood.Several Muslim residents in Maujpur and adjacent
neighborhoods said that police officers had stood by while they were
attacked. In mob lynchings of Muslims in the recent past in other parts
of India, many people have made similar accusations against officials in
Mr. Modi's party, saying that the police officers under their command
did not intervene.

India is about 80 percent Hindu and 14 percent Muslim*.*

A stretch of highway between Maujpur's Hindu neighborhood and a nearby
Muslim-dominated area called Jaffrabad now serves as a no-man's land. It
is lined by deserted shops, the asphalt marred by burn marks. Few people
dare to walk through here.

Several police officers conceded that they felt more comfortable
deployed in the Hindu crowd that had gathered at one end of the buffer
zone than with the Muslims massed at the other. While the Muslim crowd
hoisted a big Indian flag, the Hindu crowd chanted religious slogans.

Members of a Hindu mob, armed with crude weapons, begged the police to
let them attack Muslims.

``Give us permission, that's all you need to do,'' one mob leader said.
``You just stand by and watch. We will make sure you don't get hurt.
We'll settle the score.'' Then he used a slur to refer to Muslims.

Image

A man being attacked during clashes between Hindu and Muslim mobs on
Monday.Credit...Danish Siddiqui/Reuters

This kind of communal violence has left a lasting mark on Mr. Modi's
legacy. In 2002, when he was the chief minister of Gujarat State,
\href{https://www.nytimes.com/interactive/2014/04/06/world/asia/modi-gujarat-riots-timeline.html\#/\#time287_8514}{sectarian
riots} left more than 1,000 people dead --- almost 800 of them Muslims
killed by Hindu mobs.

He and his state government were accused of quietly ordering the police
to stand by as the violence raged. He has denied those accusations, and
in 2012, an investigative panel for the Supreme Court
\href{http://archive.indianexpress.com/news/SIT-report-clears-Modi--61-others/935226}{found
no evidence to charge him}. But until he won the post of prime minister
in 2014, he was banned from entering the United States because of the
suspicion hanging over him.

This week, Delhi police officials, who ultimately report to Mr. Modi's
home minister, Amit Shah, said they were determined to keep the Hindu
and Muslim mobs apart. Mr. Kumar, the police official, said he was
trying to organize a peace march between the two sides, but by nightfall
that was nowhere close to happening. Mr. Shah said
\href{https://pib.gov.in/PressReleaseIframePage.aspx?PRID=1604308\#.XlUMyhCxIX8.twitter}{in
a statement} that the violence had been spontaneous, and he appealed for
calm.

But the hatred on the streets was heavy. Several Hindu men said they
felt Muslims did not belong in India.

``Why should they?'' asked Rakesh Sharma, one of the Hindu men who had
taken it upon themselves to chase outsiders from their neighborhood.
``The Muslims have other countries they can go to, like Syria or
Nigeria. They need to get out of India.''

Many Muslims feared that once Mr. Trump left India, the violence would
get even worse.

``It's a little quiet because Trump is here,'' said Mohammed Tahir, a
rickshaw driver. ``Their side is scared to give the prime minister a bad
name.''

``But as soon as Trump leaves,'' he said, ``they will attack. They want
to uproot us. But we won't let that happen. We were born here, we live
here, this country is as much ours as theirs --- and if we need to, we
will all die here, together.''

Image

A car burns near the site of clashes between Hindu and Muslim
men.Credit...Md Meharban/Associated Press

Advertisement

\protect\hyperlink{after-bottom}{Continue reading the main story}

\hypertarget{site-index}{%
\subsection{Site Index}\label{site-index}}

\hypertarget{site-information-navigation}{%
\subsection{Site Information
Navigation}\label{site-information-navigation}}

\begin{itemize}
\tightlist
\item
  \href{https://help.nytimes.com/hc/en-us/articles/115014792127-Copyright-notice}{©~2020~The
  New York Times Company}
\end{itemize}

\begin{itemize}
\tightlist
\item
  \href{https://www.nytco.com/}{NYTCo}
\item
  \href{https://help.nytimes.com/hc/en-us/articles/115015385887-Contact-Us}{Contact
  Us}
\item
  \href{https://www.nytco.com/careers/}{Work with us}
\item
  \href{https://nytmediakit.com/}{Advertise}
\item
  \href{http://www.tbrandstudio.com/}{T Brand Studio}
\item
  \href{https://www.nytimes.com/privacy/cookie-policy\#how-do-i-manage-trackers}{Your
  Ad Choices}
\item
  \href{https://www.nytimes.com/privacy}{Privacy}
\item
  \href{https://help.nytimes.com/hc/en-us/articles/115014893428-Terms-of-service}{Terms
  of Service}
\item
  \href{https://help.nytimes.com/hc/en-us/articles/115014893968-Terms-of-sale}{Terms
  of Sale}
\item
  \href{https://spiderbites.nytimes.com}{Site Map}
\item
  \href{https://help.nytimes.com/hc/en-us}{Help}
\item
  \href{https://www.nytimes.com/subscription?campaignId=37WXW}{Subscriptions}
\end{itemize}
