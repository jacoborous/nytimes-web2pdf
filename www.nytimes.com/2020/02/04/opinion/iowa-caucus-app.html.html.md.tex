Sections

SEARCH

\protect\hyperlink{site-content}{Skip to
content}\protect\hyperlink{site-index}{Skip to site index}

\href{https://myaccount.nytimes.com/auth/login?response_type=cookie\&client_id=vi}{}

\href{https://www.nytimes.com/section/todayspaper}{Today's Paper}

\href{/section/opinion}{Opinion}\textbar{}The App That Broke the Iowa
Caucus

\url{https://nyti.ms/2Uskntx}

\begin{itemize}
\item
\item
\item
\item
\item
\item
\end{itemize}

Advertisement

\protect\hyperlink{after-top}{Continue reading the main story}

\href{/section/opinion}{Opinion}

Supported by

\protect\hyperlink{after-sponsor}{Continue reading the main story}

\hypertarget{the-app-that-broke-the-iowa-caucus}{%
\section{The App That Broke the Iowa
Caucus}\label{the-app-that-broke-the-iowa-caucus}}

Democrats desperately need to win the internet to beat Trump. Their
first big test was a massive failure.

\href{https://www.nytimes.com/by/charlie-warzel}{\includegraphics{https://static01.nyt.com/images/2019/03/15/opinion/charlie-warzel/charlie-warzel-thumbLarge-v3.png}}

By \href{https://www.nytimes.com/by/charlie-warzel}{Charlie Warzel}

Mr. Warzel is an opinion writer at large.

\begin{itemize}
\item
  Feb. 4, 2020
\item
  \begin{itemize}
  \item
  \item
  \item
  \item
  \item
  \item
  \end{itemize}
\end{itemize}

\includegraphics{https://static01.nyt.com/images/2020/02/05/opinion/04warzelWeb/merlin_168363033_9839a71a-ebea-4adb-b2d5-6ad71493a8f6-articleLarge.jpg?quality=75\&auto=webp\&disable=upscale}

A transformative piece of technology is supposed to ``disrupt'' the
unwieldy ways that came before it. On Monday evening, an app built to
deliver quicker caucus results took the Silicon Valley term of art
literally, contributing to massive delays in reporting the results in
Iowa. Hours after the caucuses ended, the state Democratic Party,
\href{https://www.nytimes.com/2020/02/04/us/politics/iowa-caucus-problems.html}{citing
inconsistencies in the reporting data}, still has not publicly reported
any results. It stressed there was no ``hack or intrusion.''

Over the past hours a disheartening game of electoral tech support,
conducted by journalists across the internet, has unfolded. And in place
of definitive results, an information war has broken out, unleashing
reckless speculation, conspiracy theories and deep anxiety.

First came the reports --- trickling in from caucus leaders, precinct
captains and observers --- that the app wasn't working properly. On
Monday evening a precinct captain told me by text that their caucus
manager was ``unable to get the app from the Democratic Party to work''
and ``had to do the math to figure out delegates `long hand.'''
FiveThirtyEight's
\href{https://fivethirtyeight.com/live-blog/iowa-caucus-2020-election-live/\#255769}{Amelia
Thomson-Deveaux spoke} to a frustrated caucus leader who suggested the
app itself wouldn't download. ``We could not problem-solve getting the
app onto one of our devices,'' he told her. NBC News
\href{https://www.nbcnews.com/politics/2020-election/live-blog/iowa-caucuses-live-updates-2020-democrats-make-their-final-pitches-n1128596/ncrd1129196\#liveBlogHeader}{reported}
some caucus leaders had missed the window to download the app
altogether. The Biden campaign
\href{https://www.nbcnews.com/politics/2020-election/live-blog/iowa-caucuses-live-updates-2020-democrats-make-their-final-pitches-n1128596/ncrd1129516}{issued
a letter} to Iowa party leaders suggesting the app had failed.

Reckless speculation followed about possible security problems with the
technology. Stories from late last month raising concerns about
\href{https://www.wsj.com/articles/dems-iowa-caucus-voting-app-stirs-security-concerns-11580063221}{the
caucus app's vulnerabilities} recirculated on Twitter. Among the chief
fears: The app was to be downloaded directly to the phones of caucus
volunteers, making it difficult to ensure the safety of the devices.

After midnight,
\href{https://www.huffingtonpost.co.uk/entry/iowa-caucus-app-shadow_n_5e390191c5b687dacc722824?ri18n=true}{The
Huffington Post} reported that Shadow, a tech company funded by the
progressive digital media firm
\href{https://www.nytimes.com/2019/11/04/us/politics/democratic-political-campaign-advertising.html}{Acronym},
was responsible for building the app. Shadow, according to
\href{https://shadowinc.io/about}{its website}, bills itself ``as
building a long-term, side-by-side `Shadow' of tech infrastructure to
the Democratic Party and the progressive community at large.'' Acronym
quickly
\href{https://twitter.com/teddyschleifer/status/1224585689586532352}{put
out a statement} distancing itself from Shadow and noting, ``We, like
everyone else, are eagerly awaiting more information from the Iowa
Democratic Party.''

There's a great deal we don't know yet about Shadow and the caucus app,
though details have trickled in. Vice attempted to install the app
itself and had similar issues to caucus leaders logging in. On Tuesday
afternoon, Shadow issued its first official statement taking
responsibility for the delayed results.
\href{https://twitter.com/ShadowIncHQ/status/1224773797380837377?s=20}{Via
Twitter,} the company stressed that ``the underlying data and collection
process via Shadow's mobile caucus app was sound and accurate, but our
process to transmit that caucus results data generated via the app to
the IDP was not.''

Its apparent failure is a nightmare scenario for Democrats and the
political left. Quite simply, the party desperately needs to
\href{https://www.nytimes.com/interactive/2019/11/01/opinion/democrats-2020-election-online.html}{win
the internet} in their race to beat President Trump. That means building
infrastructure to connect and assuage voters, controlling the narrative
and overcoming the substantial
\href{https://www.nytimes.com/interactive/2019/11/01/opinion/democrats-2020-election-online.html}{time
advantage} held by the president. In its first critical test, the party
systematically undermined each of those goals.

Shadow's failure suggests a potentially deadly combination of
techno-utopianism and laziness. The two fuel each other: The overarching
belief that software will fix everything leads to slapdash engineering,
procurement and deployment. The result is an obsession with another
Silicon Valley term of art, ``minimum viable product.'' The author Eric
Ries
\href{http://www.startuplessonslearned.com/2009/08/minimum-viable-product-guide.html}{defined}
it as ``that version of a new product which allows a team to collect the
maximum amount of validated learning about customers with the least
effort.''

Shadow's app seems to fit that definition. Reports
\href{https://twitter.com/nicoleperlroth/status/1224565448303357952?s=20}{suggest}
that the app was engineered in just the past two months. According to
cybersecurity consultants and academics
\href{https://www.nytimes.com/2020/02/03/us/politics/iowa-caucuses.html?action=click\&module=Spotlight\&pgtype=Homepage\#link-774e54b7}{interviewed
by the Times}, the app was not tested at statewide scale or vetted by
the Department of Homeland Security's cybersecurity agency. And even if
the app was working just fine,
\href{https://twitter.com/sheeraf/status/1224574088917569536?s=20}{reports
suggest} the roll out of the tool was bungled, to the point where those
tasked with reporting via the app weren't trained to know how to use it.

The process feels reckless given today's internet, where individual
devices are easy to compromise and where routine disruptions like denial
of service attacks can happen at a moment's notice. There's also
precedent for rigorous testing. A
\href{https://www.theatlantic.com/technology/archive/2012/11/when-the-nerds-go-marching-in/265325/}{2012
profile of President Obama's digital team} in The Atlantic detailed an
excruciating process by which the organization simulated ``every
possible disaster situation'' weeks before Election Day to ensure
reliability.

Since the caucus is conducted in public view and with a full paper
trail, it seems hard to imagine that the results would be lost. Still, a
critical failure like this creates credibility problems for the party
and confidence issues for voters, who woke up Tuesday morning to
uncertainty.

The cryptic nature of the digital firms and tech contractors is also
bound to raise questions. Who exactly is responsible for building the
apps intended to protect the integrity of the democratic process? Who is
funding the companies behind the tech companies? Why didn't the Iowa
Democratic Party disclose the app maker? How are procurement decisions
made? Where's the transparency? That the name of the company at the
center of the fiasco is the literal definition of opacity doesn't help
either.

Perhaps most concerning is that, on an internet engaged in a constant
information war, the Democrats' technology failure created an
information vacuum that was quickly seized upon by trolls and political
operatives alike to cast doubt on the electoral process and sow
division. ``Quality control = rigged?'' Brad Parscale, Mr. Trump's
campaign manager,
\href{https://twitter.com/parscale/status/1224533010890002434?s=20}{tweeted}
on Monday night. The message was retweeted and liked a combined 14,000
times. Both Eric Trump and Donald Trump Jr. questioned whether the
results had been ``rigged'' or fixed, as did the Trump campaign's
national press secretary, Kayleigh McEnany.

In 2020, even in periods of relative calm and certainty, conspiracy
theories abound. By failing to deliver as an anxious nation watched, the
Iowa Democratic Party helped transform the caucus into a petri dish for
conspiracies. Democrats floated suspicions of their own party;
Republicans amplified them and tried out theories of their own;
unsubstantiated claims of meddling or hacking rattled around picking up
shares, likes and retweets. Facts were scarce. Fear, uncertainty and
doubt took their place. ``Big WIN for us in Iowa tonight,'' Mr. Trump
tweeted shortly before midnight.

Disruption, indeed.

\emph{Like other media companies, The Times collects data on its
visitors when they read stories like this one. For more detail please
see}
\href{https://help.nytimes.com/hc/en-us/articles/115014892108-Privacy-policy?module=inline}{\emph{our
privacy policy}} \emph{and}
\href{https://www.nytimes.com/2019/04/10/opinion/sulzberger-new-york-times-privacy.html?rref=collection\%2Fspotlightcollection\%2Fprivacy-project-does-privacy-matter\&action=click\&contentCollection=opinion\&region=stream\&module=stream_unit\&version=latest\&contentPlacement=8\&pgtype=collection}{\emph{our
publisher's description}} \emph{of The Times's practices and continued
steps to increase transparency and protections.}

\emph{Follow}
\href{https://twitter.com/privacyproject}{\emph{@privacyproject}}
\emph{on Twitter and The New York Times Opinion Section on}
\href{https://www.facebook.com/nytopinion}{\emph{Facebook}}
\emph{and}\href{https://www.instagram.com/nytopinion/}{\emph{Instagram}}\emph{.}

\hypertarget{glossary-replacer}{%
\subsection{glossary replacer}\label{glossary-replacer}}

Advertisement

\protect\hyperlink{after-bottom}{Continue reading the main story}

\hypertarget{site-index}{%
\subsection{Site Index}\label{site-index}}

\hypertarget{site-information-navigation}{%
\subsection{Site Information
Navigation}\label{site-information-navigation}}

\begin{itemize}
\tightlist
\item
  \href{https://help.nytimes.com/hc/en-us/articles/115014792127-Copyright-notice}{©~2020~The
  New York Times Company}
\end{itemize}

\begin{itemize}
\tightlist
\item
  \href{https://www.nytco.com/}{NYTCo}
\item
  \href{https://help.nytimes.com/hc/en-us/articles/115015385887-Contact-Us}{Contact
  Us}
\item
  \href{https://www.nytco.com/careers/}{Work with us}
\item
  \href{https://nytmediakit.com/}{Advertise}
\item
  \href{http://www.tbrandstudio.com/}{T Brand Studio}
\item
  \href{https://www.nytimes.com/privacy/cookie-policy\#how-do-i-manage-trackers}{Your
  Ad Choices}
\item
  \href{https://www.nytimes.com/privacy}{Privacy}
\item
  \href{https://help.nytimes.com/hc/en-us/articles/115014893428-Terms-of-service}{Terms
  of Service}
\item
  \href{https://help.nytimes.com/hc/en-us/articles/115014893968-Terms-of-sale}{Terms
  of Sale}
\item
  \href{https://spiderbites.nytimes.com}{Site Map}
\item
  \href{https://help.nytimes.com/hc/en-us}{Help}
\item
  \href{https://www.nytimes.com/subscription?campaignId=37WXW}{Subscriptions}
\end{itemize}
