Sections

SEARCH

\protect\hyperlink{site-content}{Skip to
content}\protect\hyperlink{site-index}{Skip to site index}

\href{https://www.nytimes.com/section/world/asia}{Asia Pacific}

\href{https://myaccount.nytimes.com/auth/login?response_type=cookie\&client_id=vi}{}

\href{https://www.nytimes.com/section/todayspaper}{Today's Paper}

\href{/section/world/asia}{Asia Pacific}\textbar{}Pakistan Arrests a
Media Owner, and Journalists Cry Foul

\url{https://nyti.ms/3aRuTzz}

\begin{itemize}
\item
\item
\item
\item
\item
\end{itemize}

Advertisement

\protect\hyperlink{after-top}{Continue reading the main story}

Supported by

\protect\hyperlink{after-sponsor}{Continue reading the main story}

\hypertarget{pakistan-arrests-a-media-owner-and-journalists-cry-foul}{%
\section{Pakistan Arrests a Media Owner, and Journalists Cry
Foul}\label{pakistan-arrests-a-media-owner-and-journalists-cry-foul}}

The owner, Mir Shakil-ur-Rehman, is one of Pakistan's most influential
media figures. The prime minister, Imran Khan, has shown particular
impatience with its coverage.

\includegraphics{https://static01.nyt.com/images/2020/03/12/world/12pakistan/merlin_140469249_4fcca17a-9150-46a4-abc7-abd9499af54b-articleLarge.jpg?quality=75\&auto=webp\&disable=upscale}

By \href{https://www.nytimes.com/by/salman-masood}{Salman Masood}

\begin{itemize}
\item
  Published March 12, 2020Updated June 26, 2020
\item
  \begin{itemize}
  \item
  \item
  \item
  \item
  \item
  \end{itemize}
\end{itemize}

ISLAMABAD, Pakistan --- The authorities on Thursday arrested the owner
of the country's largest media group on three-decade-old allegations
involving a land deal, a case widely criticized by journalist groups as
an attempt to muzzle independent news reporting.

The owner, Mir Shakil-ur-Rehman, is one of Pakistan's most influential
media figures, and his company, the Jang Media Group, has run afoul of
successive governments. But the current
\href{https://www.nytimes.com/2020/06/26/world/asia/pakistan-imran-khan-bin-laden-martyr.html}{prime
minister, Imran Khan}, has shown particular impatience with its
coverage.

``This arrest over a 34-year-old land deal makes a mockery of Pakistan's
claim to be a democracy that upholds freedom of the press,'' said Steven
Butler, Asia program coordinator for the Committee to Protect
Journalists, one of several groups to denounce the arrest.

The Human Rights Commission of Pakistan, an independent group, called
the case ``another attempt to gag a beleaguered independent press.''

Officials with the National Accountability Bureau, an anti-corruption
watchdog, allege that Mr. Rehman illegally leased land from the
government of Prime Minister Nawaz Sharif in 1986 and then had ownership
transferred to him in 2016, when Mr. Sharif again headed the government.

Mr. Rehman, who is expected to appear in court Friday, has denied the
allegations and said the property was obtained from a private party.

Opposition politicians have accused the anti-corruption bureau of
working at the behest of Mr. Khan's government, bringing several
politicized cases against rivals even as inquiries against federal
ministers and allies have been put on the back burner. The bureau, which
is intended to be an independent body, has denied charges of
politicization.

Pakistan has long been considered one of the world's most dangerous
countries for journalists. Journalists for years have described an
atmosphere of pressure and intimidation by the country's powerful
military and successive governments.

But Mr. Rehman's media group said that the authorities had recently
stepped up pressure on its reporters, producers and editors, and had
threatened to use the country's media regulator to shut down its
television channels.

They note that the Khan government has halted official advertisements in
Jang media and in Dawn, the country's leading English daily, choking off
a lucrative flow of revenue and straining the companies' finances. Other
media groups in the country have also been affected by the pulling of
government advertisements, leading to layoffs.

Mr. Khan has been particularly rankled by Jang's coverage of the 2018
election, which elevated him to power, and has accused the media company
of being an ally of Mr. Sharif, who was removed from power by the
Supreme Court after corruption-related investigations.

Advertisement

\protect\hyperlink{after-bottom}{Continue reading the main story}

\hypertarget{site-index}{%
\subsection{Site Index}\label{site-index}}

\hypertarget{site-information-navigation}{%
\subsection{Site Information
Navigation}\label{site-information-navigation}}

\begin{itemize}
\tightlist
\item
  \href{https://help.nytimes.com/hc/en-us/articles/115014792127-Copyright-notice}{©~2020~The
  New York Times Company}
\end{itemize}

\begin{itemize}
\tightlist
\item
  \href{https://www.nytco.com/}{NYTCo}
\item
  \href{https://help.nytimes.com/hc/en-us/articles/115015385887-Contact-Us}{Contact
  Us}
\item
  \href{https://www.nytco.com/careers/}{Work with us}
\item
  \href{https://nytmediakit.com/}{Advertise}
\item
  \href{http://www.tbrandstudio.com/}{T Brand Studio}
\item
  \href{https://www.nytimes.com/privacy/cookie-policy\#how-do-i-manage-trackers}{Your
  Ad Choices}
\item
  \href{https://www.nytimes.com/privacy}{Privacy}
\item
  \href{https://help.nytimes.com/hc/en-us/articles/115014893428-Terms-of-service}{Terms
  of Service}
\item
  \href{https://help.nytimes.com/hc/en-us/articles/115014893968-Terms-of-sale}{Terms
  of Sale}
\item
  \href{https://spiderbites.nytimes.com}{Site Map}
\item
  \href{https://help.nytimes.com/hc/en-us}{Help}
\item
  \href{https://www.nytimes.com/subscription?campaignId=37WXW}{Subscriptions}
\end{itemize}
