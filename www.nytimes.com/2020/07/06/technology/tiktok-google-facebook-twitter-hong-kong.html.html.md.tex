Sections

SEARCH

\protect\hyperlink{site-content}{Skip to
content}\protect\hyperlink{site-index}{Skip to site index}

\href{https://www.nytimes.com/section/technology}{Technology}

\href{https://myaccount.nytimes.com/auth/login?response_type=cookie\&client_id=vi}{}

\href{https://www.nytimes.com/section/todayspaper}{Today's Paper}

\href{/section/technology}{Technology}\textbar{}TikTok to Withdraw From
Hong Kong as Tech Giants Halt Data Requests

\url{https://nyti.ms/2ZNP79m}

\begin{itemize}
\item
\item
\item
\item
\item
\end{itemize}

Advertisement

\protect\hyperlink{after-top}{Continue reading the main story}

Supported by

\protect\hyperlink{after-sponsor}{Continue reading the main story}

\hypertarget{tiktok-to-withdraw-from-hong-kong-as-tech-giants-halt-data-requests}{%
\section{TikTok to Withdraw From Hong Kong as Tech Giants Halt Data
Requests}\label{tiktok-to-withdraw-from-hong-kong-as-tech-giants-halt-data-requests}}

Google, Facebook and Twitter said they were reviewing China's punitive
new national security law for the city, a rare public questioning of
Chinese policy by major American tech companies.

\includegraphics{https://static01.nyt.com/images/2020/07/06/business/06hk-tech/merlin_174186102_b2707e58-d1fc-458c-88be-99a8b38b96c9-articleLarge.jpg?quality=75\&auto=webp\&disable=upscale}

\href{https://www.nytimes.com/by/paul-mozur}{\includegraphics{https://static01.nyt.com/images/2018/10/15/multimedia/author-paul-mozur/author-paul-mozur-thumbLarge.png}}

By \href{https://www.nytimes.com/by/paul-mozur}{Paul Mozur}

\begin{itemize}
\item
  July 6, 2020
\item
  \begin{itemize}
  \item
  \item
  \item
  \item
  \item
  \end{itemize}
\end{itemize}

\href{https://cn.nytimes.com/technology/20200707/facebook-temporarily-stops-hong-kong-data-requests/}{阅读简体中文版}\href{https://cn.nytimes.com/technology/20200707/facebook-temporarily-stops-hong-kong-data-requests/zh-hant/}{閱讀繁體中文版}

Google, Facebook and Twitter said on Monday that they would temporarily
stop processing
\href{https://www.nytimes.com/2020/07/20/world/asia/hong-kong-coronavirus.html}{Hong
Kong} government requests for user data as the companies reviewed a
\href{https://www.nytimes.com/2020/06/30/world/asia/hong-kong-security-law-explain.html}{sweeping
national security law} that has
\href{https://www.nytimes.com/2020/07/01/world/hong-kong-security-law-fear.html?action=click\&module=Top\%20Stories\&pgtype=Homepage}{chilled
political expression} in the city.

The companies said they were still assessing the law, which has already
been used
\href{https://www.nytimes.com/2020/07/01/world/asia/hong-kong-security-law-china.html?campaign_id=2\&emc=edit_th_20200702\&instance_id=19898\&nl=todaysheadlines\&regi_id=69893020\&segment_id=32410\&user_id=82de9857359202789a14fd0cdfbc6442}{to
arrest people} who have called for Hong Kong independence. Facebook said
its review would include human rights considerations.

The surprising consensus from the rival American internet giants, which
each used similar language in each statement, was a rare public
questioning of Chinese policy. It was also a stark illustration of the
deep quandaries the companies face with the
\href{https://www.nytimes.com/2020/07/07/business/hong-kong-security-law-tech.html}{sweeping,
punitive law}.

TikTok went even further than the American companies on Monday, saying
it would withdraw its app from stores in Hong Kong and make the app
inoperable to users there within a few days.

The video app is owned by the Chinese internet giant ByteDance but is
not available in mainland China. TikTok has said that managers outside
China call the shots on key aspects of its business, including rules
about content.

Late Monday, Hong Kong released new rules that give the police powers to
take down internet posts and punish internet companies that do not
comply with data requests. The new rules explicitly give the authorities
the ability to jail employees at internet companies if the firms do not
comply with requests for user data. Because the new rules apply across
the world, they open up the prospect of tech companies having to choose
between releasing data on people writing from places like the United
States or face a six-month jail sentence for an employee.

The American companies did not say whether they would ultimately decide
to cooperate with parts of the law, just that they had temporarily
stopped fielding government requests as they decided. What they decide
and the ensuing legal challenges from Hong Kong's government will most
likely chart a course for the future of internet freedoms in the city,
where the web has not been tightly censored as it has in mainland China.
Many fear the law could lead to suffocating new controls like those in
China, where Facebook, Twitter and Google are all blocked.

The companies have much to lose. Despite the blocks, Google, Facebook
and Twitter have large advertising businesses in the country.

``We are pausing the review of government requests for user data from
Hong Kong pending further assessment of the National Security Law,
including formal human rights due diligence and consultations with
international human rights experts,'' Facebook wrote in a statement.

``We believe freedom of expression is a fundamental human right and
support the right of people to express themselves without fear for their
safety or other repercussions,'' the statement added. The suspension of
data reviews also applies to the messaging app WhatsApp, the company
said.

On Monday, a Google spokesman said the company had paused processing
data requests from the Hong Kong authorities on Wednesday, and Twitter
said it had also stopped processing the requests. Telegram, a messaging
app popular with Hong Kong's protesters, said on Sunday that
\href{https://hongkongfp.com/2020/07/05/exclusive-telegram-to-temporarily-refuse-data-requests-from-hong-kong-courts-amid-security-law-terrorism-row/}{it
would suspend}the provision of user data until a consensus was reached
on the new law. Telegram has offices in the Middle East and Europe.

Some people in Hong Kong reported being unable to download the TikTok
app on Tuesday.

The national security law, adopted in part to quash the antigovernment
demonstrations that have smoldered in Hong Kong for more than a year,
was introduced last week on the anniversary of the city's return to
Chinese control. Though officials insist that the
\href{https://www.nytimes.com/2020/07/02/world/asia/hong-kong-security-china.html}{sweeping
and punitive new rules} will affect only a small number of offenders,
\href{https://www.nytimes.com/2020/07/05/world/asia/hong-kong-security-law.html}{many
worry} that it will be used to widely curb dissent in Hong Kong, which,
unlike mainland China, continues to have an array of civil liberties.

Riva Sciuto, a Google spokeswoman, said, ``Last Wednesday, when the law
took effect, we paused production on any new data requests from Hong
Kong authorities, and we'll continue to review the details of the new
law.''

The law has already cast a pall over the city's internet. Seeking safer
ways to communicate, legions have downloaded the encrypted messaging app
Signal, pushing it to the top of the list of app store downloads.
Others, fearing prosecution for speech crimes, have deleted online
posts, likes and even whole accounts.

The new rules announced by Hong Kong on Monday made clearer how the law
would apply to online discussion.

The government said that if an internet company failed to comply with a
court order to turn over data in cases related to national security, it
could be fined almost \$13,000 and an employee could face six months in
prison. If a person is ordered to remove a post and he or she refuses,
that person can face a jail sentence of one year. A separate provision
also gave the police wide powers to order the deletion of internet posts
that threaten national security. How widely the rules will be enforced
remains unclear.

The rules leave internet giants like Facebook in a tricky place. The
companies regularly provide user data to local law enforcement, yet the
vaguely written national security law has criminalized certain types of
political speech and branded some forms of vandalism terror crimes.

Going along with the law may be unpopular in the United States, where it
has received bipartisan condemnation. Yet, standing up against it could
raise the ire of Beijing, hurt companies' bottom lines and put local
employees at risk.

Daisuke Wakabayashi contributed reporting from Oakland, Calif.; Mike
Isaac from San Francisco; and Raymond Zhong from Taipei, Taiwan.

Advertisement

\protect\hyperlink{after-bottom}{Continue reading the main story}

\hypertarget{site-index}{%
\subsection{Site Index}\label{site-index}}

\hypertarget{site-information-navigation}{%
\subsection{Site Information
Navigation}\label{site-information-navigation}}

\begin{itemize}
\tightlist
\item
  \href{https://help.nytimes.com/hc/en-us/articles/115014792127-Copyright-notice}{©~2020~The
  New York Times Company}
\end{itemize}

\begin{itemize}
\tightlist
\item
  \href{https://www.nytco.com/}{NYTCo}
\item
  \href{https://help.nytimes.com/hc/en-us/articles/115015385887-Contact-Us}{Contact
  Us}
\item
  \href{https://www.nytco.com/careers/}{Work with us}
\item
  \href{https://nytmediakit.com/}{Advertise}
\item
  \href{http://www.tbrandstudio.com/}{T Brand Studio}
\item
  \href{https://www.nytimes.com/privacy/cookie-policy\#how-do-i-manage-trackers}{Your
  Ad Choices}
\item
  \href{https://www.nytimes.com/privacy}{Privacy}
\item
  \href{https://help.nytimes.com/hc/en-us/articles/115014893428-Terms-of-service}{Terms
  of Service}
\item
  \href{https://help.nytimes.com/hc/en-us/articles/115014893968-Terms-of-sale}{Terms
  of Sale}
\item
  \href{https://spiderbites.nytimes.com}{Site Map}
\item
  \href{https://help.nytimes.com/hc/en-us}{Help}
\item
  \href{https://www.nytimes.com/subscription?campaignId=37WXW}{Subscriptions}
\end{itemize}
