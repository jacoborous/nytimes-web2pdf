Sections

SEARCH

\protect\hyperlink{site-content}{Skip to
content}\protect\hyperlink{site-index}{Skip to site index}

\href{https://www.nytimes.com/section/politics}{Politics}

\href{https://myaccount.nytimes.com/auth/login?response_type=cookie\&client_id=vi}{}

\href{https://www.nytimes.com/section/todayspaper}{Today's Paper}

\href{/section/politics}{Politics}\textbar{}A \#MeToo Moment Emerges for
Military Women After Soldier's Killing

\url{https://nyti.ms/3iOMZal}

\begin{itemize}
\item
\item
\item
\item
\item
\end{itemize}

\href{https://www.nytimes.com/news-event/george-floyd-protests-minneapolis-new-york-los-angeles?action=click\&pgtype=Article\&state=default\&region=TOP_BANNER\&context=storylines_menu}{Race
and America}

\begin{itemize}
\tightlist
\item
  \href{https://www.nytimes.com/2020/07/26/us/protests-portland-seattle-trump.html?action=click\&pgtype=Article\&state=default\&region=TOP_BANNER\&context=storylines_menu}{Protesters
  Return to Other Cities}
\item
  \href{https://www.nytimes.com/2020/07/24/us/portland-oregon-protests-white-race.html?action=click\&pgtype=Article\&state=default\&region=TOP_BANNER\&context=storylines_menu}{Portland
  at the Center}
\item
  \href{https://www.nytimes.com/2020/07/23/podcasts/the-daily/portland-protests.html?action=click\&pgtype=Article\&state=default\&region=TOP_BANNER\&context=storylines_menu}{Podcast:
  Showdown in Portland}
\item
  \href{https://www.nytimes.com/interactive/2020/07/16/us/black-lives-matter-protests-louisville-breonna-taylor.html?action=click\&pgtype=Article\&state=default\&region=TOP_BANNER\&context=storylines_menu}{45
  Days in Louisville}
\end{itemize}

Advertisement

\protect\hyperlink{after-top}{Continue reading the main story}

Supported by

\protect\hyperlink{after-sponsor}{Continue reading the main story}

\hypertarget{a-metoo-moment-emerges-for-military-women-after-soldiers-killing}{%
\section{A \#MeToo Moment Emerges for Military Women After Soldier's
Killing}\label{a-metoo-moment-emerges-for-military-women-after-soldiers-killing}}

Women in the military say the horrific killing of Army Specialist
Vanessa Guillen has swept the nation, galvanizing even civilians to the
cause.

\includegraphics{https://static01.nyt.com/images/2020/07/12/us/politics/12dc-army-metoo-print1/merlin_174303915_3d41e894-f347-4e12-8579-5d984bf936cd-articleLarge.jpg?quality=75\&auto=webp\&disable=upscale}

By \href{https://www.nytimes.com/by/jennifer-steinhauer}{Jennifer
Steinhauer}

\begin{itemize}
\item
  Published July 11, 2020Updated July 31, 2020
\item
  \begin{itemize}
  \item
  \item
  \item
  \item
  \item
  \end{itemize}
\end{itemize}

WASHINGTON --- As the \#MeToo movement gained ground, propelled by
stories of women in Hollywood, the news media, restaurants and other
industries, women in the military have remained in the shadows.

Then came the killing of
\href{https://www.nytimes.com/2020/07/31/podcasts/the-daily/vanessa-guillen-military-metoo.html}{Army
Specialist Vanessa Guillen},
\href{https://www.nytimes.com/article/vanessa-guillen-fort-hood.html}{whose
remains were discovered last month} about 25 miles from Fort Hood in
central Texas, the victim, officials said, of a fellow soldier. Her
death has attracted the attention of the nation --- veterans,
active-duty service members and civilians alike.

Women in the military and those who advocate for them say the horrific
nature of the crime, perpetrated against the backdrop of a racial
equality movement sweeping the country, has galvanized many women to the
cause. The persistence of Specialist Guillen's family also has kept
front and center a case that might otherwise have left them in
grief-stricken retreat.

``I think generally the American moment we're in is inspiring collective
action in a way that we've needed,'' said Allison Jaslow, a former Army
captain and veteran of the Iraq war who has tried for years to raise
awareness of the issues. ``Women are tired of how women are still
getting deprioritized, and have lost patience.''

She said she saw a direct line from Breonna Taylor, a Black woman who
was killed by the police, to Specialist Guillen, who was Latina, to
``the women at home struggling to get our society to respond to their
needs.''

Specialist
\href{https://www.nytimes.com/article/vanessa-guillen-fort-hood.html}{Guillen},
20, was last seen on April 22 at Fort Hood. Only on July 2 did the Army
reveal that she was killed by another soldier who then tried to dispose
of her dismembered remains.

That soldier, Army Specialist Aaron Robinson, killed himself with a
pistol as police approached him this month. Authorities arrested his
girlfriend, Cecily Aguilar, after
\href{https://www.justice.gov/usao-wdtx/pr/killeen-woman-faces-federal-charge-connection-disappearance-us-army-specialist-vanessa}{Justice
Department officials}revealed in court documents that Specialist
Robinson told her he killed Specialist Guillen with a hammer and that
the couple then tried to dismember and burn her remains.

The revelations sparked immediate and widespread outrage and grief. In
Fort Worth, Houston and Austin, Texas, artists created
\href{https://twitter.com/juanitoferia/status/1280646434069544965}{murals}
in Specialist Guillen's memory. From South Sioux City, Neb., to Baldwin
Park, Calif. to East Los Angeles, several makeshift
\href{https://twitter.com/BaldwinPNews/status/1280305557371408384}{memorial
sites} have been set up by community members.
\href{https://twitter.com/hashtag/JUSTICEFORVANNESSAGUILLEN?src=hashtag_click}{On
Twitter, the hashtag \#JUSTICEFORVANNESSAGUILLEN} trended for days.

A group of female veterans created a forum for women affiliated with the
military that is calling for a congressional investigation into her
death. It quickly gained thousands of members. The actor Rose McGowan, a
prominent figure in the \#MeToo movement in Hollywood,
\href{https://twitter.com/rosemcgowan/status/1279466856479367175}{advocated
for Specialist Guillen}on Twitter. The case has made mainstream podcasts
and programs on the right and left, from the crime chronicler
\href{https://www.stitcher.com/podcast/mrw-productions-llc/crime-stories-with-nancy-grace/e/69241337}{Nancy
Grace} to the feminist podcast
\href{https://crooked.com/podcast/courting-disaster/}{Courting
Disaster.}

The presumptive Democratic presidential nominee, former Vice President
Joseph R. Biden Jr., said in a statement last week: ``We owe it to those
who put on the uniform, and to their families, to put an end to sexual
harassment and assault in the military, and hold perpetrators
accountable.''

It is, women from the military say, their Black Lives Matter moment.

``We have been swept under the rug so often,'' said Lucy C. Del Gaudio,
who served in the Army between 1990 and 1998, and was assaulted. A
friend called her when the case in Fort Hood came to light and said he
saw her face in the victim. She is now working to gather female
veterans, service members and civilians to push for a deeper
investigation into
\href{https://twitter.com/rosemcgowan/status/1279466856479367175}{Specialist
Guillen}'s killing.

``The way the social media worked in 1992, I didn't have any way to have
proof in the pudding,'' Ms. Del Gaudio said. ``We now have the proof and
means.''

It is rare to speak to a female veteran or current service member who
has not experienced sexual harassment or worse, from the elite military
academies to basic training to the barracks to the highest ranks of
service.

In 2019,
\href{https://www.stripes.com/news/us/pentagon-reports-of-sexual-assault-harassment-in-the-military-have-increased-1.627966}{the
Defense Department found,} there were 7,825 sexual assault reports
involving service members as victims or subjects, a 3 percent increase
over 2018. Reports in which survivors confidentially disclosed an
assault without starting an official investigation rose by 17 percent
increase, to 2,126 reports.

Military culture and its rules make it extremely hard for women to seek
and obtain justice in these cases, or for the military to curb the
ongoing problem of harassment and assault. Ms. Jaslow said that a
culture where ``good order and discipline'' and hierarchy rule, it is
challenging to advocate for accountability. Military women are at once
expected to be tough enough to face down harassment, and blamed for
entering a male-dominated workplace where they have long fought to be
accepted as equals.

``There are reports from Vanessa's family that she was being harassed,
but for some reason she did not feel comfortable making a credible
report,'' said Representative Elaine Luria, Democrat of Virginia, who
spent 20 years in the Navy. ``A lot of women are hesitant to make
reports and don't necessarily feel that when others make reports they
have gotten justice.''

There have been fights on Capitol Hill over changes to the way these
cases are adjudicated. Senator Kirsten Gillibrand, Democrat of New York,
and Representative Jackie Speier, Democrat of California, have
\href{https://www.nytimes.com/2013/06/15/us/politics/in-senate-complex-fight-over-curbing-sexual-military-assaults.html}{repeatedly}tried
to pass legislation that would give military prosecutors --- rather than
commanders --- the power to decide which sexual assaults to try in the
military.

A new twist in the debate is the inclusion of young, female veterans who
attended elite service academies.

Representative Mikie Sherrill, Democrat of New Jersey, and Ms. Luria
this month gave emotional
\href{https://www.dropbox.com/s/6rrnsoupppg981b/NDAA-pt2.mp4?dl=0}{testimony}
during a House Armed Services Committee hearing in support of a
\href{https://docs.house.gov/meetings/AS/AS00/20200701/110784/BILLS-116-HR6395-S001175-Amdt-227r1.pdf}{measure}
to create a pilot Office of the Chief Prosecutor at the academies for
such incidents. Seven Republicans, many of whom were initially
resistant, voted for the amendment to help it pass out of committee.

``I was harassed in the Navy and the academy,'' said Ms. Sherrill, who
went to the Naval Academy and served as a Navy helicopter pilot. ``I
have had so many friends come to me who tried to get justice who could
not get it through the chain of command. I think Vanessa has just sort
of brought so many of those emotions to a head.''

The surge of interest by nonmilitary women in the issue has been one of
the most heartening developments to stem from the tragedy, advocates for
Specialist Guillen say.

``We are now able to relate to the civilian community and say, `Yeah,
guys, this is happening, this has always been happening,''' said Army
Capt. Victoria Kositz, a West Point graduate. ``I see a shift in the
conversation from `You should have known going into the military,' to
`This is an outrage, let's make sure it never happens again.'''

Captain Kositz said that while she was at West Point, a freshman student
was raped in her room and, when it was reported, ``Her cadet chain of
command behind her back said she was a drama queen.'' While serving at
Fort Bragg, N.C., where she was the only woman in her platoon, a group
of sergeants printed out a picture of her in a swimsuit from her
Facebook account and passed it around. One of them cornered her at a
social gathering at a bar and made an obscene remark, she said. She
slapped him, and was given a letter of reprimand that, she said,
``ruined my time at that post.''

The Black Lives Movement, particularly as it pertains to Ms. Taylor, has
provided lift for this moment too.

``The momentum is definitely different this time around,'' said
\href{https://www.servicewomen.org/swan-updates/swan-names-new-ceo/}{Deshauna
Barber}, an Army Reserve captain and chief executive of the Service
Women's Action Network. ``In many ways, it is the Latino community
driving the conversation.''

Specialist Guillen's mother is Spanish-speaking and has used Latino
channels to keep attention on her daughter's case. The League of United
Latin American Citizens on Friday **** met with Army Secretary Ryan D.
McCarthy, who then
\href{https://www.military.com/daily-news/2020/07/10/army-secretary-orders-command-climate-review-amid-vanessa-guillen-investigation.html}{announced}
that he had ordered an independent review of command climate and culture
in the Army in light of the case.

``The military is seen as this very patriotic noble space,'' Ms. Barber
said. ``Well, if we are passionate about the troops, we should be
passionate about Vanessa.''

Women who never felt their cases were taken seriously say this is a
moment they hope will last.

``My voice was not heard,'' said Francine King, who said she was
assaulted while serving in the Army in 2004 at Fort Bragg. ``I would
hope more attention is given now. Before, no one wanted to hear our
cries.''

\includegraphics{https://static01.nyt.com/images/2017/01/29/podcasts/the-daily-album-art/the-daily-album-art-articleInline-v2.jpg?quality=75\&auto=webp\&disable=upscale}

\hypertarget{listen-to-the-daily-a-metoo-moment-in-the-military}{%
\subsubsection{Listen to `The Daily': A \#MeToo Moment in the
Military}\label{listen-to-the-daily-a-metoo-moment-in-the-military}}

What happened to 20-year-old Specialist Vanessa Guillen, and how the
United States responded.

Advertisement

\protect\hyperlink{after-bottom}{Continue reading the main story}

\hypertarget{site-index}{%
\subsection{Site Index}\label{site-index}}

\hypertarget{site-information-navigation}{%
\subsection{Site Information
Navigation}\label{site-information-navigation}}

\begin{itemize}
\tightlist
\item
  \href{https://help.nytimes.com/hc/en-us/articles/115014792127-Copyright-notice}{©~2020~The
  New York Times Company}
\end{itemize}

\begin{itemize}
\tightlist
\item
  \href{https://www.nytco.com/}{NYTCo}
\item
  \href{https://help.nytimes.com/hc/en-us/articles/115015385887-Contact-Us}{Contact
  Us}
\item
  \href{https://www.nytco.com/careers/}{Work with us}
\item
  \href{https://nytmediakit.com/}{Advertise}
\item
  \href{http://www.tbrandstudio.com/}{T Brand Studio}
\item
  \href{https://www.nytimes.com/privacy/cookie-policy\#how-do-i-manage-trackers}{Your
  Ad Choices}
\item
  \href{https://www.nytimes.com/privacy}{Privacy}
\item
  \href{https://help.nytimes.com/hc/en-us/articles/115014893428-Terms-of-service}{Terms
  of Service}
\item
  \href{https://help.nytimes.com/hc/en-us/articles/115014893968-Terms-of-sale}{Terms
  of Sale}
\item
  \href{https://spiderbites.nytimes.com}{Site Map}
\item
  \href{https://help.nytimes.com/hc/en-us}{Help}
\item
  \href{https://www.nytimes.com/subscription?campaignId=37WXW}{Subscriptions}
\end{itemize}
