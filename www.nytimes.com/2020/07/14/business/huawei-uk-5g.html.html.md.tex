Sections

SEARCH

\protect\hyperlink{site-content}{Skip to
content}\protect\hyperlink{site-index}{Skip to site index}

\href{https://www.nytimes.com/section/business}{Business}

\href{https://myaccount.nytimes.com/auth/login?response_type=cookie\&client_id=vi}{}

\href{https://www.nytimes.com/section/todayspaper}{Today's Paper}

\href{/section/business}{Business}\textbar{}U.K. Bars Huawei for 5G as
Tech Battle Between China and the West Escalates

\url{https://nyti.ms/2Zuq5gN}

\begin{itemize}
\item
\item
\item
\item
\item
\item
\end{itemize}

Advertisement

\protect\hyperlink{after-top}{Continue reading the main story}

Supported by

\protect\hyperlink{after-sponsor}{Continue reading the main story}

\hypertarget{uk-bars-huawei-for-5g-as-tech-battle-between-china-and-the-west-escalates}{%
\section{U.K. Bars Huawei for 5G as Tech Battle Between China and the
West
Escalates}\label{uk-bars-huawei-for-5g-as-tech-battle-between-china-and-the-west-escalates}}

Banning the use of the Chinese tech giant's equipment in high-speed
wireless infrastructure is a major reversal by Prime Minister Boris
Johnson --- and a big victory for the Trump administration.

\includegraphics{https://static01.nyt.com/images/2020/07/14/business/14ukhuawei1/14ukhuawei1-videoSixteenByNine3000.jpg}

By \href{https://www.nytimes.com/by/adam-satariano}{Adam Satariano},
\href{https://www.nytimes.com/by/stephen-castle}{Stephen Castle} and
\href{https://www.nytimes.com/by/david-e-sanger}{David E. Sanger}

\begin{itemize}
\item
  July 14, 2020
\item
  \begin{itemize}
  \item
  \item
  \item
  \item
  \item
  \item
  \end{itemize}
\end{itemize}

\href{https://cn.nytimes.com/business/20200715/huawei-uk-5g/}{阅读简体中文版}\href{https://cn.nytimes.com/business/20200715/huawei-uk-5g/zh-hant/}{閱讀繁體中文版}

LONDON --- Britain announced on Tuesday that it would ban equipment from
the Chinese technology giant Huawei from the country's high-speed
wireless network, a victory for the Trump administration that escalates
the battle between Western powers and China over critical technology.

The move reverses a decision in January, when
\href{https://www.nytimes.com/2020/01/28/technology/britain-huawei-5G.html}{Britain
said Huawei equipment} could be used in its new 5G network on a limited
basis. Since then, Prime Minister Boris Johnson has faced growing
political pressure domestically to take a harder line against Beijing,
and in May, the United States imposed
\href{https://www.nytimes.com/2020/05/15/business/economy/commerce-department-huawei.html}{new
restrictions} to disrupt Huawei's access to important components.

Britain's about-face signals a new willingness among Western countries
to
\href{https://www.nytimes.com/2020/06/29/world/asia/china-hong-kong-security-law-rules.html}{confront
China}, a determination that has grown firmer since Beijing last month
adopted a sweeping law to tighten its grip on Hong Kong, the
semiautonomous city that was a British colony until 1997. On Tuesday,
Robert O'Brien, President Trump's national security adviser, was in
Paris for meetings about China with counterparts from Britain, France,
Germany and Italy.

Huawei's critics say its close ties to the Chinese government mean
Beijing could use the equipment for espionage or to disrupt
telecommunications --- a point the company strongly disputes.

Arguing that Huawei created too much risk for such a critical,
multibillion-dollar project, the British government said Tuesday that it
would ban the purchase of new Huawei equipment for 5G networks after
December, and that existing gear already installed would need to be
removed from the networks by 2027.

``As facts have changed, so has our approach,'' Oliver Dowden, the
government minister in charge of telecommunications, told the House of
Commons on Tuesday afternoon. ``This has not been an easy decision, but
it is the right one for the U.K.'s telecoms networks, for our national
security and our economy, both now and indeed in the long run.''

After the British government announced its decision, President Trump
took aim at Huawei during a news conference at the White House, saying
the United States has ``confronted untrustworthy Chinese technology and
telecom providers.''

``We convinced many countries, and I did this myself for the most part,
not to use Huawei, because we think it's an unsafe security risk,'' Mr.
Trump said.

Mr. Trump also announced that he was issuing an executive order
formalizing a declaration from late May that the United States would
treat Hong Kong in the same manner as China rather than as a
semiautonomous territory and would impose the same tariffs that it
applies to China. He also said he was signing into law a bipartisan
congressional bill that encourages sanctions against Chinese officials
and entities that take part in the repression of Hong Kong, as well as
financial institutions that do business with those parties.

The dispute over ** Huawei, the world's largest maker of
telecommunications equipment, is an early front in a new tech cold war,
with ramifications for internet freedom and surveillance, as well as
emerging technologies like artificial intelligence and robotics.

``The democratic West has woken up late to its overdependence on a
country whose values are diametrically opposed to it,'' said Robert
Hannigan, a former head of the British digital surveillance agency GCHQ,
who is now an executive at the cybersecurity firm BlueVoyant. ``Huawei
and other Chinese companies present a real cybersecurity risk, but the
primary threat comes from the intent of the Chinese Communist Party, as
we see in Hong Kong.''

Huawei described the announcement on Tuesday as a disappointment and
``bad news for anyone in the U.K. with a mobile phone.''

``It threatens to move Britain into the digital slow lane,'' said Ed
Brewster, a spokesman for Huawei U.K. ``Regrettably our future in the
U.K. has become politicized; this is about U.S. trade policy and not
security. ''

Until the latest turn of events, Britain had been welcoming of Huawei.
In 2005, it was the first country to offer Huawei a
\href{https://www.nytimes.com/2019/01/22/technology/huawei-europe-china.html}{foothold
in Europe}, now the company's largest market outside China. Huawei
financed university research and a charity started by Prince Charles.
And just last month,
\href{https://www.huawei.com/en/news/2020/6/huawei-optoelectronics-rd-manufacturing-centre-cambridge}{Huawei
announced plans}to spend 1 billion pounds (about \$1.25 billion) on a
new research center in Cambridge.

The British experience shows the challenges nations face navigating the
United States-China rift. In moving forward with the ban, Britain risks
retaliation from China, one of its largest and fastest-growing trading
partners, when it is trying to craft a more
\href{https://www.nytimes.com/2020/07/03/world/europe/johnson-brexit-hong-kong.html}{open
trade policy}outside the European Union. China's ambassador in London,
Liu Xiaoming, recently warned that Britain would ``bear the
consequences'' of treating China with hostility.

``The Huawei issue is the first of many complicated decisions we're
going to have about striking the right balance between our commercial
and economic engagement with China, and our security concerns about how
China uses its power,'' said John Sawers, a former chief of the British
intelligence service MI6.

Huawei is the leading provider for towers, masts and other critical
equipment needed to build new wireless networks based on
fifth-generation wireless technology, known as 5G.

New 5G networks are seen as essential infrastructure in an increasingly
digital global economy. The networks will provide faster download speeds
for phone users, but offer even more important potential for commercial
applications in industries such as manufacturing, health care and
transportation.

\includegraphics{https://static01.nyt.com/images/2020/07/14/business/14ukhuawei2/merlin_168094122_c9b26ed6-454f-4f12-b04a-21b9d790133e-articleLarge.jpg?quality=75\&auto=webp\&disable=upscale}

Huawei's technological dominance in this field is viewed as a failure of
industrial policy in the West. American authorities have spent more than
a year pressuring allies to keep Huawei out of communications networks,
warning that the company is a proxy for Beijing and a threat to national
security. The Trump administration encouraged the use of other telecom
equipment makers, including Sweden's Ericsson and Finland's Nokia.

At first, countries were resistant, unconvinced that Huawei posed a
grave risk. Britain argued that it had a security system in place to
ensure all Huawei equipment was reviewed before being put inside its
communications networks. The announcement in January stipulated Huawei
would be limited to ``noncore'' parts of the network.

A turning point came in May, when the Trump administration announced a
rule that would bar Huawei and its suppliers from using American
technology and software. The decision, slated to take effect in
September, could throw Huawei's supply chain into chaos.

In Britain, the American announcement added to pressure Mr. Johnson
faced from members of his own Conservative Party to take a harder line
against China, especially after the events in Hong Kong. The government
announced a review of its January decision after the American
punishments were announced.

``American sanctions left the U.K. with little choice,'' said Priya
Guha, a former British diplomat who represented the country's interests
in Silicon Valley. ``There was a bit of checkmate by the U.S.''

The Trump administration has taken other steps, some conducted with
little publicity, to undercut China's position in communications
networks.

The U.S. government on Tuesday published an interim rule that will bar
Pentagon and NASA contractors from using technology from Huawei and
other Chinese telecommunications companies. Some government contractors
say the ban, passed into law two years ago, is too onerous, and the
administration estimates it will cost some \$12 billion.

In April, a group of agencies that calls itself Team Telecom, led by the
Justice Department, moved to remove China Telecom, another big wireless
company, from its operations inside the United States. It has long been
operating ``points of presence'' in U.S. networks that help maintain
internet connections. In a series of classified briefings, American
intelligence agencies accused it of experimenting with rerouting
American traffic through China --- though the purpose of that rerouting
was unclear.

The same group moved last month to block the
\href{https://pldcglobal.com/}{Pacific Light Cable Network}--- a
partnership involving Facebook and Google among others --- from
operating an undersea cable linking Hong Kong and the United States, in
what was supposed to be the highest-capacity undersea Pacific connection
for internet traffic.

The Trump administration asked the Federal Communications Commission to
block the connection in Hong Kong, citing concern it ``would expose U.S.
communications traffic to collection'' by China, through a Chinese firm
operating where the cable landed. Instead, it wants the commission to
approve only direct connections to Taiwan and the Philippines,
undercutting China's effort to make Hong Kong a key data transfer hub.
It cited the new national security law for Hong Kong, which at the time
was still being drafted.

But it remains unclear if the steps involving Huawei and others will
achieve Washington's objective. Chinese firms will still control much of
Asia's traffic, and that means calls, data and searches will still move
through Chinese switching systems. At best, the U.S. moves can make it
harder for China's leaders to cut off communications in times of
conflict. But it cannot protect the United States from what Sue Gordon,
the former deputy director of national intelligence, called the process
of ``living in a dirty network.''

Still, Robert B. Blair, a senior Commerce Department official who until
recently served as the Trump White House's chief telecommunications
adviser, told a meeting of the Council on Foreign Relations on Tuesday
that ``we scored a major victory'' with Britain's decision.

In Britain, officials warned its ban would add significant costs, and
delay the rollout of 5G by around two years. The new 5G wireless systems
must be built atop existing networks that Huawei had a major role in
constructing. In setting a 2027 deadline, the British government said
moving any faster to remove Huawei gear would produce a greater risk to
the security and resilience of the network.

The ban does not apply to smartphones and other consumer products made
by Huawei, or equipment used in 2G, 3G and 4G networks.

Many see the Huawei dispute as foreshadowing future conflicts, with
other prominent companies becoming entangled. Secretary of State Mike
Pompeo said the United States was considering actions against Chinese
apps, including the hugely popular social media service TikTok, which is
owned by a Chinese internet company.

Britain's decision to bar Huawei will put pressure on other European
countries. In Germany, Chancellor Angela Merkel is being urged to
\href{https://www.nytimes.com/2020/01/16/world/europe/huawei-germany-china-5g-automakers.html}{keep
the company out} of a new 5G network, but is weighing the economic
fallout for German automakers, for whom China is a critical market.

``If Huawei is stopped in its tracks, that does represent a very
important inflection point for China's ability to achieve its
objectives,'' said Nigel Inkster, a senior adviser at the International
Institute for Strategic Studies in London who has
\href{https://www.hurstpublishers.com/book/the-great-decoupling/}{written
a book} on the technology battle between the United States and China.

Mr. Inkster warned that the West risks provoking China if it feels more
economically isolated. ``There is a serious need to think hard and
deeply about whether it is realistic to disengage from China totally in
these areas,'' he said.

Julian E. Barnes and Edward Wong contributed reporting.

Advertisement

\protect\hyperlink{after-bottom}{Continue reading the main story}

\hypertarget{site-index}{%
\subsection{Site Index}\label{site-index}}

\hypertarget{site-information-navigation}{%
\subsection{Site Information
Navigation}\label{site-information-navigation}}

\begin{itemize}
\tightlist
\item
  \href{https://help.nytimes.com/hc/en-us/articles/115014792127-Copyright-notice}{©~2020~The
  New York Times Company}
\end{itemize}

\begin{itemize}
\tightlist
\item
  \href{https://www.nytco.com/}{NYTCo}
\item
  \href{https://help.nytimes.com/hc/en-us/articles/115015385887-Contact-Us}{Contact
  Us}
\item
  \href{https://www.nytco.com/careers/}{Work with us}
\item
  \href{https://nytmediakit.com/}{Advertise}
\item
  \href{http://www.tbrandstudio.com/}{T Brand Studio}
\item
  \href{https://www.nytimes.com/privacy/cookie-policy\#how-do-i-manage-trackers}{Your
  Ad Choices}
\item
  \href{https://www.nytimes.com/privacy}{Privacy}
\item
  \href{https://help.nytimes.com/hc/en-us/articles/115014893428-Terms-of-service}{Terms
  of Service}
\item
  \href{https://help.nytimes.com/hc/en-us/articles/115014893968-Terms-of-sale}{Terms
  of Sale}
\item
  \href{https://spiderbites.nytimes.com}{Site Map}
\item
  \href{https://help.nytimes.com/hc/en-us}{Help}
\item
  \href{https://www.nytimes.com/subscription?campaignId=37WXW}{Subscriptions}
\end{itemize}
