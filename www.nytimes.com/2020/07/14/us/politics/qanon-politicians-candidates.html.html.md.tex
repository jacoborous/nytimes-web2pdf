Sections

SEARCH

\protect\hyperlink{site-content}{Skip to
content}\protect\hyperlink{site-index}{Skip to site index}

\href{https://www.nytimes.com/section/politics}{Politics}

\href{https://myaccount.nytimes.com/auth/login?response_type=cookie\&client_id=vi}{}

\href{https://www.nytimes.com/section/todayspaper}{Today's Paper}

\href{/section/politics}{Politics}\textbar{}The QAnon Candidates Are
Here. Trump Has Paved Their Way.

\url{https://nyti.ms/3evRNxM}

\begin{itemize}
\item
\item
\item
\item
\item
\item
\end{itemize}

Advertisement

\protect\hyperlink{after-top}{Continue reading the main story}

Supported by

\protect\hyperlink{after-sponsor}{Continue reading the main story}

\hypertarget{the-qanon-candidates-are-here-trump-has-paved-their-way}{%
\section{The QAnon Candidates Are Here. Trump Has Paved Their
Way.}\label{the-qanon-candidates-are-here-trump-has-paved-their-way}}

The conspiracy theorists accuse Democrats and even fellow Republicans of
being beholden to a cabal of bureaucrats, pedophiles and Satanists.
President Trump has cheered them on.

\includegraphics{https://static01.nyt.com/images/2020/07/13/us/politics/00qanon-candidates1/merlin_173640276_e6359af2-2468-4ad0-9adf-251c995c5c17-articleLarge.jpg?quality=75\&auto=webp\&disable=upscale}

By \href{https://www.nytimes.com/by/matthew-rosenberg}{Matthew
Rosenberg} and
\href{https://www.nytimes.com/by/jennifer-steinhauer}{Jennifer
Steinhauer}

\begin{itemize}
\item
  July 14, 2020
\item
  \begin{itemize}
  \item
  \item
  \item
  \item
  \item
  \item
  \end{itemize}
\end{itemize}

A Republican Senate candidate recently declared herself ``one of the
thousands of digital soldiers'' in service of QAnon, a
\href{https://www.nytimes.com/2018/08/01/us/politics/what-is-qanon.html}{convoluted
pro-Trump conspiracy theory} about a ``deep state'' of child-molesting
Satanist traitors plotting against the president. A congressional
candidate in Colorado who
\href{https://www.nytimes.com/2020/06/30/us/lauren-boebert-colorado.html}{made
approving comments about QAnon} bested a five-term Republican incumbent
in a primary last month.

And then there is Marjorie Taylor Greene, a Georgia Republican who is
perhaps the most unabashedly pro-QAnon candidate for Congress and has
drawn a positive tweet from
\href{https://www.nytimes.com/interactive/2020/us/elections/donald-trump.html}{President
Trump}. She recently declared that QAnon was ``a once-in-a-lifetime
opportunity to take this global cabal of Satan-worshiping pedophiles
out.''

More than two years after QAnon, which the F.B.I. has labeled a
potential domestic terrorism threat, emerged from
\href{https://www.nytimes.com/2018/08/01/us/politics/what-is-qanon.html}{the
troll-infested corners of the internet}, the movement's supporters are
morphing from keyboard warriors into political candidates. They have
been urged on by Mr. Trump, whose own espousal of conspiracy theories
and continual railing against the political establishment have cleared a
path for QAnon candidates.

And even as party leaders publicly distance themselves from the
movement, they are quietly supporting some QAnon-linked candidates ---
demonstrating the thin line they are trying to walk between radical
elements among their base and the moderate voters they need to win over.

Precisely how many candidates are running under the banner of QAnon is
somewhat open to interpretation --- estimates range to more than a
dozen, with many more defeated in primaries --- and nearly all are
expected to lose in November. Some candidates have clear connections to
the movement and use its language and hashtags on social media and in
real-world appearances.

Scores more have cherry-picked
\href{https://www.nytimes.com/2020/02/09/us/politics/qanon-trump-conspiracy-theory.html}{some
of the movement's themes}, such as claims that Jews, and especially the
financier George Soros, are controlling the political system and
vaccines; assertions that the risk from the coronavirus is vastly
overstated; or racist theories about former President Barack Obama. Many
have appeared on QAnon-themed podcasts and in news outlets. On Monday
Jeff Sessions, caught in a tight race to reclaim his former Senate seat
in Alabama, recycled an old QAnon meme about himself in a Twitter post.

All of the candidates, though, present a fresh headache for Republican
leaders. They were already struggling to distance the party from
conspiracy theories steeped in racist and anti-Semitic messaging. Now
they must contend with candidates whose online beliefs have inspired
real-world violence, including
\href{https://www.nytimes.com/2019/07/21/nyregion/gambino-shooting-anthony-comello-frank-cali.html}{the
killing of a mob boss}.

It is a development that threatens to further alienate the kinds of
traditional Republican voters who typically care about lowering taxes,
not chasing imaginary Satanists from the government. Democrats are eager
to pounce.

``We will point it out loudly and clearly,'' said Representative Cheri
Bustos of Illinois, who leads the Democratic Congressional Campaign
Committee. ``The moral of the story is the Republican Party is silent on
all of this.''

Yet Republican leaders also cannot afford to turn off voters who share
those conspiratorial views if they hope to retain the Senate and retake
the House. So while the party has publicly sought to keep its distance
from most QAnon candidates, campaign finance filings show that some have
clearly won its tacit backing.

In April, Representative Jim Jordan of Ohio, a high-profile lawmaker and
a favorite of the president, donated \$2,000 to Ms. Greene's campaign. A
political action committee with which Mr. Jordan is associated, the
House Freedom Fund, gave her thousands of dollars more.

\includegraphics{https://static01.nyt.com/images/2020/07/13/us/politics/00qanon-candidates2/merlin_169672641_c2c4465b-bc1d-45d0-a9d7-08db4e8f64aa-articleLarge.jpg?quality=75\&auto=webp\&disable=upscale}

A month earlier, the Republican National Committee gave \$2,200 to
Angela Stanton-King, a House candidate in Georgia who has
\href{https://www.instagram.com/tv/B930CqOgc9I/?utm_source=ig_web_copy_link}{repeatedly
posted QAnon content and obscure hashtags}, such as ``\#trusttheplan.''
The Georgia Republican Party gave an additional \$2,800 to Ms.
Stanton-King, who
\href{https://www.nytimes.com/2020/02/18/us/politics/trump-pardons.html}{was
pardoned this year} by Mr. Trump for her role in a car-theft ring. She
is expected to be roundly defeated in her heavily Democratic district.

Ms. Stanton-King has since denied believing in any QAnon conspiracies.
Yet in recent days she was again tweeting about
``\href{https://twitter.com/theangiestanton/status/1281894309093167106?s=20}{global
elite pedophiles},'' as well as a new conspiracy theory involving a
purported
\href{https://twitter.com/theangiestanton/status/1281799576350076930?s=20}{child-trafficking
ring} run by an online furniture retailer.

Few of the QAnon candidates appear to share any formal ties with one
another, beyond mostly being Republicans. But as they move onto ballots
this fall, the candidates and their fellow travelers are increasingly
taking on the trappings of a discrete political movement, though one
with incoherent ideas whose adherents typically focus on wild
accusations, not policy changes.

In recent weeks QAnon followers, including a Republican Senate
candidate, have begun to publicly pledge allegiance to the movement,
posting videos of themselves reciting what they are calling the digital
soldier oath. On social media, where the conspiracy theory first took
root, QAnon candidates and followers often amplify one another.

A favored topic of the candidates on social media is Mr. Trump. From
February to June, QAnon candidates quoted, retweeted or replied to Mr.
Trump roughly 2,000 times.

In many instances, they sought to spread a core tenet of the QAnon
conspiracy: that Mr. Trump, backed by the military, ran for office to
save Americans from a so-called deep state filled with child-abusing,
devil-worshiping bureaucrats. Backing the president's enemies are
prominent Democrats who, in some telling, extract hormones from
children's blood.

The president, for his part, has
\href{https://www.politico.com/news/2020/07/12/trump-tweeting-qanon-followers-357238}{repeatedly
retweeted QAnon supporters}, and cheered on candidates who openly
support the conspiracy theory, such as Ms. Greene of Georgia.

``A big winner. Congratulations!''
\href{https://twitter.com/realDonaldTrump/status/1271428819296157697?s=20}{Mr.
Trump tweeted} after Ms. Greene, whose ads have been banned by Facebook
for violating the platform's terms of service, placed first in the
Republican primary in a deeply conservative corner of northwestern
Georgia. But she failed to clear the 50 percent mark and is now the
favorite in a
\href{https://www.nytimes.com/2020/06/17/us/marjorie-taylor-greene-georgia.html}{runoff
election} for the Republican nomination in district long held by the
party.

The movement defies easy political labels, and its adherents include a
smattering of Democrats and independents. Mostly, what unites it is a
hatred of the establishment.

``It's not like a QAnon supporter went down a path where they got into
George Bush and then started to read Ronald Reagan's speeches, and then
bought Milton Friedman's `Capitalism and Freedom,' and then believed in
satanic baby eaters,'' said Joseph Uscinski, a professor at the
University of Miami \href{https://www.joeuscinski.com/}{who studies
fringe groups}. ``It doesn't work like that.''

Mr. Trump ``won by saying that he wanted to drain the swamp,'' Dr.
Uscinski said. ``By doing that, he essentially built a coalition of
people with anti-establishment views.'' Those who believe in QAnon, the
professor added, ``are probably the most extreme part of that
coalition.''

In Western Colorado late last month, Lauren Boebert, a gun-rights
activist who has made approving comments about QAnon, beat a five-term
Republican incumbent and will now defend the sprawling district in
November. In recent weeks she told the
\href{https://www.washingtonpost.com/politics/gun-rights-activist-defeats-five-term-gop-congressman-in-colorado-primary/2020/06/30/2ffb8736-bb3d-11ea-80b9-40ece9a701dc_story.html?itid=lk_inline_manual_8}{QAnon-aligned
web show ``Steel Truth}'' that ``everything I've heard of Q --- I hope
this is real.''

In a recent interview, Ms. Boebert said she was not a follower of the
group. But, she added, ``I don't believe that's a radical notion to want
to get rid of people trying to undermine the president of the United
States.''

In Southern California, Mike Cargile, who is challenging an incumbent
Democrat for a House seat, includes \#WWG1WGA in his Twitter bio, a
shortened version of the QAnon motto ``Where We Go One We Go All.'' He
has repeated many of the group's
\href{https://www.lamag.com/citythinkblog/mike-cargile-qanon/}{racist
theories about Mr. Obama} and Black Americans.

In an emailed response to questions, Mr. Cargile said that he sought
only to discover the truth and that Americans needed to resist
``Marxists' efforts to deceive and divide.''

He said ``we'll see'' what becomes of the QAnon theories. But, he added,
all Americans should be alarmed by the efforts of the president's
opponents in Washington, ``and even more so when we discover that the
saboteurs and propagators are the very men and women tasked with
safeguarding our system of Justice.''

In Oregon, the Republicans'
\href{https://www.nytimes.com/2020/05/20/us/oregon-senate-perkins-qanon.html}{long-shot
Senate candidate}, Jo Rae Perkins,
\href{https://www.opb.org/news/article/jo-rae-perkins-qanon-oregon-republican-senate-nominee/}{posted
a video in May} declaring, ``I stand with Q and the team.''

She followed up with another video in late June in which she faced the
camera and took the QAnon digital soldier oath. The oath is lifted from
the pledge taken by senators at their swearing-in, with one small
addition tacked on at the end, the letters ``WWG1WGA.''

Though the precise origins of the oath are murky, it spread from
hard-core QAnon followers and into Republican ranks in a matter of
weeks, illustrating how adherents of the conspiracy have enmeshed
themselves --- and their theories --- in conservative circles.

There appear to be vague references to the oath on social media and
internet message boards going back to early June. But it took off on
June 24 after a so-called Q drop --- that is, a post by the person
purporting to be Q, the originator of the movement who claims to be a
high-ranking official with access to top-secret information. The post
was on 8kun, a new message board that has quickly become a home for all
flavors of conspiracy theorists and extremists, especially QAnon
followers.

Under the subject line ``Digital Soldiers: Take the Oath and Serve Your
Country,'' the user laid out the text of the oath. The user then added:
``Take the oath. Mission forward. Q.''

It quickly gained traction outside QAnon circles. Among the most recent
people to take the oath was Michael T. Flynn, the president's first
national security adviser, who is expected to soon begin campaigning for
Mr. Trump. He posted
\href{https://www.cnn.com/2020/07/07/politics/michael-flynn-qanon-video/index.html}{a
video on Twitter over the July 4 weekend}with guests reciting the oath
and intoning the phrase ``where we go one, we go all.''

His lawyer said Mr. Flynn, whose case on a charge of lying to the F.B.I.
\href{https://www.nytimes.com/2020/07/09/us/politics/michael-flynn-appeals-court.html?referringSource=articleShare}{remains
in limbo}, was interpreting the works of a 16th-century poet, though she
did not specify which bard he was referring to. Soon after the tweet,
Mr. Flynn made his Twitter account private, limiting who could see the
video.

How far QAnon candidates can go remains an open question; the vast
majority of Republican voters have shown little inclination to buy into
the movement's wildest claims. Yet some of its themes are now a regular
feature of conservative political discourse, and even candidates who
only espouse parts of QAnon's racist, anti-Semitic and violent
conspiracies could pose real threats if elected.

``It is really more like flat-earth adherents who have a different way
to interpret the world, which colors everything they see,'' said Alice
E. Marwick, a **** principal researcher at the Center for Information,
Technology and Public Life at the University of North Carolina, Chapel
Hill.

No matter how many of the candidates win, their mere presence on the
political scene is helping to further spread a conspiracy that, at its
core, sees the government as a dangerous enemy.

The latest example of how deeply QAnon themes have become embedded in
Republican politics came on Monday when Mr. Sessions, the former
attorney general who is running to reclaim his former Senate seat in
Alabama, recycled an old QAnon meme about himself, ``Sessions
Activated.''

The ``Sessions Activated'' meme first became popular in 2018 when Mr.
Sessions was still attorney general, and QAnon followers thought he was
going to lead the prosecutions of deep-state bureaucrats and their
Democratic backers. But after he resigned later that year, the meme
faded away.

Mr. Sessions grabbed the meme off the shelf this week, tweeting it just
before he faced a primary election against an opponent who has led in
polls. His post has since been retweeted nearly 9,000 times.

Ben Decker contributed reporting.

\emph{Follow Matthew Rosenberg on Twitter}
\href{https://twitter.com/allmattnyt}{\emph{@AllMattNYT}}\emph{.}

Advertisement

\protect\hyperlink{after-bottom}{Continue reading the main story}

\hypertarget{site-index}{%
\subsection{Site Index}\label{site-index}}

\hypertarget{site-information-navigation}{%
\subsection{Site Information
Navigation}\label{site-information-navigation}}

\begin{itemize}
\tightlist
\item
  \href{https://help.nytimes.com/hc/en-us/articles/115014792127-Copyright-notice}{©~2020~The
  New York Times Company}
\end{itemize}

\begin{itemize}
\tightlist
\item
  \href{https://www.nytco.com/}{NYTCo}
\item
  \href{https://help.nytimes.com/hc/en-us/articles/115015385887-Contact-Us}{Contact
  Us}
\item
  \href{https://www.nytco.com/careers/}{Work with us}
\item
  \href{https://nytmediakit.com/}{Advertise}
\item
  \href{http://www.tbrandstudio.com/}{T Brand Studio}
\item
  \href{https://www.nytimes.com/privacy/cookie-policy\#how-do-i-manage-trackers}{Your
  Ad Choices}
\item
  \href{https://www.nytimes.com/privacy}{Privacy}
\item
  \href{https://help.nytimes.com/hc/en-us/articles/115014893428-Terms-of-service}{Terms
  of Service}
\item
  \href{https://help.nytimes.com/hc/en-us/articles/115014893968-Terms-of-sale}{Terms
  of Sale}
\item
  \href{https://spiderbites.nytimes.com}{Site Map}
\item
  \href{https://help.nytimes.com/hc/en-us}{Help}
\item
  \href{https://www.nytimes.com/subscription?campaignId=37WXW}{Subscriptions}
\end{itemize}
