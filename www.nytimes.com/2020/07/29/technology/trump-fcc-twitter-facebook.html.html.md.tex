Sections

SEARCH

\protect\hyperlink{site-content}{Skip to
content}\protect\hyperlink{site-index}{Skip to site index}

\href{https://www.nytimes.com/section/technology}{Technology}

\href{https://myaccount.nytimes.com/auth/login?response_type=cookie\&client_id=vi}{}

\href{https://www.nytimes.com/section/todayspaper}{Today's Paper}

\href{/section/technology}{Technology}\textbar{}Commerce Department Asks
F.C.C. to Narrow Protections for Web Platforms

\url{https://nyti.ms/2DcOLBE}

\begin{itemize}
\item
\item
\item
\item
\item
\end{itemize}

Advertisement

\protect\hyperlink{after-top}{Continue reading the main story}

Supported by

\protect\hyperlink{after-sponsor}{Continue reading the main story}

\hypertarget{commerce-department-asks-fcc-to-narrow-protections-for-web-platforms}{%
\section{Commerce Department Asks F.C.C. to Narrow Protections for Web
Platforms}\label{commerce-department-asks-fcc-to-narrow-protections-for-web-platforms}}

The request comes as President Trump continues to accuse tech platforms
like Facebook and Twitter of suppressing conservative content.

\includegraphics{https://static01.nyt.com/images/2020/07/29/business/29DC-TrumpCensor/merlin_172932390_d5abf51f-906e-4707-872e-6e1be2d64d43-articleLarge.jpg?quality=75\&auto=webp\&disable=upscale}

By \href{https://www.nytimes.com/by/david-mccabe}{David McCabe}

\begin{itemize}
\item
  July 29, 2020
\item
  \begin{itemize}
  \item
  \item
  \item
  \item
  \item
  \end{itemize}
\end{itemize}

WASHINGTON --- The Trump administration asked the Federal Communications
Commission this week to narrow its interpretation of a law that shields
internet platforms like Facebook and YouTube from certain lawsuits over
the content they host.

The request, which stems from an executive order President Trump signed
in May, is part of a growing push by the president and his allies, who
say that tech companies are removing or suppressing conservative
content. Despite evidence that conservative sites and figures perform
well online, the president, along with much of his conservative base,
have repeatedly criticized the platforms over instances in which
conservative content was removed or otherwise moderated for violating a
platform's rules.

In a petition on Monday, the Department of Commerce asked the commission
to clarify that the law, known as Section 230, does not protect a
platform when it moderates or highlights user content based on a
``reasonably discernible viewpoint or message, without having been
prompted to, asked to, or searched for by the user.'' It would also
limit the circumstances under which platforms are protected from
liability over their users' content.

Kayleigh McEnany, the White House spokeswoman, said in a statement on
Wednesday morning that the president wants the F.C.C. ``to clarify that
Section 230 does not permit social media companies that alter or
editorialize users' speech to escape civil liability.''

The petition is now in the hands of the F.C.C., an independent agency
currently led by a Republican chairman, Ajit Pai, who was appointed to
the position by Mr. Trump. ``The F.C.C. will carefully review the
petition,'' said Brian Hart, a spokesman for the commission.

The request comes as the chief executives of Google, Facebook, Amazon
and
Apple\href{https://www.nytimes.com/2020/07/28/technology/amazon-apple-facebook-google-antitrust-hearing.html}{are
scheduled to testify} before the House Judiciary Committee on Wednesday.
While the hearing is part of the panel's ongoing antitrust investigation
into the companies, it is likely that some Republicans on the committee
will ask the executives about how their platforms treat right-leaning
content.

Mr. Trump's petition is the latest twist in an
\href{https://www.nytimes.com/2020/06/17/technology/justice-dept-urges-rolling-back-legal-shield-for-tech-companies.html}{ongoing
debate}in Washington over the future of
\href{https://www.nytimes.com/2020/05/28/business/section-230-internet-speech.html}{Section
230}, a provision of the Communications Decency Act which has long
protected platforms from certain types of lawsuits over user-generated
content. It also protects platforms from being sued over how they
moderated content they find objectionable.

In 2018, lawmakers approved a measure
\href{https://www.nytimes.com/2019/12/17/technology/fosta-sex-trafficking-law.html}{eliminating
the liability shield} in cases where a platform knowingly facilitated
sex trafficking. They have proposed other modifications to the
protections in recent years but have not passed any of them.

Mr. Trump
\href{https://www.nytimes.com/2020/05/28/us/politics/trump-order-social-media.html}{signed
the executive order} days after Twitter added information to refute the
inaccuracies in two of Mr. Trump's posts for the first time. Experts
have said that the order, which also asked the government to review its
spending on social media advertisements, would be difficult to enforce.

Internet companies and their allies in Washington have criticized the
order, saying that it would gut a crucial protection for speech online.
Twitter said in May that it was ``a reactionary and politicized approach
to a landmark law.'' The companies also argue that changing the law
could make it harder for them to moderate concerning content.

In June, the Center for Democracy and Technology filed a lawsuit against
the executive order, arguing it violated the First Amendment. Emma
Llansó, the director of the organization's Free Expression Project, said
that the petition filed Monday was ``simply the next egregious step in
the President's unconstitutional campaign to intimidate social media
platforms that are responding to hate speech and voter suppression
online.''

Advertisement

\protect\hyperlink{after-bottom}{Continue reading the main story}

\hypertarget{site-index}{%
\subsection{Site Index}\label{site-index}}

\hypertarget{site-information-navigation}{%
\subsection{Site Information
Navigation}\label{site-information-navigation}}

\begin{itemize}
\tightlist
\item
  \href{https://help.nytimes.com/hc/en-us/articles/115014792127-Copyright-notice}{©~2020~The
  New York Times Company}
\end{itemize}

\begin{itemize}
\tightlist
\item
  \href{https://www.nytco.com/}{NYTCo}
\item
  \href{https://help.nytimes.com/hc/en-us/articles/115015385887-Contact-Us}{Contact
  Us}
\item
  \href{https://www.nytco.com/careers/}{Work with us}
\item
  \href{https://nytmediakit.com/}{Advertise}
\item
  \href{http://www.tbrandstudio.com/}{T Brand Studio}
\item
  \href{https://www.nytimes.com/privacy/cookie-policy\#how-do-i-manage-trackers}{Your
  Ad Choices}
\item
  \href{https://www.nytimes.com/privacy}{Privacy}
\item
  \href{https://help.nytimes.com/hc/en-us/articles/115014893428-Terms-of-service}{Terms
  of Service}
\item
  \href{https://help.nytimes.com/hc/en-us/articles/115014893968-Terms-of-sale}{Terms
  of Sale}
\item
  \href{https://spiderbites.nytimes.com}{Site Map}
\item
  \href{https://help.nytimes.com/hc/en-us}{Help}
\item
  \href{https://www.nytimes.com/subscription?campaignId=37WXW}{Subscriptions}
\end{itemize}
