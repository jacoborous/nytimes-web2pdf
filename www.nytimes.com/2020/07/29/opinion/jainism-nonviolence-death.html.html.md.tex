\href{/section/opinion}{Opinion}\textbar{}Don't Fear Dying. Fear
Violence.

\href{https://nyti.ms/2P7mhvY}{https://nyti.ms/2P7mhvY}

\begin{itemize}
\item
\item
\item
\item
\item
\end{itemize}

\includegraphics{https://static01.nyt.com/images/2020/07/29/opinion/29stone2/merlin_174744291_46d52d68-6aef-4fdf-8bf1-82a16327ce3e-articleLarge.jpg?quality=75\&auto=webp\&disable=upscale}

Sections

\protect\hyperlink{site-content}{Skip to
content}\protect\hyperlink{site-index}{Skip to site index}

\href{/section/opinion}{Opinion}

\hypertarget{dont-fear-dying-fear-violence}{%
\section{Don't Fear Dying. Fear
Violence.}\label{dont-fear-dying-fear-violence}}

Why do millions of practitioners of the Jain religion strive to avoid
harming even microscopic creatures?

Credit...Devin Oktar Yalkin for The New York Times

Supported by

\protect\hyperlink{after-sponsor}{Continue reading the main story}

By George Yancy

Mr. Yancy is a philosopher, teacher and author.

\begin{itemize}
\item
  July 29, 2020
\item
  \begin{itemize}
  \item
  \item
  \item
  \item
  \item
  \end{itemize}
\end{itemize}

\emph{This month's conversation in our series exploring religion and
death is with Pankaj Jain (@ProfPankajJain), an associate professor in
the department of philosophy and religion at the University of North
Texas. He is the author of
``}\href{https://www.routledge.com/Dharma-in-America-A-Short-History-of-Hindu-Jain-Diaspora/Jain/p/book/9781138565456}{\emph{Dharma
in America: A Short History of Hindu-Jain Diaspora}}\emph{,''
``}\href{http://www.amazon.com/Dharma-Ecology-Hindu-Communities-Sustainability/dp/1409405915/ref=sr_1_1?ie=UTF8\&s=books\&qid=1302198725\&sr=8-1}{\emph{Dharma
and Ecology of Hindu Communities: Sustenance and
Sustainability}}\emph{'' and other works. He is currently translating
``Jain Darshan,'' a text on the philosophy of Jainism. In 2019, he
appeared in the TV series
``}\href{https://www.netflix.com/watch/80178897?source=35}{\emph{The
Story of God}}\emph{'' with the actor Morgan Freeman.}

\textbf{George Yancy:} Most Westerners know the central figures of
certain religions like the Buddha and Jesus. But less is known about
Mahavira, who is a central figure in Jainism. What should we know about
him and his role in the religion?

\textbf{Pankaj Jain:} Mahavira was a contemporary of the Buddha but was
not the founder of Jainism {[}which originated in India around the
seventh century B.C.E.{]}. He was the 24th great teacher of the
tradition and continues to be the most influential role model for
millions of Jains in India and those in other countries throughout the
rest of the world. He taught the principles of truth, nonviolence,
minimalism and celibacy that influenced global leaders such as Mahatma
Gandhi, Martin Luther King Jr. and Nelson Mandela, and demonstrated the
ultimate penance in his life with long fasts and meditations.

\textbf{Yancy:} How did he reach enlightenment?

\textbf{Jain:} Mahavira was born into a royal family but renounced all
his wealth, family and belongings when he turned 30 and became a monk.
During his wandering ascetic life, he used to be so engrossed in his
meditation that he could easily ignore all his bodily needs and even
injuries. It is mentioned in the Jain texts that he ate only rice, dates
or pulses during that time and that he accepted food only on 349 days
out of 4,380 days (almost 12 years) of his severe penance.

Because he remained silently engrossed in his contemplation and
meditation in that period, sometimes people misunderstood him and tried
to harm him in different ways. Still, nothing could disturb the
steadfast concentration that finally led him to achieve his
enlightenment around the age of 43. He obtained his liberation when he
died at the age of 72.

\textbf{Yancy:} Is Jainism theistic or nontheistic?

\textbf{Jain:} Jainism, like Buddhism, is nontheistic. In Jainism, there
is no God that created the universe and all the souls. The universe and
all its souls have existed without any beginning and will always exist
without any end. For millions of Jains, their great teachers and gurus
remain their role models. Jain temples enshrine their 24 greatest
teachers, Tirthankaras, and most of the Jains regularly visit these
temples and perform elaborate rituals, especially on their holy days and
festivals, such as the annual day of forgiveness, and many others.

\includegraphics{https://static01.nyt.com/images/2020/07/29/opinion/29stone4/merlin_174744288_14b0dce7-d41c-42fe-8480-378055cbbb8f-articleLarge.jpg?quality=75\&auto=webp\&disable=upscale}

\textbf{Yancy:} Is there a sacred text within Jainism? What does it
teach about good and evil, and how we ought to live?

\textbf{Jain:} Jainism has two prominent sects, and each has its own set
of dozens of sacred texts. The emphasis of these texts is on
nonviolence, minimalism and pluralism. We ought to live with minimal
violence toward other living beings in our thoughts, words and actions.
For Jainism, living beings include air, water, fire, earth, in addition
to humans, animals, plants and insects.

\textbf{Yancy:} This is fascinating, but how might we avoid, say,
stepping on an ant or killing microbes? Are there certain practices that
might prevent such deaths?

\textbf{Jain:} Jainism recognizes that some violence is inevitable,
especially toward microbes, as we live in the world, and therefore, lay
Jains have relaxed vows. However, for Jain ascetics, the vows and rules
are stringent, which prohibits them from riding on any vehicle. They
must only walk for any travel, and when they do, carefully clearing
their way to avoid stepping on any small insect, for instance. Some Jain
ascetics keep their mouths covered with a mask as they speak or breathe
to prevent ingesting any microbes. Devout Jains consume water only after
carefully filtering and boiling for the same reason. They also perform
forgiveness rituals periodically to repent for any unconscious violence
committed even after these precautionary practices.

\textbf{Yancy:} It is my understanding that Jainism has a rich set of
beliefs about the movement of the soul --- transmigration. What kind of
insight does this give us into how Jains conceive of death or the
afterlife?

\textbf{Jain:} In Jainism, as in some other religions, only the body
dies, but the soul continues its journey through transmigration. The
soul is distinct from the body in Jainism. The body is a tool used to
purify the soul across different lives to reach the ultimate destination
of Moksha, or liberation. The soul reincarnates in any of the millions
of species on the earth, the heavens or the hells. This infinite process
depends on how nonviolent its journey has been through different lives.

\textbf{Yancy:} How might we use our bodies to purify our souls? I ask
because often, the body is seen as ``impure'' within various religious
traditions.

\textbf{Jain:} The body is an instrument for practicing penance,
including meditation and fasting. The body is not to be indulged in
pleasures and luxuries, but rather, by courageously tolerating various
bodily inconveniences and sufferings, the soul succeeds in purifying
itself from past karmas. This purification process helps one progress
toward liberation.

\textbf{Yancy:} Why might a soul reincarnate within a particular species
--- cat, cow, bird and so on? You suggested that being nonviolent makes
a difference in the next life.

\textbf{Jain:} In the Jain system of reincarnation, the more nonviolent
and purer a soul is, the more favorable the next birth will be. The
human species is the most privileged, but other species can also
renounce violence. For example, in one of his previous births, Mahavira
was born as a lion, and that lion became vegetarian and nonviolent.

\textbf{Yancy:} The Jainist universe contains higher and lower realms,
which we might understand as heaven or hell. Is hell conceived of as a
bad place and heaven a good place?

\textbf{Jain:} Hell is a bad place where a soul is punished for its past
negative karma. Heaven is a good place where a soul is rewarded for its
positive karma. But the earth is better than both extremes because only
here a human being can strive to attain liberation, transcending all the
three realms of earth, heavens and hells.

Image

Credit...Devin Oktar Yalkin for The New York Times

\textbf{Yancy:} You and I are human beings. What does this say about our
previous reincarnations?

\textbf{Jain:} All humans are privileged as they have enough positive
karma from their past lives, allowing them in this life to strive toward
their liberation.

\textbf{Yancy:} So, because I am a philosopher, does this mean that I
was reincarnated from something ``higher'' or less violent?

\textbf{Jain:} Yes, being a philosopher would be like being a
practitioner of spirituality, which is a privilege to be able to
progress toward liberation. The opportunity to be a practitioner of
philosophy is an excellent reward for your past karma. In your previous
lives, you may have been progressively more and more nonviolent,
compassionate and minimalist.

\textbf{Yancy:} So, if one is born a dog or a fish, what sort of prior
life would they have lived?

\textbf{Jain:} For any species, the key is how nonviolently they have
lived. Even a dog or a lion can avoid hurting others unnecessarily as
Jain and Buddhist tales demonstrate.

\textbf{Yancy:} Does Jainism have a story about why there is
reincarnation at all, or is reincarnation taken as the given structure
of reality?

\textbf{Jain:} Until a soul is completely purified, it must keep
reincarnating itself through various species as there is no way out of
this infinite cycle. The body dies at the end of life, but the soul can
never die. Even after liberation, the soul lives forever in its
omniscience and omnipotence at the highest abode in the universe.

\textbf{Yancy:} What should we do now so that we can best determine the
next reincarnation of our souls? Can you provide some key precepts that
we might follow or embody?

\textbf{Jain:} We must adhere to truth, nonviolence, nonstealing,
nonpossession (minimalism) and celibacy, to the best of our competence
and capability. Renouncing meat and animal products would be the first
step toward that. Eating meat is avoided by Jain ascetics and
householders. Ascetics go to the ultimate by avoiding several kinds of
vegetarian food also to minimize all kinds of violence even toward
seeds, roots and stems of a plant.

\textbf{Yancy:} What impact does the performance of such specific
rituals have on the soul and its future reincarnations?

\textbf{Jain:} Jain rituals help practitioners celebrate the lives and
teachings of their great teachers. Many rituals involve verses and songs
based on them. Many Jains daily perform a ritual to ask for forgiveness
from millions of species that they might have unintentionally hurt on
that day. In the same spirit, they also ask for forgiveness from their
family and friends on the annual forgiveness day. These rituals help
reduce egoistic tendencies and foster a sense of interdependence and
friendliness among each other.

\textbf{Yancy:} What, according to Jainism, is living the ``good life''
or the observant life? And how does doing so or not doing so impact the
soul after death?

\textbf{Jain:} Jain teachers teach and demonstrate that one should live
with minimal violence. To that end, most of the Jains avoid meat and
other animal products to keep their souls free of negative karmas. The
Jains have demonstrated their compassion by supporting thousands of
animal sanctuaries across India and now in America.

\textbf{Yancy:} What would you say Jainism offers in the way of wisdom
regarding death, given that death is inevitable?

\textbf{Jain:} Death of the body is merely a milestone in the journey of
the soul. This journey completes when the soul achieves liberation and
arrives at the abode of the liberated beings, located at the final
frontier of the universe. Before death catches one off-guard, it is
one's duty to keep purifying the soul. The purer the soul is at the time
of death, the better the chances are to get a more favorable next birth
to continue the journey of purification. Only with 100 percent purity
based on 100 percent nonviolence a soul can achieve liberation.

\textbf{Yancy:} It seems to me that most people fear death. How does
Jainism account for this fear?

\textbf{Jain:} Like in other communities, death is not a preferred topic
of everyday discussion for Jains. This fear, however, is due to the
sudden disruption of one's life and uncertainty arising from it.

\textbf{Yancy:} But if we live our lives attempting to eliminate
violence and being successful at it, I assume that this might decrease
our fear of death. I say this because it seems we would have a better
chance of being reincarnated in a higher form.

\textbf{Jain:} Yes, you are correct. One who is steadfast in one's
practice of nonviolence would have less and less fear of death. The
death-welcoming Jain ritual of Sallekhana also underscores this
fearlessness.

\textbf{Yancy:} Are there certain ritualistic practices that Jains adopt
as they are about to die?

\textbf{Jain:} Many Jain monks and nuns in their late lives welcome
death with Sallekhana, ** the ** ritual for death. This practice
includes the ultimate renunciation of food and water, as their final
demonstration of zeroing down consumption, and related subtle violence,
of all kinds.

\textbf{Yancy:} What would you say is the most philosophically rich
takeaway from Jainism in terms of its conceptualization of death?

\textbf{Jain:} One should not fear death but celebrate it as the
ultimate demonstration of minimizing consumption and violence. Fear of
death should encourage a life that is compassionate toward other living
beings.

George Yancy is a professor of philosophy at Emory University. His
latest book is
``\href{https://rowman.com/ISBN/9781538131619/Across-Black-Spaces-Essays-and-Interviews-from-an-American-Philosopher}{Across
Black Spaces: Essays and Interviews From an American Philosopher.}''

\emph{\textbf{Now in print:}}
\emph{``}\href{http://bitly.com/1MW2kN3}{\emph{Modern Ethics in 77
Arguments}}\emph{'' and ``}\href{http://bitly.com/1MW2kN3}{\emph{The
Stone Reader: Modern Philosophy in 133 Arguments}}\emph{,'' with essays
from the series, edited by Peter Catapano and Simon Critchley, published
by Liveright Books.}

\emph{The Times is committed to publishing}
\href{https://www.nytimes.com/2019/01/31/opinion/letters/letters-to-editor-new-york-times-women.html}{\emph{a
diversity of letters}} \emph{to the editor. We'd like to hear what you
think about this or any of our articles. Here are some}
\href{https://help.nytimes.com/hc/en-us/articles/115014925288-How-to-submit-a-letter-to-the-editor}{\emph{tips}}\emph{.
And here's our email:}
\href{mailto:letters@nytimes.com}{\emph{letters@nytimes.com}}\emph{.}

\emph{Follow The New York Times Opinion section on}
\href{https://www.facebook.com/nytopinion}{\emph{Facebook}}\emph{,}
\href{http://twitter.com/NYTOpinion}{\emph{Twitter (@NYTopinion)}}
\emph{and}
\href{https://www.instagram.com/nytopinion/}{\emph{Instagram}}\emph{.}

Advertisement

\protect\hyperlink{after-bottom}{Continue reading the main story}

\hypertarget{site-index}{%
\subsection{Site Index}\label{site-index}}

\hypertarget{site-information-navigation}{%
\subsection{Site Information
Navigation}\label{site-information-navigation}}

\begin{itemize}
\tightlist
\item
  \href{https://help.nytimes.com/hc/en-us/articles/115014792127-Copyright-notice}{©~2020~The
  New York Times Company}
\end{itemize}

\begin{itemize}
\tightlist
\item
  \href{https://www.nytco.com/}{NYTCo}
\item
  \href{https://help.nytimes.com/hc/en-us/articles/115015385887-Contact-Us}{Contact
  Us}
\item
  \href{https://www.nytco.com/careers/}{Work with us}
\item
  \href{https://nytmediakit.com/}{Advertise}
\item
  \href{http://www.tbrandstudio.com/}{T Brand Studio}
\item
  \href{https://www.nytimes.com/privacy/cookie-policy\#how-do-i-manage-trackers}{Your
  Ad Choices}
\item
  \href{https://www.nytimes.com/privacy}{Privacy}
\item
  \href{https://help.nytimes.com/hc/en-us/articles/115014893428-Terms-of-service}{Terms
  of Service}
\item
  \href{https://help.nytimes.com/hc/en-us/articles/115014893968-Terms-of-sale}{Terms
  of Sale}
\item
  \href{https://spiderbites.nytimes.com}{Site Map}
\item
  \href{https://help.nytimes.com/hc/en-us}{Help}
\item
  \href{https://www.nytimes.com/subscription?campaignId=37WXW}{Subscriptions}
\end{itemize}
