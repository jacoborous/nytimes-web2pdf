The Mysterious Life of Birds Who Never Come Down

\url{https://nyti.ms/2Dcn2Bc}

\begin{itemize}
\item
\item
\item
\item
\item
\item
\end{itemize}

\includegraphics{https://static01.nyt.com/images/2020/08/02/magazine/02mag-vesper-1/02mag-vesper-1-articleLarge.jpg?quality=75\&auto=webp\&disable=upscale}

Sections

\protect\hyperlink{site-content}{Skip to
content}\protect\hyperlink{site-index}{Skip to site index}

Feature

\hypertarget{the-mysterious-life-of-birds-who-never-come-down}{%
\section{The Mysterious Life of Birds Who Never Come
Down}\label{the-mysterious-life-of-birds-who-never-come-down}}

Swifts spend all their time in the sky. What can their journeys tell us
about the future?

Credit...Illustration by Daniel Barreto

Supported by

\protect\hyperlink{after-sponsor}{Continue reading the main story}

By Helen Macdonald

\begin{itemize}
\item
  July 29, 2020
\item
  \begin{itemize}
  \item
  \item
  \item
  \item
  \item
  \item
  \end{itemize}
\end{itemize}

I found a dead common swift once, a husk of a bird under a bridge over
the River Thames, where sunlight from the water cast bright scribbles on
the arches above. I picked it up, held it in my palm, saw the dust in
its feathers, its wings crossed like dull blades, its eyes tightly
closed, and realized that I didn't know what to do. This was a surprise.
Encouraged by books, I'd always been the type of Gothic amateur
naturalist who preserved interesting bits of the dead. I cleaned and
polished fox skulls; disarticulated, dried and kept the wings of
roadkill birds. But I knew, looking at the swift, that I could not do
anything like that to it. The bird was suffused with a kind of
seriousness very akin to holiness. I didn't want to leave it there, so I
took it home, swaddled it in a towel and tucked it in the freezer. It
was in early May the next year, as soon as I saw the first returning
swifts flowing down from the clouds, that I knew what I had to do. I
went to the freezer, took out the swift and buried it in the garden one
hand's-width deep in earth newly warmed by the sun.

Swifts are magical in the manner of all things that exist just a little
beyond understanding. Once they were called the ``Devil's bird,''
perhaps because those screaming flocks of black crosses around churches
seemed pulled from darkness, not light. But to me, they are creatures of
the upper air, and of their nature unintelligible, which makes them more
akin to angels. Unlike all other birds I knew as a child, they never
descended to the ground.

When I was young, I was frustrated that there was no way for me to know
them better. They were so fast that it was impossible to focus on their
facial expressions or watch them preen through binoculars. They were
only ever flickering silhouettes at 30, 40, 50 miles an hour, a shoal of
birds, a pouring sheaf of identical black grains against bright clouds.
There was no way to tell one bird from another, nor to watch them do
anything other than move from place to place, although sometimes, if the
swifts were flying low over rooftops, I'd see one open its mouth, and
that was truly uncanny, because the gape was huge, turning the bird into
something uncomfortably like a miniature basking shark. Even so,
watching them with the naked eye was rewarding in how it revealed the
dynamism of what before was merely blankness. Swifts weigh about 1½
ounces, and their surfing and tacking against the pressures of oncoming
air make visible the movings of the atmosphere.

They still seem to me the closest things to aliens on Earth. I've seen
them up close now, held a live grounded adult in my hands before letting
it fall back into the sky. You know those deep-sea fish dragged by nets
from fathoms of blackness, how obvious it is that they aren't supposed
to exist where we are? The adult swift was like that in reverse. Its
frame was tough and spare, and its feathers were bleached by the sun.
Its eyes seemed unable to focus on me, as if it were an entity from an
alternate universe whose senses couldn't quite map onto our phenomenal
world. Time ran differently for this creature. If you record swifts'
high-pitched, insistent screaming and slow it down to human speed, you
can hear what their voices sound like as they speak to one another: a
wild, bubbling, rising and falling call, something like the song of
common loons.

\textbf{Often, during stressful} times when I was small --- while
changing schools, when bullied or after my parents had argued --- I'd
lie in bed before I fell asleep and count in my head all the different
layers between me and the center of the earth: crust, mantle, outer
core, inner core. Then I'd think upward in expanding rings of thinning
air: troposphere, stratosphere, mesosphere, thermosphere, exosphere.
Miles beneath me was molten rock, miles above me limitless dust and
vacancy, and there I'd lie with the warm blanket of the troposphere over
me and a red cotton duvet cover too, and the smell of the night's dinner
lingering upstairs, and downstairs the sound of my mother busy at her
typewriter.

This evening ritual wasn't a test of how much I could keep in my mind at
once, or of how far I could send my imagination. It had something of the
power of incantation, but it did not seem a compulsion, and it was not a
prayer. No matter how tightly the day's bad things had gripped me, there
was so much up there above me, so much below, so many places and states
that were implacable, unreachable, entirely uninterested in human
affairs. Listing them one by one built imaginative sanctuary between
walls of unknowing knowns. It helped in other ways too. Sleeping was
like losing time, somehow like not being alive, and drifting into it at
night there sometimes came a panic that I might not find my way back
from wherever I had gone. My own private vespers felt a little like
counting the steps up a flight of steep stairs. I needed to know where I
was. It was a way of bringing me home.

\textbf{Swifts nest in} obscure places, in dark and cramped spaces:
hollows beneath roof tiles, behind the intakes for ventilation shafts,
in the towers of churches. To reach them, they fly straight at the
entrance holes and enter seemingly at full tilt. Their nests are made of
things snatched from the air: strands of dried grass pulled aloft by
thermals; molted pigeon-breast feathers; flower petals, leaves, scraps
of paper, even butterflies.

During World War II, swifts in Denmark and Italy grabbed chaff,
reflective scraps of tinfoil dropped from aircraft to confuse enemy
radar, flashing and twirling as it fell. They mate on the wing. And
while young martins and swallows return to their nests after their first
flights, young swifts do not. As soon as they tip themselves free of the
nest hole, they start flying, and they will not stop flying for two or
three years, bathing in rain, feeding on airborne insects, winnowing
fast and low to scoop fat mouthfuls of water from lakes and rivers.

Common swifts spend only a few months on their breeding grounds, another
few months in winter over the forests and fields of sub-Saharan Africa,
and the rest of the time they're moving, making a mockery of borders. To
avoid heavy rain, which makes it impossible for them to feed, swifts
with nests in English roofs will fly clockwise around low-pressure
systems, traveling across Europe and back again. They love to assemble
in the complicated, unstable air behind weather depressions to feast
upon the abundance of insects there. They depart us quietly. By the
second week of August, the skies around my home are suddenly empty,
after which I'll see the occasional single straggler and think:
\emph{That's it. That's the last one}, and hungrily watch it rise and
glide through turbulent summer air.

On warm summer evenings, swifts that aren't sitting on eggs or tending
their chicks fly low and fast, screaming in speeding packs around
rooftops and spires. Later they gather higher in the sky, their calls
now so attenuated by air and distance that to the ear they corrode into
something that seems less than sound, to suspicions of dust and glass.
And then, all at once, as if summoned by a call or a bell, they fall
silent and rise higher and higher until they disappear from view. These
ascents are called vespers flights, or vesper flights, after the Latin
\emph{vesper} for evening. Vespers are evening devotional prayers, the
last and most solemn of the day, and I have always thought ``vesper
flights'' the most beautiful phrase, an ever-falling blue. Many times
I've tried to see them do it. But always the dark got too deep, or the
birds skated too wide and far across the sky for me to follow.

For years we thought vesper flights were simply swifts flying higher up
to sleep on the wind. Like other birds, they can put half of their brain
to sleep, with the other half awake. But it's possible that swifts
properly sleep up there too, drift into REM sleep in which flying is
automatic, at least for short periods. During World War I, a French
aviator on special night operations cut his engine at 10,000 feet and
glided down in silent, close circles over enemy lines, a light wind
against him, the full moon overhead. ``We suddenly found ourselves,'' he
wrote, ``among a strange flight of birds which seemed to be motionless,
or at least showed no noticeable reaction. They were widely scattered
and only a few yards below the aircraft, showing up against a white sea
of cloud underneath.''

He had flown into a small party of swifts in deep sleep, miniature black
stars illuminated by the reflected light of the moon. He managed to
catch two --- I know this is impossible, but I like to imagine that he
or his navigator simply stretched out a hand and picked them gently from
the air --- and one swift was pulled dead from the engine after the
flight returned to earth. The remote air, the coldness, the stillness
and the high birds over white cloud suspended in sleep. It's an image
that drifts in and out of my dreams.

\textbf{In the summer} of 1979, an aviator, ecologist and expert in the
science of aircraft bird strikes named Luit Buurma began making radar
observations in the Netherlands for flight-safety purposes. His plots
showed vast flocks of birds over the wide waters of the Ijsselmeer that
turned out to be swifts from Amsterdam and the surrounding region. In
the evening, they flew toward the lake, and between 9 and 10 o'clock
they hawked low over the water to feed upon swarms of freshwater midges.
Just after 10, they began to rise, until 15 minutes later, all were more
than 600 feet high, gathered together in dense, wheeling flocks. Then
the ascent began: five minutes later they were out of sight, and their
vesper flights took them to heights of up to 6,000 feet. Using a special
data processor linked to a large military air-defense radar in the north
of Friesland to more closely study their movements, Buurma discovered
that swifts weren't staying up there to sleep. In the hours after
midnight, they came down once again to feed over the water. It turns out
that swifts, beloved \emph{genii locorum} of bright summer streets, are
just as much nocturnal creatures of thick summer darkness.

But Buurma made another discovery: Swifts weren't just making vesper
flights in the evenings. They made them again just before dawn. Twice a
day, when light levels exactly mirror each other, swifts rise and reach
the apex of their flights at nautical twilight.

Since Buurma's observations, other scientists have studied these ascents
and speculated on their purpose. Adriaan Dokter, an ecologist with a
background in physics, has used Doppler weather radar to find out more
about this phenomenon. He and his co-authors have written that swifts
might be profiling the air as they rise through it, gathering
information on air temperature and the speed and direction of the wind.
Their vesper flights take them to the top of what is called the
convective boundary layer. The C.B.L. is the humid, hazy part of the
atmosphere where the ground's heating by the sun produces rising and
falling convective currents, blossoming thermals of hot air; it's the
zone of fair-weather cumulus clouds and everyday life for swifts. Once
swifts crest the top of this layer, they are exposed to a flow of wind
that's unaffected by the landscape below but is determined instead by
the movements of large-scale weather systems. By flying to these
heights, swifts cannot only see the distant clouds of oncoming frontal
systems on the twilit horizon, but they can also use the wind itself to
assess the possible future courses of these systems. What they are doing
is forecasting the weather.

And they are doing more. As Dokter and his colleagues write, migratory
birds orient themselves through a complex of interacting compass
mechanisms. During vesper flights, swifts have access to them all. At
this panoptic height, they can see the scattered patterns of the stars
overhead, and at the same time they can calibrate their magnetic
compasses, getting their bearings according to the light-polarization
patterns that are strongest and clearest in twilit skies. Stars, wind,
polarized light, magnetic cues, the distant stacks of clouds a hundred
miles out, clear cold air, and below them the hush of a world tilting
toward sleep or waking toward dawn. What they are doing is flying so
high that they can work out exactly where they are, to know what they
should do next. They're quietly, perfectly, orienting themselves.

The behavioral ecologist Cecilia Nilsson and her team have discovered
that swifts don't make these flights alone. They ascend as flocks every
evening before singly drifting down, while in the morning they fly up
alone and return to earth together. To orient themselves correctly, to
make the right decisions, they need to pay attention not only to the
cues of the world around them but also to one another. Nilsson and her
colleagues hypothesize that swifts on their vesper flights are working
according to what is called the many-wrongs principle. That is, they're
averaging all their individual assessments in order to reach the best
navigational decision. If you're in a flock, decisions about what to do
next are improved if you exchange information with those around you. We
can speak to one another; what swifts do is pay attention to what other
swifts are doing. And in the end it can be as simple as this: They
follow one another.

The realm of my own life is the quotidian, the everyday, where I sleep
and eat and work and think. Until now, I've been privileged enough to
experience it as a place of relative quiet. It's a space of rising and
falling hopes and worries, costs and benefits, plans and distractions,
and it can batter and distract me, just as high winds and rainfall send
swifts off-course. Sometimes it's a hard place to be, but it's home to
me.

Thinking about swifts has made me think more carefully about the ways in
which I've dealt with difficulty. When I was small, I comforted myself
with thoughts of layers of rising air; later I hid myself among the
whispers of recorded works of fiction, helping myself fall asleep by
playing audiobooks on my phone. We all have our defenses. Some of them
are self-defeating, but others are occasions for joy: the absorption of
a hobby, the writing of a poem, speeding on a Harley, the slow assembly
of a collection of records or shells. ``The best thing for being sad,''
said T.H. White's Merlyn, ``is to learn something.'' As my friend
Christina says, all of us have to live our lives most of the time inside
the protective structures that we have built; none of us can bear too
much reality. And with the coronavirus pandemic's terrifying grip on the
globe, as so many of us cling desperately to the remnants of what we
assumed would always be normality --- sometimes in ways that put us, our
loved ones and others in danger --- my usual defenses against difficulty
have begun to feel uncomfortably provisional and precarious.

Swifts have, of late, become my fable of community, teaching us about
how to make right decisions in the face of oncoming bad weather. They
aren't always cresting the atmospheric boundary layer at dizzying
heights; most of the time they are living below it in thick and
complicated air. That's where they feed and mate and bathe and drink and
are. But to find out about the important things that will affect their
lives, they must go higher to survey the wider scene, and there
communicate with others about the larger forces impinging on their
realm.

Not all of us need to make that climb, just as many swifts eschew their
vesper flights because they are occupied with eggs and young --- but
surely some of us are required, by dint of flourishing life and the
well-being of us all, to look clearly at the things that are so easily
obscured by the everyday. To take time to see the things we need to set
our courses toward or against; the things we need to think about to know
what we should do next. To trust in careful observation and expertise,
in its sharing for the common good. When I read the news and grieve, my
mind has more than once turned to vesper flights, to the strength and
purpose that can arise from the collaboration of numberless frail and
multitudinous souls. If only we could have seen the clouds that sat like
dark rubble on our own horizon for what they were; if only we could have
worked together to communicate the urgency of what they would become.

Advertisement

\protect\hyperlink{after-bottom}{Continue reading the main story}

\hypertarget{site-index}{%
\subsection{Site Index}\label{site-index}}

\hypertarget{site-information-navigation}{%
\subsection{Site Information
Navigation}\label{site-information-navigation}}

\begin{itemize}
\tightlist
\item
  \href{https://help.nytimes.com/hc/en-us/articles/115014792127-Copyright-notice}{©~2020~The
  New York Times Company}
\end{itemize}

\begin{itemize}
\tightlist
\item
  \href{https://www.nytco.com/}{NYTCo}
\item
  \href{https://help.nytimes.com/hc/en-us/articles/115015385887-Contact-Us}{Contact
  Us}
\item
  \href{https://www.nytco.com/careers/}{Work with us}
\item
  \href{https://nytmediakit.com/}{Advertise}
\item
  \href{http://www.tbrandstudio.com/}{T Brand Studio}
\item
  \href{https://www.nytimes.com/privacy/cookie-policy\#how-do-i-manage-trackers}{Your
  Ad Choices}
\item
  \href{https://www.nytimes.com/privacy}{Privacy}
\item
  \href{https://help.nytimes.com/hc/en-us/articles/115014893428-Terms-of-service}{Terms
  of Service}
\item
  \href{https://help.nytimes.com/hc/en-us/articles/115014893968-Terms-of-sale}{Terms
  of Sale}
\item
  \href{https://spiderbites.nytimes.com}{Site Map}
\item
  \href{https://help.nytimes.com/hc/en-us}{Help}
\item
  \href{https://www.nytimes.com/subscription?campaignId=37WXW}{Subscriptions}
\end{itemize}
