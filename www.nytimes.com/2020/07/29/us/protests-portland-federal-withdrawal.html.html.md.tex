Sections

SEARCH

\protect\hyperlink{site-content}{Skip to
content}\protect\hyperlink{site-index}{Skip to site index}

\href{https://www.nytimes.com/section/us}{U.S.}

\href{https://myaccount.nytimes.com/auth/login?response_type=cookie\&client_id=vi}{}

\href{https://www.nytimes.com/section/todayspaper}{Today's Paper}

\href{/section/us}{U.S.}\textbar{}Federal Agents Agree to Withdraw From
Portland, With Conditions

\url{https://nyti.ms/2P2jDHM}

\begin{itemize}
\item
\item
\item
\item
\item
\item
\end{itemize}

\href{https://www.nytimes.com/news-event/george-floyd-protests-minneapolis-new-york-los-angeles?action=click\&pgtype=Article\&state=default\&region=TOP_BANNER\&context=storylines_menu}{Race
and America}

\begin{itemize}
\tightlist
\item
  \href{https://www.nytimes.com/2020/07/26/us/protests-portland-seattle-trump.html?action=click\&pgtype=Article\&state=default\&region=TOP_BANNER\&context=storylines_menu}{Protesters
  Return to Other Cities}
\item
  \href{https://www.nytimes.com/2020/07/24/us/portland-oregon-protests-white-race.html?action=click\&pgtype=Article\&state=default\&region=TOP_BANNER\&context=storylines_menu}{Portland
  at the Center}
\item
  \href{https://www.nytimes.com/2020/07/23/podcasts/the-daily/portland-protests.html?action=click\&pgtype=Article\&state=default\&region=TOP_BANNER\&context=storylines_menu}{Podcast:
  Showdown in Portland}
\item
  \href{https://www.nytimes.com/interactive/2020/07/16/us/black-lives-matter-protests-louisville-breonna-taylor.html?action=click\&pgtype=Article\&state=default\&region=TOP_BANNER\&context=storylines_menu}{45
  Days in Louisville}
\end{itemize}

Advertisement

\protect\hyperlink{after-top}{Continue reading the main story}

Supported by

\protect\hyperlink{after-sponsor}{Continue reading the main story}

\hypertarget{federal-agents-agree-to-withdraw-from-portland-with-conditions}{%
\section{Federal Agents Agree to Withdraw From Portland, With
Conditions}\label{federal-agents-agree-to-withdraw-from-portland-with-conditions}}

Gov. Kate Brown of Oregon said the teams will begin a withdrawal on
Thursday. Federal officials cautioned that they will withdraw only when
they are confident the federal courthouse can be secured.

\includegraphics{https://static01.nyt.com/images/2020/07/29/us/29portland-withdrawal/29portland-withdrawal-articleLarge.jpg?quality=75\&auto=webp\&disable=upscale}

\href{https://www.nytimes.com/by/mike-baker}{\includegraphics{https://static01.nyt.com/images/2020/05/19/reader-center/author-mike-baker/author-mike-baker-thumbLarge.png}}

By \href{https://www.nytimes.com/by/mike-baker}{Mike Baker}

\begin{itemize}
\item
  July 29, 2020Updated 12:59 p.m. ET
\item
  \begin{itemize}
  \item
  \item
  \item
  \item
  \item
  \item
  \end{itemize}
\end{itemize}

Federal tactical teams that have clashed with protesters in Portland in
recent weeks will soon begin leaving the city, Gov. Kate Brown of Oregon
said Wednesday.

Under an agreement between Ms. Brown and the U.S. Department of Homeland
Security, the governor's office said the Oregon State Police will
provide security for the exterior of the city's federal courthouse,
while the usual team of federal officers that protects the courthouse
year-round will continue to provide security for the interior of the
building.

Ms. Brown said the federal tactical teams that had deployed to the city
would begin a phased withdrawal on Thursday. But Chad Wolf, the acting
secretary of the U.S. Department of Homeland Security, said in a
statement that though the department has agreed with the Oregon governor
on a plan, the department will proceed with the withdrawal of security
personnel in Portland only if federal officials are confident that
federal properties will no longer be under attack.

``State and local law enforcement will begin securing properties and
streets, especially those surrounding federal properties, that have been
under nightly attack for the past two months,'' Mr. Wolf said. ``We
anticipate the ability to change our force posture,'' he added, once
circumstances on the ground ``significantly improve'' with the
deployment of additional state and local law enforcement.

Hours before the announcement of the agreement, President Trump doubled
down on the need for the increased federal presence in Portland.

``You hear all sorts of reports about us leaving,'' Mr. Trump said.
``We're not leaving until they've secured their city. We told the
governor. We told the mayor. Secure your city. If they don't secure
their city soon, we have no choice. We're going to have to go in and
clean it out.''

The arrival of federal forces to protect the courthouse after weeks of
raucous demonstrations outside protesting the death of George Floyd in
police custody in Minneapolis infuriated local officials who did not ask
for the federal deployment. It also triggered a dramatic escalation in
the protests in downtown Portland, with demonstrators mounting nightly
rallies outside the courthouse that often included tear gas and
fireworks.

``These federal officers have acted as an occupying force, refused
accountability, and brought violence and strife to our community,'' Ms.
Brown said.

Ms. Brown said that Vice President Mike Pence was among the people
involved in the discussions to withdraw the federal officers.

The announcement came a day after officials in Washington State
announced that a federal tactical team that had arrived last week in
Seattle had since departed the city. Leaders in Seattle have dealt with
their own protests, including one over the weekend in solidarity with
Portland that included burned buildings, broken windows and local police
repeatedly firing crowd-dispersal weapons.

Portland has seen more than 60 days of consecutive protests since Mr.
Floyd's death. Much of the strife in the city had been between Portland
police officers and the protesters.

But after President Trump issued an executive order to protect statues
and federal property, federal officials sent militarized tactical teams
to Portland. They employed aggressive tactics to keep demonstrators away
from federal property. One protester was shot in the head with a
crowd-control munition, and a Navy veteran was hit repeatedly with a
baton as he stood still. In a tactic that was challenged in court by the
Oregon attorney general, the federal officers used unmarked vans to
target protesters for arrest.

Protest crowds have grown into the thousands, drawing out groups of
mothers, military veterans and nurses, though Mr. Trump has portrayed
the protest crowd as ``violent anarchists.'' Some protesters have lobbed
water bottles and fireworks while also pointing lasers at federal
officers.

In addition to the Federal Protective Service, which normally employs
mostly security contractors to guard the courthouse, the Trump
administration has also deployed tactical agents from the Homeland
Security Department and the U.S. Marshals to the courthouse and a
federal office building nearby.

Earlier this week, the U.S. Marshals said they had identified 100
additional marshals to back up the officers protecting the courthouse.

Zolan Kanno-Youngs contributed reporting.

Advertisement

\protect\hyperlink{after-bottom}{Continue reading the main story}

\hypertarget{site-index}{%
\subsection{Site Index}\label{site-index}}

\hypertarget{site-information-navigation}{%
\subsection{Site Information
Navigation}\label{site-information-navigation}}

\begin{itemize}
\tightlist
\item
  \href{https://help.nytimes.com/hc/en-us/articles/115014792127-Copyright-notice}{©~2020~The
  New York Times Company}
\end{itemize}

\begin{itemize}
\tightlist
\item
  \href{https://www.nytco.com/}{NYTCo}
\item
  \href{https://help.nytimes.com/hc/en-us/articles/115015385887-Contact-Us}{Contact
  Us}
\item
  \href{https://www.nytco.com/careers/}{Work with us}
\item
  \href{https://nytmediakit.com/}{Advertise}
\item
  \href{http://www.tbrandstudio.com/}{T Brand Studio}
\item
  \href{https://www.nytimes.com/privacy/cookie-policy\#how-do-i-manage-trackers}{Your
  Ad Choices}
\item
  \href{https://www.nytimes.com/privacy}{Privacy}
\item
  \href{https://help.nytimes.com/hc/en-us/articles/115014893428-Terms-of-service}{Terms
  of Service}
\item
  \href{https://help.nytimes.com/hc/en-us/articles/115014893968-Terms-of-sale}{Terms
  of Sale}
\item
  \href{https://spiderbites.nytimes.com}{Site Map}
\item
  \href{https://help.nytimes.com/hc/en-us}{Help}
\item
  \href{https://www.nytimes.com/subscription?campaignId=37WXW}{Subscriptions}
\end{itemize}
