Sections

SEARCH

\protect\hyperlink{site-content}{Skip to
content}\protect\hyperlink{site-index}{Skip to site index}

\href{https://www.nytimes.com/section/politics}{Politics}

\href{https://myaccount.nytimes.com/auth/login?response_type=cookie\&client_id=vi}{}

\href{https://www.nytimes.com/section/todayspaper}{Today's Paper}

\href{/section/politics}{Politics}\textbar{}Trump Says He Did Not Ask
Putin About Suspected Bounties to Kill U.S. Troops

\href{https://nyti.ms/2Ew1Kix}{https://nyti.ms/2Ew1Kix}

\begin{itemize}
\item
\item
\item
\item
\item
\item
\end{itemize}

Advertisement

\protect\hyperlink{after-top}{Continue reading the main story}

Supported by

\protect\hyperlink{after-sponsor}{Continue reading the main story}

\hypertarget{trump-says-he-did-not-ask-putin-about-suspected-bounties-to-kill-us-troops}{%
\section{Trump Says He Did Not Ask Putin About Suspected Bounties to
Kill U.S.
Troops}\label{trump-says-he-did-not-ask-putin-about-suspected-bounties-to-kill-us-troops}}

Amid no new signs of investigative developments, the president also said
for the first time that he would have acted had he known about an
earlier C.I.A. assessment.

\includegraphics{https://static01.nyt.com/images/2020/07/29/us/politics/29dc-trump/merlin_175050021_a6750b08-af23-4d9d-9bb2-0fca73c4743f-articleLarge.jpg?quality=75\&auto=webp\&disable=upscale}

\href{https://www.nytimes.com/by/charlie-savage}{\includegraphics{https://static01.nyt.com/images/2018/06/12/multimedia/author-charlie-savage/author-charlie-savage-thumbLarge-v2.png}}\href{https://www.nytimes.com/by/michael-crowley}{\includegraphics{https://static01.nyt.com/images/2019/10/25/reader-center/author-michael-crowley/author-michael-crowley-thumbLarge-v2.png}}\href{https://www.nytimes.com/by/eric-schmitt}{\includegraphics{https://static01.nyt.com/images/2018/06/12/multimedia/author-eric-schmitt/author-eric-schmitt-thumbLarge-v2.png}}

By \href{https://www.nytimes.com/by/charlie-savage}{Charlie Savage},
\href{https://www.nytimes.com/by/michael-crowley}{Michael Crowley} and
\href{https://www.nytimes.com/by/eric-schmitt}{Eric Schmitt}

\begin{itemize}
\item
  July 29, 2020
\item
  \begin{itemize}
  \item
  \item
  \item
  \item
  \item
  \item
  \end{itemize}
\end{itemize}

WASHINGTON --- President Trump said in an interview published Wednesday
that he did not bring up intelligence that
\href{https://www.nytimes.com/2020/06/26/us/politics/russia-afghanistan-bounties.html}{Russia
had covertly offered bounties to kill American troops} when he spoke
with President Vladimir V. Putin last week --- apparently his first
opportunity to directly confront Mr. Putin about the C.I.A. assessment
since its existence became public late last month.

``That was a phone call to discuss other things, and frankly, that's an
issue that many people said was fake news,'' Mr. Trump said in
\href{https://www.axios.com/trump-russia-bounties-taliban-putin-call-4a0f6110-ab58-41c0-96fc-57b507462af1.html}{an
interview with ``Axios on HBO.''}

But Mr. Trump hinted for the first time at blaming subordinates for
failing to bring the matter to his attention. ``If it reached my desk, I
would have done something about it,'' he said. Officials have said the
\href{https://www.nytimes.com/2020/06/29/us/politics/russian-bounty-trump.html}{assessment
was in his written intelligence brief} in February, although he rarely
reads it.

Mr. Trump's mixed message renewed attention on the White House's failure
to authorize any response after the C.I.A. concluded that Russia had
offered and paid bounties, which prompted a bipartisan uproar. His
administration has
\href{https://www.nytimes.com/2020/07/03/us/politics/memo-russian-bounties.html}{downplayed}
the intelligence with the apparent expectation that the furor would blow
over.

Despite public comments by top military officials in recent weeks
suggesting that the Pentagon was hunting for more information, three
senior U.S. military officials said that no single Pentagon agency or
military command was conducting a dedicated investigation into the issue
and that they were instead relying largely on the intelligence
community.

A C.I.A. spokesman declined to comment. But intelligence officials,
speaking on the condition of anonymity to discuss confidential
operations and assessments, said that the intelligence community had not
created any special task force to investigate the issue. Rather, they
described the agency as sharpening the focus in areas of regular
collection and analysis in hopes of gleaning additional evidence.

After the existence of the assessment became public, White House
officials defended their months of inaction by falsely suggesting that
no one credited the intelligence or deemed the C.I.A. assessment worthy
of sharing with Mr. Trump. Since the disclosure, no new National
Security Council interagency meetings on the topic have been scheduled,
one official said, adding that officials who were alarmed about the
bounties intelligence --- and the lack of response --- have essentially
given up because the White House's narrative has made it politically
impossible to reverse course and treat the intelligence as a serious
matter.

Senator Richard Blumenthal, Democrat of Connecticut and a member of the
Armed Services Committee, called on Wednesday for public disclosure of
the intelligence supporting the C.I.A.'s conclusion. ``Americans deserve
\& need to see the intelligence on Russians providing arms \& money to
the Taliban --- for killing American troops in Afghanistan,'' he
\href{https://twitter.com/SenBlumenthal/status/1288495888587948037?s=20}{wrote
on Twitter}.

``Declassify it right now,'' Mr. Blumenthal
\href{https://twitter.com/SenBlumenthal/status/1288495889678512129?s=20}{added},
saying the assessment would ``disprove Trump's denials.''

In the Axios interview, Mr. Trump claimed he was not told about the
bounty suspicions because intelligence officials purportedly did not
think the information was real --- apparently an exaggerated reference
to a dissent by National Security Agency analysts over the C.I.A.'s
confidence level.

``It never reached my desk,'' Mr. Trump told Axios. ``You know why?
Because they didn't think --- intelligence --- they didn't think it was
real. They didn't think --- they didn't think it was worthy of --- I
wouldn't mind --- if it reached my desk, I would have done something
about it.''

Mr. Trump did not elaborate. But speaking to reporters on the White
House lawn after Axios published the interview excerpt, Mr. Trump also
said that ``if it were true, I'd be very angry about it,'' and ``I would
respond appropriately. Nobody has been tougher on Russia than I have.''
Still, he said, ``I don't know why they'd be doing this.''

Mr. Trump is said to rarely look at his daily written briefings, though
he insisted to Axios that he did. Administration officials have
emphasized to lawmakers that none of the aides who discuss intelligence
with the president had orally drawn his attention to the matter.

The president also said in the interview that he often received oral
briefings, meandering into a discussion of violence along the border
between India and China before reiterating, ``I have so many briefings
on so many different countries, but this one didn't reach my desk.''

The New York Times first
\href{https://www.nytimes.com/2020/06/26/us/politics/russia-afghanistan-bounties.html}{reported}
in late June that the C.I.A. had assessed months ago that Russia had
covertly offered and paid bounties to a network of Afghan militants and
criminals to incentivize more frequent attacks on American and coalition
troops, citing officials familiar with the matter. Many other news
organizations confirmed that reporting.

C.I.A. analysts placed medium confidence in that assessment, which they
had reached based on analyzing evidence like the accounts of
interrogated detainees in Afghanistan;
\href{https://www.nytimes.com/2020/06/30/us/politics/russian-bounties-afghanistan-intelligence.html}{money
transfers} from a bank account controlled by Russia's military
intelligence agency, known as the G.R.U., to a Taliban-linked network;
and travel patterns such as
\href{https://www.nytimes.com/2020/07/01/world/asia/afghan-russia-bounty-middleman.html}{evidence
that a middleman suspected of handing out the cash} was now in Russia,
officials have said.

National Security Agency analysts had lower confidence in the
intelligence because they placed greater emphasis on surveillance and
wanted to see intercepts picking up explicit discussions among people
who did not know they were being eavesdropped on, officials have said.

Current and former national security officials have said that there was
rarely courtroom-level certainty in the murky world of intelligence,
that disputes over confidence levels were routine, and that
medium-confidence intelligence of this magnitude would have been briefed
to the president in previous administrations. Indeed, they said, it was
put in Mr. Trump's written daily briefing in late February and
distributed more broadly within the intelligence community in early May.

In his Axios interview, Mr. Trump claimed that former Bush
administration officials who disliked him had called the bounty
suspicions a ``fake issue.'' In his later remarks at the White House,
Mr. Trump named Colin Powell, President George Bush's national security
adviser and then secretary of state under George W. Bush.

But Mr. Powell, who has been out of office for more than a decade, did
not say that the intelligence was fake or untrue. Rather,
\href{https://www.mediaite.com/tv/retired-general-colin-powell-accuses-media-of-almost-hysterical-coverage-of-russia-taliban-bounty-program-it-got-kind-of-out-of-control/}{in
an interview with MSNBC} on July 9, he criticized news media coverage as
overhyping a complex issue.

The G.R.U.'s apparent use of bounties to drive up attacks on American
service members amid peace talks with the Taliban was seen as an
escalation of longstanding Russian assistance to the Taliban, including
covert provisions of small arms.

The National Security Council convened an interagency meeting about the
problem in late March, and then officials developed a list of potential
responses, ranging from protesting to the Kremlin to a more serious
punishment like imposing new sanctions. But months passed, and the
administration did not authorize any of them.

Now that the bounty suspicions are well-known, American intelligence
officers are most likely sorting through many new leads, some legitimate
but others from information peddlers eager to offer what they think the
Americans want to hear, said Marc Polymeropoulos, a former C.I.A. field
officer in Afghanistan who retired last year as the agency's acting
chief of operations in Europe and Eurasia.

Mr. Trump has long taken pains to avoid personally criticizing Mr. Putin
and even seemed intent on downplaying evidence of broader Russian
military and financial support for the Taliban.

Asked about claims to that effect by Gen. John W. Nicholson Jr., the
former top U.S. commander in Afghanistan, Mr. Trump dismissed the
notion. ``I didn't ask Nicholson about that,'' he said, before saying
that the general ``didn't have great success'' in his command, which
ended in 2018.

Mr. Trump also suggested to Axios that Russia's provision of arms to the
Taliban was a kind of understandable payback for the United States
backing fighters opposing the Soviet occupation of Afghanistan during
the 1980s.

``We supplied weapons when they were fighting Russia, too,'' Mr. Trump
said.

Some senior congressional Democrats said they believed that top American
officers who had spoken about the issue --- like Gen. Kenneth F.
McKenzie Jr., the head of the military's Central Command, and Gen. Mark
A. Milley, the chairman of the Joint Chiefs of Staff --- were taking it
seriously. But the lawmakers said they had much less faith in Mr. Trump
and many of his top civilian national security aides.

``I do not have confidence that the national security team writ large
within the Trump administration is committed to getting to the bottom of
this and dealing with it,'' said Representative Adam Smith, a Washington
Democrat who heads the House Armed Services Committee.

Asked about the bounty reports by Senator Christopher S. Murphy,
Democrat of Connecticut, at a July 22 Senate Foreign Relations Committee
hearing, Stephen E. Biegun, the deputy secretary of state, cautioned
that he had to avoid discussing classified information in public. But he
insisted that administration officials would take action if there were
even a suggestion that Russia was putting bounties on American service
members.

``Any suggestion that the Russian Federation, or any part of the Russian
government, is employed in providing resources to fighters from other
countries to attack American soldiers will be met,''
\href{https://www.murphy.senate.gov/newsroom/press-releases/murphy-china-is-watching-our-failure-to-hold-russia-accountable-for-bounties-on-us-soldiers-in-afghanistan}{he
said}, with ``the most severe consequences.''

Notably, Mr. Biegun added that any such ``suggestion'' would ``be the
subject of a conversation between very senior officials in both
governments, in no uncertain terms.''

Mr. Trump and Mr. Putin have stepped up their personal diplomacy since
the conclusion in 2019 of the Russia investigation by the special
counsel, Robert S. Mueller III. At the same time, broader diplomatic
relations between Washington and Moscow have remained adversarial, and
intelligence officials accuse Russia of continued
\href{https://www.nytimes.com/2020/02/20/us/politics/russian-interference-trump-democrats.html}{election
interference} and
\href{https://www.nytimes.com/2020/01/10/us/politics/russia-hacking-disinformation-election.html}{hacking
plots}.

Mr. Trump and Mr. Putin have spoken eight times this year, according to
a \href{http://kremlin.ru/catalog/persons/498/events/61270}{Kremlin
list} of the Russian leader's diplomatic activity --- twice as many
times as they spoke in all of 2019.

Several of those calls involved Mr. Trump's efforts this spring to win
Russian and Saudi support for higher global oil prices. But Mr. Trump
has shown a keen interest in a new arms control treaty with Russia that
would cap China's nuclear arsenal. Mr. Trump said their recent call was
``to discuss nuclear nonproliferation,'' which he called ``a much bigger
issue than global warming.''

During a
\href{https://www.nytimes.com/2020/06/01/us/politics/trump-putin-g7.html}{conversation}
on June 1, Mr. Trump extended an invitation to Mr. Putin to join a
gathering of Group of 7 leaders that Mr. Trump hoped to convene in
September. Russia was expelled from what was then the Group of 8 after
its annexation of Crimea in 2014. Leaders from other nations in the
group have said that Moscow has not yet earned the official readmittance
that Mr. Trump proposes.

Helene Cooper and Thomas Gibbons-Neff contributed reporting.

Advertisement

\protect\hyperlink{after-bottom}{Continue reading the main story}

\hypertarget{site-index}{%
\subsection{Site Index}\label{site-index}}

\hypertarget{site-information-navigation}{%
\subsection{Site Information
Navigation}\label{site-information-navigation}}

\begin{itemize}
\tightlist
\item
  \href{https://help.nytimes.com/hc/en-us/articles/115014792127-Copyright-notice}{©~2020~The
  New York Times Company}
\end{itemize}

\begin{itemize}
\tightlist
\item
  \href{https://www.nytco.com/}{NYTCo}
\item
  \href{https://help.nytimes.com/hc/en-us/articles/115015385887-Contact-Us}{Contact
  Us}
\item
  \href{https://www.nytco.com/careers/}{Work with us}
\item
  \href{https://nytmediakit.com/}{Advertise}
\item
  \href{http://www.tbrandstudio.com/}{T Brand Studio}
\item
  \href{https://www.nytimes.com/privacy/cookie-policy\#how-do-i-manage-trackers}{Your
  Ad Choices}
\item
  \href{https://www.nytimes.com/privacy}{Privacy}
\item
  \href{https://help.nytimes.com/hc/en-us/articles/115014893428-Terms-of-service}{Terms
  of Service}
\item
  \href{https://help.nytimes.com/hc/en-us/articles/115014893968-Terms-of-sale}{Terms
  of Sale}
\item
  \href{https://spiderbites.nytimes.com}{Site Map}
\item
  \href{https://help.nytimes.com/hc/en-us}{Help}
\item
  \href{https://www.nytimes.com/subscription?campaignId=37WXW}{Subscriptions}
\end{itemize}
