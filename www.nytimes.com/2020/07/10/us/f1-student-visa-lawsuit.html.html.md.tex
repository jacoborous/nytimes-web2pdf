Sections

SEARCH

\protect\hyperlink{site-content}{Skip to
content}\protect\hyperlink{site-index}{Skip to site index}

\href{https://www.nytimes.com/section/us}{U.S.}

\href{https://myaccount.nytimes.com/auth/login?response_type=cookie\&client_id=vi}{}

\href{https://www.nytimes.com/section/todayspaper}{Today's Paper}

\href{/section/us}{U.S.}\textbar{}As Universities Seek to Block Visa
Rules, Trump Threatens Tax Status

\url{https://nyti.ms/2DxKEQZ}

\begin{itemize}
\item
\item
\item
\item
\item
\end{itemize}

\href{https://www.nytimes.com/news-event/coronavirus?action=click\&pgtype=Article\&state=default\&region=TOP_BANNER\&context=storylines_menu}{The
Coronavirus Outbreak}

\begin{itemize}
\tightlist
\item
  live\href{https://www.nytimes.com/2020/08/02/world/coronavirus-updates.html?action=click\&pgtype=Article\&state=default\&region=TOP_BANNER\&context=storylines_menu}{Latest
  Updates}
\item
  \href{https://www.nytimes.com/interactive/2020/us/coronavirus-us-cases.html?action=click\&pgtype=Article\&state=default\&region=TOP_BANNER\&context=storylines_menu}{Maps
  and Cases}
\item
  \href{https://www.nytimes.com/interactive/2020/science/coronavirus-vaccine-tracker.html?action=click\&pgtype=Article\&state=default\&region=TOP_BANNER\&context=storylines_menu}{Vaccine
  Tracker}
\item
  \href{https://www.nytimes.com/interactive/2020/07/29/us/schools-reopening-coronavirus.html?action=click\&pgtype=Article\&state=default\&region=TOP_BANNER\&context=storylines_menu}{What
  School May Look Like}
\item
  \href{https://www.nytimes.com/live/2020/07/31/business/stock-market-today-coronavirus?action=click\&pgtype=Article\&state=default\&region=TOP_BANNER\&context=storylines_menu}{Economy}
\end{itemize}

Advertisement

\protect\hyperlink{after-top}{Continue reading the main story}

Supported by

\protect\hyperlink{after-sponsor}{Continue reading the main story}

\hypertarget{as-universities-seek-to-block-visa-rules-trump-threatens-tax-status}{%
\section{As Universities Seek to Block Visa Rules, Trump Threatens Tax
Status}\label{as-universities-seek-to-block-visa-rules-trump-threatens-tax-status}}

Harvard and M.I.T. want a court to protect foreign students taking
online classes. After a hearing, President Trump said he was ordering a
review of universities' tax-exempt status.

\includegraphics{https://static01.nyt.com/images/2020/07/10/us/10VIRUS-VISAS-harvard/merlin_174372192_e50b3589-1944-48d1-8d81-0676f3bb07db-articleLarge.jpg?quality=75\&auto=webp\&disable=upscale}

\href{https://www.nytimes.com/by/anemona-hartocollis}{\includegraphics{https://static01.nyt.com/images/2018/06/13/multimedia/author-anemona-hartocollis/author-anemona-hartocollis-thumbLarge-v3.jpg}}\href{https://www.nytimes.com/by/miriam-jordan/}{\includegraphics{https://static01.nyt.com/images/2018/02/16/multimedia/author-miriam-jordan/author-miriam-jordan-thumbLarge-v2.png}}

By \href{https://www.nytimes.com/by/anemona-hartocollis}{Anemona
Hartocollis} and \href{https://www.nytimes.com/by/miriam-jordan/}{Miriam
Jordan}

\begin{itemize}
\item
  Published July 10, 2020Updated July 14, 2020
\item
  \begin{itemize}
  \item
  \item
  \item
  \item
  \item
  \end{itemize}
\end{itemize}

A battle between the Trump administration and some of America's top
universities escalated on Friday, with Harvard and the Massachusetts
Institute of Technology
\href{https://www.nytimes.com/2020/07/08/us/harvard-mit-trump-ice-students.html}{seeking
a court order} to protect foreign students from losing their visas, and
the president threatening the tax-exempt status of institutions that he
claimed indoctrinate students.

After a brief virtual hearing, a federal judge in Boston put off a
decision Friday on the universities' challenge to new federal rules that
would
\href{https://www.nytimes.com/2020/07/09/world/international-students-visa-reaction.html}{revoke
the visas of foreign students} studying entirely online this fall, and
set another hearing for Tuesday.

Lawyers for the two universities argued in court papers that the new
rules from Immigration and Customs Enforcement, which require students
to take at least one in-person class for their F-1 student visas to
remain valid, would cruelly and recklessly upend the lives of tens of
thousands of international students and threaten public health.

The rules were issued Monday, after many if not most colleges and
universities across the country
\href{https://www.nytimes.com/2020/07/06/us/coronavirus-universities-colleges-reopening.html}{announced
reopening plans} that had been months in the making. Johns Hopkins in
Baltimore and universities in California have also sued the
administration, and several others have indicated that they intend to
support the legal efforts.

``The president is using foreign students as pawns to keep all schools
open, no matter the cost to the health and well-being of these students
and their communities,'' said Mark Rosenbaum, a lawyer with Public
Counsel, a legal aid organization in Los Angeles that filed a lawsuit
Friday on behalf of graduate students at three California universities.
``It's temper-tantrum policymaking.''

William F. Lee, the lawyer for Harvard and M.I.T., told the judge in the
case that it was important to have a decision by Wednesday, because that
is when the government is requiring schools to certify that students are
taking in-person classes to meet the visa requirements --- ``nine days
after the change was announced,'' the universities' court papers note.

\hypertarget{latest-updates-global-coronavirus-outbreak}{%
\section{\texorpdfstring{\href{https://www.nytimes.com/2020/08/01/world/coronavirus-covid-19.html?action=click\&pgtype=Article\&state=default\&region=MAIN_CONTENT_1\&context=storylines_live_updates}{Latest
Updates: Global Coronavirus
Outbreak}}{Latest Updates: Global Coronavirus Outbreak}}\label{latest-updates-global-coronavirus-outbreak}}

Updated 2020-08-02T17:52:35.962Z

\begin{itemize}
\tightlist
\item
  \href{https://www.nytimes.com/2020/08/01/world/coronavirus-covid-19.html?action=click\&pgtype=Article\&state=default\&region=MAIN_CONTENT_1\&context=storylines_live_updates\#link-34047410}{The
  U.S. reels as July cases more than double the total of any other
  month.}
\item
  \href{https://www.nytimes.com/2020/08/01/world/coronavirus-covid-19.html?action=click\&pgtype=Article\&state=default\&region=MAIN_CONTENT_1\&context=storylines_live_updates\#link-780ec966}{Top
  U.S. officials work to break an impasse over the federal jobless
  benefit.}
\item
  \href{https://www.nytimes.com/2020/08/01/world/coronavirus-covid-19.html?action=click\&pgtype=Article\&state=default\&region=MAIN_CONTENT_1\&context=storylines_live_updates\#link-2bc8948}{Its
  outbreak untamed, Melbourne goes into even greater lockdown.}
\end{itemize}

\href{https://www.nytimes.com/2020/08/01/world/coronavirus-covid-19.html?action=click\&pgtype=Article\&state=default\&region=MAIN_CONTENT_1\&context=storylines_live_updates}{See
more updates}

More live coverage:
\href{https://www.nytimes.com/live/2020/07/31/business/stock-market-today-coronavirus?action=click\&pgtype=Article\&state=default\&region=MAIN_CONTENT_1\&context=storylines_live_updates}{Markets}

Shortly after the hearing,
\href{https://twitter.com/realDonaldTrump/status/1281616586273468416}{President
Trump tweeted} that he was ordering the Treasury Department to
re-examine the tax-exempt status and funding of universities, saying too
many ``are about Radical Left Indoctrination, not Education.''

Federal law prohibits the Internal Revenue Service, part of the Treasury
Department, from scrutinizing tax-exempt organizations ``based on their
ideological beliefs.''

In recent days, Mr. Trump has shown mounting displeasure with both K-12
schools and universities, pushing them to hold in-person classes as
usual this fall and even
\href{https://www.nytimes.com/2020/07/09/us/schools-reopening-trump.html}{threatening
to withhold federal money} if they do not, although he has little power
to do so.

In the hearing Friday, Judge Allison D. Burroughs of the United States
District Court for Massachusetts said the idea that the students could
be irreparably harmed, as well as the interests of society in the
outcome, were fairly clear, leaving her basically to consider legal
arguments on how likely the case was to succeed when deciding whether to
issue a restraining order.

She said she was waiting to hear the government's arguments on those
legal arguments, and she asked other parties that might be filing briefs
in support of the universities to keep their arguments short.

``My gut on it is that the big-ticket item here is going to be a
likelihood of success on the merits,'' said Judge Burroughs, who was
nominated to the federal bench by former President Barack Obama.

The universities noted in court papers that the national emergency that
President Trump declared on March 13 for the pandemic was still in
effect, with the number of coronavirus cases in the United States having
passed three million this week.

The administration directive on visas, the university lawyers argued,
``has the hallmarks of a politically motivated maneuver'' to force
universities to hold in-person classes ``without regard to the public
health judgment of the schools and experts about whether that is safe
for students, faculty and staff.''

Harvard and M.I.T. want to welcome students back to campus, the court
papers said, but had determined that ``it is not yet prudent to do so.''
Harvard announced Monday that its undergraduate courses would be
entirely online, although some students would be invited back to campus.
M.I.T. has said that most of its courses would be taught online.

``Because higher education institutions do not exist in a vacuum, an
outbreak at one poses a threat to the health and safety of everyone in
the surrounding community,'' the universities said in court papers.

Although the government has not yet responded in court, the White House
press secretary, Kayleigh McEnany, defended the administration's actions
at a news conference earlier this week.

``You don't get a visa for taking online classes from, let's say,
University of Phoenix. So why would you if you were just taking online
classes, generally?'' she told reporters, adding, ``Perhaps the better
lawsuit would be coming from students who have to pay full tuition with
no access to in-person classes to attend.''

\href{https://www.nytimes.com/news-event/coronavirus?action=click\&pgtype=Article\&state=default\&region=MAIN_CONTENT_3\&context=storylines_faq}{}

\hypertarget{the-coronavirus-outbreak-}{%
\subsubsection{The Coronavirus Outbreak
›}\label{the-coronavirus-outbreak-}}

\hypertarget{frequently-asked-questions}{%
\paragraph{Frequently Asked
Questions}\label{frequently-asked-questions}}

Updated July 27, 2020

\begin{itemize}
\item ~
  \hypertarget{should-i-refinance-my-mortgage}{%
  \paragraph{Should I refinance my
  mortgage?}\label{should-i-refinance-my-mortgage}}

  \begin{itemize}
  \tightlist
  \item
    \href{https://www.nytimes.com/article/coronavirus-money-unemployment.html?action=click\&pgtype=Article\&state=default\&region=MAIN_CONTENT_3\&context=storylines_faq}{It
    could be a good idea,} because mortgage rates have
    \href{https://www.nytimes.com/2020/07/16/business/mortgage-rates-below-3-percent.html?action=click\&pgtype=Article\&state=default\&region=MAIN_CONTENT_3\&context=storylines_faq}{never
    been lower.} Refinancing requests have pushed mortgage applications
    to some of the highest levels since 2008, so be prepared to get in
    line. But defaults are also up, so if you're thinking about buying a
    home, be aware that some lenders have tightened their standards.
  \end{itemize}
\item ~
  \hypertarget{what-is-school-going-to-look-like-in-september}{%
  \paragraph{What is school going to look like in
  September?}\label{what-is-school-going-to-look-like-in-september}}

  \begin{itemize}
  \tightlist
  \item
    It is unlikely that many schools will return to a normal schedule
    this fall, requiring the grind of
    \href{https://www.nytimes.com/2020/06/05/us/coronavirus-education-lost-learning.html?action=click\&pgtype=Article\&state=default\&region=MAIN_CONTENT_3\&context=storylines_faq}{online
    learning},
    \href{https://www.nytimes.com/2020/05/29/us/coronavirus-child-care-centers.html?action=click\&pgtype=Article\&state=default\&region=MAIN_CONTENT_3\&context=storylines_faq}{makeshift
    child care} and
    \href{https://www.nytimes.com/2020/06/03/business/economy/coronavirus-working-women.html?action=click\&pgtype=Article\&state=default\&region=MAIN_CONTENT_3\&context=storylines_faq}{stunted
    workdays} to continue. California's two largest public school
    districts --- Los Angeles and San Diego --- said on July 13, that
    \href{https://www.nytimes.com/2020/07/13/us/lausd-san-diego-school-reopening.html?action=click\&pgtype=Article\&state=default\&region=MAIN_CONTENT_3\&context=storylines_faq}{instruction
    will be remote-only in the fall}, citing concerns that surging
    coronavirus infections in their areas pose too dire a risk for
    students and teachers. Together, the two districts enroll some
    825,000 students. They are the largest in the country so far to
    abandon plans for even a partial physical return to classrooms when
    they reopen in August. For other districts, the solution won't be an
    all-or-nothing approach.
    \href{https://bioethics.jhu.edu/research-and-outreach/projects/eschool-initiative/school-policy-tracker/}{Many
    systems}, including the nation's largest, New York City, are
    devising
    \href{https://www.nytimes.com/2020/06/26/us/coronavirus-schools-reopen-fall.html?action=click\&pgtype=Article\&state=default\&region=MAIN_CONTENT_3\&context=storylines_faq}{hybrid
    plans} that involve spending some days in classrooms and other days
    online. There's no national policy on this yet, so check with your
    municipal school system regularly to see what is happening in your
    community.
  \end{itemize}
\item ~
  \hypertarget{is-the-coronavirus-airborne}{%
  \paragraph{Is the coronavirus
  airborne?}\label{is-the-coronavirus-airborne}}

  \begin{itemize}
  \tightlist
  \item
    The coronavirus
    \href{https://www.nytimes.com/2020/07/04/health/239-experts-with-one-big-claim-the-coronavirus-is-airborne.html?action=click\&pgtype=Article\&state=default\&region=MAIN_CONTENT_3\&context=storylines_faq}{can
    stay aloft for hours in tiny droplets in stagnant air}, infecting
    people as they inhale, mounting scientific evidence suggests. This
    risk is highest in crowded indoor spaces with poor ventilation, and
    may help explain super-spreading events reported in meatpacking
    plants, churches and restaurants.
    \href{https://www.nytimes.com/2020/07/06/health/coronavirus-airborne-aerosols.html?action=click\&pgtype=Article\&state=default\&region=MAIN_CONTENT_3\&context=storylines_faq}{It's
    unclear how often the virus is spread} via these tiny droplets, or
    aerosols, compared with larger droplets that are expelled when a
    sick person coughs or sneezes, or transmitted through contact with
    contaminated surfaces, said Linsey Marr, an aerosol expert at
    Virginia Tech. Aerosols are released even when a person without
    symptoms exhales, talks or sings, according to Dr. Marr and more
    than 200 other experts, who
    \href{https://academic.oup.com/cid/article/doi/10.1093/cid/ciaa939/5867798}{have
    outlined the evidence in an open letter to the World Health
    Organization}.
  \end{itemize}
\item ~
  \hypertarget{what-are-the-symptoms-of-coronavirus}{%
  \paragraph{What are the symptoms of
  coronavirus?}\label{what-are-the-symptoms-of-coronavirus}}

  \begin{itemize}
  \tightlist
  \item
    Common symptoms
    \href{https://www.nytimes.com/article/symptoms-coronavirus.html?action=click\&pgtype=Article\&state=default\&region=MAIN_CONTENT_3\&context=storylines_faq}{include
    fever, a dry cough, fatigue and difficulty breathing or shortness of
    breath.} Some of these symptoms overlap with those of the flu,
    making detection difficult, but runny noses and stuffy sinuses are
    less common.
    \href{https://www.nytimes.com/2020/04/27/health/coronavirus-symptoms-cdc.html?action=click\&pgtype=Article\&state=default\&region=MAIN_CONTENT_3\&context=storylines_faq}{The
    C.D.C. has also} added chills, muscle pain, sore throat, headache
    and a new loss of the sense of taste or smell as symptoms to look
    out for. Most people fall ill five to seven days after exposure, but
    symptoms may appear in as few as two days or as many as 14 days.
  \end{itemize}
\item ~
  \hypertarget{does-asymptomatic-transmission-of-covid-19-happen}{%
  \paragraph{Does asymptomatic transmission of Covid-19
  happen?}\label{does-asymptomatic-transmission-of-covid-19-happen}}

  \begin{itemize}
  \tightlist
  \item
    So far, the evidence seems to show it does. A widely cited
    \href{https://www.nature.com/articles/s41591-020-0869-5}{paper}
    published in April suggests that people are most infectious about
    two days before the onset of coronavirus symptoms and estimated that
    44 percent of new infections were a result of transmission from
    people who were not yet showing symptoms. Recently, a top expert at
    the World Health Organization stated that transmission of the
    coronavirus by people who did not have symptoms was ``very rare,''
    \href{https://www.nytimes.com/2020/06/09/world/coronavirus-updates.html?action=click\&pgtype=Article\&state=default\&region=MAIN_CONTENT_3\&context=storylines_faq\#link-1f302e21}{but
    she later walked back that statement.}
  \end{itemize}
\end{itemize}

Some one million international students study in the United States each
year. Immigrant advocates say that together with delays in processing
visas as a result of the pandemic, the new visa rules, which must still
be finalized this month, might discourage many overseas students from
attending American universities, where they often pay full tuition.

Between them, Harvard and M.I.T. have 9,000 international students, some
of whom come from countries like Syria, ``where civil war and an ongoing
humanitarian crisis make internet access and study all but impossible,''
the universities said in their request for a court order. Others come
from Ethiopia, ``where the government has a practice of suspending all
internet access for extended periods, including presently.''

If the government rules are enforced, the universities said,
international students ``must abandon housing arrangements they have
made, breach leases, pay exorbitant airfares and risk Covid-19 infection
on transoceanic flights. And if their departure is not timely, they risk
detention by immigration authorities and formal removal from the country
that may bar their return to the United States for 10 years.''

The universities noted that the Trump administration was reversing an
earlier emergency decision, issued in March as the coronavirus outbreak
forced the closure of campuses across the country. Then, ICE said
students holding F-1 visas could attend remote classes while retaining
their visa status, and made clear that this arrangement was ``in effect
for the duration of the emergency,'' the universities said.

The Massachusetts attorney general said this week that she would support
Harvard and M.I.T.'s efforts to overturn the government's new rules, and
other universities, immigrant advocate groups and state attorneys
general also said they planned to get involved, either filing briefings
in support of Harvard and M.I.T. or their own lawsuits.

Late Thursday, California filed its own lawsuit seeking a preliminary
injunction against the Trump rules from the U.S. District Court for
Northern California. The state, home to the largest population of
international students in the country, called the administration's
policy change ``cruel'' and ``absurd'' and said that the government had
failed to follow the legal procedure for notice and comment required
before implementing new rules.

About 185,000 international students are enrolled in California's public
and state universities.

``Shame on the Trump administration for risking not only the education
opportunities for students who earned the chance to go to college, but
now their health and well-being as well,'' said the state's attorney
general, Xavier Becerra.

On Friday, Johns Hopkins also filed suit, saying that the administration
was undercutting the university's ability to protect ``the health and
safety of its students, faculty, and staff according to its own best
judgment, as informed by its up-to-the-minute, world-leading expertise
and data.''

Advertisement

\protect\hyperlink{after-bottom}{Continue reading the main story}

\hypertarget{site-index}{%
\subsection{Site Index}\label{site-index}}

\hypertarget{site-information-navigation}{%
\subsection{Site Information
Navigation}\label{site-information-navigation}}

\begin{itemize}
\tightlist
\item
  \href{https://help.nytimes.com/hc/en-us/articles/115014792127-Copyright-notice}{©~2020~The
  New York Times Company}
\end{itemize}

\begin{itemize}
\tightlist
\item
  \href{https://www.nytco.com/}{NYTCo}
\item
  \href{https://help.nytimes.com/hc/en-us/articles/115015385887-Contact-Us}{Contact
  Us}
\item
  \href{https://www.nytco.com/careers/}{Work with us}
\item
  \href{https://nytmediakit.com/}{Advertise}
\item
  \href{http://www.tbrandstudio.com/}{T Brand Studio}
\item
  \href{https://www.nytimes.com/privacy/cookie-policy\#how-do-i-manage-trackers}{Your
  Ad Choices}
\item
  \href{https://www.nytimes.com/privacy}{Privacy}
\item
  \href{https://help.nytimes.com/hc/en-us/articles/115014893428-Terms-of-service}{Terms
  of Service}
\item
  \href{https://help.nytimes.com/hc/en-us/articles/115014893968-Terms-of-sale}{Terms
  of Sale}
\item
  \href{https://spiderbites.nytimes.com}{Site Map}
\item
  \href{https://help.nytimes.com/hc/en-us}{Help}
\item
  \href{https://www.nytimes.com/subscription?campaignId=37WXW}{Subscriptions}
\end{itemize}
