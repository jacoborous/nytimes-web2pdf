Sections

SEARCH

\protect\hyperlink{site-content}{Skip to
content}\protect\hyperlink{site-index}{Skip to site index}

\href{https://www.nytimes.com/section/politics}{Politics}

\href{https://myaccount.nytimes.com/auth/login?response_type=cookie\&client_id=vi}{}

\href{https://www.nytimes.com/section/todayspaper}{Today's Paper}

\href{/section/politics}{Politics}\textbar{}Leading Homeland Security
Under a President Who Embraces `Hate-Filled' Talk

\url{https://nyti.ms/3fljgDB}

\begin{itemize}
\item
\item
\item
\item
\item
\end{itemize}

Advertisement

\protect\hyperlink{after-top}{Continue reading the main story}

Supported by

\protect\hyperlink{after-sponsor}{Continue reading the main story}

\hypertarget{leading-homeland-security-under-a-president-who-embraces-hate-filled-talk}{%
\section{Leading Homeland Security Under a President Who Embraces
`Hate-Filled'
Talk}\label{leading-homeland-security-under-a-president-who-embraces-hate-filled-talk}}

Elaine Duke, a lifelong Republican, was acting secretary of homeland
security for four months in 2017.

\includegraphics{https://static01.nyt.com/images/2020/07/10/us/politics/00dc-duke/00dc-duke-articleLarge.jpg?quality=75\&auto=webp\&disable=upscale}

\href{https://www.nytimes.com/by/michael-d-shear}{\includegraphics{https://static01.nyt.com/images/2018/06/13/multimedia/author-michael-d-shear/author-michael-d-shear-thumbLarge-v2.png}}

By \href{https://www.nytimes.com/by/michael-d-shear}{Michael D. Shear}

\begin{itemize}
\item
  Published July 10, 2020Updated July 28, 2020
\item
  \begin{itemize}
  \item
  \item
  \item
  \item
  \item
  \end{itemize}
\end{itemize}

WASHINGTON --- Elaine C. Duke, then President Trump's acting secretary
of homeland security, arrived at the Roosevelt Room, down the hall from
the Oval Office, on a steamy August afternoon in 2017 expecting a
discussion about President Trump's pledge to terminate
\href{https://www.nytimes.com/2020/07/28/us/politics/trump-daca.html}{DACA},
the Obama-era protections for young immigrants. Instead, she said, it
was ``an ambush.''

``The room was stacked,'' she recalled. Stephen Miller, the architect of
the president's assault on immigration, Attorney General Jeff Sessions
and other White House officials demanded that she sign
\href{https://www.nytimes.com/2017/09/05/us/politics/trump-daca-dreamers-immigration.html}{a
memo ending the program}, which they had already concluded was illegal.
She did not disagree, but she chafed at being cut out of the real
decision-making.

``President Trump believes that he can't trust,'' Ms. Duke, now a
consultant, said in a wide-ranging interview about the 14 months she
spent working for him and the consequences of the president's suspicion
of what he
\href{https://www.nytimes.com/2020/02/11/us/politics/trump-vindman.html}{calls
the ``deep state'' in government}. ``That has affected his ability to
get counsel from diverse groups of people.''

A veteran of nearly 30 years at the Departments of Homeland Security and
Defense, Ms. Duke was the deputy secretary of homeland security in the
summer of 2017 when John F. Kelly, Mr. Trump's first secretary,
\href{https://www.nytimes.com/2017/07/28/us/politics/john-kelly-chief-of-staff-donald-trump.html}{left
to become White House chief of staff}. Ms. Duke served in the top job at
the department until late 2017, when Kirstjen Nielsen was confirmed as
Mr. Kelly's permanent successor.

A lifelong Republican who describes herself as ``a kid from the
Cleveland, Ohio, area,'' Ms. Duke said she supported tougher enforcement
of immigration laws, as long as it was tempered by a sense of humanity
that she tried to exhibit when she volunteered to teach naturalization
classes. But she described an administration that is often driven by
ideology instead of deliberation, values politics over policy and is
dominated by a president who embraces ``hate-filled, angry and
divisive'' language.

``We get distracted by slogans, by maybe words we heard like the
president
\href{https://www.nytimes.com/2018/01/11/us/politics/trump-shithole-countries.html}{allegedly
saying `Haiti is a shithole,'''} Ms. Duke said from her home overlooking
the Occoquan River about 25 minutes south of Washington. ``So we get
only spun up in that, and then we never get to the issue.''

Ms. Duke is the latest in a series of senior officials who have gone
public to describe --- often in vivid, behind-the-scenes detail ---
their discomfort and sometimes shock at the inner workings of the Trump
presidency.

She said she was especially taken aback, during the response to
Hurricane Maria's devastation of Puerto Rico, when she heard Mr. Trump
raise the possibility of ``divesting'' or ``selling'' the island as it
struggled to recover.

``The president's initial ideas were more of as a businessman, you
know,'' she recalled. ``Can we outsource the electricity? Can we can we
sell the island? You know, or divest of that asset?'' (She said the idea
of selling Puerto Rico was never seriously considered or discussed after
Mr. Trump raised it.)

Like
\href{https://www.theatlantic.com/politics/archive/2020/06/james-mattis-denounces-trump-protests-militarization/612640/}{former
Defense Secretary Jim Mattis}, she chooses her words carefully. And like
John R. Bolton, the former national security adviser who
\href{https://www.nytimes.com/2020/06/23/us/politics/john-bolton-interview-trump.html}{published
a book} titled ``The Room Where It Happened,'' Ms. Duke says she is not
ready to commit to voting for Mr. Trump again.

``That's a really hard question,'' she said. ``But given the choices, I
don't know yet.''

White House officials have long expressed displeasure with Ms. Duke's
short tenure as the chief of homeland security, describing her as
unwilling to be a team player and resistant to the president's agenda.

Asked about Ms. Duke's comments, Judd Deere, a White House spokesman,
said that Mr. Trump ``has kept his promise to the American people to
reduce illegal immigration, secure the border, lower the crime rate and
maintain law and order.''

``He has never wavered in his highest obligation to the American people:
their safety and security,'' Mr. Deere added.

Ms. Duke served in the Trump administration during a key period, just as
a wave of hurricanes hit Texas, Florida and Puerto Rico. And she was
there as Mr. Trump and Mr. Miller made their earliest moves against
immigrants ---
\href{https://www.nytimes.com/2017/01/27/us/politics/trump-syrian-refugees.html}{imposing
a travel ban} on mostly Muslim countries; seeking to
\href{https://www.nytimes.com/2019/09/26/us/politics/trump-refugees.html}{sharply
limit entry by refugees}; looking for ways to block asylum seekers; and
ordering an end to DACA, or
\href{https://www.nytimes.com/article/what-is-daca.html}{the Deferred
Action for Childhood Arrivals program}.

She said she supported the president's efforts to tighten immigration
security. But the president's ``America First'' philosophy has veered
toward ``America Only,'' she said.

She said the president and Mr. Miller were right about lax immigration
laws that needed to be fixed, but she said
\href{https://www.nytimes.com/2018/06/16/us/politics/family-separation-trump.html}{the
policy of separating families along the border} --- which her successor
approved months after she left --- was discussed, and rejected, while
she was acting secretary.

``I think that we have the room to help people,'' she said. ``And one of
the ways we have the room to help people is through our immigration
system.''

\includegraphics{https://static01.nyt.com/images/2020/07/10/us/politics/00dc-duke2/merlin_171803157_b925b1f3-7642-4b53-ad38-979585e579ad-articleLarge.jpg?quality=75\&auto=webp\&disable=upscale}

One of her fondest memories, she said, was helping pass out water to
homeless people in the city of Ponce on Puerto Rico's southern coast
after Hurricane Maria, which struck there in the late summer of 2017.
But the response to the storm by the president and his top aides, beyond
the remark about selling Puerto Rico, was also a source of
disappointment.

She said that as Hurricane Maria approached Puerto Rico and Ms. Duke
argued for an
\href{https://www.whitehouse.gov/briefings-statements/president-donald-j-trump-approves-puerto-rico-emergency-declaration-2/}{emergency
declaration} before its landfall, Mick Mulvaney, then the president's
budget director, resisted.

``Quit being so emotional, Elaine, it's not about the people, it's about
the money,'' she said Mr. Mulvaney told her. Asked about the comment,
Mr. Mulvaney said on Friday: ``I never made such a remark. My experience
with the acting director was that she rarely got anything right at
D.H.S. At least she's consistent.''

The next day, Ms. Duke said she was pleased when the president himself
expressed concern about the people of Puerto Rico. But she said she grew
frustrated as Mr. Trump later
\href{https://www.nytimes.com/2017/09/30/us/politics/trump-puerto-rico-mayor.html}{traded
angry tweets} with the island's politicians.

``My thought process for both sides is all the negative energy is a
distraction,'' she said.

Ms. Duke, a soft-spoken person with little experience in the raw
political combat in Washington, said that she often found herself on the
outside of a core group of White House advisers even though she was a
member of the president's cabinet.

``There is a singular view that strength is mean,'' she said, ``that any
kind of ability to collaborate, or not be angry is a weakness.''

Ms. Duke recalled that Melania Trump, the first lady, was
\href{https://www.nytimes.com/2017/08/29/fashion/melania-trump-hurricane-harvey-heels-texas.html}{criticized
after being photographed wearing high heels} as she accompanied her
husband to tour parts of flood-ravaged Texas.

``We were talking,'' Ms. Duke said, ``and she said, `It's the White
House, and I will treat it with the respect and dignity it deserves and
I will dress accordingly.' And I thought that was beautiful.''

Ms. Duke contrasted the first lady's approach that day with Mr. Trump's
frequent use of harsh talk in person and on Twitter.

``The office of the president,'' she said, ``should have a certain
dignity to it that I think is important.''

Her public comments --- her first since leaving the administration two
years ago --- came just days after the Supreme Court
\href{https://www.nytimes.com/2020/06/18/us/trump-daca-supreme-court.html}{invalidated
the president's decision in 2017} to terminate the DACA program, handing
Mr. Trump one of his most humiliating legal defeats on a promise at the
core of his political identity.

Ms. Duke's most lasting legacy is likely to be
\href{https://www.nytimes.com/2019/11/11/us/politics/supreme-court-dreamers-case.html}{the
memo she signed} --- under pressure --- to end that program. Her
decision not to cite any specific policy reasons was at the heart of the
Supreme Court's ruling, which said the Trump administration had failed
to substantively consider the implications of terminating the program's
protections and benefits.

Ms. Duke said she did not include policy reasons in the memo because she
did not agree with the ideas being pushed by Mr. Miller and Mr.
Sessions: that DACA amounted to an undeserved amnesty and that it would
encourage new waves of illegal immigration.

She said she still agreed that DACA ``isn't a legal program,'' but hoped
that Republicans and Democrats in Congress would eventually find a way
to allow the undocumented immigrants covered by the program to live and
work permanently in the United States.

``What was missing for me is really that process of discussing it,'' she
said. ``It is a grave decision not only from a legal standpoint but from
the effect it will have on not just 700,000 people but 700,000 people
plus their families.''

Advertisement

\protect\hyperlink{after-bottom}{Continue reading the main story}

\hypertarget{site-index}{%
\subsection{Site Index}\label{site-index}}

\hypertarget{site-information-navigation}{%
\subsection{Site Information
Navigation}\label{site-information-navigation}}

\begin{itemize}
\tightlist
\item
  \href{https://help.nytimes.com/hc/en-us/articles/115014792127-Copyright-notice}{©~2020~The
  New York Times Company}
\end{itemize}

\begin{itemize}
\tightlist
\item
  \href{https://www.nytco.com/}{NYTCo}
\item
  \href{https://help.nytimes.com/hc/en-us/articles/115015385887-Contact-Us}{Contact
  Us}
\item
  \href{https://www.nytco.com/careers/}{Work with us}
\item
  \href{https://nytmediakit.com/}{Advertise}
\item
  \href{http://www.tbrandstudio.com/}{T Brand Studio}
\item
  \href{https://www.nytimes.com/privacy/cookie-policy\#how-do-i-manage-trackers}{Your
  Ad Choices}
\item
  \href{https://www.nytimes.com/privacy}{Privacy}
\item
  \href{https://help.nytimes.com/hc/en-us/articles/115014893428-Terms-of-service}{Terms
  of Service}
\item
  \href{https://help.nytimes.com/hc/en-us/articles/115014893968-Terms-of-sale}{Terms
  of Sale}
\item
  \href{https://spiderbites.nytimes.com}{Site Map}
\item
  \href{https://help.nytimes.com/hc/en-us}{Help}
\item
  \href{https://www.nytimes.com/subscription?campaignId=37WXW}{Subscriptions}
\end{itemize}
