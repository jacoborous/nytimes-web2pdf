Sections

SEARCH

\protect\hyperlink{site-content}{Skip to
content}\protect\hyperlink{site-index}{Skip to site index}

\href{https://www.nytimes.com/section/politics}{Politics}

\href{https://myaccount.nytimes.com/auth/login?response_type=cookie\&client_id=vi}{}

\href{https://www.nytimes.com/section/todayspaper}{Today's Paper}

\href{/section/politics}{Politics}\textbar{}Judge Orders Delay of First
Federal Execution in 17 Years

\url{https://nyti.ms/3gRyAs5}

\begin{itemize}
\item
\item
\item
\item
\item
\end{itemize}

\href{https://www.nytimes.com/news-event/coronavirus?action=click\&pgtype=Article\&state=default\&region=TOP_BANNER\&context=storylines_menu}{The
Coronavirus Outbreak}

\begin{itemize}
\tightlist
\item
  live\href{https://www.nytimes.com/2020/08/01/world/coronavirus-covid-19.html?action=click\&pgtype=Article\&state=default\&region=TOP_BANNER\&context=storylines_menu}{Latest
  Updates}
\item
  \href{https://www.nytimes.com/interactive/2020/us/coronavirus-us-cases.html?action=click\&pgtype=Article\&state=default\&region=TOP_BANNER\&context=storylines_menu}{Maps
  and Cases}
\item
  \href{https://www.nytimes.com/interactive/2020/science/coronavirus-vaccine-tracker.html?action=click\&pgtype=Article\&state=default\&region=TOP_BANNER\&context=storylines_menu}{Vaccine
  Tracker}
\item
  \href{https://www.nytimes.com/interactive/2020/07/29/us/schools-reopening-coronavirus.html?action=click\&pgtype=Article\&state=default\&region=TOP_BANNER\&context=storylines_menu}{What
  School May Look Like}
\item
  \href{https://www.nytimes.com/live/2020/07/31/business/stock-market-today-coronavirus?action=click\&pgtype=Article\&state=default\&region=TOP_BANNER\&context=storylines_menu}{Economy}
\end{itemize}

Advertisement

\protect\hyperlink{after-top}{Continue reading the main story}

Supported by

\protect\hyperlink{after-sponsor}{Continue reading the main story}

\hypertarget{judge-orders-delay-of-first-federal-execution-in-17-years}{%
\section{Judge Orders Delay of First Federal Execution in 17
Years}\label{judge-orders-delay-of-first-federal-execution-in-17-years}}

The Justice Department immediately appealed the injunction, issued after
family members of the victims of the crime argued that the coronavirus
made it unsafe for them to travel to witness the execution.

\includegraphics{https://static01.nyt.com/images/2020/07/10/us/politics/10dc-virus-deathpenalty/merlin_158451837_97ad5a24-ce6a-4778-a1f1-f167a3104086-articleLarge.jpg?quality=75\&auto=webp\&disable=upscale}

By Hailey Fuchs

\begin{itemize}
\item
  Published July 10, 2020Updated July 14, 2020
\item
  \begin{itemize}
  \item
  \item
  \item
  \item
  \item
  \end{itemize}
\end{itemize}

WASHINGTON --- With the Justice Department scheduled on Monday to carry
out the death penalty for the first time in 17 years, a federal judge in
Indiana temporarily suspended the plan on Friday after family members of
the victims sought a delay, saying they could not safely travel to
witness the
\href{https://www.nytimes.com/2020/07/14/us/politics/daniel-lewis-lee-execution-crime.html}{execution}
because of the coronavirus.

\href{https://www.nytimes.com/2020/07/14/us/politics/daniel-lewis-lee-execution-crime.html}{Daniel
Lewis Lee}, 47, is
\href{https://www.nytimes.com/2020/06/30/us/politics/federal-executions-pandemic.html}{set
to be executed on Monday} for his part in the 1996 murder of a family of
three, bringing an end to an informal federal moratorium on carrying out
the death penalty. But this week, family members of his victims filed a
lawsuit, asking the Justice Department to hold off until the risk of the
pandemic had subsided.

Jane E. Magnus-Stinson, the chief judge for the United States District
Court for the Southern District of Indiana,
\href{https://www.courtlistener.com/recap/gov.uscourts.insd.184466/gov.uscourts.insd.184466.20.0.pdf}{issued
a preliminary injunction on Friday,} arguing that the needs of the
victims' family outweighed the need for a speedy execution.

The Justice Department immediately filed an emergency appeal to allow
the execution to be conducted as planned.

One of the family members, Earlene Branch Peterson, 81, who lost her
daughter and granddaughter to Mr. Lee's crimes, has congestive heart
failure, among other underlying conditions. Monica Veillette, 43, a
niece and cousin to the victims, is asthmatic, and Ms. Veillette's
mother, 61, has underlying health conditions that predispose her to the
virus, they said in their suit.

\hypertarget{latest-updates-global-coronavirus-outbreak}{%
\section{\texorpdfstring{\href{https://www.nytimes.com/2020/08/01/world/coronavirus-covid-19.html?action=click\&pgtype=Article\&state=default\&region=MAIN_CONTENT_1\&context=storylines_live_updates}{Latest
Updates: Global Coronavirus
Outbreak}}{Latest Updates: Global Coronavirus Outbreak}}\label{latest-updates-global-coronavirus-outbreak}}

Updated 2020-08-02T10:04:29.623Z

\begin{itemize}
\tightlist
\item
  \href{https://www.nytimes.com/2020/08/01/world/coronavirus-covid-19.html?action=click\&pgtype=Article\&state=default\&region=MAIN_CONTENT_1\&context=storylines_live_updates\#link-34047410}{The
  U.S. reels as July cases more than double the total of any other
  month.}
\item
  \href{https://www.nytimes.com/2020/08/01/world/coronavirus-covid-19.html?action=click\&pgtype=Article\&state=default\&region=MAIN_CONTENT_1\&context=storylines_live_updates\#link-780ec966}{Top
  U.S. officials work to break an impasse over the federal jobless
  benefit.}
\item
  \href{https://www.nytimes.com/2020/08/01/world/coronavirus-covid-19.html?action=click\&pgtype=Article\&state=default\&region=MAIN_CONTENT_1\&context=storylines_live_updates\#link-2bc8948}{Its
  outbreak untamed, Melbourne goes into even greater lockdown.}
\end{itemize}

\href{https://www.nytimes.com/2020/08/01/world/coronavirus-covid-19.html?action=click\&pgtype=Article\&state=default\&region=MAIN_CONTENT_1\&context=storylines_live_updates}{See
more updates}

More live coverage:
\href{https://www.nytimes.com/live/2020/07/31/business/stock-market-today-coronavirus?action=click\&pgtype=Article\&state=default\&region=MAIN_CONTENT_1\&context=storylines_live_updates}{Markets}

They argued that traveling hundreds of miles from their homes in
Arkansas and Washington State and entering the federal penitentiary in
Terre Haute, Ind., where the execution is to be carried out, would make
it dangerous for them to attend.

``I can't understand why in the face of a roaring national pandemic the
government would seek to put the people who have the right to be there''
in harm's way, said Baker Kurrus, the family's lawyer.

Many states have postponed executions during the pandemic. But
\href{https://www.justice.gov/opa/pr/executions-scheduled-four-federal-inmates-convicted-murdering-children}{Attorney
General William P. Barr announced last month} that the Justice
Department intended to proceed with four federal executions, with Mr.
Lee's being the first.

Until recently, lawyers for the inmates scheduled to die were unable to
see their clients in the months leading up to their scheduled execution
dates because the prison barred visits from outsiders for health
reasons. The lawyers said the restrictions impeded their ability to
defend their clients.

\href{https://www.nytimes.com/2019/10/29/us/arkansas-federal-death-penalty.html}{Some
family members of Mr. Lee's victims,} including those now seeking the
delay, have long opposed his execution, but they had still planned to
attend.

\href{https://www.nytimes.com/news-event/coronavirus?action=click\&pgtype=Article\&state=default\&region=MAIN_CONTENT_3\&context=storylines_faq}{}

\hypertarget{the-coronavirus-outbreak-}{%
\subsubsection{The Coronavirus Outbreak
›}\label{the-coronavirus-outbreak-}}

\hypertarget{frequently-asked-questions}{%
\paragraph{Frequently Asked
Questions}\label{frequently-asked-questions}}

Updated July 27, 2020

\begin{itemize}
\item ~
  \hypertarget{should-i-refinance-my-mortgage}{%
  \paragraph{Should I refinance my
  mortgage?}\label{should-i-refinance-my-mortgage}}

  \begin{itemize}
  \tightlist
  \item
    \href{https://www.nytimes.com/article/coronavirus-money-unemployment.html?action=click\&pgtype=Article\&state=default\&region=MAIN_CONTENT_3\&context=storylines_faq}{It
    could be a good idea,} because mortgage rates have
    \href{https://www.nytimes.com/2020/07/16/business/mortgage-rates-below-3-percent.html?action=click\&pgtype=Article\&state=default\&region=MAIN_CONTENT_3\&context=storylines_faq}{never
    been lower.} Refinancing requests have pushed mortgage applications
    to some of the highest levels since 2008, so be prepared to get in
    line. But defaults are also up, so if you're thinking about buying a
    home, be aware that some lenders have tightened their standards.
  \end{itemize}
\item ~
  \hypertarget{what-is-school-going-to-look-like-in-september}{%
  \paragraph{What is school going to look like in
  September?}\label{what-is-school-going-to-look-like-in-september}}

  \begin{itemize}
  \tightlist
  \item
    It is unlikely that many schools will return to a normal schedule
    this fall, requiring the grind of
    \href{https://www.nytimes.com/2020/06/05/us/coronavirus-education-lost-learning.html?action=click\&pgtype=Article\&state=default\&region=MAIN_CONTENT_3\&context=storylines_faq}{online
    learning},
    \href{https://www.nytimes.com/2020/05/29/us/coronavirus-child-care-centers.html?action=click\&pgtype=Article\&state=default\&region=MAIN_CONTENT_3\&context=storylines_faq}{makeshift
    child care} and
    \href{https://www.nytimes.com/2020/06/03/business/economy/coronavirus-working-women.html?action=click\&pgtype=Article\&state=default\&region=MAIN_CONTENT_3\&context=storylines_faq}{stunted
    workdays} to continue. California's two largest public school
    districts --- Los Angeles and San Diego --- said on July 13, that
    \href{https://www.nytimes.com/2020/07/13/us/lausd-san-diego-school-reopening.html?action=click\&pgtype=Article\&state=default\&region=MAIN_CONTENT_3\&context=storylines_faq}{instruction
    will be remote-only in the fall}, citing concerns that surging
    coronavirus infections in their areas pose too dire a risk for
    students and teachers. Together, the two districts enroll some
    825,000 students. They are the largest in the country so far to
    abandon plans for even a partial physical return to classrooms when
    they reopen in August. For other districts, the solution won't be an
    all-or-nothing approach.
    \href{https://bioethics.jhu.edu/research-and-outreach/projects/eschool-initiative/school-policy-tracker/}{Many
    systems}, including the nation's largest, New York City, are
    devising
    \href{https://www.nytimes.com/2020/06/26/us/coronavirus-schools-reopen-fall.html?action=click\&pgtype=Article\&state=default\&region=MAIN_CONTENT_3\&context=storylines_faq}{hybrid
    plans} that involve spending some days in classrooms and other days
    online. There's no national policy on this yet, so check with your
    municipal school system regularly to see what is happening in your
    community.
  \end{itemize}
\item ~
  \hypertarget{is-the-coronavirus-airborne}{%
  \paragraph{Is the coronavirus
  airborne?}\label{is-the-coronavirus-airborne}}

  \begin{itemize}
  \tightlist
  \item
    The coronavirus
    \href{https://www.nytimes.com/2020/07/04/health/239-experts-with-one-big-claim-the-coronavirus-is-airborne.html?action=click\&pgtype=Article\&state=default\&region=MAIN_CONTENT_3\&context=storylines_faq}{can
    stay aloft for hours in tiny droplets in stagnant air}, infecting
    people as they inhale, mounting scientific evidence suggests. This
    risk is highest in crowded indoor spaces with poor ventilation, and
    may help explain super-spreading events reported in meatpacking
    plants, churches and restaurants.
    \href{https://www.nytimes.com/2020/07/06/health/coronavirus-airborne-aerosols.html?action=click\&pgtype=Article\&state=default\&region=MAIN_CONTENT_3\&context=storylines_faq}{It's
    unclear how often the virus is spread} via these tiny droplets, or
    aerosols, compared with larger droplets that are expelled when a
    sick person coughs or sneezes, or transmitted through contact with
    contaminated surfaces, said Linsey Marr, an aerosol expert at
    Virginia Tech. Aerosols are released even when a person without
    symptoms exhales, talks or sings, according to Dr. Marr and more
    than 200 other experts, who
    \href{https://academic.oup.com/cid/article/doi/10.1093/cid/ciaa939/5867798}{have
    outlined the evidence in an open letter to the World Health
    Organization}.
  \end{itemize}
\item ~
  \hypertarget{what-are-the-symptoms-of-coronavirus}{%
  \paragraph{What are the symptoms of
  coronavirus?}\label{what-are-the-symptoms-of-coronavirus}}

  \begin{itemize}
  \tightlist
  \item
    Common symptoms
    \href{https://www.nytimes.com/article/symptoms-coronavirus.html?action=click\&pgtype=Article\&state=default\&region=MAIN_CONTENT_3\&context=storylines_faq}{include
    fever, a dry cough, fatigue and difficulty breathing or shortness of
    breath.} Some of these symptoms overlap with those of the flu,
    making detection difficult, but runny noses and stuffy sinuses are
    less common.
    \href{https://www.nytimes.com/2020/04/27/health/coronavirus-symptoms-cdc.html?action=click\&pgtype=Article\&state=default\&region=MAIN_CONTENT_3\&context=storylines_faq}{The
    C.D.C. has also} added chills, muscle pain, sore throat, headache
    and a new loss of the sense of taste or smell as symptoms to look
    out for. Most people fall ill five to seven days after exposure, but
    symptoms may appear in as few as two days or as many as 14 days.
  \end{itemize}
\item ~
  \hypertarget{does-asymptomatic-transmission-of-covid-19-happen}{%
  \paragraph{Does asymptomatic transmission of Covid-19
  happen?}\label{does-asymptomatic-transmission-of-covid-19-happen}}

  \begin{itemize}
  \tightlist
  \item
    So far, the evidence seems to show it does. A widely cited
    \href{https://www.nature.com/articles/s41591-020-0869-5}{paper}
    published in April suggests that people are most infectious about
    two days before the onset of coronavirus symptoms and estimated that
    44 percent of new infections were a result of transmission from
    people who were not yet showing symptoms. Recently, a top expert at
    the World Health Organization stated that transmission of the
    coronavirus by people who did not have symptoms was ``very rare,''
    \href{https://www.nytimes.com/2020/06/09/world/coronavirus-updates.html?action=click\&pgtype=Article\&state=default\&region=MAIN_CONTENT_3\&context=storylines_faq\#link-1f302e21}{but
    she later walked back that statement.}
  \end{itemize}
\end{itemize}

Last-minute attempts to halt scheduled executions are common among
inmates on death row. Now, the Seventh Circuit Court of Appeals in
Chicago must decide whether to maintain the lower court's injunction or
allow the execution to proceed.

Two other executions are scheduled for next week. In both cases,
spiritual advisers for the death row inmates have filed similar
complaints seeking delays, citing their own pre-existing health
conditions and the risks of traveling to the prison during the pandemic.

Ms. Veillette said the federal government had already booked airline
tickets and hotel rooms for her family's stay near the federal
penitentiary in Terre Haute. But Friday's decision meant that those
reservations may no longer be necessary.

She said she celebrated the delay outside in her garden, tending to her
flowers.

``We can breathe and not worry about having to choose being at risk or
not being there,'' she said, acknowledging that the decision was not yet
final. ``We can breathe and put some flowers in the ground.''

Advertisement

\protect\hyperlink{after-bottom}{Continue reading the main story}

\hypertarget{site-index}{%
\subsection{Site Index}\label{site-index}}

\hypertarget{site-information-navigation}{%
\subsection{Site Information
Navigation}\label{site-information-navigation}}

\begin{itemize}
\tightlist
\item
  \href{https://help.nytimes.com/hc/en-us/articles/115014792127-Copyright-notice}{©~2020~The
  New York Times Company}
\end{itemize}

\begin{itemize}
\tightlist
\item
  \href{https://www.nytco.com/}{NYTCo}
\item
  \href{https://help.nytimes.com/hc/en-us/articles/115015385887-Contact-Us}{Contact
  Us}
\item
  \href{https://www.nytco.com/careers/}{Work with us}
\item
  \href{https://nytmediakit.com/}{Advertise}
\item
  \href{http://www.tbrandstudio.com/}{T Brand Studio}
\item
  \href{https://www.nytimes.com/privacy/cookie-policy\#how-do-i-manage-trackers}{Your
  Ad Choices}
\item
  \href{https://www.nytimes.com/privacy}{Privacy}
\item
  \href{https://help.nytimes.com/hc/en-us/articles/115014893428-Terms-of-service}{Terms
  of Service}
\item
  \href{https://help.nytimes.com/hc/en-us/articles/115014893968-Terms-of-sale}{Terms
  of Sale}
\item
  \href{https://spiderbites.nytimes.com}{Site Map}
\item
  \href{https://help.nytimes.com/hc/en-us}{Help}
\item
  \href{https://www.nytimes.com/subscription?campaignId=37WXW}{Subscriptions}
\end{itemize}
