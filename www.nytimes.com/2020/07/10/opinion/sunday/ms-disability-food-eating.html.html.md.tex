Sections

SEARCH

\protect\hyperlink{site-content}{Skip to
content}\protect\hyperlink{site-index}{Skip to site index}

\href{https://www.nytimes.com/section/opinion/sunday}{Sunday Review}

\href{https://myaccount.nytimes.com/auth/login?response_type=cookie\&client_id=vi}{}

\href{https://www.nytimes.com/section/todayspaper}{Today's Paper}

\href{/section/opinion/sunday}{Sunday Review}\textbar{}I Have M.S. This
Is What It's Like to Be Fed by Other People.

\href{https://nyti.ms/3gQ82Hu}{https://nyti.ms/3gQ82Hu}

\begin{itemize}
\item
\item
\item
\item
\item
\end{itemize}

Advertisement

\protect\hyperlink{after-top}{Continue reading the main story}

\href{/section/opinion}{Opinion}

Supported by

\protect\hyperlink{after-sponsor}{Continue reading the main story}

disability

\hypertarget{i-have-ms-this-is-what-its-like-to-be-fed-by-other-people}{%
\section{I Have M.S. This Is What It's Like to Be Fed by Other
People.}\label{i-have-ms-this-is-what-its-like-to-be-fed-by-other-people}}

I will never stop lusting for a good bite.

By Elizabeth Jameson

Ms. Jameson is an artist and writer.

\begin{itemize}
\item
  July 10, 2020
\item
  \begin{itemize}
  \item
  \item
  \item
  \item
  \item
  \end{itemize}
\end{itemize}

\includegraphics{https://static01.nyt.com/images/2020/07/12/opinion/sunday/11disability/11disability-articleLarge.jpg?quality=75\&auto=webp\&disable=upscale}

Like many people I know, I've maintained my sanity by doing things that
I am not proud of --- things I've done in secret.

Shortly after the birth of my second child, I would often go food
shopping as soon as the babysitter arrived. The true purpose of this
trip was not to get groceries; instead, I would get what I needed at the
supermarket and immediately go out to the parking lot. What was in my
grocery bag? A Diet Coke and a big bag of lime-flavored tortilla chips.
For a solid 20 minutes I would listen to the radio as I mindlessly ate
my favorite junk food in private. I would stay in this uninterrupted,
glorious space until I had to go home to relieve the babysitter.

I needed this ritual for my own sanity, but I was ashamed. I believed
that I should have been food shopping like the ``normal'' mothers in the
produce section, contemplating the greens and fresh fruit. But my
compact car, strategically nestled in a corner parking spot, felt like a
safe haven where my children and husband could not bother me. I felt
protected from the eyes of the outside world. I could breathe again. I
reveled in the fact that I could eat alone in my car. It was my dirty
little secret.

This illusion of personal space was shattered when one day, my neighbor
startled me with a knock on the car door window.

``Elizabeth?'' He asked, eyebrows raised. ``Are you OK?''

At the time I was mortified, but now I wish I could be caught with a
guilty mouth full of chips and an unhealthy soda. My approach to how and
when I eat has had to change, because now there is always a witness to
my crime. I've lost any sense of privacy or autonomy in the way I enjoy
my food, and the days of parking in the back of the grocery store lot
are long gone.

I have progressive multiple sclerosis, and I am now a quadriplegic. I
cannot hold or feel the warmth of my morning cup of coffee. I cannot
feed myself. Those blissful, secret moments I had previously saved only
for myself no longer exist. The pleasure of choosing when to eat,
whether it is alone or with company, is now over and I am always ** fed
by someone else.

My food preferences and the way I like to eat have been shaped by my
childhood, culture and inexplicable personal quirks. Now the pleasure I
get from food has less to do with the menu and everything to do with how
I am fed by others.

When I first lost the use of my hands, I was not mindful of other
people's eating habits, let alone my own. I realized this when my
husband and I were at a quiet local restaurant where we would go every
Friday for date night. I ordered my usual --- fish, brussels sprouts and
mashed potatoes --- a dish I had probably eaten a hundred times. When my
husband fed me my first bite, I did not enjoy the experience and the
night ended in an argument. In retrospect, he had no way of knowing how
to feed me, because I didn't know how to tell him. We both thought
feeding would be easy. After all, we had been married for more than 30
years, but we were both clueless.

An experience that I held so dear --- date night with my husband ---
just wasn't the same. Feelings of confusion, sadness and isolation were
now uninvited guests at the dinner table. I felt very alone.

After that evening in the restaurant, I began to combat the isolation by
observing my husband's eating habits. Once the food is on his plate he
quickly devours it; his eating is efficient and rapid. This is the
opposite of the way I like to eat, and it helped me understand what had
happened at dinner that night. I realize now that the pace was too fast
and the bites too large, not to mention the fact that the vegetables
were mashed up with the potatoes.

This realization was just the first discovery in my never-ending journey
of what it means to be fed by others, and how I've had to adjust. With
each situation, there is also a confluence of emotions --- often mutual
discomfort or frustration.

People often forget about the fact that I need to eat or drink. They may
also find it uncomfortable to feed me or simply get distracted. My
friend loads up a forkful of chocolate cake and just when she is about
to put it in my wide-open mouth, someone makes a joke or a conversation
sparks. The buttercream-frosted delight hovers two centimeters from my
face, wavering precariously. I keep my mouth open in anticipation, while
my friend is drinking a glass of chardonnay and having a great time
talking to other people at the table. She completely forgets that she is
feeding me, and the night ends as a personal tragedy --- I don't get to
enjoy my beloved chocolate cake. So close, and yet so far.

Getting the food into my mouth is just the beginning. A friend will do
me the kindness of giving me hors d'oeuvres at a party. She wants me to
partake, and she readies a huge piece of blue cheese on a crunchy piece
of bread. It has everything I want --- the smooth cheese with the
rugged, thickly sliced bread, and the crunch! She asks, ``Would you like
some?'' as she is already making a beeline for my mouth. I want the
bread, I want the cheese, but I need a smaller bite. She is trying to do
me a favor, and I don't want to lecture her or be more of a burden than
I feel I already am. I open my mouth, she inserts the bread with a smile
and then wanders off to chat with the next person. I can't manage to
swallow the entire bite and end up dumping the half-eaten bread and
cheese into my lap, in front of a room full of strangers. The scene is
both hilarious and mortifying, a common combination in my life as a
quadriplegic.

That is not to say that all of my experiences are this unsatisfying.

I was recently at a large family reunion. At the end of the event,
everyone was saying their goodbyes and I was left sitting by myself,
staring at the wall. People also forgot to ask if I wanted items from
the dessert cart, which was passing me by. My young cousin noticed my
plight and turned me around so that I could see the dessert cart and
grabbed a collection of delights for me to eat. When he realized he
might need to feed me, his eyes darted around the room looking for
someone else, anyone else, to relieve him of that duty. Tentatively, he
offered to feed me a bite of the dessert. Even though he clearly felt
awkward and embarrassed that he didn't really know how to feed me, his
gesture made me feel loved and cared for, especially because this was
totally outside of his comfort zone.

Because of my disability, I constantly worry if I am being too demanding
when I eat. When I communicate how I want to be fed, I sometimes feel I
am a burden, but I also feel that this communication is essential if I
am going to be treated as a person who deserves to enjoy her food, her
sanity. I wonder how other people feel, such as people with dementia or
other disabilities, or how children who are nonverbal express their
desires. This debate over the difference between want and need is a
distinction many of us who require assistance wrestle with every day.

Regardless of my internal monologue, which I will probably have for the
rest of my life, food means so much to me. I will never stop lusting for
a good bite. At times, I still crave my chips and the privacy of my car,
but I'm out in the open whether I like it or not. I have noticed the
myriad ways to be nourished and the feelings associated with them all
--- frustration, humiliation, humor, joy and satisfaction. And so I
continue to explore new ways of communicating so that I can enjoy the
pleasure of eating with others, while savoring each bite at my own pace.

\emph{This essay was written by the author in collaboration with}
\href{http://www.cxmdesigns.com/}{\emph{Catherine Monahon}}\emph{.}

Elizabeth Jameson (@jamesonfineart) is an artist, author and former
civil rights lawyer.

\emph{\textbf{Now in print:}}
\emph{``}\href{https://www.aboutusbook.com/}{\emph{About Us: Essays From
the Disability Series of The New York Times}}\emph{,'' edited by Peter
Catapano and Rosemarie Garland-Thomson, published by Liveright.}

\emph{The Times is committed to publishing}
\href{https://www.nytimes.com/2019/01/31/opinion/letters/letters-to-editor-new-york-times-women.html}{\emph{a
diversity of letters}} \emph{to the editor. We'd like to hear what you
think about this or any of our articles. Here are some}
\href{https://help.nytimes.com/hc/en-us/articles/115014925288-How-to-submit-a-letter-to-the-editor}{\emph{tips}}\emph{.
And here's our email:}
\href{mailto:letters@nytimes.com}{\emph{letters@nytimes.com}}\emph{.}

\emph{Follow The New York Times Opinion section on}
\href{https://www.facebook.com/nytopinion}{\emph{Facebook}}\emph{,}
\href{http://twitter.com/NYTOpinion}{\emph{Twitter (@NYTopinion)}}
\emph{and}
\href{https://www.instagram.com/nytopinion/}{\emph{Instagram}}\emph{.}

Advertisement

\protect\hyperlink{after-bottom}{Continue reading the main story}

\hypertarget{site-index}{%
\subsection{Site Index}\label{site-index}}

\hypertarget{site-information-navigation}{%
\subsection{Site Information
Navigation}\label{site-information-navigation}}

\begin{itemize}
\tightlist
\item
  \href{https://help.nytimes.com/hc/en-us/articles/115014792127-Copyright-notice}{©~2020~The
  New York Times Company}
\end{itemize}

\begin{itemize}
\tightlist
\item
  \href{https://www.nytco.com/}{NYTCo}
\item
  \href{https://help.nytimes.com/hc/en-us/articles/115015385887-Contact-Us}{Contact
  Us}
\item
  \href{https://www.nytco.com/careers/}{Work with us}
\item
  \href{https://nytmediakit.com/}{Advertise}
\item
  \href{http://www.tbrandstudio.com/}{T Brand Studio}
\item
  \href{https://www.nytimes.com/privacy/cookie-policy\#how-do-i-manage-trackers}{Your
  Ad Choices}
\item
  \href{https://www.nytimes.com/privacy}{Privacy}
\item
  \href{https://help.nytimes.com/hc/en-us/articles/115014893428-Terms-of-service}{Terms
  of Service}
\item
  \href{https://help.nytimes.com/hc/en-us/articles/115014893968-Terms-of-sale}{Terms
  of Sale}
\item
  \href{https://spiderbites.nytimes.com}{Site Map}
\item
  \href{https://help.nytimes.com/hc/en-us}{Help}
\item
  \href{https://www.nytimes.com/subscription?campaignId=37WXW}{Subscriptions}
\end{itemize}
