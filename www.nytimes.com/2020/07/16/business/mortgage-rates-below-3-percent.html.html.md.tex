Sections

SEARCH

\protect\hyperlink{site-content}{Skip to
content}\protect\hyperlink{site-index}{Skip to site index}

\href{https://www.nytimes.com/section/business}{Business}

\href{https://myaccount.nytimes.com/auth/login?response_type=cookie\&client_id=vi}{}

\href{https://www.nytimes.com/section/todayspaper}{Today's Paper}

\href{/section/business}{Business}\textbar{}Mortgage Rates Drop Below
3\% for First Time, Tempting Home Buyers

\url{https://nyti.ms/32u1IBB}

\begin{itemize}
\item
\item
\item
\item
\item
\end{itemize}

Advertisement

\protect\hyperlink{after-top}{Continue reading the main story}

Supported by

\protect\hyperlink{after-sponsor}{Continue reading the main story}

\hypertarget{mortgage-rates-drop-below-3-for-first-time-tempting-home-buyers}{%
\section{Mortgage Rates Drop Below 3\% for First Time, Tempting Home
Buyers}\label{mortgage-rates-drop-below-3-for-first-time-tempting-home-buyers}}

Thanks to Federal Reserve policy, the average rate on a 30-year fixed
mortgage fell to 2.98 percent. That could spur buying and construction.

\includegraphics{https://static01.nyt.com/images/2020/07/16/business/16mortgagerate2/merlin_174084753_4693548b-d914-4004-882e-351811156308-articleLarge.jpg?quality=75\&auto=webp\&disable=upscale}

By \href{https://www.nytimes.com/by/matt-phillips}{Matt Phillips}

\begin{itemize}
\item
  July 16, 2020
\item
  \begin{itemize}
  \item
  \item
  \item
  \item
  \item
  \end{itemize}
\end{itemize}

Home loans have never been cheaper, if you can find a willing lender.

The average rate on 30-year fixed mortgages has fallen below 3 percent
for the first time, as the Federal Reserve's recent efforts to pump
trillions of dollars into financial markets to support the economy
during the pandemic translate into lower consumer borrowing costs.

\href{http://www.freddiemac.com/pmms/\#}{Freddie Mac's nationwide survey
of mortgage rates}, released on Thursday, showed the average on a
30-year mortgage at 2.98 percent, the first time this key rate has
fallen below 3 percent since the government-backed mortgage finance firm
began publishing the data in 1971.

It was the latest in a string of record-low readings for the cost of
home loans, and a rare bright spot for the U.S. economy. Nearly 15
million jobs have disappeared since the coronavirus pandemic exploded in
March. Gross domestic product is expected to contract in the second
quarter more than it ever has before.

But for those who are still receiving a paycheck, the collapse in
mortgage rates has suddenly made homeownership more affordable, analysts
and economists say.

``If you have your job, you've got your financial house in order ---
gosh, this is a great time to go and buy a home because mortgage rates
are dirt cheap,'' said Frank Nothaft, chief economist at CoreLogic, a
real estate research firm.

The public has noticed. Mortgage applications, which fell at the start
of the pandemic, have bounced back to some of the highest levels since
the 2008 housing bubble burst.

The vast majority have been for refinancings, which allow owners to cut
their monthly housing payments, freeing up cash for spending elsewhere.
But record-low rates are stimulating more activity from first-time home
buyers, too.

``People are taking advantage of these low rates not only to refinance
but also to buy homes,'' said Laurie Goodman, co-director of the housing
finance policy center at the Urban Institute. ``You've got a lot of
first-time home buyers in the queue who see this as their opportunity.''

The savings are real. For a mortgage in the amount of the national
median home price, roughly \$285,000, the decline in rates during the
last year would save more than \$100 a month in payments, and roughly
\$50,000 over the course of the loan. In higher-cost coastal areas, the
savings can be far more substantial.

On Wednesday, the Fed's anecdotal report of economic conditions across
its 12 districts consistently spotlighted demand related to low mortgage
rates as one of the few bright spots in the American business landscape.

``Low mortgage interest rates encouraged undecided buyers to `get off
the fence,''' said the section of the Fed's report prepared by its
Cleveland branch. ``Residential realtors suggested that demand for
existing properties was robust as well, but a shortage of listings
constrained sales.''

Such reactions from consumers is precisely the way monetary policy ---
in this instance, the Fed's engineering of lower interest rates --- is
supposed to work, stimulating activity in rate-sensitive sectors of an
economy in an effort to offset weakness elsewhere.

A boom in refinancing lowers expenses for homeowners, freeing up cash
for other purchases. An increase in demand from new home buyers can spur
activity in the home building industry, lifting employment in
construction.

Analysts and economists say it's too early to tell if a sustainable
cycle of this sort is emerging. But at the very least, the Fed's efforts
to support the economy are having an effect.

``It would be troubling if the Fed had cut interest rates to zero and we
were not seeing more demand for interest-rate-sensitive consumption,''
said Ernie Tedeschi, an economist with Evercore ISI, a macroeconomic
advisory firm. ``So the fact that we are is a reassuring sign that at
least a piece of monetary policy is working as intended.''

That said, the housing market is far from immune from the nation's
economic turmoil. CoreLogic data shows that a record-high level of
mortgages --- 3.4 percent --- fell into delinquency in April, higher
than during the worst of the 2008 crisis.

Such numbers have prompted some lenders to tighten their standards for
new home loans, meaning that while average rates are at record lows,
some potential borrowers are likely to pay more or find themselves
unable to qualify.

``They're looking at ways to tighten the credit a little bit to account
for the fact that we don't know what the risks are going forward with
the economy and unemployment and potentially delinquencies,'' said Guy
Cecala, chief executive and publisher of Inside Mortgage Finance, a
trade publication.

Advertisement

\protect\hyperlink{after-bottom}{Continue reading the main story}

\hypertarget{site-index}{%
\subsection{Site Index}\label{site-index}}

\hypertarget{site-information-navigation}{%
\subsection{Site Information
Navigation}\label{site-information-navigation}}

\begin{itemize}
\tightlist
\item
  \href{https://help.nytimes.com/hc/en-us/articles/115014792127-Copyright-notice}{©~2020~The
  New York Times Company}
\end{itemize}

\begin{itemize}
\tightlist
\item
  \href{https://www.nytco.com/}{NYTCo}
\item
  \href{https://help.nytimes.com/hc/en-us/articles/115015385887-Contact-Us}{Contact
  Us}
\item
  \href{https://www.nytco.com/careers/}{Work with us}
\item
  \href{https://nytmediakit.com/}{Advertise}
\item
  \href{http://www.tbrandstudio.com/}{T Brand Studio}
\item
  \href{https://www.nytimes.com/privacy/cookie-policy\#how-do-i-manage-trackers}{Your
  Ad Choices}
\item
  \href{https://www.nytimes.com/privacy}{Privacy}
\item
  \href{https://help.nytimes.com/hc/en-us/articles/115014893428-Terms-of-service}{Terms
  of Service}
\item
  \href{https://help.nytimes.com/hc/en-us/articles/115014893968-Terms-of-sale}{Terms
  of Sale}
\item
  \href{https://spiderbites.nytimes.com}{Site Map}
\item
  \href{https://help.nytimes.com/hc/en-us}{Help}
\item
  \href{https://www.nytimes.com/subscription?campaignId=37WXW}{Subscriptions}
\end{itemize}
