Sections

SEARCH

\protect\hyperlink{site-content}{Skip to
content}\protect\hyperlink{site-index}{Skip to site index}

\href{https://www.nytimes.com/section/your-money}{Your Money}

\href{https://myaccount.nytimes.com/auth/login?response_type=cookie\&client_id=vi}{}

\href{https://www.nytimes.com/section/todayspaper}{Today's Paper}

\href{/section/your-money}{Your Money}\textbar{}Financial Brokers Must
Now Act in Your `Best Interest.' What Does That Mean?

\url{https://nyti.ms/2WtBmfo}

\begin{itemize}
\item
\item
\item
\item
\item
\end{itemize}

Advertisement

\protect\hyperlink{after-top}{Continue reading the main story}

Supported by

\protect\hyperlink{after-sponsor}{Continue reading the main story}

\hypertarget{financial-brokers-must-now-act-in-your-best-interest-what-does-that-mean}{%
\section{Financial Brokers Must Now Act in Your `Best Interest.' What
Does That
Mean?}\label{financial-brokers-must-now-act-in-your-best-interest-what-does-that-mean}}

A new standard established by the Securities and Exchange Commission may
sound better than it actually is, consumer advocates say.

\includegraphics{https://static01.nyt.com/images/2020/07/14/business/14fiduciary1/merlin_170290788_9164b021-0a71-45e8-a2d1-b11a61173db1-articleLarge.jpg?quality=75\&auto=webp\&disable=upscale}

\href{https://www.nytimes.com/by/tara-siegel-bernard}{\includegraphics{https://static01.nyt.com/images/2019/01/18/multimedia/author-tara-siegel-bernard/author-tara-siegel-bernard-thumbLarge.png}}

By \href{https://www.nytimes.com/by/tara-siegel-bernard}{Tara Siegel
Bernard}

\begin{itemize}
\item
  July 16, 2020
\item
  \begin{itemize}
  \item
  \item
  \item
  \item
  \item
  \end{itemize}
\end{itemize}

The next time you shop around for financial advice, more investment
professionals will be able to assure you that they're acting in your
``best interest.'' But what's really in your best interest is
understanding precisely what that means.

A Securities and Exchange Commission rule that took effect on June 30
created a new standard for brokers to live up to: Those who sell
financial products must act in their customers' best interest. But
consumer advocates say investors could be led to believe they're getting
more protections than the rule delivers.

And the new regulation could soon have even broader influence: A
complementary proposal from the Labor Department would allow financial
professionals to accept payments like commissions when providing advice
on your retirement money as long as they met the best-interest standard.
At the same time, more professionals may be able to skirt the rules
altogether, consumer advocates said.

``This is the new wolves in sheep's clothing,'' said Jamie Hopkins,
director of retirement research at Carson Group in Omaha.

The rule that recently took effect --- called
\href{https://www.nytimes.com/2019/06/05/your-money/sec-investment-brokers-fiduciary-duty.html}{Regulation
Best Interest} --- covers brokers, who often make commissions when they
sell things like mutual funds or stocks and bonds to average investors.

``Main Street investors will be entitled to recommendations and advice
in their best interest --- the financial professional cannot put its
interests ahead of the investor,'' said Natalie Strom, a spokeswoman for
the S.E.C.

But consumer advocates said that wasn't the same thing as putting the
client first.

``Notably, the rule does not say that best interest means that a broker
must place the customer's interests ahead of the broker's, which is what
most people would think a best-interest regulation would include,'' said
Benjamin Edwards, an associate professor of law at the University of
Nevada, Las Vegas. That still allows brokers or their firms to consider
their own pockets when making recommendations, he said.

The agency calls the rule an improvement over the old standard, which
required brokers to recommend products that were ``suitable,'' based on
factors such as the customer's age, goals and risk tolerance. The new
rule also aims to rein in certain sales contests, for example, and
requires brokers to consider costs, among other things.

Consumer advocates fear that there will be confusion, though. The ``best
interest'' rule sounds similar to the traditional gold-standard
obligation that certain other financial professionals must meet:
fiduciary duty, which typically means working solely in the interest of
the client. The kinds of professionals held to that standard include
registered investment advisers, who are often paid flat fees for the
time they take to give you advice, or a percentage of assets managed.

The S.E.C.'s own investor advocate also voiced concerns about the
potential for confusion. Customers will be harmed if the rule ``is not
enforced rigorously enough to demand behavior that matches customers'
expectations,'' the advocate, Rick Fleming, said in a statement when the
rule was proposed last year.

If you want to be certain you're working with a
\href{https://www.nytimes.com/2020/02/10/smarter-living/the-young-persons-guide-to-investing.html}{financial
professional} who is truly putting your interests first, advocates
suggest asking him or her a question: Are you acting as a
\href{https://twitter.com/FiduciaryPath/status/1136357717776457729}{fiduciary
100 percent of the time}? Then ask for a
\href{http://www.thefiduciarystandard.org/wp-content/uploads/2015/02/fiduciaryoath_individual.pdf}{signed
oath} saying as much. The broker should be able to
\href{https://www.nytimes.com/2017/02/10/your-money/the-21-questions-youre-going-to-need-to-ask-about-investment-fees.html}{fully
explain} how he or she is compensated.

The new best-interest rule could also have implications for your
retirement money. In the complementary action, the Department of Labor
\href{https://www.dol.gov/agencies/ebsa/about-ebsa/our-activities/resource-center/fact-sheets/improving-investment-advice-for-workers-and-retirees}{proposed
regulations} concerning how financial professionals acting as
fiduciaries must conduct themselves when handling their customers'
retirement accounts.

Under federal law, fiduciaries are generally prohibited from accepting
payments that would pose conflicts of interest. The proposed rule would
provide an exemption, allowing financial professionals to receive such
payments, like commissions, as long as they adhere to a best-interest
standard that generally aligns with the new S.E.C. rule.

The proposal tries to clear up any uncertainty created after a federal
appeals court
\href{https://www.nytimes.com/2018/06/22/your-money/fiduciary-rule-dies.html}{overturned},
in 2018, an Obama-era rule that was challenged by a team of lawyers led
by Eugene Scalia, the current labor secretary. The overturned rule had
allowed fiduciaries to accept commissions and similar payments only if
they entered an enforceable best-interest contract with the investor and
eliminated or more significantly reduced conflicts of interest, legal
experts said. The contract, along with other protections, would not be
required under the newly proposed regulation.

Emily Weeks, a spokeswoman for the Labor Department, said any firms or
financial professionals who relied on the exemption would still need to
acknowledge that they were acting as a fiduciary ``and adhere to these
stringent fiduciary standards as well as other consumer-protective
standards.''

But the proposal package also confirmed that fewer investment
professionals are required to act as fiduciaries when handling their
clients' retirement money, according to retirement law attorneys.

Overturning the Obama-era rule restored a large part of the Employee
Retirement Income Security Act of 1974, which governs when an investment
professional must become a fiduciary while handling retirement money.
Under that law, fiduciary duty is triggered when an investment
professional meets five conditions, including providing individual
advice on a regular basis.

But advocates say the rollback reopens loopholes that the overturned
rule was meant to close.

As a result, it may be easier for financial professionals to avoid
becoming fiduciaries, said Jason C. Roberts, chief executive officer of
the Pension Resource Institute, a consulting firm for banks, brokerage
and advisory firms. Brokers can skirt the fiduciary standard by
structuring their interactions with clients as educational in nature, he
explained, and stopping short of what might be considered advice.

``My clients, financial institutions, are going to be very pleased with
this proposal, and the investor advocates are going to hate it,'' said
Mr. Roberts, who is also a managing partner of the Retirement Law Group.
``It is not taking anything away --- or raising the bar the same way the
prior rule did.''

Barbara Roper, director of investor protection for the Consumer
Federation of America, also said the new rule appeared to make it easier
for a financial professional to avoid being a fiduciary when making
certain kinds of recommendations on one-off transactions.

For example, there would be no fiduciary duty for an insurance agent who
recommended rolling over the proceeds of a 401(k) plan into a
\href{https://www.nytimes.com/2020/03/13/business/coronavirus-scams.html}{fixed-index
annuity product} in a one-time sale, she said.

``The new D.O.L. advice rule simultaneously makes it easier for firms to
evade their fiduciary obligations and weakened those obligations where
they do apply,'' Ms. Roper said.

Stakeholders can submit comments on the new proposal for 30 days, ending
Aug. 6; the Labor Department will review those comments and evaluate
what, if any, changes are needed.

But 21
\href{https://consumerfed.org/testimonial/dol-urged-to-provide-opportunity-for-comment-on-anti-investor-rule/}{advocacy
and trade groups}
\href{https://consumerfed.org/wp-content/uploads/2020/07/DOL-Advice-Rule-Extension-Request.pdf}{wrote
a letter} last week urging the department to provide more time to digest
the proposal. ``A 30-day comment period is an unreasonably short amount
of time to provide thoughtful and comprehensive comments on this complex
and highly technical proposal, which would affect our constituencies ---
including virtually all Americans struggling to save for retirement ---
in varied and far-reaching ways,'' the groups wrote.

Advertisement

\protect\hyperlink{after-bottom}{Continue reading the main story}

\hypertarget{site-index}{%
\subsection{Site Index}\label{site-index}}

\hypertarget{site-information-navigation}{%
\subsection{Site Information
Navigation}\label{site-information-navigation}}

\begin{itemize}
\tightlist
\item
  \href{https://help.nytimes.com/hc/en-us/articles/115014792127-Copyright-notice}{©~2020~The
  New York Times Company}
\end{itemize}

\begin{itemize}
\tightlist
\item
  \href{https://www.nytco.com/}{NYTCo}
\item
  \href{https://help.nytimes.com/hc/en-us/articles/115015385887-Contact-Us}{Contact
  Us}
\item
  \href{https://www.nytco.com/careers/}{Work with us}
\item
  \href{https://nytmediakit.com/}{Advertise}
\item
  \href{http://www.tbrandstudio.com/}{T Brand Studio}
\item
  \href{https://www.nytimes.com/privacy/cookie-policy\#how-do-i-manage-trackers}{Your
  Ad Choices}
\item
  \href{https://www.nytimes.com/privacy}{Privacy}
\item
  \href{https://help.nytimes.com/hc/en-us/articles/115014893428-Terms-of-service}{Terms
  of Service}
\item
  \href{https://help.nytimes.com/hc/en-us/articles/115014893968-Terms-of-sale}{Terms
  of Sale}
\item
  \href{https://spiderbites.nytimes.com}{Site Map}
\item
  \href{https://help.nytimes.com/hc/en-us}{Help}
\item
  \href{https://www.nytimes.com/subscription?campaignId=37WXW}{Subscriptions}
\end{itemize}
