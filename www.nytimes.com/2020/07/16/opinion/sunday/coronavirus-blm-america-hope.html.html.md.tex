\href{/section/opinion/sunday}{Sunday Review}\textbar{}We Interrupt This
Gloom to Offer \ldots{} Hope

\href{https://nyti.ms/3h0YSIc}{https://nyti.ms/3h0YSIc}

\begin{itemize}
\item
\item
\item
\item
\item
\item
\end{itemize}

\includegraphics{https://static01.nyt.com/images/2020/07/20/opinion/sunday/20kristof/19kristof-articleLarge.jpg?quality=75\&auto=webp\&disable=upscale}

Sections

\protect\hyperlink{site-content}{Skip to
content}\protect\hyperlink{site-index}{Skip to site index}

\href{/section/opinion}{Opinion}

\hypertarget{we-interrupt-this-gloom-to-offer--hope}{%
\section{We Interrupt This Gloom to Offer \ldots{}
Hope}\label{we-interrupt-this-gloom-to-offer--hope}}

Yes, America is suffering needlessly. That may save us.

Credit...Illustration by Nicolas Ortega; Photograph from Getty Images

Supported by

\protect\hyperlink{after-sponsor}{Continue reading the main story}

\href{https://www.nytimes.com/column/nicholas-kristof}{\includegraphics{https://static01.nyt.com/images/2018/04/03/opinion/nicholas-kristof/nicholas-kristof-thumbLarge-v2.png}}

By \href{https://www.nytimes.com/column/nicholas-kristof}{Nicholas
Kristof}

Opinion Columnist

\begin{itemize}
\item
  July 16, 2020
\item
  \begin{itemize}
  \item
  \item
  \item
  \item
  \item
  \item
  \end{itemize}
\end{itemize}

\hypertarget{listen-to-this-op-ed}{%
\subsubsection{Listen to This Op-Ed}\label{listen-to-this-op-ed}}

Audio Recording by Audm

\emph{To hear more audio stories from publishers like The New York
Times, download}
\href{https://www.audm.com/?utm_source=nytmag\&utm_medium=embed\&utm_campaign=left_behind_draper}{**}
\href{https://www.audm.com/?utm_source=nytopinion\&utm_medium=embed\&utm_campaign=interrupt_gloom_hope}{\emph{Audm
for iPhone or Android}}\emph{.}

Just one in six Americans in a poll last month was ``proud'' of the
state of the country, and about two out of three were actually
``fearful'' about it. So let me introduce a new thought: ``hope.''

Yes, our nation is a mess, but overlapping catastrophes have also
created conditions that may finally let us extricate ourselves from the
mire. The grim awareness of national failures --- on the coronavirus,
racism, health care and jobs --- may be a necessary prelude to fixing
our country.

\includegraphics{https://static01.nyt.com/images/2020/07/16/autossell/16kristof-twitter-thumb/16kristof-twitter-thumb-videoSixteenByNine3000.jpg}

The last time our economy was this troubled, Herbert Hoover's failures
led to Franklin D. Roosevelt's election with a mandate to revitalize the
nation. The result was the New Deal, Social Security, rural
electrification, government jobs programs and a 35-year burst of
inclusive growth that built the modern middle class and arguably made
the United States the richest and most powerful country in the history
of the world.

History doesn't repeat, but it does rhyme. And when I reached out
through the gloom to consult experts, I was struck by how much hope I
heard.

``On balance, I am very hopeful and I'm very optimistic,'' Darren
Walker, the president of the Ford Foundation, told me. ``What we're
seeing today is a sort of national convulsion over the recognition that
racism in America is real and it's not a figment of the imagination of
Black people in this country.''

Marian Wright Edelman, the founder of the Children's Defense Fund, who
for six decades has been battling for a more just society, told me,
``I'm very optimistic. I think we have a chance of getting something
done.''

Like others I spoke with, she said that one reason for hope is,
paradoxically, President Trump and the way he has become the avatar of
failed ``let them eat cake'' policies and narratives. ``Mr. Trump is the
perfect opposition to have,'' Edelman said. ``He represents the
implosion of the American dream, and we can't go down his road much
farther.''

``If we can't get something done now,'' she added, ``then shame on us.''

Betting markets like PredictIt
\href{https://www.predictit.org/markets/2/Congress}{expect} Joe Biden to
sweep into the presidency in January with a Democratic House and a
Democratic Senate. By then we may have lost a quarter million Americans
to Covid-19 and remain mired in the worst economic downturn of our
lifetimes, with racial antagonisms inflamed by a president whom
\href{https://news.yahoo.com/new-yahoo-news-you-gov-poll-most-americans-say-trump-is-a-racist-and-want-him-to-stop-tweeting-160841770.html}{a
majority of Americans} regard as a racist. I've known Biden since he was
a senator, and he's no radical --- but that reassuring, boring mien may
make it easier to win a mandate and then use it to pivot the United
States onto a new path.

So perhaps today's national pain, fear and loss can also be a source of
hope: We may be so desperate, our failures so manifest, our grief so
raw, that the United States can once more, as during the Great
Depression, embrace long-needed changes that would have been impossible
in cheerier times.

\textbf{*}

The United States faces at least three simultaneous crises: more
coronavirus deaths than any other country, the worst economic slump
since the Great Depression and overflowing outrage over racial inequity.
Yet these crises are all interlinked, all facets of the same core
failure of our country, one that has its roots in President Richard
Nixon's ``Southern strategy'' of 1968 and in the racialization of social
safety net programs thereafter.

Why is the United States just about the only advanced country to lack
universal health care? Without universal paid sick leave?

Many scholars, in particular the late Alberto Alesina, a Harvard
economist, have
\href{https://scholar.harvard.edu/files/glaeser/files/why_doesnt_the_u.s._have_a_european-style_welfare_state.pdf}{argued}
that one reason for America's outlier status is race. Investing in
safety nets and human capital became stigmatized because of a perception
that African-Americans would benefit. So instead of investing in
children, we invested in a personal responsibility narrative holding
that Americans just need to lift themselves up
\href{https://www.nytimes.com/2020/02/19/opinion/economic-mobility.html}{by
their bootstraps} to get ahead.

This experiment proved catastrophic for \emph{all} Americans, especially
the working class. Marginalized groups, including African-Americans and
Native Americans, suffered the worst, but the underinvestment in health
and the lack of safety nets meant that American children today are
\href{https://www.healthaffairs.org/doi/abs/10.1377/hlthaff.2017.0767}{57
percent} more likely to die by age 19 than European children are.

This boomerang effect of obdurate white racism --- what Dr. Jonathan M.
Metzl calls \href{https://www.dyingofwhiteness.com/}{``dying of
whiteness''} --- means that Americans now are
\href{https://www.socialprogress.org/?tab=2\&code=USA}{less likely} to
graduate from high school than children in many peer countries.
Meanwhile, people die in the United States from drug overdoses at a rate
of one every seven minutes.

This is deeply personal to me. As I've written in a recent book,
\href{https://www.penguinrandomhouse.com/books/588999/tightrope-by-nicholas-d-kristof-and-sheryl-wudunn/}{``Tightrope,''}
a quarter of the children on my old No. 6 school bus in rural Yamhill,
Ore., are dead from drugs, alcohol and suicide --- deaths of despair.
Others are homeless or in prison. Although they were white, they
perished because of policy choices, partly rooted in racism, that the
United States has pursued for 50 years.

Gaps in safety nets left us in turn particularly vulnerable to a
pandemic, for underinsurance and lack of paid sick leave helped spread
the coronavirus. The pandemic then caused people to lose their jobs,
which in the United States meant that they lost health insurance just
when it was most needed. Trump bungled the pandemic, as did some local
leaders, but the failure was also 50 years in the making.

\textbf{*}

Do we now have a chance for a reset? Yes, I think we do.

It may already have been in the works. Kansas Republicans rebelled
against tax cuts that had devastated schools. Texas helped lead the way
in reversing mass incarceration. Red states like Idaho, Utah and
Oklahoma expanded Medicaid.

To the extent that America's 50 years of failures had their roots in
racism, it's also striking that the new possibilities arise in part from
mass revulsion at a short video that showed the undeniable truth of
racism today.

In 1899, W.E.B. Du Bois, writing about racial injustice, said there have
``been few other cases in the history of civilized people where human
suffering has been viewed with such peculiar indifference.'' Yet maybe
the video of George Floyd's life being snuffed out by police officers is
dispelling that indifference. The current Black Lives Matter protests,
measured by the number of participants (roughly 20 million), appear to
constitute
\href{https://www.nytimes.com/interactive/2020/07/03/us/george-floyd-protests-crowd-size.html}{the
largest movement} in American history.

``There was something about seeing a man's knee on another man's neck
that woke people up,'' said Helene Gayle, the chief executive of the
Chicago Community Trust. ``People think I'm crazy, but I have a sense of
possibility.''

The polling is striking. Sixty percent of Americans, including a
majority of white people,
\href{https://www.cbsnews.com/news/black-lives-matter-police-reform-opinion-poll-28-06-2020/}{said}
in a CBS News poll last month that they support ideas promoted by the
Black Lives Matter movement. Almost as large a majority
\href{https://www.kff.org/slideshow/public-opinion-on-single-payer-national-health-plans-and-expanding-access-to-medicare-coverage/}{supports}
a national health care plan. An astonishing
\href{https://www.politico.com/f/?id=00000169-2b4f-d6dd-ad79-3fef4cf70002}{89
percent} favor higher taxes on the rich to reduce poverty in America.

\hypertarget{the-majority}{%
\subsection{The Majority}\label{the-majority}}

Despite some vocal opposition, many sensible positions enjoy broad
support among Americans.

Net support for Black Lives Matter, in pct. pts.

(Total in support minus those against)

Early June

+20

pct.

pts.

10

More opposition

than support

0

July

2017

2018

July

2019

July

2020

Share of Americans who ...

Support higher taxes on wealthy to reduce poverty

89\%

Think masks should be worn at least some of the time

88

Trust medical scientists on the coronavirus

84

Think we're not doing enough on climate change

67

Net support for Black Lives Matter, in pct. pts.

(Total in support minus those against)

Early June

+20

pct.

pts.

10

More opposition

than support

0

July

2017

Oct.

2018

April

July

Oct.

2019

April

July

Oct.

2020

April

July

Share of Americans who ...

Support higher taxes on wealthy to reduce poverty

89\%

Think masks should be worn at least sometimes

88

Trust medical scientists on the coronavirus

84

Think we're not doing enough on climate change

67

Note: Net support among registered voters; excludes unsure and neutral
responses. Sources: Civiqs (Black Lives Matter support); Morning
Consult/Politico (poverty); Pew Research (masks, climate); New York
Times/Siena College poll (trust). \textbar{} By The New York Times

The sense of opportunity thus is emerging not solely from the wreckage
of past policies but also from new attitudes, particularly among young
people. Half a century ago, there was something to Nixon's claim of a
``silent majority'' that backed his racist dog whistles; today, polls
indicate, the silent majority want
\href{https://gssdataexplorer.norc.org/projects/82650/variables/188/vshow}{\emph{more}}
spending to address racial inequity,
\href{https://www.pewresearch.org/fact-tank/2020/04/21/how-americans-see-climate-change-and-the-environment-in-7-charts/}{\emph{more}}
** effort to address climate change and
\href{https://www.nytimes.com/2020/06/27/upshot/coronavirus-americans-trust-experts.html}{\emph{more}}
input from scientists on how to handle Covid-19.

It's not clear, of course, that these views will translate into wiser
policies. Congress is often more responsive to wealthy donors than to
voter opinions. And while white Americans may chant ``Black Lives
Matter,'' they may not want to back policies to share the bounty that
they have been hogging; few are talking about fixing our unequal system
of local school funding built to transmit advantage from one generation
to the next.

Yet this inchoate movement is gaining ground, and Trump is on the
defensive. In the rural Oregon town where I grew up, most people voted
for Trump in 2016, and until early this year they stuck with him because
they liked his nominations of conservative judges and his pro-gun
stance, but most of all they liked the roaring economy. Now the
collapsing economy and Trump's manifest failures in managing the
pandemic test that support.

In the 1930s the unequivocal nature of Hoover's failures helped win
Roosevelt his mandate and made the New Deal possible. Maybe national
anguish can again be the midwife of progress.

``It is possible that the best thing that could have happened to make
progressive change possible is the crass, self-interested, ineffective
politics of Donald Trump,'' Lizabeth Cohen, a Harvard historian, told
me.

\textbf{*}

But wait! Even if Biden wins with both chambers of Congress --- a huge
if --- this is an age of toxic polarization. Republican senators will
filibuster (if the filibuster survives), conservative judges will
overturn Biden executive orders, and Tucker Carlson and Sean Hannity
will spew venom.

Actually, that sounds rather like the 1930s. Roosevelt was (initially)
blocked by the Supreme Court, and fervently denounced by Father Charles
Coughlin on the right and Senator Huey Long on the left. F.D.R. was
regularly accused of being a ``warmonger'' and a ``fascist dictator,''
or of taking America on the road to Communism. He didn't even have the
full backing of his wife, Eleanor (history vindicated her on most of
their disagreements, such as anti-lynching legislation that she
supported and the internment of Japanese-Americans that she opposed).

Skeptics worry that Trump has permanently damaged American institutions
and norms, in ways that will impair future progress. Perhaps. But Nixon
likewise challenged institutions, norms and the rule of law, and the
result was that Americans came to value them more. One result was the
Democratic tidal wave of 1974.

Like Trump, Nixon took on journalists --- his vice president, Spiro
Agnew, excoriated critics as ``nattering nabobs of negativism'' --- but
ultimately Agnew was convicted of a felony, and Bob Woodward and Carl
Bernstein inspired a generation of kids to become journalists. Me
included.

I often hear Americans say that our country has never been so divided.
That doesn't ring true. Far more than today, households in the 1960s
were riven by civil warfare, with children denouncing parents as
murderers for supporting the Vietnam War and parents despairing of their
offspring as immoral, impractical good-for-nothings who lived in sin,
smoked pot and threatened the nation's future. If we survived the chasms
of the '60s, we can get through this.

``I know we will see a better future,'' President Jimmy Carter told me
recently. ``We have been through many painful crises, some spanning
years, but we have always gotten back on our feet. Sometimes there must
be a reckoning and course correction.''

I reached out to Carter because his administration in the late 1970s
roughly marked the end of the postwar cycle of inclusive capitalism. At
age 95, he's still guardedly optimistic, as is Walter Mondale, his vice
president, a classic liberal who at age 92 --- ``not too many more
years, and I'll be getting old,'' he told me --- said he feels ``a lot
of hope.''

\textbf{*}

History does not unfold smoothly; policies do not ``evolve'' gradually.
Rather, they develop, like animal species, through what evolutionary
biologists call ``punctuated equilibrium'' --- long periods of stasis
and short bouts of intense variation. The change is often driven by
traumas, like the American Revolution or the Great Depression.

Roosevelt was a somewhat conventional, privileged figure who seized upon
the catastrophe of the Depression to transform America. Lyndon B.
Johnson was a corrupt and manipulative Southern politician who seized
upon the Kennedy assassination to pass civil rights legislation and the
Great Society programs.

``F.D.R. wasn't by nature a revolutionary, but out of the trauma of the
Great Depression he helped unleash a revolution that made America a
richer, fairer and better country,'' said Cohen. ``The same is possible
again --- if we get everything right.''

Covid-19 and the Black Lives Matter movement, along with a broad
recognition that America has taken a wrong path, create a similar
opportunity for Joe Biden. While Biden isn't charismatic, he's a
reassuring veteran who knows how the system works and doesn't frighten
voters --- and who thus has a chance, like F.D.R., to be elected with a
mandate and make history.

Some of Biden's aides are telling him to think in such grand terms, and
he seems drawn to the idea. ``I do think we've reached a point, a real
inflection in American history,'' he
\href{https://messaging-custom-newsletters.nytimes.com/template/oakv2?campaign_id=9\&emc=edit_nn_20200714\&instance_id=20286\&nl=the-morning\&productCode=NN\&regi_id=12554495\&segment_id=33336\&te=1\&uri=nyt\%3A\%2F\%2Fnewsletter\%2Fa6802831-0f40-574e-a8a0-3f5125146b8d\&user_id=a949928698e00ffba23b348eabe9c588}{told}
reporters a few days ago. ``And I don't believe it's unlike what
Roosevelt was met with.''

Biden added that ``we have an opportunity to make some really systemic
change,'' but for now his policy positions don't show much sign of that.
He is likely to favor a public option as a path to universal health
coverage, stronger moves on climate change, a higher federal minimum
wage, easier access to college, and jobs programs to reduce inequality.
If enacted, these would put America on a path more like that of Europe
and Canada, but they would be short of Rooseveltian.

Add a universal child care/pre-K program modeled on the military's,
universal dental coverage, Canada-style child allowances to cut child
poverty in half, major investments in K-12 education for disadvantaged
children, ``baby bonds'' to reduce wealth inequality, greater union
protections and ``bandwidth for all'' --- then you are talking history.

Is that a pipe dream? Perhaps. But a series of national crises may have
exposed our failings enough to give us a chance at a do-over.

This hope is not Pollyannaish. It rests on a tragic toll of Covid-19
deaths, and it requires a thousand caveats. Trump might win in November.
If Biden wins, a Republican Senate might stymie his proposals and block
his nominees. Deficits are now so enormous that politics may become a
dispiriting fight about which programs to cut, not which dreams to
finance. Veteran liberals are scarred by memories of unfulfilled hope
that followed Barack Obama's election in 2008, Bill Clinton's in 1992,
Jimmy Carter's in 1976: Hope is the engine oil of campaigns, but it
burns up in the heat of governing.

And yet.

``Hope right now in America is bloodied and battered, but this is the
kind of hope that is successful,'' said Senator Cory Booker, Democrat of
New Jersey. ``It's hope that has lost its naïveté.''

Besieged as we are by plague and crisis, a dollop of this ``calloused
hope,'' as Booker calls it, offers an incentive to persevere. If in the
depths of the Great Depression we could claw a path out and forge a
better country, ``calloused hope'' can guide us once more to a better
place.

\emph{The Times is committed to publishing}
\href{https://www.nytimes.com/2019/01/31/opinion/letters/letters-to-editor-new-york-times-women.html}{\emph{a
diversity of letters}} \emph{to the editor. We'd like to hear what you
think about this or any of our articles. Here are some}
\href{https://help.nytimes.com/hc/en-us/articles/115014925288-How-to-submit-a-letter-to-the-editor}{\emph{tips}}\emph{.
And here's our email:}
\href{mailto:letters@nytimes.com}{\emph{letters@nytimes.com}}\emph{.}

Advertisement

\protect\hyperlink{after-bottom}{Continue reading the main story}

\hypertarget{site-index}{%
\subsection{Site Index}\label{site-index}}

\hypertarget{site-information-navigation}{%
\subsection{Site Information
Navigation}\label{site-information-navigation}}

\begin{itemize}
\tightlist
\item
  \href{https://help.nytimes.com/hc/en-us/articles/115014792127-Copyright-notice}{©~2020~The
  New York Times Company}
\end{itemize}

\begin{itemize}
\tightlist
\item
  \href{https://www.nytco.com/}{NYTCo}
\item
  \href{https://help.nytimes.com/hc/en-us/articles/115015385887-Contact-Us}{Contact
  Us}
\item
  \href{https://www.nytco.com/careers/}{Work with us}
\item
  \href{https://nytmediakit.com/}{Advertise}
\item
  \href{http://www.tbrandstudio.com/}{T Brand Studio}
\item
  \href{https://www.nytimes.com/privacy/cookie-policy\#how-do-i-manage-trackers}{Your
  Ad Choices}
\item
  \href{https://www.nytimes.com/privacy}{Privacy}
\item
  \href{https://help.nytimes.com/hc/en-us/articles/115014893428-Terms-of-service}{Terms
  of Service}
\item
  \href{https://help.nytimes.com/hc/en-us/articles/115014893968-Terms-of-sale}{Terms
  of Sale}
\item
  \href{https://spiderbites.nytimes.com}{Site Map}
\item
  \href{https://help.nytimes.com/hc/en-us}{Help}
\item
  \href{https://www.nytimes.com/subscription?campaignId=37WXW}{Subscriptions}
\end{itemize}
