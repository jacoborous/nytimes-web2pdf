Sections

SEARCH

\protect\hyperlink{site-content}{Skip to
content}\protect\hyperlink{site-index}{Skip to site index}

\href{https://myaccount.nytimes.com/auth/login?response_type=cookie\&client_id=vi}{}

\href{https://www.nytimes.com/section/todayspaper}{Today's Paper}

A Retelling of American History --- in Neon

\url{https://nyti.ms/2ZKVTxP}

\begin{itemize}
\item
\item
\item
\item
\item
\end{itemize}

Advertisement

\protect\hyperlink{after-top}{Continue reading the main story}

Supported by

\protect\hyperlink{after-sponsor}{Continue reading the main story}

True Believers

\hypertarget{a-retelling-of-american-history--in-neon}{%
\section{A Retelling of American History --- in
Neon}\label{a-retelling-of-american-history--in-neon}}

Maya Stovall, known for her dance performances in public spaces, shares
a new artwork.

\includegraphics{https://static01.nyt.com/images/2020/07/20/t-magazine/art/20tmag-artists-slide-UIOT/20tmag-artists-slide-UIOT-articleLarge.jpg?quality=75\&auto=webp\&disable=upscale}

By Maya Stovall

\begin{itemize}
\item
  Published July 20, 2020Updated July 21, 2020
\item
  \begin{itemize}
  \item
  \item
  \item
  \item
  \item
  \end{itemize}
\end{itemize}

\emph{In this new series, The Artists, an installment of which will
publish every day this week and regularly thereafter, T will highlight a
recent or little-shown work by a Black artist, along with a few words
from that artist, putting the work into context. First up is a piece by}
\href{https://mayastovall.com/home.html}{\emph{Maya Stovall}}\emph{, who
is best known for her dance performances in public spaces, as documented
in her ``Liquor Store Theater'' video series, which will screen this
summer as part of the citywide digital-art exhibition
``}\href{https://www.artmiledetroit.com/}{\emph{Art Mile
Detroit}}\emph{.''}

\href{https://www.nytimes.com/issue/t-magazine/2020/07/02/true-believers-art-issue}{\includegraphics{https://static01.nyt.com/newsgraphics/2020/06/29/tmag-art-embeds-new/assets/images/art_issue_gif_special_editon.gif}}

\textbf{Name:} Maya Stovall

\textbf{Age:} 37

\textbf{Based in:} Los Angeles

\textbf{Originally from:} Detroit

\textbf{When and where did you make this work?} ``1526 (NASDAQ: FAANG)''
began as a research project in 2018 and continues as an ongoing project.
Over the course of a year, I gathered United States historical archives
and volumes. From tens of thousands of pages of research, I developed a
series of dates, from 1526 to 2019, that reflect critical moments in
U.S. history. I am obsessed with words, numbers and time. I thought that
neon, which I've considered working with for a long time, would be a way
to emblazon these dates and facts into collective memory. The research
for this work was done in Pomona, Calif., where I'm an assistant
professor at California Polytechnic State University, and also in San
Francisco, where I spent quite a lot of time in 2019. The neons are
these tiny little dates, in buttercream with a soft glow. Postcards
accompany the neons. They summarize the actions and events associated
with each date.

\includegraphics{https://static01.nyt.com/images/2020/07/20/t-magazine/art/20tmag-artists-slide-68N4/20tmag-artists-slide-68N4-articleLarge.jpg?quality=75\&auto=webp\&disable=upscale}

\textbf{Can you describe what's going on in the work?} The work links
the past to the present and attempts to unhinge our minds from the
mythology we've been fed. And neon is fun. I like data and information,
and that is happening, too. I don't have an interest in emotions or
feelings. I like facts. These works reflect that. For instance, in xxxx
year, something monumental happened. We look at screens constantly, and
increasingly our economy is dominated by technology stocks --- of course
Facebook, Amazon, Apple, Netflix and Google are among the biggest. From
where did this outrageous accumulation of capital emerge? Well,
certainly from the ongoing exploitation of people, markets, circuits and
flows. The works move through significant but quite varied moments in
U.S. history, ranging from petitions to end human trafficking and
genocide (``1661''), to a woman human-trafficking and genocide survivor
writing a letter to an abolitionist journal scaffolding feminist theory
(``1827''), to
\href{https://www.nytimes.com/2018/06/27/magazine/adrian-pipers-self-imposed-exile-from-america-and-from-race-itself.html}{Adrian
Piper} establishing conceptual art (``1968''), to the
\href{https://www.nytimes.com/2014/07/19/us/protesters-picket-detroit-over-move-to-shut-off-water.html}{Detroit
mass water shut-off crisis}, which continues to the present day
(``2014''\emph{)} and more. There are 44 dates in the first iteration of
the series.

Image

Stovall's ``2014 (1526 NASDAQ: FAANG), no. 41'' (2019).Credit...Photo by
Clare Gatto. Courtesy of Reyes Finn

Image

The postcard that accompanies the corresponding neon date as part of the
artist's ``2014 (1526 NASDAQ: FAANG), no. 41'' (2019).Credit...Courtesy
of Reyes Finn

\textbf{What inspired you to make this work?} Imagine if Germany, for
instance, taught its schoolchildren that Nazism was an economic system
that, over time, was abolished, and that Nazism truly had ideals of hope
and freedom that were simply never realized. Imagine if Germany called
Nazism's victims slaves rather than survivors. Here in the United
States, unfortunately, our collective brains have been glued to such
hegemony that we actually teach schoolchildren such lies --- for
instance, the lie that we have a country built on democracy, freedom and
liberalism. This great big lie that must be upended if we are ever to
move forward. There is no democracy, freedom or liberalism in the United
States. The way Americans have been taught to think about United States
history is, in a word, criminal.

Today, Americans are victims and survivors of legalized human
trafficking, genocide, rape, murderous policing, exclusion from credit
markets, job markets, education, unjust incarceration policies and more,
spanning every aspect of existence --- from philosophy to city life.
There has yet to be a reckoning where the U.S. government, history
books, popular narratives, etc., reflect reality. The people
\href{https://www.nytimes.com/interactive/2020/07/03/us/george-floyd-protests-crowd-size.html}{protesting
in the streets} across the country and the world now, as a global
pandemic intersects with police brutality, know these realities, on some
level, and they are exasperated. The philosophy, the history writing,
the history-telling, the language, the marketing --- everything
absolutely must change about how we think if this country is to right
its murderous wrongs. There is nothing emotional or up for debate here.
These are facts that require much organization, documentation,
discussion, correction, **** and this series is interested in that.

\textbf{What's the work of art in any medium that changed your life?}
There are many works with which I have an intense relationship, but the
first that came to mind at this question is the work I'll share. I am
thinking about conceptual artist
\href{http://www.deshawndumas.com/}{DeShawn Dumas}'s ongoing series of
Glock-shot glass paintings called ``Ballistic Testimonies.'' The series
is this feedback loop or networked array of risk, calculation and
violence. The artist shoots his work with a Glock firearm through these
laminated glass Rothko-like monochromes that he first paints. The works
are monuments to victims of police murder, one of whom is the artist's
brother Derrick Conner. When I saw the work for the first time, I was
devastated and moved by it. I am attracted to quite intellectual and
conceptual, aesthetically cold, almost heartless-looking work. For me,
think:
\href{https://www.nytimes.com/2018/06/15/t-magazine/mary-heilmann-larry-bell-conversation.html}{Larry
Bell},
\href{https://tmagazine.blogs.nytimes.com/2012/02/24/artifacts-mary-corse/}{Mary
Corse},
\href{https://tmagazine.blogs.nytimes.com/2010/03/25/just-looking-eva-hesse/}{Eva
Hesse},
\href{https://www.nytimes.com/2018/06/18/t-magazine/glenn-ligon-adrian-piper-art.html}{Glenn
Ligon}. Dumas's works come off like this, and yet the series has an
erotic, painterly quality and crystallizes the construction of the
simultaneous logic and illogic of violence that courses through
policing, through murder and through walking down the street. This work
changed my life in some of the most profound ways you could imagine.

\hypertarget{true-believers-art-issue}{%
\subsubsection{\texorpdfstring{\href{https://www.nytimes.com/issue/t-magazine/2020/07/02/true-believers-art-issue}{True
Believers Art
Issue}}{True Believers Art Issue}}\label{true-believers-art-issue}}

Advertisement

\protect\hyperlink{after-bottom}{Continue reading the main story}

\hypertarget{site-index}{%
\subsection{Site Index}\label{site-index}}

\hypertarget{site-information-navigation}{%
\subsection{Site Information
Navigation}\label{site-information-navigation}}

\begin{itemize}
\tightlist
\item
  \href{https://help.nytimes.com/hc/en-us/articles/115014792127-Copyright-notice}{©~2020~The
  New York Times Company}
\end{itemize}

\begin{itemize}
\tightlist
\item
  \href{https://www.nytco.com/}{NYTCo}
\item
  \href{https://help.nytimes.com/hc/en-us/articles/115015385887-Contact-Us}{Contact
  Us}
\item
  \href{https://www.nytco.com/careers/}{Work with us}
\item
  \href{https://nytmediakit.com/}{Advertise}
\item
  \href{http://www.tbrandstudio.com/}{T Brand Studio}
\item
  \href{https://www.nytimes.com/privacy/cookie-policy\#how-do-i-manage-trackers}{Your
  Ad Choices}
\item
  \href{https://www.nytimes.com/privacy}{Privacy}
\item
  \href{https://help.nytimes.com/hc/en-us/articles/115014893428-Terms-of-service}{Terms
  of Service}
\item
  \href{https://help.nytimes.com/hc/en-us/articles/115014893968-Terms-of-sale}{Terms
  of Sale}
\item
  \href{https://spiderbites.nytimes.com}{Site Map}
\item
  \href{https://help.nytimes.com/hc/en-us}{Help}
\item
  \href{https://www.nytimes.com/subscription?campaignId=37WXW}{Subscriptions}
\end{itemize}
