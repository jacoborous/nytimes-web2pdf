Sections

SEARCH

\protect\hyperlink{site-content}{Skip to
content}\protect\hyperlink{site-index}{Skip to site index}

\href{https://www.nytimes.com/section/politics}{Politics}

\href{https://myaccount.nytimes.com/auth/login?response_type=cookie\&client_id=vi}{}

\href{https://www.nytimes.com/section/todayspaper}{Today's Paper}

\href{/section/politics}{Politics}\textbar{}Defying Trump, Lawmakers
Move to Strip Military Bases of Confederate Names

\url{https://nyti.ms/3fPGVfF}

\begin{itemize}
\item
\item
\item
\item
\item
\end{itemize}

Advertisement

\protect\hyperlink{after-top}{Continue reading the main story}

Supported by

\protect\hyperlink{after-sponsor}{Continue reading the main story}

\hypertarget{defying-trump-lawmakers-move-to-strip-military-bases-of-confederate-names}{%
\section{Defying Trump, Lawmakers Move to Strip Military Bases of
Confederate
Names}\label{defying-trump-lawmakers-move-to-strip-military-bases-of-confederate-names}}

The push by both Republicans and Democrats in Congress sets up an
election-year veto fight with the president, and shows how he has
isolated himself even from members of his own party on the issue.

\includegraphics{https://static01.nyt.com/images/2020/07/20/us/politics/20dc-bases/20dc-bases-articleLarge.jpg?quality=75\&auto=webp\&disable=upscale}

\href{https://www.nytimes.com/by/catie-edmondson}{\includegraphics{https://static01.nyt.com/images/2019/11/20/us/politics/catie-edmonson-twitter-chatblog/catie-edmonson-twitter-chatblog-thumbLarge.png}}\href{https://www.nytimes.com/by/emily-cochrane}{\includegraphics{https://static01.nyt.com/images/2018/11/28/multimedia/author-emily-cochrane/author-emily-cochrane-thumbLarge-v3.png}}

By \href{https://www.nytimes.com/by/catie-edmondson}{Catie Edmondson}
and \href{https://www.nytimes.com/by/emily-cochrane}{Emily Cochrane}

\begin{itemize}
\item
  Published July 20, 2020Updated July 22, 2020
\item
  \begin{itemize}
  \item
  \item
  \item
  \item
  \item
  \end{itemize}
\end{itemize}

WASHINGTON --- Representative Don Bacon, a Republican, had a blunt
message for President Trump when a White House aide called him
personally early this month and asked that he abandon legislation to
strip the names of Confederate leaders from military bases.

``You're wrong --- you need to change,'' Mr. Bacon, a second-term
Nebraskan and former Air Force brigadier general, told the official, he
said in an interview. ``We're the party of Lincoln, the party of
emancipation; we're not the party of Jim Crow. We should be on the right
side of this issue.''

The sharp exchange between the White House aide and Mr. Bacon, who is
facing an unexpectedly difficult re-election race, reflects just how
much Mr. Trump has isolated himself --- even from members of his own
party who rarely break with him --- on an issue that has come to the
forefront of the political debate amid a national outcry for racial
justice.

It will take center stage on Capitol Hill this week, when the House and
Senate each consider sweeping annual military bills that contain
bipartisan measures mandating that the Pentagon remove Confederate names
from military assets. Mr. Trump, who has sought to stoke cultural and
political divisions over symbols of the Confederacy, has said he would
veto any bill with such a requirement.

The disconnect has raised the prospect of a rare, election-year clash
between congressional Republicans and Mr. Trump on the military bill,
the measure that authorizes pay raises for American troops and is
regarded as must-pass legislation. Despite the president's unapologetic
stance, most Republicans have been unwilling to defend symbols of the
Confederacy, and some have warned the president not to force the first
veto override of his presidency.

The House voted on Monday to begin consideration of the bill and is
expected to pass it this week, as the Senate debates a similar measure
slated to be approved next week.

Mr. Trump, who has positioned himself against a growing movement for
racial justice, renewed his veto threat in an interview aired Sunday.
Mr. Trump told Fox News's Chris Wallace that he rejected the counsel of
military leaders like Gen. Mark A. Milley, the chairman of the Joint
Chiefs of Staff,
\href{https://www.nytimes.com/2020/07/09/us/politics/milley-trump-confederate-base-names.html}{who
has called for taking ``a hard look''} at changing the names of the
bases.

``We won two world wars, two world wars, beautiful world wars that were
vicious and horrible, and we won them out of Fort Bragg,'' Mr. Trump
said. ``We won them out of all of these forts, and now they want to
throw those names away.''

On Monday, Senator James M. Inhofe of Oklahoma, the Republican chairman
of the Senate Armed Services Committee, batted away the prospect of a
veto showdown.

``He's threatened several times to do that, but he also knows that's the
most important bill of the year,'' Mr. Inhofe said in a brief interview.

The measures cruising through Congress with bipartisan support,
including Mr. Bacon's proposal and a separate one in the Senate, led by
Senator Elizabeth Warren, Democrat of Massachusetts, go much further
than an
\href{https://www.nytimes.com/2020/07/17/us/politics/pentagon-trump-confederate-symbols.html}{order
issued by the Pentagon late last week} that effectively banned displays
of the Confederate flag on military installations around the world. Ms.
Warren's amendment would require the Pentagon to strip all military
assets of names and symbols of the Confederacy within three years.
Another measure in
\href{https://appropriations.house.gov/news/press-releases/appropriations-committee-releases-fiscal-year-2021-defense-funding-bill}{House
Democrats' military spending bill} would provide the Army with \$1
million to rename the installations and other assets.

Few Republicans in Congress have rallied to Mr. Trump's side on the
issue. Senate Republican leaders have moved to avoid a contentious
showdown on the issue, ducking a vote on a proposal by Senator Josh
Hawley, Republican of Missouri, to remove Ms. Warren's requirement and
replace it with a weaker measure that would instruct the Pentagon to
study the issue.

``This cancel movement seeks to divide us, not unite; to erase our
history, rather than to reckon with it,'' Mr. Hawley said in a speech on
the Senate floor, accusing proponents of Ms. Warren's measure of being
driven by ``a kind of woke fundamentalism.''

Taking such a vote on the Senate floor would have squeezed several
Republicans in tight re-election battles. And Republican leaders in both
chambers on Capitol Hill have been largely supportive of the effort to
rename the bases.

Senator Mitch McConnell of Kentucky, the majority leader,
\href{https://www.wsj.com/articles/mitch-mcconnell-signals-limits-on-race-related-policy-changes-11594733555}{told
The Wall Street Journal last week} that he would not block the effort to
rename the bases, and in an
\href{https://www.wdrb.com/news/one-on-one-sen-mitch-mcconnell-backs-rebranding-military-bases-named-for-confederate-leaders/article_78838c9c-bfd4-11ea-babc-3b697035a8fa.html}{interview
with a Louisville radio station}, he said he didn't ``have any problem''
with renaming the bases for ``people who didn't rebel against the
country.'' He has
\href{https://thehill.com/homenews/senate/505475-mcconnell-trump-shouldnt-veto-defense-bill-over-renaming-confederate-bases}{urged
the president not to veto the bill}.

``The issue of Army bases being named after Confederate generals is a
legitimate concern in the times in which we live,'' said Senator Lindsey
Graham, Republican of South Carolina. ``I'm OK with a process that the
Senate came up with. And there's a lot of good things in this bill.''

Representative Mac Thornberry of Texas, the top Republican on the House
Armed Services Committee, opposed the measure's deadlines, saying that
it did not give the Pentagon enough time to facilitate community
discussion around the change and that the end goal should be ``increased
understanding and changed hearts.''

``My personal opinion is that the names of some, if not all, of these
installations should be changed,'' Mr. Thornberry said.

Advertisement

\protect\hyperlink{after-bottom}{Continue reading the main story}

\hypertarget{site-index}{%
\subsection{Site Index}\label{site-index}}

\hypertarget{site-information-navigation}{%
\subsection{Site Information
Navigation}\label{site-information-navigation}}

\begin{itemize}
\tightlist
\item
  \href{https://help.nytimes.com/hc/en-us/articles/115014792127-Copyright-notice}{©~2020~The
  New York Times Company}
\end{itemize}

\begin{itemize}
\tightlist
\item
  \href{https://www.nytco.com/}{NYTCo}
\item
  \href{https://help.nytimes.com/hc/en-us/articles/115015385887-Contact-Us}{Contact
  Us}
\item
  \href{https://www.nytco.com/careers/}{Work with us}
\item
  \href{https://nytmediakit.com/}{Advertise}
\item
  \href{http://www.tbrandstudio.com/}{T Brand Studio}
\item
  \href{https://www.nytimes.com/privacy/cookie-policy\#how-do-i-manage-trackers}{Your
  Ad Choices}
\item
  \href{https://www.nytimes.com/privacy}{Privacy}
\item
  \href{https://help.nytimes.com/hc/en-us/articles/115014893428-Terms-of-service}{Terms
  of Service}
\item
  \href{https://help.nytimes.com/hc/en-us/articles/115014893968-Terms-of-sale}{Terms
  of Sale}
\item
  \href{https://spiderbites.nytimes.com}{Site Map}
\item
  \href{https://help.nytimes.com/hc/en-us}{Help}
\item
  \href{https://www.nytimes.com/subscription?campaignId=37WXW}{Subscriptions}
\end{itemize}
