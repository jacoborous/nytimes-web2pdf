Sections

SEARCH

\protect\hyperlink{site-content}{Skip to
content}\protect\hyperlink{site-index}{Skip to site index}

\href{https://www.nytimes.com/section/politics}{Politics}

\href{https://myaccount.nytimes.com/auth/login?response_type=cookie\&client_id=vi}{}

\href{https://www.nytimes.com/section/todayspaper}{Today's Paper}

\href{/section/politics}{Politics}\textbar{}Trump Defends His Cognitive
Testing Results on Fox News. Again.

\url{https://nyti.ms/2ZWljJf}

\begin{itemize}
\item
\item
\item
\item
\item
\item
\end{itemize}

Advertisement

\protect\hyperlink{after-top}{Continue reading the main story}

Supported by

\protect\hyperlink{after-sponsor}{Continue reading the main story}

\hypertarget{trump-defends-his-cognitive-testing-results-on-fox-news-again}{%
\section{Trump Defends His Cognitive Testing Results on Fox News.
Again.}\label{trump-defends-his-cognitive-testing-results-on-fox-news-again}}

The president again tried to defend his own mental fitness for office
--- and disparage Joe Biden's --- by frequently repeating a memory
sequence: ``Person. Woman. Man. Camera. TV.''

\includegraphics{https://static01.nyt.com/images/2020/07/22/us/politics/22dc-trumpfox/merlin_174844527_ac2275eb-3f53-4e91-90a2-39d6e1bf2cdc-articleLarge.jpg?quality=75\&auto=webp\&disable=upscale}

\href{https://www.nytimes.com/by/katie-rogers}{\includegraphics{https://static01.nyt.com/images/2018/06/12/multimedia/author-katie-rogers/author-katie-rogers-thumbLarge-v2.png}}

By \href{https://www.nytimes.com/by/katie-rogers}{Katie Rogers}

\begin{itemize}
\item
  July 22, 2020
\item
  \begin{itemize}
  \item
  \item
  \item
  \item
  \item
  \item
  \end{itemize}
\end{itemize}

WASHINGTON --- ``Person. Woman. Man. Camera. TV.''

President Trump again sought to showcase his mental fitness on
television by reciting, over and over again in an interview broadcast on
Wednesday evening, what he said was a sample cognitive testing sequence.

For the better part of a month, Mr. Trump, 74, has made repeated
appearances on Fox News to brag about acing
\href{https://www.nytimes.com/2020/07/10/us/politics/trump-cognitive-test-health.html}{a
cognitive test} he said he recently took at Walter Reed National
Military Medical Center, first with
\href{https://www.nytimes.com/2020/07/10/us/politics/trump-cognitive-test-health.html}{Sean
Hannity} and again with
\href{https://www.nytimes.com/2020/07/19/us/politics/trump-fox-interview-coronavirus-race.html}{Chris
Wallace} on ``Fox News Sunday.'' All the while, the White House has not
disclosed details about when the president underwent the testing or why.

During the most recent interview --- an appearance with Dr. Marc K.
Siegel, a professor of medicine at New York University and a medical
analyst for Fox News --- Mr. Trump tried to defend his own mental
fitness for office by outlining the particulars of the test he said he
had taken, and by questioning the acuity of former Vice President Joseph
R. Biden Jr., 77, the presumptive Democratic presidential nominee.

``It's really something that's been great,'' Mr. Trump said, referring
to being the president. ``But you need stamina. You need physical
health, and you need mental health.''

Then Mr. Trump seemed to offer differing timelines for when he had taken
the test.

First, the president said that he had asked a physician during a
hospital visit ``a little less than a year ago'' if there was a test he
could take to prove his mental acuity to the news media.

It is unclear when that visit would have occurred: Last month, the White
House released a summary of Mr. Trump's health, but not an annual
physical report, and did not explain a
\href{https://www.nytimes.com/2020/06/03/us/politics/trump-physical-hydroxychloroquine.html}{highly
unusual unannounced visit} last fall to Walter Reed. The summary also
did not indicate whether he had undergone cognitive testing.

Then, Mr. Trump said that he had asked Dr. Ronny L. Jackson, who has not
been his physician since 2018, if there was an acuity test he could
take. That year --- after the book
\href{https://www.nytimes.com/2018/01/08/books/review/michael-wolff-fire-and-fury-trump-white-house.html}{``Fire
and Fury''} described some of Mr. Trump's advisers questioning his
fitness for office --- Dr. Jackson said that the president had received
a score of 30 out of 30 on the Montreal Cognitive Assessment.

That test, also called the
\href{https://www.nytimes.com/article/trump-cognitive-test.html}{MOCA},
has been criticized by experts as too blunt an instrument that does not
rule out declines in reasoning or memory, or difficulties with planning
or judgment.

``I said to the doctor, it was Dr. Ronny Jackson. I said, `Is there some
kind of a test, an acuity test?''' Mr. Trump recalled on Wednesday.
``And he said, `There actually is,' and he named it, whatever it might
be.''

Then the president elaborated for several minutes.

``It was 30 to 35 questions,'' Mr. Trump said. ``The first questions are
very easy. The last questions are much more difficult. Like a memory
question. It's, like, you'll go: Person. Woman. Man. Camera. TV. So they
say, `Could you repeat that?' So I said, `Yeah. It's: Person. Woman.
Man. Camera. TV.'''

```OK, that's very good. If you get it in order you get extra points,'''
Mr. Trump said a doctor told him. ``OK, now he's asking you other
questions, other questions, and then, 10 minutes, 15, 20 minutes later
they say, `Remember that first question --- not the first --- but the
10th question? Give us that again. Can you do that again?'''

``And you go: `Person. Woman. Man. Camera. TV,''' Mr. Trump said. ``If
you get it in order, you get extra points.''

``They said nobody gets it in order,'' Mr. Trump said. ``It's actually
not that easy, but for me, it was easy. And that's not an easy question.
In other words, they ask it to you, they give you five names and you
have to repeat 'em. And that's OK. If you repeat 'em out of order, it's
OK, but, you know, it's not as good. But when you go back about 20, 25
minutes later and they say go back to that --- they don't tell you this
--- `Go back to that question and repeat 'em, can you do it?' And you
go: `Person. Woman. Man. Camera. TV.'

``They say, `That's amazing. How did you do that?''' Mr. Trump
continued. ``I do it because I have, like, a good memory, because I'm
cognitively there. Now, Joe should take that test, because something's
going on. And, and, I say this with respect. I mean --- going to
probably happen to all of us, right? You know? It's going to happen.''

Dr. Siegel did not ask follow-up questions.

In an interview broadcast on Sunday, when told that Mr. Biden was chosen
in a Fox poll as the more mentally sound candidate, Mr. Trump disputed
that finding and defended his cognitive test results to Mr. Wallace, who
said he had taken the same test that the president had boasted about
acing. Mr. Wallace pointed out that one of the questions asked to
identify an elephant.

``It's all misrepresentation,'' Mr. Trump said. ``Because, yes, the
first few questions are easy, but I'll bet you couldn't even answer the
last five questions. I'll bet you couldn't. They get very hard, the last
five questions.''

In \href{https://www.youtube.com/watch?v=3akO234sQi0}{another portion}
of the Wednesday interview, he defended his track record as president
during the pandemic, reiterating
\href{https://www.nytimes.com/2020/07/21/us/politics/trump-coronavirus-masks.html}{a
call for Americans to wear masks} to combat the coronavirus.

But Mr. Trump appeared to shift away from more sober assessments he has
delivered about the pandemic, which
\href{https://www.nytimes.com/interactive/2020/us/coronavirus-us-cases.html}{has
killed more than 143,000 Americans}, by saying that coronavirus testing
was ``overrated'' and ``makes us look bad.''

He then accused Democrats of sounding the alarm over a virus for
political reasons.

``Watch,'' Mr. Trump said, ``on Nov. 4, everything will open up.''

Advertisement

\protect\hyperlink{after-bottom}{Continue reading the main story}

\hypertarget{site-index}{%
\subsection{Site Index}\label{site-index}}

\hypertarget{site-information-navigation}{%
\subsection{Site Information
Navigation}\label{site-information-navigation}}

\begin{itemize}
\tightlist
\item
  \href{https://help.nytimes.com/hc/en-us/articles/115014792127-Copyright-notice}{©~2020~The
  New York Times Company}
\end{itemize}

\begin{itemize}
\tightlist
\item
  \href{https://www.nytco.com/}{NYTCo}
\item
  \href{https://help.nytimes.com/hc/en-us/articles/115015385887-Contact-Us}{Contact
  Us}
\item
  \href{https://www.nytco.com/careers/}{Work with us}
\item
  \href{https://nytmediakit.com/}{Advertise}
\item
  \href{http://www.tbrandstudio.com/}{T Brand Studio}
\item
  \href{https://www.nytimes.com/privacy/cookie-policy\#how-do-i-manage-trackers}{Your
  Ad Choices}
\item
  \href{https://www.nytimes.com/privacy}{Privacy}
\item
  \href{https://help.nytimes.com/hc/en-us/articles/115014893428-Terms-of-service}{Terms
  of Service}
\item
  \href{https://help.nytimes.com/hc/en-us/articles/115014893968-Terms-of-sale}{Terms
  of Sale}
\item
  \href{https://spiderbites.nytimes.com}{Site Map}
\item
  \href{https://help.nytimes.com/hc/en-us}{Help}
\item
  \href{https://www.nytimes.com/subscription?campaignId=37WXW}{Subscriptions}
\end{itemize}
