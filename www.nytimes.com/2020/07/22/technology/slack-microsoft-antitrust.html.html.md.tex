Sections

SEARCH

\protect\hyperlink{site-content}{Skip to
content}\protect\hyperlink{site-index}{Skip to site index}

\href{https://www.nytimes.com/section/technology}{Technology}

\href{https://myaccount.nytimes.com/auth/login?response_type=cookie\&client_id=vi}{}

\href{https://www.nytimes.com/section/todayspaper}{Today's Paper}

\href{/section/technology}{Technology}\textbar{}Slack Accuses Microsoft
of Illegally Crushing Competition

\url{https://nyti.ms/2OJnSb5}

\begin{itemize}
\item
\item
\item
\item
\item
\end{itemize}

Advertisement

\protect\hyperlink{after-top}{Continue reading the main story}

Supported by

\protect\hyperlink{after-sponsor}{Continue reading the main story}

\hypertarget{slack-accuses-microsoft-of-illegally-crushing-competition}{%
\section{Slack Accuses Microsoft of Illegally Crushing
Competition}\label{slack-accuses-microsoft-of-illegally-crushing-competition}}

The complaint, filed in Europe, threatens Microsoft's recent ability to
avoid regulatory scrutiny.

\includegraphics{https://static01.nyt.com/images/2020/07/22/business/22slack-1/22slack-1-articleLarge.jpg?quality=75\&auto=webp\&disable=upscale}

\href{https://www.nytimes.com/by/steve-lohr}{\includegraphics{https://static01.nyt.com/images/2018/02/20/multimedia/author-steve-lohr/author-steve-lohr-thumbLarge.jpg}}

By \href{https://www.nytimes.com/by/steve-lohr}{Steve Lohr}

\begin{itemize}
\item
  July 22, 2020
\item
  \begin{itemize}
  \item
  \item
  \item
  \item
  \item
  \end{itemize}
\end{itemize}

Microsoft is undeniably one of the Big Tech elite, given its size,
wealth and stock market value. But the software giant has stood apart
from Google, Facebook, Amazon and Apple in one important respect:
Microsoft, once the bully of the tech world, has escaped
\href{https://www.nytimes.com/2018/05/05/world/europe/margrethe-vestager-silicon-valley-data-privacy.html}{antitrust
scrutiny} so far.

Now Slack Technologies, whose popular chat and collaboration software
has become embedded in the daily routines of millions of workers at
thousands of companies, is hoping to change that.

Slack said on Wednesday that it had filed a complaint against Microsoft
with the European Commission, accusing the tech giant of using its
market power to try to crush the upstart rival.

Slack claims that Microsoft has illegally tied its collaboration
software, Microsoft Teams, to its dominant suite of productivity
programs, Microsoft Office, which includes Outlook, Word, Excel and
PowerPoint. That bundling tactic, Slack contends, is part of a pattern
of anticompetitive behavior by Microsoft.

``Slack threatens Microsoft's hold on business email, the cornerstone of
Office, which means Slack threatens Microsoft's lock on enterprise
software,'' Jonathan Prince, vice president of communications and policy
at Slack, said in a statement.

Slack's complaint is just a first step. The European Commission must
decide if a formal investigation is warranted. In recent years, European
regulators have more aggressively pursued antitrust actions against
large tech companies than American officials.

Federal and state regulators in the United States are investigating
whether the other tech giants have broken antitrust laws. On Monday, the
chief executives of Apple, Amazon, Alphabet (which owns Google) and
Facebook will testify before Congress, which is also looking into them.

Under European rules, the Slack complaint is not made public. But in a
news conference, David Schellhase, Slack's general counsel, said the
company sought an order to remove Teams from Office, make it a
stand-alone product and charge a ``fair price.''

In a statement, Microsoft said it was ``committed to offering customers
not only the best of new innovation, but a wide variety of choice in how
they purchase and use the product.'' And the company said it looked
forward to ``providing additional information to the European Commission
and answering any questions they may have.''

The Slack-Microsoft confrontation has some echoes of the internet
browser competition in the 1990s. The browser wars led to a landmark
federal antitrust case against Microsoft in the United States that found
the company repeatedly violated the nation's antitrust laws. Europe also
ruled against Microsoft.

The internet browser was a layer of software that could be a gateway to
online computing. Developers could write software applications that ran
on the browser, potentially undermining the role of Microsoft's Windows
operating system, the dominant technology of the personal computer era.

The browser was a rival computing platform. And Netscape Communications,
the commercial pioneer of browser software, was Microsoft's enemy.

Today, Slack serves as a gateway to online work for many people.
Developers can write apps that run on Slack. And it is a nascent
challenge to one part of Microsoft's business. Now, online cloud
software is the major platform, and Teams is included in Microsoft's
Office 365 cloud suite.

``Online collaboration platforms and related tools have become as
important to us as smartphones and computers,'' said Michael Cusumano, a
professor at the Sloan School of Management at the Massachusetts
Institute of Technology.

\includegraphics{https://static01.nyt.com/images/2020/07/22/business/22slack-2/merlin_162038460_e0324705-c95f-40af-9825-72bc254f3e35-articleLarge.jpg?quality=75\&auto=webp\&disable=upscale}

Whether a software program like Slack could emerge as a genuine business
threat to Microsoft is uncertain, industry analysts say. Microsoft has
positioned Teams as an all-in-one stop for online video meetings, calls,
chat and collaboration, and it works seamlessly with Microsoft's Office
software.

``For Microsoft, Teams is increasingly where online work is done,'' said
Wayne Kurtzman, an analyst at IDC, a technology research firm. ``It is
becoming a platform for Microsoft.''

Slack is making its complaint as adoption of collaboration technology is
surging. Slack, Microsoft Teams and Zoom are all experiencing huge
demand because of the coronavirus-induced shutdowns that have forced
much of the work force to toil from home.

In April, when the company reported its quarterly financial results,
Microsoft said Teams had 75 million daily users, more than double the
number in early March. At the time, Satya Nadella, Microsoft's chief
executive, said, ``We've seen two years' worth of digital transformation
in two months.''

Microsoft on Wednesday reported \$38 billion in revenue in the three
months that ended in June, a 13 percent jump. Its operating profit
increased 8 percent to \$13.4 billion, or \$1.46 a share. Both the
company's sales and earnings per share surpassed Wall Street estimates.

Slack, which was founded in 2014, has enjoyed rapid growth this year.
Last month, reporting the results for its quarter ended in April, Slack
said its revenue had jumped 50 percent to \$202 million. It has more
than 122,000 paying customers, typically companies with annual licenses,
which was a 28 percent increase from the year-earlier quarter.

Mr. Schellhase, the Slack general counsel, said, ``Microsoft is
reverting to past behavior.''

But the illegal-tying claim, which Slack makes in its complaint, was not
resolved in the federal browser case. Microsoft was found to have
engaged in a range of illegal tactics to thwart competition, with
contract restrictions and threats. An appeals court upheld those claims
but sent the tying claim back to the lower court for reconsideration.
The case, which was brought by the Clinton administration, was settled
early in the Bush administration.

Slack contends that bundling Teams with Office is ``clearly a violation
of European law,'' said Mr. Prince, the company's vice president for
policy.

\href{https://cloud.google.com/blog/products/g-suite/introducing-your-new-home-for-work-in-gsuite}{Google
announced last week}that it was more tightly integrating video, chat and
email into its GSuite bundle of products. But the difference, Slack
says, is that Google is not a dominant company in business software, as
Microsoft is.

Advertisement

\protect\hyperlink{after-bottom}{Continue reading the main story}

\hypertarget{site-index}{%
\subsection{Site Index}\label{site-index}}

\hypertarget{site-information-navigation}{%
\subsection{Site Information
Navigation}\label{site-information-navigation}}

\begin{itemize}
\tightlist
\item
  \href{https://help.nytimes.com/hc/en-us/articles/115014792127-Copyright-notice}{©~2020~The
  New York Times Company}
\end{itemize}

\begin{itemize}
\tightlist
\item
  \href{https://www.nytco.com/}{NYTCo}
\item
  \href{https://help.nytimes.com/hc/en-us/articles/115015385887-Contact-Us}{Contact
  Us}
\item
  \href{https://www.nytco.com/careers/}{Work with us}
\item
  \href{https://nytmediakit.com/}{Advertise}
\item
  \href{http://www.tbrandstudio.com/}{T Brand Studio}
\item
  \href{https://www.nytimes.com/privacy/cookie-policy\#how-do-i-manage-trackers}{Your
  Ad Choices}
\item
  \href{https://www.nytimes.com/privacy}{Privacy}
\item
  \href{https://help.nytimes.com/hc/en-us/articles/115014893428-Terms-of-service}{Terms
  of Service}
\item
  \href{https://help.nytimes.com/hc/en-us/articles/115014893968-Terms-of-sale}{Terms
  of Sale}
\item
  \href{https://spiderbites.nytimes.com}{Site Map}
\item
  \href{https://help.nytimes.com/hc/en-us}{Help}
\item
  \href{https://www.nytimes.com/subscription?campaignId=37WXW}{Subscriptions}
\end{itemize}
