Sections

SEARCH

\protect\hyperlink{site-content}{Skip to
content}\protect\hyperlink{site-index}{Skip to site index}

\href{https://www.nytimes.com/section/business}{Business}

\href{https://myaccount.nytimes.com/auth/login?response_type=cookie\&client_id=vi}{}

\href{https://www.nytimes.com/section/todayspaper}{Today's Paper}

\href{/section/business}{Business}\textbar{}Companies may move supply
chains out of China, but not necessarily to the U.S.

\url{https://nyti.ms/3fTcfKD}

\begin{itemize}
\item
\item
\item
\item
\item
\end{itemize}

Advertisement

\protect\hyperlink{after-top}{Continue reading the main story}

Supported by

\protect\hyperlink{after-sponsor}{Continue reading the main story}

\hypertarget{companies-may-move-supply-chains-out-of-china-but-not-necessarily-to-the-us}{%
\section{Companies may move supply chains out of China, but not
necessarily to the
U.S.}\label{companies-may-move-supply-chains-out-of-china-but-not-necessarily-to-the-us}}

\includegraphics{https://static01.nyt.com/images/2020/07/22/business/22markets-brf-reshoring/merlin_172228977_b0aad741-b56a-4965-95c0-94d8cdfd6264-articleLarge.jpg?quality=75\&auto=webp\&disable=upscale}

\href{https://www.nytimes.com/by/ana-swanson}{\includegraphics{https://static01.nyt.com/images/2018/12/10/multimedia/author-ana-swanson/author-ana-swanson-thumbLarge.png}}\href{https://www.nytimes.com/by/jim-tankersley}{\includegraphics{https://static01.nyt.com/images/2018/10/19/multimedia/author-jim-tankersley/author-jim-tankersley-thumbLarge.png}}

By \href{https://www.nytimes.com/by/ana-swanson}{Ana Swanson} and
\href{https://www.nytimes.com/by/jim-tankersley}{Jim Tankersley}

\begin{itemize}
\item
  July 22, 2020
\item
  \begin{itemize}
  \item
  \item
  \item
  \item
  \item
  \end{itemize}
\end{itemize}

As factories struggle to reopen with materials still in short supply,
some executives are questioning the
\href{https://www.nytimes.com/2020/03/05/business/coronavirus-globalism.html}{just-in-time
supply chains} they use to whisk products around the globe, rather than
keeping warehouses stocked --- and particularly how much they rely on
factories in China, where production moved en masse in previous decades.

``Covid has generated this new imagination of worst-case scenarios,''
said Emily J. Blanchard, a professor at the Tuck School of Business at
Dartmouth College.

President Trump has used the pandemic as an opportunity to encourage
more companies to bring manufacturing back to the United States. But for
all the president's criticisms of global supply chains, the economic
incentive to outsource still prevails.

The pandemic has prompted
\href{https://www.nytimes.com/2020/03/05/business/coronavirus-globalism.html}{a
broader reassessment} of the risks of global supply chains, but
executives are deeply uncertain what demand for their products will look
like in the coming months and years --- hardly the environment to
encourage big investments in new American factories.

More companies leaving China does not necessarily represent a win for
American workers. Many companies that are moving some facilities out of
China --- including \textbf{Samsung, Hasbro, Apple, Nintendo} and
\textbf{GoPro} --- are relocating to countries where wages are even
lower. While U.S. trade with China
\href{https://www.nytimes.com/2020/02/05/business/economy/trump-trade.html}{fell
sharply} last year, imports from Vietnam, Taiwan and Mexico swelled.

For many companies, making their supply chains more resilient has
actually meant spreading out production around the world, not
concentrating it in the United States, said Chris Rogers, a global trade
and logistics analyst at Panjiva.

``If you want to hedge your risks, you need to stay global,'' he said.

Advertisement

\protect\hyperlink{after-bottom}{Continue reading the main story}

\hypertarget{site-index}{%
\subsection{Site Index}\label{site-index}}

\hypertarget{site-information-navigation}{%
\subsection{Site Information
Navigation}\label{site-information-navigation}}

\begin{itemize}
\tightlist
\item
  \href{https://help.nytimes.com/hc/en-us/articles/115014792127-Copyright-notice}{©~2020~The
  New York Times Company}
\end{itemize}

\begin{itemize}
\tightlist
\item
  \href{https://www.nytco.com/}{NYTCo}
\item
  \href{https://help.nytimes.com/hc/en-us/articles/115015385887-Contact-Us}{Contact
  Us}
\item
  \href{https://www.nytco.com/careers/}{Work with us}
\item
  \href{https://nytmediakit.com/}{Advertise}
\item
  \href{http://www.tbrandstudio.com/}{T Brand Studio}
\item
  \href{https://www.nytimes.com/privacy/cookie-policy\#how-do-i-manage-trackers}{Your
  Ad Choices}
\item
  \href{https://www.nytimes.com/privacy}{Privacy}
\item
  \href{https://help.nytimes.com/hc/en-us/articles/115014893428-Terms-of-service}{Terms
  of Service}
\item
  \href{https://help.nytimes.com/hc/en-us/articles/115014893968-Terms-of-sale}{Terms
  of Sale}
\item
  \href{https://spiderbites.nytimes.com}{Site Map}
\item
  \href{https://help.nytimes.com/hc/en-us}{Help}
\item
  \href{https://www.nytimes.com/subscription?campaignId=37WXW}{Subscriptions}
\end{itemize}
