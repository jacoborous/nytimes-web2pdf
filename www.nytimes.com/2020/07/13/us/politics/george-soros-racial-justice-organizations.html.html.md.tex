Sections

SEARCH

\protect\hyperlink{site-content}{Skip to
content}\protect\hyperlink{site-index}{Skip to site index}

\href{https://www.nytimes.com/section/politics}{Politics}

\href{https://myaccount.nytimes.com/auth/login?response_type=cookie\&client_id=vi}{}

\href{https://www.nytimes.com/section/todayspaper}{Today's Paper}

\href{/section/politics}{Politics}\textbar{}George Soros's Foundation
Pours \$220 Million Into Racial Equality Push

\url{https://nyti.ms/2OfWz82}

\begin{itemize}
\item
\item
\item
\item
\item
\end{itemize}

\href{https://www.nytimes.com/news-event/george-floyd-protests-minneapolis-new-york-los-angeles?action=click\&pgtype=Article\&state=default\&region=TOP_BANNER\&context=storylines_menu}{Race
and America}

\begin{itemize}
\tightlist
\item
  \href{https://www.nytimes.com/2020/07/26/us/protests-portland-seattle-trump.html?action=click\&pgtype=Article\&state=default\&region=TOP_BANNER\&context=storylines_menu}{Protesters
  Return to Other Cities}
\item
  \href{https://www.nytimes.com/2020/07/24/us/portland-oregon-protests-white-race.html?action=click\&pgtype=Article\&state=default\&region=TOP_BANNER\&context=storylines_menu}{Portland
  at the Center}
\item
  \href{https://www.nytimes.com/2020/07/23/podcasts/the-daily/portland-protests.html?action=click\&pgtype=Article\&state=default\&region=TOP_BANNER\&context=storylines_menu}{Podcast:
  Showdown in Portland}
\item
  \href{https://www.nytimes.com/interactive/2020/07/16/us/black-lives-matter-protests-louisville-breonna-taylor.html?action=click\&pgtype=Article\&state=default\&region=TOP_BANNER\&context=storylines_menu}{45
  Days in Louisville}
\end{itemize}

Advertisement

\protect\hyperlink{after-top}{Continue reading the main story}

Supported by

\protect\hyperlink{after-sponsor}{Continue reading the main story}

\hypertarget{george-soross-foundation-pours-220-million-into-racial-equality-push}{%
\section{George Soros's Foundation Pours \$220 Million Into Racial
Equality
Push}\label{george-soross-foundation-pours-220-million-into-racial-equality-push}}

Mr. Soros's group will invest \$150 million in grants for Black-led
racial justice groups, and another \$70 million toward local grants for
criminal justice reform and civic engagement opportunities.

\includegraphics{https://static01.nyt.com/images/2020/07/13/us/politics/13soros-money1/13soros-money1-articleLarge.jpg?quality=75\&auto=webp\&disable=upscale}

\href{https://www.nytimes.com/by/astead-w-herndon}{\includegraphics{https://static01.nyt.com/images/2018/09/14/us/author-head-astead/author-head-astead-thumbLarge-v2.png}}

By \href{https://www.nytimes.com/by/astead-w-herndon}{Astead W. Herndon}

\begin{itemize}
\item
  July 13, 2020
\item
  \begin{itemize}
  \item
  \item
  \item
  \item
  \item
  \end{itemize}
\end{itemize}

The Open Society Foundations, the philanthropic group founded by the
business magnate George Soros, announced on Monday that it was investing
\$220 million in efforts to achieve racial equality in America, a huge
financial undertaking that will support several Black-led racial justice
groups for years to come.

The initiative, which comes amid national protests for racial equality
and calls for police reform ignited by the killing of George Floyd in
Minneapolis, will immediately reshape the landscape of Black political
and civil rights organizations, and signals the extent to which race and
identity have become the explicit focal point of American politics in
recent years, with no sign of receding. Mr. Soros, who has at times
faced smears and anti-Semitism over his role as a liberal megadonor, is
also positioning his foundation near the forefront of the protest
movement.

Of the \$220 million, the foundation will invest \$150 million in
five-year grants for selected groups, including progressive and emerging
organizations like the Black Voters Matter Fund and Repairers of the
Breach, a group founded by the Rev. Dr. William J. Barber II of the Poor
People's Campaign. The money will also support more established Black
civil rights organizations like the Equal Justice Initiative, which was
founded by the civil rights lawyer Bryan Stevenson and depicted in the
2019 movie ``Just Mercy.''

The Open Society Foundations will invest an additional \$70 million in
local grants supporting changes to policing and criminal justice. This
money will also be used to pay for opportunities for civic engagement
and to organize internships and political training for young people.

Patrick Gaspard, the president of the Open Society Foundations, said in
an interview that the group believed the investment was about harnessing
the momentum toward racial justice, but also giving organizations room
to think long-term. Now, he said, is ``the moment we've been investing
in for the last 25 years.''

``There is this call for justice in Black and brown communities, an
explosion of not just sympathy but solidarity across the board,'' Mr.
Gaspard said. ``So it's time to double down. And we understood we can
place a bet on these activists --- Black and white --- who see this as a
moment of not just incrementalism, but whole-scale reform.''

``The demands being made now will not be met overnight, and we know the
gaze of media and elected officials will turn in other directions,'' he
added. ``But we need these moments to be sustained. If we're going to
say `Black lives matter,' we need to say `Black organizations and
structures matter.'''

Even before Monday's announcement, progressive groups, Democratic
candidates and racial justice organizations had been flooded with
small-dollar donations, breaking giving records and allowing former Vice
President
\href{https://www.nytimes.com/interactive/2020/us/elections/joe-biden.html}{Joseph
R. Biden Jr.} as well as House and Senate candidates to post
\href{https://www.nytimes.com/2020/07/01/us/politics/trump-fundraising-2020.html}{eye-popping
fund-raising numbers}. It is the convergence of an election year in
which Democrats are desperate to defeat
\href{https://www.nytimes.com/interactive/2020/us/elections/donald-trump.html}{President
Trump} with an extraordinary protest movement that has pushed many to
action,
\href{https://www.nytimes.com/interactive/2020/06/10/upshot/black-lives-matter-attitudes.html}{changing
public opinion among white Americans} and ideological moderates in the
process.

But in making the grants last for five years, Open Society's leaders
said, the organization is freeing groups to think beyond the current
moment. Heather McGhee, who is on the foundation's domestic board and
has been on recent calls informing groups of their new grants, said the
five-year commitments let leaders feel ``they could breathe and they can
focus on the strategy and the work.'' The calls have been emotional, she
said, as groups that have long felt marginalized by mainstream
philanthropy find out their work will be sustained and supported.

``It frees up time and ensures that once the corporations stop putting
up statements and small-dollar donations stop, they can keep fighting
this fight,'' Ms. McGhee said. ``If you take the blinders of racism off,
of course you should be investing in Black leadership --- and these
organizations shouldn't be worrying about money.''

This not the first effort by Mr. Soros or his foundation to target
racial inequality, though it is the most expansive. In 1994, Mr. Soros
started Open Society's domestic work with a focus on criminal justice
reform. He has also aimed philanthropic efforts at historically
marginalized groups abroad, a nod to his own experience as a Jewish
person who survived the Nazi occupation of Hungary. In recent years, Mr.
Soros has
\href{https://www.nytimes.com/2018/10/31/us/politics/george-soros-bombs-trump.html}{become
a favorite target} of some conservatives and right-wing groups, which
have sometimes used anti-Semitic tropes to try to recast his giving as
an effort to seek world influence.

Between the local grants and the millions for Black-led organizations,
however, Mr. Soros and his foundation have helped answer the question of
whether the social justice groups that have dominated the current moment
are here to stay.

Alexander Soros, who serves alongside his father as the deputy chair of
Open Society, said in a statement that the new investment was a response
to a time ``for urgent and bold action.''

``These investments will empower proven leaders in the Black community
to reimagine policing, end mass incarceration and eliminate the barriers
to opportunity that have been the source of inequity for too long,'' he
said.

Advertisement

\protect\hyperlink{after-bottom}{Continue reading the main story}

\hypertarget{site-index}{%
\subsection{Site Index}\label{site-index}}

\hypertarget{site-information-navigation}{%
\subsection{Site Information
Navigation}\label{site-information-navigation}}

\begin{itemize}
\tightlist
\item
  \href{https://help.nytimes.com/hc/en-us/articles/115014792127-Copyright-notice}{©~2020~The
  New York Times Company}
\end{itemize}

\begin{itemize}
\tightlist
\item
  \href{https://www.nytco.com/}{NYTCo}
\item
  \href{https://help.nytimes.com/hc/en-us/articles/115015385887-Contact-Us}{Contact
  Us}
\item
  \href{https://www.nytco.com/careers/}{Work with us}
\item
  \href{https://nytmediakit.com/}{Advertise}
\item
  \href{http://www.tbrandstudio.com/}{T Brand Studio}
\item
  \href{https://www.nytimes.com/privacy/cookie-policy\#how-do-i-manage-trackers}{Your
  Ad Choices}
\item
  \href{https://www.nytimes.com/privacy}{Privacy}
\item
  \href{https://help.nytimes.com/hc/en-us/articles/115014893428-Terms-of-service}{Terms
  of Service}
\item
  \href{https://help.nytimes.com/hc/en-us/articles/115014893968-Terms-of-sale}{Terms
  of Sale}
\item
  \href{https://spiderbites.nytimes.com}{Site Map}
\item
  \href{https://help.nytimes.com/hc/en-us}{Help}
\item
  \href{https://www.nytimes.com/subscription?campaignId=37WXW}{Subscriptions}
\end{itemize}
