Sections

SEARCH

\protect\hyperlink{site-content}{Skip to
content}\protect\hyperlink{site-index}{Skip to site index}

\href{https://www.nytimes.com/section/world/asia}{Asia Pacific}

\href{https://myaccount.nytimes.com/auth/login?response_type=cookie\&client_id=vi}{}

\href{https://www.nytimes.com/section/todayspaper}{Today's Paper}

\href{/section/world/asia}{Asia Pacific}\textbar{}In Hong Kong, Arrests
and Fear Mark First Day of New Security Law

\url{https://nyti.ms/3geOHzt}

\begin{itemize}
\item
\item
\item
\item
\item
\item
\end{itemize}

Advertisement

\protect\hyperlink{after-top}{Continue reading the main story}

Supported by

\protect\hyperlink{after-sponsor}{Continue reading the main story}

\hypertarget{in-hong-kong-arrests-and-fear-mark-first-day-of-new-security-law}{%
\section{In Hong Kong, Arrests and Fear Mark First Day of New Security
Law}\label{in-hong-kong-arrests-and-fear-mark-first-day-of-new-security-law}}

Protesters deleted social media accounts, as formerly allowed speech
suddenly became a potential crime. The chill over the city has
booksellers, professors and nonprofits questioning their future.

\includegraphics{https://static01.nyt.com/images/2020/07/01/world/01hk-chill-top/01hk-chill-top-videoSixteenByNine3000.jpg}

\href{https://www.nytimes.com/by/vivian-wang}{\includegraphics{https://static01.nyt.com/images/2018/06/14/multimedia/author-vivian-wang/author-vivian-wang-thumbLarge-v2.png}}\href{https://www.nytimes.com/by/alexandra-stevenson}{\includegraphics{https://static01.nyt.com/images/2018/02/20/multimedia/author-alexandra-stevenson/author-alexandra-stevenson-thumbLarge.jpg}}

By \href{https://www.nytimes.com/by/vivian-wang}{Vivian Wang} and
\href{https://www.nytimes.com/by/alexandra-stevenson}{Alexandra
Stevenson}

\begin{itemize}
\item
  Published July 1, 2020Updated July 13, 2020
\item
  \begin{itemize}
  \item
  \item
  \item
  \item
  \item
  \item
  \end{itemize}
\end{itemize}

\href{https://cn.nytimes.com/china/20200702/hong-kong-security-law-china/}{阅读简体中文版}\href{https://cn.nytimes.com/china/20200702/hong-kong-security-law-china/zh-han}{閱讀繁體中文版}

HONG KONG --- The
\href{https://www.nytimes.com/2020/07/20/world/asia/hong-kong-coronavirus.html}{Hong
Kong} police moved swiftly on Wednesday to enforce China's new national
security rules with the first arrests under the law, as the city
immediately felt the chilling effect of Beijing's offensive to quash
dissent in the semiautonomous territory.

The law was proving effective in tamping down the anti-government
demonstrations that have wracked Hong Kong for more than a year. On
Wednesday, the anniversary of Hong Kong's return to Chinese control ---
usually observed by huge pro-democracy marches --- a scattered crowd of
thousands protested, only to be corralled by the police and risk arrest
for crimes that did not exist a day earlier.

Deploying pepper spray and water cannons to force
\href{https://www.nytimes.com/2020/07/13/world/asia/hong-kong-elections-security.html}{protesters}
off the streets, the police arrested about 370 people, including 10 over
new offenses created by the
\href{https://www.nytimes.com/2020/07/13/world/asia/hong-kong-elections-security.html}{security
law} that takes aim at political activity challenging Beijing. One of
the 10 was a 15-year-old girl waving a Hong Kong independence flag, the
police said.

Far-reaching and punitive, the law threatens the freewheeling cultural
scene and civil society that make the fabric of life in Hong Kong so
distinct from the rest of China. While officials insist that the law
will affect only a small group of offenders, many fear the government
could use the law's expansive definitions to target a wide array of
people and organizations, prompting many to take defensive action.

A museum that commemorates the 1989 Tiananmen Square massacre is rushing
to digitize its archives, afraid its artifacts could be seized.
Booksellers are nervously eyeing customers, worried they could be
government spies. Writers have asked a news site to delete more than 100
articles, anxious that old posts could be used against them.

``You can say this law is just targeting protesters and anti-Chinese
politicians, but it could be anyone,'' said Isabella Ng, a professor at
the Education University of Hong Kong who founded a
\href{https://www.facebook.com/HKSASR/?ref=page_internal}{charity} that
helps refugees in the city.

\includegraphics{https://static01.nyt.com/images/2020/07/01/world/01hk-chill-2sub/merlin_174102576_2866f9aa-737c-4f4b-924e-c261b634d254-articleLarge.jpg?quality=75\&auto=webp\&disable=upscale}

``Where is the line to draw?'' said Professor Ng, who worries that her
charity could one day come under scrutiny. ``Everything becomes very
uncertain.''

\href{http://www.xinhuanet.com/2020-06/30/c_1126179649.htm}{The law},
which went into effect as soon as it was released Tuesday night,
confirmed many residents' fears that a range of actions that they had
previously engaged in had become hazardous. Though the law specifically
bans subversion, sedition, terrorism and collusion, its definitions of
those crimes could be interpreted broadly to include various forms of
speech or organizing.

Lobbying foreign governments or publishing anti-Beijing viewpoints could
be punished by life imprisonment in serious cases. So could saying
anything seen as undermining the ruling Communist Party's authority. In
the mainland, the party has
\href{https://www.nytimes.com/2019/07/12/world/asia/china-journalists-crackdown.html}{virtually
eliminated} independent journalism and imposed
\href{https://www.nytimes.com/2016/04/29/world/asia/china-foreign-ngo-law.html}{onerous
restrictions} on nongovernmental organizations.

Citing the new law and other factors, the Trump administration is
\href{https://www.nytimes.com/2020/05/29/us/politics/trump-hong-kong-china-WHO.html}{rolling
back Hong Kong's trade privileges} with the United States.

``Free Hong Kong was one of the world's most stable, prosperous and
dynamic cities,'' Secretary of State Mike Pompeo said at a news
conference on Wednesday. ``Now it'll be just another Communist-run city
where its people will be subject to the party elite's whims.''

Even before the law was passed, activists, journalists, bookshop owners
and professors said they had begun second-guessing any speech that could
be labeled political. The human rights group Amnesty International said
it had drawn up a contingency plan.

Many Hong Kongers have expressed interest in emigration, a task that
\href{https://www.nytimes.com/2020/06/03/world/europe/boris-johnson-uk-hong-kong-china.html}{Britain
has promised to make easier}. The British foreign secretary, Dominic
Raab, said on Wednesday that some Hong Kong residents would be allowed
to live in Britain for five years --- up from six months previously ---
and then apply for citizenship.

A former British colony, Hong Kong was promised a high degree of
autonomy when it returned to Chinese control in 1997. It found success
as a bridge between the mainland and the rest of the world,
\href{https://www.nytimes.com/2019/06/18/world/asia/hong-kong-extradition-bill-china.html}{serving
as a haven} for Chinese dissidents and a base for academics, journalists
and researchers to chronicle, unfettered, the country's modernization.

But reminders of Chinese control were never far away. The
\href{https://www.nytimes.com/2018/04/03/magazine/the-case-of-hong-kongs-missing-booksellers.html}{abductions
of five Hong Kong booksellers} in 2015 by the mainland authorities
rattled others who had openly marketed salacious Chinese political
thrillers or modern historical volumes. Though Hong Kong was long a
sanctuary for books banned in the mainland, tighter border checks have
recently choked the flow of books between Hong Kong and the mainland.

Now the security push has accelerated panic and a sense of foreboding.

``If you haven't tasted what tyranny is, be prepared, because tyranny is
not comfortable,'' said Bao Pu, the founder of New Century Press, one of
the city's
\href{https://www.nytimes.com/2019/06/04/books/hong-kong-publishing-tiananmen.html}{few
surviving independent publishers}.

Image

Bao Pu founded New Century Press, a Hong Kong publisher that has
resisted censorship efforts by Beijing.~Credit...Lam Yik Fei for The New
York Times

Albert Wan, the co-owner of Bleak House Books, an independent bookstore,
said that he closely tracked all his book shipments, regardless of
whether they could be considered political, watching for any sign of
delay.

He said that he had also grown wary of unfamiliar customers, and tries
to decide if they are browsing for books or seemingly ``building a
profile'' of him and his employees.

``We are being paranoid,'' Mr. Wan said. ``I don't know how else to put
it.''

For those who built their lives and livelihoods around Hong Kong's
unique freedoms, the security law has forced them to balance two
seemingly irreconcilable goals: preserving their own safety, without
giving in to fear.

The June 4 Museum, which chronicles Beijing's
\href{https://www.nytimes.com/2019/06/03/world/asia/tiananmen-massacre-anniversary-archive.html}{bloody
military crackdown} on student protesters in 1989, has not made plans to
move its artifacts overseas for safekeeping. The Chinese government has
tried to quash any memory of the massacre, so to hide the archives would
be to admit premature defeat, said Lee Cheuk-yan, of the Hong Kong
Alliance in Support of Patriotic Democratic Movements in China, which
runs the museum.

But reality has also forced the alliance to start an
\href{https://www.kickstarter.com/projects/64museum/june-4th-museum-of-memory-and-human-rights?ref=discovery\&term=\%E5\%85\%AD\%E5\%9B\%9B}{online
fund-raiser} in support of digitizing the museum's archives, which
include video footage of the protests and letters that protesters wrote
to their families.

``We of course are racing with time,'' Mr. Lee said.

Image

Lee Cheuk-yan at the June 4 Museum in May. He said~hiding its archives
would amount to admitting premature defeat.Credit...Lam Yik Fei for The
New York Times

The chill is not limited to local groups. Large international
organizations are also evaluating their future in the city. The new law
specifically said that the government would ``strengthen the
management'' of foreign nongovernmental organizations and news agencies.

``The rule of law is going to come under very severe stress in Hong
Kong,'' said Nicholas Bequelin, the director for Amnesty's East and
Southeast Asia operations.

Concerns about the security law's reach have also forced many writers
and protesters to scrutinize their digital footprints for anything that
might now be deemed subversive. Activists deleted their accounts on
Twitter and on Telegram, a messaging app popular with protesters.

In recent weeks, around a dozen writers asked the editors of InMedia HK,
a site that posts articles **** supporting democracy, to take down some
or all of their archives, said Betty Lau, the site's editor. Editors
deleted more than 100 articles, Ms. Lau said.

Hong Kong's reputation for press freedom has long stood in contrast with
the mainland's censorship regime and routine harassment of journalists.
But the new security law has thrown the future of the city's lively news
media into question.

Image

One Hong Kong police officer pointed a pepper ball gun while another
made an arrest at a protest on Wednesday.Credit...Lam Yik Fei for The
New York Times

The Hong Kong News Executives Association, a group representing the top
editors of the city's major news outlets,
\href{https://www.scmp.com/news/hong-kong/law-and-crime/article/3090787/hong-kong-national-security-law-citys-media-bosses}{expressed
concern} about the far-reaching impact of the security law ahead of its
release. The Foreign Correspondents' Club urged the government last week
to guarantee that the authorities would not seek to interfere with the
work of reporters. The government has not responded, but officials have
sought to reassure the public that the city's civil liberties will be
protected.

During a recent end-of-semester meeting at Hong Kong University's
Journalism and Media Studies Center, staff members wondered aloud where
the red line would be and whether certain topics would be off limits,
said the center's director, Keith Richburg.

``I'd be lying if I said I don't think twice about posting something on
Twitter before pushing the button,'' said Mr. Richburg, a former foreign
correspondent with The Washington Post.

One of the starkest indicators that the national security law was
already having its intended effect came on Tuesday, directly after
lawmakers in Beijing unanimously approved it.

\href{https://www.nytimes.com/2014/10/02/world/asia/hong-kong-china-democracy-protests-students.html}{Joshua
Wong}, the 23-year-old who is perhaps Hong Kong's best-known activist,
announced on social media that he would withdraw from Demosisto, the
youth political group that he founded in 2016, citing fears for his
safety. Demosisto, which has called for greater autonomy for Hong Kong,
was for many the face of the protest movement's future.

Soon after, three other leading members of Demosisto also resigned. A
few hours later, the group announced it was disbanding altogether.

In a note explaining his decision, Mr. Wong wrote, ``Nobody can be sure
of their tomorrow.''

The crowds of protesters were small on Wednesday, relative to the
hundreds of thousands that regularly took to the streets last year. But
swarms of riot officers quickly surrounded them.

For some protesters, it's a fight they are willing to continue, even if
it means going up against Beijing. ``We have to show the people of Hong
Kong that we cannot be afraid or deterred by the national security
law,'' said Avery Ng, a leader of the League of Social Democrats, a
political party. ``We are taking a certain level of risk, being that one
of our demands is the end of one-party dictatorship.''

Image

``Lady Liberty Hong Kong,'' a statue modeled on~a woman who was hit in
the eye at a protest, on display at an exhibition in May.Credit...Lam
Yik Fei for The New York Times

Austin Ramzy, Elaine Yu and Tiffany May contributed reporting. Bella
Huang contributed research.

Advertisement

\protect\hyperlink{after-bottom}{Continue reading the main story}

\hypertarget{site-index}{%
\subsection{Site Index}\label{site-index}}

\hypertarget{site-information-navigation}{%
\subsection{Site Information
Navigation}\label{site-information-navigation}}

\begin{itemize}
\tightlist
\item
  \href{https://help.nytimes.com/hc/en-us/articles/115014792127-Copyright-notice}{©~2020~The
  New York Times Company}
\end{itemize}

\begin{itemize}
\tightlist
\item
  \href{https://www.nytco.com/}{NYTCo}
\item
  \href{https://help.nytimes.com/hc/en-us/articles/115015385887-Contact-Us}{Contact
  Us}
\item
  \href{https://www.nytco.com/careers/}{Work with us}
\item
  \href{https://nytmediakit.com/}{Advertise}
\item
  \href{http://www.tbrandstudio.com/}{T Brand Studio}
\item
  \href{https://www.nytimes.com/privacy/cookie-policy\#how-do-i-manage-trackers}{Your
  Ad Choices}
\item
  \href{https://www.nytimes.com/privacy}{Privacy}
\item
  \href{https://help.nytimes.com/hc/en-us/articles/115014893428-Terms-of-service}{Terms
  of Service}
\item
  \href{https://help.nytimes.com/hc/en-us/articles/115014893968-Terms-of-sale}{Terms
  of Sale}
\item
  \href{https://spiderbites.nytimes.com}{Site Map}
\item
  \href{https://help.nytimes.com/hc/en-us}{Help}
\item
  \href{https://www.nytimes.com/subscription?campaignId=37WXW}{Subscriptions}
\end{itemize}
