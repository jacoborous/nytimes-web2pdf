Sections

SEARCH

\protect\hyperlink{site-content}{Skip to
content}\protect\hyperlink{site-index}{Skip to site index}

\href{https://www.nytimes.com/section/business}{Business}

\href{https://myaccount.nytimes.com/auth/login?response_type=cookie\&client_id=vi}{}

\href{https://www.nytimes.com/section/todayspaper}{Today's Paper}

\href{/section/business}{Business}\textbar{}The `Rocket Ship' Economic
Recovery Is Crashing

\url{https://nyti.ms/3gjreNs}

\begin{itemize}
\item
\item
\item
\item
\item
\end{itemize}

\href{https://www.nytimes.com/news-event/coronavirus?action=click\&pgtype=Article\&state=default\&region=TOP_BANNER\&context=storylines_menu}{The
Coronavirus Outbreak}

\begin{itemize}
\tightlist
\item
  live\href{https://www.nytimes.com/2020/08/01/world/coronavirus-covid-19.html?action=click\&pgtype=Article\&state=default\&region=TOP_BANNER\&context=storylines_menu}{Latest
  Updates}
\item
  \href{https://www.nytimes.com/interactive/2020/us/coronavirus-us-cases.html?action=click\&pgtype=Article\&state=default\&region=TOP_BANNER\&context=storylines_menu}{Maps
  and Cases}
\item
  \href{https://www.nytimes.com/interactive/2020/science/coronavirus-vaccine-tracker.html?action=click\&pgtype=Article\&state=default\&region=TOP_BANNER\&context=storylines_menu}{Vaccine
  Tracker}
\item
  \href{https://www.nytimes.com/interactive/2020/07/29/us/schools-reopening-coronavirus.html?action=click\&pgtype=Article\&state=default\&region=TOP_BANNER\&context=storylines_menu}{What
  School May Look Like}
\item
  \href{https://www.nytimes.com/live/2020/07/31/business/stock-market-today-coronavirus?action=click\&pgtype=Article\&state=default\&region=TOP_BANNER\&context=storylines_menu}{Economy}
\end{itemize}

Advertisement

\protect\hyperlink{after-top}{Continue reading the main story}

Supported by

\protect\hyperlink{after-sponsor}{Continue reading the main story}

\hypertarget{the-rocket-ship-economic-recovery-is-crashing}{%
\section{The `Rocket Ship' Economic Recovery Is
Crashing}\label{the-rocket-ship-economic-recovery-is-crashing}}

Real-time data suggest a quick resurgence of business activity is
leveling off nationally --- and reversing in states like Arizona and
Texas.

\includegraphics{https://static01.nyt.com/images/2020/05/31/business/31DC-Virus-Econ-01/merlin_172963488_0429aa37-1675-4122-a1a1-93a8415bdd68-articleLarge.jpg?quality=75\&auto=webp\&disable=upscale}

\href{https://www.nytimes.com/by/jim-tankersley}{\includegraphics{https://static01.nyt.com/images/2018/10/19/multimedia/author-jim-tankersley/author-jim-tankersley-thumbLarge.png}}\href{https://www.nytimes.com/by/ben-casselman}{\includegraphics{https://static01.nyt.com/images/2018/11/09/multimedia/author-ben-casselman/author-ben-casselman-thumbLarge.png}}

By \href{https://www.nytimes.com/by/jim-tankersley}{Jim Tankersley} and
\href{https://www.nytimes.com/by/ben-casselman}{Ben Casselman}

\begin{itemize}
\item
  July 1, 2020
\item
  \begin{itemize}
  \item
  \item
  \item
  \item
  \item
  \end{itemize}
\end{itemize}

The nascent restart of America's economy has begun to stall as a surge
in new coronavirus cases dampens consumer and business activity across
states like Florida, Texas and Arizona.

After weeks of a pandemic-induced contraction, the economy had begun
rebounding faster than many economists expected from mid-April into
June, as infection rates stabilized or fell across much of the country
and the federal government injected trillions of dollars in the economy.
States began to reopen, shoppers increased their spending and employers
started to hire back furloughed workers.

But there were signs in late May and early June that the pace of
recovery was beginning to slow, even before
\href{https://www.nytimes.com/interactive/2020/us/coronavirus-us-cases.html}{another
wave of infections} swept through states that had moved quickly to ease
limits on public gatherings. In recent weeks, as that wave intensified,
real-time economic data began to show the economy moving backward as
rising infection fears spooked consumers.

The national jobs report, scheduled to be released on Thursday by the
Labor Department, is expected to obscure that reversal. Forecasters
expect the report, drawn from data compiled in the middle of the month,
to show the economy added about three million jobs in June. That would
represent progress, but nowhere close to victory against the
\href{https://www.nytimes.com/interactive/2020/05/08/business/economy/april-jobs-report.html}{more
than 20 million jobs} shed at the trough of the recession.

Recent detailed data tell a more sobering story. New job postings on the
employment platform ZipRecruiter fell in June after rising sharply in
May. Data on small business openings and employment from Homebase, which
provides scheduling and time tracking software for businesses, show that
small business employment and openings worsened over the past week,
after plateauing for much of June. The Homebase data showed a nearly 40
percent improvement for small business activity in May; across all of
June, that fell to 6 percent.

States suffering infection surges,
\href{https://www.nytimes.com/2020/06/25/business/economy/texas-economy-oil-coronavirus.html}{like
Texas}, began to see layoffs and business closings even before officials
moved to reimpose some restrictions on economic activity, such as
closing bars.

Foot traffic to retailers and other businesses declined in the third
week of June in Houston, Orlando, Jacksonville, Phoenix and other large
cities across the southern states where infections have spiked,
according to an analysis of Safegraph.com data by researchers at the
\href{https://www.aei.org/wp-content/uploads/2020/06/Nowcast-Reopening-of-Metro-Area-Economies-week-25-FINAL.pdf}{American
Enterprise Institute in Washington}. Data from 40 million households
compiled by the financial firm Commerce Signals shows that after weeks
of improvement, credit and debit card spending declined at the end of
May across most states.

\hypertarget{latest-updates-economy}{%
\section{\texorpdfstring{\href{https://www.nytimes.com/live/2020/07/31/business/stock-market-today-coronavirus?action=click\&pgtype=Article\&state=default\&region=MAIN_CONTENT_1\&context=storylines_live_updates}{Latest
Updates:
Economy}}{Latest Updates: Economy}}\label{latest-updates-economy}}

\href{https://www.nytimes.com/live/2020/07/31/business/stock-market-today-coronavirus?action=click\&pgtype=Article\&state=default\&region=MAIN_CONTENT_1\&context=storylines_live_updates\#kodaks-chief-executive-was-given-stock-options-then-the-share-price-spiked-1000-percent}{20h
ago}

\href{https://www.nytimes.com/live/2020/07/31/business/stock-market-today-coronavirus?action=click\&pgtype=Article\&state=default\&region=MAIN_CONTENT_1\&context=storylines_live_updates\#kodaks-chief-executive-was-given-stock-options-then-the-share-price-spiked-1000-percent}{Kodak's
chief executive was given stock options. Then the share price spiked
1,000 percent.}

\href{https://www.nytimes.com/live/2020/07/31/business/stock-market-today-coronavirus?action=click\&pgtype=Article\&state=default\&region=MAIN_CONTENT_1\&context=storylines_live_updates\#fitch-ratings-downgrades-its-outlook-on-us-debt}{23h
ago}

\href{https://www.nytimes.com/live/2020/07/31/business/stock-market-today-coronavirus?action=click\&pgtype=Article\&state=default\&region=MAIN_CONTENT_1\&context=storylines_live_updates\#fitch-ratings-downgrades-its-outlook-on-us-debt}{Fitch
Ratings downgrades its outlook on U.S. debt.}

\href{https://www.nytimes.com/live/2020/07/31/business/stock-market-today-coronavirus?action=click\&pgtype=Article\&state=default\&region=MAIN_CONTENT_1\&context=storylines_live_updates\#us-sanctions-more-chinese-officials-over-human-rights-violations-as-tensions-flare}{30h
ago}

\href{https://www.nytimes.com/live/2020/07/31/business/stock-market-today-coronavirus?action=click\&pgtype=Article\&state=default\&region=MAIN_CONTENT_1\&context=storylines_live_updates\#us-sanctions-more-chinese-officials-over-human-rights-violations-as-tensions-flare}{U.S.
sanctions more Chinese officials over human rights violations as
tensions flare}

\href{https://www.nytimes.com/live/2020/07/31/business/stock-market-today-coronavirus?action=click\&pgtype=Article\&state=default\&region=MAIN_CONTENT_1\&context=storylines_live_updates}{See
more updates}

More live coverage:
\href{https://www.nytimes.com/2020/08/01/world/coronavirus-covid-19.html?action=click\&pgtype=Article\&state=default\&region=MAIN_CONTENT_1\&context=storylines_live_updates}{Global}

That is a pattern economists have been dreading, and a departure from
the \href{https://www.youtube.com/watch?v=PVI0yw5olOQ}{``rocket ship''}
recovery that President Trump promised in June. Federal Reserve
officials have warned publicly that recovery appears perilous and highly
dependent on public health. ``The path forward for the economy is
extraordinarily uncertain and will depend in large part on our success
in containing the virus,'' Fed Chair Jerome H. Powell
\href{https://www.nytimes.com/2020/06/30/us/politics/mnuchin-powell-congress-economic-recovery.html}{told}
a House committee on Tuesday. ``A full recovery is unlikely until people
are confident that it is safe to re-engage in a broad range of
activities.''

The next few months of recovery could be rocky even if the current
infection surge abates.
\href{https://www.nytimes.com/2020/06/18/business/economy/coronavirus-unemployment-claims.html}{Job
losses have slowed} but remain at levels higher than in any previous
recession, and a growing share of workers now report they have been laid
off permanently, rather than temporarily furloughed. A significant share
of small businesses have still not reopened, even as states increasingly
lift restrictions on their operations, suggesting some of them may be
shuttered for good. By many measures, business activity and employment
remains down by a quarter or more from pre-crisis levels.

Child care constraints are keeping many workers, particularly Black and
Hispanic women, from returning to work, according to weekly census
survey data analyzed by Ernie Tedeschi, an economist at Evercore ISI.

Some economists say the slowdown was predictable --- and a natural
reaction to Americans attempting to rush back toward normalcy before the
virus was under control.

When it comes to the recovery, ``the virus is the boss, not the
governor, not the mayor, not the president,'' said Austan Goolsbee, a
former top economist for President Barack Obama and the author
\href{https://bfi.uchicago.edu/working-paper/2020-80/}{of a recent
study} that found fear of infection --- and not government lockdown
policies --- drove nearly all of the contraction in economic activity
this spring.

Mr. Goolsbee, who is a professor at the University of Chicago's Booth
School of Business, and his colleague Chad Syverson used cellular phone
records to track visits to businesses during the pandemic.

The research found that just over one-tenth of the drop was attributable
to lockdowns themselves, a share that held constant as areas began to
lift restrictions in May. The authors say that suggests that if
infections accelerate, public officials will not be able to avoid
another economic shock simply by refusing to shut down activity.
Consumers will make that decision for them.

When cases of the virus first began rising earlier this year, many
economists hoped that, with the right set of policies, the United States
could avoid most long-term economic damage. The idea was that by
providing trillions of dollars in support for households and businesses,
the federal government could, in effect, keep the economy in stasis
until the health crisis had passed.

There are signs that those efforts were at least partly successful.
Nearly a third of the people who lost jobs during the pandemic have
already returned to work, according to
\href{https://www.surveymonkey.com/curiosity/nyt-june-2020-cci/}{a poll
conducted for The New York Times} in early June by the online research
platform SurveyMonkey. Another quarter expected to return to their old
jobs within the next month.

But that still leaves close to half of all those who have lost jobs
still out of work, with no immediate prospects for a return. That group
is disproportionately Black and Hispanic, and concentrated in low-wage
service industries, the survey found. Perhaps unsurprisingly, those
respondents are far less sanguine about the direction of the economy
than Americans overall.

``How can we have a recovery when millions of people are now permanently
unemployed?'' asked John Singh, a survey respondent in Los Angeles.
``How can we have an economy when big companies have just thrown in the
towel?''

Mr. Singh's husband was furloughed from his job at a large corporation,
but returned to work --- from home --- this week. The break was a loss
of income, but not a major career disruption.

It is a different story for Mr. Singh. He runs a small public relations
agency --- he is the only full-time employee --- and his main client is
in film distribution. When theaters shut down in mid-March, his revenue
dried up overnight. With theaters expected to be among the last
industries to return to normal, he doesn't expect his business to bounce
back anytime soon.

The Homebase data suggest a yawning divide in the experiences of
businesses that never closed for the pandemic and those that shut down
as it began to spread. Employee hours and total number of employees are
running just above pre-crisis levels at retailers that never closed. But
many businesses have not reopened, which Homebase officials said in a
report this week could be a sign that as many as 20 percent of all small
businesses will permanently close amid the crisis.

``For many of our business owners, it doesn't yet make sense to open at
the level of customer demand they're seeing,'' said Ray Sandza, vice
president of data and analytics at Homebase.

Data from Kronos, which provides time-management software and related
services, tells a similar story. The number of shifts worked by the
company's roughly 30,000 U.S. customers have rebounded strongly since
mid-April but remain down 15 percent compared to before the crisis. Now
the pace of growth has slowed in Georgia and some other early-reopening
states, and the number of shifts worked has fallen outright in South
Carolina and Florida since the beginning of June.

``We bounced off the bottom, and it was a sharp bounce off the bottom,''
said Dave Gilbertson, vice president of strategy and operations at
Kronos. ``Now is going to be what really proves out the full pace of the
recovery, and it's going to take longer.''

Several factors could complicate that next phase. For one, many day care
centers remain closed or limited, restricting some parents' ability to
return to work. ``I don't see how parents get back to work in a
meaningful way if their kids can't be in day care or back in school,''
said Melissa S. Kearney, a University of Maryland economist who directs
the Aspen Institute's Economic Strategy Group. ``Figuring out how to
make that happen needs to be at the top of the list.''

Ms. Kearney warned in a report with several co-authors in June that the
recovery could stall if Congress fails to maintain the support for
people and businesses that has helped buoy consumer spending. Senators
are poised to leave Washington this week, returning in mid-July, with
negotiations on a new economic aid bill still in their early stages.

Nathaniel Popper contributed reporting.

Advertisement

\protect\hyperlink{after-bottom}{Continue reading the main story}

\hypertarget{site-index}{%
\subsection{Site Index}\label{site-index}}

\hypertarget{site-information-navigation}{%
\subsection{Site Information
Navigation}\label{site-information-navigation}}

\begin{itemize}
\tightlist
\item
  \href{https://help.nytimes.com/hc/en-us/articles/115014792127-Copyright-notice}{©~2020~The
  New York Times Company}
\end{itemize}

\begin{itemize}
\tightlist
\item
  \href{https://www.nytco.com/}{NYTCo}
\item
  \href{https://help.nytimes.com/hc/en-us/articles/115015385887-Contact-Us}{Contact
  Us}
\item
  \href{https://www.nytco.com/careers/}{Work with us}
\item
  \href{https://nytmediakit.com/}{Advertise}
\item
  \href{http://www.tbrandstudio.com/}{T Brand Studio}
\item
  \href{https://www.nytimes.com/privacy/cookie-policy\#how-do-i-manage-trackers}{Your
  Ad Choices}
\item
  \href{https://www.nytimes.com/privacy}{Privacy}
\item
  \href{https://help.nytimes.com/hc/en-us/articles/115014893428-Terms-of-service}{Terms
  of Service}
\item
  \href{https://help.nytimes.com/hc/en-us/articles/115014893968-Terms-of-sale}{Terms
  of Sale}
\item
  \href{https://spiderbites.nytimes.com}{Site Map}
\item
  \href{https://help.nytimes.com/hc/en-us}{Help}
\item
  \href{https://www.nytimes.com/subscription?campaignId=37WXW}{Subscriptions}
\end{itemize}
