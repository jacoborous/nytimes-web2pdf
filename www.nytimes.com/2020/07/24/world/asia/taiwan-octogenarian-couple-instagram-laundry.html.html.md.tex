\href{/section/world/asia}{Asia Pacific}\textbar{}He's 83, She's 84, and
They Model Other People's Forgotten Laundry

\url{https://nyti.ms/2CHFYaQ}

\begin{itemize}
\item
\item
\item
\item
\item
\end{itemize}

\includegraphics{https://static01.nyt.com/images/2020/07/24/world/24taiwan-laundry1a/merlin_174888354_cae974a1-0311-4912-827f-e631000138ca-articleLarge.jpg?quality=75\&auto=webp\&disable=upscale}

Sections

\protect\hyperlink{site-content}{Skip to
content}\protect\hyperlink{site-index}{Skip to site index}

The saturday profile

\hypertarget{hes-83-shes-84-and-they-model-other-peoples-forgotten-laundry}{%
\section{He's 83, She's 84, and They Model Other People's Forgotten
Laundry}\label{hes-83-shes-84-and-they-model-other-peoples-forgotten-laundry}}

The owners of a laundry shop in central Taiwan have become Instagram
stars for posing in garments left behind.

Chang Wan-ji, right, and Hsu Sho-er run a laundromat in Taichung,
Taiwan. They are also an Instagram hit.Credit...Reef Chang

Supported by

\protect\hyperlink{after-sponsor}{Continue reading the main story}

By Chris Horton

\begin{itemize}
\item
  July 24, 2020
\item
  \begin{itemize}
  \item
  \item
  \item
  \item
  \item
  \end{itemize}
\end{itemize}

\href{https://cn.nytimes.com/style/20200728/taiwan-octogenarian-couple-instagram-laundry/}{阅读简体中文版}\href{https://cn.nytimes.com/style/20200728/taiwan-octogenarian-couple-instagram-laundry/zh-hant/}{閱讀繁體中文版}\href{https://www.nytimes.com/es/2020/07/28/espanol/mundo/lavanderia-taiwanesa-instagram.html}{Leer
en español}

TAICHUNG, Taiwan --- At Wansho Laundry in central Taiwan, most dirty
clothes dropped off to be steamed or washed or dry-cleaned end up right
back in the hands of their rightful owners, cleaner than when they
arrived.

Abandoned garments, however, can end up on Instagram.

The blouses and skirts and trousers adorn the bodies of the laundry's
octogenarian owners, Chang Wan-ji and Hsu Sho-er, who have become
globally famous for modeling outfits curated from the hundreds of
forgotten items left behind by absent-minded customers.

No one is more shocked than their 31-year-old grandson and unofficial
stylist, Reef Chang, by the couple's newfound fame. ``I was really
surprised,'' the younger Mr. Chang said recently. ``I had no idea so
many foreigners would take interest in my grandparents.''

He originally came up with the idea for the Instagram account, he said.
Their business had slowed during the coronavirus pandemic, and his
grandparents were wary about going outside even
\href{https://www.nytimes.com/interactive/2020/04/09/world/asia/coronavirus-hong-kong-singapore-taiwan.html?searchResultPosition=26}{as
Taiwan} took highly effective measures to fight the virus. With nearly
24 million people, Taiwan has reported only
\href{https://www.cdc.gov.tw/En}{458 cases, 55 local transmissions and
seven deaths}.

``They had nothing to do,'' he said. ``I saw how bored they were and
wanted to brighten up their lives.''

\includegraphics{https://static01.nyt.com/images/2020/07/24/world/24taiwan-laundry2a/merlin_174867414_81c51ab8-54fe-432c-9dae-1624711482e7-articleLarge.jpg?quality=75\&auto=webp\&disable=upscale}

They are naturals in front of the camera. Ms. Hsu, 84, exudes the
haughtiness of a supermodel but retains an air of playfulness. Mr.
Chang, 83, is the perfect foil, complementing his wife's swagger with a
chill disposition while rocking bountiful eyebrows.

``His eyebrows really are something else,'' Ms. Hsu said smiling in an
interview in the rear of the laundry shop, next to a small shrine to the
earth god Tudigong, a common feature of traditional Taiwanese homes.

The clothes they model are eclectic, funky and fun. Both can be seen in
matching laced sneakers, and jauntily perched caps and hats. He
sometimes sports brightly colored shades. One photo shows her leaning
coolly against a giant washing machine, arms crossed, as he casually
holds the open door, grinning. They pose at a place they know well ---
their shop, which provides an industrious backdrop of customers'
laundry, stacked and rolled into plastic bundles or hanging from racks.

The couple's youthful attitude appeals to a growing number of followers
--- 136,000 and counting --- despite having only 19 posts on their
account,
\href{https://www.instagram.com/wantshowasyoung/?hl=en}{@wantshowasyoung},
since its inception on June 27.

``My grandson is very creative,'' Ms. Hsu said. ``His creativity has
made us happy, and other people, too.''

The account has drawn fans from around Taiwan and the wider world, with
many seeing the photos as a salve during a year made dark by worries
over a global pandemic, economic ruination, climate change and
\href{https://www.nytimes.com/2020/07/01/world/asia/taiwan-china-hong-kong.html?searchResultPosition=1}{geopolitical
tension.}

``Looking at Wan-ji and Sho-er's photos improves my mood,'' one
Instagram user named tibbar1 wrote on Thursday in response to a photo
celebrating the account's surpassing 100,000 followers. ``Their photos
really have a charming vibe to them that not just anyone can pull off.''

Image

Mr. Chang and Ms. Hsu get messages from fans all over the
world.Credit...Reef Chang

Image

He says he wishes customers would return to pick up their clothes, and
pay their bills.Credit...Reef Chang

The couple may be internet famous today, but their 61 years together had
a more traditional beginning. Their story traces that of modern Taiwan,
beginning during the repressive era when it was under martial law and
unfolding as Taiwan gradually grew more outward-looking and confident.

Mr. Chang, then 21, met Ms. Hsu in the late 1950s, when her elder sister
and aunt approached him in the couple's native Houli, a semirural
district in the north of Taichung City, with the aim of making a
marriage match. When they took him home to meet Ms. Hsu, he didn't stay
long, to her dismay.

``I wanted him to sit down with me, but he wouldn't,'' she said. Things
were more conservative back then. ``He was pretty shy,'' she added.

But he was not put off at all. ``My first time seeing her, I was
delighted,'' Mr. Chang said. ``Not long afterward, we started discussing
marriage.''

The couple wed in 1959 and became parents to two sons and two daughters,
and, eventually, grandparents to six. They worked together at the
business that he had been managing since the age of 14, doing dry
cleaning and laundry for neighbors in Houli. They built up a large
clientele, some of whom still bring their laundry there despite having
moved long ago to Downtown Taichung.

Now, Wansho Laundry, which takes its name from the second characters of
the proprietors' names, is open daily from 8 a.m. until 9 p.m., although
it sometimes closes early if it's raining, Mr. Chang said. He and his
wife are the only employees.

Image

Hundreds of garments have been left behind.Credit...An Rong Xu for The
New York Times

Image

Putting together a look from forsaken clothes.Credit...An Rong Xu for
The New York Times

In the 1980s, the two began traveling abroad after the end of 38 years
of martial law in Taiwan, visiting the United States, Japan, Europe and
Australia. Now, those travels help connect them with many of the
messages coming in from all corners of the globe over their Instagram
photos, the younger Mr. Chang said.

``I'll read them some of the messages we get and tell them where the
senders are from, and they'll say, `Ah, I've been there!''' he said.

Mr. Chang said he hoped that his and his wife's experience would inspire
other older residents in Taiwan and elsewhere to be active.

``It's better than sitting around watching TV or napping,'' he said. ``I
might be getting on in my years, but I don't feel old.''

Reef Chang said the past few weeks have been a special time for his
grandparents --- **** customers stick around and chat a little longer,
which have made the couple happier. They are also tickled by the
friendly messages sent from around the world. ``Lately, whenever we eat
together,'' he said, ``I can tell they're elated.''

Image

Mr. Chang acts as his grandparents' unofficial stylist. He says they
have been elated at the response.Credit...An Rong Xu for The New York
Times

Internet fame is famously fleeting, and the owners of Wansho Laundry
have no desire to cash in on their side gig. Although, Mr. Chang said,
he'd be happy if the hundreds of people who have forgotten to pick up
their laundry would return to pay their bills.

``It would be nice to chat with them,'' he said, arching an eyebrow.
``And to get paid.''

On Thursday morning, for the first time in nearly seven decades,
something unusual happened at Wansho Laundry. A customer who had dropped
off clothing more than a year ago and saw the couple in the local news
finally came back to collect the garments --- and to pay the bill.

Advertisement

\protect\hyperlink{after-bottom}{Continue reading the main story}

\hypertarget{site-index}{%
\subsection{Site Index}\label{site-index}}

\hypertarget{site-information-navigation}{%
\subsection{Site Information
Navigation}\label{site-information-navigation}}

\begin{itemize}
\tightlist
\item
  \href{https://help.nytimes.com/hc/en-us/articles/115014792127-Copyright-notice}{©~2020~The
  New York Times Company}
\end{itemize}

\begin{itemize}
\tightlist
\item
  \href{https://www.nytco.com/}{NYTCo}
\item
  \href{https://help.nytimes.com/hc/en-us/articles/115015385887-Contact-Us}{Contact
  Us}
\item
  \href{https://www.nytco.com/careers/}{Work with us}
\item
  \href{https://nytmediakit.com/}{Advertise}
\item
  \href{http://www.tbrandstudio.com/}{T Brand Studio}
\item
  \href{https://www.nytimes.com/privacy/cookie-policy\#how-do-i-manage-trackers}{Your
  Ad Choices}
\item
  \href{https://www.nytimes.com/privacy}{Privacy}
\item
  \href{https://help.nytimes.com/hc/en-us/articles/115014893428-Terms-of-service}{Terms
  of Service}
\item
  \href{https://help.nytimes.com/hc/en-us/articles/115014893968-Terms-of-sale}{Terms
  of Sale}
\item
  \href{https://spiderbites.nytimes.com}{Site Map}
\item
  \href{https://help.nytimes.com/hc/en-us}{Help}
\item
  \href{https://www.nytimes.com/subscription?campaignId=37WXW}{Subscriptions}
\end{itemize}
