Sections

SEARCH

\protect\hyperlink{site-content}{Skip to
content}\protect\hyperlink{site-index}{Skip to site index}

\href{https://www.nytimes.com/section/your-money}{Your Money}

\href{https://myaccount.nytimes.com/auth/login?response_type=cookie\&client_id=vi}{}

\href{https://www.nytimes.com/section/todayspaper}{Today's Paper}

\href{/section/your-money}{Your Money}\textbar{}Need Help With Your
Estate Plan? Go With the Flow, Advisers Say

\url{https://nyti.ms/3hubuIj}

\begin{itemize}
\item
\item
\item
\item
\item
\end{itemize}

\href{https://www.nytimes.com/spotlight/at-home?action=click\&pgtype=Article\&state=default\&region=TOP_BANNER\&context=at_home_menu}{At
Home}

\begin{itemize}
\tightlist
\item
  \href{https://www.nytimes.com/2020/07/28/books/time-for-a-literary-road-trip.html?action=click\&pgtype=Article\&state=default\&region=TOP_BANNER\&context=at_home_menu}{Take:
  A Literary Road Trip}
\item
  \href{https://www.nytimes.com/2020/07/29/magazine/bored-with-your-home-cooking-some-smoky-eggplant-will-fix-that.html?action=click\&pgtype=Article\&state=default\&region=TOP_BANNER\&context=at_home_menu}{Cook:
  Smoky Eggplant}
\item
  \href{https://www.nytimes.com/2020/07/27/travel/moose-michigan-isle-royale.html?action=click\&pgtype=Article\&state=default\&region=TOP_BANNER\&context=at_home_menu}{Look
  Out: For Moose}
\item
  \href{https://www.nytimes.com/interactive/2020/at-home/even-more-reporters-editors-diaries-lists-recommendations.html?action=click\&pgtype=Article\&state=default\&region=TOP_BANNER\&context=at_home_menu}{Explore:
  Reporters' Obsessions}
\end{itemize}

Advertisement

\protect\hyperlink{after-top}{Continue reading the main story}

Supported by

\protect\hyperlink{after-sponsor}{Continue reading the main story}

Wealth Matters

\hypertarget{need-help-with-your-estate-plan-go-with-the-flow-advisers-say}{%
\section{Need Help With Your Estate Plan? Go With the Flow, Advisers
Say}\label{need-help-with-your-estate-plan-go-with-the-flow-advisers-say}}

As older adults face mortality during the pandemic, lawyers and wealth
advisers are using color-coded documents and flowcharts to help them
understand estate planning.

\includegraphics{https://static01.nyt.com/images/2020/07/24/business/24Wealth-01/merlin_174870186_99002052-84b4-4864-ba37-55749eb780c0-articleLarge.jpg?quality=75\&auto=webp\&disable=upscale}

By \href{https://www.nytimes.com/by/paul-sullivan}{Paul Sullivan}

\begin{itemize}
\item
  July 24, 2020
\item
  \begin{itemize}
  \item
  \item
  \item
  \item
  \item
  \end{itemize}
\end{itemize}

Andrew D. Hendry rose through the corporate ranks to become the vice
chairman and general counsel for Colgate-Palmolive, the global consumer
products company. As a lawyer, he understood complicated legal documents
and how they guided the inner workings of a large corporation.

But when it came to his estate plan, Mr. Hendry, like many others, was
not terribly interested in digging through hundreds of pages of legal
papers.

An estate plan is intended to distribute our assets after our death. The
task can sometimes be too mundane or too macabre for many, and it is
often put off.

When the coronavirus crisis put mortality front and center, Mr. Hendry,
72, felt it was time to revisit his plan. He found comfort in what his
wealth adviser had created: a series of color-coded documents that laid
out exactly who got what, when and why.

``The flowchart is the guiding document,'' said Mr. Hendry, who lives in
Pinehurst, N.C., with his wife, Mary. ``I'm a lawyer, and I understand
estate planning documents have to be pretty heavy for the estate plan to
work. But they're really not useful to make a decision.''

Neatly diagramed flow charts and color-coded spreadsheets are not what
most people think of when they envision the densely worded documents
that will carry out their last wishes. But the mortality risk for older
adults who contract the coronavirus has pushed many people to call their
lawyers and wealth advisers to make sure their affairs are in order.
Charts are much easier to comprehend than legal jargon.

``More people are looking to review their estate plans if something
happens, but it's hard to keep track of everything without a schedule
like this,'' said John J. Voltaggio, a managing wealth adviser at
Northern Trust who creates color-coded charts and simple spreadsheets
for his clients, including Mr. Hendry. ``We have that on one page. And
then we can ask, `Should we update any of it?'''

\includegraphics{https://static01.nyt.com/images/2020/07/24/business/24Wealth-02/24Wealth-02-articleLarge.jpg?quality=75\&auto=webp\&disable=upscale}

Typical color-coded plans come in several sections. A flowchart can be a
single page. It starts with the total assets, or estate value, at the
top. When one spouse dies, if the estate is large enough, some amount
flows into a trust --- the maximum tax-free amount is \$11.58 million
per person --- and the rest flows to the surviving spouse.

When the second spouse dies, the flowchart can present options, with
more money flowing into a tax-exempt trust for heirs or going to
children outright, to charities, to estate taxes or to friends and
family as bequests.

A more detailed spreadsheet allows people to tweak how much is going to
each heir, and to see what the estate tax ramifications are depending on
what assets are transferred, how they're transferred and when they're
transferred.

Mr. Hendry has a son but no grandchildren. He has a second home in
Amelia Island, Fla., and has various charitable interests. What he finds
helpful with the spreadsheet are the detailed financial values before
and after taxes. Playing with the values in the spreadsheet, like
adjusting risk versus return preferences in an investment plan, allows
him to see the consequences of his decisions.

``Putting it together like that allows you to make a reasonable
judgment,'' he said. ``You can stress test it and understand who
benefits and what happens when you make your decisions.''

Mr. Hendry said he and his wife used the flow chart to discuss with Mr.
Voltaggio what would happen if they put their houses in trust. They are
still mulling the consequences to their estate plan.

Mr. Voltaggio said the charts might simplify things for each client, but
they are not easy to put together. Each one is created from the stack of
documents and financial statements that his clients bring.

For people who are not trained lawyers, the charts can help them
understand the plan and can serve as verification that their desires
will be fulfilled.

``Our process is we summarize it, we visualize it, and we talk through
it,'' said Joseph C. Kellogg, head of wealth planning at WE Family
Offices, which manages money for affluent families.

The color coding serves as a way to make certain points stand out. ``If
we spot it, we make sure families spot it, too,'' he said.

With investments, for example, highlighting certain areas can draw a
person's attention to who will be in charge of making investment and
distribution decisions for the estate.

``Oftentimes, the person connected with the succession planning of the
investments is not the person they thought it was,'' Mr. Kellogg said.
``If there is a trustee, they want to make sure it's the right person.''

Yet a pretty picture can be just that if people don't know what to look
for. They need to make sure the right managers are in charge of their
assets, their health care and their children, either young and in need
of care or older and awaiting an inheritance.

Different assets are transferred through different legal mechanisms.
Retirement accounts, for example, are governed by the beneficiary
designation forms. Property and collections are transferred through the
will. In certain states, not putting assets in trusts means those assets
have to go through a long and sometimes costly probate process. All of
this can be highlighted in a flowchart.

The ``if/then'' clauses that populate estate documents can be more
complicated. They're meant to trigger an action if a set of criteria is
met. One example would be more money going to someone if the value of a
certain asset rose.

Pay attention to these clauses to make sure you understand what you are
doing, said Ivan Hernandez, a co-founder of Omnia Family Wealth.
Diagraming them can be complex.

``The dream is to have everything on one page,'' he said.

For that reason, some trust and estate lawyers stick to long memos to
summarize estate plans and point-by-point conversations with their
clients. James I. Dougherty, a partner at the law firm Withersworldwide,
said that he had been sending illustrations to clients for phone
conferences, but that he always came back to memos to lay everything
out.

``If you have something where the parents' estate plan overweighs a
distribution of money to one child over another --- say because one
child got a down payment on a house --- we talk about it and we put it
in a memo,'' he said.

But with large estates, litigation is always a concern. ``Down the road,
you don't want to be on the witness stand and say, `The stuff in green
is going to go here,''' Mr. Dougherty said. ``You want to have that
lengthy memo.''

You can tweak things in the chart, but your lawyer has to put the
changes into your estate documents for them to be effective. (This is
where the exercise differs from making adjustments to investments; your
adviser can make those changes on the spot.)

At the end of the day, Mr. Hendry said, he, like any one else, just
wanted the plan to work, both on paper now and in practice later.

``If I didn't have this flowchart, I'm not sure what I would do,'' he
said. ``I'm not going to sit down and read 500 pages of documents. By
doing this, it gives my wife and me a sense of security that we have
control of this situation and have it laid out as best we can.''

Advertisement

\protect\hyperlink{after-bottom}{Continue reading the main story}

\hypertarget{site-index}{%
\subsection{Site Index}\label{site-index}}

\hypertarget{site-information-navigation}{%
\subsection{Site Information
Navigation}\label{site-information-navigation}}

\begin{itemize}
\tightlist
\item
  \href{https://help.nytimes.com/hc/en-us/articles/115014792127-Copyright-notice}{©~2020~The
  New York Times Company}
\end{itemize}

\begin{itemize}
\tightlist
\item
  \href{https://www.nytco.com/}{NYTCo}
\item
  \href{https://help.nytimes.com/hc/en-us/articles/115015385887-Contact-Us}{Contact
  Us}
\item
  \href{https://www.nytco.com/careers/}{Work with us}
\item
  \href{https://nytmediakit.com/}{Advertise}
\item
  \href{http://www.tbrandstudio.com/}{T Brand Studio}
\item
  \href{https://www.nytimes.com/privacy/cookie-policy\#how-do-i-manage-trackers}{Your
  Ad Choices}
\item
  \href{https://www.nytimes.com/privacy}{Privacy}
\item
  \href{https://help.nytimes.com/hc/en-us/articles/115014893428-Terms-of-service}{Terms
  of Service}
\item
  \href{https://help.nytimes.com/hc/en-us/articles/115014893968-Terms-of-sale}{Terms
  of Sale}
\item
  \href{https://spiderbites.nytimes.com}{Site Map}
\item
  \href{https://help.nytimes.com/hc/en-us}{Help}
\item
  \href{https://www.nytimes.com/subscription?campaignId=37WXW}{Subscriptions}
\end{itemize}
