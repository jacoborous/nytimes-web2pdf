Sections

SEARCH

\protect\hyperlink{site-content}{Skip to
content}\protect\hyperlink{site-index}{Skip to site index}

\href{https://www.nytimes.com/section/obituaries}{Obituaries}

\href{https://myaccount.nytimes.com/auth/login?response_type=cookie\&client_id=vi}{}

\href{https://www.nytimes.com/section/todayspaper}{Today's Paper}

\href{/section/obituaries}{Obituaries}\textbar{}Marlene Sekaquaptewa,
Hopi Tribal Leader and Quiltmaker, Dies at 79

\url{https://nyti.ms/2WQWxba}

\begin{itemize}
\item
\item
\item
\item
\item
\end{itemize}

\href{https://www.nytimes.com/news-event/coronavirus?action=click\&pgtype=Article\&state=default\&region=TOP_BANNER\&context=storylines_menu}{The
Coronavirus Outbreak}

\begin{itemize}
\tightlist
\item
  live\href{https://www.nytimes.com/2020/08/03/world/coronavirus-covid-19.html?action=click\&pgtype=Article\&state=default\&region=TOP_BANNER\&context=storylines_menu}{Latest
  Updates}
\item
  \href{https://www.nytimes.com/interactive/2020/us/coronavirus-us-cases.html?action=click\&pgtype=Article\&state=default\&region=TOP_BANNER\&context=storylines_menu}{Maps
  and Cases}
\item
  \href{https://www.nytimes.com/interactive/2020/science/coronavirus-vaccine-tracker.html?action=click\&pgtype=Article\&state=default\&region=TOP_BANNER\&context=storylines_menu}{Vaccine
  Tracker}
\item
  \href{https://www.nytimes.com/2020/08/02/us/covid-college-reopening.html?action=click\&pgtype=Article\&state=default\&region=TOP_BANNER\&context=storylines_menu}{College
  Reopening}
\item
  \href{https://www.nytimes.com/live/2020/08/03/business/stock-market-today-coronavirus?action=click\&pgtype=Article\&state=default\&region=TOP_BANNER\&context=storylines_menu}{Economy}
\end{itemize}

Advertisement

\protect\hyperlink{after-top}{Continue reading the main story}

Supported by

\protect\hyperlink{after-sponsor}{Continue reading the main story}

those we've lost

\hypertarget{marlene-sekaquaptewa-hopi-tribal-leader-and-quiltmaker-dies-at-79}{%
\section{Marlene Sekaquaptewa, Hopi Tribal Leader and Quiltmaker, Dies
at
79}\label{marlene-sekaquaptewa-hopi-tribal-leader-and-quiltmaker-dies-at-79}}

She played a major role in Hopi Tribal government for decades and was
governor of the village of Bacavi when she died after testing positive
for Covid-19.

\includegraphics{https://static01.nyt.com/images/2020/07/29/obituaries/24Sekaquaptewa/merlin_174870510_d8358d57-9609-4e29-9bb8-e8e7ab8af093-articleLarge.jpg?quality=75\&auto=webp\&disable=upscale}

By \href{https://www.nytimes.com/by/simon-romero}{Simon Romero}

\begin{itemize}
\item
  Published July 24, 2020Updated July 28, 2020
\item
  \begin{itemize}
  \item
  \item
  \item
  \item
  \item
  \end{itemize}
\end{itemize}

\emph{This obituary is part of a series about people who have died in
the coronavirus pandemic. Read about others}
\href{https://www.nytimes.com/interactive/2020/obituaries/people-died-coronavirus-obituaries.html}{\emph{here}}\emph{.}

Marlene Sekaquaptewa was the matriarch of a large, distinguished family,
a master quiltmaker and a political leader who played a major role in
the Hopi Tribal government for decades.

``She was a cultural ambassador, very involved in public life,'' said
her niece Patricia Sekaquaptewa, 53, a justice on the Hopi Appellate
Court and a professor specializing in tribal criminal justice at the
University of Alaska, Fairbanks. ``I was always amazed at how she could
do so many things at once.''

As the coronavirus began taking its toll in the soaring high-desert
mesas where the Hopi live in northeastern Arizona, it claimed Ms.
Sekaquaptewa, who was the governor of the Hopi village of Bacavi. She
died on June 24 in Mesa, Ariz., of Covid-19. She was 79.

Ms. Sekaquaptewa (pronounced roughly see-KIA-cwop-tee-wah) was born on
July 10, 1940, into a prominent Hopi family. Her mother, Helen, was a
homemaker who described her own life story on and off the reservation in
the 1969 \href{https://uapress.arizona.edu/book/me-and-mine}{book} ``Me
and Mine.'' Her father, Emory, was a farmer and tribal judge.

One of Ms. Sekaquaptewa's brothers, Emory Jr., was an anthropologist at
the University of Arizona who compiled the first comprehensive Hopi
dictionary; another, Abbott, was a longtime Hopi tribal chairman.

Ms. Sekaquaptewa's family was from Oraibi, a village that is one of the
oldest continuously inhabited places in the United States. Some of her
great-uncles were among the 19 Hopi men
\href{https://calendar.eji.org/racial-injustice/jan/03}{imprisoned} on
Alcatraz Island in California in the 1890s when they resisted sending
their children to assimilationist boarding schools.

Growing up in Arizona, and later as an adult, Ms. Sekaquaptewa moved
between different worlds. She graduated from Central High School in
Phoenix while her parents lived away from the reservation.

Ms. Sekaquaptewa lived briefly in Los Angeles during the
\href{https://www.archives.gov/education/lessons/indian-relocation.html}{relocation}
era in the 1950s and '60s, when United States authorities contentiously
tried to disband tribes and assimilate Native Americans in cities.

Back in Arizona, Ms. Sekaquaptewa started a family, graduated from the
tribal development program at Scottsdale Community College and got into
politics. Her husband, Leroy Kewanimptewa Sr., and two of her five
children, Kenneth and Paul, died before her. She is survived by a
daughter, Dianna Shebala; two sons, Leroy Kewanimptewa Jr. and Emory
Kewanimptewa; 14 grandchildren; and 12 great-grandchildren.

She served multiple times as governor of the village of Bacavi and was a
key figure in drafting the Hopi Tribal Constitution in 2012. She
recently helped create an assisted living facility for Hopi elders.

Ms. Sekaquaptewa was also a renowned quiltmaker whose creations have
been displayed in museums around the country. Scholars often consulted
her about Hopi culture and traditions.

In 2018, Ms. Sekaquaptewa
\href{https://www.pbs.org/video/hopi-origin-story-dc0awe/}{narrated} in
Hopi a brief description of the tribe's creation epic for PBS. ``We
lived beneath the earth and it came time for us to emerge,'' she said,
recounting how the Hopi people received guidance from the earth's
ancient caretaker, Maasaw.

``So we made a covenant to walk to the earth's farthest corners,'' she
said, ``to learn the earth with our feet and to become one with this new
world.''

\href{https://www.nytimes.com/interactive/2020/obituaries/people-died-coronavirus-obituaries.html?action=click\&pgtype=Article\&state=default\&region=BELOW_MAIN_CONTENT\&context=covid_obits_promo}{}

\hypertarget{those-weve-lost}{%
\section{Those We've Lost}\label{those-weve-lost}}

The coronavirus pandemic has taken an incalculable death toll. This
series is designed to put names and faces to the numbers.

Read more

\includegraphics{https://static01.nyt.com/images/2020/07/30/obituaries/30Pedro/30Pedro-square640.jpg}

\hypertarget{bernaldina-josuxe9-pedro}{%
\section{Bernaldina José Pedro}\label{bernaldina-josuxe9-pedro}}

d. Boa Vista, Brazil

Leader among the Indigenous Macuxi

\includegraphics{https://static01.nyt.com/images/2020/07/31/obituaries/31Swing/merlin_175167783_8913bc90-0d64-43f3-a655-1bb1bf1601c9-square640.jpg}

\hypertarget{john-eric-swing}{%
\section{John Eric Swing}\label{john-eric-swing}}

d. Fountain Valley, Calif.

Champion of Filipino-Americans

\includegraphics{https://static01.nyt.com/images/2020/07/27/obituaries/27Victor/merlin_175001436_38b11f8e-227a-4e2c-9821-7618af9b2524-square640.jpg}

\hypertarget{victor-victor}{%
\section{Victor Victor}\label{victor-victor}}

d. Santo Domingo, Dominican Republic

Beloved musician of the Dominican Republic

\includegraphics{https://static01.nyt.com/images/2020/07/31/obituaries/31Negron/merlin_175160169_516322ae-fd23-4969-b6b2-193ced371105-square640.jpg}

\hypertarget{dr-eddie-negruxf3n}{%
\section{Dr. Eddie Negrón}\label{dr-eddie-negruxf3n}}

d. Fort Walton Beach, Fla.

Internist on Florida's Emerald Coast

\includegraphics{https://static01.nyt.com/images/2020/07/30/obituaries/30Dobson/merlin_175115928_f6b9271c-8f05-4fe1-a38a-5ca4a58f8935-square640.jpg}

\hypertarget{dobby-dobson}{%
\section{Dobby Dobson}\label{dobby-dobson}}

d. Coral Springs, Fla.

Jamaican singer and songwriter

\includegraphics{https://static01.nyt.com/images/2020/08/01/obituaries/28Gonzalez/merlin_175002771_beb57888-3951-409a-ae13-03a94b2e962e-square640.jpg}

\hypertarget{waldemar-gonzalez}{%
\section{Waldemar Gonzalez}\label{waldemar-gonzalez}}

d. White Plains, N.Y.

Teacher and social worker

Advertisement

\protect\hyperlink{after-bottom}{Continue reading the main story}

\hypertarget{site-index}{%
\subsection{Site Index}\label{site-index}}

\hypertarget{site-information-navigation}{%
\subsection{Site Information
Navigation}\label{site-information-navigation}}

\begin{itemize}
\tightlist
\item
  \href{https://help.nytimes.com/hc/en-us/articles/115014792127-Copyright-notice}{©~2020~The
  New York Times Company}
\end{itemize}

\begin{itemize}
\tightlist
\item
  \href{https://www.nytco.com/}{NYTCo}
\item
  \href{https://help.nytimes.com/hc/en-us/articles/115015385887-Contact-Us}{Contact
  Us}
\item
  \href{https://www.nytco.com/careers/}{Work with us}
\item
  \href{https://nytmediakit.com/}{Advertise}
\item
  \href{http://www.tbrandstudio.com/}{T Brand Studio}
\item
  \href{https://www.nytimes.com/privacy/cookie-policy\#how-do-i-manage-trackers}{Your
  Ad Choices}
\item
  \href{https://www.nytimes.com/privacy}{Privacy}
\item
  \href{https://help.nytimes.com/hc/en-us/articles/115014893428-Terms-of-service}{Terms
  of Service}
\item
  \href{https://help.nytimes.com/hc/en-us/articles/115014893968-Terms-of-sale}{Terms
  of Sale}
\item
  \href{https://spiderbites.nytimes.com}{Site Map}
\item
  \href{https://help.nytimes.com/hc/en-us}{Help}
\item
  \href{https://www.nytimes.com/subscription?campaignId=37WXW}{Subscriptions}
\end{itemize}
