Sections

SEARCH

\protect\hyperlink{site-content}{Skip to
content}\protect\hyperlink{site-index}{Skip to site index}

\href{https://www.nytimes.com/section/us}{U.S.}

\href{https://myaccount.nytimes.com/auth/login?response_type=cookie\&client_id=vi}{}

\href{https://www.nytimes.com/section/todayspaper}{Today's Paper}

\href{/section/us}{U.S.}\textbar{}Federal Agents Push Into Portland
Streets, Stretching Limits of Their Authority

\url{https://nyti.ms/2WWSN86}

\begin{itemize}
\item
\item
\item
\item
\item
\end{itemize}

\href{https://www.nytimes.com/news-event/george-floyd-protests-minneapolis-new-york-los-angeles?action=click\&pgtype=Article\&state=default\&region=TOP_BANNER\&context=storylines_menu}{Race
and America}

\begin{itemize}
\tightlist
\item
  \href{https://www.nytimes.com/2020/07/26/us/protests-portland-seattle-trump.html?action=click\&pgtype=Article\&state=default\&region=TOP_BANNER\&context=storylines_menu}{Protesters
  Return to Other Cities}
\item
  \href{https://www.nytimes.com/2020/07/24/us/portland-oregon-protests-white-race.html?action=click\&pgtype=Article\&state=default\&region=TOP_BANNER\&context=storylines_menu}{Portland
  at the Center}
\item
  \href{https://www.nytimes.com/2020/07/23/podcasts/the-daily/portland-protests.html?action=click\&pgtype=Article\&state=default\&region=TOP_BANNER\&context=storylines_menu}{Podcast:
  Showdown in Portland}
\item
  \href{https://www.nytimes.com/interactive/2020/07/16/us/black-lives-matter-protests-louisville-breonna-taylor.html?action=click\&pgtype=Article\&state=default\&region=TOP_BANNER\&context=storylines_menu}{45
  Days in Louisville}
\end{itemize}

Advertisement

\protect\hyperlink{after-top}{Continue reading the main story}

Supported by

\protect\hyperlink{after-sponsor}{Continue reading the main story}

\hypertarget{federal-agents-push-into-portland-streets-stretching-limits-of-their-authority}{%
\section{Federal Agents Push Into Portland Streets, Stretching Limits of
Their
Authority}\label{federal-agents-push-into-portland-streets-stretching-limits-of-their-authority}}

Federal agents are venturing blocks from the buildings they were sent to
protect. Oregon officials say they are illegally taking on the role of
riot police.

\includegraphics{https://static01.nyt.com/images/2020/07/17/autossell/portland-v1-2/portland-v1-2-videoSixteenByNineJumbo1600.jpg}

By \href{https://www.nytimes.com/by/mike-baker}{Mike Baker},
\href{https://www.nytimes.com/by/thomas-fuller}{Thomas Fuller} and
Sergio Olmos

\begin{itemize}
\item
  Published July 25, 2020Updated July 31, 2020
\item
  \begin{itemize}
  \item
  \item
  \item
  \item
  \item
  \end{itemize}
\end{itemize}

PORTLAND, Ore. --- After flooding the streets around the federal
courthouse in Portland with tear gas during Friday's early morning
hours, dozens of
\href{https://www.nytimes.com/2020/07/30/nyregion/nypd-protester-van.html}{federal
officers} in camouflage and tactical gear stood in formation around the
front of the building.

Then, as one protester blared a soundtrack of ``The Imperial March,''
the officers started advancing. Through the acrid haze, they continued
to fire flash grenades and welt-inducing marble-size balls filled with
caustic chemicals. They moved down Main Street and continued up the
hill, where one of the agents announced over a loudspeaker: ``This is an
unlawful assembly.''

By the time the security forces halted their advance, the federal
courthouse they had been sent to protect was out of sight --- two blocks
behind them.

The aggressive incursion of federal officers into Portland has been
stretching the legal limits of federal law enforcement, as agents with
batons and riot gear range deep into the streets of a city whose
leadership has made it clear they are not welcome.

``I think it's absolutely improper,'' Oregon's attorney general, Ellen
Rosenblum, said in an interview on Friday. ``It's absolutely beyond
their authority.''

The state lost its bid on Friday
\href{https://www.nytimes.com/2020/07/24/us/portland-federal-jurisdiction-court-judge.html}{for
a restraining order against four federal agencies} on the grounds that
the state attorney general lacked standing, but several other challenges
are still making their way through the courts.

Federal officers who arrived this month to help control protests over
racial injustice and police violence have made dozens of arrests for
federal crimes, including assaults on federal officers and failing to
comply with law enforcement commands. More than 60 protesters have been
arrested, and 46 now face federal criminal charges, said Craig Gabriel,
an assistant U.S. attorney for the District of Oregon, in a Saturday
news conference.

One protester standing on a city street outside the federal courthouse
was shot in the head with a crowd-control munition, leaving a bloody
scene and a serious facial injury that required surgery. In another
incident, an officer was seen repeatedly using a baton to
\href{https://www.nytimes.com/2020/07/20/us/portland-protests-navy-christopher-david.html}{whack
a Navy veteran who said he had come to speak to the agents}. Videos
taken by members of the public captured
\href{https://www.nytimes.com/2020/07/17/us/portland-protests.html}{camouflaged
personnel pulling protesters into unmarked vans}.

The inspectors general of the Department of Justice and the Department
of Homeland Security
\href{https://www.nytimes.com/2020/07/23/us/seattle-protests-feds.html}{have
opened investigations into the tactics}.

During 57 consecutive nights of protests, demonstrators have squared off
first with the Portland police and then with federal agents in what at
times have been pitched battles, with protesters throwing water bottles
or fireworks and agents responding with frequent volleys of tear gas.
The arrival of the federal agents caused the protests to swell and
focused the ire of protesters onto
\href{https://www.nytimes.com/2020/07/22/us/portland-protests-courthouse.html}{the
Mark O. Hatfield U.S. Courthouse}, across from a park shaded by mature
trees.

What began as a movement for racial justice became a broader campaign to
dislodge the federal forces from the city.

The federal agents from four agencies arrived after President Trump
signed an executive order on June 26 ordering the protection of federal
monuments and buildings.

Their presence quickly became a political rallying point.

Senator Ron Wyden of Oregon, a Democrat, compared the agents to an
``occupying army.'' Speaker Nancy Pelosi, Democrat of California, called
them ``storm troopers.''

Mr. Trump criticized the police protests around the country in cities
``all run by liberal Democrats'' and defended the move to send in
federal agents, warning that with the continuing turbulence in the
streets, ``They were going to lose Portland.''

Chad F. Wolf, the acting secretary of homeland security, described the
protesters squaring off with federal agents outside the federal
courthouse in Portland as ``anarchists and criminals.''

``We will continue to take the appropriate action to protect our
facilities and our law enforcement officers,'' Mr. Wolf said at a news
briefing this past week. ``If we left tomorrow they would burn that
building down.''

There is broad agreement among legal scholars that the federal
government has the right to protect its buildings. But how far that
authority extends into a city --- and which tactics may be employed ---
is less clear.

Robert Tsai, a professor at the Washington College of Law at American
University, said the nation's founders explicitly left local policing
within the jurisdiction of local authorities.

He questioned whether the federal agents had the right to extend their
operations blocks away from the buildings they are protecting.

``If the federal troops are starting to wander the streets, they appear
to be crossing the line into general policing, which is outside their
powers,'' Professor Tsai said.

Homeland Security officials say they are operating under a federal
statute that permits federal agents to venture outside the boundaries of
the courthouse to ``conduct investigations'' into crimes against federal
property or officers.

But patrolling the streets and detaining or tear-gassing protesters go
beyond that legal authority, said David Lapan, the former spokesman for
the agency when it was led by John Kelly, Mr. Trump's first secretary of
homeland security.

``That's not an investigation,'' Mr. Lapan said. ``That's just a show of
force.''

John Malcolm, vice president for the Institute for Constitutional
Government at the conservative Heritage Foundation, and a former deputy
assistant attorney general during the George W. Bush administration,
said federal agents have clear legal authority to pursue protesters who
have damaged federal property.

``Once they have committed a crime the federal authorities have probable
cause to go arrest them,'' Mr. Malcolm said. ``I don't care how many
blocks away they are from that property.''

While federal authorities are not intended to be riot police, he said,
the federal government has the authority to send in troops in extreme
situations in which there is a breakdown of authority and local
officials are unable to effectively enforce local laws.

``But we are not there yet, and I pray that we don't get there,'' he
said.

Outraged by the federal presence, government leaders in Portland have
been looking for ways to push back against the deployment. The Portland
Police Bureau ousted federal representatives from the city's command
post. Mayor Ted Wheeler, who himself was
\href{https://www.nytimes.com/2020/07/23/us/portland-protest-tear-gas-mayor.html}{hit
with tear gas fired by federal agents on Wednesday night}, called the
federal deployment an abuse of authority.

``My colleagues and I are looking at every possible legal option we have
to get the feds out of here,'' Mr. Wheeler said in an interview.

In the state's legal challenge, Ms. Rosenblum argued that the operations
of federal authorities, using unmarked vehicles to detain protesters,
resembled abductions. The lawsuit called on the court to order the
agents to stop arresting individuals without probable cause and to
clearly identify themselves and their agency before detaining or
arresting ``any person off the streets in Oregon.''

But in his ruling on Friday, Judge Michael W. Mosman of the U.S.
District Court in Portland said the state attorney general's office did
not have standing to bring the case because it had not shown that the
issue was ``an interest that is specific to the state itself.''

In an interview, Ms. Rosenblum said that having federal agents battling
protesters in Portland was un-American because the country does not have
a tradition of a national police force.

``The police should be ideally as local as possible,'' she said. ``It's
about trust, relationships and community building.''

She warned that all Americans need to be concerned about what is
happening in Portland.

``It could be happening in your city next,'' she said.

The inspector general of the Department of Homeland Security, Joseph V.
Cuffari, told lawmakers in a letter that he planned to examine the
authority the agency used to deploy agents to Portland.

Some of the protesters who originally focused their anger on the case of
George Floyd, whose death in police custody in Minneapolis in May
sparked demonstrations around the country, now have turned much of their
attention to the presence of federal officers on Portland's streets.

On Friday night, a crowd gathered outside a fence erected around the
federal courthouse; some in the crowd lit fires, lobbed fireworks over
the fence and attempted to pull it down with power tools. Federal agents
entered the street to disperse the crowd at 2:30 a.m.

Mr. Gabriel, the assistant U.S. attorney, said that the federal officers
were forced into the streets to protect the fence. ``The officers would
love nothing more than to stay in the courthouse all night long,'' he
said. ``If the protesters don't seek to damage or destroy the fence,
then the officers have no need to go outside the fence or leave federal
property.''

Most of the demonstrations during the evening, though, were peaceful. A
\href{https://www.nytimes.com/2020/07/25/us/a-wall-of-vets-joins-the-front-lines-of-portland-protests.html}{group
of military veterans lined up along the fence}, joining a ``Wall of
Moms,'' hundreds of mothers who have linked arms to challenge the
presence of the federal agents, who had been there on previous nights.
There was also a ``Wall of Dads" carrying leaf blowers to combat the
tear gas.

Jennifer Kristiansen, a family-law attorney, was one of many women who
came out to the protests in recent days to join the ``Wall of Moms.'' In
the early morning hours on Tuesday, she said, as agents were clearing
protesters from in front of the courthouse, one of them reported to
another that Ms. Kristiansen had struck him.

Ms. Kristiansen said that she had done no such thing and that one of the
officers ended up assaulting her, groping her chest and backside during
the arrest.

``This is not creeping authoritarianism,'' Ms. Kristiansen said. ``The
authoritarianism is here.''

Kate Conger contributed reporting from Portland, and Zolan Kanno-Youngs
from Washington.

Advertisement

\protect\hyperlink{after-bottom}{Continue reading the main story}

\hypertarget{site-index}{%
\subsection{Site Index}\label{site-index}}

\hypertarget{site-information-navigation}{%
\subsection{Site Information
Navigation}\label{site-information-navigation}}

\begin{itemize}
\tightlist
\item
  \href{https://help.nytimes.com/hc/en-us/articles/115014792127-Copyright-notice}{©~2020~The
  New York Times Company}
\end{itemize}

\begin{itemize}
\tightlist
\item
  \href{https://www.nytco.com/}{NYTCo}
\item
  \href{https://help.nytimes.com/hc/en-us/articles/115015385887-Contact-Us}{Contact
  Us}
\item
  \href{https://www.nytco.com/careers/}{Work with us}
\item
  \href{https://nytmediakit.com/}{Advertise}
\item
  \href{http://www.tbrandstudio.com/}{T Brand Studio}
\item
  \href{https://www.nytimes.com/privacy/cookie-policy\#how-do-i-manage-trackers}{Your
  Ad Choices}
\item
  \href{https://www.nytimes.com/privacy}{Privacy}
\item
  \href{https://help.nytimes.com/hc/en-us/articles/115014893428-Terms-of-service}{Terms
  of Service}
\item
  \href{https://help.nytimes.com/hc/en-us/articles/115014893968-Terms-of-sale}{Terms
  of Sale}
\item
  \href{https://spiderbites.nytimes.com}{Site Map}
\item
  \href{https://help.nytimes.com/hc/en-us}{Help}
\item
  \href{https://www.nytimes.com/subscription?campaignId=37WXW}{Subscriptions}
\end{itemize}
