Sections

SEARCH

\protect\hyperlink{site-content}{Skip to
content}\protect\hyperlink{site-index}{Skip to site index}

\href{https://myaccount.nytimes.com/auth/login?response_type=cookie\&client_id=vi}{}

\href{https://www.nytimes.com/section/todayspaper}{Today's Paper}

\href{/section/opinion}{Opinion}\textbar{}What if We Could Have Meat
Without Murder?

\href{https://nyti.ms/38wbBjr}{https://nyti.ms/38wbBjr}

\begin{itemize}
\item
\item
\item
\item
\item
\end{itemize}

Advertisement

\protect\hyperlink{after-top}{Continue reading the main story}

\href{/section/opinion}{Opinion}

Supported by

\protect\hyperlink{after-sponsor}{Continue reading the main story}

THE STONE

\hypertarget{what-if-we-could-have-meat-without-murder}{%
\section{What if We Could Have Meat Without
Murder?}\label{what-if-we-could-have-meat-without-murder}}

We can, if we can agree that it doesn't need to come from the body of an
animal.

By Andy Lamey

Mr. Lamey is the author of ``Duty and the Beast: Should We Eat Meat in
the Name of Animal Rights?''

\begin{itemize}
\item
  July 2, 2020
\item
  \begin{itemize}
  \item
  \item
  \item
  \item
  \item
  \end{itemize}
\end{itemize}

\includegraphics{https://static01.nyt.com/images/2020/07/02/opinion/02Stone-Meat1/02Stone-Meat1-articleLarge.jpg?quality=75\&auto=webp\&disable=upscale}

\emph{What is meat?}

That question is unlikely to be asked along with the usual ones ---
Medium or well-done? Cheese or no cheese? --- over grills being fired up
all over the United States this summer. (Unless, of course, you invite a
philosopher to your barbecue.) But it is a timely one and how we answer
it --- how we ultimately define the word ``meat'' --- could have a
significant impact on the future of our food supply, our health and the
health of the planet.

It's no secret by now that the case against meat keeps getting stronger.
The social, environmental and ethical costs of industrial agriculture
--- exacerbated by a pandemic being traced back to a live animal market,
and a vulnerable meat processing industry --- have become too obvious
and damaging to ignore. Yet Americans on average consume more that 200
pounds of animal flesh each year. And, like it or not, it is still part
of how the United States sees itself --- cultural icons, from cowboys
and ranchers to the Golden Arches, express the country's long, tragic
love affair with meat.

But just as the meaning of American identity has changed over time, so
too has the food people eat to celebrate it. Fifty years ago, few
barbecues included burgers made of tofu or lentils for the stray
vegetarians found in so many families today.

For centuries, the definition of meat was obvious: the edible flesh of
an animal. That changed in 2013, when the Dutch scientist Mark Post
\href{https://www.nytimes.com/2013/05/14/science/engineering-the-325000-in-vitro-burger.html}{unveiled}
the first in vitro hamburger. By bathing animal stem cells with growth
serum, Dr. Post and his colleagues were able to grow a hamburger in
their lab. Their burger had essentially the same composition as a normal
hamburger but a different origin. Although Dr. Post estimated that the
first in vitro burger cost about \$325,000 to create, the price has come
down significantly and his team is one of several groups seeking to
commercialize in vitro meat and bring it to market. (Dr. Post's first
burger was grown using fetal bovine serum, a slaughterhouse byproduct;
his team and others have sought out animal-free replacements.)

This prospect has triggered opposition from the agriculture industry,
which in the past three years has petitioned lawmakers in some 25 states
to introduce bills to prevent alternative meat products being labeled
meat.

The timing of these bills is not coincidental. Lawmakers know that
plant-based meat substitutes have become big business: In 2019,
plant-based meat sales totaled
\href{https://www.gfi.org/blog-state-of-the-industry-2020\#:~:text=SPINS\%20data\%20compiled\%20by\%20GFI,has\%20greatly\%20outstripped\%20conventional\%20meat.}{\$939
million,} an 18 percent increase over the year before, while sales for
all plant-based foods reached \$5 billion. The real reason for the meat
industry's interest in grocery labels is that it is threatened by this
surge in popularity.

Missouri was the first jurisdiction where such a bill became law and it
has already been subject to a first-amendment challenge, a fate that
most likely awaits its counterparts in other states.

The debates now going on in many different state legislatures and
courthouses all revolve around this question: What is meat? The best
answer, in my view, is one that takes the arrival of in vitro flesh as
occasion to reconceive and broaden our idea of meat.

A helpful distinction is
\href{https://www.erudit.org/en/journals/ateliers/2018-v13-n1-ateliers04192/1055123ar/}{drawn}
by Jeff Sebo, the director of the animal studies program at New York
University, between a food item's origin, substance and function. The
traditional view of meat holds that its must originate in the body of an
animal. The substance of meat is what it is physically made of: muscle
tissue composed of protein, water, amino acids and the rest. Meat's
function is on one level something that we experience --- the familiar
combination of taste and texture in the mouth. Nutritionally, meat's
function varies --- it can affect our health for better or worse,
depending on how we prepare it or how much we consume.

A new framework that would allow us to classify lab grown meat as just
``meat'' would involve rethinking those principles. In vitro meat
generally satisfies the last two requirements --- substance and function
--- but not the first, origin. (I don't include plant-based products
here because they do not meet \emph{any} of the three conditions.)

It may seem like cheating to consciously redefine meat in order to
accommodate the lab-grown version. In fact, history is full of this type
of conceptual revision. Someone asking 100 years ago what a car is could
be forgiven for offering a definition that mentioned an internal
combustion engine or a human driver. In the age of self-driving and
electric cars we recognize that these are no longer defining features of
cars. Similarly, the commonly accepted definition of marriage was that
of a union between a man and a woman. When same-sex marriage was
legalized in the United States that version was reclassified as but one
option among others, all equally legitimate.

Revised understandings of cars and marriage involve the same kind of
shift. In the jargon of philosophers, we realized that we had long been
mistaking one particular \emph{conception} of cars or marriage for the
very \emph{concept}. Revising our understanding of meat to make room for
in vitro meat involves a similar move. We should strip down our
understanding of meat so that an element previously deemed essential ---
in this case, being sourced in an animal carcass --- is no longer
strictly necessary. On this updated, more minimalist understanding, all
that is necessary for something to qualify as meat is that it has a
meaty substance and function. Just as Model Ts and Teslas both qualify
as cars, animal-sourced and lab-grown versions would then both qualify
as real meat.

Two considerations support trimming the conceptual fat from our
understanding of meat in this way. The first is intuitive. Imagine you
are served two pieces of steak, one from a slaughterhouse the other from
a lab, which have an identical taste and nutritional effect. Food is by
definition what we eat, and if our experience of eating the two morsels
is the same surely they warrant a common concept.

The second is linguistic. We use the word ``milk'' to classify fluids
from cows, coconuts and nursing mothers, among other sources. If milk
can have more than one origin, why not meat?

Ludwig Wittgenstein argued in ``Philosophical Investigations'' that the
meaning of a word is its use in the language. Given that the term ``in
vitro meat'' and its synonyms (``lab-grown meat,'' ``cultured meat'')
are already widely used, it is tempting to go the full Wittgenstein and
cite common usage as grounds to declare the case for in vitro meat
closed. But, to be fair, a conceptual debate should not come down to a
popularity contest: same-sex marriage was once unpopular, yet that
hardly settled the dispute over the nature of marriage. A more cautious
handling of the linguistic evidence takes it to place the burden of
proof on those who would define ``meat'' to exclude the in vitro
version. Our default presumption should be that it is meat, barring good
arguments otherwise.

\includegraphics{https://static01.nyt.com/images/2020/07/02/opinion/02stone-meat2/02stone-meat2-articleLarge.jpg?quality=75\&auto=webp\&disable=upscale}

Such definitions are disingenuous, motived by financial considerations
rather than a good-faith inquiry into the meaning of terms.

Our ancestors regarded animals in a host of different ways --- as
currency, transportation, even objects of religious veneration --- that
may now seem strange to us. In vitro meat holds out the possibility that
our descendants may someday feel the same way about eating them.

Andy Lamey teaches philosophy at the University of California, San Diego
and is the author of ``Duty and the Beast: Should we Eat Meat in the
Name of Animal Rights?''

Now in print*: ``\emph{\href{http://bitly.com/1MW2kN3}{\emph{Modern
Ethics in 77 Arguments}}},'' and
``\emph{\href{http://bitly.com/1MW2kN3}{\emph{The Stone Reader: Modern
Philosophy in 133 Arguments}}},'' with essays from the series, edited by
Peter Catapano and Simon Critchley, published by Liveright Books.*

\emph{The Times is committed to publishing}
\href{https://www.nytimes.com/2019/01/31/opinion/letters/letters-to-editor-new-york-times-women.html}{\emph{a
diversity of letters}} \emph{to the editor. We'd like to hear what you
think about this or any of our articles. Here are some}
\href{https://help.nytimes.com/hc/en-us/articles/115014925288-How-to-submit-a-letter-to-the-editor}{\emph{tips}}\emph{.
And here's our email:}
\href{mailto:letters@nytimes.com}{\emph{letters@nytimes.com}}\emph{.}

\emph{Follow The New York Times Opinion section on}
\href{https://www.facebook.com/nytopinion}{\emph{Facebook}}\emph{,}
\href{http://twitter.com/NYTOpinion}{\emph{Twitter (@NYTopinion)}}
\emph{and}
\href{https://www.instagram.com/nytopinion/}{\emph{Instagram}}\emph{.}

\emph{The Times is committed to publishing}
\href{https://www.nytimes.com/2019/01/31/opinion/letters/letters-to-editor-new-york-times-women.html}{\emph{a
diversity of letters}} \emph{to the editor. We'd like to hear what you
think about this or any of our articles. Here are some}
\href{https://help.nytimes.com/hc/en-us/articles/115014925288-How-to-submit-a-letter-to-the-editor}{\emph{tips}}\emph{.
And here's our email:}
\href{mailto:letters@nytimes.com}{\emph{letters@nytimes.com}}\emph{.}

\emph{Follow The New York Times Opinion section on}
\href{https://www.facebook.com/nytopinion}{\emph{Facebook}}\emph{,}
\href{http://twitter.com/NYTOpinion}{\emph{Twitter (@NYTopinion)}}
\emph{and}
\href{https://www.instagram.com/nytopinion/}{\emph{Instagram}}\emph{.}

Advertisement

\protect\hyperlink{after-bottom}{Continue reading the main story}

\hypertarget{site-index}{%
\subsection{Site Index}\label{site-index}}

\hypertarget{site-information-navigation}{%
\subsection{Site Information
Navigation}\label{site-information-navigation}}

\begin{itemize}
\tightlist
\item
  \href{https://help.nytimes.com/hc/en-us/articles/115014792127-Copyright-notice}{©~2020~The
  New York Times Company}
\end{itemize}

\begin{itemize}
\tightlist
\item
  \href{https://www.nytco.com/}{NYTCo}
\item
  \href{https://help.nytimes.com/hc/en-us/articles/115015385887-Contact-Us}{Contact
  Us}
\item
  \href{https://www.nytco.com/careers/}{Work with us}
\item
  \href{https://nytmediakit.com/}{Advertise}
\item
  \href{http://www.tbrandstudio.com/}{T Brand Studio}
\item
  \href{https://www.nytimes.com/privacy/cookie-policy\#how-do-i-manage-trackers}{Your
  Ad Choices}
\item
  \href{https://www.nytimes.com/privacy}{Privacy}
\item
  \href{https://help.nytimes.com/hc/en-us/articles/115014893428-Terms-of-service}{Terms
  of Service}
\item
  \href{https://help.nytimes.com/hc/en-us/articles/115014893968-Terms-of-sale}{Terms
  of Sale}
\item
  \href{https://spiderbites.nytimes.com}{Site Map}
\item
  \href{https://help.nytimes.com/hc/en-us}{Help}
\item
  \href{https://www.nytimes.com/subscription?campaignId=37WXW}{Subscriptions}
\end{itemize}
