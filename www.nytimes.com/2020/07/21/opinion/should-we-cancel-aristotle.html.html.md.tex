Sections

SEARCH

\protect\hyperlink{site-content}{Skip to
content}\protect\hyperlink{site-index}{Skip to site index}

\href{https://myaccount.nytimes.com/auth/login?response_type=cookie\&client_id=vi}{}

\href{https://www.nytimes.com/section/todayspaper}{Today's Paper}

\href{/section/opinion}{Opinion}\textbar{}Should We Cancel Aristotle?

\href{https://nyti.ms/3fOSIuP}{https://nyti.ms/3fOSIuP}

\begin{itemize}
\item
\item
\item
\item
\item
\end{itemize}

Advertisement

\protect\hyperlink{after-top}{Continue reading the main story}

\href{/section/opinion}{Opinion}

Supported by

\protect\hyperlink{after-sponsor}{Continue reading the main story}

THE STONE

\hypertarget{should-we-cancel-aristotle}{%
\section{Should We Cancel Aristotle?}\label{should-we-cancel-aristotle}}

He defended slavery and opposed the notion of human equality. But he is
not our enemy.

By Agnes Callard

Ms. Callard is a philosopher and professor.

\begin{itemize}
\item
  July 21, 2020
\item
  \begin{itemize}
  \item
  \item
  \item
  \item
  \item
  \end{itemize}
\end{itemize}

\includegraphics{https://static01.nyt.com/images/2020/07/21/opinion/21stone-aristotle/21stone-aristotle-articleLarge.png?quality=75\&auto=webp\&disable=upscale}

The Greek philosopher Aristotle did not merely condone slavery, he
defended it; he did not merely defend it, but defended it as beneficial
to the slave. His view was that some people are, by nature, unable to
pursue their own good, and best suited to be ``living tools'' for use by
other people: ``The slave is a part of the master, a living but
separated part of his bodily frame.''

Aristotle's anti-liberalism does not stop there. He believed that women
were incapable of authoritative decision making. And he decreed that
manual laborers, despite being neither slaves nor women, were
nonetheless prohibited from citizenship or education in his ideal city.

Of course Aristotle is not alone: Kant and Hume made racist comments,
\href{https://plato.stanford.edu/entries/frege/}{Frege} made
anti-Semitic ones, and Wittgenstein was bracingly upfront about his
sexism. Should readers set aside or ignore such remarks, focusing
attention on valuable ideas to be found elsewhere in their work?

This pick-and-choose strategy may work in the case of Kant, Hume, Frege
and Wittgenstein, on the grounds that their core philosophical
contributions are unrelated to their prejudices, but I do not think it
applies so well to Aristotle: His inegalitarianism runs deep.

Aristotle thought that the value or worth of a human being --- his
virtue --- was something that he acquired in growing up. It follows that
people who can't (women, slaves) or simply don't (manual laborers)
acquire that virtue have no grounds for demanding equal respect or
recognition with those who do.

As I read him, Aristotle not only did not believe in the conception of
intrinsic human dignity that grounds our modern commitment to human
rights, he has a philosophy that cannot be squared with it. Aristotle's
inegalitarianism is less like Kant and Hume's racism and more like
Descartes's views on nonhuman animals: The fact that Descartes
characterizes nonhuman animals as soulless automata is a direct
consequence of his rationalist dualism. His comments on animals cannot
be treated as ``stray remarks.''

If cancellation is removal from a position of prominence on the basis of
an ideological crime, it might appear that there is a case to be made
for canceling Aristotle. He has much prominence: Thousands of years
after his death, his ethical works continue to be taught as part of the
basic philosophy curriculum offered in colleges and universities around
the world.

And Aristotle's mistake was serious enough that he comes off badly even
when compared to the various ``bad guys'' of history who sought to
justify the exclusion of certain groups --- women, Black people, Jews,
gays, atheists --- from the sheltering umbrella of human dignity.
Because Aristotle went so far as to think there was no umbrella.

Yet I would defend Aristotle, and his place on philosophy syllabuses, by
pointing to the benefits of engaging with him. He can help us identify
the grounds of our own egalitarian commitments; and his ethical system
may capture truths --- for instance, about the importance of aiming for
extraordinary excellence --- that we have yet to incorporate into our
own.

\includegraphics{https://static01.nyt.com/images/2019/11/19/autossell/cancelling-thumb_01/cancelling-thumb_01-videoSixteenByNine3000.png}

And I want to go a step further, and make an even stronger claim on
behalf of Aristotle. It is not only that the benefits of reading
Aristotle counteract the costs, but that there are no costs. In fact we
have no reason at all to cancel Aristotle. Aristotle is simply not our
enemy.

I, like Aristotle, am a philosopher, and we philosophers must
countenance the possibility of radical disagreement on the most
fundamental questions. Philosophers hold up as an ideal the aim of never
treating our interlocutor as a hostile combatant. But if someone puts
forward views that directly contradict your moral sensibilities, how can
you avoid hostility? The answer is to take him literally --- which is to
say, read his words purely as vehicles for the contents of his beliefs.

There is a kind of speech that it would be a mistake to take literally,
because its function is some kind of messaging. Advertising and
political oratory are examples of messaging, as is much that falls under
the rubric of ``making a statement,'' like boycotting, protesting or
publicly apologizing.

Such words exist to perform some extra-communicative task; in messaging
speech, some aim other than truth-seeking is always at play. One way to
turn literal speech into messaging is to attach a list of names:
\href{https://www.nytimes.com/2019/08/13/opinion/philosophers-petitions.html}{a
petition is an example of nonliteral speech}, because more people
believing something does not make it more true.

Whereas literal speech employs systematically truth-directed methods of
persuasion --- argument and evidence --- messaging exerts some kind of
nonrational pressure on its recipient. For example, a public apology can
often exert social pressure on the injured party to forgive, or at any
rate to perform a show of forgiveness. Messaging is often situated
within some kind of power struggle. In a highly charged political
climate, more and more speech becomes magnetically attracted into
messaging; one can hardly say anything without arousing suspicion that
one is making a move in the game, one that might call for a countermove.

For example, the words ``Black lives matter'' and ``All lives matter''
have been implicated in our political power struggle in such a way as to
prevent anyone familiar with that struggle from using, or hearing, them
literally. But if an alien from outer space, unfamiliar with this
context, came to us and said either phrase, it would be hard to imagine
that anyone would find it objectionable; the context in which we now use
those phrases would be removed.

In fact, I can imagine circumstances under which an alien could say
women are inferior to men without arousing offense in me. Suppose this
alien had no gender on their planet, and drew the conclusion of female
inferiority from time spent observing ours. As long as the alien spoke
to me respectfully, I would not only be willing to hear them out but
even interested to learn their argument.

I read Aristotle as such an ``alien.'' His approach to ethics was
empirical --- that is, it was based on observation --- and when he
looked around him he saw a world of slavery and of the subjugation of
women and manual laborers, a situation he then inscribed into his
ethical theory.

When I read him, I see that view of the world --- and that's all. I do
not read an evil intent or ulterior motive behind his words; I do not
interpret them as a mark of his bad character, or as attempting to
convey a dangerous message that I might need to combat or silence in
order to protect the vulnerable. Of course in one sense it is hard to
imagine a more dangerous idea than the one that he articulated and
argued for --- but dangerousness, I have been arguing, is less a matter
of literal content than messaging context.

What makes speech truly free is the possibility of disagreement without
enmity, and this is less a matter of what we can say, than how we can
say it. ``Cancel culture'' is merely the logical extension of what we
might call ``messaging culture,'' in which every speech act is
classified as friend or foe, in which literal content can barely be
communicated, and in which very little faith exists as to the rational
faculties of those being spoken to. In such a context, even the cry for
``free speech'' invites a nonliteral interpretation, as being nothing
but the most efficient way for its advocates to acquire or consolidate
power.

I will admit that Aristotle's vast temporal distance from us makes it
artificially easy to treat him as an ``alien.'' One of the reasons I
gravitate to the study of ancient ethics is precisely that it is
difficult to entangle those authors in contemporary power struggles.
When we turn to disagreement on highly charged contemporary ethical
questions, such as debates about gender identity, we find suspicion,
second-guessing of motives, petitioning --- the hallmarks of messaging
culture --- even among philosophers.

I do not claim that the possibility of friendly disagreement with
Aristotle offers any direct guidance on how to improve our much more
difficult disagreements with our contemporaries, but I do think
considering the case of Aristotle reveals something about what the
target of such improvements would be. What we want, when we want free
speech, is the freedom to speak literally.

Agnes Callard (@AgnesCallard), an associate professor of philosophy at
the University of Chicago and the author of ``Aspiration: The Agency of
Becoming,'' writes about public philosophy at The Point magazine.

\emph{\textbf{Now in print}}*:
``\emph{\href{http://bitly.com/1MW2kN3}{\emph{Modern Ethics in 77
Arguments}}},'' and ``\emph{\href{http://bitly.com/1MW2kN3}{\emph{The
Stone Reader: Modern Philosophy in 133 Arguments}}},'' with essays from
the series, edited by Peter Catapano and Simon Critchley, published by
Liveright Books.*

\emph{The Times is committed to publishing}
\href{https://www.nytimes.com/2019/01/31/opinion/letters/letters-to-editor-new-york-times-women.html}{\emph{a
diversity of letters}} \emph{to the editor. We'd like to hear what you
think about this or any of our articles. Here are some}
\href{https://help.nytimes.com/hc/en-us/articles/115014925288-How-to-submit-a-letter-to-the-editor}{\emph{tips}}\emph{.
And here's our email:}
\href{mailto:letters@nytimes.com}{\emph{letters@nytimes.com}}\emph{.}

\emph{Follow The New York Times Opinion section on}
\href{https://www.facebook.com/nytopinion}{\emph{Facebook}}\emph{,}
\href{http://twitter.com/NYTOpinion}{\emph{Twitter (@NYTopinion)}}
\emph{and}
\href{https://www.instagram.com/nytopinion/}{\emph{Instagram}}\emph{.}

Advertisement

\protect\hyperlink{after-bottom}{Continue reading the main story}

\hypertarget{site-index}{%
\subsection{Site Index}\label{site-index}}

\hypertarget{site-information-navigation}{%
\subsection{Site Information
Navigation}\label{site-information-navigation}}

\begin{itemize}
\tightlist
\item
  \href{https://help.nytimes.com/hc/en-us/articles/115014792127-Copyright-notice}{©~2020~The
  New York Times Company}
\end{itemize}

\begin{itemize}
\tightlist
\item
  \href{https://www.nytco.com/}{NYTCo}
\item
  \href{https://help.nytimes.com/hc/en-us/articles/115015385887-Contact-Us}{Contact
  Us}
\item
  \href{https://www.nytco.com/careers/}{Work with us}
\item
  \href{https://nytmediakit.com/}{Advertise}
\item
  \href{http://www.tbrandstudio.com/}{T Brand Studio}
\item
  \href{https://www.nytimes.com/privacy/cookie-policy\#how-do-i-manage-trackers}{Your
  Ad Choices}
\item
  \href{https://www.nytimes.com/privacy}{Privacy}
\item
  \href{https://help.nytimes.com/hc/en-us/articles/115014893428-Terms-of-service}{Terms
  of Service}
\item
  \href{https://help.nytimes.com/hc/en-us/articles/115014893968-Terms-of-sale}{Terms
  of Sale}
\item
  \href{https://spiderbites.nytimes.com}{Site Map}
\item
  \href{https://help.nytimes.com/hc/en-us}{Help}
\item
  \href{https://www.nytimes.com/subscription?campaignId=37WXW}{Subscriptions}
\end{itemize}
