Sections

SEARCH

\protect\hyperlink{site-content}{Skip to
content}\protect\hyperlink{site-index}{Skip to site index}

\href{https://www.nytimes.com/section/technology}{Technology}

\href{https://myaccount.nytimes.com/auth/login?response_type=cookie\&client_id=vi}{}

\href{https://www.nytimes.com/section/todayspaper}{Today's Paper}

\href{/section/technology}{Technology}\textbar{}Twitter Takedown Targets
QAnon Accounts

\url{https://nyti.ms/3eQ12t1}

\begin{itemize}
\item
\item
\item
\item
\item
\end{itemize}

Advertisement

\protect\hyperlink{after-top}{Continue reading the main story}

Supported by

\protect\hyperlink{after-sponsor}{Continue reading the main story}

\hypertarget{twitter-takedown-targets-qanon-accounts}{%
\section{Twitter Takedown Targets QAnon
Accounts}\label{twitter-takedown-targets-qanon-accounts}}

The company permanently suspended thousands of accounts associated with
the fringe conspiracy theory movement. Facebook was said to be preparing
to take similar action.

\includegraphics{https://static01.nyt.com/images/2020/07/21/business/21twitter1/21twitter1-articleLarge.jpg?quality=75\&auto=webp\&disable=upscale}

By \href{https://www.nytimes.com/by/kate-conger}{Kate Conger}

\begin{itemize}
\item
  Published July 21, 2020Updated July 24, 2020
\item
  \begin{itemize}
  \item
  \item
  \item
  \item
  \item
  \end{itemize}
\end{itemize}

OAKLAND, Calif. --- Twitter said Tuesday evening that it had removed
thousands of accounts that spread messages about the conspiracy theories
known as QAnon, saying their messages could lead to harm and violated
Twitter policy.

Twitter said it would also block trends related to the loose network of
QAnon conspiracy theories from appearing in its trending topics and
search, and would not allow users to post links affiliated with the
theories on its platform.

It was the first time that a social media service took sweeping action
to remove content affiliated with QAnon, which has become increasingly
popular on Twitter, Facebook and YouTube.

\begin{quote}
We've been clear that we will take strong enforcement action on behavior
that has the potential to lead to offline harm. In line with this
approach, this week we are taking further action on so-called `QAnon'
activity across the service.

--- Twitter Safety (@TwitterSafety)
\href{https://twitter.com/TwitterSafety/status/1285726277719199746?ref_src=twsrc\%5Etfw}{July
22, 2020}
\end{quote}

Facebook is preparing to take similar steps to limit the reach of QAnon
content on its platform, said two Facebook employees with knowledge of
the plans, who spoke on the condition of anonymity. The company has been
coordinating with Twitter and other social media companies and plans to
make and announcement next month, the employees said. Facebook declined
to comment.

The QAnon theories stem from an anonymous person or group of people who
use the name ``Q'' and claim to have access to government secrets that
reveal a plot against President Trump and his supporters. That
supposedly classified information was initially posted on message boards
before spreading to mainstream internet platforms and has led to
significant online harassment as well as physical violence.

``QAnon is not conventional political discourse,'' Alice Marwick, an
associate professor of communication at the University of North Carolina
at Chapel Hill. ``It's a conspiracy theory that makes wild claims and
baseless accusations about political actors and innocent people alike.''

Over several weeks, Twitter has removed 7,000 accounts that posted QAnon
material, a company spokeswoman said. The accounts had been increasingly
active, and had been involved in coordinated harassment campaigns on
Twitter or tried to evade a previous suspension by setting up new
accounts after an old account was deleted.

An additional 150,000 accounts will be hidden from trends and search on
Twitter, the spokeswoman added. The takedowns were
\href{https://www.nbcnews.com/tech/tech-news/twitter-bans-7-000-qanon-accounts-limits-150-000-others-n1234541}{reported
earlier by NBC News.}

``These accounts amplify and enable networked harassment on a level
that's clearly against the Twitter terms of service,'' Ms. Marwick said.
``But this won't stop QAnon from operating. It's multiplatform and
really good at adapting as media ecosystems change.''

In May,
\href{https://about.fb.com/news/2020/05/april-cib-report/}{Facebook
removed} a cluster of five pages, 20 Facebook accounts and six groups
affiliated with QAnon, saying they had violated its policy against
coordinated inauthentic behavior. In 2018, Reddit
\href{https://www.washingtonpost.com/news/the-intersect/wp/2018/09/12/reddit-bans-r-greatawakening-the-main-subreddit-for-qanon-conspiracy-theorists/}{banned
a handful of groups} focused on QAnon.

After years of taking a hands-off approach to content moderation,
Twitter has acted more aggressively in recent months to stem the flood
of abuse and harassment on its platform.

Since it became a venue for disinformation during the 2016 U.S.
presidential election, Twitter has cracked down on content that spreads
false information or encourages harassment. In February, it introduced a
\href{https://www.nytimes.com/2020/02/04/technology/twitter-fake-videos-photos-disinformation.html}{ban
against manipulated photos and videos}, a popular method of tricking
viewers and spreading disinformation. And in May, it began labeling some
of Mr. Trump's tweets, saying they contained false information or
promoted violence.

Twitter's aggressive enforcement actions have put it on
\href{https://www.nytimes.com/2020/05/30/technology/twitter-trump-dorsey.html}{a
collision course with Mr. Trump}, who has said that Twitter is unfairly
silencing conservative voices and has
\href{https://www.nytimes.com/2020/05/28/us/politics/trump-order-social-media.html}{encouraged
regulators to crack down} on the service. While the QAnon ban was
applauded in many circles, some conservatives said Twitter's move was
further evidence that the company unevenly enforced its rules against
Mr. Trump's supporters.

The political attention has added to Twitter's headaches. A
\href{https://www.nytimes.com/2020/07/17/technology/twitter-hackers-interview.html}{wide-ranging
hack} last week compromised the Twitter accounts of Democratic political
figures, including former Vice President Joseph R. Biden Jr. and former
President Barack Obama. Twitter also faces concerns that advertisers are
tightening spending during the coronavirus pandemic, and is expected to
report its second-quarter earnings this week.

More than two years after QAnon emerged from
\href{https://www.nytimes.com/2018/08/01/us/politics/what-is-qanon.html}{the
troll-infested corners of the internet}, supporters of the movement,
which the F.B.I. has labeled a potential domestic terrorism threat, are
\href{https://www.nytimes.com/2020/07/14/us/politics/qanon-politicians-candidates.html}{trickling
into the mainstream} of the Republican Party. Precisely how many
candidates, mostly Republicans, are running under the QAnon banner is
unclear. Some estimates put the number at a dozen, and few are expected
to win in November.

A number of the candidates have sought to spread a core tenet of the
QAnon conspiracy: that Mr. Trump ran for office to save Americans from a
so-called deep state filled with child-abusing, devil-worshiping
bureaucrats. According to QAnon, backing the president's enemies are
prominent Democrats who, in some telling, extract hormones from
children's blood.

The president has
\href{https://www.politico.com/news/2020/07/12/trump-tweeting-qanon-followers-357238}{repeatedly
retweeted QAnon supporters} and cheered on candidates who openly support
the conspiracy theory, like Marjorie Taylor Greene, a Republican House
candidate in Georgia.

``A big winner. Congratulations!''
\href{https://twitter.com/realDonaldTrump/status/1271428819296157697?s=20}{Mr.
Trump tweeted} after Ms. Greene, whose ads have been banned by Facebook
for violating its terms of service, placed first in her primary.

Some QAnon followers have diverted their attention from political
causes. Recent QAnon campaigns on Twitter have focused on
\href{https://www.nytimes.com/aponline/2020/07/16/business/ap-us-wayfair-conspiracy-theory.html}{Wayfair},
a furniture and décor company, and Chrissy Teigen, a model and cookbook
author who recently said she had blocked
\href{https://www.elle.com/uk/life-and-culture/culture/a33333809/chrissy-teigen-conspiracy-theorist-drama-jeffrey-epstein/}{one
million accounts} affiliated with QAnon. She called on Twitter to take
action after she became a target of harassment.

QAnon theories share similar themes with
\href{https://www.nytimes.com/2016/11/21/technology/fact-check-this-pizzeria-is-not-a-child-trafficking-site.html?searchResultPosition=11}{Pizzagate,
a conspiracy theory} popularized ahead of the 2016 presidential election
that advanced the baseless notion that the Democratic nominee, Hillary
Clinton, and party elites were running a child sex-trafficking ring out
of a Washington pizzeria. In December 2016, a
\href{https://www.nytimes.com/2016/12/05/business/media/comet-ping-pong-pizza-shooting-fake-news-consequences.html?searchResultPosition=12}{vigilante
gunman showed up} at the restaurant with an assault rifle and opened
fire into a closet, and social media companies fear that they could be
linked to similar incidents if they allow conspiracy theories to thrive
on their platforms.

Facebook, Twitter and YouTube managed to largely suppress that Pizzagate
conspiracy theory, but as the presidential election nears it has
appeared to rebound on those platforms and newer ones,
\href{https://www.nytimes.com/2020/06/29/technology/pizzagate-tiktok.html}{like
TikTok}.

Reporting was contributed by Sheera Frenkel, Matthew Rosenberg, Jennifer
Steinhauer and Kevin Roose.

Advertisement

\protect\hyperlink{after-bottom}{Continue reading the main story}

\hypertarget{site-index}{%
\subsection{Site Index}\label{site-index}}

\hypertarget{site-information-navigation}{%
\subsection{Site Information
Navigation}\label{site-information-navigation}}

\begin{itemize}
\tightlist
\item
  \href{https://help.nytimes.com/hc/en-us/articles/115014792127-Copyright-notice}{©~2020~The
  New York Times Company}
\end{itemize}

\begin{itemize}
\tightlist
\item
  \href{https://www.nytco.com/}{NYTCo}
\item
  \href{https://help.nytimes.com/hc/en-us/articles/115015385887-Contact-Us}{Contact
  Us}
\item
  \href{https://www.nytco.com/careers/}{Work with us}
\item
  \href{https://nytmediakit.com/}{Advertise}
\item
  \href{http://www.tbrandstudio.com/}{T Brand Studio}
\item
  \href{https://www.nytimes.com/privacy/cookie-policy\#how-do-i-manage-trackers}{Your
  Ad Choices}
\item
  \href{https://www.nytimes.com/privacy}{Privacy}
\item
  \href{https://help.nytimes.com/hc/en-us/articles/115014893428-Terms-of-service}{Terms
  of Service}
\item
  \href{https://help.nytimes.com/hc/en-us/articles/115014893968-Terms-of-sale}{Terms
  of Sale}
\item
  \href{https://spiderbites.nytimes.com}{Site Map}
\item
  \href{https://help.nytimes.com/hc/en-us}{Help}
\item
  \href{https://www.nytimes.com/subscription?campaignId=37WXW}{Subscriptions}
\end{itemize}
