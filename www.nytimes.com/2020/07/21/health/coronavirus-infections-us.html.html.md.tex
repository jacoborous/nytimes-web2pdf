Sections

SEARCH

\protect\hyperlink{site-content}{Skip to
content}\protect\hyperlink{site-index}{Skip to site index}

\href{https://www.nytimes.com/section/health}{Health}

\href{https://myaccount.nytimes.com/auth/login?response_type=cookie\&client_id=vi}{}

\href{https://www.nytimes.com/section/todayspaper}{Today's Paper}

\href{/section/health}{Health}\textbar{}Coronavirus Infections Much
Higher Than Reported Cases in Parts of U.S., Study Shows

\url{https://nyti.ms/3hl5D84}

\begin{itemize}
\item
\item
\item
\item
\item
\end{itemize}

\href{https://www.nytimes.com/news-event/coronavirus?action=click\&pgtype=Article\&state=default\&region=TOP_BANNER\&context=storylines_menu}{The
Coronavirus Outbreak}

\begin{itemize}
\tightlist
\item
  live\href{https://www.nytimes.com/2020/08/04/world/coronavirus-cases.html?action=click\&pgtype=Article\&state=default\&region=TOP_BANNER\&context=storylines_menu}{Latest
  Updates}
\item
  \href{https://www.nytimes.com/interactive/2020/us/coronavirus-us-cases.html?action=click\&pgtype=Article\&state=default\&region=TOP_BANNER\&context=storylines_menu}{Maps
  and Cases}
\item
  \href{https://www.nytimes.com/interactive/2020/science/coronavirus-vaccine-tracker.html?action=click\&pgtype=Article\&state=default\&region=TOP_BANNER\&context=storylines_menu}{Vaccine
  Tracker}
\item
  \href{https://www.nytimes.com/2020/08/02/us/covid-college-reopening.html?action=click\&pgtype=Article\&state=default\&region=TOP_BANNER\&context=storylines_menu}{College
  Reopening}
\item
  \href{https://www.nytimes.com/live/2020/08/04/business/stock-market-today-coronavirus?action=click\&pgtype=Article\&state=default\&region=TOP_BANNER\&context=storylines_menu}{Economy}
\end{itemize}

Advertisement

\protect\hyperlink{after-top}{Continue reading the main story}

Supported by

\protect\hyperlink{after-sponsor}{Continue reading the main story}

\hypertarget{coronavirus-infections-much-higher-than-reported-cases-in-parts-of-us-study-shows}{%
\section{Coronavirus Infections Much Higher Than Reported Cases in Parts
of U.S., Study
Shows}\label{coronavirus-infections-much-higher-than-reported-cases-in-parts-of-us-study-shows}}

Data from antibody tests in 10 different cities and states indicate that
many people with no symptoms may be spreading the virus.

\includegraphics{https://static01.nyt.com/images/2020/07/21/science/21VIRUS-CDC/merlin_174720729_68d62221-a7a9-4b96-86af-d24f5983afc3-articleLarge.jpg?quality=75\&auto=webp\&disable=upscale}

By \href{https://www.nytimes.com/by/apoorva-mandavilli}{Apoorva
Mandavilli}

\begin{itemize}
\item
  July 21, 2020
\item
  \begin{itemize}
  \item
  \item
  \item
  \item
  \item
  \end{itemize}
\end{itemize}

The number of people infected with the coronavirus in different parts of
the United States was anywhere from two to 13 times higher than the
reported rates for those regions, according to
\href{https://www.cdc.gov/coronavirus/2019-ncov/cases-updates/commercial-lab-surveys.html}{data
released Tuesday} by the
\href{https://www.nytimes.com/2020/07/24/health/cdc-schools-coronavirus.html}{Centers
for Disease Control and Prevention}.

\href{https://www.cdc.gov/coronavirus/2019-ncov/cases-updates/commercial-lab-surveys.html}{The
findings} suggest that large numbers of people who did not have symptoms
or did not seek medical care may have kept the virus circulating in
their communities.

The study indicates that even the hardest-hit area in the study --- New
York City, where nearly one in four people has been exposed to the virus
--- is nowhere near achieving herd immunity, the level of exposure at
which the virus would stop spreading in a particular city or region.
Experts believe 60 percent of people in an area would need to have been
exposed to the coronavirus to reach herd immunity.

The analysis, based on antibody tests, is the largest of its kind to
date; a study of a subset of cities and states was released last month.

``These data continue to show that the number of people who have been
infected with the virus that causes Covid-19 far exceeds the number of
reported cases,'' said Dr. Fiona Havers, the C.D.C. researcher who led
the study. ``Many of these people likely had no symptoms or mild illness
and may have had no idea that they were infected.''

About 40 percent of infected people do not develop symptoms, but they
may still pass the virus on to others. The United States now tests
roughly 700,000 people a day. The new results highlight the need for
much more testing to detect infection levels and contain the viral
spread in various parts of the country.

For example, in Missouri, the prevalence of infections is 13 times the
reported rate, suggesting that the state missed most people with the
virus who may have contributed to its outsized outbreak.

\hypertarget{latest-updates-global-coronavirus-outbreak}{%
\section{\texorpdfstring{\href{https://www.nytimes.com/2020/08/04/world/coronavirus-cases.html?action=click\&pgtype=Article\&state=default\&region=MAIN_CONTENT_1\&context=storylines_live_updates}{Latest
Updates: Global Coronavirus
Outbreak}}{Latest Updates: Global Coronavirus Outbreak}}\label{latest-updates-global-coronavirus-outbreak}}

Updated 2020-08-04T20:57:54.346Z

\begin{itemize}
\tightlist
\item
  \href{https://www.nytimes.com/2020/08/04/world/coronavirus-cases.html?action=click\&pgtype=Article\&state=default\&region=MAIN_CONTENT_1\&context=storylines_live_updates\#link-1228a480}{Novavax
  sees encouraging results from two studies of its experimental
  vaccine.}
\item
  \href{https://www.nytimes.com/2020/08/04/world/coronavirus-cases.html?action=click\&pgtype=Article\&state=default\&region=MAIN_CONTENT_1\&context=storylines_live_updates\#link-4825b93}{Public
  and private schools in Maryland and elsewhere are divided over
  in-person instruction.}
\item
  \href{https://www.nytimes.com/2020/08/04/world/coronavirus-cases.html?action=click\&pgtype=Article\&state=default\&region=MAIN_CONTENT_1\&context=storylines_live_updates\#link-50f7386d}{The
  United Nations calls on policymakers to `plan thoroughly for school
  reopenings.'}
\end{itemize}

\href{https://www.nytimes.com/2020/08/04/world/coronavirus-cases.html?action=click\&pgtype=Article\&state=default\&region=MAIN_CONTENT_1\&context=storylines_live_updates}{See
more updates}

More live coverage:
\href{https://www.nytimes.com/live/2020/08/04/business/stock-market-today-coronavirus?action=click\&pgtype=Article\&state=default\&region=MAIN_CONTENT_1\&context=storylines_live_updates}{Markets}

Dr. Havers emphasized that even those who do not know their infection
status should wear cloth face coverings, practice social distancing and
wash their hands frequently.

The researchers analyzed blood samples from people who had routine
clinical tests or were hospitalized to determine if they had antibodies
to the coronavirus --- evidence of prior infection. They had released
\href{https://www.nytimes.com/2020/06/27/health/coronavirus-antibodies-asymptomatic.html}{early
data from six cities and states} in June. The
\href{https://jamanetwork.com/journals/jamainternalmedicine/fullarticle/2768834?guestAccessKey=7a5c32e6-3c27-41b3-b46c-43c4a38bbe00\&utm_source=For_The_Media\&utm_medium=referral\&utm_campaign=ftm_links\&utm_content=tfl\&utm_term=072120}{study
published in the JAMA Internal Medicine journal} on Tuesday expands that
research by including four more regions. They also posted data from
later time periods for eight of those 10 sites to the C.D.C.'s website
on Tuesday.

The results indicate that in vast swaths of the country, the coronavirus
still has touched only a small fraction of the population. In Utah, for
example, just over one percent of people had been exposed to the virus
by early June. The rate was 2.2 percent for Minneapolis-St. Paul as of
the first week of June, 3.6 percent for the Philadelphia metropolitan
region as of May 30 and 1 percent for the San Francisco Bay Area as of
April 30.

In some regions, the gap between estimated infections and reported cases
decreased as testing capacity and reporting improved. New York City, for
example, showed a 12-fold difference between actual infections and the
reported rate in early April, and a 10-fold difference in early May.

``This is not coming as a shock or surprise to epidemiologists,'' Carl
Bergstrom, an infectious diseases expert at the University of Washington
in Seattle, said in an email. ``All along, we have expected that only
about 10 percent of the cases will be reported.''

Tracking the numbers over time can provide useful insights into the
virus' spread and about a region's capacity to cope with the epidemic,
other experts said.

``The fact that they're sort of marking it out over time and looking at
it over a longer duration will actually be super-informative,'' said Dr.
Rochelle Walensky, a researcher at Harvard University who wrote an
editorial accompanying the JAMA paper.

For example, South Florida ticked up to 2.9 percent as of April 24 from
1.9 percent just two weeks earlier. Missouri's numbers barely budged
from 2.7 percent as of April 26 to 2.8 percent as of May 30. Numbers for
both regions are likely to be much higher in the next round of analyses
because of the surge of infections in those regions since those dates.

New York City showed the biggest leap in its rate, from 6.9 percent as
of April 1 to 23.3 percent as of May 6, consistent with its outbreak.

The city's estimate closely matches the 22.7 percent prevalence found by
a
\href{https://www.sciencedirect.com/science/article/pii/S1047279720302015}{state
survey}, which tested patrons in supermarkets from April 19-28.

\href{https://www.nytimes.com/news-event/coronavirus?action=click\&pgtype=Article\&state=default\&region=MAIN_CONTENT_3\&context=storylines_faq}{}

\hypertarget{the-coronavirus-outbreak-}{%
\subsubsection{The Coronavirus Outbreak
›}\label{the-coronavirus-outbreak-}}

\hypertarget{frequently-asked-questions}{%
\paragraph{Frequently Asked
Questions}\label{frequently-asked-questions}}

Updated August 4, 2020

\begin{itemize}
\item ~
  \hypertarget{i-have-antibodies-am-i-now-immune}{%
  \paragraph{I have antibodies. Am I now
  immune?}\label{i-have-antibodies-am-i-now-immune}}

  \begin{itemize}
  \tightlist
  \item
    As of right
    now,\href{https://www.nytimes.com/2020/07/22/health/covid-antibodies-herd-immunity.html?action=click\&pgtype=Article\&state=default\&region=MAIN_CONTENT_3\&context=storylines_faq}{that
    seems likely, for at least several months.} There have been
    frightening accounts of people suffering what seems to be a second
    bout of Covid-19. But experts say these patients may have a
    drawn-out course of infection, with the virus taking a slow toll
    weeks to months after initial exposure. People infected with the
    coronavirus typically
    \href{https://www.nature.com/articles/s41586-020-2456-9}{produce}
    immune molecules called antibodies, which are
    \href{https://www.nytimes.com/2020/05/07/health/coronavirus-antibody-prevalence.html?action=click\&pgtype=Article\&state=default\&region=MAIN_CONTENT_3\&context=storylines_faq}{protective
    proteins made in response to an
    infection}\href{https://www.nytimes.com/2020/05/07/health/coronavirus-antibody-prevalence.html?action=click\&pgtype=Article\&state=default\&region=MAIN_CONTENT_3\&context=storylines_faq}{.
    These antibodies may} last in the body
    \href{https://www.nature.com/articles/s41591-020-0965-6}{only two to
    three months}, which may seem worrisome, but that's perfectly normal
    after an acute infection subsides, said Dr. Michael Mina, an
    immunologist at Harvard University. It may be possible to get the
    coronavirus again, but it's highly unlikely that it would be
    possible in a short window of time from initial infection or make
    people sicker the second time.
  \end{itemize}
\item ~
  \hypertarget{im-a-small-business-owner-can-i-get-relief}{%
  \paragraph{I'm a small-business owner. Can I get
  relief?}\label{im-a-small-business-owner-can-i-get-relief}}

  \begin{itemize}
  \tightlist
  \item
    The
    \href{https://www.nytimes.com/article/small-business-loans-stimulus-grants-freelancers-coronavirus.html?action=click\&pgtype=Article\&state=default\&region=MAIN_CONTENT_3\&context=storylines_faq}{stimulus
    bills enacted in March} offer help for the millions of American
    small businesses. Those eligible for aid are businesses and
    nonprofit organizations with fewer than 500 workers, including sole
    proprietorships, independent contractors and freelancers. Some
    larger companies in some industries are also eligible. The help
    being offered, which is being managed by the Small Business
    Administration, includes the Paycheck Protection Program and the
    Economic Injury Disaster Loan program. But lots of folks have
    \href{https://www.nytimes.com/interactive/2020/05/07/business/small-business-loans-coronavirus.html?action=click\&pgtype=Article\&state=default\&region=MAIN_CONTENT_3\&context=storylines_faq}{not
    yet seen payouts.} Even those who have received help are confused:
    The rules are draconian, and some are stuck sitting on
    \href{https://www.nytimes.com/2020/05/02/business/economy/loans-coronavirus-small-business.html?action=click\&pgtype=Article\&state=default\&region=MAIN_CONTENT_3\&context=storylines_faq}{money
    they don't know how to use.} Many small-business owners are getting
    less than they expected or
    \href{https://www.nytimes.com/2020/06/10/business/Small-business-loans-ppp.html?action=click\&pgtype=Article\&state=default\&region=MAIN_CONTENT_3\&context=storylines_faq}{not
    hearing anything at all.}
  \end{itemize}
\item ~
  \hypertarget{what-are-my-rights-if-i-am-worried-about-going-back-to-work}{%
  \paragraph{What are my rights if I am worried about going back to
  work?}\label{what-are-my-rights-if-i-am-worried-about-going-back-to-work}}

  \begin{itemize}
  \tightlist
  \item
    Employers have to provide
    \href{https://www.osha.gov/SLTC/covid-19/standards.html}{a safe
    workplace} with policies that protect everyone equally.
    \href{https://www.nytimes.com/article/coronavirus-money-unemployment.html?action=click\&pgtype=Article\&state=default\&region=MAIN_CONTENT_3\&context=storylines_faq}{And
    if one of your co-workers tests positive for the coronavirus, the
    C.D.C.} has said that
    \href{https://www.cdc.gov/coronavirus/2019-ncov/community/guidance-business-response.html}{employers
    should tell their employees} -\/- without giving you the sick
    employee's name -\/- that they may have been exposed to the virus.
  \end{itemize}
\item ~
  \hypertarget{should-i-refinance-my-mortgage}{%
  \paragraph{Should I refinance my
  mortgage?}\label{should-i-refinance-my-mortgage}}

  \begin{itemize}
  \tightlist
  \item
    \href{https://www.nytimes.com/article/coronavirus-money-unemployment.html?action=click\&pgtype=Article\&state=default\&region=MAIN_CONTENT_3\&context=storylines_faq}{It
    could be a good idea,} because mortgage rates have
    \href{https://www.nytimes.com/2020/07/16/business/mortgage-rates-below-3-percent.html?action=click\&pgtype=Article\&state=default\&region=MAIN_CONTENT_3\&context=storylines_faq}{never
    been lower.} Refinancing requests have pushed mortgage applications
    to some of the highest levels since 2008, so be prepared to get in
    line. But defaults are also up, so if you're thinking about buying a
    home, be aware that some lenders have tightened their standards.
  \end{itemize}
\item ~
  \hypertarget{what-is-school-going-to-look-like-in-september}{%
  \paragraph{What is school going to look like in
  September?}\label{what-is-school-going-to-look-like-in-september}}

  \begin{itemize}
  \tightlist
  \item
    It is unlikely that many schools will return to a normal schedule
    this fall, requiring the grind of
    \href{https://www.nytimes.com/2020/06/05/us/coronavirus-education-lost-learning.html?action=click\&pgtype=Article\&state=default\&region=MAIN_CONTENT_3\&context=storylines_faq}{online
    learning},
    \href{https://www.nytimes.com/2020/05/29/us/coronavirus-child-care-centers.html?action=click\&pgtype=Article\&state=default\&region=MAIN_CONTENT_3\&context=storylines_faq}{makeshift
    child care} and
    \href{https://www.nytimes.com/2020/06/03/business/economy/coronavirus-working-women.html?action=click\&pgtype=Article\&state=default\&region=MAIN_CONTENT_3\&context=storylines_faq}{stunted
    workdays} to continue. California's two largest public school
    districts --- Los Angeles and San Diego --- said on July 13, that
    \href{https://www.nytimes.com/2020/07/13/us/lausd-san-diego-school-reopening.html?action=click\&pgtype=Article\&state=default\&region=MAIN_CONTENT_3\&context=storylines_faq}{instruction
    will be remote-only in the fall}, citing concerns that surging
    coronavirus infections in their areas pose too dire a risk for
    students and teachers. Together, the two districts enroll some
    825,000 students. They are the largest in the country so far to
    abandon plans for even a partial physical return to classrooms when
    they reopen in August. For other districts, the solution won't be an
    all-or-nothing approach.
    \href{https://bioethics.jhu.edu/research-and-outreach/projects/eschool-initiative/school-policy-tracker/}{Many
    systems}, including the nation's largest, New York City, are
    devising
    \href{https://www.nytimes.com/2020/06/26/us/coronavirus-schools-reopen-fall.html?action=click\&pgtype=Article\&state=default\&region=MAIN_CONTENT_3\&context=storylines_faq}{hybrid
    plans} that involve spending some days in classrooms and other days
    online. There's no national policy on this yet, so check with your
    municipal school system regularly to see what is happening in your
    community.
  \end{itemize}
\end{itemize}

Some experts criticized the state survey at the time because people
shopping during the lockdown were more likely to be young, or might have
recovered from illness and felt safe.

``These consistent results offer mutual support to two very different
methods used,'' Eli Rosenberg, an epidemiologist at the State University
of New York at Albany and lead author of the state study.

The C.D.C. study also has limitations, Dr. Walensky said, because many
of the people who ventured out during the lockdowns for tests or were
hospitalized would have been severely ill, and might not have been
representative of the general population.

Each region also varied ``in terms of where they were on their own
epidemic curve and varied in terms of the amount of testing that they
did,'' she said.

The study also did not collect data on race, ethnicity, diagnostic and
symptom history or prevention behaviors, Dr. Rosenberg said. ``The
approach used in the grocery store study allows for these data
collections by pairing the specimen collection with a survey,'' he said.

Still, experts said the findings were valuable, despite limitations.

``This population may not be exactly representative of the population as
a whole, but the hope is that it is close enough to allow us to draw
meaningful conclusions,'' Dr. Bergstrom said.

Several recent studies have suggested that antibody levels, especially
in people with mild or no symptoms, may quickly decline. If that's true,
surveys like the C.D.C.'s might reflect only people who were infected
within the previous two to three months, Dr. Rosenberg said, ``and
complicate interpretation of results over time.''

Advertisement

\protect\hyperlink{after-bottom}{Continue reading the main story}

\hypertarget{site-index}{%
\subsection{Site Index}\label{site-index}}

\hypertarget{site-information-navigation}{%
\subsection{Site Information
Navigation}\label{site-information-navigation}}

\begin{itemize}
\tightlist
\item
  \href{https://help.nytimes.com/hc/en-us/articles/115014792127-Copyright-notice}{©~2020~The
  New York Times Company}
\end{itemize}

\begin{itemize}
\tightlist
\item
  \href{https://www.nytco.com/}{NYTCo}
\item
  \href{https://help.nytimes.com/hc/en-us/articles/115015385887-Contact-Us}{Contact
  Us}
\item
  \href{https://www.nytco.com/careers/}{Work with us}
\item
  \href{https://nytmediakit.com/}{Advertise}
\item
  \href{http://www.tbrandstudio.com/}{T Brand Studio}
\item
  \href{https://www.nytimes.com/privacy/cookie-policy\#how-do-i-manage-trackers}{Your
  Ad Choices}
\item
  \href{https://www.nytimes.com/privacy}{Privacy}
\item
  \href{https://help.nytimes.com/hc/en-us/articles/115014893428-Terms-of-service}{Terms
  of Service}
\item
  \href{https://help.nytimes.com/hc/en-us/articles/115014893968-Terms-of-sale}{Terms
  of Sale}
\item
  \href{https://spiderbites.nytimes.com}{Site Map}
\item
  \href{https://help.nytimes.com/hc/en-us}{Help}
\item
  \href{https://www.nytimes.com/subscription?campaignId=37WXW}{Subscriptions}
\end{itemize}
