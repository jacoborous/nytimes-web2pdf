Sections

SEARCH

\protect\hyperlink{site-content}{Skip to
content}\protect\hyperlink{site-index}{Skip to site index}

\href{https://www.nytimes.com/section/technology}{Technology}

\href{https://myaccount.nytimes.com/auth/login?response_type=cookie\&client_id=vi}{}

\href{https://www.nytimes.com/section/todayspaper}{Today's Paper}

\href{/section/technology}{Technology}\textbar{}Amazon, Apple, Facebook
and Google Prepare for Their `Big Tobacco Moment'

\url{https://nyti.ms/3f7lchW}

\begin{itemize}
\item
\item
\item
\item
\item
\item
\end{itemize}

Advertisement

\protect\hyperlink{after-top}{Continue reading the main story}

Supported by

\protect\hyperlink{after-sponsor}{Continue reading the main story}

\hypertarget{amazon-apple-facebook-and-google-prepare-for-their-big-tobacco-moment}{%
\section{Amazon, Apple, Facebook and Google Prepare for Their `Big
Tobacco
Moment'}\label{amazon-apple-facebook-and-google-prepare-for-their-big-tobacco-moment}}

The tech C.E.O.s will appear together at a congressional hearing on
Wednesday to argue that their companies do not stifle competition.

\includegraphics{https://static01.nyt.com/images/2020/07/24/business/24TECHCEOS-GRID/24TECHCEOS-GRID-articleLarge.jpg?quality=75\&auto=webp\&disable=upscale}

By \href{https://www.nytimes.com/by/cecilia-kang}{Cecilia Kang},
\href{https://www.nytimes.com/by/jack-nicas}{Jack Nicas} and
\href{https://www.nytimes.com/by/david-mccabe}{David McCabe}

\begin{itemize}
\item
  Published July 28, 2020Updated July 29, 2020
\item
  \begin{itemize}
  \item
  \item
  \item
  \item
  \item
  \item
  \end{itemize}
\end{itemize}

WASHINGTON --- After lawmakers collected hundreds of hours of interviews
and obtained more than 1.3 million documents about Amazon, Apple,
Facebook and Google, their chief executives will testify before Congress
on Wednesday to defend their powerful businesses from the hammer of
government.

The captains of the New Gilded Age ---
\href{https://www.nytimes.com/2020/07/27/business/jeff-bezos-amazon-congress.html}{Jeff
Bezos of Amazon}, Tim Cook of Apple, Mark Zuckerberg of Facebook and
Sundar Pichai of Google --- will appear together before Congress for the
first time to justify their business practices. Members of the House
judiciary's antitrust subcommittee
\href{https://www.nytimes.com/2019/06/11/technology/antitrust-hearing.html}{have
investigated the internet giants} for more than a year on accusations
that they stifled rivals and harmed consumers.

The hearing is the government's most aggressive show against tech power
since the
\href{https://www.nytimes.com/2000/04/04/business/us-vs-microsoft-overview-us-judge-says-microsoft-violated-antitrust-laws-with.html}{pursuit
to break up Microsoft} two decades ago. It is set to be a bizarre
spectacle, with four men who run companies worth a total of around
\$4.85 trillion --- and who include two of the world's richest
individuals --- primed to argue that their businesses are not really
that powerful after all.

And it will be a first in another way: Mr. Zuckerberg, Mr. Pichai, Mr.
Bezos and Mr. Cook will all be testifying via videoconference, rather
than rising side-by-side for a swearing-in at a witness table in
Washington. Perhaps appropriately, their reckoning will be broadcast
online.

``It has the feeling of tech's Big Tobacco moment,'' said Gigi Sohn, a
former senior adviser at the Federal Communications Commission and a
fellow at Georgetown University's law school, referring to the
\href{https://www.nytimes.com/1994/04/15/us/tobacco-chiefs-say-cigarettes-aren-t-addictive.html}{1994
congressional appearance} of top executives of the seven largest
American tobacco companies, who said they did not believe that
cigarettes were addictive.

The hearing, which caps a 13-month investigation by the House
subcommittee, will be closely watched for clues that could advance other
antitrust cases against the companies. The Federal Trade Commission, for
one,
\href{https://www.nytimes.com/2020/07/17/technology/ftc-facebook-investigation.html}{is
preparing to depose} Mr. Zuckerberg and other Facebook executives in its
13-month probe of the social network. The Justice Department may soon
unveil a case against Google. And an investigation into Apple by state
attorneys general also appears to be advancing.

As a result, preparations for the hearing have been frenetic --- even
with the event postponed by a few days this week to accommodate the
commemoration of Representative John Lewis --- as tech lobbyists
jockeyed behind the scenes to influence the types of questions that
lawmakers might ask.

At the hearing, which starts at noon on Wednesday, the 15 members of the
antitrust subcommittee will have five minutes for each question.
Representative David Cicilline, Democrat of Rhode Island and the
chairman of the subcommittee, will control the number of rounds of
questioning, potentially stretching questioning into the evening.

The length of the hearing may also be prolonged since the antitrust
issues facing Apple, Facebook,
\href{https://www.nytimes.com/2019/06/02/business/google-antitrust-investigation.html}{Google
and Amazon} are complex and vastly different.

Amazon is accused of abusing its role as both a retailer and a platform
hosting third-party sellers on its marketplace. Apple has been accused
of unfairly using its clout over its App Store to block rivals and to
force apps to pay high commissions. Rivals have said Facebook has a
monopoly in social networking. Alphabet, the parent company of Google,
is dealing with multiple antitrust allegations because of Google's
dominance in online advertising, search and smartphone software.

Democrats may also veer off the topic of antitrust and bring up concerns
about misinformation on social media. Some Republicans are expected to
sidetrack discussion with their concerns of liberal bias at the Silicon
Valley companies and accusations that conservative voices are censored.

``There was an attitude these were great American companies that created
jobs and that we should have a hands-off approach and let them
flourish,'' Mr. Cicilline said in an interview. ``But there are a lot of
serious issues we have uncovered over the course of the investigation
that weren't apparent when we first began investigating.''

Facebook, Amazon, Google and Apple declined to comment.

For the chief executives, the hearing will be a test of how they perform
under fire. Mr. Bezos, 56, has
\href{https://www.nytimes.com/2020/07/27/business/jeff-bezos-amazon-congress.html?action=click\&module=News\&pgtype=Homepage}{not
previously testified to Congress}, while Mr. Cook, 59, and Mr. Pichai,
48, have both testified once before. Mr. Zuckerberg, 36, the youngest of
the group, has the distinction of being the veteran: He has answered
questions at three congressional hearings in the past two years as
Facebook has dealt with issues such as election interference and privacy
violations.

But none are taking any chances for the event to go awry. Mr.
Zuckerberg, who had been at his 750-acre estate on the Hawaiian island
of Kauai, has been preparing for his testimony with the law firm
WilmerHale, according to people with knowledge of the matter. And a
small team is working with Mr. Bezos for his testimony in Seattle, said
people with knowledge of the matter.

For weeks, the tech giants have also waged a lobbying battle to soften
any blows. All four chief executives planned to call lawmakers on the
House subcommittee in the days before the hearing, said three people
with knowledge of the preparations who were not authorized to speak
publicly.

Apple and Amazon also recently released studies to rebut claims of
market dominance and anticompetitive practices. Last week, Apple
publicized a study by a consulting firm called Analysis Group showing
that the 30 percent commission it charges many apps for the right to
appear on iPhones is close to what other platforms charge for
distribution. The study left out that Apple helped popularize that 30
percent standard across the industry.

Amazon-funded economic consultants have in recent months argued that the
e-commerce company's business model, which is not grounded in selling
ads like Google and Facebook, makes it less likely to violate antitrust
laws. Last week, Amazon also released
\href{https://blog.aboutamazon.com/small-business/small-business-success-in-challenging-times}{a
report on small business}, saying sales by third-party sellers grew 26
percent in the past year, outpacing Amazon's own sales directly to
consumers.

Google has said that the search and advertising tech markets that it
dominates are changing fast. More than half of all searches for products
on the internet originate on Amazon, Google's lobbyists have said.

And Facebook's Washington staff has pointed to competition from China,
particularly from the popular video app
\href{https://www.nytimes.com/2020/07/26/technology/tiktok-china-ban-model.html}{TikTok},
as evidence that competition in social media abounds. The Chinese-owned
app is in the cross hairs of the Trump administration, which has
\href{https://www.nytimes.com/2020/07/26/technology/tiktok-china-ban-model.html}{threatened
to ban it} for national security reasons.

Big Tech's rivals have also jockeyed to have their gripes brought up at
the hearing, even if for just a few minutes. The House subcommittee has
been flooded with proposed questions, documents and letters from the
companies' competitors, according to congressional staff and rivals.

Spotify, for instance, submitted questions about Apple's dominance of
the App Store. GreatFire, a China-based group, sent a letter with nine
questions for Mr. Cook about Apple's censorship of certain apps in
China. Blix, a company whose email app competes with Apple and that is
suing Apple in federal court for patent infringement, sent five
questions to the subcommittee, including one on
why\href{https://www.nytimes.com/interactive/2019/09/09/technology/apple-app-store-competition.html}{Apple
ranked its own apps ahead of rivals' offerings} in its App Store.

This month, David Heinemeier Hansson, the co-founder of Basecamp, a
project-management tool, said he also briefed lawmakers
on\href{https://www.nytimes.com/2020/06/19/opinion/apple-app-store-hey.html}{a
recent public spat with Apple}. Apple had denied Basecamp's new email
app from appearing in the App Store because it charged customers outside
of Apple's payment system. After Mr. Heinemeier Hansson complained
publicly, Apple permitted the app with some minor changes.

The subcommittee's members, who have already held five hearings about
the tech giants, were informed and thoughtful during his briefing, Mr.
Heinemeier Hansson said.

``Clearly there's already a great resonance,'' he said.

Even if the hearing results in more theater than substance, some said
the greatest risk to the tech companies was increasing momentum toward
regulations.

``The C.E.O.s don't want to be testifying. Even having this collective
hearing creates a sense of quasi-guilt just because of who else has
gotten called in like this --- Big Pharma, Big Tobacco, Big Banks,''
said Paul Gallant, a tech policy analyst at the investment firm Cowen.
``That's not a crowd they want to be associated with.''

Cecilia Kang reported from Washington, Jack Nicas from Chicago and David
McCabe from Wellfleet, Mass. Daisuke Wakabayashi, Mike Isaac and Karen
Weise contributed reporting.

Advertisement

\protect\hyperlink{after-bottom}{Continue reading the main story}

\hypertarget{site-index}{%
\subsection{Site Index}\label{site-index}}

\hypertarget{site-information-navigation}{%
\subsection{Site Information
Navigation}\label{site-information-navigation}}

\begin{itemize}
\tightlist
\item
  \href{https://help.nytimes.com/hc/en-us/articles/115014792127-Copyright-notice}{©~2020~The
  New York Times Company}
\end{itemize}

\begin{itemize}
\tightlist
\item
  \href{https://www.nytco.com/}{NYTCo}
\item
  \href{https://help.nytimes.com/hc/en-us/articles/115015385887-Contact-Us}{Contact
  Us}
\item
  \href{https://www.nytco.com/careers/}{Work with us}
\item
  \href{https://nytmediakit.com/}{Advertise}
\item
  \href{http://www.tbrandstudio.com/}{T Brand Studio}
\item
  \href{https://www.nytimes.com/privacy/cookie-policy\#how-do-i-manage-trackers}{Your
  Ad Choices}
\item
  \href{https://www.nytimes.com/privacy}{Privacy}
\item
  \href{https://help.nytimes.com/hc/en-us/articles/115014893428-Terms-of-service}{Terms
  of Service}
\item
  \href{https://help.nytimes.com/hc/en-us/articles/115014893968-Terms-of-sale}{Terms
  of Sale}
\item
  \href{https://spiderbites.nytimes.com}{Site Map}
\item
  \href{https://help.nytimes.com/hc/en-us}{Help}
\item
  \href{https://www.nytimes.com/subscription?campaignId=37WXW}{Subscriptions}
\end{itemize}
