Sections

SEARCH

\protect\hyperlink{site-content}{Skip to
content}\protect\hyperlink{site-index}{Skip to site index}

\href{https://www.nytimes.com/section/movies}{Movies}

\href{https://myaccount.nytimes.com/auth/login?response_type=cookie\&client_id=vi}{}

\href{https://www.nytimes.com/section/todayspaper}{Today's Paper}

\href{/section/movies}{Movies}\textbar{}The Director Gina
Prince-Bythewood Has Always Had Game

\url{https://nyti.ms/30UUQLg}

\begin{itemize}
\item
\item
\item
\item
\item
\end{itemize}

\href{https://www.nytimes.com/spotlight/at-home?action=click\&pgtype=Article\&state=default\&region=TOP_BANNER\&context=at_home_menu}{At
Home}

\begin{itemize}
\tightlist
\item
  \href{https://www.nytimes.com/2020/07/28/books/time-for-a-literary-road-trip.html?action=click\&pgtype=Article\&state=default\&region=TOP_BANNER\&context=at_home_menu}{Take:
  A Literary Road Trip}
\item
  \href{https://www.nytimes.com/2020/07/29/magazine/bored-with-your-home-cooking-some-smoky-eggplant-will-fix-that.html?action=click\&pgtype=Article\&state=default\&region=TOP_BANNER\&context=at_home_menu}{Cook:
  Smoky Eggplant}
\item
  \href{https://www.nytimes.com/2020/07/27/travel/moose-michigan-isle-royale.html?action=click\&pgtype=Article\&state=default\&region=TOP_BANNER\&context=at_home_menu}{Look
  Out: For Moose}
\item
  \href{https://www.nytimes.com/interactive/2020/at-home/even-more-reporters-editors-diaries-lists-recommendations.html?action=click\&pgtype=Article\&state=default\&region=TOP_BANNER\&context=at_home_menu}{Explore:
  Reporters' Obsessions}
\end{itemize}

Advertisement

\protect\hyperlink{after-top}{Continue reading the main story}

Supported by

\protect\hyperlink{after-sponsor}{Continue reading the main story}

Your Weekend Watch With Tony and Manohla

\hypertarget{the-director-gina-prince-bythewood-has-always-had-game}{%
\section{The Director Gina Prince-Bythewood Has Always Had
Game}\label{the-director-gina-prince-bythewood-has-always-had-game}}

Our readers and critic revisited ``Love \& Basketball,'' the director's
feature debut about a Black girl who plays ball, falls in love and
carves out her own path to happiness.

\includegraphics{https://static01.nyt.com/images/2020/07/31/arts/31Weekend-Watch-LoveBasketball/31Weekend-Watch-LoveBasketball-articleLarge.jpg?quality=75\&auto=webp\&disable=upscale}

By \href{https://www.nytimes.com/by/manohla-dargis}{Manohla Dargis}

\begin{itemize}
\item
  Published July 28, 2020Updated July 30, 2020
\item
  \begin{itemize}
  \item
  \item
  \item
  \item
  \item
  \end{itemize}
\end{itemize}

There are different reasons I adore
\href{https://www.youtube.com/watch?v=Ur83i6_BjbE}{``Love \&
Basketball''}: its tenacious athlete-heroine, its twinned belief in
female tears and female ambition, Alfre Woodard's artful supporting
turn. The film won me over when I first saw it in 2000. But it wasn't
until I looked at it again for our
\href{https://www.nytimes.com/2020/07/23/movies/love-basketball-sanaa-lathan-omar-epps.html}{Weekend
Watch} that I realized that the director
\href{https://www.nytimes.com/2020/07/10/movies/the-old-guard-gina-prince-bythewood.html}{Gina
Prince-Bythewood} had given herself a sly cameo. She plays one of our
heroine's opponents, who dives to the floor alongside her midgame as
they scramble for the ball. It's such a perfect encapsulation of the
movie and its themes that I watched it again and again.

\includegraphics{https://static01.nyt.com/images/2020/07/28/arts/00weekend-watch2/merlin_175048623_93c75df9-1aa2-4127-9abf-a2332ce68b27-articleLarge.jpg?quality=75\&auto=webp\&disable=upscale}

I bet the moment resonated with our readers who love the movie, too.

\begin{quote}
A few years ago, I had the opportunity to spend some time with some of
the members of my college's women's basketball team. I asked how many of
them had seen ``Love and Basketball'' --- they all had, and every one of
them said that it had taken their attachment to the game to a much
higher level. That's what happens when ``entertainment'' is actually art
\ldots{} it challenges, it inspires, it creates futures. ---
\href{https://nyti.ms/2BBSKXY\#permid=108296512}{Tad La Fountain,
Penhook, VA}
\end{quote}

This scramble for the ball is also a wryly apt metaphor for the kind of
wild maneuvering that female directors --- and particularly women of
color --- need to perform to get in the industry game, then stay in and
succeed. Prince-Bythewood has managed to keep directing, for the big and
small screen, despite the industry's entrenched biases, its racism and
sexism. And now with the Netflix movie
\href{https://www.nytimes.com/2020/07/09/movies/the-old-guard-review.html}{``The
Old Guard''} she has her biggest hit. Its success makes the 20th
anniversary of ``Love \& Basketball'' especially sweet, and together the
movies show how fluidly Prince-Bythewood has pivoted from intimate drama
to full-bore action cinema.

A classic coming-of-age story, ``Love \& Basketball'' follows Monica
(Sanaa Lathan) as she pursues her passion for basketball while nurturing
a relationship with her next-door-neighbor, Quincy (Omar Epps). They
first shoot hoops together in childhood and later fall in love or, maybe
just succumb to their long-simmering feelings, all while playing very
different gendered games, on and off the court. The film borrows some
familiar genre moves --- the kids meet classically cute during a game
--- but the intensity of Monica's athletic drive quickly makes the film
skew serious and heartfelt, not funny and flirty. Part of what's
striking about ``Love \& Basketball'' is that, as the title announces,
her passions are twofold and not mutually exclusive.

\begin{quote}
One nice local detail: the woman who portrayed UCLA's coach in the
movie, Colleen Matsuhara, in real life really did coach UCLA (albeit as
an assistant coach, not the head coach) including their championship
season with Ann Meyers. ---
\href{https://nyti.ms/330VJEy\#permid=108316663}{mkt42, Portland, OR}
\end{quote}

Unlike many contemporary screen romances, ``Love \& Basketball''
unapologetically wears its full heart on its sleeve, avoiding easy
laughs and the safe distance of irony. It takes Monica seriously, which
means that it takes her life as an athlete seriously, which is expressed
in all the attention on how her physical, emotional and psychological
struggles on the court reflect --- and shape --- her life off it, too.
The movie respects Monica's struggle toward greatness in her sport, even
as those difficulties create static and worse for her with Quincy and
her mother (Woodard, the film's M.V.P.). And while some have read the
happy ending as wishful thinking, I prefer to see it as a utopian
feminist argument for a life enriched by love and by work.

Prince-Bythewood's insistence on granting Monica both is deeply
satisfying, and although happy endings may seem awfully corny, they're
foundational for some of us. American movies are so very good at
violence that we can be overly suspicious of onscreen love, embarrassed
by our sniffles and sobs. Not all of us! ``This has been my favorite
movie forever,'' Maria Teresa Alzuru
\href{https://nyti.ms/2ElrGgy\#permid=108333899}{wrote in a comment},
adding, ``I can quote it word for word.'' Another reader, Rhonda, added
\href{https://nyti.ms/3gdHvUI\#permid=108310594}{a personal note} that
cuts to why so many of us hunger for romantic stories: ``L \& B is
definitely one of my all-time faves. It kind of mimics my own earlier
life, except mine didn't have a happy ending. So, I love it, but it also
makes me sigh each time.'' More than one reader underscored the film's
status in contemporary Black American cinema.

\begin{quote}
But the fact that romance is such a convention in films and that for
eons white couples have dominated the romance category and lists of the
``10 Best'' Valentine's Day movies, it's a necessary and welcome
intervention in an ongoing saga of diversifying films. ---
\href{https://nyti.ms/3hMiNei\#permid=108346903}{BG Klinger, Chicago,
IL}

What a relief to see a movie featuring Black characters who develop on
their own and not just in relation to whites. It's a shock when the
first white character appears, well into the film, as fully intended by
the director. --- \href{https://nyti.ms/3320hL5\#permid=108331295}{Dan
DeNoon, Atlanta}
\end{quote}

The genesis of ``Love \& Basketball'' can be traced to
Prince-Bythewood's own story. She played basketball in high school and
ran track at the University of California, Los Angeles, where she
attended film school. Later, she wrote for television shows like
\href{https://ew.com/tv/a-different-world-where-are-they-now/}{``A
Different World,''}but, yes, what she really wanted to do was direct.
When she was developing the script for ``Love \& Basketball,'' the film
she thought of was ``When Harry Met Sally,''
\href{https://www.latimes.com/entertainment-arts/movies/story/2020-04-21/love-and-basketball-oral-history-20-year-anniversary}{she
told} The Los Angeles Times. But, Prince-Bythewood said, ``I wasn't
seeing myself in movies like that in love stories.'' In addition, she
said, ``there was a semi-autobiographical story in my head about a Black
girl who wanted to be the first girl in the N.B.A.''

One of the strongest, most vivid auteurist threads that runs through
Prince-Bythewood's filmography is her vision of the fully rounded,
powerful and sovereign female protagonist.
\href{https://www.nytimes.com/by/jason-bailey}{Charlize Theron's
warrior} in ``The Old Guard,'' known as Andromache of Scythia (Andy for
short), as well as her Amazonian compatriot, Nile (KiKi Lane), are of a
thematic piece with other memorable Prince-Bythewood female characters,
most notably the self-doubting yet triumphant pop star played by Gugu
Mbatha-Raw in
\href{https://www.nytimes.com/2014/11/14/movies/beyond-the-lights-a-divas-romance.html}{``Beyond
the Lights''} (2014). Like Monica, these are women with struggles that
are manifested physically in bodies that are strong, resilient and ---
instructively --- not hypersexualized or turned into fetishistic
spectacles for the viewer.

These are, then, women who sweat, characters with grace and strength and
arms that, like the rest of them, have been beautifully shaped by a
woman whose
\href{https://twitter.com/GPBmadeit?ref_src=twsrc\%5Egoogle\%7Ctwcamp\%5Eserp\%7Ctwgr\%5Eauthor}{Twitter
bio} reads: ``Used to ball. Now I write/direct and watch my boys play.''
That's characteristically unassuming for a filmmaker who has brought
change to the representation of women and to the stubbornly male, very
white figure of the American movie director. When she made ``Love \&
Basketball,'' Prince-Bythewood belonged to a tiny sorority of Black
women filmmakers that has grown --- not a lot, not nearly enough, but
enough to feel significant. There is still a great deal left to do, but
it's exhilarating to see her help mount a necessary full-court press.

Advertisement

\protect\hyperlink{after-bottom}{Continue reading the main story}

\hypertarget{site-index}{%
\subsection{Site Index}\label{site-index}}

\hypertarget{site-information-navigation}{%
\subsection{Site Information
Navigation}\label{site-information-navigation}}

\begin{itemize}
\tightlist
\item
  \href{https://help.nytimes.com/hc/en-us/articles/115014792127-Copyright-notice}{©~2020~The
  New York Times Company}
\end{itemize}

\begin{itemize}
\tightlist
\item
  \href{https://www.nytco.com/}{NYTCo}
\item
  \href{https://help.nytimes.com/hc/en-us/articles/115015385887-Contact-Us}{Contact
  Us}
\item
  \href{https://www.nytco.com/careers/}{Work with us}
\item
  \href{https://nytmediakit.com/}{Advertise}
\item
  \href{http://www.tbrandstudio.com/}{T Brand Studio}
\item
  \href{https://www.nytimes.com/privacy/cookie-policy\#how-do-i-manage-trackers}{Your
  Ad Choices}
\item
  \href{https://www.nytimes.com/privacy}{Privacy}
\item
  \href{https://help.nytimes.com/hc/en-us/articles/115014893428-Terms-of-service}{Terms
  of Service}
\item
  \href{https://help.nytimes.com/hc/en-us/articles/115014893968-Terms-of-sale}{Terms
  of Sale}
\item
  \href{https://spiderbites.nytimes.com}{Site Map}
\item
  \href{https://help.nytimes.com/hc/en-us}{Help}
\item
  \href{https://www.nytimes.com/subscription?campaignId=37WXW}{Subscriptions}
\end{itemize}
