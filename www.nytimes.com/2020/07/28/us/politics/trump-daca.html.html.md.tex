Sections

SEARCH

\protect\hyperlink{site-content}{Skip to
content}\protect\hyperlink{site-index}{Skip to site index}

\href{https://www.nytimes.com/section/politics}{Politics}

\href{https://myaccount.nytimes.com/auth/login?response_type=cookie\&client_id=vi}{}

\href{https://www.nytimes.com/section/todayspaper}{Today's Paper}

\href{/section/politics}{Politics}\textbar{}Trump Delays Effort to End
Protections for Immigrant `Dreamers'

\url{https://nyti.ms/2X2clYI}

\begin{itemize}
\item
\item
\item
\item
\item
\end{itemize}

Advertisement

\protect\hyperlink{after-top}{Continue reading the main story}

Supported by

\protect\hyperlink{after-sponsor}{Continue reading the main story}

\hypertarget{trump-delays-effort-to-end-protections-for-immigrant-dreamers}{%
\section{Trump Delays Effort to End Protections for Immigrant
`Dreamers'}\label{trump-delays-effort-to-end-protections-for-immigrant-dreamers}}

After the Supreme Court ruled that President Trump failed to present a
valid rationale for ending DACA, the administration said it would
conduct a ``comprehensive review'' of the program.

\includegraphics{https://static01.nyt.com/images/2020/07/01/us/politics/00dc-DACAhfo/00dc-DACAhfo-articleLarge.jpg?quality=75\&auto=webp\&disable=upscale}

\href{https://www.nytimes.com/by/michael-d-shear}{\includegraphics{https://static01.nyt.com/images/2018/06/13/multimedia/author-michael-d-shear/author-michael-d-shear-thumbLarge-v2.png}}\href{https://www.nytimes.com/by/caitlin-dickerson}{\includegraphics{https://static01.nyt.com/images/2018/11/06/multimedia/author-caitlin-dickerson/author-caitlin-dickerson-thumbLarge-v2.png}}

By \href{https://www.nytimes.com/by/michael-d-shear}{Michael D. Shear}
and \href{https://www.nytimes.com/by/caitlin-dickerson}{Caitlin
Dickerson}

\begin{itemize}
\item
  July 28, 2020
\item
  \begin{itemize}
  \item
  \item
  \item
  \item
  \item
  \end{itemize}
\end{itemize}

WASHINGTON --- President Trump will not try again to immediately
terminate President Barack Obama's program that protects young
undocumented immigrants, after the Supreme Court's decision to
\href{https://www.nytimes.com/2020/06/18/us/trump-daca-supreme-court.html}{invalidate
Mr. Trump's first attempt} to make good on a crackdown that is at the
core of his political identity.

Instead, the administration announced new restrictions on the program,
known as Deferred Action for Childhood Arrivals, which has allowed about
650,000 undocumented immigrants to live and work in the country legally.

Amid a ``comprehensive review'' of the program, Chad Wolf, the acting
homeland security secretary,
\href{https://www.dhs.gov/sites/default/files/publications/20_0728_s1_daca-reconsideration-memo.pdf}{announced
in a memo} that immigrants who already had protections would be allowed
to renew their status under the program for one year, rather than two.
The memo, which is intended to replace the one that originally
established DACA in 2012, also said that first-time applicants to the
program would be rejected.

The announcement appears to directly contradict an order by a federal
judge, who ruled last month that the administration must immediately
begin accepting new applications for the program. It will most likely
face immediate legal challenges.

``We obviously have no choice but to go back to court,'' said Mark
Rosenbaum, a lawyer on the case against the administration's move to
eliminate DACA that reached the Supreme Court. ``It was illegal the
first time, and now it's a constitutional crisis. It's as if a Supreme
Court decision was written with invisible ink.''

Immigrants rights groups saw the revised memo as a first step toward
completely eliminating DACA. It could also energize Mr. Trump's base by
suggesting that the program would eventually be scrapped, while helping
the president avoid the blowback from images of young people being
deported just before the election.

\href{https://www.pewresearch.org/fact-tank/2020/06/17/americans-broadly-support-legal-status-for-immigrants-brought-to-the-u-s-illegally-as-children/}{A
Pew Research Center poll} conducted in June found that three-quarters of
Americans support not only allowing so-called Dreamers to remain in the
United States, but also providing them a path to permanent residency.

``I think they made the calculation that by deferring the final blow to
DACA until after the election that they would be able to escape taking
the hardest hit,'' said Omar Jadwat, who directs the American Civil
Liberties Union's Immigrants' Rights Project.

Officials declined to say how long the review would take or whether it
would be completed before the general election in November, although the
decision to allow one-year renewals suggested that Mr. Trump and his
aides did not envision making another attempt to end the program before
the vote.

Many of the hundreds of thousands of DACA beneficiaries who have felt
their futures to be in peril reacted with exasperation to the latest
development.

``I was one of the lucky ones,'' said Pierre R. Berastaín, whose
application for a two-year renewal of his DACA status, which allows him
to work at Harvard University, his alma mater, was approved this past
month. ``I'm very cognizant of the tremendous anxiety of people who are
in limbo.''

Mr. Berastaín said that as he waited for a Supreme Court decision on his
fate, he dusted off an old safety plan he had developed as a student and
updated it with current contact information for his loved ones, his
lawyers and any elected officials who might be able to intervene on his
behalf if he was arrested by immigration authorities.

``I had this extreme anxiety and depression because of the instability
of my situation,'' he said.

Groups that favor tighter restrictions on immigration were gratified
that the program would at least be curtailed.

``It's a smart move,'' said Jessica M. Vaughan, the director of policy
studies at the Center for Immigration Studies, one such group. ``In this
way, he manages to keep the ball in Congress's court without having to
keep the program going on autopilot.''

Ms. Vaughan said she thought Congress should ultimately decide the fate
of the program, but that for now lawmakers should remain focused on
responding to the coronavirus pandemic and the resulting economic
crisis.

Mr. Trump's promised assault on immigration, including a pledge to end
DACA, was critical to his election victory in 2016 and became a central
part of his administration's agenda as he moved to keep out refugees and
asylum seekers, to build a wall across the southwestern border and to
reduce legal immigration.

Mr. Trump's
\href{https://www.nytimes.com/2017/09/05/us/politics/trump-daca-dreamers-immigration.html}{initial
move to end DACA in 2017} was blocked for more than two years by federal
judges who said he had failed to provide proper justification for ending
a program that allowed some young undocumented immigrants to live and
work in the United States without the threat of deportation.

The Supreme Court agreed last month.

The president has long argued that DACA was an illegal use of executive
authority by his predecessor. But he has wavered about what should be
done to the hundreds of thousands of immigrants who benefit from it. At
a White House news briefing Tuesday evening, Mr. Trump said, ``We are
going to make DACA happy and the DACA people and representatives happy,
and also end up with a fantastic merit-based immigration system.''

He again said he was working on a merit-based immigration bill that he
could enact without Congress, an assertion that is false.

Thousands of undocumented young people who have turned 15 since the
original rescission --- the minimum age required to apply --- had
expected the government to begin accepting new applications to comply
with the Supreme Court's decision.

However, the agency that processes DACA applications has been rejecting
them, according to immigration lawyers, while it has continued to
process renewals for current recipients.

Most legal scholars interpreted the ruling to mean that the original
program had been revived and that, therefore, new applications must be
accepted. An administration official said on Tuesday that the memo
issued that day would legally justify the decision against approving new
applications.

Mr. Obama
\href{https://www.nytimes.com/2012/06/16/us/us-to-stop-deporting-some-illegal-immigrants.html}{created
DACA in 2012} after Congress refused to provide a permanent path to
citizenship for the young immigrants. Most of the Dreamers are in many
ways indistinguishable from the American citizens with whom they
attended elementary, middle and high school.

The program has strict requirements. To be eligible, applicants have to
show that they had committed no serious crimes, had arrived in the
United States before they turned 16 and were no older than 30, had lived
in the United States for at least the previous five years, and were in
school, had graduated from high school or had received a General
Educational Development certificate, or were honorably discharged from
the military.

Mr. Trump
\href{https://twitter.com/realdonaldtrump/status/908276308265795585?lang=en}{has
at times expressed sympathy} for the immigrants who are protected by the
program even as he denounced it as an illegal use of presidential
authority.

In 2017, his hard-line advisers --- including Stephen Miller, the
architect of his immigration agenda; Stephen K. Bannon, his chief
strategist; and Jeff Sessions, his attorney general --- urged him to end
the program, which at the time faced a legal challenge from several
conservative attorneys general.

Michael Crowley contributed reporting from in New Haven, Conn.; Zolan
Kanno-Youngs from Washington; and Miriam Jordan from Los Angeles.

Advertisement

\protect\hyperlink{after-bottom}{Continue reading the main story}

\hypertarget{site-index}{%
\subsection{Site Index}\label{site-index}}

\hypertarget{site-information-navigation}{%
\subsection{Site Information
Navigation}\label{site-information-navigation}}

\begin{itemize}
\tightlist
\item
  \href{https://help.nytimes.com/hc/en-us/articles/115014792127-Copyright-notice}{©~2020~The
  New York Times Company}
\end{itemize}

\begin{itemize}
\tightlist
\item
  \href{https://www.nytco.com/}{NYTCo}
\item
  \href{https://help.nytimes.com/hc/en-us/articles/115015385887-Contact-Us}{Contact
  Us}
\item
  \href{https://www.nytco.com/careers/}{Work with us}
\item
  \href{https://nytmediakit.com/}{Advertise}
\item
  \href{http://www.tbrandstudio.com/}{T Brand Studio}
\item
  \href{https://www.nytimes.com/privacy/cookie-policy\#how-do-i-manage-trackers}{Your
  Ad Choices}
\item
  \href{https://www.nytimes.com/privacy}{Privacy}
\item
  \href{https://help.nytimes.com/hc/en-us/articles/115014893428-Terms-of-service}{Terms
  of Service}
\item
  \href{https://help.nytimes.com/hc/en-us/articles/115014893968-Terms-of-sale}{Terms
  of Sale}
\item
  \href{https://spiderbites.nytimes.com}{Site Map}
\item
  \href{https://help.nytimes.com/hc/en-us}{Help}
\item
  \href{https://www.nytimes.com/subscription?campaignId=37WXW}{Subscriptions}
\end{itemize}
