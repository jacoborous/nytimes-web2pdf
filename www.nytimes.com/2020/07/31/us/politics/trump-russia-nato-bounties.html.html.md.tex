Sections

SEARCH

\protect\hyperlink{site-content}{Skip to
content}\protect\hyperlink{site-index}{Skip to site index}

\href{https://www.nytimes.com/section/politics}{Politics}

\href{https://myaccount.nytimes.com/auth/login?response_type=cookie\&client_id=vi}{}

\href{https://www.nytimes.com/section/todayspaper}{Today's Paper}

\href{/section/politics}{Politics}\textbar{}Trump Still Defers to Putin,
Even as He Dismisses U.S. Intelligence and the Allies

\url{https://nyti.ms/30jck4Z}

\begin{itemize}
\item
\item
\item
\item
\item
\end{itemize}

\begin{itemize}
\item
  \href{https://www.nytimes.com/2020/07/31/us/elections/biden-vs-trump.html?action=click\&pgtype=Article\&state=default\&region=TOP_BANNER\&context=storylines_menu}{Election
  Updates}
\item
  \href{https://www.nytimes.com/article/biden-vice-president-2020.html?action=click\&pgtype=Article\&state=default\&region=TOP_BANNER\&context=storylines_menu}{Biden's
  V.P. Search}
\item
  \href{https://www.nytimes.com/interactive/2020/07/24/us/politics/trump-biden-campaign-donors.html?action=click\&pgtype=Article\&state=default\&region=TOP_BANNER\&context=storylines_menu}{Map
  of Donations}
\item
  \href{https://www.nytimes.com/interactive/2020/us/elections/delegate-count-primary-results.html?action=click\&pgtype=Article\&state=default\&region=TOP_BANNER\&context=storylines_menu}{Delegate
  Count}
\item
  \href{https://www.nytimes.com/interactive/2019/us/politics/2020-presidential-candidates.html?action=click\&pgtype=Article\&state=default\&region=TOP_BANNER\&context=storylines_menu}{The
  Candidates}
\item
  \href{https://www.nytimes.com/newsletters/politics?action=click\&pgtype=Article\&state=default\&region=TOP_BANNER\&context=storylines_menu}{Politics
  Newsletter}
\end{itemize}

Advertisement

\protect\hyperlink{after-top}{Continue reading the main story}

Supported by

\protect\hyperlink{after-sponsor}{Continue reading the main story}

News Analysis

\hypertarget{trump-still-defers-to-putin-even-as-he-dismisses-us-intelligence-and-the-allies}{%
\section{Trump Still Defers to Putin, Even as He Dismisses U.S.
Intelligence and the
Allies}\label{trump-still-defers-to-putin-even-as-he-dismisses-us-intelligence-and-the-allies}}

Say this about President Trump's approach to Moscow: It's been
consistent.

\includegraphics{https://static01.nyt.com/images/2020/07/31/us/politics/31dc-trumprussia/merlin_157170411_3e868d3c-9294-4286-b2a7-324bc7d61bfd-articleLarge.jpg?quality=75\&auto=webp\&disable=upscale}

\href{https://www.nytimes.com/by/david-e-sanger}{\includegraphics{https://static01.nyt.com/images/2018/10/03/multimedia/author-david-e-sanger/author-david-e-sanger-thumbLarge.png}}

By \href{https://www.nytimes.com/by/david-e-sanger}{David E. Sanger}

\begin{itemize}
\item
  July 31, 2020
\item
  \begin{itemize}
  \item
  \item
  \item
  \item
  \item
  \end{itemize}
\end{itemize}

WASHINGTON --- On the eve of accepting the Republican nomination for
president four years ago, Donald J. Trump declared that he would pull
out of NATO if American allies did not pay more for their defense,
waving away the thought that it would play into the hands of President
Vladimir V. Putin of Russia, who has spent his career trying to
dismantle the Western alliance.

Asked about his deference to the Kremlin leader, Mr. Trump responded,
``He's been complimentary of me.''

This week, as his renomination nears, Mr. Trump announced that he was
pulling a third of American troops from Germany. He declared in recent
days that
\href{https://www.axios.com/trump-russia-bounties-taliban-putin-call-4a0f6110-ab58-41c0-96fc-57b507462af1.html}{he
had never raised with Mr. Putin, during a recent phone conversation,}
American intelligence indicating that Russia was paying a bounty to the
Taliban for the killing of American soldiers in Afghanistan, because he
distrusted the information from his own intelligence agencies. Nor has
he issued warnings about what price, if any, Mr. Putin would pay for
seeking
\href{https://www.nytimes.com/2020/07/24/us/politics/election-interference-russia-china-iran.html}{to
influence the 2020 election} or
\href{https://www.nytimes.com/2020/07/28/us/politics/russia-disinformation-coronavirus.html}{pushing
disinformation about the coronavirus}. American intelligence agencies
say Russia is trying both.

Say this about Mr. Trump's approach to Moscow: It has been consistent.

With three months until Election Day, he is repeating a variant of lines
that he uttered during his first campaign. It would be ``wonderful'' if
``instead of fighting each other, we got along.'' That he and Mr. Putin
are working together to reduce the threat of nuclear war, even though
both nations have spent the past four years developing nuclear weapons
and scuttling treaties that limited their stockpiles. In recent days, he
has begun deflecting questions about Russia's cyberactivities by
repeating another line from 2016: that everyone turns a blind eye to
China.

What is striking about all these comments is that they indicate little
or no evolution in Mr. Trump's approach.

\hypertarget{latest-updates-2020-election}{%
\section{\texorpdfstring{\href{https://www.nytimes.com/2020/07/31/us/elections/biden-vs-trump.html?action=click\&pgtype=Article\&state=default\&region=MAIN_CONTENT_1\&context=storylines_live_updates}{Latest
Updates: 2020
Election}}{Latest Updates: 2020 Election}}\label{latest-updates-2020-election}}

Updated 2020-08-01T01:26:45.732Z

\begin{itemize}
\tightlist
\item
  \href{https://www.nytimes.com/2020/07/31/us/elections/biden-vs-trump.html?action=click\&pgtype=Article\&state=default\&region=MAIN_CONTENT_1\&context=storylines_live_updates\#link-29fdff45}{Kamala
  Harris, a top vice-presidential contender, confronts double
  standards.}
\item
  \href{https://www.nytimes.com/2020/07/31/us/elections/biden-vs-trump.html?action=click\&pgtype=Article\&state=default\&region=MAIN_CONTENT_1\&context=storylines_live_updates\#link-13ec3d9c}{Karen
  Bass and Susan Rice are rising on Biden's vice-presidential
  shortlist.}
\item
  \href{https://www.nytimes.com/2020/07/31/us/elections/biden-vs-trump.html?action=click\&pgtype=Article\&state=default\&region=MAIN_CONTENT_1\&context=storylines_live_updates\#link-49e9a016}{Trump
  says Russian bounties to kill U.S. troops `never took place.'}
\end{itemize}

\href{https://www.nytimes.com/2020/07/31/us/elections/biden-vs-trump.html?action=click\&pgtype=Article\&state=default\&region=MAIN_CONTENT_1\&context=storylines_live_updates}{See
more updates}

John F. Kennedy used the Cuban missile crisis to start the era of arms
control negotiations with the Soviets, and Ronald Reagan transitioned
from the hard-line anti-Communism of his party to doing business with a
reformist Kremlin leader, Mikhail S. Gorbachev. But Mr. Trump has never
wavered from a policy of praise and nonconfrontation. Just as he rarely
misses a campaign-season opportunity to take on Beijing, he has not
wavered from accommodating Moscow.

The absence of a strategy to alter Moscow's behavior at this electoral
inflection point ---
\href{https://www.nytimes.com/2016/07/27/us/politics/spy-agency-consensus-grows-that-russia-hacked-dnc.html}{four
years ago this week}, the C.I.A. was coming to the conclusion that
Russia was responsible for the hacking of the Democratic National
Committee's servers --- has been particularly evident in recent days.
After trying to raise doubts that Russia was behind the breach, the
release of emails and a social media influence campaign, Mr. Trump has
settled on a strategy of silence about evidence of an emerging new
Russian playbook.

So when he spoke with Mr. Putin recently, the White House gave no
indication that the warnings from the intelligence agencies and the
Department of Homeland Security even came up.

``At this very moment, Putin is presumably deciding how far to go in
interfering in the 2020 election,'' said David Shimer, a historian and
the author of ``Rigged: America, Russia, and One Hundred Years of Covert
Electoral Interference,'' a new study of Russian and American efforts to
influence elections around the globe. ``Leaders like Putin push as far
as they can without provoking meaningful pushback, and in that sense,
Trump's continued passivity toward Russia could embolden Putin to
proceed more aggressively.''

Not surprisingly, the administration rejects the notion that it has
given Mr. Putin free rein. Mr. Trump regularly says no American
president has been tougher on Russia than he has,
\href{https://thehill.com/homenews/administration/425034-trump-i-have-been-tougher-on-russia-that-any-other-president}{``maybe
tougher than any other president.''}

The president's advisers point out that Mr. Trump's own Justice
Department
\href{https://www.nytimes.com/2018/07/13/us/politics/mueller-indictment-russian-intelligence-hacking.html}{indicted
12 Russian intelligence officers} for breaking into the Democratic
National Committee and running the social media campaign --- though Mr.
Trump has questioned Russia's ability for both. Under authorities given
to it by the president, the director of the National Security Agency and
commander of United States Cyber Command, General Paul A. Nakasone,
briefly paralyzed the Internet Research Agency, a troll farm in St.
Petersburg, Russia, during the 2018 midterm elections to send a message.
(Mr. Trump later
said\href{https://www.nytimes.com/2020/07/11/us/politics/trump-russia-cyber-attack.html}{he
was responsible for the action.})

And Mr. Trump's secretary of state, Mike Pompeo, declared the United
States would never recognize Russia's annexation of Crimea --- ``Crimea
is Ukraine,'' he said on the sixth anniversary of the unilateral seizure
of the territory, not mentioning that Mr. Trump said
\href{https://www.nytimes.com/2016/03/27/us/politics/donald-trump-transcript.html}{in
an interview with The New York Times in 2016} that he did not understand
why the United States was penalizing Russia for events that primarily
affected allies far away.

But it is the withdrawal of troops from Germany, and the absence of any
response to the intelligence on the bounties offered to the Taliban for
killing Americans, that seems to encapsulate the administration's
absence of a strategy.

Mr. Pompeo struggled to offer up a defense on either in Senate testimony
on Thursday. He noted that as a newly-minted Army officer during the
final days of the Cold War, he himself ``fought on the border of East
Germany,'' leading Senator Jeanne Shaheen, Democrat of New Hampshire, to
note wryly that under Mr. Trump's orders ``your unit is coming back to
the United States.''

But Mr. Pompeo's testimony was more notable for what he failed to say.
He provided no strategic rationale for the reduction of 12,000 troops in
Germany, including 6,400 returning to the United States. He made the
case that they could return to Europe quickly, but never addressed the
fundamental issue: that the decision was part of presidential pique that
Chancellor Angela Merkel of Germany was not devoting a big enough
portion of the national budget to her nation's defense --- and that
reducing the American military presence in German fulfilled one of Mr.
Putin's greatest dreams.

``Germany is supposed to pay for it,'' Mr. Trump said of the American
presence, as if the forward deployment was not a central part of the
United States' own defense strategy for the past 75 years. ``Germany's
not paying for it. We don't want to be the suckers any more. The United
States has been taken advantage of for 25 years, both on trade and on
the military. So we're reducing the force because they're not paying
their bills.''

For four years, Mr. Trump has talked about ``NATO fees.'' There are no
such fees. Six years ago, alliance members agree to spend 2 percent of
their gross domestic product on their individual defense by 2024.
Germany spends 1.5 percent; Italy and Belgium, where the United States
says it is moving some of its forces, spend less.

Mr. Pompeo further muddied the waters on the intelligence surrounding
the bounties for American lives. Citing intelligence concerns, he would
not discuss the C.I.A. analysis that was included in a Presidential
Daily Brief in February that Mr. Trump says never reached his desk.

What is still missing is any statement of what the administration is
trying to accomplish with Russia. Getting Mr. Putin to back down? A new
détente? Modern containment of a declining but still disruptive nuclear
power? An end to use of its cyberpower to step into the middle of the
American elections?

Mr. Trump has had four years, and he hasn't said.

``There appears to be no overarching objective, no strategy for getting
there, no coherent policy process,'' Wendy R. Sherman, who served in the
State Department during the Obama and Clinton
administrations,\href{https://foreignpolicy.com/2020/07/31/trump-destruction-foreign-policy/}{wrote
in Foreign Policy on Friday}. ``There is no evidence, for example, of a
desire to preserve arms control with Russia or to stop Russia's (or any
other country's) persistent disinformation campaigns that are now
looming as an ever-larger threat to the integrity of the U.S.
election.''

\hypertarget{our-2020-election-guide}{%
\section{Our 2020 Election Guide}\label{our-2020-election-guide}}

Updated July 31, 2020

\begin{itemize}
\item
  \begin{center}\rule{0.5\linewidth}{\linethickness}\end{center}

  \hypertarget{the-latest}{%
  \subsection{The Latest}\label{the-latest}}

  \begin{itemize}
  \tightlist
  \item
    President Trump's assault on the Postal Service is intersecting with
    his attacks on mail-in voting.
    \href{https://www.nytimes.com/2020/07/31/us/politics/trump-usps-mail-delays.html?action=click\&pgtype=Article\&state=default\&region=BELOW_MAIN_CONTENT\&context=storylines_guide}{Voting
    rights groups say it is a recipe for disaster.}
  \end{itemize}
\item
  \begin{center}\rule{0.5\linewidth}{\linethickness}\end{center}

  \hypertarget{bidens-vp-search}{%
  \subsection{Biden's V.P. Search}\label{bidens-vp-search}}

  \begin{itemize}
  \tightlist
  \item
    \href{https://www.nytimes.com/article/biden-vice-president-2020.html?action=click\&pgtype=Article\&state=default\&region=BELOW_MAIN_CONTENT\&context=storylines_guide}{Here
    are 13 women} who have been under consideration to be Joe Biden's
    running mate, and why each might be chosen --- and might not be.
  \end{itemize}
\item
  \begin{center}\rule{0.5\linewidth}{\linethickness}\end{center}

  \hypertarget{keep-up-with-our-coverage}{%
  \subsection{Keep Up With Our
  Coverage}\label{keep-up-with-our-coverage}}

  \begin{itemize}
  \tightlist
  \item
    Get an
    \href{https://www.nytimes.com/newsletters/politics?action=click\&pgtype=Article\&state=default\&region=BELOW_MAIN_CONTENT\&context=storylines_guide}{email}
    recapping the day's news
  \end{itemize}

  \begin{itemize}
  \tightlist
  \item
    Download our mobile app on
    \href{https://apps.apple.com/us/app/nytimes/id284862083?ls=1\&mat_click_id=5c79ae7455014fd1bd66b5610c05b8f2-20191112-16948\&referrer=mat_click_id\%3D5c79ae7455014fd1bd66b5610c05b8f2-20191112-16948\%26link_click_id\%3D722930677036718082}{iOS}
    and
    \href{http://a.localytics.com/android?id=com.nytimes.android\&referrer=utm_source\%3Dother_nyt_mobile_web\%26utm_medium\%3DWeb\%2520page\%26utm_term\%3DGeneral\%2520Mobile\%2520Page\%26utm_campaign\%3DNYT\%2520Mobile\%2520General\%2520Page}{Android}
    and turn on Breaking News and Politics alerts
  \end{itemize}
\end{itemize}

Advertisement

\protect\hyperlink{after-bottom}{Continue reading the main story}

\hypertarget{site-index}{%
\subsection{Site Index}\label{site-index}}

\hypertarget{site-information-navigation}{%
\subsection{Site Information
Navigation}\label{site-information-navigation}}

\begin{itemize}
\tightlist
\item
  \href{https://help.nytimes.com/hc/en-us/articles/115014792127-Copyright-notice}{©~2020~The
  New York Times Company}
\end{itemize}

\begin{itemize}
\tightlist
\item
  \href{https://www.nytco.com/}{NYTCo}
\item
  \href{https://help.nytimes.com/hc/en-us/articles/115015385887-Contact-Us}{Contact
  Us}
\item
  \href{https://www.nytco.com/careers/}{Work with us}
\item
  \href{https://nytmediakit.com/}{Advertise}
\item
  \href{http://www.tbrandstudio.com/}{T Brand Studio}
\item
  \href{https://www.nytimes.com/privacy/cookie-policy\#how-do-i-manage-trackers}{Your
  Ad Choices}
\item
  \href{https://www.nytimes.com/privacy}{Privacy}
\item
  \href{https://help.nytimes.com/hc/en-us/articles/115014893428-Terms-of-service}{Terms
  of Service}
\item
  \href{https://help.nytimes.com/hc/en-us/articles/115014893968-Terms-of-sale}{Terms
  of Sale}
\item
  \href{https://spiderbites.nytimes.com}{Site Map}
\item
  \href{https://help.nytimes.com/hc/en-us}{Help}
\item
  \href{https://www.nytimes.com/subscription?campaignId=37WXW}{Subscriptions}
\end{itemize}
