Sections

SEARCH

\protect\hyperlink{site-content}{Skip to
content}\protect\hyperlink{site-index}{Skip to site index}

\href{https://www.nytimes.com/section/arts/music}{Music}

\href{https://myaccount.nytimes.com/auth/login?response_type=cookie\&client_id=vi}{}

\href{https://www.nytimes.com/section/todayspaper}{Today's Paper}

\href{/section/arts/music}{Music}\textbar{}Billie Eilish's Isolation
Awakening, and 8 More New Songs

\begin{itemize}
\item
\item
\item
\item
\item
\end{itemize}

\href{https://www.nytimes.com/spotlight/at-home?action=click\&pgtype=Article\&state=default\&region=TOP_BANNER\&context=at_home_menu}{At
Home}

\begin{itemize}
\tightlist
\item
  \href{https://www.nytimes.com/2020/07/28/books/time-for-a-literary-road-trip.html?action=click\&pgtype=Article\&state=default\&region=TOP_BANNER\&context=at_home_menu}{Take:
  A Literary Road Trip}
\item
  \href{https://www.nytimes.com/2020/07/29/magazine/bored-with-your-home-cooking-some-smoky-eggplant-will-fix-that.html?action=click\&pgtype=Article\&state=default\&region=TOP_BANNER\&context=at_home_menu}{Cook:
  Smoky Eggplant}
\item
  \href{https://www.nytimes.com/2020/07/27/travel/moose-michigan-isle-royale.html?action=click\&pgtype=Article\&state=default\&region=TOP_BANNER\&context=at_home_menu}{Look
  Out: For Moose}
\item
  \href{https://www.nytimes.com/interactive/2020/at-home/even-more-reporters-editors-diaries-lists-recommendations.html?action=click\&pgtype=Article\&state=default\&region=TOP_BANNER\&context=at_home_menu}{Explore:
  Reporters' Obsessions}
\end{itemize}

Advertisement

\protect\hyperlink{after-top}{Continue reading the main story}

Supported by

\protect\hyperlink{after-sponsor}{Continue reading the main story}

The Playlist

\hypertarget{billie-eilishs-isolation-awakening-and-8-more-new-songs}{%
\section{Billie Eilish's Isolation Awakening, and 8 More New
Songs}\label{billie-eilishs-isolation-awakening-and-8-more-new-songs}}

Hear tracks by Laura Veirs, A.G. Cook, Bill Frisell and others.

\includegraphics{https://static01.nyt.com/images/2020/07/31/arts/31playlist/31playlist-articleLarge.jpg?quality=75\&auto=webp\&disable=upscale}

By \href{https://www.nytimes.com/by/jon-pareles}{Jon Pareles},
\href{https://www.nytimes.com/by/jon-caramanica}{Jon Caramanica} and
\href{https://www.nytimes.com/by/giovanni-russonello}{Giovanni
Russonello}

\begin{itemize}
\item
  July 31, 2020
\item
  \begin{itemize}
  \item
  \item
  \item
  \item
  \item
  \end{itemize}
\end{itemize}

\emph{Every Friday, pop critics for The New York Times weigh in on the
week's most notable new songs and videos. Just want the music?}
\href{https://open.spotify.com/playlist/5S3JLpPLj9BsLFyaVHf8Nf?si=h85ZkCLVT7e_XvKKNuBdPQ}{\emph{Listen
to the Playlist on Spotify here}} \emph{(or find our profile: nytimes).
Like what you hear? Let us know at}
\href{mailto:theplaylist@nytimes.com}{\emph{theplaylist@nytimes.com}}
\emph{and}
\href{https://www.nytimes.com/newsletters/louder?module=inline}{\emph{sign
up for our Louder newsletter}}\emph{, a once-a-week blast of our pop
music coverage.}

\hypertarget{billie-eilish-my-future}{%
\subsection{Billie Eilish, `My Future'}\label{billie-eilish-my-future}}

``I'm in love but not with anybody else/Just want to get to know
myself,'' Billie Eilish sings on ``My Future,'' her first
self-isolation-era single --- everyone's quarantine awakening should be
so wise. Eilish's ode to loving oneself is both textured and
uncomplicated. For a full minute and a half, she leans into her crooner
side, singing deep exhales with heavy flutter. The ambience echoes the
astral, roomy R\&B of the Internet and Steve Lacy, and even when the
song kicks into something slightly more zippy, Eilish's ease is the
dominant mode --- worrying about yourself first makes for no worry at
all. JON CARAMANICA

\hypertarget{snakehips-and-jess-glynne-featuring-a-boogie-wit-da-hoodie-and-davido-lie-for-you}{%
\subsection{Snakehips and Jess Glynne featuring A Boogie Wit da Hoodie
and Davido, `Lie for
You'}\label{snakehips-and-jess-glynne-featuring-a-boogie-wit-da-hoodie-and-davido-lie-for-you}}

British dance music producers working with a British singer, an American
sing-rapper and a Nigerian Afrobeats star --- in theory, collaborations
like these are a streaming-era blessing. Music like this accelerates
cross-genre, cross-border conversations, and helps export local sounds
globally. In reality, though, what begin as regional particularities ---
unique sonic identifiers and sales pitches --- end up smeared together
so intensely here that the implicit argument ends up being that these
styles were all the same to begin with. CARAMANICA

\hypertarget{subculture-featuring-rachel-chinouriri-the-river-bend}{%
\subsection{Subculture featuring Rachel Chinouriri, `The River
Bend'}\label{subculture-featuring-rachel-chinouriri-the-river-bend}}

Rachel Chinouriri's voice goes slinking through a jazzy minefield in
``River Bend,'' a 2019 song from the British producer Subculture about
urban paranoia that was just rereleased with a video. The bass line
repeats and then multiplies, leaping around the low register as other
instruments and sounds materialize and vanish with equal suddenness:
cowbell, slide guitar, wordless voices, trumpet. She's surrounded by
phantom threats, with nothing to count on. JON PARELES

\hypertarget{laura-veirs-burn-too-bright}{%
\subsection{Laura Veirs, `Burn Too
Bright'}\label{laura-veirs-burn-too-bright}}

``Burn Too Bright'' is an up-tempo elegy for
\href{https://pitchfork.com/thepitch/why-richard-swift-was-an-indie-rock-treasure/}{Richard
Swift}, a prolific Pacific Northwest songwriter, producer and backup
musician who was 41 when he died of hepatitis in 2018 after a long
alcohol addiction. ``Who were you running from?/It was yourself all
right,'' Veirs sings amid crosscurrents of strings, electric guitars and
a teakettle-whistle synthesizer, sympathetic but also curious about how
well-loved musicianship was not enough. PARELES

\hypertarget{mina-tindle-featuring-sufjan-stevens-give-a-little-love}{%
\subsection{Mina Tindle featuring Sufjan Stevens, `Give a Little
Love'}\label{mina-tindle-featuring-sufjan-stevens-give-a-little-love}}

Sufjan Stevens wrote, produced, arranged and sang wispy backup vocals
for ``Give a Little Love'' by Mina Tindle, a.k.a. the Parisian singer
and songwriter Pauline De Lassus, who has also sung backup for the
National. It's from the watercolor zone of Stevens's catalog, made even
more diaphanous by De Lassus's guileless voice. In the first half of the
song, she realizes ``I'm all alone'' in music with a subliminally
unstable seven-beat meter. In the second, she quietly calls for what she
needs --- ``Give a little bit of your heart'' --- as the music settles
into 4/4 and she and Stevens overdub themselves into a supportive choir.
PARELES

\hypertarget{randy-travis-fools-love-affair}{%
\subsection{Randy Travis, `Fool's Love
Affair'}\label{randy-travis-fools-love-affair}}

Randy Travis was always a master of stoic, moral songs, even the ones
about misbehavior. So it is with ``Fool's Love Affair,'' his first new
song in years. Based on a demo Travis recorded in the early 1980s ---
way before his 2013 stroke, which has left him with aphasia --- it's a
reassuringly sturdy jolt of traditionalism. His grip on regret is firm,
so tight he makes it sound like decency. CARAMANICA

\hypertarget{bill-frisell-valentine}{%
\subsection{Bill Frisell, `Valentine'}\label{bill-frisell-valentine}}

The guitarist Bill Frisell tends to build his improvisations around warm
harmonic intervals and circular gestures --- not exactly the jagged
dissonances and sharp jabs of Thelonious Monk. But Monk's influence is
subtly written into a lot of Frisell's music, and it comes to the
surface on ``Valentine,'' a 12-bar blues the guitarist wrote with a
Monkish, asymmetrical wobble. In Monk's band, the other musicians
usually held fast to a swinging foundation while he dashed off
idiosyncrasies. But the members of Frisell's trio, the drummer Rudy
Royston and the bassist Thomas Morgan, take a loosened approach,
scattering and scrambling around the melody. ``Valentine'' is the title
track from this trio's first album together, due out on Blue Note on
Aug. 14. GIOVANNI RUSSONELLO

\hypertarget{ag-cook-7g-7-minute-mix}{%
\subsection{A.G. Cook, `7G (7 Minute
Mix)'}\label{ag-cook-7g-7-minute-mix}}

A.G. Cook, the metapop-loving, ultra-shiny electronics auteur who
founded PC Music, plans to release a seven-disc, 49-track collection in
August with each disc devoted to an instrument or concept --- i.e.,
drums, piano, ``extreme vocals'' --- called ``7G.'' This is a
seven-minute preview with snippets of a track from each disc --- mostly
brittle and artificial, occasionally abrasive or perky. For some people,
it'll be a promising teaser; for others, it might be all they need.
PARELES

\hypertarget{charles-tolliver-emperor-march}{%
\subsection{Charles Tolliver, `Emperor
March'}\label{charles-tolliver-emperor-march}}

The trumpeter Charles Tolliver's compositions tend to balance syncopated
elements as if they were structural beams. The drums, the piano and
bass, the horns: Each scribbles out its own pattern, contrasting with
the others and occasionally uniting. ``Connect,'' Tolliver's first album
in over a decade, stands out in large part based on the quality of his
compositions. The caliber of his company helps too: Across the LP's four
long tracks, Tolliver is joined by the immortal Buster Williams on bass
and Lenny White on drums, among others. Soloing on ``Emperor March,''
Tolliver, 78, steers neatly through the song's multipart form, starting
out sounding pebbly and piecemeal, ending up spinning triumphant blues
variations. RUSSONELLO

Advertisement

\protect\hyperlink{after-bottom}{Continue reading the main story}

\hypertarget{site-index}{%
\subsection{Site Index}\label{site-index}}

\hypertarget{site-information-navigation}{%
\subsection{Site Information
Navigation}\label{site-information-navigation}}

\begin{itemize}
\tightlist
\item
  \href{https://help.nytimes.com/hc/en-us/articles/115014792127-Copyright-notice}{©~2020~The
  New York Times Company}
\end{itemize}

\begin{itemize}
\tightlist
\item
  \href{https://www.nytco.com/}{NYTCo}
\item
  \href{https://help.nytimes.com/hc/en-us/articles/115015385887-Contact-Us}{Contact
  Us}
\item
  \href{https://www.nytco.com/careers/}{Work with us}
\item
  \href{https://nytmediakit.com/}{Advertise}
\item
  \href{http://www.tbrandstudio.com/}{T Brand Studio}
\item
  \href{https://www.nytimes.com/privacy/cookie-policy\#how-do-i-manage-trackers}{Your
  Ad Choices}
\item
  \href{https://www.nytimes.com/privacy}{Privacy}
\item
  \href{https://help.nytimes.com/hc/en-us/articles/115014893428-Terms-of-service}{Terms
  of Service}
\item
  \href{https://help.nytimes.com/hc/en-us/articles/115014893968-Terms-of-sale}{Terms
  of Sale}
\item
  \href{https://spiderbites.nytimes.com}{Site Map}
\item
  \href{https://help.nytimes.com/hc/en-us}{Help}
\item
  \href{https://www.nytimes.com/subscription?campaignId=37WXW}{Subscriptions}
\end{itemize}
