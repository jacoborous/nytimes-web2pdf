Sections

SEARCH

\protect\hyperlink{site-content}{Skip to
content}\protect\hyperlink{site-index}{Skip to site index}

\href{https://www.nytimes.com/section/books/review}{Book Review}

\href{https://myaccount.nytimes.com/auth/login?response_type=cookie\&client_id=vi}{}

\href{https://www.nytimes.com/section/todayspaper}{Today's Paper}

\href{/section/books/review}{Book Review}\textbar{}Monsters vs. Aliens

\url{https://nyti.ms/313YB14}

\begin{itemize}
\item
\item
\item
\item
\item
\end{itemize}

Advertisement

\protect\hyperlink{after-top}{Continue reading the main story}

Supported by

\protect\hyperlink{after-sponsor}{Continue reading the main story}

\href{/column/childrens-books}{Children's Books}

\hypertarget{monsters-vs-aliens}{%
\section{Monsters vs. Aliens}\label{monsters-vs-aliens}}

\includegraphics{https://static01.nyt.com/images/2020/08/02/books/02BKS-SHER-KIDS/02BKS-SHER-KIDS-articleLarge.jpg?quality=75\&auto=webp\&disable=upscale}

Buy Book ▾

\begin{itemize}
\tightlist
\item
  \href{https://www.amazon.com/gp/search?index=books\&tag=NYTBSREV-20\&field-keywords=Sia+Martinez+and+the+Moonlit+Beginning+of+Everything+Raquel+Vasquez+Gilliland}{Amazon}
\item
  \href{https://du-gae-books-dot-nyt-du-prd.appspot.com/buy?title=Sia+Martinez+and+the+Moonlit+Beginning+of+Everything\&author=Raquel+Vasquez+Gilliland}{Apple
  Books}
\item
  \href{https://www.anrdoezrs.net/click-7990613-11819508?url=https\%3A\%2F\%2Fwww.barnesandnoble.com\%2Fw\%2F\%3Fean\%3D9781534448650}{Barnes
  and Noble}
\item
  \href{https://www.anrdoezrs.net/click-7990613-35140?url=https\%3A\%2F\%2Fwww.booksamillion.com\%2Fp\%2FSia\%2BMartinez\%2Band\%2Bthe\%2BMoonlit\%2BBeginning\%2Bof\%2BEverything\%2FRaquel\%2BVasquez\%2BGilliland\%2F9781534448650}{Books-A-Million}
\item
  \href{https://bookshop.org/a/3546/9781534448650}{Bookshop}
\item
  \href{https://www.indiebound.org/book/9781534448650?aff=NYT}{Indiebound}
\end{itemize}

When you purchase an independently reviewed book through our site, we
earn an affiliate commission.

By Abby Sher

\begin{itemize}
\item
  July 31, 2020
\item
  \begin{itemize}
  \item
  \item
  \item
  \item
  \item
  \end{itemize}
\end{itemize}

Sia Martinez is trying so hard to believe in miracles.

It's been three years since her mom was taken away by ICE; almost two
since she was pronounced dead, even though there was no corpse to prove
it. All Sia has left are endless questions, the moonlit desert and the
tales her abuela told her.

\textbf{SIA MARTINEZ AND THE MOONLIT BEGINNING OF EVERYTHING (Simon
Pulse, 432 pp., \$18.99; ages 12 and up),} by Raquel Vasquez Gilliland,
is a story about searching for answers. Sia needs not only to find out
what happened to her mom, but also to confront her personal trauma, her
evolving sexuality and her wavering faith.

What's clear from the start is Sia's love and admiration for her family
and the earth. She and her father live in Arizona, next to the Sonoran
Desert, where ``the indigo of the night sky, the line of hiplike
mountains, even the cacti themselves'' call to her.

Sia is proud of her lineage of strong women who grew \emph{maiz} and
studied the land. She's determined, in turn, to make her ancestors
proud, to light candles in the desert at night both to honor her late
grandmother and to guide her mother's spirit home.

One night Sia spots mysterious blue orbs on the horizon. In all her
years of sky-gazing, she's never seen anything like them. She keeps
visiting the desert, hungry to learn more.

When the blue lights return, they get bigger and bigger, until Sia
realizes they are some sort of spacecraft, which crashes right in front
of her in the sand. Out of the wreckage crawls her mom.

Gilliland has woven together many different genres in this fascinating
debut novel --- romance, sci-fi, Mexican folklore --- all against the
backdrop of horrifically true events that have taken place at the border
in the past few years.

Her previous work as a poet and visual artist serves her well,
especially when she's describing endless stretches of desert and sky.
Her attention to light, shadow, taste, smell and sound are remarkable,
making a cricket's song seem magical. Her language exhibits a stunning
fluidity, depicting time and space and even mortality as a sort of
continuum.

Sia is a very authentic and mature 17-year-old. Her doubts, fears and
anger at her mom for trekking across the desert feel true to a young
adult.

But this book is for adults as much as it is for teenagers. Gilliland
includes a note at the beginning, warning readers that it contains
sexual assault, PTSD, physical abuse, parental death and racist
violence.

She challenges her characters and her readers to explore faith,
forgiveness and the possibility of alternate realities. Whether it's the
concept of the dead speaking to us through holes in the sky or the
creepily weird notion of cutting human blood with alien blood for
experimentation, there is no shortage of fantastical images.

At times, they can be hard to reconcile with systemic racism and the
very real tragedy of immigrant children being torn from their parents.
At others, they seem like apt metaphors.

Sia has a deep respect for the sanctity of all species; she reminds us
that we are interconnected. Humans, animals and plants should be equally
revered as part of a universal heritage we have yet to truly understand.

When she sets out to find her mom in the desert, she's inviting us to
search and hope, too.

She's teaching us about the everyday miracles that surround us if we
open ourselves to the experience. She's asking us to see ourselves as
part of a greater whole, so we can learn how to honor and respect all
beings.

As Sia muses, ``Taking walks in the starlight makes our senses raw. And
we can hear and see and feel our ancestors. They're always among us,
traveling back and forth by starlight. It's a kind of magic.''

In a world where we are so often dividing ourselves into us and them,
this book feels like a kind of magic, too, celebrating all beliefs,
ethnicities and unknowns.

Advertisement

\protect\hyperlink{after-bottom}{Continue reading the main story}

\hypertarget{site-index}{%
\subsection{Site Index}\label{site-index}}

\hypertarget{site-information-navigation}{%
\subsection{Site Information
Navigation}\label{site-information-navigation}}

\begin{itemize}
\tightlist
\item
  \href{https://help.nytimes.com/hc/en-us/articles/115014792127-Copyright-notice}{©~2020~The
  New York Times Company}
\end{itemize}

\begin{itemize}
\tightlist
\item
  \href{https://www.nytco.com/}{NYTCo}
\item
  \href{https://help.nytimes.com/hc/en-us/articles/115015385887-Contact-Us}{Contact
  Us}
\item
  \href{https://www.nytco.com/careers/}{Work with us}
\item
  \href{https://nytmediakit.com/}{Advertise}
\item
  \href{http://www.tbrandstudio.com/}{T Brand Studio}
\item
  \href{https://www.nytimes.com/privacy/cookie-policy\#how-do-i-manage-trackers}{Your
  Ad Choices}
\item
  \href{https://www.nytimes.com/privacy}{Privacy}
\item
  \href{https://help.nytimes.com/hc/en-us/articles/115014893428-Terms-of-service}{Terms
  of Service}
\item
  \href{https://help.nytimes.com/hc/en-us/articles/115014893968-Terms-of-sale}{Terms
  of Sale}
\item
  \href{https://spiderbites.nytimes.com}{Site Map}
\item
  \href{https://help.nytimes.com/hc/en-us}{Help}
\item
  \href{https://www.nytimes.com/subscription?campaignId=37WXW}{Subscriptions}
\end{itemize}
