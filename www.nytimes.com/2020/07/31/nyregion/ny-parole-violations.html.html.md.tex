Sections

SEARCH

\protect\hyperlink{site-content}{Skip to
content}\protect\hyperlink{site-index}{Skip to site index}

\href{https://www.nytimes.com/section/nyregion}{New York}

\href{https://myaccount.nytimes.com/auth/login?response_type=cookie\&client_id=vi}{}

\href{https://www.nytimes.com/section/todayspaper}{Today's Paper}

\href{/section/nyregion}{New York}\textbar{}Should N.Y. Be Jailing
Parolees for Minor Lapses During a Pandemic?

\url{https://nyti.ms/3fjBapo}

\begin{itemize}
\item
\item
\item
\item
\item
\item
\end{itemize}

\href{https://www.nytimes.com/news-event/coronavirus?action=click\&pgtype=Article\&state=default\&region=TOP_BANNER\&context=storylines_menu}{The
Coronavirus Outbreak}

\begin{itemize}
\tightlist
\item
  live\href{https://www.nytimes.com/2020/08/01/world/coronavirus-covid-19.html?action=click\&pgtype=Article\&state=default\&region=TOP_BANNER\&context=storylines_menu}{Latest
  Updates}
\item
  \href{https://www.nytimes.com/interactive/2020/us/coronavirus-us-cases.html?action=click\&pgtype=Article\&state=default\&region=TOP_BANNER\&context=storylines_menu}{Maps
  and Cases}
\item
  \href{https://www.nytimes.com/interactive/2020/science/coronavirus-vaccine-tracker.html?action=click\&pgtype=Article\&state=default\&region=TOP_BANNER\&context=storylines_menu}{Vaccine
  Tracker}
\item
  \href{https://www.nytimes.com/interactive/2020/07/29/us/schools-reopening-coronavirus.html?action=click\&pgtype=Article\&state=default\&region=TOP_BANNER\&context=storylines_menu}{What
  School May Look Like}
\item
  \href{https://www.nytimes.com/live/2020/07/31/business/stock-market-today-coronavirus?action=click\&pgtype=Article\&state=default\&region=TOP_BANNER\&context=storylines_menu}{Economy}
\end{itemize}

Advertisement

\protect\hyperlink{after-top}{Continue reading the main story}

Supported by

\protect\hyperlink{after-sponsor}{Continue reading the main story}

BIG CITY

\hypertarget{should-ny-be-jailing-parolees-for-minor-lapses-during-a-pandemic}{%
\section{Should N.Y. Be Jailing Parolees for Minor Lapses During a
Pandemic?}\label{should-ny-be-jailing-parolees-for-minor-lapses-during-a-pandemic}}

On probation since 2018, Earl Russell was sent to Rikers for sleeping in
his own bed instead of in the shelter where he was mandated to stay.

\includegraphics{https://static01.nyt.com/images/2020/07/31/nyregion/31big/merlin_170843811_2dfbdd8b-c335-4e25-a2d7-1d67e70a72e2-articleLarge.jpg?quality=75\&auto=webp\&disable=upscale}

\href{https://www.nytimes.com/by/ginia-bellafante}{\includegraphics{https://static01.nyt.com/images/2018/12/06/multimedia/author-ginia-bellafante/author-ginia-bellafante-thumbLarge.png}}

By \href{https://www.nytimes.com/by/ginia-bellafante}{Ginia Bellafante}

\begin{itemize}
\item
  July 31, 2020
\item
  \begin{itemize}
  \item
  \item
  \item
  \item
  \item
  \item
  \end{itemize}
\end{itemize}

In December, before most of us had directed our attention to the looming
terrors of the coronavirus, Earl Russell was already getting
apprehensive. At 42, he struggled with high blood pressure and was
living in a men's shelter in Brooklyn. Most people in shelters are there
because all their other housing options have run dry, but Mr. Russell
had somewhere to go --- an apartment in the Rockaways where his
girlfriend lived with their 6-year-old daughter, both of whom wanted him
home.

The bizarre vagaries of New York state's parole system were making it
impossible for him to join them, however. Returning to his family would
have been a violation of the terms of his prison release --- an action
punishable with jail time.

When Mr. Russell was paroled in 2018, after two years in prison on a
weapons-possession charge in the second degree, he was remanded to the
shelter system, where he was to remain until the fall of 2021, even
though he would be needlessly taking up space in the midst of the city's
ongoing and epic housing emergencies.

At the end of March, in an effort to curtail the spread of Covid-19
among the incarcerated, Gov. Andrew M. Cuomo announced plans to release
1,100 people from jails around the state, who were being held for
technical violations of their parole. These infractions are not crimes;
they include missing curfew, failing to contact parole officers at
designated times, testing positive for alcohol and living at addresses
other than the ones compelled, among others.

Two months later, 379 of these technical parole violators had been
released from jails in New York City. But their ranks were soon
replaced. Mr. Russell was among those who found themselves at Rikers,
detained for what logic would concede were minor infractions..

Since March 27, according to new admissions data from the
\href{https://www.vera.org/}{Vera Institute of Justice}, 295 people have
returned to city jails because of low-level parole transgressions. The
first two people to die of Covid-19 on Rikers were there precisely for
these reasons: missing parole appointments and failing a drug program.
One of them, a man named
\href{https://www.nytimes.com/2020/04/09/nyregion/rikers-coronavirus-deaths-parolees.html}{Raymond
Rivera}, had waited months for a final decision on his release and died
the day after it was rendered.

Already, by the end of May,
\href{https://justicelab.columbia.edu/two-months-later}{160 people had
been newly sent to Rikers} on technicalities, leading Vincent Schiraldi,
a director of Columbia University's Justice Lab and the city's former
probation commissioner, to correctly predict that some of the gains made
by the governor's edict would essentially be erased by the middle of the
summer.

\hypertarget{latest-updates-global-coronavirus-outbreak}{%
\section{\texorpdfstring{\href{https://www.nytimes.com/2020/08/01/world/coronavirus-covid-19.html?action=click\&pgtype=Article\&state=default\&region=MAIN_CONTENT_1\&context=storylines_live_updates}{Latest
Updates: Global Coronavirus
Outbreak}}{Latest Updates: Global Coronavirus Outbreak}}\label{latest-updates-global-coronavirus-outbreak}}

Updated 2020-08-02T06:58:18.835Z

\begin{itemize}
\tightlist
\item
  \href{https://www.nytimes.com/2020/08/01/world/coronavirus-covid-19.html?action=click\&pgtype=Article\&state=default\&region=MAIN_CONTENT_1\&context=storylines_live_updates\#link-34047410}{The
  U.S. reels as July cases more than double the total of any other
  month.}
\item
  \href{https://www.nytimes.com/2020/08/01/world/coronavirus-covid-19.html?action=click\&pgtype=Article\&state=default\&region=MAIN_CONTENT_1\&context=storylines_live_updates\#link-780ec966}{Top
  U.S. officials work to break an impasse over the federal jobless
  benefit.}
\item
  \href{https://www.nytimes.com/2020/08/01/world/coronavirus-covid-19.html?action=click\&pgtype=Article\&state=default\&region=MAIN_CONTENT_1\&context=storylines_live_updates\#link-2bc8948}{Its
  outbreak untamed, Melbourne goes into even greater lockdown.}
\end{itemize}

\href{https://www.nytimes.com/2020/08/01/world/coronavirus-covid-19.html?action=click\&pgtype=Article\&state=default\&region=MAIN_CONTENT_1\&context=storylines_live_updates}{See
more updates}

More live coverage:
\href{https://www.nytimes.com/live/2020/07/31/business/stock-market-today-coronavirus?action=click\&pgtype=Article\&state=default\&region=MAIN_CONTENT_1\&context=storylines_live_updates}{Markets}

``It was bad policy to be incarcerating so many people for noncriminal
violations even before the pandemic,'' Mr. Schiraldi told me. ``It's
absolutely ridiculous and dangerous now.''

The policy has been in place for decades. When asked for the rationale
behind it, the State Department of Corrections did not respond for
comment. In the spring, as the decision was made to refrain from issuing
arrest warrants for those who had broken the less egregious rules of
parole, certain exceptions were carved out for those considered at high
risk for offending behaviors.

Given that incidents of domestic violence were sure to rise during the
pandemic, anyone who made contact with a partner he or she had abused in
the past was eligible for re-arrest, an official in the Cuomo
administration explained. So too, was a sex offender who made contact
with a minor.

The modern parole system began in New York State in the 19th century as
a means of helping ex-convicts adjust to society. Parole officers were
volunteers; the whole idea was rooted in a benevolent paternalism.

But in the 1970s, as crime escalated, the system became more punitive
.tThe faith that people who did bad things could transform began to
wane, and the goal shifted to preventing recidivism. While the state has
succeeded in detaining far fewer people on technicalities in recent
years, the policy, exercised at the discretion of corrections officials,
nevertheless remains.

In a speech two years ago, Governor Cuomo acknowledged the problems
inherent in it. ``Jails and prisons should not be filled with people who
may have violated the conditions of their parole but present no danger
to our communities,'' he said. New York State spends hundreds of
millions of dollars each year enforcing these violations. ``Cops troll
hospitals looking for violators,'' Mr. Schiraldi said. ``While the guy
is dying, they slap a technical on him. At a certain point you routinize
the deprivation of people's liberties so much, you're just checking a
box.''

That routinization comes with all the predictable disparities. In New
York City, \href{https://www.katalcenter.org/lessismoreny}{Black people
re-enter the jail system on these technicalities at a rate more than 12
times that of whites.}

Before Christmas, Mr. Russell, who had already violated his parole on
other occasions by leaving the shelter and going home to his family,
sent a text to his parole officer explaining that he could not tolerate
his circumstance any longer and might as well be in jail. ``With this
being said,'' he wrote her, ``send me back if that's what you want to do
because I'm not returning to the shelter.'' He helpfully provided his
home address.

\href{https://www.nytimes.com/news-event/coronavirus?action=click\&pgtype=Article\&state=default\&region=MAIN_CONTENT_3\&context=storylines_faq}{}

\hypertarget{the-coronavirus-outbreak-}{%
\subsubsection{The Coronavirus Outbreak
›}\label{the-coronavirus-outbreak-}}

\hypertarget{frequently-asked-questions}{%
\paragraph{Frequently Asked
Questions}\label{frequently-asked-questions}}

Updated July 27, 2020

\begin{itemize}
\item ~
  \hypertarget{should-i-refinance-my-mortgage}{%
  \paragraph{Should I refinance my
  mortgage?}\label{should-i-refinance-my-mortgage}}

  \begin{itemize}
  \tightlist
  \item
    \href{https://www.nytimes.com/article/coronavirus-money-unemployment.html?action=click\&pgtype=Article\&state=default\&region=MAIN_CONTENT_3\&context=storylines_faq}{It
    could be a good idea,} because mortgage rates have
    \href{https://www.nytimes.com/2020/07/16/business/mortgage-rates-below-3-percent.html?action=click\&pgtype=Article\&state=default\&region=MAIN_CONTENT_3\&context=storylines_faq}{never
    been lower.} Refinancing requests have pushed mortgage applications
    to some of the highest levels since 2008, so be prepared to get in
    line. But defaults are also up, so if you're thinking about buying a
    home, be aware that some lenders have tightened their standards.
  \end{itemize}
\item ~
  \hypertarget{what-is-school-going-to-look-like-in-september}{%
  \paragraph{What is school going to look like in
  September?}\label{what-is-school-going-to-look-like-in-september}}

  \begin{itemize}
  \tightlist
  \item
    It is unlikely that many schools will return to a normal schedule
    this fall, requiring the grind of
    \href{https://www.nytimes.com/2020/06/05/us/coronavirus-education-lost-learning.html?action=click\&pgtype=Article\&state=default\&region=MAIN_CONTENT_3\&context=storylines_faq}{online
    learning},
    \href{https://www.nytimes.com/2020/05/29/us/coronavirus-child-care-centers.html?action=click\&pgtype=Article\&state=default\&region=MAIN_CONTENT_3\&context=storylines_faq}{makeshift
    child care} and
    \href{https://www.nytimes.com/2020/06/03/business/economy/coronavirus-working-women.html?action=click\&pgtype=Article\&state=default\&region=MAIN_CONTENT_3\&context=storylines_faq}{stunted
    workdays} to continue. California's two largest public school
    districts --- Los Angeles and San Diego --- said on July 13, that
    \href{https://www.nytimes.com/2020/07/13/us/lausd-san-diego-school-reopening.html?action=click\&pgtype=Article\&state=default\&region=MAIN_CONTENT_3\&context=storylines_faq}{instruction
    will be remote-only in the fall}, citing concerns that surging
    coronavirus infections in their areas pose too dire a risk for
    students and teachers. Together, the two districts enroll some
    825,000 students. They are the largest in the country so far to
    abandon plans for even a partial physical return to classrooms when
    they reopen in August. For other districts, the solution won't be an
    all-or-nothing approach.
    \href{https://bioethics.jhu.edu/research-and-outreach/projects/eschool-initiative/school-policy-tracker/}{Many
    systems}, including the nation's largest, New York City, are
    devising
    \href{https://www.nytimes.com/2020/06/26/us/coronavirus-schools-reopen-fall.html?action=click\&pgtype=Article\&state=default\&region=MAIN_CONTENT_3\&context=storylines_faq}{hybrid
    plans} that involve spending some days in classrooms and other days
    online. There's no national policy on this yet, so check with your
    municipal school system regularly to see what is happening in your
    community.
  \end{itemize}
\item ~
  \hypertarget{is-the-coronavirus-airborne}{%
  \paragraph{Is the coronavirus
  airborne?}\label{is-the-coronavirus-airborne}}

  \begin{itemize}
  \tightlist
  \item
    The coronavirus
    \href{https://www.nytimes.com/2020/07/04/health/239-experts-with-one-big-claim-the-coronavirus-is-airborne.html?action=click\&pgtype=Article\&state=default\&region=MAIN_CONTENT_3\&context=storylines_faq}{can
    stay aloft for hours in tiny droplets in stagnant air}, infecting
    people as they inhale, mounting scientific evidence suggests. This
    risk is highest in crowded indoor spaces with poor ventilation, and
    may help explain super-spreading events reported in meatpacking
    plants, churches and restaurants.
    \href{https://www.nytimes.com/2020/07/06/health/coronavirus-airborne-aerosols.html?action=click\&pgtype=Article\&state=default\&region=MAIN_CONTENT_3\&context=storylines_faq}{It's
    unclear how often the virus is spread} via these tiny droplets, or
    aerosols, compared with larger droplets that are expelled when a
    sick person coughs or sneezes, or transmitted through contact with
    contaminated surfaces, said Linsey Marr, an aerosol expert at
    Virginia Tech. Aerosols are released even when a person without
    symptoms exhales, talks or sings, according to Dr. Marr and more
    than 200 other experts, who
    \href{https://academic.oup.com/cid/article/doi/10.1093/cid/ciaa939/5867798}{have
    outlined the evidence in an open letter to the World Health
    Organization}.
  \end{itemize}
\item ~
  \hypertarget{what-are-the-symptoms-of-coronavirus}{%
  \paragraph{What are the symptoms of
  coronavirus?}\label{what-are-the-symptoms-of-coronavirus}}

  \begin{itemize}
  \tightlist
  \item
    Common symptoms
    \href{https://www.nytimes.com/article/symptoms-coronavirus.html?action=click\&pgtype=Article\&state=default\&region=MAIN_CONTENT_3\&context=storylines_faq}{include
    fever, a dry cough, fatigue and difficulty breathing or shortness of
    breath.} Some of these symptoms overlap with those of the flu,
    making detection difficult, but runny noses and stuffy sinuses are
    less common.
    \href{https://www.nytimes.com/2020/04/27/health/coronavirus-symptoms-cdc.html?action=click\&pgtype=Article\&state=default\&region=MAIN_CONTENT_3\&context=storylines_faq}{The
    C.D.C. has also} added chills, muscle pain, sore throat, headache
    and a new loss of the sense of taste or smell as symptoms to look
    out for. Most people fall ill five to seven days after exposure, but
    symptoms may appear in as few as two days or as many as 14 days.
  \end{itemize}
\item ~
  \hypertarget{does-asymptomatic-transmission-of-covid-19-happen}{%
  \paragraph{Does asymptomatic transmission of Covid-19
  happen?}\label{does-asymptomatic-transmission-of-covid-19-happen}}

  \begin{itemize}
  \tightlist
  \item
    So far, the evidence seems to show it does. A widely cited
    \href{https://www.nature.com/articles/s41591-020-0869-5}{paper}
    published in April suggests that people are most infectious about
    two days before the onset of coronavirus symptoms and estimated that
    44 percent of new infections were a result of transmission from
    people who were not yet showing symptoms. Recently, a top expert at
    the World Health Organization stated that transmission of the
    coronavirus by people who did not have symptoms was ``very rare,''
    \href{https://www.nytimes.com/2020/06/09/world/coronavirus-updates.html?action=click\&pgtype=Article\&state=default\&region=MAIN_CONTENT_3\&context=storylines_faq\#link-1f302e21}{but
    she later walked back that statement.}
  \end{itemize}
\end{itemize}

The logic behind sending him to a shelter to serve out his parole, in
the first place, involved a single domestic-violence charge that had
been filed against him years earlier. One evening in 2013, Mr. Russell
and his current girlfriend were drinking on their stoop and began to
argue. According to his recollection, the fight did not get physical,
but a neighbor called the police and once officers arrived, he became
``unruly'' with them, he said.

As a matter of precaution, it is common for the state's corrections
department to send parolees with any history of domestic incidents to
shelters rather than allow them to go home. In Mr. Russell's case, it
did not seem to matter that he had received mandated counseling. The
system demands rehabilitation and then all too often denies redemption.
In March of last year, his girlfriend wrote to parole officials asking
them to let Mr. Russell come home. The request was denied.

In mid-June he was taken into custody and spent the subsequent few weeks
in Rikers.

``It was scary. A lot of the corrections officers aren't wearing
masks,'' Mr. Russell told me. ``Ninety percent of inmates weren't
wearing masks. Then you're shackled next to people when you are
transferred,'' he said. ``Beds were less than a pinkie apart. Sleeping,
the guy beneath me, my foot could have kicked his head.''

The Legal Aid Society eventually got him out. Mr. Russell works as a
porter in a condominium building in the Ozone Park section of Queens,
and his bosses wrote a letter on his behalf explaining that they would
welcome him back if he were released.

When he left Rikers, though, he was sent back to a shelter --- a hotel
in Williamsburg, Brooklyn, far from his job and his family. Making it
back in time for curfew made it almost impossible to see his partner for
dinner or put his daughter to bed. Because parolees are not allowed to
drive, he is left incurring the risks and lags of public transportation.

A bill currently before the New York State Legislature would loosen many
of the restrictions around parole, making it easier for people exiting
the prison system to reintegrate into the real world. Many prosecutors
across the state support the bill.

Inevitably, there will be those in and out of government fearful of
reform at a moment of rising violence. But diverting technical violators
from the jail system could protect public health without endangering
public safety. Of the 791 parole violators who had their warrants lifted
since the end of March, not one was rearrested for a gun-related crime.

The convergence of the pandemic with a powerful movement to rectify the
racial injustices of the past would seem to provide the ideal moment for
rethinking the ways we manage people coming out of prison. If it cannot
happen now, it seems unlikely that it ever could.

Advertisement

\protect\hyperlink{after-bottom}{Continue reading the main story}

\hypertarget{site-index}{%
\subsection{Site Index}\label{site-index}}

\hypertarget{site-information-navigation}{%
\subsection{Site Information
Navigation}\label{site-information-navigation}}

\begin{itemize}
\tightlist
\item
  \href{https://help.nytimes.com/hc/en-us/articles/115014792127-Copyright-notice}{©~2020~The
  New York Times Company}
\end{itemize}

\begin{itemize}
\tightlist
\item
  \href{https://www.nytco.com/}{NYTCo}
\item
  \href{https://help.nytimes.com/hc/en-us/articles/115015385887-Contact-Us}{Contact
  Us}
\item
  \href{https://www.nytco.com/careers/}{Work with us}
\item
  \href{https://nytmediakit.com/}{Advertise}
\item
  \href{http://www.tbrandstudio.com/}{T Brand Studio}
\item
  \href{https://www.nytimes.com/privacy/cookie-policy\#how-do-i-manage-trackers}{Your
  Ad Choices}
\item
  \href{https://www.nytimes.com/privacy}{Privacy}
\item
  \href{https://help.nytimes.com/hc/en-us/articles/115014893428-Terms-of-service}{Terms
  of Service}
\item
  \href{https://help.nytimes.com/hc/en-us/articles/115014893968-Terms-of-sale}{Terms
  of Sale}
\item
  \href{https://spiderbites.nytimes.com}{Site Map}
\item
  \href{https://help.nytimes.com/hc/en-us}{Help}
\item
  \href{https://www.nytimes.com/subscription?campaignId=37WXW}{Subscriptions}
\end{itemize}
