Sections

SEARCH

\protect\hyperlink{site-content}{Skip to
content}\protect\hyperlink{site-index}{Skip to site index}

\href{https://www.nytimes.com/spotlight/podcasts}{Podcasts}

\href{https://myaccount.nytimes.com/auth/login?response_type=cookie\&client_id=vi}{}

\href{https://www.nytimes.com/section/todayspaper}{Today's Paper}

\href{/spotlight/podcasts}{Podcasts}\textbar{}The Reading List Behind
`Nice White Parents'

\url{https://nyti.ms/3jW2BJq}

\begin{itemize}
\item
\item
\item
\item
\item
\item
\end{itemize}

Advertisement

\protect\hyperlink{after-top}{Continue reading the main story}

Supported by

\protect\hyperlink{after-sponsor}{Continue reading the main story}

\hypertarget{the-reading-list-behind-nice-white-parents}{%
\section{The Reading List Behind `Nice White
Parents'}\label{the-reading-list-behind-nice-white-parents}}

Everyone wants what's best for their children's education. But who gets
to decide what's best? The reporter behind our new podcast from Serial
shares the books that helped her answer that question.

\includegraphics{https://static01.nyt.com/images/2019/06/02/books/review/30nwp-readinglist-top/02Kendi-articleLarge.jpg?quality=75\&auto=webp\&disable=upscale}

By Chana Joffe-Walt

\begin{itemize}
\item
  July 30, 2020
\item
  \begin{itemize}
  \item
  \item
  \item
  \item
  \item
  \item
  \end{itemize}
\end{itemize}

\textbf{``Nice White Parents'' is a new podcast from Serial Productions,
a New York Times Company, about the 60-year relationship between white
parents and the public school down the block. Listen to the first two
episodes now and keep an eye out for new episodes each Thursday,
available here and on your mobile device:}
\textbf{\href{https://podcasts.apple.com/us/podcast/nice-white-parents/id1524080195}{Via
Apple Podcasts}} \textbf{\textbar{}}
\textbf{\href{https://open.spotify.com/show/7oBSLCZFCgpdCaBjIG8mLV?si=YcEPLD3xT2ejXmpQz-tRpw}{Via
Spotify}} \textbf{\textbar{}}
\textbf{\href{https://podcasts.google.com/feed/aHR0cHM6Ly9yc3MuYXJ0MTkuY29tL25pY2Utd2hpdGUtcGFyZW50cw}{Via
Google}}

\includegraphics{https://static01.nyt.com/images/2020/07/21/podcasts/nice-white-parents-album-art/nice-white-parents-album-art-articleInline.jpg?quality=75\&auto=webp\&disable=upscale}

\hypertarget{the-book-of-statuses}{%
\subsubsection{The Book of Statuses}\label{the-book-of-statuses}}

A group of parents takes one big step together.

\includegraphics{https://static01.nyt.com/images/2020/07/21/podcasts/nice-white-parents-album-art/nice-white-parents-album-art-articleInline.jpg?quality=75\&auto=webp\&disable=upscale}

\hypertarget{i-still-believe-in-it}{%
\subsubsection{`I Still Believe in It'}\label{i-still-believe-in-it}}

White parents in the 1960s fought to be part of a new, racially
integrated school in Brooklyn. So why did their children never attend?

``I depended on several excellent archivists and historians to track the
history of one school,'' said Chana Joffe-Walt of her five-year process
reporting ``Nice White Parents,'' which follows what happened when a
group of white families arrived at a predominantly Black and Latino
school in New York City.

``And there are a few specific books that ran on a loop in my head as I
went,'' she said. ``They were essential in helping me understand what I
was seeing and learning.''

Here are a few of those books, and Chana's thoughts on each of them:

Image

Margaret A. Hagerman

\hypertarget{white-kids}{%
\subsubsection{``White Kids''}\label{white-kids}}

In episode one, a white boy, part of a large group of new white
students, talks about how the school has improved ``with us here'' and
is now ranked higher in the ``book on statuses.'' He's communicating
what he's learned about race at 10 years old, despite being in a school
where the adults are saying ``all children are equal.'' Margaret
Hagerman's ** sociological research reveals that white children learn
about race as much from the choices they see their parents make as from
what they hear their parents say.

Image

Amanda E. Lewis and John B. Diamond

\hypertarget{despite-the-best-intentions}{%
\subsubsection{``Despite the Best
Intentions''}\label{despite-the-best-intentions}}

Integration is not just bodies in the building together --- and this
book takes us to one school where the halls are racially mixed but the
classrooms are not. Lewis and Diamond explore why that is, including a
very human and helpful look at white parents and the concept of
``opportunity hoarding.''

Image

Clarence Taylor

\hypertarget{knocking-at-our-own-door}{%
\subsubsection{``Knocking at Our Own
Door''}\label{knocking-at-our-own-door}}

This is the essential history of the movement for desegregation in New
York City. It's a deeply researched book that describes the years of
organizing that led up to Freedom Day in 1964, one of the largest civil
rights demonstrations in American history. I owe so much of what I know
about the names and details of New York's movement for integration to
Clarence Taylor.

Image

Vanessa Siddle Walker

\hypertarget{the-lost-education-of-horace-tate}{%
\subsubsection{``The Lost Education of Horace
Tate''}\label{the-lost-education-of-horace-tate}}

When I first heard the archival tape of Mae Mallory, a civil rights
activist, saying her lawsuit against the New York City Board of
Education had ``nothing to do with wanting to sit next to white kids,''
I thought of this book. It's a gripping narrative about the careful,
savvy and often clandestine work of Black educators in Georgia to secure
school buses, textbooks, playgrounds and school buildings for Black
kids. These educators saw desegregation as a strategy, a means to
achieve equality. Ms. Siddle Walker argues they got ``second-class
integration'' that cost them their jobs and the very benefits they'd
fought to protect instead.

Image

Elizabeth Gillespie McRae

\hypertarget{mothers-of-massive-resistance}{%
\subsubsection{``Mothers of Massive
Resistance''}\label{mothers-of-massive-resistance}}

There's a phrase from Ms. Gillespie McRae's carefully researched book
that has stayed with me: that white women have been ``segregation's
constant gardeners.'' She documents many decades of the quiet, everyday
activism of ``good'' white mothers who helped maintain racial lines
through their work on PTAs, and as social workers, teachers and
midwives. This is a critical history that should be taught right
alongside Brown v. Board of Education.

Image

Matthew F. Delmont

\hypertarget{why-busing-failed}{%
\subsubsection{``Why Busing Failed''}\label{why-busing-failed}}

Busing was the tool of choice for white people resisting desegregation
in northern states. Mr. Delmont writes, ``White mothers in New York were
talking about busing before any school boards or courts anywhere in the
country had ordered it.'' The idea of busing took hold in subsequent
conversations about desegregation across the country. In the process,
white mothers insisted busing wasn't about race or policy or justice. It
was just about transportation.

Image

Eve L. Ewing

\hypertarget{ghosts-in-the-schoolyard}{%
\subsubsection{``Ghosts in the
Schoolyard''}\label{ghosts-in-the-schoolyard}}

While working on episode three, I put this book down and wrote
``PURPOSEFUL FORGETTING'' in red marker on a white board. Ms. Ewing
helped me understand how school policy can pretend that history doesn't
exist. It's an expertly told story about the closing of nearly 50
schools in Chicago in 2013. The Chicago schools superintendent said that
these schools were selected for closure because they were underutilized
and lacking resources. In response, Ms. Ewing asks, ``How could the
person charged with doling out resources condemn an institution for not
having enough resources?'' Then, with precision, she lays out exactly
how they got that way.

Image

Noliwe Rooks

\hypertarget{cutting-school}{%
\subsubsection{``Cutting School''}\label{cutting-school}}

Ms. Rooks began with curiosity about the rich college students she kept
meeting who were interested in working in poor schools. What is it about
the education of other people's children, she wondered, that is so
consistently compelling to the wealthy? Ms. Rooks traces the dynamics of
private money in public schools from Reconstruction to the present. She
argues that programs like Teach for America and charter schools depend
on racial and economic segregation, what she calls ``segrenomics.''

Advertisement

\protect\hyperlink{after-bottom}{Continue reading the main story}

\hypertarget{site-index}{%
\subsection{Site Index}\label{site-index}}

\hypertarget{site-information-navigation}{%
\subsection{Site Information
Navigation}\label{site-information-navigation}}

\begin{itemize}
\tightlist
\item
  \href{https://help.nytimes.com/hc/en-us/articles/115014792127-Copyright-notice}{©~2020~The
  New York Times Company}
\end{itemize}

\begin{itemize}
\tightlist
\item
  \href{https://www.nytco.com/}{NYTCo}
\item
  \href{https://help.nytimes.com/hc/en-us/articles/115015385887-Contact-Us}{Contact
  Us}
\item
  \href{https://www.nytco.com/careers/}{Work with us}
\item
  \href{https://nytmediakit.com/}{Advertise}
\item
  \href{http://www.tbrandstudio.com/}{T Brand Studio}
\item
  \href{https://www.nytimes.com/privacy/cookie-policy\#how-do-i-manage-trackers}{Your
  Ad Choices}
\item
  \href{https://www.nytimes.com/privacy}{Privacy}
\item
  \href{https://help.nytimes.com/hc/en-us/articles/115014893428-Terms-of-service}{Terms
  of Service}
\item
  \href{https://help.nytimes.com/hc/en-us/articles/115014893968-Terms-of-sale}{Terms
  of Sale}
\item
  \href{https://spiderbites.nytimes.com}{Site Map}
\item
  \href{https://help.nytimes.com/hc/en-us}{Help}
\item
  \href{https://www.nytimes.com/subscription?campaignId=37WXW}{Subscriptions}
\end{itemize}
