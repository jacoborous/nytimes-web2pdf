Sections

SEARCH

\protect\hyperlink{site-content}{Skip to
content}\protect\hyperlink{site-index}{Skip to site index}

\href{https://www.nytimes.com/section/books/review}{Book Review}

\href{https://myaccount.nytimes.com/auth/login?response_type=cookie\&client_id=vi}{}

\href{https://www.nytimes.com/section/todayspaper}{Today's Paper}

\href{/section/books/review}{Book Review}\textbar{}Isabel Wilkerson
Loves Books. That Doesn't Mean She Treats Them Gently.

\url{https://nyti.ms/33bDjBk}

\begin{itemize}
\item
\item
\item
\item
\item
\end{itemize}

Advertisement

\protect\hyperlink{after-top}{Continue reading the main story}

Supported by

\protect\hyperlink{after-sponsor}{Continue reading the main story}

\href{/column/by-the-book}{By the Book}

\hypertarget{isabel-wilkerson-loves-books-that-doesnt-mean-she-treats-them-gently}{%
\section{Isabel Wilkerson Loves Books. That Doesn't Mean She Treats Them
Gently.}\label{isabel-wilkerson-loves-books-that-doesnt-mean-she-treats-them-gently}}

\includegraphics{https://static01.nyt.com/images/2020/08/02/books/review/02ByTheBook-Wilkerson/02ByTheBook-Wilkerson-articleLarge.jpg?quality=75\&auto=webp\&disable=upscale}

July 30, 2020

\begin{itemize}
\item
\item
\item
\item
\item
\end{itemize}

\emph{``Many of them are not only dog-eared, but often
double-cornered-dog-eared, the margins marked up with my own
commentary,'' says the author, whose new book is ``Caste.''}

\textbf{What books are on your nightstand?}

I have years of catching up to do. I am especially looking forward to
reading ``The Sympathizer,'' by Viet Thanh Nguyen, ``Washington Black,''
by Esi Edugyan, and ``The Vanishing Half,'' by Brit Bennett.

\textbf{Are there any classic novels that you only recently read for the
first time?}

I am working my way through Proust because it seems that there is the
notion that every writer ought to and because I have been a Francophile
since my first French class in third grade.

\textbf{What's your favorite book no one else has heard of?}

Back in the 1930s, a Harvard-trained African-American anthropologist,
Allison Davis, and his equally refined wife, Elizabeth, risked their
lives to study life under Jim Crow in a remote section of Mississippi.
They had to sublimate their educated demeanor and act as subordinates
even to the Northern white couple, Burleigh and Mary Gardner, whom they
were teamed with as fellow anthropologists for the project, lest they
disrupt the caste system they were studying and invite danger to
themselves. After years of dedicated fieldwork, Allison Davis and the
Gardners produced a book in 1941 called ``Deep South: A Social
Anthropological Study of Caste and Class,'' perhaps the earliest study
of caste in America from both sides of the divide. It's a book that got
overshadowed upon publication and still warrants more recognition for
its groundbreaking view into life as it was in the feudal South.

\textbf{Which writers --- novelists, playwrights, critics, journalists,
poets --- working today do you admire most?}

We are in the midst of a golden age of Black intellectual abundance at
the precise moment we most need these voices, and it stresses me out to
even attempt to name the many whom I admire. I would include Suzan-Lori
Parks, Lynn Nottage, Adam Serwer, Rachel Kaadzi Ghansah, Saeed Jones,
Tracy K. Smith, Roxane Gay, Ta-Nehisi Coates, Nikole Hannah-Jones, Brent
Staples, Karen Attiah and Yamiche Alcindor. And I must add the
historians: Ibram X. Kendi, Daina Ramey Berry, Erica Armstrong Dunbar,
Blair L. M. Kelley, Carol Anderson and Stephanie Jones-Rogers.

\textbf{Which subjects do you wish more authors would write about?}

I wish we could see more books about the inner lives of everyday people
from marginalized groups in our country --- not the extremes of either
celebrity or pathology, but just regular working folks who make up, for
instance, the great bulk of African-Americans. People just going about
their days and getting through the challenges of ordinary life do not
get anywhere near the attention they deserve in the popular imagination.
And their invisibility leads to distortions in how an entire group is
seen, gives the impression that people from across the racial divide are
more fundamentally different than we actually are. Two of the most
gorgeous examples that come to mind for me are Toni Morrison's ``Jazz''
and Rita Dove's ``Thomas and Beulah,'' both of which elevate the
ordinary to the sublime.

\textbf{What books would you recommend to somebody who wants to learn
more about America's caste system?}

W. E. B. DuBois's ``Black Reconstruction'' is vital to understanding the
reinvigoration of caste after the end of the Civil War, as is Eric
Foner's ``Reconstruction.'' The late anthropologist Ashley Montagu, in
his 1942 book, ``Man's Most Dangerous Myth,'' was among the earliest to
make the case that race was a social construct and that caste was an
underlying driver of our disparities. For understanding how caste
operates in specific segments of our society, I would recommend the
following: ``Medical Apartheid,'' by Harriet A. Washington, for stunning
insights into how caste has played out in the history of health care in
our country. ``The New Jim Crow,'' by Michelle Alexander, and ``Just
Mercy,'' by Bryan Stevenson, for overwhelming evidence of caste in our
criminal justice system. ``The Color of Law,'' by Richard Rothstein for
an analysis of how caste has undergirded our country's housing policies.
And for the effect of caste in economics, the work of William A. Darity,
specifically, ``Persistent Disparity'' and ``From Here to Equality.''
Decades ago, in the seminal work ``The Annihilation of Caste,'' the late
Bhimrao Ambedkar, the revered leader of the Dalit liberation movement,
wrote of the divisive nature of caste in India, but close observers of
racial dynamics in the United States will recognize parallels with our
own country in his impassioned treatise. Gunnar Myrdal's ``An American
Dilemma'' remains perhaps the most comprehensive single work on what
Myrdal himself came to see as a caste system in America. And finally,
``The Negro in Chicago,'' the 1922 report from the Chicago Commission on
Race Relations, which convened in the aftermath of the 1919 race riots,
is as chillingly prophetic and relevant to us today as it was when it
was written nearly a century ago.

\textbf{Which genres do you especially enjoy reading? And which genres
do you avoid?}

I find myself drawn to classic, often underappreciated, novels of the
1930s and 1940s, to works like ``The Street,'' by Ann Petry, who is
deservedly experiencing a renaissance, ``If He Hollers, Let Him Go,'' by
Chester Himes, who deserves his own renaissance, and ``Black No More,''
by George S. Schuyler. The latter is a clever and biting satire in which
Schuyler imagines the social disruption of an invention that can make
Black people look like white people in a matter of days. Black people
who swear they would never do it, line up to be converted, while
paranoia spreads among white people who fear being infiltrated by Black
people who only look white. Thousands of Black people disappear into the
white world, but have trouble truly passing because they have neither
the back story nor the dominant caste perspective to pull it off, and
thus live in fear of being outed.

I am always struck by how fresh and unflinching the writing of that era
is, that, in the midst of depression and Jim Crow and war, they wrote
with a fearless straightforwardness and emotional truthtelling that
could have been written today.

\textbf{How do you organize your books?}

My books are not only dear to me, they are central to my research and to
the act of writing. Many of them are not only dog-eared, but often
double-cornered-dog-eared, the margins marked up with my own commentary.
So being able to locate a book is crucial, but my books are only
organized loosely by subject area. I have so many books on a range of
overlapping subjects, too many books for any one room, too many books
for the overflowing shelves, that, especially in the thick of writing,
hardly any room in my house is without a pile of books on whatever flat
surface happens to be available. This means that I am frequently on the
hunt through multiple rooms for a book I need, and am grateful for the
internal compass that seems always to save me. There have been times
where I have had to read a book a day for research, and this internal
compass seems to somehow remember the general vicinity of whatever book
I'm looking for and wherever it has last been seen.

\textbf{What book might people be surprised to find on your shelves?}

``Blindness,'' by José Saramago, one of my favorites in the world. From
the moment I first read it years ago while on a trip to Portugal, I have
loved it for its unsentimentally pure and raw comprehension of human
nature. It's a prophecy and a parable about the range of human reactions
when an unnamed city in an unnamed country is suddenly afflicted with a
mysterious contagion of blindness. He chooses to leave the characters
unnamed as well and thus hurls us into the isolating anonymity of the
social disorder that ensues. With the turn of each alarming or endearing
page, I thought to myself, this is exactly what humans would do. He
holds a light to the bleak underside of human frailties that get people
into trouble as they relate to one another, and he writes with a seeming
wish to believe that, despite evidence at times to the contrary, while
humans will do whatever it takes to survive, they are, in the end,
essentially good.

\textbf{What book would you recommend for America's current political
moment?}

If forced to name a single one, it would have to be Baldwin's ``The Fire
Next Time.'' He captured our present before it had even happened.

\textbf{What's the best book you've ever received as a gift?}

Years ago, I was in the earliest stages of what would become ``The
Warmth of Other Suns,'' still putting my thoughts into language, and
teaching at Princeton for a semester. One day, I was chatting with a
group of faculty members about the unnamed, embryonic book I was working
on. Soon afterward, one of them handed me a copy of ``Caste and Class in
a Southern Town,'' by John Dollard, and said it might be of some help to
me. It was the first I had heard of Dollard or seen the word ``caste''
applied to America. It felt both dissonant and intriguingly appropriate.
I realized then that what I was hearing in the interviews I was
conducting with people who had fled the Jim Crow South was, in fact, the
testimony of survivors of a caste system here in our own country. It set
me on a course of researching everything I could about caste, and I have
been using the word ever since.

\textbf{You're organizing a literary dinner party. Which three writers,
dead or alive, do you invite?}

Richard Wright, James Baldwin and Zora Neale Hurston, to sit between
them and to referee, over her favorite oysters and cornmeal dumplings
and sweet potato pone.

Advertisement

\protect\hyperlink{after-bottom}{Continue reading the main story}

\hypertarget{site-index}{%
\subsection{Site Index}\label{site-index}}

\hypertarget{site-information-navigation}{%
\subsection{Site Information
Navigation}\label{site-information-navigation}}

\begin{itemize}
\tightlist
\item
  \href{https://help.nytimes.com/hc/en-us/articles/115014792127-Copyright-notice}{©~2020~The
  New York Times Company}
\end{itemize}

\begin{itemize}
\tightlist
\item
  \href{https://www.nytco.com/}{NYTCo}
\item
  \href{https://help.nytimes.com/hc/en-us/articles/115015385887-Contact-Us}{Contact
  Us}
\item
  \href{https://www.nytco.com/careers/}{Work with us}
\item
  \href{https://nytmediakit.com/}{Advertise}
\item
  \href{http://www.tbrandstudio.com/}{T Brand Studio}
\item
  \href{https://www.nytimes.com/privacy/cookie-policy\#how-do-i-manage-trackers}{Your
  Ad Choices}
\item
  \href{https://www.nytimes.com/privacy}{Privacy}
\item
  \href{https://help.nytimes.com/hc/en-us/articles/115014893428-Terms-of-service}{Terms
  of Service}
\item
  \href{https://help.nytimes.com/hc/en-us/articles/115014893968-Terms-of-sale}{Terms
  of Sale}
\item
  \href{https://spiderbites.nytimes.com}{Site Map}
\item
  \href{https://help.nytimes.com/hc/en-us}{Help}
\item
  \href{https://www.nytimes.com/subscription?campaignId=37WXW}{Subscriptions}
\end{itemize}
