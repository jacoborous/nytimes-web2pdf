Sections

SEARCH

\protect\hyperlink{site-content}{Skip to
content}\protect\hyperlink{site-index}{Skip to site index}

\href{https://www.nytimes.com/section/books/review}{Book Review}

\href{https://myaccount.nytimes.com/auth/login?response_type=cookie\&client_id=vi}{}

\href{https://www.nytimes.com/section/todayspaper}{Today's Paper}

\href{/section/books/review}{Book Review}\textbar{}`Memorial Drive,' by
Natasha Trethewey: An Excerpt

\url{https://nyti.ms/335r3SN}

\begin{itemize}
\item
\item
\item
\item
\item
\end{itemize}

Advertisement

\protect\hyperlink{after-top}{Continue reading the main story}

Supported by

\protect\hyperlink{after-sponsor}{Continue reading the main story}

\hypertarget{memorial-drive-by-natasha-trethewey-an-excerpt}{%
\section{`Memorial Drive,' by Natasha Trethewey: An
Excerpt}\label{memorial-drive-by-natasha-trethewey-an-excerpt}}

Buy Book ▾

\begin{itemize}
\tightlist
\item
  \href{https://www.amazon.com/gp/search?index=books\&tag=NYTBSREV-20\&field-keywords=Memorial+Drive\%3A+A+Daughter\%27s+Memoir+Natastha+Trethewey}{Amazon}
\item
  \href{https://du-gae-books-dot-nyt-du-prd.appspot.com/buy?title=Memorial+Drive\%3A+A+Daughter\%27s+Memoir\&author=Natastha+Trethewey}{Apple
  Books}
\item
  \href{https://www.anrdoezrs.net/click-7990613-11819508?url=https\%3A\%2F\%2Fwww.barnesandnoble.com\%2Fs\%2FMemorial+Drive\%3A+A+Daughter\%27s+Memoir+Natastha+Trethewey}{Barnes
  and Noble}
\item
  \href{https://www.anrdoezrs.net/click-7990613-35140?url=https\%3A\%2F\%2Fwww.booksamillion.com\%2Fsearch\%3Fquery\%3DMemorial\%2BDrive\%253A\%2BA\%2BDaughter\%2527s\%2BMemoir\%2BNatastha\%2BTrethewey}{Books-A-Million}
\item
  \href{https://bookshop.org/books?keywords=Memorial+Drive\%3A+A+Daughter\%27s+Memoir}{Bookshop}
\item
  \href{https://www.indiebound.org/search/book?searchfor=Memorial+Drive\%3A+A+Daughter\%27s+Memoir+Natastha+Trethewey\&aff=NYT}{Indiebound}
\end{itemize}

When you purchase an independently reviewed book through our site, we
earn an affiliate commission.

July 30, 2020

\begin{itemize}
\item
\item
\item
\item
\item
\end{itemize}

There is a large birthmark on the back of my thigh. Even though it has
been with me over half a century, I can't recall which leg bears its
dark outline, and so I have to look at myself backward in a mirror to
remember. Seeing it is not unlike encountering a forgotten scar, a
remnant that recalls the moment of wounding. It takes me back to my
early childhood: the long, warm days in Mississippi when I wore shorts
much of the time and the birthmark was plainly visible, not hidden as it
usually is now. Though not the shape of a hand, it is the size of one,
and in exactly the spot where, if you were told to sit on your hands as
my mother was, you might leave a mark.

Across cultures myths abound about the imprint a mother can make even
before her child crosses the threshold into the world, the way her
desires or fears can be manifest on the body: birthmarks in the shape or
color of food she craved, a lock of gray hair where she tugged at her
own. To stanch the cravings, they say, eat a bit of dirt or clay; to
steady the hand that worries the hair, sit on it. Had my mother done any
of this, there might have been a single story in my family about what my
birthmark symbolizes. The only thing the elders agreed on was that it
looked like a place on a map, somewhere my mother might have dreamed of
but had never been. I've often imagined her anticipating my arrival,
both hopeful and anxious about the world, the particular time and place
I would enter: a fierce longing taking shape inside her.

In the spring of 1966, when I was born, my mother was a couple of months
shy of her twenty-second birthday. My father was out of town, traveling
for work, so she made the short trip from my grandmother's house to
Gulfport Memorial Hospital, as planned, without him. On her way to the
segregated ward she could not help but take in the tenor of the day,
witnessing the barrage of rebel flags lining the streets: private
citizens, lawmakers, Klansmen (often one and the same) raising them in
Gulfport and small towns all across Mississippi. The twenty-sixth of
April that year marked the hundredth anniversary of Mississippi's
celebration of Confederate Memorial Day---a holiday glorifying the old
South, the Lost Cause, and white supremacy---and much of the fervor was
a display, too, in opposition to recent advancements in the civil rights
movement. She could not have missed the paradox of my birth on that
particular day: a child of miscegenation, an interracial marriage still
illegal in Mississippi and in as many as twenty other states.

\emph{{[} Return to the review of}
\href{https://www.nytimes.com/2020/07/30/books/review/memorial-drive-natastha-trethewey.html}{\emph{``Memorial
Drive.''}} \emph{{]}}

Sequestered on the ``colored'' floor, my mother knew the country was
changing, but slowly. She had come of age in the summer of 1965, turning
twenty-one in the wake of Bloody Sunday, the Watts riots, and years of
racially motivated murders in Mississippi. Unlike my father, who'd grown
up a white boy in rural Nova Scotia, hunting and fishing, free to roam
the open woods, my mother had come into being a black girl in the Deep
South, hemmed in, bound to a world circumscribed by Jim Crow. Though my
father believed in the idea of living dangerously, the necessity of
taking risks, my mother had witnessed the necessity of dissembling, the
art of making of one's face an inscrutable mask before whites who
expected of blacks a servile deference. In the summer of 1955, when she
was eleven years old, she'd seen what could happen to a black child in
Mississippi who had not behaved as expected, stepping outside the
confines of racial proscription: in my grandmother's copy of \emph{Jet}
magazine, Emmett Till's battered remains, his destroyed face.

Even had my mother wanted to ignore the racial violence and increasing
turbulence around her, my grandmother would not allow it. In her house
the latest issue of \emph{Jet} lay on the coffee table beside a book of
documentary photographs of the civil rights movement, images ranging
from lynchings to peaceful protests and the resilient faces of black
Americans---constant reminders of the necessity of fighting for justice
in a state where the external reminders were increasingly unavoidable.
The year before my mother met my father, the civil rights activist
Medgar Evers had been gunned down in his driveway in Jackson. That year,
1963, my grandmother joined a group of black citizens in the Biloxi
wade-in to protest being denied the right to use the public beaches. To
mourn Evers, the protesters placed hundreds of black flags in the
sand---an image my mother, watching from the seawall, would not forget.
Nor would she forget hearing the news of the three civil rights
activists working on the Freedom Summer campaign to register black
voters in Mississippi. James Chaney, Andrew Goodman, and Michael
Schwerner had been abducted and murdered in June 1964, their bodies
found two months later, buried under the weight of an earthen berm in
Neshoba County.

When the news reached her, my mother was out of the state on a field
trip with her college theater troupe. Back home the Ku Klux Klan had
initiated its campaign of terror, the Mississippi she returned to having
grown even more frightening. That summer was a season of fires, of
danger coming ever closer: flaming crosses and black churches burning
all around the state. My mother and grandmother, living across the
street from a church, slept less soundly then, awakening often in the
night to listen.

It was against that backdrop of imminence and upheaval that my parents,
both college students at the time, fell in love. They met in a
literature course on modern drama, and their conversations on books and
theater propelled them from the classroom out into the afternoon
sunlight as they walked the campus and beyond, among the rolling green
hills of Kentucky. When they eloped in 1965, traveling across the Ohio
River into Cincinnati, where it was legal for them to be married, only
my mother fully understood what this might mean for me, the child she
was already carrying. In letters to my father during their months apart,
she was at once sanguine and practical, hopeful for a changing nation
but also aware that any child she brought into the world would have much
to learn in order to be safe. That meant I would need to understand the
realities I would face: the painful, oppressive facts of a place slow to
accept integration even as it was now the law of the land. My father,
idealistic in nature, was still naive enough to believe I could grow up
as free of the burdens of race---of blackness, that is---as he was.

They complemented each other, as opposites do: my mother graceful and
reserved, attentive to details; my father, with his rough manners, rowdy
and bookish at once, often distracted by his thoughts. It was my mother
who stanched the blood on my cheek when, after watching my father
shaving, I tried using his straight razor; it had been my father,
absentminded, who'd left the razor on the counter within my reach. One
day, when I cut my knee in the ditch outside, revealing what appeared to
be a layer of white skin underneath, I lay between them, holding their
hands up side by side, asking why they weren't the same color, why I
didn't match either of them exactly. \emph{What was I?} ``You have the
best of both worlds,'' they told me, not for the first time.

\emph{{[} Return to the review of}
\href{https://www.nytimes.com/2020/07/30/books/review/memorial-drive-natastha-trethewey.html}{\emph{``Memorial
Drive.''}} \emph{{]}}

Out in the world, alone with either of them, I was just beginning to
feel a profound sense of dislocation. If I was with my father, I
measured the polite responses from white people, the way they addressed
him as ``Sir'' or ``Mister.'' Whereas my mother would be called ``Gal,''
never ``Miss'' or ``Ma'am,'' as I had been taught was proper. So
different was the treatment I received with each of them that I was
unsure where or how I belonged. Only at home, the three of us together,
did I feel profoundly \emph{theirs}, and in that trinity of mother,
father, and child I would shut my eyes and fall asleep on the high bed
between them.

Outside that bedroom was a long, narrow hallway leading to the den and,
just inside the door, a tall bookcase that held my attention countless
afternoons. It housed my parents' books along with a ser of
encyclopedias my mother had insisted my grandmother purchase, instead of
bronzing my baby shoes, to commemorate my birth. In the earliest dream I
can recall, that hallway led to something unknown by which I was both
drawn and vaguely frightened, a hint of danger that lay before me. In
the dream I woke to a house so dark and quiet it seemed I was alone. I
rose then and stood in the doorway, peering down the length of the hall.
Opposite me, at the other end, blocking the bookcase, was a figure the
size of a man: faceless and made entirely of the crushed shells that
covered the driveway beside our house, the sharp edges I'd walked over
barefoot countless times.

It makes sense to me now that my earliest recollected dream took on such
a shape. By then my father was in graduate school part-time, working on
his PhD in English, becoming a writer. Had I told him what frightened
me, he might have reminded me, as a comfort, that the imagery resembled
some of the stories he recited to me at bedtime: the trials of Odysseus,
his encounter with the Cyclops blocking the exit to the cave; the
monster Grendel, at the entrance to the mead hall, in the legend of
Beowulf. Beyond those tales were the stories of Narcissus, Icarus,
Cassandra, the riddle of the Sphinx---stories about bravery, vanity,
hubris, knowledge.

I liked to curl up next to him in his large chair as he read. One
evening, I ran my finger along his throat, over the knot there sharp as
a knuckle.

``What's this, Daddy?'' I asked. From Sunday school I knew the story of
Adam and Eve, but not the part my father now recounted: how when Adam
bit the apple from the tree of knowledge it lodged in his throat, giving
to his descendants this lasting anatomical feature.

``Does it hurt?'' I asked.

``No,'' he said, furrowing his brow as usual. ``But it is one of the
consequences of knowledge.''

``Why don't I have one?''

``You do,'' he said, placing my hand against my own throat. ``It's just
smaller. Say something and you can feel it.''

What my father wanted me to know about the world he did not always say
explicitly and so I listened intently to his stories, finding myself in
the characters. When I swung too high on my swing set even though he
warned me not to---nearly going backward over the crossbar, the chain
buckling and sending me flailing to the ground---I heard the story of
Icarus. When I played too long before the mirror imitating my mother at
her toilette, enthralled by my own face, it would be the story of
Narcissus.

In the short stories he was writing, fictionalized accounts of our
lives, he named my character Cassandra, after the figure from Greek
mythology. For my father, the myth of Cassandra had been just another
way he sought to guide me toward what he thought I needed to know. In
some versions, Cassandra's fate is that she is merely
misunderstood---not unlike what my father imagined to be the obvious
fate of a mixed-race child born in a place like Mississippi. ``She was a
prophet,'' he told me, ``but no one would believe her.'' Over the years,
though, this second naming would come to weigh heavily on me. It was as
if, in giving me that name, he had given me not only the burden of
foresight but also the notion of causation---that whatever it was, if I
could imagine it, see it in my mind's eye, it would happen because I had
envisioned it. As if I had willed it into being.

The language of allegory and metaphor undergirded our days. ``How'd you
like to have that ball to play with?'' my father said one afternoon,
pointing to the red sun great in the sky.

``Don't be silly,'' said my mother. ``You know she'd burn her hands.''

Even then I knew something had passed between them, some difference in
how they aimed to prepare me for the world. My father believed---as the
poet Robert Frost cautioned---that one must have a thorough education in
figurative language. ``What I am pointing out,'' Frost wrote, ``is that
unless you are at home in the metaphor, unless you have had your proper
poetical education in the metaphor, you are not safe anywhere. Because
you are not at ease with figurative values: you don't know the metaphor
in its strength and its weakness. You are not safe in science; you are
not safe in history.'' My mother, who'd majored in literature and
theater in college, must have believed as well in the necessity of an
education in metaphor, and yet she was the direct one, less interested
in abstractions and figures of speech than in more practical lessons,
admonishments about dangers I could not yet imagine.

\emph{{[} Return to the review of}
\href{https://www.nytimes.com/2020/07/30/books/review/memorial-drive-natastha-trethewey.html}{\emph{``Memorial
Drive.''}} \emph{{]}}

Advertisement

\protect\hyperlink{after-bottom}{Continue reading the main story}

\hypertarget{site-index}{%
\subsection{Site Index}\label{site-index}}

\hypertarget{site-information-navigation}{%
\subsection{Site Information
Navigation}\label{site-information-navigation}}

\begin{itemize}
\tightlist
\item
  \href{https://help.nytimes.com/hc/en-us/articles/115014792127-Copyright-notice}{©~2020~The
  New York Times Company}
\end{itemize}

\begin{itemize}
\tightlist
\item
  \href{https://www.nytco.com/}{NYTCo}
\item
  \href{https://help.nytimes.com/hc/en-us/articles/115015385887-Contact-Us}{Contact
  Us}
\item
  \href{https://www.nytco.com/careers/}{Work with us}
\item
  \href{https://nytmediakit.com/}{Advertise}
\item
  \href{http://www.tbrandstudio.com/}{T Brand Studio}
\item
  \href{https://www.nytimes.com/privacy/cookie-policy\#how-do-i-manage-trackers}{Your
  Ad Choices}
\item
  \href{https://www.nytimes.com/privacy}{Privacy}
\item
  \href{https://help.nytimes.com/hc/en-us/articles/115014893428-Terms-of-service}{Terms
  of Service}
\item
  \href{https://help.nytimes.com/hc/en-us/articles/115014893968-Terms-of-sale}{Terms
  of Sale}
\item
  \href{https://spiderbites.nytimes.com}{Site Map}
\item
  \href{https://help.nytimes.com/hc/en-us}{Help}
\item
  \href{https://www.nytimes.com/subscription?campaignId=37WXW}{Subscriptions}
\end{itemize}
