Sections

SEARCH

\protect\hyperlink{site-content}{Skip to
content}\protect\hyperlink{site-index}{Skip to site index}

\href{https://www.nytimes.com/section/business/economy}{Economy}

\href{https://myaccount.nytimes.com/auth/login?response_type=cookie\&client_id=vi}{}

\href{https://www.nytimes.com/section/todayspaper}{Today's Paper}

\href{/section/business/economy}{Economy}\textbar{}Rise in Unemployment
Claims Signals an Economic Reversal

\url{https://nyti.ms/30KwCmZ}

\begin{itemize}
\item
\item
\item
\item
\item
\end{itemize}

\href{https://www.nytimes.com/news-event/coronavirus?action=click\&pgtype=Article\&state=default\&region=TOP_BANNER\&context=storylines_menu}{The
Coronavirus Outbreak}

\begin{itemize}
\tightlist
\item
  live\href{https://www.nytimes.com/2020/08/04/world/coronavirus-covid-19.html?action=click\&pgtype=Article\&state=default\&region=TOP_BANNER\&context=storylines_menu}{Latest
  Updates}
\item
  \href{https://www.nytimes.com/interactive/2020/us/coronavirus-us-cases.html?action=click\&pgtype=Article\&state=default\&region=TOP_BANNER\&context=storylines_menu}{Maps
  and Cases}
\item
  \href{https://www.nytimes.com/interactive/2020/science/coronavirus-vaccine-tracker.html?action=click\&pgtype=Article\&state=default\&region=TOP_BANNER\&context=storylines_menu}{Vaccine
  Tracker}
\item
  \href{https://www.nytimes.com/2020/08/02/us/covid-college-reopening.html?action=click\&pgtype=Article\&state=default\&region=TOP_BANNER\&context=storylines_menu}{College
  Reopening}
\item
  \href{https://www.nytimes.com/live/2020/08/03/business/stock-market-today-coronavirus?action=click\&pgtype=Article\&state=default\&region=TOP_BANNER\&context=storylines_menu}{Economy}
\end{itemize}

Advertisement

\protect\hyperlink{after-top}{Continue reading the main story}

Supported by

\protect\hyperlink{after-sponsor}{Continue reading the main story}

\hypertarget{rise-in-unemployment-claims-signals-an-economic-reversal}{%
\section{Rise in Unemployment Claims Signals an Economic
Reversal}\label{rise-in-unemployment-claims-signals-an-economic-reversal}}

Job losses showed no letup as a surge in coronavirus cases forced new
business shutdowns and a \$600 weekly federal benefit inched to its
expiration.

Initial weekly unemployment claims,

both regular and those under the Pandemic Unemployment Assistance
program

6 million

5

4

3

2

1

0

'17

'18

'19

'20

Initial weekly unemployment claims, both regular and those under the
Pandemic Unemployment Assistance program

6 million

5

4

3

2

1

0

2017

2018

2019

2020

Pandemic Unemployment Assistance extends eligibility to some workers who
would not otherwise be able to apply for unemployment benefits, such as
part-time and self-employed workers. Regular claims are seasonally
adjusted but P.U.A. claims are not.

Source: Labor Department

By The New York Times

\href{https://www.nytimes.com/by/patricia-cohen}{\includegraphics{https://static01.nyt.com/images/2018/02/16/multimedia/author-patricia-cohen/author-patricia-cohen-thumbLarge.jpg}}

By \href{https://www.nytimes.com/by/patricia-cohen}{Patricia Cohen}

\begin{itemize}
\item
  Published July 23, 2020Updated July 30, 2020
\item
  \begin{itemize}
  \item
  \item
  \item
  \item
  \item
  \end{itemize}
\end{itemize}

New state unemployment claims increased last week for the first time in
nearly four months, disturbing evidence that the struggling economy is
backsliding at a time when coronavirus cases are on the rise.

After a flood of claims as the pandemic shut businesses early in the
spring, weekly unemployment filings fell sharply before flattening in
June. But on Thursday, the
\href{https://oui.doleta.gov/press/2020/072320.pdf}{Labor Department}
reported more than 1.4 million new applications for state benefits last
week, up from about 1.3 million in the preceding two weeks.

Another 975,000 jobless workers filed for benefits through an emergency
federal program, also an increase. Unlike the figure for state claims,
that number is not seasonally adjusted.

Claims are rising just as a
\href{https://www.nytimes.com/2020/07/30/business/unemployment-payments-change.html}{\$600-a-week
federal supplement to jobless benefits} is set to expire and Republican
infighting has kept the party from putting forward a proposal for
further aid, much less negotiating with Democrats on a bill.

The discouraging news from the Labor Department followed a
\href{https://www.census.gov/householdpulsedata}{Census Bureau survey}
showing that four million fewer people were employed last week than the
week before. It was the fourth straight decline, suggesting that nearly
all the job gains since mid-May had been erased.

``At this stage, you're seeing all the wrong elements for recovery,''
said Gregory Daco, the chief U.S. economist at Oxford Economics. ``A
deteriorating health situation, a weakening labor market and a softening
path for demand.''

Mr. Daco said the rush to reopen in many states had been
counterproductive, contributing to the increasing virus caseloads,
particularly in the South and West, that are compelling businesses to
close again.

``Increasingly I fear that we're going to see net payrolls in July will
show an actual decline'' when the next monthly jobs report is released,
he added.

About 30 million people --- roughly one in five American workers --- are
drawing jobless benefits. Without congressional action to extend the
weekly federal supplement, the unemployed will be left with less money
to pay for food, medical care, rent and other bills. Also nearing an end
is the federal Paycheck Protection Program, which provided small
businesses with emergency loans that spared many workers from layoffs.

The stubbornly high rate of new unemployment claims ``suggests that the
nature of the downturn has changed from early on,'' said Ernie Tedeschi,
a policy economist at the equity research firm Evercore ISI. In addition
to reflecting renewed shutdowns, he said, the setbacks on the job front
may indicate something more fundamental.

\hypertarget{latest-updates-economy}{%
\section{\texorpdfstring{\href{https://www.nytimes.com/live/2020/08/03/business/stock-market-today-coronavirus?action=click\&pgtype=Article\&state=default\&region=MAIN_CONTENT_1\&context=storylines_live_updates}{Latest
Updates:
Economy}}{Latest Updates: Economy}}\label{latest-updates-economy}}

\href{https://www.nytimes.com/live/2020/08/03/business/stock-market-today-coronavirus?action=click\&pgtype=Article\&state=default\&region=MAIN_CONTENT_1\&context=storylines_live_updates\#the-chicago-fed-president-says-its-up-to-congress-to-save-the-economy}{13h
ago}

\href{https://www.nytimes.com/live/2020/08/03/business/stock-market-today-coronavirus?action=click\&pgtype=Article\&state=default\&region=MAIN_CONTENT_1\&context=storylines_live_updates\#the-chicago-fed-president-says-its-up-to-congress-to-save-the-economy}{The
Chicago Fed president says it's up to Congress to save the economy.}

\href{https://www.nytimes.com/live/2020/08/03/business/stock-market-today-coronavirus?action=click\&pgtype=Article\&state=default\&region=MAIN_CONTENT_1\&context=storylines_live_updates\#faa-says-boeing-has-effectively-mitigated-defects-in-the-737-max}{13h
ago}

\href{https://www.nytimes.com/live/2020/08/03/business/stock-market-today-coronavirus?action=click\&pgtype=Article\&state=default\&region=MAIN_CONTENT_1\&context=storylines_live_updates\#faa-says-boeing-has-effectively-mitigated-defects-in-the-737-max}{F.A.A.
says Boeing has `effectively mitigated' defects in the 737 Max.}

\href{https://www.nytimes.com/live/2020/08/03/business/stock-market-today-coronavirus?action=click\&pgtype=Article\&state=default\&region=MAIN_CONTENT_1\&context=storylines_live_updates\#small-businesses-got-emergency-loans-but-not-what-they-expected}{16h
ago}

\href{https://www.nytimes.com/live/2020/08/03/business/stock-market-today-coronavirus?action=click\&pgtype=Article\&state=default\&region=MAIN_CONTENT_1\&context=storylines_live_updates\#small-businesses-got-emergency-loans-but-not-what-they-expected}{Small
businesses got emergency loans, but not what they expected.}

\href{https://www.nytimes.com/live/2020/08/03/business/stock-market-today-coronavirus?action=click\&pgtype=Article\&state=default\&region=MAIN_CONTENT_1\&context=storylines_live_updates}{See
more updates}

More live coverage:
\href{https://www.nytimes.com/2020/08/04/world/coronavirus-covid-19.html?action=click\&pgtype=Article\&state=default\&region=MAIN_CONTENT_1\&context=storylines_live_updates}{Global}

``It might be that businesses are running through their first line of
credit,'' he said, ``and now they're facing the music of an economy that
has recovered a little bit but not nearly enough.''

In that case, temporary business closings and layoffs would increasingly
turn into permanent ones.

Image

Indoor dining was prohibited this week at the food court of a shopping
center in Miami.Credit...Saul Martinez for The New York Times

Image

Waiting for help with claims this month at an unemployment office in
Midwest City, Okla.Credit...Sue Ogrocki/Associated Press

On Thursday, the owner of Ann Taylor and Lane Bryant became the latest
of a string of large retailers to
\href{https://www.nytimes.com/2020/07/23/business/ascena-bankruptcy-ann-taylor-lane-bryant.html}{file
for bankruptcy}. It announced that 1,600 of its 2,800 stores around the
country would be shut.

Wieden+Kennedy, an ad agency that has worked with clients like
McDonald's, Ford and Procter \& Gamble, said this week that it had laid
off 11 percent of its work force after paring expenses and cutting
executives' pay.

``We negotiated this as long as we could, but W+K and Covid-19 have
reached an impasse,'' the company said in a statement. ``How long this
will last seems to be anybody's guess, so we have had to make some hard
choices.''

During the worst of the last recession, weekly unemployment insurance
applications never exceeded 700,000. Since mid-March, new state claims
have yet to fall below a million. States have been whittling away at a
backlog of filings, but delays in some places persist.

Behnaz Mansouri, an attorney at the Unemployment Law Project in
Washington State, said her office was still averaging 200 phone calls a
week from people who had received no benefits after waiting months, or
who had inexplicably had them cut off.

Recently there has been slow progress, she said. A number of people who
appealed a decision in March, April and May are beginning to be called
in for hearings. Workers who have waited the longest, Ms. Mansouri said,
are often those who have disabilities or don't speak English well.

In Oklahoma, hundreds of frustrated workers
\href{https://www.washingtonpost.com/national/a-very-dark-feeling-hundreds-camp-out-in-oklahoma-unemployment-lines/2020/07/20/44d59cb6-c77a-11ea-a99f-3bbdffb1af38_story.html}{camped
out overnight} hoping to sort out delays with their unemployment claims
at one of the large-scale processing sessions that officials were
holding around the state. And in Texas, applicants have taken to the
state's
\href{https://www.facebook.com/texasworkforcecommission/}{Workforce
Commission page on Facebook} to complain of having been unable to reach
the agency despite calling hundreds of times.

Daniella Knight said her husband, Nicholas, applied for unemployment
benefits in Virginia when he was laid off in June from his job as a
litigation data analyst, and ``we have not gotten one dollar.''

\includegraphics{https://static01.nyt.com/images/2020/07/23/business/23virus-jobless-3/merlin_174848520_ef9b6038-0b5a-4ee2-ab74-62a0cd2c04bb-articleLarge.jpg?quality=75\&auto=webp\&disable=upscale}

Even before this setback, ``we were already barely making it,'' said Ms.
Knight, who lives in Alexandria. She had been working part time at a
property-management company during the day while her husband took care
of their three children --- 3, 5 and 9 years old. During his 4
p.m.-to-midnight shift, she took over the household duties. After
kissing the children good night, she worked at her second job, as a
pediatric sleep consultant.

Ms. Knight called it ``tag-team parenting.'' But she put up with the
exhaustion and stress so they could save enough to stop renting and buy
a house with more than one bathroom.

The coronavirus pandemic added full-time home-schooling to their load.
She stopped going to the small property-management office during the day
to avoid contagion, instead driving there at night when it was less
crowded or empty.

``Mom and Dad were at the end of our ropes, beyond exhausted,'' she
said. ``I started to have panic attacks.''

Her husband recently found another job, working for the government, but
has to wait at least six to eight weeks for his security clearance.

Image

The Knight family at home in Alexandria, Va.~The coronavirus outbreak
added full-time home-schooling to their work load.Credit...Michael A.
McCoy for The New York Times

``We still have to pay our bills, our utilities, our rent, everything,''
Ms. Knight said. The monthly cost of their health care alone is \$1,600,
which they had to tap their savings to pay. They are hoping the
unemployment benefits come through soon. **** ``We can't get by with
another two months without that,'' she said.

The pain of job losses can be found in every corner of the country, but
Black men have had particular difficulties, said
\href{https://www.gse.harvard.edu/faculty/peter-blair}{Peter Q. Blair},
a co-director of the Project on Workforce at the Harvard Graduate School
of Education.

The government's
\href{https://www.bls.gov/news.release/empsit.t02.htm}{June jobs report}
showed that while unemployment for every other group declined from May,
the rate for African-American males over 20 rose to 15.8 percent from
15.3 percent.

``It's important that we look at the way in which this crisis is having
a disparate effect on the African-American community, particularly Black
men,'' he said.

And while the overall jobless rate dipped in June to 11.1 percent from a
peak of 14.7 percent in April, troubling weaknesses are growing more
prominent.

``The increased joblessness will certainly hinder the economic recovery,
especially if the Congress fails to extend the supplemental benefits
that were part of the CARES Act,'' said Carl Tannenbaum, chief economist
at Northern Trust.

Passed at the end of the March, the legislation created a temporary
federal jobless program, Pandemic Unemployment Assistance, to cover
freelancers, part-time workers and others who do not qualify for regular
state jobless aid. It extended jobless benefits for an extra 13 weeks
for state recipients who exhausted their aid allotment. And it helped
jobless workers survive the cash crunch by approving a weekly \$600
benefit --- a supplement that essentially expires on Saturday.

That extra money ``provided critical support over the last several
months,'' said Rubeela Farooqi, chief U.S. economist at High Frequency
Economics. Such support --- even at a reduced level --- ``is going to be
increasingly important going forward,'' she said.

Ben Casselman and Tiffany Hsu contributed reporting.

Advertisement

\protect\hyperlink{after-bottom}{Continue reading the main story}

\hypertarget{site-index}{%
\subsection{Site Index}\label{site-index}}

\hypertarget{site-information-navigation}{%
\subsection{Site Information
Navigation}\label{site-information-navigation}}

\begin{itemize}
\tightlist
\item
  \href{https://help.nytimes.com/hc/en-us/articles/115014792127-Copyright-notice}{©~2020~The
  New York Times Company}
\end{itemize}

\begin{itemize}
\tightlist
\item
  \href{https://www.nytco.com/}{NYTCo}
\item
  \href{https://help.nytimes.com/hc/en-us/articles/115015385887-Contact-Us}{Contact
  Us}
\item
  \href{https://www.nytco.com/careers/}{Work with us}
\item
  \href{https://nytmediakit.com/}{Advertise}
\item
  \href{http://www.tbrandstudio.com/}{T Brand Studio}
\item
  \href{https://www.nytimes.com/privacy/cookie-policy\#how-do-i-manage-trackers}{Your
  Ad Choices}
\item
  \href{https://www.nytimes.com/privacy}{Privacy}
\item
  \href{https://help.nytimes.com/hc/en-us/articles/115014893428-Terms-of-service}{Terms
  of Service}
\item
  \href{https://help.nytimes.com/hc/en-us/articles/115014893968-Terms-of-sale}{Terms
  of Sale}
\item
  \href{https://spiderbites.nytimes.com}{Site Map}
\item
  \href{https://help.nytimes.com/hc/en-us}{Help}
\item
  \href{https://www.nytimes.com/subscription?campaignId=37WXW}{Subscriptions}
\end{itemize}
