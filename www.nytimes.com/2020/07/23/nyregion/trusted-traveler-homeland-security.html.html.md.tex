Sections

SEARCH

\protect\hyperlink{site-content}{Skip to
content}\protect\hyperlink{site-index}{Skip to site index}

\href{https://www.nytimes.com/section/nyregion}{New York}

\href{https://myaccount.nytimes.com/auth/login?response_type=cookie\&client_id=vi}{}

\href{https://www.nytimes.com/section/todayspaper}{Today's Paper}

\href{/section/nyregion}{New York}\textbar{}Homeland Security Dept.
Admits Making False Statements in Fight With N.Y.

\url{https://nyti.ms/32M12I0}

\begin{itemize}
\item
\item
\item
\item
\item
\end{itemize}

Advertisement

\protect\hyperlink{after-top}{Continue reading the main story}

Supported by

\protect\hyperlink{after-sponsor}{Continue reading the main story}

\hypertarget{homeland-security-dept-admits-making-false-statements-in-fight-with-ny}{%
\section{Homeland Security Dept. Admits Making False Statements in Fight
With
N.Y.}\label{homeland-security-dept-admits-making-false-statements-in-fight-with-ny}}

The surprise admission came as the Trump administration unexpectedly
reversed its decision to bar New Yorkers from programs that allow
travelers to speed through airports.

\includegraphics{https://static01.nyt.com/images/2020/07/23/nyregion/23trustedtraveler/merlin_168483513_862c35bc-b6bd-4ccd-9e96-877ebed58ea0-articleLarge.jpg?quality=75\&auto=webp\&disable=upscale}

By Ed Shanahan and
\href{https://www.nytimes.com/by/zolan-kanno-youngs}{Zolan Kanno-Youngs}

\begin{itemize}
\item
  July 23, 2020
\item
  \begin{itemize}
  \item
  \item
  \item
  \item
  \item
  \end{itemize}
\end{itemize}

Homeland Security officials made false statements in a bid to justify
expelling New York residents from programs that let United States
travelers speed through borders and airport lines, federal lawyers
admitted on Thursday.

The unusual admission, contained in a court filing, said the
inaccuracies ``undermine a central argument'' in the Trump
administration's case for barring New Yorkers from the programs after
the state passed a law enabling undocumented immigrants to get driver's
licenses.

The filing was a surprising retreat by the administration in its
continuing battle with Democratic-led states and cities over immigration
policy.

Federal officials had previously insisted that New York was an outlier
in the restrictions it placed on the access the immigration authorities
have to State Department of Motor Vehicles records.

For that reason, they argued, New York was endangering national security
and could not be trusted to participate in Global Entry and related
programs.

But in their filing on Thursday, the government lawyers acknowledged
that several other states, Washington, D.C., and some U.S. territories
also limited access to motor vehicle information and had not been
subject to similar clampdowns.

Against that backdrop, the filing said, ``The acting secretary of
homeland security has decided to restore New York residents' access to''
what is officially known as the Trusted Traveler Program ``effective
immediately.''

The court filing on Thursday came in response to lawsuits filed by New
York State and the New York Civil Liberties Union over the decision to
kick New Yorkers out of the programs.

``Defendants deeply regret the foregoing inaccurate or misleading
statements and apologize to the court and plaintiffs for the need to
make these corrections at this late stage in the litigation,'' said
Audrey Strauss, the acting United States attorney in Manhattan.

A spokesman for the United States attorney's office declined further
comment.

In their own filing, Justice Department lawyers for the Homeland
Security Department echoed Ms. Strauss's letter nearly word for word,
asking the court to ``accept this notice to correct the record, and
permit defendants to withdraw the aforementioned arguments.''

Letitia James, New York's attorney general, welcomed the department's
decision to reverse course, saying the state's lawsuit had been focused
on ``stopping the president's irrational, arbitrary and retaliatory
rule.''

Christopher Dunn, the civil liberties union's legal director, described
the document filed by Ms. Strauss's office as ``extraordinary.'' He said
it proved ``what we have said from the beginning: The Trump
administration suspended Global Entry in New York in retaliation for the
state's decision to grant drivers' licenses regardless of citizenship.''

The filing was the first tangible explanation for the timing of the
Department of Homeland Security's unexpected announcement on Thursday
that it was allowing New Yorkers back into what is officially known as
the Trusted Traveler Program.

The reversal came nearly six months after
\href{https://www.nytimes.com/2020/02/06/us/politics/dhs-new-york-global-entry.html}{the
department prohibited state residents from Global Entry and other
programs} because of the state's so-called Green Light law.

Unlike its counterparts in other states, New York's law --- which took
effect in December and has
\href{https://www.nytimes.com/2019/11/14/nyregion/immigrants-drivers-license.html}{been
a contentious subject} in and outside the state --- restricted the
immigration authorities' access to state motor vehicle records without a
court order.

The suspension of New Yorkers from the travel programs came as Mr. Trump
renewed his battle against cities and states that have embraced
so-called sanctuary laws.

On Feb. 4, Mr. Trump criticized New York in his State of the Union
address for not letting the police detain undocumented immigrants until
federal agents could pick them up for deportation proceedings, and he
\href{https://www.nytimes.com/2020/02/05/us/politics/state-of-union-transcript.html}{blamed
the city's sanctuary policies} for the
\href{https://www.nytimes.com/2020/01/14/nyregion/92-year-old-woman-queens-murder.html}{rape
and murder of a 92-year-old Queens woman} by an undocumented immigrant
who was accused in the crimes.

The next day, Chad F. Wolf, the acting homeland security secretary and a
favorite of Mr. Trump's for his hard-line stance on enforcement
matters,\href{https://www.foxnews.com/politics/dhs-global-entry-trusted-traveler-new-york-ice-sanctuary-law}{said
on Fox News} that New Yorkers would be barred from the travel programs
because of the sanctuary policies.

At the time, Gov. Andrew M. Cuomo condemned the move, which had the
potential to slow the travel routines of at least 175,000 New Yorkers,
as
``\href{https://www.nytimes.com/2020/02/06/nyregion/green-light-law-global-entry.html}{a
form of extortion}.''

The dispute escalated over the course of a week, until Mr. Cuomo met
with President Trump at the White House in February
\href{https://www.nytimes.com/2020/02/12/nyregion/global-entry-cuomo-trump.html}{in
hopes of working out a compromise}.

The agreement that was eventually reached, which was enacted in a state
budget bill in April, gives the federal authorities access to the motor
vehicle records of those who apply for trusted-traveler status and for
cars being shipped in and out of the country.

Mr. Cuomo said in a statement on Thursday that it was the compromise he
helped work out that paved the way for New Yorkers to be allowed back
into the travel programs.

``I am glad that this issue has finally been resolved for all New
Yorkers,'' the governor said in a statement.

In his own statement, Mr. Wolf said that ``we appreciate the information
sharing'' that enabled the department ``to move forward and begin once
again processing New York residents'' in
\href{https://ttp.cbp.dhs.gov/}{Trusted Traveler Programs}.

Neither man mentioned the court filing and the admissions it contained,
but Mr. Wolf did take a swipe at New York, saying the state ``continues
to maintain provisions that undermine the security of the American
people and purport to criminalize information sharing between law
enforcement entities.''

On Friday, Mr. Cuomo fired back, criticizing Mr. Wolf and other
officials sharply for imposing the restrictions in the first place and
for not telling the truth in trying to justify them.

Excluding New York from the travel programs, the governor said, had
harmed the state's economy by slowing the flow of goods and people at
border crossings and might have caused the coronavirus to spread at
airports among passengers forced to wait at crowded entry points.

``It was all politics all the time,'' said Mr. Cuomo, who called on
Attorney General William P. Barr and congressional Democrats to
investigate the matter and threatened to sue the federal government over
it. ``It was all exploitation all the time. And they hurt this state
because of it.''

For Judge Jesse M. Furman of Federal District Court in Manhattan, who is
overseeing the case, the litigation is not the first case before him to
involve Trump administration officials' making false statements.

Last year, in a ruling the Supreme Court ultimately upheld, he blocked
the Commerce Department from adding a question on American citizenship
to this year's census.

In
\href{https://www.brennancenter.org/sites/default/files/legal-work/2019-01-15-574-Findings\%20Of\%20Fact.pdf}{a
lengthy and stinging opinion}, Judge Furman criticized Wilbur L. Ross
Jr., the commerce secretary, and Mr. Ross's aides for giving false or
misleading statements under oath as they struggled to explain their
rationale for adding the question.

Mr. Ross, Judge Furman wrote, had also broken ``a veritable
smorgasbord'' of federal rules in ordering that the citizenship question
be added while also cherry-picking facts to support his views, ignoring
or twisting contrary evidence and hiding deliberations from Census
Bureau experts.

For many New Yorkers, the practical impact of being allowed to
participate in the travel programs again will be minimal for now.

Many countries are not allowing American travelers in amid the pandemic
and a surge in cases in many parts of the United States.

In addition, Customs and Border Protection, the agency that oversees the
programs, said on Monday that the enrollment centers where applicants
are interviewed would be closed until at least Sept. 8.

Benjamin Weiser and Michael Gold contributed reporting.

Advertisement

\protect\hyperlink{after-bottom}{Continue reading the main story}

\hypertarget{site-index}{%
\subsection{Site Index}\label{site-index}}

\hypertarget{site-information-navigation}{%
\subsection{Site Information
Navigation}\label{site-information-navigation}}

\begin{itemize}
\tightlist
\item
  \href{https://help.nytimes.com/hc/en-us/articles/115014792127-Copyright-notice}{©~2020~The
  New York Times Company}
\end{itemize}

\begin{itemize}
\tightlist
\item
  \href{https://www.nytco.com/}{NYTCo}
\item
  \href{https://help.nytimes.com/hc/en-us/articles/115015385887-Contact-Us}{Contact
  Us}
\item
  \href{https://www.nytco.com/careers/}{Work with us}
\item
  \href{https://nytmediakit.com/}{Advertise}
\item
  \href{http://www.tbrandstudio.com/}{T Brand Studio}
\item
  \href{https://www.nytimes.com/privacy/cookie-policy\#how-do-i-manage-trackers}{Your
  Ad Choices}
\item
  \href{https://www.nytimes.com/privacy}{Privacy}
\item
  \href{https://help.nytimes.com/hc/en-us/articles/115014893428-Terms-of-service}{Terms
  of Service}
\item
  \href{https://help.nytimes.com/hc/en-us/articles/115014893968-Terms-of-sale}{Terms
  of Sale}
\item
  \href{https://spiderbites.nytimes.com}{Site Map}
\item
  \href{https://help.nytimes.com/hc/en-us}{Help}
\item
  \href{https://www.nytimes.com/subscription?campaignId=37WXW}{Subscriptions}
\end{itemize}
