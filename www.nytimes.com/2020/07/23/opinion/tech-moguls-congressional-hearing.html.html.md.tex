Sections

SEARCH

\protect\hyperlink{site-content}{Skip to
content}\protect\hyperlink{site-index}{Skip to site index}

\href{https://myaccount.nytimes.com/auth/login?response_type=cookie\&client_id=vi}{}

\href{https://www.nytimes.com/section/todayspaper}{Today's Paper}

\href{/section/opinion}{Opinion}\textbar{}Tech Is About Power. And These
Four Moguls Have Too Much of It.

\href{https://nyti.ms/3hBlD69}{https://nyti.ms/3hBlD69}

\begin{itemize}
\item
\item
\item
\item
\item
\item
\end{itemize}

Advertisement

\protect\hyperlink{after-top}{Continue reading the main story}

\href{/section/opinion}{Opinion}

Supported by

\protect\hyperlink{after-sponsor}{Continue reading the main story}

\hypertarget{tech-is-about-power-and-these-four-moguls-have-too-much-of-it}{%
\section{Tech Is About Power. And These Four Moguls Have Too Much of
It.}\label{tech-is-about-power-and-these-four-moguls-have-too-much-of-it}}

A congressional hearing on Monday should address this imbalance.

\includegraphics{https://static01.nyt.com/images/2018/08/02/opinion/02swisher/02swisher-thumbLarge.png}

By Kara Swisher

Ms. Swisher covers technology and is a contributing opinion writer.

\begin{itemize}
\item
  July 23, 2020
\item
  \begin{itemize}
  \item
  \item
  \item
  \item
  \item
  \item
  \end{itemize}
\end{itemize}

\includegraphics{https://static01.nyt.com/images/2020/07/23/opinion/23swisher1/merlin_163118160_e495a57a-95b7-4c7b-8f02-83e921e3a2c3-articleLarge.jpg?quality=75\&auto=webp\&disable=upscale}

One of the more bizarre cable-show-bro handoffs of late happened this
week between Fox News hosts Tucker Carlson and Sean Hannity concerning
the \href{https://time.com/5869262/jeff-bezos-13-billion/}{\$13 billion
one-day spike} in the Amazon chief executive Jeff Bezos' fortune.

Using a chyron calling the e-commerce mogul a ``fat cat,'' Mr. Carlson
noted that the world's wealthiest man had become ``extremely rich from
all of this, including a lot of the suffering,'' referring to the
pandemic, which has made Amazon's delivery services a must-have for many
people and has rocketed its stock to the stratosphere.

``I'm not against wealth accumulation. I'm not against free enterprise,
but \$13 billion in a day suggests something is skewed with the system,
no?,'' he asked a guest, who vehemently agreed with Mr. Carlson that
\emph{something} must be up, including floating an evidence-free idea
that Amazon was somehow wanting to keep the pandemic going so it can
continue to benefit.

Mr. Hannity came on next and quickly objected to the idea that the rich
should be hindered from getting richer: ``People can make money. They
provide goods and services people want, need and desire? That's America.
It's called freedom --- capitalism --- and as long as it's honest,
right? People decide.''

It was, how shall we say, awkward, with Mr. Carlson looking peeved to be
tweaked by a colleague. And so Mr. Hannity soon did a semi-about-face
\href{https://twitter.com/seanhannity/status/1285761201344057344}{on
Twitter}. ``I apologize for any misunderstanding to Tucker and the Fox
audience. I support freedom and capitalism,'' he wrote. ``Not people
taking advantage of a pandemic. If I see such evidence, I will obviously
condemn it.''

Obviously! But aside from the odd sight of a couple of rich guys sitting
around judging a \emph{really} rich guy, their joint fascination with
the money --- and so, so much money --- was a shiny object we all need
to stop looking at.

The wealth is extraordinarily distracting and, in fact, I wrote earlier
this year that I thought the tech leaders would be richer than ever
post-pandemic because their businesses had the heft and products to
thrive in the crisis.

That has certainly turned out to be true. According to
\href{https://www.bloomberg.com/billionaires/}{Bloomberg's Billionaires
Index}, No. 1-ranked Mr. Bezos has become close to \$70 billion richer
over the past year, for a total net worth of \$184 billion, while the
Facebook chief executive Mark Zuckerberg's net worth has jumped \$12
billion to \$91.1 billion. In fact, all but two of the top 15 on the
list are connected to American tech, including No. 13, Mr. Bezos' former
wife, Mackenzie Bezos, who is now \$24.2 billion richer in 2020 with a
\$61.3 billion fortune.

And while there is something admirable about these moguls' successes,
against the backdrop of tens of millions of Americans out of work and
seas of underpaid wage employees on the front lines of the crisis, the
income inequality feels obscene. This is especially true given the tax
breaks for the very wealthy in recent years and, really, for a long
time.

But a focus on the wealth also obscures the unprecedented accumulation
of \emph{power} by tech giants and the lack of any significant
regulation or incentives for real accountability. They are always going
to be very rich, so get used to it, but they don't necessarily have to
be as powerful if we act now.

And this must be the main topic of
\href{https://www.cnbc.com/2020/07/01/apple-google-amazon-and-facebook-ceos-to-testify-in-congress.html}{a
congressional hearing} on Monday when the House Judiciary Committee's
antitrust subcommittee questions the four top tech leaders: Mr. Bezos,
Mr. Zuckerberg, Tim Cook of Apple and Sundar Pichai of Alphabet, owner
of Google and YouTube.

The gathering of all four chief executives is a big deal, even if some
think that appearing as a group will give each individual leader cover,
resulting in less substantive questioning. And there are worries that
the event will lack the usual drama, since it is likely to be largely
remote, due to the coronavirus.

But it's critical that lawmakers block out all the noise that has grown
around the industry and aim at only discussing the repercussions of
unfettered power. All the major problems related to tech stem directly
from this, whether it be privacy violations or hate speech and
misinformation or unfair market dominance or addiction or \ldots{} fill
in the blank.

We must think of it all as \emph{systemic}, fueled by complete control
over certain areas by tech companies, without adequate guardrails from
publicly elected officials, which every other major industry has been
subject to. Tech does not play by the rules only because there are no
rules to speak of. So why shouldn't they do as they please?

Tristan Harris, a former design ethicist at Google who more recently
co-founded the Center for Humane Technology, put it perfectly in
\href{https://www.vox.com/recode/2019/5/6/18530860/tristan-harris-human-downgrading-time-well-spent-kara-swisher-recode-decode-podcast-interview}{a
podcast interview} with me last year: ``We need to move from this
disconnected set of grievances and scandals, that these problems are
seemingly separate: tech addiction, polarization, outrage-ification of
culture, the rise in vanities, micro-celebrity culture, everyone has to
be famous. These are not separate problems. They're actually all coming
from one thing, which is the race to capture human attention by tech
giants.''

And it has become a completely fixed race. Because of their heft, these
behemoths block every lane and there is no space for innovative small
companies to pass them, especially those that are faster or with better
ideas. The debate about breakup or levying fines or writing regulations
should also be a debate about innovation. What about all of the useful
inventions that do not happen when there is only one or maybe two real
games in town in social media, in search, in online video, in apps and
in e-commerce.

And in communications too, which is why the letter that Representative
Jim Jordan, who is in the Republican minority,
\href{https://abcnews.go.com/Politics/house-republicans-call-twitter-ceo-jack-dorsey-join/story?id=71925923}{wrote
Wednesday} to ask Democratic Representative Jerry Nadler to invite Jack
Dorsey of Twitter as a witness on Monday was an interesting idea.

``We believe there is a bipartisan interest to hear from Twitter about
its power in the marketplace,'' he wrote, before adding content
moderation and hacking issues to the pile. This is the tiresome and
inaccurate conservatives-being-silenced alleyway Mr. Jordan almost
always goes down, frittering away time that would be better spent
talking about power and the need to disperse it to many more.

In fact, having Mr. Dorsey there Monday to talk about that would be
interesting, since it is a far smaller company than the other four, and
it actually suffers due to its tiny size in the online advertising
market. And while Mr. Dorsey may also be very wealthy from Twitter and
his payments company called Square --- with an \$8.37 billion fortune,
which is up \$3.62 billon this year --- he and his companies are
also-rans by comparison.

So, take note Mr. Carlson and Mr. Hannity, you can follow the money, but
what you should be following is the power that makes all that money.

\emph{The Times is committed to publishing}
\href{https://www.nytimes.com/2019/01/31/opinion/letters/letters-to-editor-new-york-times-women.html}{\emph{a
diversity of letters}} \emph{to the editor. We'd like to hear what you
think about this or any of our articles. Here are some}
\href{https://help.nytimes.com/hc/en-us/articles/115014925288-How-to-submit-a-letter-to-the-editor}{\emph{tips}}\emph{.
And here's our email:}
\href{mailto:letters@nytimes.com}{\emph{letters@nytimes.com}}\emph{.}

\emph{Follow The New York Times Opinion section on}
\href{https://www.facebook.com/nytopinion}{\emph{Facebook}}\emph{,}
\href{http://twitter.com/NYTOpinion}{\emph{Twitter (@NYTopinion)}}
\emph{and}
\href{https://www.instagram.com/nytopinion/}{\emph{Instagram}}\emph{,
and sign up for the}
\href{http://www.nytimes.com/newsletters/opiniontoday/}{\emph{Opinion
Today newsletter}}\emph{.}

Advertisement

\protect\hyperlink{after-bottom}{Continue reading the main story}

\hypertarget{site-index}{%
\subsection{Site Index}\label{site-index}}

\hypertarget{site-information-navigation}{%
\subsection{Site Information
Navigation}\label{site-information-navigation}}

\begin{itemize}
\tightlist
\item
  \href{https://help.nytimes.com/hc/en-us/articles/115014792127-Copyright-notice}{©~2020~The
  New York Times Company}
\end{itemize}

\begin{itemize}
\tightlist
\item
  \href{https://www.nytco.com/}{NYTCo}
\item
  \href{https://help.nytimes.com/hc/en-us/articles/115015385887-Contact-Us}{Contact
  Us}
\item
  \href{https://www.nytco.com/careers/}{Work with us}
\item
  \href{https://nytmediakit.com/}{Advertise}
\item
  \href{http://www.tbrandstudio.com/}{T Brand Studio}
\item
  \href{https://www.nytimes.com/privacy/cookie-policy\#how-do-i-manage-trackers}{Your
  Ad Choices}
\item
  \href{https://www.nytimes.com/privacy}{Privacy}
\item
  \href{https://help.nytimes.com/hc/en-us/articles/115014893428-Terms-of-service}{Terms
  of Service}
\item
  \href{https://help.nytimes.com/hc/en-us/articles/115014893968-Terms-of-sale}{Terms
  of Sale}
\item
  \href{https://spiderbites.nytimes.com}{Site Map}
\item
  \href{https://help.nytimes.com/hc/en-us}{Help}
\item
  \href{https://www.nytimes.com/subscription?campaignId=37WXW}{Subscriptions}
\end{itemize}
