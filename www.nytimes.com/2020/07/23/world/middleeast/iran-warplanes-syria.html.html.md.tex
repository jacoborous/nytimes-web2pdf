Sections

SEARCH

\protect\hyperlink{site-content}{Skip to
content}\protect\hyperlink{site-index}{Skip to site index}

\href{https://www.nytimes.com/section/world/middleeast}{Middle East}

\href{https://myaccount.nytimes.com/auth/login?response_type=cookie\&client_id=vi}{}

\href{https://www.nytimes.com/section/todayspaper}{Today's Paper}

\href{/section/world/middleeast}{Middle East}\textbar{}Iranian Civilian
Jet Swerves to Avoid American Warplane in Syria

\url{https://nyti.ms/3jv5P6J}

\begin{itemize}
\item
\item
\item
\item
\item
\end{itemize}

Advertisement

\protect\hyperlink{after-top}{Continue reading the main story}

Supported by

\protect\hyperlink{after-sponsor}{Continue reading the main story}

\hypertarget{iranian-civilian-jet-swerves-to-avoid-american-warplane-in-syria}{%
\section{Iranian Civilian Jet Swerves to Avoid American Warplane in
Syria}\label{iranian-civilian-jet-swerves-to-avoid-american-warplane-in-syria}}

Several passengers were reported injured by the sudden drop in altitude
by the passenger plane, which landed at Beirut's airport.

By \href{https://www.nytimes.com/by/vivian-yee}{Vivian Yee},
\href{https://www.nytimes.com/by/farnaz-fassihi}{Farnaz Fassihi} and
\href{https://www.nytimes.com/by/eric-schmitt}{Eric Schmitt}

\begin{itemize}
\item
  July 23, 2020
\item
  \begin{itemize}
  \item
  \item
  \item
  \item
  \item
  \end{itemize}
\end{itemize}

BEIRUT, Lebanon --- An Iranian passenger plane en route from Iran to
Beirut swerved and dropped abruptly on Thursday to avoid a nearby
American fighter jet, injuring several passengers before landing in
Beirut.

\href{https://www.mashreghnews.ir/news/1098436/\%D9\%81\%DB\%8C\%D9\%84\%D9\%85-\%D8\%A7\%D8\%B3\%D8\%B1\%D8\%A7\%DB\%8C\%DB\%8C\%D9\%84-\%D9\%87\%D9\%88\%D8\%A7\%D9\%BE\%DB\%8C\%D9\%85\%D8\%A7\%DB\%8C-\%D9\%85\%D8\%B3\%D8\%A7\%D9\%81\%D8\%B1\%D8\%A8\%D8\%B1\%DB\%8C-\%D9\%85\%D8\%A7\%D9\%87\%D8\%A7\%D9\%86-\%D8\%B1\%D8\%A7-\%D8\%AA\%D9\%87\%D8\%AF\%DB\%8C\%D8\%AF-\%DA\%A9\%D8\%B1\%D8\%AF}{Videos}
\href{https://www.mashreghnews.ir/news/1098440/\%D9\%81\%DB\%8C\%D9\%84\%D9\%85-\%D9\%84\%D8\%AD\%D8\%B8\%D9\%87-\%D8\%AA\%D9\%87\%D8\%AF\%DB\%8C\%D8\%AF-\%D9\%87\%D9\%88\%D8\%A7\%D9\%BE\%DB\%8C\%D9\%85\%D8\%A7\%DB\%8C-\%D9\%85\%D8\%A7\%D9\%87\%D8\%A7\%D9\%86-\%D8\%AA\%D9\%88\%D8\%B3\%D8\%B7-\%D8\%AF\%D9\%88-\%D8\%AC\%D9\%86\%DA\%AF\%D9\%86\%D8\%AF\%D9\%87-\%D8\%A7\%D8\%B3\%D8\%B1\%D8\%A7\%DB\%8C\%DB\%8C\%D9\%84\%DB\%8C}{broadcast}
by Iranian and pro-Iran Lebanese media, which said the footage was taken
by passengers, showed a fighter jet flying alongside the passenger
plane, operated by Mahan Air, a privately owned Iranian airline.

Passengers then screamed as sudden turbulence seized the plane. In the
aftermath, one video showed a passenger with his face and head bloodied,
as well as a man lying down, apparently unconscious, while someone
tended to him. Oxygen masks dangled overhead.

Capt. Bill Urban, a spokesman for the U.S. Central Command, said in a
statement later Thursday that an Air Force F-15 on ``a routine air
mission'' near a small American military base in southern Syria had
conducted ``a standard visual inspection of a Mahan Air passenger
airliner.''

American fighter jets fly daily patrols near the base, Al-Tanf, where
150 to 200 U.S. troops train Syrian fighters, known as Maghawir al
Thawra, who are fighting the Islamic State.

Captain Urban said the encounter on Thursday was conducted at ``a safe
distance of approximately 1,000 meters'' and was done to ``ensure the
safety of coalition personnel.''

``Once the F-15 pilot identified the aircraft as a Mahan Air passenger
plane,'' Captain Urban added, ``the F-15 safely opened distance from the
aircraft.'' He said the encounter was done ``in accordance with
international standards.''

A U.S. military official said a second U.S. Air Force F-15 was at least
two miles away from the Mahan Air jetliner, but only one of the U.S.
aircraft closed to about 1,000 meters.

Iranian state television, citing an interview with the Mahan Air pilot,
had reported earlier that the fighter jets were American.

Lebanese media said an elderly passenger had been taken to a hospital
affiliated with Hezbollah, the Iran-backed Lebanese militia and
political party. Lebanese civil aviation authorities said the plane had
been carrying 150 passengers, some of whom suffered minor injuries.

The American base at Al-Tanf, which sits near southern Syria's border
with Jordan and Iraq, is strategically located to block Iran from
controlling a land route through Iraq to Syria and Lebanon.

In recent months, the American-backed Syrian forces have skirmished with
roving bands of suspected Islamic State fighters near Al-Tanf. In May
2017, American warplanes attacked a pro-Syrian government convoy that
ignored warnings and violated a restricted zone around the base.

The American-led international coalition fighting the Islamic State in
Syria flies combat air patrols over northeastern Syria to support about
500 American troops who carry out missions with Syrian Kurdish allies on
the ground to counter pockets of ISIS fighters.

The Mahan Air encounter came as Iran was already on edge after a series
of mysterious
\href{https://www.nytimes.com/2020/07/15/world/middleeast/iran-ships-fire-explosions.html}{explosions}
and violent attacks against its civilian and military infrastructure,
including at a
\href{https://www.nytimes.com/2020/07/05/world/middleeast/iran-Natanz-nuclear-damage.html}{nuclear
fuel enrichment complex} in early July. Iranian officials have
attributed some of the recent attacks to sabotage, though they have not
said whether they suspect the United States.

The encounter could amplify tensions between Iran and the United States,
which is pursuing a hard-line campaign of sanctions and other actions
against Iran that the Trump administration calls a ``maximum pressure''
strategy.

Abbas Mousavi, a spokesman for Iran's Foreign Ministry, said Iran had
contacted the Swiss Embassy in Tehran, which protects American interests
in Iran, to warn that the United States would be held accountable if
anything happened to the Mahan Air flight, which later left Beirut empty
to return to Iran.

Mr. Mousavi said Iran's mission to the United Nations had also conveyed
this message to the U.N. secretary general, António Guterres.

``We are investigating the details of this incident and after
information is complete we will take necessary legal and political
measures,'' Mr. Mousavi said.

A number of analysts said the Mahan Air episode appeared to fit a
pattern of recent efforts to unnerve and destabilize Iran.

``The timing of this incident is revealing,'' said Nader Hashemi,
director of the Center for Middle East Studies at the University of
Denver. ``It takes place against the backdrop of recent bombings in Iran
that are widely attributed to Israel with the blessing of the U.S.A.''

It remained far from clear, however, whether the warplane action was
deliberate.

Hesameddin Ashena, a senior adviser to President Hassan Rouhani, wrote
on Twitter, ``Those who love the lives of their leaders don't play with
the lives of our passengers.''

While Iran and the United States have many longstanding grievances, for
Iran one of the most potent remains the
\href{https://timesmachine.nytimes.com/timesmachine/1988/07/04/issue.html}{1988
shooting down of an Iranian passenger plane} by the Vincennes, an
American warship that had been patrolling in the Persian Gulf.

The plane, Iran Air Flight 655, carried 290 people. The United States
later called it ``a tragic and regrettable accident,'' and subsequently
paid millions to settle \href{https://www.icj-cij.org/en/case/79}{a
lawsuit that Iran filed} at the International Court of Justice.

Vivian Yee reported from Beirut, Farnaz Fassihi from New York, and Eric
Schmitt from Washington. Reporting was contributed by Hwaida Saad from
Beirut, Lebanon, Ronen Bergman from Tel Aviv, and Thomas Gibbons-Neff
from Washington.

Advertisement

\protect\hyperlink{after-bottom}{Continue reading the main story}

\hypertarget{site-index}{%
\subsection{Site Index}\label{site-index}}

\hypertarget{site-information-navigation}{%
\subsection{Site Information
Navigation}\label{site-information-navigation}}

\begin{itemize}
\tightlist
\item
  \href{https://help.nytimes.com/hc/en-us/articles/115014792127-Copyright-notice}{©~2020~The
  New York Times Company}
\end{itemize}

\begin{itemize}
\tightlist
\item
  \href{https://www.nytco.com/}{NYTCo}
\item
  \href{https://help.nytimes.com/hc/en-us/articles/115015385887-Contact-Us}{Contact
  Us}
\item
  \href{https://www.nytco.com/careers/}{Work with us}
\item
  \href{https://nytmediakit.com/}{Advertise}
\item
  \href{http://www.tbrandstudio.com/}{T Brand Studio}
\item
  \href{https://www.nytimes.com/privacy/cookie-policy\#how-do-i-manage-trackers}{Your
  Ad Choices}
\item
  \href{https://www.nytimes.com/privacy}{Privacy}
\item
  \href{https://help.nytimes.com/hc/en-us/articles/115014893428-Terms-of-service}{Terms
  of Service}
\item
  \href{https://help.nytimes.com/hc/en-us/articles/115014893968-Terms-of-sale}{Terms
  of Sale}
\item
  \href{https://spiderbites.nytimes.com}{Site Map}
\item
  \href{https://help.nytimes.com/hc/en-us}{Help}
\item
  \href{https://www.nytimes.com/subscription?campaignId=37WXW}{Subscriptions}
\end{itemize}
