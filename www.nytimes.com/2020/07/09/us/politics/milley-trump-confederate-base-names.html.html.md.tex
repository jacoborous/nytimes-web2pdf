Sections

SEARCH

\protect\hyperlink{site-content}{Skip to
content}\protect\hyperlink{site-index}{Skip to site index}

\href{https://www.nytimes.com/section/politics}{Politics}

\href{https://myaccount.nytimes.com/auth/login?response_type=cookie\&client_id=vi}{}

\href{https://www.nytimes.com/section/todayspaper}{Today's Paper}

\href{/section/politics}{Politics}\textbar{}Milley Calls for `Hard Look'
at Renaming Bases Honoring Confederates

\url{https://nyti.ms/3gOfzXz}

\begin{itemize}
\item
\item
\item
\item
\item
\end{itemize}

Advertisement

\protect\hyperlink{after-top}{Continue reading the main story}

Supported by

\protect\hyperlink{after-sponsor}{Continue reading the main story}

\hypertarget{milley-calls-for-hard-look-at-renaming-bases-honoring-confederates}{%
\section{Milley Calls for `Hard Look' at Renaming Bases Honoring
Confederates}\label{milley-calls-for-hard-look-at-renaming-bases-honoring-confederates}}

Gen. Mark A. Milley, the chairman of the Joint Chiefs of Staff, told a
House hearing that ``there is no place in our armed forces for
manifestations or symbols of racism, bias or discrimination.''

\includegraphics{https://static01.nyt.com/images/2020/07/09/us/politics/09dc-military/09dc-military-videoSixteenByNine3000.jpg}

\href{https://www.nytimes.com/by/helene-cooper}{\includegraphics{https://static01.nyt.com/images/2018/08/24/multimedia/author-helene-cooper/author-helene-cooper-thumbLarge.png}}

By \href{https://www.nytimes.com/by/helene-cooper}{Helene Cooper}

\begin{itemize}
\item
  July 9, 2020
\item
  \begin{itemize}
  \item
  \item
  \item
  \item
  \item
  \end{itemize}
\end{itemize}

WASHINGTON --- The top military official in the United States called on
Thursday for taking ``a hard look'' at changing the names of Army bases
honoring Confederate officers who had fought against the Union during
the Civil War, disagreeing with President Trump and further exposing a
divide between the military and the president.

Gen. Mark A. Milley, the chairman of the Joint Chiefs of Staff and Mr.
Trump's senior military adviser, told a House hearing that the base
names had become an issue of ``divisiveness.''

Ten Army bases that honor Confederate generals who fought to defend the
slaveholding South have been the focus of a growing movement for change.

``There is no place in our armed forces for manifestations or symbols of
racism, bias or discrimination,'' General Milley said.

``The Confederacy, the American Civil War, was fought, and it was an act
of rebellion,'' he said. ``It was an act of treason, at the time,
against the Union, against the Stars and Stripes, against the U.S.
Constitution. Those officers turned their back on their oath.''

General Milley had warned White House officials this month that he
planned to give his unvarnished opinion to Congress if the base issue
came up, an administration official said. But his assessment was
nonetheless likely to anger the president, who has made clear his
disdain for both the waves of demonstrations for racial justice that
swept the country last month and the calls to rename the Confederate
bases.

But just as Mr. Trump has shown an
\href{https://www.nytimes.com/2020/06/23/us/politics/trump-race-racism-protests.htmlhttps://www.nytimes.com/2020/06/23/us/politics/trump-race-racism-protests.html}{increasing
willingness to air divisive and even racist viewpoints}, military
leaders have also shown more willingness to publicly express views at
odds with their commander in chief's.

General Milley infuriated the president last month when he issued a
\href{https://www.nytimes.com/2020/06/11/us/politics/trump-milley-military-protests-lafayette-square.html}{public
apology} for taking part in Mr. Trump's walk across Lafayette Square for
a photo op after the authorities used tear gas and rubber bullets to
clear the area of peaceful protesters. ``I should not have been there,''
General Milley said later.

The
\href{https://www.nytimes.com/2020/06/11/us/military-bases-confederates.html}{10
bases named after Confederate generals} are all in the South: Fort Bragg
in North Carolina; Fort Hood in Texas; Fort Benning and Fort Gordon in
Georgia; Fort A.P. Hill, Fort Pickett and Fort Lee in Virginia; Camp
Beauregard and Fort Polk in Louisiana; and Fort Rucker in Alabama.

Critics argue that the men lionized by these base names were traitors
who fought the very military that now honors them, and that glorifying
them is a boon to racist groups.

Maj. Gen. George Pickett, for instance, led an infantry assault against
Union soldiers at the Battle of Gettysburg, while Col. Edmund Rucker,
who was wounded and captured during the Battle of Nashville in 1864, was
later released in a prisoner exchange organized by the Ku Klux Klan's
first grand wizard, Gen. Nathan Bedford Forrest.

At the Pentagon, Defense Secretary Mark T. Esper and General Milley, as
well as senior Army, Navy and Air Force officials, have been anxious to
show understanding of the public anger over racial inequity that has
also manifested itself among those in uniform. They have held meetings
to discuss the
\href{https://www.nytimes.com/2020/05/25/us/politics/military-minorities-leadership.html}{gap
in the military} between its mostly white officer corps and its diverse
ranks, where 43 percent are people of color.

But Mr. Trump grew upset when he saw articles about the possibility of
renaming bases, according to administration officials.

Mark Meadows, the White House chief of staff, encouraged the president
to block any attempt to change the names, the officials said. Mr. Trump
has tweeted several times to voice his ire about renaming the bases, in
posts that have infuriated senior defense officials.

``The United States of America trained and deployed our HEROES on these
Hallowed Grounds, and won two World Wars,''
\href{https://twitter.com/realDonaldTrump/status/1270787974880526337}{Mr.
Trump wrote} in
\href{https://twitter.com/realDonaldTrump/status/1270787975719391233}{a
string} of
\href{https://twitter.com/realDonaldTrump/status/1270787978626052096}{Twitter
posts}. ``Therefore, my Administration will not even consider the
renaming of these Magnificent and Fabled Military Installations. Our
history as the Greatest Nation in the World will not be tampered with.
Respect our Military!''

The president even threatened to veto the military spending bill passed
by Congress if it contained a requirement to rename the bases.

Advertisement

\protect\hyperlink{after-bottom}{Continue reading the main story}

\hypertarget{site-index}{%
\subsection{Site Index}\label{site-index}}

\hypertarget{site-information-navigation}{%
\subsection{Site Information
Navigation}\label{site-information-navigation}}

\begin{itemize}
\tightlist
\item
  \href{https://help.nytimes.com/hc/en-us/articles/115014792127-Copyright-notice}{©~2020~The
  New York Times Company}
\end{itemize}

\begin{itemize}
\tightlist
\item
  \href{https://www.nytco.com/}{NYTCo}
\item
  \href{https://help.nytimes.com/hc/en-us/articles/115015385887-Contact-Us}{Contact
  Us}
\item
  \href{https://www.nytco.com/careers/}{Work with us}
\item
  \href{https://nytmediakit.com/}{Advertise}
\item
  \href{http://www.tbrandstudio.com/}{T Brand Studio}
\item
  \href{https://www.nytimes.com/privacy/cookie-policy\#how-do-i-manage-trackers}{Your
  Ad Choices}
\item
  \href{https://www.nytimes.com/privacy}{Privacy}
\item
  \href{https://help.nytimes.com/hc/en-us/articles/115014893428-Terms-of-service}{Terms
  of Service}
\item
  \href{https://help.nytimes.com/hc/en-us/articles/115014893968-Terms-of-sale}{Terms
  of Sale}
\item
  \href{https://spiderbites.nytimes.com}{Site Map}
\item
  \href{https://help.nytimes.com/hc/en-us}{Help}
\item
  \href{https://www.nytimes.com/subscription?campaignId=37WXW}{Subscriptions}
\end{itemize}
