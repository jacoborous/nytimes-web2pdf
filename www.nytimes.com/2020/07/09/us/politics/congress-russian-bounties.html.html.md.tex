Sections

SEARCH

\protect\hyperlink{site-content}{Skip to
content}\protect\hyperlink{site-index}{Skip to site index}

\href{https://www.nytimes.com/section/politics}{Politics}

\href{https://myaccount.nytimes.com/auth/login?response_type=cookie\&client_id=vi}{}

\href{https://www.nytimes.com/section/todayspaper}{Today's Paper}

\href{/section/politics}{Politics}\textbar{}Congress Presses Military
Leaders on Suspected Russian Bounties

\url{https://nyti.ms/3egH5Lk}

\begin{itemize}
\item
\item
\item
\item
\item
\end{itemize}

Advertisement

\protect\hyperlink{after-top}{Continue reading the main story}

Supported by

\protect\hyperlink{after-sponsor}{Continue reading the main story}

\hypertarget{congress-presses-military-leaders-on-suspected-russian-bounties}{%
\section{Congress Presses Military Leaders on Suspected Russian
Bounties}\label{congress-presses-military-leaders-on-suspected-russian-bounties}}

Two House hearings grappled with a C.I.A. assessment that Russia offered
payments to kill American troops in Afghanistan --- and White House
inaction on the months-old judgment.

\includegraphics{https://static01.nyt.com/images/2020/07/09/us/politics/09dc-intel/09dc-intel-articleLarge.jpg?quality=75\&auto=webp\&disable=upscale}

By \href{https://www.nytimes.com/by/charlie-savage}{Charlie Savage} and
\href{https://www.nytimes.com/by/eric-schmitt}{Eric Schmitt}

\begin{itemize}
\item
  July 9, 2020
\item
  \begin{itemize}
  \item
  \item
  \item
  \item
  \item
  \end{itemize}
\end{itemize}

WASHINGTON --- The top American military officer delivered the
Pentagon's strongest public expression of concern to date on Thursday
about the C.I.A.'s assessment that
\href{https://www.nytimes.com/2020/07/29/us/politics/trump-putin-bounties.html}{Russia
offered bounties to Afghan militants} for killing American troops, even
as former top intelligence and military leaders questioned the Trump
administration's inaction.

The current and former officials testified at a pair of House hearings
as lawmakers grappled with the continuing fallout from the disclosure of
the intelligence assessment. Gen. Mark A. Milley, the chairman of the
Joint Chiefs of Staff, and Defense Secretary Mark T. Esper testified in
front of the Armed Services Committee, and former national security
officials before the Foreign Affairs Committee.

``If in fact there's bounties, I am an outraged general,'' said General
Milley, who served three tours in Afghanistan. ``If, in fact, there's
bounties directed by the government of Russia or any of their
institutions to kill American soldiers, that's a big deal. That's a real
big deal.''

He also said that while the government so far lacks proof that any
Russian bounties caused specific military casualties, ``we are still
looking.''

``We're not done,'' he continued. ``We're going to run this thing to
ground.''

General Milley's comments stood in contrast to efforts by the Trump
administration to dismiss the C.I.A.'s judgment and to justify the White
House's failure to authorize any response to Moscow by downplaying the
assessment as uncorroborated. After The New York Times first reported on
the suspected bounties on June 26,
\href{https://twitter.com/realDonaldTrump/status/1278284552679624705}{President
Trump deemed the matter a ``hoax.''}

``It is outrageous to me that we ask our servicemen and women to put
their lives in danger for our peace and security, and yet the
administration won't believe a credible piece of intelligence putting
bounties on their heads,'' Representative Gregory W. Meeks, Democrat of
New York, said at the foreign affairs hearing. ``How was Congress never
briefed until the claim was leaked to the press at great risk to
whistle-blowers?''

Based on intelligence that included accounts from interrogated detainees
and electronic intercepts of data showing payments from a bank account
linked to Russia's military intelligence agency, the G.R.U., to the
Taliban, the C.I.A. concluded that Russia had escalated its support to
the Taliban to include financial incentives for killing Americans and
other coalition troops.

The C.I.A. --- as well as analysts at the National Counterterrorism
Center --- expressed medium or moderate confidence in that conclusion.
The National Security Agency, which puts greater stock in surveillance
intercepts, was more skeptical, officials have said.

The White House initially denied that Mr. Trump had been briefed, but
did not deny a subsequent Times report that the intelligence was
included in his written daily briefing in late February. But Mr. Trump
often chooses not to read his written briefing, and White House aides
later stressed to lawmakers that a C.I.A. official who delivers his oral
briefings did not bring it up.

Michael J. Morell, a former acting C.I.A. director, portrayed that as
scapegoating in testimony to the Foreign Affairs Committee. He pointed
out that the briefer is the lowest-ranking person in the room during the
president's regular intelligence briefings, and said the national
security adviser, the C.I.A. director, the director of national
intelligence or the White House chief of staff could also have brought
the suspicions about Russian bounties to the president's attention. He
also noted that Mr. Esper receives a copy of the written President's
Daily Brief.

Mr. Morell also disputed the White House's suggestion that an
intelligence assessment had to be unanimously backed by intelligence
agencies to be taken seriously. In previous administrations, he said, if
the intelligence community assessed such information at any level of
confidence, officials would have told both the president and
congressional leaders immediately about that judgment and any dissent.
If the confidence level were low, he said, an administration would seek
more information before acting, while a medium- or high-level assessment
would most likely result in a response.

``You never have certainty in intelligence,'' Mr. Morell added.

John W. Nicholson Jr., a retired general who led coalition forces in
Afghanistan from 2016 to the middle of 2018, testified before the
Foreign Affairs Committee that Russia grew bolder over his tenure.
Afghan governors, he said, brought him weapons and other military
equipment and said Russians had provided them to the Taliban.

General Nicholson
\href{https://www.bbc.com/news/world-asia-43500299}{talked about Russian
support for the Taliban publicly} while still in that role, and he said
on Thursday that it was important to respond to such findings ---
including by going public with accusations to elicit a response from
Russia.

``It may just be denial, but you've got it on their radar screen,'' he
said. ``They know they're being watched. They know you're pushing back.
So these kinds of actions are extremely important. And of course, the
higher up you go, the more powerful the response is.''

Later, pressed by a Republican lawmaker, Representative Lee Zeldin of
New York, to comment on leaks of classified intelligence, General
Nicholson noted that they were unhelpful --- his comments were
\href{https://www.youtube.com/watch?v=9m_qt2lPmvU}{garbled in the live
video} of the hearing that was conducted remotely because of the
pandemic --- but then praised the hearing itself. He said that drawing
attention to the American suspicions of the bounty plot could cause the
Russians ``to dial down what they are doing.''

The ranking Republican on the Foreign Affairs Committee, Representative
Michael McCaul of Texas, said the government needed to take the
intelligence seriously given Russia's track record. He criticized Mr.
Trump's idea of inviting Russia to rejoin the economic alliance known as
the Group of 7, and noted that the administration already had legal
authority to impose new sanctions.

``While it's not news that Moscow has provided the Taliban with weapons
and other support,'' Mr. McCaul said, ``Russia paying bounties for the
murder of American service members would be an unacceptable escalation.
If true, the administration, in my judgment, must take swift and certain
action to hold the Putin regime accountable.''

The tone was different in the Armed Services Committee hearing, where
several Republicans sought to play down the significance of the
intelligence assessment, pointing to a lack of concrete evidence linking
Russian payments to specific American combat casualties.

Early in that hearing, a Republican asked Mr. Esper if he had ever
received a briefing that included the word ``bounties'' in connection
with killing American troops, and he said he had not. But later, under
skeptical questioning from Democrats, he acknowledged that he had been
briefed in February about intelligence that Russia was making
``payments'' to Taliban-linked militants for killing American troops.

Commanders in the region heard about the reports a month earlier, Mr.
Esper said, and had taken steps to safeguard American troops, who he
noted were already at the ``highest levels'' of force protection given
they were operating in a combat zone. Still, he stressed, the
intelligence was uncorroborated.

The discussion at both hearings repeatedly returned to Russia's broad
range of misbehavior, such as election interference and disinformation
campaigns as well as its suspected covert support for the Taliban that
the C.I.A. now thinks has escalated to offering bounties to incentivize
more frequent attacks amid peace talks.

The issue now, General Milley said, is what should or could be done to
push back --- through steps like diplomatic protests, the imposition of
new economic sanctions or other forms of pressure.

``Some of that is done,'' General Milley said. ``Are we doing as much as
we could or should? Perhaps not.''

Advertisement

\protect\hyperlink{after-bottom}{Continue reading the main story}

\hypertarget{site-index}{%
\subsection{Site Index}\label{site-index}}

\hypertarget{site-information-navigation}{%
\subsection{Site Information
Navigation}\label{site-information-navigation}}

\begin{itemize}
\tightlist
\item
  \href{https://help.nytimes.com/hc/en-us/articles/115014792127-Copyright-notice}{©~2020~The
  New York Times Company}
\end{itemize}

\begin{itemize}
\tightlist
\item
  \href{https://www.nytco.com/}{NYTCo}
\item
  \href{https://help.nytimes.com/hc/en-us/articles/115015385887-Contact-Us}{Contact
  Us}
\item
  \href{https://www.nytco.com/careers/}{Work with us}
\item
  \href{https://nytmediakit.com/}{Advertise}
\item
  \href{http://www.tbrandstudio.com/}{T Brand Studio}
\item
  \href{https://www.nytimes.com/privacy/cookie-policy\#how-do-i-manage-trackers}{Your
  Ad Choices}
\item
  \href{https://www.nytimes.com/privacy}{Privacy}
\item
  \href{https://help.nytimes.com/hc/en-us/articles/115014893428-Terms-of-service}{Terms
  of Service}
\item
  \href{https://help.nytimes.com/hc/en-us/articles/115014893968-Terms-of-sale}{Terms
  of Sale}
\item
  \href{https://spiderbites.nytimes.com}{Site Map}
\item
  \href{https://help.nytimes.com/hc/en-us}{Help}
\item
  \href{https://www.nytimes.com/subscription?campaignId=37WXW}{Subscriptions}
\end{itemize}
