\href{/section/us}{U.S.}\textbar{}John Lewis, Towering Figure of Civil
Rights Era, Dies at 80

\href{https://nyti.ms/32xSLXY}{https://nyti.ms/32xSLXY}

\begin{itemize}
\item
\item
\item
\item
\item
\item
\end{itemize}

\includegraphics{https://static01.nyt.com/images/2020/01/14/obituaries/14Lewis-John1-sub/merlin_166560153_c207f2db-7cc6-4567-94ab-58f4ce96db16-articleLarge.jpg?quality=75\&auto=webp\&disable=upscale}

Sections

\protect\hyperlink{site-content}{Skip to
content}\protect\hyperlink{site-index}{Skip to site index}

\hypertarget{john-lewis-towering-figure-of-civil-rights-era-dies-at-80}{%
\section{John Lewis, Towering Figure of Civil Rights Era, Dies at
80}\label{john-lewis-towering-figure-of-civil-rights-era-dies-at-80}}

Images of his beating at Selma shocked the nation and led to swift
passage of the 1965 Voting Rights Act. He was later called the
conscience of the Congress.

Credit...Andrea Mohin

Supported by

\protect\hyperlink{after-sponsor}{Continue reading the main story}

\href{https://www.nytimes.com/by/katharine-q-seelye}{\includegraphics{https://static01.nyt.com/images/2020/03/23/reader-center/author-katharine-q-seelye/author-katharine-q-seelye-thumbLarge.png}}

By \href{https://www.nytimes.com/by/katharine-q-seelye}{Katharine Q.
Seelye}

\begin{itemize}
\item
  Published July 17, 2020Updated Aug. 4, 2020
\item
  \begin{itemize}
  \item
  \item
  \item
  \item
  \item
  \item
  \end{itemize}
\end{itemize}

\href{https://www.nytimes.com/2020/07/30/us/john-lewis-live-funeral.html}{Representative
John Lewis}, a son of sharecroppers and an apostle of nonviolence who
was bloodied at Selma and across the Jim Crow South in the historic
struggle for racial equality, and who then carried a mantle of moral
authority into Congress, died on Friday. He was 80.

His death was confirmed
\href{https://www.speaker.gov/newsroom/71720-}{in a statement} by Nancy
Pelosi, the speaker of the House of Representatives.

\href{https://www.nytimes.com/2020/08/04/us/politics/trump-john-lewis-axios.html}{Mr.
Lewis}, a Georgia Democrat, announced on Dec. 29 that he had
\href{https://www.nytimes.com/2019/12/29/us/politics/rep-john-lewis-pancreatic-cancer.html}{Stage
4 pancreatic cancer} and vowed to fight it with the same passion with
which he had battled racial injustice. ``I have been in some kind of
fight --- for freedom, equality, basic human rights --- for nearly my
entire life,'' he said.

On the front lines of the bloody campaign to end Jim Crow laws, with
blows to his body and a fractured skull to prove it, Mr. Lewis was a
valiant stalwart of the civil rights movement and the last surviving
speaker from the March on Washington for Jobs and Freedom in 1963.

More than a half-century later, after
\href{https://www.nytimes.com/2020/07/08/us/george-floyd-body-camera-transcripts.html}{the
killing in May of George Floyd}, a Black man in police custody in
Minneapolis, Mr. Lewis welcomed the resulting global demonstrations
against police killings of Black people and, more broadly, against
systemic racism in many corners of society. He saw those protests as a
continuation of his life's work, though his illness had left him to
watch from the sidelines.

``It was very moving, very moving to see hundreds of thousands of people
from all over America and around the world take to the streets --- to
speak up, to speak out, to get into what I call `good trouble,''' Mr.
Lewis
\href{https://www.cbs.com/shows/cbs_this_morning/video/XpTqoXY0ErhQ0nqUvSJoL5n9sbgrvXLT/rep-john-lewis-message-to-protesters-fighting-for-racial-equality/}{told
``CBS This Morning'' in June.}

``This feels and looks so different,'' he said of the Black Lives Matter
movement, which drove the anti-racism demonstrations. ``It is so much
more massive and all inclusive.'' He added, ``There will be no turning
back.''

He died on the same day as another civil rights stalwart,
\href{https://www.nytimes.com/2020/07/17/us/ct-vivian-dead.html}{the
Rev. C.T. Vivian}, a close associate of the Rev. Dr. Martin Luther King
Jr.

Mr. Lewis's personal history paralleled that of the civil rights
movement. He was
\href{https://www.nytimes.com/2020/07/18/us/freedom-riders-john-lewis-work.html}{among
the original 13 Freedom Riders}, the Black and white activists who
challenged segregated interstate travel in the South in 1961. He was a
founder and early leader of the \href{https://snccdigital.org/}{Student
Nonviolent Coordinating Committee}, which coordinated lunch-counter
sit-ins. He helped organize the March on Washington, where Dr. King was
the main speaker, on the steps of the Lincoln Memorial.

Mr. Lewis led demonstrations against racially segregated restrooms,
hotels, restaurants, public parks and swimming pools, and he rose up
against other indignities of second-class citizenship. At nearly every
turn he was beaten, spat upon or burned with cigarettes. He was
tormented by white mobs and absorbed body blows from law enforcement.

On March 7, 1965, he led
\href{https://www.archives.gov/exhibits/eyewitness/html.php?section=2}{one
of the most famous marches in American history}. In the vanguard of 600
people demanding the voting rights they had been denied, Mr. Lewis
marched partway across the Edmund Pettus Bridge in Selma, Ala., into a
waiting phalanx of state troopers in riot gear.

Ordered to disperse, the protesters silently stood their ground. The
troopers responded with tear gas and bullwhips and rubber tubing wrapped
in barbed wire. In the melee, which came to be known as Bloody Sunday, a
trooper cracked Mr. Lewis's skull with a billy club, knocking him to the
ground, then hit him again when he tried to get up.

\includegraphics{https://static01.nyt.com/images/2020/07/19/obituaries/19Lewis-jp-print3/merlin_92926661_c0d916af-0237-4636-816a-a8a166e33399-articleLarge.jpg?quality=75\&auto=webp\&disable=upscale}

Televised images of the beatings of Mr. Lewis and scores of others
outraged the nation and galvanized support for the
\href{https://www.ourdocuments.gov/doc.php?flash=true\&doc=100}{Voting
Rights Act}, which President Lyndon B. Johnson presented to a joint
session of Congress eight days later and signed into law on Aug. 6. A
milestone in the struggle for civil rights, the law struck down the
literacy tests that Black people had been compelled to take before they
could register to vote and replaced segregationist voting registrars
with federal registrars to ensure that Black people were no longer
denied the ballot.

Once registered, millions of African-Americans began transforming
politics across the South. They gave Jimmy Carter, a son of Georgia, his
margin of victory in the 1976 presidential election. (A popular poster
proclaimed, ``Hands that once picked cotton now can pick a president.'')
And their voting power opened the door for Black people, including Mr.
Lewis, to run for public office. Elected in 1986, he became the second
African-American to be sent to Congress from Georgia since
Reconstruction, representing a district that encompassed much of
Atlanta.

\hypertarget{conscience-of-the-congress}{%
\subsection{`Conscience of the
Congress'}\label{conscience-of-the-congress}}

While Mr. Lewis represented Atlanta, his natural constituency was
disadvantaged people everywhere. Known less for sponsoring major
legislation than for his relentless pursuit of justice, he was called
``the conscience of the Congress'' by his colleagues.

When the House voted in December 2019 to impeach President Trump, Mr.
Lewis's words rose above the rest. ``When you see something that is not
right, not just, not fair, you have a moral obligation to say
something,'' he said on the House floor. ``To do something. Our children
and their children will ask us, `What did you do? What did you say?' For
some, this vote may be hard. But we have a mission and a mandate to be
on the right side of history.''

His words resonated as well after he saw the
\href{https://www.nytimes.com/2020/05/31/us/george-floyd-investigation.html}{video
of a Minneapolis police officer kneeling on Mr. Floyd's neck} for more
than eight minutes as Mr. Floyd gasped for air.

``It was so painful, it made me cry,'' Mr. Lewis told ``CBS This
Morning.'' ``People now understand what the struggle was all about,'' he
said. ``It's another step down a very, very long road toward freedom,
justice for all humankind.''

Image

Mr. Lewis, third from left, marching with the Rev. Dr. Martin Luther
King Jr., right, from Selma to Montgomery, Ala., on March 21,
1965.Credit...William Lovelace/Daily Express, via Getty Images

When he was younger, his words could be more militant. History remembers
the
\href{https://archive.nytimes.com/www.nytimes.com/interactive/2013/08/24/us/march-on-washington-original-coverage.html}{March
on Washington} for Dr. King's ``I Have a Dream'' speech, but Mr. Lewis
startled and energized the crowd with his own passion.

``By the force of our demands, our determination and our numbers,'' he
told the cheering throng that August day, ``we shall splinter the
segregated South into a thousand pieces and put them together in the
image of God and democracy. We must say: `Wake up, America. Wake up!'
For we cannot stop, and we will not and cannot be patient.''

\href{https://www.crmvet.org/info/mowjl.htm}{His original text} was more
blunt. ``We will march through the South, through the heart of Dixie,
the way Sherman did,'' he had written. President John F. Kennedy's civil
rights bill was ``too little, too late,'' he had written, demanding,
``Which side is the federal government on?''

But Dr. King and other elders --- Mr. Lewis was just 23 --- worried that
those first-draft passages would offend the Kennedy administration,
which they felt they could not alienate in their drive for federal
action on civil rights. They told him to tone down the speech.

Still, the crowd, estimated at more than 200,000, roared with approval
at his every utterance.

An earnest man who lacked the silver tongue of other civil rights
orators, Mr. Lewis could be pugnacious, tenacious and single-minded, and
he led with a force that commanded attention.

Image

Mr. Lewis and a fellow Freedom Rider, James Zwerg, after they were
attacked by segregationists in Montgomery, Ala., in May
1961.Credit...Bettmann/Corbis

He gained a reputation for having an almost mystical faith in his own
survivability. One civil rights activist who knew him well
\href{https://www.nytimes.com/1976/11/14/archives/john-lewis-is-a-personal-symbol-for-a-historic-period-black-passage.html}{told
The New York Times} in 1976: ``Some leaders, even the toughest, would
occasionally finesse a situation where they knew they were going to get
beaten or jailed. John never did that. He always went full force into
the fray.''

Mr. Lewis was arrested 40 times from 1960 to 1966. He was repeatedly
beaten senseless by Southern policemen and freelance hoodlums. During
the Freedom Rides in 1961, he was left unconscious in a pool of his own
blood outside the Greyhound Bus Terminal in Montgomery, Ala., after he
and others were attacked by hundreds of white people. He spent countless
days and nights in county jails and 31 days in Mississippi's notoriously
brutal Parchman Penitentiary.

Once he was in Congress, Mr. Lewis voted with the most liberal
Democrats, though he also showed an independent streak. In his quest to
build what Dr. King called ``the beloved community'' --- a world without
poverty, racism or war (Mr. Lewis adopted the phrase) --- he routinely
voted against military spending. He opposed the Persian Gulf war of 1991
and the North American Free Trade Agreement, which was signed in 1992.
He refused to take part in the
\href{https://www.britannica.com/event/Million-Man-March}{``Million Man
March''} in Washington in 1995, saying that statements made by the
organizer,
\href{https://www.nytimes.com/2018/03/09/us/louis-farrakhan-facts-history.html?smid=fb-nytimes\&smtyp=cur\&fbclid=IwAR0hOQH--3CNneVBqRv4lvO0uwg1rgo2r4OFACTuZQN4ZjROabr8hWpfhOs}{Louis
Farrakhan,} leader of the Nation of Islam, were ``divisive and
bigoted.''

In 2001, Mr. Lewis skipped the inauguration of George W. Bush, saying he
thought that Mr. Bush, who had become president after the Supreme Court
halted a vote recount in Florida, had not been truly elected.

In 2017 he boycotted Mr. Trump's inauguration, questioning the
legitimacy of his presidency because of evidence that Russia had meddled
in the 2016 election on Mr. Trump's behalf.

That earned him a derisive
\href{https://twitter.com/realDonaldTrump/status/820251730407473153}{Twitter
post} from
\href{https://twitter.com/realDonaldTrump/status/820255947956383744}{the
president}: ``Congressman John Lewis should spend more time on fixing
and helping his district, which is in horrible shape and falling apart
(not to mention crime infested) rather than falsely complaining about
the election results. All talk, talk, talk --- no action or results.
Sad!''

Mr. Trump's attack marked a sharp detour from the respect that had been
accorded Mr. Lewis by previous presidents, including, most recently,
Barack Obama. Mr. Obama awarded Mr. Lewis the Presidential Medal of
Freedom, the nation's highest civilian honor, in 2011.

Image

President Barack Obama was joined by Mr. Lewis in Selma, Ala., in 2015
to observe the 50th anniversary of the Voting Rights Act of
1965.Credit...Doug Mills/The New York Times

In bestowing the honor in a
\href{https://obamawhitehouse.archives.gov/the-press-office/2011/02/15/remarks-president-honoring-recipients-2010-medal-freedom}{White
House ceremony}, Mr. Obama said: ``Generations from now, when parents
teach their children what is meant by courage, the story of John Lewis
will come to mind --- an American who knew that change could not wait
for some other person or some other time; whose life is a lesson in the
fierce urgency of now.''

\hypertarget{to-his-family-preacher}{%
\subsection{To His Family, `Preacher'}\label{to-his-family-preacher}}

John Robert Lewis grew up with all the humiliations imposed by
segregated rural Alabama. He was born on Feb. 21, 1940, to Eddie and
Willie Mae (Carter) Lewis near the town of Troy on a sharecropping farm
owned by a white man. After his parents bought their own farm --- 110
acres for \$300 --- John, the third of 10 children, shared in the farm
work, leaving school at harvest time to pick cotton, peanuts and corn.
Their house had no plumbing or electricity. In the outhouse, they used
the pages of an old Sears catalog as toilet paper.

John was responsible for taking care of the chickens. He fed them and
read to them from the Bible. He baptized them when they were born and
staged elaborate funerals when they died.

``I was truly intent on saving the little birds' souls,'' he wrote in
his memoir, ``Walking With the Wind'' (1998). ``I could imagine that
they were my congregation. And me, I was a preacher.''

His family called him ``Preacher,'' and becoming one seemed to be his
destiny. He drew inspiration by listening to a young minister named
Martin Luther King on the radio and reading about the 1955-56
\href{https://www.history.com/topics/black-history/montgomery-bus-boycott}{Montgomery
bus boycott}. He finally wrote a letter to Dr. King, who sent him a
round-trip bus ticket to visit him in Montgomery, in 1958.

By then, Mr. Lewis had begun his studies at American Baptist Theological
Seminary (now American Baptist College) in Nashville, where he worked as
a dishwasher and janitor to pay for his education.

In Nashville, Mr. Lewis met many of the civil rights activists who would
stage the lunch counter sit-ins, Freedom Rides and voter registration
campaigns. They included the
\href{https://jameslawsoninstitute.org/history/}{Rev. James M. Lawson
Jr.,} who was one of the nation's most prominent scholars of civil
disobedience and who led workshops on Gandhi and nonviolence. He
mentored a generation of civil rights organizers, including Mr. Lewis.

Image

Mr. Lewis, right, and a fellow student demonstrator, James Bevel, stood
inside the door of a Nashville restaurant in 1960 during a sit-in to
protest the establishment's refusal to serve Black people.Credit...Jack
Corn/The Tennessean, via USA Today Network

Mr. Lewis's first arrest came in February 1960, when he and other
students demanded service at whites-only lunch counters in Nashville. It
was the first prolonged battle of the movement that evolved into the
Student Nonviolent Coordinating Committee.

David Halberstam, then a reporter for The Nashville Tennessean, later
\href{https://www.nytimes.com/1995/05/01/opinion/nashville-revisited-lunchcounter-days.html}{described
the scene}: ``The protests had been conducted with exceptional dignity,
and gradually one image had come to prevail --- that of elegant,
courteous young Black people, holding to their Gandhian principles,
seeking the most elemental of rights, while being assaulted by young
white hoodlums who beat them up and on occasion extinguished cigarettes
on their bodies.''

In three months,
\href{https://www.blackpast.org/african-american-history/nashville-sit-ins-1960/}{after
repeated well-publicized sit-ins}, the city's political and business
communities gave in to the pressure, and Nashville became the first
major Southern city to begin desegregating public facilities.

But Mr. Lewis lost his family's good will. When his parents learned that
he had been arrested in Nashville, he wrote, they were ashamed. They had
taught him as a child to accept the world as he found it. When he asked
them about signs saying ``Colored Only,'' they told him, ``That's the
way it is, don't get in trouble.''

But as an adult, he said, after he met Dr. King and
\href{https://www.nytimes.com/2005/10/25/us/rosa-parks-92-founding-symbol-of-civil-rights-movement-dies.html}{Rosa
Parks}, whose refusal to give up her bus seat to a white man was a flash
point for the civil rights movement, he was inspired to ``get into
trouble, good trouble, necessary trouble.''

Getting into ``good trouble'' became his motto for life. A
\href{https://variety.com/2019/film/news/john-lewis-documentary-cnn-films-1203434157/}{documentary
film,} ``John Lewis: Good Trouble,'' was released this month.

Despite the disgrace he had brought on his family, he felt that he had
been ``involved in a holy crusade'' and that getting arrested had been
``a badge of honor,'' he said in
\href{http://repository.wustl.edu/concern/videos/vt150m28q}{a 1979 oral
history interview} housed at Washington University in St. Louis.

In 1961, when he graduated from the seminary, he joined a Freedom Ride
organized by the Congress of Racial Equality, known as CORE. He and
others were beaten bloody when they tried to enter a whites-only waiting
room at the bus station in Rock Hill, S.C. Later, he was jailed in
Birmingham, Ala., and beaten again in Montgomery, where several others
were badly injured and one was paralyzed for life.

``If there was anything I learned on that long, bloody bus trip of
1961,'' he wrote in his memoir, ``it was this --- that we were in for a
long, bloody fight here in the American South. And I intended to stay in
the middle of it.''

At the same time, a schism in the movement was opening between those who
wanted to express their rage and fight back and those who believed in
pressing on with nonviolence. Mr. Lewis chose nonviolence.

Image

Mr. Lewis in June 1967. He had been ``involved in a holy crusade,'' he
later said, and getting arrested had been ``a badge of
honor.''Credit...Sam Falk/The New York Times

\hypertarget{overridden-by-black-power}{%
\subsection{Overridden by `Black
Power'}\label{overridden-by-black-power}}

But by the time of the urban race riots of the 1960s, particularly in
the
\href{https://learning.blogs.nytimes.com/2011/08/11/aug-11-1965-riots-in-the-watts-section-of-los-angeles/}{Watts}
section of Los Angeles in 1965, many Black people had rejected
nonviolence in favor of direct confrontation. Mr. Lewis was ousted as
chairman of the Student Nonviolent Coordinating Committee in 1966 and
replaced by the fiery
\href{https://www.nytimes.com/1998/11/16/us/stokely-carmichael-rights-leader-who-coined-black-power-dies-at-57.html}{Stokely
Carmichael}, who popularized the phrase ``Black power.''

Mr. Lewis spent a few years out of the limelight. He headed the
\href{https://www.georgiaencyclopedia.org/articles/history-archaeology/voter-education-project}{Voter
Education Project,} registering voters, and finished his bachelor's
degree in religion and philosophy at Fisk University in Nashville in
1967.

During this period he met Lillian Miles, a librarian, teacher and former
Peace Corps volunteer. She was outgoing and political and could quote
Dr. King's speeches verbatim. They were married in 1968, and she became
one of his closest political advisers.

\href{https://patch.com/georgia/decatur/congressman-john-lewis-wife-passed-away-monday-morning-1231}{She
died in 2012.} Mr. Lewis's survivors include several siblings and his
son, John-Miles Lewis.

Mr. Lewis made his first attempt at running for office in 1977, an
unsuccessful bid for Congress. He won a seat on the Atlanta City Council
in 1981, and in 1986 he ran again for the House. It was a bitter race
that pitted against each other two civil rights figures, Mr. Lewis and
\href{https://www.nytimes.com/2015/08/17/us/julian-bond-former-naacp-chairman-and-civil-rights-leader-dies-at-75.html}{Julian
Bond}, a friend and former close associate of his in the movement. The
charismatic Mr. Bond, more articulate and polished than Mr. Lewis, was
the perceived favorite.

``I want you to think about sending a workhorse to Washington, and not a
show horse,'' Mr. Lewis said
\href{https://www.nytimes.com/1986/08/09/us/campaign-in-georgia-strains-black-political-ties.html}{during
a debate}. ``I want you to think about sending a tugboat and not a
showboat.''

Mr. Lewis won in an upset, with 52 percent of the vote. His support came
from Atlanta's white precincts and from working-class and poor Black
voters who felt more comfortable with him than with Mr. Bond, though Mr.
Bond won the majority of Black voters.

Not surprisingly, Mr. Lewis's long congressional career was marked by
protests. He was arrested in Washington several times, including outside
the South African Embassy for demonstrating against apartheid and at
Sudan's Embassy while protesting genocide in Darfur.

In 2010 he supported Mr. Obama's health care bill, a divisive measure
that drew angry protesters, including many from the right-wing Tea
Party, to the Capitol. Some demonstrators
\href{https://www.washingtonpost.com/wp-dyn/content/article/2010/03/20/AR2010032002556.html}{shouted
obscenities and racial slurs} at Mr. Lewis and other members of the
Congressional Black Caucus.

``They were shouting, sort of harassing,'' Mr. Lewis told reporters at
the time. ``But it's OK. I've faced this before.''

Image

Mr. Lewis with other members of Congress staging a sit-in on the floor
of the House of Representatives in June 2016, demanding that the
Republican-led body vote on gun control legislation after the Orlando
nightclub massacre.~Credit...Office of Representative Elizabeth Esty,
via Agence France-Presse --- Getty Images

In 2016, after a
\href{https://www.nytimes.com/news-event/2016-orlando-shooting}{massacre
at an Orlando, Fla., nightclub} left 49 people dead, he
\href{https://www.cnn.com/2016/06/22/politics/john-lewis-sit-in-gun-violence/index.html}{led
a sit-in on the House floor} to protest federal inaction on gun control.
The demonstration drew the support of 170 lawmakers, but Republicans
dismissed it as a publicity stunt and squelched any legislative action.

Through it all, the events of Bloody Sunday were never far from his
mind, and every year Mr. Lewis traveled to Selma to commemorate its
anniversary. Over time, he watched attitudes change. At the ceremony in
1998,
\href{https://www.nytimes.com/2005/09/13/us/joseph-smitherman-mayor-in-selma-strife-dies-at-75.html}{Joseph
T. Smitherman,} who had been Selma's segregationist mayor in 1965 and
was still mayor --- though a repentant one --- gave Mr. Lewis a key to
the city.

``Back then, I called him an outside rabble-rouser,'' Mr. Smitherman
said of Mr. Lewis. ``Today, I call him one of the most courageous people
I ever met.''

Mr. Lewis was a popular speaker at college commencements and always
offered the same advice --- that the graduates get into ``good
trouble,'' as he had done against his parents' wishes.

Image

Mr. Lewis in 2017. ``Our struggle is not the struggle of a day, a week,
a month, or a year,'' he said, ``it is the struggle of a
lifetime.''Credit...Al Drago/The New York Times

He put it this way
\href{https://twitter.com/repjohnlewis/status/1011991303599607808?lang=en}{on
Twitter in 2018}:

``Do not get lost in a sea of despair. Be hopeful, be optimistic. Our
struggle is not the struggle of a day, a week, a month, or a year, it is
the struggle of a lifetime. Never, ever be afraid to make some noise and
get in good trouble, necessary trouble.''

Roy Reed, who covered the civil rights movement for The New York Times
and who died in 2017, contributed reporting from an earlier version of
this obituary. Sheryl Gay Stolberg also contributed reporting.

Advertisement

\protect\hyperlink{after-bottom}{Continue reading the main story}

\hypertarget{site-index}{%
\subsection{Site Index}\label{site-index}}

\hypertarget{site-information-navigation}{%
\subsection{Site Information
Navigation}\label{site-information-navigation}}

\begin{itemize}
\tightlist
\item
  \href{https://help.nytimes.com/hc/en-us/articles/115014792127-Copyright-notice}{©~2020~The
  New York Times Company}
\end{itemize}

\begin{itemize}
\tightlist
\item
  \href{https://www.nytco.com/}{NYTCo}
\item
  \href{https://help.nytimes.com/hc/en-us/articles/115015385887-Contact-Us}{Contact
  Us}
\item
  \href{https://www.nytco.com/careers/}{Work with us}
\item
  \href{https://nytmediakit.com/}{Advertise}
\item
  \href{http://www.tbrandstudio.com/}{T Brand Studio}
\item
  \href{https://www.nytimes.com/privacy/cookie-policy\#how-do-i-manage-trackers}{Your
  Ad Choices}
\item
  \href{https://www.nytimes.com/privacy}{Privacy}
\item
  \href{https://help.nytimes.com/hc/en-us/articles/115014893428-Terms-of-service}{Terms
  of Service}
\item
  \href{https://help.nytimes.com/hc/en-us/articles/115014893968-Terms-of-sale}{Terms
  of Sale}
\item
  \href{https://spiderbites.nytimes.com}{Site Map}
\item
  \href{https://help.nytimes.com/hc/en-us}{Help}
\item
  \href{https://www.nytimes.com/subscription?campaignId=37WXW}{Subscriptions}
\end{itemize}
