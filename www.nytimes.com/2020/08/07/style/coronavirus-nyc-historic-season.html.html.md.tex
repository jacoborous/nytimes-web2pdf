Sections

SEARCH

\protect\hyperlink{site-content}{Skip to
content}\protect\hyperlink{site-index}{Skip to site index}

\href{/section/style}{Style}\textbar{}A Season of Grief and Release: 5
Months of the Virus in New York City

\href{https://nyti.ms/2PuJ8l1}{https://nyti.ms/2PuJ8l1}

\begin{itemize}
\item
\item
\item
\item
\item
\item
\end{itemize}

\hypertarget{spring-to-summer-2020}{%
\subsection{Spring to Summer 2020}\label{spring-to-summer-2020}}

March

\includegraphics{https://static01.nyt.com/packages/flash/multimedia/ICONS/transparent.png}

\includegraphics{https://static01.nyt.com/newsgraphics/2020/08/03/nyc-spring-topper/assets/images/march-2000.jpg}

March

A Season of Grief and Release:

5 Months of the Virus

\includegraphics{https://static01.nyt.com/packages/flash/multimedia/ICONS/transparent.png}

\includegraphics{https://static01.nyt.com/newsgraphics/2020/08/03/nyc-spring-topper/assets/images/april-2000.jpg}

April

\includegraphics{https://static01.nyt.com/packages/flash/multimedia/ICONS/transparent.png}

\includegraphics{https://static01.nyt.com/newsgraphics/2020/08/03/nyc-spring-topper/assets/images/may-alt-2000.jpg}

May

\includegraphics{https://static01.nyt.com/packages/flash/multimedia/ICONS/transparent.png}

\includegraphics{https://static01.nyt.com/newsgraphics/2020/08/03/nyc-spring-topper/assets/images/june-2000.jpg}

June

in New York City

A photographic timeline\\
of a historic half-year.

\includegraphics{https://static01.nyt.com/packages/flash/multimedia/ICONS/transparent.png}

\includegraphics{https://static01.nyt.com/newsgraphics/2020/08/03/nyc-spring-topper/assets/images/july-2000.jpg}

July

\hypertarget{a-season-of-grief-and-release-5-months-of-the-virus-in-new-york-city}{%
\section{A Season of Grief and Release: 5 Months of the Virus in New
York
City}\label{a-season-of-grief-and-release-5-months-of-the-virus-in-new-york-city}}

Photographs by Daniel Arnold

Text by Dodai Stewart

Aug. 7, 2020

The restaurant was empty, save for two other tables. Spoons slid through
our dessert, split three ways. We were anxious but determined to enjoy
it, somehow. It was March 10, and the coronavirus had just hit New York
City. We laughed, but with worry in our brows and the preamble of panic
in our eyes.

The next day, I packed up my desk at work: the computer, the mouse, the
keyboard. I left the giant monitor and two boxes of Girl Scout cookies
behind.

March

\includegraphics{https://static01.nyt.com/images/2020/08/09/fashion/NYC-HISTORIC-MARCH1/NYC-HISTORIC-SPRING1-articleLarge.jpg?quality=75\&auto=webp\&disable=upscale}

Image

Image

Image

\includegraphics{https://static01.nyt.com/images/2020/07/23/fashion/NYC-HISTORIC-MARCH-07/NYC-HISTORIC-MARCH-07-mobileMasterAt3x.jpg}\includegraphics{https://static01.nyt.com/images/2020/07/23/fashion/NYC-HISTORIC-MARCH-10/NYC-HISTORIC-MARCH-10-mobileMasterAt3x.jpg}\includegraphics{https://static01.nyt.com/images/2020/07/23/fashion/NYC-HISTORIC-MARCH-03/NYC-HISTORIC-MARCH-03-mobileMasterAt3x.jpg}\includegraphics{https://static01.nyt.com/images/2020/08/09/fashion/NYC-HISTORIC-MARCH-09/NYC-HISTORIC-MARCH-09-mobileMasterAt3x.jpg}

Daniel Arnold

By the end of the week, schools, restaurants, bars, courthouses, night
clubs, Broadway theaters, nail salons --- almost everything was closed.

Toilet paper was nowhere to be found.

New York State had 950 confirmed coronavirus cases, officials said.

A rush for jobless benefits by those who became immediately unemployed
crashed the New York State website.

Image

Image

My brother's birthday arrived, and instead of the usual dinner and
drinks, we saw each other on a FaceTime call. He was downtown and I was
uptown. It felt wrong for his birthday to pass with such little fanfare,
so I sent him two pints of ice cream through Postmates, as a surprise.
He called to report that when the bell rang, what he saw was jarring: a
man wearing a mask, gloves, goggles and white hooded disposable
coveralls, handing over a brown paper bag.

In bed at night, in my Manhattan apartment where I live alone, I stared
at the ceiling, unable to sleep. The silence was unnerving. Many of my
neighbors, in my building and on my block, had vanished. I grew up in
New York and have lived nowhere else since I was 7. Not once in that
time --- not during the snowstorm of 1983 nor the snowstorm of 1996, not
after Sept. 11 or during the blackout of 2003, not after Hurricane Sandy
in 2012 --- had the streets ever been so quiet. The moan of a siren
would speed by intermittently. Within days, they became more frequent,
the wailing echoing off near-empty buildings.

April

\includegraphics{https://static01.nyt.com/images/2020/07/23/fashion/NYC-HISTORIC-APRIL/NYC-HISTORIC-APRIL-mobileMasterAt3x.jpg}\includegraphics{https://static01.nyt.com/images/2020/08/09/fashion/NYC-HISTORIC-APRIL-02/NYC-HISTORIC-APRIL-02-mobileMasterAt3x-v2.jpg}\includegraphics{https://static01.nyt.com/images/2020/08/09/fashion/NYC-HISTORIC-APRIL-05/NYC-HISTORIC-APRIL-05-mobileMasterAt3x.jpg}

Daniel Arnold

New York City had become the epicenter of the coronavirus outbreak in
the United States.

The governor was counting, counting, counting, as the numbers of
infected people went up, up, up.

My friends and co-workers and their loved ones were sick. TV screens
showed ambulances, doctors, tears.

Image

Image

Image

Image

Image

Image

By the third week in April, there were
\href{https://www1.nyc.gov/assets/doh/downloads/pdf/imm/covid-19-daily-data-summary-04142020-1.pdf}{more
than 100,000} cases in New York. More than 6,000 people had died. I dug
my thermometer out of the toiletries drawer and kept it on my desk as I
worked, checking my temperature repeatedly, afraid the virus was somehow
sneaking up on me.

Delicate pink cherry blossoms unfurled in Central Park. I walked among
them, bandanna over my nose and mouth, taking pictures, hypervigilant of
staying at least 12 feet away from the other folks strolling.

Attending my first Zoom birthday party involved arranging comfortable
seating and flattering lighting, and smiling at the one-square-inch
rectangle of laptop screen that contained a pixelated version of the
birthday girl, even as the audio glitched.

May

\includegraphics{https://static01.nyt.com/images/2020/08/09/fashion/00NYC-HISTORIC-MAY-20/00NYC-HISTORIC-MAY-20-mobileMasterAt3x-v2.jpg}\includegraphics{https://static01.nyt.com/images/2020/07/24/fashion/00NYC-HISTORIC-MAY-02/00NYC-HISTORIC-MAY-02-mobileMasterAt3x.jpg}\includegraphics{https://static01.nyt.com/images/2020/07/24/fashion/00NYC-HISTORIC-MAY-13/00NYC-HISTORIC-MAY-13-mobileMasterAt3x.jpg}\includegraphics{https://static01.nyt.com/images/2020/07/24/fashion/00NYC-HISTORIC-MAY-24/00NYC-HISTORIC-MAY-24-mobileMasterAt3x-v2.jpg}

Daniel Arnold

May brought warmer weather, and the cases and deaths started dropping.

But the virus was killing Black and Latino people at double the rate.

There were equipment shortages at the hospitals, infections spreading in
prisons and jails.

In a livestream, the governor's face appeared next to a slide with the
words ``When is it over?''

Image

Image

Image

Birds chirped merrily as I edited articles about the virus's impact in
New York, facing a ``Rear Window''-style view where I saw no signs of
life except for the 7 p.m. clap. That's when my friend who lives one
flight up would lean out of the window as far as she could around the
child safety guard, and I would do the same, until we could both see a
sliver of each other's faces. We would wave, and shout: ``Hiiii!''
Sometimes she looked a bit haunted --- her position as a physical
therapist in a hospital put her dangerously close to the I.C.U., and she
would remove her clothes in the hallway before going inside her
apartment to join her son and her mother. Three generations under one
roof in the ghostly city.

There were
\href{https://www.nytimes.com/2020/04/30/nyregion/coronavirus-nyc-funeral-home-morgue-bodies.html}{bodies
stacked in refrigerated trucks} in Queens and
\href{https://www.nytimes.com/2020/04/30/nyregion/coronavirus-nj-hunger.html}{food
lines a mile long} in New Jersey. A slew of small businesses shut down,
including personal favorites: Lucky Strike, Gem Spa,
\href{https://untappedcities.com/2020/06/10/closed-record-mart-manhattans-oldest-record-store-located-in-times-square-subway-station/}{Record
Mart}. Rent was due and jobless claims surged.

How does one mourn in isolation? How does one process grief for an
entire city?

\includegraphics{https://static01.nyt.com/images/2020/08/09/fashion/00NYC-HISTORIC-MAY-04/00NYC-HISTORIC-MAY-04-mobileMasterAt3x.jpg}\includegraphics{https://static01.nyt.com/images/2020/07/24/fashion/00NYC-HISTORIC-MAY-08/00NYC-HISTORIC-MAY-08-mobileMasterAt3x-v2.jpg}\includegraphics{https://static01.nyt.com/images/2020/07/24/fashion/00NYC-HISTORIC-MAY-03/00NYC-HISTORIC-MAY-03-mobileMasterAt3x-v2.jpg}\includegraphics{https://static01.nyt.com/images/2020/07/24/fashion/00NYC-HISTORIC-MAY-06/00NYC-HISTORIC-MAY-06-mobileMasterAt3x.jpg}

Daniel Arnold

Graduations, proms, weddings canceled. Basketball hoops removed from
public parks.

The days became blurred by uniformity.

Wake up, sit at the computer, watch news conferences, eat dinner in
front of the evening news.

Weekends were for donning disposable gloves and a disposable mask (not
the nice mask used for dog walking) and venturing into the perilous
germ-minefields of the grocery store and the pharmacy.

Image

Image

Image

Image

Credit...Daniel Arnold

I had spent decades as a die-hard New Yorker, defending my hometown
against the smack-talking of malcontents and come-latelys, and my heart
ached for my fellow citizens, for the transit workers and the funeral
directors, the unsheltered and the first responders.

Privately, I railed against my own internal upheaval, mad at how selfish
it sounded in my head: Did I still love the city if
\href{https://www.nytimes.com/2020/05/21/nyregion/nyc-bars-coronavirus.html}{bars}
and museums were closed? If there were no plays, musicals or movies? No
Queens Night Market, no cocktails on the roof of the Met, no tennis
lessons in Central Park, no burlesque shows in Bushwick, no dim sum in
Chinatown, no Donna Summer dance parties in Bed-Stuy? If the spontaneity
of running into an old friend or making a brand-new one had evaporated?

In mid-May, there was talk of reopening. I hadn't been on the subway in
over 75 days, and I couldn't remember the last time I'd worn closed-toe
shoes or pants. (Caftans --- day in, day out.) Seldom-seen gray hairs,
usually disguised expertly in a salon, met me in the mirror each
morning. Both my toenails and my dog's were in desperate need of
professional help. I wondered if for my own birthday, in June, it might
be possible to do something sort of social.

Then, on Memorial Day, George Floyd died after being pinned under the
knee of a Minneapolis police officer.

Image

June

\includegraphics{https://static01.nyt.com/images/2020/08/09/fashion/00NYC-HISTORIC-JUNE-16/00NYC-HISTORIC-JUNE-16-mobileMasterAt3x.jpg}\includegraphics{https://static01.nyt.com/images/2020/07/24/fashion/00NYC-HISTORIC-JUNE-02/00NYC-HISTORIC-JUNE-02-mobileMasterAt3x.jpg}\includegraphics{https://static01.nyt.com/images/2020/08/09/fashion/00NYC-HISTORIC-JUNE-13/00NYC-HISTORIC-JUNE-13-mobileMasterAt3x.jpg}\includegraphics{https://static01.nyt.com/images/2020/07/24/fashion/00NYC-HISTORIC-JUNE/00NYC-HISTORIC-JUNE-mobileMasterAt3x.jpg}

Daniel Arnold

Demonstrations exploded around the country and in New York. It felt like
something had been unleashed.

At first it seemed ominous.

I watched on the Citizen app as a succession of reports came in: A fire
was set, then a store was broken into, one after another, in a line
across NoHo and Soho.

But as the week went on, it was clear that this was an anomaly. The
protests were largely peaceful, though passionate.

Image

Image

Image

Image

\includegraphics{https://static01.nyt.com/images/2020/07/24/fashion/00NYC-HISTORIC-JUNE-06/00NYC-HISTORIC-JUNE-06-mobileMasterAt3x.jpg}\includegraphics{https://static01.nyt.com/images/2020/07/24/fashion/00NYC-HISTORIC-JUNE-05/00NYC-HISTORIC-JUNE-05-mobileMasterAt3x.jpg}\includegraphics{https://static01.nyt.com/images/2020/07/24/fashion/00NYC-HISTORIC-JUNE-07/00NYC-HISTORIC-JUNE-07-mobileMasterAt3x.jpg}\includegraphics{https://static01.nyt.com/images/2020/07/24/fashion/00NYC-HISTORIC-JUNE-04/00NYC-HISTORIC-JUNE-04-mobileMasterAt3x-v2.jpg}\includegraphics{https://static01.nyt.com/images/2020/07/24/fashion/00NYC-HISTORIC-JUNE-11/00NYC-HISTORIC-JUNE-11-mobileMasterAt3x-v2.jpg}

Daniel Arnold

On my birthday, I stood in front of Gracie Mansion and filmed hundreds
of people sitting, silently, their fists in the air, for more than half
an hour.

There were demonstrations by day and fireworks all night.

In a way, it felt like proof: New York had not been conquered.

Twenty-three thousand had died of coronavirus, more than the seating
capacity of Madison Square Garden.

Still, thousands filled the streets to demand an end to police violence
against Black people.

Image

Image

****

Image

Image

Following the demonstrations and late-night destruction --- shattered
windows and stolen merchandise --- the mayor instituted a curfew.

It was New York City's first curfew since World War II.

The protests continued, day after day. Doctors and nurses knelt in Times
Square with raised fists; thousands of cyclists pedaled while shouting
``no justice, no peace.''

Encounters with the authorities were captured by protesters and shared
on social media. A police officer was suspended after video emerged of
him pushing a woman to the ground in Brooklyn. Another officer was
suspended for pulling down a man's face mask and spraying pepper spray
directly in his face.

The cases of the virus in the city kept dropping.

Image

July

\includegraphics{https://static01.nyt.com/images/2020/08/09/fashion/00NYC-HISTORIC-JULY-02/00NYC-HISTORIC-JULY-02-mobileMasterAt3x.jpg}\includegraphics{https://static01.nyt.com/images/2020/07/27/fashion/00NYC-HISTORIC-JULY-09/00NYC-HISTORIC-JULY-09-mobileMasterAt3x.jpg}\includegraphics{https://static01.nyt.com/images/2020/07/27/fashion/00NYC-HISTORIC-JULY/00NYC-HISTORIC-JULY-mobileMasterAt3x.jpg}\includegraphics{https://static01.nyt.com/images/2020/07/27/fashion/00NYC-HISTORIC-JULY-06/00NYC-HISTORIC-JULY-06-mobileMasterAt3x.jpg}\includegraphics{https://static01.nyt.com/images/2020/07/27/fashion/00NYC-HISTORIC-JULY-13/00NYC-HISTORIC-JULY-13-mobileMasterAt3x-v2.jpg}

Daniel Arnold

At some point it dawned on me that I was working as an editor at the New
York Times, on the New York desk, during a gigantic moment in New York's
history.

We'd started a daily live briefing in March.

And we had continuously updated it throughout each day, keeping it going
for nearly three months, seven days a week.

``We did it,'' the governor said in the middle of June.

``This is one of the best days for New York since we started.''

Image

Image

Image

Image

Image

In July,
\href{https://www.nytimes.com/2020/07/06/nyregion/nyc-phase-3-reopening-coronavirus.html}{we
hit Phase 3}: Restaurant dining
(\href{https://www.nytimes.com/2020/06/23/dining/outdoor-restaurants-nyc-coronavirus.html}{outdoors})
returned. Pedicures were back. Beaches
\href{https://www.nytimes.com/interactive/2020/07/07/nyregion/nyc-beaches-open.html?smid=nytcore-ios-share}{opened},
then
\href{https://www.nytimes.com/2020/07/24/nyregion/nyc-pools-open.html}{pools}.
There were kids eating Italian ices in the street and
\href{https://www.tiktok.com/@colita03/video/6854966008730389766}{men
opening up fire hydrants}. It felt like the Old New York.

It's not, of course.

I don't know when I'll ever share dessert in a restaurant again.

The city has changed, the people have changed. The country has changed.
The world has changed. We walk outside masked, coated in a residue of
terror and grief.

Image

Image

Image

But on July 22, I saw a metaphor in
\href{https://twitter.com/_Mikey_Cee/status/1286074510488350720}{a video
on Twitter}, and it's cheesy, but I took it as a sign.

During a thunderstorm, lightning struck the Statue of Liberty, the bolt
slicing through an immense and menacing cloud. The statue stood
steadfast and unmovable. She didn't budge an inch.

Image

\begin{center}\rule{0.5\linewidth}{\linethickness}\end{center}

Produced by Eve Lyons, Natalie Shutler and Tracy Ma.

Daniel Arnold is a photographer in New York. Dodai Stewart is a deputy
editor on the Metro desk of The Times.

Advertisement

\protect\hyperlink{after-bottom}{Continue reading the main story}

\hypertarget{site-index}{%
\subsection{Site Index}\label{site-index}}

\hypertarget{site-information-navigation}{%
\subsection{Site Information
Navigation}\label{site-information-navigation}}

\begin{itemize}
\tightlist
\item
  \href{https://help.nytimes.com/hc/en-us/articles/115014792127-Copyright-notice}{©~2020~The
  New York Times Company}
\end{itemize}

\begin{itemize}
\tightlist
\item
  \href{https://www.nytco.com/}{NYTCo}
\item
  \href{https://help.nytimes.com/hc/en-us/articles/115015385887-Contact-Us}{Contact
  Us}
\item
  \href{https://www.nytco.com/careers/}{Work with us}
\item
  \href{https://nytmediakit.com/}{Advertise}
\item
  \href{http://www.tbrandstudio.com/}{T Brand Studio}
\item
  \href{https://www.nytimes.com/privacy/cookie-policy\#how-do-i-manage-trackers}{Your
  Ad Choices}
\item
  \href{https://www.nytimes.com/privacy}{Privacy}
\item
  \href{https://help.nytimes.com/hc/en-us/articles/115014893428-Terms-of-service}{Terms
  of Service}
\item
  \href{https://help.nytimes.com/hc/en-us/articles/115014893968-Terms-of-sale}{Terms
  of Sale}
\item
  \href{https://spiderbites.nytimes.com}{Site Map}
\item
  \href{https://help.nytimes.com/hc/en-us}{Help}
\item
  \href{https://www.nytimes.com/subscription?campaignId=37WXW}{Subscriptions}
\end{itemize}
