Sections

SEARCH

\protect\hyperlink{site-content}{Skip to
content}\protect\hyperlink{site-index}{Skip to site index}

\href{https://www.nytimes.com/section/sports/baseball}{Baseball}

\href{https://myaccount.nytimes.com/auth/login?response_type=cookie\&client_id=vi}{}

\href{https://www.nytimes.com/section/todayspaper}{Today's Paper}

\href{/section/sports/baseball}{Baseball}\textbar{}Horace Clarke,
Standout in a Dismal Yankee Era, Dies at 82

\href{https://nyti.ms/31xI1Hb}{https://nyti.ms/31xI1Hb}

\begin{itemize}
\item
\item
\item
\item
\item
\end{itemize}

Advertisement

\protect\hyperlink{after-top}{Continue reading the main story}

Supported by

\protect\hyperlink{after-sponsor}{Continue reading the main story}

\hypertarget{horace-clarke-standout-in-a-dismal-yankee-era-dies-at-82}{%
\section{Horace Clarke, Standout in a Dismal Yankee Era, Dies at
82}\label{horace-clarke-standout-in-a-dismal-yankee-era-dies-at-82}}

He was a solid, dependable player, but he had the misfortune of joining
the Yankees just as they tumbled from greatness.

\includegraphics{https://static01.nyt.com/images/2020/08/08/obituaries/00ClarkeH4/00ClarkeH4-articleLarge.jpg?quality=75\&auto=webp\&disable=upscale}

By Mathew Brownstein

\begin{itemize}
\item
  Published Aug. 7, 2020Updated Aug. 8, 2020, 1:49 a.m. ET
\item
  \begin{itemize}
  \item
  \item
  \item
  \item
  \item
  \end{itemize}
\end{itemize}

Horace Clarke, a dependable though light-hitting second baseman for the
Yankees who became indelibly and ingloriously associated with the team's
lean years in the 1960s and '70s --- what some sardonically labeled
``the Horace Clarke era'' --- died on Wednesday at his home in Laurel,
Md. He was 82.

His death was confirmed by the office of his cousin, Stacey E. Plaskett,
the Democratic delegate who represents the Virgin Islands in Congress.
His son Jeffrey said the cause was complications of Alzheimer's disease.

At the time of his debut, in 1965, Clarke, an undersized middle
infielder, was one of just five players born in the U.S. Virgin Islands
to make it to the major leagues. He played 10 seasons in the majors, all
but part of the last season for the Yankees.

What he lacked in power as a hitter --- he had only 27 career home runs
--- he made up for with a sure-handed glove and excellent speed. His
stolen-base totals were in double digits in seven seasons, and he was
among the American League's top 10 base stealers four times.

But he had the misfortune of joining the Yankees just as the team was
about to tumble from the heights of greatness. Preceding his rookie
season of 1965, the Yankees, led by the likes of
\href{https://www.nytimes.com/1995/08/14/obituaries/mickey-mantle-great-yankee-slugger-dies-at-63.html\#:~:text=Mickey\%20Mantle\%2C\%20the\%20most\%20powerful,died\%20at\%202\%3A10\%20A.M.\&text=9\%2C\%20the\%20hospital\%20said\%20the\%20cancer\%20had\%20spread\%20to\%20his\%20abdomen.}{Mickey
Mantle} and Whitey Ford, had won the American League pennant five
straight seasons.

During Clarke's 10-year tenure, however, New York failed to make the
postseason once. The team wouldn't get there again until 1976, two years
after Clarke retired. In between came that so-called Horace Clarke era.

Speaking to a reporter for The Daily News in 2010, Clarke admitted that
it was frustrating to be labeled a scapegoat for those underachieving
Yankee teams. But he added: ``I know --- New York is New York. You don't
win, you're going to hear about it. I was in the middle.''

Horace Meredith Clarke was born on June 2, 1938, in Frederiksted, St.
Croix, to Dennis and Vivian (Woods) Clarke. He was the youngest of six
children.

He attended Christiansted High School and went to a baseball tryout camp
in 1957 but was not signed. The next January, he was signed by the
Yankee scout Jose Seda.

From 1958 to 1965, Clarke showcased his speed and his ability to get on
base in the minor leagues. He made his major league debut on May 13,
1965, against the Boston Red Sox. In his first at-bat, he pinch-hit for
the pitcher
\href{https://fritzpetersondotorg.wordpress.com/2015/07/02/remembering-hal-reniff-on-his-birthday/}{Hal
Reniff} in the seventh inning and hit an infield single.

\includegraphics{https://static01.nyt.com/images/2020/08/08/obituaries/00ClarkeH1/merlin_166614141_059a09fa-37e0-4450-9833-0b91f9c9ca65-articleLarge.jpg?quality=75\&auto=webp\&disable=upscale}

Clarke began his big-league career as a reserve, appearing mostly at
shortstop and as a pinch-hitter. He was made the full-time second
baseman in 1967, succeeding the Yankee stalwart Bobby Richardson, who
had retired after the 1966 season. Playing alongside teammates like
Ruben Amaro, Joe Pepitone, Roy White and Tom Tresh, Clarke proceeded to
lead the club in at-bats, hits, runs, stolen bases and batting average
in 1967, playing in more games than any teammate except Mantle.

From 1965 to 1974, Clarke was one of just 10 players who posted 150 or
more stolen bases and 1,200 or more hits --- a list that also includes
the Hall of Fame players Joe Morgan and Lou Brock.

His best overall season was in 1969, when he appeared in 156 games,
posting a career-high .285 batting average and .339 on-base percentage.
His 183 hits were second among American League hitters that year.

A pesky switch-hitter, Clarke broke up three potential no-hitters during
the 1970 season, all in the ninth inning, with singles off Jim Rooker,
Sonny Siebert and the knuckleballer
\href{https://www.nytimes.com/2006/10/29/sports/baseball/29niekro.html\#:~:text=He\%20was\%2061.,aneurysm\%2C\%20the\%20news\%20agency\%20reported.}{Joe
Niekro} --- and all, remarkably, within one month.

\href{https://www.nytimes.com/1970/07/03/archives/clarke-foils-bid-on-one0ut-single-breaks-up-nohitter-for-3d-time-in.html}{In
Niekro's no-hit bid}, a road game in Detroit on July 2, Clarke was at
bat with one out in the ninth and the count at one ball and no strikes
when he pulled a ground ball between first and second. The Tiger second
baseman Dick McAuliffe corralled the baseball on the outfield grass and
tossed it to Niekro, covering first base. But the throw was low and
pulled Niekro off the bag, enabling the hustling Clarke to reach base
safely and end the no-hitter.

Since 1961 only one other player has broken up three potential
no-hitters in the ninth inning, the Minnesota Twins All-Star Joe Mauer,
though only Clarke did it in one season.

After playing in more than 1,200 games in his 10 seasons with the
Yankees, Clarke was dealt to the San Diego Padres in May 1974. He
appeared in just 42 games with the Padres, batting below .200 before
retiring at the end of that season.

Among players born in the U.S. Virgin Islands, a relatively small
roster, Clarke is the leader in games played, hits, runs, R.B.I.s and
stolen bases.

Image

From 1965 to 1974, Clarke was one of just 10 players who posted 150 or
more stolen bases and 1,200 or more hits.

On his retirement from the game, Clarke returned home and ran baseball
programs for the young. Two participants,
\href{https://sabr.org/bioproj/person/jerry-browne/}{Jerry Browne} and
\href{https://sabr.org/bioproj/person/midre-cummings/}{Midre Cummings},
went on to have major league careers.

In addition to his son Jeffrey, Clarke is survived by another son,
Jason; his sisters, Violet Armstrong and Hollis Jefferson; and four
grandchildren.

While his tenure with the Yankees came during a low point in team
history, Clarke recalled his time in the Bronx fondly, relishing in
particular the fact that he had played for the same storied organization
as his boyhood hero.

``Walking onto the field at the stadium that first time was one of the
biggest things for me,'' he told The Daily News in 2010. ``I grew up
listening to the Yankees on the radio, and
\href{http://www.nydailynews.com/topics/Phil+Rizzuto}{Phil Rizzuto} was
my idol. I associated with him, because he was small and I was small,
and I played shortstop then, too.''

Johnny Diaz contributed reporting.

Advertisement

\protect\hyperlink{after-bottom}{Continue reading the main story}

\hypertarget{site-index}{%
\subsection{Site Index}\label{site-index}}

\hypertarget{site-information-navigation}{%
\subsection{Site Information
Navigation}\label{site-information-navigation}}

\begin{itemize}
\tightlist
\item
  \href{https://help.nytimes.com/hc/en-us/articles/115014792127-Copyright-notice}{©~2020~The
  New York Times Company}
\end{itemize}

\begin{itemize}
\tightlist
\item
  \href{https://www.nytco.com/}{NYTCo}
\item
  \href{https://help.nytimes.com/hc/en-us/articles/115015385887-Contact-Us}{Contact
  Us}
\item
  \href{https://www.nytco.com/careers/}{Work with us}
\item
  \href{https://nytmediakit.com/}{Advertise}
\item
  \href{http://www.tbrandstudio.com/}{T Brand Studio}
\item
  \href{https://www.nytimes.com/privacy/cookie-policy\#how-do-i-manage-trackers}{Your
  Ad Choices}
\item
  \href{https://www.nytimes.com/privacy}{Privacy}
\item
  \href{https://help.nytimes.com/hc/en-us/articles/115014893428-Terms-of-service}{Terms
  of Service}
\item
  \href{https://help.nytimes.com/hc/en-us/articles/115014893968-Terms-of-sale}{Terms
  of Sale}
\item
  \href{https://spiderbites.nytimes.com}{Site Map}
\item
  \href{https://help.nytimes.com/hc/en-us}{Help}
\item
  \href{https://www.nytimes.com/subscription?campaignId=37WXW}{Subscriptions}
\end{itemize}
