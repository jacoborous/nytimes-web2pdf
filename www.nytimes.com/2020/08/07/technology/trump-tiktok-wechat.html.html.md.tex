Sections

SEARCH

\protect\hyperlink{site-content}{Skip to
content}\protect\hyperlink{site-index}{Skip to site index}

\href{https://www.nytimes.com/section/technology}{Technology}

\href{https://myaccount.nytimes.com/auth/login?response_type=cookie\&client_id=vi}{}

\href{https://www.nytimes.com/section/todayspaper}{Today's Paper}

\href{/section/technology}{Technology}\textbar{}Trump Swings Against
TikTok, WeChat

\href{https://nyti.ms/2PxVf0T}{https://nyti.ms/2PxVf0T}

\begin{itemize}
\item
\item
\item
\item
\item
\end{itemize}

Advertisement

\protect\hyperlink{after-top}{Continue reading the main story}

Supported by

\protect\hyperlink{after-sponsor}{Continue reading the main story}

on tech

\hypertarget{trump-swings-against-tiktok-wechat}{%
\section{Trump Swings Against TikTok,
WeChat}\label{trump-swings-against-tiktok-wechat}}

President Trump signed executive orders against two popular China-based
apps. Let me try to explain.

\includegraphics{https://static01.nyt.com/images/2020/08/07/business/07ontech-videostill/07ontech-videostill-threeByTwoMediumAt2X.png}

\href{https://www.nytimes.com/by/shira-ovide}{\includegraphics{https://static01.nyt.com/images/2020/03/18/reader-center/author-shira-ovide/author-shira-ovide-thumbLarge-v2.png}}

By \href{https://www.nytimes.com/by/shira-ovide}{Shira Ovide}

\begin{itemize}
\item
  Aug. 7, 2020
\item
  \begin{itemize}
  \item
  \item
  \item
  \item
  \item
  \end{itemize}
\end{itemize}

\emph{This article is part of the On Tech newsletter. You can}
\href{https://www.nytimes.com/newsletters/signup/OT}{\emph{sign up
here}} \emph{to receive it weekdays.}

The most honest explanation for the White House's latest move is this:
????? But I'll do the best I can.

On Thursday, the Trump administration
\href{https://www.nytimes.com/2020/08/06/technology/trump-wechat-tiktok-china.html}{signed
executive orders} that seemed to seek a ban on two China-based
smartphone apps, TikTok and WeChat, from operating in the United States
or interacting with U.S. companies. This step was a long time coming,
but it still felt surprising.

The practical significance of these White House mandates and a similar
\href{https://www.state.gov/announcing-the-expansion-of-the-clean-network-to-safeguard-americas-assets/}{State
Department policy statement} this week aren't exactly clear yet. Two
high-profile apps might be gone from the United States in 45 days, or
not. But the philosophical implications are concerning.

On paper at least, internet policy in the United States is creeping a
little closer to what happens in countries like Russia and India: The
government makes draconian rules about what technology its citizens are
allowed to use. And it can be hard for people to know if those rules are
based on legitimate national security concerns or expressions of
nationalism.

If you've been following this newsletter, you know that some U.S.
officials are
\href{https://www.nytimes.com/2020/07/26/technology/tiktok-china-ban-model.html}{worried
about TikTok} potentially handing over reams of data collected on
Americans to the Chinese government, and helping spread a
Chinese-friendly view of the world. There are similar fears about
WeChat, which is
\href{https://www.nytimes.com/2020/08/07/business/trump-china-wechat-tiktok.html}{widely
used in China and also by Chinese people abroad} and those who do
business there.

What we have now is a series of
\href{https://www.nytimes.com/2020/08/03/technology/tiktok-microsoft.html}{just
plain weird things} that could get even weirder. First, in the case of
TikTok, the U.S. government is effectively negotiating --- in public ---
\href{https://www.nytimes.com/2020/08/03/technology/trump-tiktok-microsoft.html}{a
sale between a foreign company}, ByteDance, and an American company,
Microsoft.

And I don't know what will happen to WeChat, or how either order would
be enforced. Is the White House going to demand that Apple, Google or
U.S. internet providers prevent people from downloading WeChat and
TikTok? Would this apply to Apple and Google \emph{outside} the United
States, too?

And what happens to video games like League of Legends and Fortnite, and
the carmaker Tesla? WeChat's parent company, Tencent, owns all or parts
of those properties.

It's possible none of this will happen: Executive orders sometimes never
become action.

What we're seeing now is a struggle to figure out the right approach to
U.S. digital security. (At least, I think that's the goal.)

TikTok is a black box that siphons large amounts of data about users
(like other social networks) and is likely beholden to orders of the
Chinese government.

Chinese security forces have used WeChat to spy on and intimidate
members of the Chinese diaspora. Chinese-backed hackers regularly try to
steal
\href{https://www.nytimes.com/2015/06/05/us/breach-in-a-federal-computer-system-exposes-personnel-data.html}{sensitive
digital information} about
\href{https://www.nytimes.com/2020/02/10/us/politics/equifax-hack-china.html}{Americans}
and
\href{https://www.nytimes.com/2020/07/21/us/politics/china-hacking-coronavirus-vaccine.html}{secret
business information} of American companies.

The internet, like everything else, should be subject to appropriate
government regulation and rules. I mentioned
\href{https://www.nytimes.com/2018/04/18/world/europe/russia-telegram-shutdown.html}{Russia}
and India because those countries sometimes cross the line between
protecting citizens and
\href{https://www.nytimes.com/2019/02/14/technology/india-internet-censorship.html}{cutting
them off from the outside world}.

The United States isn't threatening to turn off the internet,
\href{https://www.nytimes.com/2020/01/26/world/asia/kashmir-internet-shutdown-india.html}{as
India sometimes does}. But the U.S. government is taking a hard line to
cut off apps that Americans rely on, and it's not clear how much
protection Americans will gain in return.

It feels like there must be a middle ground --- as my colleague
\href{https://www.nytimes.com/2020/07/26/technology/tiktok-china-ban-model.html}{Kevin
Roose recently suggested} --- that addresses some of the potential risks
of foreign-owned apps without resorting to extreme measures.

\emph{If you don't already get this newsletter in your inbox,}
\href{https://www.nytimes.com/newsletters/signup/OT}{\emph{please sign
up here}}\emph{.}

\begin{center}\rule{0.5\linewidth}{\linethickness}\end{center}

\includegraphics{https://static01.nyt.com/images/2020/08/07/business/07ontech-octopus1/merlin_175350516_face028d-ddaa-47e9-a437-83e97a3492ac-articleLarge.jpg?quality=75\&auto=webp\&disable=upscale}

\hypertarget{when-antimonopoly-memes-ruled}{%
\subsection{When antimonopoly memes
ruled}\label{when-antimonopoly-memes-ruled}}

There are tempting comparisons between today's American tech superpowers
and the industrial monopolies of a century ago. What we really need are
the old memes.

In the late 19th and early 20th centuries, industrial conglomerates like
Standard Oil and U.S. Steel were regularly portrayed as octopuses or
other tentacled creatures in illustrations, editorial cartoons and other
visual depictions.

I kept coming across these octopus drawings as I dug into the history of
U.S. antitrust law, which was a response to those old industrial
monopolies. This history is relevant today, of course, because a handful
of American tech superpowers are now being compared to those Gilded Age
trusts.

\href{https://www.holycross.edu/academics/programs/history/faculty/edward-t-odonnell}{Edward
O'Donnell}, the chair of the history department at College of the Holy
Cross, said tentacled monsters were very popular images a century ago to
depict anxieties about big corporations unfairly wielding power and
poisoning food, mistreating workers and bending government officials and
competitors to their will.

The octopus was, yes,
\href{https://en.wikipedia.org/wiki/Internet_meme}{a meme}.

There's this famous 1904 illustration of Standard Oil (see above) with
one tentacle wrapped around the U.S. Capitol building, another squeezing
white collar and blue collar workers and others grabbing at depictions
of the shipping, steel and other industries.

``\href{http://nationalhumanitiescenter.org/pds/gilded/power/text1/octopusimages.pdf}{The
Curse of California},'' which appeared in a satirical magazine in 1882,
showed the bosses of the Southern Pacific railroad in the eyes of a
monster that was squeezing miners, farmers, wealthy power brokers, wheat
exports and stage coaches.

Illustrators of the age knew what
\href{https://www.nytimes.com/2020/05/07/style/memers-are-taking-over-tiktok.html}{teens
on TikTok} and
\href{https://www.nytimes.com/2017/11/01/us/politics/russia-2016-election-facebook.html}{Russian
propagandists on Facebook} figured out nearly a century later: A
powerful visual can influence public opinion about important topics.

You may recognize that the tentacled creature is also a visual used in
anti-Semitic tropes. This is not a coincidence, Dr. O'Donnell said.
Institutions and people who were believed to have achieved power
unfairly and abused it have often been depicted with octopus images.

In addition to Jews and monopolists, subjects that got the tentacles
treatment included
\href{https://fineartamerica.com/featured/imperialism-cartoon-1882-granger.html}{British
imperialism} in the 19th century, the
\href{https://www.popsci.com/article/technology/brief-history-octopi-taking-over-world/}{Soviet
Union under Stalin} and
\href{https://digital.library.cornell.edu/catalog/ss:3293931}{1940s
Japan}. And now, it seems,
\href{https://www.esquire.com/news-politics/a15895746/bust-big-tech-silicon-valley}{the
21st century technology superpowers}.

\begin{center}\rule{0.5\linewidth}{\linethickness}\end{center}

\hypertarget{before-we-go-}{%
\subsection{Before we go \ldots{}}\label{before-we-go-}}

\begin{itemize}
\item
  \textbf{It's not an election day. It's election days/weeks:} The
  institutions most responsible for our understanding of government ---
  the news media and, yes, social networks --- are preparing for how to
  communicate the results of this November's U.S. elections.

  My colleague Ben Smith wrote earlier this week about
  \href{https://www.nytimes.com/2020/08/02/business/media/election-coverage.html}{news
  organizations planning for voting results to take days or weeks}, in
  part because of heavier use of mail-in voting during the pandemic.
  BuzzFeed News also wrote that some Facebook employees are worried that
  longer ballot counting will create opportunities for
  \href{https://www.buzzfeednews.com/article/craigsilverman/facebook-zuckerberg-what-if-trump-disputes-election-results}{people
  to use Facebook to undermine the legitimacy of the election}.
\item
  \textbf{The risks of turning mental health care into an app:}
  Talkspace, the app that lets people text and chat with a licensed
  therapist, has made therapy much more accessible, but my colleagues
  Kashmir Hill and Aaron Krolik also found cases in which the company
  treated patient confidentiality cavalierly and prioritized marketing
  over patient welfare.

  Their reporting, Kash and Aaron wrote, suggested ``that the needs of a
  venture capital-backed start-up to grow quickly can sometimes be in
  conflict with the core values of professional therapy.''
\item
  \textbf{Microsoft Excel is changing our genes (sort of):} Scientists
  have run into trouble when alphanumeric symbols of genes like
  ``MARCH1'' are misread as dates when they're typed into Excel
  spreadsheets. And because it's apparently easier to change scientific
  nomenclature than to change the settings in Excel, the
  \href{https://www.theverge.com/2020/8/6/21355674/human-genes-rename-microsoft-excel-misreading-dates}{names
  of about 27 genes have been changed to avoid spreadsheet
  misinterpretation}, according to The Verge.
\end{itemize}

\hypertarget{hugs-to-this}{%
\subsubsection{Hugs to this}\label{hugs-to-this}}

A
\href{https://www.facebook.com/FortWorthZoo/videos/422445615335977/}{young
gorilla is trying, and failing, to get his father to goof around with
him}. (The kid gorilla is Gus, by the way. Elmo is the big poppa.)

\begin{center}\rule{0.5\linewidth}{\linethickness}\end{center}

\emph{We want to hear from you. Tell us what you think of this
newsletter and what else you'd like us to explore. You can reach us at}
\href{mailto:ontech@nytimes.com?subject=On\%20Tech\%20Feedback}{\emph{ontech@nytimes.com.}}
**

\emph{If you don't already get this newsletter in your inbox,}
\href{https://www.nytimes.com/newsletters/signup/OT}{\emph{please sign
up here}}\emph{.}

Advertisement

\protect\hyperlink{after-bottom}{Continue reading the main story}

\hypertarget{site-index}{%
\subsection{Site Index}\label{site-index}}

\hypertarget{site-information-navigation}{%
\subsection{Site Information
Navigation}\label{site-information-navigation}}

\begin{itemize}
\tightlist
\item
  \href{https://help.nytimes.com/hc/en-us/articles/115014792127-Copyright-notice}{©~2020~The
  New York Times Company}
\end{itemize}

\begin{itemize}
\tightlist
\item
  \href{https://www.nytco.com/}{NYTCo}
\item
  \href{https://help.nytimes.com/hc/en-us/articles/115015385887-Contact-Us}{Contact
  Us}
\item
  \href{https://www.nytco.com/careers/}{Work with us}
\item
  \href{https://nytmediakit.com/}{Advertise}
\item
  \href{http://www.tbrandstudio.com/}{T Brand Studio}
\item
  \href{https://www.nytimes.com/privacy/cookie-policy\#how-do-i-manage-trackers}{Your
  Ad Choices}
\item
  \href{https://www.nytimes.com/privacy}{Privacy}
\item
  \href{https://help.nytimes.com/hc/en-us/articles/115014893428-Terms-of-service}{Terms
  of Service}
\item
  \href{https://help.nytimes.com/hc/en-us/articles/115014893968-Terms-of-sale}{Terms
  of Sale}
\item
  \href{https://spiderbites.nytimes.com}{Site Map}
\item
  \href{https://help.nytimes.com/hc/en-us}{Help}
\item
  \href{https://www.nytimes.com/subscription?campaignId=37WXW}{Subscriptions}
\end{itemize}
