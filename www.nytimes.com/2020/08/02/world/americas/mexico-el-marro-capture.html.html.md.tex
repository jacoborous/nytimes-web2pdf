Sections

SEARCH

\protect\hyperlink{site-content}{Skip to
content}\protect\hyperlink{site-index}{Skip to site index}

\href{https://www.nytimes.com/section/world/americas}{Americas}

\href{https://myaccount.nytimes.com/auth/login?response_type=cookie\&client_id=vi}{}

\href{https://www.nytimes.com/section/todayspaper}{Today's Paper}

\href{/section/world/americas}{Americas}\textbar{}Mexico Seizes Crime
Boss El Marro, Under Pressure to Cut Violence

\url{https://nyti.ms/33iDL0w}

\begin{itemize}
\item
\item
\item
\item
\item
\end{itemize}

Advertisement

\protect\hyperlink{after-top}{Continue reading the main story}

Supported by

\protect\hyperlink{after-sponsor}{Continue reading the main story}

\hypertarget{mexico-seizes-crime-boss-el-marro-under-pressure-to-cut-violence}{%
\section{Mexico Seizes Crime Boss El Marro, Under Pressure to Cut
Violence}\label{mexico-seizes-crime-boss-el-marro-under-pressure-to-cut-violence}}

While the arrest of José Antonio Yépez delivers a major blow to a
cartel, analysts say officials lack a cohesive strategy against
debilitating crime.

\includegraphics{https://static01.nyt.com/images/2020/08/02/world/02mexico2/merlin_170327064_c214ce7c-82c0-4a6b-ab03-3dc20289260d-articleLarge.jpg?quality=75\&auto=webp\&disable=upscale}

By \href{https://www.nytimes.com/by/azam-ahmed}{Azam Ahmed}

\begin{itemize}
\item
  Aug. 2, 2020
\item
  \begin{itemize}
  \item
  \item
  \item
  \item
  \item
  \end{itemize}
\end{itemize}

MEXICO CITY --- The Mexican federal authorities captured José Antonio
Yépez, the criminal boss known as El Marro, on Sunday, landing a major
blow against a cartel whose struggle for control helped spur record
violence in the midst of the coronavirus pandemic.

After his arrest in an early-morning raid, low-resolution photographs of
his capture were released by law enforcement agencies eager to highlight
the latest success in their campaign against organized crime.

What is less clear is whether Mr. Yépez's imprisonment will make any
meaningful difference in the violence that has subsumed Mexico --- or in
the prevalence of organized crime more broadly.

``This is basically a short-lived P.R. victory, but it doesn't provide a
solution,'' said Falko Ernst, a Mexico analyst for the International
Crisis Group. ``The big worry is that there is no backing in terms of a
more cohesive security strategy.''

Mexico's president, Andrés Manuel López Obrador, has been dogged by
criticism that he almost completely lacks a security strategy. When
challenged about rising violence and the government's response to it,
Mr. López Obrador has always said he would take a nonconfrontational
approach that focused on the causes of crime: hugs and not bullets, in
the president's words.

The strategy, such as it is, is at its core a reaction to the failed
strategies of his predecessors. Since 2006, when the Mexican government
declared a war on drugs, it has focused on arresting and killing
traffickers. And yet in the first months of this year, Mexico registered
\href{https://www.reuters.com/article/us-mexico-violence/murders-in-mexico-reach-record-levels-in-first-four-months-of-2020-idUSKBN22W2JC}{more
homicides} than at any point in the last two decades.

\includegraphics{https://static01.nyt.com/images/2020/08/02/world/02mexico/merlin_175231914_92dfcc35-9e2d-4179-87bd-21419e25a40f-articleLarge.jpg?quality=75\&auto=webp\&disable=upscale}

The president, also known by his initials, AMLO, vowed not to conduct
arrests as public spectacles, or otherwise continue on the same path as
previous leaders. But the sudden arrest of Mr. Yépez, the leader of the
Santa Rosa de Lima cartel, seems to run contrary to that mantra.

``It shows how desperate AMLO is to show he is doing something,'' said
David Shirk, a professor of political science at the University of San
Diego. ``The fact is he just did something that he said he would never
do. It's the same old playbook as before.''

The Santa Rosa de Lima cartel began its reign in the state of
Guanajuato, pilfering oil from pipelines that crisscross that area of
central Mexico and siphoning off amounts estimated at one point to be
valued at nearly \$2 million a day.

As the head of a small start-up cartel, which analysts say was largely
run as a family crime group, Mr. Yépez showed uncharacteristic pluck,
challenging both the government and much larger and more diversified
criminal groups.

In emotional videos, Mr. Yépez has often lashed out at his enemies and
even threatened the president himself if federal troops were not
withdrawn from his native state, where they had been sent to fight fuel
theft.

But the government of Mr. López Obrador, which has placed paramount
importance on the oil economy, kept targeting the oil racket. Already
this year, the authorities had arrested Mr. Yépez's mother and sister,
prompting additional emotional videos.

The revenue from the oil theft, meanwhile, was too lucrative for other
criminal organizations to resist --- specifically the much larger and
more prominent New Generation cartel of Jalisco. The fight between the
two groups made Guanajuato the country's deadliest state last year, with
more than 3,000 killings. This year, it is on track to exceed that
figure.

``We are talking about a state with 12 homicides a day, and 360 murders
in the last month alone,'' said Eduardo Guerrero, a security analyst in
Mexico City. ``That's 15 percent of the nation's homicides.''

The New Generation cartel also tried to
\href{https://www.nytimes.com/2020/06/26/world/americas/mexico-city-police-chief-shot.html}{assassinate
the head of security} in Mexico City in a brazen daytime fusillade in
June.

Mr. Yépez's arrest is certain to set off the shifting of key criminal
players in the state of Guanajuato, the forging of new alliances and a
splintering of groups.

Past captures of kingpins have seldom improved the dynamic in Mexico.
New players enter, old ones exit and the same patterns repeat on a loop.
Drugs flow north of the border, guns flow south and Mexicans die in the
arbitration of who gets to control what.

The authorities similarly crowed about
\href{https://www.nytimes.com/2019/02/12/nyregion/el-chapo-verdict.html}{the
conviction of Joaquín Guzmán Loera}, the drug lord known as El Chapo,
who was
\href{https://www.nytimes.com/2019/07/17/nyregion/el-chapo-sentencing.html}{sentenced
last year to life in prison} in the United States. And yet the fight
waged by the New Generation cartel for primacy since his departure has
left more bodies than ever in its wake.

No one knows what will follow this most recent arrest. Analysts are
split.

In one possible outcome, infighting among the remnants of the Santa Rosa
de Lima cartel could lead to a greater fracturing and diffusion --- and
therefore increases --- in violence.

In another, the New Generation takes over, gains control over the state
and violence drops because it has no rivals.

``A reduction in violence there would be a very important achievement
for the federal government,'' Mr. Guerrero said.

Natalie Kitroeff and Paulina Villegas contributed reporting.

Advertisement

\protect\hyperlink{after-bottom}{Continue reading the main story}

\hypertarget{site-index}{%
\subsection{Site Index}\label{site-index}}

\hypertarget{site-information-navigation}{%
\subsection{Site Information
Navigation}\label{site-information-navigation}}

\begin{itemize}
\tightlist
\item
  \href{https://help.nytimes.com/hc/en-us/articles/115014792127-Copyright-notice}{©~2020~The
  New York Times Company}
\end{itemize}

\begin{itemize}
\tightlist
\item
  \href{https://www.nytco.com/}{NYTCo}
\item
  \href{https://help.nytimes.com/hc/en-us/articles/115015385887-Contact-Us}{Contact
  Us}
\item
  \href{https://www.nytco.com/careers/}{Work with us}
\item
  \href{https://nytmediakit.com/}{Advertise}
\item
  \href{http://www.tbrandstudio.com/}{T Brand Studio}
\item
  \href{https://www.nytimes.com/privacy/cookie-policy\#how-do-i-manage-trackers}{Your
  Ad Choices}
\item
  \href{https://www.nytimes.com/privacy}{Privacy}
\item
  \href{https://help.nytimes.com/hc/en-us/articles/115014893428-Terms-of-service}{Terms
  of Service}
\item
  \href{https://help.nytimes.com/hc/en-us/articles/115014893968-Terms-of-sale}{Terms
  of Sale}
\item
  \href{https://spiderbites.nytimes.com}{Site Map}
\item
  \href{https://help.nytimes.com/hc/en-us}{Help}
\item
  \href{https://www.nytimes.com/subscription?campaignId=37WXW}{Subscriptions}
\end{itemize}
