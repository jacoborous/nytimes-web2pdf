Sections

SEARCH

\protect\hyperlink{site-content}{Skip to
content}\protect\hyperlink{site-index}{Skip to site index}

\href{https://myaccount.nytimes.com/auth/login?response_type=cookie\&client_id=vi}{}

\href{https://www.nytimes.com/section/todayspaper}{Today's Paper}

\href{/section/opinion}{Opinion}\textbar{}Real Life Horror Stories From
the World of Pandemic Motherhood

\href{https://nyti.ms/3km6tE2}{https://nyti.ms/3km6tE2}

\begin{itemize}
\item
\item
\item
\item
\item
\item
\end{itemize}

Advertisement

\protect\hyperlink{after-top}{Continue reading the main story}

\href{/section/opinion}{Opinion}

Supported by

\protect\hyperlink{after-sponsor}{Continue reading the main story}

\hypertarget{real-life-horror-stories-from-the-world-of-pandemic-motherhood}{%
\section{Real Life Horror Stories From the World of Pandemic
Motherhood}\label{real-life-horror-stories-from-the-world-of-pandemic-motherhood}}

`I have been given two options: either resign or get fired.'

By Joan C. Williams

Ms. Williams runs the Center for WorkLife Law at the University of
California, Hastings College of the Law.

\begin{itemize}
\item
  Aug. 6, 2020
\item
  \begin{itemize}
  \item
  \item
  \item
  \item
  \item
  \item
  \end{itemize}
\end{itemize}

\includegraphics{https://static01.nyt.com/images/2020/08/04/opinion/04williams/04williams-articleLarge.jpg?quality=75\&auto=webp\&disable=upscale}

Employers are using the pandemic to get rid of mothers, and our attempts
to protect them are failing.

The Families First Coronavirus Response Act was enacted this spring for
the express purpose of providing workers with expanded family and sick
leaves for reasons related to Covid-19 and its accompanying school and
child care closings. But between April and June, caregiver-related calls
to our \href{https://worklifelaw.org/covid19/}{hotline} at the Center
for WorkLife Law, which provides legal resources to help workers claim
workplace accommodations and family leaves, increased 250 percent
compared to the same time last year. We've heard from lots and lots of
workers, many of them mothers. And the stories they're sharing make it
clear that Families First is falling short.

One single mom is ineligible for Families First, which excludes health
care workers, emergency responders and those who work for companies with
over 500 employees. She has no child care options for her 6-year-old and
8-month-old. She exhausted all of her paid leave options while on
maternity leave. ``I have been given two options: either resign or get
fired,'' she told us. She resigned. She's one of an estimated
\href{https://www.americanprogress.org/issues/economy/news/2020/04/17/483287/coronavirus-paid-leave-exemptions-exclude-millions-workers-coverage/}{106
million people} not guaranteed coverage under the act.

Even those who appear to be covered by Families First often end up
losing their jobs. A single mom wanted to begin to work part time,
taking Families First leave for a few days each week. She felt this
worked well, but at the time, taking leave in chunks was allowed only if
the employer agreed to it. Hers ultimately didn't --- and she was fired.
(On Monday, a federal judge in New York
\href{https://www.law360.com/employment/articles/1297787}{ruled it
illegal} for employers to refuse intermittent leaves; the Trump
administration will likely appeal that decision.)

One grocery worker was able to return to work --- provided it was on the
same part-time schedule she had always worked. But when she asked for
that, her employer cut her to zero hours and ghosted her, refusing to
respond to queries about why those hours had been reduced, whether she
was laid off, what was happening. She's out of luck unless she can prove
her termination was discriminatory, which is often hard and sometimes
impossible.

We heard from another single mother whose daughter has a disability that
makes her especially vulnerable to Covid, and who had successfully
worked from home since near the beginning of the pandemic. She was fired
because her employer insisted she return to the office, which she
couldn't do without putting her daughter at risk. If a worker has an
underlying medical condition, sometimes we can get them telecommuting as
an accommodation under the Americans With Disabilities Act. But if they
need to telecommute to protect the health of a relative, typically
they're out of luck.

We know that Covid-related job loss has
\href{https://nyti.ms/2LijQEA}{disproportionately affected women}. We
also know that the women we're hearing from aren't quitting because they
don't want to work; they're being driven out by a combination of family
care requirements and employer rigidity. And when workers try to push
back, they face a labyrinth of laws that are often ineffectual.

Figuring out whether you're eligible for Families First is so
complicated that a
\href{https://www.lw.com/thoughtLeadership/lw-ffcra-leave}{chart}
explaining the program looks like a game of Chutes and Ladders. It seems
clear that many states understand neither Families First nor a companion
program known as Pandemic Unemployment Assistance.

Traditionally, workers have been denied unemployment when they leave a
job because of a lack of child care; Pandemic Unemployment Assistance
explicitly reversed this until the end of the year. If calls to our
hotline are any indication, many employers don't know that, and some
states have set up Byzantine systems that ask workers to apply for
standard unemployment and get rejected before they apply for Pandemic
Unemployment Assistance. (To add to the chaos, Virginia
\href{https://www.richmond.com/news/virginia/when-the-school-year-ends-parents-cant-claim-unemployment-for-lack-of-child-care-state/article_c9d39f96-adcf-5054-8b44-b9a19fa425c8.html}{announced
a policy} of denying unemployment insurance to workers whose kids' camps
are closed --- a clear violation of the act, as the
\href{https://www.dol.gov/sites/dolgov/files/WHD/legacy/files/fab_2020_4.pdf}{Department
of Labor} recently reiterated.) The end result is that many mothers find
that once they \emph{have} been pushed out, employers derail their
unemployment claims on the grounds that they left their jobs for
personal reasons.

One single mother of two found herself without day care and had no
income for two months while the state twice deemed her ineligible for
unemployment benefits. Another couldn't even appeal her state's decision
because of a faulty internet connection. We hear from low-income women
who have to return to work, leaving small children home alone. Now they
worry someone will call Child Protective Services and they will lose
their children.

Recently, we're hearing a lot from mothers whose 12-week Families First
leaves are running out, and who still have no option for child care. If
schools aren't given the resources to open safely this fall, there's
going to be a blood bath. As it is, we may well be facing a generational
wipeout of mothers' careers:
\href{https://www.americanprogress.org/issues/early-childhood/reports/2016/06/21/139731/calculating-the-hidden-cost-of-interrupting-a-career-for-child-care/}{research}
shows that when mothers leave the labor force it hurts their economic
prospects for decades, often permanently. A society that pushes mothers
out of their jobs is a society that impoverishes both mothers and
children.

We're in this mess because, even before coronavirus, the legal
protections for working mothers consisted of a convoluted matrix of
federal, state and local laws. Mothers who want time and space for
pumping breast milk turn to not-very-enforceable provisions of the
Affordable Care Act. Mothers who need pregnancy accommodations often
turn to the Americans With Disabilities Act. Mothers fired when a
disabled child's health care costs cause their employer's insurance
costs to skyrocket turn to a tax law. The lack of straightforward legal
protections is just one of many ways that public policy fails mothers;
the haphazard nature of Families First is merely one symptom of a
broader problem.

This crisis should help us finally recognize that mothers are raising
the next generation of citizens; motherhood is not a private frolic like
hang gliding. In June, Senator Cory Booker introduced
\href{https://www.insidernj.com/press-release/sen-booker-introduces-legislation-protecting-family-caregivers-discrimination/}{legislation}
that would, in a simple and straightforward way, protect all mothers ---
and fathers, and other family caregivers --- from employment
discrimination. That's long overdue but we need much more. If, God and
Wisconsin willing, Democrats win in November, we also need
\href{https://www.pewresearch.org/fact-tank/2019/12/16/u-s-lacks-mandated-paid-parental-leave/}{nationwide
paid family leave} and what many other advanced industrial countries
also have: neighborhood-based, nationally financed child care to replace
the patched-together
\href{https://news.bloomberglaw.com/daily-labor-report/without-child-care-back-to-work-parents-have-few-legal-options}{Rube
Goldberg machine that just broke}.

Joan C. Williams is a professor of law and director of the Center for
WorkLife Law at the University of California, Hastings, College of the
Law.

\emph{The Times is committed to publishing}
\href{https://www.nytimes.com/2019/01/31/opinion/letters/letters-to-editor-new-york-times-women.html}{\emph{a
diversity of letters}} \emph{to the editor. We'd like to hear what you
think about this or any of our articles. Here are some}
\href{https://help.nytimes.com/hc/en-us/articles/115014925288-How-to-submit-a-letter-to-the-editor}{\emph{tips}}\emph{.
And here's our email:}
\href{mailto:letters@nytimes.com}{\emph{letters@nytimes.com}}\emph{.}

\emph{Follow The New York Times Opinion section on}
\href{https://www.facebook.com/nytopinion}{\emph{Facebook}}\emph{,}
\href{http://twitter.com/NYTOpinion}{\emph{Twitter (@NYTopinion)}}
\emph{and}
\href{https://www.instagram.com/nytopinion/}{\emph{Instagram}}\emph{.}

Advertisement

\protect\hyperlink{after-bottom}{Continue reading the main story}

\hypertarget{site-index}{%
\subsection{Site Index}\label{site-index}}

\hypertarget{site-information-navigation}{%
\subsection{Site Information
Navigation}\label{site-information-navigation}}

\begin{itemize}
\tightlist
\item
  \href{https://help.nytimes.com/hc/en-us/articles/115014792127-Copyright-notice}{©~2020~The
  New York Times Company}
\end{itemize}

\begin{itemize}
\tightlist
\item
  \href{https://www.nytco.com/}{NYTCo}
\item
  \href{https://help.nytimes.com/hc/en-us/articles/115015385887-Contact-Us}{Contact
  Us}
\item
  \href{https://www.nytco.com/careers/}{Work with us}
\item
  \href{https://nytmediakit.com/}{Advertise}
\item
  \href{http://www.tbrandstudio.com/}{T Brand Studio}
\item
  \href{https://www.nytimes.com/privacy/cookie-policy\#how-do-i-manage-trackers}{Your
  Ad Choices}
\item
  \href{https://www.nytimes.com/privacy}{Privacy}
\item
  \href{https://help.nytimes.com/hc/en-us/articles/115014893428-Terms-of-service}{Terms
  of Service}
\item
  \href{https://help.nytimes.com/hc/en-us/articles/115014893968-Terms-of-sale}{Terms
  of Sale}
\item
  \href{https://spiderbites.nytimes.com}{Site Map}
\item
  \href{https://help.nytimes.com/hc/en-us}{Help}
\item
  \href{https://www.nytimes.com/subscription?campaignId=37WXW}{Subscriptions}
\end{itemize}
